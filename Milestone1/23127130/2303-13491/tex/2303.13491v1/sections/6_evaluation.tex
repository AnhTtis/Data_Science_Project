\section{Evaluation}\label{sec:eval}

To demonstrate the effectiveness of \textit{FraudAuditor}, we conducted two case studies and expert interviews using the real-world health insurance dataset. We sampled the records of 1,035 patients in a district from 2019 to 2020, which contained more than 46,000 visit records and more than 300,000 drug records. \rc{In actual scenarios, experts will select data of similar scale for analysis by spatio-temporal filtering.}

\subsection{Case Studies}

\subsubsection{Examining Crafty Fraud Group}

\begin{figure*}[htb!]
    \centering % avoid the use of \begin{center}...\end{center} and use \centering instead (more compact)
    \includegraphics[width=2\columnwidth]{figures/case2.pdf}
    \caption{The visual analysis process of case 1 mainly consists of (a) selecting two suspicious groups and their neighbors, (b) filtering groups, (c) browsing patients' attributes, (d) checking frequent co-visit records of short duration, and (e) verifying co-visit behaviors in multiple time periods.}
    \label{fig:case1}
\end{figure*}

A health insurance auditor wanted to detect complex patterns of fraudulent behavior. As shown in the \autoref{fig:teaser}-a1, he chose the drugstore and community hospital in the patient attributes view, as they are poorly regulated and fraud-prone.
He noticed a significant decrease in the number of patients with 1-20 visits. The majority visited drugstores and community hospitals concentrated in the range of 21-60 times. 
Based on the prior knowledge, he set the maximum time gap for a co-visit to 1 hour and the number of co-visit to at least 2.
To avoid missing collusive fraud, while excluding single patients and small groups of only two patients, he set the minimum group size as 3 and adjusted the minimum number of co-visits from 4 to 2 to generate the co-visit network. He then clicked on the button to generate the co-visit network and browsed the results of the collusive fraud detection models, which contained 48 groups.

As shown in \autoref{fig:case1}-b, to filter the groups with short visit intervals, he selected the 0-1 minute bar in the chart about the minimum gap within the co-visit in the group attributes view.
In the remaining groups, he noticed that the two groups in the patient co-visit network are very closely connected (\autoref{fig:case1}-a). Four surrounding points connected to them aroused the expert's interest and were therefore added for analysis.
% He selects these points to analyze the individual patients in the patient comparison view.
In the patient comparison view, he noticed that the selected patients had low disease similarity and high drug similarity (\autoref{fig:teaser}-c). To understand why diseases and drugs did not correspond, he selected all the patients and found that their fees fluctuated wildly.

He then went to the patient behavior view to investigate the details of visits. He noticed a spike in the timeline of the number of visits and dense co-visit links in the visit sequence visualization around December 2019 (\autoref{fig:teaser}-d).
Patients P-0037 and P-0960, in particular, had no medical records from January 2019 to November 2019 and rarely visited medical institutions after 2020. 
Their first medical visit was a co-visit in December 2019.
So he narrowed down the selected time range (\autoref{fig:case1}-d).
After adjusting the time granularity to a week, he noticed multiple co-visits in the weeks of December 19 and 26, indicating suspicious frequent co-visits in a short period.
Such frequent co-visits in a short time were very suspicious.
So he adjusted the co-visit time gap to 15 minutes and found that the co-visits of these patients all took place in Institution 9505010 (\autoref{fig:case1}-e), which was the drugstore they often visited (\autoref{fig:case1}-c).
He hovered over the bar of co-visit to check their specific time of visits and found that their time for medicine purchase was very close, and many times within 1 minute. 
The cost of each visit was also within a certain range.
He clicked on patients to view the specific information about diseases and fees.
Results showed that all the diseases had no specific directivity and were messy, and the cost of each visit was relatively fixed. 
Therefore, he speculated that these patients were likely a collusive fraud group and tagged with the corresponding labels.
The subsequent investigation confirmed that these fraudsters collude with salespeople to make money by frequently buying drugs to resell while also helping the drugstore increase sales for kickbacks.

% 调查结果

\begin{figure}[tb]
    \centering % avoid the use of \begin{center}...\end{center} and use \centering instead (more compact)
    \includegraphics[width=\columnwidth]{figures/case-chronic.pdf}
    \caption{The visual analysis process in case 2 mainly consists of (a) filtering patients based-on attributes, (b) detecting groups, (c) identifying groups based on connections, (d) selecting suspicious groups, (e) comparing intra-group similarity, (f) exploring detailed patient behaviors, and (g) reasoning through contextual information.}
    \label{fig:case2}
\end{figure}

\subsubsection{Excluding False-positive Group with Chronic Diseases}

% The user wants to explore whether the patients would seek medical attention in a group due to chronic diseases. 
Another health insurance auditor wanted to exclude false-positive groups, misclassified as collusive fraud because of collective visit behavior for chronic diseases.
After reviewing the patient attributes view, she unchecked patients younger than 30 years old who were less prone to chronic disease.
The bar chart on the number of visits and the number of people showed that the majority of patients with more than 60 visits were the patients retained after filtering and were also the primary visitors to various medical institutes.
After understanding the basic characteristics, she proceeded to observe the co-visit network between these patients.
The resulting network showed that almost everyone was connected.
She attributed this mainly to the fact that the data contained numerous public hospital visits, making it easy for patients to have co-visits with others.
\rc{She, therefore, excluded records of visits to public hospitals (visits to other institutions were still retained) (\autoref{fig:case2}-a) because the large number of visits to these hospitals could easily lead to connections between patients and the likelihood of fraud is lower in well-regulated institutions.} The regenerated network had significantly fewer edges (\autoref{fig:case2}-b), and suspicious groups marked in purple were clearly visible. 

She then found some outliers in the group projection results.
She hovered over one outlier and found that it ranked \#2 on the group list, and had 4 anomalous metrics in the radar chart (\autoref{fig:case2}-d), such as 140 co-visits.
So she clicked on that radar chart and decided to further analyze the patients within that group.
Using the disease bar chart, she found that the primary diseases in this group were chronic diseases such as back pain and hypertension.
The top five drugs purchased included acupuncture, traditional Chinese fumigation medicine, and Qianghuo~\footnote{\url{https://en.wikipedia.org/wiki/Notopterygium_incisum}}, consistent with chronic diseases.
In the matrix view, after clustering the patients by disease, she found that P0056, P0252, and P0514 had a high disease and drug similarity (\autoref{fig:case2}-e), with P0514 and P0056 having up to 90\% disease similarity and the rest having more than 70\% disease similarity between them.
She then selected these three patients, and the highlighted area in the bar chart showed their distribution in each attribute relative to the entire group. Most of these patients' visits were distributed in a community hospital numbered 9202.
The area chart about fees reflected that their medical insurance expenses are relatively stable.

She suspected the three patients might have been mistakenly detected as a collusive fraud group, so she moved to the patient behavior view to investigate visit behavior for further confirmation.
She found that the pink bars representing back pain dominated the visit sequence visualization and appeared at regular intervals (\autoref{fig:case2}-f).
Between September 2019 and March 2020, there were multiple co-visit links between P0252 and P0514, so she swiped the timeline to this period.
From the co-visit view, she learned that these patients visited the community hospital (ID 9202) every week or so. The height of the bar also showed that they spent about the same amount each time, consistent with the usual regular drug change cycle.
Clicking on patients P0252 and P0514 revealed that their ages and total fees were similar (\autoref{fig:case2}-g), so she inferred that they might be patients who knew each other and went to the doctor together.
So she marked the group as normal in the annotation view and gave the corresponding reason.


\subsection{Expert Interview}

To assess the usefulness and effectiveness of \textit{FraudAuditor}, we conducted interviews with six domain experts. These experts are our collaborators but have not been involved in the design process. Two of them (E1, E2) are experts in health-related products and are familiar with the actual scenarios and business of health insurance.
The other four (E3-E6) are experts in health insurance fraud detection, understanding common fraud patterns, and detection algorithms. 

\textbf{Procedure.} Each interview lasted for 90 minutes. \rc{First, we started with a 30-minute training on \textit{FraudAuditor}'s purpose and usage flow.} Then the experts spent 45 minutes exploring and discovering suspected group fraud using the system. In the last 15 minutes, we asked each expert to complete a questionnaire consisting of quantitative evaluations based on a five-point Likert scale and qualitative questions. We summarize the insights in the following four aspects.

\textbf{Suspicious group mining model.} %Experts agree a typical characteristic of group fraud suspects is that they frequently visit the same institution on the same day. 
Experts agree that the detection algorithm, considering both the number of co-visits and the time gap of a co-visit, can identify suspicious groups more accurately (ratings: 4/5). Experts said most existing automated algorithms were black boxes without mechanical explanation. With \textit{FraudAuditor}, experts can understand the detection mechanics and interactively adjust them (e.g., setting the number of co-visits and the time gap threshold of a co-visit based on their own experience).
The effects of different algorithm parameters on the results are visible in real-time, which can boost experts' trust in the algorithm.
E3 said, \textit{``The co-visit network can help verify many of my hypotheses. For example, public hospitals cause many co-visits, and excluding records in them wouldn't have much effect on the numbers of groups.''} E3 added that \textit{``if the algorithm can take into account the continuity of visits, it will be more effective.''}

\textbf{Visualization and interaction.} Experts believe \textit{FraudAuditor} can meet the needs of analysis. 
% They also appreciate the learnability and usability of \textit{FraudAuditor}. 
\textit{FraudAuditor} uses basic visualizations (e.g., bar charts, line charts, and node-link diagrams) and common statistical metrics (e.g., median and mean).
Since healthcare experts are not familiar with visualization, these simple visual designs lower the learning barrier.
E1 mentioned that \textit{``the patient behavior view (ratings: 4.5/5) is beautiful and practical in design, which can help assess the evolution of the disease and co-visit behavior.''}
E4 noted that \textit{``it takes some exploration to understand the coordination of the system. I never used similar functions before and gradually got familiar with them after a period of use.''}
E2 stated that \textit{``usually we don't have much time to verify each group in detail in real work.
The interactive filtering of patients (ratings: 4.3/5) and groups in the system, as well as the ranking in the customized radar chart (ratings: 4/5), enabled me to select some groups of higher suspicion, which enhanced my analysis efficiency.''}
The feedback from the experts supports the usefulness of \textit{FraudAuditor} to help discover patterns in the data efficiently without having to directly query and interpret the boring raw data.

\textbf{System usability.} Experts agree that the proposed visual analytics approach follows a real health insurance audit workflow.
From overviews to details, each step is easy to understand (ratings: 4.3/5).
\textit{``Compared to the traditional audit process, the system can greatly shorten the time of manual analysis, and the information provided by the system is what I need in the analysis,''} E1 remarked.
E6 commented: \textit{``the system delivers user-friendly interactions, and the analysis process is reasonable.''}
\rc{Evaluations on the learning cost were also presented.}
E5 stated, \textit{``When I first started using this system, I had to read the documentation frequently. The system has a different interface than the health insurance database management system. It is recommended that the visual analysis system guide users during usage.''}
E5 added, \textit{``I want to use the system for my work. Once I am familiar with the process, I can better assess the effectiveness of fraud detection algorithms.''}
Based on experts' feedback, \textit{FraudAuditor} can assist experts in detecting, analyzing, and labeling fraudulent groups.

\rc{\textbf{Suggestions.} Experts also offered valuable suggestions for \textit{FraudAuditor}. E3 suggested that the system could integrate more details of tests, examinations, and procedures in healthcare (currently not available in our dataset), which would help improve the accuracy of fraud verification. E1 also requested additional granular data filtering options, such as selecting data from a particular pharmacy, because sometimes they receive reports of fraud cases and then need to independently extract the relevant data for review. E2 added, \textit{``if the system could support real-time fraud detection and make timely warnings of fraud, it would be more effective in reducing the loss of health insurance funds.''}}