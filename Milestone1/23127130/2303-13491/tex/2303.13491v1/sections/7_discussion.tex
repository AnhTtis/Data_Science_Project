\section{Discussion}\label{sec:discussion}

This section discusses the advantages and limitations of \textit{FraudAuditor} from the perspectives of generalizability and scalability. We also summarize the lessons learned from the implementation process and shed light on future work.

\textbf{Generalizability.} 
\rc{FraudAuditor is designed to detect collusive fraud in healthcare, where patients often visit certain medical institutes together at similar times. By adjusting the co-visit definition, the detection approach can uncover other types of healthcare fraud. For example, in the case where fraudsters work asynchronously, co-visit behaviors can be detected by increasing the parameter of the co-visit time gap or using time alignment methods (e.g., DTW~\cite{muller2007dynamic}); in the case where fraudsters spatially disperse, the location constraint can be relaxed to the same area by considering the geographic information of the medical institutes.}
%For other cunning fraud variants, such as different fraudsters within a group acting at different times in different locations. Our approach is also adaptable to such ``soft'' co-visits, for example by adjusting parameters such as the co-visit time gap and by using time alignment methods such as DTW to cope with different initial time of fraud. In addition, considering the geographical information of medical institutes allows relaxing the same fraudulent location to the same area.

Our approach can also be generalized to other application domains where the frauds share similar characteristics of group and simultaneity, e.g., electronic commerce fraud, spam detection, and telecommunication fraud. Taking spam as an example, spammers always use botnets to send group emails by controlling multiple bot accounts. Similar to the co-visit network in health insurance, a co-sending network between email accounts can be constructed by considering the sending interval and the number of co-sendings. %so it is promising to detect bot accounts from it.

\textbf{Scalability.} %Due to the large volume of data, the computation of the co-visit network could take a couple of seconds. Possible solutions to this problem include: 1) leveraging multi-core CPU and GPU to accelerate the algorithm, 2) introducing a distributed computation scheme by dividing the graph into multiple subgraphs, and 3) integrating more efficient algorithms.
\rc{Due to enormous volumes of raw data, our system faces scalability issues in data processing and data visualization. We allow users to use data filtering to focus on a small set of records. Since the percentage of fraud is small, reducing the data scope through spatio-temporal segmentation and attribute filtering, such as removing visit records from highly regulated public hospitals, is in line with the practical workflow and the principle of overview to detail in visualization~\cite{shneiderman2003eyes}. In the extreme case of a group with many fraudsters, the patient behavior view may encounter visual clutter and rendering bottlenecks, which can be mitigated using data sampling and progressive visualization~\cite{zgraggen2016progressive}.}

% Besides, the presentation of medical behavior in the patient behavior view also has a scalability issue. When the space is limited, the timeline of medical treatment may overlap. \textit{FraudAuditor} supports interactive observation from multiple time granularity. However, the rendering efficiency in a large time period is still not satisfying. In order to display more data, we can speed up the interactive response by pre-loading and pre-drawing.

\textbf{Lessons learned.} %We summarized the lessons learned in developing \textit{FraudAuditor}.
First, multi-level views should be provided for the visualization of complex high-dimensional data to support progressive analysis.
Directly presenting all information to users increases cognitive load.
The health insurance data used in this paper involves associations among multiple subjects and has a large number of attribute dimensions.
We split the analysis tasks and coordinated views into three levels (i.e., overview-level, group-level, and patient-level).
%Multiple coordinated views are designed.
Thus, both the visual analytics approach and the system design follow the overview-to-detial principle.

Second, intuitive and effective visualization helps users learn quickly.
Because users of our system are experts with a background in health insurance audit, they don't know much about visualization.
The charts in \textit{FraudAuditor} are mainly common and popular charts, such as bar charts and node-link diagrams, which help lower the learning cost for users and increase their trust in the system.

\textbf{Future work.} In the future, \rc{to detect other types of fraud, such as doctor-patient collusion}, we plan to provide more detailed contexts of medical records by constructing a dynamic heterogeneous network of patients, doctors, and medical institutions. Another possible direction is to reduce the cost of manually labeling the dataset by leveraging active learning techniques to improve the efficiency of data instance selection. Additionally, the precision of group detection can be further improved by semi-supervised algorithms. \rc{We also plan to add more guidance and annotations to the system to further improve its usability.}
