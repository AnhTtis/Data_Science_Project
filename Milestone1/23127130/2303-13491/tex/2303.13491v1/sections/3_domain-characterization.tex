\section{Domain Characterization}\label{sec:domain}

Through intensive collaboration with health insurance experts, we get access to real-world health insurance data, learn about the patterns of collusive fraud, and summarize a set of design requirements.


\subsection{Data Description}

The data used in this paper are from the local Healthcare Security Administration we collaborate with. Under experts' guidance, we excluded irrelevant fields and anonymized identity information. Two tables hold the processed data:
% The processed data are stored in two tables:

\begin{itemize}[noitemsep,topsep=0pt]
    \item \textbf{Patient visit table}: Rows represent patient visits. Columns include time, patient ID, medical institution, diagnosed disease, and total fee. For example, at 16:23 on August 12, 2021, patient $P_1$ went to a hospital. $P_1$ was diagnosed with hypertension and was reimbursed 36.35 yuan for drugs through health insurance. 
    
    \item \textbf{Drug table}: Doctors prescribe drugs for each patient visit. The name and dosage of each drug are recorded in this table. Note that a prescription can include multiple drugs. For example, the doctor prescribed $P_1$ antihypertensive drugs, consisting of perindopril and enalapril.
\end{itemize}

\subsection{Problem Specification}
\label{sec:pro}
Over the past year, we have worked closely with two health insurance audit experts with four years of work experience. Through multiple interviews with them, we learned about the health insurance system, examined existing fraud cases, and summarized patterns of collusive fraud behaviors.

In our scenario, a part of the medical expense can be reimbursed when insured patients pay for their drugs. Fraud occurs when insured patients sell the drugs instead of taking them after reimbursement. 
Seeking to cash out quickly, fraudsters need to gather a large number of drugs. Thus, fraudsters always collude with each other to purchase sufficient drugs within a short period. To avoid scrutiny, they prefer to visit clinics or pharmacies in poorly regulated rural or community areas. 
However, \rc{the behavior of fraudulent groups can be confused with the normal visit behavior of patient groups with chronic diseases.} For example, if the doctors treating the chronic disease see patients only at specific times, there is a high probability that certain patients will go to the same location at a similar time.
\rc{Therefore, normal patients can also have the characteristics of \textbf{spatio-temporal connections} and \textbf{group actions} similar to the collusive fraud group, i.e., patients may visit certain medical institutes together on similar cycles.}
To avoid misjudgment, \rc{it is indispensable for experts to review and validate suspicious groups by referring to the contextual information of patients' visit behavior.}

To better understand the needs of audit experts in detecting and analyzing collusive fraud, we interviewed them and summarized the current audit process into three steps.
% we interviewed experts about the current audit process and summarized it into three stages. 

1. Experts use the audit system (a graphical interface of the database) to learn the overall characteristics of the health insurance data, such as the cost of claims. Then, they narrow the scope of the investigation by filtering out patients of interest, such as those whose costs exceed 10,000 yuan. Since fraudsters may have multiple tricks, experts require repeated attempts to avoid omission. 
Then, experts leverage manual rules to filter patients to further identify suspicious groups. For example, experts can identify patients with an unusually high number of drug purchases over a period of time or sequentially separate out groups with a high overlap of visit locations and similar visit times \rc{(e.g., a group of five patients frequently visit a specific drugstore within 1 hour to purchase medications)}. 

2. Experts browse the list of suspicious groups to 
start with the groups with serious hazards potentially and seize the opportunity to stop loss in time. The potential hazard of a group can be estimated according to the group size and the total claim expense. Besides, experts may also merge groups with similar characteristics during browsing to improve analysis efficiency.


3. Experts judge whether a group is a collusive fraud or a false positive by examining details of patients' visit behaviors (diseases, drugs, hospitals, etc.). Highly suspicious groups will be further investigated (e.g., surveillance video inspections and phone/on-site interviews with the patients).


\subsection{Requirement Analysis}

As mentioned above, the audit process requires a large number of manual inspections, which are extremely time-consuming and labor-intensive.
Based on discussions with experts, we summarized requirements at three levels.

Users need to learn about the patients statistics and their behavior connections from the \textbf{overview} level.

\begin{compactitem}
    \item[\textbf{R1}]
    \textbf{Show attribute distribution of medical records.} An overview allows users to understand the dataset and find a starting point to detect fraud. For example, users can learn a reasonable expense range from the distribution of patient expenses. Then, the patients whose expenses exceed the threshold should be reviewed.
    
    \item[\textbf{R2}]
    \textbf{Allow flexible data filtering.} Our dataset includes many patients' visit records. Some of them do not need to be audited because of small expenses or limited numbers of visits. Filtering patients by appropriate user-specified conditions can improve analysis efficiency. 
    
    \item[\textbf{R3}]
    \textbf{Identify the behavioral connections among patients.} Connections of visiting behaviors and drug purchasing behaviors are the basis for detecting fraud groups. Thus, these connections should be identified according to expert knowledge, namely, user-specified restrictions. For instance, patients are considered to be potentially associated only if they visit the same location in less than 15 minutes more than five times.
    
    
\end{compactitem}

Further fraud detection processes should be implemented at the \textbf{group} level.

\begin{compactitem}
    \item[\textbf{R4}]
    \textbf{Detect patient groups.} Patient groups can be detected based on various user-specified rules (e.g., whether there exist specific behavior connections or whether the total expense exceeds a limit). Automation can be leveraged to guarantee the efficiency of group detection. 


    \item[\textbf{R5}]
    \textbf{Support suspicious group selection.}
    Given a large number of detected candidate groups, users need to quickly locate those with the most suspicion or hazards. Recommending groups using effective ranking approaches can accelerate group selection.
    
    

\end{compactitem}

Finally, users need to check group details from the \textbf{patient} level and find evidence of fraud.

\begin{compactitem}
    \item[\textbf{R6}]
    \textbf{Support suspicious patients verification.} Understanding the intra-group similarities of patient behaviors, such as prescribed diseases, drug purchases, and selection of medical institutions, can help users exclude irrelevant patients and examine suspicious patients.
    
    \item[\textbf{R7}]
    \textbf{Visualize the visit records of an individual patient.} As mentioned earlier, auto-detection can hardly differentiate fraud groups from patients with specific visit needs, which leads to false positives. Users should examine the identified suspicious fraudsters to prove or disprove their suspicions. Visualizing patients' histories of medical visits could help users gather evidence regarding the continuity and rationality of the visits. In this way, fraudulent groups and false positive groups can be differentiated based on expert knowledge.

\end{compactitem}