
%% bare_jrnl_compsoc.tex
%% V1.4b
%% 2015/08/26
%% by Michael Shell
%% See:
%% http://www.michaelshell.org/
%% for current contact information.
%%
%% This is a skeleton file demonstrating the use of IEEEtran.cls
%% (requires IEEEtran.cls version 1.8b or later) with an IEEE
%% Computer Society journal paper.
%%
%% Support sites:
%% http://www.michaelshell.org/tex/ieeetran/
%% http://www.ctan.org/pkg/ieeetran
%% and
%% http://www.ieee.org/

%%*************************************************************************
%% Legal Notice:
%% This code is offered as-is without any warranty either expressed or
%% implied; without even the implied warranty of MERCHANTABILITY or
%% FITNESS FOR A PARTICULAR PURPOSE! 
%% User assumes all risk.
%% In no event shall the IEEE or any contributor to this code be liable for
%% any damages or losses, including, but not limited to, incidental,
%% consequential, or any other damages, resulting from the use or misuse
%% of any information contained here.
%%
%% All comments are the opinions of their respective authors and are not
%% necessarily endorsed by the IEEE.
%%
%% This work is distributed under the LaTeX Project Public License (LPPL)
%% ( http://www.latex-project.org/ ) version 1.3, and may be freely used,
%% distributed and modified. A copy of the LPPL, version 1.3, is included
%% in the base LaTeX documentation of all distributions of LaTeX released
%% 2003/12/01 or later.
%% Retain all contribution notices and credits.
%% ** Modified files should be clearly indicated as such, including  **
%% ** renaming them and changing author support contact information. **
%%*************************************************************************


% *** Authors should verify (and, if needed, correct) their LaTeX system  ***
% *** with the testflow diagnostic prior to trusting their LaTeX platform ***
% *** with production work. The IEEE's font choices and paper sizes can   ***
% *** trigger bugs that do not appear when using other class files.       ***                          ***
% The testflow support page is at:
% http://www.michaelshell.org/tex/testflow/


\documentclass[10pt,journal,compsoc]{IEEEtran}
%
% If IEEEtran.cls has not been installed into the LaTeX system files,
% manually specify the path to it like:
% \documentclass[10pt,journal,compsoc]{../sty/IEEEtran}





% Some very useful LaTeX packages include:
% (uncomment the ones you want to load)

% 自己添加的包

% 注释
\usepackage{comment}

% 支持表格多行合并
\usepackage{multirow}

% 需求列表
\usepackage{paralist} 

% 图片位置
\usepackage{float}

% 算法
\usepackage{algorithm}
\usepackage{algpseudocode}
\renewcommand{\algorithmicrequire}{\textbf{Input:}}  % Use Input in the format of Algorithm
\renewcommand{\algorithmicensure}{\textbf{Output:}} % Use Output in the format of Algorithm

% 公式文字
\usepackage{amsmath}

% 链接
\usepackage{hyperref}

% url
\usepackage{url}

% 枚举
\usepackage{enumitem}

% 表格
\usepackage{booktabs} 

% 颜色
\usepackage{xcolor}

% 指示函数
\usepackage{bbm}

% *** MISC UTILITY PACKAGES ***
%
%\usepackage{ifpdf}
% Heiko Oberdiek's ifpdf.sty is very useful if you need conditional
% compilation based on whether the output is pdf or dvi.
% usage:
% \ifpdf
%   % pdf code
% \else
%   % dvi code
% \fi
% The latest version of ifpdf.sty can be obtained from:
% http://www.ctan.org/pkg/ifpdf
% Also, note that IEEEtran.cls V1.7 and later provides a builtin
% \ifCLASSINFOpdf conditional that works the same way.
% When switching from latex to pdflatex and vice-versa, the compiler may
% have to be run twice to clear warning/error messages.






% *** CITATION PACKAGES ***
%
\ifCLASSOPTIONcompsoc
  % IEEE Computer Society needs nocompress option
  % requires cite.sty v4.0 or later (November 2003)
  \usepackage[nocompress]{cite}
\else
  % normal IEEE
  \usepackage{cite}
\fi
% cite.sty was written by Donald Arseneau
% V1.6 and later of IEEEtran pre-defines the format of the cite.sty package
% \cite{} output to follow that of the IEEE. Loading the cite package will
% result in citation numbers being automatically sorted and properly
% "compressed/ranged". e.g., [1], [9], [2], [7], [5], [6] without using
% cite.sty will become [1], [2], [5]--[7], [9] using cite.sty. cite.sty's
% \cite will automatically add leading space, if needed. Use cite.sty's
% noadjust option (cite.sty V3.8 and later) if you want to turn this off
% such as if a citation ever needs to be enclosed in parenthesis.
% cite.sty is already installed on most LaTeX systems. Be sure and use
% version 5.0 (2009-03-20) and later if using hyperref.sty.
% The latest version can be obtained at:
% http://www.ctan.org/pkg/cite
% The documentation is contained in the cite.sty file itself.
%
% Note that some packages require special options to format as the Computer
% Society requires. In particular, Computer Society  papers do not use
% compressed citation ranges as is done in typical IEEE papers
% (e.g., [1]-[4]). Instead, they list every citation separately in order
% (e.g., [1], [2], [3], [4]). To get the latter we need to load the cite
% package with the nocompress option which is supported by cite.sty v4.0
% and later. Note also the use of a CLASSOPTION conditional provided by
% IEEEtran.cls V1.7 and later.





% *** GRAPHICS RELATED PACKAGES ***
%
\ifCLASSINFOpdf
  \usepackage[pdftex]{graphicx}
  % declare the path(s) where your graphic files are
  \graphicspath{{../pdf/}{../jpeg/}}
  % and their extensions so you won't have to specify these with
  % every instance of \includegraphics
  \DeclareGraphicsExtensions{.pdf,.jpeg,.png}
\else
  % or other class option (dvipsone, dvipdf, if not using dvips). graphicx
  % will default to the driver specified in the system graphics.cfg if no
  % driver is specified.
  \usepackage[dvips]{graphicx}
  % declare the path(s) where your graphic files are
  \graphicspath{{../eps/}}
  % and their extensions so you won't have to specify these with
  % every instance of \includegraphics
  \DeclareGraphicsExtensions{.eps}
\fi
% graphicx was written by David Carlisle and Sebastian Rahtz. It is
% required if you want graphics, photos, etc. graphicx.sty is already
% installed on most LaTeX systems. The latest version and documentation
% can be obtained at: 
% http://www.ctan.org/pkg/graphicx
% Another good source of documentation is "Using Imported Graphics in
% LaTeX2e" by Keith Reckdahl which can be found at:
% http://www.ctan.org/pkg/epslatex
%
% latex, and pdflatex in dvi mode, support graphics in encapsulated
% postscript (.eps) format. pdflatex in pdf mode supports graphics
% in .pdf, .jpeg, .png and .mps (metapost) formats. Users should ensure
% that all non-photo figures use a vector format (.eps, .pdf, .mps) and
% not a bitmapped formats (.jpeg, .png). The IEEE frowns on bitmapped formats
% which can result in "jaggedy"/blurry rendering of lines and letters as
% well as large increases in file sizes.
%
% You can find documentation about the pdfTeX application at:
% http://www.tug.org/applications/pdftex






% *** MATH PACKAGES ***
%
%\usepackage{amsmath}
% A popular package from the American Mathematical Society that provides
% many useful and powerful commands for dealing with mathematics.
%
% Note that the amsmath package sets \interdisplaylinepenalty to 10000
% thus preventing page breaks from occurring within multiline equations. Use:
%\interdisplaylinepenalty=2500
% after loading amsmath to restore such page breaks as IEEEtran.cls normally
% does. amsmath.sty is already installed on most LaTeX systems. The latest
% version and documentation can be obtained at:
% http://www.ctan.org/pkg/amsmath





% *** SPECIALIZED LIST PACKAGES ***
%
%\usepackage{algorithmic}
% algorithmic.sty was written by Peter Williams and Rogerio Brito.
% This package provides an algorithmic environment fo describing algorithms.
% You can use the algorithmic environment in-text or within a figure
% environment to provide for a floating algorithm. Do NOT use the algorithm
% floating environment provided by algorithm.sty (by the same authors) or
% algorithm2e.sty (by Christophe Fiorio) as the IEEE does not use dedicated
% algorithm float types and packages that provide these will not provide
% correct IEEE style captions. The latest version and documentation of
% algorithmic.sty can be obtained at:
% http://www.ctan.org/pkg/algorithms
% Also of interest may be the (relatively newer and more customizable)
% algorithmicx.sty package by Szasz Janos:
% http://www.ctan.org/pkg/algorithmicx




% *** ALIGNMENT PACKAGES ***
%
%\usepackage{array}
% Frank Mittelbach's and David Carlisle's array.sty patches and improves
% the standard LaTeX2e array and tabular environments to provide better
% appearance and additional user controls. As the default LaTeX2e table
% generation code is lacking to the point of almost being broken with
% respect to the quality of the end results, all users are strongly
% advised to use an enhanced (at the very least that provided by array.sty)
% set of table tools. array.sty is already installed on most systems. The
% latest version and documentation can be obtained at:
% http://www.ctan.org/pkg/array


% IEEEtran contains the IEEEeqnarray family of commands that can be used to
% generate multiline equations as well as matrices, tables, etc., of high
% quality.




% *** SUBFIGURE PACKAGES ***
%\ifCLASSOPTIONcompsoc
%  \usepackage[caption=false,font=footnotesize,labelfont=sf,textfont=sf]{subfig}
%\else
%  \usepackage[caption=false,font=footnotesize]{subfig}
%\fi
% subfig.sty, written by Steven Douglas Cochran, is the modern replacement
% for subfigure.sty, the latter of which is no longer maintained and is
% incompatible with some LaTeX packages including fixltx2e. However,
% subfig.sty requires and automatically loads Axel Sommerfeldt's caption.sty
% which will override IEEEtran.cls' handling of captions and this will result
% in non-IEEE style figure/table captions. To prevent this problem, be sure
% and invoke subfig.sty's "caption=false" package option (available since
% subfig.sty version 1.3, 2005/06/28) as this is will preserve IEEEtran.cls
% handling of captions.
% Note that the Computer Society format requires a sans serif font rather
% than the serif font used in traditional IEEE formatting and thus the need
% to invoke different subfig.sty package options depending on whether
% compsoc mode has been enabled.
%
% The latest version and documentation of subfig.sty can be obtained at:
% http://www.ctan.org/pkg/subfig




% *** FLOAT PACKAGES ***
%
%\usepackage{fixltx2e}
% fixltx2e, the successor to the earlier fix2col.sty, was written by
% Frank Mittelbach and David Carlisle. This package corrects a few problems
% in the LaTeX2e kernel, the most notable of which is that in current
% LaTeX2e releases, the ordering of single and double column floats is not
% guaranteed to be preserved. Thus, an unpatched LaTeX2e can allow a
% single column figure to be placed prior to an earlier double column
% figure.
% Be aware that LaTeX2e kernels dated 2015 and later have fixltx2e.sty's
% corrections already built into the system in which case a warning will
% be issued if an attempt is made to load fixltx2e.sty as it is no longer
% needed.
% The latest version and documentation can be found at:
% http://www.ctan.org/pkg/fixltx2e


%\usepackage{stfloats}
% stfloats.sty was written by Sigitas Tolusis. This package gives LaTeX2e
% the ability to do double column floats at the bottom of the page as well
% as the top. (e.g., "\begin{figure*}[!b]" is not normally possible in
% LaTeX2e). It also provides a command:
%\fnbelowfloat
% to enable the placement of footnotes below bottom floats (the standard
% LaTeX2e kernel puts them above bottom floats). This is an invasive package
% which rewrites many portions of the LaTeX2e float routines. It may not work
% with other packages that modify the LaTeX2e float routines. The latest
% version and documentation can be obtained at:
% http://www.ctan.org/pkg/stfloats
% Do not use the stfloats baselinefloat ability as the IEEE does not allow
% \baselineskip to stretch. Authors submitting work to the IEEE should note
% that the IEEE rarely uses double column equations and that authors should try
% to avoid such use. Do not be tempted to use the cuted.sty or midfloat.sty
% packages (also by Sigitas Tolusis) as the IEEE does not format its papers in
% such ways.
% Do not attempt to use stfloats with fixltx2e as they are incompatible.
% Instead, use Morten Hogholm'a dblfloatfix which combines the features
% of both fixltx2e and stfloats:
%
% \usepackage{dblfloatfix}
% The latest version can be found at:
% http://www.ctan.org/pkg/dblfloatfix




%\ifCLASSOPTIONcaptionsoff
%  \usepackage[nomarkers]{endfloat}
% \let\MYoriglatexcaption\caption
% \renewcommand{\caption}[2][\relax]{\MYoriglatexcaption[#2]{#2}}
%\fi
% endfloat.sty was written by James Darrell McCauley, Jeff Goldberg and 
% Axel Sommerfeldt. This package may be useful when used in conjunction with 
% IEEEtran.cls'  captionsoff option. Some IEEE journals/societies require that
% submissions have lists of figures/tables at the end of the paper and that
% figures/tables without any captions are placed on a page by themselves at
% the end of the document. If needed, the draftcls IEEEtran class option or
% \CLASSINPUTbaselinestretch interface can be used to increase the line
% spacing as well. Be sure and use the nomarkers option of endfloat to
% prevent endfloat from "marking" where the figures would have been placed
% in the text. The two hack lines of code above are a slight modification of
% that suggested by in the endfloat docs (section 8.4.1) to ensure that
% the full captions always appear in the list of figures/tables - even if
% the user used the short optional argument of \caption[]{}.
% IEEE papers do not typically make use of \caption[]'s optional argument,
% so this should not be an issue. A similar trick can be used to disable
% captions of packages such as subfig.sty that lack options to turn off
% the subcaptions:
% For subfig.sty:
% \let\MYorigsubfloat\subfloat
% \renewcommand{\subfloat}[2][\relax]{\MYorigsubfloat[]{#2}}
% However, the above trick will not work if both optional arguments of
% the \subfloat command are used. Furthermore, there needs to be a
% description of each subfigure *somewhere* and endfloat does not add
% subfigure captions to its list of figures. Thus, the best approach is to
% avoid the use of subfigure captions (many IEEE journals avoid them anyway)
% and instead reference/explain all the subfigures within the main caption.
% The latest version of endfloat.sty and its documentation can obtained at:
% http://www.ctan.org/pkg/endfloat
%
% The IEEEtran \ifCLASSOPTIONcaptionsoff conditional can also be used
% later in the document, say, to conditionally put the References on a 
% page by themselves.




% *** PDF, URL AND HYPERLINK PACKAGES ***
%
%\usepackage{url}
% url.sty was written by Donald Arseneau. It provides better support for
% handling and breaking URLs. url.sty is already installed on most LaTeX
% systems. The latest version and documentation can be obtained at:
% http://www.ctan.org/pkg/url
% Basically, \url{my_url_here}.





% *** Do not adjust lengths that control margins, column widths, etc. ***
% *** Do not use packages that alter fonts (such as pslatex).         ***
% There should be no need to do such things with IEEEtran.cls V1.6 and later.
% (Unless specifically asked to do so by the journal or conference you plan
% to submit to, of course. )


% correct bad hyphenation here
\hyphenation{op-tical net-works semi-conduc-tor}

% rc: 较大的改动内容
% \newcommand{\rc}[1]{\textcolor{red}{#1}}
\newcommand{\rc}[1]{\textcolor{black}{#1}}

\begin{document}
%
% paper title
% Titles are generally capitalized except for words such as a, an, and, as,
% at, but, by, for, in, nor, of, on, or, the, to and up, which are usually
% not capitalized unless they are the first or last word of the title.
% Linebreaks \\ can be used within to get better formatting as desired.
% Do not put math or special symbols in the title.
\title{FraudAuditor: A Visual Analytics Approach for Collusive Fraud in Health Insurance}
%
%
% author names and IEEE memberships
% note positions of commas and nonbreaking spaces ( ~ ) LaTeX will not break
% a structure at a ~ so this keeps an author's name from being broken across
% two lines.
% use \thanks{} to gain access to the first footnote area
% a separate \thanks must be used for each paragraph as LaTeX2e's \thanks
% was not built to handle multiple paragraphs
%
%
%\IEEEcompsocitemizethanks is a special \thanks that produces the bulleted
% lists the Computer Society journals use for "first footnote" author
% affiliations. Use \IEEEcompsocthanksitem which works much like \item
% for each affiliation group. When not in compsoc mode,
% \IEEEcompsocitemizethanks becomes like \thanks and
% \IEEEcompsocthanksitem becomes a line break with idention. This
% facilitates dual compilation, although admittedly the differences in the
% desired content of \author between the different types of papers makes a
% one-size-fits-all approach a daunting prospect. For instance, compsoc 
% journal papers have the author affiliations above the "Manuscript
% received ..."  text while in non-compsoc journals this is reversed. Sigh.

\author{Jiehui~Zhou,
        Xumeng~Wang,
        Jie~Wang,
        Hui~Ye,
        Huanliang~Wang,
        Zihan~Zhou,
        Dongming~Han,
        Haochao~Ying,
        Jian~Wu,
        and~Wei~Chen% <-this % stops a space
\IEEEcompsocitemizethanks{\IEEEcompsocthanksitem J. Zhou, H. Wang, Z. Zhou, D. Han, and W. Chen are with the State Key Lab of CAD\&CG, Zhejiang University.
% note need leading \protect in front of \\ to get a newline within \thanks as
% \\ is fragile and will error, could use \hfil\break instead.
E-mail: \{zhoujiehui, wanghuanliang, zhouzihan, dongminghan, chenvis\}@zju.edu.cn.
\IEEEcompsocthanksitem X. Wang is with TMCC, CS, Nankai University.
E-mail: wangxumeng@nankai.edu.cn.
\IEEEcompsocthanksitem J. Wang is with Alibaba Group, Hangzhou.
E-mail: siwei.wj@alibaba-inc.com.
\IEEEcompsocthanksitem H. Ye is with Tencent, Shenzhen.
E-mail: hazelye@tencent.com.
\IEEEcompsocthanksitem H. Ying is with the School of Public Health, Zhejiang University. He is also with the Key Laboratory of Intelligent Preventive Medicine of Zhejiang Province.
E-mail: haochaoying@zju.edu.cn.
\IEEEcompsocthanksitem J. Wu is with Second Affiliated Hospital School of Medicine, School of Public Health, and Institute of Wenzhou, Zhejiang University.
E-mail: wujian2000@zju.edu.cn.
\IEEEcompsocthanksitem Wei Chen and Haochao Ying are the corresponding authors.
}% <-this % stops an unwanted space
\thanks{Manuscript received xx xx, 20xx; revised xx xx, 20xx.}}

% note the % following the last \IEEEmembership and also \thanks - 
% these prevent an unwanted space from occurring between the last author name
% and the end of the author line. i.e., if you had this:
% 
% \author{....lastname \thanks{...} \thanks{...} }
%                     ^------------^------------^----Do not want these spaces!
%
% a space would be appended to the last name and could cause every name on that
% line to be shifted left slightly. This is one of those "LaTeX things". For
% instance, "\textbf{A} \textbf{B}" will typeset as "A B" not "AB". To get
% "AB" then you have to do: "\textbf{A}\textbf{B}"
% \thanks is no different in this regard, so shield the last } of each \thanks
% that ends a line with a % and do not let a space in before the next \thanks.
% Spaces after \IEEEmembership other than the last one are OK (and needed) as
% you are supposed to have spaces between the names. For what it is worth,
% this is a minor point as most people would not even notice if the said evil
% space somehow managed to creep in.



% The paper headers
\markboth{Journal of \LaTeX\ Class Files,~Vol.~xx, No.~x, xx~20xx}%
{Shell \MakeLowercase{\textit{et al.}}: Bare Demo of IEEEtran.cls for Computer Society Journals}
% The only time the second header will appear is for the odd numbered pages
% after the title page when using the twoside option.
% 
% *** Note that you probably will NOT want to include the author's ***
% *** name in the headers of peer review papers.                   ***
% You can use \ifCLASSOPTIONpeerreview for conditional compilation here if
% you desire.



% The publisher's ID mark at the bottom of the page is less important with
% Computer Society journal papers as those publications place the marks
% outside of the main text columns and, therefore, unlike regular IEEE
% journals, the available text space is not reduced by their presence.
% If you want to put a publisher's ID mark on the page you can do it like
% this:
%\IEEEpubid{0000--0000/00\$00.00~\copyright~2015 IEEE}
% or like this to get the Computer Society new two part style.
%\IEEEpubid{\makebox[\columnwidth]{\hfill 0000--0000/00/\$00.00~\copyright~2015 IEEE}%
%\hspace{\columnsep}\makebox[\columnwidth]{Published by the IEEE Computer Society\hfill}}
% Remember, if you use this you must call \IEEEpubidadjcol in the second
% column for its text to clear the IEEEpubid mark (Computer Society jorunal
% papers don't need this extra clearance.)



% use for special paper notices
%\IEEEspecialpapernotice{(Invited Paper)}



% for Computer Society papers, we must declare the abstract and index terms
% PRIOR to the title within the \IEEEtitleabstractindextext IEEEtran
% command as these need to go into the title area created by \maketitle.
% As a general rule, do not put math, special symbols or citations
% in the abstract or keywords.
\IEEEtitleabstractindextext{%
\begin{abstract}
Collusive fraud, in which multiple fraudsters collude to defraud health insurance funds, threatens the operation of the healthcare system. 
However, existing statistical and machine learning-based methods have limited ability to detect fraud in the scenario of health insurance due to the high similarity of fraudulent behaviors to normal medical visits and the lack of labeled data.
To ensure the accuracy of the detection results, expert knowledge needs to be integrated with the fraud detection process. By working closely with health insurance audit experts, we propose \textit{FraudAuditor}, a three-stage visual analytics approach to collusive fraud detection in health insurance. Specifically, we first allow users to interactively construct a co-visit network to holistically model the visit relationships of different patients. Second, an improved community detection algorithm that considers the strength of fraud likelihood is designed to detect suspicious fraudulent groups. Finally, through our visual interface, users can compare, investigate, and verify suspicious patient behavior with tailored visualizations that support different time scales. We conducted case studies in a real-world healthcare scenario, i.e., to help locate the actual fraud group and exclude the false positive group. The results and expert feedback proved the effectiveness and usability of the approach.
\end{abstract}

% Note that keywords are not normally used for peerreview papers.
\begin{IEEEkeywords}
Visual analytics, collusive fraud, fraud detection, health insurance.
%Computer Society, IEEE, IEEEtran, journal, \LaTeX, paper, template.
\end{IEEEkeywords}}


% make the title area
\maketitle


% To allow for easy dual compilation without having to reenter the
% abstract/keywords data, the \IEEEtitleabstractindextext text will
% not be used in maketitle, but will appear (i.e., to be "transported")
% here as \IEEEdisplaynontitleabstractindextext when the compsoc 
% or transmag modes are not selected <OR> if conference mode is selected 
% - because all conference papers position the abstract like regular
% papers do.
\IEEEdisplaynontitleabstractindextext
% \IEEEdisplaynontitleabstractindextext has no effect when using
% compsoc or transmag under a non-conference mode.



% For peer review papers, you can put extra information on the cover
% page as needed:
% \ifCLASSOPTIONpeerreview
% \begin{center} \bfseries EDICS Category: 3-BBND \end{center}
% \fi
%
% For peerreview papers, this IEEEtran command inserts a page break and
% creates the second title. It will be ignored for other modes.
\IEEEpeerreviewmaketitle



% \begin{figure}[t]
%     % \begin{subfigure}{1\linewidth}
%     %   \centering
%     % %   \includegraphics[width=1\linewidth]{figs/fig_1_moti_textattn.pdf}  
%     % %   \includegraphics[width=1\linewidth]{figs/fig_1_moti_textattn_v2.pdf}  
%     %   \includegraphics[width=1\linewidth]{figs/fig_1_moti_textattn_v5.pdf}  
%     %   \vspace{-0.5cm}
%     %     \caption{Amount of attention added to each video clip from the source video and query text in the self-attention layers of Moment-DETR encoder.}
%     %     % \caption{Distribution of attention for source and query in Moment-DETR encoder}
%     %     % Visualization of video clip's self-attention score in Moment-DETR encoder.
%     %   \label{fig:fig1_text_attn_ex}
%     % \end{subfigure}%\hfill% or  or \hspace{0.3\textwidth}
%     \vspace{0.2cm}
%     % \begin{subfigure}{1\linewidth}
%       \centering
%     %   \includegraphics[width=1\linewidth]{figs/fig1_moti_negattn.pdf}  
%       \includegraphics[width=1\linewidth]{figs/fig1_moti_negattn_v3.pdf}  
%       \vspace{-0.4cm}
%     %   \caption{Correspondence of saliency scores on the relevance between video clips and the text query.}
%     % \caption{Predicted saliency scores against the video relevant positive query and video irrelevant negative query}
%       \label{fig:fig1_neg_attn_ex}
%     % \end{subfigure}%\hfill% or  or \hspace{0.3\textwidth}
%     \caption{
%     % 원준 원본
%     % (a) Comparison between attention scores of source and query for each video clip~(We sum the attention scores from video and text). 
%     % We observe that the attention scores are dominated by other clips in the source video. 
%     % Text queries do not account for much attention regardless of the relevance to the video clips.
%     % \textbf{(a)} Inspection of the query dependency in Moment-DETR encoder.
%     % % We visualize the attention score of video tokens in the transformer encoder and observe that text query accounts for only a low portion of attention.
%     % % This tendency occurs regardless of the relevance between the text query and video clips. 
%     % We visualize the attention score of video tokens in the transformer encoder and observe 1) text query only accounts for a low portion of attention, and 2) relevance between video-query pair does not affect the attention scores ratio of text.
%     \textbf{(b)} Comparison of highlight-ness when relevant and non-relevant queries are input.
%     As observed in , existing work only uses queries to play an insignificant role, thereby may not be capable of detecting false queries and considering the video-query relevance even when the problem in (a) is resolved. 
%     % \SE{} % 이 부분이 "not capable of" 란 용어가 세다는 피드백이 있는 듯 합니다. 이러한 능력이 없다는 것은 굉장히 강한 어조인거 같기는 하고, 이러한 경우들이 종종 있다거나 좀 약화시킬 필요가 있어보이긴 하네요.
%     On the other hand, our QD-DETR yields a query-dependent representation that the relevance between the source video and query text is updated in the saliency scores.
%     There is a large gap between positive and negative saliency scores, and scores are consistent since the clips are all highly correlated to others.
%     }
%     \label{fig:motivation_ex}
%     % \captionsetup{belowskip=13pt}
%     % \setlength{\belowcaptionskip}{-10pt}
% \end{figure}
\begin{figure}
    \centering
    \includegraphics[width=1\linewidth]{figs/fig1_moti_negattn_1111.pdf}
    % \includegraphics[width=1\linewidth]{figs/fig1_moti_negattn_1109.pdf}
    % \includegraphics[width=1\linewidth]{figs/fig1_moti_negattn_stat.pdf}
    \vspace{-0.6cm}
    \caption{
        % \SE{} % 수정 필요
        Comparison of highlight-ness~(saliency score) when relevant and non-relevant queries are given.
        We found that the existing work only uses queries to play an insignificant role, thereby may not be capable of detecting negative queries and video-query relevance; saliency scores for clips in ground-truth~(GT) moments are low and equivalent for positive and negative queries.
        % This also results in mispredicted moments when ground-truth~(GT) moment is dominated by clips unrelated to GT since their prediction is highly focused on the video.
        % \SE{} % 여기 한번 더 보면 좋을 듯 합니다. GT moment에 unrelated한 clip이 많으면? label이 틀렷을 경우를 말씀하시는건지?
        % As observed in saliency graph, existing work only uses queries to play an insignificant role, thereby may not be capable of detecting false queries and considering the video-query relevance.
        On the other hand, query-dependent representations of QD-DETR result in corresponding saliency scores to the video-query relevance and precisely localized moments.
        % On the other hand, our QD-DETR yields a query-dependent representation that the
        % saliency scores are in accordance with the relevance between the video and query.
        % text is in accordance with the saliency scores.
        % There is a large gap between positive and negative saliency scores, and scores are consistent since the clips are all highly correlated to others.
}
    \label{fig:motivation_ex}
\end{figure}


\section{Introduction}
% 원준 원본
% Along with the advance of digital devices and platforms, video is now one of the most desired data type for consumers. However, although the large information capacity of videos may be beneficial in many aspects, e.g., informative and entertaining, on the contrary perspective, videos are time-consuming, and hard to search for desirable moments. 
% This has led many creators to use extra manpower to crop and edit the video to generate highlight clips to gain the consumer’s attention.
Along with the advance of digital devices and platforms, video is now one of the most desired data types for consumers~\cite{apostolidis2021video,wu2017deep}.
% SE: Video aware deep learning application & survey papers?
Although the large information capacity of videos might be beneficial in many aspects, e.g., informative and entertaining, inspecting the videos is time-consuming, so that it is hard to capture the desired moments~\cite{anne2017localizing,apostolidis2021video}. 
% This has led many creators to use extra manpower to crop and edit the video to generate highlight clips to gain the consumer’s attention.


% On the other side, 
Indeed, the need to retrieve user-requested or highlight moments within videos is greatly raised.
Numerous research efforts were put into the search for the requested moments in the video~\cite{anne2017localizing, gao2017tall, liu2015multi, escorcia2019temporal} and summarizing the video highlights~\cite{zhang2016video, mahasseni2017unsupervised, badamdorj2022contrastive, wei2022learning}.
% Numerous research efforts were put into the search for the requested moments in the video~\cite{anne2017localizing, gao2017tall, liu2015multi, escorcia2019temporal}, summarizing the video to generate highlights was another popular topic~\cite{zhang2016video, mahasseni2017unsupervised, badamdorj2022contrastive, wei2022learning}.
Recently, Moment-DETR~\cite{momentdetr} further spotlighted the topic by proposing a QVHighlights dataset that enables the model to perform both tasks, retrieving the moments with their highlight-ness, simultaneously.

% 원준 원본
% To detect the desired moments, previous works employed transformer encoder-decoder architectural designs to fuse the text query into the video representations. Moment-DETR~\cite{mDETR} modified detection transformer to process capture the moment as a set, and UMT~\cite{umt} implemented transformer decoder as to output clip-wise saliency. 
% Yet to their outstanding breakthroughs in the literature of moment retrieval with the seminal architectures, their limitation is that the role of the given text query is insignificant in representing the query-conditioned video representation; the attention mechanism of moment DETR is not explicitly conditioned on the text query, and the text query is conditioned on multi-modal clips where the differences between the clips are smoothed after encoding process in UMT.



% \begin{figure}[t]
% \centering
%     \begin{subfigure}[l]{0.37\linewidth}
%       \centering
%       \vspace{0.20cm}
%     %   \includegraphics[width=1\linewidth]{figs/fig_1_moti_textattn.pdf}  
%     %   \includegraphics[width=1\linewidth]{figs/fig_1_moti_textattn_v2.pdf}  
%       \includegraphics[width=1\linewidth]{figs/fig1_moti_violin_a.pdf}  
%       \vspace{-0.60cm}
%     %   \caption{text attention}
%         \caption{Importance of queries in video representation}
%       \label{fig:fig1_text_attn}
%     \end{subfigure}%\hfill% or  or \hspace{0.3\textwidth}
%     \vspace{0.2cm}
%     \begin{subfigure}[r]{0.61\linewidth}
%       \centering
%     %   \includegraphics[width=1\linewidth]{figs/fig1_moti_negattn.pdf}  
%       \includegraphics[width=1\linewidth]{figs/fig1_moti_violin_b.pdf}  
%     %   \caption{neg attention}
%         % \caption{Relation between the highlight-ness and the relevance between videos and query texts.}
%         \caption{Highlight-ness~(saliency) histogram of positive and negative video-query pairs\SE{}}
%       \label{fig:fig1_neg_attn}
%     \end{subfigure}%\hfill% or  or \hspace{0.3\textwidth}
%     % \vspace{-0.2cm}
%     \caption{Overall statistics for attention scores in Fig.~\ref{fig:motivation_ex} in QVHighlights dataset. 
%     (a) For the attention scores that measure how much the text query is generally involved in video representation, we use violin plots to show the probability density. We plot the score for each layer in the encoder.
%     % (b) Using the histogram, we compare how the baseline and QD-DETR yield different salient scores given the positive and negative video-text pairs.
%     (b) Saliency histogram shows the distributional gap between positive and negative video-text query pairs of baseline~(Moment-DETR) and proposed QD-DETR.\SE{}
%     }
%     \label{fig:motivation}
%     % \captionsetup{belowskip=13pt}
%     % \setlength{\belowcaptionskip}{-10pt}
% \end{figure}

% \begin{figure}[t]
% \centering

%     \begin{subfigure}[r]{1\linewidth}
%       \centering
%       \hspace{-0.2cm}
%     %   \includegraphics[width=1\linewidth]{figs/fig1_moti_negattn.pdf}  
%       \includegraphics[width=1.1\linewidth]{figs/fig1_moti_violin_a_v2.pdf}  
%     %   \caption{neg attention}
%         % \caption{Relation between the highlight-ness and the relevance between videos and query texts.}
%         \vspace{-0.5cm}
%         % \caption{Saliency histogram of positive and negative video-query pairs}
%         \caption{We plot the histograms and its average value~(dotted line) to compare saliency scores when true and false text queries are given for each method. (left) Since the video representations do not include much textual information, both the true and false queries yield similar saliency scores. (Middle) Even when the video representation is enforced to be updated with the textual information, the issue is not much resolved. (Right) By extracting discriminative features in the text query, distributions are differentiated.
%         % \SE{} % R1@0.5 설명
%         Also, R1@0.5 indicates evaluation metric, Recall at 1 with IoU 0.5 threshold on QVhighlight \textit{val} set.
%         }
%       \label{fig:fig1_neg_attn}
%     \end{subfigure}%\hfill% or  or \hspace{0.3\textwidth}
%     \\
%     \begin{tabular}{cc}
%     \hspace{-0.2cm}
%         \begin{minipage}{.4\linewidth}
%             \begin{subfigure}[l]{1\linewidth}
%               \centering
%             %   \vspace{0.20cm}
%             %   \includegraphics[width=1\linewidth]{figs/fig_1_moti_textattn.pdf}  
%             %   \includegraphics[width=1\linewidth]{figs/fig_1_moti_textattn_v2.pdf}  
%               \includegraphics[width=1\linewidth]{figs/fig1_moti_violin_a.pdf}  
%               \vspace{-0.60cm}
%             %   \caption{text attention}
%                 \caption{Importance of queries in video representation}
%               \label{fig:fig1_text_attn}
%             \end{subfigure}%\hfill% or  or \hspace{0.3\textwidth}
%         \end{minipage}
        
%         \begin{minipage}{.6\linewidth}
%             \vspace{-0.2cm}
%             \caption{Overall statistics of Fig.~\ref{fig:motivation_ex} in QVHighlights dataset. 
%             (a) Saliency histogram shows the distributional gap between positive and negative video-text query pairs.
%             % (a) For the attention scores that measure how much the text query is generally involved in video representation, we use violin plots to show the probability density. We plot the score for each layer in the encoder.
%             % (b) Using the histogram, we compare how the baseline and QD-DETR yield different salient scores given the positive and negative video-text pairs.
%             % (b) Text ratio in self-attention layer to  of Moment-DETR
%             % (b) Ratio of text when representing video tokens in self-attention of Moment-DETR.
%             % (b) Magnitude of attention text query involved.
%             % (b) Attention score of video tokens
%             % (b) Magnitude of text query to refine the video tokens in self-attention layer of Moment-DETR.
%             (b) Probability density depicting the weight of the text query in attention score for video clips. Scores are from the self-attention layers in Moment-DETR encoder.
%             % (b) The text query ratio in attention score of video clips (Self-attention layer in Moment-DETR encoder). We use violin plots to show probability density.
%             % 텍스트 쿼리가, 비디오 피쳐에 얼만큼 attend 하는지
%             }
%         \end{minipage}
    
%     \end{tabular}
%     \vspace{-0.5cm}
%     \label{fig:moti}
%     % \captionsetup{belowskip=13pt}
%     % \setlength{\belowcaptionskip}{-10pt}
% \end{figure}


% \begin{figure}
%     \centering
%     % \includegraphics[width=1\linewidth]{figs/fig1_moti_negattn_1109.pdf}
%     \includegraphics[width=1\linewidth]{figs/fig1_moti_negattn_stat_v2.pdf}
%     \vspace{-0.8cm}
%     \caption{
%         Histogram of saliency when the positive and negative queries are given. We plot the histograms and its average value~(dotted line) to compare saliency scores when relevant~(positive) and irrelevant~(negative) text queries are given for each method. (Left) Since the video representations do not properly reflect textual information, both the positive and negative queries yield similar saliency scores. 
%         % (Middle) Even when the video representation is enforced to be updated with the textual information, the issue is not much resolved. 
%         (Right) By representing video clips in query-dependent manner, distributions are differentiated.
%     }
%     \vspace{-0.6cm}
%     \label{fig:motivation}
% \end{figure}


% One of the demanding task is moment retrieval task, which is detecting the desired moments from the given query, typically the text query.
When describing the moment, one of the most favored types of query is the natural language sentence~(text)\cite{anne2017localizing}. 
While early methods utilized convolution networks~\cite{zhang2020learning, gao2021fast, wang2020temporally}, recent approaches have shown that deploying the attention mechanism of transformer architecture is more effective to fuse the text query into the video representation.
% To handle these modalities, previous works simply employed the attention mechanism of transformer architecture to fuse the text query into the video representation.
For example, Moment-DETR~\cite{momentdetr} introduced the transformer architecture which processes both text and video tokens as input by modifying the detection transformer~(DETR), and UMT~\cite{umt} proposed transformer architectures to take multi-modal sources, e.g., video and audio. 
Also, they utilized the text queries in the transformer decoder.
Although they brought breakthroughs in the field of MR/HD with seminal architectures, they overlooked the role of the text query.
To validate our claim, we investigate the Moment-DETR~\cite{momentdetr} in terms of the impact of text query in MR/HD~(Fig.\ref{fig:motivation_ex}).
Given the video clips with a relevant positive query and an irrelevant negative query, we observe that the baseline often neglects the given text query when estimating the query-relevance scores, i.e., saliency scores, for each video clip.
% the output saliency score, i.e. query-relevance scores.
% Based on the observation, we traced the actual saliency prediction of the model against both the video-relevant query and the irrelevant dummy one where we find that the baseline often neglects the given text query when estimating the query-relevance scores of video clips.
% For example, in Fig.~\ref{fig:motivation_ex}, saliency scores are not affected even when the query is substituted with the dummy.
% % General statistics for Fig.~\ref{fig:motivation_ex} is shown in Fig.~\ref{fig:motivation}. 
% General statistics corresponding to Fig.~\ref{fig:motivation_ex} are also shown in Fig.~\ref{fig:motivation}.



% The limitation of the concrete baseline~\cite{momentdetr} is inspected in two different aspects; 1) Utilization of text-query in the encoding process and 2) the output saliency score, i.e. query-relevance scores.
% Firstly, we visualize the attention score when video clips are given as a query in self-attention. 
% We observe that the text queries have relatively small impacts compared to other video features, as shown in Fig.~\ref{fig:fig1_text_attn_ex}.
% That is, the text does not account for much in representing every video clip, although the goal of MR/HD is to detect query-relevant moments.
% Based on the observation, we traced the actual saliency prediction of the model against both the video-relevant query and the irrelevant dummy one where we find that the baseline often neglects the given text query when estimating the query-relevance scores of video clips.
% For example, in Fig.~\ref{fig:motivation_ex}, saliency scores are not affected even when the query is substituted with the dummy.
% % General statistics for Fig.~\ref{fig:motivation_ex} is shown in Fig.~\ref{fig:motivation}. 
% General statistics are also shown in Fig.~\ref{fig:motivation}.

% Consequently, in Fig.~\ref{fig:fig1_neg_attn_ex}~(b), we found that the baseline often neglects the given text query when estimating the query-relevance scores of video clips; 
% For example, 


% We validate the previous work sometimes neglects the given query when estimating the saliency of video clips.
% For example, there is an example that the saliency scores from positive and negative queries cannot be distinguishable, as shown in Fig.~\ref{fig:fig1_neg_attn_ex}.
% % 우리는 추가로 text attention을 추가도 해봤지만, 효과가 있긴 했으나, still 이슈가 있는 것을 확인하였다?
% % Still, we observe that assuring the high attendance of text queries does not resolve the overlap which motivates us to question the quality of the naive use of task-agnostic text representation~\cite{momentdetr, umt}.
% We found that introducing the text-attention for ensuring the high attendance of text queries relieve the overlap, but there still be a severe overlap.


% To validate their limitations, we inspect the impacts of text queries in the concrete baseline~\cite{momentdetr} with the two different aspects, 1) tendency of attention in self-attention layer and 2) saliency score, i.e. query-relevance scores. \SE{} % attention 이 갑자기 등장하는가?
% Firstly, we visualize the attention score when video clips are given as a query in self-attention. We observe the text queries have relatively low attention scores compared to the video features, as shown in Fig.~\ref{fig:fig1_text_attn_ex}.
% That is, the text does not account for much in representing every video clip, although the goal of MR/HD is to detect query-relevant moments.
% Based on this observation, we trace the actual saliency prediction of the model against both positive and negative text queries.
% We validate the previous work sometimes neglects the given query when estimating the saliency of video clips.
% For example, there is an example that the saliency scores from positive and negative queries cannot be distinguishable, as shown in Fig.~\ref{fig:fig1_neg_attn_ex}.
% % 우리는 추가로 text attention을 추가도 해봤지만, 효과가 있긴 했으나, still 이슈가 있는 것을 확인하였다?
% % Still, we observe that assuring the high attendance of text queries does not resolve the overlap which motivates us to question the quality of the naive use of task-agnostic text representation~\cite{momentdetr, umt}.
% We found that introducing the text-attention for ensuring the high attendance of text queries relieve the overlap, but there still be a severe overlap.



% Thus, we 
% query dependency를 높이기 위해 
% Cross-attention? text-attention? detailed explanation on text-attention should be needed?
% By handling these two issues, we find that more precise retrieval can be achieved.
% 
% 
%
% By projecting video-discriminative text features with high text attendance to source video, we f 
% We also find the need to improve the quality of query features since assuring high text attendance also results in...
% pairs are not finetuned to be discriminative that even the similarity within the pairs does not reflect the relevance between the query and the video clips.
% General statistics for Fig.~\ref{fig:motivation_ex} is shown in Fig.~\ref{fig:motivation}. 
% \SE{} % 이거 ??로 뜨는데, 위처럼 figure 그리면 label이 안되는걸까요
% \SE{}
% 형님 아래 사항 생각 좀 해보는게 좋을 거 같아요.
% fig 1. (a) 그림만 봤을 때 모든 clip에 대해 text attention이 일정이상 존재하긴 하니까, 뭔가 not assured to be conditioned가 와닿지 않는거 같아요.
% + 왜 text가 항상 attend 해야하나?
% not assured to be conditioned --> text shows relatively low affects compared to video 같이 실제 나타난 현상까지 같이 적으면 어떨까 싶어요.
% fig 1. (b) 덜 반영한다?

% \SU{}
% 일단 text가 attend 잘 되어야 한다는 것에 좀 궁금점이 생깁니다. 결국에는 text와 관련있는 frame들을 attend해서 higlight를 찾아야 하는게 아닐까요? 그리고, 현제 저희의 모델 구조상 text query가 Key와 Value로 거의 활용되고 있는데 그렇다면 결국에는 해당 모델은 text에 대한 attention이 전혀 없다고 봐도 무방하지 않을까요? 그런 면에서 text attention을 강조하는게 좀 걸리긴 합니다.

% Specifically, the text query is not assured to be explicitly conditioned on every clip of the video, and as the query texts are evenly treated, discriminative keywords may not be spotlighted.
% attention mechanism of Moment-DETR is not explicitly conditioned on the text query as shown in Fig~\ref{}(d), and in UMT, the text are only used for conditioning the queries while the video representation are refined itself by self-attention.

% \begin{figure}[t]
%     \begin{subfigure}{1\linewidth}
%       \centering
%     %   \includegraphics[width=1\linewidth]{figs/fig_1_moti_textattn.pdf}  
%     %   \includegraphics[width=1\linewidth]{figs/fig_1_moti_textattn_v2.pdf}  
%       \includegraphics[width=1\linewidth]{figs/fig_1_moti_textattn_v4.pdf}  
%       \vspace{-0.5cm}
%     %   \caption{text attention}
%         \caption{Distribution of attention scores in Moment-DETR encoder}
%       \label{fig:fig1_text_attn}
%     \end{subfigure}%\hfill% or  or \hspace{0.3\textwidth}
%     \vspace{0.2cm}
%     \begin{subfigure}{1\linewidth}
%       \centering
%     %   \includegraphics[width=1\linewidth]{figs/fig1_moti_negattn.pdf}  
%       \includegraphics[width=1\linewidth]{figs/fig1_moti_negattn_v2.pdf}  
%       \vspace{-0.5cm}
%     %   \caption{neg attention}
%         \caption{Saliency score against positive and negative text queries}
%       \label{fig:fig1_neg_attn}
%     \end{subfigure}%\hfill% or  or \hspace{0.3\textwidth}
%     \vspace{0.2cm}
%     \begin{subfigure}{1\linewidth}
%       \centering
%     %   \includegraphics[width=1\linewidth]{figs/fig1_moti_violin.pdf}  
%       \includegraphics[width=1\linewidth]{figs/fig1_moti_violin_v2.pdf}  
%       \vspace{-0.5cm}
%       \caption{violin}
%       \label{fig:fig1_violin}
%     \end{subfigure}%\hfill% or  or \hspace{0.3\textwidth}
%     \vspace{-0.2cm}
%     \caption{(a) 1. portion of text attention vs. video attention 2. relation with text query and content (e.g. fg, bg) of clip seems not to affect the attention score
%     (b) 1. high variability even though entire clips are highly correlated with the given text query 2. positive and negative query makes overlaps on saliency score distribution
%     (3) actual distribution on validation dataset.}
%     \label{fig:motivation}
%     % \captionsetup{belowskip=13pt}
%     % \setlength{\belowcaptionskip}{-10pt}
% \end{figure}

To this end, we propose Query-Dependent DETR~(QD-DETR) that produces query-dependent video representation.
% Our key focus is to ensure each clip in predicted moments is explicitly conditioned by the query, particularly on the video-descriptive portion of the text query.
% Our key focus is to ensure that query-relevant clips are predicted by enforcing each clip to be explicitly conditioned by the query.
%Our key focus is to ensure that the model prediction for each clip is highly relevant to the query.
Our key focus is to ensure that the model's prediction for each clip is highly dependent on the query.
% by enforcing each clip to be explicitly conditioned by the query. :)
% hmm...
% \SE {} % "query-relevant clips are predicted" 이 문장이 좀 애매한거 같습니다. relevant 클립을 놓지지 않고 찾는 것을 보장한다? 이런 느낌인지 아니면 높은 saliency 를 주는게 목적이다? model prediction이 query-relevance를 반영하는 것을 보장한다?
% Our key focus is to ensure that the model prediction reflects query-relevance of clips by enforcing each clip to be explicitly conditioned by the query.
First, to fully utilize the contextual information in the query, we revise the transformer encoder to be equipped with cross-attention layers at the very first layers.
% 상익's thought :  single video - query간의 관계만 고려 - 같은 word가 더 많이 쓰이는 것을 보고 
% 교수님's thought : neg pair 를 쓰면 쿼리를 보지 않고서는 video clip간만 고려하는 것이 사라짐. 왜냐면 0으로 내보내야 하기 때문. --> SE: relative difference 만 고려하다가, 
By inserting a video as the query and a text as the key and value of the cross-attention layers, our encoder enforces the engagement of the text query in extracting video representation.
% 원준 교수님 코멘트 반영해서 다시
Then, in order to not only inject a lot of textual information into the video feature but also make it fully exploited, we leverage the negative video-query pairs generated by mixing the original pairs.
Specifically, the model is learned to suppress the saliency scores of such  negative~(irrelevant) pairs.
Our expectation is the increased contribution of the text query in prediction since the videos will be sometimes required to yield high saliency scores and sometimes low ones depending on whether the text query is relevant or not.
% \SE{}
% learns to?
% By suppressing the saliency scores of the irrelevant video-query pairs, the model learns to spotlight only the video-specific discriminative words in the query.
% % \SE{} % ====================== 상익 수정 ========================
% However, this architectural design still lacks the capability of identifying the video-descriptive keywords in the query.
% % However, this architectural design still lacks in identifying proper query relevance.
% This is because the current training scheme only focuses on the interactions of video and clips within a single video while neglecting information shared throughout the entire video.
% % We argue the problem of the current training scheme that only focuses on distinguishing the clips in a single video while neglecting information shared throughout the entire video.
% Therefore, we leverage the negative video-query relationships to enhance the capability of identifying the contextual similarity of query and video clips.
% 
% 원준 원본 
% However, this architectural design heavily relies on the quality of the text query.
% Therefore, we leverage the negative video-query relationships to enable the model to emphasize key corresponding query features.
% By suppressing the saliency scores of the irrelevant video-query pairs, the model learns to spotlight only the video-specific discriminative words in the query.
% =========================================================
Lastly, to apply the dynamic criterion to mark highlights for each instance, we deploy a saliency token to represent the entire video and utilize it as an input-adaptive saliency criterion. 
With all components combined, our QD-DETR produces query-dependent video representation by integrating source and query modalities.
This further allows the use of positional queries~\cite{dabdetr} in the transformer decoder.
% Furthermore, we can exploit the advanced DETR decoder architectures using the positional information, e.g., DAB-DETR, since our encoded tokens consist of identical position representations from a single modality.
% \SE{} % ====================== 상익 수정 ========================
% Furthermore, we can exploit the advanced DETR decoder architectures using the positional information, e.g., DAB-DETR, since our video clip tokens consist of identical position representations from a single modality.
% 원준 원본
% It also enables the use of advanced DETR decoder architectures, e.g., DAB-DETR, for the first time, as these works exploit the position information within a single modality.
% =========================================================
Overall, our superior performances over the existing approaches validate the significance of the role of text query for MR/HD.
% Our extensive experiments on QVHighlights, TVSum, and Charades-STA datasets validate the significance of considering the role and the quality of text query.

% All components combined with dynamic anchor moments for the query of decoder, our FOQUE fosters the query-dependent video representation, thereby making the 
% All components combined, our modified transformer encoding process fosters the query-dependent video representation thereby achieving the state-of-the-art results on various benchmarks of moment-retrieval and highlight detection.
	
% -	Video Platform & Streamer & Consumer의 증가. 
% Video는 다른 데이터 타입보다 정보가 많아 유용하지만, 이는 다른 말로 해석하면 video를 보는 것은 time-consuming 하고, 원하는 것을 찾아보기에는 힘들 수 있음.
% 따라서, 많은 매체에서는 사람들의 더 많은 이목을 끌기 위해 highlight 비디오라는 것을 편집하여 공유도 함.
% 하지만, highlight video를 만들기 위해 사람의 노력이 필요한 현 시점에서, This spotlights the need to retrieve the user-requested / Highlight moments in the video.

% -	이전에도 이러한 문제를 해결하기 위해 (asdfasdf) for moment retrieval, (asdfasdf) for highlight detection 등이 제안 되었지만, 이들은 비디오의 특정 영역을 찾는다는 공통된 목적을 가지고 있으면서도, 데이터 셋의 한계로 인해 따로 연구되었음. 이를 문제 삼으며, 최근에는 두 task를 동시에 학습할 수 있는 dataset이 소개 되었는데, 컴퓨터비전에서 최근 각광을 받고 있는 Transformer 모델 도입과 함께 큰 발전을 거듭하고 있음.

% -	구체적으로, 이 두가지 task를 수행하기 위해서는 transformer를 두가지 방법으로 이용할 수 있는데, moment-DETR 처럼 moment 를 clip의 set 단위로 예측할 수 있고, UMT 처럼 clip-wise prediction을 할 수 있음. 하지만, 이들은 query를 condition이 아닌 video와 동등한 레벨로 취급하거나 [mDETR], 매 클립이 self-attention으로 mixing 된 후에 condition을 걸어주어 clip간의 차이를 확실하지 이용하지 못하였고, 또한, 확실하게 condition으로 주지 못하였고, video와 query 사이의 관계를 한정적으로만 이용하였다.

% -	따라서, we explore three different ways to fully exploit query information. First, we design one-way cross-attention layer to condition every clip with the query features. Then, we utilized the negative video-text pairs to better model the relationships between the video and the text embeddings. Lastly, we define the saliency token to be the video-query dependent saliency estimator.


















% ===================== neg pair 부분 ===========================
% Nevertheless, the current training scheme, only considering the given video-query pair, still disturbs the model from identifying proper query-relevance prediction.
% In detail, the model focus on learning the fine-grained discrepancy between video clips, while neglecting the information they share, which contains significant clues to understand the context of video.
% Therefore, we leverage the negative video-query relationships to enhance the capability of identifying the contextual similarity of query and video clips.
% Therefore, we leverage the negative video-query relationships by suppressing those pairs, so that enhance the capability of identifying the contextual similarity of query and video clips.
% We hypothsize the diversity in query-video pairs are insufficient to learn the general relationship between text query and video.
% Therefore, we leverage the negative video-query relationships by suppressing the saliency scores of the irrelevant video-query pairs.
% However, this architectural design still lacks in identifying proper query relevance.
% We argue that the current training scheme only focuses on learning the fine-grained discrepancy between clips in a single video, while neglecting the information they share, which contains significant clues to understand the context of the video.
% Therefore, we leverage the negative video-query relationships to enhance the capability of identifying the contextual similarity of query and video clips.
% However, this architectural design still lacks in identifying proper query relevance.
% We argue the problem of the current training scheme that only focuses on learning the fine-grained discrepancy between clips in a single video.
% That is, the current design neglects the information shared throughout the video, although it contains significant clues to understand the context of the video.

\section{Related Work}\label{sec:rel}

% This section summarizes related work on collusive fraud detection models and visual analytics approaches for fraud detection.

\subsection{Collusive Fraud Detection Models}

Anomaly detection models are widely applied to detect collusive fraud from graph data that records interpersonal events by identifying groups with unexpected behaviors~\cite{bindu2018discovering, molloy2016graph, li2012mining, joudaki2015using, akoglu2015graph}. Related methods can be divided into statistics-based models and ML methods.

Statistics-based models identify anomalies through the statistical information of nodes, edges, or sub-graphs. For instance, Akoglu et al.~\cite{akoglu2010oddball} extracted structural features, such as node degree or centrality, from the graph to find egonets. SpamCom~\cite{bindu2018discovering} identified spammer communities on Twitter by using structure and attribute features such as Twitter content similarity, user topology, and user profile. In healthcare scenarios, Chen et al.~\cite{chen2013novel} applied a spectrum analysis-based community detection method to detect patient referral fraud cases from a bipartite graph of physicians and specialists. Zhao et al.~\cite{zhao2019health} generated a dynamic heterogeneous information network containing patients, hospitals, and diseases. Then, they identified anomalies that fit predefined fraud patterns (e.g., the high-cost single treatment) over fixed or variable periods. Statistics-based methods can produce initial fraud candidates but may have erroneous results, requiring further validation by experts.

ML methods typically use GNN to detect fraud, as it is powerful for learning a deep representation of nodes. Previous studies are either conducted on homogeneous~\cite{wang2019fdgars, ding2019deep} or heterogeneous graphs~\cite{xu2021towards, wang2019semi, zhong2020financial}. Wang et al.~\cite{wang2019fdgars} constructed a network of reviewers in online app stores, where nodes (i.e., reviewers) are connected if they have reviewed the same app. The reviews and behavioral features of reviewers are extracted from review logs. Then, a graph convolutional network model is trained and used to detect more fraudsters based on the identified fraud. To detect collusion for fraudulent consumer loans from individuals with various roles (e.g., sellers and intermediaries), Xu et al.~\cite{xu2021towards} propose GRC, a novel GNN model, that learns representations of different types of individuals and detects loan fraud by using attention mechanisms and imposing conditional random fields. However, these ML methods are supervised or semi-supervised and thus require fraud-labeled data, which is lacking in our health insurance scenario.

Since the boundary between fraudulent and normal behavior in healthcare insurance could be unclear, automated models can hardly learn to judge correctly and achieve satisfying accuracy. Therefore, our approach integrates a graph-based detection model with a visual interface, which supports interactive data exploration, model optimization, and result validation.

\subsection{Visual Analytics Approaches for Fraud Detection}
For human-in-the-loop fraud detection, existing studies employ visual analytics to help users understand and implement detection tasks from the perspectives of individual portraits, dramatic changes, and interpersonal events.

Individual portraits include high-dimensional records, which can be described and compared by glyph representations~\cite{ko2014analyzing, cao2015targetvue, maccas2020vabank}.
TargetVue~\cite{cao2015targetvue}'s three circular glyphs depict Twitter users' communication activities, features, and social interactions. Juxtaposed glyphs allow users to compare the behaviors of different individuals and discover possible fraud, such as social bots.
To analyze and detect fraud patterns in banking transactions, Macas et al.~\cite{maccas2020vabank} offered different glyphs to characterize bank clients. Depending on the transaction amount, beneficiaries, and transaction time, the glyph has a circular or rectangular shape complemented with a series of symbols, which enhance the analyst's understanding of typical/atypical transaction profiles.

Dramatic changes are also an important point cut of fraudulent behaviors. Previous studies have designed multiple representation techniques to visualize temporal information, such as sequence visualization~\cite{DBLP:journals/ivs/MacasPM22, zhao2014fluxflow}, radial layouts~\cite{bertini2007spiralview, silva2021visualisation}, and calendar~\cite{lin2020taxthemis}, etc. FluxFlow~\cite{zhao2014fluxflow} demonstrates the impact of anomalous information (e.g., rumors) spreading through colored circles packed on a timeline. Bertini et al.~\cite{bertini2007spiralview} proposed SpiralView, which uses radar charts with spiral time axes to show how alerts change over time to detect suspicious periodic patterns. TaxThemis~\cite{lin2020taxthemis} uses calendar heatmaps to show evidence of transferring revenue through related taxpayers.

Interpersonal events can be summarized by graph visualization. For instance, 
financial fraud between buyers and sellers can be reflected by anomalous structural patterns composed of nodes and edges~\cite{didimo2011advanced}. Niu et al.~\cite{niu2018visual} used a node-link diagram to demonstrate the loan guarantee network, where each node belongs to a community defined by a random walk algorithm and is encoded with the corresponding color. In order to identify collective anomalies, Tao et al.~\cite{tao2018visual} proposed a high-order correlation graph to support analysis processes starting with an abnormal node. Corresponding nodes that contribute to the anomaly can be easily identified through the high-order correlation graph.

Our system incorporates graph and sequence visualization. To focus on collusive fraud in health insurance scenarios, our system provides richer contextual information, such as disease, drugs, and visit frequency.


\section{Domain Characterization}\label{sec:domain}

Through intensive collaboration with health insurance experts, we get access to real-world health insurance data, learn about the patterns of collusive fraud, and summarize a set of design requirements.


\subsection{Data Description}

The data used in this paper are from the local Healthcare Security Administration we collaborate with. Under experts' guidance, we excluded irrelevant fields and anonymized identity information. Two tables hold the processed data:
% The processed data are stored in two tables:

\begin{itemize}[noitemsep,topsep=0pt]
    \item \textbf{Patient visit table}: Rows represent patient visits. Columns include time, patient ID, medical institution, diagnosed disease, and total fee. For example, at 16:23 on August 12, 2021, patient $P_1$ went to a hospital. $P_1$ was diagnosed with hypertension and was reimbursed 36.35 yuan for drugs through health insurance. 
    
    \item \textbf{Drug table}: Doctors prescribe drugs for each patient visit. The name and dosage of each drug are recorded in this table. Note that a prescription can include multiple drugs. For example, the doctor prescribed $P_1$ antihypertensive drugs, consisting of perindopril and enalapril.
\end{itemize}

\subsection{Problem Specification}
\label{sec:pro}
Over the past year, we have worked closely with two health insurance audit experts with four years of work experience. Through multiple interviews with them, we learned about the health insurance system, examined existing fraud cases, and summarized patterns of collusive fraud behaviors.

In our scenario, a part of the medical expense can be reimbursed when insured patients pay for their drugs. Fraud occurs when insured patients sell the drugs instead of taking them after reimbursement. 
Seeking to cash out quickly, fraudsters need to gather a large number of drugs. Thus, fraudsters always collude with each other to purchase sufficient drugs within a short period. To avoid scrutiny, they prefer to visit clinics or pharmacies in poorly regulated rural or community areas. 
However, \rc{the behavior of fraudulent groups can be confused with the normal visit behavior of patient groups with chronic diseases.} For example, if the doctors treating the chronic disease see patients only at specific times, there is a high probability that certain patients will go to the same location at a similar time.
\rc{Therefore, normal patients can also have the characteristics of \textbf{spatio-temporal connections} and \textbf{group actions} similar to the collusive fraud group, i.e., patients may visit certain medical institutes together on similar cycles.}
To avoid misjudgment, \rc{it is indispensable for experts to review and validate suspicious groups by referring to the contextual information of patients' visit behavior.}

To better understand the needs of audit experts in detecting and analyzing collusive fraud, we interviewed them and summarized the current audit process into three steps.
% we interviewed experts about the current audit process and summarized it into three stages. 

1. Experts use the audit system (a graphical interface of the database) to learn the overall characteristics of the health insurance data, such as the cost of claims. Then, they narrow the scope of the investigation by filtering out patients of interest, such as those whose costs exceed 10,000 yuan. Since fraudsters may have multiple tricks, experts require repeated attempts to avoid omission. 
Then, experts leverage manual rules to filter patients to further identify suspicious groups. For example, experts can identify patients with an unusually high number of drug purchases over a period of time or sequentially separate out groups with a high overlap of visit locations and similar visit times \rc{(e.g., a group of five patients frequently visit a specific drugstore within 1 hour to purchase medications)}. 

2. Experts browse the list of suspicious groups to 
start with the groups with serious hazards potentially and seize the opportunity to stop loss in time. The potential hazard of a group can be estimated according to the group size and the total claim expense. Besides, experts may also merge groups with similar characteristics during browsing to improve analysis efficiency.


3. Experts judge whether a group is a collusive fraud or a false positive by examining details of patients' visit behaviors (diseases, drugs, hospitals, etc.). Highly suspicious groups will be further investigated (e.g., surveillance video inspections and phone/on-site interviews with the patients).


\subsection{Requirement Analysis}

As mentioned above, the audit process requires a large number of manual inspections, which are extremely time-consuming and labor-intensive.
Based on discussions with experts, we summarized requirements at three levels.

Users need to learn about the patients statistics and their behavior connections from the \textbf{overview} level.

\begin{compactitem}
    \item[\textbf{R1}]
    \textbf{Show attribute distribution of medical records.} An overview allows users to understand the dataset and find a starting point to detect fraud. For example, users can learn a reasonable expense range from the distribution of patient expenses. Then, the patients whose expenses exceed the threshold should be reviewed.
    
    \item[\textbf{R2}]
    \textbf{Allow flexible data filtering.} Our dataset includes many patients' visit records. Some of them do not need to be audited because of small expenses or limited numbers of visits. Filtering patients by appropriate user-specified conditions can improve analysis efficiency. 
    
    \item[\textbf{R3}]
    \textbf{Identify the behavioral connections among patients.} Connections of visiting behaviors and drug purchasing behaviors are the basis for detecting fraud groups. Thus, these connections should be identified according to expert knowledge, namely, user-specified restrictions. For instance, patients are considered to be potentially associated only if they visit the same location in less than 15 minutes more than five times.
    
    
\end{compactitem}

Further fraud detection processes should be implemented at the \textbf{group} level.

\begin{compactitem}
    \item[\textbf{R4}]
    \textbf{Detect patient groups.} Patient groups can be detected based on various user-specified rules (e.g., whether there exist specific behavior connections or whether the total expense exceeds a limit). Automation can be leveraged to guarantee the efficiency of group detection. 


    \item[\textbf{R5}]
    \textbf{Support suspicious group selection.}
    Given a large number of detected candidate groups, users need to quickly locate those with the most suspicion or hazards. Recommending groups using effective ranking approaches can accelerate group selection.
    
    

\end{compactitem}

Finally, users need to check group details from the \textbf{patient} level and find evidence of fraud.

\begin{compactitem}
    \item[\textbf{R6}]
    \textbf{Support suspicious patients verification.} Understanding the intra-group similarities of patient behaviors, such as prescribed diseases, drug purchases, and selection of medical institutions, can help users exclude irrelevant patients and examine suspicious patients.
    
    \item[\textbf{R7}]
    \textbf{Visualize the visit records of an individual patient.} As mentioned earlier, auto-detection can hardly differentiate fraud groups from patients with specific visit needs, which leads to false positives. Users should examine the identified suspicious fraudsters to prove or disprove their suspicions. Visualizing patients' histories of medical visits could help users gather evidence regarding the continuity and rationality of the visits. In this way, fraudulent groups and false positive groups can be differentiated based on expert knowledge.

\end{compactitem}

\section{Our Approach}\label{sec:approach}
This section provides an overview of the visual analytics approach and introduces the two employed models.

\subsection{Approach Overview}

Based on the design requirements elaborated in \autoref{sec:domain}, we propose a visual analytics approach (see \autoref{fig:workflow}), which enables users to analyze health insurance records at multiple levels to drill down into the data of interest, locate suspicious fraudulent groups, and find patient-level evidence to verify collusive fraud. Our approach consists of three stages: (1) co-visit network overview, (2) suspicious groups identification, and (3) suspicious patients examination.

\begin{figure*}[ht!]
    \centering % avoid the use of \begin{center}...\end{center} and use \centering instead (more compact)
    \includegraphics[width=2\columnwidth]{figures/fraudauditor-approach-3.pdf}
    \caption{The three-stage approach that can help users identify, examine, and verify suspicious groups of collusive fraud in health insurance.}
    \label{fig:workflow}
\end{figure*}

\textbf{Co-visit network overview.} In the first stage, users seek a general understanding of the data by checking the attribute distributions (\textbf{R1}, \autoref{fig:workflow}-a). According to data distribution and domain knowledge, users then filter patients for analysis (\textbf{R2}, \autoref{fig:workflow}-a). %, which helps to improve audit efficiency and reduce the interference of irrelevant data. 
Next, users check connections between the filtered patients interactively (\textbf{R3}, \autoref{fig:workflow}-b). Collusive behaviors can be disclosed by time gaps of visits or the number of co-visits. Users are supported to specify the definition of complicit behaviors by setting thresholds for time gaps and the co-visits number. 
%This stage provides an overview of patients from multiple perspectives to enable analysts to filter patients by relevant attributes (\textbf{R1, R2}) for group detection.

\textbf{Suspicious groups identification.} In the second stage, our system employs a group mining method (see \autoref{sec:sgm}) to detect complicit groups according to the user-specified definition (\textbf{R4}). %Based on user-constructed connections between patients, a suspicious group detection models that take into account the strength of the connections is used to identify fraudulent groups that have frequent visits in the same place and within short intervals (\textbf{R4}).
Then, our system provides multiple selection strategies to help users locate target groups from the list of detected groups (\textbf{R5}). Feasible strategies are multi-attribute filtering (\autoref{fig:workflow}-c), group comparison, and group ranking (\autoref{fig:workflow}-d). Users can also add neighboring patients or groups to optimize detection results (\autoref{fig:workflow}-e).
% Based on user settings of the auto-detection parameters, such as detecting groups that visit the same medical institution within two-hour interval for more than five times  (\textbf{R3}), \textit{FraudAuditor} then auto-detects and visualizes groups for analysis (\textbf{R4}). \textit{FraudAuditor} visualizes groups in coordinated views to enable examination and analysis of the characteristics of groups, such as attribute distributions on a group level and group similarities (\textbf{R4}).

\textbf{Suspicious patients examination.} In the third stage, our system calculates the similarity of the prescribed diseases and drugs between each pair of patients in a group (see \autoref{sec: dds}). According to the similarity, users can assess the likelihood of collusive fraud (\textbf{R6}, \autoref{fig:workflow}-f). Patients with a low likelihood can be excluded interactively. Next, users can investigate the rest patients by inspecting their visit behaviors at different time granularities (\textbf{R7}, \autoref{fig:workflow}-g). Our system allows users to quickly understand the time periods and frequency of co-visits among them. 
%Our system supports users to investigate patient visit behavior from different time granularity through tailored visit event visualization and to quickly identify the time period and frequency of suspicious behavior through co-visit links.
We also provide contextual information, including disease, drug, and fee, to help users reason and annotate whether the suspicious behavior is collusive fraud (\textbf{R7}, \autoref{fig:workflow}-h).
%  Analysts can select target groups from the group comparison module to explore individual patients. The patient examination module depicts patients' disease and drug similarities and the distribution of other attributes, such as patients' visiting frequencies of various medical institutions, for comparison and analysis (\textbf{R5}). Analysts can select patients of interest to further explore their historical records in a timeline (\textbf{R6}). The timeline provides the individual visiting histories as well as the aggregated visiting histories of all selected patients.
% Additionally, \textit{FraudAuditor} records analysts reasoning process through interaction logs and enables annotation on top of the charts.

\subsection{Suspicious Group Mining}
\label{sec:sgm}
\rc{To detect suspicious groups with spatio-temporal connections and group action characteristics (see \autoref{sec:pro})}, we propose a suspicious group mining method to detect collusive fraud in health insurance (\textbf{R4}). Our method first builds a co-visit network to represent the spatio-temporal relationship among patients. Based on the co-visit network, the method uses a modularity optimization-based community detection algorithm to mine suspicious groups. For clarity of description, we have listed the notations in \autoref{tab:notations}. See Algorithm~\autoref{alg:sgm} for the pseudo-code.

\begin{table}[htb]
  \centering
  \caption{Notation Definitions.}
  \label{tab:notations}
  \small
  \begin{tabular}{p{0.15\linewidth} p{0.75\linewidth}   }
  \toprule
  \textbf{Notation}  & \textbf{Description}  \\ 
  \midrule
      $\mathbf{P}$ & The patients set \\
      $m$ & The number of patients \\
      $\mathbf{V}$ & The visits set \\
      $n$ & The number of visits \\
      $t_i$ & The time of visit $v_i$ \\
      $\theta_1$ & The maximum time gap for a co-visit \\
      $\theta_2$ & The minimum number of co-visits \\
      $\mathbf{CV}(p_i,p_j)$ & The co-visit behaviors between patient $p_i$ and $p_j$\\
      $w(v_i,v_j)$ & The weight of a co-visit about visit $v_i$ and $v_j$ \\
      $w(p_i, p_j)$ & The weight of co-visits between patient $p_i$ and $p_j$ \\
      $\mathbf{W}$ & The weight between patients \\
      $\mathbf{D}$ & The diseases set \\
      $\mathbf{C}$ & The number of visits for each disease in $\mathbf{D}$ \\
      $w(d_i)$ &  The contribution of disease $d_i$ to the similarity of patients $p_i$ and $p_j$ \\
      $c_i$ &  The number of visits of disease $d_i$ \\
      $sim(p_i, p_j)$ &  The similarity of patient $p_i$ and $p_j$ \\
  \bottomrule
  \end{tabular}
\end{table}

\textbf{Co-visit network construction}. Patients in a collusive fraud group frequently visit the same medical institution within relatively short time periods. Considering such a characteristic, we construct a co-visit network $\mathbf{G}$ among patients to summarize the co-visit behaviors and detect collusive fraud. A node in the network represents a patient. An edge between two patients records the co-visit behaviors between the two patients. \rc{If the medical institutions of the two corresponding visits of two patients are the same, and the time gap is less than a threshold $\theta_1$ (the default is 1 hour, which can be adjusted to 6, 12, or 24 hours), it is considered a co-visit.} For patients $p_i$ and $p_j$, their co-visit behaviors are represented as
$\mathbf{CV}(p_i, p_j) = \{(v_{i1}, v_{j1}), \cdots, (v_{is}, v_{js})\}$ and $s$ is the total number of visits they made together. 

\textbf{Edge weight calculation}. The edge weight indicates the likelihood that the two patients belong to the same group. We calculated the weights of the edge $w(p_i, p_j)$ based on the number of co-visits and the visiting time gap. 
% We then build a community detection algorithm based on the weighted network.
As shown in the \autoref{eq:1}, the weight of a co-visit is inversely proportional to the visit time gap. To avoid the impact of occasional visits with a small time gap on the weight, \rc{inspired by the ReLU activation function, we set the cutoff time to 10 minutes based on expert experience, and weights less than that interval are considered to be the same.}
%maximum time gap threshold $\theta_1$ (default as 1 hour). The time gaps less than $\theta_1$ are modified as $\theta_1$.

% \begin{equation} \label{eq:1}
%     w(v_i, v_j) = \frac{1}{max(\theta_1, |t_i - t_j|)}
% \end{equation}

\rc{
\begin{equation} \label{eq:1}
    w(v_i, v_j)=\left\{
    \begin{array}{rcl}
        \frac{1}{max(10 \, \rm minutes, |t_i - t_j|)} &  & {|t_i - t_j| \leq \theta_1} \\
        0                                             &  & {otherwise}
    \end{array} \right.
\end{equation}
}

The edge weight $w(p_i, p_j)$ between two patients is the total of their co-visit weights, defined as \autoref{eq:2}.
An adjustable threshold $\theta_2$ (default as 4) for the minimum number of co-visits is set here to avoid random factors. The co-visit weight being less than the threshold indicates the low probability of both belonging to the same group.

\begin{equation} \label{eq:2}
    w(p_i, p_j)=\left\{
    \begin{array}{rcl}
        \sum_{z=1}^{|\mathbf{CV}(p_i, p_j)|} w(v_{iz}, v_{jz}) &  & {|\mathbf{CV}(p_i, p_j)| \geq \theta_2} \\
        0                                             &  & {otherwise}
    \end{array} \right.
\end{equation}

\textbf{Community detection}. \rc{In order to mine suspicious groups from the co-visit network, we use Louvain~\cite{blondel2008fast}, an community detection algorithm based on modularity optimization. The algorithm is applicable to weighted graphs and supports the exclusion of non-community nodes, which can yield clear detection results since most patients in the healthcare scenario are normal.}
%Not all community detection algorithms can be applied to weighted graphs. For example, the Newman-Girvan algorithm~\cite{girvan2002community} based on the betweenness centrality does not consider edge weights. Thus, it is not applicable in the health insurance scenario, where fraudulent groups are rare, and most of the patients are normal. After the survey and experiments, we choose the Louvain algorithm~\cite{blondel2008fast} based on modularity optimization, which supports weighted graphs and can exclude communities with smaller granularity, which makes the detection results clearer.


\begin{algorithm}[]
    \caption{Suspicious Group Mining}
    \small
    \label{alg:sgm}
    {{
                \begin{algorithmic}[1]
                    \Require
                    $\mathbf{P}$: the patient set;
                    $\mathbf{V}$: the visit records;
                    $\mathbf{W}$: the weight between patients;
                    $\theta_1$: the maximum time gap;
                    $\theta_2$: the minimum co-visit times.
                    \Ensure
                    $\mathbf{G}$: the co-visit network;
                    $\mathbf{SG}$: the suspicious groups.
                    \State $\mathbf{CV} \gets$ extract co-visit behavior from $\mathbf{V}$
                    \State $\mathbf{W} \gets \mathbf{0}$
                    \For{each patient pair $(p_i, p_j)$}
                    \State $w(p_i, p_j) \gets 0$
                    \If {$|\mathbf{CV}(p_i, p_j)| \geq \theta_2$}
                    \For{each co-visit ($v_{ik}, v_{jk}$) in $\mathbf{CV}(p_i, p_j)$}
                    \State $w(v_{ik}, v_{jk}) = \mathbbm{1}(|t_i - t_j| \le \theta_1)\frac{1}{max(10 \, \rm minutes, |t_i - t_j|)}$
                    \State $w(p_i, p_j) \mathrel{+}= w(v_{ik}, v_{jk})$
                    \EndFor
                    \EndIf
                    \EndFor
                    \State $\mathbf{G} \gets (\mathbf{P}, \mathbf{CV}, \mathbf{W})$
                    \State $\mathbf{SG} \gets \Call{cdlib.algorithms.louvain}{\mathbf{G}, \mathbf{W}}$ %// \textit{Calculates the community using louvain algorithm.}
                    \State \Return {$\mathbf{G}, \mathbf{SG}$}
                \end{algorithmic}}}
\end{algorithm}

\subsection{Disease and Drug Similarity} \label{sec: dds}

To verify collusive fraud groups, we need to calculate the similarity among patients based on their prescribed diseases and corresponding drugs (\textbf{R6}). At first, we tried to calculate the similarity according to the string texts of diseases and drugs, but the results were not satisfactory. For example, such a calculation would lead to the headache being similar to the stomachache and not similar to the stroke, when in fact both headaches and strokes are brain disorders with a closer relationship. Later, we found that either diseases or drugs have hierarchical encoding, which can reflect the similarity information. The ICD10~\footnote{\url{https://en.wikipedia.org/wiki/ICD-10}} coding of diseases and the standard coding of drugs~\footnote{\url{https://code.nhsa.gov.cn/toDetail.html?infoId=5546&CatalogId=2}} encode diseases and drugs hierarchically by large, medium, and small class.
For example, diseases J11 (influenza) and J18 (pneumonia) are similar, but they are very different from M54 (back pain).

Hence, we propose a nearest-match-based similarity calculation method that considers the disease/drug codes and the number of visits for the corresponding diseases. For each disease/drug, we need to find the most similar one in another \rc{patient's disease/drug set, so it is not affected by the specific order of visits}. Assume that a patient has been prescribed several diseases $\mathbf{D} = \{d_1, d_2, \cdots, d_l \}$. The corresponding numbers of visits for each disease are $\mathbf{C} = \{c_1, c_2, \cdots, c_l\}$. 
%To calculate disease similarity between two patients $p_i$ and $p_j$, we first grouped the diseases of the two patients according to the first letter of the disease codes $\mathbf{D}=\{\mathbf{D_a}, \mathbf{D_b}, \cdots, \mathbf{D_z} \}$.

As is shown in \autoref{fig:algorithm}, from the first letter of $p_i$'s and $p_j$'s diseases, we can see that both of them have been treated for diseases beginning with ``K''. However, $p_i$'s E10 (i.e., Type 1 diabetes), and $p_j$'s M54 (i.e., back pain) are not shared with each other. Thus, E10 and M54 contribute no disease similarity. Next, according to the code of the second letter, the disease set beginning with K can be divided into three medium-class sets: K0, K1, and K2. The K02 disease of $p_j$ does not correspond to the remained diseases of $p_i$, while the remaining diseases are further calculated according to the letters of the third letter until the diseases of the two patients in the set are exactly the same. The closer the two diseases are, the greater their contribution to the calculation of similarity. Therefore, the weight of each disease is determined by the longest prefix of the set in which it last stays, having

\begin{equation}
    w(d_i)= \frac{\text{the length of the prefix letter}}{\text{the total length of the coded letter}}.
\end{equation}

\begin{figure}[h]
    \centering % avoid the use of \begin{center}...\end{center} and use \centering instead (more compact)
    % width=0.5\textwidth
    \includegraphics[width=\columnwidth]{figures/algorithm-v2.pdf}
    \caption{An example of disease weight calculation.}
    \label{fig:algorithm}
\end{figure}

For example, the weight of disease K02 is 1/3$\approx$0.33, K12 is 2/3$\approx$0.66, and K13 is 1.00. The same goes for drugs. The number of medical visits for diseases/drugs also reflects their similarity. The similarity between two patients is calculated as follows:

\rc{
\begin{equation}
    sim(p_i, p_j) = \dfrac{\sum w(d_k^{p_i}) \cdot c_k^{p_i} + \sum w(d_k^{p_j}) \cdot c_k^{p_j}}{\sum c_k^{p_i} + \sum c_k^{p_j}}.
\end{equation}
}

%We use the hierarchical clustering method to sort the patients by their similarities in the matrix.

% First, we calculated the distance matrix between patients $dist(p_i, p_j) = 1-sim (p_i, p_j)$.
% To generate patient clusters, we first treated each patient as a cluster. The distance between two clusters was defined as the maximum distance between two patients from the two clusters. The method merges the two clusters closest to each other iteratively until one cluster remains.
% By setting the distance threshold(in practice, we set it to half of the maximum distance), we could get various numbers of clusters and the patient numbers of each cluster, and rearrange the order of patients in the matrix accordingly.

% 先注释掉伪代码
\begin{comment}
The algorithm pseudocode is as follows:

\begin{algorithm}[]
    \caption{Disease and Drug Similarity}
    \small
    \label{alg:ddsm}
    {{
                \begin{algorithmic}[1]
                    \Require
                    % $D$ and $C$ for two patients $p_1$ and $p_2$
                    $D_{p_1}, D_{p_2}$: the disease set for two patients;
                    $C$: the number of visits for each disease.
                    \Ensure
                    $sim$: similarity between two patients $p_1$ and $p_2$.
                    \State $sim \gets 0$
                    \State $s \gets 0$
                    \For{each disease $d_i$ in $D_{p_1}$($D_{p_2}$)}
                    \State $d_j \gets$ the most similar disease in $D_{p_2}$($D_{p_1}$)
                    \State $w(d_i) \gets$ the weight of $d_i$ // \textit{Based on the calculation described above, consider the longest prefixes of $d_i$ and $d_j$.}
                    \State $sim \gets sim + w(d_i) * c_i$
                    \State $sim \gets s + c_i$
                    \EndFor
                    \State $sim \gets \dfrac{sim}{s}$
                    \State \Return{$sim$}
                \end{algorithmic}}}
\end{algorithm}
\end{comment}

\section{System Design}\label{sec:design}

To help users implement the approach mentioned in \autoref{sec:approach}, we developed an interactive prototyping system, \textit{FraudAuditor}. This section presents a system overview and introduces the details of the visual design and interactions.

\subsection{System Overview}

The system contains four views, as shown in \autoref{fig:teaser}: the network analysis view, the group comparison view, the patient comparison view, and the patient behavior view. We describe an analysis flow to demonstrate how these four views help the user discover, analyze, and validate suspicious groups of collusive fraud based on health insurance data. The user first learns the data distribution from the bar chart of patient attributes in the network analysis view (\textbf{R1}), based on which he can interactively filter the data of interest (\textbf{R2}). Then he sets parameters on co-visit behavior and generates the co-visit network between patients in the patient co-visit network view (\textbf{R3}). The results of the automatic detection model are also displayed in the network in real-time by highlighting. The attribute distribution, similarity, and ranking of detected groups can be viewed in the group comparison view (\textbf{R5}). He clicks on the top-ranked group, and its position in the network is highlighted simultaneously. In the patient comparison view, he compares different patients in the group using the disease and drug similarity matrix, stacked bar chart, and area chart (\textbf{R6}). From there, he selects several suspicious patients and goes to the patient behavior view for further investigation. He visually analyzes the visit pattern and co-visit distribution of patients from the visit sequence visualization. Combined with the visualization of contextual information such as diseases and drugs, he infers and labels these patients as engaging in collusive fraud (\textbf{R7}).

\begin{figure*}[ht!]
    \centering % avoid the use of \begin{center}...\end{center} and use \centering instead (more compact)
    \includegraphics[width=\linewidth]{figures/teaser.pdf}
    \caption{\textit{FraudAuditor} facilitates the identification, examination, and annotation of collusive fraud in health insurance. (a) The network analysis view supports interactive filtering of patient attributes and constructing a co-visit network. (b) The group comparison view provides interactive filtering, similarity analysis, and ranking of groups. (c) The patient comparison view helps users analyze the similarity and distribution of diseases, drugs, and other attributes among patients within a group and helps select suspicious patients to be analyzed. (d) The patient behavior view supports the inspection and annotation of detailed patient visit records and the analysis of co-visit behaviors.}
    \label{fig:teaser}
\end{figure*}

\subsection{Network Analysis View}

The network analysis view (\autoref{fig:teaser}-a) has two parts: (1) The patient attributes view gives an overview of patients by showing attribute distributions and supports interactive filtering of data of interest (\textbf{R1, R2}). (2) The patient co-visit network lets users interactively define co-visit behaviors, browse the resulting co-visit network, and visually inspect suspicious groups detected by automated algorithms (\textbf{R3}).

In the patient attributes view (\autoref{fig:teaser}-a1), bars indicate the distribution of patient attributes, including the distribution of patients in terms of the number of visits, age, and total fee, as well as the number of visits to different medical institutions. At first, all patients are selected, and users can click on a bar to deselect/re-select the corresponding patients. Patients that are not selected are represented by a translucent background, and a mouse hovering over the corresponding bar will bring up a tooltip showing the total number of patients belonging to the original and current patients in the interval, making it easy for users to compare the distribution of patients under different filtering conditions.

In the patient co-visit network (\autoref{fig:teaser}-a2), the control panel above allows users to interactively configure the definition of co-visit behavior. A slider controls the minimum number of co-visits, and a drop-down menu sets the maximum time gap, such as 1 hour, 6 hours, etc.
% The maximum time gap for a co-visit can be adjusted via a drop-down menu, including 1 hour, 6 hours, etc., and the minimum number of co-visits can be controlled via a slider. 
Users can browse the co-visit network by clicking "Generate Graph" and iteratively changing the co-visit definition or filtering conditions in the patient attributes view if the results are unsatisfactory.
% Clicking on the "Generate Graph" button allows users to browse the currently defined co-visit network and if the results are unsatisfactory, users can iteratively modify the definition or select different filtering conditions in the patient attributes view.
Suspicious groups detected by the automated algorithm are displayed in the network simultaneously and marked in purple, where small groups can be filtered by adjusting the minimum component size slider. \rc{The node-link diagram was applied to visualize the graph. Because node-link diagram performs well in showing direct and indirect relationships among patients, which can support tasks on connection analysis and topology exploration.} Each node in the network represents a patient, and the edge between two nodes reflects their co-visit relationship, whose width represents the strength of the co-visit relationship (see \autoref{sec:sgm}). The network supports zooming and panning for navigation. Patients are highlighted, and patient ID and group ID (if any) are displayed when the mouse hovers over a node. If the node belongs to a group, all nodes in that group are highlighted. 
To help users understand the analysis provenance, selected, suspicious, normal, and other nodes are mapped to different visual encodings.
% Nodes with different statuses, e.g., selected, suspicious group, marked as normal, etc., are mapped to different visual encodings to facilitate users' understanding of the analysis provenance.

\subsection{Group Comparison View}

The group comparison view (\autoref{fig:teaser}-b) consists of three parts: (1) The group attributes view provides an overview of group-level attributes as well as interactive filtering capabilities. (2) The group projection supports similarity analysis of groups (3) The group rank allows sorting groups across multiple dimensions. Through initial filtering and careful selection, users can identify suspicious groups that need to be focused on for analysis (\textbf{R5}).

The group attributes view (\autoref{fig:teaser}-b1) uses bar charts to show the distribution of groups on various metrics, including the number of patients (p), the total fee per capita (f), the number of co-visits (c), the average number of days between multiple co-visits (d), and the minimum time gap within a co-visit (g), which are common metrics used by health insurance audit experts to evaluate how suspicious a group is. The view also supports interactive filtering to help narrow down the groups to be analyzed.

The group projection (\autoref{fig:teaser}-b2) maps groups to a two-dimensional plane, helping users to compare similarities and differences between groups and to detect clusters or outliers that are worthy of analysis. Using the group-level features mentioned above, the original groups are represented as a set of feature vectors. \rc{To compare the distribution of groups over different features, we used kernel PCA~\cite{scholkopf1998nonlinear} because it maintains the covariance of the data and is able to handle linearly indistinguishable cases.}
%To allow users to perceive the distribution of groups in a multidimensional space, the Kernel PCA~\cite{scholkopf1998nonlinear} is employed because it can handle linearly indistinguishable cases by using kernel functions compared to the traditional PCA~\cite{wold1987principal}. 
In the projection result, each triangle represents a group, and the distance between them reflects their similarity to some extent. Depending on the group status, e.g., filtered-out or user-selected, different visual encoding is applied to the group. The group that needs further investigation can be selected by clicking on it.

The group rank view (\autoref{fig:teaser}-b3) provides a drop-down menu for selecting the ranking keyword, where single group metrics or overall scores can be used.
We created a customized radar chart to visually compare groups across multiple criteria.
% To help users compare groups, we designed a customized radar chart to visually represent the group characteristics across multiple metrics. 
In \autoref{fig:radar}-a, each axis represents the metric mentioned above. Five gray lines run through each axis from inside to outside, statistically representing the lower fence, first, second, third quartile, and upper fence~\footnote{\url{https://en.wikipedia.org/wiki/Quartile}} of each group on each metric. For comparison, data transformations (e.g., adding negative signs) make all metrics more anomalous as they grow. A gray area indicates each metric's average value for reference.
% The average value on each metric is enclosed in a gray area as a reference.
The blue area shows the current group, whose size visually reflects the group's degree of suspicion.
A special arrow alerts users to outliers above the upper fence.
% For outliers beyond the upper fence, a special arrow is designed to draw the user's attention.
Hovering over this glyph displays details of the group and each metric. 
% When this glyph is hovered, the details of the group and each metric are displayed.
Users can click to select the corresponding group for subsequent analysis.


\begin{figure}[tb]
    \centering % avoid the use of \begin{center}...\end{center} and use \centering instead (more compact)
    % width=0.5\textwidth
    \includegraphics[width=\columnwidth]{figures/glyph-v3.pdf}
    \caption{Explanation of visual design. (a) Customized radar chart. Five axes represent different group metrics. The darker gray area represents the average of all groups on these metrics, and the blue area represents the current group. If the value on one axis exceeds $Q3 + 1.5\mathrm{IQR}$, it is marked with an arrow as an outlier. (b)The co-visit link of note metaphor, and the width represents the number of co-visits.}
    % \caption{Customized radar chart and the co-visit diagram of note metaphor.}
    \label{fig:radar}
\end{figure}

\subsection{Patient Comparison View}

The patient comparison view (\autoref{fig:teaser}-c) has two types of visualizations: (1) A similarity matrix of diseases and drugs that supports comparisons between patients and helps users determine whether diseases and drugs correspond to each other. (2) Stacked bar charts and area charts of the patient attribute to help analyze the contribution of different patients to the group. Through similarity and attribute comparison, users can identify suspicious patients (\textbf{R6}).

\rc{To further select suspicious patients and exclude innocent ones, users need to study the similarity of disease and drugs within a group. We show the similarity between each pair of patients in a matrix} (\autoref{fig:teaser}-c1). Cells are divided into two categories, with the upper left corner representing drug similarity and the lower right corner representing disease similarity. Both the horizontal and vertical axes of the matrix are the patients within the selected group and have the same order. Each cell is color-coded in green with the degree of similarity between the corresponding patients; the darker, the more similar. When the mouse hovers over a cell, the specific values of diseases and drug similarity and the corresponding patient IDs are displayed, and the diagonally symmetrical cell is also highlighted for comparison. To discover patterns of patient clustering on diseases and drugs, the system supports matrix reordering, where hierarchical clustering methods are used to determine new patient orders. The patient can be selected for a more detailed analysis by clicking on labels. \rc{Since there are usually at most a dozen fraudsters in a group, the matrix is less prone to visual clutter.}

\rc{If suspicious patients are found (e.g., with similar diseases but large differences in drugs), users need contextual information to further assess the rationality of drug purchase behavior.} In the stacked bar chart and stacked area chart (\autoref{fig:teaser}-c2), users can study contextual information about diseases, drugs, medical institutes visited, total fees, etc., in order to verify suspicious patients. The data corresponding to patients selected by users is highlighted, while the data for unselected patients becomes translucent to provide a clear picture of the proportion of these patients in the overall group on different attributes. When the mouse hovers over a bar or area, information about each selected patient and the entire group in that attribute interval is displayed in a tooltip.


% The \textit{Patient Comparison View} enables users to compare drug and disease similarities among the patients of the selected groups and select patients of interest to explore the detailed attributes (\textbf{R5}), such as medical costs, and their treatment history in the \textit{Patient Behavior View}.
% \textit{FraudAuditor} visualizes drug and disease similarities between patients of selected groups in a matrix-based view, as the matrix view is scalable to show various numbers of patients. Individual patients are represented by rows and columns in the same order from left to right and from bottom to top. The upper-left part of the table cells depicts the drug similarity whereas the bottom-right part describes disease similarities between two patients. The darker the table cell is, the more similar the disease or drug between the two patients.
% Additionally, the left and bottom sides of the matrix summarize the counts of drugs and diseases in bar charts where the x-axis lists the top 5 diseases or drugs in descending order of their number and the y-axis shows the corresponding quantities.
% The right side view summarizes the medical institutions and medical costs of selected groups in bar charts and time-varying costs in an area chart. The bar charts and the area chart also act as stacked charts where the detailed attributes of user-selected patients are shown as a new layer over the original bars or areas as stacked bars or areas.

% \textit{Interactions.} User interactions support patient-level comparison and patient selection for detailed explorations. Users can sort the patients in the matrix based on their drug and disease similarities to facilitate pattern discovery using the drop-down menu at the top. Mousing over a cell displays the pair of patient IDs and their drug and disease similarity values. Clicking on the patient labels, users select the patients to explore the detailed data in the \textit{Patient Behavior View}. Hovering over the labels of selected patients in the matrix view highlights the corresponding stacked bars and areas in the right charts. Hovering over the charts shows the attribute and its total value along with the values of selected patients.

\subsection{Patient Behavior View}

The patient behavior view (\autoref{fig:teaser}-d) contains: (1) a line chart of the number of visits over time for locating anomalies and navigation; and (2) a visualization of the patients' visit sequence that shows the evolution of patients' visits and highlights co-visits. Users can analyze visit behavior at the different temporal granularity and combine rich contextual information and domain knowledge to reason and verify whether it is collusive fraud (\textbf{R7}).

The line chart (\autoref{fig:teaser}-d1) at the bottom reflects the temporal change in the number of visits to support users in locating time periods with significant fluctuations and detecting periodic visits, etc. The gray boxes represent selected time periods and allow for range swiping and panning.

In the visit sequence visualization (\autoref{fig:teaser}-d2), each timeline corresponds to one patient's visit history during the selected time period. Different time periods lead to different view granularities, such as a month, week, day, etc., which helps improve readability and reduce the cognitive load. The bars on the timeline represent the patient's visit behavior, where the position encodes the visit time, the color represents the disease type, and the height refers to the number of visits for the corresponding disease. This allows users to quickly grasp information about the patient's main diseases, frequency of visits, etc. Since there are many possible disease types, to avoid visual clutter, we give different colors to the top 5 diseases, while the rest are represented in gray. Users can view specific information about diseases through the legend on the right side. The number of patients in the visible area can be adjusted using the plus or minus buttons on the right, and clicking the full-screen button displays all selected patients in the current window.

To represent the co-visit behavior between patients, we designed a co-visit link to explicitly show this suspicious behavior. As shown in \autoref{fig:radar}-b, if there is co-visit behavior between two patients, we extend a line at the corresponding position of each of the two patients' timelines and link them to each other with a vertical line. In addition, since the current timeline may contain aggregated visit events, the width of the vertical line is used to encode the number of co-visits. We did not use arcs to connect co-visit behaviors because they tend to cause more crossings and visual clutter.

To support an in-depth analysis of co-visit behavior, users can select the threshold in the co-visit time gap drop-down box to switch to the co-visit view (see \autoref{fig:case1}-e) and examine all co-visit behaviors within the selected time period.
% where all co-visit behaviors during the selected time period are revealed. 
When the mouse hovers above the bar, further contextual information about this co-visit appears, including each patient's visit time, medical institution, list of drugs, etc. Users can check a patient's age, gender, disease, and drug costs by clicking on the patient ID.
% Clicking on the patient ID allows users to check their age, gender, and cost information for the corresponding diseases and drugs. 
This supporting information helps users reason and verify if they are committing collusive fraud. If the review is complete, users can click the pencil button in the upper right corner to annotate these patients with a reason on the pop-up labeling page.


% The patient behavior view enables users to analyze individual patient's treatment history in a timeline (\textbf{R6}). Each row represents the treatment timeline of a selected patient. A bar in the timeline represents the number of visits for a specific disease during a certain time frame. We color-encoded the bars of the top 5 diseases and show the rest of the diseases in gray. Through exploring the timeline of a patient, users can see the main diseases the patient has suffered from and the patterns of his/her medical visits.
% If co-visits exist in a certain time frame, the corresponding patients will be connected with a vertical line; the thickness of the line encodes the number of co-visits. The line chart at the bottom summarizes the total number of visits of selected patients during the entire time frame.

% \textit{Interactions.} Interactions allow users to compare and analyze patients’ treatment histories in various time frames (\textbf{R6}). The line chart at the bottom provides the navigation control enabling users to select a certain time frame of interest. Specifically, users can pan and change the width of the gray area on the line chart to explore the detailed patient behaviors of a selected time frame in the upper timelines. As the aggregation granularity of the timeline changes, such as the granularity of one day, one month, or one year, the bars and vertical lines expand and merge accordingly to show the data in a specific period.
% The right side lists other controllers which allow users to zoom in and out the patient timelines vertically to compare multiple patients or analyze a single patient or revert back to the original state using the refresh button. The legend of the top 5 diseases on the right side enables users to turn on/off the color-encodings of the diseases to focus on one or multiple diseases in the timelines.

% Users can click on the pencil icon on the top right corner to make annotations. The annotation view lists the selected groups and their patients. Users can choose to take notes for a whole group or individual patients, such as annotating about group fraud or regular prescription change of traditional Chinese medicines. Marked patients will be updated in the patient co-visit network and group projection view.

\textbf{Alternative design.} Before we finalized the bar chart for the patient visit record, we had three alternative designs: the pie chart, the treemap, and Nightingale's rose chart. In \autoref{fig:alter}, the number of visits is denoted by height, angle, area, and radius, respectively. Although the latter three representations are more compact, the visual elements representing each disease have different angles, making comparisons difficult. Also, for the pie chart and Nightingale's rose chart, the radius and area are squared, which can easily mislead users. The bar chart, on the other hand, has a fixed orientation, which is suitable for disease comparison.


% The horizontal axis represents time, and the color represents disease. Each block reflects the top 5 diseases and the number of visits under a certain aggregation granularity of time. 
% The number of visits is 
% For treemap, the area of each rectangle is proportional to the percentage of visits for the corresponding disease in the total number of visits. 
% For the pie chart, the size of the circle represents the proportion of the disease in the whole. For the rose chart, the length of the fan radius represents the number of visits. 
% However, with treemap, pie chart and Nightingale's rose, each disease has a different perspective and it is difficult to compare the number of visits between different diseases. On the other hand, for pie chart and Nightingale's rose, the relationship between radius and area is square, which is easy for users to misunderstand, while for bar charts, the orientation of the columns is the same, so it is easy to compare the same and different diseases.

\begin{figure}[tb]
    \centering % avoid the use of \begin{center}...\end{center} and use \centering instead (more compact)
    % width=0.5\textwidth
    \includegraphics[width=0.9\columnwidth]{figures/alternative-design.pdf}
    \caption{Alternative design of patient medical visit behavior. The number of visits is indicated by (a) the height of the bar, (b) the angle of the sector, (c) the area of the rectangle, and (d) the radius of the sector.}
    \label{fig:alter}
\end{figure}

\setlength{\tabcolsep}{3.5pt}
\begin{table*}[]
 \centering
 % \resizebox{\columnwidth}{!}{%
 \resizebox{\linewidth}{!}{
 \begin{tabular}{l | c | c c c c | c c c c c c c}
 \hline
 \multirow{2}{*}{\textbf{Group}} &  \multirow{2}{*}{\textbf{\#KP}} &\textbf{Sal.} & \textbf{Cov.} & \textbf{Sal.+Cov.} & \textbf{Cov.} & \multirow{2}{*}{\textbf{Nat.} $\uparrow$} & \multirow{2}{*}{\textbf{Faith.} $\uparrow$} & \multicolumn{2}{c}{\textbf{Diversity}} & \multicolumn{2}{c}{\textbf{Utility}} \\
& & $SemP \uparrow$ & $SemR \uparrow$ & $SemF1 \uparrow$ & $SemCov \uparrow$ & & & $dup \downarrow$ &$emb\_sim \downarrow$ & $RR@5 \uparrow$ & $Spare_5@5 \uparrow$ \\
 \hline
 \multicolumn{8}{l}{\textbf{\textit{KP20k}}} \\
 \hdashline
 Human & 5.3 &  - & - & - & - & 0.884 & 0.788 & \textbf{0.073} & 0.133 & 0.698 & 0.500 \\
 KPE & 8.7 & 0.471 & 0.514 & 0.469 & 0.773 & 0.790 & 0.821 & 0.303 & 0.201 & 0.828 & 0.431 \\
 KPG-NN & 7.3 & 0.572 & 0.577 & 0.562 & 0.817 & \textbf{0.894} & 0.832 & 0.307 & 0.240 & 0.851 & 0.578 \\
 KPG-PLM & 6.2 & \textbf{0.584} & \textbf{0.605} & \textbf{0.582} & \textbf{0.828} & 0.891 & 0.837 & 0.139 & 0.167 & 0.864 & 0.572 \\
 GPT-3.5 & 8.0 & 0.464 & \textbf{0.605} & 0.514 & 0.825 & 0.876  & \textbf{0.860} & 0.115 & 0.112 & \textbf{0.982} & \textbf{0.609} \\
 API & 10.0 & 0.306 & 0.430 & 0.345 & 0.789 & 0.868 & 0.769 & 0.317 & \textbf{0.086} & 0.760 & 0.603 \\
 \hline
 \multicolumn{8}{l}{\textbf{\textit{KPTimes}}} \\
 \hdashline
 Human & 5.0 & - & - & - & - &  0.882 & 0.757 & 0.069 & 0.203 & 0.515 & 0.420 \\
 KPE & 8.5 & 0.488 & 0.540 & 0.490 & 0.751 & 0.796 & 0.763 & 0.192 & 0.258 & 0.725 & 0.396 \\
 KPG-NN & 6.5 & 0.708 & 0.741 & 0.713 & 0.862 & 0.871 & 0.723 & 0.171 & 0.240 & 0.541 & 0.457 \\
 KPG-PLM & 5.2 & \textbf{0.771} & \textbf{0.777} & \textbf{0.762} & \textbf{0.879} & 0.875 & 0.757 & 0.073 & 0.205 & 0.626 & 0.479 \\
 GPT-3.5 & 10.4 & 0.466 & 0.632 & 0.525 & 0.817 & \textbf{0.902} & \textbf{0.792} & \textbf{0.063} & \textbf{0.163} & \textbf{0.850} & \textbf{0.589} \\
 API & 10.0 & 0.285 & 0.386 & 0.324 & 0.772 & 0.858 & 0.736 & 0.216 & 0.168 & 0.528 & 0.231 \\
 \hline
 \end{tabular}
 }
 % \vspace{-2mm}
 \caption{Result for all dimensions for human references and five model groups. While reference-based metrics prefer pre-trained language models, GPT-3.5 is highly preferred in reference-free evaluations. "\#KP"=Number of keyphrases, "Sal."=Saliency, "Cov."=Coverage, "Nat."=Naturalness, "Faith."=Faithfulness, $dup$=$dup\_token\_ratio$. We use $\uparrow$ for the higher the better and $\downarrow$ for the reverse. The best entry in each column is boldfaced.}
 \label{tab:grouped-results-all-dims}
 % \vspace{-3mm}
 
\end{table*}
% 
\setlength{\tabcolsep}{3pt}
\begin{table}[]
    \centering
    \resizebox{\columnwidth}{!}
    {%
    \begin{tabular}{l | c  c  c}
    \hline
    Dimension & $1^{st}$ & $2^{nd}$ &$3^{rd}$ \\
    \hline
    \textbf{Naturalness} & CatSeqTF+2RF1 & SetTrans & ExHiRD-h\\
    \textbf{Faithfulness} & BERT+CRF & GPT-3.5 (5-shot) & GPT-3.5 (0-shot)\\
    \textbf{Saliency} & BERT+CRF & SciBART+OAGKX & CatSeq \\
    \textbf{Coverage} & SetTrans & SciBART+OAGKX & GPT-3.5 (0-shot) \\
    \textbf{Diversity} & GPT-3.5 (0-shot) & SciBART+OAGKX & TextRank \\
    \textbf{Utility} & GPT-3.5 (5-shot) & GPT-3.5 (0-shot) & Azure Cognitive Services \\
    \hline
    \end{tabular}
    }
    % \vspace{-2mm}
    \caption{Top three models in each evaluation dimension for KP20k. For diversity and utility, the models are ranked by the average of all the evaluated metrics.}
    \label{tab:winners}
    % \vspace{-2mm}
\end{table}

\section{Fine-grained re-evaluation of keyphrase extraction and generation systems}
\label{evaluation_human_study}

Using the proposed framework, we conduct a thorough re-evaluation of 18 keyphrase systems, grouped into keyphrase extraction (\textbf{KPE}, M1-M5), keyphrase generation with neural networks trained from scratch (\textbf{KPG-NN}, M6-M10), keyphrase generation with pre-trained language models (\textbf{KPG-PLM}, M11-M14), prompting \textbf{GPT-3.5} (M15-M16), and calling keyphrase extraction \textbf{API}s (M17-M18). Table \ref{tab:grouped-results-all-dims} presents the metric results on all dimensions. We also evaluate human-written keyphrase labels with reference-free metrics. The per-model performance is recorded in appendix \ref{all-evaluation}. % Table \ref{tab:winners} shows the highest performing model in all evaluated dimensions. 

\subsection{Reference-based evaluation}
From Table 4, we first observe that the performance underestimation problem of exact matching is alleviated by semantic matching, with the best model achieving around 0.6 $SemF1$ on KP20k and 0.8 on KPTimes. The difference between KPE and KPG models is also clearly distinguished compared to BertScore reported in \citet{koto-etal-2022-lipkey} and \citet{glazkova2022applying}. For all the reference-based metrics, we observe that the KPG-PLM family is consistently the best. However, comparing the best performing models in both families (Table \ref{tab:all-results-ref-based}), we find that they achieve the same level of coverage, while KPG-PLM models are better at generating phrases with higher saliency. 

For GPT-3.5, we observe that it achieves a competitive coverage in both 0-shot and 5-shot setting. The zero-shot setting achieves a low saliency as the model is not shown any demonstrations. With five examples, the saliency significantly increases. For the Amazon and Azure APIs, we find that they cannot outperform zero-shot prompting GPT-3.5 in the reference-based setting (appendix \ref{all-evaluation}).


\subsection{Reference-free evaluation}
\paragraph{Naturalness} We find that models trained on the sequence generation objective have the highest naturalness. By contrast, KPE models overall exhibit poor naturalness. Pre-trained language models such as BERT and BART do not significant outperform models trained from scratch on KP20k or KPTimes. Noteably, GPT-3.5 exhibits the best performance on KPTimes while performs worse than KPG-NN models on KP20k, suggesting that the model is more prone to generating unnatural phrases in the specialized science domain.

\paragraph{Faithfulness} In terms of faithfulness, GPT-3.5 leads the other models by a large margin. We hypothesize that the GPT-3.5 model has obtained a strong ability of extracting the concepts from the input with minimal paraphrasing. Surprisingly, KPE models outperform KPG-PLM/NN on KPTimes, but not on KP20k. One explanation is that KPE models' prediction may group words that do not belong to the same phrase, which may likely be deemed unfaithful in the scientific domain.

In addition, human references do not obtain a high score, which can be caused by humans writing more abstract absent keyphrases. This suggests that the metric model can be further improved for judging the faithfulness of concepts absent from the input. We also note that UniEval may inherently prefer similar pre-trained language models like GPT-3.5 and is not suitable as a single gold standard \citep{deutsch-etal-2022-limitations}.

\paragraph{Diversity} For diversity, we find that $dup\_token\_ratio$ and $embed\_sim$ do not always agree. While the former prefers humans and GPT-3.5, the latter prefers GPT-3.5 and keyphrase APIs. In addition, KPG-PLM models have much higher diversity compared to KPG-NN and KPE models. After a manual inspection, we find that the major reason is that KPE-PLM models generate much less duplications compared to KPG-NN even if greedy decoding is used for both. In addition, some KPE models use ranking heuristics that rank similar phrases together, causing a high duplication. From this perspective, methods that explicitly model topical diversity (such as M3) has a great advantage. 

\paragraph{Utility} The last two columns of Table \ref{tab:grouped-results-all-dims} show the performance of the keyphrase predictions for downstream document retrieval. For both datasets, we use the training and the testing corpus as $C$ and report the scores for $k=5$. It is notable that the \textbf{utility evaluation does not agree with reference-based evaluation}, i.e., scoring high against human references does not guarantee good downstream performance. For both retrieval effectiveness ($RR$) and efficiency ($Spare$), we observe that GPT-3.5 leads the other types of models by a large margin, indicating its an outstanding ability to pick useful keyphrases and rank them properly. Human-written keyphrases have the lowest $RR$, possibly due to the low number of keyphrases per example. This issue is mitigated with $Spare_5$, which always focuses on the top five keyphrases. 

\subsection{Discussion}
In the previous two subsections, we have benchmarked a range of keyphrase systems using the proposed framework. What insights can these results provide for NLP practitioners?

\paragraph{No one-winning-for-all model. Think about what you need.}
As shown earlier, different models have different assumptions and capabilities. Therefore, we define fine-grained goals and select proper tools for evaluating them. If a "human-like behavior" as defined by a large dataset is desired, then KPG-PLM models are the best choice. However, if the goal is to have a system that can generate natural, faithful, and diverse keyphrases for humans and robust to inputs from multiple domains, then prompting GPT-3.5 with a few examples should be preferred. If the keyphrases are merely used for IR tasks and inference cost is a concern, then a keyphrase extraction model such as M3 or M18 would be the best choice (Table \ref{tab:all-results-ref-free-kp20k} and \ref{tab:all-results-ref-free-kptimes}). 

\paragraph{Do not overtrust human references.} Our results suggest that judging the models based on their similarity to human references is insufficient for evaluating keyphrase systems. Particularly, we have shown that human references themselves are suboptimal in terms of diversity and utility. 
 

\paragraph{Nuances in model selection.} In general, model selection decisions are affected by both the metric and the benchmarking dataset. Our framework produces different model ranking than previous work such as \citet{ye-etal-2021-one2set} and \citet{2212.10233}. We thus warn that even for the same dimension, different metrics (e.g., F1 calculated by different matching strategies) have different connotations and one should not assumption a metric's ranking be identical to another. In addition, different model rankings can be concluded for highly specialized domains (such as KP20k) compared to general domains (such as KPTimes). 


% As a side note, for a more specialized domain (KP20k), we observe a higher retrieval performance (0.1 to 0.2 absolute points higher) than a more general domain (KPTimes), as scientific papers are more likely to be uniquely identified by specific keyphrases. 


% For instance, \citet{2212.10233} report a better performance of M14 compared to M10 in terms of F1@M, while the two have a similar $SemF1$. \citet{ye-etal-2021-one2set} report that M7 is more superior than M8, while we observe the reverse for $SemF1$. It shows that our metrics have different assumptions compared to $F1@k$ and their correlation should not be assumed in practice.



\section{Discussion}\label{sec:discussion}

This section discusses the advantages and limitations of \textit{FraudAuditor} from the perspectives of generalizability and scalability. We also summarize the lessons learned from the implementation process and shed light on future work.

\textbf{Generalizability.} 
\rc{FraudAuditor is designed to detect collusive fraud in healthcare, where patients often visit certain medical institutes together at similar times. By adjusting the co-visit definition, the detection approach can uncover other types of healthcare fraud. For example, in the case where fraudsters work asynchronously, co-visit behaviors can be detected by increasing the parameter of the co-visit time gap or using time alignment methods (e.g., DTW~\cite{muller2007dynamic}); in the case where fraudsters spatially disperse, the location constraint can be relaxed to the same area by considering the geographic information of the medical institutes.}
%For other cunning fraud variants, such as different fraudsters within a group acting at different times in different locations. Our approach is also adaptable to such ``soft'' co-visits, for example by adjusting parameters such as the co-visit time gap and by using time alignment methods such as DTW to cope with different initial time of fraud. In addition, considering the geographical information of medical institutes allows relaxing the same fraudulent location to the same area.

Our approach can also be generalized to other application domains where the frauds share similar characteristics of group and simultaneity, e.g., electronic commerce fraud, spam detection, and telecommunication fraud. Taking spam as an example, spammers always use botnets to send group emails by controlling multiple bot accounts. Similar to the co-visit network in health insurance, a co-sending network between email accounts can be constructed by considering the sending interval and the number of co-sendings. %so it is promising to detect bot accounts from it.

\textbf{Scalability.} %Due to the large volume of data, the computation of the co-visit network could take a couple of seconds. Possible solutions to this problem include: 1) leveraging multi-core CPU and GPU to accelerate the algorithm, 2) introducing a distributed computation scheme by dividing the graph into multiple subgraphs, and 3) integrating more efficient algorithms.
\rc{Due to enormous volumes of raw data, our system faces scalability issues in data processing and data visualization. We allow users to use data filtering to focus on a small set of records. Since the percentage of fraud is small, reducing the data scope through spatio-temporal segmentation and attribute filtering, such as removing visit records from highly regulated public hospitals, is in line with the practical workflow and the principle of overview to detail in visualization~\cite{shneiderman2003eyes}. In the extreme case of a group with many fraudsters, the patient behavior view may encounter visual clutter and rendering bottlenecks, which can be mitigated using data sampling and progressive visualization~\cite{zgraggen2016progressive}.}

% Besides, the presentation of medical behavior in the patient behavior view also has a scalability issue. When the space is limited, the timeline of medical treatment may overlap. \textit{FraudAuditor} supports interactive observation from multiple time granularity. However, the rendering efficiency in a large time period is still not satisfying. In order to display more data, we can speed up the interactive response by pre-loading and pre-drawing.

\textbf{Lessons learned.} %We summarized the lessons learned in developing \textit{FraudAuditor}.
First, multi-level views should be provided for the visualization of complex high-dimensional data to support progressive analysis.
Directly presenting all information to users increases cognitive load.
The health insurance data used in this paper involves associations among multiple subjects and has a large number of attribute dimensions.
We split the analysis tasks and coordinated views into three levels (i.e., overview-level, group-level, and patient-level).
%Multiple coordinated views are designed.
Thus, both the visual analytics approach and the system design follow the overview-to-detial principle.

Second, intuitive and effective visualization helps users learn quickly.
Because users of our system are experts with a background in health insurance audit, they don't know much about visualization.
The charts in \textit{FraudAuditor} are mainly common and popular charts, such as bar charts and node-link diagrams, which help lower the learning cost for users and increase their trust in the system.

\textbf{Future work.} In the future, \rc{to detect other types of fraud, such as doctor-patient collusion}, we plan to provide more detailed contexts of medical records by constructing a dynamic heterogeneous network of patients, doctors, and medical institutions. Another possible direction is to reduce the cost of manually labeling the dataset by leveraging active learning techniques to improve the efficiency of data instance selection. Additionally, the precision of group detection can be further improved by semi-supervised algorithms. \rc{We also plan to add more guidance and annotations to the system to further improve its usability.}


\section{Conclusion}
\label{sec:conclusion}

We consider top-down attention by explaining from an Analysis-by-Synthesis (AbS) view of vision. Starting from previous work on the functional equivalence between visual attention and sparse reconstruction, we show that AbS optimizes a similar sparse reconstruction objective but modulates it with a goal-directed top-down modulation, thus simulating top-down attention. We propose \model, a top-down modulated ViT model that variationally approximates AbS. We show that \model achieves controllable top-down attention and improves over baselines on V\&L tasks as well as image classification and robustness.



% if have a single appendix:
%\appendix[Proof of the Zonklar Equations]
% or
%\appendix  % for no appendix heading
% do not use \section anymore after \appendix, only \section*
% is possibly needed

% use appendices with more than one appendix
% then use \section to start each appendix
% you must declare a \section before using any
% \subsection or using \label (\appendices by itself
% starts a section numbered zero.)
%


% \appendices
% \section{Proof of the First Zonklar Equation}
% Appendix one text goes here.

% you can choose not to have a title for an appendix
% if you want by leaving the argument blank
% \section{}
% Appendix two text goes here.


% use section* for acknowledgment
\ifCLASSOPTIONcompsoc
  % The Computer Society usually uses the plural form
  \section*{Acknowledgments}
\else
  % regular IEEE prefers the singular form
  \section*{Acknowledgment}
\fi

% We would like to thank all the reviewers for their constructive comments. 
This work was supported by the NSFC (62132017, 62202244).

% Can use something like this to put references on a page
% by themselves when using endfloat and the captionsoff option.
\ifCLASSOPTIONcaptionsoff
  \newpage
\fi



% trigger a \newpage just before the given reference
% number - used to balance the columns on the last page
% adjust value as needed - may need to be readjusted if
% the document is modified later
%\IEEEtriggeratref{8}
% The "triggered" command can be changed if desired:
%\IEEEtriggercmd{\enlargethispage{-5in}}

% references section

% can use a bibliography generated by BibTeX as a .bbl file
% BibTeX documentation can be easily obtained at:
% http://mirror.ctan.org/biblio/bibtex/contrib/doc/
% The IEEEtran BibTeX style support page is at:
% http://www.michaelshell.org/tex/ieeetran/bibtex/
\bibliographystyle{IEEEtran}
% argument is your BibTeX string definitions and bibliography database(s)
%\bibliography{IEEEabrv,../bib/paper}
\bibliography{template}
%
% <OR> manually copy in the resultant .bbl file
% set second argument of \begin to the number of references
% (used to reserve space for the reference number labels box)
% \begin{thebibliography}{1}

% \bibitem{IEEEhowto:kopka}
% H.~Kopka and P.~W. Daly, \emph{A Guide to \LaTeX}, 3rd~ed.\hskip 1em plus
%   0.5em minus 0.4em\relax Harlow, England: Addison-Wesley, 1999.

% \end{thebibliography}

% biography section
% 
% If you have an EPS/PDF photo (graphicx package needed) extra braces are
% needed around the contents of the optional argument to biography to prevent
% the LaTeX parser from getting confused when it sees the complicated
% \includegraphics command within an optional argument. (You could create
% your own custom macro containing the \includegraphics command to make things
% simpler here.)
% \begin{IEEEbiography}[{\includegraphics[width=1in,height=1.25in,clip,keepaspectratio]{mshell}}]{Michael Shell}
% or if you just want to reserve a space for a photo:

% \begin{IEEEbiography}{Jiehui Zhou}
% Biography text here.
% \end{IEEEbiography}


% if you will not have a photo at all:
\begin{IEEEbiography}[{\includegraphics[width=1in,height=1.25in,clip,keepaspectratio]{figures/bio/zhoujiehui.png}}]{Jiehui Zhou}
is a Ph.D. candidate at the State Key Laboratory of CAD\&CG, Zhejiang University, and works under the supervision of Prof. Wei Chen. He holds a bachelor's degree in computer science and technology from Central South University. His main research focuses on visual analytics, decision intelligence, and human-centered AI, with a special interest in their applications in healthcare.
\end{IEEEbiography}

\begin{IEEEbiography}[{\includegraphics[width=1in,height=1.25in,clip,keepaspectratio]{figures/bio/wangxumeng.jpg}}]{Xumeng Wang}
is a lecturer of computer science in Nankai University. She received the Ph.D. degree in computer science and technology from Zhejiang University in 2021. Her research interests are visual analytics and privacy preservation.
\end{IEEEbiography}

\begin{IEEEbiography}[{\includegraphics[width=1in,height=1.25in,clip,keepaspectratio]{figures/bio/wangjie.jpg}}]{Jie Wang}
is a development engineer for business intelligence at Alibaba Group, Hangzhou. He earned a M.S. degree in software engineering from Zhejiang University in 2021. His research focuses on information visualization and augmented analytics.
\end{IEEEbiography}

\begin{IEEEbiography}[{\includegraphics[width=1in,height=1.25in,clip,keepaspectratio]{figures/bio/yehui.jpeg}}]{Hui Ye}
is a development engineer at Tencent, Shenzhen. She earned a M.S. degree in software engineering from Zhejiang University in 2021. Her research focuses on data visualization.

\end{IEEEbiography}

\begin{IEEEbiography}[{\includegraphics[width=1in,height=1.25in,clip,keepaspectratio]{figures/bio/wanghuanliang.jpg}}]{Huanliang Wang}
is currently working toward the M.S. degree with the State Key Lab of CAD\&CG, Zhejiang University, Hangzhou, China. His research interests are data visualization and visual analytics.
\end{IEEEbiography}
% insert where needed to balance the two columns on the last page with
% biographies
%\newpage

% \newpage

\begin{IEEEbiography}[{\includegraphics[width=1in,height=1.25in,clip,keepaspectratio]{figures/bio/zhouzihan.jpg}}]{Zihan Zhou}
is a Ph.D. candidate in the State Key Lab of CAD\&CG at Zhejiang University, Hangzhou. She earned a B.S. degree in digital media technology from Zhejiang University in 2021. Her research interests are data visualization and visual analytics.
\end{IEEEbiography}

\begin{IEEEbiography}[{\includegraphics[width=1in,height=1.25in,clip,keepaspectratio]{figures/bio/handongming.jpeg}}]{Dongming Han}
is currently working toward the Ph.D. degree with the State Key Lab of CAD\&CG, Zhejiang University, Hangzhou, China. His research interests include information visualization, graph visualization, and visual analytics. He received a B.S. degree in software engineering from Zhejiang University in 2017.
\end{IEEEbiography}

\begin{IEEEbiography}[{\includegraphics[width=1in,height=1.25in,clip,keepaspectratio]{figures/bio/yinghaochao.jpeg}}]{Haochao Ying}
is currently an assistant professor in the School of Public Health, Zhejiang University. He received the Ph.D. degree in the College of Computer Science from Zhejiang University in 2019, and the B.S. degree in computer science and technology from Zhejiang University of Technology in 2014. His research interests include data mining for healthcare and personalized recommender systems. He has authored some papers at prestigious international conferences and journals, such as TKDE, TCBB, JBHI, IJCAI, ICML, and CVPR.
\end{IEEEbiography}

\begin{IEEEbiography}[{\includegraphics[width=1in,height=1.25in,clip,keepaspectratio]{figures/bio/wujian.jpeg}}]{Jian Wu}
is a full professor at Zhejiang University. He is currently the director of the Real Doctor AI Research Centre of Zhejiang University. His research interests include artificial intelligence, data mining, and their applications in healthcare and biomedicine. He received his Ph.D. degree in Computer Science and Technology from Zhejiang University. He has published more than 200 papers in some prestigious refereed journals and conference proceedings, such as IEEE Transactions on Knowledge and Data Engineering, IEEE Transactions on Medical Imaging, CVPR, IJCAI, AAAI, ICML, and MICCAI. He is a distinguished member of the CCF.
\end{IEEEbiography}

\begin{IEEEbiography}[{\includegraphics[width=1in,height=1.25in,clip,keepaspectratio]{figures/bio/chenwei.jpg}}]{Wei Chen} is a professor at the State Key Lab of CAD\&CG, Zhejiang University. His research interests include visualization and visual analytics. He has published more than 90
IEEE/ACM Transactions and IEEE VIS papers. He actively served as a guest or associate editor of
IEEE Transactions on Visualization and Computer Graphics, IEEE Transactions on Intelligent Transportation Systems, and Journal of Visualization. For more information, please refer to
\url{http://www.cad.zju.edu.cn/home/chenwei/}
%He received the PhD degree from Zhejiang University in 2002. His research interests include visualization, visual analytics, and biomedical image computing. He has authored or coauthored more than 90 IEEE/ACM Transactions and IEEE VIS papers.
\end{IEEEbiography}
% You can push biographies down or up by placing
% a \vfill before or after them. The appropriate
% use of \vfill depends on what kind of text is
% on the last page and whether or not the columns
% are being equalized.

%\vfill

% Can be used to pull up biographies so that the bottom of the last one
% is flush with the other column.
%\enlargethispage{-5in}



% that's all folks
\end{document}


