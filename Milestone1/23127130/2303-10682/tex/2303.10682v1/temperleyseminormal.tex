\documentclass[10pt,hidelinks]{article}

\usepackage{cancel}
\usepackage{pstricks-add}
\usepackage{yfonts}
\usepackage{pstricks-add}

\usepackage{pst-func}

\usepackage{bbold}
\usepackage[normalem]{ulem}
\usepackage{etex}
%\usepackage[left=3cm,top=2cm,right=3cm,bottom=2cm]{geometry}
\usepackage[left=2cm,top=2cm,right=2cm,bottom=2cm]{geometry}
\usepackage{amsmath, amsthm, amssymb}
\allowdisplaybreaks
\usepackage{pst-all}
\usepackage[textwidth=0.5in]{todonotes}
\usepackage[rightcaption]{sidecap}
\renewcommand\sidecaptionsep{0.1cm}
\usepackage{relsize}
\usepackage{float}
\usepackage[caption = false]{subfig}

\usepackage[hypertexnames=false]{hyperref}


%%%%%%%%%%%%%%%%%%%%%



%%%%%%%%%%%%%%%%%%%%%

%\usepackage[left=3cm,top=3cm,right=3cm,bottom=3cm]{geometry}



\usepackage[scaled=1]{helvet}
\usepackage{titlesec}
\usepackage{multido}
\usepackage{graphicx}
\usepackage{multirow}
\usepackage{blindtext}
\usepackage{color}
\usepackage{lipsum}
\usepackage{mathtools}
\usepackage[vcentermath]{youngtab}
\usepackage{young}
\usepackage[all,ps,dvips,graph]{xy}
\usepackage[misc,geometry]{ifsym}
%\usepackage{genyoungtabtikz}


\usepackage{titlesec}
\usepackage{lineno,hyperref}



\let\OLDthebibliography\thebibliography
\renewcommand\thebibliography[1]{
  \OLDthebibliography{#1}
  \setlength{\parskip}{0pt}
  \setlength{\itemsep}{0pt plus 0.3ex}
}



\newenvironment{demespecial}{\noindent \textit{\bf{Proof of Theorem} }\ref{theorem first generators for lambda singular}: }{\quad \hfill $\square$}

\newenvironment{demespecialregular}{\noindent \textit{\bf{Proof of Theorem} }\ref{main theorem regular}: }{\quad \hfill $\square$}
\newcommand*{\Lbullet}{\raisebox{-0.25ex}{\scalebox{.7}{$\bullet$}}}


\titleformat{\section} {\normalfont\scshape \large \centering}{ \thesection}{1em}{}
\definecolor{morado}{rgb}{0.5,0,0.5}

%%%%%%%%%%%%%%%%%%%%%%%%%%%%%%%%%%%%%%%%%%



\newcommand{\class}{\mathcal{CL} }
\newcommand{\comu}{\mathbb C}
\newcommand{\QQ}{\mathbb Q}
\newcommand{\cupdot}{\mathbin{\mathaccent\cdot\cup}}


\newcommand{\once}{11}
\newcommand{\onceB}{\color{blue}11}
\newcommand{\diez}{10}
\newcommand{\diezR}{\color{red}10}
\newcommand{\doce}{12}
\newcommand{\doceB}{\color{blue} 12}
\newcommand{\doceR}{\color{red} 12}
\newcommand{\trece}{13}
\newcommand{\quince}{15}
\newcommand{\catorce}{ 14}
\newcommand{\cuarentacuatro}{ 44}
\newcommand{\veintidos}{22}
\newcommand{\treintatres}{33}
\newcommand{\treintatresR}{\color{red}33}
\newcommand{\cincuentacinco}{55}
\newcommand{\cincuentacincoR}{\color{red}55}
\newcommand{\catorceB}{\color{red} 14}
\newcommand{\quinceB}{\color{red} 15}
\newcommand{\sixteenB}{\color{red} 16}
\newcommand{\Soergelcat}{ {\mathcal D}_{(W,S)}}
\newcommand{\SoergelcatC}{ {\mathcal D}_{(W,S)}^{\comu}}

\newcommand{\tildeSoergelcat}{ \tilde{{\mathcal D}}_{(W,S)}}
\newcommand{\tildeSoergelcatC}{ \tilde{{\mathcal D}}_{(W,S)}^{\comu}}




\newcommand{\II}{I_e}
\newcommand{\shape}{\text{shape}}
\newcommand{\MP}{{\rm Par }_{l,n}}
\newcommand{\FF}{ { \mathbb F}_{\! p}}
\newcommand{\G}{ { \mathcal G}}
\newcommand{\N}{ { \mathbb N}}
\newcommand{\No}{ { \mathbb N}_0}
\newcommand{\NB}{ {{\mathbb{NB}}_n}}
\newcommand{\NBgr}{ {{\mathbb{NB}_n^{gr}}}}
\newcommand{\NBnngr}{ {{\mathbb{NB}_{n-1}^{gr}}}}
\newcommand{\NNB}{ {{\mathbb{NB}}}}
\newcommand{\BRn}{ {\mathbb{B}^{x,y}_n}}
\newcommand{\BRgr}{ {\mathbb{B}^{gr, x,y}_n}}
\newcommand{\BRnn}{ {\mathbb{B}^{x,y}_{n-1}}}
\newcommand{\BRnngr}{ {\mathbb{B}^{gr, x,y}_{n-1}}}
\newcommand{\BRnnn}{ {\mathbb{B}^{y,x}_{n-1}}}
\newcommand{\BRcinco}{ {\mathbb{B}^{x,y}_5}}


\newcommand{\JWk}{ {\mathbf{JW}_{\! k}}}
\newcommand{\JWn}{ {\mathbf{JW}_{\! n}}}
\newcommand{\JWcuatro}{ {\mathbf{JW}_{\! 4}}}
\newcommand{\JWtres}{ {\mathbf{JW}_{\! 3}}}
\newcommand{\JWnn}{ {\mathbf{JW}_{n-1}}}
\newcommand{\JWdos}{ {\mathbf{JW}_{2}}}
\newcommand{\JWdoce}{ {\mathbf{JW}_{\! 12}}}


\newcommand{\TLkQ}{ {\mathbb{TL}_k^{\! \mathbb Q}}}
\newcommand{\TLnQ}{ {\mathbb{TL}_n^{\! \mathbb Q}}}
\newcommand{\TLnk}{ {\mathbb{TL}_n^{\! \Bbbk}}}
\newcommand{\TLKn}{  {\mathbb {TL}}_n^{\!\mathcal K}(q) }
\newcommand{\TLZpn}{  {\mathbb {TL}}_n^{\! {\mathbb Z}_{(p)} }}
\newcommand{\TLZpntwo}{  {\mathbb {TL}}_{N_2}^{\! {\mathbb Z}_{(p)} }}
\newcommand{\TLQpntwo}{  {\mathbb {TL}}_{N_2}^{\! \mathbb  Q  }}


\newcommand{\TLn}{ {\mathbb{TL}_n}}
\newcommand{\TLndiag}{ {\mathbb{TL}_n^{\! diag}}}
\newcommand{\TLnF}{ {\mathbb{TL}_n^{\! {\mathbb F}_p}}}

\newcommand{\TLnZp}{ {\mathbb{TL}_n^{\! {\mathbb Z}_{(p)}}}}

\newcommand{\TLtrestres}{ {\mathbb{TL}_3^{\! {\mathbb F}_3}}}

\newcommand{\Blob}{ {\mathbb{B}}}



\newcommand{\TRnn}{ {\mathbb{TL}_{n-1}}}
\newcommand{\TRn}{ {\mathbb{TL}_{n}}}
\newcommand{\TRcinco}{ {\mathbb{TL}_{5}}}

\newcommand{\yvc}{\Yvcentermath1}
\newcommand{\seq}{{\rm seq}_n}
\newcommand{\Li}{\mathcal{L}}
\newcommand{\K}{\mathcal{K}}
\newcommand{\kkk}{k-1}
\newcommand{\OO}{\mathcal{O}}
\newcommand{\kk}{\mathcal{K}}
\newcommand{\A}{\mathcal{A}}
\newcommand{\Z}{\mathbb{Z}}
\newcommand{\e}{\mathfrak{e}}
%\newcommand{\C}{\mathcal{C}_{\rm{YH}}}
\newcommand{\BB}{\mathcal{B}^{\F}_{l,n}(\kappa)}
\newcommand{\B}{\mathbb{B}}
\newcommand{\BBk}{\mathcal{B}_k}
\newcommand{\BBB}{\mathcal{B}_{n+1}}
\newcommand{\Basis}{\mathcal{C}_n}
\newcommand{\BasisOne}{\mathcal{C}^{1,\K}_n}
\newcommand{\res}{ \textrm{res} }
\newcommand{\q}{\hat{q}}
\newcommand{\m}{\mathfrak{m}}
\newcommand{\Bchico}{\mathcal{B}_{n-1}}
\newcommand{\Comp}{{\mathcal Comp}_n}
\newcommand{\Par}{{\rm Par}_n   }
\newcommand{\ParTwo}{{\rm Par}^{\le 2}_n}
\newcommand{\ParTwoNuno}{{\rm Par}^{\le 2}_{N_2}}
\newcommand{\CC}{ \mathbb C }

\newcommand{\RKLR}{{ \mathcal R}_n }
\newcommand{\RKLRZ}{ {\mathcal R}_{n}^{ \Z_{(p)}}}

\newcommand{\End}{{\rm End}}
\newcommand{\spa}{{\rm span}}
\newcommand{\id}{{\rm id}}
\newcommand{\ind}{{\rm ind}}

\newcommand{\Der}{{\rm Der}}

\newcommand{\gl}{\mathfrak{gl}}
\newcommand\es{\mathbbm{s}}
\newcommand\et{\mathbbm{t}}
\newcommand\eu{\mathbbm{u}}
\newcommand\ev{\mathbbm{v}}
\newcommand\bs{\mathbf{s}}
\newcommand\bt{\mathbf{t}}
\newcommand\bu{\mathbf{u}}
\newcommand\bv{\mathbf{v}}
\newcommand{\s}{\mathfrak{s}}

\newcommand{\V}{\mathfrak{v}}
\newcommand{\T}{  \mathfrak{t}}
\newcommand{\U}{  \mathfrak{u}}
\newcommand{\g}{  \mathfrak{g}}
\newcommand{\bb}{  \mathfrak{b}}
\newcommand\bbs{\mathsf{s}}
\newcommand\bbt{\boldsymbol{\mathsf{t}}}
\newcommand\bi{\boldsymbol{i}}
\newcommand\bn{\boldsymbol{n}}
\newcommand\bj{\boldsymbol{j}}
\newcommand\bbu{\mathsf{u}}
\newcommand\bbv{\mathsf{v}}

\newcommand{\aaa}{\mathfrak{a}}
\newcommand{\bbb}{\mathfrak{b}}
\newcommand{\ccc}{\mathfrak{c}}
\newcommand{\tr}{{\rm \textbf{tr}}}

\newcommand{\ch}{{\rm char}}
\newcommand{\Rad}{{\rm Rad}}

\newcommand{\OnePar}{{ \rm Par}^1_{n}}
\newcommand{\OneParSing}{{ \rm Par}^1_{\overline{n}}}
\newcommand{\MC}{{ {\rm Comp}}_{l,n}}
\newcommand{\MCm}{{ {\rm Comp}}_{l,m}}
\newcommand{\op}{\otimes}
\newcommand{\Si}{\mathfrak{S}}
\newcommand{\std}{{\rm Std}}
\newcommand{\nstd}{{\rm NStd}}
\newcommand{\tab}{{\rm Tab}}
\newcommand{\Snake}{{\rm Snake}}
\newcommand{\inv}{{\rm inv}}
\newcommand{\h}{{h}}
\newcommand{\HH}{ \mathcal{H}_n}
\newcommand{\HHO}{ \mathcal{H}^{\OO}_n}
\newcommand{\HHK}{ \mathcal{H}^{\K}_n}
\newcommand{\HHtwo}{ \mathcal{H}_2}
\newcommand{\HHKtwo}{ \mathcal{H}^{\K}_2}
\newcommand{\HHOtwo}{ \mathcal{H}^{\OO}_2}
\newcommand{\HHKOne}{ \mathcal{H}^{1,\K}_n}

\newcommand{\Hecken}{ {\mathcal{H}_n(q)}}
\newcommand{\HeckenK}{ {\mathcal{H}^{\cal K}_n(q)}}
\newcommand{\R}{ \mathcal{R}_n}


\newcommand{\E}{ {\mathcal E}_n(q)}






\newcommand\bS{\Sigma}
\newcommand\blambda{{\boldsymbol\lambda}}

\newcommand\btau{{\boldsymbol\tau}}
\newcommand\bnu{{\boldsymbol\nu}}
\newcommand\be{\mathbb{E}}
\newcommand\bmu{{\boldsymbol\mu}}
\newcommand{\YL}{{\mathcal Y}^\blambda}


\newcommand{\bT}{\pmb{\mathfrak{t}}}
\newcommand{\Bg}{\pmb{\mathfrak{g}}}

\newcommand{\bTI}{ \bT^{-1} (\bT_{\theta}^{\blambda}(1))}
\newcommand{\bTII}{ \bT^{-1} (\bT_{\theta}^{\blambda}(2))}
\newcommand{\bTj}{ \bT^{-1} (\bT_{\theta}^{\blambda}(j))}
\newcommand{\bTn}{ \bT^{-1} (\bT_{\theta}^{\blambda}(n))}


\newcommand{\Bs}{\pmb{\mathfrak{s}}}

\newcommand{\Ba}{\pmb{\mathfrak{a}}}
\newcommand{\Bb}{\pmb{\mathfrak{b}}}
\newcommand{\Bu}{\pmb{\mathfrak{u}}}
\newcommand{\Bv}{\pmb{\mathfrak{v}}}




%%%%%%%%%%%%%%%%%%%%%%%%%%%%%%%%%%%%%%%%%%%


\newcommand{\trunc}{ {\B} (\blambda ) }
\newcommand{\truncPrime}{ {\mathbb B}_n^{\prime} (\blambda ) }
\newcommand{\truncSing}{ {\mathbb B}_{\bar n} (\overline{\blambda} ) }
\newcommand{\sign}{-}
\newcommand{\UU}{\mathbb{u}}
\newcommand{\VV}{\mathbb{V}}
\newcommand{\UUU}{\mathbb{U}}

\newcommand{\YY}{\mathbb{Y}}
\newcommand{\LL}{\mathbf{L}}
\newcommand{\LLL}{ \textswab{L}}  
\newcommand{\EE}{\mathbb{E}}
\newcommand{\EEE}{ \mathbf{E}}  
\newcommand{\ee}{\mathbf{e}}
\newcommand{\TT}{\mathbb{T}}
\newcommand{\JM}{ \mathbf{JM} }
\newcommand{\Exp}{ {\rm \bf exp} }
\newcommand{\one}{\mathbb{1}}
\newcommand{\sss}{\mathbb{s}}

\newcommand{\greekrho}{\varrho}    

\renewcommand{\baselinestretch}{1}
\newcommand{\botts}[2]{ \mbox{Hom}_{{\mathcal D}} ( \underline{#1}, \underline{#2} ) }
\newcommand{\Std}{{\rm Std}}
%\newcommand{\T}{\mathfrak{t}}
\newcommand{\Yla}{Y_n(\lambda)}
\newcommand{\ide}{e(\bs{i}^\lambda)}


\newcommand{\bip}{ \operatorname{Bip}_1(n)}
\newcommand{\ydl}{Y_n(\lambda)}
\newcommand{\alg}{ \mbox{End}(BS(w))}
\newcommand{\ese}{  {\color{red} s}}
\newcommand{\te}{ {\color{blue} t}}
\newcommand{\al}{ A_{w}}

\newcommand{\X}{ \mathcal{X}}

\newcommand{\ii}{\mathbf{i}}
\newcommand{\jj}{\mathbf{j}}

\newcommand{\blob}{ b_n }
\newcommand{\laba}[1]{ \underline{#1}}
\modulolinenumbers[5]


\newtheorem{theorem}{Theorem}[section]
\newtheorem{lemma}[theorem]{Lemma}
\newtheorem{observation}[theorem]{Observation}
\newtheorem{proposition}[theorem]{Proposition}
\newtheorem{definition}[theorem]{Definition}
\newtheorem{corollary}[theorem]{Corollary}
\newtheorem{example}[theorem]{Example}
\newtheorem{remark}[theorem]{Remark}
\newtheorem{conjecture}[theorem]{Conjecture}
\newtheorem{claim}[theorem]{Claim}
\newtheorem{conditions}[theorem]{Conditions}
\newtheorem{algorithm}[theorem]{Algorithm}
\newtheorem{notation}[theorem]{Notation}
\newenvironment{dem}{\noindent \textit{Proof:} }{\quad \hfill $\square$}


\numberwithin{equation}{section}



\newgray{plomoclaro}{.90}
\newgray{plomooscuro}{.70}
\newgray{blanco}{1}

\begin{document}
\Yvcentermath1
\sidecaptionvpos{figure}{lc}

%\begin{frontmatter}




\title{Seminormal forms for the Temperley-Lieb algebra. }


\author{
  \, Katherine Orme\~no Bast\'ias{\thanks{Supported in part by ANID beca
doctorado nacional 21202166
  }} \, and  Steen Ryom-Hansen{\thanks{Supported in part by FONDECYT grant 1171379  }}}



\date{\vspace{-5ex}}
\maketitle
\begin{abstract}
Let $\TLnQ$ be the rational Temperley-Lieb algebra, with loop parameter $ 2 $. 
In the first part of the paper we study the seminormal idempotents $ E_{\T} $ for $\TLnQ$ for 
$ \T $ running over two-column standard tableaux. 
Our main result is here a concrete combinatorial construction of $ E_{\T} $ 
using Jones-Wenzl idempotents $ \JWk $ for $\TLkQ$ where $ k \le n $. 

In the second part of the paper
we consider 
the Temperley-Lieb algebra $\TLnF$ over the finite field $ \FF$, where $ p>2$.
The KLR-approach to $\TLnF$ gives rise to an action 
of a symmetric group $ \Si_m $ on $\TLnF$, for some $ m < n $. 
We show that the $ E_{\T} $'s from the first part of the paper are 
simultaneous eigenvectors for the associated Jucys-Murphy elements for $ \Si_m $. This leads to 
a KLR-interpretation of the $p$-Jones-Wenzl idempotent 
$ ^{p}\!\JWn $ for $\TLnF$, that was introduced recently by Burull, Libedinsky and Sentinelli. 



\end{abstract}

\section{Introduction}
In the present paper we introduce a new perspective on both the semisimple and the non-semisimple
representation theory of the Temperley-Lieb algebra $ \TLn$. Some of the ingredients of
this new perspective are very classical and well-known, whereas other ingredients
are based on ideas developed within the last few couple of years. The unifying element of these ingredients
are {\it seminormal forms}. 

\medskip
In the representation theory of the symmetric group $ \Si_n$, 
the seminormal form has a long history 
going back to
Young's papers a century ago. A milestone in this history was 
Murphy's discovery in the eighties, 
realizing the {\it seminormal form} as common eigenvectors for the Jucys-Murphy elements
of the symmetric group. 
Later on, Mathas showed that Murphy's results hold in the 
general framework of a {\it cellular algebra} $ \A $ endowed with a {\it family of $ \JM$-elements}.
In the present paper we show that $ \TLn$ fits into this framework, with
cellular structure given by the diagram basis and $ \JM $-elements induced
from the $ \JM $-element from the symmetric group. 

\medskip
In Mathas' framework there is a dichotomy between the 'separated' and the 'unseparated' cases 
and our treatment of $ \TLn$ follows this dichotomy,
with $ \TLnQ$ corresponding to the separated case and $  \TLnF$ to the unseparated case. 
In the separated case, $ \A $ is semisimple and 
there exists a {\it complete} set of {\it orthogonal idempotents}
$ \{ \EE_{\T} \, | \, \T \in \std(\Lambda) \} $ that are common eigenvectors for the $ \JM $-elements. 
In the unseparated case however, $ \A $ is not semisimple and the $ \EE_{\T} $'s are undefined, but
summing over a {\it $p$-class $ [\T]$ of standard tableaux} we have well defined
{\it class idempotents}
in $ \A $, given by 
\begin{equation}
\EE_{[\T]} := \sum_{ \s \in [\T]  } \EE_{\s}  
\end{equation}


\medskip
In general one is especially interested in the {\it primitive idempotents }in $ \A $, since
they correspond to the {\it indecomposable projective modules} for $ \A$, that carry
essential relevant information about the representation theory of $\A$, such as decomposition
numbers, etc. The primitive idempotent for the projective cover of the trivial $ \TLnF$-module,
the {\it $p$-Jones-Wenzl idempotent} $\,  ^{p}\!\JWn $, 
was determined recently
by Burull, Libedinsky and Sentinelli in \cite{BLS}, via a recursive construction involving the
expansion of $ n+1 $ in base $ p$. 
The class idempotents $ \EE_{[\T]} $
are not always primitive, 
but in our final Theorem \ref{finalTheo}
we show that $ ^{p}\!\JWn $ can in fact be built from
the class idempotents, via a  
recursive construction in KLR-theory. 


\medskip

In the separated case, our main results are  
Theorem \ref{mentionedabove} and Corollary \ref{finalcorsection2} that together give a 
realization of the idempotents $ \EE_{\T} $ in terms 
of a concrete diagrammatic construction involving 
{\it Jones-Wenzl} idempotents
$ \JWk $ for $\TLkQ$ where $ k \le n $. A key ingredient for the proof of these results is
contained in our Theorem \ref{YSFfirst}, 
giving a surprisingly close connection between the Jones-Wenzl idempotents
and the seminormal form, in which 
the well-known recursive formula for $ \JWn$
\begin{equation}
  \raisebox{-.35\height}{\includegraphics[scale=0.7]{dib9.pdf}}
\end{equation}
becomes exactly the classical formula for the action of the symmetric group on the seminormal form,
see Corollary \ref{YSFsecond}. 


\medskip
Our treatment of the unseparated case $ \TLnF$ relies heavily on the KLR-algebra (Khovanov-Lauda-Rouquier)
approach to $ \TLnF$. The KLR-algebra was introduced independently
by Khovanov-Lauda and Rouquier in
\cite{KhovanovLauda} and \cite{Rouquier}, 
in order to categorify the positive part of the type
$ A$ quantum
group. In seminal papers by Brundan-Kleshchev and Rouquier, see \cite{brundan-klesc}
and \cite{Rouquier}, an isomorphism $  \R \cong  \FF \Si_n  $ was established (in fact 
in the greater generality of cyclotomic Hecke algebras).  
Hu and Mathas gave in \cite{hu-mathas2} a new
simpler proof of this isomorphism
{\it using seminormal forms}, and via this they were able to lift it to an
isomorphism $\RKLRZ \cong   \Z_{(p)} \Si_n   $, where $ \RKLRZ $
is an integral version of $ \RKLR $, defined over the localization $ \Z_{(p)} $ of $ \Z $ at the prime $p$.
This isomorphism and its proof is important to us. 
It firstly induces an isomorphism $ \RKLRZ \!  / {\cal I }_n \cong  \TLnZp  $ where $ {\cal I }_n $
is an ideal of $ \RKLRZ $, generalizing a result by D. Plaza and the second author, see
Theorem \ref{steendavid} and
\cite{PlazaRyom}.


\medskip
Under the isomorphism $  \RKLRZ \! /  {\cal I }_n   \cong   \TLnZp $, the KLR-generator
$ e(\ii) $ for $ \ii =(0,-1,-2, \ldots, -n+1) $ corresponds to the class idempotent $ \ee= \EE_{[ \T_n] } $
where $ \T_n $ is the unique one-column standard tableau of length $ n $, and so we consider the
idempotent truncated subalgebra $ \ee \TLnZp \ee $ of $ \TLnZp $. It contains block intertwining
elements $ \UUU_i $, that are represented diagrammatically by 'diamonds' as follows, for $ p=3$. 
\begin{align}
& \UUU_1 = \raisebox{-.5\height}{\includegraphics[scale=0.75]{dib76.pdf}} & 
& \UUU_2 = \raisebox{-.5\height}{\includegraphics[scale=0.75]{dib77.pdf}} 
\end{align} 
Similar elements have been already been considered before for example
in \cite{KMR}, \cite{LiPl}, \cite{Lo} and \cite{LPR},
but only in the context of the original KLR-algebra $ \R $ defined over a field.

\medskip
Our main results in the 
unseparated case 
revolve around the
action of the $ \UUU_i $'s on 
the seminormal form for $  \TLnQ$, given by Hu-Mathas' isomorphism. The key result 
is our Theorem \ref{YSFthird},
which establishes a formula reminiscent of the classical formula for the
action of the symmetric group on 
Young's seminormal form.
The fact that such a formula is possible
in the representation theory of $ \TLnZp $, and hence also 
of 
$ \TLnF$, is highly surprising,
and could surely not have been conceived before
the introduction of KLR-algebra in representation theory, in particular before Hu-Mathas'
proof of the isomorphism $\RKLRZ \cong   \Z_{(p)} \Si_n   $. 


\medskip
As a first consequence of Theorem \ref{YSFthird} we obtain in
Corollary \ref{thereisaninjection}
an injection
$\iota_{KLR}:   {\mathbb {TL}}_{n_2}^{\! {\mathbb Z}_{(p)} } \rightarrow \TLZpn $ for 
$ n_2 $ a concretely given number $ n_2 < n$. 
The small Temperley-Lieb algebra $ {\mathbb {TL}}_{n_2}^{\! {\mathbb Z}_{(p)} } $
is of course endowed with its own
$ \JM$-elements $ \LLL_i $, with associated seminormal form idempotents $ \{ \EEE_{s} \} $
that apriori are unrelated to the original seminormal form idempotents $  \{ \EE_{\T} \} $
for $\TLnQ$. But nevertheless we establish in Theorem \ref{indutionseminormalKLR}, 
Corollary \ref{idempotentJMcor} and
Corollary \ref{idempotentJMcordos}
another series of surprising facts, showing that the
$   \EE_{\T}  $'s are eigenvectors for the $ \LLL_i $'s, and that the product
$  \EEE_{\s} \EE_{\T} $ is given by a simple formula. 

\medskip

These last results are the essence of our recursive construction of $   ^{p}\!\JWn $, 
We obtain a chain 
\begin{equation} 
0 \subseteq   {\mathbb {TL}}_{n^{k-1}_2}^{\! {\mathbb Z}_{(p)} } \subseteq
  {\mathbb {TL}}_{n^{k-2}_2}^{\! {\mathbb Z}_{(p)} } \subseteq  \cdots \subseteq  
  {\mathbb {TL}}_{n_2^0}^{\! {\mathbb Z}_{(p)} }  =
  {\mathbb {TL}}_{n}^{\! {\mathbb Z}_{(p)} }
\end{equation}  
of subalgebras, and let $ \EEE_{[\T_{n^{i}_2}]} $ be the class idempotent for 
$ {\mathbb {TL}}_{n_2^i}^{\! {\mathbb Z}_{(p)} }  $, considered as element of 
$   {\mathbb {TL}}_{n}^{\! {\mathbb Z}_{(p)} } $. In our final Theorem \ref{finalTheo}
we show that $\,  ^{p}\!\JWn $ is the product of these $ \EEE_{[\T_{n^{i}_2}]} $'s. 

\medskip
The organization of our paper is as follows. In section \ref{basic notions}
we fix notation and recall some basic results concerning the objects that are studied in the paper, that is
the Temperley-Lieb algebra $\TLn$, the group algebra of the symmetric group, and so on.
In section \ref{separated case}, we construct for each two-column standard tableau $ \T $ an
element $ \EE_{\T}^{\prime} \in \TLnQ$ and show that $  \EE_{\T}^{\prime} = \EE_{\T}$.
In section \ref{The unseparated case}, we initiate our study of the unseparated case. We recall
the construction of $\,  ^{p}\!\JWn $ and give various examples that motivate and illustrate the main
results of the last section. In section \ref{The integral KLR-algebra}, we recall the KLR algebra
$ \R $ and in particular Hu and Mathas' integral version $\RKLRZ $ of $ \R$. We also recall the basic
ingredients in Hu and Mathas' proof of the isomorphism $\RKLRZ \cong   \Z_{(p)} \Si_n   $. They
involve a series of formulas for the action of the KLR-generators on the seminormal basis.


In section \ref{Seminormal form for}, we prove our main results concerning the unseparated
case,
starting with Theorem \ref{YSFthird}, which describes the action of the $ \UUU_i $ on the seminormal
basis. We found the results of this section with 
considerable aid from the
Sage computer system. The proof of Theorem \ref{YSFthird} is a lengthy calculation involving Hu-Mathas' formulas. 
We firmly believe that it is possible to generalize all our results of the present paper to hold 
for the 
Tempeley-Lieb algebra 
at loop parameter $ q + q^{-1} $ and have in fact already made some progress in this direction.
On the other hand,
we also realized that the generalization of Theorem \ref{YSFthird} to loop parameter
$ q+ q^{-1} $ involves substantially more
calculations
and we therefore decided to develop
our results firstly in the loop parameter 2 setting.

\medskip
We thank N. Libedinsky, D. Plaza and S. Griffeth for useful comments on a previous version
of this work. We specially thank G. Burull for explaining us the construction of $ ^{p}\!\JWn $.

\section{Basic notions}\label{basic notions}
In this section we fix notation and recall the basic notions related to the
symmetric group and 
Temperley-Lieb algebra and their representation theories. 



\medskip
Let $ n $ be a positive integer. A partition of $ n $ of {\it length} $ k $ is a weakly
decreasing sequence $ \lambda = (\lambda_1, \lambda_2,
\ldots, \lambda_k ) $ of positive integers with total sum $  n$. 
The set of partitions of $ n $ is denoted
$ \Par$. A partition $ \lambda  = (\lambda_1, \lambda_2,
\ldots, \lambda_k )\in \Par $ is represented diagrammatically by its {\it Young diagram}, which consists 
of $ k $, left aligned, rows of nodes with $ \lambda_1 $ nodes in the first row, $ \lambda_2 $
nodes in the second row, and so on. For example, 
$ \lambda = (7,4,2^2) = (7,4,2,2) \in {\rm Par}_{15} $ is represented by 
the following Young diagram
\begin{equation}\label{dib47}
\lambda = \raisebox{-.5\height}{\includegraphics[scale=0.7]{dib47.pdf}}  
\end{equation}
As in \eqref{dib47}, we shall identify $ \lambda \in \Par$ with its Young diagram.
The set of partitions of $n$ whose Young diagram has less than two columns 
is denoted $ {\rm Par}_n^{\le k } $. For $ \lambda, \mu \in \Par $ we write 
$ \lambda \trianglelefteq \mu $
if $ \lambda_1 \le \mu_1 $, $ \lambda_1 + \lambda_2 \le \mu_1 + \mu_2 $,
$ \lambda_1 + \lambda_2 + \lambda_3 \le \mu_1 + \mu_2 +\mu_3 $,
and so on. This is the
{\it dominance order} on $ \Par$. For $ \lambda \in \Par $, a $ \lambda$-tableau $ \T $ is a filling of
the nodes of the Young diagram of $ \lambda$ using the numbers
from the set $ \{ 1,2,\ldots, n \}$, with each number occurring 
once. A $ \lambda$-tableau $ \T $ is said to be {\it standard} if the numbers from
$ \{ 1,2,\ldots, n \}$ appear increasingly from left to right along the rows of $ \lambda $, and increasing from top to
bottom along the columns of $ \lambda $. The set of standard $ \lambda$-tableaux is denoted $ \std(\lambda)$. 
Here is an example of a standard $ \lambda$-tableau $ \T $, for $ \lambda $ as in \eqref{dib47}
 \begin{equation}\label{dib48}
\T = \raisebox{-.5\height}{\includegraphics[scale=0.7]{dib48.pdf}}  
\end{equation}
 We define $ \shape(\T) = \lambda $ if $ \T \in \Std(\lambda) $. For $ \T $ any standard tableau we define
 $ \T |_{\le m } $ to be the tableau obtained from $ \T $ by deleting all the nodes that contain numbers
 strictly larger than $ m $. 
 We then extend the dominance order to standard tableaux
 via $ \s \trianglelefteq \T $ if $ \shape( \s |_{\le m } ) \trianglelefteq \shape( \T |_{\le m } ) $ for
 all $ m=1,2,\ldots, n$. 

\medskip
For $\lambda \in \Par$ we define $ \T^{\lambda} \in  \Std(\lambda) $ as the {\it row-reading} tableau,
and similarly we define $ \T_{\lambda} \in  \Std(\lambda) $ as the {\it column-reading} tableau.
 When restricted to $ \std(\lambda) $, we have that $ \T^{\lambda} $ is the unique maximal tableau
 and $ \T_{\lambda} $ is the unique minimal tableau, with respect to $\trianglelefteq $.
For example, for $ \lambda $ as in \eqref{dib47} we have 
 \begin{equation}\label{dib48second}
   \T^{\lambda} = \raisebox{-.5\height}{\includegraphics[scale=0.7]{dib49.pdf}}, \, \, \, \, \, \, \, \, 
\T_{\lambda} = \raisebox{-.5\height}{\includegraphics[scale=0.7]{dib50.pdf}}     
\end{equation}




\medskip
The Temperley-Lieb algebra was introduced in the seventies from motivations in statistical mechanics.
In this paper we shall use the variation of the Temperley-Lieb algebra that has loop parameter
equal to $2$. It is defined as follows.  
\begin{definition}\label{defTL}
  The Temperley-Lieb algebra $ \TLn$ is the associative unitary $ \Z $-algebra on 
  generators $ \UU_1, \UU_2, \ldots, \UU_{n-1} $ subject to the relations
	\begin{align}
\label{eq oneTL}	\UU_i^2	& =   2\UU_i,  & &  \mbox{if }  1\leq i < n \\
\label{eq twoTL}	\UU_i\UU_j\UU_i & =\UU_i,      &  & \mbox{if } |i-j|=1   \\
\label{eq threeTL}	\UU_i\UU_j& = \UU_j\UU_i,      & &   \mbox{if } |i-j|>1
	\end{align}
\end{definition}
For $ \Bbbk $ a commutative ring containing a nonzero element $ \xi $, we shall also consider
the {\it specialized version} $ \TLnk$ of $ \TLn $, defined as 
\begin{equation}
 \TLnk:=  \TLn \otimes_{\Z} \Bbbk 
\end{equation}  
Here we are mostly interested in the cases 
where 
$  \Bbbk $ is the rational field $ \QQ $, 
the finite field with
$ p $ elements $ \FF $ or the localization $ \Z_{(p)} $ of $ \Z $ at the prime $ p$, and $ \xi = 1$. 
The corresponding Temperley-Lieb algebras are  
$ \TLnQ $, $  \TLnF $ and $ \TLZpn$. 


\medskip
A well-known and important feature of $ \TLn $ is the fact that it is a diagram algebra. Concretely, 
$ \TLn $ is isomorphic to the diagrammatically defined algebra $ \TLndiag $ with basis given by {\it non-crossing
planar matchings} of $ n $ northern points of a(n invisible) rectangle with $ n $ southern points of the rectangle. 
Here are three examples for $ n= 5$. 
\begin{equation}\label{dib1}
\raisebox{-.5\height}{\includegraphics[scale=0.75]{dib1.pdf}}  
\end{equation}
We refer to such matchings as {\it Temperley-Lieb diagrams}. 
For two Temperley-Lieb diagrams $ D_1 $ and $ D_2 $ the product $ D_1 D_2 $ in $ \TLndiag $
is given by concatenation
with $ D_1 $ on top of $ D_2 $. For example, choosing $ D_1 $ and $ D_2 $ as the first two diagrams in
\eqref{dib1} we have that
\begin{equation}\label{dib2}
D_1 D_2 \, \, =\, \,  \raisebox{-.4\height}{\includegraphics[scale=0.75]{dib2.pdf}}   \, \, =\, \,D_3
\end{equation}
where $ D_3 $ is the third diagram of \eqref{dib1}. This concatenation product may give rise to
diagrams with internal loops. Each internal loop is removed from
the diagram, and the resulting diagram is multiplied by the scalar $ 2 \in \Z $.
For example, if $ D_1 $ and $ D_3 $ are as above, we have that 
\begin{equation}\label{dib3}
D_1 D_3 \, \, =\, \,  \raisebox{-.4\height}{\includegraphics[scale=0.75]{dib3.pdf}}   \, \, =\, \, 2D_3
\end{equation}
The isomorphism between $ \TLn$ and $ \TLndiag $ is given by 
\begin{equation}\label{dib4} 
  \one \, \, \, \,  \mapsto\, \, \,  \, \raisebox{-.35\height}{\includegraphics[scale=0.75]{dib5.pdf}}, \, \, \, 
    \UU_i  \, \, \mapsto\, \,  \raisebox{-.35\height}{\includegraphics[scale=0.75]{dib4.pdf}} 
\end{equation}
where $ \one $ is the one-element of $ \TLn$.  
From now on we shall identify $ \TLn$ with $ \TLndiag$ 
via this isomorphism. There is similar
isomorphism for the specialized Temperley-Lieb algebra $ \TLnk $
and here we shall also identify $ \TLnk $ with the corresponding diagrammatic algebra, defined over $ \Bbbk $. 

\medskip
Throughout the paper we shall be interested in the {\it Jones-Wenzl idempotent}
$\JWn $ of $ \TLnQ $. It is the unique nonzero idempotent of 
$ \TLnQ $ satisfying
\begin{equation}\label{definingpropertyJW}
 \UU_i  \JWn = \JWn \UU_i  = 0  \mbox{ for all } i 
\end{equation}
We use the following standard diagrammatic notation
for $\JWn $ 
\begin{equation}
\JWn = 
  \raisebox{-.35\height}{\includegraphics[scale=0.7]{dib6.pdf}} \in \TLnQ
\end{equation}
For example we have that 
\begin{equation}\label{dib7}
   \raisebox{-.35\height}{\includegraphics[scale=0.7]{dib7.pdf}} 
 \end{equation}
\begin{equation}\label{dib8}
   \raisebox{-.35\height}{\includegraphics[scale=0.7]{dib8.pdf}} 
 \end{equation}
In general, as one already observes in \eqref{dib7} and \eqref{dib8}, 
when expanding $ \JWn  $ in terms of the diagram basis for
$  \TLnQ $, 
the coefficient of $ \one $ is 1, whereas the other coefficients in general
are rational numbers $ \dfrac{a}{b} $ with 
non-trivial denominator. These denominators $ b $ prohibit the
specialization of $ \JWn  $ to fields $ \Bbbk $ whose characteristic $ p $ divides $ b $. 

\medskip
On the other hand, 
we always have $ \mathbf{JW}_n^{\ast} = \JWn $ where $ \ast $ is the antiautomorphism of $ \TLnQ $
given by reflection
through a horizontal axis, and similarly $ \JWn $ is symmetric with respect to reflection through
a vertical axis.
These properties can be observed in \eqref{dib7} and \eqref{dib8}. 

\medskip
For general $ n $ there is no known closed formula for calculating 
the coefficients of $ \JWn  $ in terms of the diagram basis for $ \TLnQ$; all known formulas are recursive.
We shall need the following recursive formula that goes back to
Jones and Wenzl, see \cite{Jo} and \cite{Wenzl}. 


\begin{equation}\label{goesbackto}
  \raisebox{-.35\height}{\includegraphics[scale=0.7]{dib9.pdf}}
\end{equation}
Combining it with \eqref{eq oneTL} 
we obtain the following well known formula 
\begin{equation}\label{closing}
  \raisebox{-.35\height}{\includegraphics[scale=0.7]{dib10.pdf}}
\end{equation}
which can be repeated to arrive at 
\begin{equation}\label{basicelements}
  \raisebox{-.35\height}{\includegraphics[scale=0.7]{dib11.pdf}}
\end{equation}

Using \eqref{goesbackto} one proves that
for $ m<n $ we have 
$ \mathbf{JW}_{m} \mathbf{JW}_{n} =
\mathbf{JW}_{n} $, or diagrammatically 
\begin{equation}\label{absorbtion}
  \raisebox{-.35\height}{\includegraphics[scale=0.7]{dib27.pdf}}
\end{equation}



We next recall the basic elements of the representation theory of $ \TLn$, using the language
of {\it cellular algebras}. The notion of cellular algebras was introduced by Graham and Lehrer in
\cite{GL} and in fact $ \TLn $ was one of their motivating examples. 

\begin{definition}\label{cellular}
  Let $ A $ be an associative $ R$-algebra with unit, where $ R $ is a commutative ring. A cell datum for
  $ A $ is a triple $ (\Lambda, T, C) $ where $ \Lambda = (\Lambda, >) $ is a finite poset, 
  $ T$ is a function from $ \Lambda $ to finite sets and $ C $ is an injective function 
\begin{equation}  
  C: \prod_{\lambda \in \Lambda} T(\lambda) \times T(\lambda) \rightarrow A,  (s,t) \in T(\lambda ) \times
T(\lambda )  \mapsto C_{s t}^{\lambda} 
\end{equation}    
These data should satisfy that
\begin{enumerate}
\item  The image of $ C $, that is $im \, C= \{ C_{st}^{\lambda}  \, | \, (s,t) \in
 T(\lambda) \times T(\lambda), \lambda \in \Lambda \} $, is an $ R $-basis for $ A $.
\item For any $ a \in A $ we have
\begin{equation}\label{mult}  C_{st}^{\lambda}a 
  = \sum_{v \in T(\lambda)} r_{tva}C_{sv}^{\lambda}  \, \mbox{ mod }  A^{\lambda}
\end{equation}  
where $ A^{\lambda} $ is the $ R$-submodule of $ A $ spanned by $ \{ C_{st}^{\mu}  \, | \, \mu > \lambda \}$.
\item The $R$-linear map $ \ast$ of $ A$ given by $ C_{st}^{\lambda} \mapsto C_{ts}^{\lambda} $ is an algebra 
anti-isomorphism of $A$. 
\end{enumerate}
\end{definition}
An algebra endowed with a cell datum $ (\Lambda, T, C) $ is called a cellular algebra, with
cellular basis $ im \, C$. 



\medskip
$\TLn $ is an example of a cellular algebra. To see this one lets $ \Lambda $ be the set of two-column
integer partitions $ \ParTwo $,  
endowed with the usual dominance order,
and for $ \lambda \in \ParTwo $ one lets $ T(\lambda) $ be the set of standard $\lambda$-tableaux
$ \std(\lambda) $. 
To explain $ C $, one first constructs for $ \lambda \in \ParTwo $ and $ \T \in \std(\lambda) $ 
a {\it Temperley-Lieb half-diagram} $ C_{\T}^{\lambda}$ for $ \TLn$ as follows. Going through the numbers
$ \{1, 2\, \ldots, n \} $ in increasing order, one raises for any $ i $ occurring in the first column of $ \T $
a vertical line from the $ i $'th lower position of the rectangle 
and for any $ i $ occurring in the second column of $ \T $, one joins the $i$'th lower position with the 
top end of the first vacant line to the left, always avoiding line crossings. Here is an example for
$ \lambda = (2^4,1^3) \in {\rm Par}_{11} $. 
\begin{equation}
  \T :=  \raisebox{-.5\height}{\includegraphics[scale=0.7]{dib12.pdf}}
  \mapsto  \, \, \, \,   C_{\T}^{\lambda} = \,    \raisebox{-.3\height}{\includegraphics[scale=0.7]{dib13.pdf}}
\end{equation}
For a pair of standard $\lambda$-tableaux $ (\s, \T) $, one then defines $ C_{\s \T}^{\lambda} $ as the
diagram obtained from $  C_{\s}^{\lambda} $ and $ C_{\T}^{\lambda}  $ by reflecting
$ C_{\s}^{\lambda} $ horizontally and concatenating below with $C_{\T}^{\lambda} $. 
Here is an example. 
\begin{equation}\label{diagrambasisTL}
 (\s,  \T) :=  \raisebox{-.5\height}{\includegraphics[scale=0.7]{dib14.pdf}}
  \mapsto  \, \, \, \,   C_{\s\T}^{\lambda} = \,    \raisebox{-.5\height}{\includegraphics[scale=0.7]{dib15.pdf}}
\end{equation}
Using the multiplication rules explained in \eqref{dib2} and \eqref{dib3}, 
one now checks that $ \TLn $ indeed is a cellular algebra over $ \Z$, with the ingredients just introduced,
and similarly $ \TLnk $ is a cellular algebra over $ \Bbbk $. 



\medskip
In general, for a cellular algebra $ A $ there is a family of {\it cell modules}
$ \{ \Delta(\lambda) \, | \, \lambda \in \Lambda \} $ that play a key role
when studying the representation theory of $ A $. To define $ \Delta(\lambda) $ one chooses an
arbitrary $ s_0 \in T(\lambda) $ and sets
\begin{equation}
\Delta(\lambda) := {\rm span}_R \{ C_{ s_0 t}^{\lambda} \, | \, t \in T(\lambda) \}
\end{equation}
The action of $ A $ on $ \Delta(\lambda) $ is given by
$ C_{ s_0 t}^{\lambda} a  =  \sum_{v \in T(\lambda)} r_{tva}C_{s_0v}^{\lambda} $ where 
$ r_{tva} \in R$ is the scalar that appears in \eqref{mult}. Note that $ \Delta(\lambda) $ is a right module.
We shall sometimes write $  \Delta^R(\lambda) $ for $  \Delta(\lambda) $ to indicate the dependence on
the ground ring $ R $. 

\medskip
In the case of $ \TLn$, we identify for $ \s_0, \T \in \std(\lambda) $ the $ \Delta(\lambda) $ basis element 
$ C_{\s_0 \T}^{\lambda} $ with the half-diagram $ C_{ \T}^{\lambda} $. Under this identification, 
for a Temperley-Lieb diagram $ D $ we have that $ C_{ \T}^{\lambda} D $ 
is the concatenation with $ C_{ \T}^{\lambda}$ on top of $ D $, where internal loops are removed
by multiplying by $ 2$, and where half-diagrams that do not belong to $ \{ C_{ \T}^{\lambda} \,| \, \T
\in \std(\lambda) \} $ are set equal to zero.

\medskip
In the present paper, Jucys-Murphy elements play an important role. Their key properties
were developed in Murphy's papers in the eighties,
see \cite{Murphy2}, \cite{Murphy}, \cite{Murphy1}. 
These properties were formalized by Mathas as follows, see \cite{Mat-So}. 
\begin{definition}\label{JM}
  Let $ A $ be a cellular algebra over $ R $ with triple $ (\Lambda, T, C) $, and suppose that for 
  each $ \lambda \in \Lambda $ there is a poset structure on $ T(\lambda) $ with order relation $ <$.
  Then a family of elements $ \{L_1, L_2, \ldots, L_M\} $
  is called a set of $\JM$-elements for $ A $ if it satisfies the following conditions. 
\begin{enumerate}
\item The $L_i  $'s commute and satisfy $ L_i^{\ast} = L_i$.
\item For each $ t \in T(\Lambda) := \coprod_{\lambda \in \Lambda} T(\lambda) $ there
  is a function $ c_{t}: \{1, 2, \ldots, M\} \rightarrow R $ such that the following
  triangularity formula holds in $ \Delta(\lambda) $
\begin{equation}  
C^{\lambda}_t L_i = c_t(i) C^{\lambda}_t + \sum_{s\in T(\lambda), s>t } a_s C_s^{\lambda}
\end{equation}  
for some $ a_s \in R$. 
\end{enumerate}
\end{definition}
If $ \{ L_1, L_2, \ldots, L_M \} $ is a family of $\JM$-elements for $ A $, then the
corresponding functions $ c_{t}$ are called {\it content functions}. 

\medskip
$\JM$-elements were first constructed for the group algebra of the symmetric group,
and
from these one obtains $\JM$-elements for $ \TLn$,
as we shall shortly see.

\medskip
Let $ \Si_n $ be the symmetric group of bijections $ \{1,2, \ldots, n \} $. For $ \sigma \in \Si_n $ and
$ i \in \{1,2, \ldots, n \} $ we write $ (i)\sigma $ for the image of $i $ under $ \sigma$. This notation
reflects our general preference for right actions and right modules over left actions and left modules.
We shall use standard cycle notation
for elements of $ \Si_n $, that is $ \sigma = (i_1, i_2, \ldots, i_k ) $ is the element of $ \Si_n $
defined by $ (i_1)\sigma = i_2, (i_2) \sigma = i_3, \ldots, (i_k)\sigma = i_1 $.
For elements $ \sigma_1, \sigma_2 \in \Si_n $ the product $ \sigma_1 \sigma_2 \in \Si_n $ is given by
$ (i) (\sigma_1 \sigma_2) = ((i) \sigma_1 ) \sigma_2 $.

\medskip
Let 
$ \{ L_1, L_2, \ldots, L_n \}  \subseteq  \Z \Si_n \,  {(\mbox{or }} \Bbbk \Si_n) $ be defined by 
\begin{equation}\label{defJM}
L_1 := 0, \mbox{ and } L_i := (1,i) + (2,i) + \ldots + (i-1, i ) \mbox{ for } i = 2,3, \ldots, n 
\end{equation}  
Define moreover for $\T \in \std(\lambda) $ the function 
$ c_{\T}:  \{1,2,\ldots, n\} \rightarrow \Z \, {(\mbox{or }} \Bbbk) $ by
\begin{equation}\label{contentdef}
 c_{\T}(i) := c-r  \mbox{ for } \T[r,c] = i
\end{equation}
where $ \T[r,c] $ is the number that appears in the $ r$'th row of $ \T $, counted from top to bottom,
and in the $ c$'th column of $ \T$, counted from the left to right. 


\medskip
$ \Si_n $ is a Coxeter group on the {\it simple transpositions} $ s_i := (i,i+1) $. 
We need the following well known fundamental Lemma, which is easily verified. 
\begin{lemma}\label{wellknownfunda}
There is a 
surjection $ \Phi: \Z \Si_n \rightarrow \TLn $, given by $ s_i \mapsto \UU_i -\one$.
The kernel of $ \Phi $ is the ideal in $  \Z \Si_n  $ generated by
$ s_1 s_2 s_1 +s_1 s_2 + s_2 s_1  +s_1 + s_2 + 1 $. 
A similar statement holds over $ \Bbbk $. 
 \end{lemma}  

Let $ \LL_i := \Phi(L_i) $. We represent  $   \LL_i $ diagrammatically as follows
\begin{equation}
\LL_i =  \raisebox{-.35\height}{\includegraphics[scale=0.7]{dib36.pdf}}
\end{equation}


We now have the following key result. 
\begin{theorem}\label{followingkeyresult}
$ \{ \LL_1,    \LL_2,   \ldots,  \LL_n \} $ 
is a family of $\JM$-elements for $ \TLn$ with respect to the content functions
$ c_{\T}$, defined in \eqref{contentdef}. 
\end{theorem}
\begin{dem}
  $ \{ L_1, L_2, \ldots, L_n \} $ is known to be a family of $\JM$-elements
  for the cellular structure on $ \Z \Si_n $ given by the specialization $ q = 1 $ of
  Murphy's standard basis for the Hecke algebra, see \cite{Mat} and \cite{Murphy1}. 
  On the other hand, $ \Phi: \Z \Si_n  \rightarrow \TLn $ maps the standard basis cellular structure
  on $ \Z \Si_n $ to the diagram basis cellular structure on $ \TLn $ and therefore
  $ \{ \LL_1,    \LL_2,   \ldots,  \LL_n \} $ is a family of $\JM$-elements
  for $ \TLn$, as claimed. For more details one should consult \cite{Orm}. 
\end{dem}  
\bibliographystyle{myalpha}
 
\bibliography{mybibfile}

\section{The separated case}\label{separated case}
In this section we consider the rational Temperley-Lieb algebra $\TLnQ $.
The ground ring for $ \TLnQ $ is $ \QQ $ which implies that for
two-column partitions $ \lambda $ and $ \mu $ and for 
standard tableaux
$ \s \in \std(\lambda) $ and $ \T \in \std(\mu) $ we have that 
\begin{equation}\label{separationcondition}
c_{\s}(i) = c_{\T}(i) \mbox{ for } i= 1, 2, \ldots, n \Longrightarrow \s= \T
\end{equation}
In other words, the {\it separation condition} in \cite{Mat-So} is fulfilled and so
$ \TLnQ$ is semisimple. The separation condition also implies that
the simultaneous action of the $ \LL_i$'s
on $  \TLnQ $ via right multiplication is diagonalizable
with eigenvalues given by the $ c_{\T}(i)$'s, and similarly for the left action.
Moreover, under the separation condition we have the
following expression for the idempotent projector $ \EE_{\T } $ for the common eigenvector for
all the $ \LL_i$'s with eigenvalues $ c_{\T}(i)$.
\begin{equation}\label{IdempotentHecke1}
  \EE_{\T} = \prod_{c \in {\cal C}} \, \, \prod_{\substack{i=1, \ldots, n\\ c \neq c_{\T}(i)} } \dfrac{\LL_i-c}{c_{\T}(i)-c}
  \in \TLnQ
\end{equation}
where $ {\cal C } $ is the set of contents for standard tableaux of two-column partitions, that is 
\begin{equation}\label{IdempotentHecke2}
{\cal C }:= \{ c_{\T}(i) \, | \,  i=1,2, \ldots, n \mbox{ and } \T \in \std(\ParTwo) \} \, \, \mbox{where} \, \, \, 
\std(\ParTwo) := \bigcup_{\lambda \in \ParTwo} \std(\lambda)
\end{equation}
  With $ \T $ running over $\std(\ParTwo)$, the $ \EE_{\T}$'s form a 
complete set of orthogonal primitive idempotents for $ \TLnQ $, that is we have 
\begin{equation}\label{IdempotentHecke3}
  \one = \sum_{\T \in \std(\ParTwo)} \EE_{\T}, \, \, \, \, \,  \LL_i \EE_{\T} = \EE_{\T} \LL_i=  c_{\T}(i) \EE_{\T} , \,\, \, \, \,  \EE_{\s} \EE_{\T} =
  \delta_{\s \T} \EE_{\s}
\end{equation}
where $  \delta_{\s \T}  $ is the Kronecker delta. 
  
\medskip
The formulas in \eqref{IdempotentHecke1} and \eqref{IdempotentHecke3} are
consequences of the general theory for $ \JM$-elements developed in \cite{Mat-So}. 
For $ \QQ \Si_n $ the analogues of \eqref{IdempotentHecke1} and 
\eqref{IdempotentHecke3} were first found by Murphy in \cite{Murphy3}. We find it worthwhile to mention
that the corresponding 
properties do not hold for the Young symmetrizer idempotents for $ \QQ \Si_n $, since these are
not orthogonal.

\medskip
The expression for $ \EE_{\T} $ given in \eqref{IdempotentHecke1} contains many redundant factors and
is in general intractable, in the symmetric group case as well as in the Temperley-Lieb case.


\medskip
The purpose of this section is to give a new expression for $ \EE_{\T} $ in the Temperley-Lieb case,
using Jones-Wenzl idempotents. In view of this, one may now consider seminormal forms and Jones-Wenzl
idempotents as two aspects of the same theory. 


\medskip
Let $ \T \in \std(\lambda) $ be a two-column standard tableau. Then $ \T $ can be written in the form 
\begin{equation}\label{intheform}
\T = \raisebox{-.5\height}{\includegraphics[scale=0.75]{dib16.pdf}}
\end{equation}
where each $ D_i $ and $ M_i $ is a non-empty block
of consecutive natural
numbers, except that $ M_k $ is allowed to be empty, satisfying
that the numbers of both $ D_i $ and $ M_i $ are less than the numbers of
both $ D_j $ and $ M_j$ if $ i < j $. Conversely, each sequence of blocks $ D_1, M_1, D_2, \ldots, M_k $
satisfying these conditions give rise to a two-column standard tableau $ \T$. 



Let $  d_i := | D_i| $  and $ m_i  = | M_i| $ be the cardinalities of $ D_i $ and $ M_i $
and define $ n_i $ for $ i=1, 2, \ldots, k $ via
\begin{equation}\label{wenowassociate}
n_1 := d_1 \mbox{ and }  n_i = (d_1+d_2+ \ldots + d_i) - ( m_1+m_2+ \ldots + m_{i-1}) \mbox{ for } i >1 
\end{equation}
We now associate with $ \T $ an element $ f_{\T} \in \Delta^{\QQ}(\lambda) $ in the following recursive way. 
Suppose first that $  M_k \neq \emptyset  $. If 
$ k= 1 $ we set
\begin{equation}\label{exA}
f_\T = \raisebox{-.35\height}{\includegraphics[scale=0.7]{dib17.pdf}}
\end{equation}
and if $ k= 2 $ we set
\begin{equation}\label{exB}
f_\T \, \, \, = \, \, \,  \raisebox{-.5\height}{\includegraphics[scale=0.7]{dib18.pdf}}
\end{equation}
We repeat this recursively, that is in the $i$'th step we first concatenate on top with $ \mathbf{JW}_{\! n_i } $
and then bend down the $ m_i$ top and rightmost lines 
to the bottom. If $ M_k = \emptyset $ the construction is the same as for
$ M_k \neq \emptyset  $, except that in the last step the bending down of the $ m_k$ top and rightmost 
lines is omitted. 

\medskip
For example,
\begin{align}\label{exC}
& \text{if }\T \,  := \,   \raisebox{-.5\height}{\includegraphics[scale=0.7]{dib19.pdf}} & 
& \text{we have that} & 
& f_\T \, \, \, = \, \, \,  \raisebox{-.5\height}{\includegraphics[scale=0.6]{dib20.pdf}}
\end{align}
In general, if $ i $ appears in the first column of $ \T $ then $ i $ is connected
in $ f_\T $ to the southern border of a Jones-Wenzl element, and 
if $ i $ appears in the second column of $ \T $ then $ i $ is connected
in $ f_\T $ to the northern border of a Jones-Wenzl element. We have 
indicated this in \eqref{exC}, using colors. 

\medskip
In general, for $ \T $ as in \eqref{intheform}
we shall sometimes represent $ f_\T $ in the following schematic way
\begin{equation}\label{shorthand}
f_\T = \raisebox{-.5\height}{\includegraphics[scale=0.7]{dib23.pdf}}
\end{equation}
where $ n $ is a shorthand for $ \JWn $ and 
where $m_i $ indicates the number of lines being bent down, which may be zero for $ m_k $. 


\medskip
For $ \T $ a two-column standard tableau we set 
\begin{equation}
\gamma_{\T} := \prod_{j=1}^{k }\dfrac{n_i +1}{n_i-m_i +1} 
\end{equation}
We define 
$  f_{\T \T} $ as the concatenation of $ f_{\T}^{\ast} $ with $ f_{\T} $ with
$ f_{\T}^{\ast} $ on top of $ f_{\T} $ and 
finally we define $ \EE_{\T}^{\prime} \in \TLnQ$ as 
$ \EE_{\T}^{\prime} := \dfrac{1}{\gamma_{\T}} f_{\T \T} $, 
or diagrammatically
\begin{equation}\label{mainconstruction} 
  \EE_{\T}^{\prime} := \dfrac{1}{\gamma_{\T}}
\raisebox{-.5\height}{\includegraphics[scale=0.7]{dib24.pdf}}
\end{equation}




The elements $ \EE_{\T}^{\prime} $ have already appeared in the literature, see \cite{GW} and \cite{PMartin},
although our approach to them is quite different from the previous approaches.
The purpose of this section is to show that $ \EE_{\T}  = \EE_{\T}^{\prime} $. This is a new result. 

\medskip

We start out by proving the following Theorem. 
\begin{theorem}\label{startby}
  $ \{ \EE_{\T}^{\prime} \, | \, \T \in \std(\ParTwo) \} $ is a complete set of primitive orthogonal idempotents
  for $ \TLnQ$. 
\end{theorem}
\begin{dem}
  We first observe that \eqref{basicelements} implies 
$ f_{\T \T}^2 = \gamma_\T f_{\T \T} $ and so $ \EE_{\T}^{\prime} $ is indeed an idempotent.
  Similarly, we observe that $ f_{\T}^{ \ast} C_{\T }^{} = \gamma_{\T} { \rm JW}_{n_k -m_k } $
  from which it follows that $ f_{\T}  \neq 0 $, and hence also
  $ \EE_{\T}^{\prime}  \neq 0 $.
  
\medskip
We next assume that $ \T \neq \overline{\T} $ and must show 
that $ \EE_{\T}^{\prime} \EE_{\overline{\T}}{}^{\! \! \prime} = 0 $ which can be done by showing
that $ f_{\T}^{} f_{\overline{\T}}{}^{\! \! \ast} = 0 $.  
Letting $ \{\overline{D}_i \, | \, i=1,2, \ldots, \overline{k} \} $ and 
$ \{\overline{M}_i \, | \, i=1,2, \ldots, \overline{k} \} $
be the blocks for $ \overline{\T} $, as in 
\eqref{intheform},
and defining $ \overline{n}_i $ and $ \overline{m}_i $ as in
\eqref{wenowassociate}, 
we must show that the following diagram is zero
\begin{equation}\label{mustshow}
f_{\T}^{} f_{\overline{\T}}{}^{\! \! \ast} = \raisebox{-.5\height}{\includegraphics[scale=0.65]{dib25.pdf}}
\end{equation}
If $ n_1 = \overline{n}_1 $ and $ m_1 = \overline{m}_1 $ then \eqref{mustshow} is equal
to $ \dfrac{n_1 +1}{n_1-m_1 +1}  $ times 
$ f_{\T_1}^{} f_{\overline{\T}_1}{}^{\! \! \ast}  $ 
where $ \T_1 $ and $ \overline{\T}_1 $ are the standard tableaux obtained from
$ \T $ and $ \overline{\T} $  
by removing the blocks $ M_1, D_1 $ and $ \overline{M_1}, \overline{D_1} $
and so 
we may assume that $ n_1 \neq \overline{n}_1  $ or $ m_1 \neq \overline{m}_1 $. 
If $ n_1 < \overline{n}_1 $ then at least one line from $ \mathbf{JW}_{n_1} $ is bent down to
$ \mathbf{JW}_{n_1^{\prime}} $, and so it follows from
\eqref{absorbtion} that 
the resulting diagram is zero: to illustrate this we take 
$ n_1 = 3 $ and $ n_1^{\prime} = 4 $ where the relevant part of
$ f_{\T_1}^{} f_{\overline{\T}_1}{}^{\! \! \! \! \ast}  $ is 
\begin{equation}\label{mustshow1}
\raisebox{-.55\height}{\includegraphics[scale=0.65]{dib26.pdf}} \, \, = \, \, 0
\end{equation}  
If $ n_1 > \overline{n}_1 $ one applies $ \ast $ to \eqref{mustshow} and is then reduced to
the previous case $ n_1 < \overline{n}_1$.
If $ n_1 = \overline{n}_1 $ and $ m_1>  \overline{m}_1 $ then
a line from $ \mathbf{JW}_{n_1} $ is bent down to $ \mathbf{JW}_{n_2^{\prime}} $ 
and so the resulting diagram is also zero in this case. Let us illustrate this using $ n_1 = \overline{n}_1 = 4 $ and
$ m_1 =3, \overline{m}_1 =2 $ and $ n_2  = 3$ where the relevant part of 
\eqref{mustshow} is as follows
\begin{equation}
\raisebox{-.5\height}{\includegraphics[scale=0.7]{dib28.pdf}} 
\end{equation}  
Finally, if $ n_1 = \overline{n}_1 $ and $ m_1<  \overline{m}_1 $ we once again first apply $ \ast$ and
are then reduced to the previous case. This proves that $\{ \EE_{\T}^{\prime}\, | \, \T \in \std(\ParTwo) \} $
is a set of orthogonal idempotents. 

\medskip
The proof of the remaining parts of the Theorem, that is the statement that the 
$\{ \EE_{\T}^{\prime}\, | \, \T \in \std(\ParTwo) \} $ are a set of 
complete and primitive idempotents, is postponed to Corollary
\ref{finalcorsection2}.
\end{dem}


\medskip
\begin{corollary}
  Let $ \lambda $ be a two-column partition. Then 
  $ \{ f_\T \, | \, \T \in \std(\lambda) \} $ is a $\QQ$-basis for $ \Delta^{\QQ}(\lambda)$. 
\end{corollary}
\begin{dem}
  We have that $  f_{\T} \EE^{\prime}_{\s} = \delta_{ \s \T} f_{\T} $ and so it
  follows from Theorem \ref{startby} that 
    $ \{ f_\T \, | \, \T \in \std(\lambda) \} $ is a $ \QQ$-linearly independent subset of $ \Delta^{\QQ}(\lambda)$. 
Since $ \dim \Delta^{\QQ}(\lambda) = | \std(\lambda) | $ it is also a basis for $ \Delta^{\QQ}(\lambda)$. 
\end{dem}  


\medskip
The action of $ \Si_n $ on $ \{ 1,2, \ldots, n \} $ extends naturally to an action
of $ \Si_n $ on $ \lambda$-tableaux, by permuting the entries. For $ \T $ a $ \lambda$-tableau
and $ \sigma \in \Si_n $, we denote by $ \T \sigma $ the action of $ \sigma $ on $ \T$. 



\medskip
We now set out to prove 
$ \EE_{\T}^{\prime} = \EE_{\T} $. Our proof will be an induction over the dominance order on standard
tableaux and for this the following Theorem is a key ingredient.



\begin{theorem}\phantomsection\label{YSFfirst}
  Suppose that $ \T \in \std(\lambda) $ where $ \lambda \in \ParTwo$. Suppose first that
  for a simple transposition $ s_i \in \Si_n$ we have that $ \T s_i \in \std(\lambda) $ and that 
  $ \T  \unlhd \T s_i $. Then, setting $ \T_d := \T, \T_u := \T s_i $
and $ r := c_{\T_u}(i) - c_{\T_d}(i)  $, the following formulas hold  
  \begin{description}
  \item[a)]
$  f_{\T_d} \UU_i=
\dfrac{ r +1 }{ r }
f_{\T_d} + \dfrac{ r^2-1}{ r^2} f_{\T_u}$
  \item[b)]
$  f_{\T_u} \UU_i=
\dfrac{ r -1 }{ r }
f_{\T_u} +  f_{\T_d}$
    \end{description}
Suppose next that $ \T s_i \notin \std(\lambda) $. Then 
  \begin{description}
  \item[c)]
    $  f_{\T} \UU_i= 0  \,\, \, \,\, \, \,   \mbox{ if } i, i+1 \mbox{ are in the same column of } \T $ 
 \item[d)]
    $  f_{\T} \UU_i= 2 f_{\T}  \, \mbox{ if } i, i+1 \mbox{ are in the same row of } \T $ 
    \end{description}
\end{theorem}
\begin{dem}
  We first show {\bf a)}.
We have blocks $ D_j $ and $ M_j $ for $ \T $, as in \eqref{intheform}.
  By the assumptions, $ i $ lies in the first column of $ \T$,
  as the biggest number of a block $ D_j $, whereas $ i+1 $ lies in the second column of $ \T$, as 
the smallest number of the block $ M_j $, as indicated in the example below. 
\begin{equation}\label{2.17}
\T= \raisebox{-.5\height}{\includegraphics[scale=0.7]{dib29.pdf}}, \, \, \,  \, \, \,  \, \, \, 
f_\T= \raisebox{-.5\height}{\includegraphics[scale=0.7]{dib30.pdf}} 
\end{equation}
In \eqref{2.17}, we have indicated the corresponding $ f_{\T} $ and have singled out the lines for $ f_{\T} $ 
that are connected to $ i $ and $ i+1 $.
We now get, using \eqref{closing}
\begin{equation}\label{inserting}
  f_\T \UU_i= \raisebox{-.5\height}{\includegraphics[scale=0.7]{dib31.pdf}} = 
\raisebox{-.5\height}{\includegraphics[scale=0.7]{dib32.pdf}}   
\end{equation}  
On the other hand, bending down the last top line of the recursive formula \eqref{goesbackto}
for $ \JWn$ we have
\begin{equation}\label{bendingdown}
\raisebox{-.5\height}{\includegraphics[scale=0.7]{dib33.pdf}}   
\end{equation}
and inserting this in the right hand side of \eqref{inserting} we obtain
\begin{equation}\label{followsfrom}
\begin{array}{ll}     f_\T \UU_i  &= 
   \raisebox{-.35\height}{\includegraphics[scale=0.75]{dib34.pdf}} \\& \\
   &= \dfrac{ n_2+1}{ n_2} f_{\T_u}+ \dfrac{ n_2^2 -1}{ n_2^2} f_{\T_d}
\end{array}
\end{equation}
One finally checks that $ n_2 =  c_{\T_u}(i) - c_{\T_d}(i)  = r $ and so
{\bf a)} follows 
from \eqref{followsfrom}, at least for $ \T $ as in \eqref{2.17}.
For general $ \T$ the proof of {\bf a)} is carried out the same way. 
From this {\bf b)} follows by applying $ \UU_i $ to both sides of {\bf a)}. Finally, 
{\bf c)} is a consequence of the definitions. 
\end{dem}

\medskip
Theorem \ref{YSFfirst} is an analogue of Young's seminormal form known from the representation
theory of $ \QQ \Si_n $. 
To make this explicit we set 
\begin{equation}
\sss_i := \Phi(s_i ) = \UU_i - \one
\end{equation}
Then we have the following Corollary to Theorem \ref{YSFfirst}. 
\begin{corollary}\phantomsection\label{YSFsecond}
  {\rm (Young's seminormal form YSF for $\TLnQ$)}.
  Let $ \T, s_i, \T_u, \T_d $ and $ r $ be as in Theorem \ref{YSFfirst}. Then we have 
  \begin{description}
  \item[a)]
$  f_{\T_d} \sss_i=
\dfrac{ 1 }{ r }
f_{\T_d} + \dfrac{ r^2 -1 }{ r^2 } f_{\T_u}$ 
  \item[b)]
    $  f_{\T_u} \sss_i= -\dfrac{ 1 }{ r }
f_{\T_u} +  f_{\T_d}
$
  \end{description}
Suppose next that $ \T s_i \notin \std(\lambda) $. Then 
  \begin{description}
  \item[c)]
    $  f_{\T} \sss_i= -f_{\T}  \,\, \, \, \, \,   \mbox{ if } i, i+1 \mbox{ are in the same column of } \T $ 
 \item[d)]
    $  f_{\T} \sss_i= f_{\T}    \,\, \, \,\, \, \,\, \, \, \mbox{ if } i, i+1 \mbox{ are in the same row of } \T $ 
    \end{description}
\end{corollary}  

\begin{dem}
This follows immediately from Theorem \ref{YSFfirst}. 
\end{dem}

\medskip
Note that the main ingredient for proving Theorem \ref{YSFfirst}, and hence also 
Corollary \ref{YSFsecond}, was the 
recursive formula \eqref{goesbackto}
for $ \JWn$. In view of this one main consider \eqref{goesbackto} and YSF,
that is Corollary \ref{YSFsecond}, as two sides of the same coin. 



\medskip
We next aim at proving that $ f_\T$'s is an eigenvector for $ \LL_i$ with eigenvalue $ c_\T(i) $.
The argument for this will be an induction on $ \std(\lambda)$ over $\trianglelefteq $.
We may either carry out this induction from top to bottom, using
$ \T^{\lambda} $ as inductive basis, or from bottom to top, using $ \T_{\lambda} $ as inductive basis.
In either case it turns
out that the inductive step, using Theorem \ref{YSFfirst}, is relatively straightforward and
similar to the inductive step for the $ \QQ \Si_n $-case, whereas the
inductive basis is the most complicated part of the proof.
The $ \T_{\lambda} $-case is slightly simpler than the $ \T^{\lambda} $-case
and so we choose to carry out the induction from bottom to top. In other words, to prove the inductive basis
we should take $ \T= \T_{\lambda} $ where $ \lambda = \ParTwo $ and must show that
$ f_{\T_{\lambda}} \LL_i = c_{\T_{\lambda}}(i) f_{\T_{\lambda}}$ for all $ i =1, 2, \ldots, n$. This is the content(!)
of the next Lemma. 
\begin{lemma}\label{2.4}
  Let the situation be as just described, that is $ \T= \T_{\lambda} $
  where $ \lambda = (2^{l_2} , 1^{l_1-l_2} )  \in \ParTwo$ and $ l_1 $ and
$ l_2 $ are the lengths of the two columns of $ \lambda$. 
  Then
  we have that $   f_{\T} \LL_{i} = (1-i)  f_{\T} $  for $ i=1,\ldots, l_1 $ 
  and $   f_{\T} \LL_{i} = (2 -i + l_1 )  f_{\T} $
  for $ i = l_1+1, \ldots, n$, that is 
  \begin{equation}\label{thatis}
  f_{\T} \LL_{i} = c_{\T}(i)  f_{\T}
  \end{equation}
\end{lemma}  
\begin{dem}
We have that 
\begin{equation}\label{2.23}
f_{\T } =   \raisebox{-.47\height}{\includegraphics[scale=0.7]{dib35.pdf}}, \, \, 
f_{\T } {\mathbf{L}}_{l_1+1} = 
   \raisebox{-.4\height}{\includegraphics[scale=0.7]{dib37.pdf}} 
\end{equation}
Since the product $f_{\T } \LL_{l_1 +1} $
only involves the leftmost $ l_1 +1 $ bottom lines of $ f_{\T }  $ we   
assume from now on that $ l_2 =1$. We therefore prove by induction on $ l_1 $ that 
$   f_{\T} \LL_{l_1+1} = c_{\T}(i) f_{\T} $ where $ \lambda= (2, 1^{l_1 -1}) $. 
For this the basis case $ l_1=1 $ is the claim that  
\begin{equation}
 \raisebox{-.5\height}{\includegraphics[scale=0.75]{dib38.pdf}} 
\end{equation}
or equivalently that $ \UU_1 \LL_{2} =  \UU_1 ( \UU_1 - \one )$ 
which follows immediately from the definitions.

\medskip
To prove the induction step we assume 
the statement for $ l_1 -1 $ and prove it for $ l_1 $.
From the definition of the $\JM$-elements
\eqref{defJM} we have the following formula, valid for $ i=1, 2, \ldots, n-1 $. 
\begin{equation}\label{225}
  \LL_{i+1} = \sss_i \LL_i \sss_i + \sss_i = (\UU_i-\one) \LL_i (\UU_i-\one) +  \UU_i-\one 
\end{equation}
But $ f_{\T }  \UU_{i } = 0 $ for $ i = 1, 2, \ldots, l_1-1 $ and so  
we
deduce from \eqref{225} that
\begin{equation}\label{wededucefrom225}
  f_{\T }  \LL_{i } = (1-i ) f_{\T }  \mbox{ for } i= 1,2\ldots, l_1
\end{equation}  
which shows \eqref{thatis} except when $ i = l_1+1  $. 

\medskip
In order to treat the case $ i = l_1+1 $, in view of \eqref{225}, 
we first calculate an expression for $ f_{\T }  (\UU_{l_1 }-\one ) $. 
We find 
\begin{equation} \begin{array}{l}
f_{\T }  (\UU_{l_1 }-\one ) =  \raisebox{-.6\height}{\includegraphics[scale=0.7]{dib39.pdf}} 
\end{array}
  \end{equation}
\begin{equation} \hspace{-5.8cm}  
=  \raisebox{-.55\height}{\includegraphics[scale=0.7]{dib40.pdf}} 
  \end{equation}
\begin{equation}\label{2.27}  
=  \raisebox{-.38\height}{\includegraphics[scale=0.7]{dib41.pdf}} 
  \end{equation}
\begin{equation}\label{2.28}
\hspace{-5.2cm}
=  \raisebox{-.38\height}{\includegraphics[scale=0.7]{dib42.pdf}}   
\end{equation}
where we used the \eqref{bendingdown} variation of 
\eqref{goesbackto} for \eqref{2.27}. We next apply $ \LL_{l_1} $ to 
\eqref{2.28} in order to arrive at an expression for $  f_{\T}( \UU_{l_1}-\one) \LL_{l_1} $. 
Using \eqref{wededucefrom225} we see that 
$ \LL_{l_1} $ acts on the first term of \eqref{2.28} by multiplication with $ 1-l_1$ 
and, by inductive hypothesis, it acts on the second term of
\eqref{2.28} by multiplication with $ 1$. Combining, we get that 
\begin{equation}\label{2.29}
\hspace{-3cm}
f_{\T} (\UU_{l_1}-\one) \LL_{l_1} =  \raisebox{-.38\height}{\includegraphics[scale=0.7]{dib43.pdf}}   
\end{equation}
We now get
\begin{equation}\label{2.30}
\hspace{-3cm}
f_{\T} (\UU_{l_1}-\one) \LL_{l_1}  (\UU_{l_1}-\one)=
\raisebox{-.73\height}{\includegraphics[scale=0.7]{dib44.pdf}}   
\end{equation}
\begin{equation}\label{2.31}
 =  \raisebox{-.4\height}{\includegraphics[scale=0.7]{dib45.pdf}}   
\end{equation}
Finally, adding \eqref{2.28} and \eqref{2.31} we get, using \eqref{225}
\begin{equation}\label{2.32} 
  f_{\T}  \LL_{l_1+1}
  = \raisebox{-.5\height}{\includegraphics[scale=0.7]{dib46.pdf}}   = f_{\T }
\end{equation}
This proves the induction step
and then also the Lemma.

\end{dem}  

\medskip


\begin{lemma}\phantomsection\label{commutation}
  We have the following commutation relations between $ \LL_k $ and $ \UU_i $. 
  \begin{description}
  \item[a)] If $ k \neq i, i+1 $ then $   \UU_i \LL_k = \LL_k \UU_i$
  \item[b)] We have $(\UU_i -\one)  \LL_i  =  \LL_{i+1} ( \UU_i -\one) -1 $
     \item[c)] We have $(\UU_i -\one)  \LL_{i+1}  =  \LL_{i} ( \UU_i -\one) +1 $
  \end{description}    
\end{lemma}  
\begin{dem}
  This follows immediately from $ \LL_k = \Phi(L_i ) $ and the definition of $ L_i $ in
  \eqref{defJM}. 

\end{dem}  

\medskip
We can now show the Theorem that was mentioned above. 
\begin{theorem}\label{mentionedabove}
  Let $ \T \in \std(\lambda) $ where $ \lambda \in \ParTwo$. Then
for all $ i =1,2, \ldots, n $ 
  we have 
  \begin{equation}\label{2.34}
  f_{\T} \LL_{i} = c_{\T}(i)  f_{\T}
  \end{equation}  
\end{theorem}  
\begin{dem}  
As already mentioned, we show the formula \eqref{2.34} by upwards induction on $ \std(\lambda)$. The basis
  case $ \T = \T_{\lambda} $ is given by Lemma \ref{2.4}, so let us assume that
 $ \T \neq \T_{\lambda} $ and that 
  \eqref{2.34} holds
  for all $ \s $ such that $ \s \lhd \T $. We must then check it for $ \T$. Since
  $ \T \neq \T_{\lambda} $ there is an $ i $ appearing in the second column of $ \T $, but with $ i+1$
  appearing in the first column of $ \T$, in a lower position, 
  and so $ \T s_i \lhd \T $. Setting $ f_d = f_{\T  s_i} $, $ f_u = f_{\T} $
  and $ r :=    c_u(i ) -  c_d(i ) $ where $  c_u(k ) := c_\T(k ) $ and $  c_d(k ) := c_{\T s_i}   (k ) $, 
  we have from {\bf a}) of Theorem \ref{YSFfirst} that
  \begin{equation}\label{YSFinduction}
 f_{d} \UU_i=\dfrac{ r +1 }{ r }
f_{d} + \dfrac{ r^2-1}{ r^2} f_{u}
  \end{equation}

By  
induction hypothesis we have that $ f_d \LL_k = c_d (k) f_u$ for all $ k $. 
  Suppose first that $ k \neq i, i+1 $. Then we get from Lemma \ref{commutation} that
  $ \UU_i \LL_k =  \LL_k \UU_i   $. Acting upon $ f_d $, this equation becomes via
\eqref{YSFinduction}
 \begin{equation}\label{compare}
\dfrac{ r +1 }{ r }
c_d(k)   f_{d} + \dfrac{ r^2-1}{ r^2}  f_{u} \LL_k = \dfrac{ r +1 }{ r }
c_d(k) f_{d} + \dfrac{ r^2-1}{ r^2} c_d(k) f_{u}
 \end{equation}
from which we deduce 
that $ f_u \LL_k = c_d(k ) f_u $. But in this case $ c_u(k) = c_d(k ) $ and so $ f_u \LL_k = c_u(k ) f_u $,
as claimed.

Suppose now that $ k = i $.
We have from Lemma \ref{commutation}
that $ (\UU_i - \one ) \LL_i = \LL_{i+1} (\UU_i - \one ) -1 $. Acting upon $ f_d $, this becomes 
$ f_d (\UU_i - \one ) \LL_i  = f_d (\LL_{i+1} (\UU_i - \one ) -1 )  $. 
Using \eqref{YSFinduction}, the left hand side of this is 
\begin{equation}\label{compareA}
{\rm LHS}= \dfrac{ 1 }{ r } c_d(i) 
f_{d} + \dfrac{ r^2-1}{ r^2} f_{u} \LL_i^{} 
\end{equation}
whereas the right hand side is 
\begin{equation}\label{compareB}
 \begin{split}
{\rm RHS}=&  \dfrac{ 1 }{ r } c_d(i+1) 
f_{d} + \dfrac{ r^2-1}{ r^2} c_d(i+1)  f_{u} - f_d
= \dfrac{- r+ c_d(i+1) }{ r } 
f_{d} + \dfrac{ r^2-1}{ r^2} c_d(i+1)  f_{u}  \\
=&  \dfrac{ c_d(i) }{ r } 
f_{d} + \dfrac{ r^2-1}{ r^2} c_d(i+1)  f_{u} 
 \end{split}
 \end{equation}
where we used $ c_d(i+1) = c_u(i) $ and $ r =    c_u(i ) -  c_d(i ) $ for the last equality. 
Comparing \eqref{compareA} and \eqref{compareB}
we conclude that $ f_u \LL_i  = c_d(i+1 ) f_d =  c_u(i ) f_d $, proving the Theorem in this case
as well.

\medskip
Finally, the case $ k = i+1 $ is proved the same way. The Theorem is proved.
\end{dem}

\medskip

\begin{corollary}\label{finalcorsection2}
  For $ \lambda $ a two-column partition and $ \T \in \std(\lambda) $ we have that
  $ \EE^{\prime}_{\T} = \EE_{\T}  $. In particular, the $ \{  \EE_{\T}^{\prime} \} $ form
  a complete set of primitive idempotents for $ \TLnQ$. 
\end{corollary}
\begin{dem}
  It follows from Theorem \ref{mentionedabove} and
  the formula $ \EE_{\T}^{\prime} := \dfrac{1}{\gamma_{\T}} f_{\T \T} $ that
  $ \EE^{\prime}_{\T} \LL_k = \LL_k  \EE^{\prime}_{\T} = c_\T(k)  \EE^{\prime}_{\T} $ for all $ k$. 
  But this property characterizes the idempotent $ \EE_\T $ and so $ \EE^{\prime}_{\T} = \EE_{\T} $,
  as claimed. 
\end{dem}    

\section{The unseparated case}\label{The unseparated case}
We shall from now on focus on the Temperley-Lieb algebra $\TLnF $ defined over the finite
field $ \FF$, where $ p>2$. We are interested in idempotents in $\TLnF $. 


\medskip
If $ p > n $ the condition \eqref{separationcondition} still holds and 
so $\TLnF $ is a semisimple algebra and in fact all the results from the
previous section remain valid. Let us therefore assume that $ p \le n $.
Under that assumption \eqref{separationcondition} does not hold,
and so we are in the {\it unseparated case}
in the notation of \cite{Mat-So}.
Moreover, the coefficients of $ \JWn $ and of $ \EE_{\T} $ 
cannot be reduced from $ \QQ $ to $ \FF$, and hence these idempotents {\it do not exist} in 
$\TLnF $. In fact, if $ p \le n $
there are in general no nonzero idempotents in $\TLnF $ satisfying \eqref{definingpropertyJW}. 

\medskip
On the other hand, we can still apply the general theory
of $\JM$-elements to construct idempotents for $\TLnF $. Let us briefly explain this. 

\medskip
For $ \T \in \std(\lambda) $ where $ \lambda \in \ParTwo$ we define the {\it $p$-class} $ [ \T]  $ of $ \T$
via
\begin{equation}\label{firstclass}
[\T]  = \{ \s \in \std(\ParTwo) \, | \, c_{\s}(i) \equiv c_{\T}(i)  \mbox{ mod }  p \mbox{ for all } i=1,2,\ldots, n  \} 
\end{equation}
We now set 
\begin{equation}\label{mainprinciple}
\EE_{[\T]} := \sum_{ \s \in [\T]  } \EE_{\s}  
\end{equation}
By definition $ \EE_{[\T]} \in \TLnQ $, but 
it follows from the general theory developed in \cite{Mat-So} that $ \EE_{[\T]} $ in fact belongs to 
$ \TLZpn $ where $ \Z_{(p)} :=\{ \dfrac{a}{b} \in \QQ \, | \,   p
\mbox{ does not divide } b \} $. 
We have that $  \Z_{(p)} $ is a local ring with maximal ideal $ \pi := p \Z_{(p)} $ and
$   \Z_{(p)}/\pi \cong \FF$.
and hence $  \EE_{[\T]}  $ can be reduced to an element of $ \TLnF $, that we shall also denote 
$   \EE_{[\T]}  $. 

\medskip
The $   \EE_{[\T]}  $'s clearly are idempotents in $ \TLnF $, called {\it class idempotents}, 
but they are 
not primitive idempotents in general, as we shall shortly see. 

\medskip
Let $ M_{triv} := \Delta^{\FF}( 1^{n} ) $
be the trivial $ \TLnF$-module, in other words $ M_{triv} $ is the one-dimensional $ \TLnF$-module on which
$ \UU_k $ acts as zero for all $ k $. 
Let $ P_{triv} $ be the projective cover for $ M_{triv} $. 
By general principles there exists 
a primitive idempotent $ ^{p} \EE_{triv} \in \TLnF$ such that $   ^{p} \EE_{triv} \TLnF = P_{triv} $.
Recently, it was  observed in \cite{StuSpe} that the idempotent $ ^{p} \EE_{triv} $ coincides with the 
{\it $p$-Jones-Wenzl idempotent} $ ^{p}\!\JWn  $ that was introduced by Burull, Libedinsky and Sentinelli,
see  
\cite{BLS}. We need this fact in the following, 
and shall therefore recall the definition of $ ^{p}\!\JWn  $.

\medskip
For $ n \in \N$ we define non-negative integers $ a_i $ satisfying $ 0 \le a_i < p, a_k \neq  0 $
and 
\begin{equation}\label{padic}
 n+1 = a_k p^k + a_{k-1} p^{k-1} + \ldots +a_1 p + a_0 
\end{equation}
In other words, $ (a_k, a_{k-1}, \ldots, a_1, a_0 )$ are the coefficients
of $ n+1 $ when written in base $p$. We then define $ {\mathcal I}_n  \subseteq \N $ via 
\begin{equation}\label{mathcal I}
{\mathcal I}_n := \{ a_k p^k \pm  a_{k-1} p^{k-1} \pm  \ldots  \pm a_1 p  \pm a_0 \} -1
\end{equation}
One checks that each $ m \in {\mathcal I}_n $ is given uniquely by the corresponding
sequence of signs for the nonzero
$ a_k$'s. Using this, 
for $  m \in {\mathcal I}_n $ 
we now define a tableau $ \T_m \in \std(\ParTwo) $
in terms of a block decomposition for standard tableaux as in \eqref{intheform}, 
using 
blocks $ D_1, M_1, D_2, M_2, \ldots, D_k, M_k $ of consecutive numbers, as follows. 


\medskip
Suppose first that $ i_1 \ge  0 $ is maximal such that $ (a_k, a_{k-1}, \ldots, a_{k-i_1}) $ all
appear in $ m $ with non-negative sign. Then $ D_1 $ is defined by the condition that it be of cardinality 
$ | D_1|  = a_k p^k  + \ldots + a_{k-i_1} p^{k-i_1}  -1 $. 
Suppose next that $ i_2 > i_1 $ is maximal such that
$ (a_{k-i_1-1}, a_{k-i_1-2}, \ldots, a_{k-i_2 }) $ all appear in $ m $ with non-positive sign. Then we define
$ M_1 $ by the condition that it be of cardinality 
$ | M_1|  = a_{k-i_1-1} p^{k-i_1-1} +  \ldots + a_{k-i_2} p^{k-i_2}   $. We then continue the same way, defining
$ D_2, M_2, \ldots $ 
except that the $ -1 $ term should only appear for $ D_1$. 

\medskip
The $p$-Jones-Wenzl idempotent $ ^{p}\!\JWn  $ is now defined as follows
\begin{equation}\label{pJW}
 ^{p}\!\JWn  := \sum_{ m \in {\cal I}_n } \EE_{ \T_m}^{\prime} 
\end{equation}

Note that, unlike our definition \eqref{pJW},
the original definition of $  ^{p}\!\JWn  $
in \cite{BLS} was formulated recursively,
and did not use standard tableaux.
Note also that the original definition of $  ^{p}\!\JWn  $ was carried out for the Temperley-Lieb
algebra with loop parameter $ -2 $, as opposed to loop parameter $ 2 $ as in the present paper. 
To switch between the two settings one
should apply the isomorphism $ \UU_i \mapsto - \UU_i $. 
But modulo these observations, the two definitions can quickly be seen  to coincide. 

\medskip
Let us give a couple of examples. If $ n= 3 $ and $ p= 3 $ we have $ {\cal I }_3 = \{ 3 \pm 1 \} -1 = 
\{ 3,1 \} $. The tableaux corresponding to the elements of $  {\cal I }_3 $ are as follows
\begin{equation}\label{dib51}
  \T_3 =  \raisebox{-.4\height}{\includegraphics[scale=0.7]{dib51.pdf}}, \, \, \, \, 
    \T_1 = \raisebox{-.45\height}{\includegraphics[scale=0.7]{dib52.pdf}}
\end{equation}
and so we get 
\begin{equation}\label{dib53}
  ^{3}\!\JWtres     =\EE_{\T_3}^{\prime} + \EE_{\T_1}^{\prime}   = 
  \raisebox{-.42\height}{\includegraphics[scale=0.7]{dib53.pdf}}
\end{equation}
To verify that  $ ^{3}\!\JWtres $ belongs to $ \TLtrestres$, one 
uses \eqref{dib7} and \eqref{dib8} to expand $ \JWdos $ and $ \JWtres $ and finds
\begin{equation}\label{dib54}
 ^{3}\!\JWtres     = \raisebox{-.42\height}{\includegraphics[scale=0.7]{dib54.pdf}}
\end{equation}
which indeed belongs to $ \TLtrestres$.

\medskip

In the tableaux in \eqref{dib51} we have indicated with color red, for each $ i =1,2,3 $, 
the {\it residue} $ c_{\T}(i) \, \,  {\rm   mod  }\, \, p $
of the content $  c_{\T}(i) $. Using this
we get that the $3$-class of $ \T_3 $ is $ [ \T_3] = \{ \T_3, \T_1 \} $.
We now 
use Corollary
\ref{finalcorsection2} and get that 
\begin{equation}
\EE_{[\T_3]}  = \,  ^{3} \! \JWtres     
\end{equation}
Thus in this case
the class idempotent $ \EE_{[\T_3]} $ is in fact primitive. 

\medskip
To give an example where the class idempotent is not primitive we
choose $ p= 3 $ and $ n=12$. 
We then have $ n+1 = 9+3+1 $ and so $ {\cal I}_{n} = \{ 9 \pm 3 \pm 1 \} -1 = \{ 12, 10, 6,4 \} $
and so we have that $    ^{3} \! \JWdoce =
\EE_{\T_{12}}^{\prime} + \EE_{\T_{10}}^{\prime} +
\EE_{\T_{6}}^{\prime} + \EE_{\T_{4}}^{\prime}  $ 

\medskip
The corresponding standard tableaux, with $3$-residues indicated with color red as before, are
as follows
\begin{align}\label{redcolortableaux1}
&    \T_{12}=  \raisebox{-.5\height}{\includegraphics[scale=0.7]{dib55.pdf}}  & 
&   \T_{10}=  \raisebox{-.455\height}{\includegraphics[scale=0.7]{dib56.pdf}} & 
&      \T_{6}=  \raisebox{-.34\height}{\includegraphics[scale=0.7]{dib58.pdf}} & 
&   \T_{4}=  \raisebox{-.26\height}{\includegraphics[scale=0.7]{dib57.pdf}}
 \end{align}
Note that
$ \T_{12}, \T_{10}, \T_{16}$ and $ \T_{4} $ all belong to the same $3$-class, as can be seen
by comparing the residues modulo $3$. But the class $ [\T_{12}] $ 
contains two more tableaux, namely 
\begin{align}\label{redcolortableaux2}
&     \s=  \raisebox{-.50\height}{\includegraphics[scale=0.7]{dib60.pdf}} & 
&   \T=  \raisebox{-.44\height}{\includegraphics[scale=0.7]{dib59.pdf}}
\end{align}
obtained by interchanging $ \{6,7,8\} $ and $ \{9,10,11\} $ in $ \T_6 $ and $ \T_4$. From this we get that
\begin{equation}\label{notprimitive}
 \EE_{[\T_{12}]} = \,  ^{3} \!  \JWdoce  + \EE_{\s}^{\prime} +
 \EE_{\T}^{\prime}
\end{equation}
which shows that
$ \EE_{\s}^{\prime} +
\EE_{\T}^{\prime} \in \TLnF $. By expanding in terms of the diagram basis
for $ \TLnF $, one gets $  \EE_{\s}^{\prime} +
 \EE_{\T}^{\prime} \neq 0 $ in $ \TLnF $ and clearly
$ \JWdoce $ and $  \EE_{\s}^{\prime} +
 \EE_{\T}^{\prime}  $ are orthogonal. 
Hence  
$ \EE_{[\T_{12}]} $ is not a primitive idempotent in $ \TLnF $.  

\medskip
The purpose of the rest of the paper is to show that a variation of the principle for constructing
idempotents given in \eqref{mainprinciple}, this time using KLR-theory, 
can be applied recursively to derive the $p$-Jones-Wenzl idempotents
for $ \TLnF$, that is the primitive idempotents. 

\medskip
Let us start out by proving the following Lemma, which is a generalization of
\eqref{notprimitive}.
\begin{lemma}
  Let $ \EE_{[\T_n]} \in \TLnF  $ be the class idempotent for the $p$-class $ [ \T_n] $, given by
  the one-column tableau $ \T_{n} = \T_{ 1^n} = \raisebox{-.5\height}{\includegraphics[scale=0.5]{dib61.pdf}}$.
  Then $ \EE_{[\T_n ]} = \,   ^{p}\!\JWn + \EE $ for some idempotent $ \EE $
  in $ \TLnF $, orthogonal to $  ^{p}\!\JWn $. 
\end{lemma}
\begin{dem}
  We must show that $ \T_m \in [ \T_{n}] $ for all $ m \in{ \cal I }_n $ as in \eqref{mathcal I}.
  Let $ D_1, M_1, \ldots, D_k, M_k $ be the sequence of blocks defining $ \T_m $,
  as in the paragraph preceding \eqref{pJW}. Then clearly $ c_{\T_{n}}(i) \equiv  c_{\T_m}(i)  \mbox{ mod } p$ for
  $ i \in D_1 $, since in fact $ c_{\T_{n}}(i) =  c_{\T_m}(i) $ for these $ i $.
  Suppose now that $ M_1 \neq \emptyset $ and that $ m_{1, min} $ is the first number in $M_1 $. Then
  by the cardinality of $ D_1 $ we have that
  $ c_{\T_{n}}( m_{1, min})  \equiv  c_{\T_m}( m_{1, min}) \equiv 1  \mbox{ mod } p$ and
  then $ c_{\T_n}(m)  \equiv  c_{\T_m}(m)  \mbox{ mod } p$ for all $ m \in M_1 $. 
  This patterns repeats itself. If
  $ D_2 \neq \emptyset $ we let $ d_{2,min} $ be the first number of $ D_2$. Then by the cardinality of
  $  D_1 \cup M_1 $
  we have that $ c_{\T_n}(d_{2,min})  \equiv  c_{\T_m}(d_{2,min}) \equiv 1  \mbox{ mod } p$
  and then $ c_{\T_n}(d)  \equiv  c_{\T_m}(d)  \mbox{ mod } p$ for all $ d\in D_2 $, and so
  on recursively. This
  proves the Lemma. 
\end{dem}  


\medskip

For the rest of the article we fix the following notation. We set $ N:= n $
and define $ N_1, N_2, R \in \N $ using integer division as follows 
\begin{equation}\label{fixnotation}
N= (p-1) + N_1, \,  N_1 = p N_2 + R \, \mbox{ where }  0 \le R < p
\end{equation}
Recall that we have $ n \ge p $ so that indeed $ N_1 \in \N$. 


\medskip
The next Lemma gives us a kind of recursive description of the class $ [ \T_{N}] $.



\begin{lemma}\label{kindofrecursive}
If $ R = 0 $ there is a bijection
  \begin{equation}\label{313}
f_1: [ \T_{N}] \rightarrow \std(\ParTwoNuno) 
  \end{equation}
Otherwise, if $ R > 0 $, there is a bijection 
  \begin{equation}\label{314} 
f_2: [ \T_{N}] \rightarrow \std(\ParTwoNuno) \times \{1,2\}
  \end{equation}
\end{lemma}
\begin{dem}
  Suppose first that $ R=0$ and let $ \T \in [ \T_{N}] $. We must define $ f_1(\T) $
  and must show that it is a bijection.
Since $  \T \in [ \T_{N}] $, 
the numbers
$ (1,2,\ldots, p-1) $, whose content residues in $ \T$ are $ (0,p-1,\ldots, 2) $, 
all appear in the first column 
of $ \T$. We now consider consecutive blocks of consecutive numbers $ B_1, B_2, \ldots, B_{n_2} $ in $ \T$, 
all of length $ p $, starting with the block
$ B_1 := (p,p+1, \ldots, 2p-1) $. 
For each $B_i$, the 
content residues are $ (1,0,p-1,p-2,\ldots, 3,2)$. The numbers of each $ B_i $ may appear in either column
of $ \T$, 
but they all 
appear in the same column of $ \T$, since $ \T \in [ \T_{N}] $.
Using this observation, 
we can define $ f_1(\T) $ as the two-column standard
tableau that has $ i $ in the
first column iff the numbers of $ B_i $ are in the first column of $ \T$.


Here are two examples 
of $ f_1(\T) $, using $ p=3 $, in which we have indicated the blocks $ B_1, B_2, B_3 $ and $ B_4 $ with colors.


\begin{equation}
f_1:  \raisebox{-.5\height}{\includegraphics[scale=0.7]{dib62.pdf}} \mapsto
      \raisebox{-.4\height}{\includegraphics[scale=0.7]{dib63.pdf}}, \, \, \,  \, \, \,  \, \, \,  \, \, \, 
f_1:  \raisebox{-.5\height}{\includegraphics[scale=0.7]{dib65.pdf}} \mapsto
      \raisebox{-.4\height}{\includegraphics[scale=0.7]{dib64.pdf}} 
\end{equation}
One readily checks that $ f_1 $, defined this way, is a bijection, proving \eqref{313}.

In order to show \eqref{314}, we choose $ \T \in  [\T_{N}] $ and proceed as before, defining blocks
$ B_1, B_2, \ldots, B_{n_2} $ of consecutive numbers of length $ p $. But since $ R > 0 $ there will this
time be an 'extra' block $ B_{n_2+1} $ of length $ R $. The numbers of $ B_{n_2+1} $ may appear in
either column of $ \T$, but they all appear in the same column. 
Let $ \T_1 :=  \T |_{\le n-R } $. We now define $ f_2(\T) := (f_1(\T_1), 1) $ if the 
numbers of $ B_{n_2+1} $ are all in the
first column of $\T$, and otherwise we define $ f_2(\T) := (f_1(\T_1), 2) $. Here are two examples, using
$ p= 3$ and $ R=2 $. 



\begin{equation}
f_2:  \raisebox{-.5\height}{\includegraphics[scale=0.7]{dib66.pdf}} \mapsto
     \left( \raisebox{-.4\height}{\includegraphics[scale=0.7]{dib64.pdf}},1\right), \, \, \,  \, \, \,  \, \, \,  \, \, \, 
f_2:  \raisebox{-.5\height}{\includegraphics[scale=0.7]{dib67.pdf}} \mapsto
    \left(  \raisebox{-.4\height}{\includegraphics[scale=0.7]{dib64.pdf}}, 2 \right)
\end{equation}
Once again, one checks that $ f_2 $ is a bijection, which proves \eqref{314}, and hence the Lemma. 

\end{dem}

\medskip
Returning to the examples \eqref{redcolortableaux1} and \eqref{redcolortableaux2},
where $ n=12 $ and $ p = 3$, 
we have that $ [ \T_{12}] = \{ \T_{12}, \T_{10}, \T_{6}, \T_{4}, \s, \T \} $ and writing $ f= f_2 $ we get
\begin{equation}\label{tableauxclassA}
  f(\T_{12}) =     \left( \raisebox{-.45\height}{\includegraphics[scale=0.7]{dib68.pdf}},1\right), \, 
  f(\T_{10}) =     \left( \raisebox{-.45\height}{\includegraphics[scale=0.7]{dib68.pdf}},2\right), \, 
  f(\T_{6}) =     \left( \raisebox{-.45\height}{\includegraphics[scale=0.7]{dib69.pdf}},1\right), \, 
    f(\T_{4}) =     \left( \raisebox{-.45\height}{\includegraphics[scale=0.7]{dib69.pdf}},2\right)
\end{equation}
whereas 
\begin{equation}\label{tableauxclassAB}
  f(\s) =     \left( \raisebox{-.45\height}{\includegraphics[scale=0.7]{dib70.pdf}},1\right), \, 
  f(\T) =     \left( \raisebox{-.45\height}{\includegraphics[scale=0.7]{dib70.pdf}},2\right), \, 
\end{equation}
Note now that $ [ \T_{3} ] =\left\{ 
\raisebox{-.45\height}{\includegraphics[scale=0.7]{dib68.pdf}},
\raisebox{-.45\height}{\includegraphics[scale=0.7]{dib69.pdf}} 
\right\} $, which are the two tableaux that appear in \eqref{tableauxclassA}, but that 
$ \raisebox{-.45\height}{\includegraphics[scale=0.7]{dib70.pdf}} $ does not belong
to $ [ \T_{3} ]  $. Our second goal is to explain, in general,
that this is the reason why the tableaux $ \s $ and $ \T $ should
not be taken into account when giving the primitive idempotent. 


\section{The integral KLR-algebra}\label{The integral KLR-algebra}
Brundan-Kleshchev and independently Rouquier found a new presentation
for the group algebra $ \FF \Si_n $, proving that it is isomorphic to the {\it KLR-algebra} $ \RKLR $
(in fact they worked in the greater generality of cyclotomic Hecke algebras). 
If $ n\ge p $ it follows from
their work that $  \FF \Si_n $ is endowed with a non-trivial $ \Z$-grading, since
$ \RKLR $ is endowed with a non-trivial $ \Z$-grading in that case.
The isomorphism $ \FF \Si_n \cong \RKLR $ is important to us since it induces, via
Lemma \ref{wellknownfunda}, an isomorphism $ \TLnF \cong \RKLR/ {\mathcal I}_{n} $
where $ {\mathcal I}_{n} $ is a graded ideal in $ \RKLR $, and hence in particular 
$ \TLnF  $ inherits a $\Z$-grading from $ \FF \Si_n  $, see \cite{PlazaRyom} for more details on this.

\medskip
Hu and Mathas gave in \cite{hu-mathas2} a new simpler proof of the Brundan-Kleshchev and Rouquier isomorphism
using seminormal forms, and via this they were able to lift it to an
isomorphism $  \Z_{(p)} \Si_n \cong \RKLRZ $, where $ \RKLRZ $
is an integral version of $ \RKLR $ (once again the result was proved in the greater generality
of cyclotomic Hecke algebras).
We shall need this isomorphism and its proof so let us recall
the precise definition of $ \RKLRZ $ from \cite{hu-mathas2}.

\medskip

We first arrange the elements of $ \FF= \{0,1,2,\ldots, p-1 \} $ in a cyclic quiver as follows
\begin{equation}
  \raisebox{-.45\height}{\includegraphics[scale=0.7]{dib71.pdf}}
 \end{equation}
and for $ i, j \in \FF $ we 
write $ i \rightarrow j $ if $ i $ and $ j $ are adjacent in the quiver in the way that the arrows
indicate. We shall refer to the elements $ \ii
=(i_1, i_2, \ldots, i_n ) $ of $    \FF^n $ as residue sequences. 



\begin{definition}\label{RnZ}
  The integral KLR-algebra $ \RKLRZ $ is the $  \Z_{(p)} $-algebra generated
  by the elements
\begin{equation}  
 \{ e( \ii ) \,  | \,  \ii \in  \FF^n \} \cup \{ \psi_k \,  | \,  1 \le k < n \} \cup
  \{  y_l \,  | \,  1 \le l \le n  \} 
\end{equation}    
with identity 
$   1  = \sum_{\ii \in  \FF^n } e(\ii) $,
subject to the relations
\begin{align}
\label{firstR} e(\ii) e(\jj) & = \delta_{\ii, \jj} e(\ii)  &   y_l e(\ii) & = e(\ii) y_l  \\
  \psi_k e(\ii) & =  e(\ii \cdot s_k) \psi_k    &   y_l y_m &= y_m y_l \\
  \psi_k  y_{k+1} e(\ii) & = (y_k \psi_k + \delta_{i_k, i_{k+1}}) e(\ii)     &
  y_{k+1} \psi_k e(\ii) & = (\psi_k y_k  + \delta_{i_k, i_{k+1}} )e(\ii)       \\
  \psi_k y_l & =  y_l \psi_k     &    \mbox{ if }&  l  \neq k, k+1 \\
   \psi_k \psi_m   & =   \psi_m \psi_k      &    \mbox{ if }&      | l-m |  > 1 \\ 
  e(\ii)  & = 0      &    \mbox{ if } &  i_1  \neq 0 \mbox{\rm \,\, mod } p \\
      y_1 e(\ii) &= 0  & &  
\end{align}
\begin{align}
      \big(\psi_k\psi_{k+1}\psi_k-\psi_{k+1}\psi_k\psi_{k+1}\big)e(\ii)=
        \begin{cases}
          -e(\ii)&\text{if }i_{k+2}=i_k\rightarrow i_{k+1}\\
          e(\ii)&\text{if }i_{k+2}=i_r\leftarrow i_{k+1}\\
          0&\text{otherwise}
        \end{cases}
\end{align}


\begin{align}\label{finalrelations}
  \psi_k^2 e(\ii) =
\begin{cases}
        (y_k-y_{k+1}) e(\ii) & \mbox{if }i_k\rightarrow i_{k+1}\neq0\\
        (y_{k}+p-y_{k+1})e (\ii)& \mbox{if }i_k\rightarrow i_{k+1}=0\\
        (y_{k+1}-y_k)e(\ii)& \mbox{if }0\neq i_k\leftarrow i_{k+1}\\
        (y_{k+1}+p-y_k)e(\ii) & \mbox{if } 0=i_k\leftarrow i_{k+1}\\
        0&\mbox{if }i_k=i_{k+1}\\
        e(\ii)&\mbox{otherwise}
\end{cases}        
\end{align}
\end{definition}
\noindent
where $ \ii \cdot s_k = (i_1, i_2, \ldots, i_k, i_{k+1}, \ldots, i_n ) \cdot s_k :=
(i_1, i_2, \ldots, i_{k+1}, i_{k}, \ldots, i_n ) $. 

\medskip
It is easy to check that $ \RKLRZ \otimes_{\Z_{p}} \FF \cong \RKLR $ where $  \RKLR $ is the original
cyclotomic KLR-algebra. Note however, that the degree function for
$  \RKLR $ does {\it not} induce a $ \Z$-grading on $\RKLRZ $, since the relations in 
\eqref{finalrelations} are not homogeneous.

\medskip
We have already alluded to the following Theorem, that was proved by Hu and Mathas in
\cite{hu-mathas2}. 
\begin{theorem}\label{humathas}
There is an isomorphism of $ \Z_{(p)}$-algebras $F: \RKLRZ   \cong  \Z_{(p)}\Si_n    $. 
\end{theorem}

We next recall the diagrammatics for $ \RKLRZ$, as given in \cite{hu-mathas2}.
It is an 
extension of the diagrammatics for $ \RKLR$. 
A {\it KLR-diagram} $ D $ for $ \RKLR$ consists of $ n $ strands connecting $ n $ northern
points with $n$ southern 
points of a(n invisible) rectangle. Crossings are allowed in $ D $, but only crossings involving two strands.
Isotopic diagrams are considered to be equal. 
The strands of $ D $ are decorated with elements of $ \FF$, and the segments
of a strand are decorated with a nonnegative number of dots.
The product $ D_1 D_2 $ of KLR-diagrams $ D_1 $ and $ D_2 $ is realized by vertical concatenation
with $ D_1 $ on top of $D_2 $ where $ D_1 D_2 $ is set to zero if the bottom residue sequence
for $ D_1 $ does not coincide with the top
residue sequence for $ D_2 $. 
Here is an example of a KLR-diagram, using $ n= 6$
and $  p = 3$. 
\begin{equation}
  \raisebox{-.45\height}{\includegraphics[scale=0.7]{dib72.pdf}}
\end{equation}
The diagrammatics for $ \RKLRZ$ is given by 


\begin{equation}
 \begin{split}
    &e(\ii) \mapsto  \raisebox{-.5\height}{\includegraphics[scale=0.8]{dib73.pdf}}, \, \, \, \, \, \, 
    y_l e(\ii) \mapsto  \raisebox{-.5\height}{\includegraphics[scale=0.8]{dib74.pdf}} \\
    &  \psi_k e(\ii) \mapsto  \raisebox{-.5\height}{\includegraphics[scale=0.8]{dib75.pdf}}    
 \end{split}
 \end{equation}
Via this, one can convert the relations \eqref{firstR} -- \eqref{finalrelations}
into a set of  diagrammatic relations for $ \RKLRZ $. 

\medskip

We now have the following Theorem which is an extension of
Theorem 3.2 and Remark 3.7 of \cite{PlazaRyom} to the integral case. 
\begin{theorem}\label{steendavid}
  Let $ n \ge 3$. If $ p > 3 $ then
the homomorphism $ \Phi $ from Lemma \ref{wellknownfunda} induces 
an isomorphism between $ \TLnZp $ and the quotient of $ \RKLRZ  $ given by the relation
\begin{align}\label{rel1}
  e(\ii)  & = 0      &    \mbox{ if } &  i_1  = 0 \mbox{\rm \, mod }  p, \, \, \, 
  i_2  =1 \mbox{\rm \, mod }  p  \mbox{ and }
  i_3  = 2 \mbox{\rm \, mod }  p 
\end{align}  
If $ p=3 $ then $ \Phi $ induces an isomorphism between $ \TLnZp $ and
the quotient of $ \RKLRZ \! \!  $
given by the relation
\begin{align}\label{rel2}
  y_3 e(\ii)  & = 0      &    \mbox{ if } &  i_1  = 0 \mbox{\rm \, mod }  p, \, \, \, 
  i_2  = 1 \mbox{\rm \, mod }  p  \mbox{ and }
  i_3  = 2 \mbox{\rm \, mod }  p 
\end{align}  
\end{theorem}
\begin{dem}
  The proof from \cite{PlazaRyom} carries over.
  It uses properties of Murphy's standard basis 
  that also hold in the present case. These properties lead to a description of 
  $ {\rm ker }\,  \psi $ as the ideal in $  \RKLRZ \! $, given by \eqref{rel1} and
  \eqref{rel2}.
\end{dem}



\medskip
We need the basic ingredients in Hu-Mathas's proof of \ref{humathas}, in the special
case $ \Z_{(p)}  \Si_n$ that we are considering. 


\medskip
Let $\{ x_{\s \T}^{\lambda} \, | \, ( \s, \T) \in \std(\lambda)^{\times 2}, \lambda \in \Par \} $
be the specialization $ q= 1 $ of Murphy's standard basis for the Hecke algebra of type $ A_n$,
see \cite{Mat} and \cite{Murphy1}. As already mentioned in the proof of 
Theorem 
\ref{followingkeyresult}, 
it is a cellular basis for $ \Z_{(p)}  \Si_n $ on poset $ (\Par,
\trianglelefteq) $, and the elements $ \{ L_1, L_2, \ldots, L_n \} $ defined in \eqref{defJM}
form a family of $ \JM$-elements for $ \Z_{(p)} \Si_n $ with respect 
to the content function defined in \eqref{contentdef}. 
For $ \QQ \Si_n $, these $ \JM$-elements 
are separating, and so for $ \T \in \std(\lambda) $ we have an idempotent $ E_{\T} \in \QQ \Si_n$,
using the formula in \eqref{IdempotentHecke1}. For $ \s, \T \in \std(\Par) $ we define
\begin{equation} 
f_{\s \T}:= E_{ \s} x_{ \s \T } E_{\T} \in \QQ \Si_n
\end {equation}
Then the elements $ \{ f_{\s \T} \, | \, (\s, \T) \in \std(\lambda)^2, \lambda \in \Par \} $ form
a $ \QQ$-basis for $ \QQ \Si_n$.


\medskip
For $ \lambda \in \ParTwo $ and 
$ \s, \T \in \std(\lambda) $ we define similarly elements $ f_{\s \T} $ in $  \TLnQ$, denoted the
same way, via
\begin{equation}\label{denoted the same way}
f_{\s \T}:= \EE_{ \s} C^{\lambda}_{ \s \T } \EE_{\T} \in \TLnQ
\end {equation}
that form 
a $ \QQ$-basis for $ \TLnQ$. For $ \Phi: \QQ \Si_n \rightarrow \TLnQ $ the homomorphism
from Lemma \ref{wellknownfunda} we have that
$
\Phi(x_{\s \T}^{\lambda} ) = C_{\s \T}^{\lambda} + \mbox{ higher terms}  
$, 
where the higher terms are a linear combination of $ C_{\s_1 \T_1} $ with
$ \s_1 \rhd \s $ and $ \T_1 \rhd \T $, see Theorem 9 of \cite{Harterich}. Using this,
and that $ \Phi(L_i) = \LL_i$ and therefore $ \Phi(E_\T) = \EE_\T $ for $ \T \in \std(\ParTwo) $ we get that 
\begin{equation}\label{compatibility}
  \Phi(f_{\s \T}) = f_{\s \T} \, \, \, \, \, \, \mbox{ for } \s, \T \in \std(\ParTwo)
\end{equation}  

\medskip
For $ p $ a prime and $ \T \in \std(\Par)$ we define the $p$-class $ [ \T] \subseteq \std(\Par) $, 
as in \eqref{firstclass}. There is a well-defined function
from $p$-classes to residue sequences, given by $ [ \T] \mapsto \ii^{\T} := (c_\T(1), c_\T(2), \ldots, c_\T(n) )$. 


\medskip
In the proof of the isomorphism Theorem in \ref{humathas}, Hu and Mathas construct 
left and right actions of 
$e(\ii) $, $ y_l $ and $\psi_k  $ 
on $   \Z_{(p)}  \Si_n   $, by defining their actions
on $ \{ f_{\s \T } \}$. Let us explain the
formulas that they used for this.

\medskip
The formulas for $  e(\ii) $ are the simplest. They are given by 
\begin{align}\label{fofei}
&  e(\ii)  f_{\s \T  } :=
\begin{cases} 
    f_{\s \T }  &\mbox{ if } \ii^{\s} = \ii \\
  0 & \mbox{ if } \ii^{\s} \neq \ii
\end{cases}   
&   f_{\s \T  }  e(\ii) :=
\begin{cases} 
   f_{\s \T }  &\mbox{ if } \ii^{\T} = \ii \\
  0 & \mbox{ if } \ii^{\T} \neq \ii
\end{cases}
\end{align} 


\medskip
The formulas for $  y_l  $ correspond to taking the nilpotent part
of the $ \JM$-element $ L_i$, just as in the proof of the original isomorphism
Theorem. For $ i \in \Z $  
let $ \hat{i} \in  \Z$ be given via integer division such that 
$ 0 \le  \hat{i}  \le p-1$ and $ \hat{i} \equiv i \, \, {\rm mod }\, \, p   $ and 
consider $ \hat{i} $ as an element of $ \Z_{(p)} $. 
Then 
\begin{align}\label{fofeiJM}
& y_l  f_{\s \T  }:=
     \left(c_{\s}(l) - \widehat{c_{\s}(l)}\right)  f_{\s \T  }   &
  f_{\s \T  } y_l :=
     \left(c_{\T}(l) - \widehat{c_{\T}(l)}\right)  f_{\s \T  }  
\end{align}  






The formulas for $  \psi_k $ are a bit more complicated, but also the
most important for us. 





\medskip
For $ \s \in \std(\lambda) $ where $ \lambda \in \Par$
and $ k =1,2, \ldots, n-1 $ we set $ \T := \s s_k $ and
$ r= r_\s(k) := c_{\s}(k) - c_{\T}(k)  $. We then 
define $\alpha =  \alpha_\s(k) \in \QQ$ via
\begin{align}\label{alphaF}
  &   \alpha_\s(k) :=
\begin{cases}  
  1 & \text{ if } \T\in \std(\lambda) \text{ and } \T \lhd \s \\
  \dfrac{r^2-1}
       {r^2} & \text{ if } \T\in \std(\lambda) \text{ and } \T \rhd \s \\
0 & \text{ otherwise } 
\end{cases}    
\end{align}  
In the terminology of \cite{hu-mathas2}, 
$ \alpha_\s(k)  $ is a choice of a {\it seminormal coefficient system}.
It is the 'canonical choice' of a seminormal coefficient system, 
since it corresponds to the 'non-diagonal' part of YSF,
see Corollary \ref{YSFsecond}.



\medskip
In order to define the action of $  \psi_k   $ it is enough to define
the left action of 
$ \psi_k e(\ii)    $
and the right action of $  e(\ii) \psi_k    $. 
Suppose that $ \ii^\s=(i_1, i_2,\ldots, i_k , i_{k+1}, \ldots, i_n) $. 
We first define  
$ \beta= \beta_\s(k) \in \QQ $ and $ \widehat{\beta}= \widehat{\beta}_\s(k) \in \QQ $ via
\begin{align}\label{thendefinebeta}
&     \beta_{\s}(k) := 
\begin{cases}  
  \dfrac{\alpha}{1-r}  & \text{ if } i_k \equiv i_{k+1}   \, \,     {\rm mod } \, \, p \\
 \alpha r  & \text{ if } i_k \equiv i_{k+1} +1    \, \,     {\rm mod } \, \,  p \\
 \dfrac{\alpha r}{1-r} & \text{ otherwise } 
\end{cases} 
& \widehat{\beta}_{\s}(k) := 
\begin{cases}  
  \dfrac{\alpha}{1+r}  & \text{ if } i_k \equiv i_{k+1}   \, \,     {\rm mod } \, \, p \\
- \alpha r  & \text{ if } i_k \equiv i_{k+1} -1    \, \,     {\rm mod } \, \,  p \\
 -\dfrac{\alpha r}{1+r} & \text{ otherwise } 
\end{cases}
\end{align}
Let $ \aaa \in \std(\lambda)$. 
We then have 
\begin{align}
&   \psi_k e(\ii) f_{\s \aaa  } :=
\begin{cases} 
    \beta  f_{\T \aaa  }   - \delta_{i_k , i_{k+1}} \dfrac{1}{r} f_{\s \aaa  } &\mbox{ if } \ii^{\s} = \ii \\
  0 & \mbox{ if } \ii^{\s} \neq \ii
\end{cases} \label{fofei2}    \\
& f_{\aaa \s   }   e(\ii) \psi_k :=
\begin{cases} 
    \widehat{\beta}  f_{\aaa \T   }   - \delta_{i_k , i_{k+1}} \dfrac{1}{r} f_{\aaa \s   } &\mbox{ if } \ii^{\s} = \ii \\
  0 & \mbox{ if } \ii^{\s} \neq \ii
\end{cases} \label{fofei2B}
\end{align}  



The formulas in  \eqref{fofei} -- \eqref{fofei2B} are a key
ingredient in Hu and Mathas' proof of Theorem \ref{humathas}, see
Lemma 4.23 in \cite{hu-mathas2}. Note that 
the formulas \eqref{fofei} -- \eqref{fofei2B} in fact over-determine $ F(e(\ii) ),  F(y_l  ) $ and $ F( \psi_k ) $,
since already the left action on the basis $ \{ f_{\s \T} \} $ is enough to determine 
$ F(e(\ii) ),  F(y_l  ) $ and $ F( \psi_k ) $. In other words, the left action determines the right action and
vice versa. 

\medskip

We now return to the homomorphism $ \Phi: \Z_{ (p)} \Si_n \rightarrow \TLnZp$ from Lemma
\ref{wellknownfunda}. We have the following compatibility Theorem. 
\begin{theorem}
 The actions of $ \Phi( e(\ii)) $,  $ \Phi( y_l ) $ and  $ \Phi(  \psi_k ) $ 
are given by the formulas in 
\eqref{fofei} -- \eqref{fofei2},
with the only difference that $ f_{\s \T} $ is now the element of $ \TLnQ$ defined in \eqref{compatibility}.
\end{theorem}
\begin{dem}
  This is an immediate consequence of \eqref{compatibility} and
  the definitions in \eqref{fofei} -- \eqref{fofei2}. 
\end{dem}  


\section{Seminormal form for $  \ee   \TLnF   \ee$}\label{Seminormal form for}
We write for simplicity $ \ee := \EE_{ [ \T_N]} \in \TLnZp$, that is 
$ \ee := \Phi( e(\ii) ) $ where $ \ii=(0,-1, -2, \ldots, -n+1 ) $ is the decreasing residue sequence.
This is an idempotent in $ \TLnZp$ and so we obtain an 
{\it idempotent truncated} subalgebra $ \ee   \TLnZp   \ee$ of $ \TLnZp $.
This subalgebra plays an important role for what follows. 
To a certain extent, this runs parallel
to several recent papers, for example \cite{LiPl} and \cite{LPR}, where 
similar idempotent 
truncated algebras have been studied. 
By general principles, 
$ \ee   \TLnZp   \ee$ 
is a subalgebra of $ \TLnZp $, but with 
one-element $ \ee $.

\medskip
Under the isomorphism from Theorem \ref{steendavid}, the elements
$ \ee   \TLnZp   \ee$ are linear combinations of KLR-diagrams that have top
and bottom residue sequences both equal to
$ \ii = (0,-1, -2, \ldots, -n+1 ) $. 

\medskip
Recall from \eqref{kindofrecursive} that we have fixed natural numbers
$ N_1 $, $ N_2 $ and $ R $ such that $ N= (p-1) +N_1 $ and
$ N_1= pN_2 +R $. As in Lemma \ref{kindofrecursive} 
we furthermore have blocks $ B_1, B_2. \ldots, B_{N_2} $ of length $ p$ of consecutive natural numbers. 
The largest number of $ B_i $ is $ I := (i+1)p-1 $ and we define $ S_i \in \Si_n $ as 
\begin{equation}
  S_i := s_I (s_{I-1}s_{I+1}) \cdots (s_{I-p+1} s_{I-p+3} \cdots s_{I+p-3} s_{I+p-1}) \cdots
 (s_{I-1}s_{I+1})    s_I 
\end{equation}
$ S_I $ is a reduced expression for the element of $ \Si_n $ that interchanges the blocks $ B_i $ and $ B_{i+1} $,
respecting the orders of the elements of each block.
We then define $ \UUU_i $ as the element of $ e(\ii ) \RKLRZ \!  e(\ii ) $
that is obtained from $ S_i $ by converting each $ s_j $ to
$ \psi_j $, and finally multiplying on the left and on the right by $ \ee$.
Similar elements have been considered before in \cite{KMR}, \cite{LiPl} and \cite{LPR},
but only for the original KLR-algebra $ \RKLR$ defined over a field. In \cite{LiPl} and \cite{LPR}, 
the $ \UUU_i$'s are called {\it diamonds}. For example, for $ n = 14 $ and $ p=3 $ we have
\begin{align}\label{diamonds} 
& \UUU_1 = \raisebox{-.5\height}{\includegraphics[scale=0.75]{dib76.pdf}} & 
& \UUU_2 = \raisebox{-.5\height}{\includegraphics[scale=0.75]{dib77.pdf}} 
\end{align} 

\medskip 
Our goal is to describe the left and right actions
on the $ \QQ $-basis $ \{f_{\s \T } \} $ for $ \TLnQ $. 
For this we have the following surprising Theorem, which may be
viewed as 
a generalization of Theorem \ref{YSFfirst}, and then also of
Corollary \ref{YSFsecond},
that is YSF, to the non-semisimple setting. As we shall see, its proof
relies on \eqref{fofei} and \eqref{fofei2}, and so
ultimately on Hu and Mathas's proof of the isomorphism Theorem 
\ref{humathas}. It is valid for $ \ee   \TLnZp   \ee$ and $ \ee   \TLnF   \ee$. 





\begin{theorem}\phantomsection\label{YSFthird}
 Suppose that $ \s , \aaa \in [\T_N] \cap \std(\ParTwo)$, 
 and that $ i=1,2, \ldots, N_2 -1$. Let $ \T := \s \cdot S_i $ and
 suppose that $ \T $ is a standard tableau. 
If  $ \s \rhd \T $
set $ \s_u := \s $, otherwise 
set $ \s_u := \T   $. Let $ \s_d = \s_u \cdot S_i$. 
In the notation of Lemma \ref{kindofrecursive},
define $ f $ as $ f_1 $ if $ R= 0 $, otherwise as the first component of $ f_2 $. 
Define $ \greekrho := c_{ f( \s_u )}(i ) - c_{ f( \s_u )}(i+1 )  $ and $ X \in \QQ$ via 
\begin{equation}
  X:= \dfrac{  \Bigl( (\greekrho+1)p -1 \Bigr) \Bigl( (\greekrho+1)p -2 \Bigr)  \cdots \Bigl( \greekrho p +1\Bigr) }
  { \Bigl(\greekrho p -1 \Bigr) \Bigl(\greekrho p -2\Bigr)  \cdots \Bigl( (\greekrho-1)p +1\Bigr) }
\end{equation}
with $ p-1 $ factors in decreasing order in numerator as well as denominator. 
Then the left action of $ \UUU_i  $ is given by 
  \begin{description}
  \item[a)]
$ \UUU_i  f_{ \s_d  \aaa} = 
\dfrac{ \greekrho +1 }{ \greekrho }
f_{\s_d \aaa } + \dfrac{ \greekrho^2-1}{X \greekrho^2} f_{\s_u \aaa }$
  \item[b)]
$ \UUU_i f_{\s_u  \aaa }  = 
\dfrac{ \greekrho -1 }{ \greekrho }
f_{ \s_u \aaa } + X f_{ \s_d \aaa}$
    \end{description}
Suppose next that $ \T  $ is not standard. Then $  \UUU_i  $ acts via 
  \begin{description}
  \item[c)]
$ \UUU_i f_{ \s \aaa} = 
 0  \,\, \, \,\, \, \, \, \, \,  \mbox{ if } i, i+1 \mbox{ are in the same column of } f(\s) $ 
 \item[d)] $ \UUU_i f_{ \s \aaa} = 
 2 f_{ \s \aaa}  \, \mbox{ if } i, i+1 \mbox{ are in the same row of } f(\s) $ 
    \end{description}
\end{theorem}
\begin{dem}
  Let us first prove $ {\bf b) } $. The proof is a book-keeping of the coefficients
  that arise from the applications via \eqref {fofei2} of the $ \psi_i $'s that appear in   
  $ \UUU_i$. By the assumptions, in $\s_u $ the block of numbers $ B_i$ is positioned above the
  block of numbers $ B_{i+1} $
  as indicated below. 
\begin{equation}\label{diamond} 
  \s_u= \, \, \,  \raisebox{-.5\height}{\includegraphics[scale=0.7]{dib78.pdf}}\, \, \, 
S_i= \,  \raisebox{-.49\height}{\includegraphics[scale=0.7]{dib79.pdf}}
\end{equation}

\medskip
For simplicity we write $ f_{\s_u} = f_{ \s_u \aaa} $ and $ f_{\s_d} = f_{ \s_d \aaa} $. 
We first claim that $ \UUU_i  $ maps 
$ f_{\s_u }   $ to a linear combination of $  f_{\s_u} $ and $ f_{\s_d } $, disregarding the coefficients
for the time being. 

\medskip
To show this claim we proceed as follows. When applying $ \psi_I $ to $ f_{\s_u} $,
corresponding to the top row in the
'diamond' for $ S_i $ in \eqref{diamond}, the residue difference is $ 1 $, as can be read off
from the red numbers in \eqref{diamond}, 
and so by \eqref{fofei2} the result is a scalar multiple of $  f_{\s_u \cdot s_I} $, that is one term.
Next when applying $ \psi_{I+1} $ and $ \psi_{I-1} $ to $ f_{\s_u \cdot s_I } $,
corresponding to the second row in the 
diamond for $ S_i $ in \eqref{diamond}, the residue difference is $ 2 $
and so by \eqref{fofei2} the result is a multiple of $  f_{\s_u \cdot (s_I s_{I-1} s_{I+1})}  $, that is
one term once again. This pattern repeats itself until 
we reach the middle row of the diamond where the  
residue differences are all $ p $, and so by \eqref{fofei2} these $ \psi_i $'s 
produce two terms each, corresponding to the
two terms in \eqref{fofei2}. The tableau of the first term is given by the action by $ s_i $
whereas the tableau of the second is given by the omission of $ s_i $. 
On the other hand, the $ \psi_i $'s in the lower part of the diamond once again only
produce one term each. This pattern of residue differences can be read off from the KLR-diamonds as well,
see \eqref{diamonds}.

\medskip
We conclude from this that $ \UUU_i $ maps $ f_{\s_u} $ to a linear combination of
$  f_{\s_u \cdot \sigma} $ where $ \sigma $ is a subexpression of $ S_i $ obtained from $ S_i $ by deleting
certain of the $ s_i $'s from the middle row of $ S_i $ and
where $ \s_u \cdot \sigma $ is standard. If $ \sigma $ is the subexpression obtained by 
deleting all the $ s_i $'s of the 
middle row, the resulting term is $   f_{\s_u \cdot \sigma} =  f_{\s_u }  $ and if no $ s_i $ is deleted
the resulting term is $   f_{\s_u \cdot \sigma} =  f_{\s_d }  $, of course.

\medskip
We must however also consider the {\it mixed} cases where some of the $ s_i$'s from
the middle row of $ S_i$ are deleted, but not all. In these cases
we may use Coxeter relations to move a generator
$ s_i \neq s_I $ to the top of $ S_i $ and so we deduce that $ \s_u \cdot \sigma $ is not standard.
Here is an example, using $ p = 5$, and
the indicated tableau $ \s_u$. 

\begin{equation}\label{here is ok} 
  \s_u= \, \, \,  \raisebox{-.5\height}{\includegraphics[scale=0.6]{dib80.pdf}} \, \,
      \raisebox{-.5\height}{\includegraphics[scale=0.6]{dib81.pdf}}
\end{equation}
It follows from this observation that the part of the action of
$  \psi_{ \sigma}  $ on $ f_{\s_u} $ that gives rise to $ f_{ \s_u \cdot \sigma } $ 
must involve the third case of \eqref{alphaF}, for
at least one of the $ \psi_i$'s, since the other cases produce standard tableaux.
But then the result is zero, proving that the mixed cases do not contribute to the action of $ \UUU_i  $ 
and so the claim is proved.



\begin{figure}[h]
   \raisebox{-.5\height}{\includegraphics[scale=0.8]{dib82.pdf}} \, \,  
\centering
\caption{Values of $ r $ and $ \beta $ for each row of the diamond.}
\label{fig:here is}
\end{figure}



\medskip
Let us now calculate the coefficent of $ f_{\s_d}  $ under the action of $ \UUU_i  $ on $ f_{\s_u}$.
The contribution to this coefficient for each $ \psi_i $ of the middle row of the diamond is given by
always choosing the first term of 
\eqref{fofei2}. This implies that the coefficient of $ f_{\s_d}  $ always comes from 'going down' and so
$\alpha = 1 $ for all occurrences of  
\eqref{alphaF} involved in the coefficient of $ f_{\s_d}  $.
The value of $ \beta $, according to \eqref{thendefinebeta}, therefore only depends on $ r $ and 
the relevant residue differences, that are constant along the rows of the diamond. 

\medskip

The table in Figure
\ref{fig:here is}
gives the values of $ r $ and $ \beta $ for each row of the diamond, 
where we write $ P := \greekrho p$, for simplicity. 
The colors in the table correspond to the three cases in the definition of $ \beta $ in
\eqref{thendefinebeta}, with red corresponding to the first case, blue to the second case and
black to the third case. 
To get the coefficient of $ f_{\s_d}  $ we must now multiply all the $ \beta$'s of the table in Figure
\ref{fig:here is}, with multiplicities given by the cardinalities of the rows of the diamond.















\medskip
We first claim 
that the sign of this product is $ +$. To show this
we observe that the number of black or red $ \beta$'s in the table in 
\eqref{thendefinebeta} is
$ p^2$ minus the number of blue $ \beta$'s, that is 
$ p^2 - p = p (p-1) $ which is even, proving the claim. 

\medskip
The product of the $ \beta$'s is therefore
\begin{align}
\begin{split}  
&  {\color{blue}{ \frac{ (P-p+1)} {1}}}  \frac{ (P-p+2)^2}{(P-p+1)^2}  \frac{ (P-p+3)^3}{(P-p+2)^3}
\cdots \frac{ (P-1)^{p-1}}{(P-2)^{p-1}}
       {\color{red}{ \frac{ 1} { (P-1)^p }  }}     {\color{blue}{ \frac{ (P+1)^{p-1} } { 1  }  }}
  \cdots 
 \frac{ (P+p-3)^3} {(P+p-4)^3 }  \frac{ (P+p-2)^2} {(P+p-3)^2 } \frac{ (P+p-1)} {(P+p-2) }  = \\ & \\
 &   {\color{blue}{ \frac{ \cancel{(P-p+1)}} {1}}}
 \frac{ \cancel{ (P-p+2)^2}}{(P-p+1)^{ \cancel{2}}}  \frac{ \cancel{(P-p+3)^3}}{(P-p+2)^{\cancel{3}}}
\cdots \frac{\cancel{ (P-1)^{p-1}}}{(P-2)^{ \cancel{p-1}}}
       {\color{red}{ \frac{ 1} { (P-1)^{\cancel{p }}}  }}     {\color{blue}{ \frac{ (P+1)^{\cancel{p-1}} } { 1  }  }}
  \cdots 
 \frac{ (P+p-3)^{\cancel{3}}} {\cancel{(P+p-4)^3 }}  \frac{ (P+p-2)^{\cancel{2}}} {\cancel{(P+p-3)^2} } \frac{ (P+p-1)} {\cancel{(P+p-2)} } =  
\\ & \\
 & \frac{ (P+1)} {(P-p+1)} \frac{ (P+2)} {(P-p+2)} \cdots
 \frac{ (P+p-2)} {(P-2) }  \frac{ (P+p-1)} {(P-1) } 
\end{split}
\end{align}
Remembering that $ P= \greekrho p $, we conclude from this that the coefficient of
$  f_{\s_d}  $ is 
$ X $ as claimed.

\medskip
In order to determine the coefficient of $  f_{\s_u}  $ we use the same method as for
the coefficient of $  f_{\s_d}  $, with the difference
that this time $ \alpha $ is 'going down' only until reaching the middle row of diamond in which it 
'stand still', and after this point, corresponding to the lower part
of the diamond, $ \alpha $ is 'going up' again. 
Thus the table for
$  f_{\s_u}  $ coincides with the table in Figure \ref{fig:here is} in the upper half of the diamond, but
differs from it in the middle row and below. 
Using the definitions of $ r $, $ \alpha $ and $ \beta$, we then get the following table, where we use
the same color scheme as in Figure \ref{fig:here is}, and once again $ P := \greekrho p $. 



\begin{equation}\label{here is once again} 
   \raisebox{-.5\height}{\includegraphics[scale=0.8]{dib83.pdf}} \, \,
\end{equation}

We must calculate the product of the $\beta$'s that appear in 
\eqref{here is once again}. There is only one $ \beta$
appearing with a positive sign in \eqref{here is once again}, namely the one 
in the first row,  
and so the sign of the product of all the $ \beta$'s is $ (-1)^{ p^2 -1} = 1 $, since $ p $ is an odd
prime. It is now easy to calculate the product of the $ \beta$'s: indeed multiplying the $ \beta $
of the first row with the 
$\beta$ of the last row, the $ \beta$'s of the second row with the $ \beta$'s of the second last row, and so on, 
we find that the product of the $ \beta$'s is  
\begin{equation}
\dfrac{P-p}{P} = \dfrac{\greekrho p -p}{\greekrho p } = \dfrac{\greekrho  -1}{\greekrho} 
\end{equation}  
which proves $ {\bf b) } $

\medskip
The other parts of the Theorem are proved with the same methods and are left to the reader.
\end{dem}

\medskip
We have the following variant of Theorem \ref{YSFthird} describing the right action
of $ \UUU_i $ on $ \{ f_{\s \T} \} $. Note that the formulas for the right action are the same as the formulas
for the left action, except that $ X $ should be replaced by $ \dfrac{1}{X}$. 

\begin{theorem}\phantomsection\label{YSFthirdB}
Let the notation be the same as in Theorem \ref{YSFthird}. 
Then the right action of $ \UUU_i  $ is given by 
  \begin{description}
  \item[a)]
$  f_{\aaa  \s_d  }  \UUU_i = 
\dfrac{ \greekrho +1 }{ \greekrho }
f_{\aaa \s_d  } + \dfrac{X (\greekrho^2-1)}{ \greekrho^2} f_{\aaa \s_u  }$
  \item[b)]
$  f_{\aaa \s_u   }  \UUU_i = 
\dfrac{ \greekrho -1 }{ \greekrho }
f_{ \aaa \s_u  } + \dfrac{1}{X} f_{\aaa  \s_d }$
    \end{description}
Suppose that $ \T  $ is not standard. Then $ \UUU_i  $ acts via 
  \begin{description}
  \item[c)]
$f_{ \aaa  \s}  \UUU_i = 
 0  \,\, \, \,\, \, \, \, \, \,  \mbox{ if } i, i+1 \mbox{ are in the same column of } f(\s) $ 
 \item[d)] $ f_{\aaa  \s } \UUU_i = 
 2 f_{\aaa  \s }  \, \mbox{ if } i, i+1 \mbox{ are in the same row of } f(\s) $ 
    \end{description}
\end{theorem}



\medskip
Statements similar to the one of the following Corollary, but for
the original KLR-algebra $ \RKLR$ defined over a field, are already present in literature,
see for example \cite{KMR}, \cite{LPR} and \cite{LiPl}, although the proofs in these references
are different from ours, since they rely on KLR-diagrammatics. 
\begin{corollary}\label{corKLRincl}
  Let $ N_2 $ be chosen as \eqref{fixnotation}. 
  Then there is an injection of Temperley-Lieb algebras 
\begin{equation}\label{thereisaninjection}
  \iota_{KLR}:  \TLZpntwo \rightarrow
       {\mathbb {TL}}_{N}^{\! {\mathbb Z}_{(p)} }, \, \, \,  \UU_i \mapsto
\left\{ f_{\s \T} \mapsto \UUU_i f_{\s \T} \right\} 
       \mbox{ for } i =1,2, \ldots, N_2 -1
\end{equation}    
\end{corollary}

\begin{dem}
  We must show that the left action of the $\UUU_i$'s verify the Temperley-Lieb relations \eqref{eq oneTL},
\eqref{eq twoTL} and \eqref{eq threeTL}. The quadratic relation \eqref{eq oneTL} follows
  immediately from Theorem \ref{YSFthird}, since the $ 2 \times 2 $-matrix $ { \mathbf{M}_{\UUU_i}} $
  expressing the left action of $ \UUU_i $ in terms of $ \{ f_{\aaa \T_d }, f_{\aaa \T_u } \} $ has the form 
\begin{equation}
 \mathbf{M}_{\UUU_i} =  \begin{bmatrix}
\dfrac{\greekrho +1}{\greekrho} & X  \\ 
\dfrac{\greekrho^2 -1}{X\greekrho} &   \dfrac{\greekrho -1}{\greekrho}
\end{bmatrix}
\end{equation}
which satisfies $  \mathbf{M}_{\UUU_i}^2 =  \mathbf{M}_{\UUU_i} $.

\medskip
In order to show relation \eqref{eq twoTL}, we choose $ \s , \T \in [\T_N] \cap \std(\ParTwo)$ 
and show that the left action of $ \UUU_i\UUU_{i \pm 1}\UUU_i $ on $f_{ \s \T} $ is equal to the left action
of $ \UUU_i $ on $f_{ \s \T} $. 
Let us focus on  $ \UUU_i\UUU_{i+ 1}\UUU_i $.
We then consider the positions of $ i, i+1 $ and $ i+2 $ in $ f(\s) $ where
$ f $ is as in Theorem \ref{YSFthird}. If $ i, i+1 $ and $ i+2 $ are in different rows of $ f(\s) $,
we have the following possibilities $ \s_1, \s_2, \ldots, \s_6 $ for $ f(\s)$. 


\begin{equation}\label{hereA} 
  \raisebox{-.5\height}{\includegraphics[scale=0.65]{dib84.pdf}} \, \,
\end{equation}
One now checks for all $ j = 1,2,\ldots, 6 $ that indeed $  \UUU_i\UUU_{i+ 1}\UUU_i f_{\s_j \T}  = 
 \UUU_i f_{\s_j \T}  $. 
For example, using $ \greekrho := c_{\s_2}(i) - c_{\s_1}(i) $ one gets, using Theorem \ref{YSFthird}
repeatedly
\begin{align}
\begin{split}
  &   \UUU_i\UUU_{i+ 1}\UUU_i  f_{\s_1 \T} =
   \UUU_{i+ 1}\UUU_i  \left(\dfrac{\greekrho +1 }{\greekrho} f_{\s_1 \T}+
\dfrac{\greekrho^2 -1 }{X\greekrho^2}
f_{\s_2 \T}\right)   =
\dfrac{\greekrho^2 -1 }{X\greekrho^2}  \UUU_{i+ 1}\UUU_i
f_{\s_2 \T}    \\
& =   
\dfrac{\greekrho^2 -1 }{X\greekrho^2}  \UUU_i \left(  \dfrac{\greekrho  }{\greekrho-1} f_{ \s_2 \T} +
\dfrac{(\greekrho -1)^2 -1 }{X_1(\greekrho -1)^2} 
f_{  \s_3 \T} \right)   = \UUU_i  \left( \dfrac{\greekrho +1 }{X\greekrho}f_{ \s_2 \T} 
\right)   \\
& 
=  \dfrac{\greekrho +1 }{X\greekrho}  \left( \dfrac{\greekrho -1 }{\greekrho} f_{ \s_2  \T} + X f_{ \s_1 \T} \right )
=  \dfrac{\greekrho^2 -1 }{X\greekrho^2}   f_{ \s_2 \T} +  \dfrac{\greekrho +1 }{\greekrho} f_{ \s_1 \T} 
\end{split}
\end{align}
which equals $ \UUU_i  f_{ \s_1 \T}  $. For the other $ \s_j$'s, the verification of
$  \UUU_i\UUU_{i+ 1}\UUU_i f_{\s_j \T}  = 
 \UUU_i  f_{\s_j \T}  $. 
is
done the same way. 

\medskip
If two of the numbers $ i, i+1 $ and $ i+2$ are in the same 
row of $ f(\T)$ we have the following possibilities

\begin{equation}\label{hereAB} 
  \raisebox{-.5\height}{\includegraphics[scale=0.65]{dib85.pdf}} \, \,
\end{equation}
and in each case one checks that $   \UUU_i\UUU_{i+ 1}\UUU_i  $ and
$   \UUU_i $ act the same way.
The verification of $ \UUU_i\UUU_{i-1 }\UUU_i  = \UUU_i $ is done the same way, 
and finally the verification of relation \eqref{eq threeTL} is trivial. 


\medskip
In order to show injectivity of $ \iota_{KLR} $ one first checks 
that throughout the above arguments, one may always replace left actions by right actions.
(This also follows from the theory in 
\cite{hu-mathas2}).

\medskip
Let now $ \{ C_{ \s \T} \, |\,  \s, \T \in \std(\lambda), \lambda \in \ParTwoNuno \} $
be the basis for $\TLZpntwo $, as introduced in the paragraph before 
\eqref{diagrambasisTL}.
From the formulas in 
Theorem \ref{YSFthird} we have that $ \U, \V \in \std(\mu) $ with $ \mu \in \ParTwoNuno $ 
and $ C_{ \s \T}  f_{\U \V } \neq 0 $ implies $ \U \unrhd \T$, and similarly, from the formulas in
Theorem \ref{YSFthirdB}, we have that
$ f_{\U \V } C_{ \s \T}  \neq 0 $ implies $ \V \unrhd \s$.
Moreover, we also that $ C_{ \T^{\lambda} \U}  f^{}_{\U \V } = \mu^l_{ \U} f_{\T^{\lambda} \V}  $ where $ \mu^l_{  \U} \neq 0$
and that $ f^{}_{\U  \V } C_{ \V \T^{\lambda}} =  \mu^r_{ \V } f_{\U \T^{\lambda} }  $ where $  \mu^r_{ \V } \neq 0$
and where $ \U, \V $ are of shape $ \lambda$. 

\medskip
Suppose now that  
$ 0 \neq C=  \sum_{\s,\T} \lambda_{\s \T} C_{\s \T}
\in  {\rm ker} \, \iota_{KLR}  $. Choose $ (\s_0, \T_0 ) $ such that  
$  \lambda_{\s_0 \T_0} \neq 0 $ and such that
$ (\s_0, \T_0 ) $ is minimal with respect to this property. Then, using 
$ C f_{\s_0 \T_0 } =0  $ we get
$0=  f_{\T_0 \s_0 } C f_{\s_0 \T_0 } = \lambda_{ \s_0 \T_0}c \, \mu^l_{  \s_0} \mu^r_{ \T_0 } f_{\T_0 \T_0}   $, 
where $ c \neq 0 $, 
which implies $ \lambda_{ \s_0 \T_0}  =0 $. This is however 
a contradiction, and so the injectivity of $  \iota_{KLR} $ has been proved. 


\end{dem}


\medskip
Let $ \{  \LLL_1, \LLL_2, \ldots, \LLL_{N_2}  \} $ be the family 
of $ \JM$-elements in $ \TLZpntwo $ 
given by
$ \LLL_i := \Phi( L_i) $ where $ \{  L_1, L_2, \ldots, L_{N_2}  \} 
\subseteq   \Z_{(p)} \Si_{N_2}$ is the original family of $ \JM$-elements in 
\eqref{defJM} and where 
$\Phi: \Z_{(p)} \Si_{N_2} \rightarrow  \TLZpntwo $
is the surjection from Lemma \ref{wellknownfunda}. 
Using the general theory in \cite{Mat-So}, 
we then obtain idempotents $ \EEE_{\T} \in \TLQpntwo$
for $ \T \in \ParTwoNuno$  
that are common eigenvectors for
the $   \LLL_i$'s, via the construction in \eqref{mainconstruction} and
Corollary \ref{finalcorsection2}.
On the other hand,
the inclusion $  \iota_{KLR}:  \TLZpntwo \rightarrow \TLZpn$ from Corollary
\ref{corKLRincl} induces an inclusion $\iota_{KLR}^{\QQ}: \TLQpntwo \subseteq \TLnQ $ and
so we may view the $  \EEE_{\T}$'s as idempotents in $ \TLnQ $ via $ \iota_{KLR}^{\QQ} $.

\medskip
Our next goal is to show that, quite surprisingly,
these new idempotents $ \{  \EEE_{\T} \, | \,  \T \in \std(\ParTwoNuno) $,
viewed as elements
in $ \TLnQ $,  
are closely related to the first idempotents $\{  \EE_{\T} \, | \, \T \in \std(\ParTwo) \}$
in $ \TLnQ $. We start with the following Lemma, which should be compared with
Lemma \ref{2.4}. 

\begin{lemma}\label{inductionbasis}
  Let $ \lambda \in \std(\ParTwo) $ and suppose that 
  $ \T= \T_{\lambda}  \in [\T_N] \cap \std(\ParTwo)$ and that
  $ \aaa \in  [\T_N] \cap \std(\lambda)$. 
Set $ \s:=  f(\T_{\lambda}) \in \ParTwoNuno $ where
$ f $ is as in Theorem \ref{YSFthird}. Let $f_{\T \aaa} $ and $f_{ \aaa \T} $
be as in \eqref{denoted the same way}. 
Then for for $ i =1,2,\ldots, N_2 $ we have that 
\begin{equation}
  \LLL_i f_{\T \aaa}  =  c_{ \s }(i) f_{\T \aaa} \, \, \, \, \,\, \,    \mbox{and} \, \, \, \, \, \, \,
f_{\T \aaa} \LLL_i  =  c_{ \s }(i) f_{\T \aaa}
  \end{equation}  
 \end{lemma}
\begin{dem}
Let us show the formula for the left action of $ \LLL_i $. 
Letting $ l_1 $ and $ l_2 $ be the column lengths of $ \s $ we have that
\begin{equation}\label{hereABC} 
\s=  \raisebox{-.5\height}{\includegraphics[scale=0.6]{dib86.pdf}} \, \,
\end{equation}
Once again, we use the recursive formula 
$
  \LLL_{i+1} = (\UUU_i-\one) \LLL_i (\UUU_i-\one) +  \UUU_i-\one. 
$
Together with $ {\bf c) } $ of Theorem \ref{YSFthird}, it reduces the proof to the case $ l_2 = 1 $
and $ i = l_1+1$ where we must show that 
\begin{equation}\label{518}
\LLL_{l_1 +1} f_{\T \T}   =  f_{\T \T} 
\end{equation}
We do so by induction over $ l_1 $. 
The basis of the induction, corresponding to $ l_1 =1 $, is the affirmation that
$ \LLL_{2}  f_{\T \aaa}  =  f_{\T \aaa} \Longleftrightarrow  (\UUU_1- \one) f_{\T\aaa}  =  f_{\T \aaa} $ which is true
by $ {\bf d) } $ of Theorem \ref{YSFthird}.

\medskip
To show the inductive step
$ l_1 -1 \Longrightarrow l_1 $ we write for simplicity $ l:= l_1 $, 
$ \T_d := \T $ and $ \T_u := \T \cdot s_l $ and get via
$ {\bf a) } $ and $ {\bf b) } $ of Theorem \ref{YSFthird} that 
\begin{align}
\begin{split}
&       \LLL_{l +1} f_{\T \aaa} =  \Big(\UUU_l-\one) \LLL_l (\UUU_l-\one) +  \UUU_l-\one \Big)  f_{\T_d \aaa} 
  = \LLL_l (\UUU_l-\one)  \Big(  \dfrac{1}{l} f_{\T_d \T} + \dfrac{l^2-1}{X\, l^2}  f_{\T_u \T}\Big) +
  \Big(  \dfrac{1}{l} f_{\T_d \aaa} + \dfrac{l^2-1}{X \, l^2}  f_{\T_u \aaa}\Big)  \\
& = (\UUU_l-\one) \Big(  \dfrac{1-l}{l} f_{\T_d \aaa} + \dfrac{l^2-1}{X\, l^2}  f_{\T_u \aaa}\Big)   +
  \Big(  \dfrac{1}{l} f_{\T_d \aaa} + \dfrac{l^2-1}{X \, l^2}  f_{\T_u \aaa}\Big)  =
 \UUU_l \Big(  \dfrac{1-l}{l} f_{\T_d \aaa} + \dfrac{l^2-1}{X\, l^2}  f_{\T_u \aaa}\Big)  + f_{\T_d \aaa} \\
 & =  \dfrac{1-l}{l} \UUU_l  \Big(   f_{\T_d \aaa} - \dfrac{l+1}{X\, l} f_{\T_u \aaa}\Big) + f_{\T_d} = f_{\T_d \aaa} =
 f_{\T \aaa}
\end{split}
\end{align}
The proof of the formula for the right action is done the same way. 

\end{dem}  

\medskip
The previous Lemma is the basis step
for the inductive proof of 
the following Theorem which should be compared with
Theorem \ref{mentionedabove}. 
\begin{theorem}\label{indutionseminormalKLR}
Suppose that 
$ \T, \aaa   \in [\T_N] \cap \std(\ParTwo)$. 
Set $ \s:=  f(\T) \in \ParTwoNuno $ where
$ f $ is as in Theorem \ref{YSFthird}.
Then for $ i =1,2,\ldots, N_2  $ we have that 
\begin{equation}
  \LLL_i f_{\T \aaa }   = c_{ \s }(i) f_{\T \aaa } \, \, \, \, \,\, \,    \mbox{and} \, \, \, \, \, \, \, 
  f_{ \aaa \T} \LLL_i  = c_{ \s }(i) f_{ \aaa \T }   
\end{equation}
\end{theorem}
\begin{dem}
  As already indicated, the proof is by upwards induction over the dominance
  order in $ \std(\lambda) $, with Lemma \ref{inductionbasis} corresponding to the induction basis.
  The induction step is carried out the same way as the induction step in the proof of
  Theorem \ref{mentionedabove}, with Theorem \ref{YSFthird} replacing Theorem \ref{YSFsecond}. 
  The extra factors $ X$ or $ 1/X $ 
  in the equations corresponding to
 \eqref{YSFinduction}--\eqref{compareB} do not affect the conclusion. 
\end{dem}

\medskip
We have two Corollaries to Theorem \ref{indutionseminormalKLR}. 
\begin{corollary}\label{idempotentJMcor}
  Let $ \T $ and $ \s $ be as in Theorem \ref{indutionseminormalKLR}
  and let $ \EE_{\T} \in \TLnQ$ be the idempotent from Corollary \ref{finalcorsection2}. 
  Then we have
\begin{equation}
 \LLL_i \EE_{\T}   = \EE_{\T}  \LLL_i = c_{ \s }(i) \EE_{\T} \, \mbox{ for } i =1,2,\ldots, N_2 
\end{equation}
\end{corollary}
\begin{dem}
  This follows directly from
Theorem \ref{indutionseminormalKLR} together with 
  the construction of $ \EE_{\T}$ in \eqref{mainconstruction}
  and Corollary \ref{finalcorsection2}. 
\end{dem}

\begin{corollary}\label{idempotentJMcordos}
Suppose that 
$ \T  \in [\T_N] \cap \std(\ParTwo)$ and that 
$ \s \in \ParTwoNuno $ where $ N_2 $ is as in Lemma \ref{313}. 
 Let $\iota_{KLR}^{\QQ}: \TLQpntwo \subseteq \TLnQ $ be the inclusion
given by Corollary
\ref{corKLRincl}. Then we have
\begin{align}
 &  \iota_{KLR}^{\QQ}  \big( \EEE_{\s} \big) \cdot \EE_{\T} =\EE_{\T} \,\cdot  \iota_{KLR}^{\QQ} \big( \EEE_{\s} \big) 
= \begin{cases}
 \EE_{\T} & \mbox{ if } f(\T) = \s \\
    0 & \text{ if } f(\T)  \neq \s 
\end{cases}
\end{align}  
\end{corollary}
\begin{dem}
  Suppose first that $ f(\T) = \s$. Using 
  \eqref{IdempotentHecke1} and Corollary \ref{idempotentJMcor}
  we then get
\begin{equation} \iota_{KLR}^{\QQ}  \big( \EEE_{\s} \big) \cdot \EE_{\T} = 
  \left( \prod_{c \in {\cal C}} \, \, \prod_{\substack{i=1, \ldots, N_2\\ c \neq c_{\s}(i)} } \dfrac{\LLL_i-c}{c_{\s}(i)-c}
  \right)   \EE_{\T} =
\left( \prod_{c \in {\cal C}} \, \, \prod_{\substack{i=1, \ldots, N_2\\ c \neq c_{\s}(i)} } \dfrac{c_{\s}(i)-c}{c_{\s}(i)-c}
  \right)   \EE_{\T} =\EE_{\T} 
\end{equation}
as claimed.
Suppose next that $ f(\T) \neq \s$. Then there is $ i \in \{1,2,\ldots, N_2 \} $ such that
$ c_{ f(\T)}(i) \neq c_\s(i)$, since the separability condition \eqref{separationcondition}
is fulfilled, and so $  \iota_{KLR}^{\QQ}  \big( \EEE_{\s} \big)  $ has $ \big(\LLL_i - c_{ f(\T)}(i)  \big) $
as a factor. But by Corollary \ref{idempotentJMcor} we have
$ \big(\LLL_i - c_{ f(\T)}(i)  \big) E_{\T} = 0 $ which implies $ \iota_{KLR}^{\QQ}  \big( \EEE_{\s} \big) E_{\T} =0 $.
The formula for the right action is proved the same way. 

\end{dem}

\medskip
In the rest of the article we shall show how $ ^{p}\!\JWn  $ fits into the picture. Recall from 
\eqref{padic} the expansion of $ n+1 $ in base $ p$
\begin{equation}
 n+1 = a_k p^k + a_{k-1} p^{k-1} + \ldots +a_1 p + a_0 
\end{equation}
From \eqref{fixnotation} we have $ n=N=N_1 + (p-1) $ and $ N_1 = pN_2 +R$
and so we get 
\begin{equation}
 R = a_0\, \,\mbox{ and }    N_2+1 = a_k p^{k-1} + a_{k-1} p^{k-2} + \ldots +a_1
\end{equation}
Repeating this process we find that $ N, N_1, N_2  $ and $ R $ belong to sequences of natural numbers
$ N^i, N_1^i, N_2^i $ and $ R^i$ where 
$ N:=N^0, N_1=N_1^0, N_2=N_2^0 $ and $ R=R^0 $ and where 
\begin{equation}
N^i=N_1^i + (p-1), \,\, \,   N^i_1 = pN^i_2 +R^i, \, \, \, N^{i+1} = N_2^i \,  \mbox{  for }  i =0,1,\ldots, k-1 
\end{equation}
In fact we have 
\begin{equation}
 R^i = a_i\, \,\mbox{ and }    N_2^i+1 = a_k p^{k-i-1} + a_{k-1} p^{k-i-2} + \ldots +a_{i+1}
\end{equation}
Using Corollary \ref{thereisaninjection} we then get a chain of injections
\begin{equation}\label{thechain}
  {\mathbb {TL}}_{N^{k-1}_2}^{\! {\mathbb Z}_{(p)} } \subseteq
  {\mathbb {TL}}_{N^{k-2}_2}^{\! {\mathbb Z}_{(p)} } \subseteq  \cdots \subseteq  
  {\mathbb {TL}}_{N_2^0}^{\! {\mathbb Z}_{(p)} }  =
  {\mathbb {TL}}_{N}^{\! {\mathbb Z}_{(p)} } =   {\mathbb {TL}}_{n}^{\! {\mathbb Z}_{(p)} } 
\end{equation}
Let $ \EEE_{[\T_{N^{i}_2}]} $ be the class idempotent for
$ {\mathbb {TL}}_{N_2^i}^{\! {\mathbb Z}_{(p)} }  $. We may view it as an idempotent of
$  {\mathbb {TL}}_{n}^{\! {\mathbb Z}_{(p)} } $ via the chain in \eqref{thechain}.

\medskip
We can now prove the following main Theorem. It  establishes the promised connection between the 
$p$-Jones-Wenzl idempotents and KLR-theory for the Temperley-Lieb algebra, via the seminormal form
approach to KLR-theory. 
\begin{theorem}\label{finalTheo}
  In the above setting we have 
\begin{equation}\label{mustshoweq}
  ^{p}\!\JWn  = \prod_{i=0}^{k-1}  \EEE_{[\T_{N^i_2}]}
\end{equation}  
\end{theorem}
\begin{dem}
$ \EEE_{[\T_{N^0_2}]} = \EEE_{[\T_{n}]} = \sum_{ \T \in [ \T_n ] }\EE_{ \T } $ is the class
  idempotent for $ \TLnZp$ and so we get from Corollary \ref{idempotentJMcordos} that
  the right hand side RHS of \eqref{mustshoweq} has the form 
  \begin{equation}
{\rm RHS} =  \sum_{ \T \in \A_n} \EE_{\T} 
  \end{equation}
  for $ \A $ a subset of $ [ \T_n] $. We must show that $ \A_n = {\mathcal I}_n $, where
  $ {\mathcal I}_n  $ is as in \eqref{mathcal I}.
  Given $ \T \in \A_n $ we 
  must therefore show that $ \T $ is given by blocks 
  $ D_1, M_1, D_2, M_2, \ldots, D_k, M_k $ of consecutive numbers, of the cardinalities specified
  in the paragraphs following \eqref{mathcal I}.

\medskip  
  Let us first
  focus on the numbers $ \{1, 2, \ldots, (a_k p^k-1) \} $ in $ \T $.
  We must show that they all appear in the first column of  $\T$. But 
  as already observed in the preceding sections, since
  $ \EEE_{[\T_{N^0_2}]} $ is a factor of RHS 
  certainly the numbers $ \{1, 2,\ldots, p-1 \} $ all
  appear in the first column of $ \T$.

  \medskip
  We then consider the 
  numbers $ \{p,p+1, \ldots (a_k p^k-1) \} $ grouped in blocks of cardinality $ p$. But 
  $ \EEE_{[\T_{N^1_2}]} $ is a factor of RHS, from which we deduce that the first $ p-1 $ of these blocks
  also appear in the first column of $ \T$. We have $ (p-1)+(p-1)p = p^2-1$ and so we have shown that 
  the numbers $ \{1, 2, \ldots, p^2-1 \} $ appear in the first column of $ \T$.
  We next consider blocks of cardinality $ p^2 $ and use the factor 
  factor $ \EEE_{[\T_{N^2_2}]} $ to show that in fact
  the numbers $ \{1, 2, \ldots, p^3-1 \} $ all appear in the first column of $ \T$. Repeating
  this argument, we finally get that the numbers $ \{1, 2, \ldots, (a_k p^k-1) \} $ all appear in
  the first column of $ \T $, as claimed.

  \medskip
  But using that $ \EEE_{[\T_{N^{k-2}_2}]} $ is a factor of RHS, we now get 
that 
  the $ a_{k-1} p^{k-1}  $ numbers following
  $ \{1, 2, \ldots, (a_k p^k-1) \} $ all appear either in first or in the second column of
  $ \T$. Repeating this last argument recursively, we finally find that $ \T \in \A_n $, as claimed.
  The Theorem is proved.


  
\end{dem}
  \begin{thebibliography}{X}


  






  

 \bibitem{brundan-klesc} J. Brundan, A. Kleshchev,
 \textit{Blocks of cyclotomic Hecke algebras and Khovanov-Lauda algebras}, Invent. Math. {\bf 178}, (2009), 451-484.


\bibitem{BCH} C. Bowman, A. Cox, A. Hazi, \textit{Path isomorphisms between quiver Hecke and diagrammatic Bott-Samelson endomorphism algebras}, arXiv:2005.02825. 



\bibitem{BLS}
G. Burrull, N. Libedinsky, P. Sentinelli, \textit{p-Jones-Wenzl idempotents}, 
Adv. Math. {\bf 352}, (2019),  246-264.



\bibitem{EL} B. Elias, N. Libedinsky, {\it  Indecomposable Soergel bimodules for universal Coxeter groups.
  With an appendix by Ben Webster}, Trans. Am. Math. Soc.
{\bf 369}(6) (2017), 3883-3910. 



\bibitem{EhrigStroppel}
M. Ehrig, C. Stroppel, \textit{Koszul gradings on Brauer algebras}, 
Int. Math. Res. Not. {\bf 2016} (13), (2016), 3970-4011. 

\bibitem{EH} K. Erdmann, A. Henke, \textit{On Ringel duality for Schur algebras, Math. Proc. Cambridge Philos. Soc}. {\bf 132}, (2002), 97–116.



\bibitem{GL} J. J. Graham, G. I. Lehrer,
  \textit{Cellular algebras}, Inventiones Mathematicae {\bf 123}, (1996), 1-34.


\bibitem{GL2} J. J. Graham, G. I. Lehrer,
  \textit{The representation theory of affine Temperley-Lieb algebras},
  Enseign. Math., II. S\'er. {\bf 44}(3-4), (1998), 173-218.


  
\bibitem{GW} F. M. Goodman, H. Wenzl,
  \textit{The Temperley-Lieb Algebra at roots of unity}, 
Pacific Journal of Mathematics {\bf 161}(2), (1993), 307-334. 

\bibitem{GW2} F. M. Goodman, H. Wenzl,
\textit{Ideals in the Temperley-Lieb category}, appendix to: \textit{A mathematical
  model with a possible Chern-Simons phase} by M. Freedman, Comm. Math. Phys. {\bf 234}, (2003), 
129–183.




\bibitem{HMP}  
A. Hazi, P. Martin, A. Parker, \textit{Indecomposable tilting modules for the blob algebra},
Journal of Algebra {\bf 568}, (2021), 273-313. 


\bibitem{hu-mathas} J. Hu, A. Mathas, \textit{Graded cellular bases for the cyclotomic Khovanov-Lauda-Rouquier
  algebras of type $A$}, Adv. Math., {\bf 225}, (2010), 598-642.

\bibitem{hu-mathas2}
 J. Hu, A. Mathas, \textit{
Seminormal forms and cyclotomic quiver Hecke algebras of type A}, 
Math. Ann. {\bf 364}(3-4), (2016), 1189-1254. 

\bibitem{Harterich} M. H\"arterich, 
\textit{Murphy bases of generalized Temperley-Lieb algebras}, 
Arch. Math. {\bf 72}(5), (1999), 337-345.


\bibitem{H1} J. E. Humphreys, {\it Reflection groups and Coxeter groups}, volume {\bf 29} of Cambridge Studies in
Advanced Mathematics. Cambridge University Press, Cambridge, (1990).


\bibitem{Jo} V.F.R. Jones, {\it Index for subfactors}, Invent. Math. {\bf 72}(1), (1983), 1–25.


\bibitem{KhovanovLauda} M. Khovanov, A. Lauda, \textit{A diagrammatic approach to categorification of quantum groups I}, Represent. Theory {\bf 13}, (2009), 309-347.

\bibitem{KMR}  
A. Kleshchev, A. Mathas, A. Ram, {\it 
Universal graded Specht modules for cyclotomic Hecke algebras}, 
Proc. Lond. Math. Soc. (3) {\bf 105}(6),  (2012), 1245-1289. 

  
  
\bibitem{LiPl} N. Libedinsky, D. Plaza, \textit{Blob algebra approach to modular representation theory},
Proc. of the London Math. Soc. ({\bf 121})(3), (2020), 656-701.

\bibitem{Lo} D. Lobos, \textit{On Generalized blob algebras: Vertical idempotent truncations and Gelfand-Tsetlin subalgebras}, arXiv:2203.15139. 


\bibitem{LPR} D. Lobos, D. Plaza, S. Ryom-Hansen,
\textit{The nil-blob algebra: an incarnation of type $\tilde{A}_1$
Soergel calculus and of the truncated blob algebra}, 
Journal Algebra {\bf 570}, (2021),  297-365. 




\bibitem{Lobos-Ryom-Hansen} D. Lobos, S. Ryom-Hansen,
  \textit{Graded cellular basis and Jucys-Murphy elements
  for generalized blob algebras},
Journal of Pure and Applied Algebra, 
{\bf 224}(7), (2020), 106277, 1-40.





\bibitem{PMartin} P. Martin, 
  \textit{Potts models and related problems in statistical mechanics},
  Series on Advances in Statistical Mechanics, 5. Singapore etc.: World Scientific. xiii, 344 pages, 
  (1991).
  


\bibitem{Mat} A. Mathas,
\textit{Hecke algebras and Schur algebras of the symmetric group},
Univ. Lecture Notes, 15, A.M.S., Providence, R.I., (1999).



\bibitem{Mat-So} A. Mathas, \textit{Seminormal forms and Gram determinants for cellular algebras}, J.
Reine Angew. Math., {\bf 619}, (2008), 141-173.  With an appendix by M. Soriano.



\bibitem{Murphy2} G. E. Murphy, \textit{A new construction of Young’s seminormal representation of the symmetric groups},
J. of Algebra {\bf 69}, (1981), 287-297.


\bibitem{Murphy} G. E. Murphy, \textit{The idempotents of the symmetric group and
Nakayama's conjecture}, J. of Algebra {\bf 81}, (1983), 258-265.

\bibitem{Murphy1} G. E. Murphy, \textit{The Representations of Hecke Algebras of type
$ A_n $}, J. of Algebra {\bf 173}, (1995), 97-121.

\bibitem{Murphy3} G. E. Murphy, \textit{On the Representation Theory of the Symmetric Groups and associated Hecke Algebras},
J. of Algebra {\bf 152}, (1992),, 492-513.


\bibitem{Orm} K. Orme\~no, \textit{Elementos de Jucys-Murphy en el \'algebra de Temperley-Lieb},
Tesis de magister, Universidad de Talca.




\bibitem{PlazaRyom}
D. Plaza,  S. Ryom-Hansen, \textit{Graded cellular bases for Temperley-Lieb algebras of type A and B},
Journal of Algebraic Combinatorics, {\bf 40}(1), (2014), 137-177.

\bibitem{Rouquier}
R. Rouquier, \textit{2-Kac-Moody algebras}, arXiv:0812.5023.




\bibitem{steen} S. Ryom-Hansen, \textit{Jucys-Murphy elements for Soergel bimodules}, Journal of Algebra, {\bf 551}, (2020), 154-190.

\bibitem{StuSpe}  
M. Stuart, R. A. Spencer, 
\textit{$(\ell,p)$-Jones-Wenzl idempotents}, 
J. Algebra {\bf 603}, (2022), 41-60. 


\bibitem{SuTuWeZhu}
L. Sutton, D. Tubbenhauer, P. Wedrich, J. Zhu, 
\textit{$SL_2 $ tilting modules in the mixed case},
arXiv:2105.07724. 




\bibitem{TuWe1}
D. Tubbenhauer, P. Wedrich, 
\textit{Quivers for $SL_2$
tilting modules}, 
Represent. Theory {\bf 25}, (2021), 440-480.





\bibitem{TuWe}
D. Tubbenhauer, P. Wedrich, 
\textit{The center of $SL_2$
tilting modules}, 
Glasg. Math. J. {\bf 64}(1), (2022), 165-184.



\bibitem{Wenzl} H. Wenzl, {\it On sequences of projections}, C. R. Math. Rep. Acad. Sci. Can. 9 {\bf 1},
  (1987), 5–9.





  \end{thebibliography}

 \end{document}





