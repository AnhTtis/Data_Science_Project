%%%%%%%%%%%%%%%%%%%%%%%%%%%%%%%%%%%%%%%%%%%%%%%%%%%%%%%%%%%%%%%%%%%%%%%%%%%%%%%%
%2345678901234567890123456789012345678901234567890123456789012345678901234567890
%        1         2         3         4         5         6         7         8

\documentclass[letterpaper, 10 pt, conference]{ieeeconf}  % Comment this line out
                                                          % if you need a4paper
%\documentclass[a4paper, 10pt, conference]{ieeeconf}      % Use this line for a4
                                                          % paper

\IEEEoverridecommandlockouts                              % This command is only
                                                          % needed if you want to
                                                          % use the \thanks command
\overrideIEEEmargins
% See the \addtolength command later in the file to balance the column lengths
% on the last page of the document



% The following packages can be found on http:\\www.ctan.org
%\usepackage{graphics} % for pdf, bitmapped graphics files
%\usepackage{epsfig} % for postscript graphics files
%\usepackage{mathptmx} % assumes new font selection scheme installed
%\usepackage{times} % assumes new font selection scheme installed
%\usepackage{amsmath} % assumes amsmath package installed
%\usepackage{amssymb}  % assumes amsmath package installed

\title{\LARGE \bf
A Bandit Learning Method for Continuous Games under Feedback Delays with Residual Pseudo-Gradient Estimate 
}

%\author{ \parbox{3 in}{\centering Huibert Kwakernaak*
%         \thanks{*Use the $\backslash$thanks command to put information here}\\
%         Faculty of Electrical Engineering, Mathematics and Computer Science\\
%         University of Twente\\
%         7500 AE Enschede, The Netherlands\\
%         {\tt\small h.kwakernaak@autsubmit.com}}
%         \hspace*{ 0.5 in}
%         \parbox{3 in}{ \centering Pradeep Misra**
%         \thanks{**The footnote marks may be inserted manually}\\
%        Department of Electrical Engineering \\
%         Wright State University\\
%         Dayton, OH 45435, USA\\
%         {\tt\small pmisra@cs.wright.edu}}
%}

\usepackage{hyperref}
\usepackage{url}
\usepackage{amssymb}
% \usepackage[english]{babel}

\usepackage[dvipsnames]{xcolor}
\usepackage{mathrsfs}
\usepackage{bbm, dsfont}
\let\labelindent\relax % Error: Command \labelindent already defined
\usepackage{amsmath, amsfonts, outlines, mathtools, outlines, cancel, enumitem, float, stackengine, graphicx, romannum}
% accents: \bar below symbol
\usepackage{subcaption}
\usepackage[english]{babel}
\usepackage[utf8]{inputenc}
\usepackage[T1]{fontenc}
\usepackage{newtxtext,newtxmath}
\usepackage{algorithm, algpseudocode}
% \usepackage[ruled,vlined]{algorithm2e}
% \usepackage{algorithmic}

\setenumerate[1]{label=(\roman*)}

% % calligraphic latters
% % Ralph Smith’s Formal Script Font (rsfs)
% \usepackage{mathrsfs} % $\mathscr{ABCDEFGHIJKLMNOPQRSTUVWXYZ}$
% % Euler Script font: Use the “euscript” package
% \usepackage[mathscr]{euscript} % $\mathscr{ABCDEFGHIJKLMNOPQRSTUVWXYZ}$

\newtheorem{theorem}{Theorem}
\newtheorem{example}{Example}
\newtheorem{assumption}{Assumption}
\newtheorem{lemma}{Lemma}
\newtheorem{remark}{Remark}
\newtheorem{propst}{Proposition}
\newtheorem{envdef}{Definition}
\newtheorem{coro}{Corollary}
\raggedbottom

\DeclareMathSizes{10}{9}{6}{5}
\allowdisplaybreaks


\definecolor{purple}{rgb}{1, 0, 1}

\newcommand{\ie}{\emph{i.e.,}\xspace}
\newcommand{\eg}{\emph{e.g.,}\xspace}
\newcommand{\abr}{\emph{abbr.}\xspace}
\newcommand{\ea}{\emph{et al.}\xspace}
\newcommand{\gensync}{\emph{GenSync}\xspace}
\newcommand{\colosseum}{\emph{Colosseum}\xspace}
\newcommand{\srep}{\emph{SREP}\xspace} % Set Reconciliation Enhances
\newcommand{\srepsim}{\emph{SREPSim}\xspace}
% Propagation
\newcommand{\esrep}{\emph{E-SREP}\xspace}
\newcommand{\epsrep}{\emph{EP-SREP}\xspace}
\newcommand{\mesrep}{\emph{ME-SREP}\xspace}
\newcommand{\mempoolsync}{\emph{MempoolSync}}

\newcommand{\fref}[1]{Fig.~\ref{#1}}
\newcommand{\tref}[1]{Table~\ref{#1}}
\newcommand{\aref}[1]{Algorithm~\ref{#1}}
\newcommand{\procref}[1]{Procedure~\ref{#1}}
\newcommand{\sref}[1]{Section~\ref{#1}}
\newcommand{\lineref}[1]{line~\ref{#1}}
\newcommand{\appref}[1]{Appendix~\ref{#1}}

% Change \eqref
\LetLtxMacro{\originaleqref}{\eqref}
\renewcommand{\eqref}{Eq.~\originaleqref}

% Theorems and corollaries
\newcounter{theoremcount}
\setcounter{theoremcount}{0}
\DeclareRobustCommand{\theorem}[1]{%
  \refstepcounter{theoremcount}%
  \noindent\textit{\textbf{Theorem \thetheoremcount\label{theorem:#1}: }}%
}
\DeclareRobustCommand{\theoremref}[1]{Theorem~\ref{theorem:#1}}

\DeclareRobustCommand{\proof}{\emph{Proof:}\xspace}
\DeclareRobustCommand{\qqed}{\hfill$\blacksquare$}

\newcounter{corollcount}
\setcounter{corollcount}{0}
\DeclareRobustCommand{\coroll}[1]{%
  \refstepcounter{corollcount}%
  \noindent\textit{\textbf{Corollary \thecorollcount\label{coroll:#1}: }}%
}
\DeclareRobustCommand{\corollref}[1]{Corollary~\ref{coroll:#1}}

\newcounter{lemmacount}
\setcounter{lemmacount}{0}
\DeclareRobustCommand{\lemma}[1]{%
  \refstepcounter{lemmacount}%
  \noindent\textit{\textbf{Lemma \thelemmacount\label{lemma:#1}: }}%
}
\DeclareRobustCommand{\lemmaref}[1]{Lemma~\ref{lemma:#1}}

\newcounter{definitioncount}
\setcounter{definitioncount}{0}
\DeclareRobustCommand{\definition}[1]{%
  \refstepcounter{definitioncount}%
  \noindent\textit{\textbf{Definition \thedefinitioncount\label{definition:#1}: }}%
}
\DeclareRobustCommand{\defref}[1]{Definition~\ref{definition:#1}}

%notes of different authors
\newif\ifnotes
\notestrue
\notesfalse

\newif\ifdiff
\difftrue
\difffalse

\newcommand{\anote}[1]{\ifnotes $\ll$\textsf{\textcolor{purple}{Ari: {#1}}}$\gg$ \fi}
\newcommand{\nnote}[1]{\ifnotes $\ll$\textsf{\textcolor{orange}{Novak: {#1}}}$\gg$ \fi}
\newcommand{\diff}[1]{\ifdiff\textcolor{orange}{#1}\else#1\fi}

%%% Local Variables:
%%% mode: latex
%%% TeX-master: "main"
%%% End:

\pdfminorversion=4

\author{Yuanhanqing Huang$^{1}$ and Jianghai Hu$^{1}$% <-this % stops a space
\thanks{This work was supported by the National Science Foundation under Grant No. 2014816 and No.2038410. }% <-this % stops a space
\thanks{$^{1}$The authors are with the Elmore Family School of Electrical and Computer Engineering, Purdue University, West Lafayette, IN, 47907, USA 
        {\tt\small \{huan1282, jianghai\}@purdue.edu}}%
}


%% Switch between the conference proceeding version and the arXiv version
\newif\ifproceeding
\newif\ifarxiv

% Uncomment to generate the version for the conference proceeding
% \proceedingtrue
% \arxivfalse

% Uncomment to generate the version for arXiv 
\proceedingfalse
\arxivtrue

\begin{document}



\maketitle
\thispagestyle{empty}
\pagestyle{empty}


% %%%%%%%%%%%%%%%%%%%%%%%%%%%%%%%%%%%%%%%%%%%%%%%%%%%%%%%%%%%%%%%%%%%%%%%%%%%%%%%%
\begin{abstract}
Learning in multi-player games can model a large variety of practical scenarios, where each player seeks to optimize its own local objective function, which at the same time relies on the actions taken by others. 
Motivated by the frequent absence of first-order information such as partial gradients in solving local optimization problems and the prevalence of asynchronicity and feedback delays in multi-agent systems, we introduce a bandit learning algorithm, which integrates mirror descent, residual pseudo-gradient estimates, and the priority-based feedback utilization strategy, to contend with these challenges. 
We establish that for pseudo-monotone plus games, the actual sequences of play generated by the proposed algorithm converge a.s. to critical points. 
Compared with the existing method, the proposed algorithm yields more consistent estimates with less variation and allows for more aggressive choices of parameters. 
Finally, we illustrate the validity of the proposed algorithm through a thermal load management problem of building complexes. 


\end{abstract}


% %%%%%%%%%%%%%%%%%%%%%%%%%%%%%%%%%%%%%%%%%%%%%%%%%%%%%%%%%%%%%%%%%%%%%%%%%%%%%%%%
\section{INTRODUCTION}

With the proliferation of cyber-physical engineering systems and modern network applications, the non-cooperative multi-player game has emerged as a valuable tool for modeling and investigating the decision-making process of agents with interest conflicts \cite{li2022confluence}. 
Each participant in the game seeks to unilaterally optimize its own objective, whose value also depends on the action taken by others. 
Notable practical applications include thermal load management of autonomous buildings \cite{jiang2021game}, supply-side risk management in power markets \cite{kannan2013addressing}, power control in wireless communication \cite{zhou2021robust}, path planning and control of self-driving cars \cite{liniger2019noncooperative}, etc. 

Over the past few decades, the control and optimization communities have devoted significant effort to developing solution algorithms for non-cooperative games by reformulating them as variational inequalities \cite{facchinei2003finite}. 
Recently, there has been growing interest in distributed solutions under partial information settings, as they offer advantages in scalability and privacy preservation \cite{pavel2019distributed, bianchi2022fast, huang2022distributed}. 
Despite their promise in some cases, the applicability of these methods is often limited by the requirement for the existence of first-order/pseudo-gradient oracles or the full knowledge of the objectives, which may not be available in practical settings. 
Prompted by the need to relax the information requirement, researchers approximate the missing pseudo-gradient information with the actions taken and the resulting objective values. 
This problem can then be fit into the framework of bandit online learning \cite{shalev2012online}, where at every updating step, each player selects an action, observes the realized objective value, and updates its strategy according to the observed result and the process repeats. 

Another practical challenge that hinders the implementation in real-world scenarios is the latency between taking action and receiving bandit feedback, which is further exacerbated in multi-agent systems, where agents could experience heterogeneous delays. 
Latency can arise as a result of significant communication delays or the fundamental limitation that certain actions take time to manifest their effects. 
In the context of routing problems \cite{vu2021fast}, assessing the effectiveness of a navigation strategy entails waiting for a driver to execute the instructions, operate the vehicle, and record the time elapsed. 
In light of the preceding consideration, the primary objective of this work is to propose a bandit online learning algorithm for multi-player continuous games that can ensure convergence despite the presence of feedback delays.  

\textit{Related Work:} 
\Tblue{
In the context of bandit learning in games with instantaneous feedback, Bravo et al. \cite{bravo2018bandit} introduced a bandit mirror descent (MD) method that ensures a.s. convergence when the game is strictly monotone. 
The single-point pseudo-gradient estimate is obtained via the simultaneous perturbation stochastic approximation (SPSA) approach \cite{agarwal2010optimal}. 
In the context of strongly monotone games and their variants, the algorithms proposed in \cite{lin2021optimal, tatarenko2022rate, tatarenko2023convergence, drusvyatskiy2022improved} similarly employ single-point estimates of the pseudo-gradient and attain a $\mathcal{O}(1/t^{1/2})$ convergence rate.  
The single-point estimates are also applied in \cite{tatarenko2020bandit} and \cite{gao2022bandit} for merely monotone games and their variants. 
% In terms of convergence rate for strongly monotone games, Lin et al.'s \cite{lin2021optimal} contribution was notable for raising the rate from $O(1/t^{1/3})$ to $O(1/t^{1/2})$ via a barrier-based method. 
% Furthermore, Tatarenko et al. \cite{tatarenko2022rate} also reported a comparable acceleration in convergence rate for strongly monotone games. 
Given the susceptibility of single-point estimates to large variances, a critical factor impacting the efficiency of algorithms, Tatarenko et al. \cite{tatarenko2022rate} introduced the two-point estimate. 
This strategy mitigates variance-related issues and enhances the convergence rate to $\mathcal{O}(1/t)$ for strongly monotone games. 
In the field of zeroth-order optimization, Zhang et al. \cite{zhang2022new} considered a residual feedback scheme to control the estimation variance. 
By integrating residual pseudo-gradient estimate into the single-call extra-gradient scheme, Huang et al. \cite{huang2023zeroth} developed two bandit algorithms.
The proposed algorithms only require a single query per iteration and ensure a.s. convergence for pseudo-monotone plus games and achieve $O(1/t^{1 - \epsilon})$ convergence rate for strongly monotone games. 
% To extend the scope of games beyond strictly monotone cases, Tatarenko et al. \cite{tatarenko2020bandit} and Gao et al. \cite{gao2022bandit} leverage Tikhonov regularization and second-order learning dynamics in their proposed algorithm, respectively. 
}

To contend with the feedback delays in games, Huang et al. \cite{huang2022convergence} proposed an algorithm based on the improved accelerated gradient descent for potential games, which can tackle cases ranging from sublinear delays to superlinear delays. 
Zhang et al. \cite{zhang2022multi} focused on the general-sum Markov games where the agents are impacted by heterogeneous reward delays and proposed the delay-adaptive multi-agent V-learning to procure coarse-correlated equilibria. 
Of particular relevance is \cite{heliou20agradient}, in which Helious et al. delved into the development of a no-regret bandit learning algorithm for strictly monotone games corrupted by homogeneous sublinear reward delays. 
Nevertheless, the delicate balance between bias and variance of the proposed method is elusive and requires careful calibration. 
Moreover, its stringent requirements on step sizes and query radius hinder its applicability. 

\textit{Contributions:} 
\Tblue{
First, we propose a bandit learning algorithm under feedback delays, where the delays can be heterogeneous but upper-bounded by a constant or homogeneous with a sublinearly growing upper bound. 
Our algorithm integrates mirror descent, residual pseudo-gradient estimates, and the priority-based feedback utilization strategy. 
It is the first algorithm that employs the variance control strategy via single-point residual estimates in the scenario of bandit learning with delays. 
Second, we establish the a.s. convergence of the proposed algorithm for pseudo-monotone plus games. 
While some of the proving techniques have been previously established in \cite{huang2023zeroth}, this paper places additional emphasis on addressing the error caused by delays, which can complicate the problem, particularly when two subsequent realized objective values are required for each single estimate. 
Compared to the existing method in \cite{heliou20agradient}, the proposed algorithm in this work maintains a constant upper bound for the estimation variance and relaxes the conditions on step size and query radius by incorporating the residual pseudo-gradient estimates. 
In addition, we evaluate the performance of the solution algorithms using the thermal load management problem of buildings. 
Compared to the existing work, the proposed algorithm achieves faster and more consistent convergence. 
\ifproceeding
Due to the page limit, complete proofs are presented in \cite{huang2023bandit}. 
\fi
}


\textit{Basic Notations:} 
For a set of vectors $\{v_i\}_{i \in S}$, $[v_i]_{i \in S}$ or $[v_1; \cdots; v_{|S|}]$ denotes their vertical stack. 
For a vector $v$ and a positive integer $i$, $[v]_i$ denotes the $i$-th entry of $v$. 
Denote $\nset{}{+} \coloneqq \nset{}{} \backslash \{0\}$ and $\rset{}{++} \coloneqq (0, +\infty)$. 
We let $\norm{\cdot}_2$ represent the Euclidean norm, $\norm{\cdot}$ a general norm, and $\norm{\cdot}_*$ its dual.
For a set $\mathcal{S}$, let $\mathds{1}_{\mathcal{S}}$ denote the indicator function for this set, i.e., $\mathds{1}_{\mathcal{S}}(x) = 1$ if $x \in \mathcal{S}$ and $0$ otherwise. 
Let $\cl{\mathcal{S}}$ denote the closure of set $\mathcal{S}$, $\text{int}(\mathcal{S})$ the interior, and $\partial \mathcal{S}$ the boundary. 
The symbols $a \wedge b$ and $a \vee b$ stand for the lesser and the greater of the two real numbers $a$ and $b$, respectively. 

%%%%%%%%%%%%%%%%%%%%%%%%%%%%%%%%%%%%%%%%%%%%%%%%%%%%%%%%%%%%%%%%%%%%%%%%%%%%%%%%
\section{SETUP AND PRELIMINARIES}

\subsection{Problem Setup}
In this subsection, we formalize the multi-player continuous game with feedback delays that we will investigate and introduce the assumptions to impose. 
In this $N$-player game $\mathcal{G}$, with the player set given by $\mathcal{N} \coloneqq \{1, \ldots, N\}$, each player $i$ needs to optimize its own local objective by determining its local action $x^i \in \mathcal{X}^i$, where $\mathcal{X}^i \subseteq \rset{n^i}{}$ represents the local strategy space of player $i$. 
For brevity, we let the stack vector $x \coloneqq [x^j]_{j \in \mathcal{N}}$ denote the global action, the stack vector $x \coloneqq [x^j]_{j \in \mathcal{N}_{-i}}$ denote the action taken by all players except player $i$ with $\mathcal{N}_{-i} \coloneqq \mathcal{N} \backslash \{i\}$. 
Similarly, denote the global strategy space $\mathcal{X} \coloneqq \prod_{j \in \mathcal{N}} \mathcal{X}^j \subseteq \rset{n}{}$ with $n \coloneqq \sum_{j \in \mathcal{N}} n^j$. 
Formally, given the action $x^{-i}$ taken by other players, each player $i$ aims to solve the following local problem: 
\begin{align}
\minimize_{x^i \in \mathcal{X}^i} J^i(x^i; x^{-i}). 
\end{align}
The following conditions are imposed regarding the smoothness of objective $J^i$'s and the properties of $\mathcal{X}^i$'s.

\begin{assumption}\label{asp:objt-set}
For each player $i$, the local objective function $J^i$ is continuously differentiable $(C^1)$ in $x$ over the strategy space $\mathcal{X}$.
The individual strategy space $\mathcal{X}^i$ is compact and convex.
Moreover, each $\mathcal{X}^i$ possesses a non-empty interior. 
\end{assumption}

The underlying probability space is given by $(\Omega, \mathcal{F}, \mathbb{P})$. 
One operator we will leverage throughout is the pseudo-gradient operator $F:\mathcal{X} \to \rset{n}{}$, which is defined as the stack of the partial gradient given the smoothness imposed in Assumption~\ref{asp:objt-set}, i.e., 
\begin{align}
F: x \mapsto [\nabla_{x^i} J^i(x^i; x^{-i})]_{i \in \mathcal{N}}. 
\end{align}
The Lipschitz continuity of $F$ then entails the fact that each $J^i$ is $C^1$ and $\mathcal{X}^i$ compact, i.e., there exists some constant $L$, such that for arbitrary $x$ and $y \in \mathcal{X}$, $\norm{F(x) - F(y)}_* \leq L\norm{x - y}$. 
In the same vein, the gradient $\nabla_x J^i: \mathcal{X} \to \rset{n^i}{}$ is also Lipschitz continuous and admits a tighter Lipschitz constant denoted by $L^i$. 
Throughout this work, we will concentrate on the solution concept known as critical points (CPs) \cite[Section~2]{mertikopoulos2022learning}, whose definition is given as follows. 
\begin{envdef}\label{def:cps} (Critical Points)
A decision profile $x_* \in \mathcal{X}$ is a critical point of the non-cooperative game $\mathcal{G}$ if it solves the associated (Stampacchia) variational inequality (VI), i.e., 
\begin{align}\label{eq:cps}
\langle F(x_*), x - x_*\rangle \geq 0, \; \forall x \in \mathcal{X}, 
\end{align}
which is typically denoted by the abbreviation $\text{VI}(\mathcal{X}, F)$. 
\end{envdef}
Besides, the following assumption is postulated regarding the monotonicity of $F$ to facilitate the convergence analysis. 
\begin{assumption}
The pseudo-gradient $F$ is pseudo-monotone plus on $\mathcal{X}$, i.e., $F$ is pseudo-monotone, i.e., for all $x, y \in \mathcal{X}$, $\langle F(y), x - y\rangle\geq 0\implies \langle F(x), x - y\rangle\geq 0$, and satisfies for any action profiles $x, y \in \mathcal{X}$, $\langle F(y), x-y\rangle \geq 0$ and $\langle F(x), x - y\rangle = 0 \implies F(x) = F(y)$. 
\end{assumption}

\Tblue{Pseudo-monotone plus games are a broader class of games than strictly monotone games, but they are not a subset or a superset of merely monotone games. 
Examples of pseudo-monotone plus games that are not merely monotone can be found in \cite[Section~V.A]{huang2023zeroth}\cite{kannan2019optimal}. }

\subsection{Setup for Feedback Delays}
In this work, we consider the scenario where there exists some time lag between the time when an action is taken and the time when the associated realized objective value is received by the player. 
To simplify notation, we let the realized objective value of player $i$ at the $k$-th iteration be denoted by $\hat{J}^i_k$. 
Then, for player $i$, the delay time of $\hat{J}^i_k$ is denoted by $d^i_k$, and this piece of bandit information is available at iteration $\ceil{k + d^i_k}$.
We impose that the delay time should grow at most sublinearly in the iteration $k$ when the delays are homogeneous or be upper bounded by some constant when the delays are heterogeneous, which is formally stated in the assumptions below. 
\begin{assumption}\label{asp:delay}
For each player $i$, the feedback delay $d^i_k$ associated with the realized objective value $\hat{J}^i_k$ is a random variable and $d^i_k \in [0, \bar{d}(k)]$, where $\bar{d}(k) \coloneqq k^{\alpha_d} + \bar{d}$, for some constants $\bar{d} \geq 0$ and $0 \leq \alpha_d < 1$. 
\end{assumption}

\begin{assumption}\label{asp:delay-plus}
Either one of the following statements holds:
\begin{outline}[enumerate]
\1 the delay $d^i_k$ is upper-bounded by a constant $\bar{d}$;
\1 all the players experience the same delay, i.e., $d^1_k = \cdots = d^N_k = d_k$. 
\end{outline}
\end{assumption}

\Tblue{
The issue of handling delays that grow sublinearly or even superlinearly relative to a global clock is receiving increasing attention in the realm of distributed systems \cite{zhou2022distributed}.
For example, in volunteer computing grids, the participation of new and faster workers in the network can undermine the performance of slower workers, causing their computation requests to accumulate quickly over time and resulting in growing delays. 
}

\subsection{Mirror Map and Mirror Descent}
To streamline our subsequent discussion, we briefly introduce mirror descent and related concepts in this subsection. 
The interested readers are referred to \cite[Ch.~4]{bubeck2014theory} for more detailed information. 
Let $\mathcal{B}$ denote a Banach space and $\mathcal{B}^*$ its dual. 
We first let $\psi:\dom\psi \to \rset{}{}$ with $\dom\psi \subseteq \mathcal{B}$ denote a distance generating function (DGF). 
Here, $\dom\psi$ refers to the set where $\psi$ is well-defined and is assumed to be convex and open. 
The DGF $\psi$ satisfies: 
$(\romannum{1})$ $\psi$ is differentiable and $\Tilde{\mu}$-strongly convex for some $\Tilde{\mu} > 0$; 
$(\romannum{2})$ $\nabla \psi(\dom \psi) = \rset{n}{}$; 
$(\romannum{3})$ $\cl{\dom \psi} \supseteq \mathcal{X}$ and $\lim_{x \to \partial (\dom \psi)}$ $\norm{\nabla \psi(x)}_* = +\infty $. 
With the DGF $\psi$ in hand, the mirror map $\nabla \psi^*$ can be defined as:
\begin{align}
\nabla \psi^*(z) = \argmax_{x \in \mathcal{X}} \{\langle z, x\rangle - \psi(x)\}, 
\end{align}
which can be regarded as an extension of projection in general spaces. 
We let $D(\cdot, \cdot): \mathcal{B} \times \mathcal{B} \to \rset{}{}$ represent the Bregman divergence, whose formal expression is given by: 
\begin{align}
D(p,x) = \psi(p) - \psi(x) - \langle \nabla \psi(x), p - x\rangle, \forall p, x \in \dom \psi.
\end{align}
\begin{assumption}\label{asp:recip}
(Bregman Reciprocity) The chosen DGF $\psi$ satisfies that when the sequence $(x_k)_{k \in \nset{+}{}}$ converges to some point $p$, i.e., $\norm{x_k - p} \to 0$, then $D(p, x_k) \to 0$.
\end{assumption}
% We further postulate the following to enable the Bregman divergence $D(p, \cdot)$ to reflect a neighborhood of $p$. 
% The above assumption is imposed to let the Bregman divergence $D(p, \cdot)$ represent a certain distance measure to $p$ and then indicate some neighborhood of $p$. 
The above assumption is introduced to enable the Bregman divergence $D(p, \cdot)$ to function as a specific distance metric with respect to $p$, thereby delineating a particular vicinity around $p$.
The prox-mapping $P_{x, \mathcal{X}}: \mathcal{B}^* \to \dom \psi \cap \mathcal{X}$, induced by the Bregman divergence, is defined as:
\begin{align}
P_{x, \mathcal{X}}(y) = \argmin_{x^\prime \in \mathcal{X}}\{\langle y, x - x^\prime\rangle + D(x^\prime, x)\}, 
\end{align}
which plays an essential role in mirror descent and its variants. 
A lemma characterizing mirror maps and prox-mappings that will be frequently used in the subsequent analysis is given below. 
\begin{lemma}\label{le:md-lips}
Consider the ambient Banach space $\mathcal{B}$ equipped with norm $\norm{\cdot}$ and a closed and convex feasible set $\mathcal{X} \subseteq \cl{\dom \psi} \subseteq \mathcal{B}$. 
Suppose $\psi: \dom \psi \to \rset{}{}$ is a DGF, then the mirror map $\nabla \psi^*$ is $1/\Tilde{\mu}$-Lipschitz continuous, i.e., $\forall y_1, y_2 \in \mathcal{B}^*$, $\norm{\nabla \psi^*(y_1) - \nabla \psi^*(y_2)} \leq (1/\Tilde{\mu})\norm{y_1 - y_2}_*$. 
\end{lemma}
\begin{proof}
See \cite[Lemma~A.1]{huang2023zeroth}. 
\end{proof}
\Tblue{
To solve $\text{VI}(\mathcal{X}, F)$, the mirror descent can be expressed as: 
\begin{align}\label{eq:md}
X_{k+1} = P_{X_k, \mathcal{X}}(-\gamma_k g_k) = \nabla\psi^*(\nabla\psi(x_k) - \gamma_k g_k), 
\end{align}
where in the literature of stochastic VI, $g_k$ usually denotes some noise-corrupted first-order information queried at $X_k$ and $\gamma_k$ an appropriate updating step size. }
One prevalent assumption is that there exists a first-order oracle to generate $g_k$ after observing $X_k$, and given some proper filtration $(\mathcal{F}_k)_{k \in \nset{}{+}}$, it holds that $\expt{}{g_k \mid \mathcal{F}_k} = F(X_k)$ and $\expt{}{\norm{g_k}^2_* \mid \mathcal{F}_k}$ is a.s. bounded. 
The convergence properties of the actual sequences and the ergodic sequences have been extensively studied in \cite{mertikopoulos2019learning, mertikopoulos2018optimistic, juditsky2022unifying}. 

\section{BANDIT MIRROR DESCENT WITH FEEDBACK DELAYS}

\subsection{Residual Pseudo-Gradient Estimate}\label{subsec:rpg}
% u^i_k ... 
% perturbation ... 
% domain shrinkage ...
Our blanket assumption throughout is that the first-order oracle that returns $g_k$ is unavailable, and each player can only observe its realized objective value associated with the action taken. 
To address the absence of first-order information, we leverage a pseudo-gradient estimate called the residual pseudo-gradient estimate (RPG) \cite{huang2023zeroth} to approximate the missing information from the observed objective values. 
At each iteration $k$, initially, it is necessary to undertake the following perturbation step: 
\begin{align}\label{eq:perb}
\begin{split}
& \hat{X}^i_{k} = (1 - \frac{\delta_k}{r^i})X^i_{k}+\frac{\delta_k}{r^i}(p^i + r^iu^i_k) = \bar{X}^i_{k} + \delta_k u^i_k,
\end{split}
\end{align}
where 
$u^i_k$ is randomly sampled from the unit sphere in the $n^i-$dimensional Euclidean space and we define $u_k \coloneqq [u^i_k]_{i \in \mathcal{N}}$; 
$\delta_k$ represents the random query radius at iteration $k$;
$\mathbb{B}(p^i, r^i) \subseteq \mathcal{X}^i$ is an arbitrary fixed ball within the feasible set $\mathcal{X}^i$ that centers at $p^i$ with radius $r^i$; 
$\bar{X}^i_{k} \coloneqq (1 - \delta_k/r^i)X^i_{k} + (\delta_k/r^i) p^i$. 
The RPG associated with the states $X_{k}$ at $k$-th iteration leverages the realized objective values from the current iteration $\hat{J}^i_k \coloneqq J^i(\hat{X}^i_k; \hat{X}^{-i}_k)$ and the previous iteration $\hat{J}^i_{k-1} \coloneqq J^i(\hat{X}^i_{k-1}; \hat{X}^{-i}_{k-1})$, which is formally given by 
\begin{align}\label{eq:rpe}
G^i_k \coloneqq \frac{n^i}{\delta_k}(\hat{J}^i_k - \hat{J}^i_{k-1})u^i_k.
\end{align}

To analyze the properties of RPG, a smoothed version for each local objective function $J^i$ is leveraged:
\begin{align}
\Tilde{J}^i_{\delta}(x^i; x^{-i}) \coloneqq \frac{1}{\mathbb{V}^i_{\delta}} \int_{\delta \mathbb{S}_{-i}} \int_{\delta \mathbb{B}_i} J^i(x^i + \Tilde{\tau}^i; x^{-i} + \tau^{-i})d\Tilde{\tau}^i d\tau^{-i}, 
\end{align}
where $\mathbb{S}_{-i} \coloneqq \prod_{j \in \mathcal{N}^{-i}} \mathbb{S}_j \subseteq \rset{n^{-i}}{}$ with each $\mathbb{S}_j$ representing a unit sphere centered at the origin within $\rset{n^j}{}$; $\mathbb{B}_i$ denotes the unit ball centered at the origin inside $\rset{n^i}{}$; $\mathbb{V}^i_{\delta} \coloneqq \vol{\delta \mathbb{B}_i} \cdot \vol{\delta \mathbb{S}_{-i}}$ is the normalizing volume constant of the area that we are integrating over. 
One widely employed decomposition in the existing literature is that 
\begin{align*}
G^i_k =& \nabla_{x^i} J^i(X_{k}) + \big(G^i_k - \expt{}{G^i_k \mid \mathcal{F}_k}\big)  + \big(\expt{}{G^i_k \mid \mathcal{F}_k} - \nabla_{x^i} J^i(X_{k})\big),
\end{align*}
where we let $B^i_k \coloneqq \expt{}{G^i_k \mid \mathcal{F}_k} - \nabla_{x^i} J^i(X_{k})$ represent the systematic error and $V^i_k \coloneqq G^i_k - \expt{}{G^i_k \mid \mathcal{F}_k}$ the stochastic error. 
Denote $B_k \coloneqq [B^i_k]_{i \in \mathcal{N}}$ and $V_k \coloneqq [V^i_k]_{i \in \mathcal{N}}$. 
Let $(\mathcal{F}_k)_{k \in \nset{}{+}}$ be the filtration concerning the random exploration factor, i.e., $\mathcal{F}_k \coloneqq \sigma\{X_0, u_1, \ldots, u_{k-1}\}$. 
Then we have the following lemma to characterize the properties of $B_k$. 
\begin{lemma}\label{le:bias}
Suppose that Assumption~\ref{asp:objt-set} holds. 
Then at each iteration $k$, the conditional expectation satisfies $\expt{}{G^i_k \mid \mathcal{F}_k} = \nabla_{x^i} \Tilde{J}^i_{\delta_k}(\bar{X}_{k})$ a.s. for every $i \in \mathcal{N}$. 
Moreover, the systematic error $B_k$ possesses a decaying upper bound $\norm{B_k} \leq \alpha_B \delta_k$ for some positive constant $\alpha_B$.
\end{lemma}
\begin{proof}
See the proof of \cite[Lemma~1 \& Lemma~2]{huang2023zeroth}. 
\end{proof}

\subsection{Feedback Utilization Strategy}

% describe how the current utilization strategy works 
The systematic error $B^i_k$ and stochastic error $V^i_k$ rooted in the estimate \eqref{eq:rpe} make it inappropriate to merely leverage the most recent first-order estimate multiple times until a more recent one arrives as what is done in \cite{huang2022convergence}; otherwise, the error will accumulate and endanger the convergence of the iterations. 
In view of this, we adopt the priority-based feedback utilization strategy: at each update, the first-order estimate with the earliest timestamp will be used and then discarded, similar to the approach employed in \cite{heliou20agradient}. 
\Tblue{However, the single-point estimate strategy used in \cite{heliou20agradient} mandates solely one realized function value, in which case it suffices to maintain a priority queue exclusively for these values. 
In contrast, the RPG adopted in this work requires two consecutive realized function values to obtain one estimate, which necessitates maintaining a cache to store observed function values and another priority queue for the resulting RPG estimates. 
}

\Tblue{In our feedback utilization strategy}, two information caches $\mathcal{P}^i_J$ and $\mathcal{P}^i_G$ are endowed for each player $i$. 
As reflected in \eqref{eq:rpe}, two subsequent objective values ($\hat{J}^i_k$ and $\hat{J}^i_{k-1}$) are a prerequisite to compute $G^i_k$, and it is possible that one arrives much earlier than the other. 
As such, cache $\mathcal{P}^i_J$ will store all the objective values received and pop out the ones that have been used twice in computing \eqref{eq:rpe}. 
For another thing, caused by the uncertainty in the feedback delay $d^i_k$, it is possible that at some iteration, player $i$ has no available first-order estimates, while for some other iterations, multiple estimates are at player $i$'s disposal. 
This motivates us to design $\mathcal{P}^i_G$ as a priority queue with the timestamp of each pseudo-gradient estimate as the key value. 
For notational convenience, we introduce a map $s^i: \nset{}{+} \to \nset{}{+}$ that maps from the current iteration to the iteration where the first-order estimate originates from. 
When \Tblue{$P^i_G$} is empty at iteration $k$, $s^i(k) = 1$ and the action remains unchanged. 
We also note that the map $s^i$ is implicitly parameterized by the random sample $\omega \in \Omega$ and could vary across this group of players under Assumption~\ref{asp:delay-plus} $(\romannum{1})$. 
To account for the heterogeneity in feedback delay $(d^i_k)_{i \in \mathcal{N}}$, we introduce a group iteration index map $s: \nset{}{+} \to \nset{N}{+}$, that projects from a certain iteration index $k$ to the stack of originated indices $[s^i(k)]_{i \in \playerN}$. 

%% TODO: may introduce $\mathcal{F}_{s(k)}$ here

Below, we present two lemmas that characterize the priority-based feedback utilization strategy, which our subsequent convergence analysis hinges upon. 
\ifarxiv
The proof is reported in Appendix~\ref{appd:delay-strategy}. 
\fi
\ifproceeding
The proof is reported in \cite[Appendix~A]{huang2023bandit}. 
\fi
% le:cnt-no-update
% le:gap-play-receive
\begin{lemma}\label{le:fdbk-prop}
For each player $i$ and arbitrary iteration $k \in \nset{}{+}$, we have the following: \\
$(\romannum{1})$ $K^i_{\varnothing}(k) \coloneqq \abs{\{s: \mathcal{P}^i_G = \varnothing, 1 \leq s \leq k\}} \leq \min\{k, \bar{d}(k) + 1\}$; \\ 
$(\romannum{2})$ if $s^i(k) \neq 1$, then $s^i(k) + \bar{d}(s^i(k)) \geq k$. 
\end{lemma}

% Two Lemmas about the delayed information utilization strategy

\subsection{The MD Algorithm with Feedback Delays} 

\begin{algorithm}
\caption{Bandit Learning with Reward Delays of CPs Based on Mirror Descent (Player $i$)}\label{alg:md-delay}
\begin{algorithmic}[1]
\State \textbf{Initialize:} $X^i_1 \in \mathcal{X}^i \cap \dom{\psi^i}$ chosen arbitrary; 
take action $\hat{X}^i_1$ and $\hat{J}^i_1 = J^i(X^i_{1}; X^{-i}_{1})$ will arrive $\ceil{d^i_1}$ iterations later; 
$G^i_{1} = \boldsymbol{0}_{n^i}$; 
$p^i, r^i$ to be the center and radius of an arbitrary ball within the set $\mathcal{X}^i$
\Procedure{At the $k$-th iteration ($k \in \nset{}{+}$)}{}
\State Receive $\mathcal{R}^i_k \coloneqq \{(t, \hat{J}^i_t, u^i_t): k-1 < t + d^i_t \leq k\}$ 
\State $\mathcal{P}_J^i \leftarrow \mathcal{P}_J^i \cup \mathcal{R}^i_k$
\For{$(t, \hat{J}^i_t, u^i_t) \in \mathcal{R}^i_k$}
    \If{$(t+1, \hat{J}^i_{t+1}, u^i_{t+1}) \in \mathcal{P}^i_J$}
        \State $G^i_{t+1} \leftarrow \frac{n^i}{\delta_{t+1}}(\hat{J}^{i}_{t+1} - \hat{J}^{i}_{t})u^{i}_{t+1}$, $\mathcal{P}_G^i \coloneqq \mathcal{P}_G^i \cup \{G^i_{t+1}\}$
    \EndIf
    \If{$(t-1, \hat{J}^i_{t-1}, u^i_{t-1}) \in \mathcal{P}^i_J$}
        \State $G^i_{t} \leftarrow \frac{n^i}{\delta_t}(\hat{J}^{i}_t - \hat{J}^{i}_{t-1})u^{i}_t$, $\mathcal{P}_G^i \coloneqq \mathcal{P}_G^i \cup \{G^i_{t}\}$
    \EndIf
    \State $\mathcal{P}_J^i$ clears up the received feedback that has been utilized twice
\EndFor
\If{$\mathcal{P}_G^i \neq \varnothing$}
\State $s^i(k) \leftarrow \text{earliest index in }\mathcal{P}_G^i$, $\mathcal{P}_G^i \leftarrow \mathcal{P}_G^i \backslash \{ G^i_{s^i(k)} \}$
\Else
\State $s^i(k) \leftarrow 1$ \Comment{No update at this iteration}
\EndIf
\State $X^i_{k+1} \leftarrow P_{X^i_k, \mathcal{X}^i}(-\gamma_k G^i_{s^i(k)})$
\State Randomly sample the direction $u^i_{k+1}$ from $\mathbb{S}_i$
\State $\hat{X}^{i}_{k+1} \leftarrow (1 - \frac{\delta_{k+1}}{r^i})X^{i}_{k+1} + \frac{\delta_{k+1}}{r^i}(p^i + r^i u^{i}_{k+1})$ 
\State Take action $\hat{X}^{i}_{k+1}$ and the realized objective value $\hat{J}^{i}_{k+1} \coloneqq J^i(\hat{X}^{i}_{k+1}; \hat{X}^{-i}_{k+1})$ will arrive $\ceil{d^i_{k+1}}$ iterations later
\EndProcedure
\State \textbf{Return:} $\{\hat{X}^i_{k}\}_{i \in \playerN}$
\end{algorithmic}
\end{algorithm}

The fusion of MD, RPG, and the priority-based feedback utilization strategy results in the proposed algorithm for bandit learning in continuous games with feedback delays, which is detailed in Algorithm~\ref{alg:md-delay}. 
As has been discussed in \cite{huang2023zeroth}, one prominent benefit we can reap from RPG is that the associated stochastic error $V_k$ enjoys bounded variance if the decaying rate of step size is faster than that of query radius. 
It is worth mentioning that, Algorithm~\ref{alg:md-delay} leverages $\hat{G}_k = G_{s(k)} \coloneqq [G^i_{s^i(k)}]_{i \in \mathcal{N}}$ rather than $G_k$ to implement the action update at the $k$-th iteration, which is susceptible to the approximation errors stemming from bandit estimation and feedback delays. 
The existence of feedback delays then disrupts the recurrent relation characterizing $(\hat{G}_k)_{k \in \nset{}{+}}$, as a result of which, the analysis of the boundedness of the stochastic error and the estimates $G^i_k$ in \cite{huang2023zeroth} cannot be directly carried over. 
To facilitate later analysis, we set $\hat{G}_1 = G_1 = G_0 = \boldsymbol{0}_{n}$ and  $\hat{J}^i_0 = \hat{J}^i_1$. 
In the lemma below, we will present the sufficient condition to guarantee that the estimates $\hat{G}_k$ enjoy a uniform upper bound across $k \in \nset{}{+}$ and $\omega \in \Omega$. 
\ifarxiv
The proof is reported in Appendix~\ref{appd:bounded-rpg}. 
\fi
\ifproceeding
The proof is reported in \cite[Appendix~B]{huang2023bandit}. 
\fi

\begin{lemma}\label{le:bounded-rpg}
Suppose that Assumptions~\ref{asp:objt-set} and \ref{asp:delay} hold. 
Moreover, step size $(\gamma_k)_{k \in \nset{}{+}}$ and query radius $(\delta_k)_{k \in \nset{}{+}}$ are monotonically decreasing and satisfy: 
$\lim_{k\to\infty}\gamma_k = 0$, $\sum_{k \in \nset{}{+}} \gamma_k = \infty$, $\lim_{k\to\infty}\delta_k = 0$, $\delta_k/\delta_{k+1}$ is uniformly bounded for all $k \in \nset{}{+}$, $\lim_{k \to \infty} \gamma_k/\delta_k = 0$. 
Considering $(\hat{G}_{k})_{k \in \nset{}{+}}$ generated by Algorithm~\ref{alg:md-delay}, we have $\sup_{k \in \nset{}{+}} \norm{\hat{G}_{k}}_* < \infty$. 
\end{lemma}

For the feedback-delay scenario, the randomness originates from two sources: the random exploration factor at each iteration $u^i_k$ and the feedback delay $d^i_k$ associated with the realized objective value $\hat{J}^i_k$. 
Let the $\sigma$-field reflecting the delay information up to iteration $k$ be denoted as:
\begin{align}
\mathcal{F}^d_k \coloneqq \sigma\{d^i_t: \forall i \in \mathcal{N}, 1\leq t\leq k\}  
\end{align}
Note that $s^i(t) \in \mathcal{F}^d_k$ for all $1 \leq t \leq k$ and the available information respecting random exploration factors $u^i_t$ depends on $\mathcal{F}^d_k$. 
Based on the observation, we are prompted to consider a more suitable $\sigma$-field $\Tilde{\mathcal{F}}_{s(k)}$ for this specific problem, rather than the $\sigma$-field $\mathcal{F}_{k}$ previously discussed in Sec.~\ref{subsec:rpg}, which is defined as: 
\begin{align}
 \Tilde{\mathcal{F}}_k \coloneqq \sigma\big(\mathcal{F}^d_k \cup \{u^i_{s^i(t)}: \forall i \in \mathcal{N}, 1 \leq t \leq k-1\}\big). 
\end{align} 
With this definition in hand, we can then proceed to discuss the asymptotic convergence results for the actual sequence of play generated by Algorithm~\ref{alg:md-delay}. 
\ifarxiv
The proof can be found in Appendix~\ref{appd:convg}. 
\fi
\ifproceeding
The proof can be found in \cite[Appendix~C]{huang2023bandit}. 
\fi
\begin{theorem}\label{thm:as-convg}
Suppose the game $\mathcal{G}$ under consideration satisfies Assumptions~\ref{asp:objt-set} to \ref{asp:recip} and all the players of $\mathcal{G}$ follow Algorithm~\ref{alg:md-delay} throughout the process. 
Moreover, the step size $(\gamma_k)_{k \in \nset{}{+}}$ and the query radius $(\delta_k)_{k \in \nset{}{+}}$ are chosen as $\gamma_k = \gamma_0/(k + K_{\gamma})^{\alpha_\gamma}$ and $\delta_k = \delta_0/(k + K_{\delta})^{\alpha_\delta}$, respectively. 
The selected parameters satisfy 
$
0.5 < \alpha_\gamma \leq 1, \alpha_{\gamma} > \alpha_{\delta}, \alpha_{\gamma} + \alpha_{\delta} > 1, 2\alpha_{\gamma} - \alpha_d > 1.
$
Then the actual sequence of play $(\hat{X}_k)_{k \in \nset{}{+}}$ converges to one of the CP $x_*$ almost surely. 
\end{theorem}

%%%%%%%%%%%%%%%%%%%%%%%%%%%%%%%%%%%%%%%%%%%%%%%%%%%%%%%%%%%%%%%%%%%%%%%%%%%%%%%%

% \section{Convergence Analysis}

% properties of the pseudo-gradient estimate under the current setup

% decaying systematic error

% constant range upper bound for G_k

% a.s. convergence

%%%%%%%%%%%%%%%%%%%%%%%%%%%%%%%%%%%%%%%%%%%%%%%%%%%%%%%%%%%%%%%%%%%%%%%%%%%%%%%%

\section{Numerical Experiments}

To illustrate the effectiveness of the proposed algorithm, we provide a numerical example of the thermal load management problem in a building complex. 
Suppose the load aggregator under study consisting of $N$ buildings, indexed by $\mathcal{N} \coloneqq \{1, \ldots, N\}$. 
Over a given time horizon $\mathcal{T} \coloneqq \{1, \ldots, T\}$, we use $x^i_t$ to represent the power consumption of building $i$ at a certain time slot $t \in \mathcal{T}$.
Moreover, the concatenations $x^i \coloneqq [x^i_t]$ and $x \coloneqq [x^i]$ denote the power profile of building $i$ for all time slots and the energy profile of all buildings in this load aggregator, respectively. 
The internal pricing mechanism under consideration \cite{jiang2021game} discourages peak-demand usage by incorporating an approximate version of Shapley value, where each building $i$'s share of peak demand is defined as 
$R^i(x) = \sum_{\mathcal{C}_j: i \in \mathcal{C}_j} \frac{(N - \abs{\mathcal{C}_j})!(\abs{\mathcal{C}_j} - 1)!}{N!}\Big(V(\mathcal{C}_j, x) - V(\mathcal{C}_j\backslash \{i\}, x)\Big),
$
where $\mathcal{C} \coloneqq \{\mathcal{C}_{1}, \ldots \mathcal{C}_{n_c}\}$ with each $\mathcal{C}_j \subseteq \mathcal{N} (j = 1, \ldots, n_c)$ denotes the clique set; 
the function $V$ is defined as $V(\mathcal{C}_j, x) = \frac{1}{C}\log \Big( \sum_{t \in \mathcal{T}} \exp\big(\sum_{l \in \mathcal{C}_j} C x^l_{t}\big) \Big)$, \Tblue{where $C \in \rset{}{++}$ is a constant sufficiently large to make the log-sum-exp function a proper smooth approximation to the maximum function. }

With knowledge of the power profile $x^{-i}$ of other buildings, each building $i$ seeks to find an optimal power control strategy, which can be expressed as follows: 
\begin{align}
\begin{split}
& \minimize_{x^i \in \mathcal{X}^i} \;(p_e)^Tx^i + Q^i(x^i) + p_d \cdot R^i(x)  \\
& \subj \;r^i_{t} = a^i r^i_{t-1} + b^i x^i_{t}, \; y^i_{t} = c^i r^i_{t},  \\
& \qquad \qquad  \ubar{y}^i_{t} \leq y^i_{t} \leq \bar{y}^i_{t}, \; 0 \leq x^i_{t} \leq \bar{x}^{i}, \forall t \in \mathcal{T}, 
\end{split}
\end{align}
where $p_e \in \rset{T}{++}$ denotes the energy price and $p_d \in \rset{}{++}$ penalized the peak electricity usage of the aggregator; 
a strongly convex quadratic function $Q^i$ is introduced for the convergence purpose; 
$y^i_t$ denotes the temperature of building $i$ at the $t$-th time slot and its dynamics are characterized by the first and second equality constraints; 
the third constraint enforces that the temperature $y^i_t$ should be within a comfort zone $[\ubar{y}^i_{t}, \bar{y}^i_{t}]$; 
the last constraint reflects the system power capacity for each building. 
It can be proved that this multi-player game admits a potential function 
$\Phi(x) = \sum_{i \in \playerN} \Big((p_e)^Tx^i + Q^i(x^i)\Big) + p_d \cdot \sum_{\mathcal{C}_j \in \mathcal{C}} \frac{(N - \abs{\mathcal{C}_j})!(\abs{\mathcal{C}_j} - 1)!}{N!}V(\mathcal{C}_j, x)$. 

In the experiments, twenty buildings $(N=20)$ are involved in this game, and each building needs to determine its energy profile for four different time slots $(T=4)$. 
Suppose there are six cliques and the number of buildings within each clique ranges from three to eight. 
For $Q^i(x^i) = (x^i)^T\diag{\lambda_{i1}, \ldots, \lambda_{in^i}}x^i$, each diagonal entry $\lambda_{ij}$ is randomly sampled from $[0.04, 0.06]$. 
The query radius $\delta_k$ and the step size $\gamma_k$ are set to be $\delta_k = 1/(k+10)^{0.6}$ and $\gamma_k = 1/(k+10^3)^{0.9}$, respectively. 
Regarding the feedback delay $d^i_k$, we consider the case when $d^i_k$ is upper bounded by $\bar{d}_k = 10^3$ while the realized values of $d^i_k$ vary across different buildings. 
In addition, several experiments are conducted under the setup that $d^i_k$ is homogeneous in this group of buildings and grows sublinearly. 
To compare with the existing work, we implement the method in \cite{heliou20agradient} with $\delta_k = 1/(k+10)^{0.35}$ and $\gamma_k = 1/(k+10^3)^{0.9}$ as required by the associated convergence theorem. 
Two metrics are employed to measure the performance of Algorithm~\ref{alg:md-delay}, which include the relative distance between the NE and the perturbed actions, $\norm{\hat{X}_{k} - x_*}_2 / \norm{x_*}_2$, and the difference between the potential function's optimal value and the values at the perturbed actions, $\Phi(\hat{X}_{k}) - \Phi_*$. 

The numerical results are illustrated in Fig~\ref{fig:thm_ctrl4}. 
It can be observed that when the feedback delay $d^i_k$ grows no faster than $O(\sqrt{k})$, the convergence rates of the generated sequences are dominated by the first-order estimation error and no significant difference is noted among $\bar{d}_k = 10^3$, $d_k = k^{0.1}$, $d_k = 5 k^{0.5}$, and $d_k = 10 k^{0.5}$. 
When the delay time $d^i_k$ grows faster and even approaches the rate of $O(k)$, the errors induced by the feedback delay outweigh those induced by the estimation error, as reflected in the curves associated with $d_k = 5 k^{0.75}$ and $d_k = 5 k^{0.99}$. 
Furthermore, the results in Fig.~\ref{fig:thm_ctrl4} indicate that Algorithm~\ref{alg:md-delay} exhibits reduced variance, more consistent sequences of play, and faster convergence compared to the existing method in \cite{heliou20agradient}.  


\begin{figure}
    \centering
    \includegraphics[width=0.45\textwidth]{figs/thermal_ctrl4_delay_1674428500.png}
    \caption{Performance of the Proposed Algorithm Confronted with Homogeneous and Heterogeneous Feedback Delays}
    \label{fig:thm_ctrl4}
\end{figure}

% thermal control in buildings with merely convex potential

%%%%%%%%%%%%%%%%%%%%%%%%%%%%%%%%%%%%%%%%%%%%%%%%%%%%%%%%%%%%%%%%%%%%%%%%%%%%%%%%

\section{Conclusion}
This paper studies the problem of bandit learning in multi-player continuous games, which is further complicated by information delays. 
Compared with the existing method introduced in \cite{heliou20agradient}, the algorithm proposed in this paper incorporates the residual pseudo-gradient estimation strategy and the mirror descent iteration, which loosens the conditions imposed upon the query radius and the step sizes. 
The a.s. convergence of the actual sequences of play generated by the proposed algorithm is established for pseudo-monotone plus games. 
One important direction for future research concerns the case where the feedback delays grow as the iteration proceeds and at the same time, they are heterogeneous across the participants. 
Another potential future direction resides in designing an algorithm that could tackle a more general class of multi-player games, such as merely monotone games, which are prevalent in the modeling of practical problems. 
Nevertheless, when applied to merely monotone games, mirror descent and most of its variants fail to converge and are prone to be trapped in spurious solutions.  

\ifarxiv
\appendices 
\section{Appendix for Proofs}

\paragraph{Proof of Theorem \ref{thm:main}.}

\begin{proof}
\label{proof:main}
Our proof has two steps. In Step 1, we will show that SimCLR is equivalent to minimizing the cross entropy loss defined in Eqn.~(\ref{eqn:cross-entropy}). 
In Step 2, we will show  that minimizing the cross-entropy loss 
is equivalent to spectral clustering on $\bfpi$. 
Combining the two steps together, we have proved our theorem. 

\textbf{Step 1: } SimCLR is equivalent to minimizing the cross entropy loss.

The cross-entropy loss takes expectation over 
$\bfW_\bfX\sim \mathbb{P}(\cdot ; \bfpi)$, 
which means $\bfW_\bfX$ has exactly one non-zero entry in each row $i$. By Lemma~\ref{lem:multinomial}, we know every row $i$ of $\bfW_\bfX$ is independent of other rows. Moreover, 
$\bfW_{\bfX,i}\sim \mathcal{M}(1, \bfpi_i/\sum_j \bfpi_{i,j})=\mathcal{M}(1, \bfpi_i)$, because $\bfpi_i$ itself is a probability distribution.
Similarly, we know $\bfW_\bfZ$ also has the row-independent property by sampling over $\mathbb{P}(\cdot;\bfK_\bfZ)$.
Therefore, by Lemma~\ref{lem:cross_split}, we know Eqn.~(\ref{eqn:cross-entropy}) is equivalent to:
\[
 -\sum_{i=1}^n \mathbb{E}_{\bfW_{\bfX,i}}[\log \mathbb{P}(\bfW_{\bfZ,i}=\bfW_{\bfX,i};\bfK_\bfZ)],
\]

This expression takes expectation over $\bfW_{\bfX,i}$ for the given row $i$. Notice that 
$\bfW_{\bfX,i}$ has exactly one non-zero entry, which equals $1$ (same for $\bfW_{\bfZ,i}$). 
As a result
we expand the above expression to be:
\begin{equation}
 -\sum_{i=1}^n \sum_{j\neq i} \Pr(\bfW_{\bfX,i,j}=1)\log \Pr(\bfW_{\bfZ,i,j}=1).
\label{eqn:detailed-expansion}    
\end{equation}


By Lemma~\ref{lem:multinomial}, $\Pr(\bfW_{\bfZ,i,j}=1)=\bfK_{\bfZ,i,j}/\|\bfK_{\bfZ,i}\|_1$ for $j\neq i$. Recall that $\bfK_\bfZ=(k(\bfZ_i-\bfZ_j))_{(i,j)\in[n]^2}$, which means 
$\bfK_{\bfZ,i,j}/\|\bfK_{\bfZ,i}\|_1=\frac{\exp(-\|\bfZ_i-\bfZ_j\|^2/{2\tau})}{\sum_{k\neq i}
\exp(-\|\bfZ_i-\bfZ_k\|^2/{2\tau})
}$ for $j\neq i$, when $k$ is the Gaussian kernel with variance $\tau$. 

Notice that $\bfZ_i=f(\bfX_i)$, so we know
\begin{equation}
-\log \Pr(\bfW_{\bfZ,i,j}=1)=
-\log \frac{\exp(-\|f(\bfX_i)-f(\bfX_j)\|^2/{2\tau})}{\sum_{k\neq i}
\exp(-\|f(\bfX_i)-f(\bfX_k)\|^2/{2\tau}),
}
\label{eqn:infonce-equivalence}    
\end{equation}


The right hand side is exactly the InfoNCE loss defined in Eqn.~(\ref{eqn:infonce}).
Inserting Eqn.~(\ref{eqn:infonce-equivalence}) into Eqn.~(\ref{eqn:detailed-expansion}), we get the SimCLR algorithm, which first samples augmentation pairs $(i,j)$ with $\Pr(\bfW_{\bfX,i,j}=1)$ for each row $i$, and then optimize the InfoNCE loss. 

\textbf{Step 2: } minimizing the cross entropy loss 
is equivalent to spectral clustering on $\bfpi$.


By Lemma~\ref{lem:convert_to_spectral}, we may further convert the loss to 
\begin{equation}
\label{eqn:main-theorem-repul-attr}
\min_{\bfZ}
-\sum_{(i,j)\in [n]^2} \mathbf{P}_{i,j}
\log k (\bfZ_i-\bfZ_j)+\log \mathbf{R}(\bfZ).
\end{equation}
Since $k$ is the Gaussian kernel, this reduces to \[
\min_\bfZ \mathrm{tr}(\bfZ^\top \mathbf{L}(\bfpi) \bfZ)
+\log \mathbf{R}(\bfZ),
\]

where we use the fact that $\mathbb{E}_{\bfW_\bfX\sim \mathbb{P}(\cdot; \bfpi)}[\mathbf{L}(\bfW_\bfX)]
=\mathbf{L}(\bfpi)
$, because the Laplacian operator is linear and $
\mathbb{E}_{\bfW_\bfX\sim \mathbb{P}(\cdot; \bfpi)}(\bfW_\bfX)=\bfpi
$.
\end{proof}

\paragraph{Proof of Theorem \ref{thm:clip}.}
\begin{proof}
Since $\bfW_\bfX\sim \mathbb{P}(\cdot;\bfpi_{\mathbf{A}, \mathbf{B}})$, we know 
$\bfW_\bfX$ has exactly one non-zero entry in each row, denoting the pair that got sampled. 
A notable difference compared to the previous proof is we now have $n_\mathcal{A}+n_\mathcal{B}$ objects in our graph. CLIP deals with this by taking a mini-batch of size $2N$, 
such that $n_\mathcal{A}=n_\mathcal{B}=N$, and adding the $2N$ InfoNCE losses together. We label the objects in $\mathcal{A}$ as $[n_\mathcal{A}]$, and the objects in $\mathcal{B}$ as $\{n_\mathcal{A}+1, \cdots, n_\mathcal{A}+n_\mathcal{B}\}$. 

Notice that $\bfpi_{\mathbf{A}, \mathbf{B}}$ is a bipartite graph, so the edges of objects in $\mathcal{A}$ will only connect to object in $\mathcal{B}$ and vice versa. We can define the similarity matrix in $\cZ$ as $\bfK_\bfZ$, 
where $\bfK_\bfZ(i, j+n_\mathcal{A})=\bfK_\bfZ(j+n_\mathcal{A},i)= k(\bfZ_i-\bfZ_j)$ for $i\in [n_\mathcal{A}], j\in [n_\mathcal{B}]$, and otherwise we set $\bfK_\bfZ(i,j)=0$. 
The rest is same as the previous proof. 
\end{proof}

\paragraph{Proof of Theorem \ref{thm:exponential}.}

\begin{proof}
\label{proof:exponential}
Since the objective function consists of a linear term combined with an entropy regularization, which is a strongly concave function, the maximization problem is a convex optimization problem. Owing to the implicit constraints provided by the entropy function, the problem is equivalent to having only the equality constraint. We then introduce the Lagrangian multiplier $\lambda$ and obtain the following relaxed problem:

$$
\widetilde{E}(\boldsymbol{\alpha})=\psi_{1}-\sum_{i=1}^n \alpha_{i} \psi_{i}+\tau \sum_{i=1}^n \alpha_{i}\log \alpha_{i}+\lambda\left(\boldsymbol{\alpha}^{\top} \mathbf{1}_n-1\right).
$$

As the relaxed problem is unconstrained, taking the derivative with respect to $\alpha_{i}$ yields

$$
\frac{\partial \widetilde{E}(\boldsymbol{\alpha})}{\partial \alpha_{i}}=-\psi_{i}+\tau\left(\log \alpha_{i}+\alpha_{i} \frac{1}{\alpha_{i}}\right)+\lambda=0.
$$

Solving the above equation implies that $\alpha_{i}$ takes the form
$
\alpha_{i}=\exp \left(\frac{1}{\tau} \psi_{i}\right) \exp \left(\frac{-\lambda}{\tau}-1\right).
$ Since $\alpha_{i}$ lies on the probability simplex, the optimal $\alpha_{i}$ is explicitly given by
$
\alpha^{*}_{i}=\frac{\exp \left(\frac{1}{\tau} \psi_{i}\right)}{\sum_{i^{\prime}=1}^n \exp \left(\frac{1}{\tau} \psi_{i^{\prime}}\right)} .
$ Substituting the optimal point into the objective function, we obtain
$$
\begin{aligned}
E\left(\boldsymbol{\alpha}^*\right)  &=\psi_1-\sum_{i=1}^n \frac{\exp \left(\frac{1}{\tau} \psi_{i}\right)}{\sum_{i^{\prime}=1}^n \exp \left(\frac{1}{\tau} \psi_{i^{\prime}}\right)} \psi_{i}+\tau \sum_{i=1}^n \frac{\exp \left(\frac{1}{\tau} \psi_{i}\right)}{\sum_{i^{\prime}=1}^n \exp \left(\frac{1}{\tau} \psi_{i^{\prime}}\right)}\log \frac{\exp \left(\frac{1}{\tau} \psi_{i}\right)}{\sum_{i^{\prime}=1}^n \exp \left(\frac{1}{\tau} \psi_{i^{\prime}}\right)} \\
& =\psi_1 - \tau \log \left(\sum_{i=1}^n \exp \left(\frac{1}{\tau} \psi_{i}\right)\right).
\end{aligned}
$$
Thus, the Lagrangian dual function is given by
\begin{equation*}
-E\left(\boldsymbol{\alpha}^*\right)= -\tau \log \frac{\exp \left(\frac{1}{\tau} \psi_{1}\right)}{\sum_{i=1}^n \exp \left(\frac{1}{\tau} \psi_{i}\right)}.\qedhere
\end{equation*}
\end{proof}



\section{More on Experiments} \label{section: experiment_details}

\paragraph{CIFAR-10 and CIFAR-100} CIFAR-10 ~\citep{krizhevsky2009learning} and CIFAR-100 ~\citep{krizhevsky2009learning} are well-known classic image classification datasets. Both CIFAR-10 and CIFAR-100 contain a total of 60k $32 \times 32$ labeled images of different classes, with 50k for training and 10k for testing. CIFAR-10 is similar to CIFAR-100, except there are 10 different classes in CIFAR-10 and 100 classes in CIFAR-100.

\paragraph{TinyImageNet} TinyImageNet ~\citep{le2015tiny} is a subset of ImageNet ~\citep{deng2009imagenet}. There are 200 different object classes in TinyImageNet, with 500 training images, 50 validation images, and 50 test images for each class. All the images in TinyImageNet are colored and labeled with a size of $64 \times 64$.

\textbf{Pseudo-code.} Algorithm \ref{alg:Training Procedure} presents the pseudo-code for our empirical training procedure.

\begin{algorithm}[!htbp]
\caption{Training Procedure}
\label{alg:Training Procedure}
\begin{algorithmic}[1]
\REQUIRE trainable encoder network $f$, batch size $N$, augmentation strategy \textit{aug}, loss function $L$ with hyperparameters \textit{args}
\FOR {sampled minibatch ${x_i}_{i=1}^N$}
\FORALL{$i \in { 1, ..., N }$}
\STATE draw two augmentations $t_i = \textit{aug}\left(x_i\right) $, $t_i' = \textit{aug}\left(x_i\right) $
\STATE $z_i = f\left(t_i\right)$, $z_i' = f\left(t_i'\right)$
\ENDFOR
\STATE compute loss $\mathcal{L} = L(N, z, z', \textit{args})$
\STATE update encoder network $f$ to minimize $\mathcal{L}$
\ENDFOR
\STATE \textbf{Return} encoder network $f$
\end{algorithmic}
\end{algorithm}

We also provide the pseudo-code for our core loss function used in the training procedure in Algorithm \ref{alg:Core loss}. The pseudo-code is almost identical to SimCLR's loss function, with the exception of an extra parameter $\gamma$.

\begin{algorithm}[!htbp]
\caption{Core loss function $\mathcal{C}$}
\label{alg:Core loss}
\begin{algorithmic}[1]
\REQUIRE batch size $N$, two encoded minibatches $z_1, z_2$, $\gamma$, temperature $\tau$
\STATE $z = \textit{concat}\left(z_1, z_2\right)$
\FOR {$i \in {1, ..., 2N }, j \in {1, ..., 2N}$ }
\STATE $s_{i,j} = \Vert z_i - z_j \Vert_2^{\gamma}$
\ENDFOR
\STATE \textbf{define} $l(i, j)$ \textbf{as} $l(i, j) = - \log \frac{exp\left(s_{i,j}/\tau \right)}{\sum_{k=1}^{2N} \mathbf{1}{[k \ne i]} exp\left(s{i, j} / \tau \right)} $
\STATE \textbf{Return} $\frac{1}{2N} \sum_{k=1}^N\left[l(i, i+N) + l(i+N, i)\right]$
\end{algorithmic}
\end{algorithm}

Utilizing the core loss function $\mathcal{C}$, we can define all kernel loss functions used in our experiments in Table \ref{table: loss definition}. For all $z_i \in z$ with even dimensions $n$, we define $z_{L_i} = z_i\left[0:n/2\right]$ and $z_{R_i} = z_i\left[n/2:n\right]$.

\begin{table}[ht]
\centering
\begin{tabular}{{@{}l|l@{}}}
Kernel  &  Loss function \\ \midrule
Laplacian & $\mathcal{C}\left(N, z, z', \gamma=1, \tau\right)$\\ \midrule
Sum       & $\lambda * \mathcal{C}\left(N, z, z', \gamma=1, \tau_1\right) + (1-\lambda) * \mathcal{C}\left(N, z, z', \gamma=2, \tau_2\right)$  \\ \midrule
Concatenation Sum&$\lambda * \mathcal{C}\left(N, z_L, z'_L, \gamma=1, \tau_1\right) + (1-\lambda) * \mathcal{C}\left(N, z_R, z'_R, \gamma=2, \tau_2\right)$\\ \midrule
$\gamma = 0.5$ & $\mathcal{C}\left(N, z, z', \gamma=0.5, \tau\right)$          \\ 

\end{tabular}

\caption{Definition of kernel loss functions in our experiments}
\label {table: loss definition}
\end{table}

\textbf{Baselines.} We reproduce the SimCLR algorithm using PyTorch Lightning~\citep{PytorchLightning}.

\textbf{Encoder details.}
The encoder $f$ consists of a backbone network and a projection network. We employ ResNet50~\citep{ResNet} as the backbone and a 2-layer MLP (connected by a batch normalization~\citep{ioffe2015batch} layer and a ReLU \cite{nair2010rectified} layer) with hidden dimensions 2048 and output dimensions 128 (or 256 in the concatenation kernel case).

\textbf{Encoder hyperparameter tuning.}
For each encoder training case, we randomly sample 500 hyperparameter groups (sample details are shown in Table \ref{table: Hyperparameter sample}) and train these samples simultaneously using Ray Tune ~\citep{RayTune}, with the ASHA scheduler~\citep{li2018massively}. Ultimately, the hyperparameter group that maximizes the online validation accuracy (integrated in PyTorch Lightning) within 5000 validation steps is chosen for the given encoder training case.

\begin{table}[ht]
\centering

\begin{tabular}{@{}l|l|l@{}}
\midrule
Hyperparameter  & Sample Range & Sample Strategy \\ \midrule
start learning rate & $\left[10^{-2}, 10\right]$ & log uniform \\ \midrule
$\lambda$       & $\left[0, 1\right]$ & uniform \\ \midrule
$\tau$, $\tau_1$, $\tau_2$ & $\left[0, 1\right]$ & log uniform \\ \midrule
\end{tabular}

\caption{Hyperparameters sample strategy}
\label {table: Hyperparameter sample}
\end{table}

\textbf{Encoder training.} 
We train each encoder using the LARS optimizer~\citep{LARSOptimizer}, LambdaLR Scheduler in PyTorch, momentum 0.9, weight decay $10^{-6}$, batch size 256, and the aforementioned hyperparameters for 400 epochs on a single A-100 GPU.

\textbf{Image transformation.} The image transformation strategy, including augmentation, is identical to the default transformation strategy provided by PyTorch Lightning.

\textbf{Linear evaluation.}
The linear head is trained using the SGD optimizer with a cosine learning rate scheduler, batch size 64, and weight decay $10^{-6}$ for 100 epochs. The learning rate starts at $0.3$ and ends at $0$.

\textbf{Moco Experiments.} We also tested our method based on MoCo~\citep{he2019moco}. The results are summarized in Table \ref{tab:results-moco}. Here we choose ResNet18~\citep{ResNet} as the backbone and set a temperature of $0.1$ as default. For our simple sum kernel, we set $\lambda=0.8$. The results show that our method outperforms the original MoCo method.

\begin{table}[thb]
\centering
\caption{MoCo Experiment Results on CIFAR-10 and CIFAR-100.}
\label{tab:results-moco}
\resizebox{\textwidth}{!}{%
\begin{tabular}{@{}c|ccc|ccc@{}}
\toprule
\multirow{3}{*}{Method} & \multicolumn{3}{c|}{CIFAR-10} & \multicolumn{3}{c}{CIFAR-100} \\ \cmidrule(lr){2-4} \cmidrule(lr){5-7} 
                        & 200 epochs & 400 epochs    & 1000 epochs   & 200 epochs & 400 epochs & 1000 epochs         \\ \midrule
MoCo (repro.)         & $76.41 \pm 0.12$    & $80.01 \pm 0.15$          & $84.45 \pm 0.08$    & $\mathbf{47.02 \pm 0.11}$ & $52.50 \pm 0.07$ & $57.62 \pm 0.15$            \\
\midrule
Laplacian Kernel        & ${78.09 \pm 0.10}$    & $\mathbf{83.85 \pm 0.09}$          & $\mathbf{88.34 \pm 0.16}$    & $46.12 \pm 0.22$   & $53.44 \pm 0.17$ & $59.10 \pm 0.14$        \\
Simple Sum Kernel & $\mathbf{78.12 \pm 0.15}$   & $83.23 \pm 0.18$ & $87.50 \pm 0.20$ & $46.65 \pm 0.06$ & $\mathbf{53.62 \pm 0.19}$ & $\mathbf{59.83 \pm 0.12}$\\
\bottomrule
\end{tabular}
}
\end{table}



\section{More Experiments on Synthetic Data}


Consider a scenario with $n$ clusters, each containing $k$ vertices. Let the probability of vertices $u$ and $v$ from the same cluster belonging to $\bfpi$ be $p$. Conversely, for vertices $u$ and $v$ from different clusters, let the probability of belonging to $\pi$ be $q$. We generate the graph $\bfpi$ randomly, based on $p$ and $q$. We experiment with values of $k=100$ and $n=6$ for ease of visualization, embedding all points in a two-dimensional space. Each vertex's initial position originates from a normal distribution. In each iteration, we sample a subgraph of $\bfpi$ uniformly, ensuring each vertex has an out-degree of $1$. We then optimize the corresponding vectors using InfoNCE loss with an SGD optimizer and iterate until convergence. Our experimental setup consists of an SGD learning rate of $1$, an InfoNCE loss temperature of $0.5$, and a batch size of $50$. We evaluate two scenarios with different $p$ and $q$ values: $p=1$, $q=0$, and $p=0.75$, $q=0.2$. The results of these experiments are visualized in Figure \ref{fig:vis-spectral-cluster}. The obtained embeddings exhibit the hallmark pattern of spectral clustering of graph $\bfpi$.

\begin{figure}[!tb]
\centering
\subfigure{
\includegraphics[width=1\textwidth]{Figures/cluster_pi.png}
\label{fig:vis-cluster}
}
\subfigure{
\includegraphics[width=1\textwidth]{Figures/noised_cluster_pi.png}
\label{fig:vis-noised-cluster}
}
\caption{Visualizations of the optimization process using InfoNCE Loss on the vectors corresponding to $\bfpi$. Points of identical color belong to the same cluster within $\bfpi$. To showcase the internal structure of $\bfpi$, we randomly select 10 vertices from each cluster to display the edge distribution of $\bfpi$.}
\label{fig:vis-spectral-cluster}
\end{figure}


\fi

%%%%%%%%%%%%%%%%%%
% References
%%%%%%%%%%%%%%%%%%
% \nocite{*}
\bibliographystyle{IEEEtran}
\bibliography{IEEEabrv,references}

\end{document}
