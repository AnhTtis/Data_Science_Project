% CVPR 2023 Paper Template
% based on the CVPR template provided by Ming-Ming Cheng (https://github.com/MCG-NKU/CVPR_Template)
% modified and extended by Stefan Roth (stefan.roth@NOSPAMtu-darmstadt.de)

\documentclass[10pt,twocolumn,letterpaper]{article}

%%%%%%%%% PAPER TYPE  - PLEASE UPDATE FOR FINAL VERSION
%\usepackage[review]{cvpr}      % To produce the REVIEW version
%\usepackage{cvpr}              % To produce the CAMERA-READY version
\usepackage[pagenumbers]{cvpr} % To force page numbers, e.g. for an arXiv version
\makeatletter
\@namedef{ver@everyshi.sty}{}
\makeatother
\usepackage{tikz}
\usepackage{comment}
% Include other packages here, before hyperref.
\usepackage{graphicx}
%\usepackage{amsmath}
%\usepackage{amssymb}
\usepackage{booktabs}
\usepackage{amsfonts}
\usepackage{bbding}
\usepackage{algorithmic}
\usepackage{algorithm}
\usepackage{array}
\usepackage[title]{appendix}
% \usepackage{subfigure}
%\usepackage[caption=false,font=normalsize,labelfont=sf,textfont=sf]{subfig}
\usepackage{textcomp}
\usepackage{stfloats}
\usepackage{url}
\usepackage{verbatim}
\usepackage{graphicx}
%\usepackage{cite}
%\usepackage{color}
%\usepackage{times}
\usepackage{soul}
\usepackage{url}
%\usepackage[hidelinks]{hyperref}
\usepackage[utf8]{inputenc}
\RequirePackage{caption}
\usepackage{amssymb,amsmath,amsthm}
\newtheorem{theorem}{Theorem}
\usepackage{mathtools} % Bonus
\usepackage{booktabs}
% DO NOT USE \usepackage{times}, it will be removed by typesetters
%\usepackage{times}
\usepackage{amsthm}
\usepackage{tikz}
% \usepackage{footnote}
\usepackage{amsmath,amssymb} % define this before the line numbering.
\usepackage{color}
% The "axessiblity" package can be found at: https://ctan.org/pkg/axessibility?lang=en
%\usepackage[accsupp]{axessibility} 
% It is strongly recommended to use hyperref, especially for the review version.
% hyperref with option pagebackref eases the reviewers' job.
% Please disable hyperref *only* if you encounter grave issues, e.g. with the
% file validation for the camera-ready version.
%
% If you comment hyperref and then uncomment it, you should delete
% ReviewTempalte.aux before re-running LaTeX.
% (Or just hit 'q' on the first LaTeX run, let it finish, and you
%  should be clear).
\usepackage[pagebackref,breaklinks,colorlinks]{hyperref}
% Support for easy cross-referencing
\usepackage[capitalize]{cleveref}
\crefname{section}{Sec.}{Secs.}
\Crefname{section}{Section}{Sections}
\Crefname{table}{Table}{Tables}
\crefname{table}{Tab.}{Tabs.}
\def\confName{CVPR}
\def\confYear{2023}
\pdfoutput=1
\begin{document}
	%%%%%%%%% TITLE - PLEASE UPDATE
	\title{Weakly-supervised Single-view Image Relighting}
	\author{
		%Authors
		% All authors must be in the same font size and format.
		Renjiao Yi\thanks{Co-first authors.}, Chenyang Zhu\footnotemark[1],  Kai Xu\thanks{Corresponding author: kevin.kai.xu@gmail.com. }\\
		National University of Defense Technology\\
		%{\tt\small yirenjiao, zhuchenyang07@nudt.edu.cn, kevin.kai.xu@gmail.com}
	}
\maketitle
%%%%%%%%% ABSTRACT
\begin{abstract}
	We present a learning-based approach to relight a single image of Lambertian and low-frequency specular objects. 
	Our method enables inserting objects from photographs into new scenes and relighting them under the new environment lighting, which is essential for AR applications. To relight the object, we solve both inverse rendering and re-rendering. 
	%There are two main challenges to solve this problem. The first one is the shortage of labeled training data for inverse rendering. The second one is the shortage of differentiable specular renderers for environment lighting. 
	To resolve the ill-posed inverse rendering, we propose a weakly-supervised method by a low-rank constraint.  
	%Based on the observation that reflectance is invariant to illumination change, it imposes that the reflectance maps of the same object under changing illuminations are linearly correlated. %(i.e., the reflectance matrix storing different reflectance maps to be rank one). 
	%We have proven that the low-rank loss is mathematically correct and leads to training convergence. 
	%Specular component can be separated similarly by low-rank loss on the diffuse chromaticity. 
	To facilitate the weakly-supervised training, we contribute Relit, a large-scale (750K images) dataset of videos with aligned objects under changing illuminations.
	%The capturing of Relit is effortless in the data capture setting. 
	For re-rendering, we propose a differentiable specular rendering layer to render low-frequency non-Lambertian materials under various illuminations of spherical harmonics. 
	The whole pipeline is end-to-end and efficient, allowing for a mobile app implementation of AR object insertion. Extensive evaluations demonstrate that our method achieves state-of-the-art performance. Project page: \href{https://renjiaoyi.github.io/relighting/}{https://renjiaoyi.github.io/relighting/}. 
\end{abstract}
%%%%%%%%% BODY TEXT
% \section{Introduction}
\label{sec:introduction}
% \begin{itemize}
%     % Diffusion of FL
%     \item {\st{Diffusion of FL}}
%     % Security threats to FL
%     \item {\st{Security threats to FL with particular focus on model poisoning}}
%     % Limitations of existing countermeasures
%     \item {\st{Current countermeasures (e.g., KRUM) and their limitations}}
%     % Proposed method and its advantages
%     \item {\st{Intuitive description of the proposed method and its difference (i.e., advantages) w.r.t. state of the art}}
%     % Main contributions
%     \item {\st{Summary of the main contributions of this work}}
%     % Paper's structure and organization
%     \item {\st{Paper's structure and organization}}
% \end{itemize}

% Diffusion of FL
Recently, {\em federated learning} (FL) has emerged as the leading paradigm for training distributed, large-scale, and privacy-preserving machine learning (ML) systems~\cite{mcmahan2017googleai,mcmahan2017aistats}. 
The core idea of FL is to allow multiple edge clients to collaboratively train a shared, global model without disclosing their local private training data.
%Specifically, an FL system consists of a central server and many edge clients; 
A typical FL round involves the following steps: {\em(i)} the server randomly picks some clients and sends them the current, global model; {\em(ii)} each selected client locally trains its model with its own private data; then, it sends the resulting local model to the server;\footnote{Whenever we refer to global/local model, we mean global/local model {\em parameters}.} {\em(iii)} the server updates the global model by computing an \emph{aggregation function}, usually the average (FedAvg), on the local models received from clients.
% \begin{enumerate}
%     \item[{\em(i)}] the server sends the current, global model to the clients and appoints some of them for training;
%     \item[{\em(ii)}] each selected client locally trains its copy of the global model with its own private data; then, it sends the resulting local model back to the server;\footnote{Whenever we refer to global/local model, we mean global/local model {\em parameters}.}
%     \item[{\em(iii)}] the server updates the global model by computing an \emph{aggregation function} on the local models received from clients (by default, the average, also referred to as FedAvg~\cite{mcmahan2017aistats}).
% \end{enumerate}
This process goes on until the global model converges. %(e.g., after a certain number of rounds or other similar stopping criteria).
%\\
% The advantages of FL over the traditional, centralized learning paradigm are undoubtedly clear in terms of flexibility/scalability (clients can join/disconnect from the FL network dynamically), network communications (only model weights\footnote{We will use \textit{parameters} and \textit{weights} interchangeably.} are exchanged between clients and server), and privacy (each client's private training data is kept local at the client's end and not uploaded to the server).
\\
% Security threats to FL
%However, the growing adoption of FL also raises security concerns~\cite{costa2022covert}, particularly about its confidentiality, integrity, and availability.
Although its advantages over standard ML, FL also raises security concerns~\cite{costa2022covert}. %, particularly about its confidentiality, integrity, and availability~\cite{costa2022covert}.
% OLD, LONG VERSION
% Indeed, some work deals with privacy leakage that may expose the local data of some clients~\cite{melis2019sp}. 
% A large body of work, instead, investigates attacks that usually aim to detriment the predictive accuracy of the learned global model. For instance, \emph{data poisoning} attacks achieve this goal by letting an adversary pollute the training set of some corrupt FL clients with maliciously crafted examples~\cite{jagielski2018sp}.
% Similarly, in \emph{model poisoning} the attacker attempts to tweak the global model weights~\cite{bhagoji2019pmlr} by directly perturbing the local model's weights of some infected FL clients before these are sent to the central server for aggregation, usually via so-called Byzantine attacks. 
% It turns out that Byzantine model poisoning attacks severely impact standard FedAvg; therefore, more robust aggregation functions must be designed to make FL systems secure.
Here, we focus on \emph{untargeted model poisoning} attacks~\cite{bhagoji2019pmlr}, where an adversary attempts to tweak the global model weights %\footnote{We will use the terms \textit{parameters} and \textit{weights} interchangeably.} 
by directly perturbing the local model's parameters of some infected clients before these are sent to the central server for aggregation.
In doing so, the adversary aims to jeopardize the global model \textit{indiscriminately} at inference time.
Such model poisoning attacks severely impact standard FedAvg; therefore, more robust aggregation functions must be designed to secure FL systems.
\\
% In this paper, we focus on designing a novel robust aggregation scheme at the server's end to contrast the effect of Byzantine model poisoning attacks.
%
% Current countermeasures and their limitations
%Several countermeasures have been proposed in the literature to combat model poisoning attacks on FL systems.
% Some methods use simple statistics more robust than plain average to smooth the impact of malicious updates (e.g., Trimmed Mean and FedMedian~\cite{yin2018icml}). 
% Other defenses implement outlier detection techniques to discard malicious updates from the aggregation performed at the server's end. Those are either based on heuristics (e.g., Krum/Multi-Krum~\cite{blanchard2017nips} and Bulyan~\cite{mhamdi2018pmlr}) or data-driven approaches (e.g., K-means clustering~\cite{shen2016acm} or DnC via spectral analysis~\cite{shejwalkar2021ndss}). 
% Finally, some strategies rely on a centralized ``source of trust'' to spot potential malicious updates (e.g., FLTrust~\cite{cao2020fltrust}).
% Several countermeasures have been proposed in the literature to combat model poisoning attacks on FL systems, i.e., to discard possible malicious local updates from the aggregation performed at the server's end. 
% These techniques range from simple statistics more robust than plain average (e.g., Trimmed Mean and FedMedian~\cite{yin2018icml}) to outlier detection heuristics (e.g., Krum/Multi-Krum~\cite{blanchard2017nips} and Bulyan~\cite{mhamdi2018pmlr}) or data-driven approaches (e.g., spectral analysis via K-means clustering~\cite{shen2016acm} or spectral analysis), or methods based on ``source of trust'' (e.g., FLTrust~\cite{cao2020fltrust}).
% OLD, LONG VERSION
%Several countermeasures have been proposed in the literature to combat Byzantine model poisoning attacks on FL systems.
% Descriptive statistics
% For example, Trimmed Mean and FedMedian aggregate local model updates using more robust statistics than standard average~\cite{yin2018icml}.
%
% % Heuristics for outlier detection
% Many existing Byzantine-resilient strategies implement some outlier detection heuristics to discard the model updates sent by potentially malicious clients from the input of the aggregation function.
% One of the most popular heuristics is Krum~\cite{blanchard2017nips}.
% This strategy tries to mitigate the impact of Byzantine attacks by selecting as a global model the local model with the smallest sum of Euclidean distances to {\em all} the other local models.
% Although powerful, Krum requires the server to know (or, at least, estimate) the number of malicious FL clients upfront, which is generally impossible in a realistic attack scenario. %
% Moreover, Krum may become ineffective for complex, high-dimensional model parameter spaces due to the curse of dimensionality.
% Bulyan~\cite{mhamdi2018pmlr} tries to overcome this issue by combining Krum with a variant of Trimmed Mean.
% % Data-driven outlier detection
% Other strategies use data-driven outlier detection techniques -- e.g., via K-means clustering~\cite{shen2016acm} -- to spot potential malicious local model updates. 
% %For instance, Shen et al. propose to cluster local model updates with K-means and thus identify outliers.
%
% % Other techniques
% As far as the server is concerned, any local model received can be from a potential malicious client. 
% FLTrust~\cite{cao2020fltrust} assumes the server acts as a client, i.e., trains a local model on an additional {\em trustworthy} dataset at the server's end and compares it against all the local models from other clients. 
% This way, the server can rely on some ``source of trust'' when discarding potentially malicious clients.
%\\
% Limitations of existing Byzantine-resilient strategies
Unfortunately, existing defense mechanisms either rely on simple heuristics (e.g., Trimmed Mean and FedMedian by~\cite{yin2018icml}) or need strong and unrealistic assumptions to work effectively (e.g., foreknowledge or estimation of the number of malicious clients in the FL system, as for Krum/Multi-Krum~\cite{blanchard2017nips} and Bulyan~\cite{mhamdi2018pmlr}, which, however, cannot exceed a fixed threshold).
Furthermore, outlier detection methods using K-means clustering~\cite{shen2016acm} or spectral analysis like DnC~\cite{shejwalkar2021ndss} do not directly consider the temporal evolution of local model updates received.
Finally, strategies like FLTrust~\cite{cao2020fltrust} require the server to collect its own dataset and act as a proper client, thereby altering the standard FL protocol.
\\
% OLD, LONG VERSION
% Overall, existing Byzantine-resilient strategies are either simple heuristics (e.g., FedMedian) or, if they are more complex, they rely on strong and unrealistic assumptions to work effectively (e.g., knowing the number of malicious clients in the FL system in advance, as for Krum and alike).
% Furthermore, data-driven outlier detection methods do not consider the temporary evolution of local model updates received (e.g., K-means clustering). 
% Finally, strategies like FLTrust requires the server to collect its own dataset and act as a proper client, thereby altering the standard FL protocol.
%
% Description of the proposed method
This work introduces a novel pre-aggregation \textit{filter} robust to untargeted model poisoning attacks. Notably, this filter $(i)$ operates without requiring prior knowledge or constraints on the number of malicious clients and $(ii)$ inherently integrates temporal dependencies. 
The FL server can employ this filter as a preprocessing step before applying \textit{any} aggregation function, be it standard like FedAvg or robust like Krum or Bulyan.
Specifically, we formulate the problem of identifying corrupted updates as a multidimensional (i.e., matrix-valued) time series anomaly detection task. 
The key idea is that legitimate local updates, resulting from well-calibrated iterative procedures like stochastic gradient descent (SGD) with an appropriate learning rate, show \textit{higher predictability} compared to malicious updates. This hypothesis stems from the fact that the sequence of gradients (thus, model parameters) observed during legitimate training exhibit regular patterns, as validated in Section~\ref{subsec:intuition}. %until convergence. 
%This regularity may be more pronounced for smooth convex loss functions, but it can still be captured within an appropriate time window, even for more complex and convoluted loss surfaces. 
%We provide evidence of this claim in Appendix~B, where we show that the average mutual information (i.e., ``predictability''), calculated over pairs of legitimate model updates sent at different FL rounds, is significantly higher than the corresponding computation for a malicious client.
\\
Inspired by the matrix autoregressive (MAR) framework for multidimensional time series forecasting~\cite{chen2021je}, we propose the FLANDERS ({\em \textbf{F}ederated \textbf{L}earning meets \textbf{AN}omaly \textbf{DE}tection for a \textbf{R}obust and \textbf{S}ecure}) filter.
The main advantages of FLANDERS over existing strategies like FLDetector~\cite{zhao2020multivariate} are its resilience to large-scale attacks, where $50\%$ or more FL participants are hostile, and the capability of working under realistic non-iid scenarios.
We attribute such a capability to two key factors: $(i)$ FLANDERS works without knowing a priori the ratio of corrupted clients, and $(ii)$ it embodies temporal dependencies between intra- and inter-client updates, quickly recognizing local model drifts caused by evil players. Below, we summarize our main contributions:

\begin{itemize}
\item[{\em(i)}]
We provide empirical evidence that the sequence of models sent by legitimate clients is more predictable than those of malicious participants performing untargeted model poisoning attacks.
\\
\item[{\em(ii)}] 
We introduce FLANDERS, the first pre-aggregation filter for FL robust to untargeted model poisoning based on multidimensional time series anomaly detection.
\\
\item[{\em(iii)}] 
We integrate FLANDERS into Flower,\footnote{\scriptsize{\url{https://flower.dev/}}} a popular FL simulation framework for reproducibility.
\\
\item[{\em(iv)}] 
We show that FLANDERS improves the robustness of the existing aggregation methods under multiple settings: different datasets, client's data distribution (non-iid), models, and attack scenarios.
\\
\item[{\em(v)}] 
We publicly release all the implementation code of FLANDERS along with our experiments.\footnote{\scriptsize{\url{https://anonymous.4open.science/r/flanders_exp-7EEB}}}
\end{itemize}

% Paper's structure and organization
The remainder of the paper is structured as follows. %some related work and the current state-of-the-art solutions to security issues that FL entails. 
Section~\ref{sec:background} covers background and preliminaries. 
In Section~\ref{sec:related}, we discuss related work.
Section~\ref{sec:problem} and Section~\ref{sec:method} describe the problem formulation and the method proposed. % to tackle it. 
Section~\ref{sec:experiments} gathers experimental results. %, and Section~\ref{sec:limitations} discusses some limitations of this work.
Finally, we conclude in Section~\ref{sec:conclusion}.
 %discusses the limitations of this work and draws future research directions.
%reports conclusions and draws perspectives for future research directions.

%%%%%%% OLD %%%%%%%
%to overcome the resilience of Byzantine failures in distributed Stochastic Gradient Descent computations. 
% The strength of Krum is its time complexity, which is linear in the gradient dimension. 
% However, the robustness of the approach is guaranteed for gradient-based learning applications only when the majority of the clients are not compromised. 
% Besides, the aggregation mechanism of Krum, as well as that of similar methods, is robust from a coarse-grained perspective and does not provide solutions to errors and perturbations that may occur at inference time.
%A related approach to~\cite{blanchard2017nips} is the work of Su et al.~\cite{su2016dc}. Here, the authors propose an iterated approximate agreement to tackle a multi-layer scenario attacked by Byzantine agents. 
%However, the method works efficiently on the sole discrete context and it is inapplicable to continuous state environments.
%\gabri{Maybe, we should just talk about the main limitations of existing countermeasures without digging into their details (or, we can just mention Krum as this is the most popular one). I will move the description of all these methods to the Related Work section.}
% \section{Related work}
\noindent \textbf{Video foundation models.}
With sufficient computational power and an abundant source of data, there have been attempts to build a single large-scale foundation model that can be adapted to diverse downstream tasks.
Along with the success of foundations models in the natural language processing domain~\cite{brown2020language,chen2021evaluating,devlin2019bert} and in computer vision~\cite{bertasius2021space,jia2021scaling,radford2021learning}, video data has become another data type of interest, as it has grown in scale due to numerous internet video-sharing platforms.
Accordingly, several methods to train a video foundation model have been proposed.
Due to the innate multi-modality of video data, \textit{i.e.}, a combination of visual $\cdot$ vocal $\cdot$ textual context, most works have centered around the variations of the cross-modal attention mechanism \cite{akbari2021vatt,bertasius2021space,gabeur2020multi,luo2020univl,neimark2021video,tan2021look,wei2020multi,yang2021taco}.
In addition, as most video data lack proper labels or descriptions, contrastive learning methods were studied to learn meaningful feature representations or enhance video-text alignment in a self-supervised manner \cite{akbari2021vatt,kuang2021video,luo2020univl,yang2021taco}.

More specifically, MERLOT \cite{zellers2021merlot} proposed a multi-modal representation learning method for visual commonsense reasoning, which also performed well in twelve video reasoning tasks.
VATT \cite{akbari2021vatt} introduced a multi-modal learning method via contrastive learning. 
The pre-trained model performed well in a variety of vision tasks from image classification to video action recognition and zero-shot video retrieval.
Another representative work, UniVL \cite{luo2020univl} proposed a straightforward pre-training method with auxiliary loss functions. 
After fine-tuning on a specific task, the pre-trained model performed outstandingly in a wide range of tasks of text-to-video retrieval, action segmentation, action step localization, video sentiment analysis, and video captioning.
Other foundation models for multiple video tasks include \cite{li2020hero,sun2019learning,sun2019videobert,zhu2020actbert,fu2021violet,wang2022all}. 

\noindent \textbf{Auxiliary learning.}
In order to enhance the performance of one or a multitude of primary tasks, auxiliary learning methods can be incorporated.
\cite{ruder2017overview} introduced Multi-task learning (MTL) to the deep neural networks by training a single model with multiple task losses to assist learning on the main task.
Such a method is generally adapted to pre-train the foundation models in the self-supervised manner~\cite{li2020hero,sun2019learning,sun2019videobert,zhu2020actbert,fu2021violet,wang2022all}.
However, these various pretext task losses used in the pre-training phase are ignored in the fine-tuning phase, and only the primary task loss is minimized.

Recently, meta-learning methods have been introduced for auxiliary learning.
\cite{liu2019self,navon2020auxiliary,shu2019meta} proposed a meta-learning method in which the model learns auxiliary tasks to generalize well to unseen data. 
In these settings, a separate subset of data is held out as the primary task, while the others are used as auxiliary tasks that aid the primary task's performance.
Similar methods were adopted for computer vision tasks such as semantic segmentation \cite{xu2021leveraging}.
Other domain applications include navigation tasks with reinforcement learning \cite{ye2021auxiliary}, or self-supervised learning methods on graph data \cite{hwang2020self}.
% \input{src/method_new}
% \section{Experimental Results}
\label{sec:experiments}
\subsection{Training Details}
\cite{Kalantari2017DeepHD} provides the first dataset specifically designed for multi-exposure HDR fusion under large motion. It consists of 74 training sets, which we use to supervise the training of our model. We crop the input images to patches of size \(256 \times 256\) at a step size of 64. This totally generates 20128 training samples. To augment training samples, we randomly rotate and flip the training images. The training adopts Adam optimizer. The learning rate is initialized to \(10^{-4}\) and is reduced to \(10^{-5}\) after 20 epochs. It is observed that 40 epochs are sufficient for the training to converge.    

\subsection{Numerical Evaluation}
We numerically measure the performance of our method on the 15 test sets of \cite{Kalantari2017DeepHD}, by Peak Signal-to-Noise Ratio (PSNR) and Structure Similarity, computed in both tonemapping domain (-\(\mu\)) and HDR linear domain (-L). Visual difference metric HDR-VDP-2 is also adopted, where the parameters are set as same as in previous works \cite{wu2018end} and \cite{niu2021hdrgan}. 

Table \ref{table_metrics} compares our model with state-of-the-art models. For \cite{yan2020nonlocal} and \cite{xiong2021hierarchical}, we use the results reported in the publications. Note that \cite{sen2012robust} and \cite{hu2013hdr} are not machine learning based methods. Moreover,  \cite{Kalantari2017DeepHD} and \cite{wu2018end} apply optical flow and homography transformation to preprocess the input images respectively, and hence entail extra computation. 

Table \ref{table_metrics} shows that our method outperforms competing method in terms of PSNR-L, SSIM-$\mu$, SSIM-L and HDR-VDP-2. It ranks the second best in PSNR-$\mu$, being slightly (0.1dB) inferior to \cite{xiong2021hierarchical}. Note that \cite{xiong2021hierarchical} utilizes a pretrained model to detect ghosting regions for training, whereas our method does not require any pretrained model. The high PSNR and SSIM scores varify that our model has strong HDR reconstruction ability and can accurately restore the radiance and structure of the scene in both tonemapping domain and HDR linear domain. Furthermore, its high performance in term of HDR-VDP-2\cite{mantiuk2011hdr} performance indicates that our method can generate HDR image visually close to the target image.

\begin{table*}[ht]
\centering
\begin{tabular}{l|c|c|c|c|c}
\hline
& PSNR-$\mu$ & PSNR-L & SSIM-$\mu$ & SSIM-L & HDR-VDP-2 \\
\hline
\bfseries Sen & 40.97 & 38.36 & 0.9830 & 0.9746 & 60.60\\
\hline
\bfseries Hu  & 35.65 & 30.80 & 0.9725 & 0.9491 & 58.34\\
\hline
\bfseries Kalantari & 42.69 & 41.22 & 0.9888 & 0.9845 & 65.05\\
\hline
\bfseries DeepHDR& 41.99 & 41.22 & 0.9878 & 0.9859 & \underline{65.91}\\
\hline
\bfseries AHDR & 43.62 & 41.03 & 0.9900  &\underline{0.9883} & 63.85 \\
\hline 
\bfseries NHDRRNet& 42.414 & - & 0.9887 & - & 61.21 \\
\hline 
\bfseries HDR-GAN &43.92 & \underline{41.57} &\underline{0.9905} &0.9865 & 65.45\\
\hline 
\bfseries HFNet & \textbf{44.28} & 41.47 & - & - & - \\
\hline 
\bfseries Ours & \underline{44.18} & \textbf{42.19}&\textbf{0.9912} & \textbf{0.9883}& \textbf{67.07} \\
\hline
\end{tabular}
\caption{Numerical performance of the proposed model, evaluated on the dataset by Kalantari-Ramamoorthi. The best and second best results for each metric are marked in \textbf{bold} and \underline{underlined}, respectively}
\label{table_metrics}
\end{table*}

\subsection{Visual Performance Evaluation}

\begin{figure*}[!htb]
\centering
\includegraphics[width=\textwidth]{experiments/kalantari_test.png}
\caption{Visual comparison on the test set of Kalantari-Ramamoorthi dataset. Zoom-in views of reconstruction by each method are presented on the saturated regions that contain moving objects. Our network built with gated Swin Transformer yields noticeably better visual results than other methods.}
\label{fig_kalantari_test}
\end{figure*}
Fig. \ref{fig_kalantari_test} present the visual performance of our method and comparable methods on two examples from \cite{Kalantari2017DeepHD}. We present the zoom-in views of two challenging cases, where large saturated regions contain substantial non-rigid motion in the reference image. The two patch-based methods do not reconstruct the missing details in the saturated regions, as they heavily rely on the details provided by the reference image for registration. Image reconstructed by the optical flow based method \cite{Kalantari2017DeepHD} suffers motion blur artifacts. This is because the convolutions of DeepHDR and HDR-GAN have limited receptive fields, and hence are hampered to repair missing content in misaligned regions by aligned regions. The gating mechanism of AHDR is only applied to low-level features, so the high-level outliers may deteriorate the HDR fusion. In contrast to comparable methods, our model remarkably overcomes the ghosting artifacts.

\begin{figure}[ht]
\centering
\includegraphics[width=\columnwidth]{experiments/sen_test.pdf}
\caption{Visual performance comparison on example images from the dataset by Sen et al. Zoom in views on challenging areas are presented. Although the ground truth is unavailable, it can be clearly observed that our method visually performs better than comparable methods.}
\label{sen_test}
\end{figure}

\begin{figure}[ht]
\centering
\includegraphics[width=\columnwidth]{experiments/tursun_test.pdf}
\caption{Visual performance comparison on example images from the dataset by Tursun et al. Compared to state of the art methods, our method suffers less ghosting artifact.}
\label{tursun_test}
\end{figure}

Fig.\ref{sen_test} and Fig.\ref{tursun_test} present visual performance of our method on two examples from benchmark datasets \cite{sen2012robust} and \cite{tursun2016objective}. As these test datasets   do not provide ground truth image. we mark the visual difference on the results generated by different methods. It can be seen that our method suffers less artifacts than other methods in various scenes with various motion patterns, achieving better visual results. Our method creates high-quality HDR more robustly and generalizes well. 

\subsection{Ablation Study}

\begin{table}[h]
\centering
\resizebox{\columnwidth}{!}{
\begin{tabular}{l|c|c|c|c|c}
\hline
                         & PSNR-$\mu$ & PSNR-l & SSIM-$\mu$ & SSIM-l & HDR-VDP-2 \\ \hline
restormer(w/o ssim loss) & 44.00  & 41.5   & 0.9906 & 0.9873 & 64.72  \\ \hline
Ours(w/o ssim loss)      & 44.07  & 41.83  & 0.9909 & 0.9879 &  64.78  \\ \hline
Ours                     & 44.18  & 42.19  & 0.9912 & 0.9883 & 67.07      \\ \hline
\end{tabular}
}
\caption{Experimental results of ablation study. We compare using Gated Swin Transformer v.s. Gated Transformer, and the combined loss function v.s. the traditional $l_{1}$ norm loss function.}
\label{table_ablation_block_loss}
\end{table}

We verify various components of our method, including Swin Transformer, loss function, and gating mechanism by ablation study.

\subsubsection{Ablation Study on Block Design}
Our model has similar architecture to Restormer, which uses modified Transformer, whereas we use modified Swin Transformer as the building unit. For comparison, we replace the residual modules in each block in our model with multiple transformer layers as in Restormer, with same number of transformer layers. Table \ref{table_ablation_block_loss} presents the results, which show that using Swin Transformer achieves superior performance in all measures. The reason is that the attention module of Restormer is computed channel-wise, but forgoes the cross-exposure spatial dependency to repair the non-aligned area. 

\subsubsection{Ablation Study on Loss Function}
We trained our model under different loss function configurations, as shown in \ref{table_ablation_block_loss}. The results validate that the SSIM loss benefits detail reconstruction.

\subsubsection{Ablation Study on Gating Mechanism}
\begin{table}[h]
\resizebox{\columnwidth}{!}{
\begin{tabular}{l|c|c|c|c|c}
\hline
           & PSNR-$\mu$ & PSNR-l & SSIM-$\mu$ & SSIM-l & HDR-VDP-2 \\ \hline
w/o gating & 43.14  & 41.03  & 0.9904 & 0.9868 &     64.88      \\ \hline
one gating & 43.44  & 41.42  & 0.9909 & 0.9882 &    67.13   \\ \hline
Ours       & 43.61  & 41.74  & 0.9909 & 0.9881 & 66.96     \\ \hline
\end{tabular}
}
\caption{Ablation experimental results to verify the effectiveness of the gating mechanism}
\label{table_ablation_gating}
\end{table}

The gating mechanism is an important component in our model. Ablation study is conducted in the gating mechanism as follows.

\textbf{w/o gating}: The gating mechanism is not used in the feed forward network of all transformer layers in the model, that it, our GST unit degenerate to the vanilla Swin Transformer.

\textbf{one gating}: The gating mechanism is only used in the first Swin Transformer layers subsequent to the embedding layer, but not used for other layers. 

 Table \ref{table_ablation_gating} shows the results of the ablation experiments, where the model is trained for 20 epochs. By removing the gating mechanism, the network relies on self-attention for image alignment, resulting in the lowest performance. On top of it, adding gates to low level layers notably improves the HDR reconstruction. Furthermore, by integrating the gating mechanism with all Swin Transformer layers, the model effectively inpaints information in non-aligned regions and obtains the highest HDR reconstruction results, thus validates the effectiveness of the gating mechanism in our model.

% \section{Conclusion}\label{sec:conclusion}
In this work, we focus on addressing the fundamental challenge of OOD detection tasks, which is how to fully understand the semantic discrepancy between the ID/OOD samples. We reveal that the key to success in the realistic SCOOD task is to allocate as many ID samples in the unlabeled set correctly as possible. To this end, we propose a novel uncertainty-aware optimal transport scheme that introduces class-specific energy scores as guidance for effective label assignment. Experimental results show that our method achieves better performance than previous state-of-the-art methods on SCOOD benchmarks.

\textbf{Limitations.} In addition to temperature scaling, other techniques such as feature clipping applied in ReAct~\cite{sun2021react} also enhance the performance of energy score, so how to obtain an OOD score that best fits the SCOOD task can be further explored. Moreover, a setting highly related to SCOOD has been proposed in \cite{katz2022training} and formulated as a constrained optimization problem. We will also theoretically analyze these practical OOD settings in our feature work.

% \section*{Acknowledgments}
\textbf{Acknowledgments.} 
This work is supported by National Key R\&D Program of China under Grant 2020AAA0105701, National Natural Science Foundation of China (NSFC) under Grants 61872327, Major Special Science and Technology Project of Anhui, National Natural Science Foundation of China (62033012) and Ant Group through Ant Research Intern Program.

\section{Introduction}
%Image-based relighting is a hot-topic in the Augmented Reality and Extended Reality. Relighting from a single image is the most common scenario of mobile AR apps. It is a highly ill-posed problem since relighting requires re-render the objects under new lightings, but all we get is one observed image under unknown lighting condition. To tackle this problem, single-image inverse rendering is performed. Inverse rendering aims to recover the essential inputs for re-rendering such as material, shape and illumination, from a single image observation.
%The recovered 3D properties facilitate Augmented Reality (AR) applications such as inserting objects vividly into arbitrary scenes. %which is common in commercial design and mobile amusement.

Object insertion finds extensive applications in Mobile AR.
Existing AR object insertions require a perfect mesh of the object being inserted. Mesh models are typically built by professionals and are not easily accessible to amateur users. Therefore, in most existing AR apps such as SnapChat and Ikea Place, users can use only built-in virtual objects for scene augmentation.
This may greatly limit user experience.
A more appealing setting is to allow the user to extract objects from a photograph and insert them into the target scene with proper lighting effects. This calls for a method of inverse rendering and relighting based on a single image, which has so far been a key challenge in the graphics and vision fields.

Relighting real objects requires recovering lighting, geometry and materials which are intertwined in the observed image; it involves solving two problems, inverse rendering~\cite{patow2003survey} and re-rendering. Furthermore, to achieve realistic results, the method needs to be applicable for non-Lambertian objects. In this paper, we propose a pipeline to solve both problems, weakly-supervised inverse rendering and non-Lambertian differentiable rendering for Lambertian and low-frequency specular objects. 

\begin{figure}
	\centering
	\includegraphics[width=\linewidth]{teaser-new2.pdf}\
	\caption{Our method relights real objects into new scenes from single images, which also enables editing materials from diffuse to glossy with non-Lambertian rendering layers. }
	\label{fig:teaser}
	\vspace{-1.2em}
\end{figure}

\begin{figure*}[t]
	\centering
	\includegraphics[width=\linewidth]{rj_pipeline2.pdf}\
	\caption{Overview of our method. At training time, Spec-Net separates input images into specular and diffuse branches. Spec-Net, Normal-Net and Light-Net are trained in a self-supervised manner by the Relit dataset. At inference time, inverse rendering properties are predicted to relight the object under novel lighting and material. The non-Lambertian render layers produce realistic relit images. }
	\label{fig:pipeline}
	\vspace{-0.2cm}
\end{figure*}

Inverse rendering is a highly ill-posed problem, with several unknowns to be estimated from a single image. Deep learning methods excel at learning strong priors for reducing ill-posedness. However, this comes at the cost of a large amount of labeled training data, which is especially cumbersome to prepare for inverse rendering since ground truths of large-scale real data are impossible to obtain. Synthetic training data brings the problem of domain transfer. Some methods explore self-supervised pipelines and acquire geometry supervisions of real data from 3D reconstruction by multi-view stereo (MVS)~\cite{yu2019inverserendernet,yu2020self}. Such approaches, however, have difficulties in handling textureless objects.

To tackle the challenge of training data shortage, 
%we collect a large-scale video dataset Relit of objects well-alined among frames under changing illumination. Based on the dataset,
we propose a \emph{weakly-supervised inverse rendering pipeline} based on a novel low-rank loss and a re-rendering loss. 
%Most prior arts in inverse rendering are designed for outdoor buildings ~\cite{yu2019inverserendernet,yu2020self}, where multi-view stereo (MVS) is used to reconstruct 3D geometry as supervision. This method, however, may not work as well for household objects since smooth and textureless surfaces would make accurate and complete reconstruction especially challenging for MVS.
%We base our work on a
For low-rank loss, a base observation here is that the material reflectance is invariant to illumination change, as an intrinsic property of an object. We derive a low-rank loss for inverse rendering optimization which imposes that \emph{the reflectance maps of the same object under changing illuminations are linearly correlated}. 
In particular, we constrain the reflectance matrix with each row storing one of the reflectance maps to be rank one.
This is achieved by minimizing a low-rank loss defined as the Frobenius norm between the reflectance matrix and its rank-one approximation.
%We have proved that the rank-one approximation of the reflectance matrix can be recovered from its singular value decomposition.
We prove the convergence of this low-rank loss. In contrast, traditional Euclidean losses lack a convergence guarantee.

%The low rank constraint entails the linear dependence among the reflectance maps of an object under different illuminations, rather than enforcing them to be identical. This helps resolve the intensity (scale) ambiguity between reflectance and lighting. In the field of intrinsic image decomposition, a scale-invariant loss is usually used to resolve the ambiguity~\cite{grosse2009ground}. Here, our low rank constraint is also scale-invariant.
To facilitate the learning, we contribute Relit, a large-scale dataset of videos of real-world objects with changing illuminations. We design an easy-to-deploy capturing system: a camera faces toward an object, both placed on top of a turntable. Rotating the turntable will produce a video with the foreground object staying still and the illumination changing. To extract the foreground object from the video, manual segmentation of the first frame suffices since the object is aligned across all frames. 


As shown in Figure~\ref{fig:pipeline}, a fixed number of images under different lighting are randomly selected as a batch. We first devise a Spec-Net to factorize the specular highlight, trained by the low-rank loss on the chromaticity maps of diffuse images (image subtracts highlight) which should be consistent within the batch. With the factorized highlight, we further predict the shininess and specular reflectance, which is self-supervised with the re-rendering loss of specular highlight. For the diffuse branch, we design two networks, Normal-Net and Light-Net, to decompose the diffuse component by predicting normal maps and spherical harmonic lighting coefficients, respectively. The diffuse shading is rendered by normal and lighting, and diffuse reflectance (albedo) is computed by diffuse image and shading. Both networks are trained by low-rank loss on diffuse reflectance. 
%and we are the first deep inverse rendering methods for natural objects. Object inverse rendering is more flexible and practical in AR object insertions, which enables inserting and relighting multiple objects from different photos into the same scene. However, the problem setting is ill-posed so the problem is challenging and leads to a lot of ambiguities. Meanwhile, the performance of deep learning methods rely on the quality of training data. For several sub-problems in inverse rendering, such as intrinsic image decomposition or surface normal estimation, some existing techniques rely on supervision from synthesized dataset~\cite{shi2017learning,narihira2015direct,janner2017self}. However, synthesizing large-scale ground-truth datasets for training deep neural networks is expensive in both computation and time, and deep neural networks trained on synthetic data usually have a domain gap while testing on real images. This motivates recent works to developing unsupervised schemes for intrinsic image decomposition\cite{ma2018single,li2018learning,yi2020leveraging}. Since there are natural ambiguities between reflectance and shading, and to simplify the decomposition, some methods propose to add priors such as piece-wise reflectance and smoothness in shading, or use deep neural networks to learn the appearance of each terms\renjiao{cite the paper that uses image translation to solve intrinsic image}. This priors help to generate reasonable decomposition but they still cannot ensure the principal correctness behind the decomposition, which is the rendering/reflection process. In order to solve inverse rendering respecting to the physical reflection behind, we propose to reconstruct every properties ( i.e. lighting, shading, reflectance...) according to the rendering equation.
%\renjiao{This para is about the structure}
%\renjiao{This para is about the dataset}
%Firstly, an optional specularity separation step is performed to separate the possible highlights on the input image.


Regarding the re-rendering phase, 
%we design an efficient differentiable specular rendering layer based on normal maps. In our pipeline, 
the main difficulty is the missing of 3D information of the object given a single-view image. The Normal-Net produces a normal map which is a partial 3D representation, making the neural rendering techniques and commercial renderers inapplicable. The existing diffuse rendering layer for normal maps of~\cite{ramamoorthi2001efficient} cannot produce specular highlights. Pytorch3D and \cite{li2022phyir,li2020inverse} render specular highlights for point lights only. 
%Representing environment lighting as combinations of point lights are inefficient.

%\renjiao{This para is about the religting, the difference between scene and object inverse rendering}
%For relighting, since geometry from a single image is represented as normal map from inverse rendering, without mesh, voxel or other kinds of full 3D representations. Many neural rendering techniques are not applicable.
To this end,
%for the diffuse reflections, we use a Lambertian-based differentiable renderer based on inverse rendering outputs. For non-Lambertian reflections,
we design a \emph{differentiable specular renderer} from normal maps, based on the Blinn-Phong specular reflection~\cite{blinn1977models} and spherical harmonic lighting~\cite{green2003spherical}. Combining with the differentiable diffuse renderer, we can render low-frequency non-Lambertian objects with prescribed parameters under various illuminations, and do material editing as byproduct.

%Our non-Lambertian rendering takes $0.35$ seconds per image. 
We have developed \emph{an Android app} based on our method which allows amateur users to insert and relight arbitrary objects extracted from photographs in a target scene. Extensive evaluations on inverse rendering and image relighting demonstrate the state-of-the-art performance of our method. 
%Both video and app demos can be found in the supplementary materials.
%
%\renjiao{Evaluations show normal estimation of the proposed method on the synthetic dataset by Janner et al.~\shortcite{janner2017self} outperforms the state-of-the-art supervised and unsupervised inverse rendering methods by $10\%$ for MSE and $11.8\%$ for DSSIM. Our method outperforms the state-of-the-art inverse rendering and unsupervised intrinsic image decomposition methods by $12.5\%$ for LMSE and $3.6\%$ for SMSE on the MIT Intrinsics dataset on intrinsic image decomposition. }
%Meanwhile, the codes and demo videos of relighting results are in the supplements. Due to size limit, the Relit dataset will be made public online upon publication.
%
%

Our contributions include:
\begin{itemize}
	%\item We propose a physically-motivated single-image inverse rendering pipeline, which is self-.
	\vspace{-5pt}\item A weakly-supervised inverse rendering pipeline trained with a low-rank loss. The correctness and convergence of the loss are mathematically proven.
	\vspace{-5pt}\item A large-scale dataset of foreground-aligned videos collecting $750K$ images of $100$+ real objects under different lighting conditions.
	%The dataset can potentially benefit many vision tasks such as inverse rendering and object segmentation.
	\vspace{-5pt}\item An Android app implementation for amateur users to make a home-run.
	%\item We realize end-to-end real object relighting from a single image, and implement an AR relighting app.
	%Comparing to virtual object insertion in current AR applications, real object insertion is more flexible and can be customized by amatuer users.
\end{itemize}
\section{Related Work}

{\bf{Inverse rendering.}}
As a problem of inverse graphics, inverse rendering aims to solve geometry, material and lighting from images. This problem is highly ill-posed. Thus some works tackle the problem by targeting a specific class of objects, such as faces~\cite{shu2017neural,tewari2017mofa} or planar surfaces~\cite{aittala2016reflectance}. For inverse rendering of general objects and scenes, most prior works~\cite{barron2015shape,janner2017self,li2018learning2,Lichy_2021_CVPR} require direct supervisions by synthesized data. However, networks trained on synthetic data have a domain gap for real testing images. Ground truths of real images are impossible to obtain, and it calls for self-supervised methods training on real images. Recently, self-supervised methods~\cite{yu2019inverserendernet,yu2020self} explore self-supervised inverse rendering for outdoor buildings, where the normal supervision is provided by reconstructing the geometry by MVS. However, they do not work well for general objects, which is reasonable because object images are unseen during training. However, applying the pipelines for objects meet new problems. Textureless regions on objects are challenging for MVS due to lack of features. It motivates our work on weakly-supervised inverse rendering for general objects. To fill the blank of real-image datasets on this topic, we capture a large-scale real-image datasets Relit to drive the training.  

There are also many works addressing inverse rendering as several separated problems, such as intrinsic image decomposition~\cite{shi2017learning,yi2020leveraging,liu2020unsupervised,li2018cgintrinsics}, specularity removal~\cite{shen2013real,shi2017learning,yamamoto2019general} or surface normal estimation~\cite{li2018learning2}. In order to compare with more related methods, we also evaluate these tasks individually in experiments. %Intrinsic image decomposition solves diffuse reflectance and shading from input images.  

\begin{comment}
	{\bf{Differentiable lighting rendering. }}
	%Rendering is the computational process of transforming geometry, material and lighting into an image observation, simulating real-life scenarios. 
	Rendering bridges 3D scenes and 2D images, and there are many commercial renderers, such as 3D Max and Mitsuba, for high quality lighting rendering. However, such  commercial renderers are unable to embed to deep learning based pipelines for end-to-end training, and they require full 3D models such as meshes. Partial 3D representations such as normal maps are unable to render in these commercial renderers. To make the whole inverse rendering and relighting process end-to-end and differentiable, we need differentiable renderers working as network modules and propagate the gradients. 
	
	In our pipeline, after we get the normal map and reflectances from inverse rendering, parametric reflection models such as Phong~\cite{phong1975illumination}, Blinn-Phong~\cite{blinn1977models} and Lambertian can be performed to re-render objects under different lighting, or with different materials. Although they are all differentiable, they can be computational expensive and inaccurate while performing Monte Carlo integration~\cite{kajiya1986rendering} of point lights~\cite{deschaintre2018single} for the whole lighting sphere, as in PyTorch3D. Spherical harmonics are commonly used to model environmental lighting, and achieves better affects than point lights. A computationally efficient method is proposed by \cite{ramamoorthi2001efficient} which performs Lambertian rendering in real-time by spherical harmonic lighting. It is commonly adopted in recent methods~\cite{yu2019inverserendernet,yu2020self} and computer games. %In order to render non-Lambertian materials, we propose a differentiable specular renderer, which performs in real-time as well. 
	Currently, there are no non-Lambertian differentiable renderers for spherical harmonic lighting, and it motivates our work on the differentiable specular renderer. 
\end{comment}

{\bf{Image relighting.}}
%Image-based relighting methods usually perform inverse rendering as the first step. 
Most prior methods in image-based relighting require multi-image inputs~\cite{azinovic2019inverse,xu2018deep}. For example, in \cite{xu2018deep}, a scene is relit from a sparse set of five images under the optimal light directions predicted by CNNs. Single-image relighting is highly ill-posed, and needs priors. \cite{philip2019multi,yu2020self} target outdoor scenes, and benefit from priors of outdoor lighting models. \cite{meka2019deep,shu2017neural,shu2017portrait,sun2019single,sengupta2018sfsnet} target at portrait images, which is also a practical application for mobile AR. Single image relighting for general scenes have limited prior works. Yu et al. \cite{yu2020self} takes a single image as inputs, with the assumption of Lambertian scenes. In this work, we propose a novel non-Lambertian render layer, and demonstrate quick non-Lambertian relighting of general objects.  
%enables not only relighting but also material editing, 
% implemented as an Android App.   



\section{Overview}



We propose a deep neural network to solve single-image inverse rendering and object-level relighting. 
%A novel differentiable image rendering model which enable the relighting application is proposed is Section~\ref{sec:diff_render}. 
The overall pipeline is shown in Figure~\ref{fig:pipeline}. 
%We design a lightweight inverse rendering network consisting of only three learnable modules, Normal-Net, Light-Net and Spec-Net. All other parts of the network are non-parametric. Diffuse reflectance and shading are directly rendered from the normal maps and lighting predicted by the networks.
%To train Normal-Net and Light-Net in an unsupervised manner, we randomly take frames from the Relit videos to form training batches.
%The images in the same batch are well-aligned objects under different illuminations. Based on the observation that under different illuminations the surface reflectance should remain identical, we propose a low rank constraint on the reflectance output of the images in one batch.
%Normal-Net and Light-Net are trained in an interleaving manner, both based on the low rank loss on reflectance. %Since predicting normal and lighting is a chicken-and-egg problem, we use a small amount of synthetic data to pretrain the Normal-Net to warm up the training. After that, the two network modules are trained alternatively.
%For image inputs with non-Lambertian objects, a Spec-Net is devised to factor out the highlights before inverse rendering, which is trained similarity in self-supervised fashion.  
%Spec-Net is finetuned in a self-supervised manner on the pretrained model of~\cite{yi2020leveraging}.
%After training, reflectance, shading, normal and lighting are predicted at once from a single image.
% The whole pipeline is weakly-supervised, driven by the proposed Relit Dataset. 
The whole pipeline is weakly-supervised with a supervised warm-up of Normal-Net, and self-supervised training of the whole pipeline. The self-supervised training is driven by the Relit Dataset. 
The details of Relit Dataset is intoduced in Section~\ref{sec:dataset}. 
In Section~\ref{sec:method}, we introduce the proposed pipeline following the order from single-image inverse rendering to differentiable non-Lambertian relighting. The weakly-supervised inverse rendering, including the proofs of theoretical fundamentals and convergence of the low-rank loss, are introduced in Section~\ref{sec:network}. 
%The novel low rank constraint is proposed to drive the self-supervised training, whose theoretical fundamentals and convergence are proven in Section~\ref{sec:unsupervised}. 
The differentiable non-Lambertian rendering layers are introduced in Section~\ref{sec:relighting}. 
%, where we can relight objects in a wide range of different materials from diffuse to glossy. 
% The proposed single-image relighting is implemented as an Android app, demos can be found in the video. 

\section{The Relit Dataset}\label{sec:dataset}
\begin{figure*}[t]
	\centering
	\includegraphics[width=0.95 \linewidth]{RelitGAN_dataset_device_small.pdf}
	\caption{Left: The data capture set-up. Right: Selected objects in Relit dataset. The last row shows selected frames from one video. }
	\label{fig:dataset}
	%\vspace{-0.5cm}
\end{figure*}
% \begin{figure}
	% 	\centering
	% 	\includegraphics[width= \linewidth]{figure/device_new.pdf}\
	% 	%\caption{The set-up of capturing foreground-aligned videos in the Relit dataset. }
	% 	\caption{The capture set-up of our Relit dataset. }
	% 	\label{fig:device}
	% \end{figure}
To capture foreground-aligned videos of objects under changing illuminations, we design an automatic device for data capture, as shown in Figure~\ref{fig:dataset} (left). The camera and object are placed on the turntable, and videos are captured as the turntable rotating. The target object stays static among the frames in captured videos, with changing illuminations and backgrounds. 
%The central part is an electric turntable painted black to avoid strong reflections. 
%Objects are taped on the turntable, and the camera is fixed on the turntable by a screw.
%While capturing data, objects and camera are fixed on the turntable. The turntable rotates at a uniform angular velocity of $12.6$ rad/s, controlled by a remote to avoid shaking. For each video, the device is rotated by $360^{\circ}$ for 50 seconds. 
%The device is chargeable and portable, enabling us to capture data under arbitrary scenes easily. The target object stays static in the image coordinate system in captured videos, with changing illuminations and backgrounds. 
In summary, the Relit dataset consists of 500 videos for more than 100 objects under different indoor and outdoor lighting. Each video is 50 seconds, resulting in 1500 foreground-aligned frames under various lighting. In total, the Relit dataset consists of $750K$ images. %, which benefits many applications in deep learning. 
%with one corresponding object mask (only one mask is needed for each video since objects are aligned among frames)
%In pre-processing, we segment the mask for one frame of each video and apply it to all frames to remove the changing backgrounds. 
Selected objects are shown in Figure~\ref{fig:dataset} (right). The objects cover a wide variety of shapes, materials, and textures. In Section~\ref{sec:method}, we introduce how to leverage Relit dataset to drive the self-supervised training. 
% The dataset are released on the project page. % of our networks. 
It can facilitate many tasks, such as image relighting and segmentation. %More details about the dataset are in the supplementary materials. 

\section {Our Method}\label{sec:method}

\subsection{Image formation model}

A coarse-level image formation model for inverse rendering is intrinsic image decomposition (IID), which is a long-standing low-level vision problem, decomposing surface reflectance from other properties, assuming Lambertian surfaces. For non-Lambertian surfaces, the model can be improved by adding a specular highlight term:

\begin{equation}
	I= I_d + H, \quad I_d = \mathcal{A}\odot S, \label{equation:iid}
\end{equation}

\noindent where $H$ is the specular highlight, $\mathcal{A}$ is the surface reflectance map, i.e. albedo map in IID, and $S$ is a term describing the shading related to illumination and geometry. Here $\odot$ denotes the Hadamard product. To be more specific, according to the well-known Phong model\cite{phong1975illumination} and Blinn-Phong model\cite{blinn1977models}, the image can be formulated as the sum of a diffuse term and a specular term:

\begin{equation}
	\centering
	\begin{aligned}
		I(p)=&I_d(p)+H(p),\\
		I_d(p)=&\mathcal{A}(p)S(p)=\mathcal{A}(p)\sum_{\omega \in \mathcal{L}}l_\omega(L_\omega\cdot n(p)),\\ H(p)=&\sum_{\omega \in \mathcal{L}}s_pl_\omega(\frac{L_\omega+v}{\|L_\omega+v\|}\cdot n(p))^\alpha,\label{equation:phong}
	\end{aligned}
\end{equation}

\noindent where $I(p)$ is the observed intensity and $n_p=(x,y,z)$ is the surface normal at pixel $p$. $\mathcal{L}$ is a set of sampled point lights in the lighting environment. $L_\omega$ and $l_\omega$ describe lighting direction and intensity of one point light $\omega$ in $\mathcal{L}$ respectively. $\mathcal{A}(p)$ and $s_p$ are defined as the diffuse and specular reflectance at pixel $p$, respectively. The specular term is not view independent, view direction $v$ is needed to calculate the reflectance intensity and $\alpha$ is a shininess constant. The differentiable approximation for Equation~(\ref{equation:phong}) is introduced in Section~\ref{sec:diffuserender}-\ref{sec:specrender}.%Our main concern here is formulating a differentiable approximation for Equation(\ref{equation:phong}).

\subsection{Inverse rendering from a single image}\label{sec:network}

For relighting, we first inverse the rendering process to get 3D properties including geometry, reflectance, shading, illumination and specularities, then we can replace the illumination and re-render the objects. Following this order, we firstly introduce inverse rendering. 

For non-Lambertian object, we can perform specular highlight separation first by the Spec-Net. The specular parameters are then predicted in the specular branch, which is introduced in Section~\ref{sec:spec}. 

%The specular highlight separation works as an optional step in our inverse rendering pipeline, 
%We first introduce our main pipeline. Specular highlight separation is optional depending on the input image.  and we can inverse render them using our main pipeline. It is easy for users to tell when their are strong specular reflections, where there are usually many pixels saturated and tends to be white in specular highlight regions. In these cases, we adopt an optional Spec-Net as the first step to remove the specularities. The training of Spec-Net is initialized from the network of Yi et al. \shortcite{yi2020leveraging}, enhanced with the set of non-Lambertian objects in our Relit Dataset by unsupervised finetuning. Similar to the unsupervised loss on reflectance, the training loss here is the low rank constraint defining on rg-chromaticity of diffuse reflections.}

For diffuse branch, adopting separate networks to predict normal, lighting, shading, reflectance is the most straightforward choice. However, in this way, the diffuse component in the rendering equation (Equation (\ref{equation:phong})) is not respected, since relations between these properties are not constrained. 
Thus, we design a lightweight physically-motivated inverse rendering network, respecting the rendering equation strictly, as shown in Figure~\ref{fig:pipeline}. 
There are only two learnable network modules in our end-to-end diffuse inverse rendering pipeline.
Here we adopt spherical harmonics~\cite{ramamoorthi2001efficient} to represent illumination $\mathcal{L}$ in Equation (\ref{equation:phong})), which is calculated more efficiently than Monte Carlo integration of point lights:

%\vspace{-0.5cm}
\begin{equation}
\mathcal{L}=\sum_{l=0}^{\infty}\sum_{m=-l}^{l}C_{l,m}Y_{l,m}, \label{equation:SH}
\end{equation}
%\vspace{-0.5cm}

\noindent where $Y_{l,m}$ is the spherical harmonic basis of degree $l$ and order $m$, $C_{l,m}$ is the corresponding coefficient. Each environment lighting can be represented as the weighted sum of spherical harmonics. The irradiance can be well approximated by only 9 coefficients, 1 for $l = 0, m = 0$, 3 for $l = 1, -1 \leq m \leq 1$, and 5 for $l = 2, -2 \leq m \leq 2$. %However, the specular componet is view dependent, which can not be simply parameterized with $Y_{l,m}$ and $C_{l,m}$.

%Only two modules, Normal-Net and Light-Net, in our end-to-end inverse rendering pipeline require training. 
Normal-Net predicts surface normal maps $n$, and Light-Net regresses lighting coefficients $C_{l,m}$ in spherical harmonic representation. A total of 12 coefficients are predicted by Light-Net, where the last 3 coefficients present the illumination color. 
%and other outputs are computed by the image formation models  since there are natural ambiguities between reflectance and shading decomposition, we train one encoder-decoder Normal-Net for normal prediction only. In order to complete the inverse rendering task in a light-weight fashion, and avoid the ambiguities between terms, other than the Normal-Net, we only have a encoder Light-Net that is trainable. 
The shading $S$ is then rendered from the predicted normal and lighting, by a hard-coded differentiable rendering layer (no learnable parameters) in Section~\ref{sec:diffuserender}, following Equation (\ref{equation:phong}). The reflectance $\mathcal{A}$ is computed by Equation (\ref{equation:iid}) after rendering shading. The pipeline design is based on the physical rendering equation (Equation (\ref{equation:phong})), where relations among terms are strictly preserved. 

%The end-to-end inverse rendering network is shown in . Normal, lighting coefficients, shading and reflectance are predicted in the end-to-end fashion. 
%For relighting, after end-to-end inverse rendering, the lighting coefficients are replaced by those of new scenes, and the object is re-rendered by the differentiable render layer, as shown at the bottom half in Figure~\ref{fig:pipeline}. The whole process takes only about 0.35 seconds.
%The inverse rendering and relighting of one image takes about 0.35 seconds.




%\subsection{The differetiable renderer}\label{sec:render}


%In the rendering, all computations are differentiable, and gradients are able to back-propagate through rendering layers.

\subsubsection{Self-supervised low-rank constraint}\label{sec:unsupervised}

%\renjiao{firstly introduce the dataset processing, then observation, than define the loss, discussion about mean, median albedo? drive the whole network}

We have foreground-aligned videos of various objects under changing illuminations in Relit dataset. The target object is at a fixed position in each video, which enables pixel-to-pixel losses among frames. %Note that the changing backgrounds are deleted after pre-processing.

For each batch, $N$ images $I_1$, $I_2$, ..., $I_N$ are randomly selected from one video. Since the object is aligned in $N$ images under different lighting, one observation is that the reflectance should remain unchanged as an intrinsic property, and the resulting reflectance $\mathcal{A}_1$, $\mathcal{A}_2$, ..., $\mathcal{A}_N$ should be identical. However, due to the scale ambiguity between reflectance and lighting intensities, i.e., estimating reflectance as $\mathcal{A}$ and lighting as $\mathcal{L}$, is equivalent to estimating them as $w \mathcal{A}$ and $\frac{1}{w}\mathcal{L}$. A solution for supervised methods is defining a scale-invariant loss between ground truths and predictions. However the case is different here, there are no predefined ground truths. While adopting traditional Euclidean losses between every pair in $\mathcal{A}_1$, $\mathcal{A}_2$, ..., $\mathcal{A}_N$, it leads to degenerate results where all reflectance are converged to zero. To solve the problem, here we enforce $\mathcal{A}_1$, $\mathcal{A}_2$,..., $\mathcal{A}_N$ to be linearly correlated and propose a rank constraint as training loss. Therefore, a scaling factor $w$ does not affect the loss. 
%, i.e., scaling factor $\alpha$ applying to each reflectance does not affect the losses.

We can compose a matrix $R$ with each reflectance $\mathcal{A}_i$ storing as one row. Ideally, rows in $R$ should be linearly correlated, i.e., $R$ should be rank one. 
We formulate a self-supervised loss by the distance between $R$ and its rank-one approximation. 
%We can measure how close $R$ is rank one by measuring the distance to its rank 1 approximation. 
We introduce Theorem 1 below. 
%, along with its proofs.  

\begin{theorem}\label{theorem:rankone}
\textbf{Optimal rank-one approximation.} By SVD,  $R = U\Sigma V^T$,  $\Sigma=diag(\sigma_1,\sigma_2,...\sigma_k)$, $\Sigma'=diag(\sigma_1,0,...)$, $\bar{R} = U\Sigma'V^T$ is the optimal rank-one approximation for R, which meets:

\begin{equation}
	\label{equation:optimal}
	||\bar{R}-R||^2_F = \min_{\scriptscriptstyle b\in\mathcal{R}^{N},c\in\mathcal{R}^{d}}||bc^T-R||^2_F,
\end{equation}

\noindent where $||\cdot||_F$ denotes the Frobenius norm of a matrix. %The proof of $\bar{R}$ being the optimal rank 1 approximation of $R$ is in the supplements. 


\end{theorem}

The proof of Theorem~\ref{theorem:rankone} can be found in Appendix~\ref{sec:supp}. 
\begin{comment}

\begin{proof}
	The objective in (\ref{equation:optimal}) can be written as following:
	
	\begin{equation}
		\label{equation:optimal_sigma}
		||bc^T-R||^2_F = \sum_{i=1}^{d}||c_i\cdot b-r_i||^2_2. 
	\end{equation}
	
	\noindent To minimize $||c_i\cdot b-r_i||^2_2$ while $b$ is a fixed unit vector, $c_i\cdot b$ should be the projection of $r_i$ onto $b$ ($r_i$ is the $i^{th}$ column of $R$). It is equivalent to $c=b^TR$. Then we reduce the optimization problem (\ref{equation:optimal}) as:
	
	\begin{equation}
		\label{equation:reduced}
		\min_{\scriptscriptstyle b\in\mathcal{R}^{N},||b||_2=1}||bb^TR-R||^2_F.
	\end{equation}
	
	\noindent Since $b$ is a unit vector and  V are orthonormal, we can rewrite $||bb^TR||^2_F$ as:
	
	\begin{equation}
		\label{equation:svd_bbt}
		\begin{aligned}
			||bb^TR||^2_F&=||b^TR||^2_2=||b^TU\Sigma V^T||^2_2\\
			&=||b^TU\Sigma||^2_2=\sum_{i=1}^k(b^Tu_i)^2\sigma_i^2.
		\end{aligned}	
	\end{equation}
	
	\noindent By Pythagorean Theorem, and $||R||^2_F\geq  {\textstyle \sum_{i=1}^{n}} ||c_ib||_2$, optimization problem (\ref{equation:reduced}) is equivalent to maximizing (\ref{equation:svd_bbt})). Since $\sum_{i=1}^k(b^Tu_i)^2=1$ and $\{\sigma_i\}$ are descending, Equation (\ref{equation:svd_bbt}) is maximized when $(b^Tu_1)^2=1$. It can be accomplished by setting $b=u_1$ and $c=\sigma_1v_1^T$, i.e., $\bar{R} = U\Sigma'V^T = bc^T$ is the optimal rank 1 approximation for $R$. 
\end{proof} 

%\textbf{\textit{Proof.}} 
\begin{proof}
	The objective in (\ref{equation:optimal}) can be written as following:
	
	\begin{equation}
		\label{equation:optimal_sigma}
		||bc^T-R||^2_F = \sum_{i=1}^{d}||c_i\cdot b-r_i||^2_2. 
	\end{equation}
	
	\noindent To minimize $||c_i\cdot b-r_i||^2_2$ while $b$ is a fixed unit vector, $c_i\cdot b$ should be the projection of $r_i$ onto $b$ ($r_i$ is the $i^{th}$ column of $R$). It is equivalent to $c=b^TR$. Then we reduce the optimization problem (\ref{equation:optimal}) as:
	
	\begin{equation}
		\label{equation:reduced}
		\min_{\scriptscriptstyle b\in\mathcal{R}^{N},||b||_2=1}||bb^TR-R||^2_F.
	\end{equation}
	
	\noindent Since $b$ is a unit vector and  V are orthonormal, we can rewrite $||bb^TR||^2_F$ as:
	
	\begin{equation}
		\label{equation:svd_bbt}
		\begin{aligned}
			||bb^TR||^2_F&=||b^TR||^2_2=||b^TU\Sigma V^T||^2_2\\
			&=||b^TU\Sigma||^2_2=\sum_{i=1}^k(b^Tu_i)^2\sigma_i^2.
		\end{aligned}	
	\end{equation}
	
	\noindent By Pythagorean Theorem, and $||R||^2_F\geq  {\textstyle \sum_{i=1}^{n}} ||c_ib||_2$, optimization problem (\ref{equation:reduced}) is equivalent to maximizing (\ref{equation:svd_bbt})). Since $\sum_{i=1}^k(b^Tu_i)^2=1$ and $\{\sigma_i\}$ are descending, Equation (\ref{equation:svd_bbt}) is maximized when $(b^Tu_1)^2=1$. It can be accomplished by setting $b=u_1$ and $c=\sigma_1v_1^T$, i.e., $\bar{R} = U\Sigma'V^T = bc^T$ is the optimal rank 1 approximation for $R$. 
\end{proof} 
\end{comment}

Therefore, we define the low-rank loss as:

%\vspace{-0.5cm}
\begin{equation}
f(R) = ||\bar{R}-R||^2_F. 
\end{equation} \label{equation:loss}
%\vspace{-0.5cm}

%In Section~\ref{sec:discussions}, we provide robustness evaluations of three losses. 
%Here, the proposed formation of low rank constraint is robust. 
Its convergence is proven as below, fitting the needs of learning-based approaches training by gradient descents. 

Since the gradient of $\bar{R}$ is detached from the training, the derivative of $f(R)$ can be accomplished as $\nabla f(R)=-2(\bar{R}-R)$. According to the gradient descent algorithm, with a learning rate  $\eta$, the result $R^{(n+1)}$ ($R$ after $n+1$ training iterations), can be deduced as: 

\begin{equation}
\label{equation:loss_gd}
R^{(n+1)} = R^{(n)} + 2\eta(\bar{R}-R^{(n)}),
\end{equation} 

%\noindent where $\eta$ is the learning rate. 

\begin{theorem}
\textbf{Convergence of $f(R)$.} 
The loss $f(R)$ would converge to a fixed point, which is $\bar{R}$ while $0<\eta<0.5$, 
\begin{equation}
	\label{equation:theorem2}
	\lim_{n \to \infty}R^{(n)}=\bar{R} \Leftarrow  0<\eta<0.5, R^{(0)}=R. 
\end{equation} 
\end{theorem}

\begin{proof}
According to Equation (\ref{equation:loss_gd}), $R = U\Sigma V^T$ and $\bar{R} = U\Sigma' V^T$, we have:
%\vspace{-0.2cm}
\begin{equation}
	\label{equation:r1}
	\begin{aligned}
		R^{(1)}&=R+2\eta(\bar{R}-R)\\
		&=U \mbox{diag} \{\sigma_1, (1-2\eta)\sigma_2,\dots, (1-2\eta)\sigma_k\}V^T.
	\end{aligned}
\end{equation} 

\noindent Since $0<\eta<0.5$, we have $1-2\eta < 1$, $\{\sigma_1, (1-2\eta)\sigma_2,\dots, (1-2\eta)\sigma_k\}$ are still descending. Therefore, Equation (\ref{equation:r1}) is the SVD form for $R^{(1)}$. Similarly, we have: 
\vspace{-0.2cm}
\begin{equation}
	\label{equation:r2}
	\begin{array}{l}
		R^{(2)}=U \mbox{diag} \{\sigma_1, (1-2\eta)^2\sigma_2,\dots, (1-2\eta)^2\sigma_k\}V^T. 
	\end{array}
\end{equation}

\noindent Repeat $n$ iterations, we have the expression for $R^{(n)}$: 
\vspace{-0.2cm}
\begin{equation}
	\label{equation:rn}
	R^{(n)}=U \mbox{diag} \{\sigma_1, (1-2\eta)^n\sigma_2,\dots, (1-2\eta)^n\sigma_k\}V^T. %\\=U\Sigma\V^T=R'
\end{equation}

\noindent Since $|1-2\eta|<1$, Equation (\ref{equation:rn}) can be reduced to:
\begin{equation}
	\label{equation:rn_final}
	\begin{aligned}
		\lim_{n \to \infty}R^{(n)}&=U \mbox{diag} \{\sigma_1, 0,\dots, 0\}V^T\\
		&=U\Sigma' V^T=\bar{R}. 
	\end{aligned}
\end{equation}
\vspace{-0.5cm}
\end{proof}
%We can also visualize the first row of $R'$ by reshaping it back to the image size, and we call it ``singular reflectance''. In Section~\ref{sec:discussions}, we have discussions and comparisons about using ``singular reflectance'' versus ``mean reflectance'' and ``median reflectance''.

In our diffuse branch, the low-rank loss of reflectance back-propagates to Normal-Net and Light-Net, and trains both in self-supervised manners. 
\subsubsection{Specularity separation}\label{sec:spec}

%Many real images have strong specular highlights, in which case the Lambertian assumption of the above diffuse inverse rendering pipeline would not hold. 
To deal with the specular highlights, we add a Spec-Net, to remove the highlights before diffuse inverse rendering. On highlight regions, pixels are usually saturated and tends to be white. Based on it, we automatically evaluate the percentage of saturated pixels on the object image. If the percentage exceeds $5\%$, Spec-Net will be performed, otherwise the object is considered as diffuse and Spec-Net will not be performed. We found that under this setting the results are better than performing Spec-Net on all images, since learning-based highlight removal methods tend to overextract highlights on diffuse images.  
%we apply a specularity separation first. In mobile AR apps, it is easy for users to tell whether there are strong specular reflections, where pixels are usually saturated and tends to be white. We adopt an optional Spec-Net as the first step to remove the specular highlights before diffuse inverse rendering. 
The training of Spec-Net is initialized from the highlight removal network of Yi et al. \cite{yi2020leveraging}, enhanced with images of non-Lambertian objects in our Relit Dataset by self-supervised finetuning. From the Di-chromatic reflection model~\cite{shafer1985color}, if illumination colors remain unchanged, the rg-chromaticity of Lambertian reflection should be unchanged as well. Thus the finetuning can be driven by the low-rank constraint on rg-chromaticity of diffuse images after removing specular highlights, following the image formation model in Equation~(\ref{equation:iid}). 

With the separated specular highlight, we can further predict specular reflectance $s_p$ and shininess (smoothness) $\alpha$ in Equation~(\ref{equation:phong}). The training is self-supervised by re-rendering loss between the separated highlight by Spec-Net, and the re-rendered specular highlight by the predicted $s_p$, $\alpha$, lighting coefficients $C_{l,m}$ from Light-Net via the specular rendering layer in Section~\ref{sec:specrender}. 

\subsubsection{Joint training}

%As introduced later in Section~\ref{sec:diffuserender}, 
%$F$ are the resulting matrices of normals timing with every Spherical Harmonic bases. 

Firstly, the Spec-Net is trained to separate input images into specular highlight and diffuse images, as the first phase. Since training to predict specular reflectance and smoothness requires lighting coefficients from Light-Net, Light-Net and Normal-Net in the diffuse branch are trained as the second phase. Training to predict specular reflectance and smoothness is the last phase. 

In the second phase, Light-Net predicts spherical harmonic lighting coefficients $C_{l,m}$ corresponding to each basis $Y_{l,m}$. There is an axis ambiguity between Normal-Net and Light-Net predictions. For example, predicting a normal map with the $x$-axis pointing right with positive coefficients of the bases related to $x$, is equivalent to predicting a normal map with $x$-axis pointing left with corresponding coefficients being negative. They would render the same shading results. Normal-Net and Light-Net are in a chicken-and-egg relation and cannot be tackled simultaneously.
We employ a joint training scheme to train Normal-Net and Light-Net alternatively. To initialize the coordinate system in Normal-Net, we use a small amount of synthetic data (50k images) from LIME~\cite{meka2018lime} to train an initial Normal-Net. Then we freeze Normal-Net and train Light-Net from scratch by our low-rank loss on reflectance, as the $1^{st}$ round joint training. Then Light-Net is frozen and Normal-Net is trained from the initial model by the same low-rank loss on reflectance. The joint training is driven by the Relit dataset, using 750k unlabeled images. Normal-Net is weakly-supervised due to the pretraining and all other nets are self-supervised. 
The joint training scheme effectively avoids the axis ambiguity and the quantitative ablation studies are shown in Section~\ref{exp:inv}. 
%and performs better than training both nets simultaneously. The ablations are shown in Section~\ref{exp:inv}. 

\subsection{Non-Lambertian object relighting}\label{sec:relighting}

After inverse rendering, an input photo is decomposed into normal, lighting, reflectance, shading and a possible specular component by our network. 
With these predicted properties, along with the lighting of new scenes, the object is re-rendered and inserted into new scenes. We propose a specular rendering layer in Section~\ref{sec:specrender}. Given specularity parameters (specular reflectance and smoothness), we can relight the object in a wide range of materials.  

Both diffuse and specular render layers take spherical harmonic coefficients as lighting inputs, which present low-frequency environment lighting. The transformations from HDR lighting paranomas to SH coefficients are pre-computed offline. 
%This pipeline also enables multiple object insertion from different input images; video demos can be found on the project page. 
%and we show a lot of relighting videos in the supplements, as mentioned in Section
We also implement a mobile App, whose details are in Appendix~\ref{sec:supp}. 
%We can also insert and relight multiple objects from different photos into the same scene, and manipulate the lay-outs and sizes through simple dragging, tailored for amateur users. 
%The object is relit in about 0.34 seconds with the chosen lighting environment. 

\subsubsection{Diffuse rendering layer}\label{sec:diffuserender}

In order to encode the shading rendering while keeping the whole network differentiable, we adopt a diffuse rendering layer respecting to Equation (\ref{equation:phong})-(\ref{equation:SH}), based on \cite{ramamoorthi2001efficient}. %The rendering is differentiable efficient, taking 0.29 seconds each. 
%By this way, a image can be represented as:
%\begin{equation}
%I = A\odot f(N,L),
%\label{equation:sh1}
%\end{equation}
%where $A$ is the albedo map and $N$ is the normal map, $L$ is the lighting. Shading can be represented by a function $f$ of $N$ and $L$.
The rendering layer takes the spherical harmonic coefficients as lighting inputs. Combining Equation (\ref{equation:phong})-(\ref{equation:SH}), introducing  coefficients $\hat{A}_l$ from \cite{ramamoorthi2001relationship}, and incorporating normal into the spherical harmonic bases, the shading and the diffuse component of relit images are rendered by:
%\vspace{-0.2cm}
% \scalebox{0.9}{
\begin{equation}
	\label{equation:sh2}
	%\begin{align}
	I_d(p)=\mathcal{A}(p)\sum_{\omega \in \mathcal{L}}l_\omega(L_\omega\cdot n(p))=\mathcal{A}(p)\sum_{l,m}\hat{A}_l C_{l,m}Y_{l,m}(\theta,\phi),
	%S =\sum_{l,m}\hat{A}_l C_{l,m}Y_{l,m}= C \cdot F, \\
	%		I = R \odot (C\cdot F),
	%\end{align}
\end{equation} 


\noindent where $(\theta,\phi)$ is the spherical coordinates where $(x,y,z)=(\sin\theta\cos\phi, \sin\theta\sin\phi,\cos\theta)$ and $n(p)=(x,y,z)$.  

%\noindent where $C$ is the matrix of Spherical Harmonic coefficients, $F$ are the resulting matrices of normals timing with every Spherical Harmonic bases, and $R$ is the diffuse reflectance. Here $\odot$ is Hadamard product and $\cdot$ is matrix product. A common observation of Spherical Harmonic representation for lighting is that coefficients decay fast after order 2. 
%Up to order 2 of Spherical Harmonics are enough to represent most low-frequency lighting. Thus, $F=[1,n_x,n_y,n_z,3n_z^2-1,n_xn_y,n_xn_z,n_yn_z,n_x^2-n_y^2]^T$. 

%\vspace{-0.2cm}
\subsubsection{Specular rendering layer}\label{sec:specrender}
\begin{figure}
	\centering
	\includegraphics[width= \linewidth]{spec_illust.pdf}\
	\caption{The visual illustration of $b$, $L_\omega$ and $v$. (a) The view point $v$ is set as $[0, 0, 1]$ in our inverse rendering problem. $b$ is the bisector of $L_\omega$ and $v$. (b) Observe $b$, $L_\omega$ and $v$ in the $yz$ plane view. We find that the polar angle $\theta_{L_\omega}=2\theta_{b}$. (c) We find that the azimuth angle $\phi_{L_\omega}=\phi_{b}$ from the $xy$ plane view. (d) Then we get spherical harmonic basis $Y_{l,m}(\theta_{b},\phi_{b})$ for differentiable rendering of the specular component.}
	\label{fig:spec}
	%\vspace{-0.5cm}
\end{figure}
Since the specular componet is view dependent, which can not be simply parameterized with $Y_{l,m}$ and $C_{l,m}$ as in the diffuse renderer. 
With the assumption of distant lighting, the view point is fixed. As shown in Figure~\ref{fig:spec}, $b = \frac{L_\omega+v}{\|L_\omega+v\|}$ is the bisector of light direction $L_\omega$ and view point $v$. Note that, $b$ has the same azimuth angle $\phi$ as $L_\omega$ while polar angle $\theta$ is only a half under spherical coordinate system as shown in Figure~\ref{fig:spec}. 
Since the predicted normal map has a pixel-to-pixel correspondence to the input image, which means the normal map is projected perspectively. We only need to apply orthogonal projection in the re-rendering step by assuming viewing the object the $z$ direction, which means $v = [0, 0, 1]$. The re-rendered images share a pixel-wise correspondence to observed images, following a perspective projection. 

Now we can modify $Y_{l,m}$ into $\hat{Y}_{l,m}$, and use $\hat{A}_l\hat{Y}_{l,m}$ to describe the distribution of all possible $b$ as well, keeping lighting coefficients $C_{l,m}$ unchanged for sharing between both renderers. %Note that $Y_{l,m}$ are the cartesian components of $(x, y, z)$.
%\begin{equation}
\begin{align}
	%(x,y,z)&=(\sin\theta\cos\phi, \sin\theta\sin\phi,\cos\theta)\nonumber\\
	\hat{Y}_{0,0}(\theta, \phi)&=Y_{0,0}(2\theta, \phi) = c_0\nonumber\\
	\hat{Y}_{1,1}(\theta, \phi)&= Y_{1,1}(2\theta, \phi) = c_1\sin 2\theta\cos\phi = 2c_1xz\nonumber\\
	\hat{Y}_{1,-1}(\theta, \phi)&=Y_{1,-1}(2\theta, \phi) = c_1\sin 2\theta\sin\phi = 2c_1yz\nonumber\\
	\hat{Y}_{1,0}(\theta, \phi)&=Y_{1,0}(2\theta, \phi) = c_1\cos 2\theta = c_1(2z^2-1)\nonumber\\
	\hat{Y}_{2,-2}(\theta, \phi) &=Y_{2,-2}(2\theta, \phi)= 4c_2xyz^2\nonumber\\
	\hat{Y}_{2,1}(\theta, \phi) &=Y_{2,1}(2\theta, \phi)= c_2(4xz^3-2xz)\nonumber\\
	% \end{align}
% \begin{align}
	\hat{Y}_{2,-1}(\theta, \phi) &=Y_{2,-1}(2\theta, \phi)= c_2(4yz^3-2yz)\nonumber\\
	\hat{Y}_{2,0}(\theta, \phi) &=Y_{2,0}(2\theta, \phi)= c_3(3(4z^4-4z^2+1)-1)\nonumber\\
	\hat{Y}_{2,2}(\theta, \phi) &=Y_{2,2}(2\theta, \phi) = c_4(4x^2z^2-4y^2z^2)\nonumber\\
	c_0&=0.282095,c_1=0.488603\nonumber
	\\
	c_2=1.092&548,c_3=0.315392,c_5=0.546274\nonumber
\end{align}
%\end{equation}

%Hence, with $\mathbf{n}_p^t=(x,y,z,1)$, we can write the differentiable rendering approximation for the specular component as:
Hence, we can write the differentiable rendering approximation for the specular component similar as Equation~(\ref{equation:sh2}):
\begin{equation}
	\begin{aligned}
		H(p)&=s_p\sum_{\omega \in \mathcal{L}}l_\omega(\frac{L_\omega+v}{\|L_\omega+v\|}\cdot n(p))^\alpha \\
		&\approx s_p\sum_{l,m} C_{l,m}(\hat{A}_l\hat{Y}_{l,m}(\theta,\phi))^\alpha.
	\end{aligned}
\end{equation}




%\noindent where $M$ is a symmetric 4x4 matrix. Each color has an independent matrix $M$. The matrix $M$ can be obtained similar as \cite{ramamoorthi2001efficient}. The detailed form of $M$ can be found in the supplementary material. 




%\subsection{Image relighting}\label{sec:relighting}

%\renjiao{describe the relighting process, relighting multiple images, video demo, android app}

%Demo videos and implementation details are in the supplements.
%We can also rotate the lighting environment to generate a relit video. With a sampling rate of 36 samples in 360 degrees, a relit video is generated in about 10 seconds. 
%In detail, we convert the network models to Pytorch Mobile and package them inside the application as assets.
%For object photos captured from camera, an on-device GrabCut in OpenCV is applied to obtain the object mask. For photos loading from memory, the object mask is also required as an additional input.
%The application is implemented in Java, using \renjiao{android gradle plugin of version 3.5.0 with gradle dependencies androidx.appcompat:appcompat:1.1.0, org.pytorch:pytorch_android:1.8.0 and org.pytorch:pytorch_android_torchvision:1.8.0.} 


\section {Experiments}


%\renjiao{remember to include training details in the supplements}
%We first introduce the details of our Relit Dataset in Section \ref{sec:dataset}.
In this section, we evaluate the performance of inverse rendering and image relighting. The inverse rendering evaluation with a series of state-of-the-art methods
%, quantitatively and qualitatively, 
is presented in Section~\ref{exp:inv}, along with several ablations. 
For image relighting, we provide quantitative evaluations on a synthetic dataset in Section~\ref{exp:relighting}, and real object insertion is demonstrated in Figure~\ref{fig:teaser} and the project page. 
%More evaluations, and discussions are in the Supplementary. %From the experiments, the proposed method outperforms the state-of-the-arts methods, benefiting from the large-scale real-image training data, driven by low-rank losses. Unlike synthetic data, real-image data has no domain gap between training and testing sets, leading a better performance on unseen natural images. 
%Codes of the proposed method are provided in the supplementary materials. 
%we demonstrate AR effects of object insertion and compare with a state-of-the-art method and naive insertion in Section~\ref{exp:relighting}. 
%More relighting video demos and Android App demo are in the supplements. 
%Additional discussions of technical details in Section~\ref{sec:discussions}. 
%Ablations are performed to show the effectiveness of our method design. 
%Additional discussions about the proposed method, implementation details, more ablations, additional results, video and mobile App demos and codes are in the supplementary materials.

\subsection{Inverse rendering}\label{exp:inv}
Many prior works address surface normal estimation or intrinsic image decomposition but not both, 
and there are no benchmark datasets for inverse rendering, 
we evaluate these two tasks individually. Evaluations on lighting and specularity are in Appendix~\ref{sec:supp}. %We also compare end-to-end inverse rendering results qualitatively on natural images. %Evaluations of specular highlight separation are included in the supplements. 
The end-to-end inverse rendering takes 0.15 seconds per image at $256\times 256$ on a Titan T4 GPU. 


\noindent{\bf{Intrinsic image decomposition.}} We compare our self-supervised intrinsic image decomposition to several inverse rendering methods (InverseRenderNet~\cite{yu2019inverserendernet}, RelightNet~\cite{Lichy_2021_CVPR}, ShapeAndMaterial~\cite{Lichy_2021_CVPR}), and intrinsic image decomposition methods~\cite{shi2017learning,yi2020leveraging,liu2020unsupervised,li2018cgintrinsics} on MIT Intrinsics dataset, which is a commonly-used benchmark dataset for IID. 
%Unlike the comparisons in some previous papers, where all methods are finetuned on one half of the dataset and tested on the other half. 
To evaluate the performances and cross-dataset generalizations, all methods are not finetuned on this dataset. 
We adopt scale-invariant MSE (SMSE) and local scale-invariant MSE (LMSE) as error metrics, which are designed for this dataset~\cite{grosse2009ground}. 
%The errors are computed on shadings. 
As shown in Table~\ref{table:MIT} (visual comparisons are in the supplementary material), our method outperforms all unsupervised and self-supervised methods and has comparable performance with supervised ones. 
Note that the assumptions of white illumination and Lambertian surfaces in this dataset fit the cases of synthetic data, which benefit supervised methods. However, self-supervised and unsupervised methods enable training on unlabeled real-image datasets, which produce better visual results on unseen natural images.
As shown in Figure~\ref{fig:endtoend}, 
SIRFS~\cite{barron2015shape}, a method based on scene priors, fails to decompose reflectance colors. InverseRenderNet~\cite{yu2019inverserendernet} and RelightingNet~\cite{yu2020self} tend to predict a similar color of shading and reflectance, leading to unnatural reflectance colors.  ShapeAndMaterial~\cite{Lichy_2021_CVPR} generates visually good results but has artifacts on reflectance due to specular highlights. Our method decomposes these components by considering non-Lambertian cases. 
%Reflectance also implies our Light-Net results, since it is computed from the diffuse image, normal and lighting coefficients. More visual results are in the Supplementary. 

\begin{figure*}
	\centering
	\includegraphics[width=\linewidth]{inv_new2.pdf}\
	\caption{Qualitative comparisons on an unseen image, comparing with state-of-the-art methods. The first row shows the reflectance of all methods and specular highlights of our method. The second row shows estimated normal maps and the colormap for reference. }
	\label{fig:endtoend}
\end{figure*}

\begin{figure*}
	\centering
	\includegraphics[width=\linewidth]{relight_comp1.pdf}\
	\caption{Comparisons of object relighting with RelightNet~\cite{yu2020self} and ground truths. }
	\label{fig:relighting}
\end{figure*}


\begin{table}[t]
	\caption{Quantitative comparisons with state-of-the-art alternatives and ablation study of  intrinsic image decomposition on MIT intrinsic dataset. }\label{table:MIT}
	%\begin{minipage}[t]{.6\linewidth}
	%\caption{Quantitative evaluation of intrinsic image decomposition on MIT intrinsic dataset. The ablations are shown in the last four rows.}
	\vspace{0pt}
	\centering
	\scalebox{0.8}{
		\begin{tabular}{c c c c c } 
			\hline 
			Methods&Supervision&Data type&SMSE&LMSE\\
			\hline
			Shi et al.~\cite{shi2017learning}&Sup.& Synthetic&0.0194&0.0318\\
			Li et al.~\cite{li2018cgintrinsics}&Sup.&Synthetic&0.0186&\bf{0.0259}\\
			Shape\&Material~\cite{Lichy_2021_CVPR}&Sup.&Synthetic&\bf{0.0150}&0.0309\\
			\hline
			RelightingNet~\cite{yu2020self}&Self-sup.&Real&0.0368&0.1077\\
			Yi et al.~\cite{yi2020leveraging}&Unsup.&Real&0.0231&0.0422\\
			InverseRenderNet~\cite{yu2019inverserendernet}&Self-sup.&Real&0.0299&0.0855\\
			Liu et al.~\cite{liu2020unsupervised}&Unsup.&Real&0.0193&0.0428\\
			Ours&Self-sup.&Real&\bf{0.0186}&\bf{0.0369}\\
			\hline
			$1^{st}$ round training&Self-sup.&Real&0.0224&00420\\
			%Yi\cite{yi2020leveraging}&0.0274&0.0476\\
			w/o joint training&Self-sup.&Real&0.0216&0.0399\\
			loss$^+$ ($\sigma_2$)&Self-sup.&Real&0.0357&0.0513\\
			loss* ($\sigma_2/\sigma_1$)&Self-sup.&Real&0.0808&0.2137\\
			\hline
	\end{tabular}}
	%\end{minipage}%
	%\begin{minipage}[t]{.4\linewidth}
\end{table}
\begin{table}[t]
	\caption{Quantitative comparisons with state-of-the-art alternatives and ablation study of surface normal estimation on the dataset from Janner et al.\cite{janner2017self}.}\label{table:janner}
	\centering
	\scalebox{0.8}{
		%\caption{Quantitative evaluation and ablations of surface normal estimation on dataset from Janner et al.\cite{janner2017self}. The ablations are in the last four rows.}
		\begin{tabular}{c c c } 
			\hline 
			Methods&MSE&DSSIM\\
			\hline
			SIRFS~\cite{barron2013intrinsic}&0.0230&0.0243\\
			SVBRDF~\cite{li2018learning2}&0.0144&0.0278\\
			InverseRenderNet~\cite{yu2019inverserendernet}&0.0084&0.0272\\
			RelightNet~\cite{yu2020self}&0.0080&0.0265\\
			ShapeAndMaterial~\cite{Lichy_2021_CVPR}&0.0060&0.0228\\
			Ours&\bf{0.0054}&\bf{0.0201}\\
			%Yi\cite{yi2020leveraging}&0.0274&0.0476\\
			\hline
			$1^{st}$ round training&0.0061&00219\\
			w/o joint training&0.0065&0.0228\\ 
			loss$^+$ ($\sigma_2$)&0.0059&0.0213\\
			loss* ($\sigma_2/\sigma_1$)&0.0083&0.0309\\
			\hline
			
	\end{tabular}}
	%\end{minipage} 
	%\vspace{-15pt}
\end{table}




% \begin{table}[t]
	% 	\centering 
	% 	\scalebox{0.7}{
		% 		\begin{tabular}{c c c c c } 
			% 			\hline 
			% 			Methods&Supervision&Data type&SMSE&LMSE\\
			% 			\hline
			% 			Shi et al.~\cite{shi2017learning}&Sup.& Synthetic&0.0194&0.0318\\
			% 			Li et al.~\cite{li2018cgintrinsics}&Sup.&Synthetic&0.0186&\bf{0.0259}\\
			% 			Shape\&Material&Sup.&Synthetic&\bf{0.0150}&0.0309\\
			% 			\hline
			% 			RelightingNet&Self-sup.&Real&0.0368&0.1077\\
			% 			Yi et al.~\cite{yi2020leveraging}&Unsup.&Real&0.0231&0.0422\\
			% 			InverseRenderNet&Unsup.&Real&0.0299&0.0855\\
			% 			Liu et al.~\cite{liu2020unsupervised}&Unsup.&Real&0.0193&0.0428\\
			% 			Ours&Unsup.&Real&\bf{0.0186}&\bf{0.0369}\\
			% 			\hline
			% 			$1^{st}$ round training&Unsup.&Real&0.0224&00420\\
			% 			%Yi\cite{yi2020leveraging}&0.0274&0.0476\\
			% 			w/o joint training&Unsup.&Real&0.0216&0.0399\\
			% 			loss$^+$ ($\sigma_2$)&Unsup.&Real&0.0357&0.0513\\
			% 			loss* ($\sigma_2/\sigma_1$)&Unsup.&Real&0.0808&0.2137\\
			% 			\hline
			% 		\end{tabular}
		% 	}
	% 	\caption{Quantitative evaluation of intrinsic image decomposition on MIT intrinsic dataset. 
		% 		%Shi et al., Li et al., ShapeAndMaterial, RelightNet, Yi et al., InverseRenderNet and USI$^3$D denote \cite{shi2017learning, li2018cgintrinsics, Lichy_2021_CVPR, yu2020self, yi2020leveraging, yu2019inverserendernet,liu2020unsupervised} respectively. 
		% 		The ablations are shown in the last four rows.  }\label{table:MIT} 
	% \end{table}

% \begin{table}[t]
	% 	\centering 
	% 	\scalebox{0.9}{
		% 		\begin{tabular}{c c c } 
			% 			\hline 
			% 			Methods&MSE&DSSIM\\
			% 			\hline
			% 			SIRFS~\cite{barron2013intrinsic}&0.0230&0.0243\\
			% 			SVBRDF~\cite{li2018learning2}&0.0144&0.0278\\
			% 			InverseRenderNet~\cite{yu2019inverserendernet}&0.0084&0.0272\\
			% 			RelightNet~\cite{yu2020self}&0.0080&0.0265\\
			% 			ShapeAndMaterial~\cite{Lichy_2021_CVPR}&0.0060&0.0228\\
			% 			Ours&\bf{0.0054}&\bf{0.0201}\\
			% 			%Yi\cite{yi2020leveraging}&0.0274&0.0476\\
			% 			\hline
			% 			$1^{st}$ round training&0.0061&00219\\
			% 			w/o joint training&0.0065&0.0228\\ 
			% 			loss$^+$ ($\sigma_2$)&0.0059&0.0213\\
			% 			loss* ($\sigma_2/\sigma_1$)&0.0083&0.0309\\
			% 			\hline
			
			% 		\end{tabular}
		% 	}
	% 	\caption{Quantitative evaluation and ablations of surface normal estimation on dataset from Janner et al.\cite{janner2017self}. The ablations are in the last four rows. }\label{table:Normal} 
	% \end{table}
\begin{comment}
	
	\begin{figure*}
		\centering
		\includegraphics[width=1\linewidth]{figures/normal_janner_6data.pdf}\
		\caption{Normal estimation comparisons with SIRFS\cite{barron2013intrinsic}, SVBRDF\cite{li2018learning2}, InverseRenderNet\cite{yu2019inverserendernet}, RelightNet\cite{yu2020self}and ShapeAndMaterial\cite{Lichy_2021_CVPR} on selected data from Janner et al.~\cite{janner2017self}. The reference color map can be found in Figure~\ref{fig:endtoend}. }
		\label{fig:normal}
	\end{figure*}
	content...
\end{comment}
\begin{comment}
	\begin{figure}
		\centering
		\includegraphics[width=\linewidth]{figure/normal_janner_2data_single2.pdf}\
		\caption{Visualizations of results on synthetic data from Janner et al. \shortcite{janner2017self}. }
		\label{fig:normal}
	\end{figure}
\end{comment}

\noindent{\bf{Normal estimation.}} We compare our method with several inverse rendering methods~\cite{barron2015shape,li2018learning2,yu2019inverserendernet,yu2020self,Lichy_2021_CVPR} on synthetic dataset from Janner et al.~\cite{janner2017self}. Since the dataset is too large (95k), and SIRFS~\cite{barron2015shape} takes one minute for each data, a testing set of 500 images is uniformly sampled, covering a wide variety of shapes. In Table~\ref{table:janner}, the evaluations are reported with two error metrics, MSE and DSSIM, measuring pixel-wise and overall structural distances. Our method yields the best performance. 
%Qualitative comparisons are included in the supplements due to size limits. 
Qualitative comparisons are shown in Appendix~\ref{sec:supp}. %Comparisons on natural images are shown in Figure~\ref{fig:endtoend}, as discussed in last paragraph. 


%\subsection{Evaluation of Spec-Net}

% \noindent{\bf{Lighting and Specular highlight separation.}} 
% Evaluations of lighting and specular highlights are in the supplement. 
\begin{comment}
	To evaluate the performance of specular highlight extraction of Spec-Net, we compare with several prior methods on a real-image dataset from \cite{yi2020leveraging}. As shown in Table~\ref{table:specnet}, Spec-Net outperforms other methods in both SMSE and DSSIM. In the Supplementary, we show qualitative comparisons on this evaluation set. On real images where highlights are strong, and highlight regions are saturated, most methods tend to over-extract specular highlights, while the Spec-Net performs well due to the training on a large scale of real images. 
	
	
	\begin{table}[t]
		\caption{Quantitative evaluation of specular highlight separation on real images. 
		}
		\centering 
		\scalebox{0.8}{
			\begin{tabular}{c |c c c c} 
				\hline 
				&\cite{shen2013real}&\cite{shi2017learning}&\cite{yamamoto2019general}&Ours\\
				\hline
				MSE&0.0334&0.0305&0.0334&\textbf{0.0148}\\
				DSSIM&0.1745&0.2087&0.1743&\textbf{0.1500}\\
				\hline
			\end{tabular}
		}
		\label{table:specnet} 
	\end{table}
\end{comment}
\begin{comment}
	
	\begin{figure}
		\centering
		\includegraphics[width=1\linewidth]{figure/specnet.pdf}\
		\caption{Qualitative comparisons on four data from real-image specularity separation dataset from \cite{yi2020leveraging}, captured by cross-polarization. From left to right, there are input images and diffuse components after removing specular highlights by different methods. }
		\label{fig:specnet}
	\end{figure}
	
	content...
\end{comment}


{\noindent\bf{Ablations.}} %To evaluate the effectiveness of our unsupervised joint training scheme and loss definition. 
We present ablations in the last four rows in Table~\ref{table:MIT}-\ref{table:janner}.  $1^{st}$ round training denotes the networks of initial Normal-Net and self-supervised Light-Net. ``w/o joint training'' denotes training Normal-Net and Light-Net simultaneously, rather than alternatively. Previous works propose different formulations of low-rank loss, as the second singular value ($\sigma_2$)~\cite{yi2018faces,zhu2020adacoseg} or the second singular value normalized by the first one ($\frac{\sigma_2}{\sigma_1}$)~\cite{yi2020leveraging} to enforce a matrix to be rank one. 
As discussed in the original papers, these losses are unstable in training and would degenerate to local optima. 
The proposed low-rank constraint is more robust as proven, not suffering from local optimas. More discussions and visual comparisons of these low-rank losses are in Appendix~\ref{sec:supp}. 
%They may converge to different results since there are more than one local minima (e.g. predicting reflectance maps as all-zeros or all-whites are both rank one). 
%Loss$^+$ and loss* denote replacing the low rank loss by those in~\cite{yi2018faces} and~\cite{yi2020leveraging}. 
%Loss$^+$ and loss* both suffer from degenerating cases, as mentioned in original papers. 
%Table~\ref{table:MIT} %-\ref{table:Normal} shows that loss* performs worse than initial Normal-Net due to degeneration during training. Loss$^+$ is effective with a hand-picked learning rate. Ours yields the best performance on IID and normal estimation among all ablations. %Training Normal-Net and Light-Net simultaneously suffer from ambiguities between normal and lighting. 
%The visual comparisons are shown in the supplements, from which we can see the final model generates better visual decompositions than ablations.  

%The comparison is similar for unseen natural images, after unsupervised training, the albedo/reflectance predictions are smoother and more flattened, and the normal predictions have more details (see the data ``bread'' in Figure~\ref{fig:endtoend}, and correct some imprecise predictions (see the reflectance of the data ``garlic'', the right part of the normal prediction in data ``brick''). 

%From the last row in Table~\ref{table:MIT}-\ref{table:Normal}, where the Normal-Net and Light-Net are trained unsupervisedly together, the performance of normal estimation is getting worse than the pretrained performance (see the values in the last second row, which are the values of the first round training where Light-Net is trained and Normal-Net is fixed), it is because the ambiguities between normal and lighting. In detail, the low rank constraint is defined on the albedos, and the gradients are propagated to shadings. However, to generate a target shading, there are infinity combinations of normal and lighting. Thus we need the joint finetuning scheme to make sure the training is converged to the optimas we wanted. \renjiao{add discussion about reflectance according to the figure}

\begin{comment}
	
	\begin{figure}
		\centering
		\includegraphics[width=1\linewidth]{figure/12.pdf}\
		\caption{Realistic object insertion comparing with naive insertion. }
		\label{fig:relit}
	\end{figure}
	
\end{comment}

\begin{table}[t]
	\caption{Quantitative evaluation on relighting. % with and without specularity. %RelightNet~\cite{yu2020self} can only provide diffuse relighting. Baseline* denotes naive insertion without relighting as demonstrated in the supplementary video.  
	}\label{table:relighting} 
	\centering 
	\scalebox{0.7}{
		\begin{tabular}{c |c c| c c |c c} 
			\hline 
			&\multicolumn{2}{c|}{Baseline*}&\multicolumn{2}{c|}{RelightNet}&\multicolumn{2}{c}{Ours}\\
			&MSE&DSSIM&MSE&DSSIM&MSE&DSSIM\\
			\hline
			Diffuse &0.2210&0.1350&0.1144&0.0788&\textbf{0.0926}&\textbf{0.0616}\\
			With specularity&0.2152&0.1272&-&-&\textbf{0.0876}&\textbf{0.0720}\\
			\hline
		\end{tabular}
	}
	\vspace{-0.5cm}
\end{table}

\subsection{Image relighting}\label{exp:relighting}


%The transformation from HDR environment lighting to Spherical Harmonics coefficients is pre-computed offline, and is not included in the network since it is not related to the objects. 
After inverse rendering, a differentiable non-Lambertian renderer is used to relight the object under new lighting. 
The rendering is efficient, taking 0.35 seconds per image at $256\times 256$ on a single Tesla T4 GPU. 
For quantitative evaluations, we rendered an evaluation set of 100 objects under 30 lighting environments, with various materials. 
%The evaluation set have 6000 images. 
For each object, we use one image under one lighting as input, and relight it under the other 29 lighting for evaluation. We compare our method with a state-of-the-art method RelightNet~\cite{yu2020self}, which only provides diffuse relighting. To be fair, we compare them on diffuse relighting only. Ours is evaluated for both diffuse and non-Lambertian relighting. %The results are in Table~\ref{table:relighting}. 
Comparisons are shown in Figure~\ref{fig:relighting} and Table~\ref{table:relighting}, more in Appendix~\ref{sec:supp}. Baseline* in the table denotes naive insertions without relighting.
%We also demonstrate our object insertion results on natural images in the teaser and the video (compared with naive insertion). 
% Naive insertion directly places the object under original lighting into the scene, as many AR applications do, leading to inconsistencies with the target scene.  %RelightNet~\cite{yu2020self} is trained on buildings under outdoor lighting, having a domain gap to uncontrolled lighting here. As in Figure~\ref{fig:endtoend}, they have difficulties decomposing reflectance and shading and predict a normal map that is not smoothly changed, leading to unrealistic relighting results. 
Object insertion and App demos are on the project page, where our method relights and inserts objects into new scenes realistically. %The supplementary materials provide additional video demos to show the smoothly-changing relighting results while the lighting environment is rotating. 
%and interesting AR effects such as material editing. 
%For the mobile App, we provide video demos of users using the App to create vivid object insertion results, on the project page. 
%object insertion, we provide a video demo of the AR App we designed in Android mobile systems. 


%We also provide a demo of the Android App to show how users can easily take object photos, and automatically relight them into new scenes. When relighting multiple objects from different input images into the same scene, we can further manipulate the lay-out by simply dragging the objects on the screen. 
\begin{comment}
	
	\subsection{More discussions}
	
	\noindent{\bf{Multi-view stereo as normal supervision.}} Previous method \cite{yu2019inverserendernet} uses multi-view stereo to reconstruct normal maps on outdoor building images in MegaDepth dataset\cite{li2018megadepth}, where ground truth depth maps are also available. Features on outdoor buildings are rich, which are suitable for multi-view stereo to reconstruct. 
	
	For object images, we explored similar approaches and found it not working for our scenarios. we use a reconstruction pipeline of adopting VisualSFM\cite{wu2011visualsfm} to reconstruct sparse point clouds, then PMVS2\cite{furukawa2010accurate} to further reconstruct dense point clouds. Applying the pipeline needs multi-view images as inputs, which would introduce a heavy workload for capturing multi-view images for all objects. For demonstration, we capture additional multi-view images and test the pipeline on several objects. For each object, we capture about 50 multi-view images as inputs. From the results, we find the point clouds are very sparse due to lack of features. A example is shown in Figure~\ref{fig:sfm}, textureless regions are quite common on natural objects, where the features are sparse, and reconstruction results have many holes on the resulting dense point clouds. For some other object, due to the lack of features, VisualSFM even fails to reconstruct a initial point cloud. Thus, adopting SFM and MVS to reconstruct geometry is not an option for our cases. 
	
	
	\begin{figure}
		\centering
		\includegraphics[width=1\linewidth]{figure/sfm.pdf}\
		\caption{Object reconstructed by VisualSFM and PMVS2. Selected multi-view image inputs are shown on the left and reconstructed dense point clouds are on the right. }
		\label{fig:sfm}
	\end{figure}
	
	\begin{figure}
		\centering
		\includegraphics[width=1\linewidth]{figure/singular_white.pdf}\
		\caption{On each row, selected images from one batch are shown at the left. Corresponding mean image, median image and singular image are at the right.  }
		\label{fig:meanimage}
	\end{figure}
	
	
	\noindent{\bf{Using median or mean reflectance vs. the singular reflectance.}} One may wonder whether using median or mean images of reflectance predictions in one batch will have similar results with our low rank constraint. Firstly, losses between median or mean reflectance of one batch and predicted reflectance are not scale-invariant. Secondly, median image is not differentiable. Thirdly, we perform a large amount of testing on our Relit dataset and found that singular reflectance is more robust to shadows, intensity saturations and uneven lighting, which are common cases in natural images. Some visual comparisons are shown in Figure~\ref{fig:meanimage}, we can see that mean image may generates incorrect reflectance in some regions due to above reasons while dominant singular reflectance generates much more reasonable reflectance maps. It is because SVD solves the dominant direction of reflectance maps, better than naive averaging. Note that we show cases on input images in Figure~\ref{fig:meanimage} because that at the beginning of joint training, the network initializes from predicting reflectance same as input images. We can see that using singular reflectance is much better visually, with convergence proven. \\
\end{comment}
\begin{comment}
	\subsection{More discussions}\label{sec:discussions}
	
	\noindent{\bf{Comparisons to other low rank losses. }}
	Other than the quantitative ablations in Section~\ref{exp:inv}, we also evaluate the robustness of our low rank loss with losses from \cite{yi2018faces} and \cite{yi2020leveraging}. Previous low rank losses have more than one local optima as mentioned in \cite{yi2020leveraging}. Thus they have to use a pretraining phase to initialize the training, and the learning rates are hand-picked to make sure the final models converge to the local optima near the pretraining results. In Table~\ref{table:lowrankloss}, we found it is not easy to find a suitable learning rate. For loss$^+$ in the table, a learning rate smaller than $10^{-8}$ would work. For loss*, we test learning rates from $10^{-2}$ to $10^{-8}$, and all cases degenerate to predict all-white or all-zero shadings. Setting a small learning rate also makes the training time much longer. Our loss has only one global and local optima, and it is promised to converge, and it does not suffer from degenerating. 
	
	\begin{table}[t]
		\centering 
		\scalebox{0.9}{
			\begin{tabular}{c| c c c c} 
				\hline 
				&$10^{-2}$&$10^{-4}$&$10^{-6}$&$10^{-8}$\\
				\hline
				loss$^+$ ($\sigma_2$)&\XSolidBrush&\XSolidBrush&\XSolidBrush&\Checkmark\\
				
				loss* ($\sigma_2/\sigma_1$)&\XSolidBrush&\XSolidBrush&\XSolidBrush&\XSolidBrush\\
				Ours&\Checkmark&\Checkmark&\Checkmark&\Checkmark\\
				\hline
			\end{tabular}
		}
		\caption{The robustness of different loss formulations. \XSolidBrush means the training degenerates to generating an invalid shading and \Checkmark means the training are converging.  }\label{table:lowrankloss} 
	\end{table}
	
	More discussions about using multi-view stereo to supervise normal, and replacing singular reflectance by mean or median images, are included in the supplements. 
\end{comment}
\begin{comment}
	\subsection{Limitations}
	
	The proposed method enables lighting and material editing of a single object image. There are still limitations that can be further explored in the future. One limitation is that cast shadows are not considered in relighting, which sometimes plays an essential part while inserting objects into new scenes realistically. In the future, we plan to detect a target plane to cast the object shadows on, which should be able to further improve object insertion effects. 
	
	
	\renjiao{the following discussions move the rest to supp?}
	
	
	\noindent{\bf{Multi-view stereo as normal supervision.}} Here we discuss the potential of using multi-view stereo to supervise Normal-Net, as in InverseRenderNet\cite{yu2019inverserendernet}. InverseRenderNet is trained on MegaDepth dataset\cite{li2018megadepth}, which are mostly outdoor building images, and the ground truth depth maps are also available. Features on outdoor buildings are rich, which are suitable for multi-view stereo to reconstruct. For object images, we also explore similar approaches for object images. we use a reconstruction pipeline of using VisualSFM\cite{wu2011visualsfm} to reconstruct sparse point clouds, then PMVS2\cite{furukawa2010accurate} to further reconstruct dense point clouds. Applying the pipeline needs multi-view images as inputs, which would introduce a heavy workload for capturing multi-view images for all objects. Here for testing, we test the pipelien on several objects. For each object, we capture about 50 multi-view images as inputs, and we found many issues. A example is shown in Figure~\ref{fig:sfm}, textureless regions are quite common on natural objects, where the features are sparse, and reconstruction results have many holes on the resulting dense point clouds. For some other object, due to the lack of features, VisualSFM even fails to reconstruct a initial point cloud. 
	
	\begin{figure}
		\centering
		\includegraphics[width=1\linewidth]{sfm.pdf}\
		\caption{Object reconstructed by VisualSFM and PMVS2. Selected multi-view image inputs are shown on the left and reconstructed dense point clouds are on the right. }
		\label{fig:sfm}
	\end{figure}
	
	\begin{figure}
		\centering
		\includegraphics[width=1\linewidth]{singular_white.pdf}\
		\caption{On each row, selected images from one batch are shown one the left. The mean image, median image and singular image are on the right. At the beginning of joint training, the network initailzes from predicting reflectance same as input images. }
		\label{fig:meanimage}
	\end{figure}
	
	\noindent{\bf{Using median or mean reflectance vs. the singular reflectance.}} One may wonder whether using median or mean images of reflectance predictions in one batch have similar results with our low rank constraint. Firstly, losses between median or mean reflectance of one batch and predicted reflectance are not scale-invariant. Secondly, median image is not differentiable. Thirdly, we perform a large amount of testing on our Relit dataset and found that singular reflectance is more robust to shadows, intensity saturations and uneven lightings in one batch. Some visual comparisons are shown in Figure~\ref{fig:meanimage}, we can see that mean image may generates incorrect reflectance in some regions due to above reasons while dominant singular reflectance generates piecewise reflectance maps. It is because singular decomposition solves the dominant direction of reflectance maps, better than naive averaging. 
\end{comment}
\vspace{-0.2cm}
\section{Conclusions}

We present a single-image relighting approach based on weakly-supervised inverse rendering, driven by a large foreground-aligned video dataset and a low-rank constraint. %Convergence of the loss is proven mathematically. 
We propose the differentiable specular renderer for low-frequency non-Lambertian rendering. Limitations including shadows and parametric models are discussed in Appendix~\ref{sec:supp}. %Image and video demos show realistic AR effects of object insertions by the proposed approach. 
%An Android app is implemented, for amateur users to capture data and get the realistic AR affects. 
%\vspace{-0.3cm}
\paragraph{Acknowledgements. }  
We thank Kun Xu for the helpful discussions. We thank Sisi Dai and Kunliang Xie for data capturing. This work is supported in part by the National Key Research and Development Program of China (2018AAA0102200), NSFC (62002375, 62002376, 62132021), Natural Science Foundation of Hunan Province of China (2021JJ40696, 2021RC3071, 2022RC1104) and NUDT Research Grants (ZK22-52).
\begin{comment}
	
	\textbf{Limitations. }
	%There are several limitations, as well as future directions of the proposed method. 
	One limitation is that, cast shadows (visibility) are not considered, which can further narrow the gap between relighting results and reality. 
	%sometimes plays an essential role while inserting objects into new scenes realistically. 
	%In the future, we plan to embed cast shadow rendering into single image relighting. 
	Furthermore, parametric models such as Blinn-Phong and Phong are difficult to model semitransparent and transparent materials, which are also common in real scenarios. Spherical harmonics are also limited to model high-frequency lighting components. We plan to explore these directions in the future. 
	%We are exploring region-adaptive Spherical Harmonics models to fit different level of Harmonics to different environment regions. 
	%detect a target plane to cast the object shadows on, which should be able to further improve object insertion effects. %In the future, we plan to add more splits into Relit dataset, such as multi-view videos under static lighting, facilitating many multi-view vision tasks. 
	
\end{comment}
%%%%%%%%% REFERENCES
{\small
	\bibliographystyle{ieee_fullname}
	\bibliography{egbib}
}
\clearpage
\begin{appendices}
	
	\section{Supplementary material}\label{sec:supp}
	In this appendix, we introduce additional experiments, discussions, details of relighting video demos, Android app implementation, the Relit dataset, as well as network and training details. 
	
	\subsection{Mathematical proofs of Theorem~1}
	\begin{theorem}\label{theorem:rankone}
		\textbf{Optimal rank-one approximation.} By SVD,  $R = U\Sigma V^T$,  $\Sigma=diag(\sigma_1,\sigma_2,...\sigma_k)$, $\Sigma'=diag(\sigma_1,0,...)$, $\bar{R} = U\Sigma'V^T$ is the optimal rank-one approximation for R, which meets:
		
		\begin{equation}
			\label{equation:optimal}
			||\bar{R}-R||^2_F = \min_{\scriptscriptstyle b\in\mathcal{R}^{N},c\in\mathcal{R}^{d}}||bc^T-R||^2_F,
		\end{equation}
		
		\noindent where $||\cdot||_F$ denotes the Frobenius norm of a matrix. %The proof of $\bar{R}$ being the optimal rank 1 approximation of $R$ is in the supplements. 
		
		
	\end{theorem}
	
	\begin{proof}
		The objective in (\ref{equation:optimal}) can be written as following:
		
		\begin{equation}
			\label{equation:optimal_sigma}
			||bc^T-R||^2_F = \sum_{i=1}^{d}||c_i\cdot b-r_i||^2_2. 
		\end{equation}
		
		\noindent To minimize $||c_i\cdot b-r_i||^2_2$ while $b$ is a fixed unit vector, $c_i\cdot b$ should be the projection of $r_i$ onto $b$ ($r_i$ is the $i^{th}$ column of $R$). It is equivalent to $c=b^TR$. Then we reduce the optimization problem (\ref{equation:optimal}) as:
		
		\begin{equation}
			\label{equation:reduced}
			\min_{\scriptscriptstyle b\in\mathcal{R}^{N},||b||_2=1}||bb^TR-R||^2_F.
		\end{equation}
		
		\noindent Since $b$ is a unit vector and  V are orthonormal, we can rewrite $||bb^TR||^2_F$ as:
		
		\begin{equation}
			\label{equation:svd_bbt}
			\begin{aligned}
				||bb^TR||^2_F&=||b^TR||^2_2=||b^TU\Sigma V^T||^2_2\\
				&=||b^TU\Sigma||^2_2=\sum_{i=1}^k(b^Tu_i)^2\sigma_i^2.
			\end{aligned}	
		\end{equation}
		
		\noindent By Pythagorean Theorem, and $||R||^2_F\geq  {\textstyle \sum_{i=1}^{n}} ||c_ib||_2$, optimization problem (\ref{equation:reduced}) is equivalent to maximizing (\ref{equation:svd_bbt})). Since $\sum_{i=1}^k(b^Tu_i)^2=1$ and $\{\sigma_i\}$ are descending, Equation (\ref{equation:svd_bbt}) is maximized when $(b^Tu_1)^2=1$. It can be accomplished by setting $b=u_1$ and $c=\sigma_1v_1^T$, i.e., $\bar{R} = U\Sigma'V^T = bc^T$ is the optimal rank 1 approximation for $R$. 
	\end{proof} 
	
	
	
	
	\subsection{Additional experiments}
	
	
	
	\subsubsection{Evaluation of Light-Net}
	
	To evaluate the performance of Light-Net, we randomly sampled a testing set of 200 images from LIME~\cite{meka2018lime}. It is a synthetic dataset of Bigbird~\cite{singh2014bigbird} and ShapeNet~\cite{chang2015shapenet} objects with ground truth normal, albedo, and shading. 
	We use lighting coefficients predicted by Light-Net to render shading with ground truth normal maps. By comparing the rendered shading and ground truths, we can evaluate the accuracy of the estimated lighting coefficients. 
	For quantitative evaluation, we adopt three metrics, including MSE, scale-invariant MSE, and SSIM. The results are in Table~\ref{table:lightnet}. We compare to two ablations from Table~\ref{table:MIT}-\ref{table:janner}, which are ``loss+'' and ``without joint training''. We can see that for lighting evaluation, our final model produces the lowest MSE and scale-invariant MSE, and comparable SSIM to ``without joint training''. From visual examples in Figure~\ref{fig:lighteval}, our model renders similar shading with ground truths, while the predicted lighting is more directional than ground truths. 
	
	Although part of the LIME dataset is used in the pretraining of Normal-Net, here we only use Light-Net for this evaluation, for which the dataset is completely unseen.  
	
	
	
	\begin{figure}
		\centering
		\includegraphics[width=1\linewidth]{lighteval.pdf}\
		\caption{Visual examples of Light-Net evaluation. We render shading from ground truth normal and predicted lighting, and produce close results with ground truth shading. }
		\label{fig:lighteval}
	\end{figure}
	
	
	\begin{table}[h]
		\caption{Quantitative evaluation of Light-Net. 
		}
		\centering 
		\scalebox{0.8}{
			\begin{tabular}{c |c c c } 
				\hline 
				&The final model&loss+&w/o joint training\\
				\hline
				MSE $\downarrow$&\textbf{0.0403}&0.0452&0.0414\\
				SMSE $\downarrow$&\textbf{0.0336}&0.0368&0.0345\\
				SSIM $\uparrow$&0.8684&0.8652&\textbf{0.8686}\\
				\hline
			\end{tabular}
		}
		\label{table:lightnet} 
	\end{table}
	
	\subsubsection{Evaluation of Spec-Net}
	To evaluate the performance of specular highlight extraction of Spec-Net, we compare with several prior methods on a real-image dataset from \cite{yi2020leveraging}. As shown in Table~\ref{table:specnet}, Spec-Net outperforms other methods in both SMSE and DSSIM. Visual comparisons are in Figure~\ref{fig:specnet}. On real images where highlights are strong, and highlight regions are saturated, most methods tend to over-extract specular highlights, while the Spec-Net performs well due to the training on a large scale of real images. 
	
	\begin{table}
		\caption{Quantitative evaluation of specular highlight separation on real images. 
		}
		\centering 
		\scalebox{1}{
			\begin{tabular}{c |c c c c} 
				\hline 
				&\cite{shen2013real}&\cite{shi2017learning}&\cite{yamamoto2019general}&Ours\\
				\hline
				MSE&0.0334&0.0305&0.0334&\textbf{0.0148}\\
				DSSIM&0.1745&0.2087&0.1743&\textbf{0.1500}\\
				\hline
			\end{tabular}
		}
		\label{table:specnet} 
	\end{table}
	
	
	
	\begin{figure}
		\centering
		\includegraphics[width=1\linewidth]{specnet.pdf}\
		\caption{Qualitative comparisons on four data from real-image specularity separation dataset from \cite{yi2020leveraging}, captured by cross-polarization. From left to right, there are input images and diffuse components after removing specular highlights by \cite{shi2017learning,shen2013real,yamamoto2019general}, ours, and ground truths.  }
		\label{fig:specnet}
	\end{figure}
	
	\subsubsection{Additional results of experiments in the main paper}
	
	We show additional results for experiments in the main paper. In Figure~\ref{fig:mit}, there are visual comparisons of two data from MIT intrinsics~\cite{grosse2009ground}. Here all methods are not fine-tuned on MIT dataset. Here SIRFS~\cite{barron2013intrinsic} and DI~\cite{narihira2015direct} are supervised methods. Yi~\cite{yi2020leveraging} and ours are self-supervised, while they predict shading by a Shading-Net, and our shading is rendered from predicted normal and lighting. In Figure~\ref{fig:normal}, there are visual comparisons of normal estimation to several state-of-the-art methods, for the quantitative evaluation in Table \ref{table:janner}, on unseen data from Janner et al.~\cite{janner2017self}. Our method produces more details in normal maps. In Figure~\ref{fig:insertion}, we compare with a full relighting pipeline RelightingNet~\cite{yu2020self} on real object insertion. 
	
	\begin{figure*}
		\centering
		\includegraphics[width=1\linewidth]{mit_comparison.pdf}\
		\caption{Qualitative comparisons on two data from MIT Intrinsics. Odd rows are input images, albedos and even rows are shadings. The normal predicted by our method is shown at right.  }
		\label{fig:mit}
	\end{figure*}
	
	\begin{figure}
		\centering
		\includegraphics[width=1\linewidth]{relight_comparison.pdf}\
		\caption{Qualitative comparisons on object insertion by ours, RelightingNet~\cite{yu2020self} and naive insertion without relighting. }
		\label{fig:insertion}
	\end{figure}
	
	\begin{figure*}
		\centering
		\includegraphics[width=1\linewidth]{normal_janner_6data.pdf}\
		\caption{Normal estimation comparisons with SIRFS\cite{barron2013intrinsic}, SVBRDF\cite{li2018learning2}, InverseRenderNet\cite{yu2019inverserendernet}, RelightNet\cite{yu2020self}and ShapeAndMaterial\cite{Lichy_2021_CVPR} on selected data from Janner et al.~\cite{janner2017self}. The reference color map can be found in Figure~\ref{fig:endtoend}, where the red channel is x-axis pointing right, green channel corresponds to y-channel pointing down, and the blue channel is z-axis pointing out from the image plane. }
		\label{fig:normal}
	\end{figure*}
	
	% \subsection{Qualitative evaluation of rendering layers}
	
	In Figure~\ref{fig:relight}, we provide supplementary results of Table~\ref{table:relighting} and Figure ~\ref{fig:relighting}. Our diffuse and non-Lambertian rendering layers produce similar results with GT renderers. GT renderers are implemented by Monte-Carlo sampling of point lights following the Blinn-Phong model. 
	
	
	
	\subsubsection{More discussions}
	
	\noindent{\bf{Multi-view stereo as normal supervision.}} Previous method \cite{yu2019inverserendernet} uses multi-view stereo to reconstruct normal maps on outdoor building images in MegaDepth dataset\cite{li2018megadepth}, where ground truth depth maps are also available. Features on outdoor buildings are rich, which are suitable for multi-view stereo to reconstruction. 
	
	For object images, we explored similar approaches and found it not working for our scenarios. we use a reconstruction pipeline of adopting VisualSFM\cite{wu2011visualsfm} to reconstruct sparse point clouds, then PMVS2\cite{furukawa2010accurate} to further reconstruct dense point clouds. Applying the pipeline needs multi-view images as inputs, which would introduce a heavy workload for capturing multi-view images for all objects. For demonstration, we capture additional multi-view images and test the pipeline on several objects. For each object, we capture about 50 multi-view images as inputs. From the results, we find the point clouds are very sparse due to lack of features. A example is shown in Figure~\ref{fig:sfm}, textureless regions are quite common on natural objects, where the features are sparse, and reconstruction results have many holes on the resulting dense point clouds. For some other object, due to the lack of features, VisualSFM even fails to reconstruct a initial point cloud. Thus, adopting SFM and MVS to reconstruct geometry is not an option for our cases. 
	
	
	\begin{figure}
		\centering
		\includegraphics[width=1\linewidth]{sfm.pdf}\
		\caption{Object reconstructed by VisualSFM and PMVS2. Selected multi-view image inputs are shown on the left and reconstructed dense point clouds are on the right. }
		\label{fig:sfm}
		\vspace{-0.3cm}
	\end{figure}
	
	\begin{figure}
		\centering
		\includegraphics[width=1\linewidth]{singular_white.pdf}\
		\caption{On each row, selected images from one batch are shown at the left. Corresponding mean image, median image and singular image are at the right.  }
		\label{fig:meanimage}
	\end{figure}
	
	
	\noindent{\bf{Using median or mean reflectance vs. the singular reflectance.}} One may wonder whether using median or mean images of reflectance predictions in one batch will have similar results with our low-rank constraint. Firstly, losses between the median or mean reflectance of one batch and predicted reflectance are not scale-invariant. Secondly, the median image is not differentiable. Thirdly, we perform a large amount of testing on our Relit dataset and found that singular reflectance is more robust to shadows, intensity saturations and uneven lighting, which are common cases in natural images. Some visual comparisons are shown in Figure~\ref{fig:meanimage}, we can see that mean image may generate incorrect reflectance in some regions due to the above reasons while dominant singular reflectance generates much more reasonable reflectance maps. It is because SVD solves the dominant direction of reflectance maps, better than naive averaging. Note that we show cases on input images in Figure~\ref{fig:meanimage} because at the beginning of joint training, the network initializes from predicting reflectance the same as input images. We can see that using singular reflectance is much better visually, with convergence proven. \\
	
	\noindent{\bf{Comparisons to other low-rank losses. }}
	As mentioned in Section~\ref{exp:inv}, our definition of low-rank constraint is more robust and easy to converge. We evaluate the robustness of our low-rank loss with losses from \cite{yi2018faces} and \cite{yi2020leveraging}. Previous low-rank losses have more than one local optima as mentioned in \cite{yi2020leveraging}. Thus they have to use a pretraining phase to initialize the training, and the learning rates are hand-picked to make sure the final models converge to the local optima near the pretraining results. In Table~\ref{table:lowrankloss}, we found the learning rate has to be tuned carefully. For loss$^+$ in the table, a learning rate smaller than $10^{-8}$ would work. For loss*, we test learning rates from $10^{-2}$ to $10^{-8}$, and all cases degenerate to predict all-white or all-zero shadings. Setting a small learning rate also makes the training time much longer. Our loss has only one global and local optima, and it is promised to converge, and it does not suffer from degenerating. 
	
	Visual comparisons to previous low-rank losses (loss+ and loss*) from \cite{yi2018faces,yi2020leveraging} are in Figure~\ref{fig:losses}. We can see that loss+ gives similar results to ours, while albedo by our method is more smooth in color, and our normal is more accurate from Table~\ref{table:janner}. Note that here loss+ is trained in a small learning rate of $10^{-8}$ to prevent degeneration. It also benefits from our large-scale Relit dataset. However, even by a small learning rate of $10^{-8}$, loss* still degenerates and starts to predict all black albedo maps, as in Figure~\ref{fig:losses}. 
	
	\begin{figure*}
		\centering
		\includegraphics[width=1\linewidth]{loss_comparison.pdf}\
		\caption{Visual comparison of three low-rank losses on two unseen images. Benefiting from the Relit dataset, loss* produces similar results while setting a small learning rate. }
		\label{fig:losses}
	\end{figure*}
	
	
	\begin{figure*}
		\centering
		\includegraphics[width=1\linewidth]{relight_comp_supp.pdf}\
		\caption{Quantitative evaluation on relighting with and without specularity. RelightNet~\cite{yu2020self} can only provide diffuse relighting. Baseline* denotes the images under the original lighting. }
		\label{fig:relight}
	\end{figure*}
	\begin{table}[t]
		\centering 
		\scalebox{0.9}{
			\begin{tabular}{c| c c c c} 
				\hline 
				&$10^{-2}$&$10^{-4}$&$10^{-6}$&$10^{-8}$\\
				\hline
				loss$^+$ ($\sigma_2$)&\XSolidBrush&\XSolidBrush&\XSolidBrush&\Checkmark\\
				
				loss* ($\sigma_2/\sigma_1$)&\XSolidBrush&\XSolidBrush&\XSolidBrush&\XSolidBrush\\
				Ours&\Checkmark&\Checkmark&\Checkmark&\Checkmark\\
				\hline
			\end{tabular}
		}
		\caption{The robustness of different loss formulations. \XSolidBrush means the training degenerates to an invalid shading and \Checkmark means the training is converging.  }\label{table:lowrankloss} 
	\end{table}
	\subsection{Limitations}
	There are several limitations, as well as future directions of the proposed method. 
	One limitation is that, cast shadows (visibility) are not considered, which can further narrow the gap between relighting results and reality. 
	%sometimes plays an essential role while inserting objects into new scenes realistically. 
	%In the future, we plan to embed cast shadow rendering into single image relighting. 
	Furthermore, parametric models such as Blinn-Phong and Phong are difficult to model semitransparent and transparent materials, which are also common in real scenarios. Spherical harmonics are also limited to model high-frequency lighting components. We plan to explore these directions in the future. 
	%We are exploring region-adaptive Spherical Harmonics models to fit different level of Harmonics to different environment regions. 
	%detect a target plane to cast the object shadows on, which should be able to further improve object insertion effects. %In the future, we plan to add more splits into Relit dataset, such as multi-view videos under static lighting, facilitating many multi-view vision tasks. 
	
	\subsection{Relighting demos}
	
	On the project page \footnote{\href{https://renjiaoyi.github.io/relighting/}{https://renjiaoyi.github.io/relighting/}}, we include many relighting videos under changing backgrounds. Relit images are inserted to target scenes to show a seamless AR object insertion effect. We demonstrate single-object insertion and multi-object insertion where multiple objects are from different input images. We also demonstrate editing the materials of objects. Object insertion is quite popular in AR applications, and most AR Apps simply adopt naive insertion without relighting, such as the dancing hotdog in SnapChat, and furniture in Ikea Place. From the video, we can see our method generates much better object insertion results than naive insertion, demonstrating the importance of this problem. 
	
	Note that the backgrounds are cropped from HDR lighting panoramas, after Gamma corrections with $\gamma$ as $2.2$. Codes for pre-computation of Spherical Harmonic coefficients, and end-to-end inverse rendering and relighting will be released on the project page. 
	
	\subsection{App implementation}
	
	To implement the object relighting app in the Android mobile system, we convert the network models to Pytorch Mobile and package them inside the application as assets. For object photos captured from the camera, an on-device GrabCut in OpenCV is applied to obtain the object mask. To ensure acceptable automatic segmentation results, we require users to capture the objects under a background of solid colors. For photos loading from memory, the object mask is required as an additional input. We can insert and relight single or multiple objects from different photos into the same scene, and manipulate the layouts and sizes through simple dragging, tailored for amateur users. 
	
	The application is implemented in Java, using the Android Gradle plugin of version 3.5.0 with several additional Gradle and Pytorch dependencies. The app demo video is also on the project page. 
	
	\subsection{The Relit dataset}\label{sec:dataset}
	
	To capture foreground-aligned videos of objects under changing illuminations, we design an automatic device for data capture, as shown in Figure~\ref{fig:dataset} (left). The main part is an electric turntable painted black to avoid strong reflections. 
	%Objects are taped on the turntable, and the camera is fixed on the turntable by a screw.
	While capturing data, objects and the camera are fixed on the turntable. The turntable rotates at a uniform angular velocity of $12.6$ rad/s, controlled by a remote to avoid shaking. For each video, the device is rotated by $360^{\circ}$ for 50 seconds. 
	
	The device is chargeable and portable, enabling us to capture data under arbitrary scenes easily. The target object stays static in the image coordinate system in captured videos, with changing illuminations and backgrounds. These foreground-aligned videos can facilitate many tasks, such as image relighting, segmentation, and inverse rendering. 
	
	In summary, the Relit dataset consists of 500 videos for more than 100 objects under different indoor and outdoor lighting. Each video is 50 seconds, resulting in 1500 foreground-aligned frames under various lighting. In total, the Relit dataset consists of $750K$ images. %, which benefits many applications in deep learning. 
	%with one corresponding object mask (only one mask is needed for each video since objects are aligned among frames)
	In pre-processing, we segment the mask for one frame of each video and apply it to all frames to remove the changing backgrounds. 
	Selected objects are shown in Figure~\ref{fig:dataset} (right) The objects cover a wide variety of shapes, materials, and textures. 
	
	Some foreground-aligned images in Relit dataset are shown in Figure~\ref{fig:dataset4}-\ref{fig:dataset7}. These are selected frames from some videos after preprocessing. Sample videos from the dataset are shown on the project page, where the device is very stable, making sure the foreground objects are staying well-aligned among all frames. The dataset is released on the project page.  
	
	\subsection{Network structure and training details}
	
	Normal-Net and Light-Net are the only two learnable modules in our diffuse pipeline, and an optional specular branch may be used depending on the materials of target objects. The structures are in Figure~\ref{fig:structure}. Spec-Net shares the same structure with \cite{yi2020leveraging}. The network to regress specular reflectance $S_p$ and smoothness $\alpha$ shares the same structure of Light-Net, while changing the output to 4 channels (3 for specular reflectance and 1 for smoothness). 
	
	In pretraining of Normal-Net, $50K$ synthetic images from LIME~\cite{meka2018lime} are used for training. The learning rate is $10^{-4}$ without further adjustments. The training lasts for 50 epochs, by Adam optimizer. 
	
	In our joint training, we use the large-scale foreground-aligned images from Relit dataset. Light-Net is initialized from scratch and Normal-Net is initialed by the pre-trained model. The learning rate is $10^{-6}$ without further adjustments. Each round of joint training last for 3 epochs, taking 60 minutes per epoch on Tesla P40 GPU. The joint training process driven by the proposed low-rank loss converges rapidly, which takes 6 hours in total, thanks to the convergence proven in Section~\ref{sec:unsupervised}. 
	
	% data of Figure 9 in the main paper is shown in Figure \ref{fig:relight}. 
	
	\begin{figure*}
		\centering
		\includegraphics[width=\linewidth]{structure-both.pdf}\
		\caption{(a) Structure of Normal-Net. (b) Structure of Light-Net.}
		\label{fig:structure}
	\end{figure*}
	
	% \begin{figure}
		% 	\centering
		% 	\includegraphics[width=0.8\linewidth]{figures/LightNet.pdf}\
		% 	\caption{Structure of Light-Net. }
		% 	\label{fig:lightnet}
		% \end{figure}
	
	% \begin{figure*}
		% 	\centering
		% 	\includegraphics[width=0.9\linewidth]{figures/NormalNet.pdf}\
		% 	\caption{Structure of Normal-Net. Spec-Net shares the same structure of Normal-Net with the output channel from the last layer changing to 3. }
		% 	\label{fig:normalnet}
		% \end{figure*}
	
	
	
	%\section{Sample images from Relit dataset}
	
	\begin{figure*}
		\centering
		\includegraphics[width=0.8\linewidth]{dataset4.pdf}\
		\caption{Selected frames from one video in Relit dataset. }
		\label{fig:dataset4}
	\end{figure*}
	
	\begin{figure*}
		\centering
		\includegraphics[width=0.8\linewidth]{dataset5.pdf}\
		\caption{Selected frames from one video in Relit dataset. }
		\label{fig:dataset5}
	\end{figure*}
	
	\begin{figure*}
		\centering
		\includegraphics[width=\linewidth]{dataset6.pdf}\                
		\caption{Selected frames from one video in Relit dataset. }
		\label{fig:dataset6}
	\end{figure*}
	
	\begin{figure*}
		\centering
		\includegraphics[width=\linewidth]{dataset7.pdf}\
		\caption{Selected frames from one video in Relit dataset. }
		\label{fig:dataset7}
	\end{figure*}
	
\end{appendices}
\end{document}