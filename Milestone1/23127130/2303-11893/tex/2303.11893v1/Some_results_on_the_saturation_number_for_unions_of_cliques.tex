\documentclass[12pt]{article}
\usepackage[misc]{ifsym}
\usepackage[T1]{fontenc}
\usepackage[utf8]{inputenc}
\usepackage{authblk}
\usepackage{indentfirst}
\usepackage[top=1in, bottom=1in, left=1.00in, right=1.00in]{geometry}
\renewcommand{\baselinestretch}{1.5}
\usepackage{amsmath}
\usepackage{enumerate}
\usepackage{amsthm}
\usepackage{abstract}
\renewcommand{\abstractnamefont}{\Large\bfseries}
\usepackage{bm}
\usepackage{graphics}
\usepackage{graphicx}
\usepackage{subfigure}
\usepackage{caption}
\usepackage{amsthm,amsmath,amssymb}
\usepackage{float}
\usepackage[section]{placeins}
\usepackage{multirow}
\usepackage{amssymb}
\usepackage{IEEEtrantools}
\usepackage{cite}
\usepackage{pifont}
\pdfoutput=1

\title{Some results on the saturation number for unions of cliques}
\date{}
\author{Fan Chen, Xiying Yuan\thanks{Corresponding author. Email address: xiyingyuan@shu.edu.cn (Xiying Yuan)\\ \indent chenfan@shu.edu.cn (Fan Chen)\\ \indent This work was supported by the National Nature Science Foundation of China (Nos.11871040,
12271337)}}
\affil{\small{\emph{Department of Mathematics, Shanghai University, Shanghai 200444, P.R. China}}}

\begin{document}
\newtheorem{theorem}{Theorem}[section]
\newtheorem{assumption}[theorem]{Assumptio}
\newtheorem{corollary}[theorem]{Corollary}
\newtheorem{proposition}[theorem]{Proposition}
\newtheorem{lemma}[theorem]{Lemma}
\newtheorem{definition}[theorem]{Definition}
\newtheorem{remark}[theorem]{Remark}
\newtheorem{problem}[theorem]{Problem}
\newtheorem{claim}{Claim}
\newtheorem{conjecture}[theorem]{Conjecture}

\maketitle
\noindent\rule[0pt]{16.5cm}{0.09em}

\noindent{\bf Abstract}

\noindent The graph $G$ is $H$-saturated if $H$ is not a subgraph of $G$ and $H$ is a subgraph of $G+e$ for any edge $e$ not in $G$. The saturation number for a graph $H$ is the minimum number of edges in any $H$-saturated graph of order $n$. In this paper, the saturation number for $K_p\cup (t-1)K_q$ ($2\leqslant p<q$) is determined, and when $t=3$ the extremal graph is determined. Moreover, the saturation number and the extremal graph for $K_p\cup K_q\cup K_r$ ($2\leqslant p\leqslant q\leqslant r-2$) are completely determined.

\noindent{\bf Keywords:} Saturation number, Disjoint union of cliques, Extremal graph

\noindent\rule[0pt]{16.5cm}{0.05em}

\section{Introduction}
The vertex set and the edge set of a graph $G$ are denoted by $V(G)$ and $E(G)$. The order and size of $G$ are usually denoted by $|V(G)|=v(G)$ and $|E(G)|=e(G)$. For any two graphs $G$ and $H$, $G\cup H$ is the union of the graph $G$ and $H$ with $V(G\cup H)=V(G)\cup V(H)$ and $E(G\cup H)=E(G)\cup E(H)$. The join of the graph $G$ and $H$, denoted by $G\vee H$, is the graph obtained from $G\cup H$ by adding edges between $V(G)$ and $V(H)$.

We denote the complement of $G$ by $\overline{G}$. $G$ is $H$-saturated if $H$ is not a subgraph of $G$ and $H$ is a subgraph of $G+e$ for any edge $e$ of $E(\overline{G})$ (\cite{PAJ}). The saturation number for a graph $H$, denoted by $sat(n,H)$, is the minimum number of edges in any $H$-saturated graph of order $n$. Saturation numbers were firstly studied by P. Erd\H{o}s, A. Hajnal and J. W. Moon in \cite{PAJ}. In this paper, we mainly focus on the results about the saturation numbers for unions of cliques. Denote a complete graph and an independent set of order $n$ by $K_n$ and $I_n$. In \cite{PAJ}, it was proved that $sat(n,K_r)=(r-2)(n-r+2)+\binom{r-2}{2}$ and the unique extremal graph is $K_{r-2}\vee I_{n-r+2}$. L. K\'{a}szonyi and Z. Tuza in \cite{LZ} determined the value of $sat(n,tK_2)$ and the extremal graph. In \cite{RMRM}, J. Faudree, R. Ferrara and R. Gould et al., obtained the value of $sat(n,tK_p)$ and determined the extremal graph for $3K_p$. They also determined the extremal graph for $K_p\cup K_q$. We refer the reader to \cite{C, CFFGJM, CLLYZ}, \cite{JRJ}, \cite{ HL}, and \cite{LGS, LSWZ, O, PS} for more results about the saturation number, and \cite{JRJ} is an excellent dynamic survey. J. Faudree, R. Faudree and J. Schmitt proposed an open problem about $sat(n,K_p\cup K_q\cup K_r)$ and $sat(n,2K_p\cup K_q)$ (see Problem 3 in \cite{JRJ}).
\begin{problem}\cite{JRJ}\label{c}
Investigate $sat(n,K_p\cup K_q\cup K_r)$ and $sat(n,2K_p\cup K_q)$.
\end{problem}

We always suppose that $2\leqslant p_1\leqslant \cdots\leqslant p_t$. Write
$$H(n;p_1,p_2,\cdots,p_t)\cong K_{p_1-2}\vee (K_{p_2+1}\cup \cdots\cup K_{p_t+1}\cup I_{n-\sum_{i=1}^tp_i-t+3}).$$ Motivated by Problem \ref{c}, when $2\leqslant p<q$, we prove that
$$sat(n,K_p\cup (t-1)K_q)=(p-2)(n-p+2)+\binom{p-2}{2}+(t-1)\binom{q+1}{2},$$
where $n>q(q+1)(t-1)+3(p-2)$ (see Theorem \ref{ptq}). When $t=3$,
the unique extremal graph is $H(n;p,q,q)$ (see Theorem \ref{p2q}). When $2\leqslant p\leqslant q\leqslant r-2$, we prove that
$$sat(n,K_p\cup K_q\cup K_{r})=(p-2)(n-p+2)+\binom{p-2}{2}+\binom{q+1}{2}+\binom{r+1}{2},$$
and the unique extremal graph is $H(n;p,q,r)$ where $n>q(q+1)+r(r+1)+3(p-2)$ (see Theorem \ref{pqr}).

\section{Preliminaries}
For any vertex $v$ in $V(G)$, $N_G(v)$ is the set of the neighbors of $v$ and $N_G[v]=N_G(v)\cup\{v\}$. The degree of a vertex $v$ is $d_G(v)=|N_G(v)|$. If there is no ambiguity, then we write $N_G(v)=N(v)$ and $d_G(v)=d(v)$. Let the minimum degree of $G$ be $\delta(G)$. For $S\subset V(G)$, $G[S]$ is the subgraph of $G$ induced by the vertices in $S$.

\begin{lemma}\label{saturated} If $p_2=\cdots =p_t$ or $p_{i+1}-p_i\geqslant2$ for any $2\leqslant i\leqslant t-1$, then $H(n;p_1,p_2,\cdots,p_t)$ is a $K_{p_1}\cup\cdots\cup K_{p_t}$-saturated graph.
\end{lemma}
\begin{proof}[\rm{\textbf{Proof}.}]
Write $G_1\cong K_{p_1-2}$ and $G_i\cong K_{p_i+1}$ for any $2\leqslant i\leqslant t$. For brevity, write $G= H(n;p_1,\cdots,p_t)=G_1\vee(G_2\cup\cdots \cup G_t\cup I_{n-\sum_{i=1}^tp_i-t+3})$.

Suppose that $p_2=\cdots=p_t$. If $G$ has a subgraph isomorphic to $K_{p_1}\cup K_{p_2}\cup\cdots\cup K_{p_2}$, then the copy of $K_{p_1}$ lies in some $G_1\vee G_{\ell}$, say $G_1\vee G_2$. The copy of $K_{p_2}$ does not lie in $G_1\vee G_2$ since there are only $p_2-1$ vertices in $V(G_1\vee G_2)\setminus V(K_{p_1})$. Then there is no subgraph $(t-1)K_{p_2}$ in $G[V(G)\setminus V(K_{p_1})]$. Hence, $G$ is $K_{p_1}\cup\cdots\cup K_{p_t}$-free when $p_2=\cdots=p_t$.

Suppose that $p_{i+1}-p_i\geqslant2$ for any $2\leqslant i\leqslant t-1$. If $G$ has a subgraph isomorphic to $K_{p_1}\cup K_{p_2}\cup\cdots\cup K_{p_t}$, then the copy of $K_{p_1}$ lies in $G_1\vee G_{\ell}$ for some $2\leqslant \ell\leqslant t$. Suppose $\ell=t$. The copy of $K_{p_t}$ does not lie in $G_1\vee G_t$ or in $G_{\ell'}$ for any $\ell'\neq t$, since there are only $p_t-1$ vertices in $V(G_1\vee G_t)\setminus V(K_{p_1})$, and $p_t-p_{\ell'}\geqslant 2$ for any $2\leqslant \ell'\leqslant t-1$. Hence, there is no copy of $K_{p_t}$ in $G$.
Suppose $2\leqslant \ell\leqslant t-1$. The copy of $K_{p_{\ell}}$ does not lie in $G_1\vee G_{\ell}$ or in $G_{\ell_1}$ for any $\ell_1<\ell$ since there are only $p_{\ell}-1$ vertices in $V(G_1\vee G_{\ell})\setminus V(K_{p_1})$ and $p_{\ell}-p_{\ell_1}\geqslant 2$ for any $\ell_1<\ell$. Hence $K_{p_{\ell}}$ lies in $G_{\ell_2}$ for some $\ell_2>\ell$. Similarly, the copy of $K_{p_{\ell_2}}$ lies in $G_{\ell_3}$ for some $\ell_3>\ell_2$. Continuing this process, there is a copy of $K_{p_{\ell_a}}$ lies in $G_t$ for some $2\leqslant \ell_a\leqslant t-1$. So there is no copy of $K_{p_t}$ in $G$. Hence, $G$ is also $K_{p_1}\cup\cdots\cup K_{p_t}$-free when $p_{i+1}-p_i\geqslant2$ for any $2\leqslant i\leqslant t-1$.

For any nonadjacent $u$ and $v$ in $G$, then $u$ and $v$ lie in distinct $G_i$'s. Thus, in the graph $G^*=G+uv$, we have $G^*[V(G_1)\cup\{u,v\}]\cong K_{p_1}$ and there is a subgraph isomorphic to $K_{p_2}\cup \cdots\cup K_{p_t}$ in $G[V(G)\setminus (V(G_1)\cup\{u,v\})]$. Hence, $G$ is $K_{p_1}\cup\cdots\cup K_{p_t}$-saturated.
\end{proof}

When $G$ is an extremal graph for $K_{p_1}\cup \cdots\cup K_{p_t}$, we may give an elementary characterization for $G$ in Lemma \ref{detla}.
\begin{lemma}\label{detla}
Suppose that $G$ is a $K_{p_1}\cup \cdots\cup K_{p_t}$-saturated graph with $e(G)=sat(n,K_{p_1}\cup\cdots\cup K_{p_t})$, $n>\sum_{i=2}^tp_i(p_i+1)+3(p_1-2)$ and $t\geqslant 2$. If
$$e(G)\leqslant e(H(n;p_1,p_2,\cdots,p_t))=(p_1-2)(n-p_1+2)+\binom{p_1-2}{2}+\sum_{i=2}^t\binom{p_i+1}{2},$$
then we have

\noindent(1) $\delta(G)=p_1-2$,

\noindent(2) $G[N(v)]\cong K_{p_1-2}$,
$$e(G[N(v),\overline{N(v)}])=(p_1-2)(n-p_1+2),$$
and
$$e(G[\overline{N(v)}])\leqslant\sum_{i=2}^t\binom{p_i+1}{2},$$
where $v$ is a vertex of minimum degree in $V(G)$.
\end{lemma}
\begin{proof}[\rm{\textbf{Proof}.}]
Write
\begin{equation}\label{a}
e(G)\leqslant (p_1-2)(n-p_1+2)+\binom{p_1-2}{2}+\sum_{i=2}^t\binom{p_i+1}{2}=a.
\end{equation}
Since $G$ is $K_{p_1}\cup\cdots\cup K_{p_t}$-saturated, for any vertex $u$ in $\overline{N[v]}$, $G+uv$ contains a subgraph isomorphic to $K_{p_1}\cup \cdots\cup K_{p_t}$. Then $uv$ lies in a clique of order at least $p_1$ in $G+uv$. Hence, $u$ has at least $p_1-2$ neighbors in $N(v)$, and there is a copy of $K_{p_1-2}$ in $G[N(v)]$.

(1). We have proved $d(v)=\delta(G)\geqslant p_1-2$. Assume to the contrary that $d(v)\geqslant p_1-1$. Write $\delta(G)=\delta$ for brevity. We have
\begin{align*}
\sum_{w\in N(v)}d_G(w)&=2e(G[N(v)])+e(G[N(v),\overline{N(v)}])\\
&\geqslant2\binom{p_1-2}{2}+(n-\delta-1)(p_1-2)+\delta.
\end{align*}
This yields that
\begin{align}\label{1}
2e(G)&=\sum_{w\in N(v)}d_G(w)+\sum_{w\in \overline{N(v)}}d_G(w)\notag\\
&\geqslant2\binom{p_1-2}{2}+(n-\delta-1)(p_1-2)+\delta+\delta(n-\delta) \triangleq f(\delta).
\end{align}
Since $p_1-1\leqslant\delta \leqslant \frac{2e(G)}{n}\leqslant\frac{2a}{n}\leqslant 2p_1-3\leqslant \frac{n-p_1+3}{2}$, we have
\begin{equation}\label{2}
\text{min}f(\delta)=f(p_1-1)=2\binom{p_1-2}{2}+(n-p_1)(p_1-2)+(p_1-1)+(p_1-1)(n-p_1+1).
\end{equation}
Combining (\ref{a}) (\ref{1}) and (\ref{2}), we get
$$n\leqslant \sum_{i=2}^tp_i(p_i+1)+3(p_1-2).$$
This is a contradiction to the assumption of $n>\sum_{i=2}^tp_i(p_i+1)+3(p_1-2)$. So, $\delta(G)=p_1-2$.

(2). For any vertex $u$ in $\overline{N[v]}$, $u$ has at least $p_1-2$ neighbors in $N(v)$ and $uv$ lies in a clique of order at least $p_1$ in $G+uv$. Combining the result $|N(v)|=p_1-2$, then each vertex in $N(v)$ is adjacent to the vertices in $\overline{N(v)}$ and $G[N(v)]\cong K_{p_1-2}$. Hence,
$$e(G[N(v),\overline{N(v)}])=(p_1-2)(n-p_1+2).$$
By (\ref{a}), we get
\begin{align*}
e(G) &=e(G[N(v)])+e(G[N(v),\overline{N(v)}])+e(G[\overline{N(v)}])\\
&=\binom{p_1-2}{2}+(p_1-2)(n-p_1+2)+e(G[\overline{N(v)}])\leqslant a.
\end{align*}
Hence,
\begin{equation*}
e(G[\overline{N(v)}])\leqslant\sum_{i=2}^t\binom{p_i+1}{2}.
\end{equation*}
\end{proof}


\section{The saturation number for $K_p\cup (t-1)K_q$ ($2\leqslant p< q$)}

L. K\'asonyi and Z. Tuza determined $sat(n,tK_2)=3t-3$, and the extremal graph is $(t-1)K_3\cup (n-3t+3)K_1$ when $n\geqslant 3t-3$ (see Corollary 5 in \cite{LZ}). For general $tK_p$, R. Faudree, M. Ferrara and R. Gould et al., determined that
$$sat(n,tK_p)=(p-2)(n-p+2)+\binom{p-2}{2}+(t-1)\binom{p+1}{2},$$
and the extremal graph is $H(n;p,p,p)$ when $t=3$ where $n\geqslant tp(p+1)-p^2+2p-6$ and $p\geqslant 3$ (see Theorem 2.1 and Theorem 2.3 in \cite{RMRM}). We will further consider the saturation number for $K_p\cup (t-1)K_q$ when $2\leqslant p<q$ in this section.
\begin{theorem}\label{ptq}
Suppose $2\leqslant p< q$ and $n> q(q+1)(t-1)+3(p-2)$. Then
$$sat(n,K_p\cup (t-1)K_q)=(p-2)(n-p+2)+\binom{p-2}{2}+(t-1)\binom{q+1}{2}.$$
\end{theorem}
\begin{proof}[\rm{\textbf{Proof}.}]
Let $G$ be a $K_p\cup (t-1)K_q$-saturated graph of order $n> q(q+1)(t-1)+3(p-2)$ with $e(G)=sat(n,K_p\cup (t-1)K_q)$. By Lemma \ref{saturated}, $H(n;p,q,\cdots,q)$ is $K_p\cup (t-1)K_q$-saturated. Then $e(G)\leqslant e(H(n;p,q,\cdots,q))$. Suppose that $v$ is a vertex of minimum degree in $G$. Write $S=N(v)$ and $\overline{S}=\overline{N(v)}$. By Lemma \ref{detla}, we have $|S|=p-2$, $G[S]\cong K_{p-2}$ and
\begin{equation}\label{g}
e(G[\overline{S}])\leqslant (t-1)\binom{q+1}{2}.
\end{equation}
Note that $G+vw$ contains a subgraph $K_p\cup (t-1)K_q$ and the edge $vw$ lies in the copy of $K_p$ for a vertex $w\in \overline{S}\setminus\{v\}$. Hence, there is a subgraph $H_{vw}= H_1\cup H_2\cup \cdots \cup H_{t-1}$ with $H_i\cong K_q$ ($i=1,2,\cdots,t-1$) in $G$.

For any vertex $x\in V(H_i)$, $S\cup V(H_i)\subseteq N[x]$ and $d_G(x)\geqslant p+q-3$ hold. For some vertex $x'\in (V(H_i)\setminus \{x\})$, considering the graph $G_1=G+vx'$, it contains a subgraph isomorphic to $K_p\cup (t-1)K_q$, and $G_1[S\cup\{v,x'\}]$ is the copy of $K_p$. If $d_G(x)=p+q-3$, then $x$ is not contained in $(t-1)K_q$ since $|N(x)\setminus (S\cup\{x'\})|=p+q-3-(p-1)<q-1$. Consequently, replacing $G_1[S\cup\{v,x'\}]$ with $G[S\cup\{x,x'\}]$, we find a subgraph of $G$ isomorphic to $K_p\cup (t-1)K_q$. So, we have $d(x)\geqslant p+q-2$ for any vertex $x\in V(H_{vw})$.

Let $R=\big\{u\, |\, u\in \overline{S}\setminus V(H_{vw})\ \text{and}\ u\ \text{has at least one neighbor in $V(H_{vw})$}\big\}$. For any vertex $x\in V(H_i)$, the fact $d_G(x)\geqslant p+q-2$ implies that $x$ has at least one neighbor in $R\cup (V(H_{vw})\setminus V(H_i))$.

If each $x\in V(H_{vw})$ has a neighbor in $R$, we have
$$e(G[\overline{S}])\geqslant e(H_{vw})+e(G[V(H_{vw}),R])\geqslant e(H_{vw})+v(H_{vw})=(t-1)\binom{q}{2}+(t-1)q=(t-1)\binom{q+1}{2}.$$
Combining (\ref{g}), we have $e(G[\overline{S}])=(t-1)\binom{q+1}{2}$.
Consequently,
\begin{align*}
e(G)=e(G[S])+e(G[S,\overline{S}])+e(G[\overline{S}])=\binom{p-2}{2}+(p-2)(n-p+2)+(t-1)\binom{q+1}{2}.
\end{align*}

If $x$ has a neighbor in $V(H_{vw})\setminus V(H_i)$ for some vertex $x\in V(H_i)$ ($1\leqslant i\leqslant t-1$), then the subgraph $H'$ given by $G[V(H_{vw})]\setminus E(H_{vw})$ is not empty. Let $Q$ be a component of $H'$. If $Q$ is a tree with at least two vertices, then we will prove that some vertex of $Q$ has a neighbor in $R$. Suppose to the contrary that no vertex of $Q$ has a neighbor in $R$. Let $u$ be an end vertex of one longest path of $Q$. Without loss of generality, assume that $u\in V(H_1)$. Suppose that $z$ is the neighbor of $u$ in $Q$. Then $z\in V(H_i)$ ($i\geqslant 2$). Write $V(H_1)=\{u,u_2,\cdots,u_q\}$. We claim that $z\sim u_i$ for $2\leqslant i\leqslant q$. Since $u$ is a pendant vertex of $Q$, we have $N_G[u]=S\cup V(H_1)\cup \{z\}$ and $d(u)=p+q-2$. Assume that $z\nsim u_2$. Considering the graph $G_2=G+vu_3$, $G_2[S\cup\{v,u_3\}]\cong K_p$, and there is a subgraph $H_{vu_3}$ isomorphic to $(t-1)K_q$ in $G$. If $u$ is contained in $H_{vu_3}$, then it is contained in the clique induced by $(V(H_1)\setminus \{u_3\})\cup\{z\}$. This is a contradiction to the assumption $z\nsim u_2$. Hence, $u$ is not contained in $H_{vu_3}$. Then $G[S\cup\{u,u_3\}]\cup  H_{vu_3}$ is a subgraph of $G$ isomorphic to $K_p\cup (t-1)K_q$. This is a contradiction to the choice of the graph $G$. Hence, the vertex $z$ is adjacent to all vertices in $V(H_1)$. Note that all but one of the neighbors of $z$ are the pendant vertices of $Q$. Without loss of generality, assume that $\{u,u_2,\cdots,u_{q-1}\}$ are pendant vertices of $Q$. For any vertex $z'\in V(H_i)\setminus\{z\}$, we have $u\nsim z'$. Hence, $G_3=G+uz'$ contains a subgraph $H_{uz'}^*$ isomorphic to $K_p\cup (t-1)K_q$. Let $W$ denote the clique of $H_{uz'}^*$ that contains the edge $uz'$ and $W'$ denote $G_3[V(H_{uz'}^*)\setminus V(W)]$. Note that $N_G(u)\cap N_G(z')\subseteq S\cup \{z,u_q\}$. Hence, $p\leqslant v(W)\leqslant p+2$. Now we distinguish three cases according to $v(W)$.

\textbf{Case 1.} $v(W)=p+2$.

In this case, $W=G_3[S\cup\{z,u_q,u,z'\}]$ is a clique of order $p+2$, and $q=p+2$ holds. Note that $N_G[u]=N_G[u_i]=S\cup V(H_1)\cup\{z\}$ for $2\leqslant i\leqslant q-1$. If $u_2$ is contained in $W'$, then $u_2$ is in the clique $G[\{u_2,\cdots,u_{q-1}\}]\cong K_p$. Therefore, $G[S\cup\{z,z'\}]\cup G[\{u,u_2,\cdots,u_q\}]\cup G[V(W')\setminus\{u_2,\cdots,u_{q-1}\}]$ is a subgraph of $G$ isomorphic to $K_p\cup (t-1)K_q$. Hence, the vertex $u_2$ is not contained in the subgraph $W'$. Then $G[S\cup\{z,u_q,u,u_2\}]\cup W'$ is a subgraph isomorphic to $K_p\cup (t-1)K_q$ in $G$, which is a contradiction.

\textbf{Case 2.} $v(W)=p+1$.

\textbf{Subcase 2.1} $W=G_3[S\cup\{z,u,z'\}]$ or $W=G_3[S\cup\{u_q,u,z'\}].$

Assume that $W=G_3[S\cup\{z,u,z'\}]$. Then $W$ is a clique of order $p+1$, and $q=p+1$ holds. Note that $N_G[u]=N_G[u_i]=S\cup V(H_1)\cup\{z\}$ for $2\leqslant i\leqslant q-1$. If $u_2$ is contained in $W'$, then $u_2$ is contained in the clique $G[\{u_2\cdots,u_{q-1},u_q\}]$. Therefore, $G[S\cup\{z,z'\}]\cup G[\{u,u_2,\cdots,u_q\}]\cup G[V(W')\setminus\{u_2,\cdots,u_{q-1},u_q\}]$ is a subgraph of $G$ isomorphic to $K_p\cup (t-1)K_q$. Hence, the vertex $u_2$ is not contained in $W'$. Then $G[S\cup\{z,u,u_2\}]\cup W'$ is a subgraph isomorphic to $K_p\cup (t-1)K_q$ in $G$, which is a contradiction. If $W=G_3[S\cup\{u_q,u,z'\}]$, then we may also obtain a contradiction as above.

\textbf{Subcase 2.2} $W=G_3[S'\cup\{z,u_q,u,z'\}]$ with $S'\subseteq S$ and $|S'|=p-3$ when $p\geqslant 3$.

Write $\{s\}=S\setminus S'$. Now $W$ is a clique of order $p+1$ and $q=p+1$ holds. If $u_2$ is contained in $W'$, then $u_2$ is contained in the clique $G[\{u_2\cdots,u_{q-1},s\}]$. Therefore, $G[S\cup\{z,z'\}]\cup G[\{u,u_2,\cdots,u_q\}]\cup G[V(W')\setminus\{u_2,\cdots,u_{q-1},s\}]$ is a subgraph of $G$ isomorphic to $K_p\cup (t-1)K_q$. Hence, the vertex $u_2$ is not contained in $W'$. Then $G[S'\cup\{z,u_q,u,u_2\}]\cup W'$ is a subgraph isomorphic to $K_p\cup (t-1)K_q$ in $G$, which is a contradiction.

\textbf{Case 3.} $v(W)=p$.

\textbf{Subcase 3.1} $W=G_3[S\cup\{u,z'\}]$.

Note that $G[\{u_2,\cdots,u_{q-1},u_q,z\}]$ is a clique of order $q$. If $u_2$ is contained in $W'$, then $u_2$ is contained in the clique $G[\{u_2,\cdots,u_{q-1},u_q,z\}]$. Thus $G[S\cup\{z,z'\}]\cup G[\{u,u_2,\cdots,u_{q-1},u_q\}]\cup G[V(W')\setminus\{u_2,\cdots,u_{q-1},u_q,z\}]$ is a subgraph of $G$ isomorphic to $K_p\cup (t-1)K_q$. Hence, the vertex $u_2$ is not contained in $W'$. Then $G[S\cup\{u,u_2\}]\cup W'$ is a subgraph isomorphic to $K_p\cup (t-1)K_q$, which is a contradiction.

\textbf{Subcase 3.2} $W=G_3[S'\cup\{z,u,z'\}]$ or  $W=G_3[S'\cup\{u_q,u,z'\}]$ with $S'\subseteq S$ and $|S'|=p-3$ when $p\geqslant 3$.

Write $\{s\}=S\setminus S'$. Assume that $W=G_3[S'\cup\{z,u,z'\}]$. Then $W$ is a clique of order $p$. Now $G[\{u_2,\cdots,u_{q-1},u_q,s\}]$ is a clique of order $q$ in $G$. If $u_2$ is contained in $W'$, then $u_2$ is contained in the clique $G[\{u_2,\cdots,u_{q-1},u_q,s\}]$. Thus $G[S\cup\{z,z'\}]\cup G[\{u,u_2,\cdots,u_{q-1},u_q\}]\cup G[V(W')\setminus \{u_2,\cdots,u_{q-1},u_q,s\}]$ is a subgraph isomorphic to $K_p\cup (t-1)K_q$ in $G$. Therefore, the vertex $u_2$ is not contained in $W'$. Hence, $G[S'\cup\{z,u_q,u,u_2\}]\cup W'$ is a subgraph of $G$ isomorphic to $K_p\cup (t-1)K_q$, which is a contradiction. If $W=G_3[S'\cup\{u_q,u,z'\}]$, then we may also obtain a contradiction above.

\textbf{Subcase 3.3} $W=G_3[S''\cup\{z,u_q,u,z'\}]$, with $S''\subseteq S$ and $|S''|=p-4$ when $p\geqslant 4$.

Write $\{s_1,s_2\}=S\setminus S''$. Now $G[\{u_2,\cdots,u_{q-1},s_1,s_2\}]$ is a clique of order $q$. If $u_2$ is contained in $W'$, then $u_2$ is contained in the clique $G[\{u_2,\cdots,$ $u_{q-1},s_1,s_2\}]$. Thus $G[S\cup\{z,z'\}]\cup G[\{u,u_2,\cdots,u_{q-1},u_q\}]\cup G[V(W')\setminus \{u_2,\cdots,u_{q-1},s_1,s_2\}]$ is a subgraph isomorphic to $K_p\cup (t-1)K_q$ in $G$. Hence, the vertex $u_2$ is not contained in $W'$. Then $G[S\cup\{u,u_2\}]\cup W'$ is a subgraph of $G$ isomorphic to $K_p\cup (t-1)K_q$, which is a contradiction.

From the above arguments, for a component $Q$ of $H'$, if $Q$ is a tree with at least two vertices, then $e(Q)+e(G[V(Q),R])\geqslant v(Q)$. If $Q$ is an isolated vertex $y$, then $y$ has a neighbor in $R$ since $d_G(y)\geqslant p+q-2$. Hence, $e(Q)+e(G[V(Q),R])\geqslant v(Q)$. If $Q$ contains a cycle, then $e(Q)\geqslant v(Q)$. So, we always have $e(H')+e(G[V(H'),R])\geqslant v(H')$.
Hence, we know
$$e(G[\overline{S}])\geqslant e(H_{vw})+e(H')+e(G[V(H'),R])\geqslant e(H_{vw})+v(H')\geqslant(t-1)\binom{q+1}{2}.$$
Consequently, we have
$$e(G[\overline{S}])=(t-1)\binom{q+1}{2},$$
and
\begin{align*}
e(G)=e(G[S])+e(G[S,\overline{S}])+e(G[\overline{S}])=\binom{p-2}{2}+(p-2)(n-p+2)+(t-1)\binom{q+1}{2}.
\end{align*}\end{proof}

\begin{theorem}\label{p2q}
Suppose $2\leqslant p< q$ and $n> 2q(q+1)+3(p-2)$. Then $H(n;p,q,q)$ is the unique extremal $K_p\cup 2K_q$-saturated graph.
\end{theorem}
\begin{proof}[\rm{\textbf{Proof}.}]
Let $G$ be a $K_p\cup 2K_q$-saturated graph of order $n> 2q(q+1)+3(p-2)$ with $e(G)=sat(n,K_p\cup 2K_q)$. By Lemma \ref{detla}, we may let $v$ be a vertex of degree $p-2$ in $G$. Write $S=N(v)$ and $\overline{S}=\overline{N(v)}$. For any vertex $w\in \overline{S}\setminus\{v\}$, the graph $G+vw$ contains a subgraph isomorphic to $K_p\cup 2K_q$ and $vw$ lies in the copy of $K_p$. Hence, there is a subgraph $H_{vw}=H_{vw,1}\cup H_{vw,2}$ with $H_{vw,1}\cong H_{vw,2}\cong K_q$ in $G$. Write $V(H_{vw,1})=\{u_1,\cdots,u_q\}$ and $V(H_{vw,2})=\{v_1,\cdots,v_q\}$. By Theorem \ref{ptq}, we have
\begin{equation}\label{2q}
e(G[\overline{S}])-e(H_{vw})= 2q.
\end{equation}

We claim that $N(x)\subseteq S\cup V(H_{vw})$ for any vertex $x\in \overline{S}\setminus V(H_{vw})$. If there is an edge $e=xy$ such that $y\in \overline{S}\setminus V(H_{vw})$. Then $G[S\cup\{x,y\}]\cup H_{vw}\cong K_p\cup 2K_q$, which is a contradiction. Let $R=\big\{u\, |\, u\in \overline{S}\setminus V(H_{vw})\ \text{and}\ u\ \text{has at least one neighbor in $V(H_{vw})$}\big\}$. Now we will show that $|R|\geqslant2$. Assume that $|R|\leqslant1$. If $u_i$ is not adjacent to $v_j$ for some $u_i\in V(H_{vw,1})$ and $v_j\in V(H_{vw,2})$, then the graph $G+u_iv_j$ contains $K_p\cup 2K_q$ as a subgraph. While the vertices of degree at least $p-1$ only lie in $S\cup V(H_{vw})\cup R$, and
$$|S\cup V(H_{vw})\cup R|\leqslant (p-2)+2q+1<p+2q.$$
There is not a subgraph $K_p\cup 2K_q$ in $G+u_iv_j$. If $u_i$ is adjacent to $u_j$ for any $1\leqslant i,j\leqslant q$, then
$$e(G[\overline{S}])-e(H_{vw})\geqslant q^2>2q,$$
which is a contradiction to (\ref{2q}). So $|R|\geqslant 2$ holds.

Let $y$ be a neighbor of $x$ in $V(H_{vw})$ for some vertex $x\in R$. Note that $N(x)\subseteq S\,\cup V(H_{vw})$. If $d_{H_{vw}}(x)\leqslant q-1$, then $d_G(x)\leqslant p+q-3$. Considering the graph $G_1=G+vy$, then $G_1[S\cup\{v,y\}]$ is the clique of order $p$. Hence, there is a subgraph $H_{vy}$, distinct from $H_{vw}$, isomorphic to $2K_q$ in $G$. Note that $x$ is not in $H_{vy}$, since there are at most $p+q-3-(p-1)=q-2$ neighbors of $x$ in $G[V(G)\setminus (S\cup\{v,y\})]$. Hence, $G[S\cup\{x,v\}]\cup H_{vy}$ is a subgraph isomorphic to $K_p\cup 2K_q$ in $G$, which is a contradiction to the choice of $G$. Hence, $d_{H_{vw}}(x)\geqslant q$ for any vertex $x\in R$.

Combining the fact $|R|\geqslant2$, $d_{H_{vw}}(x)\geqslant q$ for any vertex $x\in R$ and (\ref{2q}), we have
$$|R|q\leqslant e(G[\overline{S}])-e(H_{vw})= 2q.$$
Then $|R|=2$, $d_{H_{vw}}(x)=q$ and $d_G(x)=p+q-2$ for any vertex $x\in R$. It is obvious that $u_i\nsim v_j$ for any $1\leqslant i,j\leqslant q$. Otherwise, $e(G[\overline{S}])-e(H_{vw})>2q$, which is a contradiction to (\ref{2q}). Let $R=\{x_1,x_2\}$. Without loss of generality, assume that $u_1$ is a neighbor of $x_1$ in $H_{vw}$. Considering the graph $G+vu_1$, there is a subgraph $H_{vu_1}$, distinct from $H_{vw}$, isomorphic to $2K_q$ in $G$. If $x_1$ is not in $H_{vu_1}$, then $G[S\cup\{u_1,x_1\}]\cup H_{vu_1}$ is a subgraph isomorphic to $K_p\cup 2K_q$ in $G$, which is a contradiction. Hence, the vertex $x_1$ is contained in a clique of order $q$ in $H_{vu_1}$. Noting that the graph $G[N(x_1)\setminus (S\cup\{u_1\})]$ is a clique of order $q-1$ in $H_{vu_1}$, we have $N(x_1)\setminus(S\cup \{u_1\})\subseteq V(H_{vw,1})$ or $N(x_1)\setminus(S\cup \{u_1\})\subseteq V(H_{vw,2})$. Suppose $N(x_1)\setminus(S\cup \{u_1\})\subseteq V(H_{vw,2})$. Then we assume that $x_1\sim v_i$ ($i=1,\cdots,q-1$) and consider the auxiliary graph $G_2=G+vv_1$. Since $N(x_1)=S\cup\{u_1,v_1,\cdots,v_{q-1}\}$ and the vertices of $S\cup \{v_1\}$ are contained in the clique of order $p$ in $G_2$, we have $G[\{u_1,v_2,\cdots,v_{q-1},x_1\}]\cong K_q$. There is a contradiction to the fact $u_1\nsim v_i$ for any $1\leqslant i\leqslant q$. Hence, $N(x_1)\setminus(S\cup \{u_1\})\subseteq V(H_{vw,1})$. Since $d_G(x_1)=p+q-2$, we have $N(x_1)=S\cup V(H_{vw,1})$. Similarly, $N(x_2)=S\cup V(H_{vw,i})$ for some $1\leqslant i\leqslant2$. We will further prove that $N(x_2)=S\cup V(H_{vw,2})$. Otherwise, we have $N(x_1)=N(x_2)=S\cup V(H_{vw,1})$. Notice that $N[v_2]=S\cup V(H_{vw,2})$ and $d_G(v_2)=p+q-3$. Then there are at most $q-2$ neighbors of $v_2$ in $G[V(G)\setminus (S\cup \{v,v_1\})]$, which implies that $v_2$ is not contained in $H_{vv_1}$. Hence, $G[S\cup\{v_1,v_2\}]\cup H_{vv_1}$ is a subgraph isomorphic to $K_p\cup 2K_q$ in $G$, which is a contradiction. Thus, $N(x_1)=S\cup V(H_{vw,1})$ and $N(x_2)=S\cup V(H_{vw,2})$. Hence, $$G=G[S]\vee\left(G[V(H_{vw,1})\cup\{x_1\}]\cup G[V(H_{vw,2})\cup\{x_2\}]\cup I_{n-p-2q}\right)\cong H(n;p,q,q).$$
\end{proof}

\section{The saturation number for $K_p\cup K_q\cup K_r$ ($2\leqslant p\leqslant q\leqslant r-2$)}
There is a $K_p\cup K_q$ in $K_{p-2}\vee K_{q+2}$, and then there is a subgraph $K_p\cup K_q\cup K_{q+1}$ in $K_{p-2}\vee(K_{q+1}\cup K_{q+2})$. Thus, $H(n;p,q,q+1)$ is not $K_p\cup K_q\cup K_{q+1}$-free. While for $q\leqslant r-2$, $H(n;p,q,r)$ is a $K_p\cup K_q\cup K_r$-saturated graph (see Lemma \ref{saturated}). In this section, we will further prove that $H(n;p,q,r)$ is the extremal $K_p\cup K_q\cup K_r$-saturated graph with minimal edges when $q\leqslant r-2$.

\begin{theorem}\label{pqr}
Suppose $2\leqslant p\leqslant q\leqslant r-2$ and $n>q(q+1)+r(r+1)+3(p-2)$. Then
$$sat(n,K_p\cup K_q\cup K_r)=(p-2)(n-p+2)+\binom{p-2}{2}+\binom{q+1}{2}+\binom{r+1}{2}.$$
Furthermore, $H(n;p,q,r)$ is the unique extremal $K_p\cup K_q\cup K_r$-saturated graph.
\end{theorem}
\begin{proof}[\rm{\textbf{Proof}.}]
Let $G$ be a $K_p\cup K_q\cup K_r$-saturated graph of order $n$ with $e(G)=sat(n,K_p\cup K_q\cup K_r)$. By Lemma \ref{saturated}, $H(n;p,q,r)$ is a $K_p\cup K_q\cup K_r$-saturated graph. Hence, $e(G)\leqslant e(H(n;p,q,r))$. Let $v$ be a vertex with minimum degree in $G$. Write $S=N(v)$ and $\overline{S}=\overline{N(v)}$. By Lemma \ref{detla}, we have $|S|=p-2$, $G[S]\cong K_{p-2}$ and
\begin{equation}\label{qr}
e(G[\overline{S}])\leqslant\binom{q+1}{2}+\binom{r+1}{2}.
\end{equation}
Choosing a vertex $w\in \overline{S}\setminus\{v\}$, then $G+vw$ contains a subgraph isomorphic to $K_p\cup K_q\cup K_r$ and the edge $vw$ lies in the copy of $K_p$. Hence, $G$ has a subgraph $H_{vw}= H_{vw,1}\cup H_{vw,2}$ with $H_{vw,1}\cong K_q$ and $H_{vw,2}\cong K_r$. Let $V(H_{vw,1})=\left\{u_1, u_2, \cdots, u_q\right\}$ and $V(H_{vw,2})=\left\{v_1, v_2, \cdots, v_r\right\}$. For any vertex $u_i$, the graph $G+vu_i$ contains a copy of $K_p\cup K_q\cup K_r$ and $vu_i$ lies in a copy of $K_p$. Hence, $G$ has a subgraph $H_{vu_i}= H_{vu_i,1}\cup H_{vu_i,2}$ with $H_{vu_i,1}\cong K_q$ and $H_{vu_i,2}\cong K_r$.

\begin{claim}\label{nothing}
$|V(H_{vu_i,1})\cap V(H_{vw,2})|=0$ for $1\leqslant i\leqslant q$.
\end{claim}
Write $V(H_{vu_1,i})\cap V(H_{vw,j})=L_{ij}$ and $|L_{ij}|=\ell_{ij}$ for $1\leqslant i,j\leqslant 2$.
Then we have
\begin{align}\label{S-}
e(G[\overline{S}])\geqslant & \binom{q}{2}+\binom{r}{2}+\binom{q-\ell_{11}-\ell_{12}}{2}+\binom{r-\ell_{21}-\ell_{22}}{2} \notag\\
 &+(q-\ell_{11}-\ell_{12})(\ell_{11}+\ell_{12})+(r-\ell_{21}-\ell_{22})(\ell_{21}+\ell_{22})+\ell_{11}\ell_{12}+\ell_{21}\ell_{22} \notag\\
 =&\binom{q}{2}+\binom{r}{2}+f(\ell_{11},\ell_{12},\ell_{21},\ell_{22}),
\end{align}
where
\begin{align*}
f(\ell_{11},\ell_{12},\ell_{21},\ell_{22})=&\frac{1}{2}\left[(q^2-q)+(r^2-r)-(\ell_{11}+\ell_{12})^2-(\ell_{21}+\ell_{22})^2\right.\\
 & \left.+2(\ell_{11}\ell_{12}+\ell_{21}\ell_{22})+(\ell_{11}+\ell_{12}+\ell_{21}+\ell_{22})\right].
\end{align*}

If $\ell_{11}+\ell_{12}\leqslant q-2$ or $\ell_{21}+\ell_{22}\leqslant r-2$, then there is an edge, say $xy$, in $G[V(H_{vu_1})\setminus V(H_{vw})]$. Then $G[S\cup\{x,y\}]\cup H_{vw}$ is a subgraph isomorphic to $K_p\cup K_q\cup K_r$ in $G$, which is a contradiction. Hence,  $\ell_{11}+\ell_{12}\geqslant q-1$ and $\ell_{21}+\ell_{22}\geqslant r-1$. Then we have
\begin{equation}\label{k1112}
\left\{
\begin{array}{ll}
      q-1\leqslant\ell_{11}+\ell_{12}\leqslant q; & \hbox{} \\
      r-1\leqslant\ell_{21}+\ell_{22}\leqslant r; & \hbox{} \\
      \ell_{11}+\ell_{21}\leqslant q-1; & \hbox{} \\
      \ell_{12}+\ell_{22}\leqslant r. & \hbox{}
    \end{array}
\right.
\end{equation}

Without loss of generality, it suffices to show $\ell_{12}=0$ to prove Claim \ref{nothing}. We will complete the proof by obtaining contradictions when $\ell_{12}\leqslant 2$ and $\ell_{12}=1$.

\noindent\textbf{Case 1.} $\ell_{12}\geqslant 2$.

Then $\ell_{11}\leqslant q-2$ and $\ell_{22}\leqslant r-2$. Since $\ell_{11}+\ell_{12}+\ell_{21}+\ell_{22}\leqslant q+r-1$, the case $\ell_{11}+\ell_{12}=q$ and $\ell_{21}+\ell_{22}=r$ does not appear. In this case, we distinguish three subcases according to $\ell_{11}+\ell_{12}$ and $\ell_{21}+\ell_{22}$.

\noindent\textbf{Subcase 1.1.} $\ell_{11}+\ell_{12}= q-1$ and $\ell_{21}+\ell_{22}= r-1$.

In this subcase, $f(\ell_{11},\ell_{12},\ell_{21},\ell_{22})=\ell_{11}\ell_{12}+\ell_{21}\ell_{22}+q+r-2$. The assumption $\ell_{12}\geqslant 2$ implies that $q\geqslant 3$ and $r\geqslant 5$. If $\ell_{21}=0$, then $\ell_{22}=r-1$, which is a contradiction to $\ell_{22}\leqslant r-2$. If $\ell_{22}=0$, then $\ell_{21}=r-1$, which is a contradiction to $\ell_{21}\leqslant q-1$. Hence, we have $\ell_{21}$, $\ell_{22}\geqslant 1$.
Combining (\ref{S-}), we have
\begin{align*}
e(G[\overline{S}]) &\geqslant \binom{q}{2}+\binom{r}{2}+f(\ell_{11},\ell_{12},\ell_{21},\ell_{22})\\
&\geqslant \binom{q}{2}+\binom{r}{2}+f(0,\,q-1,\,1,\,r-2)\\
&= \binom{q}{2}+\binom{r}{2}+q+2r-4\\
& >\binom{q+1}{2}+\binom{r+1}{2},
\end{align*}
which is a contradiction to (\ref{qr}).


\noindent\textbf{Subcase 1.2.} $\ell_{11}+\ell_{12}=q$ and $\ell_{21}+\ell_{22}= r-1$.


In this subcase, $f(\ell_{11},\ell_{12},\ell_{21},\ell_{22})=\ell_{11}\ell_{12}+\ell_{21}\ell_{22}+r-1$. If $\ell_{11}\geqslant1$, then $q=\ell_{11}+\ell_{12}\geqslant 3$ and $r\geqslant 5$. Hence, we have
\begin{align*}
e(G[\overline{S}]) &\geqslant\binom{q}{2}+\binom{r}{2}+f(\ell_{11},\ell_{12},\ell_{21},\ell_{22})\\
&\geqslant\binom{q}{2}+\binom{r}{2}+f(1,\,q-1,\,1,\,r-2)\\
&=\binom{q}{2}+\binom{r}{2}+q+2r-4\\
& >\binom{q+1}{2}+\binom{r+1}{2}.
\end{align*}
This is a contradiction to (\ref{qr}).

If $\ell_{11}=0$, then $\ell_{12}=q$. By (\ref{k1112}), we know that $0\leqslant \ell_{21}\leqslant q-1$ and $0\leqslant \ell_{22}\leqslant r-q$. Combining the assumption $\ell_{21}+\ell_{22}= r-1$, we have $\ell_{21}=q-1$ and $\ell_{22}=r-q$. Then we have
\begin{align*}
e(G[\overline{S}])& \geqslant\binom{q}{2}+\binom{r}{2}+f(\ell_{11},\ell_{12},\ell_{21},\ell_{22})\\
&=\binom{q}{2}+\binom{r}{2}+f(0,\,q,\,q-1,\,r-q)\\
&=\binom{q+1}{2}+\binom{r+1}{2}-q^2+(q-1)r-1.
\end{align*}

If $q\geqslant 4$, we have $-q^2+(q-1)r-1>0$.
Then,
$$e(G[\overline{S}])>\binom{q+1}{2}+\binom{r+1}{2}.$$

If $q=3$, then we have $r\geqslant 5$ and
$$e(G[\overline{S}])\geqslant\binom{4}{2}+\binom{r+1}{2}+2r-10.$$
Without loss of generality, assume that $L_{12}=\{v_1,v_2,v_3\}$. There exists some vertex, say $v_1$, in $\{v_1,v_2,v_3\}$ such that $u_1\nsim v_1$ and $N(u_1)\cap N(v_1)=S$. Otherwise,
$$e(G[\overline{S}])\geqslant\binom{4}{2}+\binom{r+1}{2}+2r-7>\binom{4}{2}+\binom{r+1}{2},$$
which is a contradiction.
Considering the graph $G_1=G+u_1v_1$, we have $K_p\cup K_3\cup K_r\subseteq G_1$ and $G_1[S\cup \{u_1,v_1\}]\cong K_p$. Note that $|S\cup V(H_{vw})\cup V(H_{vu_1})|=p+r+2$. So, there is a vertex $x$ in $\overline{S}\setminus(V(H_{vw})\cup V(H_{vu_1}))$, and $x$ has at least $q-1=2$ neighbors in $V(H_{vw})\cup V(H_{vu_1})$. Hence,
$$e(G[\overline{S}])\geqslant\binom{4}{2}+\binom{r+1}{2}+2r-8>\binom{4}{2}+\binom{r+1}{2},$$
which is a contradiction.

If $q=2$, we may use the similar arguments as that of the case $q=3$ to obtain a contradiction.

\noindent\textbf{Subcase 1.3.} $\ell_{11}+\ell_{12}= q-1$ and $\ell_{21}+\ell_{22}=r$.

In this subcase, $f(\ell_{11},\ell_{12},\ell_{21},\ell_{22})=\ell_{11}\ell_{12}+\ell_{21}\ell_{22}+q-1$. By the assumption $\ell_{12}\geqslant 2$, then $q\geqslant 3$ and $r\geqslant 5$. Note that $\ell_{22}\leqslant r-2$. Then we have
\begin{align*}\label{b}
e(G[\overline{S}])& \geqslant\binom{q}{2}+\binom{r}{2}+f(\ell_{11},\ell_{12},\ell_{21},\ell_{22})\\
&\geqslant\binom{q}{2}+\binom{r}{2}+f(0,\,q-1,\,2,\,r-2)\\
&=\binom{q}{2}+\binom{r}{2}+q+2r-5\notag
\end{align*}
Note that there are nonadjacent vertices $u_i$ and $v_j$ for some $u_i\in V(H_{vw,1})$ and some $v_j\in V(H_{vw,2})$. Otherwise, we have
$$e(G[\overline{S}])>\binom{q}{2}+\binom{r}{2}+q+2r-5\geqslant\binom{q+1}{2}+\binom{r+1}{2},$$
which is a contradiction. Considering the graph $G+u_iv_j$, we have $K_p\cup K_q\cup K_r\subseteq G+u_iv_j$. Note that $|S\cup V(H_{vw})\cup V(H_{vu_1})|=p+q+r-1$. So, there is a vertex $x\in \overline{S}\setminus(V(H_{vw})\cup V(H_{vu_1}))$, and $x$ has at least one neighbor in $V(H_{vw})\cup V(H_{vu_1})$. Hence, we have
$$e(G[\overline{S}])>\binom{q+1}{2}+\binom{r+1}{2},$$
which is a contradiction.

\noindent\textbf{Case 2.} $\ell_{12}=1$.

Without loss of generality, assume that $L_{12}=\{v_1\}$. By (\ref{k1112}), we have $q-2 \leqslant\ell_{11}\leqslant q-1$. We distinguish two subcases according to $\ell_{11}$.

\noindent\textbf{Subcase 2.1.} $\ell_{11}=q-2$

Assume that $L_{11}=\{u_2,\cdots,u_{q-1}\}$. In this subcase, we have $0\leqslant\ell_{21}\leqslant 1$ by (\ref{k1112}). If $\ell_{21}=0$, then $u_1,u_q\notin V(H_{vu_1})$. Hence, $G[S\cup\{u_1,u_q\}]\cup H_{vu_1}$ is a subgraph of $G$ isomorphic to $K_p\cup K_q\cup K_r$, which is a contradiction. If $\ell_{21}=1$, then $L_{21}=\{u_q\}$. Furthermore, we have $r-2\leqslant\ell_{22}\leqslant r-1$ by (\ref{k1112}).

If $\ell_{22}=r-2$, then we have
\begin{align*}
e(G[\overline{S}]) & \geqslant\binom{q}{2}+\binom{r}{2}+f(\ell_{11},\ell_{12},\ell_{21},\ell_{22})\\
&=\binom{q}{2}+\binom{r}{2}+f(q-2,\,1,\,1,\,r-2)\\
&=\binom{q}{2}+\binom{r}{2}+2q+2r-6.
\end{align*}
Suppose that $L_{22}=\{v_2,\cdots,v_{r-1}\}$. Since $\ell_{11}+\ell_{12}=q-1$ and $\ell_{21}+\ell_{22}=r-1$, there are vertices $x$, $y\in \overline{S}\setminus V(H_{vw})$ such that $G[\{u_2,\cdots,u_{q-1},v_1,x\}]\cong K_q$ and $G[\{v_2,\cdots,v_{r-1},u_q,y\}]\cong K_r$. Note that there is some vertex, say $v_2$, in $\{v_2,\cdots,v_{r-1},y\}$ such that $u_1\nsim v_2$ and $N(u_1)\cap N(v_2)=S\cup \{u_q\}$. Otherwise, we have
$$e(G[\overline{S}])>\binom{q}{2}+\binom{r}{2}+2q+2r-6\geqslant\binom{q+1}{2}+\binom{r+1}{2},$$
which is a contradiction. Considering the graph $G+u_1v_2$, the edge $u_1v_2$ is not contained in a clique of order $r$ since $N(u_1)\cap N(v_2)=S\cup \{u_q\}$ and $|S\cup\{u_1,v_2,u_q\}|=p+1<r$. Moreover, we have $G[\{u_2,\cdots,u_{q-1},v_1,x\}]\cong K_q$. Note that $G[\{v_1,v_2,\cdots,v_r\}]$ and $G[\{u_q,v_2,\cdots,v_{r-1},y\}]$ are the two cliques of order $r$ in $G[\overline{S}\setminus\{u_1\}]$. In fact, $G[\{v_1,v_2,\cdots,v_r\}]$ and $G[\{u_q,v_2,\cdots,v_{r-1},y\}]$ are the only two cliques of order $r$ in $G[\overline{S}\setminus\{u_1\}]$. Otherwise, there is at least one vertex different from $\{v_1,v_2,\cdots,v_r,u_1,u_q,y\}$ contained in a subgraph $K_r$. Then, we have
$$e(G[\overline{S}])\geqslant \binom{q}{2}+\binom{r}{2}+(2q+2r-6)+(r-1)> \binom{q+1}{2}+\binom{r+1}{2},$$
which is a contradiction. Hence, $G[\{v_1,v_2,\cdots,v_r\}]$ and $G[\{u_q,v_2,\cdots,v_{r-1},y\}]$ are the only two cliques of order $r$ in $G[\overline{S}\setminus\{u_1\}]$, and both contain the vertex $v_2$. Hence, there is not a clique of order $r$ in $G[\overline{S}\setminus\{u_1,v_2\}]$. Thus, there is not a $K_p\cup K_q\cup K_r$ in $G+u_1v_2$, which is a contradiction.

If $\ell_{22}=r-1$, we have
\begin{align*}
e(G[\overline{S}])& \geqslant \binom{q}{2}+\binom{r}{2}+f(\ell_{11},\ell_{12},\ell_{21},\ell_{22})\\
&=\binom{q}{2}+\binom{r}{2}+f(q-2,\,1,\,1,\,r-1)\\
&=\binom{q}{2}+\binom{r}{2}+2q+r-4.
\end{align*}
The assumption $\ell_{22}=r-1$ implies that $L_{22}=\{v_2,\cdots,v_r\}$. Since $\ell_{11}+\ell_{12}=q-1$, there is a vertex $x\in \overline{S}\setminus V(H_{vw})$ such that $G[\{u_2,\cdots,u_{q-1},v_1,x\}]\cong K_q$. Note that there is some vertex in $\{v_2,\cdots,v_r\}$, say $v_2$, such that $u_q\nsim v_2$ and $N(x)\cap N(v_2)=S\cup \{v_1\}$. Otherwise, we have
$$e(G[\overline{S}])\geqslant \binom{q}{2}+\binom{r}{2}+(2q+r-4)+(r-1)>\binom{q+1}{2}+\binom{r+1}{2},$$
which is a contradiction. Considering the graph $G+xv_2$, the edge $xv_2$ is not contained in a clique of order $r$ since $N(x)\cap N(v_2)=S\cup \{v_1\}$ and $|S\cup \{x,v_1,v_2\}|=p+1<r$. Moreover, we have $G[\{u_1,u_2,\cdots,u_q\}]\cong K_q$. Note that $G[\{v_1,v_2,\cdots,v_r\}]$ and $G[\{u_q,v_2,\cdots,v_r\}]$ are the only two cliques of order $r$ in $G[\overline{S}\setminus \{x\}]$. Otherwise, we have
$$e(G[\overline{S}])\geqslant \binom{q}{2}+\binom{r}{2}+(2q+r-4)+(r-1)>\binom{q+1}{2}+\binom{r+1}{2},$$
which is a contradiction. Hence, there is not a clique of order $r$ in $G[\overline{S}\setminus\{x,v_2\}]$. Thus, there is not a $K_p\cup K_q\cup K_r$ in $G+u_1v_2$, which is a contradiction..

\noindent\textbf{Subcase 2.2.} $\ell_{11}=q-1$.

By (\ref{k1112}), we have $\ell_{21}=0$ and $\ell_{22}=r-1$. Furthermore, we have
\begin{align*}
e(G[\overline{S}])&\geqslant\binom{q}{2}+\binom{r}{2}+f(\ell_{11},\ell_{12},\ell_{21},\ell_{22})\\
&=\binom{q}{2}+\binom{r}{2}+f(1,\,q-1,\,0,r-1)\\
&=\binom{q}{2}+\binom{r}{2}+q+r-2.
\end{align*}
Noting that $v_1\sim u_i$ ($2\leqslant i\leqslant q$), then we have $G[\{v_1,u_2,\cdots,u_q\}]\cong K_q$ and $G[\{v_2,\cdots,v_r\}]\cong K_{r-1}$. Since $\ell_{21}+\ell_{22}=r-1$, there is a vertex, say $y\in \overline{S}\setminus V(H_{vw})$ contained in a clique of order $r$ in $G$. Note that there is some vertex in $\{v_2,\cdots,v_r,y\}$, say $v_2$, such that $u_1\nsim v_2$ and $N(u_1)\cap N(v_2)=S$. Otherwise, each vertex in $\{v_2,\cdots,v_r,y\}$ either is adjacent to $u_1$ or has a common neighbor with $u_1$ in $G[\overline{S}]$. Then, we always have
$$e(G[\overline{S}])\geqslant\binom{q}{2}+\binom{r}{2}+(q+r-2)+r>\binom{q+1}{2}+\binom{r+1}{2}.$$
This is a contradiction.

Considering the graph $G_2=G+u_1v_2$, then $G_2[S\cup\{u_1,v_2\}]\cong K_p$. If $y$ is not adjacent to $v_1$, then $G[\{v_2,\cdots,v_r,y\}]\cong K_r$. Note that $G[\{v_1,v_2,\cdots,v_r\}]$ and $G[\{v_2,\cdots,v_r,y\}]$ are the only two cliques of order $r$ in $G[\overline{S}\setminus \{u_1\}]$. Otherwise, we have
$$e(G[\overline{S}])\geqslant\binom{q}{2}+\binom{r}{2}+(q+r-2)+(r-1)>\binom{q+1}{2}+\binom{r+1}{2}.$$
This is a contradiction. Hence, the graph $G_2$ does not contain a $K_r$ as a subgraph, which is a contradiction.

If $y$ is adjacent to $v_1$, then $G[\{v_1,v_2,\cdots,v_r,y\}]\cong K_{r+1}$ and
$$e(G[\overline{S}])\geqslant\binom{q}{2}+\binom{r}{2}+(q+r-2)+1.$$
Note that $G[\{v_1,v_3,\cdots,v_r,y\}]$ is only clique of order $r$ in $G[\overline{S}\setminus\{v_2\}]$. Otherwise, we have
$$e(G[\overline{S}])\geqslant\binom{q}{2}+\binom{r}{2}+(q+r-1)+(r-1)>\binom{q+1}{2}+\binom{r+1}{2},$$
which is a contradiction. Hence, there is a vertex $x$ in $\overline{S}\setminus(V(H_{vw})\cup\{y\})$ such that $G[\{u_2,\cdots,u_q,x\}]\cong K_q$. Furthermore, we have
$$e(G[\overline{S}])\geqslant \binom{q}{2}+\binom{r}{2}+(q+r-1)+(q-1)\geqslant \binom{q+1}{2}+\binom{r+1}{2}.$$
Note that $u_2\nsim v_j$ and $u_2\nsim y$ for $2\leqslant j\leqslant r$. Otherwise, we have
$$e(G[\overline{S}])>\binom{q+1}{2}+\binom{r+1}{2},$$
which is a contradiction. Considering the graph $G_3=G+u_2v_2$, the edge $u_2v_2$ is not contained in a clique of order $r$ since $N(u_2)\cap N(v_2)=S\cup \{v_1\}$ and $|S\cup\{u_2,v_1,v_2\}|=p+1<r$. If $G_3[S\cup \{u_2,v_1,v_2\}]\cong K_{p+1}$, then there is a subgraph $K_r$ distinct from $G[\{v_1,v_3,\cdots,v_r,y\}]$ in $G[\overline{S}\setminus\{v_2\}$], which is a contradiction. Then $G_3[S\cup \{u_2,v_2\}]\cong K_p$ and $G[\{v_1,v_3,\cdots,v_r,y\}]\cong K_r$ in the subgraph $K_p\cup K_q\cup K_r$ in $G_3$. Note that $N(x)=S\cup \{u_2,\cdots,u_q\}$ and the vertices in $S\cup \{u_2\}$ are contained in the subgraph $K_p\cup K_q\cup K_r$, then $x$ is not contained in the subgraph $K_p\cup K_q\cup K_r$. Note that $|S\cup V(H_{vw})\cup V(H_{vu_1})|=p+q+r-1$,
then there is a vertex $x'$ in $\overline{S}\setminus (V(H_{vw})\cup\{x,y\})$, and $x'$ has at least $q-1$ neighbors in $V(H_{vw})\cup V(H_{vu_1})$. Hence, we have
$$e(G[\overline{S}])\geqslant \binom{q+1}{2}+\binom{r+1}{2}+(q-1)>\binom{q+1}{2}+\binom{r+1}{2},$$
which is a contradiction. So, we have proved $\ell_{12}=0$.

Similarly, we may prove $|V(H_{vv_j,2})\cap V(H_{vw,1})|=0$ for any vertex $v_j\in V(H_{vw,2})$.

\begin{claim}\label{q-1}
$|V(H_{vu_i,1})\cap V(H_{vw,1})|=q-1$ for $1\leqslant i\leqslant q$.
\end{claim}
Without loss of generality, we will prove $|V(H_{vu_1,1})\cap V(H_{vw,1})|=q-1$. Since $u_1\in V(H_{vw,1})$ and $u_1\notin V(H_{vu_1,1})$, $|V(H_{vu_1,1})\cap V(H_{vw,1})|\leqslant q-1$ holds. If $|V(H_{vu_1,1})\cap V(H_{vw,1})|<q-1$, then there is an edge, say $xy$, in $G[V(H_{vu_1,1})\setminus V(H_{vw,1})]$. Hence, $G[S\cup\{x,y\}]\cup H_{vw}$ is a subgraph isomorphic to $K_p\cup K_q\cup K_r$ in $G$. This is a contradiction. Hence, $|V(H_{vu_1,1})\cap V(H_{vw,1})|=q-1$. Similarly, we may prove $|V(H_{vv_j,2})\cap V(H_{vw,2})|=r-1$ for any $v_j\in V(H_{vw,2})$.

Let $V(H_{vu_i,1})\setminus V(H_{vw,1})=\{x_i\}$ and $V(H_{vv_j,2})\setminus V(H_{vw,2})=\{y_j\}$ hold for any $u_i\in V(H_{vw,1})$ and $v_j\in V(H_{vw,2})$. Suppose that there are two different $x_i$s', say $x_1$, $x_2$. Without loss of generality, we assume that $V(H_{vu_1,1})\setminus V(H_{vw,1})=\{x_1\}$ and $V(H_{vu_2,1})\setminus V(H_{vw,1})=\{x_2\}$. Then $u_1x_2\in E(G)$. Furthermore, the subgraph $G[S\cup \{u_1,x_2\}]\cup H_{vu_1}$ of $G$ is isomorphic to $K_p\cup K_q\cup K_r$, which is a contradiction. Hence, there is a unique vertex $x$ such that $G[V(H_{vw,1})\cup\{x\}]\cong K_q$. Similarly, there is a unique vertex $y$ such that $G[V(H_{vw,2})\cup\{y\}]\cong K_r$.

Assume $x=y$. If $u_i\nsim v_j$ for some $u_i\in V(H_{vw,1})$ and $v_j\in V(H_{vw,2})$, then we consider the graph $G+u_iv_j$. Furthermore, we have $K_p\cup K_q\cup K_r\subseteq G+u_iv_j$. Note that $|S\cup V(H_{vw})\cup V(H_{vu_1})|=p+q+r-1$. So, there is a vertex $x'$ in $\overline{S}\setminus(V(H_{vw})\cup \{x\})$, and $x'$ has at least one neighbor in $V(H_{vw})\cup V(H_{vu_1})$ then we have
$$e(G[\overline{S}])\geqslant\binom{q}{2}+\binom{r}{2}+q+r+1>\binom{q+1}{2}+\binom{r+1}{2},$$
which is a contradiction. Hence, $u_i\sim v_j$ for any $u_i\in V(H_{vw,1})$ and $v_j\in V(H_{vw,2})$. Then
$$e(G[\overline{S}])\geqslant\binom{q}{2}+\binom{r}{2}+q+r+qr>\binom{q+1}{2}+\binom{r+1}{2},$$
which is a contradiction. Hence $x\neq y$, and
$$e(G[\overline{S}])\geqslant\binom{q}{2}+\binom{r}{2}+q+r=\binom{q+1}{2}+\binom{r+1}{2}.$$
By (\ref{qr}), we have
$$e(G[\overline{S}])=\binom{q+1}{2}+\binom{r+1}{2}.$$
Furthermore, $u_i\nsim v_j$ for any $u_i\in V(H_{vw,1})$ and $v_j\in V(H_{vw,2})$.
So,
$$G=G[S]\vee\left(G[V(H_{vw,1})\cup\{x\}]\cup G[V(H_{vw,2})\cup\{y\}]\cup I_{n-p-q-r}\right)\cong H(n;p,q,r).$$
Then, we have
\begin{align*}
e(G)=e(H(n;p,q,r))=(p-2)(n-p+2)+\binom{p-2}{2}+\binom{q+1}{2}+\binom{r+1}{2}.
\end{align*}
\end{proof}

\section{Conclusion}
Combining the results of Theorems \ref{ptq}, \ref{p2q}, and \ref{pqr}, the unsettled case of Problem 1.1 is to investigate $sat(n,K_p\cup K_q\cup K_{q+1})$ ($2\leqslant p\leqslant q$). Let $G$ be a $K_p\cup K_q\cup K_{q+1}$-saturated graph with $e(G)=sat(n,K_p\cup K_q\cup K_{q+1})$. It is easy to check that $H(n;p,2q+1)$ is a $K_p\cup K_q\cup K_{q+1}$-saturated graph. So we have
$e(G)\leqslant e(H(n;p,2q+1))$. By using the similar arguments as that of Lemma \ref{detla}, we may give an elementary characterization of the extremal graph $G$.
\begin{lemma}\label{pqq+1}
Suppose that $G$ is a $K_p\cup K_q\cup K_{q+1}$-saturated graph with $e(G)=sat(n,K_p\cup K_q\cup K_{q+1})$ and $n>2(q+1)^2+3(p-2)$. Then, we have

\noindent(1) $\delta(G)=p-2$,

\noindent(2) $G[N(v)]\cong K_{p-2}$,
$$e(G[N(v),\overline{N(v)}])=(p-2)(n-p+2),$$
and
$$e(G[\overline{N(v)}])\leqslant\binom{2q+2}{2},$$
where $v$ is a vertex of minimum degree in $V(G)$.
\end{lemma}

Combining the related results of \cite{PAJ}, \cite{LZ}, and \cite{RMRM}, Theorems \ref{ptq}, \ref{p2q}, and \ref{pqr}, about the saturation number for unions of cliques, we may list the related results in Table 1.
\begin{figure}[h]
  \centering
  \caption*{Table 1: The saturation number for unions of cliques.}
  \includegraphics[width=1.05\textwidth]{table1.pdf}
\end{figure}

From Table 1, we know when $t=1,\,2,\,3$, $H(n;p,\underbrace{q,\cdots,q}_{t-1})$ is the unique extremal $K_p\cup (t-1)K_q$-saturated graph with minimum edges. For general case, we give the following conjecture.

\begin{conjecture}
Suppose $2\leqslant p\leqslant q$, $n>(t-1)(q+1)^2+3(p-2)$ and $t\geqslant 4$.
$H(n;p,q,\cdots,q)$ is the unique extremal $K_p\cup (t-1)K_q$-saturated graph with minimum edges.
\end{conjecture}

\begin{thebibliography}{99}

\bibitem{C} Y. Chen, Minimum $C_5$-saturated graphs. J. Graph Theory 61 (2009) 111-126.

\bibitem{CFFGJM} G. Chen, J. Faudree, R. Faudree, et al., Results and problems on saturation numbers for linear forests. Bull. Inst. Combin. Appl. 75 (2015) 29-46.

\bibitem{CLLYZ} S. Cao, H. Lei, X. Lian, et al., Saturation numbers for $tP_k$ with $k$ less than 6. Discrete Appl. Math. 325 (2023) 108-119.

\bibitem{PAJ} P. Erd\H os, A. Hajnal, J. W. Moon, A problem in graph theory. Amer. Math. Monthly 71 (1964) 1107-1110.

\bibitem{JRJ} J. Faudree, R. Faudree, J. Schmitt, A survey of minimum saturated graphs. Electron. J. Combin. 18 (2011), Dynamic Survey 19, 36 pages.

\bibitem{RMRM} R. Faudree, M. Ferrara, R. Gould, et al., $tK_p$-saturated graphs of minimum size. Discrete Math. 309 (2009) 5870-5876.

\bibitem{HL} Z. He, M. Lu, Saturation number of $tK_{\ell,\ell,\ell}$ in the complete tripartite graph. Electron. J. Combin. 28 (2021) 4-20.

\bibitem{LZ} L. K\'asonyi, Z. Tuza, Saturated graphs with minimal number of edges. J. Graph Theory 10 (1986) 203-210.

\bibitem{LGS} T. {\L}uczak, R. Gould, J. Schmitt, Constructive upper bounds for cycle saturated graphs of minimum size. Electron. J. Combin. 13 (2006) R29.

\bibitem{LSWZ} Y. Lan, Y. Shi, Y. Wang, et al., The saturation number of $C_6$. https://arxiv.org/abs /2108.03910.

\bibitem{O} L. T. Ollmann, $K_{2,2}$-saturated graphs with a minimal number of edges. in: Proc. 3rd Southeastern Conference on Combinatorics, Graph Theory and Computing, (1972) 367-392.

\bibitem{PS} O. Pikhurko, J. Schmitt, A note on minimum $K_{2,3}$-saturated graphs. Australasian J. Combin. 40 (2008) 211-215.


\end{thebibliography}


\end{document} 