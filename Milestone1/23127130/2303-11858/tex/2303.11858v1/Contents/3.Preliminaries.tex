\paragraph{Knowledge Graph}
 A KG $\mathcal{G} \subseteq \mathcal{E} \times \mathcal{R} \times \mathcal{E}$, where $\mathcal{E}$ and $\mathcal{R}$ represent the set of entities and relations respectively, can be defined as a set of subject-predicate-object $\bigl\langle s,p,o \bigr\rangle$ triples. For each triple $\bigl\langle e_i,r,e_j \bigr\rangle$, $e_{i,j} \in \mathcal{E}$ and $r \in \mathcal{R}$, it exists if and only if $e_i$ is linked to $e_j$ by relation $r$.

% \paragraph{Relation patterns}
% Going beyond triple representation, KGs also exhibit relational patterns. We introduce the three most common patterns as follows:\\
% 1. Symmetry/anti-symmetry: 
% A relation $r$ is symmetric (anti-symmetric) if
% \(\forall x, y: r(x,y) \rightarrow r(y,x) (r(x,y) \rightarrow \neg r(y,x) )\)\\
% 2. Inversion: relation $r_1$ is inverse to relation $r_2$ if
% \(\forall x, y: r_{1}(x,y) \rightarrow r_{2}(y,x)\)\\
% 3. Composition: relation $r_3$ is  composition of relation $r_1$ and $r_2$ if 
% \(\forall x, y, z: r_{1}(x,y) \land r_{2}(y,z) \rightarrow r_{3}(x,z)\)


% first order logical queries
\paragraph{First-Order Logical Queries involving constants}
First-Order Logical Queries are broad and here we consider answering a subset, i.e., 
multi-hop queries with constants and first-order logical operations including conjunction ($\land$), disjunction ($\lor$), existential quantification ($\exists$), and negation ($\neg$). 
The query consists of a set of constants (anchor entities) $\mathcal{E}_a \subset \mathcal{E}$, a set of existentially quantified bound variables $V_1,..., V_m$ and a single target answer variable $V_?$. The disjunctive normal form of this subset of FOL queries is namely the disjunction of conjunctive formulas, and can be expressed as
\begin{equation}
    q[V_t] = V_t .\exists V_1,...,V_m: c_1 \lor c_2 \lor ... \lor c_n
\end{equation}
where $c_i$, $i\in \{1,...,n\}$ corresponds to a conjunctive query with one or more atomic queries $d$ i.e. \(c_i= d_{i1} \land d_{i2}\land ... \land d_{im}\). For each atomic formula, $d_{ij}$ = $(e_a,r,V)$ or $\neg (e_a,r,V)$ or $(V^{'},r,V)$ or $\neg (V^{'},r,V)$, where $e_a \in \mathcal{E}_a$, $V \in \{V_t, V_1, ..., V_k\}$, $V^{'} \in \{V_1, ..., V_k\}$, $r\in \mathcal{R}$. 
The goal of logical query embedding is to find a set of answer entities 
$\{e_{t1},e_{t2},...\}$ for $V_t$, such that $q[V_t]=\operatorname{True}$.


% Logical operators
% \paragraph{Logical Operators}
% As illustrated in Figure \ref{fig:FOL query example}, each logical query can be represented as a directed acyclic graph (DAG) tree, where the tree nodes correspond to constants/anchor node entities or variables, and the edges correspond to atom relations or logical operations in a query. Logical operations are performed along the DAG tree from constants to the target answer variable. Following \citep{BetaE}, the logical operators can be defined as:\\
% 1. Relational Projector: Given a set of entities $\mathcal{S} \subset \mathcal{E}$ and relation $r \in \mathcal{R}$, selects the neighbouring entities $\mathcal{S}^{'}\subset\mathcal{E}$ such that $\mathcal{S}^{'} = \{e \in \mathcal{S}, e^{'} \in \mathcal{S}^{'}: r(e, e^{'}) = True\}$\\
% 2. Intersector: For sets of entities $\{S_{1},...,S_{n}\}$, their intersection is $\cap_{i=1}^{n} S_{i}$\\
% 3. Union: Given sets of entities $\{S_{1},...,S_{n}\}$, their union is $\cup_{i=1}^{n} S_{i}$\\
% 4. Negation: Given a set of entities $\mathcal{S} \subset \mathcal{E}$, its negation can be defined as $\bar{S} = \mathcal{E}\setminus\mathcal{S}$
