
% SIAM Article Template
\documentclass{article}

% Information that is shared between the article and the supplement
% (title and author information, macros, packages, etc.) goes into
% ex_shared.tex. If there is no supplement, this file can be included
% directly.

\usepackage[utf8]{inputenc}
\usepackage[margin=1.5in]{geometry}
\usepackage{amsmath}
\usepackage{amsfonts}
\usepackage{amssymb}
\usepackage{mathrsfs}
\usepackage{subcaption}
\usepackage{amsthm}

\usepackage{mathtools}
\usepackage{natbib}
\usepackage[pdftex]{graphicx}
%\usepackage{listings}
\usepackage{color}
\usepackage{longtable}
\usepackage{tikz}
\usepackage{comment}
\usepackage{bbm}
\usepackage{tabularx}
\usepackage{multirow}
\usepackage{url}
%\usepackage{ntheorem}

\definecolor{codegreen}{rgb}{0,0.6,0}
\definecolor{codegray}{rgb}{0.5,0.5,0.5}
\definecolor{codepurple}{rgb}{0.58,0,0.82}
\definecolor{backcolour}{rgb}{0.95,0.95,0.92}

%\usepackage{draftwatermark}
%\usepackage[printfigures]{figcaps}
\newtheorem{thm}{Theorem}[subsection]
\newtheorem{defn}[thm]{Definition}
\newtheorem{lem}{Lemma}
%\newtheorem{lem*}[theorem]{Lemma}
%\newtheorem{theorem}{Theorem}[section]
%\newtheorem{thm}[theorem]{Theorem}
%\newtheorem{lemma*}[theorem]{Lemma}
%\newtheorem{proposition}[theorem]{Proposition}\
%\newtheorem{corollary}[theorem]{Corollary}
%\newtheorem{defn}[theorem]{Definition}
%\newtheorem{example}[theorem]{Example}
%\newtheorem{remark}[theorem]{Remark}
%\newtheorem{proof}[theorem]{Proof}
%\newtheorem*{theorem*}{Theorem}

\newcommand{\act}{\sigma}
\newcommand{\be}{\begin{equation}}
\newcommand{\ee}{\end{equation}}
\def \bsp {\begin{split}}
\def \esp {\end{split}}
\def \bea {\begin{eqnarray}}
\def \eea {\end{eqnarray}}
\DeclareMathOperator*{\argmin}{argmin}
\DeclareMathOperator*{\mini}{minimize}


\title{A Unified Framework for Fast Large-Scale Portfolio Optimization}
 
% \author{Weichuan Deng\thanks{Stony Brook University, Department of Applied Mathematics and Statistics, Email: weichuan.deng@stonybrook.edu}\and Ronakdilip Shah\thanks{Stony Brook University, Department of Applied Mathematics and Statistics; Institute for Advanced Computational Science} \and Pawe\l\ Polak \textsuperscript{a}\footnote{\textsuperscript{\dag}Corresponding author at: Department of Applied Mathematics and Statistics, and Institute for Advanced Computational Science, Stony Brook University, Stony Brook, New York, United States. E-mail address: \url{pawel.polak@stonybrook.edu}.}\thanks{Stony Brook University, Department of Applied Mathematics and Statistics} \and Abolfazl Safikhani \thanks{George Mason University, Department of Statistics}
%     }

\author{\vspace{-5mm}{Weichuan Deng}$^{a}$ \hspace*{2mm}  %\addtocounter{footnote}{-2}
{Ronakdilip Shah}$^{a}$\hspace*{2mm}
%\addtocounter{footnote}{+1}
{Pawe\l \ Polak}$^{a, c}$\footnote{Corresponding author at Department of Applied Mathematics and Statistics at Stony Brook University, United States. E-mail address: \texttt{pawel.polak@stonybrook.edu}.}\hspace*{2mm}
{Abolfazl Safikhani}$^{b}$
\\[5mm]
$^{a}$\textit{\vspace{-2mm} \small Department of Applied Mathematics and Statistics, Stony Brook University, United States}\\[2mm]
$^{b}$\textit{\vspace{-2mm} \small Department of Statistics, George Mason University, United States}\\[2mm]
$^{c}$\textit{\vspace{-2mm} \small Institute for Advanced Computational Science, Stony Brook University, United States}}






\begin{document}
\date{\today }
\maketitle
\begin{abstract}
\noindent We develop a unified framework for fast large-scale portfolio optimization with shrinkage and regularization for different objectives such as minimum variance, mean-variance, and maximum Sharpe ratio with various constraints on the portfolio weights. For all of the optimization problems, we derive the corresponding quadratic programming problems and implement them in an open-source Python library. We use the proposed framework to evaluate the out-of-sample portfolio performance of popular covariance matrix estimators such as sample covariance matrix, linear and nonlinear shrinkage estimators, and the covariance matrix from the instrumented principal component analysis (IPCA). We use 65 years of monthly returns from (on average) 585 largest companies in the US market, and 94 monthly firm-specific characteristics for the IPCA model. We show that the regularization of the portfolio norms greatly benefits the performance of the IPCA model in portfolio optimization, resulting in outperformance  linear and nonlinear shrinkage estimators even under realistic constraints on the portfolio weights.
\end{abstract}

\vspace{1cm}
\noindent\textbf{Keywords:} IPCA; Large-Scale Portfolio Optimization; l1 and l2 Regularization; Quadratic Programming; Shrinkage Covariance Matrix Estimator.



\section{Introduction}\label{sec:introduction}
Institutional investors often hold portfolios consisting of hundreds of assets, and the performance of such portfolios is frequently evaluated through backtesting exercises. These backtests rely on a large number of optimizations that are repeatedly performed in a rolling-window scheme using a long history of returns data. In this paper, we develop a unified framework for portfolio optimization that utilizes Quadratic Programming (QP) methods to compute portfolios with $\ell_1$ and $\ell_2^2$ regularization, long-short constraints, and various portfolio objective functions, such as minimum-variance, mean-variance, and maximum-Sharpe ratio. Thanks to very fast QP optimization algorithms, the proposed models can be applied in realistic setting of large dimensional portfolios with various additional constraints, and repeatedly in a rolling window scheme for backtesting evaluations and fine-tuning of the investment strategy.

In our framework, all portfolio optimization methods require the estimation of a covariance matrix. When the number of assets in the portfolio is comparable to the number of observations, two main approaches to this problem are shrinkage covariance matrix estimation and financial factor modeling. The former uses information contained in the assets returns only. It has been studied extensively starting from linear shrinkage covariance matrix estimator by \cite{Ledoit:04}, nonlinear shrinkage estimators such as \cite{Ledoit:12}, and \cite{LedoitWolf:20}, up to the most recent nonlinear quadratic shrinkage estimator proposed by \cite{LedoitWolf:22} (see Section \ref{sec:covariance_model} for more details). The latter approach uses common risk factors with financial or economic interpretations,
which are well-known to capture large amounts of variation in the returns.  Among the most famous models are CAPM-model of  \cite{treynor1961market}, \cite{sharpe1964capital}, \cite{lintner1965security}, and \cite{mossin1966equilibrium}, the three-factor, four-factor, and the five-factor model by \cite{FAMA1993}, \cite{carhart1997persistence} and \cite{fama2015five}, respectively. The extensions of these models under the non-Gaussianity assumption for the asset returns and factors are given in \citet{hediger2021heterogeneous}. There is also the momentum factor, which extends the three-factor model to the Carhart four-factor model. It was first introduced and analyzed by \cite{jegadeesh1993mom}, see also \citet{FracMom:22} and the references therein for momentum-based portfolio strategy without crashes.

While the aforementioned classical common risk factors remain among the most important, a large literature now exists on how to decide whether or not to include a particular factor among the dozens, if not hundreds, available: see, e.g., 
\cite{bai2002determining}, \cite{stock2002forecasting}, \cite{tsai2010constrained}, \cite{bai2013principal}, \cite{bai2016efficient}, and the references therein. 
The amount of available alternative data, the computational power, and the statistical techniques which can estimate sparse models as in \citet{tibshirani1996} and 
\citet{Hastie:2015} lead to the proliferation of different factor models. Resulting in, as \citet{feng2020taming} refer to, a zoo of factors.

% {\color{blue}DONE. Above I cite a few papers and briefly explain shrinkage estimator. We have more details in Section \ref{sec:covariance_model}} 
% {\color{red}  The factor modeling framework is explained nicely, but there is no discussion on shrinkage cov matrix estimation here. Since both are mentioned in the previous page, would be nice to have a paragraph describing that as well and list some references.}

% {\color{red} Also, there seems to be a bit of discontinuity here ... we describe two main approaches and then suddenly start explaining our data. We could add a small paragraph and motivate what we plan to do next. Based on our discussion, the motivation could be that the recent proposal of IPCA seems to be working well but no rigorous evaluation exists in the literature re its application in portfolio optimization. Further, there is no notion of adding regularization to IPCA while regularization has shown to be effective in portfolio optimization [REFERENCE]. To that end, we investigate the performance of IPCA with/without regularization in portfolio optimization ... }

In this paper, we consider a large universe of US stocks and 94 asset-specific factors, as listed in Table \ref{tab:94features} in the Appendix. To extract relevant information from this large number of factors while capturing the dynamics in the dependency between factors and returns in a large portfolio of assets, we use the Instrumented Principal Component Analysis (IPCA)---an approach introduced in \citet{KELLY2019501}. Instead of selecting a sparse model with a small number of factors, IPCA assumes that a high-dimensional set of asset-specific factor characteristics is implicitly related to the returns via a lower-dimensional linear subspace of instruments. To model the dynamics in the dependency, the loadings that capture the relation between these lower-dimensional instruments and the stock returns are allowed to change over time, as in the BARRA's approach to factor modeling. 
\citet{KELLY2019501} show that their IPCA model outperforms the common risk factors models mentioned earlier in terms of higher in-sample and predicted $R^2$ values, leading to better out-of-sample portfolio performance. Recently, \citet{goyal2022equity} use IPCA to explain the returns of option contracts and achieved a significantly better out-of-sample $R^2$. Motivated by these recent successes of the IPCA model and its flexibility to capture information from a large number of factors, we forgo the aforementioned common risk factors models and focus on the IPCA in our unified portfolio optimization framework. Our contribution is to study the performance of this emerging model in portfolio optimization under more realistic portfolio constraints, with and without regularization.

% {\color{blue}DONE. I added the last sentence above. I think we explain a lot of what you wrote in the two paragraphs below.} {\color{red} This paragraph is written nicely; couple of comments: (1) Based on our discussion, the motivation could be that the recent proposal of IPCA seems to be working well but no rigorous evaluation exists in the literature re its application in portfolio optimization. Further, there is no notion of adding regularization to IPCA while regularization has shown to be effective in portfolio optimization [REFERENCE]. To that end, we investigate the performance of IPCA with/without regularization in portfolio optimization ... this way, we mention again what we said in the abstract (regularization helps IPCA); (2) Now that regularization is important, we should say that its implementation is computationally expensive, but we have a solution. Our solution as mentioned before is that we combine all different regularization and utilize QP to solve the optimization problem. This is specifically important in practice since these optimization problems need to be solved repeatedly.}

The portfolio performance evaluation of the IPCA model in \citet{KELLY2019501} is done using the tangent portfolio, which is a closed-form portfolio that allows for unbounded long and short positions. In large-dimensional setup without additional portfolio constraints, the tangent portfolio often results in highly leveraged positions which are difficult to implement in practice and can heavily skew the out-of-sample portfolio results. In this paper we want to illustrate the flexibility of the proposed unified portfolio optimization framework in a practical portfolio problem of large institutional investors. Therefore, we compare  the portfolio performance of the IPCA model against commonly used benchmarks such as the shrinkage covariance matrix estimator, and we use a rolling window exercise on a long history of a large set of monthly US equity returns with realistic constraints on the individual positions, and long-short constraints to avoid any highly leveraged positions.

Our paper makes three main contributions. First, we develop a unified framework for large-scale, fast portfolio optimization with realistic constraints and important regularizations. This framework is important for institutional investors who manage portfolios consisting of hundreds or even thousands of assets and need to make investment decisions quickly and efficiently. The corresponding Python implementation is open-source and available online.\footnote{See \url{https://github.com/PawPol/PyPortOpt} for the latest version of the code.} Second, we provide new insights regarding the performance of the recently proposed IPCA model, showing the importance of regularization in more realistic portfolio optimization problems. Finally, our framework offers many different combinations of portfolio problems with different portfolio objective functions, regularizations, and constraints. Among them are the $\ell_1$ and $\ell_2^2$ regularized portfolio problems, introduced by \citet{demiguel2009generalized} in the context of minimum-variance portfolio, and extend in this paper to the $\ell_1$+$\ell_2^2$ regularized maximum-Sharpe ratio portfolios and the whole $\ell_1$+$\ell_2^2$ regularized mean-variance portfolio frontier.

The remainder of the paper is organized as follows. Section \ref{sec:portfolio} introduces our general framework for portfolio optimization, while Section \ref{sec:covariance_model} discusses different covariance matrix estimators considered in this paper. Section \ref{sec:empirics} provides an empirical comparison of the estimators for different portfolio optimization problems. Finally, Section \ref{sec:conclusions} summarizes some concluding remarks. The Appendix gathers information about the asset-specific factors.

\section{Portfolio Optimization Framework}\label{sec:portfolio}
We consider a universe of $N$ assets, with prices observed over a given period of time with $T$ observations. Let $P_{t,i}$ be the price of asset $i=1,\ldots,N$ at time index $t=1,\ldots,T$, where the time index $t$ corresponds to a fixed unit of time such as days, weeks, or months. The corresponding simple returns (also known as linear or net returns) are given by $R_{t,i} =\frac{P_{t,i} - P_{t-1,i}}{P_{t-1,i}}=\frac{P_{t,i}}{P_{t-1,i}}-1$, and the log-returns (also known as continuously compounded returns) are 
$r_{t,i} = \log \frac{P_{t,i}}{P_{t-1,i}} = \log(1+R_{t,i}).$

We denote the vector of $\log$-returns of $N$ assets at time $t$ with $\mathbf{r}_t\in \mathbb{R}^{N}$. It is a multivariate stochastic process with conditional mean and covariance matrix denoted by
\[ \mathbb{E} [\mathbf{r}_t\mid \mathcal{F}_{t-1}]=\boldsymbol{\mu}_t =\begin{bmatrix}\mu_{t,1}\\ \vdots\\ \mu_{t,N}\end{bmatrix}\nonumber \]
and
\[ Cov[\mathbf{r}_t\mid \mathcal{F}_{t-1}] =  \mathbb{E}
[(\mathbf{r}_t - \boldsymbol{\mu}_t)(\mathbf{r}_t - \boldsymbol{\mu}_t)^T \mid \mathcal{F}_{t-1}]=\boldsymbol{\Sigma}_t =\begin{bmatrix}\sigma_{t,11} &\cdots& \sigma_{t,1N}\\ \vdots&\ddots&\vdots\\
\sigma_{t,N1}&\cdots&\sigma_{t,NN}\end{bmatrix}\nonumber, \]
where $\mathcal{F}_{t-1}$ denotes the previous historical data. In this work, except for the IPCA model, we will drop the subscript $t$ on the mean and covariance matrix since all models assume $iid$ returns. For more general multivariate time-series models of returns with the dynamics in the conditional mean and covariance matrix together with their applications in portfolio optimization, we refer to \citet{PaPo:15c}, \citet{paolella2019regime}, and \citet{Paolella:2021}.

% $p$ risky assets: $i=1,2,\ldots,p$
% Single-Period Returns: $p$-variate random vector
% $$\mathbf{R} = [R_1,R_2,\ldots,R_p]'$$
% Mean and Covariance of Returns:
% $$\mathbb{E}[\mathbf{R}] = \boldsymbol{\mu} = \begin{bmatrix}\mu_1\\ \vdots\\ \mu_p\end{bmatrix}\quad Cov[\mathbf{R}] = \boldsymbol{\Sigma} = \begin{bmatrix}\sigma_{11} &\cdots& \sigma_{1p}\\ \vdots&\ddots&\vdots\\
% \sigma_{p1}&\cdots&\sigma_{pp}\end{bmatrix}$$

The investment portfolio is usually summarized by an $N$-vector of weights $\mathbf{w} = [w_1,\ldots,w_N]^{\prime}$ indicating the fraction of the total wealth of the investor held in each asset. If the investor is assumed to hold her total wealth in the portfolio, then $\mathbf{w}'\mathbf{1}_N = 1$, where $\mathbf{1}_N$ denotes an $N$-vector of ones. The corresponding portfolio return $r_t(\mathbf{w}) = \mathbf{w}^{\prime}\mathbf{r}_t$ is a random variable with the mean and variance given by $\mu_{\mathbf{w}} = \mathbb{E}[r_t(\mathbf{w})] = \mathbf{w}^{\prime}\boldsymbol{\mu}$ and $\sigma^2_{\mathbf{w}} = Var[r_t(\mathbf{w})] = \mathbf{w}^{\prime}\boldsymbol{\Sigma}\mathbf{w}$, respectively.

The general theory of portfolio optimization, as introduced in a seminal work by \citet{Ma52}, summarizes the trade-off between risk and investment return using the portfolio's mean and variance. In particular, for a given choice of target mean return $\alpha_0$, in Markowitz portfolio optimization, one chooses the optimal portfolio as 
\begin{equation}\label{eq:meanVariancePortOpt}
\mathbf{w}^* = \arg\min_{\mathbf{w}\in\mathcal{W}}\frac{1}{2}\mathbf{w}^\prime\mathbf{\Sigma}\mathbf{w},
\end{equation}
where  $\mathcal{W}:=\left\{\mathbf{w}\in \mathbb{R}^N:\mathbf{w}^{\prime}\boldsymbol{\mu} \geq \alpha_0\mbox{ and } \mathbf{w}^{\prime}\mathbf{1}_{N} = 1\right\}$ is a set of constraints on the portfolio weights which correspond to a fully invested portfolio with the expected return above the $\alpha_0$ threshold. Under these constraints, \eqref{eq:meanVariancePortOpt} has a closed-form solution given by
\begin{equation}\label{eq:long_short_closed_form}
\mathbf{w}^* = \left\{{B}\boldsymbol{\Sigma}^{-1}\mathbf{1} - {A} \boldsymbol{\Sigma}^{-1}\boldsymbol{\mu} + \alpha_0({C}\boldsymbol{\Sigma}^{-1}\boldsymbol{\mu} - {A}\boldsymbol{\Sigma}^{-1} \mathbf{1})\right\}/{D},
\end{equation}
where ${A}=\boldsymbol{\mu}\boldsymbol{\Sigma}^{-1}\mathbf{1} =\mathbf{1}^{\prime} \boldsymbol{\Sigma}^{-1}\boldsymbol{\mu}$, ${B}=\boldsymbol{\mu}^{\prime}\boldsymbol{\Sigma}^{-1}\boldsymbol{\mu}$, ${C}=\mathbf{1}^{\prime}\boldsymbol{\Sigma}^{-1}\mathbf{1} $, ${D}={B}{C}-{A}^2$.


The minimum-variance portfolio ($Min-Var$ in Figure \ref{fig:portfoliofrontier}) is a solution to \eqref{eq:meanVariancePortOpt} with 
$\mathcal{W}:=\left\{\mathbf{w}\in \mathbb{R}^N: \mathbf{w}^{\prime}\mathbf{1}_{N} = 1\right\}$. The solution to this problem also has a closed-form expression given by
\begin{equation}\label{eq:minimumvariance_closed_form}
    \mathbf{w}^* = \boldsymbol{\Sigma}^{-1} \mathbf{1}/C,
\end{equation}
where $C$ is defined above. However, when short-selling is not allowed, i.e., $\mathbf{w}\geq \boldsymbol{0}_N$, or when it is constrained, e.g., as in Section \ref{sec:long_short_portfolio}, then the optimization problem \eqref{eq:meanVariancePortOpt} does not have a closed-form solution and needs to be solved numerically.

Nevertheless, \eqref{eq:meanVariancePortOpt} is a QP problem with convex constraints (hence also a convex problem). It has closed-form expressions for the gradient and hessian of the objective function, and a unique global optimal portfolio satisfying the constraints in $\mathcal{W}$. In particular, by changing $\alpha_0$, one can derive a whole portfolio frontier of optimal investments $\mathbf{w}^*(\alpha_0)$ summarizing the risk-return trade-off. 


Figure \ref{fig:portfoliofrontier} displays two long-only mean-variance portfolio efficient frontiers with and without $\ell_2^2$ regularization discussed in Section \ref{sec:l2_constraint_portfolio_norms}, where for different levels of portfolio variances the expected return of the best performing portfolio is drawn. Together with the frontiers, we show different optimal portfolios that are discussed in this paper and a cloud of points depicting means and variances of randomly drawn $25000$ $iid$ Dirichlet distributed portfolios, i.e., $\mathbf{w}_k\overset{iid}{\sim} Dir(\mathbf{1}_N)$, for $k=1,\ldots,25000$. In this example, the portfolios consist of 8 stocks from the US market (tickers: AMZN, MSFT, GOOGL, F, TM, AAPL, KO, and PEP) with the mean and covariance matrix estimated using daily returns over the period of 2015-01-01-2022-01-01.


\begin{figure}
\centering
{\includegraphics[width=1.1\textwidth, height=0.8\textwidth]{figure/frontier_dirichlet.png}}\hspace{5pt}
\caption{\textit{Portfolio frontier (with and without $\ell_2^2$ regularization) and all of the optimal portfolios considered in our portfolio framework with the long-only constraints and $\ell_1$, $\ell_2^2$, and $\ell_1$+$\ell_2^2$ regularization for eight stocks (AMZN, MSFT, GOOGL, F, TM, AAPL, KO, and PEP), with the mean and covariance matrix estimated using daily returns over eight years (2015/01/01-2022/01/01). Among them are two optimal portfolios: the minimum-variance portfolio and the maximum Sharpe ratio portfolio, and a collection of random portfolios.}} \label{fig:portfoliofrontier}
\end{figure}



In practice, investment portfolios consist of a much larger number of assets than in the example in Figure \ref{fig:portfoliofrontier}, and one often uses only monthly instead of daily returns to obtain the necessary parameter estimates, which largely reduces the number of observations relative the dimensionality of the problem. Figure \ref{fig:differentassetsportfolio} illustrates portfolio frontiers together with $25000$ $iid$ Dirichlet distributed\footnote{Here we use Dirichlet distributed random vectors to guarantee uniform sampling on the $N$ dimensional simplex ($\mathbf{w}'\mathbf{1}_N=1$). The results for the weights sampled from uniform distribution normalized on the simplex, i.e., $\mathbf{w}= \mathbf{x}/(\mathbf{x}'\mathbf{1}_N)$, where $\mathbf{x}=[x_1,\ldots,x_N]$ and $x_i\overset{iid}{\sim}  U([0,1])$; and for the weights sampled from the absolute value of standard normal distribution normalized on the simplex, i.e., $\mathbf{w}= \left|\mathbf{x}\right|/\left\|\mathbf{x}\right\|_1$, where $\mathbf{x}=[x_1,\ldots,x_N]$ and $x_i\overset{iid}{\sim}  N(0,1)$ are similar.} portfolios $\mathbf{w}_k\overset{iid}{\sim} Dir(\mathbf{1}_N)$, for $k=1,\ldots,25000$, and for the different number of assets $N=10, 20, 50, 500$ selected from the largest market-capitalization stocks in the US market with mean and covariance matrix estimated using ten years of daily returns (which is a much larger number of observations than all of our monthly data used in Section \ref{sec:empirics}). As can be seen from different panels in Figure \ref{fig:differentassetsportfolio}, the dimensionality of the portfolio has two major impacts. First, the larger the assets universe, the further away and the more concentrated around $1/N$ are the random portfolios. This shows that in a large-dimensional setup without proper portfolio optimization, one cannot expect to achieve any optimal risk-reward profile and that even the $1/N$ portfolio, which is so often advocated as a naive-diversification and well-performing portfolio---see \citealp{DGU:09} and the references therein---is in fact as good as any random guess. Figure \ref{fig:differentassetsportfolio} also depicts the equal volatility contribution portfolio, which is a special case of the risk parity portfolio (see, e.g., \citealp{roncalli_RP:13} and \citealp{PaPoPoWa:19}). It is slightly better than random portfolios or the $1/N$. However,  based on the distance between the equal volatility contribution portfolio and the mean-variance portfolio frontier, and even with the uncertainty about the actual frontier, there is still a lot of opportunity for improved portfolio allocation. Hence, the importance of proper portfolio optimization in large dimensions, as opposed to naive diversification, is evident. The second major impact is that the closed-form long-short frontier from equation \eqref{eq:long_short_closed_form} depicted with a dotted red line is becoming almost vertical compared with the long-only portfolio when the number of assets increases. Therefore, small changes in the optimal portfolio volatility translate to theoretically disproportionately large gains in the expected returns of the optimal portfolio. This implies that estimates of the optimal portfolio weights are sensitive to new data points, and the weights can change a lot over the rolling windows. Of course, this is the artifact of high-dimensionality and relatively close to non-singular covariance matrix estimates. Proper covariance matrix estimation in high dimensions and long-short constraints help in avoiding some of these over-leveraged and unrealistic but theoretically optimal portfolios.



\begin{figure}
\centering
\begin{subfigure}{0.45\textwidth}
\includegraphics[width=\textwidth]{figure/frontier_a_dirichlet.png}
\caption{\textit{N = 10}}
\label{fig:1a}
\end{subfigure}%
\begin{subfigure}{0.45\textwidth}
\includegraphics[width=\textwidth]{figure/frontier_b_dirichlet.png}
\caption{\textit{N = 20}}
\label{fig:1b}
\end{subfigure}\\
\begin{subfigure}{0.45\textwidth}
\includegraphics[width=\textwidth]{figure/frontier_c_dirichlet.png}
\caption{\textit{N = 50}}
\label{fig:1c}
\end{subfigure}%
\begin{subfigure}{0.45\textwidth}
\includegraphics[width=\textwidth]{figure/frontier_d_dirichlet.png}
\caption{\textit{N = 500}}
\label{fig:1d}
\end{subfigure}
\caption{\textit{Four plots of two different portfolio frontiers (long-only and the closed-form long-short from \eqref{eq:long_short_closed_form}), together with different optimal long-only portfolios (maximum-Sharpe ratio \eqref{eq:maximumSharperatio} and minimum-variance), the equally weighted portfolio ($1/N$), equal volatility contribution portfolio (Equal-Var), and $25000$ $iid$ Dirichlet distributed portfolios $\mathbf{w}\sim Dir(\mathbf{1}_N)$, for different number of assets $N=10, 20, 50, 500$ selected from the largest market-capitalization stocks in the US market. Mean and covariance matrix estimated from 10 years of daily returns (2520 observations).}} \label{fig:differentassetsportfolio}
\end{figure}



%The equal volatility contribution portfolio $\mathbf{w} = [\frac{\sigma_1^{-1}}{\sum_{i=1}^p \sigma_i^{-1}},\ldots,\frac{\sigma_p^{-1}}{\sum_{i=1}^p \sigma_i^{-1}}]$, also given in Figure \ref{fig:portfoliofrontier},  has already a better risk-return trade-off than most of the randomly selected portfolios. Nevertheless, as discussed in \citet{demiguel2009optimal} the $1/p$ portfolio can be a very competitive, well diversified, low turnover, and simple to construct investment alternative that performs well out-of-sample. 

Also, the true mean vector and covariance matrix are unknown in practice, and one needs to rely on their estimates. Financial markets, especially at low frequencies, are highly efficient---or, as suggested by \citet{Pedersen:15}, they are at least ``efficiently-inefficient''. Therefore, prediction of the mean vector is very difficult, and we do not attempt to construct any improved mean vector or momentum prediction signal---for the latter, we refer to recent results in \citet{FracMom:22}. Instead, we take the historical mean of the returns for the estimator and focus on different covariance matrix estimators. The general idea is that the covariance matrix, because of the stylized facts of persistency of the volatilities and correlations between assets, should be relatively easier to predict than the mean. However, as discussed above, one needs to account for the dimensionality of the problem by relying either on factor models or shrinkage methods to have any chance of constructing optimal portfolios that perform well out-of-sample. Before we turn to the covariance matrix estimators, we introduce the rest of our general portfolio optimization framework.




\subsection{Portfolio Constraints}\label{sec:minimumvarianceportfolio}


The set of feasible portfolio weights $\mathcal{W}:=\left\{\mathbf{w}\in \mathbb{R}^N: \mathbf{w}^{\prime}\mathbf{1}_{N} = 1\right\}$ usually includes additional constraints. Among the most commonly used are:
\begin{itemize}
    \item Long only: %$\mathbf{w}: w_j\geq 0, \forall j$
    $$\mathcal{W}:=\left\{\mathbf{w}\in \mathbb{R}^N: \mathbf{w}^{\prime}\mathbf{1}_{N} = 1 \mbox{ and } w_i\geq 0, \forall i\right\}.$$
    \item Asset specific holding constraints: 
    $$\mathcal{W}:=\left\{\mathbf{w}\in \mathbb{R}^N: \mathbf{w}^{\prime}\mathbf{1}_{N} = 1  \mbox{ and } L_i\leq w_i\leq U_i, \forall i\right\},$$
    where $\mathbf{U}=(U_1,\ldots,U_N)$ and $\mathbf{L}=(L_1,\ldots,L_N)$ are upper and lower bounds for the $N$ portfolio positions.
    \item Turnover constraints: 
    \begin{itemize}
        \item[-] for individual assets limits
        $$\mathcal{W}:=\left\{\mathbf{w}\in \mathbb{R}^N: \mathbf{w}^{\prime}\mathbf{1}_{N} = 1  \mbox{ and } |\Delta w_i|\leq U_i, \forall i\right\},$$
        where $\Delta w_i$ denotes the change in the portfolio weight from the current position to the optimal value and $U_i$ are the turnover limits for individual positions;
    \item[-] for the total portfolio limit
    $$\mathcal{W}:=\left\{\mathbf{w}\in \mathbb{R}^N: \mathbf{w}^{\prime}\mathbf{1}_{N} = 1  \mbox{ and } \left\|\mathbf{w}\right\|_1 = \sum_{i=1}^N|\Delta w_i|\leq U_*\right\},$$
    where $U_*$ is the turnover limit for the entire portfolio.
    \end{itemize}
    \item Benchmark exposure constraints: 
     $$\mathcal{W}:=\left\{\mathbf{w}\in \mathbb{R}^N: \mathbf{w}^{\prime}\mathbf{1}_{N} = 1 \mbox{ and }\left\|\mathbf{w} - \mathbf{w}_{B}\right\|_1 = \sum_{i=1}^N|w_i - w_{B,i}|\leq U_B\right\},$$
    where, $\mathbf{w}_B$ are the weights of the benchmark portfolio, and $U_B$ is the total error bound.
    \item Tracking error constraints: for a given benchmark portfolio $B$ with  weights $\mathbf{w}_B$, 
    $r_B = \mathbf{w}_B' \mathbf{r}$ is the return of the benchmark portfolio, e.g., S\&P 500 Index, NASDAQ 100, Russell 1000/2000. One can compute the variance of the Tracking Error $Var(TE) = (\mathbf{w}-\mathbf{w}_B)'\boldsymbol{\Sigma}(\mathbf{w}-\mathbf{w}_B),$ and include the corresponding constraint into to the set of feasible portfolio weights 
     $$\mathcal{W}:=\left\{\mathbf{w}\in \mathbb{R}^N: \mathbf{w}^{\prime}\mathbf{1}_{N} = 1 \mbox{ and }(\mathbf{w}-\mathbf{w}_B)'\boldsymbol{\Sigma}(\mathbf{w}-\mathbf{w}_B)\leq \sigma_{TE}^2\right\},$$
     where $\sigma_{TE}^2>0$ is the variance tracking-error of the portfolio.
    \item Risk factor constraints: estimate the risk factors exposure for all the assets in the portfolio, e.g., via the following regression (see \eqref{eq:Fama_French_Factor_Model} for details)
    \[r_{i,t} = \alpha_i + \sum_{k=1}^K \beta_{i,k} f_{k,t} + \epsilon_{i,t}.\] 
    Given these estimates, one can 
    \begin{itemize}
    \item[(i)] constrain the exposure to a given factor $k$ by 
    $$\mathcal{W}:=\left\{\mathbf{w}\in \mathbb{R}^N: \mathbf{w}^{\prime}\mathbf{1}_{N} = 1 \mbox{ and }|\sum_{i=1}^N \beta_{i,k}w_i|\leq U_k\right\}.$$
    \item[(ii)] neutralize the exposure to all the risk factors by  
    $$\mathcal{W}:=\left\{\mathbf{w}\in \mathbb{R}^N: \mathbf{w}^{\prime}\mathbf{1}_{N} = 1 \mbox{ and } |\sum_{i=1}^N \beta_{i,k}w_i|=0\ \forall k\right\}.$$
    \end{itemize}
    %\item minimum transaction size
    %\item minimum holding size
    %\item integer constraints
\end{itemize}
All the constraints listed above (including those that involve the absolute value function---see the remarks in Section \ref{sec:l1_constraint_portfolio_norms})  can be written as linear or quadratic constraints, i.e.,
\begin{itemize}
        \item linear constraints: we can specify $N$-columns matrices $A_w$ and $A_B$  and vectors $u_w$, $u_B$ to introduce linear inequality constraints for the relative positions between the assets or the benchmark
     $$\mathcal{W}:=\left\{\mathbf{w}\in \mathbb{R}^N: \mathbf{w}^{\prime}\mathbf{1}_{N} = 1 \mbox{ and } A_w\mathbf{w}\leq u_w,\ \ A_B(\mathbf{w}-\mathbf{w}_B)\leq u_B\right\}.$$
    \item quadratic constraints: we can specify $N\times N$ matrices $Q_w$, $Q_B$ and scalars $q_w$, $q_B$ to build constraints
    $$\mathcal{W}:=\left\{\mathbf{w}\in \mathbb{R}^N: \mathbf{w}^{\prime}\mathbf{1}_{N} = 1 \mbox{ and } \mathbf{w}'Q_w\mathbf{w}\leq q_w,  (\mathbf{w}-\mathbf{w}_B)'Q_B(\mathbf{w} - \mathbf{w}_B) \leq q_B\right\}.$$
\end{itemize}
Once the constraints are converted into these standard forms, they can be easily combined and incorporated into our portfolio optimization framework. We consider next, a different type of constraint that is often incorporated into portfolio optimization using the method of Lagrange multipliers. These constraints are not imposed by the portfolio manager because of her trading goals or position requirements. They are added because they are a form of regularization of the problem in high dimensions, and they help to improve the out-of-sample portfolio performance in large dimensions.

%In the following section, we will discuss different regularizations using the minimum-variance portfolio as the building block. Then in the next section, we will introduce mean-variance and maximum Sharpe ratio portfolios and show how to formulate them as QP problems. So the same approach with all the possible constraints and regularizations can also be used for these cases.


\subsection{Portfolio Optimization with $\ell_2^2$ Penalized Portfolio Norms}\label{sec:l2_constraint_portfolio_norms}
Consider now an $\ell_2^2$-constrained (also called the ridge penalty) portfolio optimization problem for the minimum-variance portfolio \eqref{eq:meanVariancePortOpt}. Using the method of Lagrange multipliers, we can write the corresponding optimization problem as
\begin{equation}\label{eq:l2largrangian}
\mathbf{w}^*=\arg\min_{\mathbf{w}\in\mathcal{W}}\mathbf{w}'\boldsymbol{\Sigma}\mathbf{w} + \lambda \left\|\mathbf{w}\right\|_2^2,
\end{equation}
where $\lambda\geq 0$ is the penalty strength parameter and $\left\|\mathbf{w}\right\|_2^2 = \sum_{i=1}^N w_i^2$. Using the spectral decomposition of $\boldsymbol{\Sigma}=\mathbf{P}\boldsymbol{\Lambda}\mathbf{P}'$, where $\mathbf{P}\mathbf{P}'=\mathbb{I}_N$ and $\boldsymbol{\Lambda}=diag(\delta_1,\ldots,\delta_N)$, and since $\left\|\mathbf{w}\right\|_2^2=\mathbf{w}'\mathbf{w}=(\mathbf{P}'\mathbf{w})'(\mathbf{P}'\mathbf{w})$, we can rewrite the $\ell_2^2$ penalized objective function as
\begin{equation}\label{eq:l2meanvariance}
    \mathbf{w}^*=\arg\min_{\mathbf{w}\in\mathcal{W}} \mathbf{w}'\tilde{\boldsymbol{\Sigma}}\mathbf{w},
\end{equation}
where $\tilde{\boldsymbol{\Sigma}} = \mathbf{P}\left[\boldsymbol{\Lambda}+\lambda\mathbb{I}_N\right]\mathbf{P}'$ has all the eigenvalues shifted up by $\lambda\geq 0$. This is, again, a QP optimization problem that falls into our unified framework.

\subsection{Portfolio Optimization with $\ell_1$ Penalized Portfolio Norms}\label{sec:l1_constraint_portfolio_norms}


Similarly to the $\ell_2^2$-constraint, we can write the Lagrangian of $\ell_1$-constrained minimum-variance portfolio optimization problem as
\begin{equation}\label{eq:l1Lagrangian}
\mathbf{w}^*=\arg\min_{\mathbf{w}\in\mathcal{W}}\mathbf{w}'\boldsymbol{\Sigma}\mathbf{w} + \lambda \left\|\mathbf{w}\right\|_1,
\end{equation}
where $\lambda\geq 0$ is the penalty strength parameter and $\left\|\mathbf{w}\right\|_1 = \sum_{i=1}^N \left|w_i\right|$. The main difference compared to \eqref{eq:l2largrangian} is that the objective function in \eqref{eq:l1Lagrangian} is non-differentiable because of the kinks in the absolute value function, and the spectral decomposition will not help in converting \eqref{eq:l1Lagrangian} into a standard QP problem. Instead, we define $\mathbf{w}_+ = \max(0,\mathbf{w})\in\mathbb{R}_{0,+}^N$,
$\mathbf{w}_- = -\min(0,\mathbf{w})\in\mathbb{R}_{0,+}^N$
and $\mathbf{w}_+\cdot\mathbf{w}_-=\boldsymbol{0}$. Then $\mathbf{w}=\mathbf{w}_+-\mathbf{w}_-\mbox{ and } \left\|\mathbf{w}\right\|_1=\mathbf{w}_+ + \mathbf{w}_-$.
We can rewrite the $\ell_1$-regularized objective function as
\begin{equation}\label{eq:l1meanvariance_old}
\left(\mathbf{w}^*,\mathbf{w}^*_+, \mathbf{w}^*_-\right)=\arg\min_{(\mathbf{w},\mathbf{w}_+,\mathbf{w}_-)\in\tilde{\mathcal{W}}}\mathbf{w}'\boldsymbol{\Sigma}\mathbf{w} + \lambda \left(\mathbf{w}_+ + \mathbf{w}_-\right),
\end{equation}
where $\tilde{\mathcal{W}}=\left\{(\mathbf{w},\mathbf{w}_+,\mathbf{w}_-)\in\mathbb{R}^{3N}: \mathbf{w}=\mathbf{w}_+-\mathbf{w}_-, \mathbf{w}_+\geq 0, \mathbf{w}_- \geq 0, \mbox{ and } \mathbf{w}\in\mathcal{W}\right\}.$ This way, we rewrote the original non-differentiable problem in $N$ variables as a QP problem in $3N$ variables with additional $N$ equality constraints.\footnote{It is possible to further simplify the optimization problem from $3N$ to $2N$ variables by incorporating these constraints explicitly. But we tested this empirically, and it slows down the algorithms because one needs to use then the $2N\times 2N$ matrix instead of $\boldsymbol{\Sigma}$ in \eqref{eq:l1meanvariance_old}. The same argument applies to the similar optimization problems below.} 

The following remarks can be made about this new optimization problem:
    \begin{itemize} 
        \item[(i)] Note that we do not have to include the constraint 
$\mathbf{w}_+\cdot\mathbf{w}_-=\boldsymbol{0}$ into the definition of the set of feasible weights $\tilde{\mathcal{W}}$ since any portfolio with
$\mathbf{w}_+\cdot\mathbf{w}_-\neq \boldsymbol{0}$ is strictly dominated in terms of the value of the objective function by an analogous portfolio with 
$\mathbf{w}_+\cdot\mathbf{w}_-=\boldsymbol{0}$. Hence, the optimizer will never stop at $\mathbf{w}_+\cdot\mathbf{w}_- \neq \boldsymbol{0}$.
\item[(ii)] If the portfolio is long-only, the $\ell_1$ norm for the feasible portfolios reduces to the sum of portfolio weights, and the optimization problem \eqref{eq:l1Lagrangian} becomes differentiable. In this case, we observe empirically that optimal portfolio weights will never change when $\lambda$ grows---see the left panel in Figure \ref{fig:allocation_l1} (see also Figure \ref{fig:portfoliofrontier} where some optimal portfolios are $\ell_1$+$\ell_2^2$ regularized, and they coincide with the $\ell_2^2$ regularized portfolios). This is because the constraints will disappear if we assume that $\mathbf{w}^{\prime}\mathbf{1}_{N} = 1$ and $\mathbf{w} \geq \mathbf{0}_N$. Even when short positions are allowed, the optimization problem will have only \emph{partially} sparse solutions. In both cases, as opposed to a usual LASSO problem, the solution will not converge to $\mathbf{0}$ when $\lambda$ goes to infinity because we have another constraint in $\mathcal{W}$ that $\mathbf{w}'\mathbf{1}_N = 1$, and one will never get all the optimal weights equal to zero. As shown in the right panel in Figure \ref{fig:allocation_l1}, in the long-short portfolio, only all the initially (when $\lambda=0$) negative weights will converge to zero. Some of the initially positive weights will go to zero too. At the same time, the remaining positive weights will converge to a long-only minimum-variance portfolio. Importantly, some intermediate levels of $\lambda$ and the corresponding non-zero optimal weights can perform well out-of-sample.
\item[(iii)] Note that any of the constraints listed in Section \ref{sec:minimumvarianceportfolio} such that it involves an absolute value function, can be rewritten using the $\mathbf{w}^+$ and $\mathbf{w}^-$. Hence, the corresponding optimization problem can be solved using the QP methods.
\end{itemize}


\begin{figure}
\centering
\begin{subfigure}{0.45\textwidth}
\includegraphics[width=\textwidth]{figure/allocation2.png}
\caption{\textit{Long-only portfolio allocation}}
\label{fig:allocation_long}
\end{subfigure}%
\begin{subfigure}{0.45\textwidth}
\includegraphics[width=\textwidth]{figure/allocation1.png}
\caption{\textit{Long-short portfolio allocation}}
\label{fig:allocation_longshort}
\end{subfigure}
\caption{\textit{Portfolio weights of N = 50 assets as a function of the regularization strength parameter $\lambda$ of $\ell_1$ penalty in minimum-variance $\ell_1$ regularized portfolio (long-only vs. long-short with $\vartheta=0.2$), the $x$-axis are in $\log$ scale.}} \label{fig:allocation_l1}
\end{figure}


\subsection{Portfolio Optimization with $\ell_1$+$\ell_2^2$ Penalized Portfolio Norms}\label{sec:l1_l2_regularization}

Naturally, we can consider both the $\ell_1$-constrained and $\ell_2^2$-constrained, which we call $\ell_1$+$\ell_2^2$-constrained portfolio. For that purpose, we  modify our objective function \eqref{eq:meanVariancePortOpt} to
\begin{equation}\label{eq:ela_Lagrangian}
\mathbf{w}^*=\arg\min_{\mathbf{w}\in\mathcal{W}}\mathbf{w}'\boldsymbol{\Sigma}\mathbf{w} + \lambda_1 \left\|\mathbf{w}\right\|_1 + \lambda_2 \left\|\mathbf{w}\right\|_2^2.
\end{equation}
By combining \eqref{eq:l2meanvariance} and \eqref{eq:l1meanvariance_old}, we can use again the eigenvalues decomposition of $\boldsymbol{\Sigma}=\mathbf{P}\boldsymbol{\Lambda}\mathbf{P}'$, where $\mathbf{P}\mathbf{P}'=\mathbb{I}_N$, $\boldsymbol{\Lambda}=diag(\delta_1,\ldots,\delta_N)$
and $\left\|\mathbf{w}\right\|_2^2=\mathbf{w}'\mathbf{w}=(\mathbf{P}'\mathbf{w})'(\mathbf{P}'\mathbf{w})$. %Then, supplementing again $\mathbf{w}=\mathbf{w}_+-\mathbf{w}_-\mbox{ and } \left\|\mathbf{w}\right\|_1=\mathbf{w}_+ + \mathbf{w}_-$ we can rewrite the objective function as

\begin{equation}\label{eq:ela_meanvariance}
    \mathbf{w}^*=\arg\min_{(\mathbf{w},\mathbf{w}_+,\mathbf{w}_-)\in\tilde{\mathcal{W}}} \mathbf{w}'\tilde{\boldsymbol{\Sigma}}\mathbf{w}+\lambda \left(\mathbf{w}_+ + \mathbf{w}_-\right),
\end{equation}
where $\tilde{\mathcal{W}}=\left\{(\mathbf{w},\mathbf{w}_+,\mathbf{w}_-)\in\mathbb{R}^{3N}: \mathbf{w}=\mathbf{w}_+-\mathbf{w}_-\mbox{ and } \mathbf{w}\in\mathcal{W}\right\}$ and $\tilde{\boldsymbol{\Sigma}} = \mathbf{P}\left[\boldsymbol{\Lambda}+\lambda_2\mathbb{I}_N\right]\mathbf{P}'$ has shifted by $\lambda_2\geq 0$ all the eigenvalues.


\subsection{Long-Short Constrained Portfolio}\label{sec:long_short_portfolio}

The long-short constrained minimum-variance portfolio optimization from \eqref{eq:meanVariancePortOpt} is defined as
\begin{equation}\label{eq:longshortmeanvariance}
    \mathbf{w}^*(\vartheta)=\arg\min_{\mathbf{w}\in\mathcal{W}_{LS}(\vartheta)}\mathbf{w}'\boldsymbol{\Sigma}\mathbf{w},
\end{equation}
where $\mathcal{W}_{LS}(\vartheta)=\left\{\mathbf{w}\in\mathbb{R}^N:\sum_{i:w_i>0} w_i \leq 1+\vartheta\mbox{ and }\sum_{i:w_i<0} w_i\geq -\vartheta \right\}$. This is a different type of portfolio weights constraint that aggregates them based on their sign. Long-only portfolio constraint is a special case given by $\mathcal{W}_{LS}(\vartheta)$ for $\vartheta=0$. 
We can take again $\mathbf{w}_+ = \max(0,\mathbf{w})\in\mathbb{R}_{0,+}^N \mbox{ and }\mathbf{w}_- = -\min(0,\mathbf{w})\in\mathbb{R}_{0,+}^N$ and $\mathbf{w}_+\cdot\mathbf{w}_-=\boldsymbol{0}$. So, $\sum_{i:w_i>0} w_i \leq 1+\vartheta \iff \mathbf{w}_+'\boldsymbol{1}_N-1\leq \vartheta$ and  
$\sum_{i:w_i<0} w_i\geq -\vartheta \iff \mathbf{w}_-'\boldsymbol{1}_N\leq \vartheta$. 

Hence, we can replace the $\mathcal{W}_{LS}(\vartheta)$ with a new constraint set given by
$$\tilde{\mathcal{W}}_{LS}(\vartheta)=\left\{\mathbf{w}\in\mathbb{R}^{N}:\mathbf{w} = \mathbf{w}_+ - \mathbf{w}_-\mathbf{w}_+\geq 0, \mathbf{w}_- \geq 0, \mathbf{w}_+'\boldsymbol{1}_N\leq 1+ \vartheta \mbox{ and }  \mathbf{w}_-'\boldsymbol{1}_N\leq \vartheta\right\}$$
and solve the corresponding QP problem.


\subsection{Mean-Variance Optimization with Risk-Free Asset}

In mean-variance portfolio in \eqref{eq:meanVariancePortOpt}, the goal is to optimize the trade-off between portfolio returns and risk. In other words, the mean-variance method looks for a portfolio with the lowest variance while the expected portfolio returns $\mathbf{w}^{\prime}\boldsymbol{\mu}$ is constraint from below by $\alpha_0$. Because of the convexity of the problem, the optimal value corresponds to the minimum volatility portfolio under the target return level.

In addition to the risky assets ($i=1,\ldots,N$) we can assume there is a risk-free asset for which $R_f = r_f, \mbox{ i.e., } \mathbb{E}[R_f] = r_f \mbox{ and } Var(R_f) = 0.$ Suppose the investor can invest in the $N$ risky investments as well as in the risk-free asset. The portfolio with investment in risk-free assets consists of two parts: $\mathbf{w}'\mathbf{1}_N = \sum_{i=1}^N w_i$ (invested in risky assets) and $1 - \mathbf{w}'\mathbf{1}_N$ (risk-free asset).

If borrowing is allowed, $(1-\mathbf{w}^{\prime}\mathbf{1}_N)$ can be negative. Long-short portfolio with return $R_{\mathbf{w}} = \mathbf{w}^{\prime}\mathbf{R} + (1-\mathbf{w}^{\prime}\mathbf{1}_N)R_f$ where $\mathbf{R}=[R_1,\ldots,R_N]^{\prime}$, has expected return 
$\mu_{\mathbf{w}} = \mathbf{w}^{\prime}\boldsymbol{\mu} + (1- \mathbf{w}^{\prime}\mathbf{1}_N)r_f$ and variance $\sigma_{\mathbf{w}} = \mathbf{w}^{\prime}\boldsymbol{\Sigma}\mathbf{w}$.

For a given choice of target mean return $\alpha_0$, choose the portfolio $\mathbf{w}^*$ to 
\begin{equation}\label{eq:meanvarianceLongShort}
    \mathbf{w}^*  = \arg\min_{\mathbf{w}^{\prime}\in\mathcal{W}} \frac{1}{2}\mathbf{w}^{\prime}\boldsymbol{\Sigma}\mathbf{w},
\end{equation}
 where $\mathcal{W}=\left\{\mathbf{w}\in \mathbb{R}^N: \mathbf{w}^{\prime}\boldsymbol{\mu} + (1-\mathbf{w}^{\prime}\mathbf{1}_N)r_f =\alpha_0\right\}$.
Then we can derive the Lagrangian as
\begin{equation}\label{eq:meanvariance_long-short_largrange}
    L(\mathbf{w},\lambda_1) = \frac{1}{2}\mathbf{w}^{\prime}\boldsymbol{\Sigma}\mathbf{w} - \gamma[(r_f - \alpha_0) + \mathbf{w}^{\prime}(\boldsymbol{\mu} - \mathbf{1}_N r_f)].
\end{equation}
Solving the Lagrangian, we get $\mathbf{w}^* = \gamma^* \boldsymbol{\Sigma}^{-1}(\boldsymbol{\mu} - \mathbf{1}_N r_f)$
and $\gamma^* = (\alpha_0 - r_f) / [(\boldsymbol{\mu} - \mathbf{1}_Nr_f)^{\prime}\boldsymbol{\Sigma}^{-1}(\boldsymbol{\mu} - \mathbf{1}_N r_f)]$. So the expected return and the variance of the optimal portfolio are given by $\mathbb{E}(R_N) = \mathbf{w}^{*\prime}\mathbf{R} + (1- \mathbf{w}^{*\prime}\mathbf{1}_N)r_f$,
$Var(R_N) = (\alpha_0 - r_f)^2/[(\boldsymbol{\mu} - \mathbf{1}_Nr_f)'\boldsymbol{\Sigma}^{-1}(\boldsymbol{\mu}-\mathbf{1}_Nr_f)]$, respectively.

Note that because of the risk-free asset, the resulting portfolio frontier will be a line (it is the so-called one fund theorem) connecting two points in the mean-variance plane: the $(0,r_f)$ where all the money is invested only in the risk-free asset; and the mean and variance of so called market portfolio $\mathbf{w}_0 = \boldsymbol{\Sigma}^{-1}(\boldsymbol{\mu} - \mathbf{1}_N r_f)/[\mathbf{1}'\boldsymbol{\Sigma}^{-1}(\boldsymbol{\mu} - \mathbf{1}_N r_f)]$ which is the tangent point to the portfolio frontier without the risk-free asset. So in order to find solutions for different $\alpha_0$,  it suffices to solve for the portfolio without risk-free asset, and take linear combinations of that portfolio with the risk-free investment. Hence, again this can be considered as part of our general portfolio framework.




\subsection{Maximum Sharpe Ratio Portfolio}\label{sec:maxSharpeRatioPortfolio}

Markowitz’s mean-variance framework in \eqref{eq:meanVariancePortOpt} provides portfolios along the optimal frontier, and the choice of the specific portfolio depends on the risk-aversion of the investor. Typically one measures the investment performance using the Sharpe ratio, and there is only one portfolio on the optimal frontier that achieves the maximum Sharpe ratio
\begin{equation}\label{eq:maximumSharperatio}
\arg\max_{\mathbf{w}\in\mathcal{W}}\frac{\mathbf{w}'\boldsymbol{\mu} - r_f}{\sqrt{\mathbf{w}'\boldsymbol{\Sigma}\mathbf{w}}},
\end{equation}
where $\mathcal{W}=\left\{\mathbf{w}\in\mathbb{R}^N: \mathbf{1}_N\mathbf{w}=1, \mathbf{w}\geq \boldsymbol{0}\right\}$, and $r_f$ is the return for a risk-free asset.

This problem -- although nonconvex -- belongs to the family of so called Fractional Programming (FP) optimization problems that involve ratios. It is a concave-convex single-ratio and can be solved by different approaches. This particular FP problem is still simple to solve using a reparametrization trick. One can note that the objective function in \eqref{eq:maximumSharperatio} is homogeneous of degree zero, and reformulate this problem as a QP problem. If there exists at least one portfolio vector  $\mathbf{w}$ such that $\mathbf{w}'\boldsymbol{\mu} - r_f> 0$, then for $\mathbf{w}'\boldsymbol{\mu} - r_f\neq 0$, and $\mathbf{w}\in\mathcal{W}$, we can change the maximization problem into an equivalent minimization
\begin{equation}\label{eq:maximumSharperatio_transformed}
\arg\min_{\mathbf{w}\in\mathcal{W}}\frac{\sqrt{\mathbf{w}'\boldsymbol{\Sigma}\mathbf{w}}}{\mathbf{w}'(\boldsymbol{\mu} - r_f\mathbf{1}_N)},
\end{equation}
where $\mathcal{W}=\left\{\mathbf{w}\in\mathbb{R}^N: \mathbf{w}'\mathbf{1}_N=1, \mathbf{w}\geq \boldsymbol{0}\right\}$. Now by the homogeneity of degree zero of the objective function, we can choose the proper scaling factor for our convenience. We define $\tilde{\mathbf{w}}=\gamma \mathbf{w}$ with scaling factor $\gamma = 1/\mathbf{w}'(\boldsymbol{\mu}-r_f\mathbf{1}_N)>0$. So that the objective becomes $\tilde{\mathbf{w}}'\boldsymbol{\Sigma}\tilde{\mathbf{w}}$, the sum constraint $\mathbf{1}_N'\tilde{\mathbf{w}}=\gamma$, and the above problem is equivalent to 
\begin{equation}\label{ep:maximumShaperation_final}
\arg\min_{\mathbf{w}\in\mathcal{W}}\frac{\sqrt{\gamma\mathbf{w}'\boldsymbol{\Sigma}\mathbf{w}\gamma}}{\gamma\mathbf{w}'(\boldsymbol{\mu} - r_f\mathbf{1}_N)} \iff \arg\min_{[\tilde{\mathbf{w}},\gamma]'\in\tilde{\mathcal{W}}}\tilde{\mathbf{w}}'\boldsymbol{\Sigma}\tilde{\mathbf{w}},
\end{equation}
where $\tilde{\mathcal{W}}=\left\{[\tilde{\mathbf{w}},\gamma]'\in\mathbb{R}^{N+1}: 1 = \tilde{\mathbf{w}}'(\boldsymbol{\mu} - r_f\mathbf{1}_N), \mathbf{1}_N'\tilde{\mathbf{w}}=\gamma, \tilde{\mathbf{w}}\geq \boldsymbol{0}\right\}$.
 
The optimal portfolio weights $\mathbf{w}^*$  are recovered after doing the optimization through the transformation $\mathbf{w}^*=\tilde{\mathbf{w}}^*/\gamma^*$.
Importantly note that all the aforementioned constraints and regularizations can also be incorporated into this optimization problem  \eqref{ep:maximumShaperation_final}, and it will remain equivalent to the original maximum Sharpe ratio portfolio with the same regularizations and constraints properly rescaled as in \eqref{eq:maximumSharperatio_l1l2}. In Section \ref{sec:empirics}, we will provide a more detailed and precise presentation. The advantage of \eqref{ep:maximumShaperation_final} is that even with these constraints and regularizations, it will be easy to solve numerically using QP methods.


\begin{table}[ht]
\centering
\caption{\textit{Summary of the total running time (in seconds) for $100$ rolling windows of three different portfolio optimization problems from our general framework described in Section \ref{sec:portfolio} for different dimensions of the problem ($N=10,20,50,500$), and two different levels of tolerance and precision in the optimizer: (i) default precision used in the OSQP package \protect\url{https://osqp.org} ; (ii) high precision with $10^{4}$ maximum iterations, and the absolute and relative tolerance set to $10^{-8}$. The latter is needed to generate convex portfolio frontiers in simulations for large $N$, and we use it in all our empirical studies. Computations are done using a single core of the AMD Ryzen Threadripper 2990WX Processor.}}
\label{table:time_table}
\resizebox{\columnwidth}{!}{%
\begin{tabular}{lrrrr|rrrr}
\hline
 &     \multicolumn{4}{c}{{Default Precision}} & \multicolumn{4}{c}{{High Precision}} \\
 & & & & & & & &\\
{Portfolio  Objective Function}&  N=10 &      N=20 &      N=50 &     N=500 & N=10 &      N=20 &      N=50 &     N=500 \\
\hline
Long-Only Min-Variance         &  0.14 &  0.16 &  0.20 &  10.77  &  0.15 &  0.16 &  0.21 &  11.96\\
& & & & \\
Long-Only Max-Sharpe Ratio          &  0.12 &  0.13 &  0.21 &  29.94  &  0.17 &  0.19 &  0.26 &  63.02 \\
& & & & \\
Long-Short Max-Sharpe Ratio with $\ell_1$$+\ell_2^2$  &  0.17 &  0.21 &  0.45 &  62.64 &  0.21 &  0.26 &  0.53 &  222.17\\
\hline
\end{tabular}
}
\end{table}

In our portfolio optimization framework, once the portfolio problems are turned into standard QP problems, we use the OSQP solver from \citet{osqp} to solve them. The solver uses ADMM algorithm for the optimization (see \citealp{boydADMM:2011} and references therein for the detail introduction of the algorithm). It is an open-source solver available at \url{https://osqp.org/docs/solver/index.html}. As summarized in Table \ref{table:time_table}, portfolios with 50 assets or less can be optimized with very h\section{Covariance Matrix Estimation}\label{sec:covariance_model}

In Markowitz's portfolio theory, mean vector $\boldsymbol{\mu}$ and covariance matrix $\boldsymbol{\Sigma}$ are assumed to be known. While in practice, they need to be estimated from the data, and a common approach is to use historical sample mean and sample covariance matrix with $iid$ assumption. However, this often leads to poor out-of-sample performance. As discussed in the introduction, there are different estimators that lead to better out-of-sample performance. In the empirical section below, using our portfolio optimization framework, we compare the sample covariance matrix with three other covariance matrix estimators in a rolling window scheme.

First, we use the classical linear shrinkage covariance matrix estimator \citet{Ledoit:04} defined as
\begin{equation}
    \hat{\boldsymbol{\Sigma}} = \hat{\delta}\hat{\mathbf{F}} + (1-\hat{\delta}\mathbf{S}),
\end{equation}
where $\mathbf{S} =\frac{1}{T} \sum_{t=1}^T(\mathbf{r}_t - \Bar{\mathbf{r}})(\mathbf{r}_t - \Bar{\mathbf{r}})^{\prime}$ and $\hat{\mathbf{F}}$ is the estimated structured covariance matrix. In particular, $\hat{\mathbf{F}} = trace(S)/N$, and $\hat{\delta}$ denotes the estimator of optimal shrinkage constant $\delta$. In practice, the authors propose to use $\hat{\delta} = \max\{0, \min\{\frac{\hat{\kappa}}{T}, 1\}\}$, where $\hat{\kappa} = \frac{\hat{\pi}-\hat{\rho}}{\hat{\gamma}}$, and $\hat{\pi}, \hat{\rho}$ and $\hat{\gamma}$ be estimated as $\hat{\pi} = \sum_{i=1}^N\sum_{j=1}^N\hat{\pi}_{ij}$ with $ \hat{\pi}_{ij} = \frac{1}{T}\sum_{t=1}^T\{(r_{it}-\Bar{r}_{i.})(r_{jt}-\Bar{r}_{j.}) - s_{ij}\}$, $\hat{\rho} = \sum_{i=1}^N\hat{\pi}_{ij} + \sum_{i=1}^N\sum_{j=1,j\neq i}^N\frac{\Bar{r}}{2}(\sqrt{\frac{s_{jj}}{s_{ii}}}\hat{\theta}_{ii,ij} + \sqrt{\frac{s_{ii}}{s_{jj}}}\hat{\theta}_{jj,ij})$ with $\hat{\theta}_{ii,ij} = \frac{1}{T}\sum_{t=1}^T\{(r_{it}-\Bar{r}_{i.})^2-s_{ii}\}\{(r_{it}-\Bar{r}_{i.})(r_{jt}-\Bar{r}_{j.}) - s_{ij}\}$, $\hat{\theta}_{jj,ij} = \frac{1}{T}\sum_{t=1}^T\{(r_{jt}-\Bar{r}_{j.})^2-s_{jj}\}\{(r_{it}-\Bar{r}_{i.})(r_{jt}-\Bar{r}_{j.}) - s_{ij}\}$, and $\hat{\gamma} = \sum_{i=1}^N\sum_{j=1}^N(f_{ij} - s_{ij})^2$.

In situations when the number of assets (variables) is commensurate with the sample size, the sample covariance matrix is usually not well-conditioned and not invertible. Getting the linear combination of the sample covariance matrix and identity matrix is a way to shrink the eigenvalues of the sample covariance matrix away from zero and towards their average in  $\hat{\mathbf{F}} = trace(S)/N$, with $\delta\in[0,1]$ denoting the shrinkage intensity. As a result, we get a well-conditioned covariance matrix estimator that has a lower mean-square error than the sample covariance matrix, and, in large dimensions, when $N$ grows asymptotically with $T$, it is a consistent estimator of the covariance matrix.

Second, we consider a more recent nonlinear shrinkage covariance matrix estimator---the quadratic inverse shrinkage estimator from \citet{ledoit2020analytical}. The estimator can be written as
\begin{equation}
    \hat{\boldsymbol{\Sigma}}_t := \mathbf{U}_t\hat{\boldsymbol{\Delta}}_t\mathbf{U}_t^{\prime},
\end{equation}
where $\hat{\boldsymbol{\Delta}}_t := diag(\hat{\delta}_t(\lambda_{1,t}), \ldots, \hat{\delta}_t(\lambda_{N,t}))$, and $\hat{\delta}_t$ is a real univariate function of $\lambda_{i,t}$ for $i = 1,\ldots, N$. $\boldsymbol{\lambda} = (\lambda_1, \ldots, \lambda_N)$ denotes the eigenvalues and $\mathbf{U}_t = [u_{1,t}, \ldots, u_{N,t}]$ are the corresponding eigenvectors. By introducing the nonlinear transformation (Hilbert transform) of the sample eigenvalues, this method helps with the curse of dimensionality. %The advantage of the quadratic inverse shrinkage estimator over the classical shrinkage estimator is that the nonlinear one can handle larger covariance matrices. 


Finally, the third approach to covariance matrix estimation that we use is based on the factor model.
In classical factor modeling from \cite{FAMA1993}, \cite{carhart1997persistence}, and \cite{fama2015five}, the returns are assumed to follow the linear model
\begin{equation} \label{eq:Fama_French_Factor_Model}
r_{i, t} = \alpha_{i} + \boldsymbol{\beta}_{i}' \mathbf{f}_{t} + \epsilon_{i, t},
\end{equation}
where $\mathbf{f}_t \in \mathbb{R}^{K\time 1}$ is a  vector of observed factors; $\epsilon_{i,t}$ is the mean zero noise capturing the idiosyncratic component that is uncorrelated with the observed factors; $\boldsymbol{\beta}_{i} \in \mathbb{R}^{K\times 1}$ is a vector of unknown factor loadings, and in most of these models $\alpha_{i}$ is set to $0$ for all assets $i$. This is a linear regression problem, and since the factors are assumed to be uncorrelated with the $\epsilon_{i,t}$, the corresponding covariance matrix of the returns decomposes into the part explained by the factors and the idiosyncratic part. If, in addition, the $\epsilon_{i,t}$ are assumed to be uncorrelated across assets, then the covariance matrix of the idiosyncratic component can be easily estimated from the regression residuals. Hence, the model can be used even when $N$ is much larger than $T$.

One of the limitations of the simple factor model in \eqref{eq:Fama_French_Factor_Model} is that the factors are assumed to be known and common for all the assets. In other words, they can only partially explain the risk, and they are not always strongly correlated with the true risk in a specific market scenario. Secondly, the factor loadings $\boldsymbol{\beta}_{i}$ are assumed to be constant over time.


An alternative approach that addresses the first limitation is to use the PCA model to get the latent factors directly from the portfolio of specific returns without extra information. However, PCA models still assume static loadings, and they lack accuracy and flexibility because they use only the information from the returns of assets in the portfolio.  

In a similar way \citet{KELLY2019501} motivated their IPCA model, where asset returns are assumed to admit the following factor structure 
\begin{equation}\label{eq:IPCA_model}
     r_{i, t+1} = \alpha_{i,t} + \boldsymbol{\beta}'_{i,t} \mathbf{f}_{t+1} + \epsilon_{i, t+1}, \quad \forall i=1,\ldots,N \mbox{ and } t=1,\ldots,T.
\end{equation}

The major differences compared with the classical factor models described above are:
\begin{itemize}
\item[(i)] the IPCA model, similarly to BARRA's factor model, assumes that the alphas $\alpha_{i,t}$ and the factor loadings $\boldsymbol{\beta}_{i,t} \in \mathbb{R}^{K\times 1}$ are time dependent; however, differently than the BARRA's factor model, it assumes that they are implicitly observed via
\[\alpha_{i,t} = \mathbf{z}_{i,t}^{\prime}\boldsymbol{\Gamma}_{\alpha} + v_{\alpha,i,t}, \quad \boldsymbol{\beta}_{i,t} = \mathbf{z}_{i,t}^{\prime}\boldsymbol{\Gamma}_{\beta} + \mathbf{v}_{\beta,i,t},\]
where $\mathbf{z}_{i,t}\in \mathbb{R}^{1\times L}$ are observed asset specific characteristics, and 
$\boldsymbol{\Gamma}_{\alpha}\in \mathbb{R}^{L\times 1}$ and $\boldsymbol{\Gamma}_{\beta}\in \mathbb{R}^{L\times K}$ are matrices of parameters estimated from the data;
\item[(ii)] due to the dimension reduction introduced via the matrix $\boldsymbol{\Gamma}_{\beta}\in \mathbb{R}^{L\times K}$,  the number of observed factors $L$ can be assumed to be much larger than the number of factor loadings $K$; 
\item[(iii)] the factors $\mathbf{f}_t\in \mathbb{R}^{K\times 1}$ are time dependent and they are estimated from the data;
\item[(iv)] this is a predictive model where the observable factors are lagged by one period relative to the returns that they explain;
    \item[(v)] $\epsilon_{i, t+1}$, $v_{\alpha,i,t}$, and $\mathbf{v}_{\beta,i,t}$ are the mean zero random noises that come from the estimation of factors and loadings. The $\epsilon_{i, t+1}$ reveal the firm-level risk, and $v_{\alpha,i,t}$ and $\mathbf{v}_{\beta,i,t}$ represent the residual between true factor model parameters and observable firm characteristics.
\end{itemize}



The intuition behind the IPCA model is that in a high dimensional factor model, too many characteristics will lead to too much noise and colinearity among factors which makes the results hard to interpret and deteriorates the out-of-sample performance of the model. Therefore, $\boldsymbol{\Gamma}_{\beta}$ is introduced to aggregate large dimensional characteristics to the linear combination of exposure risks, and any errors orthogonal to the dynamic loadings will fall into the $\mathbf{v}_{\beta,i,t}$. 


In the empirical analysis, we assume that $\boldsymbol{\Gamma}_{\alpha} = \mathbf{0}$ while focusing on the estimation of $\boldsymbol{\Gamma}_{\beta}$. Hence, for the restricted model ($\boldsymbol{\Gamma}_{\alpha} = \mathbf{0}$), we have
\begin{equation}
    r_{i, t+1} =  \mathbf{z}_{i,t}^{\prime}\boldsymbol{\Gamma}_{\beta} \mathbf{f}_{t+1} + \epsilon_{i, t+1}^{*},
\end{equation}
where $\epsilon_{i, t+1}^{*} = \epsilon_{i, t+1} +v_{\alpha,i,t}+\mathbf{v}_{\beta,i,t}\mathbf{f}_{t+1}$. We can derive this based on the vector form
\[ \mathbf{r}_{t+1} =  \mathbf{Z}_{t}^{\prime}\boldsymbol{\Gamma}_{\beta} \mathbf{f}_{t+1} + \boldsymbol{\epsilon_{t+1}^{*}}, \]
where $\boldsymbol{r_{t+1}}$ is an $N\times 1$ vector of assets returns, $\mathbf{Z}_t$ is an $N\times L$ vector of observable characteristics and $\boldsymbol{\Gamma}_{\beta}$ is an $L\times K$ mapping matrix, $\mathbf{f}_{t+1}$ is an $K\times 1$ vector of the combination latent factor. Then we can write the objective function of IPCA model as
\begin{equation}\label{eq: IPCA_objective}
    \min_{\boldsymbol{\Gamma}_{\beta}, F}\sum_{t=1}^{T-1}(\mathbf{r}_{t+1} - \mathbf{Z}_{t}^{\prime}\boldsymbol{\Gamma}_{\beta} \mathbf{f}_{t+1})^{\prime}(\mathbf{r}_{t+1} - \mathbf{Z}_{t}^{\prime}\boldsymbol{\Gamma}_{\beta} \mathbf{f}_{t+1}),
\end{equation}
with constrain $\boldsymbol{\Gamma}_{\beta}^{\prime}\boldsymbol{\Gamma}_{\beta} = \boldsymbol{I}_k$ and $\mathbf{FF}^{\prime} = diag(\lambda_1, \ldots, \lambda_k)$. To minimize the objective function \eqref{eq: IPCA_objective}, one iterates
\begin{subequations}\label{eq:IPCA_iterate}
    \begin{equation}\label{eq:IPCA_latent_factor}
        \hat{\mathbf{f}}_{t+1} = (\hat{\boldsymbol{\Gamma}}_{\beta}^{\prime}\mathbf{Z}_t^{\prime}\mathbf{Z}_t\hat{\boldsymbol{\Gamma}}_{\beta})^{-1}\hat{\boldsymbol{\Gamma}}_{\beta}^{\prime}\mathbf{Z}_t^{\prime}\mathbf{r}_{t+1}, \quad \text{for all} \,\, t,
    \end{equation}
 \mbox{and}   \begin{equation}\label{eq:IPCA_dimensional_reduction}
        \mbox{vec}(\hat{\boldsymbol{\Gamma}}_{\beta}^{\prime}) = (\sum_{t=1}^{T-1}\mathbf{Z}_t^{\prime}\mathbf{Z}_t\otimes \hat{\mathbf{f}}_{t+1}\hat{\mathbf{f}}_{t+1}^{\prime})^{-1}(\sum_{t=1}^{T-1}[\mathbf{Z}_t\otimes\hat{\mathbf{f}}_{t+1}^{\prime}]^{\prime}\mathbf{r}_{t+1}),
    \end{equation}
\end{subequations}
where $\otimes$ denotes the Kronecker product of matrices. Formula \eqref{eq:IPCA_latent_factor} shows that latent factors represent the coefficients of returns regressed on the latent loading matrix $\boldsymbol{\beta}_t \in \mathbb{R}^{N \times L}, t = (1, \ldots, T)$. Meanwhile, $\boldsymbol{\Gamma}_{\beta}$ denotes the regression coefficients of $\mathbf{r}_{t+1}$ on the combination of latent factors and firm characteristics. This first-order condition system does not have a close form solution, but it can be solved numerically by alternating the least squares method. Given the estimators of $\mathbf{f}_{t+1}$ and $\mathbf{\Gamma}_{\beta}$, we can obtain the estimator of the covariance matrix from
\begin{equation}
    \hat{\boldsymbol{\Sigma}}_{r_{t+1}} = \mathbf{Z}_t\hat{\boldsymbol{\Gamma}}_{\beta}\mbox{cov}(\hat{\mathbf{F}})\hat{\boldsymbol{\Gamma}}_{\beta}^{\prime}\mathbf{Z}_t^{\prime} + \hat{\mathbf{D}},
\end{equation}
where $\hat{\mathbf{D}} = diag(\mbox{cov}(\boldsymbol{r}_{t+1} - \mathbf{Z}_{t}^{\prime}\hat{\boldsymbol{\Gamma}}_{\beta} \hat{\mathbf{f}}_{t+1}))$
is the diagonal matrix for the covariance of the residuals. We will compare the $\hat{\boldsymbol{\Sigma}}_{r_{t+1}}$ with two aforementioned shrinkage covariance matrix estimators and the sample covariance matrix estimator in our portfolio analysis below using our QP framework.


\section{Empirical Results}\label{sec:empirics}


In our empirical analysis, we use split and dividend adjusted stock returns data from the Center for Research in Security Prices (CRSP). Our sample period begins in March 1957 and ends in December 2021, which corresponds to almost 65 years of monthly returns data. Naturally, in such a long sample, the number of assets changes over time. It ranges from 329 to 931 symbols every month, with an average number of 585 stocks. The universe is therefore similar to the S$\&$P500 index of mid- and large-cap stocks. Importantly for the portfolio performance evaluation, the dataset is free from survivorship bias as new firms are being listed and some existing ones are dropping from the dataset. For the estimation of the covariance matrix from the IPCA model, in addition to the returns data, we use the 94 stock-level monthly predictive characteristics from \citet{guxiu:20}--see Appendix for a full list of the factors together with the references where they were introduced.


\begin{figure}[ht]
\centering
\includegraphics[width=0.6\textwidth]{figure/rollingwindow.png}
\caption{\textit{Summary of the rolling window analysis. We use data going back to 1957 and slide 30 years of monthly returns to estimate the parameters, with monthly rebalancing and performance updates.}} \label{fig:rollingwindow_plot}
\end{figure}

In portfolio optimization, one often is interested in the evaluation of the performance of a given model in out-of-sample. For that purpose, one works with a so-called rolling window backtest analysis. In Figure \ref{fig:rollingwindow_plot}, we summarize our rolling window scheme. We divide the 65 years of the data into 30 years of training samples (1957-1986), and the remaining 35 years (1987-2021) for out-of-sample rolling window analysis with monthly reestimation of all the model parameters and portfolio weights optimization. Regarding the portfolio optimization objective function, from all the functions in our portfolio framework, we focus on a fully-invested maximum Sharpe ratio portfolio with long-short constraint and bounds for individual assets. These constraints are the only difference compared to the closed-form tangent portfolio used originally in \citet{KELLY2019501}. Similar or even more restrictive constraints are often imposed in practice, and we want to evaluate the impact of them on the performance. The maximum Sharpe ratio portfolio corresponds to maximizing the risk adjusted return of the portfolio strategy, i.e., it provides the biggest return for each unit of risk measured in terms of portfolio volatility. It is always in the central part of the portfolio efficient frontier and it is one of the most computationally demanding portfolio problems from our framework because it requires reparametrization into a larger dimensional problem. Hence, we consider it a good representative example for our whole portfolio framework. The corresponding optimization problem can be written as
\begin{align}\label{eq:maximumSharperatio_used}
&\arg\max_{\mathbf{w}\in\mathcal{W}_{LS}(\vartheta)}\frac{\mathbf{w}'\widehat{\boldsymbol{\mu}} - r_f}{\sqrt{\mathbf{w}'\widehat{\boldsymbol{\Sigma}}\mathbf{w}}},\\
\mathcal{W}_{LS}(\vartheta)=&\left\{\mathbf{w}\in\mathbb{R}^N:  \mathbf{w}'\mathbf{1}_N=1, \sum_{i:w_i>0} w_i \leq 1+\vartheta, \sum_{i:w_i<0} w_i\geq -\vartheta, L_j\leq w_j\leq U_j, \forall j \right\} \nonumber 
\end{align}
with $\vartheta=0.2$,  $r_f = 0$, $L_j=-0.08$, and $U_j=0.08$ for all $j=1,\ldots,N$, we have chosen these values as they approximate the standard industry setting, as described in detail in \cite{lunde2016econometric}. In order to solve it efficiently, we reformulate it into an equivalent QP problem from Section \ref{sec:maxSharpeRatioPortfolio} with constraints rewritten as in Section \ref{sec:long_short_portfolio}.

The $\widehat{\boldsymbol{\mu}}$ and $\widehat{\boldsymbol{\Sigma}}$ in 
\eqref{eq:maximumSharperatio_used} are obtained in the aforementioned rolling window exercise using monthly returns from the last 30 years of data. In particular, we use the sample mean for $\widehat{\boldsymbol{\mu}}$, and for $\widehat{\boldsymbol{\Sigma}}$ we use four different 
covariance matrix estimators from Section \ref{sec:covariance_model}, respectively. In the case of the sample covariance matrix, even with $30$ years of monthly returns, the estimated matrix is singular ($N>T$). In order to avoid numerical problems, we replace the zero eigenvalues with small positive numbers. These four models for the covariance matrix estimation, and the S\&P500 Index as the approximation for the market form the five different portfolio benchmarks. 

The last model used in our comparison is the $\ell_1$+$\ell_2^2$ regularized version of the same maximum Sharpe ratio portfolio with the sample mean estimator and the IPCA covariance matrix estimator, i.e.,
%  \begin{align}\label{eq:maximumSharperatio_l1_l2_used}
% &\arg\max_{\mathbf{w}\in\mathcal{W}_{LS}(\vartheta)}\frac{\mathbf{w}'\widehat{\boldsymbol{\mu}} - r_f}{\sqrt{\mathbf{w}'\widehat{\boldsymbol{\Sigma}}\mathbf{w}}}+ \lambda_1 \left\|\mathbf{w}\right\|_1 + \lambda_2 \left\|\mathbf{w}\right\|_2^2
% \end{align}

\begin{equation}\label{eq:maximumSharperatio_l1l2}
\arg\min_{\mathbf{w}\in\mathcal{W}}-\frac{\mathbf{w}'\boldsymbol{\mu} - r_f}{\sqrt{\mathbf{w}'\boldsymbol{\Sigma}\mathbf{w}}} + \delta_1\frac{1}{\mathbf{w}^{\prime}\boldsymbol{\mu}-r_f}\left\|\mathbf{w}\right\|_1 + \delta_2\frac{1}{(\mathbf{w}^{\prime}\boldsymbol{\mu} - r_f)^2}\left\|\mathbf{w}\right\|_2^2,
\end{equation}
where $\mathcal{W}=\left\{\mathbf{w}\in\mathbb{R}^N: \mathbf{1}_N\mathbf{w}=1, \mathbf{w}\geq \boldsymbol{0}\right\}$. The regularization terms in \eqref{eq:maximumSharperatio_l1l2} are $\left\|\mathbf{w}\right\|_1$ and $\left\|\mathbf{w}\right\|_2^2$. Using $\gamma = 1/\mathbf{w}'(\boldsymbol{\mu}-r_f\mathbf{1}_N)>0$, we can reparameterize this objective function to
\begin{equation}\label{eq:MaximumSharperatio_reparameterize}
\arg\min_{\mathbf{w}\in\mathcal{W}}-\frac{\gamma\mathbf{w}'(\boldsymbol{\mu} - r_f\mathbf{1}_N)}{\sqrt{\gamma\mathbf{w}'\boldsymbol{\Sigma}\mathbf{w}\gamma}} + \delta_1\left\|\gamma\mathbf{w}\right\|_1 + \delta_2\left\|\gamma\mathbf{w}\right\|_2^2,   
\end{equation}
then we obtain the simplified objective function
\begin{equation}\label{eq:MaximumSharperatio_model}
\arg\min_{[\tilde{\mathbf{w}},\gamma]'\in\tilde{\mathcal{W}}_{LS}(\vartheta)}{\tilde{\mathbf{w}}'\boldsymbol{\Sigma}\tilde{\mathbf{w}}} + \lambda_1\left\|\tilde{\mathbf{w}}\right\|_1 + \lambda_2\left\|\tilde{\mathbf{w}}\right\|_2^2,     
\end{equation}
where $\tilde{\mathcal{W}}_{LS}(\vartheta)=\{[\tilde{\mathbf{w}},\gamma]'\in\mathbb{R}^{N+1}: 1 = \tilde{\mathbf{w}}'(\boldsymbol{\mu} - r_f\mathbf{1}_N), \mathbf{1}_N'\tilde{\mathbf{w}}=\gamma, \sum_{i:\tilde{w}_i>0} \tilde{w}_i \leq \gamma(1+\vartheta), \sum_{i:\tilde{w}_i<0} \tilde{w}_i\geq -\gamma\vartheta, L_j\leq \tilde{w}_j\leq U_j, \forall j \}$, with $\vartheta=0.2$,  $r_f = 0$, $L_j=-0.08$, and $U_j=0.08$ for all $j=1,\ldots,N$. We solve it efficiently by transforming it into an equivalent QP problem from Section \ref{sec:maxSharpeRatioPortfolio} with constraints rewritten as in Section \ref{sec:long_short_portfolio}, and with additional regularization from Section \ref{sec:l1_l2_regularization}.



Figure \ref{fig:heatmap_benchmark} shows the heatmaps of relative improvement of the regularized maximum Sharpe ratio portfolio with IPCA estimated covariance matrix compared to the four benchmarks without regularization (sample covariance matrix, linear shrinkage, quadratic inverse shrinkage, and IPCA). The regularization strength parameters in the regularized maximum Sharpe ratio IPCA portfolio are fixed using the exponential grids $\lambda_1=10^{-6},\ldots,0.1$ and $\lambda_2 = 10^{-2},\ldots,5$. Empirically we observe that in these ranges of parameters, the regularization has the largest effect on the portfolio weights. Also, based on the heatmaps in Figure  \ref{fig:heatmap_benchmark} this is the range for $\lambda_1$ and $\lambda_2$, where the most improvement in terms of the out-of-sample Sharpe ratio from the regularized portfolio is realized.

Further, Figure \ref{fig:heatmap_benchmark} illustrates that for a broad range of regularization parameters, the regularized maximum Sharpe ratio portfolio with IPCA estimates of the covariance matrix performs much better than all the benchmark models and it has a large improvement relative to the portfolios from the original IPCA model without regularization. The best performing among the benchmarks is the portfolios with shrinkage covariance matrix estimators. In particular, the recently introduced non-linear shrinkage estimator by \citet{LedoitWolf:22} performs second to our regularized IPCA portfolio. Note that the shrinkage estimator does not use any additional information besides the history of stock returns. While the IPCA incorporates information about the asset specific factors. Although the dimension reduction is embedded in the IPCA model, the regularization of the portfolio weights in the portfolio optimization leads to large improvements, and without it, IPCA estimator is outperformed by all other benchmark estimators.

\begin{figure}
\centering
\resizebox*{14.5cm}{!}{\includegraphics{figure/heatmap.png}}
\caption{\textit{Four heatmaps of out-of-sample performance gains from monthly rebalanced portfolio in terms of the difference in the annualized Sharpe ratios of $\ell_1+\ell_2^2$ regularized maximum Sharpe ratio portfolio strategy with IPCA estimated covariance matrix for different regularization strength parameters $\lambda_1$ and $\lambda_2$.
In all the cases, we use the maximum Sharpe ratio portfolio estimated based on the last 30 years of data with short-selling constraint $\vartheta=0.2$ and no additional constraints.
\textbf{Top Left}: Difference in Sharpe ratios between IPCA$+\ell_1+\ell_2^2$ and the original IPCA.
\textbf{Top Right}:  Difference in Sharpe ratios between IPCA$+\ell_1+\ell_2^2$ and linear shrinkage.
\textbf{Bottom Left}:  Difference in Sharpe ratios between IPCA$+\ell_1+\ell_2^2$ and non-linear shrinkage.
\textbf{Bottom Right}:  Difference in Sharpe ratios between IPCA$+\ell_1+\ell_2^2$ and sample covariance matrix.}} \label{fig:heatmap_benchmark}
\end{figure}


Table \ref{table:quanstat_table} presents a  comparison of key performance metrics for the S\&P 500 index, three benchmark portfolios with sample covariance matrix, linear and non-linear shrinkage covariance matrix estimators discussed in \ref{sec:covariance_model}; and three different IPCA based portfolios.
The first IPCA portfolio is an unconstrained closed-form tangency portfolio with the covariance matrix estimated using the IPCA model, the second IPCA portfolio is similar but it has constraints on the portfolio weights as in \eqref{eq:maximumSharperatio_used}, and the third IPCA portfolio is also constraint as in \eqref{eq:maximumSharperatio_used} but with additional regularization on the portfolio weights as in \eqref{eq:MaximumSharperatio_model}.


The unconstrained IPCA portfolio performs best in most of the metrics. However, it is in general difficult to implement in practice because of the lack of constraints on the individual positions and no constraints on the long-short leverage. The same portfolio when such constraints are introduced (IPCA Constrained in Table \ref{table:quanstat_table}) performs much worse. In fact, almost all the key metrics drop significantly below the market performance. Finally, the proposed in this paper $\ell_1+\ell_2$ regularized IPCA portfolio restores the performance of the IPCA portfolio under realistic constraints. The strategy surpasses all the benchmarks, the other constrained IPCA portfolio, and the S\&P 500 index across a wide range of significant metrics, including Sharpe ratio, Sortino ratio, strategy alpha, shortest maximum drawdown period, daily Value-at-Risk (VaR), Expected Shortfall (cVaR), average drawdown size, and both the average and maximum number of days in drawdown periods.

These results show that the unconstrained IPCA model demonstrates strong performance primarily due to the inclusion of numerous unrealistic portfolios that investors would not typically consider. The IPCA model with long-short constraints and without regularization exhibits markedly inferior results compared to all other approaches and it is only through the application of regularization techniques that the IPCA model can effectively outperform the established benchmarks under realistic constraints on the portfolio weights.


\begin{table}
\caption{
\textit{Key performance metrics from a rolling window exercise on monthly returns from 1987-31-03 until 2021-31-12, for \textbf{First Column:} S\&P500 index as a long-only market benchmark. The maximum Sharpe ratio (tangency) optimal portfolio strategies with long-short constraints as in \eqref{eq:maximumSharperatio_used} and the covariance matrix estimated via: \textbf{Second Column:} Sample covariance matrix; \textbf{Third Column:} Linear shrinkage estimator; \textbf{Fourth Column:} Non-linear shrinkage estimator. In the 
\textbf{Fifth Column:} are the results for the closed form tangency portfolio with IPCA covariance matrix without any constraints; and in the \textbf{Sixth Column:} tangency portfolio with IPCA covariance matrix with long-short constraints as in \eqref{eq:maximumSharperatio_used}; and in the \textbf{Seventh Column:} results for the same portfolio but estimated with the best performing  $\ell_1$+$\ell_2^2$ regularization.}}
\resizebox{\textwidth}{!}{%
\begin{tabular}{lll | ll | lll}
\hline
{} &   Market  &  Sample &\multicolumn{2}{c |}{Shrinkage} &  \multicolumn{3}{c}{IPCA}  \\
{} &     S\&P 500 &      Cov. &      Linear &  Non-linear & Unconstrained &   Constrained    &   $\ell_1+\ell_2$ \\
\hline

Time in Market            &         1.0 &         1.0 &         1.0 &         1.0 &              1.0 &         1.0 &         1.0 \\
Cumulative Return         &       14.48 &        39.0 &       69.02 &       32.39 &           \textbf{139.08} &       23.33 &       66.34 \\
CAGR                     &        0.08 &        0.11 &        0.13 &        0.11 &             \textbf{0.15} &         0.1 &        0.13 \\
Sharpe                    &         0.6 &        0.68 &        0.71 &        0.59 &             \textbf{0.85} &        0.47 &        \textbf{0.85} \\
Sortino                   &        0.88 &         1.0 &        1.21 &        1.01 &             \textbf{1.44} &        0.81 &        1.29 \\
Omega                     &        1.93 &        1.76 &        1.86 &        1.69 &             \textbf{1.95} &         1.5 &        1.93 \\
Max Drawdown              &       -0.53 &       -0.59 &       -0.63 &       -0.61 &            \textbf{-0.51} &       \textbf{-0.51} &       -0.53 \\
Longest DD Days           &        2433 &        2316 &        1461 &        2101 &             1250 &        1980 &         \textbf{882} \\
Volatility (ann.)         &        \textbf{0.15} &        0.18 &         0.2 &        0.21 &             0.19 &        0.27 &        0.16 \\
Calmar                    &        0.16 &        0.19 &        0.21 &        0.17 &              \textbf{0.3} &        0.19 &        0.24 \\
Kurtosis                  &        \textbf{2.47} &        6.29 &       13.98 &       14.16 &              3.1 &       12.01 &        4.59 \\
Expected Yearly           &        0.08 &        0.11 &        0.13 &        0.11 &             \textbf{0.15} &         0.1 &        0.13 \\
Kelly Criterion           &        0.05 &        0.23 &        0.27 &        0.24 &             0.25 &        0.18 &        \textbf{0.29} \\
Daily Value-at-Risk       &       \textbf{-0.06} &       -0.08 &       -0.08 &       -0.09 &            -0.08 &       -0.12 &       \textbf{-0.06} \\
Expected Shortfall &       \textbf{-0.06} &       -0.08 &       -0.08 &       -0.09 &            -0.08 &       -0.12 &       \textbf{-0.06} \\
Avg. Drawdown             &       \textbf{-0.07} &       \textbf{-0.07} &       -0.09 &       -0.08 &            \textbf{-0.07} &       -0.14 &       \textbf{-0.07} \\
Avg. Drawdown Days        &         189 &         208 &         206 &         208 &              163 &         327 &         \textbf{158} \\
Beta                      &           - &        \textbf{0.02} &        0.07 &        0.06 &             0.05 &       -0.11 &        \textbf{0.02} \\
Alpha                     &           - &        0.12 &        0.14 &        0.12 &             \textbf{0.16} &        0.14 &        0.13 \\
Correlation               &           - &       \textbf{0.02\%} &       0.05\% &       0.05\% &            0.04\% &      -0.06\% &       \textbf{0.02\%} \\
Treynor Ratio             &           - &    1569.26\% &     1035.7\% &     517.66\% &         2544.32\% &    -221.43\% &    \textbf{3500.28\%} \\
\hline
\end{tabular}}
\label{table:quanstat_table}
\end{table}



Figure \ref{fig:cumu_return} summarizes the evolution over time of the key characteristics of our best performing maximum-Sharpe ratio portfolio with $\ell_1$+$\ell_2^2$ regularization portfolio relative to the returns of the market (S\&P500 index). The strategy has much higher cumulative and end-of-year returns---see Figures \ref{fig:cumu_return1a}, and \ref{fig:cumu_return1b}. Figure \ref{fig:cumu_return1c} shows that the 6-month rolling window Sharpe ratio remains mostly positive. The 6-month and 12-month rolling window market beta is close to zero for most of the months. The 6-month rolling window volatility in Figure \ref{fig:cumu_return1e} oscillates around $13\%$---a level acceptable by quantitative portfolio managers without additional (de-)leveraging, and finally, the underwater plot in Figure \ref{fig:cumu_return1f} shows all the drawdowns which, despite very good cumulative returns and a low beta of our portfolio strategy, happen mainly during the major market turbulences. Therefore, further refinement of our portfolio strategy could include better market-beta neutralization via our risk factor constraint listed in Section \ref{sec:minimumvarianceportfolio}, or using more advanced drawdown-beta neutralization from \citet{DingUryasev:22}. We leave these ideas for future extensions of our portfolio framework.

\begin{figure}
\centering
\begin{subfigure}{0.45\textwidth}
\includegraphics[width=\textwidth]{figure/cumreturn.png}
\caption{\textit{Cumulative Returns}}
\label{fig:cumu_return1a}
\end{subfigure}%
\begin{subfigure}{0.45\textwidth}
\includegraphics[width=\textwidth]{figure/EOYreturn.png}
\caption{\textit{EOY Returns}}
\label{fig:cumu_return1b}
\end{subfigure}\\

\begin{subfigure}{0.45\textwidth}
\includegraphics[width=\textwidth]{figure/rolling_shar.png}
\caption{\textit{Rolling Sharpe (6M)}}
\label{fig:cumu_return1c}
\end{subfigure}%
\begin{subfigure}{0.45\textwidth}
\includegraphics[width=\textwidth]{figure/rolling_beta.png}
\caption{\textit{Rolling Beta (6M\&12M)}}
\label{fig:cumu_return1d}
\end{subfigure}\\

\begin{subfigure}{0.45\textwidth}
\includegraphics[width=\textwidth]{figure/rolling_vol.png}
\caption{\textit{Rolling Volatility (6M)}}
\label{fig:cumu_return1e}
\end{subfigure}%
\begin{subfigure}{0.45\textwidth}
\includegraphics[width=\textwidth]{figure/underwater.png}
\caption{\textit{Underwater Plot}}
\label{fig:cumu_return1f}
\end{subfigure}
\caption{\textit{Six plots of different out-of-sample performance measures for our long-short maximum-Sharpe ratio with $\ell_1$+$\ell_2^2$ regularization and IPCA covariance matrix estimator portfolio strategy from \eqref{eq:MaximumSharperatio_model} versus the S\&P500 Index as a benchmark.}}
\label{fig:cumu_return}
\end{figure}

\section{Concluding Remarks}\label{sec:conclusions}

The paper proposes a unified framework for portfolio optimization using quadratic programming. Our framework includes several different classical portfolio optimization objective functions, portfolio constraints, and regularizations which are often employed in practice. It is suitable for rapid backtesting of large-scale portfolio problems with high accuracy and computational speed. 

The proposed framework is used to evaluate the IPCA factor model introduced in \citet{KELLY2019501} in the high dimensional portfolio problem of maximum Sharpe ratio portfolio with long-short constraints, and with/without regularization. We show that the portfolio performance of that model under long-short constraints can be significantly improved by considering a regularized portfolio optimization problem. As a result of the proposed portfolio wights regularization, the IPCA portfolio is much better than the (non-)linear shrinkage covariance matrix estimators.  

In future research, it would be interesting to review other portfolio problems from our general framework and the impact of different constraints on the portfolio performance of different covariance matrix estimators. Furthermore, it is important to explore which asset-specific factors in the IPCA model are essential for the returns predictability and portfolio performance. In general, sparsifying the factors should improve the model's signal-to-noise ratio. One can also achieve it by introducing sparse PCA extensions of the IPCA model.


\newpage 
%\begin{thebibliography}{}
\bibliographystyle{chicago}
\bibliography{main}
%\end{thebibliography}

\newpage

\section*{Appendix: Description of Asset Specific Factors}

In Table \ref{tab:94features}, we list the details of the firm specific characteristics used in the IPCA model.

\scriptsize
\begin{longtable}{llll}%[H]
\centering
No.&Acronym &Firm Characteristic &	Literature\\
\noalign{\smallskip}\hline\noalign{\smallskip}
1&absacc&Absolute accruals&\citep{bandy:10}\\
2&acc&Working capital accruals&\citep{sloan:96}\\
3&aeavol&Abnormal earnings announcement volume&\citep{lerman:08}\\
4&age&Years since first Compustat coverage&\citep{jiang:05}\\
5&agr&Asset growth&\citep{cooper:08}\\
6&baspread&Bid-ask spread&\citep{amihud:89}\\
7&beta&Beta&\citep{fama:73}\\
8&betasq&Beta squared&\citep{fama:73}\\
9&bm&Book-to-market&\citep{rosenberg:85}\\
10&bmia&Industry-adjusted book-to-market&\citep{asness:00}\\
11&cash&Cash holdings&\citep{palazzo:12}\\
12&cashdebt&Cash flow to debt&\citep{ou:89}\\
13&cashpr&Cash productivity&\citep{chandrashekar2009productivity}\\
14&cfp&Cash flow to price ratio&\citep{desai:04}\\
15&cfpia&Industry-adjusted cash flow to price ratio&\citep{asness:00}\\
16&chatoia&Industry-adjusted change in asset turnover&\citep{soliman:08}\\
17&chcsho&Change in shares outstanding&\citep{pontiff:08}\\
18&chempia&Industry-adjusted change in employees&\citep{asness:00}\\
19&chinv&Change in inventory&\citep{thomas:02}\\
20&chmom&Change in 6-month momentum&\citep{gettle:06}\\
21&chpmia&Industry-adjusted change in profit margin&\citep{soliman:08}\\
22&chtx&Change in tax expense&\citep{thomas:11}\\
23&cinvest&Corporate investment&\citep{titman:04}\\
24&convind&Convertible debt indicator&\citep{valta:16}\\
25&currat&Current ratio&\citep{ou:89}\\
26&depr&Depreciation / PP\&E&\citep{holthausen:92}\\
27&divi&Dividend initiation&\citep{michaely:95}\\
28&divo&Dividend omission&\citep{michaely:95}\\
29&dolvol&Dollar trading volume&\citep{chordia:01}\\
30&dy&Dividend to price&\citep{litzenberger:82}\\
31&ear&Earnings announcement return&\citep{kishore:08}\\
32&egr&Growth in common shareholder equity&\citep{richardson:05}\\
33&ep&Earnings to price&\citep{basu:77}\\
34&gma&Gross profitability&\citep{NOVYMARX:13}\\
35&grCAPX&Growth in capital expenditures&\citep{anderson:06}\\
36&grltnoa&Growth in long term net operating assets&\citep{fairfield:03}\\
37&herf&Industry sales concentration&\citep{hou:06}\\
38&hire&Employee growth rate&\citep{belo:14}\\
39&idiovol&Idiosyncratic return volatility&\citep{ali:03}\\
40&ill&Illiquidity&\citep{amihud:02}\\
41&indmom&Industry momentum&\citep{moskowitz:99}\\
42&invest&Capital expenditures and inventory&\citep{chen:10}\\
43&lev&Leverage&\citep{bhandari:88}\\
44&lgr&Growth in long-term debt&\citep{richardson:05}\\
45&maxret&Maximum daily return&\citep{bali:11}\\
46&mom12m&12-month momentum&\citep{jegadeesh1993mom}\\
47&mom1m&1-month momentum&\citep{jegadeesh1993mom}\\
48&mom36m&36-month momentum&\citep{jegadeesh1993mom}\\
49&mom6m&6-month momentum&\citep{jegadeesh1993mom}\\
50&ms&Financial statement score&\citep{mohanram:05}\\
51&mvel1&Size&\citep{banz:81}\\
52&mveia&Industry-adjusted size&\citep{asness:00}\\
53&nincr&Number of earnings increases&\citep{finn:99}\\
54&operprof&Operating profitability&\citep{fama2015five}\\
55&orgcap&Organizational capital&\citep{eisfeldt:13}\\
56&pchcapxia&Industry adjusted change in capital exp.&\citep{abarbanell:98}\\
57&pchcurrat&Change in current ratio&\citep{ou:89}\\
58&pchdepr&Change in depreciation&\citep{holthausen:92}\\
59&pchgmpchsale&Change in gross margin - change in sales&\citep{abarbanell:98}\\
60&pchquick&Change in quick ratio&\citep{ou:89}\\
61&pchsalepchinvt&Change in sales - change in inventory&\citep{abarbanell:98}\\
62&pchsalepchrect&Change in sales - change in A/R&\citep{abarbanell:98}\\
63&pchsalepchxsga&Change in sales - change in SG\&A&\citep{abarbanell:98}\\
64&ppchsaleinv&Change sales-to-inventory&\citep{ou:89}\\
65&pctacc&Percent accruals&\citep{hafzalla:10}\\
66&pricedelay&Price delay&\citep{hou:05}\\
67&ps&Financial statements score&\citep{piotroski:00}\\
68&quick&Quick ratio&\citep{ou:89}\\
69&rd&R\&D increase&\citep{eberhart:04}\\
70&rdmve&R\&D to market capitalization&\citep{guo:06}\\
71&rdsale&R\&D to sales&\citep{guo:06}\\
72&realestate&Real estate holdings&\citep{tuzel:10}\\
73&retvol&Return volatility&\citep{ang:06}\\
74&roaq&Return on assets&\citep{balakrishnan:10}\\
75&roavol&Earnings volatility&\citep{francis:04}\\
76&roeq&Return on equity&\citep{hou:15}\\
77&roic&Return on invested capital&\citep{brown:07}\\
78&rsup&Revenue surprise&\citep{kama:09}\\
79&salecash&Sales to cash&\citep{ou:89}\\
80&saleinv&Sales to inventory&\citep{ou:89}\\
81&salerec&Sales to receivables&\citep{ou:89}\\
82&secured&Secured debt&\citep{valta:16}\\
83&securedind&Secured debt indicator&\citep{valta:16}\\
84&sgr&Sales growth&\citep{lakonishok:94}\\
85&sin&Sin stocks&\citep{hong:09}\\
86&sp&Sales to price&\citep{barbee:96}\\
87&stddolvol&Volatility of liquidity (dollar trading volume)&\citep{chordia:01}\\
88&stdturn&Volatility of liquidity (share turnover)&\citep{chordia:01}\\
89&stdacc&Accrual volatility&\citep{bandy:10}\\
90&stdcf&Cash flow volatility&\citep{huang:09}\\
91&tang&Debt capacity/firm tangibility&\citep{almeida:07}\\
92&tb&Tax income to book income&\citep{lev:04}\\
93&turn&Share turnover&\citep{datar:98}\\
94&zerotrade&Zero trading days&\citep{liu:06}\\
\noalign{\smallskip}\hline
%\end{tabular}
\caption{\textit{Details of the firm specific characteristics used in the IPCA model. \textbf{Columns:} ID of the characteristic; acronym used in the paper; the name of the characteristic; reference to the paper that introduced it.}}
\label{tab:94features} 
\end{longtable}

\end{document}

