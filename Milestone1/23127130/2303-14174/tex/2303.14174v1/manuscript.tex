%%%% Generic manuscript mode, required for submission
%%%% and peer review
\documentclass[manuscript,screen,nonacm]{acmart}
%% Fonts used in the template cannot be substituted; margin 
%% adjustments are not allowed.
%%
%% \BibTeX command to typeset BibTeX logo in the docs
\AtBeginDocument{%
  \providecommand\BibTeX{{%
    \normalfont B\kern-0.5em{\scshape i\kern-0.25em b}\kern-0.8em\TeX}}}

%% Rights management information.  This information is sent to you
%% when you complete the rights form.  These commands have SAMPLE
%% values in them; it is your responsibility as an author to replace
%% the commands and values with those provided to you when you
%% complete the rights form.
%%\setcopyright{none}
\usepackage{hyperref}
\usepackage[
    type={CC},
    modifier={by-nc-sa},
    version={4.0},
]{doclicense}

%\copyrightyear{2023}
%\acmYear{2023}
% \acmDOI{XXXXXXX.XXXXXXX}
%% These commands are for a PROCEEDINGS abstract or paper.
% \acmConference[Conference acronym 'XX]{Make sure to enter the correct
  % conference title from your rights confirmation emai}{June 03--05,
  % 2018}{Woodstock, NY}
%
%  Uncomment \acmBooktitle if th title of the proceedings is different
%  from ``Proceedings of ...''!
%
%\acmBooktitle{Designing Platform Technology and Policy Simultaneously: A CHI'23 workshop} 
% \acmPrice{15.00}
% \acmISBN{978-1-4503-XXXX-X/18/06}


%%
%% Submission ID.
%% Use this when submitting an article to a sponsored event. You'll
%% receive a unique submission ID from the organizers
%% of the event, and this ID should be used as the parameter to this command.
%%\acmSubmissionID{123-A56-BU3}

%%
%% For managing citations, it is recommended to use bibliography
%% files in BibTeX format.
%%
%% You can then either use BibTeX with the ACM-Reference-Format style,
%% or BibLaTeX with the acmnumeric or acmauthoryear sytles, that include
%% support for advanced citation of software artefact from the
%% biblatex-software package, also separately available on CTAN.
%%
%% Look at the sample-*-biblatex.tex files for templates showcasing
%% the biblatex styles.
%%

%%
%% The majority of ACM publications use numbered citations and
%% references.  The command \citestyle{authoryear} switches to the
%% "author year" style.
%%
%% If you are preparing content for an event
%% sponsored by ACM SIGGRAPH, you must use the "author year" style of
%% citations and references.
%% Uncommenting
%% the next command will enable that style.
%%\citestyle{acmauthoryear}
\author{Camilo Sanchez}
\orcid{0000-0002-8486-031X}
\title{}
\email{camilo.sanchez@aalto.fi}
\affiliation{%
  \institution{Aalto University}
  %\streetaddress{Otaniementie 14}
  \city{Espoo}
  %\state{Uusimaa}
  \country{Finland}
  %\postcode{02150}
}

\author{Felix A. Epp}
\orcid{0000-0001-6252-7244}
\title{}
\email{felix.epp@aalto.fi}
\affiliation{
  \institution{Aalto University}
  %\streetaddress{Otaniementie 14}
  \city{Espoo}
  %\state{Uusimaa}
  \country{Finland}
  %\postcode{02150}
}


%%
%% end of the preamble, start of the body of the document source.
\begin{document}
\doclicenseThis
%%
%% The "title" command has an optional parameter,
%% allowing the author to define a "short title" to be used in page headers.
%\title{Enabling democratic futures visions through experiential futures and HCI prototypes}
%\title{Field Trials as Experiential Futures to Inform Policy Design}
\title{Experiential Futures In-the-wild to Inform Policy Design}

%%
%% The "author" command and its associated commands are used to define
%% the authors and their affiliations.
%% Of note is the shared affiliation of the first two authors, and the
%% "authornote" and "authornotemark" commands
%% used to denote shared contribution to the research.
%%
%% The code below is generated by the tool at http://dl.acm.org/ccs.cfm.
%% Please copy and paste the code instead of the example below.
%%
\begin{CCSXML}
<ccs2012>
   <concept>
       <concept_id>10003120.10003121.10003122.10011750</concept_id>
       <concept_desc>Human-centered computing~Field studies</concept_desc>
       <concept_significance>500</concept_significance>
       </concept>
   <concept>
       <concept_id>10003456.10003462.10003588.10003589</concept_id>
       <concept_desc>Social and professional topics~Governmental regulations</concept_desc>
       <concept_significance>500</concept_significance>
       </concept>
 </ccs2012>
\end{CCSXML}

\ccsdesc[500]{Human-centered computing~Field studies}
\ccsdesc[500]{Social and professional topics~Governmental regulations}

%%
%% Keywords. The author(s) should pick words that accurately describe
%% the work being presented. Separate the keywords with commas.
\keywords{Possible Futures, Anticipation, Prototypes, Experiential Futures}

%% A "teaser" image appears between the author and affiliation
%% information and the body of the document, and typically spans the
%% page.

% \begin{teaserfigure}
%   \includegraphics[width=\textwidth]{cover.jpeg}
%   \caption{Prototype "in-the-wild".}
%   \Description{A public transportation scene with a person in the foreground wearing a smart garment prototype}
%   \label{fig:teaser}
% \end{teaserfigure}

%\received{23 February 2023}
% \received[revised]{12 March 2009}
% \received[accepted]{5 June 2009}

%%
%% This command processes the author and affiliation and title
%% information and builds the first part of the formatted document.
\maketitle

\section{Messy Experiential Futures}
Technological innovation shapes the world around us at an accelerating pace. 
By building novel technologies, HCI materialises visions of the future in the present \cite{alvial-palavicinoFuturePracticeFramework2016, gustonDaddyCanHave2013, salovaaraScalingHCIPrototype2020}. 
The unintended consequences of these technologies then require policy makers to attend to them retroactively.
However, policy development has been struggling to keep up with the pace of the innovation cycles, which increases uncertainty and limits evidence-based regulation \cite{owen2013framework}.
To address the policy-novelty gap, R. von Schomberg proposed the Responsible Research Innovation (RRI) to drive science and technology innovation towards socially desirable goals and respond to the "grand challenges" of our time \cite{vonschombergVisionResponsibleResearch2013}.
While adopting the RRI poses challenges in terms of scale, temporality, and resources \cite{grimpeCloserDialoguePolicy2014a}, HCI can offer participatory, sustainable, and value-laden perspectives \cite{batesResponsibleInnovationAgenda2019} that policy designers could adopt to operate on the dimensions proposed by RRI \cite{owen2013framework}.
% However, HCI research is still overly optimistic about technological visions and primarily considers short-term time horizons. 
To better understand the impacts of technology in long-term and uncertain futures, HCI should consider other driving forces of society beyond the technological~\cite{salovaaraScalingHCIPrototype2020, mankoffLookingYesterdayTomorrow2013}
%To fill the futures gap in HCI, \cite{salovaaraEvaluationPrototypesProblem2017} propose looking at policy where possible futures are considered according to various levels of uncertainty. 

% OPPORTUNITY FROM FUTURES STUDIES
Future studies established that considering futures as uncertain and open-ended can help identify fluctuating knowledge still subject to change~\cite{adamFutureMattersAction2007, andersonPreemptionPrecautionPreparedness2010}, which relates to RRI by anticipating emerging socio-ethical concerns to alter policy goals \cite{sykesResponsibleInnovationOpening2013}.
Anticipation can be exercised through scenario planning, prototypes, life cycle assessments, role-playing and other methods~\cite{alvial-palavicinoFuturePracticeFramework2016} to adjust our assumptions and relations to possible futures~\cite{andersonPreemptionPrecautionPreparedness2010}.
%We can use the RRI anticipatory dimension to grasp this dynamic knowledge about the future.
%Anticipation is examining the present for possible futures that may emerge within it and adjusting our assumptions and relations to those futures \cite{andersonPreemptionPrecautionPreparedness2010}.
%Anticipation can be exercised through scenario planning, prototypes, life cycle assessments, role-playing and other methods. \cite{alvial-palavicinoFuturePracticeFramework2016}. 
%However, we must be critical of how 
As future narratives may reflect biases %and choices that are 
not representative of diverse socio-cultural groups \cite{kinsleyFuturesMakingPractices2012, lightCollaborativeSpeculationAnticipation2021, mazePoliticsDesigningVisions2019}, futures studies scholars are developing more participatory methods %to which the future belongs to 
\cite{candyTurningForesightOut2019, millerTransformingFutureAnticipation2018, dahle50KeyWorks1996}.
An emerging collection of participatory methods is Experiential Futures, which addresses the predominance of expert-based textual construction of futures by engaging socially diverse stakeholders in the future-making process through tangible, performative and interactive experiences of everyday life visions~\cite{gardunogarciaDesigningFutureExperiences2021, candyTurningForesightOut2019, jenkinsFutureSupermarketCase2020}. 

% This participatory factor in future-making resonates with RRI's aim to involve the discussions with the public earlier in the process when uncertainty is palpable, and emerging socio-ethical concerns can be identified to alter policy goals \cite{sykesResponsibleInnovationOpening2013}.
%Experiential Futures tries to address the expert predominance in future visions by engaging socially diverse groups of stakeholders in the future-making process \cite{gardunogarciaDesigningFutureExperiences2021}.
% Building on the proposition of more plural future visions, Ethnographic Experiential Futures (EXF) formalises the Experiential Futures concept into a cycle process to render everyday life future visions as tangible, performative and interactive \cite{candyTurningForesightOut2019}.
%Such experiences use transmedia to captivate the general audience interest in contrast to the expert-based textual construction of futures of traditional foresight \cite{candyTurningForesightOut2019}.

% HOW HCI CAN FILL THE GAP: AGENDA
Akin to adopting HCI methods in Responsible Innovation~\cite{batesResponsibleInnovationAgenda2019}, HCI provides opportunities to incorporate everyday life experiences and participatory approaches in foresight. 
Considering ubiquitous computing, HCI developed capacities to study the "messiness"~\cite{bellYesterdayTomorrowsNotes2007} of everyday life, which provides useful avenues for experiential futures.
% We especially want to emphasise the opportunities that future-oriented in-the-wild evaluations of technology could offer as another medium of experiential futures interventions. 
Field trials and design interventions study socio-political and cultural factors in users' ordinary life to examine the understanding and adoption of technologies~\cite{brownWildChallengesOpportunities2011, eppAdornedMemesExploring2022, salovaaraScalingHCIPrototype2020}. 
These field interventions as experiential futures may provide a new opportunity for policy design.
% Hereby, extending HCI's characteristic to evaluate novel technologies by addressing other sociopolitical factors enables the possibility to study possible futures in-action ~\cite{salovaaraEvaluationPrototypesProblem2017, eppReinventingWheelFuture2022, mankoffLookingYesterdayTomorrow2013, salovaaraScalingHCIPrototype2020}.
%In contrast, the urban labs~\cite{karvonenUrbanLaboratoriesExperiments2014} in policy design focus on the local factors that influence scientific knowledge and less on the systemic factors that determine collective practices. 
%Field trials offer situated knowledge but open the possibility of empirically studying possible futures~\cite{salovaaraEvaluationPrototypesProblem2017}.
% However, this requires HCI field studies to address different STEEPLE factors (societal, technological, economical, ethical, political, legal, and environmental) more holistically and how they may influence the future~\cite{eppReinventingWheelFuture2022, salovaaraScalingHCIPrototype2020}. 
% This type of future-oriented evaluation that examines users' understanding and practices with technologies can reveal a way to study the futures "in-action"~\cite{salovaaraScalingHCIPrototype2020}.

%In our ongoing research, we use interactive prototypes in experiential interventions to provide grounds for more plausible renderings of futures and allow users to experiment in real life and, thus, question new future imaginaries. 

%In our ongoing research towards developing methods to study possible futures in-action ~\cite{salovaaraEvaluationPrototypesProblem2017}. we diversify future vision through STEEPLE factors (societal, technological, economical, ethical, political, legal, and environmental) and ~\cite{eppReinventingWheelFuture2022, salovaaraScalingHCIPrototype2020}. 


\section{Case Study: Data Leakage in Sustainable Smart Garment Futures}
To this end, we are developing and conducting a field trial study, which fosters the idea of anticipatory practice~\cite{alvial-palavicinoFuturePracticeFramework2016}. 
Using the Future Ripples method \cite{eppReinventingWheelFuture2022}, we explored the trajectories of wearable technology futures from diverse STEEPLE factors (societal, technological, economical, ethical, political, legal, and environmental). Consequently, we identified the collision between data privacy and recycled electronics in the context of the Circular Economy.

While our study aims to materialise an experiential future in-the-wild, we iteratively developed future scenarios in parallel to the concretisation of an interactive prototype. 
% Building the prototype simultaneously with the scenario development helped us recognize design decisions that were not supporting the factors reflected in the original scenario.34
%Noticing ungrounded decisions in the prototype that did not support the factors of the original scenario made us question our concretisation of the future.
Similarly, to the “multiply” step in Ethnographic Experiential Futures \cite{candyTurningForesightOut2019}, the initial scenario was refined into four future scenarios by using consequences mapped out in five Future Ripples workshops and identifying the critical uncertainties~\cite{vanderheijdenScenariosArtStrategic2005} of surveillance and social divide.
The four scenarios, enriched with weak signals and trends, 
%and combined the critical uncertainties at different degrees of impact.
%Representing four possible futures related to the critical uncertainties 
guided us to reiterate the prototype and field intervention. 
%Heightening the features that represented these uncertainties in an everyday setting while we leaving other elements flexible to accommodate to varied factors within the four scenarios
% Being flexible to iterate on the prototype and mindful about different futures served us to consider the prototype as a knowledge artefact to better understand future trajectories and users' expectations \cite{zimmermanResearchDesignMethod2007}.
% The exercise of comprehensively informing an HCI evaluation study beyond the technological dimension facilitated the justification for design and methodological choices of the evaluation study.

By placing ubiquitous technology in the form of a tracking garment and a stationary device for reflections in the home for a 2-week period, we confront participants with a future scenario of data leakage through recycled electronics. This way, we aim for a pervasive experiential future that taps into the routines, values, and meanings that people give to their everyday interaction with technology before embedding it into society~\cite{wongElicitingValuesReflections2017, grimpeCloserDialoguePolicy2014a}. 
%Evaluating technology in terms of everydayness and socio-cultural factors opens a window to consider groups, practices, and psychological dynamics overlooked by more traditional and quantifying foresight and technological methodologies \cite{mazePoliticsDesigningVisions2019, slaughterFarewellAlternativeFutures2020}. 
%By asking the participants to adopt a ubiquitous technology that prompts them with a possible future scenario, this case study shows how in-the-wild studies can become a pervasive intervention of the future into the user's ordinary life.  
%Extending the evaluation of technology into mundane experiences not only provides us access to routines, values, and meanings that people give to their everyday interaction with technology.  
%Evaluating technology in terms of everydayness and socio-cultural factors opens a window to consider groups, practices, and psychological dynamics overlooked by more traditional and quantifying foresight and technological methodologies \cite{mazePoliticsDesigningVisions2019, slaughterFarewellAlternativeFutures2020}. 
The insights we hope to learn from the field intervention and reflections on the futures should inform policies regarding possible unintended consequences and peoples' values, fears, and assumptions. % about the future they experienced.

\section{Proposition for Further Inquiry}
Although there have been proposals in HCI to attend socio-political reflections on future visions through experiential futures \cite{jenkinsFutureSupermarketCase2020} and counterfactual actions \cite{forlanoSpeculativeHistoriesJust2023}, there is still much ground to empirically research HCI's role in the materialisation of collective future visions. 
This work could be a dualistic endeavour between policy design and HCI. On the one hand, policy design could benefit from more agile and responsive evaluation environments. On the other, HCI studies can improve the impact of their findings beyond their own discipline. 
For both fields, engaging in pluralistic futures rendering means questioning which and how futures are constructed, who are they benefiting, and how are the findings of these interventions interpreted towards other futures? 
For HCI's cross-disciplinary quality, adopting futuring methods to aid policy design should not be an upstream task.  
However, HCI itself should give more weight to its own capacity to study futures empirically and the responsibility this entails. 

%% The acknowledgments section is defined using the "acks" environment
%% (and NOT an unnumbered section). This ensures the proper
%% identification of the section in the article metadata, and the
%% consistent spelling of the heading.
\begin{acks}
This work has been supported by the Academy of Finland grant Future Methods (330124).
\end{acks}

%%
%% The next two lines define the bibliography style to be used, and
%% the bibliography file.
\bibliographystyle{ACM-Reference-Format}
\bibliography{manuscript}

\end{document}
\endinput
%%
%% End of file `sample-authordraft.tex'.
