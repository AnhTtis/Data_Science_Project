\cite{HD2021} presented the first results of our \mk observations of A2626, including online atlas pages of the \hi detected galaxies. In this work, we characterised \hi asymmetries in various ways and calculated \hi deficiencies. We then related these to local and global environment and specific galaxy properties such as SFR, stellar mass etc. As discussed in the review paper by \cite{cortese2021}, during the last few Gyrs in today's passive galaxies, it is inevitable that most of the primary quenching mechanisms must have played a role in shaping their properties (e.g. \citealt{mcgee2009, han2018}). Since multiple processes can be at play simultaneously, it is difficult to identify the dominant process observationally. Still, as mentioned before, by exploring various gas and star formation properties of galaxies, we can investigate which plausible physical mechanisms are acting on their \hi gas reservoirs. Mechanical gas removal processes like ram-pressure stripping and tidal stripping are likely to push or pull the \hi gas away from the stellar body, causing offset and asymmetric \hi discs, or uni-directional tails. Sometimes, however, when exposed to ram-pressure for a long time, \hi discs lose most of the gas from the outer disc and become truncated. Processes like starvation, harassment or thermal evaporation mainly occur in dense cluster environments and also give rise to truncated or small \hi discs, rather than offset \hi discs. So, if the \hi deficient galaxies have settled, small \hi discs, a gas removal process like ram-pressure stripping probably has happened a while ago and / or other more subtle processes may be at play such as starvation or thermal evaporation. In this section, we address some of these observational findings and questions raised by our analysis. 

\subsection{Trends with projected distance}
First of all, the clearest result we found while examining the \hi properties of the galaxies in A2626 is the increase of \hi deficient galaxies towards the cluster core in Fig. \ref{fig:HIdef_plot1}. This trend probably indicates an increase of gas removal efficiencies in the cluster core, similar to what previous observations have found (e.g. \citealt{Solanes2001, chung2009}). Moreover, \hi detected galaxies in the substructures are mainly found in the cluster outskirts. Interestingly, beyond R$_{200}$, both non-substructure and substructure galaxies in A2626 as well as the galaxies in the Swarm have a similar range of \hi deficiencies. This hints at pre-processing in some of the substructure galaxies in A2626 and in the Swarm galaxies, before they are fully exposed to the intra-cluster medium. 

 Next, we have investigated whether the shapes of the outer \hi discs are also affected by the different physical mechanisms driving \hi deficiencies in the various environments in and around A2626. In Fig. \ref{fig:HIdef_plot2}, we have explored how \hi deficiency vs. projected distance from the cluster centre depends on visual classes (top panel), \hi offset (middle panel), and \amod (bottom panel). In the top panel of Fig. \ref{fig:HIdef_plot2}, the VClass 3 or unsettled galaxies (stars) seem to have \hi deficiencies similar to field galaxies and even tend to be slightly more gas rich on average. Thus, these unsettled galaxies are still in an active stage of evolution and their \hi discs still contain a large gas reservoir for star formation. The occurrence of more \hi asymmetric galaxies in substructures of A2626 is possibly due to more effective tidal interactions among the neighbouring galaxies in the substructure environment than the non-substructure galaxies in A2626. But these interactions do not remove the \hi gas efficiently, making the galaxies in the substructures only moderately \hi deficient. This hints at ongoing pre-processing in the substructure galaxies in A2626 before they fall into the cluster core. Considering the most \hi deficient galaxies (HIdef$>$0.3) within $\sim$ 1.5 R$_{200}$, we find that 12 out of the 17 galaxies have disturbed \hi discs (VClass 2), indicating that they may be still under the influence of ram-pressure stripping. The other 5 galaxies have settled \hi discs (VClass 1), but their \hi discs seem to be small with respect to their stellar discs, indicating that they are in an  advanced stage of ram-pressure stripping or under the influence of processes like starvation or thermal evaporation. In the middle panel of Fig. \ref{fig:HIdef_plot2}, we do not observe any clear correlation between \hi deficiency and the amount of \hi offset. This can be understood by realizing that the direction of ram-pressure stripping is not always perpendicular to the line-of-sight and the small offsets induced by harassment would not have a preferred direction either. In the bottom panel of Fig. \ref{fig:HIdef_plot2}, we find no clear relation of \hi deficiency with \amod although the most \hi deficient galaxy is highly asymmetric.
 
\subsection{Trends with respect to the SFMS}
 In Fig. \ref{fig:SFR_Mstar_tdep_HIdef} and \ref{fig:SFR_Mstar_tdep_Amod}, we have explored the location of the A2626 and the Swarm galaxies with respect to the SFMS as well as the relation between their \hi depletion times and stellar masses. The fact that the galaxies in all three environments (both the substructure and non-substructure galaxies in A2626 and the Swarm galaxies) tend to be below the SFMS from \cite{cluver2020} means that both the cluster and group environments are inducing slightly lower SFR than the expected SFR for normal galaxies. There is no clear relation between \hi deficiency and offset from the SFMS, i.e. SF deficiency, for both the cluster and the Swarm galaxies. This implies that the mechanisms causing \hi deficiencies are probably not immediately impacting the SFR and vice versa, irrespective of the environment. 
 
 However, some galaxies (e.g. substructure galaxy 19 and non-substructure galaxy 88 in A2626 in Fig. \ref{fig:SFR_Mstar_tdep_HIdef}) have higher SFR than the SFMS. Their \hi morphologies appear to be fairly regular (insets of the top left panel in Fig. \ref{fig:SFR_Mstar_tdep_HIdef}) while they are moderately \hi deficient and have short \hi depletion times. Galaxy 88 was identified as a jellyfish candidate galaxy from the B-band optical image by \cite{Poggianti2016}. However, since its \hi morphology is regular and it is only moderately \hi deficient, this galaxy was later identified as a non-jellyfish galaxy in \cite{Deb2022}. We have further inspected \hi maps of some other outlier galaxies from the SFMS. Galaxy 150 is below the quenching threshold, yet it is not particularly \hi deficient. The \hi disc of this galaxy probably has a low column density, below the star formation limit which results in a large depletion time because of the lower SFR.
 
 We also investigated the \hi maps of the most \hi deficient galaxies in A2626 (galaxy 16, 65, 110, 128, see Fig. \ref{fig:SFR_Mstar_tdep_HIdef} and \ref{fig:HIdef_plot1}), all four of them located below the SFMS. Though the \hi disc of galaxy 65 seems small and regular, the other three galaxies have asymmetric \hi morphologies. As already shown with \hi contours overlaid on the optical colour images in Fig. \ref{fig:HIdef_plot1}, all these galaxies have small, offset, or truncated \hi discs with yellowish optical colours. As mentioned before, these three galaxies are below the SFMS, signifying low relative SFRs, although they are still above the star formation quenching threshold. However, because of their high \hi deficiencies, they have very short depletion times. Thus, all these four galaxies are plausibly in the last stages of exhausting their gas and on their way to drop below the star formation quenching threshold. 
 
 Focusing on the \hi morphologies of the most \hi deficient galaxies in the Swarm (galaxy 25, 60, 70, 105 in Fig. \ref{fig:SFR_Mstar_tdep_HIdef}), we note that all of them have nearby companions, and two of these galaxies (galaxy 60, 105) have asymmetric or offset \hi discs. From the optical colour, galaxy 105 seems dusty, that might have caused some internal reddening. Galaxy 60 is tidally interacting, with a visible stellar stream to the west, similar to the tidally interacting system Arp 282 \citep{Zaragoza-Cardie2018}. Though we do not have optical or \hi redshift information for the companions of galaxies 25 and 70, these galaxies are clearly experiencing tidal interactions, given their \hi morphologies.

Finally, we have investigated the relation between SF deficiency and \hi morphology using the three different morphological definitions (visual classes, \hi offset, \amod, see Fig. \ref{fig:SFR_Mstar_tdep_Amod}). The fact that there is no correlation between SF deficiency and \hi offset or \amod implies that an enhancement or quenching of their star formation rates are not directly related to their \hi morphologies or the environment in which they reside.  
 
\bigskip We conclude this discussion by noting that the \hi content and \hi morphologies of galaxies are useful tracers of various environmental effects. Although the galaxies are \hi deficient close to the cluster centre, those \hi deficient galaxies do not always have disturbed or unsettled \hi discs. Consequently, apart from processes like ram-pressure stripping and tidal interactions, more subtle processes like harassment and thermal evaporation may be at play in A2626. The presence of more \hi asymmetric galaxies in the substructures of A2626 indicates that tidal interactions between neighbouring galaxies are already pre-processing the substructure galaxies before they fall into the cluster core. The fact that SF deficiency is not related to the \hi deficiency or \hi morphology of galaxies in all three environments suggests that the physical mechanisms causing disturbed \hi morphologies take some time to affect the star formation activity. Although investigating the \hi deficiencies and morphologies of galaxies is important, the interpretation is difficult without ancillary data to better understand the effect of nurture on galaxy evolution.