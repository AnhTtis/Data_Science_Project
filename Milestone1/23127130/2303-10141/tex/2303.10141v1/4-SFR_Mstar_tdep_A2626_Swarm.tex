\begin{figure*}[t!]
 {
    \includegraphics[width=\textwidth]{images/SFR_Mstar_tdep_HIdef_Feb24.pdf} 
  } 
  \caption{\footnotesize{Top left panel: SFMS for the non-substructure (filled circles) and substructure (open circles) galaxies in A2626. The SFMS relation and quenching threshold are taken from \cite{cluver2020} which is calibrated using the WISE data and the same methodologies that is used for stellar mass and SFR calculations for the galaxies in our survey. The colour scale (shown in the right) represents \hi deficiency for the galaxies. The down-arrows are 2$\sigma$ upper limits on the SFR from WISE observations. Both the non-substructure and substructure galaxies seem to have slight lower SFR than the SFR expected for normal galaxies. In the top panels we show \hi maps overlaid on DECaLS colour images for some outlier galaxies. Top right panel: Similar to the plot on the top left, except it presents galaxies in the Swarm. Bottom left panel: \hi depletion time vs stellar mass of the non-substructure (filled circles) and substructure (open circles) galaxies in A2626. The colour scale presents the \hi deficiency of the galaxies. Bottom right panel: Similar to the bottom left plot, except it presents the galaxies in the Swarm.} }
  \label{fig:SFR_Mstar_tdep_HIdef} 
 \end{figure*}

\subsection{Galaxies on the SFMS }
 
We have combined the information on star formation rates (SFR), stellar masses with the \hi properties of the galaxies in A2626 and the Swarm to investigate the extent of the effect of different environments on the star formation properties of the \hi detected galaxies. The star formation activity and the stellar mass of a galaxy are related in a systematic way, described by a well-established scaling relation, the `star formation main sequence’ (SFMS, e.g. \citealt{brinchmann2004, Noeske2007, elbaz2007, speagle2014, tomczak2016}). We investigated the location of the galaxies in different environments (non-substructure and sub-structure galaxies in A2626, and the Swarm galaxies) with respect to the SFMS. We have adopted the SFMS and the star formation quenching threshold relation from \citet{cluver2020}, which are calibrated using WISE data. For consistency, we have also adopted stellar masses and SFRs based on WISE data derived with the same methodology as developed by \citet{cluver2014} (see Sect. 1.3).

We have also investigated the relation between \hi depletion time and stellar mass for galaxies in different environments as a function of \hi deficiency. Depletion time is the time required for a galaxy to deplete its \hi gas due to star formation and is defined as $t_{dep}$ = M$_{HI}$/ SFR. The depletion timescale of normal star-forming galaxies spans a range of $\sim $2-10 Gyr \citep{kennicutt1989, Kennicutt1998, bigiel2008}.

\subsection{HI deficiency versus SF deficiency }

 In Fig. \ref{fig:SFR_Mstar_tdep_HIdef}, we have plotted SFMS in the top two panels and the depletion time vs stellar mass in the bottom two panels for the galaxies in A2626 and the Swarm. To represent the two different environments in A2626, we have used filled and open circles respectively for the non-substructure and sub-structure galaxies in A2626 in the top and bottom left panel of Fig. \ref{fig:SFR_Mstar_tdep_HIdef}. In all the panels, we have colour coded the \hi deficiencies. The orange and black dotted lines are the SFMS and the quenching threshold from \citet{cluver2020} respectively. The downward pointing arrows are the 2$\sigma$ upper limits on the SFR. Interestingly, galaxies in all three environments -- the non-substructure and sub-structure galaxies in A2626, and the Swarm galaxies are mostly below the SFMS from \citet{cluver2020}, thus show an overall SF deficiency.
 
 \begin{figure}
 {\includegraphics[width=0.45\textwidth]{images/A2626_Swarm_offset_from_SFMS_MHI_Mstar_Feb24.pdf} 
  } 
  \caption{Offset from SFMS i.e. SF deficiency in A2626 and the Swarm galaxies. Galaxies above the horizontal dashed line lie below the SFMS and can therefore be considered to have a positive SF deficiency. The colour coding is similar to the colour coding in Fig. \ref{fig:Amod_offset_diff_env}.}
  \label{fig:offset_SFMS_A2626_Swarm} 
 \end{figure}
 
  \begin{figure*}[ht!]
 {\includegraphics[width=\textwidth]{images/SFR_Mstar_tdep_Amod_Feb24.pdf} 
  } 
  \caption{Similar to Fig. \ref{fig:SFR_Mstar_tdep_HIdef}, with additional information regarding \hi morphologies of galaxies in A2626 and the Swarm.} 
  \label{fig:SFR_Mstar_tdep_Amod} 
 \end{figure*}
 
 To examine if there is any difference in the distribution of galaxies in the three environments with respect to the SFMS, in Fig. \ref{fig:offset_SFMS_A2626_Swarm}, we have plotted the offset of the galaxies from the SFMS, i.e. the SF deficiency, in those three environments (green: non-substructure galaxies in A2626, magenta: sub-structure galaxies in A2626, orange: the Swarm galaxies). A positive SF deficiency (galaxies above the horizontal dotted line) indicates that these galaxies are below the SFMS and a negative SF deficiency (galaxies below the horizontal dotted line) indicates that these galaxies are above the SFMS. Along the horizontal axis we plotted M$_{HI}$/M$_{\star}$ to investigate if the SF deficiency relates to the relative \hi content of those galaxies. We do not observe any differences for the galaxies in different environments. This means that the star formation activity of galaxies with a similar stellar mass has no dependence on environment. 
 
 \subsection{\hi depletion time versus \hi morphology}
 In the bottom panels of Fig. \ref{fig:SFR_Mstar_tdep_HIdef}, we do not observe any clear trend between the depletion time and stellar mass for the galaxies, regardless of their environment. The downward pointing arrows are 2$\sigma$ upper limits on the depletion times. The figures show that there are no galaxies with large stellar masses and large depletion times. Massive galaxies tend to have a high SFR as expected from the SFMS and thus exhaust their \hi fuel relatively fast. There is no strong correlation of depletion time with \hi deficiency. However, we note that the most \hi deficient galaxies in A2626 have a short depletion time while the three galaxies in the Swarm with the longest depletion time (purple symbols) have a negative \hi deficiency. 
 
 \subsection{\hi morphology versus SF deficiency}
 
 In Fig. \ref{fig:SFR_Mstar_tdep_Amod}, we present similar SFMS plots in six different panels for three different environments, using the same symbols as Fig. \ref{fig:SFR_Mstar_tdep_HIdef}. The only differences are that in the two top panels of Fig. \ref{fig:SFR_Mstar_tdep_Amod} we have used different symbols for the three visual classifications, in the middle panels we have colour coded with \hi offset, and in the bottom panels we have colour coded with \amod. Here again, we do not observe any meaningful correlation between the SFR and \hi offset or \amod, as a function of environment or stellar mass.