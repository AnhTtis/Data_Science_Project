

\hi asymmetries are thought to be indicative of environmental effects. We aim to study them in relation with the local environment and then explore which method(s) are robust enough to assess \hi morphologies in the A2626 galaxies given the limitations of the available \mk data (good sensitivity though variable across the field of view, moderate linear resolution). To be able to do this, we first need to examine the detected sources carefully and prune the sample so that only the objects with sufficient sensitivity and resolution are considered for further analysis. 

\begin{figure}[t!]
\begin{center}
 {
    \includegraphics[width=0.45\textwidth]{images/3_histograms_A2626_Feb24.pdf}
  } 
  \caption{Histograms of three observational properties of galaxies in and around A2626. (a) The distribution of the peak signal-to-noise of the \hi maps. (b) The number of beams enclosed by the 3$\sigma$ contours of the \hi maps. (c) The distribution of 3$\sigma$ column density levels in the \hi maps.} 
  \label{fig:S2N_histogram} 
  \end{center}
 \end{figure}

In the observed volume, 219 \hi sources identified by \sofia \citep{Serra2015} have optical counterparts (see \citealt{HD2021}). Among those 219 sources, we first identified the uncertain \hi detections with low signal-to-noise, with only a few very high signal-to-noise pixels, or sources missing one peak of the double horn profile in the \sofia mask as revealed in the position-velocity diagram, and excluded them from our analysis. We then were left with 177 reliable \hi sources with a peak signal-to-noise in their \hi map $\geq$ 5, as indicated in the signal-to-noise in the atlas pages in \citep{HD2021}. Among those 177 galaxies, 108 galaxies are in the A2626 cluster and the Swarm, the number of \hi sources in each of these environments is given in Table \ref{tab:hi_detections}. In the left panel of Fig. \ref{fig:S2N_histogram}, we present a histogram with a distribution of the peak signal-to-noise values for those 177 galaxies. Therefore, this histogram gives an impression of the quality of the data. The median peak signal-to-noise of the galaxies in the surveyed volume is 8.6. The middle panel of Fig. \ref{fig:S2N_histogram} shows the histogram of the number of beams within the 3$\sigma$ contour of the \hi map (see \citealt{HD2021} for details). The median number of beams for these 177 galaxies is 3.1.

\renewcommand{\arraystretch}{1.1}
\begin{table}
    \centering
    \caption{\hi sources in different environments.}
    \begin{tabular}{lc}\hline
   Environment & No. of \hi sources                   \\ \hline
  Non-substructure in A2626  & 46\\
Substructures in A2626  & 30\\
 The Swarm  & 32\\
     \hline
    \end{tabular}
    \label{tab:hi_detections}
\end{table}

\subsection{Visual classifications}

\label{subsec:visual_classes}

At first we have classified our entire sample visually based on their \hi morphologies. For visual classification, we have assessed \hi morphologies based on the 3$\sigma$ column density contours. We have visually classified in three different classes: settled, one-sided / disturbed, and unsettled, similar to the work by \cite{Molnar2021}, who classified \hi morphologies of the galaxies in the Coma cluster.  

\begin{enumerate}
    \item \textbf{Settled sources} (VClass 1) - These are sources in which the \hi distribution is already `settled' i.e. they have symmetric \hi morphologies centred on a stellar disc and a  velocity gradient consistent with rotation.  Spatially unresolved \hi sources were also included to this category as well.

\item \textbf{Disturbed sources} (VClass 2) - These \hi sources either have a regular disc component with an additional one-sided asymmetry in their \hi morphology, or the \hi distribution is regular with a significant excess of \hi flux on one side of the stellar disc.

\item \textbf{Unsettled sources} (VClass 3) - These are \hi sources with an irregular \hi morphology and/or kinematics or with \hi flux extensions in multiple directions from the stellar disc, or an extreme 3D asymmetry / displacement relative to the
optical light with an unclear \hi disc component.

\end{enumerate}

Table. \ref{tab:hi_visual_classes} shows the number of galaxies with different visual classifications in different environments. Thus, substructure galaxies are more disturbed / unsettled than non-substructure galaxies in A2626 or the Swarm. Though the statistics is limited for the galaxies in these three environments, in substructures there are significantly more ($\sim$ 84\%) disturbed / unsettled galaxies than among the non-substructure galaxies (65\%) in A2626 or the Swarm  galaxies (60\%). In substructures, only 17\% $\pm$ 8\% of the galaxies have settled \hi discs while in the other two environments 37\% $\pm$ 8\% of the galaxies have settled \hi discs. This difference is at the 2$\sigma$ level. 

\renewcommand{\arraystretch}{1.4}
\begin{table*}
    \centering
        %\small\addtolength{\tabcolsep}{0pt}
    \caption{No. of galaxies with different Visual classifications in different environments.}
    \begin{tabular}{lccc}\hline 
   Environment & No. of & No. of  & No. of   \\ 
   & VClass 1 sources & VClass 2 sources & VClass 3 sources\\     \hline \hline
  non-substructure in A2626  & 16 (35\%) & 25 (54\%) & 5 (11\%)\\
Substructures in A2626   & 5 (16.7\%) & 23 (76.7\%)& 2 (6.6\%) \\
 The Swarm   & 13 (40\%) & 12 (38\%) & 7 (22\%) \\
     \hline
    \end{tabular}
    \label{tab:hi_visual_classes}
\end{table*}

\subsection{\hi offsets}

Another effective way to quantify asymmetry in the \hi distribution is to measure the offset of the \hi centre from the optical centre. \hi centres are calculated by fitting 2D Gaussians to the \hi maps while optical centres are taken from the SDSS.
A significant offset of the \hi centre from the optical centre may signify an external environmentally induced disturbance in the \hi morphology. We note that with a small \hi offset, a galaxy can still undergo subtle environmental processes like thermal evaporation, or starvation or can be at a late stage of ram-pressure stripping. Fig. \ref{fig:offset_histogram} presents the distributions of \hi offset for non-substructure, substructure, and the Swarm galaxies. All the galaxies in these three environments display a range of \hi offsets, signifying there is no obvious correlation of \hi offset with the different environments in and around A2626.

\begin{figure}[t]
\centering
 {
    \includegraphics[clip, trim=0cm 1cm 0cm 0cm, width=0.5\textwidth]{images/A2626_The_Swarm_offset_histogram_Feb24.pdf} 
  } 
  \caption{Distribution of \hi offset in kpc presented as histograms for the galaxies in and around A2626. Top, middle, and bottom panel shows \hi offset for the non-substructure (green), substructure (magenta) galaxies in A2626 and galaxies in the Swarm (orange) respectively.} 
  \label{fig:offset_histogram} 
 \end{figure}

\subsection{Quantifying asymmetries}

\textbf{Defining \hi asymmetry}
\newline \noindent
To investigate the environmental impact on the \hi discs of galaxies, it is important to quantify the severity of the asymmetry/ disturbance in the outer \hi distribution of individual galaxies. The optical/ infrared morphologies of galaxies are generally quantified with the concentration, asymmetry, and smoothness (CAS) parameters \citep{conselice2003} and the Gini and M20 parameters \citep{lotz2004}. \cite{holwerda2011, holwerda2013} used these parameters to quantify the \hi morphologies of galaxies from multiple \hi surveys (WHISP, LITTLE-THINGS, VLA-ANGST). In addition, \cite{holwerda2011, holwerda2013} and \cite{giese2016} used the definition of asymmetry from \cite{conselice2003}. \cite{giese2016} found that the asymmetry parameter is much more robust than any of the other parameters for \hi imaging data.
\begin{equation}
    A =  \frac{\sum_{i,j}^{N}|{I(i,j) - I_{180}(i,j)}|}{\sum_{i,j}^{N}|{I(i,j)}|}
\end{equation}
where \textit{I(i, j)} denotes the value of the pixel at the \textit{i, j} position of the original image of the galaxy and  $I_{180}(i, j)$ is the value of the pixel in the same position of the image rotated by 180$^{\circ}$ around the centre of the galaxy. So, asymmetry is measured by summing the pixel-by-pixel difference of the original and rotated image and normalising that by the total intensity in the image. Hence, the asymmetry index can have a value between 0 and 1. Asymmetries in the fainter outer regions of the \hi discs give a negligible contribution to the global asymmetry index compared to the brighter ($\sim$ 2 orders of magnitude or more) inner regions. Environmental processes, however, influence the extended outer parts of the \hi disc in a galaxy much more easily than the inner parts. In order to give proper weight to the asymmetries in the outer \hi disc, \cite{Lelli2014} introduced a modified asymmetry index ($A_{mod}$):

\begin{equation}
    A_{mod} = \frac{1}{N}\sum_{i,j}^{N} \frac{|{I(i,j) - I_{180}(i,j)}|}{|{I(i,j) + I_{180}(i,j)}|}
\end{equation}
where $N$ is the total number of pixels in the image. Thus, this definition of $A_{mod}$ normalizes the intensity differences at the position $(i, j)$ with the `local' intensity of the pixels, contrary to the total intensity of all the pixels in the \hi map. For example, for a highly lopsided galaxy with \hi emission exclusively on one side, $A_{mod}$ will obtain the maximum value of 1. However, we recognize that the value of \amod depends on some selection criteria.

(i) The value of \amod depends on the \hi column density above which it is measured. Depending on the observational setup and the sensitivity of the telescope, different \hi observations have different column density sensitivities. Moreover, even with the same observational settings, galaxies may have diverse 3$\sigma$ column density thresholds, depending on the local noise in the vicinity of the galaxy and the location of the galaxy with respect to the pointing centre which affects the local primary beam attenuation. 

(ii) The choice of the galaxy centre around which the \hi disc is rotated also effects the measurement of \amod. For disturbed and unsettled \hi distributions, the \hi centre often does not coincide with the optical centre. Moreover, the determination of the optical or the \hi centre also depends on the adapted method of calculation. 

(iii) The reliability of the \amod measurement also depends on how well resolved the galaxy is, both in terms of the beam size in kpc and the number of beams across the \hi map. If the galaxy is only marginally resolved and detected at low signal-to-noise, the asymmetry in the outer \hi disc might be dominated by noise, rather than the asymmetry induced by environmental processes.

\cite{bilimogga2022} have investigated the dependence of \amod on signal-to-noise, column density threshold and angular resolution of the \hi observations. Using mock galaxies from the EAGLE simulations \citep{schaye2015, crain2015}, they suggest an optimal combination of the observational constraints that are required for a robust measurement of the \amod value of the outer \hi disc of a galaxy: a column density threshold of 5 $\times$ 10$^{19}$cm$^{-2}$ or lower at a minimal signal-to-noise of 3 and a galaxy resolved with at least 11 beams. These are not `hard' limits as they depend on what one considers to be acceptable deviations of the measured \amod from the intrinsic \amod of a galaxy. Our observations do not reach this column density sensitivity and most galaxies are not resolved by 11 beams. Consequently, we measure \hi asymmetries of the inner \hi discs. 

\medskip

\noindent
\textbf{Measuring asymmetry in galaxies in and around A2626}

\label{subsec:measuring_hi_asym}
\noindent
To compute \amod, we need to adopt the position of the centre of the galaxy as well as a column density threshold above which we would consider the intensity of the \hi emission. All 219 \hi detected galaxies in our survey have optical counterparts within the footprints of the Sloan Digital Sky Survey (SDSS, \citealt{York2000, Aguado2018}) and the DECam Legacy Survey (DECaLS, \citealt{dey2019}). For our analysis, we adopted the optical centres from the SDSS for all 219 galaxies. We did not consider the \hi kinematic centres since most of the galaxies in our sample have disturbed/ unsettled \hi discs. Moreover, optical centres are generally better tracers of the dynamical centre of a galaxy, compared to the \hi centres. Using the optical centre sometimes results in a high \amod value for the galaxies with a strong offset between the \hi distribution and the stellar body, that may indicate a recent interaction or accretion event. 

It is crucial to calculate \amod above a specific column density for all the galaxies in the sample to make a fair comparison between different \amod values. But that specific column density will correspond to different signal-to-noise levels for different galaxies, depending on the local noise properties and the location of the galaxy within the primary beam. We have a range of 3$\sigma$ column density sensitivities (9-65) $\times$ 10$^{19}$ cm$^{-2}$ in our \mk observations \citep{HD2021}. To measure \amod reliably, we need adequate signal-to-noise images and we need as low an $N_{HI}$ limit as possible to measure the outer parts of galaxies well. The bottom panel of Fig. \ref{fig:S2N_histogram} shows the distribution of the 3$\sigma$ \hi column density levels for the reliable \hi detections with a minimum signal-to-noise $\geq$ 5 (177 galaxies among the 219 galaxies). 

The vertical dotted line in the bottom panel of Fig. \ref{fig:S2N_histogram} represents the \hi column density level of 25 $\times$ 10$^{19}$ cm$^{-2}$. Galaxies to the left of this dotted line have 3$\sigma$ \hi column densities below 25 $\times$ 10$^{19}$ cm$^{-2}$. Hence, their \amod values will be reliable when including pixels above 25 $\times$ 10$^{19}$ cm$^{-2}$ in the calculation of \amod. The galaxies to the right of the dotted line have 3$\sigma$ \hi column density levels above 25 $\times$ 10$^{19}$ cm$^{-2}$, thus, will have noisy pixels (below their own 3$\sigma$ contour) included in the \amod calculation when including pixels with values as low as 25 $\times$  10$^{19}$ cm$^{-2}$. By adopting a threshold of 25 $\times$ 10$^{19}$, we retain 71\% of the galaxies for which the \amod calculation only includes pixels with signal-to-noise >3 in the \hi map. Hence we consider the corresponding \amod values as reliable, provided the galaxy is sufficiently resolved (see below).

Notably, in a statistical sense, for a sample of galaxies with random inclinations, 25 $\times$ 10$^{19}$ cm$^{-2}$ corresponds to 1 \msun/pc$^{2}$ in a face-on orientation, which is the typical column density at which the diameters of \hi discs are measured. 1 \msun/pc$^{2}$ surface density corresponds to \hi column density of 12.5 $\times$ 10$^{19}$ cm$^{-2}$. Since the galaxies in our sample cover the full range of inclinations, by considering the \textit{observed} column density along the line-of-sight, we do not always consider the \hi column density perpendicular to the plane of those galaxies. Since most of the galaxies in our survey are barely resolved, we can not make meaningful inclination corrections for the \hi column densities of individual galaxies. So, we have adopted a statistical approach. The median inclination of a randomly oriented sample of galaxies is i = 60$^{\circ}$. This means that a face-on column density of 12.5 $\times$ 10$^{19}$ cm$^{-2}$ increases to a line-of-sight column density of 12.5/cos(60$^{\circ}$) = 25 $\times$ 10$^{19}$ cm$^{-2}$. Thus, 25 $\times$ 10$^{19}$ cm$^{-2}$ is also the practical level adopted for measuring \hi diameters.

Another important aspect for a reliable \amod measurement is angular resolution. For poorly resolved galaxies, \amod is not very meaningful. So, in addition to signal-to-noise, we need to restrict the sample to sufficiently resolved galaxies. We have considered the following criteria to identify the galaxies for which the \amod value is reliable: galaxies with a 3$\sigma$ column density level of $\leq$ 25 $\times$ 10$^{19}$ cm$^{-2}$, with a minimum peak signal-to-noise of $\geq$5, and resolved by more than 3 beams. Consequently, we are left with 33 galaxies for which the calculation of \amod is meaningful. 

\begin{figure}[t!]
 {
    \includegraphics[width=0.45\textwidth]{images/Amod_offset_Feb20_modified.pdf} 
  } 
  \caption{Distribution of \amod and \hi offset for the galaxies in and around A2626. The galaxies above the horizontal dashed line (\amod>0.4) are considered as \hi disturbed / unsettled galaxies. Galaxies on the right side of the vertical dashed line are considered as high \hi offset galaxies. \hi maps of a few example galaxies (see Section \ref{subsec:measuring_hi_asym}) are included as insets in the top panel. In the bottom right corner we reported the Pearson's cofficient and p-value for the relation between \amod and \hi offset. \amod and \hi offset seem to have a very strong correlation.} 
  \label{fig:Amod_offset_class1} 
 \end{figure}

\medskip

\noindent
\textbf{The relation between visual classifications, offsets, and \amod}

\noindent
We have explored the relation between our different methods of classifying \hi morphologies : visual classes, \hi offset, and \amod. Fig. \ref{fig:Amod_offset_class1} shows \amod as a function of offset between the \hi and the optical centres. We observe a strong correlation between \amod and offset for these 33 galaxies. The Spearman correlation coefficient is 0.81 which is shown in the bottom-left of Fig. \ref{fig:Amod_offset_class1}. This correlation is expected since a high offset between the \hi and optical centres would result in a high \amod value, though a galaxy with a high \amod value does not necessarily have a strongly offset \hi disc. For practical purposes, we are classifying a galaxy as \hi asymmetric when its \amod value is $\geq$0.4. Such an \hi asymmetric galaxy will be located above the horizontal dashed line in Fig. \ref{fig:Amod_offset_class1}. Similarly, we have defined galaxies to the right of the vertical dashed line in Fig. \ref{fig:Amod_offset_class1} as the high \hi offset galaxies.

These thresholds as indicated by the horizontal and vertical dashed lines have separated the galaxies in different classes. In particular, we found three different areas in the plot representing three different types of galaxies. 
\begin{enumerate}
    \item In the first quadrant (q1), we find high \amod, low \hi offset galaxies. For these galaxies, a high value of \amod is driven by the asymmetry in the outer \hi disc. So, these are \hi asymmetric galaxies. For example, in Fig. \ref{fig:Amod_offset_class1}, we observe an asymmetry in the outer \hi contour of galaxy 184 which results in a high \amod value. 
    
    
    \item In the second quadrant (q2), we find high \amod, high \hi offset galaxies. For these galaxies, a high \amod is driven by the high \hi offset (sometimes due to the combination of an \hi offset and an outer disc \hi asymmetry). For example, galaxy 105 in  Fig. \ref{fig:Amod_offset_class1} has a strongly offset \hi disc with respect to the optical centre.
     
    \item In the fourth quadrant (q4), we find low \amod, low \hi offset galaxies. These galaxies have an \hi disc that is neither asymmetric, nor offset. Hence, these are \hi normal galaxies (e.g. galaxy 76 in Fig. \ref{fig:Amod_offset_class1}), although their \hi discs can be small with respect to the stellar disc.

\end{enumerate}

We note that there are no galaxies in the third quadrant (q3), i.e. galaxies with low \amod and high \hi offset. This is expected since a high \hi offset would automatically cause a high \amod value. 

\begin{figure*}[t!]
 {
    \includegraphics[width=1\textwidth]{images/Amod_offset_vclass_Feb18.pdf} 
  } 
  \caption{Distribution of \amod and \hi offset as a function of visual classes for the galaxies in and around A2626. Left panel: settled galaxies (in circles), in top panel \hi maps of settled galaxies with high \amod. Right panel: disturbed (diamonds) and unsettled (stars) galaxies, in top panel \hi maps of disturbed and unsettled galaxies with low \amod.   } 
  \label{fig:Amod_offset_visual_class1} 
 \end{figure*}
Next, we have explored the same offset-\amod relation with our visual classifications of \hi asymmetries as an additional parameter. The left and right panel of Fig. \ref{fig:Amod_offset_visual_class1} show the galaxies with settled \hi discs as circles and the galaxies with disturbed / unsettled \hi discs as diamonds and stars respectively, based on our visual classifications as mentioned in Sec. \ref{subsec:visual_classes}. So, if there is a complete overlap of our visual classification with the \hi asymmetries based on the \amod values, all the diamond and star symbols should be above the \amod = 0.4 horizontal dashed line and all the circle symbols should be below the \amod = 0.4 horizontal dashed line. However, in the bottom left corner in the right panel of Fig. \ref{fig:Amod_offset_visual_class1}, two galaxies (31 and 45) that are visually classified as galaxies with a disturbed or an unsettled \hi disc, have nevertheless a low \amod value. These galaxies have a low \amod value since \amod is calculated within the green contour that corresponds to 25 $\times$ 10$^{19}$ cm$^{-2}$, which is above the 3$\sigma$ level (because it is inside the outermost black contour). These galaxies, however, have disturbed / unsettled \hi discs if we consider the outer \hi contour that corresponds to the 3$\sigma$ column density. This illustrates that the choice of 25 $\times$ 10$^{19}$ cm$^{-2}$ means that for some galaxies, the visual classification based on the outer \hi disc will not correspond to the \amod value calculated with pixels above the 25 $\times$ 10$^{19}$ cm$^{-2}$ contour. Furthermore, in the left panel of Fig. \ref{fig:Amod_offset_visual_class1}, we find three galaxies with \amod $>$0.4 yet they are visually classified to have settled \hi discs. For galaxies 147 and 120, the \hi disc seems somewhat offset from the optical centre. Galaxy 191 has an \hi cloud detached from the \hi disc, yet it was included in the calculation of \amod. However, the visual impression of the \hi discs of these galaxies are settled. This illustrates that the \amod values and the visual classifications both are valuable tracers of \hi morphologies and largely follow each other.

\begin{figure}[t!]
\centering
 {
\includegraphics[width=0.5\textwidth]{images/Amod_offset_diff_envs_Sep.pdf} 
  } 
  \caption{Distribution of \amod and \hi offset, colour coded with three different environments for the galaxies residing in and around A2626. The green, magenta, and orange circles represent non-substructure, substructure galaxies in A2626 and galaxies in the Swarm respectively. Substructure galaxies seem to have more asymmetric \hi discs with high \amod compared to non-substructure galaxies in A2626.} 
  \label{fig:Amod_offset_diff_env} 
 \end{figure}
 
 As a next step, we are considering the environment as a parameter in the same \amod vs \hi offset plot. We have used three different colours to represent the three different environments as described in Sec. \ref{subsec:diff_envs}. In Fig. \ref{fig:Amod_offset_diff_env}, green, magenta and the orange circles represent the non-substructure galaxies in A2626, the substructure galaxies in A2626, and the Swarm galaxies respectively. Although the statistics is limited, interestingly, all the galaxies in the substructure for which we have reliable \amod values are \hi asymmetric galaxies. At the same time, only a fraction of the non-substructure galaxies in A2626 have an \hi asymmetric disc. The occurrence of more \hi asymmetric galaxies in substructures of A2626 (all magenta symbols are above the dashed line) is possibly due to more effective tidal interactions in the substructure environment than the non-substructure galaxies in A2626 experience. However, galaxies in the Swarm and non-substructure galaxies in A2626 are found to have both disturbed and undisturbed \hi discs. 