%                                                                 aa.dem
% AA vers. 9.1, LaTeX class for Astronomy & Astrophysics
% demonstration file
%                                                       (c) EDP Sciences
%-----------------------------------------------------------------------
%
%\documentclass[referee]{aa} % for a referee version
%\documentclass[onecolumn]{aa} % for a paper on 1 column  
%\documentclass[longauth]{aa} % for the long lists of affiliations 
%\documentclass[letter]{aa} % for the letters 
%\documentclass[bibyear]{aa} % if the references are not structured 
%                              according to the author-year natbib style

%
%\documentclass[referee]{aa}  
%\documentclass[onecolumn]{aa}
\documentclass{aa}  

\usepackage{amsmath, amsfonts, amssymb}
\usepackage{graphicx,color}
\usepackage[dvipsnames]{xcolor}
%%%%%%%%%%%%%%%%%%%%%%%%%%%%%%%%%%%%%%%%
\usepackage{txfonts}
\usepackage{bbding}
\usepackage{pifont}
%%%%%%%%%%%%%%%%%%%%%%%%%%%%%%%%%%%%%%%%
\usepackage{hyperref}
\usepackage{xspace}
\usepackage{subfig}
\usepackage{caption}
\usepackage{rotating}
%\usepackage[options]{hyperref}
% To add links in your PDF file, use the package "hyperref"
% with options according to your LaTeX or PDFLaTeX drivers.

\newcommand{\hi}{\textsc{H$\,$i}\xspace}
\newcommand{\numunit}[2]{\mbox{\ensuremath{#1\,#2}\xspace}}
\newcommand{\mstar}{\ensuremath{\text{M}_\star}\xspace}
\newcommand{\mhi}{\ensuremath{\text{M}_\hi}\xspace}
\newcommand{\kms}{\ensuremath{\text{km}\,\text{s}^{-1}}\xspace}
\newcommand{\msun}{\ensuremath{\text{M}_\odot}\xspace}
\newcommand{\mk}{MeerKAT\xspace}
\newcommand{\meerlicht}{MeerLICHT\xspace}
\newcommand{\rvir}{\ensuremath{R_{200}}\xspace}
\newcommand{\caracal}{\texttt{CARAcal}\xspace}
\newcommand{\gipsy}{\texttt{GIPSY}\xspace}
\newcommand{\mhz}{\text{MHz}\xspace}
\newcommand{\mpc}{\text{Mpc}\xspace}
\newcommand{\pybdsf}{\texttt{PyBDSF}\xspace}
\newcommand{\sofia}{\texttt{SoFiA}\xspace}
\newcommand{\wfifty}{\ensuremath{\text{W}_{50}}\xspace}
\newcommand{\wtwenty}{\ensuremath{\text{W}_{20}}\xspace}
\newcommand{\decals}{DECaLS\xspace}
\newcommand{\mhilr}{\ensuremath{\mhi/L_r}\xspace}
\newcommand{\zcl}{\ensuremath{z_{\rm cl}}\xspace}
\newcommand{\scl}{\ensuremath{\sigma_{\rm cl}}\xspace}
\newcommand{\hwone}{The Swarm\xspace}
\newcommand{\amod}{\ensuremath{A_{mod}}\xspace}

\newcommand{\secref}[1]{Sec.~\ref{#1}}
\newcommand{\figref}[1]{Fig.~\ref{#1}}
\newcommand{\tabref}[1]{Table~\ref{#1}}

\graphicspath{{images/}}

\newcommand{\xmark}{\ding{55}}%

%
\begin{document} 


   \title{An \hi\ story of galaxies in Abell 2626 and beyond}

   %\subtitle{I. Overviewing the $\kappa$-mechanism}
   
   \author{T. Deb,\inst{1,2}\thanks{contact: tirna1106@gmail.com}
          M.A.W. Verheijen,\inst{1} 
          J.M. van der Hulst\inst{1}
          }
\titlerunning{\hi\ story of galaxies in Abell 2626 and beyond}
   \authorrunning{Deb et al.}
   
  \institute{Kapteyn Astronomical Institute, University of Groningen, Landleven 12, 9747 AV Groningen, The Netherlands
  \and
  Department of Physics and Astronomy, University of the Western Cape, Robert Sobukwe Road, Bellville 7535, South Africa}


  

   \date{Received xxx; accepted xxx}

% \abstract{}{}{}{}{} 
% 5 {} token are mandatory
 
  \abstract
  {To study the effects of environment on galaxies we use \hi observations of galaxies in and around the cluster Abell 2626 (A2626). The cluster can effectively be divided in three different environments: the cluster itself, a group environment in the periphery of the cluster (we call it the Swarm) and substructure in the cluster itself. We use these to study the dependence of galaxy properties on environment. }
  % aims heading (mandatory)
   {We have explored the relationship between \hi deficiency, \hi morphology, and star formation deficiency for the galaxies in and around the A2626 galaxy cluster to investigate the environmental effects on those properties. } 
    % methods heading (mandatory)
   {To quantify asymmetries of the outer \hi disc of a galaxy, we used 1) three visual classes based on the outermost reliable \hi contour (settled, disturbed, unsettled \hi discs), 2) the offset between the \hi centre and the optical centre of a galaxy, and 3) the modified asymmetry parameter Amod as defined by Lelli et al. (2014). }
   % results heading (mandatory)
   {The \hi deficiency of a galaxy is strongly correlated with the projected distance from the centre of A2626. Furthermore, substructure galaxies tend to be more asymmetric than the isolated galaxies in A2626, plausibly because of more efficient tidal interactions within substructures than outside substructures. Moreover, asymmetric, offset, and smaller \hi discs are not necessarily the result of the cluster environment, as they are also observed in substructures in A2626 and in the Swarm. This signifies that ‘pre-processing’ of the \hi discs of galaxies in groups or substructures plays an important role, together with the ‘processing’ in the cluster environment. Finally, the galaxies in all three environments have slightly lower star-formation rates (SFRs) than the typical SFR for normal galaxies as manifested by their offset from the star formation main sequence, implying effective gas removal mechanisms in all three environments.}
  % context heading (optional)
  {} %leave it empty if necessary  
  
 % conclusions heading (optional), leave it empty if necessary 
   %{}

   \keywords{Galaxy: evolution --
                galaxies: clusters: intracluster medium
               }

   \maketitle
%
%-------------------------------------------------------------------

 \section{Introduction}
   \label{sec:intro}
    
    
  
  \section{Introduction}
\IEEEPARstart{T}{he} method Neural Radiance Fields (NeRF)~\cite{mildenhall2020nerf} is proposed for photorealistic novel view synthesis. Given many views of the scene, it creates implicit multi-view geometry and learns for view synthesis. However, it has poor generalizations to new scenes and requires retraining or fine-tuning on each scene. 
 
 Recent work~\cite{Yu_2021_CVPR,Trevithick_2021_ICCV} has explored the ways of using a single image to train NeRF. They introduce a convolutional feature encoder to learn the image representation which gives it some limited generalization abilities to unseen scenes.  But, without fine-tuning, these methods produce many floats and artifacts in rendering novel views. 
 
  Multi-Plane Images (MPI) representation that learns multiple RGB images from a single image is also used in \cite{Wu_2021_ICCV,Tucker_2020_CVPR,wu2022remote} for  novel view synthesis. However, MPI heavily relies on the qualities of the planar images and needs plenty of image planes to avoid blurs. There is no strong 3D geometry constraint and it fails in many complex scenes.
  
  MINE~\cite{Li_2021_ICCV2} introduces the volume rendering of NeRF into the MPI. It runs faster and produces better depth rendering quality compared with single-view NeRFs~\cite{Yu_2021_CVPR,Trevithick_2021_ICCV}. However, the rendering quality heavily relies on the number of image planes. It needs high-resolution 4D volumes to store the 4-channel  (RGB and volume density) image planes that cost a large amount of GPU memory in both training and 
 prediction.  
 

 
 \begin{figure}[t]
\setlength{\abovecaptionskip}{7pt}
\setlength{\belowcaptionskip}{0pt}
	\centering
% 	\subfigure[MINE (PSNR:14.9)]{  % for AAAI
	\subfloat[MINE (PSNR:14.9)]{
%			\centering
			\includegraphics[width=0.23\textwidth]{figure/intro/DJI_20200223_163206_598_0_MINE.png}
%			\label{subfig:pixelnerf}
	}\subfloat[MINE (depth)]{
%			\centering
			\includegraphics[width=0.23\textwidth]{figure/intro/MINE_disp.png}
%			\label{subfig:mpi}
	}
	\\[-3mm]
	\subfloat[Ours (PSNR:17.0)]{
%			\centering
			\includegraphics[width=0.23\textwidth]{figure/intro/DJI_20200223_163206_598_0_ours.png} 
	}\subfloat[Ours (depth)]{
%			\centering
			\includegraphics[width=0.23\textwidth]{figure/intro/ours_disp.png}
	}
	\caption{Comparison with state-of-the-art methods. (a-b) RGB and depth rendering results of  \cite{Li_2021_ICCV2}. It produces many blurs and floats in the occluded regions and at the object/depth edges. 
	(c-d) Our method employs a joint rendering mechanism that preserves more image details and predicts sharp depth edges.}
	\label{fig:performance_illustration}
\end{figure}
 
 In this paper, we propose a joint rendering mechanism that takes the MPI strategy for coarse sampling proposals and the MLP\&volume-based rendering~\cite{mildenhall2020nerf} for fine sampling and rendering. Then, both the coarse point samples and the fine samples are combined according to their geometry distribution to realize a more accurate joint rendering. More importantly, we introduce a depth teacher net that serves as the guidance for the joint rendering. The monocular depth teacher predicts dense pseudo depth maps that assist the consistent 3D geometry learning between the MPI, the fine volume, and the joint rendering. It also boosts the multi-view geometry consistency between the source view and the target novel views that 
helps handle the occlusions, reduce the blurs and floats, and render accurate depths. 
 
In the experiments,  we verify the effectiveness of our method on three challenging real-scene datasets (RealEstate10K~\cite{zhou2018stereo}, NYU~\cite{silberman2012indoor} and  NeRF-LLFF~\cite{mildenhall2020nerf}) for novel view synthesis or depth estimation. Given a single image as input, our method is shown able to produce higher qualities in both the RGB image rendering and depth map prediction. It far outperforms state-of-the-art methods~\cite{Li_2021_ICCV2,Yu_2021_CVPR} with improvements of 5$\sim$20\% in PSNR and SSIM for the RGB rendering and reduces 20$\sim$50\% of the errors for the depth prediction.
  
  %--------------------------------------------------------------------
\section{Classifying \hi morphologies
}
 \label{sec:hi_morphologies}
 
 

\hi asymmetries are thought to be indicative of environmental effects. We aim to study them in relation with the local environment and then explore which method(s) are robust enough to assess \hi morphologies in the A2626 galaxies given the limitations of the available \mk data (good sensitivity though variable across the field of view, moderate linear resolution). To be able to do this, we first need to examine the detected sources carefully and prune the sample so that only the objects with sufficient sensitivity and resolution are considered for further analysis. 

\begin{figure}[t!]
\begin{center}
 {
    \includegraphics[width=0.45\textwidth]{images/3_histograms_A2626_Feb24.pdf}
  } 
  \caption{Histograms of three observational properties of galaxies in and around A2626. (a) The distribution of the peak signal-to-noise of the \hi maps. (b) The number of beams enclosed by the 3$\sigma$ contours of the \hi maps. (c) The distribution of 3$\sigma$ column density levels in the \hi maps.} 
  \label{fig:S2N_histogram} 
  \end{center}
 \end{figure}

In the observed volume, 219 \hi sources identified by \sofia \citep{Serra2015} have optical counterparts (see \citealt{HD2021}). Among those 219 sources, we first identified the uncertain \hi detections with low signal-to-noise, with only a few very high signal-to-noise pixels, or sources missing one peak of the double horn profile in the \sofia mask as revealed in the position-velocity diagram, and excluded them from our analysis. We then were left with 177 reliable \hi sources with a peak signal-to-noise in their \hi map $\geq$ 5, as indicated in the signal-to-noise in the atlas pages in \citep{HD2021}. Among those 177 galaxies, 108 galaxies are in the A2626 cluster and the Swarm, the number of \hi sources in each of these environments is given in Table \ref{tab:hi_detections}. In the left panel of Fig. \ref{fig:S2N_histogram}, we present a histogram with a distribution of the peak signal-to-noise values for those 177 galaxies. Therefore, this histogram gives an impression of the quality of the data. The median peak signal-to-noise of the galaxies in the surveyed volume is 8.6. The middle panel of Fig. \ref{fig:S2N_histogram} shows the histogram of the number of beams within the 3$\sigma$ contour of the \hi map (see \citealt{HD2021} for details). The median number of beams for these 177 galaxies is 3.1.

\renewcommand{\arraystretch}{1.1}
\begin{table}
    \centering
    \caption{\hi sources in different environments.}
    \begin{tabular}{lc}\hline
   Environment & No. of \hi sources                   \\ \hline
  Non-substructure in A2626  & 46\\
Substructures in A2626  & 30\\
 The Swarm  & 32\\
     \hline
    \end{tabular}
    \label{tab:hi_detections}
\end{table}

\subsection{Visual classifications}

\label{subsec:visual_classes}

At first we have classified our entire sample visually based on their \hi morphologies. For visual classification, we have assessed \hi morphologies based on the 3$\sigma$ column density contours. We have visually classified in three different classes: settled, one-sided / disturbed, and unsettled, similar to the work by \cite{Molnar2021}, who classified \hi morphologies of the galaxies in the Coma cluster.  

\begin{enumerate}
    \item \textbf{Settled sources} (VClass 1) - These are sources in which the \hi distribution is already `settled' i.e. they have symmetric \hi morphologies centred on a stellar disc and a  velocity gradient consistent with rotation.  Spatially unresolved \hi sources were also included to this category as well.

\item \textbf{Disturbed sources} (VClass 2) - These \hi sources either have a regular disc component with an additional one-sided asymmetry in their \hi morphology, or the \hi distribution is regular with a significant excess of \hi flux on one side of the stellar disc.

\item \textbf{Unsettled sources} (VClass 3) - These are \hi sources with an irregular \hi morphology and/or kinematics or with \hi flux extensions in multiple directions from the stellar disc, or an extreme 3D asymmetry / displacement relative to the
optical light with an unclear \hi disc component.

\end{enumerate}

Table. \ref{tab:hi_visual_classes} shows the number of galaxies with different visual classifications in different environments. Thus, substructure galaxies are more disturbed / unsettled than non-substructure galaxies in A2626 or the Swarm. Though the statistics is limited for the galaxies in these three environments, in substructures there are significantly more ($\sim$ 84\%) disturbed / unsettled galaxies than among the non-substructure galaxies (65\%) in A2626 or the Swarm  galaxies (60\%). In substructures, only 17\% $\pm$ 8\% of the galaxies have settled \hi discs while in the other two environments 37\% $\pm$ 8\% of the galaxies have settled \hi discs. This difference is at the 2$\sigma$ level. 

\renewcommand{\arraystretch}{1.4}
\begin{table*}
    \centering
        %\small\addtolength{\tabcolsep}{0pt}
    \caption{No. of galaxies with different Visual classifications in different environments.}
    \begin{tabular}{lccc}\hline 
   Environment & No. of & No. of  & No. of   \\ 
   & VClass 1 sources & VClass 2 sources & VClass 3 sources\\     \hline \hline
  non-substructure in A2626  & 16 (35\%) & 25 (54\%) & 5 (11\%)\\
Substructures in A2626   & 5 (16.7\%) & 23 (76.7\%)& 2 (6.6\%) \\
 The Swarm   & 13 (40\%) & 12 (38\%) & 7 (22\%) \\
     \hline
    \end{tabular}
    \label{tab:hi_visual_classes}
\end{table*}

\subsection{\hi offsets}

Another effective way to quantify asymmetry in the \hi distribution is to measure the offset of the \hi centre from the optical centre. \hi centres are calculated by fitting 2D Gaussians to the \hi maps while optical centres are taken from the SDSS.
A significant offset of the \hi centre from the optical centre may signify an external environmentally induced disturbance in the \hi morphology. We note that with a small \hi offset, a galaxy can still undergo subtle environmental processes like thermal evaporation, or starvation or can be at a late stage of ram-pressure stripping. Fig. \ref{fig:offset_histogram} presents the distributions of \hi offset for non-substructure, substructure, and the Swarm galaxies. All the galaxies in these three environments display a range of \hi offsets, signifying there is no obvious correlation of \hi offset with the different environments in and around A2626.

\begin{figure}[t]
\centering
 {
    \includegraphics[clip, trim=0cm 1cm 0cm 0cm, width=0.5\textwidth]{images/A2626_The_Swarm_offset_histogram_Feb24.pdf} 
  } 
  \caption{Distribution of \hi offset in kpc presented as histograms for the galaxies in and around A2626. Top, middle, and bottom panel shows \hi offset for the non-substructure (green), substructure (magenta) galaxies in A2626 and galaxies in the Swarm (orange) respectively.} 
  \label{fig:offset_histogram} 
 \end{figure}

\subsection{Quantifying asymmetries}

\textbf{Defining \hi asymmetry}
\newline \noindent
To investigate the environmental impact on the \hi discs of galaxies, it is important to quantify the severity of the asymmetry/ disturbance in the outer \hi distribution of individual galaxies. The optical/ infrared morphologies of galaxies are generally quantified with the concentration, asymmetry, and smoothness (CAS) parameters \citep{conselice2003} and the Gini and M20 parameters \citep{lotz2004}. \cite{holwerda2011, holwerda2013} used these parameters to quantify the \hi morphologies of galaxies from multiple \hi surveys (WHISP, LITTLE-THINGS, VLA-ANGST). In addition, \cite{holwerda2011, holwerda2013} and \cite{giese2016} used the definition of asymmetry from \cite{conselice2003}. \cite{giese2016} found that the asymmetry parameter is much more robust than any of the other parameters for \hi imaging data.
\begin{equation}
    A =  \frac{\sum_{i,j}^{N}|{I(i,j) - I_{180}(i,j)}|}{\sum_{i,j}^{N}|{I(i,j)}|}
\end{equation}
where \textit{I(i, j)} denotes the value of the pixel at the \textit{i, j} position of the original image of the galaxy and  $I_{180}(i, j)$ is the value of the pixel in the same position of the image rotated by 180$^{\circ}$ around the centre of the galaxy. So, asymmetry is measured by summing the pixel-by-pixel difference of the original and rotated image and normalising that by the total intensity in the image. Hence, the asymmetry index can have a value between 0 and 1. Asymmetries in the fainter outer regions of the \hi discs give a negligible contribution to the global asymmetry index compared to the brighter ($\sim$ 2 orders of magnitude or more) inner regions. Environmental processes, however, influence the extended outer parts of the \hi disc in a galaxy much more easily than the inner parts. In order to give proper weight to the asymmetries in the outer \hi disc, \cite{Lelli2014} introduced a modified asymmetry index ($A_{mod}$):

\begin{equation}
    A_{mod} = \frac{1}{N}\sum_{i,j}^{N} \frac{|{I(i,j) - I_{180}(i,j)}|}{|{I(i,j) + I_{180}(i,j)}|}
\end{equation}
where $N$ is the total number of pixels in the image. Thus, this definition of $A_{mod}$ normalizes the intensity differences at the position $(i, j)$ with the `local' intensity of the pixels, contrary to the total intensity of all the pixels in the \hi map. For example, for a highly lopsided galaxy with \hi emission exclusively on one side, $A_{mod}$ will obtain the maximum value of 1. However, we recognize that the value of \amod depends on some selection criteria.

(i) The value of \amod depends on the \hi column density above which it is measured. Depending on the observational setup and the sensitivity of the telescope, different \hi observations have different column density sensitivities. Moreover, even with the same observational settings, galaxies may have diverse 3$\sigma$ column density thresholds, depending on the local noise in the vicinity of the galaxy and the location of the galaxy with respect to the pointing centre which affects the local primary beam attenuation. 

(ii) The choice of the galaxy centre around which the \hi disc is rotated also effects the measurement of \amod. For disturbed and unsettled \hi distributions, the \hi centre often does not coincide with the optical centre. Moreover, the determination of the optical or the \hi centre also depends on the adapted method of calculation. 

(iii) The reliability of the \amod measurement also depends on how well resolved the galaxy is, both in terms of the beam size in kpc and the number of beams across the \hi map. If the galaxy is only marginally resolved and detected at low signal-to-noise, the asymmetry in the outer \hi disc might be dominated by noise, rather than the asymmetry induced by environmental processes.

\cite{bilimogga2022} have investigated the dependence of \amod on signal-to-noise, column density threshold and angular resolution of the \hi observations. Using mock galaxies from the EAGLE simulations \citep{schaye2015, crain2015}, they suggest an optimal combination of the observational constraints that are required for a robust measurement of the \amod value of the outer \hi disc of a galaxy: a column density threshold of 5 $\times$ 10$^{19}$cm$^{-2}$ or lower at a minimal signal-to-noise of 3 and a galaxy resolved with at least 11 beams. These are not `hard' limits as they depend on what one considers to be acceptable deviations of the measured \amod from the intrinsic \amod of a galaxy. Our observations do not reach this column density sensitivity and most galaxies are not resolved by 11 beams. Consequently, we measure \hi asymmetries of the inner \hi discs. 

\medskip

\noindent
\textbf{Measuring asymmetry in galaxies in and around A2626}

\label{subsec:measuring_hi_asym}
\noindent
To compute \amod, we need to adopt the position of the centre of the galaxy as well as a column density threshold above which we would consider the intensity of the \hi emission. All 219 \hi detected galaxies in our survey have optical counterparts within the footprints of the Sloan Digital Sky Survey (SDSS, \citealt{York2000, Aguado2018}) and the DECam Legacy Survey (DECaLS, \citealt{dey2019}). For our analysis, we adopted the optical centres from the SDSS for all 219 galaxies. We did not consider the \hi kinematic centres since most of the galaxies in our sample have disturbed/ unsettled \hi discs. Moreover, optical centres are generally better tracers of the dynamical centre of a galaxy, compared to the \hi centres. Using the optical centre sometimes results in a high \amod value for the galaxies with a strong offset between the \hi distribution and the stellar body, that may indicate a recent interaction or accretion event. 

It is crucial to calculate \amod above a specific column density for all the galaxies in the sample to make a fair comparison between different \amod values. But that specific column density will correspond to different signal-to-noise levels for different galaxies, depending on the local noise properties and the location of the galaxy within the primary beam. We have a range of 3$\sigma$ column density sensitivities (9-65) $\times$ 10$^{19}$ cm$^{-2}$ in our \mk observations \citep{HD2021}. To measure \amod reliably, we need adequate signal-to-noise images and we need as low an $N_{HI}$ limit as possible to measure the outer parts of galaxies well. The bottom panel of Fig. \ref{fig:S2N_histogram} shows the distribution of the 3$\sigma$ \hi column density levels for the reliable \hi detections with a minimum signal-to-noise $\geq$ 5 (177 galaxies among the 219 galaxies). 

The vertical dotted line in the bottom panel of Fig. \ref{fig:S2N_histogram} represents the \hi column density level of 25 $\times$ 10$^{19}$ cm$^{-2}$. Galaxies to the left of this dotted line have 3$\sigma$ \hi column densities below 25 $\times$ 10$^{19}$ cm$^{-2}$. Hence, their \amod values will be reliable when including pixels above 25 $\times$ 10$^{19}$ cm$^{-2}$ in the calculation of \amod. The galaxies to the right of the dotted line have 3$\sigma$ \hi column density levels above 25 $\times$ 10$^{19}$ cm$^{-2}$, thus, will have noisy pixels (below their own 3$\sigma$ contour) included in the \amod calculation when including pixels with values as low as 25 $\times$  10$^{19}$ cm$^{-2}$. By adopting a threshold of 25 $\times$ 10$^{19}$, we retain 71\% of the galaxies for which the \amod calculation only includes pixels with signal-to-noise >3 in the \hi map. Hence we consider the corresponding \amod values as reliable, provided the galaxy is sufficiently resolved (see below).

Notably, in a statistical sense, for a sample of galaxies with random inclinations, 25 $\times$ 10$^{19}$ cm$^{-2}$ corresponds to 1 \msun/pc$^{2}$ in a face-on orientation, which is the typical column density at which the diameters of \hi discs are measured. 1 \msun/pc$^{2}$ surface density corresponds to \hi column density of 12.5 $\times$ 10$^{19}$ cm$^{-2}$. Since the galaxies in our sample cover the full range of inclinations, by considering the \textit{observed} column density along the line-of-sight, we do not always consider the \hi column density perpendicular to the plane of those galaxies. Since most of the galaxies in our survey are barely resolved, we can not make meaningful inclination corrections for the \hi column densities of individual galaxies. So, we have adopted a statistical approach. The median inclination of a randomly oriented sample of galaxies is i = 60$^{\circ}$. This means that a face-on column density of 12.5 $\times$ 10$^{19}$ cm$^{-2}$ increases to a line-of-sight column density of 12.5/cos(60$^{\circ}$) = 25 $\times$ 10$^{19}$ cm$^{-2}$. Thus, 25 $\times$ 10$^{19}$ cm$^{-2}$ is also the practical level adopted for measuring \hi diameters.

Another important aspect for a reliable \amod measurement is angular resolution. For poorly resolved galaxies, \amod is not very meaningful. So, in addition to signal-to-noise, we need to restrict the sample to sufficiently resolved galaxies. We have considered the following criteria to identify the galaxies for which the \amod value is reliable: galaxies with a 3$\sigma$ column density level of $\leq$ 25 $\times$ 10$^{19}$ cm$^{-2}$, with a minimum peak signal-to-noise of $\geq$5, and resolved by more than 3 beams. Consequently, we are left with 33 galaxies for which the calculation of \amod is meaningful. 

\begin{figure}[t!]
 {
    \includegraphics[width=0.45\textwidth]{images/Amod_offset_Feb20_modified.pdf} 
  } 
  \caption{Distribution of \amod and \hi offset for the galaxies in and around A2626. The galaxies above the horizontal dashed line (\amod>0.4) are considered as \hi disturbed / unsettled galaxies. Galaxies on the right side of the vertical dashed line are considered as high \hi offset galaxies. \hi maps of a few example galaxies (see Section \ref{subsec:measuring_hi_asym}) are included as insets in the top panel. In the bottom right corner we reported the Pearson's cofficient and p-value for the relation between \amod and \hi offset. \amod and \hi offset seem to have a very strong correlation.} 
  \label{fig:Amod_offset_class1} 
 \end{figure}

\medskip

\noindent
\textbf{The relation between visual classifications, offsets, and \amod}

\noindent
We have explored the relation between our different methods of classifying \hi morphologies : visual classes, \hi offset, and \amod. Fig. \ref{fig:Amod_offset_class1} shows \amod as a function of offset between the \hi and the optical centres. We observe a strong correlation between \amod and offset for these 33 galaxies. The Spearman correlation coefficient is 0.81 which is shown in the bottom-left of Fig. \ref{fig:Amod_offset_class1}. This correlation is expected since a high offset between the \hi and optical centres would result in a high \amod value, though a galaxy with a high \amod value does not necessarily have a strongly offset \hi disc. For practical purposes, we are classifying a galaxy as \hi asymmetric when its \amod value is $\geq$0.4. Such an \hi asymmetric galaxy will be located above the horizontal dashed line in Fig. \ref{fig:Amod_offset_class1}. Similarly, we have defined galaxies to the right of the vertical dashed line in Fig. \ref{fig:Amod_offset_class1} as the high \hi offset galaxies.

These thresholds as indicated by the horizontal and vertical dashed lines have separated the galaxies in different classes. In particular, we found three different areas in the plot representing three different types of galaxies. 
\begin{enumerate}
    \item In the first quadrant (q1), we find high \amod, low \hi offset galaxies. For these galaxies, a high value of \amod is driven by the asymmetry in the outer \hi disc. So, these are \hi asymmetric galaxies. For example, in Fig. \ref{fig:Amod_offset_class1}, we observe an asymmetry in the outer \hi contour of galaxy 184 which results in a high \amod value. 
    
    
    \item In the second quadrant (q2), we find high \amod, high \hi offset galaxies. For these galaxies, a high \amod is driven by the high \hi offset (sometimes due to the combination of an \hi offset and an outer disc \hi asymmetry). For example, galaxy 105 in  Fig. \ref{fig:Amod_offset_class1} has a strongly offset \hi disc with respect to the optical centre.
     
    \item In the fourth quadrant (q4), we find low \amod, low \hi offset galaxies. These galaxies have an \hi disc that is neither asymmetric, nor offset. Hence, these are \hi normal galaxies (e.g. galaxy 76 in Fig. \ref{fig:Amod_offset_class1}), although their \hi discs can be small with respect to the stellar disc.

\end{enumerate}

We note that there are no galaxies in the third quadrant (q3), i.e. galaxies with low \amod and high \hi offset. This is expected since a high \hi offset would automatically cause a high \amod value. 

\begin{figure*}[t!]
 {
    \includegraphics[width=1\textwidth]{images/Amod_offset_vclass_Feb18.pdf} 
  } 
  \caption{Distribution of \amod and \hi offset as a function of visual classes for the galaxies in and around A2626. Left panel: settled galaxies (in circles), in top panel \hi maps of settled galaxies with high \amod. Right panel: disturbed (diamonds) and unsettled (stars) galaxies, in top panel \hi maps of disturbed and unsettled galaxies with low \amod.   } 
  \label{fig:Amod_offset_visual_class1} 
 \end{figure*}
Next, we have explored the same offset-\amod relation with our visual classifications of \hi asymmetries as an additional parameter. The left and right panel of Fig. \ref{fig:Amod_offset_visual_class1} show the galaxies with settled \hi discs as circles and the galaxies with disturbed / unsettled \hi discs as diamonds and stars respectively, based on our visual classifications as mentioned in Sec. \ref{subsec:visual_classes}. So, if there is a complete overlap of our visual classification with the \hi asymmetries based on the \amod values, all the diamond and star symbols should be above the \amod = 0.4 horizontal dashed line and all the circle symbols should be below the \amod = 0.4 horizontal dashed line. However, in the bottom left corner in the right panel of Fig. \ref{fig:Amod_offset_visual_class1}, two galaxies (31 and 45) that are visually classified as galaxies with a disturbed or an unsettled \hi disc, have nevertheless a low \amod value. These galaxies have a low \amod value since \amod is calculated within the green contour that corresponds to 25 $\times$ 10$^{19}$ cm$^{-2}$, which is above the 3$\sigma$ level (because it is inside the outermost black contour). These galaxies, however, have disturbed / unsettled \hi discs if we consider the outer \hi contour that corresponds to the 3$\sigma$ column density. This illustrates that the choice of 25 $\times$ 10$^{19}$ cm$^{-2}$ means that for some galaxies, the visual classification based on the outer \hi disc will not correspond to the \amod value calculated with pixels above the 25 $\times$ 10$^{19}$ cm$^{-2}$ contour. Furthermore, in the left panel of Fig. \ref{fig:Amod_offset_visual_class1}, we find three galaxies with \amod $>$0.4 yet they are visually classified to have settled \hi discs. For galaxies 147 and 120, the \hi disc seems somewhat offset from the optical centre. Galaxy 191 has an \hi cloud detached from the \hi disc, yet it was included in the calculation of \amod. However, the visual impression of the \hi discs of these galaxies are settled. This illustrates that the \amod values and the visual classifications both are valuable tracers of \hi morphologies and largely follow each other.

\begin{figure}[t!]
\centering
 {
\includegraphics[width=0.5\textwidth]{images/Amod_offset_diff_envs_Sep.pdf} 
  } 
  \caption{Distribution of \amod and \hi offset, colour coded with three different environments for the galaxies residing in and around A2626. The green, magenta, and orange circles represent non-substructure, substructure galaxies in A2626 and galaxies in the Swarm respectively. Substructure galaxies seem to have more asymmetric \hi discs with high \amod compared to non-substructure galaxies in A2626.} 
  \label{fig:Amod_offset_diff_env} 
 \end{figure}
 
 As a next step, we are considering the environment as a parameter in the same \amod vs \hi offset plot. We have used three different colours to represent the three different environments as described in Sec. \ref{subsec:diff_envs}. In Fig. \ref{fig:Amod_offset_diff_env}, green, magenta and the orange circles represent the non-substructure galaxies in A2626, the substructure galaxies in A2626, and the Swarm galaxies respectively. Although the statistics is limited, interestingly, all the galaxies in the substructure for which we have reliable \amod values are \hi asymmetric galaxies. At the same time, only a fraction of the non-substructure galaxies in A2626 have an \hi asymmetric disc. The occurrence of more \hi asymmetric galaxies in substructures of A2626 (all magenta symbols are above the dashed line) is possibly due to more effective tidal interactions in the substructure environment than the non-substructure galaxies in A2626 experience. However, galaxies in the Swarm and non-substructure galaxies in A2626 are found to have both disturbed and undisturbed \hi discs. 
    %--------------------------------------------------------------------
%-----------------------------------------------------------------

\section{\hi\ deficiency in A2626 and the Swarm galaxies}
 \label{sec:HI_def}
 
In addition to exploring different \hi morphological classifications and their significance, we have explored the \hi deficiency as another indicator of environmentally induced gas depletion and removal processes. We examined a galaxy's \hi deficiency as a function of its projected distance from the centre of A2626, its position in the phase-space diagram of A2626, and its \hi morphology.

\hi deficiency is defined as a logarithmic quantity (Haynes \& Giovanelli 1983), a difference between the log of expected \hi mass and the log of observed \hi mass of a galaxy.

\begin{equation}
    HIdef = log[M_{HI}^{exp}] - log[M_{HI}^{obs}]
\end{equation}
 
 \begin{figure*}[ht!]
 {
    \includegraphics[width=\textwidth]{images/HI_def_plot1_Apr16.pdf} 
  } 
  \caption{\footnotesize{Top left: \hi deficiency vs projected distance normalised by R$_{200}$ for the non-substructure (green circles) and substructure (magenta circles) in A2626. Horizontal dashed lines presents the range of \hi deficiencies of the field galaxies. There is a strong correlation between \hi deficiency and projected distance - \hi deficient galaxies seem to reside close to the cluster core. Top right: Histogram of the distribution of \hi deficiencies of galaxies in the Swarm. Bottom left: The distribution of the non-substructure (green) and substructure (magenta) galaxies in A2626 in the projected phase-space. The black dashed lines indicate the escape velocity. There is no clear difference between the distribution of galaxies in the two different environments apart from the previously observed trend of the proximity of non-substructure galaxies to the cluster core. Bottom right: Six \hi maps overlaid on DECaLS colour images for the most \hi deficient (top three) and the most \hi rich (bottom three) galaxies. The \hi deficient galaxies seem to be bright and yellowish with offset or truncated \hi discs. The \hi rich galaxies are fainter and bluish, with extended \hi discs.} }
  \label{fig:HIdef_plot1} 
 \end{figure*}

 \noindent Thus \hi deficiency is positive for \hi deficient galaxies and negative for galaxies with excess \hi gas. $M_{HI}^{exp}$ is generally calculated from \hi-optical scaling relations (e.g. \citealt{Haynes1984, chamaraux1986, Batuski1985, solanes1996, denes2014}). We used the scaling relation from \citet{denes2014} (see their Table 3), a multiwavelength scaling relation between the \hi content and the optical diameter of the galaxies. The scaling relation of Denes 2014 is based on the r-band diameters reported by HyperLEDA.
 
 \begin{equation}
   log[M_{HI}^{exp}] = \alpha_{\lambda} + \beta_{\lambda} log[D_{\lambda}]  
 \end{equation}
 
\noindent where $M_{HI}^{exp}$ is the expected \hi mass, $D_{\lambda}$ is the optical diameter (in kpc) in a particular band, $\alpha_{\lambda}$ and $\beta_{\lambda}$ are the parameters for this band. To calculate the expected \hi mass, we have measured the diameters ($D_{r}$=2R$_{25}$) from the DECaLS r-band images where R$_{25}$ is the radius of the 25th mag/arcsec$^2$ isophote.

\subsection{\hi deficiency in substructures}

The top left panel of Fig. \ref{fig:HIdef_plot1} shows the \hi deficiency of galaxies vs. their projected distance from the center of A2626, normalised by R$_{200}$ of A2626. R$_{200}$ is the projected radius of a sphere with mean density 200 times the critical density of the universe. Light and magenta symbols represent the non-substructure and the sub-structure galaxies in A2626 respectively. The area between the three horizontal dashed lines represents the \hi deficiency range for field galaxies. The vertical dashed line indicates R$_{200}$ of A2626. The \hi deficiencies of the Swarm galaxies are presented in the orange histogram in the top right panel of Fig. \ref{fig:HIdef_plot1}. First of all, we observe a correlation between \hi deficiency and the projected distance from the center of A2626 (Spearman's coefficient= -0.377, p-value= 0.0012). This clearly illustrates that galaxies become \hi deficient towards the cluster core. Moreover, within 1.5 R$_{200}$, the most \hi deficient galaxies seem to be the non-substructure galaxies. On average, non-substructure galaxies are more \hi deficient ($< HIdef>_{non-substructure}$ = 0.27, $<HIdef>_{ss}$=0.15) than sub-structure galaxies in A2626. We note, however, that the substructure galaxies are more prevalent at larger cluster centric radii. The six panels in the bottom-right of Fig. \ref{fig:HIdef_plot1} illustrate \hi maps overlaid on DECaLS colour images for the most \hi deficient (top three) and the most \hi rich (bottom three) galaxies. Focusing on them, we observe that the \hi deficient galaxies are bright and have a yellowish colour, and have small or truncated \hi discs. On the contrary, the \hi rich galaxies are fainter with a bluish colour and have extended \hi discs.

\begin{figure}[t!]
\begin{center}

 {
    \includegraphics[width=0.45\textwidth]{images/A2626_HIdefs_Feb18.pdf} 
  } 
  \caption{\footnotesize{HI deficiency vs projected distance normalised by R$_{200}$ (similar to Fig. \ref{fig:HIdef_plot1}), with symbols and colour codes to include additional information regarding \hi morphologies of galaxies in A2626. Top panel: Different symbols (shown in the top right corner) to represent non-substructure (green) and substructure (magenta) galaxies of different visual classes. Middle panel: The colour scale (colourbar shown in the right) represents \hi offsets of non-substructure (filled circles) and substructure (open circles) galaxies respectively. Bottom panel: The colour scale (colourbar shown in the right) represents \amod value of non-substructure (filled circles) and substructure (open circles) galaxies respectively.}} 
  \label{fig:HIdef_plot2} 
  \end{center}
 \end{figure}

\subsection{\hi deficiency in phase-space}

In the bottom left panel of Fig. \ref{fig:HIdef_plot1}, we show the phase-space diagram for non-substructure and sub-structure galaxies in A2626. The black dashed lines indicate the escape velocity for A2626 galaxies, calculated using the formalism from \citet{Jaffe2015}. We assumed a concentration index C = 6 and the mass enclosed by R$_{200}$ is calculated using:

\begin{equation}
    M_{200} = \frac{4}{3} \pi R^{3}_{200}200\rho_{c}
\end{equation}

\noindent where M$_{200}$ is the mass enclosed by R$_{200}$ and $\rho_{c}$ is the critical density. The vertical grey dashed line in the phase-space diagram represents R$_{200}$ of A2626. In the phase-space diagram, there is no clear additional trend in the distribution of non-substructure/ sub-structure galaxies.

\subsection{\hi deficiency and \hi morphology}

As a next step, in Fig. \ref{fig:HIdef_plot2}, we have explored the same \hi deficiency vs. projected distance plot with colour codes and symbols to include additional information regarding the \hi morphologies of those 76 galaxies. 

\textbf{\hi deficiency and visual classifications}

In the top  panel of Fig.  \ref{fig:HIdef_plot2}, we present the \hi detected galaxies of the three visual classes as defined in Sec. \ref{subsec:visual_classes} with circle, triangle, and star symbols respectively. The light and magenta symbols in each class stand for the non-substructure and sub-structure galaxies in the A2626 cluster. Interestingly, the VClass 3 or unsettled galaxies (star symbols) seem to have \hi deficiencies similar to the field galaxies (i.e. within the three horizontal dashed lines). Moreover, substructures contain more disturbed / unsettled galaxies compared to the non-substructure galaxies in A2626 as we have seen in Sec. \ref{subsec:visual_classes}. However, there is no obvious correlation of the visual classes of the \hi detected galaxies with their \hi deficiencies or with their projected distance from the cluster centre. Thus, the visually classified \hi morphologies are not directly related to the \hi deficiencies of the galaxies or their location in the A2626 cluster. 

\textbf{\hi deficiency and \hi offsets}

In the middle panel of Fig. \ref{fig:HIdef_plot2}, we have again plotted \hi deficiency vs. projected distance from the center of A2626 for the \hi detected galaxies, with a colour coding according to the offset (in units of kpc) of the \hi centers of those galaxies with respect to the optical centers. The filled and open circles are non-substructure and substructure galaxies respectively. There is no correlation between \hi deficiency and \hi offset for the galaxies in A2626. This implies that the \hi deficient galaxies are not necessarily \hi offset and vise versa.

\textbf{\hi deficiency and \amod}

In the bottom panel of Fig. \ref{fig:HIdef_plot2}, we examine the \hi deficiency vs projected distance, with a colour coding indicating the \amod values of the \hi discs of galaxies in A2626. Since \amod is meaningful for only 11 well-resolved, high signal-to-noise galaxies in A2626, we have plotted only those galaxies. The filled and open symbols are non-substructure and sub-structure galaxies in A2626. Similar to what we see in two other panels in Fig. \ref{fig:HIdef_plot2}, there is no clear trend between \hi deficiency, projected distance from the cluster centre, and \amod in the galaxies in A2626. Interestingly, the most \hi deficient galaxy in this plot is also the most \hi asymmetric galaxy with a high \amod value (\amod = 0.6).

\bigskip Considering all panels of Fig. \ref{fig:HIdef_plot2}, apart from the clear trend between \hi deficiency and the projected distance from the cluster centre, we conclude that there are no further obvious correlations between \hi deficiency, \hi morphology, and location of the galaxy in the cluster, except that VClass 3 galaxies are not \hi deficient.
%--------------------------------------------------------------------


\section{Star formation rates and depletion times in A2626 and the Swarm galaxies}
\label{sec:sfr_dep_time}

\begin{figure*}[t!]
 {
    \includegraphics[width=\textwidth]{images/SFR_Mstar_tdep_HIdef_Feb24.pdf} 
  } 
  \caption{\footnotesize{Top left panel: SFMS for the non-substructure (filled circles) and substructure (open circles) galaxies in A2626. The SFMS relation and quenching threshold are taken from \cite{cluver2020} which is calibrated using the WISE data and the same methodologies that is used for stellar mass and SFR calculations for the galaxies in our survey. The colour scale (shown in the right) represents \hi deficiency for the galaxies. The down-arrows are 2$\sigma$ upper limits on the SFR from WISE observations. Both the non-substructure and substructure galaxies seem to have slight lower SFR than the SFR expected for normal galaxies. In the top panels we show \hi maps overlaid on DECaLS colour images for some outlier galaxies. Top right panel: Similar to the plot on the top left, except it presents galaxies in the Swarm. Bottom left panel: \hi depletion time vs stellar mass of the non-substructure (filled circles) and substructure (open circles) galaxies in A2626. The colour scale presents the \hi deficiency of the galaxies. Bottom right panel: Similar to the bottom left plot, except it presents the galaxies in the Swarm.} }
  \label{fig:SFR_Mstar_tdep_HIdef} 
 \end{figure*}

\subsection{Galaxies on the SFMS }
 
We have combined the information on star formation rates (SFR), stellar masses with the \hi properties of the galaxies in A2626 and the Swarm to investigate the extent of the effect of different environments on the star formation properties of the \hi detected galaxies. The star formation activity and the stellar mass of a galaxy are related in a systematic way, described by a well-established scaling relation, the `star formation main sequence’ (SFMS, e.g. \citealt{brinchmann2004, Noeske2007, elbaz2007, speagle2014, tomczak2016}). We investigated the location of the galaxies in different environments (non-substructure and sub-structure galaxies in A2626, and the Swarm galaxies) with respect to the SFMS. We have adopted the SFMS and the star formation quenching threshold relation from \citet{cluver2020}, which are calibrated using WISE data. For consistency, we have also adopted stellar masses and SFRs based on WISE data derived with the same methodology as developed by \citet{cluver2014} (see Sect. 1.3).

We have also investigated the relation between \hi depletion time and stellar mass for galaxies in different environments as a function of \hi deficiency. Depletion time is the time required for a galaxy to deplete its \hi gas due to star formation and is defined as $t_{dep}$ = M$_{HI}$/ SFR. The depletion timescale of normal star-forming galaxies spans a range of $\sim $2-10 Gyr \citep{kennicutt1989, Kennicutt1998, bigiel2008}.

\subsection{HI deficiency versus SF deficiency }

 In Fig. \ref{fig:SFR_Mstar_tdep_HIdef}, we have plotted SFMS in the top two panels and the depletion time vs stellar mass in the bottom two panels for the galaxies in A2626 and the Swarm. To represent the two different environments in A2626, we have used filled and open circles respectively for the non-substructure and sub-structure galaxies in A2626 in the top and bottom left panel of Fig. \ref{fig:SFR_Mstar_tdep_HIdef}. In all the panels, we have colour coded the \hi deficiencies. The orange and black dotted lines are the SFMS and the quenching threshold from \citet{cluver2020} respectively. The downward pointing arrows are the 2$\sigma$ upper limits on the SFR. Interestingly, galaxies in all three environments -- the non-substructure and sub-structure galaxies in A2626, and the Swarm galaxies are mostly below the SFMS from \citet{cluver2020}, thus show an overall SF deficiency.
 
 \begin{figure}
 {\includegraphics[width=0.45\textwidth]{images/A2626_Swarm_offset_from_SFMS_MHI_Mstar_Feb24.pdf} 
  } 
  \caption{Offset from SFMS i.e. SF deficiency in A2626 and the Swarm galaxies. Galaxies above the horizontal dashed line lie below the SFMS and can therefore be considered to have a positive SF deficiency. The colour coding is similar to the colour coding in Fig. \ref{fig:Amod_offset_diff_env}.}
  \label{fig:offset_SFMS_A2626_Swarm} 
 \end{figure}
 
  \begin{figure*}[ht!]
 {\includegraphics[width=\textwidth]{images/SFR_Mstar_tdep_Amod_Feb24.pdf} 
  } 
  \caption{Similar to Fig. \ref{fig:SFR_Mstar_tdep_HIdef}, with additional information regarding \hi morphologies of galaxies in A2626 and the Swarm.} 
  \label{fig:SFR_Mstar_tdep_Amod} 
 \end{figure*}
 
 To examine if there is any difference in the distribution of galaxies in the three environments with respect to the SFMS, in Fig. \ref{fig:offset_SFMS_A2626_Swarm}, we have plotted the offset of the galaxies from the SFMS, i.e. the SF deficiency, in those three environments (green: non-substructure galaxies in A2626, magenta: sub-structure galaxies in A2626, orange: the Swarm galaxies). A positive SF deficiency (galaxies above the horizontal dotted line) indicates that these galaxies are below the SFMS and a negative SF deficiency (galaxies below the horizontal dotted line) indicates that these galaxies are above the SFMS. Along the horizontal axis we plotted M$_{HI}$/M$_{\star}$ to investigate if the SF deficiency relates to the relative \hi content of those galaxies. We do not observe any differences for the galaxies in different environments. This means that the star formation activity of galaxies with a similar stellar mass has no dependence on environment. 
 
 \subsection{\hi depletion time versus \hi morphology}
 In the bottom panels of Fig. \ref{fig:SFR_Mstar_tdep_HIdef}, we do not observe any clear trend between the depletion time and stellar mass for the galaxies, regardless of their environment. The downward pointing arrows are 2$\sigma$ upper limits on the depletion times. The figures show that there are no galaxies with large stellar masses and large depletion times. Massive galaxies tend to have a high SFR as expected from the SFMS and thus exhaust their \hi fuel relatively fast. There is no strong correlation of depletion time with \hi deficiency. However, we note that the most \hi deficient galaxies in A2626 have a short depletion time while the three galaxies in the Swarm with the longest depletion time (purple symbols) have a negative \hi deficiency. 
 
 \subsection{\hi morphology versus SF deficiency}
 
 In Fig. \ref{fig:SFR_Mstar_tdep_Amod}, we present similar SFMS plots in six different panels for three different environments, using the same symbols as Fig. \ref{fig:SFR_Mstar_tdep_HIdef}. The only differences are that in the two top panels of Fig. \ref{fig:SFR_Mstar_tdep_Amod} we have used different symbols for the three visual classifications, in the middle panels we have colour coded with \hi offset, and in the bottom panels we have colour coded with \amod. Here again, we do not observe any meaningful correlation between the SFR and \hi offset or \amod, as a function of environment or stellar mass.

%--------------------------------------------------------------------
\section{Interesting galaxies}
\label{sec:interesting_gals}
Before we discuss our results in detail, we first present the detailed \hi maps of several interesting galaxies to illustrate the diversity of the \hi morphologies. These galaxies are resolved in our \mk \hi observations and we compare them with the existing DECaLS optical colour images, star formation rate from WISE observations, and their location in phase-space to infer the ongoing physical mechanisms acting on those galaxies. 

\begin{figure*}[t!]
 {
    \includegraphics[width=\textwidth]{images/interesting_galaxies_Feb18.pdf} 
  } 
  \caption{Interesting galaxies with diverse \hi morphologies in and around A2626.} 
  \label{fig:interesting_gals} 
 \end{figure*}

 \subsection{Potential ram-pressure stripped galaxies}
 
 In Fig. \ref{fig:interesting_gals} (a), we present several galaxies that are potentially experiencing ram-pressure stripping or other effects like thermal evaporation or starvation. Galaxies 110 and 128 are non-substructure galaxies in A2626, located close to the cluster core (within R$_{200}$) and fully exposed to the ICM. Moreover, they are highly \hi deficient, have small \hi discs with yellowish optical discs hinting at a quenched star formation, confirmed by their low star formation rates inferred from the WISE fluxes. Hence, they probably have experienced  ram-pressure stripping or thermal evaporation for a long time which has resulted in the removal of a significant amount of \hi gas. Galaxy 178 is located in a substructure in A2626. Although it is not very close to the cluster core ($\sim$ 1.5 R$_{200}$), the \hi morphology hints at ongoing ram-pressure stripping. Its \hi contours are compressed on the northern side of the \hi disc and the outer \hi contour seems to suggest that the \hi gas is pushed to the south. Interestingly, galaxy 178 is moving at a velocity similar to the mean velocity of the cluster and has a low star formation rate. Therefore one may conclude that galaxy 178 is a backsplash galaxy, but the \hi content of the galaxy suggests it is still on its first infall trajectory.
 
 \subsection{Galaxies with unsettled \hi discs}
 
 In Fig. \ref{fig:interesting_gals} (b), we present some well-resolved galaxies with unsettled \hi discs (Vclass 3) and high \amod values (\amod$>$0.5). Galaxy 87 is a face-on system in the Swarm with a regular optical morphology, but an extended unsettled \hi disc. We note a nearby companion $\sim$ 70" to the south-west. Galaxy 99 is in a substructure in A2626, not only does it have an unsettled \hi disc, but optically it is also very disturbed. It is likely that it has experienced a tidal interaction with another galaxy in the substructure. Galaxy 130 is a foreground galaxy at a redshift of z$_{HI}$=0.0382, and therefore is excluded from our overall analysis. Its optical morphology is that of an early-type, barred ring galaxy. \hi emission is not only found in the bright central regions of the galaxy, but clumps of \hi gas are also found in the faint outer stellar ring surrounding the bar, similar to NGC 4736 \citep{bosma1977}.
 
\subsection{Interacting galaxies}

In Fig. \ref{fig:interesting_gals} (c), we present examples of interacting galaxies in our sample that are confirmed to have similar velocities based on the MMT as well as \hi redshifts \citep{HealySS2021}. Among the interacting galaxies shown in Fig. \ref{fig:interesting_gals} (c), galaxies 15, 122, 135, 138, 163 and 164 are in A2626 itself. Galaxies 52 and 57 are in the Swarm while galaxies 93, 96, 108, 109, and 174 are in the background cluster A2637. The \hi emission of galaxies 15, 122, and 174 is blended with that of their companion galaxies. We conclude that the gravitational interactions are prevalent in both the group and the cluster environment in and around A2626.

%--------------------------------------------------------------------
\section{Discussion}
\label{sec:discusiions}


\section{Future Directions}
Based on the results from our analysis, we suggest several future directions for both Vietnamese monolingual language models and Vietnamese MRC benchmarks.
\subsection{Language Models}
Our analysis shows that monolingual models, especially PhoBERT, acquire comparable abilities in recognizing the differences in lexical information between unanswerable questions and the given context. However, monolingual models show poor performances when encountering unanswerable questions that require the ability to comprehend a bigger ``picture''. For example, while monolingual models perform very well on unanswerable questions that use explicit antonyms, they often have difficulties in recognizing unanswerable questions when these questions are created using implicit antonyms. We explain this phenomenon by the findings of \citet{zhang-etal-2021-need} as pre-training language models on larger text copora results in significant improvement on downstream tasks that require high-level semantic and factual knowledge such as Machine Reading Comprehension. Therefore, when encountering unanswerable questions that require ability to grasp big ``picture,'' models pre-trained with smaller text corpora will show lower performances. Hence, the small size of pre-training corpora of PhoBERT and WikiBERT may be the main reason for their poor performances in MRC.

Scaling the pre-training data size of PhoBERT will further develop this model and empower it to achieve state-of-the-art performances on different benchmarks of Machine Reading Comprehension. Besides, we believe that introducing a new unsupervised task for the pre-training phase that focuses on improving the high-level semantic and factual knowledge of pre-trained models also plays an integral role in developing language models in the future.
\subsection{Benchmarks}
\textbf{Unanswerable Questions. } Although UIT-ViQuAD 2.0 successfully further introduced new types of artificially unanswerable questions, our work in Section 5 shows that current unanswerable questions in the development test of UIT-ViQuAD 2.0 are still not challenging enough. In order to increase the challenging levels of unanswerable questions, we believe that more high-quality works on adversarial human annotation for unanswerable questions are needed. These works can follow the guidelines of adversarial human annotation for answerable questions \cite{bartolo-etal-2020-beat}. Results of these works can reveal different techniques to annotate hard unanswerable questions and therefore be valuable for improving the guidelines for unanswerable questions annotation for Machine Reading Comprehension.\\
\textbf{Quality of Benchmark. } On the other hand, as we have shown in section 5, although PhoBERT and XLM-RoBERTa achieve high performances on the UIT-VinewsQA development set, our unanswerable questions reveal that these two models do not truly understand the context to give the correct answer for questions in the original development set. We hypothesize that questions in UIT-VinewsQA enable machine reading comprehension systems with shortcut learning knowledge \cite{lai-etal-2021-machine} to achieve high performance due to biases in annotating process. Therefore, we believe that studies on how Vietnamese machine reading comprehension systems are currently evaluated are important for tracking the progress of Vietnamese language systems.


%--------------------------------------------------------------------

\section{Summary}
\label{sec:summary}

In this work, we have presented \hi deficiencies, \hi morphologies, and star formation deficiencies of the \hi detected galaxies in and around the A2626 cluster observed with \mk. There are three main environments in our surveyed volume: non-substructure or isolated galaxies in A2626 (cluster environment), substructure galaxies in A2626 (groups influenced by the cluster environment) and the Swarm galaxies (group environment). We are interested in understanding if there is any correlation between the \hi deficiency, \hi morphology, and the star formation deficiency of the \hi detected galaxies and the environment in which the galaxies are residing. 

To characterise the \hi morphology of the outer \hi distribution of individual galaxies, we have used three approaches. First, we have used three visual classifications based on the outermost reliable \hi contour (settled, one-sided asymmetric, unsettled galaxies). Second, we have measured the offset of the \hi distribution with respect to the optical centre of the galaxy. Third, we have calculated the modified asymmetry parameter \amod (as introduced by \citealt{Lelli2014}), which depends on the choice of the galaxy centre, the \hi column density above which it is measured, and how well the galaxy is spatially resolved. We chose an \hi column density threshold of 25 $\times 10^{19}$ cm$^{-2}$ to calculate the \amod value for galaxies with at least three beams in size and with a peak signal-to-noise$\geq$ 5. 

First, we have explored the relationship between our three different types of classifications of the \hi morphologies. We have plotted \amod vs. \hi offset from the optical centre and found a very strong correlation between them as expected. We have found that high \amod (\amod $\geq$ 0.4), low \hi offset galaxies are strongly \hi asymmetric galaxies. In case of galaxies with both high \amod and high \hi offsets, the \amod value is driven by the large offset of the \hi disc compared to the stellar disc. Galaxies with a low \amod value and a small \hi offset are mostly \hi symmetric galaxies with settled \hi discs. There is a strong correspondence between \amod and our visual classifications; indicating that all three characterisations of \hi morphology are useful tracers of \hi asymmetries. 

In Fig. \ref{fig:HIdef_plot2} we have investigated whether an environmental dependence exists on the \hi asymmetry and \hi offset. We find that substructures contain a higher fraction of asymmetric galaxies compared to the population of non-substructure galaxies in A2626. This suggests that tidal interactions are more efficient in substructures than outside substructures. This might be expected because the relative velocity differences of the neighbouring galaxies in the substructures tend to be lower compared to the cluster galaxies outside the substructures.

There is a strong correlation between \hi deficiency and projected distance from the cluster centre. Outside R$_{200}$, non-substructure and substructure galaxies in A2626, and galaxies in the Swarm have a similar range of \hi deficiencies. This signifies that the mechanisms that cause \hi deficiencies are possibly pre-processing the galaxies prior to their infall into the cluster. Focusing on the most \hi deficient and most \hi rich galaxies in A2626, we observe that the \hi deficient galaxies have yellowish optical colours and have small and truncated \hi discs while the \hi rich galaxies have bluish optical colours and extended \hi discs. This clearly shows that the \hi deficient galaxies are in a later stage of their evolution while \hi rich galaxies are actively star forming with sufficient amounts of \hi gas.

Galaxies with an offset or asymmetric \hi disc are found in all three environments that we considered, implying that pre-processing of the \hi discs plays an important role. However, though the substructure galaxies in A2626 are more \hi asymmetric than the non-substructure galaxies, the asymmetric galaxies are not necessarily \hi deficient. This suggests that, although the substructure environment instigates more interactions among galaxies, those interactions do not deplete the \hi gas out of those galaxies completely. 

The galaxy population in all three different environments (substructure, non-substructure, and the Swarm) are mostly below the SFMS from \cite{cluver2020}. This means that the gas removal mechanisms in the cluster environment as well as pre-processing in the substructures and the Swarm are causing slightly lower SFRs than the usual SFR for normal galaxies. There is no clear relation between \hi deficiency, \hi asymmetry or \hi offset with star formation deficiency for the galaxies in all the three environments. This signifies that the environmental mechanisms causing \hi deficiencies or asymmetries are probably not immediately impacting the star formation activity and vice versa.

We conclude that \hi deficiency and \hi morphology are good tracers of different environmental mechanisms acting on galaxies and their \hi characteristics and star formation deficiencies are affected by pre-processing before the galaxies enter the cluster environment. 

Encouraged by the high number of direct \hi detections, thanks to the unprecedented sensitivity of the \mk telescope, the \hi study of galaxies in and around A2626 will be expanded by four additional \mk pointings in 32k-mode covering the ENE-SSW sector of A2626. The primary motivation of those observations is to enable a detailed investigation of so called pre-processing of \hi discs of galaxies before they have enter the cluster environment. The higher spectral resolution of these observations will supplement the analysis of \hi morphologies presented in this paper and, additionally, will allow for detailed investigations of the kinematical asymmetries of the detected galaxies. 


%--------------------------------------------------------------------

\begin{acknowledgements}
This paper makes use of the MeerKAT data (Project ID: SCI-20190418-JH-01). The MeerKAT telescope is operated by the South African Radio Astronomy Observatory, which is a facility of the National Research Foundation, an agency of the Department of Science and Innovation. MV acknowledges the Netherlands Foundation for Scientific Research support through VICI grant 016.130.338 and the Leids Kerkhoven-Bosscha Fonds (LKBF) for travel support. JMvdH acknowledges support from the European Research Council under the European Union’s Seventh Framework Programme (FP/2007-2013)/ERC Grant Agreement nr. 291531 (HiStoryNU)     
\end{acknowledgements}

% WARNING
%-------------------------------------------------------------------
% Please note that we have included the references to the file aa.dem in
% order to compile it, but we ask you to:
%
% - use BibTeX with the regular commands:
%   \bibliographystyle{aa} % style aa.bst
%   \bibliography{Yourfile} % your references Yourfile.bib
%
% - join the .bib files when you upload your source files
%-------------------------------------------------------------------

\bibliographystyle{aa}
\bibliography{references}

\appendix
\onecolumn
\section{Catalogue of \hi detections.}

\centering
\setlength{\tabcolsep}{6pt} % Default value: 6pt
\renewcommand{\arraystretch}{1}

\centering
\setlength{\tabcolsep}{3pt} % Default value: 6pt
\renewcommand{\arraystretch}{1}

\begin{longtable}[t]{ccccccccccccccc}
\caption[]{A table that lists all of the relevant \hi and stellar physical and morphological characteristics of the galaxies used in this study. The column entries of table \ref{tab:catalogue} are as follows: Column (1): The assigned \hi identifier from  \citealt{HD2021}. Column (2): The SDSS identifier for the optical counterpart of the \hi detected galaxies, based on their Right Ascension and Declination (J2000.0). Column (3): Substructure identifier. SS=`0',`1': non-substructure and substructure galaxies in A2626, SS=`2': galaxies in the Swarm. Column (4): Redshift measured as the midpoint of the 20\% line width of the \hi global profile. Column (5,6): Log of \hi mass and uncertainty as mentioned in \citealt{HD2021}. Column (7): \hi deficiency using the scaling relation from \citet{denes2014}. Column (8): Calculated modified asymmetry using the definition of \citet{Lelli2014}. Column (9): Measured offset of the \hi centre from the optical centre (in kpc). Column (10): Visual classifications of galaxies. `1': settled sources, `2': disturbed sources, `3': unsettled sources. Column (11,12): Log of stellar mass and uncertainty derived from M/L from \citet{cluver2014}. Column (13,14): 12 $\mu$m star formation rates star formation rate and uncertainty from WISE observations (provided by T. Jarrett, private communication). Column (15): g-r magnitude from DECaLS \citep{dey2019} survey (corrected for extinction.)} \\\hline
\multicolumn{1}{c}{HI ID} &
\multicolumn{1}{c}{Name} &
\multicolumn{1}{c}{SS} &
\multicolumn{1}{c}{z} &
\multicolumn{1}{c}{Log(M$\rm_{HI}$}) &
\multicolumn{1}{c}{$\pm$} &
\multicolumn{1}{c}{HIdef} &
\multicolumn{1}{c}{Amod} &
\multicolumn{1}{c}{\hi offset} &
\multicolumn{1}{c}{VClass} &
\multicolumn{1}{c}{Log(M$_{*}$}) &
\multicolumn{1}{c}{$\pm$} &
\multicolumn{1}{c}{SFR} &
\multicolumn{1}{c}{$\pm$} &
\multicolumn{1}{c}{g-r} \\
%\noalign{\vspace{0.8mm}}
& & 
\multicolumn{2}{c}{} &
\multicolumn{2}{c}{} &
\multicolumn{2}{c}{} &
\multicolumn{1}{c}{(kpc)} &
\multicolumn{1}{c}{} &
\multicolumn{2}{c}{} &
\multicolumn{2}{c}{(M$_{\odot}$/yr)} &
\multicolumn{1}{c}{mag}\\
\multicolumn{1}{c}{(1)}&
\multicolumn{1}{c}{(2)}&
\multicolumn{1}{c}{(3)}&
\multicolumn{1}{c}{(4)}&
\multicolumn{1}{c}{(5)}&
\multicolumn{1}{c}{(6)}&
\multicolumn{1}{c}{(7)}&
\multicolumn{1}{c}{(8)}&
\multicolumn{1}{c}{(9)}&
\multicolumn{1}{c}{(10)}&
\multicolumn{1}{c}{(11)}&
\multicolumn{1}{c}{(12)}&
\multicolumn{1}{c}{(13)}&
\multicolumn{1}{c}{(14)}&
\multicolumn{1}{c}{(15)}\\

%\noalign{\vspace{0.5mm}}
\hline
\hline
\endfirsthead

\noalign{\vspace{0.7mm}}
\multicolumn{15}{l}{Continued from previous page.} \\
\hline
%\noalign{\vspace{0.2mm}}
\multicolumn{1}{c}{HI ID} &
\multicolumn{1}{c}{Name} &
\multicolumn{1}{c}{SS} &
\multicolumn{1}{c}{z} &
\multicolumn{1}{c}{Log(M$\rm_{HI}$)} &
\multicolumn{1}{c}{$\pm$} &
\multicolumn{1}{c}{HIdef} &
\multicolumn{1}{c}{Amod} &
\multicolumn{1}{c}{\hi offset} &
\multicolumn{1}{c}{VClass} &
\multicolumn{1}{c}{Log(M$_{*}$)} &
\multicolumn{1}{c}{$\pm$} &
\multicolumn{1}{c}{SFR} &
\multicolumn{1}{c}{$\pm$} &
\multicolumn{1}{c}{g-r} \\
%\noalign{\vspace{0.8mm}}
& & 
\multicolumn{2}{c}{} &
\multicolumn{2}{c}{} &
\multicolumn{2}{c}{} &
\multicolumn{1}{c}{(kpc)} &
\multicolumn{1}{c}{} &
\multicolumn{2}{c}{} &
\multicolumn{2}{c}{(M$_{\odot}$/yr)} &
\multicolumn{1}{c}{mag}\\
\multicolumn{1}{c}{(1)}&
\multicolumn{1}{c}{(2)}&
\multicolumn{1}{c}{(3)}&
\multicolumn{1}{c}{(4)}&
\multicolumn{1}{c}{(5)}&
\multicolumn{1}{c}{(6)}&
\multicolumn{1}{c}{(7)}&
\multicolumn{1}{c}{(8)}&
\multicolumn{1}{c}{(9)}&
\multicolumn{1}{c}{(10)}&
\multicolumn{1}{c}{(11)}&
\multicolumn{1}{c}{(12)}&
\multicolumn{1}{c}{(13)}&
\multicolumn{1}{c}{(14)}&
\multicolumn{1}{c}{(15)}\\
%\noalign{\vspace{0.5mm}}
\hline
\hline
\endhead
\endfoot
\endlastfoot
10 & J233409.36+211641.9 & 1 & 0.0524 & 9.24 & 0.09 & 0.12 & 0.60 & 2.79 & 2 & 8.83 & 1.28 & 0.23 & -.-- & 0.42\\
  12 & J233413.05+212327.5 & 1 & 0.0515 & 9.60 & 0.05 & 0.07 & 0.48 & 2.75 & 2 & 9.52 & 0.13 & 0.55 & 0.22 & 0.55\\
  13 & J233425.70+213122.9 & 1 & 0.0553 & 9.74 & 0.04 & -0.02 & 0.43 & 3.35 & 2 & 9.56 & 1.27 & 0.37 & -.-- & 0.46\\
  15 & J233438.15+211851.7 & 1 & 0.0538 & 9.93 & 0.03 & -0.07 & 0.45 & 2.06 & 2 & 10.36 & 0.15 & 3.33 & 1.15 & 0.73\\
  16 & J233438.80+211721.0 & 1 & 0.0529 & 9.07 & 0.08 & 0.64 & 0.87 & 5.85 & 2 & 10.22 & 0.11 & 2.0 & 0.70 & 0.73\\
  17 & J233440.31+203710.8 & 0 & 0.0573 & 9.95 & 0.04 & 0.01 & 0.64 & 6.18 & 2 & 10.46 & 0.10 & 3.08 & 1.07 & 0.70\\
  19 & J233453.14+213344.9 & 1 & 0.0554 & 9.38 & 0.06 & 0.39 & 0.49 & 0.85 & 2 & 9.78 & 0.13 & 2.03 & 0.71 & 0.77\\
  21 & J233455.82+212245.8 & 1 & 0.0525 & 9.50 & 0.04 & 0.21 & 0.54 & 3.38 & 2 & 9.63 & 1.27 & 0.36 & -.-- & 0.52\\
  23 & J233500.37+205908.6 & 0 & 0.0588 & 9.31 & 0.06 & 0.34 & 0.39 & 1.80 & 1 & 9.18 & 1.27 & 0.27 & -.-- & 0.40\\
  24 & J233510.54+212147.8 & 1 & 0.0527 & 9.35 & 0.05 & 0.42 & 0.37 & 0.93 & 2 & 9.99 & 0.26 & 0.74 & 0.28 & 0.63\\
  25 & J233512.34+214628.1 & 2 & 0.0649 & 9.58 & 0.08 & -.-- & 0.29 & 0.79 & 1 & 10.23 & 0.2 & 2.14 & 0.76 & 0.60\\
  29 & J233525.93+204419.5 & 2 & 0.0617 & 9.26 & 0.09 & 0.07 & 0.34 & 2.16 & 1 & 8.81 & 1.27 & 0.31 & -.-- & 0.37\\
  31 & J233526.81+211638.3 & 2 & 0.0650 & 9.87 & 0.03 & -0.23 & 0.35 & 0.31 & 2 & 10.12 & 0.14 & 1.34 & 0.48 & 0.83\\
  32 & J233527.64+204059.2 & 2 & 0.0662 & 9.54 & 0.09 & -0.36 & 0.87 & 37.91 & 1 & 8.85 & 1.29 & 0.33 & -.-- & 0.36\\
  33 & J233528.63+203803.2 & 2 & 0.0625 & 9.54 & 0.09 & -0.04 & 0.21 & 1.10 & 1 & 9.22 & 1.27 & 0.42 & -.-- & 0.29\\
  36 & J233532.73+211011.3 & 2 & 0.0659 & 10.21 & 0.01 & -0.51 & 0.35 & 1.31 & 1 & 9.91 & 0.31 & 1.01 & 0.39 & 0.46\\
  37 & J233533.49+210252.1 & 0 & 0.0567 & 9.63 & 0.04 & 0.13 & 0.46 & 2.08 & 2 & 10.40 & 0.24 & 2.19 & 0.77 & 0.71\\
  39 & J233535.77+204159.6 & 1 & 0.0613 & 9.99 & 0.04 & -0.10 & 0.50 & 2.56 & 3 & 10.72 & 0.10 & 3.88 & 1.35 & 0.78\\
  40 & J233535.93+203807.1 & 2 & 0.0626 & 10.02 & 0.03 & -0.51 & 0.29 & 2.50 & 1 & 10.12 & 0.26 & 0.37 & 0.17 & 0.48\\
  42 & J233536.98+210440.3 & 0 & 0.0545 & 9.23 & 0.04 & -0.04 & 0.33 & 1.93 & 2 & 9.44 & 1.27 & 0.35 & -.-- & 0.54\\
  43 & J233537.03+204639.6 & 0 & 0.0486 & 9.21 & 0.06 & 0.46 & 0.47 & 1.50 & 2 & 9.33 & 1.27 & 0.29 & -.-- & 0.53\\
  44 & J233537.44+211025.1 & 2 & 0.0654 & 9.12 & 0.08 & 0.17 & 0.86 & 1.75 & 2 & 8.99 & 1.29 & 0.39 & -.-- & 0.29\\
  45 & J233539.99+210844.6 & 0 & 0.0613 & 9.81 & 0.02 & -0.17 & 0.25 & 1.05 & 3 & 9.54 & 1.27 & 0.26 & -.-- & 0.49\\
  46 & J233540.43+205357.4 & 0 & 0.0578 & 8.90 & 0.11 & -.-- & 0.97 & 1.94 & 1 & -.-- & -.-- & -.-- & -.-- & 0.29\\
  52 & J233544.62+210228.7 & 2 & 0.0663 & 9.55 & 0.04 & 0.18 & 0.13 & 0.54 & 1 & 9.69 & 0.25 & 0.73 & 0.29 & 0.62\\
  53 & J233545.43+205045.5 & 2 & 0.0639 & 9.10 & 0.09 & 0.13 & 0.64 & 2.15 & 2 & -.-- & -.-- & -.-- & -.-- & 0.47\\
  54 & J233546.25+210031.0 & 1 & 0.0540 & 9.34 & 0.04 & -0.04 & 0.30 & 0.98 & 1 & 8.63 & 1.28 & 0.27 & -.-- & 0.32\\
  55 & J233546.32+203026.4 & 2 & 0.0655 & 10.16 & 0.04 & -0.48 & 0.57 & 0.75 & 3 & 10.34 & 0.14 & 2.07 & 0.73 & 0.59\\
  57 & J233547.60+210207.0 & 2 & 0.0636 & 9.28 & 0.05 & 0.39 & 0.72 & 3.94 & 3 & 10.38 & 0.13 & 2.38 & 0.84 & 0.72\\
  58 & J233547.85+204226.6 & 2 & 0.0619 & 10.02 & 0.03 & 0.24 & 0.29 & 3.37 & 1 & 11.23 & 0.10 & 5.04 & 1.76 & 0.87\\
  59 & J233549.79+211356.4 & 2 & 0.0661 & 8.96 & 0.08 & 0.43 & 1.00 & 3.91 & 2 & 9.47 & 1.14 & 0.4 & 0.2 & 0.44\\
  60 & J233550.08+202318.3 & 2 & 0.0639 & 10.09 & 0.05 & -.-- & 0.74 & 7.28 & 3 & 10.54 & 0.12 & 2.75 & 0.97 & 0.82\\
  62 & J233556.63+211333.0 & 2 & 0.0660 & 9.53 & 0.04 & 0.01 & 0.13 & 0.24 & 1 & 9.68 & 1.27 & 0.29 & -.-- & 0.44\\
  63 & J233557.05+211105.5 & 2 & 0.0649 & 9.94 & 0.03 & -0.06 & 0.33 & 1.10 & 1 & 10.42 & 0.13 & 3.80 & 1.33 & 1.17\\
  64 & J233557.38+205307.1 & 0 & 0.0597 & 9.42 & 0.05 & 0.31 & 0.55 & 2.61 & 2 & 9.89 & 0.35 & 0.41 & 0.19 & 0.60\\
  65 & J233557.68+211703.5 & 0 & 0.0501 & 8.99 & 0.06 & 0.71 & 0.22 & 0.92 & 1 & 9.44 & 0.38 & 0.26 & 0.14 & 0.53\\
  70 & J233602.82+210821.1 & 2 & 0.0644 & 9.35 & 0.05 & -.-- & 0.30 & 0.98 & 1 & 9.81 & 0.27 & 0.72 & 0.31 & 0.26\\
  71 & J233604.52+210613.3 & 2 & 0.0661 & 9.65 & 0.03 & 0.25 & 0.90 & 4.83 & 3 & 10.77 & 0.10 & 0.24 & 0.13 & 0.92\\
  73 & J233606.20+203113.8 & 2 & 0.0650 & 9.61 & 0.07 & -0.19 & 0.54 & 3.40 & 2 & 8.98 & 1.27 & 0.34 & -.-- & 0.38\\
  75 & J233607.00+210500.1 & 0 & 0.0576 & 9.44 & 0.04 & 0.00 & 0.27 & 0.56 & 1 & 8.89 & 1.27 & 0.30 & -.-- & 0.30\\
  77 & J233609.55+205428.4 & 0 & 0.0492 & 9.05 & 0.06 & -.-- & 0.83 & 2.78 & 1 & -.-- & -.-- & -.-- & -.-- & 0.32\\
  79 & J233611.37+205702.0 & 0 & 0.0536 & 9.22 & 0.06 & 0.10 & 0.56 & 4.16 & 2 & -.-- & -.-- & -.-- & -.-- & 0.28\\
  81 & J233614.66+214434.3 & 1 & 0.0560 & 9.71 & 0.04 & -0.11 & 0.45 & 2.75 & 2 & 9.70 & 1.27 & 0.50 & -.-- & 0.37\\
  84 & J233615.59+205047.1 & 0 & 0.0492 & 9.27 & 0.05 & 0.19 & 0.52 & 1.97 & 2 & 9.32 & 0.28 & 0.52 & 0.20 & 0.38\\
  86 & J233616.89+215412.1 & 1 & 0.0565 & 9.35 & 0.10 & -0.11 & 0.37 & 0.93 & 1 & 9.12 & 1.27 & 0.24 & -.-- & 0.23\\
  87 & J233618.06+203054.6 & 2 & 0.0649 & 10.15 & 0.03 & -0.25 & 0.49 & 2.45 & 3 & 10.37 & 0.12 & 1.90 & 0.68 & 0.94\\
  88 & J233618.57+210402.6 & 0 & 0.0533 & 9.77 & 0.02 & 0.20 & 0.56 & 3.40 & 2 & 9.89 & 0.14 & 2.31 & 0.81 & 0.47\\
  90 & J233619.79+204950.3 & 0 & 0.0514 & 8.84 & 0.12 & 0.60 & 1.00 & 4.29 & 2 & 9.53 & 1.27 & 0.36 & -.-- & 0.48\\
  95 & J233624.23+204355.5 & 2 & 0.0634 & 9.22 & 0.07 & 0.21 & 0.49 & 1.43 & 1 & 9.34 & 1.27 & 0.31 & -.-- & 0.38\\
  99 & J233626.71+215204.9 & 1 & 0.0570 & 10.26 & 0.03 & -0.38 & 0.69 & 6.67 & 3 & 9.90 & 0.14 & 1.10 & 0.40 & 0.06\\
  102 & J233631.03+205750.0 & 0 & 0.0585 & 9.21 & 0.07 & 0.49 & 0.70 & 6.88 & 2 & 10.03 & 0.14 & 2.02 & 0.71 & 0.99\\
  103 & J233631.53+205325.7 & 2 & 0.0624 & 9.10 & 0.07 & 0.59 & 1.00 & 5.30 & 2 & 10.05 & 0.14 & 1.66 & 0.59 & 0.61\\
  105 & J233632.47+204110.8 & 2 & 0.0646 & 9.98 & 0.04 & -.-- & 0.77 & 6.74 & 2 & 10.24 & 0.10 & 5.75 & 1.99 & 0.81\\
  108 & J233632.86+203424.6 & 2 & 0.0661 & 9.13 & 0.09 & 0.57 & 0.88 & 4.97 & 2 & 10.18 & 0.11 & 5.29 & 1.84 & 0.81\\
  109 & J233634.21+203441.1 & 2 & 0.0673 & 9.55 & 0.05 & 0.22 & 0.29 & 0.44 & 2 & 10.22 & 0.10 & 6.65 & 2.31 & 0.59\\
  110 & J233639.74+210606.8 & 0 & 0.0589 & 9.46 & 0.04 & 0.65 & 0.57 & 3.86 & 2 & 10.98 & 0.11 & 6.35 & 2.20 & 0.74\\
  111 & J233642.05+212810.6 & 0 & 0.0531 & 9.33 & 0.05 & 0.26 & 0.68 & 3.28 & 2 & 9.88 & 0.11 & 0.69 & 0.26 & 0.45\\
  113 & J233642.70+212212.0 & 0 & 0.0550 & 9.58 & 0.03 & 0.16 & 0.39 & 2.17 & 1 & 10.08 & 0.13 & 2.52 & 0.88 & 0.63\\
  115 & J233645.05+205108.7 & 0 & 0.0530 & 9.28 & 0.05 & -0.21 & 0.90 & 5.14 & 3 & 9.06 & 0.52 & 0.37 & 0.17 & 0.56\\
  117 & J233647.54+212506.6 & 0 & 0.0563 & 9.02 & 0.10 & 0.29 & 1.00 & 5.84 & 2 & -.-- & -.-- & -.-- & -.-- & 0.40\\
  118 & J233647.73+204136.1 & 0 & 0.0539 & 9.91 & 0.02 & 0.11 & 0.28 & 1.65 & 1 & 10.18 & 0.11 & 1.27 & 0.45 & 0.61\\
  119 & J233648.56+205031.3 & 0 & 0.0521 & 8.77 & 0.12 & 0.56 & 0.41 & 1.47 & 1 & 8.98 & 1.28 & 0.34 & -.-- & 0.25\\
  122 & J233650.53+203534.6 & 0 & 0.0571 & 9.88 & 0.04 & -0.10 & 0.68 & 6.92 & 3 & 9.69 & 1.27 & 0.23 & -.-- & 0.51\\
  124 & J233652.31+204719.4 & 0 & 0.0592 & 9.39 & 0.06 & 0.29 & 0.30 & 0.86 & 1 & 9.70 & 1.27 & 0.47 & -.-- & 0.56\\
  126 & J233652.59+203726.2 & 0 & 0.0506 & 9.43 & 0.06 & 0.31 & 0.54 & 3.68 & 2 & 9.11 & 1.27 & 0.39 & -.-- & 0.44\\
  127 & J233654.75+203635.9 & 0 & 0.0517 & 9.23 & 0.08 & 0.24 & 0.49 & 3.24 & 2 & 9.05 & 1.27 & 0.4 & -.-- & 0.28\\
  128 & J233655.10+212708.9 & 0 & 0.0528 & 9.46 & 0.05 & 0.67 & 0.52 & 2.75 & 2 & 10.99 & 0.21 & 4.43 & 1.54 & 0.79\\
  129 & J233655.43+204826.1 & 0 & 0.0579 & 9.94 & 0.02 & 0.09 & 0.30 & 1.05 & 1 & 10.02 & 0.16 & 1.35 & 0.48 & 0.52\\
  131 & J233657.81+204254.2 & 2 & 0.0633 & 9.57 & 0.05 & -0.05 & 0.51 & 2.57 & 3 & 9.56 & 1.27 & 0.48 & -.-- & 0.40\\
  132 & J233658.57+203442.4 & 0 & 0.0517 & 9.72 & 0.04 & 0.08 & 0.83 & 9.59 & 3 & 9.49 & 0.16 & 0.38 & 0.17 & 0.31\\
  134 & J233702.06+204756.0 & 0 & 0.0516 & 9.55 & 0.03 & -.-- & 0.70 & 6.93 & 2 & -.-- & -.-- & -.-- & -.-- & 0.00\\
  135 & J233703.24+205326.8 & 1 & 0.0576 & 9.04 & 0.06 & 0.39 & 1.00 & 1.73 & 2 & 9.43 & 0.84 & 0.43 & 0.21 & 0.26\\
  138 & J233703.72+205304.1 & 1 & 0.0580 & 9.48 & 0.04 & 0.01 & 0.54 & 3.44 & 1 & 9.74 & 0.46 & 0.34 & 0.17 & 0.47\\
  139 & J233705.27+210739.1 & 2 & 0.0638 & 9.38 & 0.05 & 0.05 & 0.36 & 1.30 & 2 & 9.27 & 1.27 & 0.32 & -.-- & 0.50\\
  142 & J233711.63+204243.3 & 2 & 0.0641 & 9.68 & 0.04 & -0.69 & 0.77 & 6.52 & 2 & 9.35 & 1.27 & 0.29 & -.-- & 0.72\\
  143 & J233712.13+204554.1 & 1 & 0.0581 & 9.15 & 0.11 & 0.02 & 0.88 & 2.86 & 2 & -.-- & -.-- & -.-- & -.-- & 0.34\\
  144 & J233712.73+210406.5 & 0 & 0.0584 & 9.35 & 0.05 & 0.31 & 0.48 & 1.90 & 1 & 9.68 & 0.18 & 0.84 & 0.32 & 0.54\\
  148 & J233720.02+204934.0 & 1 & 0.0583 & 9.17 & 0.07 & 0.12 & 0.88 & 4.31 & 2 & 9.45 & 0.22 & 0.86 & 0.32 & 0.70\\
  149 & J233720.15+213258.3 & 0 & 0.0577 & 9.31 & 0.08 & 0.28 & 0.68 & 3.57 & 2 & 9.67 & 1.27 & 0.25 & -.-- & 0.39\\
  150 & J233721.12+205717.6 & 0 & 0.0563 & 9.84 & 0.02 & 0.20 & 0.27 & 1.86 & 1 & 10.81 & 0.10 & 1.17 & 0.42 & 1.04\\
  153 & J233723.08+205719.7 & 0 & 0.0581 & 9.18 & 0.06 & 0.28 & 0.48 & 1.66 & 1 & -.-- & -.-- & -.-- & -.-- & 0.33\\
  157 & J233726.14+210016.1 & 0 & 0.0525 & 8.78 & 0.10 & -.-- & 0.98 & 2.70 & 1 & -.-- & -.-- & -.-- & -.-- & 0.42\\
  158 & J233731.24+205616.2 & 1 & 0.0575 & 9.26 & 0.09 & 0.00 & 0.29 & 1.61 & 2 & 8.63 & 1.28 & 0.31 & -.-- & 0.49\\
  159 & J233733.15+203213.4 & 1 & 0.0593 & 9.52 & 0.08 & -0.21 & 0.56 & 3.52 & 2 & 9.12 & 0.48 & 0.37 & 0.19 & 0.47\\
  160 & J233735.81+210101.8 & 0 & 0.0527 & 8.92 & 0.08 & 0.49 & 0.33 & 1.01 & 1 & 9.66 & 1.19 & 0.59 & 0.24 & 0.48\\
  161 & J233739.94+203118.8 & 1 & 0.0573 & 9.73 & 0.07 & 0.10 & 0.72 & 1.26 & 2 & 10.16 & 0.15 & 2.08 & 0.73 & 0.60\\
  162 & J233741.04+213051.6 & 0 & 0.0538 & 9.50 & 0.05 & 0.08 & 0.71 & 3.38 & 2 & 9.46 & 0.12 & 0.35 & 0.17 & 0.54\\
  166 & J233745.24+210742.1 & 2 & 0.0640 & 9.50 & 0.04 & 0.10 & 0.53 & 2.72 & 1 & 9.43 & 1.27 & 0.38 & -.-- & 0.39\\
  169 & J233746.51+212126.6 & 0 & 0.0533 & 9.07 & 0.06 & 0.33 & 0.62 & 3.37 & 2 & -.-- & -.-- & -.-- & -.-- & 0.38\\
  170 & J233748.75+204013.3 & 1 & 0.0561 & 9.64 & 0.04 & 0.15 & 0.38 & 2.82 & 1 & 10.12 & 0.12 & 2.27 & 0.79 & 0.53\\
  172 & J233750.60+212454.3 & 0 & 0.0553 & 9.32 & 0.06 & 0.09 & 0.35 & 0.53 & 2 & -.-- & -.-- & -.-- & -.-- & 0.29\\
  173 & J233751.32+211127.1 & 0 & 0.0553 & 9.58 & 0.04 & -0.04 & 0.36 & 1.49 & 2 & 9.11 & 1.27 & 0.24 & -.-- & 0.39\\
  176 & J233756.65+205436.8 & 1 & 0.0559 & 9.21 & 0.06 & 0.49 & 0.54 & 2.55 & 2 & 9.99 & 0.24 & 0.70 & 0.28 & 0.62\\
  177 & J233756.99+212234.8 & 0 & 0.0556 & 9.16 & 0.06 & -.-- & 0.97 & 6.03 & 2 & -.-- & -.-- & -.-- & -.-- & 0.26\\
  178 & J233758.90+204000.6 & 1 & 0.0551 & 9.60 & 0.05 & 0.13 & 0.72 & 4.04 & 2 & 9.60 & 0.14 & 0.4 & 0.20 & 0.60\\
  179 & J233800.62+205722.0 & 0 & 0.0559 & 9.52 & 0.05 & 0.55 & 0.58 & 2.00 & 2 & 10.72 & 0.10 & 5.07 & 1.77 & 0.79\\
  182 & J233808.40+205755.0 & 0 & 0.0551 & 9.39 & 0.06 & 0.19 & 1.00 & 12.62 & 3 & 9.98 & 0.32 & 0.38 & 0.19 & 0.60\\
  191 & J233825.56+204911.2 & 1 & 0.0564 & 9.86 & 0.03 & 0.05 & 0.56 & 3.09 & 1 & 9.81 & 0.16 & 0.54 & 0.21 & 0.52\\
  192 & J233826.00+203517.6 & 1 & 0.0577 & 9.83 & 0.06 & -0.10 & 0.39 & 1.42 & 2 & 9.71 & 1.27 & 0.51 & -.-- & 0.57\\
  194 & J233828.47+212848.0 & 0 & 0.0533 & 9.41 & 0.10 & 0.20 & 0.84 & 6.03 & 2 & 9.89 & 0.10 & 0.91 & 0.34 & 0.61\\
  196 & J233832.29+203644.9 & 1 & 0.0565 & 9.51 & 0.09 & 0.16 & 0.68 & 5.22 & 2 & 9.52 & 1.27 & 0.49 & -.-- & 0.34\\
  198 & J233841.11+212336.3 & 2 & 0.0633 & 9.57 & 0.07 & 0.16 & 1.00 & 17.94 & 3 & -.-- & -.-- & -.-- & -.-- & 0.86\\
  200 & J233846.64+204855.3 & 1 & 0.0555 & 9.31 & 0.07 & 0.09 & 0.67 & 6.26 & 2 & 8.96 & 1.27 & 0.23 & -.-- & 0.41\\
  201 & J233847.26+204720.0 & 1 & 0.0568 & 9.50 & 0.06 & 0.26 & 0.65 & 2.95 & 2 & 10.07 & 0.24 & 0.57 & 0.23 & 0.43\\
  204 & J233848.60+212311.0 & 2 & 0.0637 & 9.41 & 0.09 & -0.01 & 0.62 & 2.84 & 2 & 9.90 & 0.33 & 2.22 & 0.79 & 0.36\\
  205 & J233848.61+205550.2 & 1 & 0.0562 & 9.73 & 0.05 & -0.08 & 0.60 & 3.16 & 2 & 9.99 & 0.14 & 2.43 & 0.85 & 0.50\\
  210 & J233859.60+211723.6 & 1 & 0.0558 & 9.52 & 0.06 & 0.19 & 0.61 & 4.74 & 2 & 9.07 & 1.27 & 0.42 & -.-- & 0.49\\
  218 & J233948.55+205857.4 & 0 & 0.0565 & 9.88 & 0.09 & -0.10 & 2.00 & 1.42 & 1 & 9.90 & 0.16 & 1.26 & 0.45 & 0.44\\
\hline
\hline
\label{tab:catalogue}
\end{longtable}

\newpage

\end{document}