In this work, we have presented \hi deficiencies, \hi morphologies, and star formation deficiencies of the \hi detected galaxies in and around the A2626 cluster observed with \mk. There are three main environments in our surveyed volume: non-substructure or isolated galaxies in A2626 (cluster environment), substructure galaxies in A2626 (groups influenced by the cluster environment) and the Swarm galaxies (group environment). We are interested in understanding if there is any correlation between the \hi deficiency, \hi morphology, and the star formation deficiency of the \hi detected galaxies and the environment in which the galaxies are residing. 

To characterise the \hi morphology of the outer \hi distribution of individual galaxies, we have used three approaches. First, we have used three visual classifications based on the outermost reliable \hi contour (settled, one-sided asymmetric, unsettled galaxies). Second, we have measured the offset of the \hi distribution with respect to the optical centre of the galaxy. Third, we have calculated the modified asymmetry parameter \amod (as introduced by \citealt{Lelli2014}), which depends on the choice of the galaxy centre, the \hi column density above which it is measured, and how well the galaxy is spatially resolved. We chose an \hi column density threshold of 25 $\times 10^{19}$ cm$^{-2}$ to calculate the \amod value for galaxies with at least three beams in size and with a peak signal-to-noise$\geq$ 5. 

First, we have explored the relationship between our three different types of classifications of the \hi morphologies. We have plotted \amod vs. \hi offset from the optical centre and found a very strong correlation between them as expected. We have found that high \amod (\amod $\geq$ 0.4), low \hi offset galaxies are strongly \hi asymmetric galaxies. In case of galaxies with both high \amod and high \hi offsets, the \amod value is driven by the large offset of the \hi disc compared to the stellar disc. Galaxies with a low \amod value and a small \hi offset are mostly \hi symmetric galaxies with settled \hi discs. There is a strong correspondence between \amod and our visual classifications; indicating that all three characterisations of \hi morphology are useful tracers of \hi asymmetries. 

In Fig. \ref{fig:HIdef_plot2} we have investigated whether an environmental dependence exists on the \hi asymmetry and \hi offset. We find that substructures contain a higher fraction of asymmetric galaxies compared to the population of non-substructure galaxies in A2626. This suggests that tidal interactions are more efficient in substructures than outside substructures. This might be expected because the relative velocity differences of the neighbouring galaxies in the substructures tend to be lower compared to the cluster galaxies outside the substructures.

There is a strong correlation between \hi deficiency and projected distance from the cluster centre. Outside R$_{200}$, non-substructure and substructure galaxies in A2626, and galaxies in the Swarm have a similar range of \hi deficiencies. This signifies that the mechanisms that cause \hi deficiencies are possibly pre-processing the galaxies prior to their infall into the cluster. Focusing on the most \hi deficient and most \hi rich galaxies in A2626, we observe that the \hi deficient galaxies have yellowish optical colours and have small and truncated \hi discs while the \hi rich galaxies have bluish optical colours and extended \hi discs. This clearly shows that the \hi deficient galaxies are in a later stage of their evolution while \hi rich galaxies are actively star forming with sufficient amounts of \hi gas.

Galaxies with an offset or asymmetric \hi disc are found in all three environments that we considered, implying that pre-processing of the \hi discs plays an important role. However, though the substructure galaxies in A2626 are more \hi asymmetric than the non-substructure galaxies, the asymmetric galaxies are not necessarily \hi deficient. This suggests that, although the substructure environment instigates more interactions among galaxies, those interactions do not deplete the \hi gas out of those galaxies completely. 

The galaxy population in all three different environments (substructure, non-substructure, and the Swarm) are mostly below the SFMS from \cite{cluver2020}. This means that the gas removal mechanisms in the cluster environment as well as pre-processing in the substructures and the Swarm are causing slightly lower SFRs than the usual SFR for normal galaxies. There is no clear relation between \hi deficiency, \hi asymmetry or \hi offset with star formation deficiency for the galaxies in all the three environments. This signifies that the environmental mechanisms causing \hi deficiencies or asymmetries are probably not immediately impacting the star formation activity and vice versa.

We conclude that \hi deficiency and \hi morphology are good tracers of different environmental mechanisms acting on galaxies and their \hi characteristics and star formation deficiencies are affected by pre-processing before the galaxies enter the cluster environment. 

Encouraged by the high number of direct \hi detections, thanks to the unprecedented sensitivity of the \mk telescope, the \hi study of galaxies in and around A2626 will be expanded by four additional \mk pointings in 32k-mode covering the ENE-SSW sector of A2626. The primary motivation of those observations is to enable a detailed investigation of so called pre-processing of \hi discs of galaxies before they have enter the cluster environment. The higher spectral resolution of these observations will supplement the analysis of \hi morphologies presented in this paper and, additionally, will allow for detailed investigations of the kinematical asymmetries of the detected galaxies. 
