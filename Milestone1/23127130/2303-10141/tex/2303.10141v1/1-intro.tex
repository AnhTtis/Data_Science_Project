\subsection{General science background}
 %--------------------------------------------------------------------

%Ettori 2005 The baryon fraction in hydrodynamical simulations of galaxy clusters 
The well-known morphology-density relation \citep{Dressler1980} provides direct evidence of the notion that the cosmic environment of galaxies influences their constitutional properties and star formation activity -- both during their formation (nature) and subsequent evolution and interaction with their surroundings (nurture). In the hierarchical structure formation scenario \citep{Toomre1972}, galaxy clusters grow by accreting galaxies and galaxy groups and other clusters along the filaments of the cosmic web. Diverse astrophysical processes impact the morphologies, gas content, and star formation properties of the galaxies when they move into the dense cluster environment. Within $R_{200}$ ($\approx 0.75 R_{vir}$) of a cluster, gravitational perturbations such as tidal galaxy-galaxy interactions and mergers (e.g. \citealt{Spitzer1951,Tinsley1979,Merritt1983,Springel2000}), tidal galaxy-cluster interactions \citep{Byrd1990,Valluri1993}, and galaxy-galaxy interactions (sometimes called galaxy `harassment', e.g. \citealt{Moore1996, Jaffe2016}), affect both the stellar and gaseous components of the galaxies. The hydro-dynamical processes such as `starvation’ (e.g. \citealt{Larson1980, Balogh2000}), `thermal evaporation' \citep{Cowie1977}, `ram pressure stripping' (e.g. \citealt{Gunn1972}) and `viscous stripping' \citep{Nelson1982} only affect the gas content of galaxies. Moreover, sometimes these processes are effective in transporting cold gas to the centers of galaxies and may trigger AGN activity \citep{Baldry2004, Balogh2009, Poggianti2017}.

While the neutral atomic hydrogen or \hi gas in galaxies provides the raw fuel for star formation, the morphologies and kinematics of the extended, collisional and fragile \hi gas discs serve as sensitive diagnostic tracers of the environment driven processes (e.g. \citealt{Davies1973, Haynes1984, Williams1987, Abramson2011, Serra2013, Jaffe2015}). One of the nice illustrations of \hi in galaxies in a cluster environment was by the VLA Imaging survey of Virgo galaxies in Atomic gas (VIVA) survey \citep{chung2009}: galaxies near the core of the Virgo cluster show small or disturbed \hi discs of low column density while in many cases, the \hi gas is displaced from the stellar body, trailing the galaxy along its infall trajectory and providing evidence of ram pressure stripping. Occasionally, star formation occurs \textit{in situ} in these gas tails, ionising the gas and giving rise to the `jellyfish' phenomenon (e.g. \citealt{Poggianti2017, ramatsoku2019, ramatsoku2020, Deb2020}). In two other nearby clusters, Coma (e.g. \citealt{Molnar2021}) and Fornax \citep{loni2021}, at least half of the \hi detections show a disturbed \hi morphology, including several cases of asymmetries, tails, offsets between \hi and optical centres, and truncated \hi discs. While most of the \hi-selected Coma galaxies have both enhanced star formation rates and are \hi deficient compared to field galaxies of the same stellar mass \citep{Molnar2021}, Fornax galaxies are \hi deficient and have low star formation rate \citep{loni2021}. This means that the cluster environment affects both the \hi gas content and the star formation rate in galaxies. The extent of the disturbance on \hi morphologies and star formation depends on the properties of the cluster environment and the galaxies. However, \hi deficient and quenched galaxies are also observed in less dense environments including cluster outskirts, galaxy groups and the filaments of the cosmic web, both in observations and simulations (e.g. \citealt{Solanes2001, Tonnesen2007, Yoon2017}). So, the galaxies may already be pre-processed, i.e. have lost their \hi gas and halted their star formation activity prior to entering the dense cluster environment. It is important to understand the nature and efficiency of pre-processing that happens at larger distances from the cluster centers, beyond $R_{200}$, where galaxies can be relatively isolated or reside in groups with miscellaneous dynamic histories and galaxy populations (e.g. \citealt{Laigle2018, Kraljic2018, Alpaslan2016, Sarron2018, Kleiner2017, Vulcani2019}). Thus, wide-area, volume limited, \hi imaging surveys of galaxy clusters that probe the galaxies beyond the virial radius of the cluster, are indispensable to investigate different environmental mechanisms in different clusters for a better understanding of the impact of the different environments on galaxies. 



\begin{figure}[ht!]
 {
    \includegraphics[width=0.5\textwidth]{images/A2626_The_Swarm_histogram_Feb18.pdf}
  } 
  \caption{Redshift distribution of three different environments in and around A2626. green, magenta, and orange histograms present non-substructure, substructure galaxies in A2626 and galaxies in the Swarm respectively. } 
  \label{fig:A2626_TS_histogram} 
 \end{figure}

However, it is difficult to identify the ongoing astrophysical mechanisms for individual galaxies within an observed volume around a cluster only from their \hi morphologies. Moreover, if the galaxies are not well resolved, it is even difficult to unambiguously classify their \hi morphologies. Though it is not possible to confirm in detail the corresponding environmental mechanisms by only investigating the \hi morphologies of the galaxies, it is still possible to broadly identify the plausible environmental mechanisms acting on galaxies based on their \hi morphologies. Moreover, if we also consider \hi deficiency, star formation rate, optical colours and morphologies, and the location of the galaxies with respect to the cluster core, together with the \hi morphologies, we would gain further insight in the corresponding astrophysical interactions with the environment. 

In this work, we will explore the \hi deficiencies and morphologies of galaxies and relate those to the corresponding environmental mechanisms for the \hi detected galaxies in and around the A2626 cluster, recently observed with \mk \citep{HD2021}. For this purpose, we will employ different methods to classify or quantify \hi morphologies  to understand which method is the most suitable to identify a particular environmental process. 

\subsection{A2626 and its neighbourhood}

\label{subsec:diff_envs}

 A2626 is a fairly massive  ($\sim$ 5 $\times$ 10$^{14}$ \msun) galaxy cluster located at a  redshift of z=0.055 \citep{HealySS2021}.  With \mk we surveyed a large cosmic volume, containing three distinct over-densities that represent three different global environments. Galaxies within the A2626 cluster are in the redshift range $0.0475 < z < 0.0615$. Behind A2626, at the redshift range $0.0615 < z < 0.0675$, we find a collection of galaxy groups that we refer to as the Swarm. Behind the Swarm, we find another over-density, in the redshift range $0.0675 < z < 0.0745$, associated with the cluster A2637 (see Fig. 6 and 11 in \citealt{HD2021} and Fig. 5(a,b,c) in \citealt{HealySS2021} ). This cluster is located to the north-east of A2626, near the FWHM of the primary beam of \mk, hence its galaxies suffer from significant primary beam attenuation. Therefore we only consider galaxies at the redshifts in the over-densities of A2626 and \hwone (see Fig. \ref{fig:A2626_TS_histogram}). Table \ref{tab:cluster_properties} presents different properties of A2626 and the Swarm. \citet{HealySS2021} have identified six different substructures in the A2626 cluster based on the  Dressler-Shectman (DS) test (see Fig. 13 in \citealt{HealySS2021}). We note that in one of the substructures, close to the cluster centre,  none of the galaxies are detected in \hi. Due to a relatively low number of \hi detections in each substructure, we have combined the \hi detections of all the substructures together in order to analyse the \hi properties of the substructure galaxies. 
  
  Within these two over-densities (A2626 and the Swarm), we identify three classes of galaxies, depending on their environment. 
  
  \renewcommand{\arraystretch}{1}
\begin{table}
    \centering
    \caption{Properties of A2626 and the Swarm.}
    \begin{tabular}{lclc}\hline 
   Environment & cz ($km/s$) & $\sigma$ ($km/s)$ & R$_{200}$ (Mpc)                 \\ \hline 
 A2626  & 16576 &660 $\pm$ 26  & 1.59\\
 The Swarm   & 19247 & 397 $\pm$ 22 & 0.95\\
     \hline
    \end{tabular}
    \label{tab:cluster_properties}
\end{table}


\begin{enumerate}
    \item  non-substructure or `isolated' galaxies in A2626: 57 galaxies that are in the cluster environment but not a member of a substructure. However,we note that, these non-substructure galaxies are still in a high density environment.
    \item  Substructure galaxies in A2626: 34 galaxies that are within the substructures in A2626. They are in groups within the cluster environment. 
    \item  The Swarm galaxies: 31 galaxies that are within the group environment associated with the Swarm, a structure behind the cluster.
\end{enumerate}

 \noindent We will explore the environmental impact on the \hi deficiencies and morphologies of these 122 galaxies: processing due to the cluster environment as well as pre-processing of galaxies before they fall into the cluster.
 
 \subsection{Description of available data products}
 
 The detailed description of \hi observations, data reduction, and \hi data products are provided in \cite{HD2021}. We have \hi maps, \hi signal-to-noise maps, velocity fields, position velocity diagrams, and \hi global profiles for all 219 \hi detections in the A2626 volume (see \citealt{HD2021} for detail). The sensitivity of MeerKAT, enables the detection of \hi emission well beyond the FWHM of the primary beam and, hence, we imaged an area of 2 $\times$ 2 deg$^{2}$ centred on A2626. For this work, we analysed the \hi data at an angular  resolution of 15" which corresponds to a spatial resolution of $\sim$15 kpc at the distance of A2626. At the centre of the field-of-view and at the redshift of A2626, the $3\sigma$ \hi mass detection limit for a galaxy with a line width of \numunit{300}{\kms} is \numunit{\mhi = 2\times 10^8}{\msun}, with a 5$\sigma$ column density sensitivity of \numunit{\mathrm{N}_\hi = 2.5 \times 10^{19}}{\text{cm}^{-2}} per channel (207 kHz wide corresponding to 45 km s$^{-1}$) when the data are smoothed to $30''$ resolution.
 
Photometric imaging data were taken from the DECam Legacy Survey \citep{dey2019} and the Sloan Digital Sky Survey (SDSS, \citealt{York2000, Aguado2018}). Optical spectroscopic data were mainly taken from the SDSS, and a new dedicated MMT/ Hectospec survey \citep{HealySS2021}. Stellar masses and star formation rates are calculated from WISE photometry kindly provided by T. Jarrett (private communication). Stellar masses are calculated from W1$(3.4 \mu$m) and W2 (4.6$\mu$m) fluxes and SFRs are based mainly on the 12 $\mu$m emission. 


For our analysis, we have adopted optical centres from the SDSS and we have measured the \hi centres by fitting 2D Gaussians to the \hi maps of the galaxies. To quantify how well an \hi map is resolved, we have divided the number of pixels above a certain column density level in the \hi map (e.g. the column density corresponding to 3 times signal-to-noise) by the number of pixels in the \mk synthesized beam. 

\bigskip The remainder of the paper is structured as follows: Sec. \ref{sec:hi_morphologies} presents different ways of classifying \hi morphologies. In Sec. \ref{sec:HI_def} we explore \hi deficiency of galaxies in A2626 and the Swarm. In Sec. \ref{sec:sfr_dep_time}, we investigate the star formation rates and depletion times in the A2626 and the Swarm galaxies. In Sec. \ref{sec:interesting_gals}, we showcase some interesting \hi morphologies of selected galaxies. In Sec. \ref{sec:discusiions}, we discuss our observational findings and address questions raised by our results. Finally, in Sec. \ref{sec:summary} we summarise our work. In this paper, we have used \numunit{{H}_0 = 70}{\kms} and $\Omega_{m} = 0.3$.