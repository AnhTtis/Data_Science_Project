Before we discuss our results in detail, we first present the detailed \hi maps of several interesting galaxies to illustrate the diversity of the \hi morphologies. These galaxies are resolved in our \mk \hi observations and we compare them with the existing DECaLS optical colour images, star formation rate from WISE observations, and their location in phase-space to infer the ongoing physical mechanisms acting on those galaxies. 

\begin{figure*}[t!]
 {
    \includegraphics[width=\textwidth]{images/interesting_galaxies_Feb18.pdf} 
  } 
  \caption{Interesting galaxies with diverse \hi morphologies in and around A2626.} 
  \label{fig:interesting_gals} 
 \end{figure*}

 \subsection{Potential ram-pressure stripped galaxies}
 
 In Fig. \ref{fig:interesting_gals} (a), we present several galaxies that are potentially experiencing ram-pressure stripping or other effects like thermal evaporation or starvation. Galaxies 110 and 128 are non-substructure galaxies in A2626, located close to the cluster core (within R$_{200}$) and fully exposed to the ICM. Moreover, they are highly \hi deficient, have small \hi discs with yellowish optical discs hinting at a quenched star formation, confirmed by their low star formation rates inferred from the WISE fluxes. Hence, they probably have experienced  ram-pressure stripping or thermal evaporation for a long time which has resulted in the removal of a significant amount of \hi gas. Galaxy 178 is located in a substructure in A2626. Although it is not very close to the cluster core ($\sim$ 1.5 R$_{200}$), the \hi morphology hints at ongoing ram-pressure stripping. Its \hi contours are compressed on the northern side of the \hi disc and the outer \hi contour seems to suggest that the \hi gas is pushed to the south. Interestingly, galaxy 178 is moving at a velocity similar to the mean velocity of the cluster and has a low star formation rate. Therefore one may conclude that galaxy 178 is a backsplash galaxy, but the \hi content of the galaxy suggests it is still on its first infall trajectory.
 
 \subsection{Galaxies with unsettled \hi discs}
 
 In Fig. \ref{fig:interesting_gals} (b), we present some well-resolved galaxies with unsettled \hi discs (Vclass 3) and high \amod values (\amod$>$0.5). Galaxy 87 is a face-on system in the Swarm with a regular optical morphology, but an extended unsettled \hi disc. We note a nearby companion $\sim$ 70" to the south-west. Galaxy 99 is in a substructure in A2626, not only does it have an unsettled \hi disc, but optically it is also very disturbed. It is likely that it has experienced a tidal interaction with another galaxy in the substructure. Galaxy 130 is a foreground galaxy at a redshift of z$_{HI}$=0.0382, and therefore is excluded from our overall analysis. Its optical morphology is that of an early-type, barred ring galaxy. \hi emission is not only found in the bright central regions of the galaxy, but clumps of \hi gas are also found in the faint outer stellar ring surrounding the bar, similar to NGC 4736 \citep{bosma1977}.
 
\subsection{Interacting galaxies}

In Fig. \ref{fig:interesting_gals} (c), we present examples of interacting galaxies in our sample that are confirmed to have similar velocities based on the MMT as well as \hi redshifts \citep{HealySS2021}. Among the interacting galaxies shown in Fig. \ref{fig:interesting_gals} (c), galaxies 15, 122, 135, 138, 163 and 164 are in A2626 itself. Galaxies 52 and 57 are in the Swarm while galaxies 93, 96, 108, 109, and 174 are in the background cluster A2637. The \hi emission of galaxies 15, 122, and 174 is blended with that of their companion galaxies. We conclude that the gravitational interactions are prevalent in both the group and the cluster environment in and around A2626.