In addition to exploring different \hi morphological classifications and their significance, we have explored the \hi deficiency as another indicator of environmentally induced gas depletion and removal processes. We examined a galaxy's \hi deficiency as a function of its projected distance from the centre of A2626, its position in the phase-space diagram of A2626, and its \hi morphology.

\hi deficiency is defined as a logarithmic quantity (Haynes \& Giovanelli 1983), a difference between the log of expected \hi mass and the log of observed \hi mass of a galaxy.

\begin{equation}
    HIdef = log[M_{HI}^{exp}] - log[M_{HI}^{obs}]
\end{equation}
 
 \begin{figure*}[ht!]
 {
    \includegraphics[width=\textwidth]{images/HI_def_plot1_Apr16.pdf} 
  } 
  \caption{\footnotesize{Top left: \hi deficiency vs projected distance normalised by R$_{200}$ for the non-substructure (green circles) and substructure (magenta circles) in A2626. Horizontal dashed lines presents the range of \hi deficiencies of the field galaxies. There is a strong correlation between \hi deficiency and projected distance - \hi deficient galaxies seem to reside close to the cluster core. Top right: Histogram of the distribution of \hi deficiencies of galaxies in the Swarm. Bottom left: The distribution of the non-substructure (green) and substructure (magenta) galaxies in A2626 in the projected phase-space. The black dashed lines indicate the escape velocity. There is no clear difference between the distribution of galaxies in the two different environments apart from the previously observed trend of the proximity of non-substructure galaxies to the cluster core. Bottom right: Six \hi maps overlaid on DECaLS colour images for the most \hi deficient (top three) and the most \hi rich (bottom three) galaxies. The \hi deficient galaxies seem to be bright and yellowish with offset or truncated \hi discs. The \hi rich galaxies are fainter and bluish, with extended \hi discs.} }
  \label{fig:HIdef_plot1} 
 \end{figure*}

 \noindent Thus \hi deficiency is positive for \hi deficient galaxies and negative for galaxies with excess \hi gas. $M_{HI}^{exp}$ is generally calculated from \hi-optical scaling relations (e.g. \citealt{Haynes1984, chamaraux1986, Batuski1985, solanes1996, denes2014}). We used the scaling relation from \citet{denes2014} (see their Table 3), a multiwavelength scaling relation between the \hi content and the optical diameter of the galaxies. The scaling relation of Denes 2014 is based on the r-band diameters reported by HyperLEDA.
 
 \begin{equation}
   log[M_{HI}^{exp}] = \alpha_{\lambda} + \beta_{\lambda} log[D_{\lambda}]  
 \end{equation}
 
\noindent where $M_{HI}^{exp}$ is the expected \hi mass, $D_{\lambda}$ is the optical diameter (in kpc) in a particular band, $\alpha_{\lambda}$ and $\beta_{\lambda}$ are the parameters for this band. To calculate the expected \hi mass, we have measured the diameters ($D_{r}$=2R$_{25}$) from the DECaLS r-band images where R$_{25}$ is the radius of the 25th mag/arcsec$^2$ isophote.

\subsection{\hi deficiency in substructures}

The top left panel of Fig. \ref{fig:HIdef_plot1} shows the \hi deficiency of galaxies vs. their projected distance from the center of A2626, normalised by R$_{200}$ of A2626. R$_{200}$ is the projected radius of a sphere with mean density 200 times the critical density of the universe. Light and magenta symbols represent the non-substructure and the sub-structure galaxies in A2626 respectively. The area between the three horizontal dashed lines represents the \hi deficiency range for field galaxies. The vertical dashed line indicates R$_{200}$ of A2626. The \hi deficiencies of the Swarm galaxies are presented in the orange histogram in the top right panel of Fig. \ref{fig:HIdef_plot1}. First of all, we observe a correlation between \hi deficiency and the projected distance from the center of A2626 (Spearman's coefficient= -0.377, p-value= 0.0012). This clearly illustrates that galaxies become \hi deficient towards the cluster core. Moreover, within 1.5 R$_{200}$, the most \hi deficient galaxies seem to be the non-substructure galaxies. On average, non-substructure galaxies are more \hi deficient ($< HIdef>_{non-substructure}$ = 0.27, $<HIdef>_{ss}$=0.15) than sub-structure galaxies in A2626. We note, however, that the substructure galaxies are more prevalent at larger cluster centric radii. The six panels in the bottom-right of Fig. \ref{fig:HIdef_plot1} illustrate \hi maps overlaid on DECaLS colour images for the most \hi deficient (top three) and the most \hi rich (bottom three) galaxies. Focusing on them, we observe that the \hi deficient galaxies are bright and have a yellowish colour, and have small or truncated \hi discs. On the contrary, the \hi rich galaxies are fainter with a bluish colour and have extended \hi discs.

\begin{figure}[t!]
\begin{center}

 {
    \includegraphics[width=0.45\textwidth]{images/A2626_HIdefs_Feb18.pdf} 
  } 
  \caption{\footnotesize{HI deficiency vs projected distance normalised by R$_{200}$ (similar to Fig. \ref{fig:HIdef_plot1}), with symbols and colour codes to include additional information regarding \hi morphologies of galaxies in A2626. Top panel: Different symbols (shown in the top right corner) to represent non-substructure (green) and substructure (magenta) galaxies of different visual classes. Middle panel: The colour scale (colourbar shown in the right) represents \hi offsets of non-substructure (filled circles) and substructure (open circles) galaxies respectively. Bottom panel: The colour scale (colourbar shown in the right) represents \amod value of non-substructure (filled circles) and substructure (open circles) galaxies respectively.}} 
  \label{fig:HIdef_plot2} 
  \end{center}
 \end{figure}

\subsection{\hi deficiency in phase-space}

In the bottom left panel of Fig. \ref{fig:HIdef_plot1}, we show the phase-space diagram for non-substructure and sub-structure galaxies in A2626. The black dashed lines indicate the escape velocity for A2626 galaxies, calculated using the formalism from \citet{Jaffe2015}. We assumed a concentration index C = 6 and the mass enclosed by R$_{200}$ is calculated using:

\begin{equation}
    M_{200} = \frac{4}{3} \pi R^{3}_{200}200\rho_{c}
\end{equation}

\noindent where M$_{200}$ is the mass enclosed by R$_{200}$ and $\rho_{c}$ is the critical density. The vertical grey dashed line in the phase-space diagram represents R$_{200}$ of A2626. In the phase-space diagram, there is no clear additional trend in the distribution of non-substructure/ sub-structure galaxies.

\subsection{\hi deficiency and \hi morphology}

As a next step, in Fig. \ref{fig:HIdef_plot2}, we have explored the same \hi deficiency vs. projected distance plot with colour codes and symbols to include additional information regarding the \hi morphologies of those 76 galaxies. 

\textbf{\hi deficiency and visual classifications}

In the top  panel of Fig.  \ref{fig:HIdef_plot2}, we present the \hi detected galaxies of the three visual classes as defined in Sec. \ref{subsec:visual_classes} with circle, triangle, and star symbols respectively. The light and magenta symbols in each class stand for the non-substructure and sub-structure galaxies in the A2626 cluster. Interestingly, the VClass 3 or unsettled galaxies (star symbols) seem to have \hi deficiencies similar to the field galaxies (i.e. within the three horizontal dashed lines). Moreover, substructures contain more disturbed / unsettled galaxies compared to the non-substructure galaxies in A2626 as we have seen in Sec. \ref{subsec:visual_classes}. However, there is no obvious correlation of the visual classes of the \hi detected galaxies with their \hi deficiencies or with their projected distance from the cluster centre. Thus, the visually classified \hi morphologies are not directly related to the \hi deficiencies of the galaxies or their location in the A2626 cluster. 

\textbf{\hi deficiency and \hi offsets}

In the middle panel of Fig. \ref{fig:HIdef_plot2}, we have again plotted \hi deficiency vs. projected distance from the center of A2626 for the \hi detected galaxies, with a colour coding according to the offset (in units of kpc) of the \hi centers of those galaxies with respect to the optical centers. The filled and open circles are non-substructure and substructure galaxies respectively. There is no correlation between \hi deficiency and \hi offset for the galaxies in A2626. This implies that the \hi deficient galaxies are not necessarily \hi offset and vise versa.

\textbf{\hi deficiency and \amod}

In the bottom panel of Fig. \ref{fig:HIdef_plot2}, we examine the \hi deficiency vs projected distance, with a colour coding indicating the \amod values of the \hi discs of galaxies in A2626. Since \amod is meaningful for only 11 well-resolved, high signal-to-noise galaxies in A2626, we have plotted only those galaxies. The filled and open symbols are non-substructure and sub-structure galaxies in A2626. Similar to what we see in two other panels in Fig. \ref{fig:HIdef_plot2}, there is no clear trend between \hi deficiency, projected distance from the cluster centre, and \amod in the galaxies in A2626. Interestingly, the most \hi deficient galaxy in this plot is also the most \hi asymmetric galaxy with a high \amod value (\amod = 0.6).

\bigskip Considering all panels of Fig. \ref{fig:HIdef_plot2}, apart from the clear trend between \hi deficiency and the projected distance from the cluster centre, we conclude that there are no further obvious correlations between \hi deficiency, \hi morphology, and location of the galaxy in the cluster, except that VClass 3 galaxies are not \hi deficient.