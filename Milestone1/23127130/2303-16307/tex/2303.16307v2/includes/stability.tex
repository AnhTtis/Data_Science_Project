\section{Stability}
\label{appendix:A}

A linear differential equation is said to be stable, in the
sense of Laplace \cite{struble, teschl2012ordinary}, if all solutions are bounded as $t\to\infty.$  We will show that Equation (\ref{eq:00}) is stable due to the non-negativity of both $\malware(t)$ and $\bonware(t)$.  Our approach is to consider the simplified cases where each of $\malware(t)$ and $\bonware(t)$ are zero, show that each of these equations is stable, and then show that this implies Equation (\ref{eq:00}) is stable.

% \begin{equation}
%    \frac{d\functionality}{dt} + (\malware(t)+\bonware(t)) \functionality(t) = \Fnominal \bonware (t).
%    \label{eq:00}
% \end{equation}
% has solution
% \begin{equation*}
%     F(t)  = \expminus \left( F(0) + \bintegral \right).
% \end{equation*}
% First, let's look at $\malware(t)$ positive and $\bonware(t)=0$.  Then we can reason about both non-negative!
Consider Equation (\ref{eq:00}) with $\bonware(t)=0.$

\begin{equation}
    \frac{d\functionality}{dt} + \malware(t) \functionality(t) = 0.
    \label{eq:11}
\end{equation}
The solution is 
\begin{equation}
  \functionality(t) = F_0 e^{-\int_0^t M(p)\, dp},
    \label{eq:22}
\end{equation}
and since $\malware(t) \ge 0$,  $\functionality(t)$ is bounded everywhere ($0  \le \functionality(t) \le F(0)$).

Let us now discuss Equation (\ref{eq:00}) with $\malware(t)=0.$

Then 

\begin{equation}
    \frac{d\functionality}{dt} + \bonware(t)(\functionality(t) - \Fnominal) =0.
    \label{eq:001}
\end{equation}
or
\begin{equation}
    \frac{d\Phi}{dt} + \bonware(t)\Phi(t) =0. \quad \text{where } \Phi(t) = F(t) - \Fnominal
    \label{eq:002}
\end{equation}

Equation (\ref{eq:002}) has solution
\begin{equation*}
    \Phi(t)  = (F(0)-\Fnominal) e^{-  \int_0^t B(p) \, dp }.
\end{equation*}
so that 
\begin{equation*}
    \functionality(t)  = \Fnominal+ (F(0)-\Fnominal) e^{-  \int_0^t B(p) \, dp }.
\end{equation*}
%
Since we require $\bonware(t) \ge 0$, $\functionality(t)$ is bounded
everywhere ($F(0) \le F(t) \le \Fnominal$).

We've shown that in intervals when either $\malware(t)$ or $\bonware(t)$ are
zero, the solution to the differential equation is bounded.  Now
consider the case where both $\malware(t)$ and $\bonware(t)$ are strictly
positive on an interval $(t_1,t_2)$, initially with $F(t_1)>0$  Let $\functionality(t)$ be a solution to
(\ref{eq:00}) and let $\functionality_1(t)$ be a solution to (\ref{eq:001}).
Subtracting, we obtain:
\begin{equation}
  \frac{d}{dt} (\functionality_1(t)-\functionality(t)  )
  + \bonware (t) (\functionality_1(t)-\functionality(t)  ) =
  \malware(t) \functionality(t).
    \label{eq:005}
\end{equation}
Assume that there exists a time that $\functionality(t)=0.$  Then by
(\ref{eq:005}), 
\begin{equation*}
\frac{d \functionality_1(t) }{dt} + \bonware(t) \functionality_1(t) =0.
\end{equation*}
But $\functionality_1(t)$ satisfies equation (\ref{eq:001}).  So
$\bonware(t)$ must be zero contradicting our assumption that both
$\malware(t)$ and $\bonware(t)$ are positive.  Thus
$\functionality(t)>0$ on $(t_1,t_2)$ and thus is bounded below.
%

Assuming again that $\functionality(t)$ satisfies (\ref{eq:00})
and $\functionality_1(t)$ satisfies (\ref{eq:001}).  We further assume,
$\functionality_1(t_1)=\functionality(t_1).$  Then at $t_1$, we have
$\frac{d\functionality(t)}{dt} < \frac{d\functionality_1(t)}{dt}$
implying that at time $t^\star>t_1$,
$\functionality(t^\star)<\functionality_1(t^\star)$.  So $\Fnominal \ge
F_1(t) \ge F(t)$ everywhere, and $\functionality(t)$ is bounded above
by $\Fnominal.$  We have thus shown that if $\functionality(t)$ is a
solution to (\ref{eq:00}), then $\Fnominal \ge \functionality(t) \ge
0$ and thus Equation (\ref{eq:00}) is stable in the sense of Laplace. 
