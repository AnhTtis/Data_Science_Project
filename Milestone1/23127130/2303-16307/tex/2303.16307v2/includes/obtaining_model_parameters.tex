\subsection{Obtaining model parameters}\label{sec:parameters}
Given data that represents functionality over the course of an incident where malware and bonware are active, we develop a fast method to estimate the continuous model parameters for a curve that approximates the data, and use these parameters to generate further realizations based on this model.  In Fig.~\ref{fig:notional}, notional data is shown (in orange) and the parameters $\malware$ and $\bonware$ are estimated and a fit for the functionality $F(t)$ is found that solves the piecewise constant model expressed by Eq.~(\ref{eq:000}). 
In this section, we illustrate our fast method to extract the model parameters from this curve.  %The method was used to determine the coefficients $\malware_i$ and $\bonware_i$ in the piecewise model fit to our experimental data.  For example, in Fig.~\ref{fig:model fits}, a continuous fit to the experimental data is shown, and the parameters $\malware_i$ and $\bonware_i$ are estimated.

The set $P=\{t_0, \hdots, t_K\}$ partitions the scenario timeline, and malware and bonware are constant in each interval $(t_{i-1},t_i), \,\, i=1, \hdots, K. \,\,$  In each interval, $\Qware_i=\malware_i+\bonware_i$ and the differential equation governing Continuous Model I is $\frac{dF(t)}{dt}+\Qware_i F(t) = \Fnominal (t) \bonware_i.$  Thus, in each interval $(t_{i-1},t_i),$ the solution is 
\begin{equation*}
    \functionality(t)     =
    \left(\functionality(t_{i-1}) - \frac{\Fnominal \bonware_i}{
        \Qware_i } \right)  e^{-\Qware_i (t-t_{i-1})} + \frac{\Fnominal \bonware_i}{\Qware_i}.
\end{equation*}
We compute the \impact{} of malware $\malware_i$ and the \impact{} of bonware $\bonware_i$ in each interval.

We observe that there is a unique switching time $\tchange$ where the functionality's trend reverses, and thus we take $K=2.$ Before the switch, the \impact{} of malware is greater than that of bonware. From the time of the switch until the end of the run, bonware is stronger.  To estimate the switching time $\tchange$, we find the minimum of the data to occur over the interval from 64 s to 75 s. There, the minimum value of the data curve is $m=0.27$.  Taking the midpoint, our estimate for $\tchange$ is 69.5 s.

We numerically solve this system of equations:
\begin{align*}
    \alpha m & = \Fnominal \bonQone,                                           \\
    m        & = \left(\functionality(0)  -\Fnominal \bonQone\right) e^{-\Qware_1 \tchange}+\Fnominal \bonQone.                      
\end{align*}
The first equation says that where the curve meets the minimum of the data, it has experienced exponential decay of $\alpha$ toward the asymptotic minimum.  We take $\alpha$ to be $\alpha=1-\sfrac{1}{e}$. The second equation says that the minimum occurs at the switching time (the time when the model switches from malware dominating bonware, to bonware dominating malware).  Solving this system of equations yields (with $\malware_1=\Qware_1-\bonware_1$), $\malware_1 \approx  0.025$ and $\bonware_1 \approx 0.005$.  To the right of $t^\star$, we fit an exponentially increasing function by numerically solving this system of equations:
\newcommand\FBonQtwo[1]{\frac{#1 \bonware_2}{\Qware_2}}
\begin{align*}
    \zeta                & = \FBonQtwo{F(0)},                     \\
    \tilde{\alpha} \zeta & = \left(m- \FBonQtwo{\Fnominal}\right) 
    \!\! \left(e^{-\Qware_2 (125-t^\star)}+\FBonQtwo{\Fnominal}\right).
\end{align*}
We have found that $\tilde{\alpha}=1-e^{-4}$ and $\zeta=0.95$ are satisfactory values to use for these hyperparameters.  We compute $\malware_2 \approx 0.005$ and $\bonware_2 \approx 0.088$.
