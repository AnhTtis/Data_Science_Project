\subsection{Universality of the disk structure in the disk with grown dust grains}
In this study, we propose the new disk evolutionary picture ``co-evolution of dust grains and protoplanetary disks" based on non-ideal dust-gas two-fluid MHD simulations considering dust growth. The dust growth changes the gas-phase ionization degree, magnetic resistivity, and evolution of the disk.
In the co-evolution process, the microscopic dust grains of micrometer size couple with the macroscopic disks of $100$ AU size and they co-evolve. The size scale difference between two objects is $10^{19}$, which is astounding compared to the well-known co-evolution of super massive black holes and galaxies (size scale difference is $\sim 10^{10}$).

Furthermore, once the dust grains grow sufficiently, the structure of protoplanetary disks is well described by the non-trivial power laws, which we analytically derive in equations (\ref{solution_first}) to  (\ref{solution_last}) (figure \ref{1D_time_evolution_q25_a100nm} and \ref{rad_prof_all_models}).
From the assumptions adopted in the analytical solutions, we conclude that the disk structures will emerge when
\begin{enumerate}
\item Dust grains grow sufficiently and adsorption of charged particles by the dust grains becomes negligible,
\item The toroidal magnetic field in the disk is determined by the balance between vertical shear (of the order of $(H/r)^2$) and ambipolar diffusion, and 
 \item Angular momentum transport mechanisms other than magnetic braking (such as turbulent viscosity) are negligible.
\end{enumerate}

We believe that the discovery of this new disk structure is a theoretical breakthrough for star and planet formation theory.
The disk structure is determined only by observable parameters such as the central star mass, mass accretion rate, disk temperature, and cosmic-ray ionization rate, without including difficult-to-determine parameters such as the viscous parameter $\alpha$.
Using the analytcal soluiton, we can study the planet formation process in the realistic disk
and evolution of the magnetic flux during protostellar evolution.
In the future, we will discuss the broad implications of this disk model for the formation and evolution of protostars and planets.

\subsection{Assumptions employed in the dust growth model and their uncertainty}

Our simulations make several simplifications to the dust growth and dust size distribution.
The largest simplification is the representative size approximation for dust growth in which we assume that the representative size corresponds to the peak dust size of the mass distribution (note that the peak of the dust mass distribution corresponds to the maximum dust size if $q<4$).
Our approximated equation for dust growth can be derived from the coagulation equation (for derivation, see \citep{2016A&A...589A..15S}).
In this study, we solve the evolution of the representative size and regard it to be the maximum dust size $a_{\rm max}$, and set the minimum dust size and power as parameters.
Moreover we implicitly assume that the size distribution can be described by a single power law.
More realistically, the time evolution of the dust size distribution should be considered, and the validity of the simplifications employed in this study should be investigated in future more realistic studies.

However, our claim, "the disk structure converges to the analytical solution once the dust has grown sufficiently", remains valid regardless of the specific details of the dust distribution and dust growth model.
This is because the essential physics required for the disk to converge to the analytical solutions is that the total dust surface area becomes sufficiently small and adsorption of charged particles by the dust grains becomes negligible.
Under these conditions, the  ambipolar resistivity converges to $\eta_A \propto B^2 \rho^{-3/2}$ and we obtain the analytical solutions.
In this sense, our results are universal, regardless of the details of the dust growth model.
