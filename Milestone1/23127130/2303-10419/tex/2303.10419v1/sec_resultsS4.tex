\subsection{Time evolution}
In this section, we examine the time evolution of disk angular momentum (and radius), disk  mass, typical dust size and dust abundance in disks. We also investigate the mass ejection rate by outflows from the disks.

\subsubsection{Time evolution of disk size}
Figure \ref{disk_size} shows the time evolution of
centrifugal radius and the angular momentum of the disk.
The angular momentum of disk $J(\rho_{\rm disk})$ is calculated as
\begin{eqnarray}
J(\rho_{\rm disk})\equiv \left| \int_{\rho_g>\rho_{\rm disk}} \rho_g (\rad \times \vel) d V \right|.
\end{eqnarray}
For the density threshold of the disk, we choose $\rho_{\rm disk}=10^{-13} \gcm$.
The centrifugal radius is calculated as
\begin{eqnarray}
  %r_{\rm cent}=  \frac{\bar{j}(\rho_{\rm disk})^2}{G(M_{\rm star}+\epsilon)}.
r_{\rm disk} \equiv  r_{\rm cent}=  \frac{\bar{j}(\rho_{\rm disk})^2}{G M_{\rm star}}.
\end{eqnarray}
Here $\bar{j}(\rho_{\rm disk})=J(\rho_{\rm disk})/M_{\rm disk}$, where $M_{\rm disk}$ denotes the enclosed gas mass within the region $\rho_g>\rho_{\rm disk}$.
Comparing the radius of the region with a density of $\rho_g > \rho_{\rm disk}$   (figure \ref{rad_prof_all_models}) with the centrifugal radius, the former is about 2 times larger than the latter owing to radial density distribution and temporal density oscillation by the non-axisymmetric structures. In this study, we consider the centrifugal radius as an estimate of the disk size.

The left panel of figure \ref{disk_size} shows the results of the models with $a_{\rm min}=100~\nm$. 
In ModelA100Q25 (red line),  the  angular momentum of the disk continues to increase until $t_* \sim 7\times 10^3$ yr, and then it shows a sharp decrease.
This is because the dust grains grows to $a_{\rm d}\gtrsim 10 \mum$ (figure \ref{dust_size_mass}), which causes a sudden decrease in magnetic resistivity and the extraction of angular momentum by magnetic field.
Interestingly, although the angular momentum has decreased by a factor of $\sim 1/3$ from $t_* \sim 7\times 10^3$ yr to $t_* \sim 9 \times 10^3$ yr, the centrifugal radius has decreased by only a factor of $\sim 1/2$.
This indicates that the disk mass also decreases rapidly during this period (figure \ref{disk_mass}).
For ModelA100Q35, no such rapid decrease of angular momentum is observed.
This is because $a_{\rm min}$  is also responsible for the total dust surface area, and thus the decrease in magnetic resistivity is not so drastic \citep{2022ApJ...934...88T}.
As exhibited by  ModelZeta-18 (green line), the low cosmic-ray ionization rate can contribute to maintaining the angular momentum.
This is in agreement with previous studies \citep{2018MNRAS.476.2063W,2020A&A...639A..86K, 2023MNRAS.tmp..691K}.
The decrease in the disk size of ModelA100Fixed is caused by the pseudo-disk warp and associated inward magnetic flux drag \citep{2020ApJ...896..158T}.
In this model, the angular momentum decreases less than a factor of two, which is not significant compared to ModelA100Q25.

The right panel of figure \ref{disk_size} shows the results of the models with $a_{\rm min}=5~\nm$. 
Interestingly, the relationship between the disk size and power exponent $q$ is different from the models with $a_{\rm min}=100 \nm$.
the disk size of the model with $q=3.5$ (ModelA5Q35) is significantly smaller
than that in the model with $q=2.5$ (ModelA5Q25).
This is because when there is a large amount of small dust ($\sim \nm$), dust grains are responsible for the conductivity and reduce the magnetic resistivity.
This allows magnetic braking to work more effectively in ModelA5Q35 and reduce the disk size. 


\subsubsection{Time evolution of the disk mass}
Figure \ref{disk_mass} shows the time evolution of disk mass $M_{\rm disk}$ (solid), protostar mass $M_{\rm star}$ (dashed), and total mass $M_{\rm star}+M_{\rm disk}$ (dotted). 
In ModelA100Q25 (red), the disk mass continues to increase and reaches $\sim 0.15 \msun$ at $t_* \sim 7 \times 10^3$ yr. Then, it drops sharply to $M_{\rm disk}\sim 0.03 \msun$ at $t_* \sim 9 \times 10^3$ yr.
Meanwhile, mass accretion onto protostars is enhanced and the protostellar mass rapidly increases from $0.1 \msun$ to 0.2 $\msun$ in this period, giving a mass accretion rate of $\gtrsim 10^{-5} \msunyear$ for this model.
In ModelZeta18 (green line) and ModelA5Q25 (black line), it appears that the disk mass also begins to decrease towards the end of the simulations. However, the time at which the decrease begins is later than in ModelA100Q25.

In ModelA100Fixed, the disk radius decreases in $t_*\gtrsim 6\times 10^3$ yr (figure \ref{disk_size}) but the disk mass does not significantly change.
This suggests that the disk evolution without dust growth is different from the models that include dust growth and that are consistent consistent with the analytic solution.
In ModelA5Q35, the mass (and radius) evolution differs from the other models.
This is due to inefficient ambipolar diffusion since disk formation. Thus, the situation close to the ideal MHD is realized.



\subsubsection{time evolution of dust size and dust abundance in the disk}

Figure \ref{dust_size_mass} shows the time evolution of the dust-to-gas mass ratio and mean dust size of the disks.
The dust mass and mean dust size of the disk is calculated as
\begin{eqnarray}
\bar{a}_{\rm disk}\equiv \frac{1}{M_{\rm disk}}\int_{\rho_g>\rho_{\rm disk}} \rho_g a_d d V,
\end{eqnarray}
and 
\begin{eqnarray}
M_{\rm dust, disk} \equiv \int_{\rho_g>\rho_{\rm disk}} \rho_d d V,
\end{eqnarray}
respectively.

The figure shows that the increase in the dust-to-gas mass ratio occurs later in the simulation.
This increase begins when the average dust size in the disk exceeds $\sim 100 \mum$.
This is due to the ``ash-fall phenomenon" proposed in our previous study \citep{2021ApJ...920L..35T}.
The largest increase is observed in ModelA100Q25, where the dust-to-gas mass ratio increases to 1.04\% at the end of the simulation.
Some readers may think that this value is small and irrelevant.
However, we only considered $\sim 10^3$ yr after the ratio started to increase.
If this event continues for, for instance, $10^5$ yr (i.e., during the Class 0/I phase), it can cause a significant increase in the dust abundance.

The increase of the dust-to-gas mass ratio is slower in ModelA100Q35, ModelA5Q25, and ModelZeta18, compared to ModelA100Q25 owing to the lower outflow activity (i.e., weaker coupling between the magnetic field and gas in the upper layers of the disk;  figure \ref{Fout}) in these models.



\subsubsection{Time evolution of the mass ejection rate by outflow}
Figure \ref{Fout} shows the time evolution of the mass ejection rate due to the molecular outflow.
The mass ejection rate is calculated as
\begin{eqnarray}
\dot{M}_{\rm out} \equiv\int_{r=100 {\rm AU}} \rho v^+_r dS,
\end{eqnarray}
where $v_r^+ \equiv \max(\vel\cdot \rad,0)$ denotes the positive radial velocity, and we perform the surface integral on a sphere with radius of $100 {\rm AU}$.


The mass ejection rate is highly variable and has a peak value of $\gtrsim 10^{-5} \msunyear$. This is comparable to the mass accretion rate in the disk.
Comparing the mass ejection rate with the epoch when the dust-to-gas mass ratio begins to increase (at $t_* \sim 8\times 10^3$ yr; figure \ref{dust_size_mass}), a mass ejection rate of $10^{-6}$ to $10^{-5} \msunyear$ and the mean dust size of $\sim 100 \mum$ are required for the "ash-fall" phenomenon to happen, in which the disk gas is selectively ejected into interstellar space by the outflow and the dust grains are resupplied to the disk.


