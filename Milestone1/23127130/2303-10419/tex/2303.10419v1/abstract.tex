We propose a new evolutionary process of protoplanetary disks "co-evolution of dust grains and protoplanetary disks", revealed by dust-gas two-fluid non-ideal magnetohydrodynamics simulations considering the growth of dust and associated changes in magnetic resistivity. We found that the dust growth significantly affects disk evolution by changing the coupling between the gas and magnetic field. Moreover, once the dust grains sufficiently grow, the physical quantities (e.g., density and magnetic field) of the disk are well described by nontrivial power laws, regardless of the details of the dust model. In this disk structure, the radial profile of density is steeper and the disk mass is smaller than those of the model ignoring dust growth and they are more consistent with the disk observations.
We analytically derive these power laws from the basic equations of non-ideal magnetohydrodynamics. The analytical power laws are determined only by observable physical quantities, e.g., central stellar mass and mass accretion rate, and do not include difficult-to-determine parameters e.g., viscous parameter $\alpha$. Therefore, they are applicable to various stages of disk evolution. We believe that the disk structure provides a new basis for future studies on star and planet formation.
