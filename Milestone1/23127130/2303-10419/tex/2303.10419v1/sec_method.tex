\subsection{Numerical methods}
\subsubsection{Two-fluid magneto-hydrodynamics simulations}
We solve two-fluid magnetohydrodynamics equations for the dust-gas mixture.
The governing equations are given as
\begin{eqnarray}
  \label{single_fluid_continum}
  \frac{D \rho}{D t}  &=& - \rho \nabla \cdot \vel, \\
  \label{single_fluid_eps}
  \frac{D \epsilon}{D t} &=& -\frac{1}{\rho}\cdot \{\epsilon(1-\epsilon)\rho \Delta \vel \}, \\
  \label{single_fluid_motion1}
  \frac{D \vel }{D t} &=&  -\frac{1}{\rho} \{\nabla P-  \frac{\cul \times \magB}{c} \}, \\ 
  \label{single_fluid_dv2}
  \frac{D \dv }{D t} &=&-\frac{\dv}{\tstop}+\frac{1}{\rho_g}[-\nabla P+ \frac{\cul \times \magB}{c}],\\
  \frac{D \magB}{Dt}&=&-\magB (\nabla \cdot \vel)+(\magB\cdot \nabla)\vel \nonumber \\
  &+& c \nabla \times \{  \eta_{O} \cul +\eta_A (\cul \times {\hat \magB}) \times {\hat \magB}  \},
\end{eqnarray}
where $\rho_{[g, d]}$ denotes the mass densities and subscripts $[g, d]$ denote gas and dust components, respectively. $\rho=\rho_g+\rho_d$ denotes the total density.
$\epsilon=\rho_d/\rho$ denotes the dust-to-total-mass ratio,
$\vel=(\rho_g \vel_g+\rho_d \vel_d)/(\rho_g+\rho_d)$  denotes the barycentric velocity of dust gas mixture where $\vel_{[g, d]}$ denotes the gas and dust velocity,
$\dv=(\vel_d-\vel_g)$  denotes the the velocity difference between gas and dust,
$P$  denotes the gas pressure.
$\cul$  denotes the electric current.
$c$  denotes the speed of light.
$\eta_O$ and $\eta_A$   denote the Ohmic and ambipolar resistivities, respectively.
The details of the approximations adopted in the governing equations and numerical method are described in \citet{2021ApJ...913..148T}.
Our numerical simulations consider the Ohmic and ambipolar diffusions, but ignore the  Hall effect.

\subsubsection{Dust growth}
We consider dust growth
with single-size approximation \citep{2016A&A...589A..15S,2016ApJ...821...82O, 2017ApJ...838..151T}.
The governing equation of dust growth is
\begin{eqnarray}
  \label{dadt}
  \frac{D_d a_{d}}{Dt}=A_{\rm gain/loss} \frac{a_d}{3 t_{\rm growth}},
\end{eqnarray}
where $a_{d}$ denotes the representative dust size, $t_{\rm growth}=1/(\pi a_d^2 n_d \Delta v_{\rm dust})$, $n_d$ denotes the dust number density,
$\Delta v_{\rm dust}$ denotes the collision velocity between the dust grains,
$D_d/Dt=\partial/\partial t +\vel_d \cdot \nabla$,
and
\begin{eqnarray}
  \label{collision_factor}
A_{\rm gain/loss}={\rm min}(1,-\frac{\ln(\Delta v_{\rm dust}/\Delta v_{\rm frag})}{\ln 5}),
\end{eqnarray}
which models the collisional mass gain and loss \citep{2016ApJ...821...82O}.
We assume $v_{\rm frag}=30 \ms$.
For the dust relative velocity $\Delta v_{\rm dust}$, we consider
the sub-grid scale turbulence and Brownian motion.

For the turbulent-induced dust relative velocity $\Delta v_{\rm turb}$,
we adopt the prescription presented by \citet{2007A&A...466..413O},
\begin{align}
\Delta v_{\rm turb}&= \nonumber \\
&\begin{cases}
  \frac{\delta v_{\rm Kol}}{t_{\rm Kol}} (t_{\rm stop, 1}-t_{\rm stop, 2}) & (t_{\rm stop, 1}<t_{\rm Kol}) \nonumber \\
  1.5 \delta v_L \sqrt{\frac{t_{\rm stop, 1}}{t_L}} & (t_{\rm Kol}< t_{\rm stop, 1}<t_L)  \nonumber \\
  \delta v_L \sqrt{\frac{1}{1+t_{\rm stop, 1}/t_L}+\frac{1}{1+t_{\rm stop, 2}/t_L}} & (t_L< t_{\rm stop, 1})  \nonumber 
\end{cases}
\\
\end{align}
where  $\delta v_{\rm Kol}=Re_L^{-1/4}$ and $t_{\rm Kol}=Re_L^{-1/2} t_L$ denote the
eddy velocity and eddy turn-over timescale at dissipation scale and $Re_L=L v_L/\nu$ denotes the Reynolds number.
We set the stopping time $t_{\rm stop, 1}=t_{\rm stop}(a_d)$
and  $t_{\rm stop, 2}=1/2~ t_{\rm stop}(a_d)$ referring to \citet{2016A&A...589A..15S}, where
$t_{\rm stop}(a_d)$ is calculated as in our previous study \citep{2021ApJ...913..148T}.

We assume sub-grid turbulence of the ``$\alpha$ turbulence model "\citep{2021ApJ...920L..35T}, in which we assume
\begin{eqnarray}
  \delta v_L&=&\sqrt{\alpha_{\rm turb}} c_s, \\
  \label{tL_eq}
  t_L&=&\frac{c_s}{a_g}, \\
  L&=&\delta v_L t_L.
\end{eqnarray}
$\alpha_{\rm turb}=2 \times 10^{-3}$ denotes the dimensionless parameter that determines the strength
of the sub-grid turbulence and  $a_g$ denotes the gravitational acceleration.
(see \citep{2021ApJ...920L..35T} for the underlying physical assumptions for this turbulence model).


\subsubsection{Resistivity calculations}
For the resistivity model, we adopt the analytical resistivity formula described in \citet{2022ApJ...934...88T} in which we analytically solve the 
equations for chemical equilibrium in the gas phase and
detailed balance equations for dust charging. 
The dust size distribution considered in resistivity calculations is set to be
\begin{eqnarray}
  \label{power_ad}
  \frac{{\rm d} n_{\rm d}}{{\rm d} a} =  A~ a^{-q}  (a_{\rm min}<a<a_{\rm max}),
\end{eqnarray}
where $A$ denotes a constant for normalization.
Here we assume that the maximum dust size is $a_{\rm max}=a_{\rm d}$ of equation (\ref{dadt}) (which is valid when $q<4$).
The minimum dust size $a_{\rm min}$ and power exponet $q$ are the parameters of this study.

\subsubsection{Sink particle}
A sink particle is dynamically introduced when the density exceeds $\rho_{\rm sink}=10^{-12} \gcm$.
The sink particle absorbs SPH particles with $\rho>\rho_{\rm sink}$ within $r_{\rm sink}<1$ AU.

\subsection{Initial conditions}
We adopt the density-enhanced Bonnor-Ebert sphere, which is surrounded by a medium with a steep density profile used in \citet{2021ApJ...920L..35T}  as a initial condition. 

The radius of the core
is $R_c=4.8\times 10^3$ AU and the enclosed mass within $R_c$ is $M_c=1 \msun$.
We adopt an angular velocity  profile of $\Omega(d)=\Omega_0/[\exp[10(d/(1.5 R_c))-1]+1]$
with $d=\sqrt{x^2+y^2}$ and  $\Omega_0=2.3\times 10^{-13} ~{\rm s^{-1}}$.
We assume a constant magnetic field $(B_x,B_y,B_z)=(0,0,83 ~\mu G)$.

The parameter $\alpha_{\rm therm}$ ($\equiv E_{\rm therm}/E_{\rm grav}$) is $0.4$, where
$E_{\rm therm}$ and $E_{\rm grav}$ denote the thermal and gravitational energies of the central core
(without surrounding medium), respectively.
The parameter $\beta_{\rm rot}$ ($\equiv E_{\rm rot}/E_{\rm grav}$) within the core is $0.03$,
where $E_{\rm rot}$ denotes the rotational energy of the core.
The mass-to-flux ratio of the core normalized by the critical value is  $\mu/\mu_{\rm crit}=3$.

We adopt a dust density profile of $\rho_d(r)=f_{dg} \rho_g(r)/[\exp[10(r/(1.5 R_c))-1]+1]$,
where $f_{dg}=10^{-2}$ denotes the dust-to-gas mass ratio.
The dust density profile  has the same shape
with the gas density profile in $r\lesssim 1.5 R_c$
but is truncated at $r \geq 1.5 R_c$. The initial (maximum) dust size is assumed to be $a_d=0.1 \mum$.

We resolve 1 $\msun$ with $3\times 10^6$ SPH particles.
The model names and parameters of the models are listed in Table 1.
