\subsection{Diversity and universality of disk evolution}
As  seen in the previous section, the disk evolution is significantly affected by the dust growth in it.
Furthermore, once the dust grains sufficiently grows,
the disk structure of our fiducial model is well described by the power laws that we analytically derive in supplementary materials.
In this section, we examine the impact of the minimum dust size and the  dust power exponent on the evolution of the disk.
We also examine the effect of the cosmic-ray ionization rate.
Through these considerations, we discuss the diversity and universality of protoplanetary disk evolution.


\subsubsection{Density structure of all models at the end of the simulations}
Figure \ref{2D_density_all_models_end} shows the density map at the end of the simulations.
Although the disk size is different among the simulation, the density structure of inner $\sim 20$ AU region of panels a (ModelA100Q25), b (ModelA100Q35), c (ModelA5Q25), f (ModelZeta18) are very similar.
Actually, in these simulations, the density structures are consistent with the analytical solution.

On the other hand, figure \ref{2D_density_all_models_end} d (ModelA5Q35) shows the formation of a very small disk of $\lesssim 10 {\rm AU}$ and bubble-like structures around it.
This bubble-like structure is created by magnetic interchange instability \citep{2012ApJ...757...77K}.
The large amounts of small dust grains make ambipolar diffusion (and Ohmic diffusion) ineffective in the high density region and leads to the development of interchange instability as a redistribution mechanism of magnetic flux.

The figure \ref{2D_density_all_models_end} e (ModelA100Fixed) in which the dust growth is artificially neglected shows that the disk is relatively compact, dense and massive.
This massive disk is consistent with the previous theoretical studies but such massive disk seems to be inconsistent with the observations \citep{2022arXiv220913765T}.




\subsubsection{Universality of the disk structure}
Figure \ref{rad_prof_all_models} shows the azimuthally averaged radial profiles at the end of the simulations.
The dashed lines show the power laws of our analytical solutions (in this figure, we only show the power laws as a reference, because the parameters such as central star mass or mass accretion rate are differ among the models). 

Of the five models that consider dust growth, four models have an inner $\sim 20 {\rm AU}$ region consistent with the analytical solution (red, green, black, orange lines).
If we regard the disk radius as the radius at which the density distribution deviates from the power law of the analytical solution, the disk size is largest for ModelZeta18, and ModelA100Q35, ModelA100Q25, and ModelA5Q25 have smaller disk sizes, in that order (see also figure \ref{disk_size}).
This difference may be due to the difference in resistivity in the regions where dust grains have not grown (outer region of the disk and envelope). Despite the difference in disk size, the universality of the disk structure in the inner region is noteworthy.

Note also that the disk structure of  ModelZeta18 (green lines) is
more consistent with the power laws of the analytical solutions than the fiducial model. For example, $\eta_A$ clearly has a positive power exponent. This is due to the fact that this model is more evolved than the fiducial model and that the approximation on $\eta_A$ in the analytical solution is better validated.

Figure \ref{rad_prof_all_models} shows that the disk in the ModelA5Q25 (yellow) is very small and has a very different structure from the other four models.
The large amount of small dust in this model makes ambipolar diffusion very ineffective from the beginning of disk formation in the high density region.  The disk shrinks rapidly before the dust grows and before $\eta_A$ is well described by the analytical form of Shu \citep{1983ApJ...273..202S}. This makes the disk evolution of the model very different.

It would be instructive to see the differences in disk structure between the model ignoring the dust growth (ModelA100Fixed; magenta) and those that well described by the analytical solutions.
In the model ignoring dust growth, the density and $\eta_A$ are very large, and the magnetic field and radial velocity $v_r$ are small.
This is because in the absence of dust growth, the ambipolar diffusion in the disk is extremely effective.
It suppresses the magnetic braking in the disk, resulting in a smaller $v_r$ and an increased density due to gas accumulation in the disk.
Furthermore, the magnetic flux is extracted from the disk by the ambipolar diffusion causing the low value of the disk magnetic field.
The density of the disk in ModelA100Fixed is large and Toomre $Q$ parameter is $O(1)$.

The disk density profile  in ModelA100Fixed is approximately $ \rho_g \propto r^{-1}$ meaning that the surface density $\Sigma_g $ is approximately $\Sigma_g \propto r^{1/4}$ (here we simply assume $\rho_g= \Sigma_g/H$ and $H=c_s/\Omega\propto r^{5/4}$).
While this shallow profile is consistent with previous theoretical studies \citep{2017ApJ...835L..11T}, it is inconsistent with the disk observation ($\Sigma_g \propto r^{-0.9}$) \citep{2011ARA&A..49...67W}.
Our analytical solution, on the other hand, has deeper power law of $\rho_g \propto r^{-9/5}$ and is therefore more consistent with the observations.


