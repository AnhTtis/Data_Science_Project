\subsection{Radial disk structure and comparison with analytical solutions}
Figure \ref{1D_time_evolution_q25_a100nm} shows the azimuthally averaged radial profiles at the midplane of ModelA100Q25.
The red, orange, magenta, and green lines show profiles at the epochs of panels 1-4 in figure \ref{2D_time_evolution_q25_a100nm}, respectively.

In the early evolutionary phase when the dust size is sufficiently small ($t_* \lesssim 5\times 10^3$ yr; red and orange lines),
the gas density has a relatively shallow profile with power of  $D_{\rho_g} \sim -1$ (red and orange lines; where $D_f$ denotes the power exponent of a quantity $f$ as $f(r) \propto r^{D_f}$ ).
As the dust grains grow, the gas density profile becomes steeper and appears to converge to a power law with  $D_{\rho_g} \sim -2$ (at $t_*= 9\times 10^3$ yr; green line).


To explain this disk structure,
we analytically derive the new steady state solutions of the disk in Appendix A.
The important assumptions in deriving the steady state solutions are that
(1) the magnetic braking determines the disk angular momentum evolution (and angular momentum transfer by the viscosity is negligible), (2) radial magnetic flux transport is determined by the balance between gas advection and ambipolar diffusion, and (3) the adsorption of charged particles on the dust grains is negligible, and the ionization degree is determined by cosmic ray ionization and gas phase recombination ( for the details of the derivation, see Appendix A).


The derived solution of the density profile predicts $D_{\rho_g} = -\frac{135}{82}$ (equation (\ref{solution_first})). 
The black dotted line and hatched areas indicate the analytical solutions and a region within a factor of {\rm three} of the solution, respectively.
Here, we have chosen the following values for the analytical solution:
$\rho_c=4\times 10^{-14} \gcm$,
$\zeta_{\rm CR}=10^{-17}  {\rm s^{-1}}$,
$c_{s, ref}=190 \ms$,
which are led from the simulation setup and 
$\dot{M}=2\times 10^{-5} \msunyear$ and  $M=0.3 \msun$ to be approximately consistent with the values at $t_* = 9 \times 10^3$ yr.
Our analytical solution well agrees with the simulation result not only in terms of the power, but also in terms of the exact value.

Panel (b) shows that in the early evolutionary phase (red and orange lines),
the magnetic field in the disk is almost constant ($D_{B_z} \sim 0$), which is in agreement with previous studies \citep{2016A&A...587A..32M,2015MNRAS.452..278T} in which the dust growth is ignored.
In contrast, as the dust grains grow, the vertical magnetic field profile becomes steeper and converges to a power law with $D_{B_z} \sim -1$.
Our analytical solution suggests the power exponent of $D_{B_z} = -\frac{177}{164}$ (equation (\ref{solution_second})) and agree with the simulation results at $t_*=9.0 \times 10^3$ yr (green line).
The black dotted line indicates our analytical solution with the same parameters used in the density profile and the hatched area are a region within a factor of  three from the solution.
The black dotted line confirms that our analytical solution simultaneously reproduces the density and magnetic field with a single set of parameters and quantitative agrees with the simulation results.

Panel (c) shows that, when the dust size is small, the absolute value of radial velocity is  $ |v_r|\lesssim 10 \ms$ (red line).
As the dust grows, it increases in the order of $100 \ms$.
The proposed analytical solution suggests a power of $D_{v_r} = -\frac{25}{82}$ and value of $\sim 100 \ms$. Although $v_r$ possesses relatively strong time fluctuation, the simulation results at $t_*=9.0 \times 10^3$ yr (green line) agrees with the analytical solutions.


Panel (d) shows that the $\eta_A$ profile in the simulation is almost radially constant once the dust grains sufficiently grow (magenta and green lines).
On the other hand, our analytical model predicts slightly positive power law with $D_{\eta_A}=\frac{57}{82}$.
Compared to other physical quantities, the difference between the simulation result and the analytical solution is relatively large, but still within a factor of three
in region $r \lesssim 10$ AU.

%Note also that the $\eta_A$ relatively strongly depends both on the gas density ($\rho_g^{-\frac{3}{2}}$) and magnetic field  ($B^2$), and slight differences of the density and the magnetic field between simulations and analytical solutions can lead to an accentuated difference in the $\eta_A$.

