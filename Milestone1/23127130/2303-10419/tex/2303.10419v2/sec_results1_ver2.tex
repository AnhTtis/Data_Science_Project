\subsection{Co-evolution of dust grains and protoplanetary disks}


Figure \ref{2D_time_evolution_q25_a100nm} shows the time evolution of (a) density, (b) total magnetic resistivity ($\eta_O+\eta_A$), (c) dust size, and (d) plasma $\beta$ in our fiducial model, ModelA100Q25.
Panel 1-a shows the formation of a disk with a size of $\sim 20$ AU at $t_*=1.1 \times 10^{3}$ yr (where $t_*$ denotes the time after protostar formation).
At this stage, the total magnetic resistivity  is $\sim 10^{19} \etacm$ and relatively large (panel 1-b).
This large value is attributed to the fact that the maximum dust size remains $a_d<1 \mum$ within the disk (panel 1-c) and dust adsorption of charged particles is effective.
Owing to the efficient magnetic diffusion, the plasma $\beta$ inside the disk is significantly high (panel 1-d).

As the time progresses, a remarkable change in the magnetic resistivity occurs.
Panel 2-b shows a significant decrease in the magnetic resistivity within the disk.
This is caused by the dust growth within the disk ($ a_d$ reaches up to $10 \mum$) and reduction in the dust adsorption efficiency
(see Appendix B and \citet{2022ApJ...934...88T} for the impact of dust size on the ambipolar resistivity).
However, at this epoch, the decrease in the magnetic diffusion efficiency does not lead to a decrease in the disk size; instead, the disk continues to expand. The plasma $\beta$ also remains high.

The decrease in the magnetic resistivity leads to a better coupling between the magnetic field and gas in the disk.
This coupling promotes gas accretion, which in turn transports the magnetic flux to the center and reduces the plasma $\beta$ in the disk (panel 3-d).
The increase of magnetic field amplifies the magnetic resistivity  ($\eta_A$) in the central region (panel 3-b).
Once the dust grows sufficiently, $\eta_A$ is proportional to the square of the magnetic field strength even inside the disk.
Even at this point, the gas density map (panel 3-a) indicates the presence of a relatively large disk of $\sim 50$ AU.

The strong magnetic field in the disk causes efficient magnetic braking and efficient mass accretion over the entire disk, leading to a decrease in the disk size from 3-a to 4-a.
However, the density structure of the central region is very similar in panels 3-a and 4-a.
On the other hand, between panels 2-a and 4-a, the disk size is similar, but the density structure of the inner region is different.
This indicates that the inner disk structure transits from 2-a to 3-a as the dust grains grow.

In this way, the growth of the dust grains causes a decrease in the magnetic resistivity and changes the magnetic activity of the disk,
and ultimately determines the evolution of the disk.
Conversely, changes in the disk structures affect the dust growth in the disk (figure \ref{dust_size_mass}).
Based on these results, we propose a new evolutionary process of the protoplanetary disk: ``co-evolution of dust grains and protoplanetary disks".

