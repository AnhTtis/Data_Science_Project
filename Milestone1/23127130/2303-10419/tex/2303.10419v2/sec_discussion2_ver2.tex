\subsection{Maximum dust size lifted up by the outflow}
Figure \ref{dust_size_mass} as well as our previous study \citep{2021ApJ...920L..35T} shows that the dust-to-gas mass ratio increases when the mean dust size in the disk reaches $a_d\gtrsim 100 \mum$.
This is caused by the selective fall of dust grains from the dust-gas mixture lifted up by the outflow.
The simulations show that, once the dust grows to $\sim 100 \mum$, "ash-fall" phenomenon occurs, in which the dust grains and gas are decoupled in the outflow, and only the dust falls back into the disk.
Hence, the minimum dust size for the protostellar ash-fall is $\sim 100 \mum$.
Then, how large is the maximum dust size that can be lifted by the outflow or the maximum dust size of the "falling ash"?
This is particularly important in explaining the recent observations of the presence of grown dust in the envelope \citep{2009ApJ...696..841K, 2019A&A...632A...5G,2019MNRAS.488.4897V}.

The maximum dust size lifted up by the outflow can be estimated from the following considerations.
For the dust grain to be lifted up by the outflow, the dust grains must couple to the gas at the outflow driving point (or root).
Thus, the stopping time of the dust grains should be less than the orbital period at the root (otherwise, outflow driving causes dust grains to remain in the disk and only the gas is ejected).

To estimate the stopping time, we need the density at the outflow root.
As shown in figure \ref{Fout} and by the observations \citep{2004A&A...426..503W}, the mass ejection rate of the outflow in young protostars (e.g., their age is $t\lesssim 10^5$ yr) is in the range $\dot{M}_{\rm out}=10^{-6}-10^{-5} \msunyear$.
Thus, by assuming that the outflow velocity is comparable to the orbital velocity at the radius of the root \citep{1997ApJ...474..362K},
the density at the root of the outflow $\rho_{\rm dp}$ can be estimated as,
\begin{align}
  \rho_{\rm dp}&=\frac{\dot{M}_{\rm out}}{v_{\rm out} \pi r_{\rm disk}^2}  \\
  &=1.9 \times 10^{-15} \nonumber \\
  &\left(\frac{\dot{M}_{\rm out}}{10^{-5} \msunyear}\right)  \left(\frac{M_{*}}{\rm 0.2 \msun}\right)^{-1/2} \left(\frac{r_{\rm disk}}{50 {\rm AU}}\right)^{-3/2} \gcm, \nonumber
\end{align}
where we assume that $v_{\rm out}=\sqrt{G M_*/r}$ is the Keplerian velocity at disk outer edge $r_{\rm disk}$.
Hence, the stopping time at the root is estimated as 
\begin{align}
  t_{\rm stop}&=  \frac{\rho_{\rm mat} a_d}{\rho_{\rm dp} \sqrt{8/\pi}c_s} \\
  &= 3.3 \times 10^2 \left(\frac{\rho_{\rm mat}}{ 2 \g}\right)\left(\frac{a_{d}}{ 5 \mm}\right) \nonumber \\
  &\left(\frac{M_{*}}{\rm 0.2 \msun}\right)^{1/2} \left(\frac{r_{\rm disk}}{50 {\rm AU}}\right)^{12/7} \left(\frac{\dot{M}_{\rm out}}{10^{-5} \msunyear}\right)^{-1}  {\rm yr}, \nonumber
\end{align}
where the sound velocity and temperature is assumed to be $c_s=190 (T/10 {\rm K})^{1/2} \ms$ and $T=150(r/\au)^{-3/7} {\rm K}$, respectively.
Then, the ratio of the stopping time to the orbital period at $r_{\rm disk}$ is calculated as,
\begin{align}
  \frac{t_{\rm stop}}{t_{\rm orb}}&=0.41 \left(\frac{\rho_{\rm mat}}{ 2 \g}\right) \\
  &\left(\frac{a_{d}}{ 5 \mm}\right) \left(\frac{M_{*}}{\rm 0.2 \msun}\right) \nonumber \\
  &\left(\frac{r_{\rm disk}}{50 {\rm AU}}\right)^{3/14} \left(\frac{\dot{M}_{\rm out}}{10^{-5} \msunyear}\right)^{-1}, \nonumber
\end{align}
or $t_{\rm stop} \sim t_{\rm orb}$ is realized when
\begin{align}
  a_{d} &\sim 1.2 \left(\frac{\rho_{\rm mat}}{ 2 \g}\right)^{-1}\left(\frac{M_{*}}{\rm 0.2 \msun}\right)^{-1} \\
  &\left(\frac{r_{\rm disk}}{50 {\rm AU}}\right)^{-3/14} \left(\frac{\dot{M}_{\rm out}}{10^{-5} \msunyear}\right)\cm. \nonumber
\end{align}
This indicates that the dust size of at maximum $a_d\sim 1 \cm$ can be entrained by the outflow with $\dot{M}_{\rm out} \sim  10^{-5} \msunyear$  from the disk with a size of $\sim 50$ AU.
This size is larger than the wavelength of sub-millimeter observations such as with ALMA and may cause the decrease of the spectral index of dust opacity in the outflow and the envelope.
Thus, ``ash-fall" can explain the presence of grown dust in the envelope suggested by the observations.


%\subsection{Disk magnetic flux and magnetic flux problem}
%Our important finding in this simulation is that the magnetic field profile obeys the power law of $B_z \propto r^{-13/11}$ which is  shallower than $B_z \propto r^{-2}$ even in a
%disk with grown dust. $B_z$ is expected to be shallower when the dust growth is neglected because it causes the decrease of magnetic resistivity.

%The shallower vertical magnetic field profile means that the total magnetic flux is determined by the disk outer radius because of
%\begin{eqnarray}
%\Phi= \int^{r_{\rm max}}_{r_{\rm min}} B_z 2 \pi r dr \sim  \frac{1}{2-q}B_{\rm z, edge} r_{\rm disk}^2,
%\end{eqnarray}
%where $q$ is the power of $B_z$, and $B_{\rm z, edge}$ is $B_z$ at $r_{\rm disk}$.
%This showss that the total magnetic flux is determined by the magnetic field at the outer edge.

%More quantitatively, with our disk profile Appendix (xxx), $\Phi$ is given as
%\begin{eqnarray}
%\Phi= \int^{r_{\rm max}}_{r_{\rm min}} B_z 2 \pi r dr \sim  \frac{1}{2-q}B_{\rm z, edge} r_{\rm disk}^2,
%\end{eqnarray}
%Thus, the mass-to-flux ratio within the  disk radius can be estimated as,
%\begin{eqnarray}
%  \frac{M}{\Phi}= 
%\end{eqnarray}
%This clearly shows that the  magnetic flux 
