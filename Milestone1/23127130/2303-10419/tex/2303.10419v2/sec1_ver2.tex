\label{intro}
%Protoplanetary disks are the birthplaces of planets. Planets come into being as a result of the growth of dust grains in the disk.
%During their growth, like the baby kicks, dust grains can influence the evolution of the protoplanetary disks.

In protoplanetary disks, dust grains are not only the building blocks of the planets, but also play a key role in determining the ionization degree of the disk gas by adsorbing charged particles (ions and electrons) in the gas phase. Since the ionization degree determines the magnetic resistivity, i.e., the degree of coupling between the gas and magnetic field, the microscopic nature ($\mum$ to $\cm$ scale) of the dust grains is expected to influence the macroscopic disk evolution ($100$ AU, i.e., $10^{15} \cm$ scale) via magnetic resistivity \citep{2016MNRAS.460.2050Z,2020ApJ...900..180M,2020A&A...643A..17G,2022ApJ...934...88T}.

Previous studies show that non-ideal magnetohydrodynamics (MHD) effects arising from finite resistivity, specifically ambipolar diffusion, dramatically weaken the coupling between the magnetic field and gas, thereby enabling the formation of a disk \citep{2015ApJ...801..117T, 2015MNRAS.452..278T, 2016MNRAS.457.1037W,2016A&A...587A..32M}.
Moreover ambipolar diffusion determines the magnetic flux evolution in protostars \citep{1998ApJ...497..850L,2020ApJ...896..158T}.

Previous studies on the formation and evolution of protoplanetary disks assumed that dust grains possess the properties (such as the size distribution) of  the interstellar medium (ISM) dust grains. However, in the disk, the dust growth timescale is $\sim 10^4$ years, which is much shorter than the lifetime of the disks ($\sim 10^6$ years). Thus, it is unsatisfactory to study disk evolution with resistivity assuming ISM dust \citep{2022arXiv220913765T}.

How would the dust growth affect the ionization degree? As dust grains merge and grow, their total surface area decreases.
Therefore, the adsorption of charged particles by the dust grains becomes ineffective, and the gas-phase ionization degree is expected to increase and magnetic resistivity to decrease accordingly. Recent studies on dust growth and associated changes in magnetic resistivity have shown a decrease in magnetic resistivity \citep{2016MNRAS.460.2050Z,2020ApJ...900..180M,2020A&A...643A..17G,2022ApJ...934...88T, 2022MNRAS.515.2072K}, and some studies have also shown changes in gas dynamics as a result \citep{2023MNRAS.518.3326L,2023arXiv230101510M}.

However, the effect of the dust growth on the evolution of the protoplanetary disk is still unclear, because the calculations in the aforementioned studies were performed assuming spherical symmetry \citep{2023MNRAS.518.3326L} or 3D simulation until the prestellar or first core formation stage in which the gas is supported by the pressure gradient force \citep{2023arXiv230101510M}.

In this study, we report simulation results of the formation and evolution of protoplanetary disks of $\sim 10^4$ years after the formation of protostars considering dust growth inside the disk, the associated change of magnetic resistivity, and its feedback on the disk dynamics.
Moreover we present an analytical argument that explains the resulting disk structures.
Based on these results, we propose a new evolutionary process for protostars: "co-evolution of dust grains and protoplanetary disks".
