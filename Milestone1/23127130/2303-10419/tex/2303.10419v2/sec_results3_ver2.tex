\subsection{Diversity and universality of disk evolution}
As  seen in the previous section, the disk evolution is significantly affected by the dust growth.
Furthermore, once the dust grains sufficiently grows,
the disk structure of our fiducial model is well described by the power laws analytically derived in Appendix A.
In this section, we examine the impact of the minimum dust size and dust power exponent on the evolution of the disk.
Moreover we examine the effect of the cosmic-ray ionization rate.
Through these considerations, we discuss the diversity and universality of protoplanetary disk evolution.

Before discussing the simulation results, we summarize how the minimum dust size $a_{\rm min}$ and the power exponent $q$ affect the magnetic resistivity based on our previous studies \citep{2022ApJ...934...88T}.
  Figures which show how the resistivities depend on $a_{\rm max}$ with various $q$ and $a_{\rm min}$ can be found in \citet{2022ApJ...934...88T}.



The adsorption of dust grains, which plays the primary role in determining the resistivity depends on the dust total cross-section, which depends on $q$ and $a_{\rm min}$.
In the case where $q$ is less than 3, the maximum dust size determines the total surface area. Conversely, in a case where $q$ is larger than 3, both the maximum and minimum dust sizes affect the total surface area. Consequently, when $q$ is large (or the size distribution is steep), the influence of dust growth tends to be weaker.

Another important factor that influences resistivity is conductivity generated by the dust grains.
If a significant amount of dust with a size of  $\lesssim 10 \nm$ is present, the dust grains contributes to conductivity. In such case, resistivity at a high density
is smaller than that when the minimum size is, for instance, $100 ~\nm$.
This effect is pronounced in a case when the dust grains have not grown and the size distribution is steep (i.e., $q$ is large).

\subsubsection{Comparison of density structures}
Figure \ref{2D_density_all_models_end} shows the density map of all models.
Although the disk size is different among the simulation, the density structure of inner $\sim 20$ AU region of panels a (ModelA100Q25), b (ModelA100Q35), c (ModelA5Q25), and f (ModelZeta18) are very similar.
In these simulations, the inner density structures are consistent with the analytical solutions.
The spiral patterns in the outer regions of the disks are created by gravitational instability. We confirm that Toomre's $Q$ parameter in these regions are $Q \sim 1$ in the outer regions of these disks.




On the other hand, figure \ref{2D_density_all_models_end} d (ModelA5Q35) shows the formation of a very small disk of $\lesssim 10 {\rm AU}$ and bubble-like structures around it.
This bubble-like structure is created by magnetic interchange instability \citep{2012ApJ...757...77K}.
The large amounts of small dust grains make ambipolar diffusion (and Ohmic diffusion) ineffective in the high-density region and leads to the development of interchange instability as a redistribution mechanism of magnetic flux.


Figure \ref{2D_density_all_models_end} e (ModelA100Fixed) in which the dust growth is artificially ignored shows that the disk is relatively compact, dense and massive.
This massive disk is consistent with previous theoretical studies; however such massive disk seems to be inconsistent with the observations \citep{2022arXiv220913765T}.




\subsubsection{Universality of the disk structure}
Figure \ref{rad_prof_all_models} shows the azimuthally averaged radial profiles of the all models.
The dashed lines show the power laws of our analytical solutions (in this figure, we plot the power laws just as a reference because parameters such as central star mass or mass accretion rate differ among the models). 

Of the five models that consider dust growth, four models have an inner region of $\sim 20 {\rm AU}$ consistent with the analytical solution (red, green, black, and orange lines).
If we regard the disk radius as the radius at which the density distribution deviates from the power law of the analytical solution, ModelZeta18 has the largest disk size, and ModelA100Q35, ModelA100Q25, and ModelA5Q25 have smaller disk sizes in this order (see also figure \ref{disk_size}).
This difference may be due to the difference in resistivity in the regions where dust grains have not grown (outer region of the disk and envelope). Despite the difference in disk size, the universality of the disk structure in the inner region is noteworthy.

Note that the disk structure of  ModelZeta18 (green lines) is
more consistent with the power laws of the analytical solutions than the fiducial model. For example, $\eta_A$ has a positive power exponent. This is because this model is more evolved than the fiducial model and approximation on $\eta_A$ in the analytical solution is better validated.

Figure \ref{rad_prof_all_models} shows that the disk of ModelA5Q25 (yellow) is very small and has a different structure from the other four models.
A large amount of small dust in this model makes ambipolar diffusion ineffective from the beginning of disk formation in the high-density region.  The disk rapidly shrinks before the dust grows and $\eta_A$ is well described by the analytical form of Shu \citep{1983ApJ...273..202S}. Thus, the disk evolution of the model is different from other models.

It would be instructive to see the differences in disk structure between the model ignoring the dust growth (ModelA100Fixed; magenta) and those that well described by the analytical solutions.
In the model without the dust growth, the density and $\eta_A$ are large, and the magnetic field and radial velocity $v_r$ are small.
This is because in the absence of the dust growth, the ambipolar diffusion in the disk is extremely effective.
It suppresses magnetic braking in the disk, resulting in smaller $v_r$ and an increased density due to gas accumulation in the disk.
Furthermore, the magnetic flux is extracted from the disk by the ambipolar diffusion causing the low value of the disk magnetic field.
The density of the disk in ModelA100Fixed is large and Toomre's $Q$ parameter is $Q\sim 1$  even in the inner region.

%The disk density profile  in ModelA100Fixed is approximately $ \rho_g \propto r^{-1}$, meaning that the surface density $\Sigma_g $ is approximately $\Sigma_g \propto r^{1/4}$ (here we assume $\rho_g= \Sigma_g/H$ and $H=c_s/\Omega\propto r^{5/4}$).
%While this shallow profile is consistent with previous theoretical studies \citep{2017ApJ...835L..11T}, it is inconsistent with the disk observation ($\Sigma_g \propto r^{-0.9}$)\citep{2011ARA&A..49...67W}.
%Our analytical solution and the simulation results with dust growth, on the other hand, has deeper power law of $\rho_g \propto r^{-9/5}$ and is therefore more consistent with the observations.

