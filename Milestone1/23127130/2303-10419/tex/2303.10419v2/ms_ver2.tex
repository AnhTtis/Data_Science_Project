%\documentclass[usenatbib,preprint,onecolumn,doublespacing]{mn2e}

%\documentclass[usenatbib,preprint,onecolumn]{mn2e}
\documentclass[usenatbib,preprint]{mn2e}

%\documentclass[usenatbib,preprint]{emulateapj}


\usepackage[switch]{lineno}
%\linenumbers
%\usepackage[dvipdfmx]{graphicx}
\usepackage{graphicx}
%%for arxiv
\usepackage {natbib,aas_macros}
\usepackage {mediabb}
\bibliographystyle{apj}
\usepackage {bm}
\usepackage {amsmath,amssymb}
%\usepackage {setspace}

%\doublespacing
%\paperwidth  597pt
%\paperheight 845pt
%\hoffset   -14.0pt
%\voffset    14.5pt
% \oddsidemargin 0.0pt
% \evensidemargin 0.0pt
% \topmargin     0.0pt
% \headheight    0.0pt
% \headsep       0.0pt
%\textheight 671.0pt 
%\textwidth  480.5pt
% \marginparsep   0.0pt
% \marginparwidth 0.0pt
% \footskip       0.0pt
%For Nature letter 1500 words,with method 3500 words 4 item
%For Nature astronmy letter 2000 words, without
%introduction (abstract+introduction 200 words) with method 3500 words 4 item
%For Nature astronmy article 3000-3500 words, with introduction
%with method 3500 words 6-9 item with abstract of 150 words
%For ApJ letter 3500 words, 5 figure

\newcommand{\pasa}{PASA} 

\newcommand{\msun}{\thinspace M_\odot} 
\newcommand{\gcm}{~{\rm g~cm}^{-3} }
\newcommand{\St}{~{\rm St} } 
\newcommand{\Jang}{\mathbf{J_{\rm ang}}}
\newcommand{\rR}{\frac{r}{R_0}}

\newcommand{\magB}{\mathbf{B}}
\newcommand{\eleE}{\mathbf{E}}
\newcommand{\cul}{\mathbf{J}}
\newcommand{\vel}{\mathbf{v}}
\newcommand{\rad}{\mathbf{r}}
\newcommand{\etacm}{~{\rm cm}^{2} ~{\rm s}^{-1} } 
\newcommand{\cms}{~{\rm cm} ~{\rm s}^{-1} } 
\newcommand{\cmcms}{~{\rm cm}^2 ~{\rm s}^{-1} } 
\newcommand{\ms}{~{\rm m} ~{\rm s}^{-1} } 
\newcommand{\kms}{~{\rm km} ~{\rm s}^{-1} }
\newcommand{\gcms}{~{\rm g~ cm} ~{\rm s}^{-1} }

\newcommand{\mum}{{\rm \mu} {\rm m} }
\newcommand{\mm}{{\rm mm}}
\newcommand{\nm}{{\rm nm}}
\newcommand{\cm}{{\rm cm}}
\newcommand{\au}{{\rm AU}}
\newcommand{\g}{{\rm ~g}}


\newcommand{\msunyear}{\thinspace M_\odot~{\rm yr}^{-1}}
\newcommand{\acc}{\mathbf{a_0}}
\newcommand{\dt}{\Delta t}
\newcommand{\dv}{\Delta \vel}
\newcommand{\expm}{{\rm expm1}}
\newcommand{\tstop}{t_{\rm stop}}
\newcommand{\tgrowth}{t_{\rm growth}}
\newcommand{\fgp}{\mathbf{f}_{g, p}}
\newcommand{\fgem}{\mathbf{f}_{g, em} }
\newcommand{\fdem}{\mathbf{f}_{d, em}}
\newcommand{\fge}{\mathbf{f}_{g, e} }
\newcommand{\fgm}{\mathbf{f}_{g, m}}

%\newcommand{\fd}{\mathbf{f}_d}

\title{Co-evolution of dust grains and protoplanetary disks}

\author[Tsukamoto et al]{
Yusuke Tsukamoto$^{1}$, Masahiro N. Machida$^{2}$, and  Shu-ichiro Inutsuka$^{3}$ \\
$^1$Graduate Schools of Science and Engineering, Kagoshima University, Kagoshima, Japan  \\
$^2$Department of Earth and Planetary Sciences, Kyushu University, Fukuoka, Japan \\
$^3$Department of Physics, Nagoya University, Aichi, Japan  \\
}

\begin{document}
\maketitle

\begin{abstract}
\input abstract_ver3.tex
\end{abstract}

\begin{keywords}
star formation -- circum-stellar disk -- methods: magnetohydrodynamics -- smoothed particle hydrodynamics -- protoplanetary disk
\end{keywords}


\section{Introduction}
\input sec1_ver2.tex

\section{Methods and initial condition}
\input sec_method_ver2.tex
\input table1_ver2.tex

\section{Results}
\input sec_results1_ver2.tex
\input fig_results1_ver2.tex
\input sec_results2_ver2.tex

\input sec_results3_ver2.tex
\input fig_results3_ver2.tex
\input sec_resultsS4_ver2.tex
\input fig_resultsS4_ver2.tex
%Appendix B
%It would be useful to compare the above results with simulation results in which dust growth does not occur.
%Figure 2 shows the time evolution of the density, magnetic resistivity, dust size, and plasma $\beta$ of model 2 in which dust size is fixed to be $a_d=0.1 \mum$,
%the initial dust size of model 1.

%The disk structures in panels 1-a to 1-d of show that the structures are very similar to those shown in panels 1-a to 1-d,
%because there is dust growth and associated change in magnetic resistivity does not play the role in figure 1. 
%Note, nevertheless, that the magnetic resistivity is larger in the center of model 2 on closer inspection, becuase of slight increase of dust size as shown 1-c of figure 1.

%In the epoch of panel 2, the difference in the magnetic resistivity of the disks between Fig. 1 and Fig. 2 is striking.
%By fixing the dust size, the magnetic resistivity in the disk is kept high.
%As a result, the magnetic field and gas are not strongly coupled inside the disk.
%Thereby, even in the epoch of panel 3, the decrease of plasma $\beta$ inside the disk is not seen in this model (panel 3-d).

%Notable changes have occurred from panels 3 to panels 4. When the dust size is fixed at submicron size,
%ambipolar diffusion becomes rather inefficient (lower ambipolar resistivity) at a density of $\sim 10^{-14} \gcm$
%(see, for example, figure 1 in Tsukamoto+20 ).
%Therefore, a ring-like structure with a lower magnetic resistivity is formed from the outer edge of the disk to the envelope (panel 3-b).
%Certain instabilities in this region cause the formation of a warped envelope in the z-direction (panel 4-c).
%This kind of warp structures has been observed in a several previous studies \citep{2013ApJ...763....6T,2016MNRAS.457.1037W,2018MNRAS.473.4868Z, Tsukamoto+20}.
%This warp forms a compact and dense disk (panel 4-a). Note, however, that even in such a compact disk,
%the internal magnetic resistivity and plasma $\beta$ remains high (panel 4-b and 4-d).

%\begin{figure*}
%  \includegraphics[clip,trim=0mm 0mm 0mm 0mm,width=150mm]{fig2.png}
%\caption{
%}
%\label{2D_time_evolution_fixed}
%\end{figure*}
%%%%%%%%%%%%%%%%%%%%%%%%%%%%%%%%%%%Appendix B




\section{Discussion}
\input sec_discussion1_ver2.tex

\input sec_discussion2_ver2.tex


\section*{Acknowledgments}
We thank Dr. Shinsuke Takasao and Mr. Ryoya Yamamoto for the fruitful discussion.
%We also thank anonymous referee for helpful comments.
The computations were performed on the Cray XC50 system at CfCA of NAOJ.
This work is supported by JSPS KAKENHI  grant number 18H05437, 18K13581, 18K03703.

%\section*{Author contribution}
%Y.T led the project and conducted the simulations.
%All authors discussed the results and commented on the manuscript.


\appendix
\input sec_appendixA_ver2.tex

\bibliography{article}


\end{document}
