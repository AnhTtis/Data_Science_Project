\section{Analytic solution of steady state disks with magnetic braking and ambipolar diffusion}
In this section, we derive the power laws of steady-state circumstellar disks which is determined by the angular momentum removal by magnetic braking and magnetic field structure determined by the balance between gas advection and ambipolar diffusion.
We start from MHD equation with ambipolar diffusion,
\begin{eqnarray}
  \frac{\partial \rho}{\partial t}+\nabla \cdot (\rho \vel)&=&0,\\
  \rho \left( \frac{\partial \vel}{\partial t} +\vel \cdot \nabla \vel \right)&=&-\rho \nabla \Phi -\nabla p +\frac{1}{4 \pi} (\nabla \times \magB)\times \magB, \nonumber \\ \\
  \frac{\partial \magB}{\partial t} &=& \nabla \times \left(\vel \times \magB \right) \\
  &-& \nabla \times \left( \frac{\eta_A}{|\magB|^2} ((\nabla \times \magB)\times \magB) \times \magB \right), \nonumber \\
  \nabla \cdot \magB&=&0.
\end{eqnarray}
In this appendix,  $\rho$ denotes the gas density,
$\vel$ denotes the gas velocity,
$p$ denotes the gas pressure,
$\Phi$ denotes the gravitational potential,
$\magB$ denotes the magnetic field.
Hereafter, we assume the steady state i.e., $\frac{\partial}{\partial t}=0$.

Because we focus on the structure of the circumstellar disk,
we make the assumptions below following \citet{2012MNRAS.424.2097G,2013MNRAS.430..822G}.
We introduce a small dimensionless parameter $\epsilon=O(H/r)$, where $H$ denotes the gas scale height of the disk.
When the disk self-gravity is negligible, the gravitational potential is expanded as
\begin{eqnarray}
  \Phi(r,z)=\Phi_0(r)+\frac{1}{2}\Phi_2(r) z^2+O\left(\left(\frac{z}{r}\right)^4\right) \nonumber,\\
  \therefore   \Phi(r,\zeta)=\Phi_0(r)+\epsilon^2 \frac{1}{2}\Phi_2(r)\zeta^2+O(\epsilon^4),
\end{eqnarray}
where we introduces the rescaled vertical coordinate $\zeta=\epsilon^{-1} z$, and
$\Phi_0(r)=-(GM/r)$ and $\Phi_2(r)=(GM/r^3)$.
This dependence of the gravitational potential on $\epsilon$
is the guiding principle that determines the order of other physical quantities.

We assume that the vertical gravity is balanced by thermal pressure at the leading order.
Thus, the scaling of pressure can be assumed to be
\begin{eqnarray}
  p(r,\zeta)=\epsilon^2 p_2(r,\zeta)+\epsilon^4 p_4(r,\zeta)+O(\epsilon^6).
\end{eqnarray}

We assume the scaling of the density to be
\begin{eqnarray}
  \rho(r,\zeta)=\rho_0(r,\zeta)+\epsilon^2 \rho_2(r,\zeta)+O(\epsilon^4).
\end{eqnarray}
Then, the sound velocity $c_s$ scales as
\begin{eqnarray}
  c_s(r)=\epsilon c_{s,1}(r)+\epsilon^3 c_{s,3}(r)+O(\epsilon^5).\\
\end{eqnarray}
Here, we assume the disk to be vertically isothermal.

The leading order of the radial velocity is assumed to be $\epsilon c_s$, and thus,
\begin{eqnarray}
  v_r(r,\zeta)=\epsilon^2 v_{r,2}(r,\zeta)+\epsilon^4 v_{r,4}(r,\zeta)+O(\epsilon^6).
\end{eqnarray}

The leading order of the azimuthal velocity is assume to be balanced with the leading order of $\Phi(r,\zeta)$ and hence,
\begin{eqnarray}
  v_\phi(r,\zeta)=r \Omega_0(r)+ \epsilon^2 v_{\phi,2}(r,\zeta)+\epsilon^4 v_{\phi,4}(r,\zeta)+O(\epsilon^6).
\end{eqnarray}

The leading order of the vertical velocity is assumed to be smaller than $v_r$,
\begin{eqnarray}
  v_z(r,\zeta)=\epsilon^3 v_{z,3}(r,\zeta)+\epsilon^5 v_{r,5}(r,\zeta)+O(\epsilon^7).
\end{eqnarray}

For the magnetic field, we assume that the vertical magnetic field
is the dominant component and the scaling of magnetic field is assumed to be 
\begin{eqnarray}
  B_r(r,\zeta)&=&\epsilon^2 B_{r,2}(r,\zeta)+\epsilon^4 B_{r,4}(r,\zeta)+O(\epsilon^6),\\
  B_\phi(r,\zeta)&=&\epsilon^2 B_{\phi,2}(r,\zeta)+\epsilon^4 B_{\phi,4}(r,\zeta)+O(\epsilon^6),\\
  B_z(r,\zeta)&=&\epsilon B_{z,1}(r)+\epsilon^3 B_{z,3}(r,\zeta)+O(\epsilon^5),
\end{eqnarray}
where the leading term of $B_z(r,\zeta)$ does not depend on $\zeta$ because the leading order of the divergence free condition gives,
\begin{eqnarray}
 \partial_\zeta B_{z,1}=0.
\end{eqnarray}
The underlying assumption that leads to this ordering is that the leading order of Alfven velocity ($\propto B_z/\sqrt{\rho_g}=O(\epsilon)$) is the order of the sound velocity.


  Once the dust grains have grown sufficiently and the adsorption of charged particles by grains becomes negligible, the ionization degree is determined by the balance between cosmic-ray ionization and gas-phase recombination. In this case, $\eta_A$ is given as
\begin{align}
  \eta_A&=\frac{\magB^2}{4 \pi C \gamma \rho^{3/2}} \nonumber \\
  &=\epsilon^2 \eta_{A,2} + O(\epsilon^4).
\end{align}
Here $C$ is given as 
\begin{align}
C=\sqrt{\frac{m_i^2 \zeta_{\rm CR}}{m_g \beta_r}},
\end{align}
where $m_i$ and $m_g$ are the mass of ion and neutral particles and we assume $m_i=29 m_p$ and $m_g=2.34 m_p$ assuming that the major ion is HCO$^+$ where $m_p$ is the proton mass. $\zeta_{\rm CR}$ is the cosmic ray ionization rate. $\beta_r$ is the recombination rate and assumed to be
\begin{align}
 \label{beta_UMIST}
\beta_r=\beta_{r,0} \left(\frac{T}{300 {\rm K}}\right)^{-0.69} \sim \beta_{r, 0} \left(\frac{T}{300 {\rm K}}\right)^{-7/10},
\end{align}
where $\beta_{r,0}=2.4 \times 10^{-7} {\bf cm^3 s^{-1}}$ taken from UMIST database \citep{2013A&A...550A..36M}.
\begin{align}
\gamma=\frac{\langle \sigma v \rangle_{in}}{(m_g+m_i)},
\end{align}
where $\langle \sigma v \rangle_{in}$ is the rate coefficient for collisional momentum transfer between ions and neutrals.
We assume  $\langle \sigma v \rangle_{in} = 1.3 \times 10^{-9} {\rm cm^3 s^{-1}}$ which is calculated from the Langevin rate \citep{2008A&A...484...17P}.
Hence $\eta_A$ has the weak temperature dependence of approximately $\propto T^{-\frac{7}{20}}$.


The leading order of $\eta_A$ is $\epsilon^2$ because of the leading order of $B_z(r,\zeta)$ and $\rho(r,\zeta)$ are  $O(\epsilon)$ and  $O(1)$, respectively.

The radial components of the equation of motion at the leading order,
\begin{eqnarray}
  -\rho_0 r \Omega_0^2=-\rho_0 \partial_r \Phi_0,
\end{eqnarray}
gives
\begin{eqnarray}
  \Omega_0=\sqrt{\frac{G M}{r^3}}.
\end{eqnarray}

The vertical components of equation of motion at the leading order
\begin{eqnarray}
\rho_0 \Phi_2 \zeta =-\partial_\zeta p_2,
\end{eqnarray}
leads
\begin{eqnarray}
\rho_0=\frac{\Sigma_0}{\sqrt{2 \pi} H_1}\exp\left(-\frac{\zeta^2}{2H_1^2}\right)\equiv \tilde{\rho}\exp\left(-\frac{\zeta^2}{2H_1^2}\right),
\end{eqnarray}
where $H_1(r)=c_{s,1}/\Phi_2^{1/2}=c_{s,1}/\Omega_0$, where $c_{s,1}$ is the vertically
isothermal sound velocity defined by $p_2=c_{s,1}^2\rho_0$ and $H_1(r)$ is related to the scale height as $H= \epsilon H_1+O(\epsilon^3)$.
$\Sigma_0$ is given as $\Sigma_0=\int_{-\infty}^\infty \rho_0 d\zeta$.

The second order of the radial and azimuthal components 
of the equation of motion and of the induction equation are written as
\begin{eqnarray}
  \label{eqs_second_order_first}
  -2\rho_0 \Omega_0 v_{\phi,2}&=&-\frac{1}{2} \rho_0 \partial_r\Phi_2 \zeta^2 \nonumber \\
  &-&\partial_r\left(p_2+\frac{B_{z,1}^2}{8\pi}\right)+\frac{B_{z,1}}{4 \pi}\partial_\zeta B_{r,2}, \\
  \rho_0 v_{r,2} \frac{1}{r} \partial_r(r^2 \Omega_0) &=&\frac{B_{z,1}}{4 \pi}\partial_\zeta B_{\phi,2},\\
  0=B_{z,1}\partial_\zeta v_{r,2} &+&\partial_\zeta[\eta_{A,2} (\partial_\zeta B_{r,2}-(\partial_r B_{z,1})]. \\
  \label{eqs_second_order_last}
  0=B_{r,2} r \partial_r \Omega_0  &+& B_{z,1} \partial_\zeta v_{\phi,2}+\partial_\zeta (\eta_{A,2} \partial_\zeta B_{\phi,2})
\end{eqnarray}
These equations correspond to equation (28) to (31) of
\citet{2012MNRAS.424.2097G}, if we assume viscous parameter $\alpha$ to be $0$.

Then, we rescale the vertical coordinate with
\begin{eqnarray}
\hat{z} \equiv \frac{\zeta}{H_1}=\frac{z}{H}.
\end{eqnarray}

To evaluate the single power law for each physical quantity,
we need to further simplify the equations (\ref{eqs_second_order_first}) to (\ref{eqs_second_order_last}).
Here, we assume that the radial thermal pressure gradient is much larger
than the magnetic pressure gradient, and the terms with $B_{r,2}$ can be neglected.
%Neglecting $B_{r,2}$ does not necessarily violate the divergence free condition, because its second order relation is
%\begin{eqnarray}
%\frac{1}{r}\partial_r (r B_{r,2})+\partial_\zeta B_{z,3}=0,
%\end{eqnarray}
%and $B_{r,2}$ is related to $B_{z,3}$, and has the freedom to have the finite value.

Then equations (\ref{eqs_second_order_first}) to (\ref{eqs_second_order_last}) are rewritten as
\begin{eqnarray}
  \label{eqs_second_order_simplified_first}
  -2\rho_0 \Omega_0 v_{\phi,2}&=&\frac{3}{2 r} \rho_0 \Omega_0^2 H_1^2 \hat{z}^2-\partial_r(\rho_0 c_{s,1}^2),\\
  \frac{1}{2} \rho_0 v_{r,2} \Omega_0 &=&\frac{B_{z,1}}{4 \pi H_1}\partial_{\hat{z}} B_{\phi,2},\\
  B_{z,1}\partial_{\hat{z}} v_{r,2} &=& (\partial_{\hat{z}} \eta_{A,2}) (\partial_r B_{z,1}), \\
  B_{z,1} \partial_{\hat{z}} v_{\phi,2}&=&\frac{1}{H_1} \partial_{\hat{z}} (\eta_{A,2} \partial_{\hat{z}} B_{\phi,2}).
  \label{eqs_second_order_simplified_last}
\end{eqnarray}

Furthermore, from the conservation of the mass, we have
\begin{eqnarray}
  \label{eqs_mass_accretion}
-2\pi r  \int^{H}_{-H} \rho v_r d z  = \dot{M} \equiv \epsilon^3 \dot{M}_{3} +O(\epsilon^5).
\end{eqnarray}
where we approximated the integral range from $-H$ to $H$ instead of from $-\infty$ to $\infty$
because of the vertical expansion below.
$\dot{M}$ is the mass accretion rate within the disk which is assumed to be constant and
$\dot{M}_{3}$ is its leading term.

Using $\hat{z}$, we consider the vertical expansion.
Because the azimuthal component of the magnetic field should be odd with respect to the midplane,
we assume a vertical dependence of the magnetic field as
\begin{eqnarray}
  B_{\phi,2}=B_{\phi,\hat{z}1} \hat{z} +B_{\phi,\hat{z}3} \hat{z}^3  +O(\hat{z}^5).
\end{eqnarray}
Note that $B_{r,2}$ has disappeared from the equations and cannot be determined from our assumptions above.

The radial and azimuthal components of the velocity should be even with respect to the midplane.
Thus, we assume the vertical dependence of the velocity as
\begin{eqnarray}
  v_{r,2}=v_{r,\hat{z}0} +v_{r,\hat{z}2} \hat{z}^2 + O(\hat{z}^4),\\
  v_{\phi,2}=v_{\phi,\hat{z}0} +v_{\phi,\hat{z}2} \hat{z}^2  +O(\hat{z}^4).
\end{eqnarray}

Then we assume the power law for the vertical magnetic field $B_z$ and midplane density $\rho_{\rm mid}\equiv \rho(r,z=0)$
with respect to the radius as 
\begin{eqnarray}
  B_z(r)=B_{\rm z, ref}\left(\frac{r}{r_{\rm ref}}\right)^{D_{B_z}},\\
  \rho_{\rm mid}(r)=\rho_{\rm mid, ref}\left(\frac{r}{r_{\rm ref}}\right)^{D_{\rho_g}}.
\end{eqnarray}

To be consistent with our simulations, we assume the polytropic relation for the sound velocity
\begin{eqnarray}
  \label{cs_polytropic}
  c_s(r)=c_{\rm s,ref} \left(\frac{\rho_{\rm mid}(r)}{\rho_c}\right)^{1/3},
\end{eqnarray}
     and temperature
\begin{eqnarray}
\label{Tgas_polytropic}
  T(r)=T_{\rm ref} \left(\frac{\rho_{\rm mid}(r)}{\rho_c}\right)^{2/3},
\end{eqnarray}
where we assume $T_{\rm ref}=10$ K. Note that although we assume the temperature distribution in this Appendix, 
the model described here can be applied to different temperature distribution.

From the analytical form, $\eta_A$ is vertically expanded as 
\begin{eqnarray}
  \eta_{A,2}&=&\frac{B_{z,1}(r)^2}{C \gamma \rho_0(r,\hat{z})^{3/2}} =B_{z,1}^2 \tilde{\rho}^{-(3/2+7/30)} \exp((\frac{3}{4}+\frac{7}{60})\hat{z}^2) \nonumber \\
  &=& (C \gamma)^{-1} B_{z,1}^2 \tilde{\rho}^{-3/2}\left(1+(\frac{3}{4}+\frac{7}{60})\hat{z}^2+O(\hat{z}^4)\right),
\end{eqnarray}
where the factor $\frac{7}{60}$ comes from the temperature dependence of the recombination rate.


By substituting these power laws and taking the leading terms with respect to the $\epsilon$ of equation (\ref{eqs_mass_accretion}),
we obtain the solutions of the equations (\ref{eqs_second_order_simplified_first}) to (\ref{eqs_mass_accretion}),

\begin{eqnarray}
  v_{\phi,\hat{z}2}&=&\frac{82}{150} v_{\phi,\hat{z}0},\\
  v_{r,\hat{z}2}&=&\frac{1}{2} v_{r,\hat{z}0},\\
  B_{\phi,\hat{z}3}&=&0,
\end{eqnarray}
and, 
\begin{align}
 \label{solution_first}
 \rho_{\rm mid}(r)&=5.1 \times 10^{-1}\dot{M}^{\frac{15}{41}} \rho_c^{\frac{37}{82}} \Omega_0^{\frac{45}{41}} m_g^{\frac{15}{82}} m_i^{-\frac{15}{41}}\nonumber \\
 &\zeta_{\rm CR}^{-\frac{15}{82}} \beta_{r,0}^{-\frac{35}{164}} \gamma^{-\frac{15}{41}} c_{\rm s,ref}^{-\frac{45}{41}}  \left(\frac{r}{r_{\rm ref}}\right)^{-\frac{135}{82}},\\
   \label{solution_second}
  B_z(r)&= 3.6 \times 10^{-1} \dot{M}^{\frac{47}{82}} \rho_c^{\frac{23}{164}} \Omega_0^{\frac{59}{82}} m_g^{-\frac{35}{164}} m_i^{\frac{35}{82}} \nonumber \\
  &\zeta_{\rm CR}^{-\frac{35}{164}}  \beta_{r,0}^{-\frac{35}{164}} \gamma^{\frac{35}{82}} c_{\rm s,ref}^{-\frac{59}{82}} \left(\frac{r}{r_{\rm ref}}\right)^{-\frac{177}{164}},\\
  v_\phi(r,z=0)-v_K&=-8.8 \times 10^{-1} \dot{M}^{\frac{10}{41}} \rho_c^{-\frac{15}{41}} \Omega_0^{\frac{30}{41}} m_g^{\frac{5}{41}} m_i^{-\frac{10}{41}} \nonumber \\
  &\zeta_{\rm CR}^{-\frac{5}{41}}  \beta_{r,0}^{\frac{5}{41}} \gamma^{-\frac{10}{41}} c_{\rm s,ref}^{\frac{11}{41}} \left(\frac{H_{\rm ref}}{r_{\rm ref}}\right) \left(\frac{r}{r_{\rm ref}}\right)^{-\frac{49}{82}},\\
  v_{r}(r,z=0)&=- 2.1 \times 10^{-1} \dot{M}^{\frac{21}{41}} \rho_c^{-\frac{11}{41}} \Omega_0^{\frac{22}{41}} m_g^{-\frac{10}{41}} m_i^{-\frac{20}{41}} \nonumber \\
  &\zeta_{\rm CR}^{\frac{10}{41}}  \beta_{r,0}^{-\frac{10}{41}} \gamma^{\frac{20}{41}} c_{\rm s,ref}^{-\frac{22}{41}} \left(\frac{H_{\rm ref}}{r_{\rm ref}}\right) \left(\frac{r}{r_{\rm ref}}\right)^{-\frac{25}{82}},\\
  \label{solution_last}
  B_{\phi}(r,z)&= -1.5 \times 10^{-1} \dot{M}^{\frac{35}{82}} \rho_c^{-\frac{23}{164}} \Omega_0^{\frac{105}{82}} m_g^{\frac{35}{164}} m_i^{-\frac{35}{82}} \nonumber \\
  &\zeta_{\rm CR}^{-\frac{35}{164}}  \beta_{r,0}^{\frac{35}{164}} \gamma^{-\frac{35}{82}} c_{\rm s,ref}^{-\frac{23}{82}} \left(\frac{H_{\rm ref}}{r_{\rm ref}}\right) \left(\frac{z}{H}\right) \left(\frac{r}{r_{\rm ref}}\right)^{-\frac{233}{164}},
\end{align}
where $H_{\rm ref}\equiv c_{\rm s, ref}/\Omega(r_{\rm ref})$.

The following estimates are obtained by substituting numerical values for the parameters.
\begin{align}
  \rho_{\rm mid}(r)&=1.7\times 10^{-12} \nonumber \\
  &\left(\frac{\dot{M}}{10^{-5} \msun {\rm yr}^{-1}}\right)^{\frac{15}{41}} \left(\frac{\rho_c}{10^{-13} \gcm}\right)^{\frac{37}{82}} \nonumber \\
  &\left(\frac{M_{\rm star}}{0.2 \msun}\right)^{\frac{45}{82}} \left(\frac{\zeta_{\rm CR}}{10^{-17} s^{-1}}\right)^{-\frac{15}{82}} \nonumber \\
  &\left(\frac{c_{\rm s,ref}}{190 \ms}\right)^{-\frac{45}{41}} \left(\frac{r}{10 {\rm AU}}\right)^{-\frac{135}{82}} \gcm,\\
  B_z(r)&= 2.5\times 10^{-1} \left(\frac{\dot{M}}{10^{-5} \msun {\rm yr}^{-1}}\right)^{\frac{47}{82}} \nonumber\\
  &\left(\frac{\rho_c}{10^{-13} \gcm}\right)^{\frac{23}{164}} \left(\frac{M_{\rm star}}{0.2 \msun}\right)^{\frac{59}{164}} \left(\frac{\zeta_{\rm CR}}{10^{-17} s^{-1}}\right)^{\frac{35}{164}} \nonumber \\
  &\left(\frac{c_{\rm s,ref}}{190 \ms}\right)^{-\frac{59}{82}} \left(\frac{r}{10 {\rm AU}}\right)^{-\frac{177}{164}} {\rm G},\\
  v_\phi (r,z)  &= v_K -80 \left(1+\frac{82}{150} \left(\frac{z}{H}\right)^2\right)\left(\frac{\dot{M}}{10^{-5} \msun {\rm yr}^{-1}}\right)^{\frac{10}{41}} \nonumber \\
  &\left(\frac{\rho_c}{10^{-13} \gcm}\right)^{-\frac{15}{41}} \left(\frac{M_{\rm star}}{0.2 \msun}\right)^{\frac{15}{41}} \left(\frac{\zeta_{\rm CR}}{10^{-17} s^{-1}}\right)^{-\frac{5}{41}} \nonumber \\
  &\left(\frac{c_{\rm s,ref}}{190 \ms}\right)^{\frac{11}{41}} \left(\frac{r}{10 {\rm AU}}\right)^{-\frac{49}{82}} \ms,\\
  v_{r}(r,z)&= -110 \left(1+\frac{1}{2} \left(\frac{z}{H}\right)^2\right)\left(\frac{\dot{M}}{10^{-5} \msun {\rm yr}^{-1}}\right)^{\frac{21}{42}} \nonumber \\
  &\left(\frac{\rho_c}{10^{-13} \gcm}\right)^{-\frac{11}{41}} \left(\frac{M_{\rm star}}{0.2 \msun}\right)^{\frac{11}{41}} \left(\frac{\zeta_{\rm CR}}{10^{-17} s^{-1}}\right)^{\frac{22}{41}} \nonumber \\
  &\left(\frac{c_{\rm s,ref}}{190 \ms}\right)^{-\frac{22}{41}}  \left(\frac{r}{10 {\rm AU}}\right)^{-\frac{25}{82}} \ms,\\
  B_{\phi}(r,z)&= -2.4\times 10^{-2}  \left(\frac{z}{H}\right) \left(\frac{\dot{M}}{10^{-5} \msun {\rm yr}^{-1}}\right)^{\frac{35}{82}} \nonumber \\
  &\left(\frac{\rho_c}{10^{-13} \gcm}\right)^{-\frac{23}{164}} \left(\frac{M_{\rm star}}{0.2 \msun}\right)^{\frac{105}{164}} \left(\frac{\zeta_{\rm CR}}{10^{-17} s^{-1}}\right)^{-\frac{35}{164}} \nonumber \\
  &\left(\frac{c_{\rm s,ref}}{190 \ms}\right)^{-\frac{23}{82}} \left(\frac{r}{10 {\rm AU}}\right)^{-\frac{233}{164}} {\rm G}.
\end{align}
These equations reproduce our simulation results very well.

So far, we have derived the solution from the basic equations with the assumptions explicitly stated,
which, however, may be intuitively difficult to understand.
The aforementioned power laws can be derived from a simple and intuitive extension of the viscous accretion disk model.
Besides the numerical factors, the analytic solutions can also be derived by the following equations:
\begin{eqnarray}
  \label{a7}
  -2 \pi r v_r \Sigma =\dot{M},\\
  \label{a8}
  v_\phi =\sqrt{\frac{G M_c}{r}}\equiv r \Omega,\\
  \label{a9}
  H=\frac{c_s}{\Omega},\\
  \label{a10}
   v_r =-\frac{B_z B_{\phi, s}}{\pi \Sigma \Omega},\\
  \label{a11}
  B_z v_r=-\frac{\eta_A}{r} B_z,\\
  \label{a12}
  B_{\phi, s} =\left( \frac{H}{r}\right)^2 \frac{B_z H}{\eta_A} v_\phi.
\end{eqnarray}
The equations (\ref{a7}) to (\ref{a10}) have a similar form
of the standard viscous accretion disk model \citep{1973A&A....24..337S,1974MNRAS.168..603L},
in which equation (\ref{a10}) is
\begin{eqnarray}
v_r= -\frac{3}{2} \frac{\alpha c_s H}{r},
\end{eqnarray}
where $\alpha$ is the Shakura-Sunyaev viscous parameter.

In our model, equations  (\ref{a11}) and  (\ref{a12}) describe the radial and azimuthal balance
between the magnetic field advection by the gas motion and magnetic field drift by ambipolar diffusion.
Note that the right hand side of equation (\ref{a12}) has the factor of $(H/r)^2$, reflecting that the balance between the vertical shear motion (not  Keplerian rotation itself) and magnetic field drift due to ambipolar diffusion determines the toroidal magnetic field $B_{\phi}$.
This is the key to deriving the disk structure we have identified.

Instead of equation (\ref{a12}), it is often assumed that
\begin{eqnarray}
  \label{a12_prev}
  B_{\phi, s} = \frac{B_z H}{\eta_A} v_\phi,
\end{eqnarray}
which corresponds to the assumption that the Keplerian rotation balances the magnetic field drift by ambipolar diffusion \citep{2002ApJ...580..987K,2012MNRAS.422..261B,2016ApJ...830L...8H}.
We found that this relation leads to considerably large $v_r$ (of the order of the Keplerian velocity), which is inconsistent with the simulation results.
The reason why this relation is inappropriate for the circumstellar disk is that
at the leading order of $\epsilon$,  the gravity does not depend on $z$ and is canceled out by the centrifugal force in the circumstellar disk.
Thus, we should consider the balance between rotation and field drift at the order of $\epsilon^2$ to estimate the toroidal magnetic field in the circumstellar disk (see the derivations above for details  and see also equation (27) of \citet{2021MNRAS.508.2142X}). 

The solutions given by equations (\ref{solution_first}) to (\ref{solution_last}) well reproduce our three dimensional simulations.
Furthermore, they are specified only by the central star mass, mass accretion rate, equation of state (or gas temperature), and ionization and recombination rate,
and {\it do not} contain the viscous $\alpha$ parameter, which is usually extremely difficult to determine.
Therefore, we believe that the analytical solutions are useful for investigating the longer-term evolution of circumstellar disks than we have investigated in this paper by the simulations.


\section{dust size dependence of ambipolar resistivity}
It would be insightful to see how  $\eta_A$ changes depeding on the maximum dust size with a simple one zone model.
Figure \ref{etaA_amax} shows the $\eta_A$ dependence on the maximum dust size with the parameter adopted in ModelA100Q25 (the fiducial model).

Here, we assume that the temperature
is given as $T=10(1+\gamma (\rho_{\rm g}/\rho_c)^{(\gamma-1)}) ~{\rm K}$, where $\gamma=5/3$ and $\rho_c=4\times 10^{-13} \gcm$.
The magnetic field is given as $0.2 n_{\rm g}^{1/2} \mu G$  \citep[i.e., assuming flux freezing; see e.g.,][]{2002ApJ...573..199N}.  $\zeta_{\rm CR}=10^{-17} {\rm s^{-1}}$.
With these assumptions of the temperature and magnetic field, and once the adsorption of charged particles by grains becomes negligible, $\eta_A$ obeys
\begin{eqnarray}
\eta_A =\frac{B^2}{4 \pi C \gamma \rho^{\frac{3}{2}}} &\propto&
  \begin{cases}
    \rho^{-\frac{1}{2}} & (\rho<\rho_c)\\
    \rho^{-\frac{22}{30}} & (\rho>\rho_c),
  \end{cases}
\end{eqnarray}
where the temperature dependence of $C$ is included (equation (\ref{beta_UMIST})).

We can see that $\eta_A$ is well described by this power law in $\rho<10^{-11} \gcm$ (i.e., the simulated disk density region)
once $a_{\rm max} \gtrsim 250 \mum$.
Therefore, our assumption in Appendix A seems justified when the dust size exceeds $\gtrsim 100 \mum$ (at least for ModelA100Q25).
See \citet{2022ApJ...934...88T} for the results with a wider variety of parameters.

\begin{figure*}
    \includegraphics[clip,trim=0mm 0mm 0mm 0mm,width=100mm,angle=-90]{figB1.png}
    \caption{
     $\eta_{\rm A}$ as a function of the density with parameters of ModelA100Q25 in which we assume $a_{\rm min}=100 \nm$, $q=2.5$, and $\zeta_{\rm CR}=10^{-17} {\rm s^{-1}}$
      The black, blue, red, cyan, orange lines show the results of $a_{\rm max}=2.5\times 10^{-1} \mum,2.5\times 10^{0} \mum,2.5\times 10^{1} \mum, 2.5\times 10^{2} \mum, 2.5\times 10^{3} \mum$, respectively.
}
\label{etaA_amax}
\end{figure*}

