\subsection{Universality of the disk structure in the disk with grown dust grains}
In this study, we propose the new disk evolutionary picture ``co-evolution of dust grains and protoplanetary disks" based on non-ideal dust-gas two-fluid MHD simulations considering dust growth. The dust growth changes the gas-phase ionization degree, magnetic resistivity, and evolution of the disk.
In the co-evolution process, the microscopic dust grains of micrometer size couple with the macroscopic disks of $100$ AU size and they co-evolve. The size scale difference between two objects is $10^{19}$, which is astounding compared to the well-known co-evolution of super massive black holes and galaxies (size scale difference is $\sim 10^{10}$).

Furthermore, once the dust grains grow sufficiently, the structure of protoplanetary disks is well described by the non-trivial power laws, which we analytically derive in equations (\ref{solution_first}) to  (\ref{solution_last}) (figure \ref{1D_time_evolution_q25_a100nm} and \ref{rad_prof_all_models}).
From the assumptions adopted in the analytical solutions, we conclude that the disk structures will emerge when
\begin{enumerate}
\item Dust grains grow sufficiently and adsorption of charged particles by the dust grains becomes negligible,
\item The toroidal magnetic field in the disk is determined by the balance between vertical shear (of the order of $(H/r)^2$) and ambipolar diffusion, and 
 \item Angular momentum transport mechanisms other than magnetic braking (such as turbulent viscosity) are negligible.
\end{enumerate}

We believe that the discovery of this new disk structure is a theoretical breakthrough for star and planet formation theory.
The disk structure is determined only by observable parameters such as the central star mass, mass accretion rate, disk temperature, and cosmic-ray ionization rate, without including difficult-to-determine parameters such as the viscous parameter $\alpha$.
Using the analytical solution, we can study the planet formation process in the realistic disk
and evolution of the magnetic flux during protostellar evolution.
In the future, we will discuss the broad implications of this disk model for the formation and evolution of protostars and planets.

\subsection{Assumptions employed in the dust growth model and their uncertainty}

Our simulations make several simplifications to the dust growth and dust size distribution.
The largest simplification is the representative size approximation for dust growth in which we assume that the representative size corresponds to the peak dust size of the mass distribution (note that the peak of the dust mass distribution corresponds to the maximum dust size if $q<4$).
Our approximated equation for dust growth can be derived from the coagulation equation (for derivation, see \citep{2016A&A...589A..15S}).
In this study, we solve the evolution of the representative size and regard it to be the maximum dust size $a_{\rm max}$, and set the minimum dust size and power as parameters.
Moreover we implicitly assume that the size distribution can be described by a single power law.

More realistically, the time evolution of the dust size distribution should be considered, and the validity of the simplifications employed in this study should be investigated in future more realistic studies.
Detailed modeling of dust fragmentation may be important because the dust fragmentation can cause a variety of dust size distributions \citep{2011A&A...525A..11B}.
In particular, it is possible to have a large number of small dust grains \citep{2018ApJ...869L..45B}. If this is the case,
the adsorption of charged particles by dust grains is not negligible. 


However, our claim, "the disk structure converges to the analytical solution once the dust has grown sufficiently  and the adsorption of charged particles by the dust grains becomes negligible", remains valid regardless of the specific details of the dust distribution and dust growth model.
This is because the essential physics required for the disk to converge to the analytical solutions is that  the ambipolar resistivity is  determined by the balance between ionization and recombination and can be written as $\eta_A = B^2/(C \gamma  \rho^{3/2})$.
In this sense, our results are universal.


  \subsection{Comparisons with previous studies}

  Recently, \citet{2023MNRAS.518.3326L} performed spherically symmetric 1D simulations of collapsing cloud core with considering the coagulation and fragmentation of dust grains. They also calculated the change of resistivities due to the dust growth. They pointed out that
dust growth is a critical process for the resistivity in the protostellar evolution. Furthermore, they also pointed out that dust fragmentation if it happens strongly affects the magnetic resistivities profiles. 

\citet{2023A&A...670A..61M} investigated the time evolution of the collapse of the cloud cores until about $1000$ yr after the formation of the first cores with 3D simulations that consider the dust growth.
They found that the grain sizes reach more than 100 $\mum$ in the inner dense region only in $1000$ yr, and the dust growth significantly affects the resistivities. The timescale of dust growth is consistent with our simulations.

In contrast to those previous studies, we investigated the disk evolution for a longer time after protostar formation with 3D simulations.
In particular, the dominant gravitational source in our simulations is the central protostar (sink) and the gas rotation becomes Keplerian, which is necessary for the simulation results to converge to an analytical solution (see Appendix A).  Thus, future studies should include the numerical treatment of the central star that determines the gravity near the center.

\subsection{Importance of future validation}
The impact of numerical resolution or numerical methods on the simulation results were not explored in the paper because we need (additional) enormous computational costs.
Therefore, it is very important to validate our results (especially convergence to the analytical solution) with other numerical method and/or higher numerical resolution in future studies.

Nevertheless, we expect that the convergence of the simluated disk structures to the analytical solutions is robust for the following reasons.
In our simulations that converged to the analytical solution (thick lines in figure \ref{rad_prof_all_models}),
the scale heights of the disks  were resolved with different numerical resolutions (with $\sim 4$ to $10$ smoothing lengths)
due to different densities at the midplane.  Nevertheless, convergence to the power law is observed in all those simulations.
We think this point indirectly reinforces our claim.

