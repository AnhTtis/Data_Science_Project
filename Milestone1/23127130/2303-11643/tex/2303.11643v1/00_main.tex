% CVPR 2023 Paper Template
% based on the CVPR template provided by Ming-Ming Cheng (https://github.com/MCG-NKU/CVPR_Template)
% modified and extended by Stefan Roth (stefan.roth@NOSPAMtu-darmstadt.de)

\documentclass[10pt,twocolumn,letterpaper]{article}

%%%%%%%%% PAPER TYPE  - PLEASE UPDATE FOR FINAL VERSION
% \usepackage[review]{cvpr}      % To produce the REVIEW version
% \usepackage{cvpr}              % To produce the CAMERA-READY version
\usepackage[pagenumbers]{cvpr} % To force page numbers, e.g. for an arXiv version

% Include other packages here, before hyperref.
\usepackage{graphicx}
\usepackage{amsmath}
\usepackage{amssymb}
\usepackage{booktabs}
\usepackage{makecell}

% \usepackage[latin1]{inputenc}
\usepackage[british]{babel}
\usepackage[all]{xy}
\usepackage{amscd}
\usepackage{amssymb}
\usepackage{amsthm}
\usepackage{enumitem}
\usepackage{mathrsfs,bbm}
\usepackage{xcolor,graphicx}
\usepackage{graphics}
\usepackage{soul}
\usepackage{comment}
\usepackage[all]{xy}
\usepackage{amscd}
\usepackage{amssymb,amsmath,latexsym}
\usepackage{amsthm}
\usepackage{enumitem}
\usepackage{mathrsfs,bbm}
\usepackage{dsfont}
\usepackage{tikz-cd}
\usepackage[T1]{fontenc}
\usepackage[utf8]{inputenc}  
 %
%%%%%%%%%%%%%%%%%%%%%%%%%%%%%%%%%%
%pagestyle
%%%%%%%%%%%%%%%%%%%%%%%%%%%%%%%%%%
%\pagestyle{plain}
\textwidth=430pt
\headsep=.7cm
\evensidemargin=15pt
\oddsidemargin=15pt
\leftmargin=0cm
\rightmargin=0cm
%%
%%%%%%%%%%%%%%%%%%%%%%%
\newcommand*\fixitem {\item[]%
  \refstepcounter{enumi}\hskip-\leftmargin\labelenumi\hskip\labelsep}
\newtheorem*{mainthm}{Main Theorem}
\newtheorem*{mainthm1}{Theorem}
\newtheorem*{maincor}{Corollary}
\usepackage[colorlinks=true]{hyperref}
\DeclareMathOperator{\Forall}{\forall}
\DeclareMathOperator{\Exists}{\exists}
\DeclareMathOperator{\ord}{ord}
\newcommand{\phiD}{\varphi_D}
\newcommand{\phiDI}{\varphi_{\mathbf{D}_I}}
\newcommand{\phiDIj}{\varphi_{\mathbf{D}_I (j)}}
\newcommand{\phiH}{\varphi_H}
\newcommand{\phiTimes}{\phiD \otimes \phiH}
\newcommand{\phiTimesDI}{\varphi_{\mathbf{D}_I} \otimes \phiH}
\newcommand{\R}{\mathscr{A}}
\newcommand{\X}{\mathscr{X}}
\newcommand{\Xf}{\mathscr{X}_{(k_0 ,i)}[r_0]}
\newcommand{\Xfr}{\mathscr{X}_{(k_0,i)}[r]}
\newcommand{\hotimes}{\widehat{\otimes}}
\newcommand{\C}{\mathbb{C}_p}
\newcommand{\V}{\mathscr{V}}
\newcommand{\B}{\mathscr{B}}
\newcommand{\dualD}{\mathfrak{D}}
\newcommand{\Dg}{\mathbf{D}}
\newcommand{\DD}{\mathcal{D}^0}
\newcommand{\DDg}{\mathcal{D}}
\newcommand{\DV}{\mathcal{D}}
\newcommand{\W}{\mathscr{W}_N}
\newcommand{\Ao}{\mathbf{A}^\circ}
\newcommand{\AoK}{\mathbf{A}^\circ_{\K}}
\newcommand{\AK}{\mathbf{A}_{/\K}}
\newcommand{\OOO}{\mathscr{A}^\circ}
\newcommand{\K}{\mathcal{K}} 
\newcommand{\OK}{\mathcal{O}_{\K}}
\newcommand{\varprojlog}[1]{\underleftarrow{\log\!^{#1}}}
\newcommand{\T}{\mathscr{T}}
\newcommand{\TT}{\mathbf{T}}
\newcommand{\VV}{\mathbf{V}}
\newcommand{\HH}{\mathcal{H}}
\newcommand{\hh}{\mathcal{H}^+}
\newcommand{\HG}[2]{\mathcal{H}_{#1}(#2)}
\newcommand{\hhl}{\mathcal{H}^{+,[l]}}
\newcommand{\hhj}{\mathcal{H}^{+,[j]}}
\newcommand{\hhjj}{\mathcal{H}^{+,[l,l']}}
\newcommand{\GS}{G_{\mathbb{Q},S}}
\newcommand{\Rf}{R_{(k_0 ,i)}[r_0]}
\newcommand{\Rfr}{R_{(k_0 ,i)}[r]}
\newcommand{\parT}{\langle T\rangle}
\newcommand{\Zf}{Z_{(k_0 ,i)}[r_0]}
\newcommand{\Zfr}{\mathscr{Z}_{(k_0 ,i)}[r]}
\newcommand{\ZFf}{\mathscr{Z}_{(k_0 ,i)}[r_0]}
\newcommand{\ZFfr}{\mathscr{Z}_{(k_0 ,i)}[r]}
\newcommand{\ZF}{\mathscr{Z}}


\renewcommand{\todo}[1]{{\color{red} TODO: {#1}}}
\newcommand{\anote}[1]{\textcolor{blue}{\bf \emph{Anshuman: #1}}}
\newcommand{\snote}[1]{\textcolor{blue}{\bf \emph{Suya: #1}}}
\newcommand{\revision}[1]{{\color{blue} {#1}}}


% \renewcommand{\todo}[1]{}
% \renewcommand{\anote}[1]{}
% \renewcommand{\snote}[1]{}
% \renewcommand{\ynote}[1]{}
% \renewcommand{\dnote}[1]{}


% It is strongly recommended to use hyperref, especially for the review version.
% hyperref with option pagebackref eases the reviewers' job.
% Please disable hyperref *only* if you encounter grave issues, e.g. with the
% file validation for the camera-ready version.
%
% If you comment hyperref and then uncomment it, you should delete
% ReviewTempalte.aux before re-running LaTeX.
% (Or just hit 'q' on the first LaTeX run, let it finish, and you
%  should be clear).
%\usepackage[pagebackref,breaklinks,colorlinks]{hyperref}

\usepackage[pagebackref,breaklinks,hidelinks]{hyperref}

\renewcommand{\sectionautorefname}{Section}
\renewcommand{\subsectionautorefname}{Section}

% Support for easy cross-referencing
\usepackage[capitalize]{cleveref}
\crefname{section}{Sec.}{Secs.}
\Crefname{section}{Section}{Sections}
\Crefname{table}{Table}{Tables}
\crefname{table}{Tab.}{Tabs.}


%%%%%%%%% PAPER ID  - PLEASE UPDATE
\def\cvprPaperID{11010} % *** Enter the CVPR Paper ID here
\def\confName{CVPR}
\def\confYear{2023}


\begin{document}

%%%%%%%%% TITLE - PLEASE UPDATE
\title{Manipulating Transfer Learning for Property~Inference}

% \author{First Author\\
% Institution1\\
% Institution1 address\\
% {\tt\small firstauthor@i1.org}
% % For a paper whose authors are all at the same institution,
% % omit the following lines up until the closing ``}''.
% % Additional authors and addresses can be added with ``\and'',
% % just like the second author.
% % To save space, use either the email address or home page, not both
% \and
% Second Author\\
% Institution2\\
% First line of institution2 address\\
% {\tt\small secondauthor@i2.org}
% }
\author{
	Yulong Tian\textsuperscript{1}, Fnu Suya\textsuperscript{2}, Anshuman Suri\textsuperscript{2}, Fengyuan Xu\textsuperscript{1\thanks{Indicates the corresponding author.}} , David Evans\textsuperscript{2} \\
	\textit{\textsuperscript{1}State Key Laboratory for Novel Software Technology, Nanjing University, China} \\
	\textit{\textsuperscript{2}University of Virginia, USA} \\
        \tt\small{yulong.tian@smail.nju.edu.cn, \{suya, anshuman\}@virginia.edu, fengyuan.xu@nju.edu.cn, evans@virginia.edu} \\ 
}

\maketitle

%%%%%%%%% ABSTRACT
\begin{abstract}
Transfer learning is a popular method for tuning pretrained (upstream) models for different downstream tasks using limited data and computational resources. We study how an adversary with control over an upstream model used in transfer learning can conduct property inference attacks on a victim's tuned downstream model. For example, to infer the presence of images of a specific individual in the downstream training set. We demonstrate attacks in which an adversary can manipulate the upstream model to conduct highly effective and specific property inference attacks (AUC score $> 0.9$), without incurring significant performance loss on the main task. The main idea of the manipulation is to make the upstream model generate activations (intermediate features) with different distributions for samples with and without a target property, thus enabling the adversary to distinguish easily between downstream models trained with and without training examples that have the target property. Our code is available at \url{https://github.com/yulongt23/Transfer-Inference}. %\dnote{could this be renamed to a shorter URL that would fit on the line?}
%We study several possible detection methods against our proposed attacks and demonstrate that they can easily be evaded by the adaptive attacks, highlighting the stealthiness of our attacks. 
\end{abstract}

\section{Introduction}

The ability to reason about plans is critical for performing long-horizon tasks \citep{erol1996hierarchical, sohn2018hierarchical, sharma-etal-2022-skill}, compositional generalization \citep{corona-etal-2021-modular} and generalization to unseen tasks and environments \citep{shridhar2020alfred}.
Consider a simple long-horizon planning scenario where a robot is tasked with preparing a meal and serving it on the table. 
This presents a non-trivial planning problem since the agent needs to understand the sequence of operations required to perform the task and search for the relevant objects in the unfamiliar environment by interacting with various objects. %



Large language models have been recently shown to possess commonsense knowledge about the world such as object affordances and physical dynamics \citep{ouyang2022training,chowdhery2022palm}.
Early approaches considered text based environments and fine-tuned PLMs to predict actions given the history of past observations and actions \citep{jansen-2020-visually,micheli-fleuret-2021-language,yao-etal-2020-keep}.
Recent work has used this ability to reason about plans from text instructions in simulated household environments with simplifying assumptions such as text-only environment observations or feedback \citep{huang2022language,ahn2022can,li2022pre,logeswaran-etal-2022-shot}.


We focus on \emph{visually grounded planning} with PLMs --- the ability to adapt plans based on interaction and visual feedback from the environment.
While PLMs have strong planning commonsense priors, predictions from a PLM may not be directly realizable in the environment since the observation and action spaces are unknown.
This requires \emph{grounding} the PLM in the environment and adapting it to observe visual feedback, which is highly non-trivial.
Some prior works assume the availability of a pre-trained affordance function \citep{ahn2022can} or a success detector \citep{mirchandani2021ella}.
Notably, SayCan \citep{ahn2022can} completely decouples the PLM from observation information by selecting actions that have both high affordability (through a pre-trained affordance model) and high PLM likelihood.
Although this partially addresses the grounding problem, the use of visual feedback for action affordance alone is limited.
Often an agent must choose one of many affordable actions using information from observations.
For example, a driving agent should re-navigate and possibly turn around when encountering a ``road closed'' sign, but both turning around and driving forward are indistinguishable to SayCan because they are both affordable and the PLM is blind to observations.

Another workaround explored in prior work is translating the information in the visual observations to text using a pre-trained captioning system \citep{shridhar2021alfworld,huang2022language}.
However, it can be difficult to faithfully describe an image in words and information is lost in this inherently noisy process, which limits the information available to the planner.



Recent work shows that PLMs can be adapted for various natural language tasks by inserting tunable embeddings or soft prompts at the input of the PLM (also called prompt tuning or prefix tuning)~\citep{li-liang-2021-prefix,lester-etal-2021-power}.
This approach also extends to multi-modal understanding tasks such as image captioning \citep{mokady2021clipcap} and VQA \citep{tsimpoukelli2021multimodal} where images are encoded as soft prompts and finetuned for the target task.
Transformer based architectures have also been successfully applied to offline Reinforcement Learning in recent work \citep{chen2021decision,janner2021offline,li2022pre,reid2022can}.

Taking inspiration from these works, we propose the simple approach of embedding visual observations (`visual prompts') and \textit{directly inserting them as PLM input embeddings}.
The visual encoder and PLM are jointly trained for the target task, an approach we call \textbf{\oursfull}~(\ours).
By teaching the PLM to use observations for planning in an end to end manner, we remove the dependency on external data such as captions and affordability information that was used in prior work.
We show that this simple approach performs better than prior PLM-based planning approaches on two embodied planning benchmarks based on ALFWorld~\citep{shridhar2021alfworld} and Virtualhome~\cite{puig2018virtualhome}.



\section{Related Work}

%Here we summarize prior work on transfer learning and property inference.

%\shortsection{Transfer Learning}
%%Transfer learning reuses features learned by pre-trained models for new tasks, with the pretext that inherent similarities in the generic features will be useful for the downstream tasks and hence reducing their cost of downstream training. Specifically, the downstream model trainer will use a pre-trained upstream model as the starting point for the downstream training, with inclusion of (or replacement with) the task-specific classification layer/module. The downstream model is then trained by either updating all layers of the model (including ones reused from upstream model) or freezing some earlier layers of the reused parts as the ``feature extractor'' and only updating the rest. The latter approach is more popular as the reused feature extractors can already learn useful feature representations and the training cost is also much lower and affordable for individuals with limited computational resources. We study the vulnerability of the latter transfer learning approach in this paper. 


%\shortsection{Transfer Learning} 
Several works have demonstrated risks associated with transfer learning across a variety of attack goals. Wang et al.~\cite{wang2018great} and Yao et al.~\cite{yao2019latent} consider manipulating the upstream model such that the fine-tuned downstream models contain backdoors, misclassifying test inputs that contain predefined backdoor triggers. These transfer manipulations are tailored to their particular attack goals and cannot be applied for the property inference goal considered in this paper. Zou et al.~\cite{zou2020privacy} study the threat of membership inference attacks on transfer learning, but with normally trained upstream models.  
%\dnote{its clear that the goals are different for these attacks, but how similar are the methods?} \ynote{similarity of the methods? more details about the methods? do not know what is expected here}
%In contrast, we investigate the possibility of boosting the effectiveness of property inference by manipulating the upstream model training. % Schuster et al.~\cite{schuster2020humpty} show that the attacker can modify the corpus on which the word embedding is trained such that the downstream NLP models which use that embedding will behave abnormally.

%\shortsection{Property Inference}
The risk of property inference was introduced by Ateniese et al.~\cite{ateniese2015hacking}, % introduces the threat of inferring properties of the training data from pre-trained models, 
and several subsequent works have developed property inference (also known as distribution inference) attacks~\cite{Wang2022GroupPI, suri2022formalizing, Jurez2022BlackBoxAF, Hartmann2022DistributionIR}.
% Ganju et al.~\cite{ganju2018property} and Suri and Evans~\cite{suri2022formalizing} 
These works study property inference against normally trained models, and they launch attacks using a variety of black-box and white-box attacks. All the white-box attacks use meta-classifiers, which take the permutation-invariant representation~\cite{ganju2018property} of the model parameters as the features. We use the state-of-the-art white-box attack~\cite{suri2022formalizing} in our experiments.
%We will use the state-of-the-art white-box method proposed by Ganju et al.~\cite{ganju2018property} and later extended by suri et al.~\cite{suri2022formalizing} in this paper.
%\dnote{do we use these attacks?} 
Melis et al.~\cite{melis2019exploiting} and Zhang et al.~\cite{zhang2021leakage} focus on property inference in distributed training scenarios. In their settings, the attacker is a participant in the global model training and conducts property inference using meta-classifiers that are trained on model outputs or gradients. Similarly, Suri et al.~\cite{suri2022subject} focus on federated learning settings where the attacker is a participant (or the central server) that utilizes black-box attacks for inferring membership of data from particular subjects. %\dnote{if we use black-box attacks, explain which ones, or how ours are related to previous ones} 
For our experiments, We improve the black-box meta-classifier proposed by Zhang et al.~\cite{zhang2021leakage} using the ``query tuning'' technique in Xu et al.~\cite{xu2019detecting}. 

The closest works to ours are Chase et al.~\cite{saeed} and Chaudhari et al.~\cite{Chaudhari2022SNAPEE}, which both consider a scenario where the attacker can manipulate some of the training data of the model to induce a model that significantly increases property inference risk.
% \dnote{it enables precise property inference attacks?}.
These works assume an adversary with the ability to poison the victim's training data, while the adversary in our scenario has no access to the victim's training data, and therefore, their methods are not applicable.
% \dnote{example how different from ours, and why the methods are not applicable}
%Thus, their methods are not applicable to our transfer learning scenario.
%Their methods rely on inducing certain behavior correlated with the properties to be inferred, and thus are not applicable to our transfer learning scenario. \anote{Still a bit unclear why that is the case.}
%
There are also works similar to ours that leverage ``adversarial initializations'' for attack purposes.
% \cite{grosse2019adversarial, boenisch2021curious, wen2022fishing, fowl2021robbing}.
Grosse et al.~\cite{grosse2019adversarial} focus on scenarios where the attacker can control the parameter initialization of a model, and demonstrate that the attacker can use special initializations to damage the performance of the trained model. %This attack is orthogonal to ours.
Other works \cite{boenisch2021curious, wen2022fishing, fowl2021robbing} show that the malicious central server in a federated learning protocol can reconstruct some training samples via falsifying the global model in some training rounds and then analyzing the submitted gradients. These kinds of attacks do not apply to our transfer-learning scenario since the attacker cannot access the downstream gradients, and can only manipulate the upstream training.

\iffalse %%%%%%%%%%%%%%%%%%%%%%%%%%%%%%%%

In this section, we provide the background and also the summary of prior attacks on transfer learning (Section~\ref{sec:transfer_learning}) and property inference (Section~\ref{sec:property_inference}). Then, we introduce the closely related manipulation attacks against machine learning models to boost different privacy risks in Section~\ref{sec:active_inference_attacks}.

%\anote{Do we really need a dedicated section for this? It's barely 2 paragraphs right now.}

%\dnote{the most closely related work to ours are works that attempt to amplify inference attacks by poisoning models, the two most relevant I know of are \url{https://www.computer.org/csdl/proceedings-article/sp/2022/131600b569/1CIO8nmuota} and \url{https://arxiv.org/abs/2204.00032}, but need to look thoroughly for others. We should definitely be describing this and relating it to our work, probably in the introduction. Most of what is here is Background, but should be clear what this section is for (not muddling background and related work)}

\subsection{Transfer Learning} \label{sec:transfer_learning}
Transfer learning reuses features learned by pre-trained models for new tasks, with the pretext that inherent similarities in generic features can be useful for downstream tasks, thus reducing the cost of downstream training. Specifically, the downstream model trainer uses a pre-trained upstream model as the starting point for downstream training, with the inclusion (or replacement) of task-specific classification layers/modules. The downstream model is then trained by either updating all layers of the model (including ones reused from the upstream model) or freezing some earlier layers of the reused parts as the ``feature extractor'' and only updating the rest. The latter approach is more popular as the reused feature extractors can already learn useful feature representations and the training cost is also much lower and affordable for individuals with limited computational resources. We study the vulnerability of the latter transfer learning approach in this paper. 
%mainly in two ways:  1) all the layers (including ones reused from ) and tune the full model; the other one is to freeze some earlier layers of the model as the feature extractor and only tune the rest later layers. The second update strategy could achieve better efficiency since the frozen layers can already produce meaningful feature representations~\cite{wang2018great,yao2019latent}, and we will study the transfer learning using this strategy. 

Recently, various attacks have been proposed for the transfer learning setting, but with different attack goals from ours. Wang et al.~\cite{wang2018great} generate adversarial examples against black-box student models that transfer knowledge from publicly available teacher models without repeated queries. Yao et al.~\cite{yao2019latent} propose to manipulate the upstream model such that the downstream models derived from the upstream model contain backdoors, which would misclassify test inputs that contain some predefined backdoor triggers. Zou et al.~\cite{zou2020privacy} study the threat of membership inference attacks on transfer learning and the upstream models are trained normally. In contrast, we investigate the possibility of boosting the effectiveness of property inference by manipulating the upstream model training. Schuster et al.~\cite{schuster2020humpty} show that the attacker can modify the corpus on which the word embedding is trained such that the downstream NLP models which use that embedding will behave abnormally.

%This additionally allows model trainers to achieve satisfactory performance with limited training samples, leading to reduced computational costs. The most common approach reuses parameters in the earlier layers of the pre-trained model, either by fixing them as the feature extractor or just using them for initialization, to conduct downstream training.

\subsection{Property Inference} \label{sec:property_inference}

\shortsection{Property Inference Attacks} In property inference attacks, the adversary aims to infer some sensitive properties of some data, given a model trained on it. For example, the adversary may be interested in sensitive properties like the presence of people of a specific race in the dataset~\cite{ateniese2015hacking, melis2019exploiting}), or even be curious about the 
the statistics of the training set (e.g, the ratio of people with a specific gender~\cite{saeed, ganju2018property, suri2022formalizing, zhang2021leakage}).


Ateniese et al.~\cite{ateniese2015hacking} were the first to identify the threat of inferring properties of the training data from pre-trained models. Ganju et al.~\cite{ganju2018property} and Suri and Evans~\cite{suri2022formalizing} 
study property inference against normally trained models, and they launch attacks using white-box meta-classifiers, which utilize the permutation-invariance representation~\cite{ganju2018property} of the model parameters, while other works focus on distributed training~\cite{zhang2021leakage} where the attacker is a participant in the global model training and conducts property inference using meta-classifiers trained on model outputs. Similarly, Suri et al.~\cite{suri2022subject} focus on federated learning, where the attacker is a participant (or the central server) that utilizes black-box attacks for inferring membership of data from particular subjects. Chase et al.~\cite{saeed} propose an active property inference attack for data poisoning scenarios, which we will cover and compare to in Section~\ref{sec:active_inference_attacks}.

%The closest work to ours are by Chase et al.~\cite{saeed} and Tramer et al.~\cite{tramer2022truth}. In their work, the attacker can manipulate some of the training data of the model such that a model trained (from scratch) on the poisoned data has an increased inference risk. However, their methods are not applicable to the transfer learning scenario. 
%In this work, we will focus on the property inference in transfer learning scenarios in which the attacker releases the upstream model and infer sensitive properties of the downstream models tuned from that upstream model.
% 

\shortsection{Defenses}
Defending against property inference attacks is an open problem. There are no studies in the current literature on active adversaries, and only a couple on passive ones. Ma et. al.~\cite{ma2021nosnoop} propose a defense against property inference attacks on data batches in the  collaborative learning setting. However, adversaries in the transfer-learning setting do not have access to batch-wise gradients of the downstream trainer. Chen and Ohrimenko~\cite{chen2022protecting} utilize mechanisms that add carefully-crafted noise to features to provide theoretical guarantees against inference adversaries, but focus on query-based access to the underlying dataset, not a machine learning model trained on it. These existing defenses thus do not apply to our threat model.

%propose a framework that reduces property inference to Boolean functions of individual members, posing the ratio of members satisfying the given function in a dataset as the property. These property inference attacks have since then been proposed as distribution inference attacks~\cite{suri2022formalizing}, presenting such attacks as inferring properties of the distributions used to sample datasets, differentiating them from exact inference attacks like dataset inference~\cite{maini2021dataset}. Nearly all property inference attacks use meta-classifiers to perform inference: training models on versions of datasets with and without the target property, followed by training a meta-classifier on top of these classifiers's model representations. These representations can take several forms: using model weights themselves with permutation-invariance~\cite{ganju2018property}, or model activations or logits for a generated set of query points~\cite{xu2019detecting}. However, the capability of such approaches is limited: the most that these attacks have been shown to work is medium-sized convolutional networks on the CelebA dataset~\cite{suri2022formalizing}.


\subsection{Active Privacy Attacks} \label{sec:active_inference_attacks}
% Perhaps the closely related works to ours as ones that proactively enhance the effectiveness of privacy attacks by manipulating the model training process in certain ways~\cite{saeed, melis2019exploiting, nasr2019comprehensive, tramer2022truth}. 
%shown that the adversary can, by using proactive ways, achieve stronger attacks that infer private information from deep learning systems~\cite{nasr2019comprehensive, melis2019exploiting, tramer2022truth, saeed}. In this section, we introduce the ones that are close to ours.

In the decentralized federated learning training, by submitting specially crafted gradients to the central server, malicious agents can increase membership inference risk~\cite{nasr2019comprehensive} and property inference risks~\cite{melis2019exploiting} of other benign agents' training data. However, these attacks do not apply to transfer learning scenario, as the attacker cannot control model gradients of downstream training. In the centralized setting, researchers propose attacks to poison the victim's training data such that the impacts of attribute inference and membership inference~\cite{tramer2022truth} and property inference~\cite{saeed} attacks are amplified on the poisoned model.
The ability to poison the victim's data is a threat model orthogonal to ours, since we have no access to the victim's downstream data. While there is scope to combine such approaches for stronger attacks (albeit with stronger access assumptions), we choose to focus on the scenario with no read/write access to the victim's data.

\fi %%%%%%%%%%%%%%%%%%%%%%%%%%%%%%%%

\iffalse
\begin{figure*}[htb]
\centering
% \includegraphics[width=0.95\columnwidth]{fig/system/Flow Diagram.pdf}
\includegraphics[width=0.85\linewidth,scale=0.7]{fig/system/scenario.pdf}
\caption{Overview of the attack. The adversary first trains a specially-crafted upstream model (whose activations are manipulated considering an inference goal) and releases it. Then, a victim tunes that model for its downstream task on a private dataset. At last, the adversary leverages its different levels of access to the tuned downstream model to infer sensitive information about the victim's training dataset. 
%The adversary trains its model on the upstream task such that activations after $f(\cdot)$ are different for data with and without the target property. The victim performs transfer learning starting with the adversary's model to train the downstream model, which the adversary can then inspect to infer the presence of data with the target property. 
% \dnote{can the figure show the attack? the adversary is using $g_d$ to decide if $\mathcal{D}_{vic} \cap (\mathcal{D}_w - \mathcal{D}_{wo})$ is non-empty? (the subtracting the distributions like this isn't quite right, but if there was a better notation to make $\mathcal{D}_w = \mathcal{D}_{wo} \cup \mathcal{W}$? }
}
\label{fig:threat_model}
\end{figure*}
\fi

\section{Threat Model} \label{sec:threat_model}
% \begin{figure}
% \centering
% \includegraphics[width=0.95\columnwidth]{fig/system/Trainsfer_Inference_Diag.pdf}
% \caption[Flow of information in our threat model. The adversary $\mathcal{A}$ releases a carefully-manipulated model, which the victim then re-uses partially to finetune the model on its data.]{Flow of information in our threat model. The adversary $\mathcal{A}$ releases a carefully-manipulated model, which the victim then re-uses partially to finetune the model on its data\protect\footnotemark.}
% \label{fig:transer_inf_flow}
% \end{figure}
% \footnotetext{This diagram is adapted from the illustration of transfer learning of~\cite{wang2018great}}

The adversary $\mathcal{A}$ trains and releases a specially crafted upstream model $g_{u}(f(\cdot))$ that is used by a victim $\mathcal{B}$ to fine-tune a model $g_{d}(f(\cdot))$ for a downstream task on a downstream training set $\mathrm{D}$. This model is then exposed to $\mathcal{A}$, with varying levels of knowledge and access (discussed below), who performs property inference attacks to learn some desired property of $\mathrm{D}$. %\dnote{the adversary's tuning data, but this doesn't seem to be included in the above? should be explicit, and have a notation for the training dataset} \ynote{the victim's tuning data?}
As is common in many transfer learning settings, the upstream model includes $f(\cdot)$, a fixed feature-extraction component that is not modified by the downstream tuning process~\cite{schuster2020humpty, wang2018great,yao2019latent}. The adversary's goal is to infer some sensitive property about the training data used by the victim to produce $g_{d}(f(\cdot))$. %such as whether images of a specific individual or individuals with a specific property are used in the downstream training set.
For example, the adversary can release a general vision model (e.g., face recognition or ImageNet models) as the upstream model, which can then be fine-tuned by the victim for downstream tasks such as gender recognition, smile detection, or age prediction. The attacker's goal could be to infer whether or not images of a specific individual or individuals with a specific property are included in the downstream training set for tuning. 
%\dnote{this sounds like we are focused on face recognition? is that true? not mentioned earlier. If claiming a general attack, not specific to face recognition, should have other types of examples (including in the experiments). I think we have a more specific scenario in mind - is it a general vision model as pretrained model, and then tuned to do face recognition (but isn't the set of labels already revealing the specific faces?) I think we need to explain a concrete scenario that matches our experiments and motivate it well} \ynote{Yeah, we need to motivate the story well. We are not focused on face recognition but face related downstream tasks}
This is different from commonly studied membership inference attacks---in membership inference %for the given example,
the attacker is assumed to know a specific image and aims to infer if that specific image was included in the training set; in property inference, the attacker does not presume knowledge of specific training images, but wants to determine if any images having a given property were used in training. 
%\dnote{I think we need to be careful about the language here - it is inferring a property of the training data set, which we are viewing as sampled from a training distribution. I don't think we really need to get into these distinctions here, but should use the "distribution" language carefully, or just avoid it in this paper.}
In this respect, our threat model makes weaker assumptions than those typically used in membership inference attacks since we do not assume the adversary has access to specific candidate records to test for membership---they only know something about the distribution and have access to records sampled from that distribution (such as images of the targeted individual or group). 
%\anote{This is similar to the concept of subject-level privacy, which has been explored in federated learning~\cite{marathesubject}.}
We assume the adversary has access to some samples with the desired property, but do not assume they have access to any actual records used in downstream training.

% Dave - I'm cutting this from here, it doesn't belong here and might be confusing. Should be mentioned when we talk about defenses.
%We also assume the victim is unaware of the property targeted by the adversary, as this is specific to an attacker's goal and the possible properties can be very diverse for a rich training set. 
%\dnote{need to be much more explicit about what the downstream task is here - if it is face recognition, this seems trivial - if the model was well trained to recognize that individual, just query the model of images of that individual and see if it outputs the expected label}

%(such as the presence of a certain individual in victim's data).
% To formalize this, let $\mathcal{D}$ denote the natural distribution of data, which can be further partitioned into $\mathcal{D}_{w}$ and $\mathcal{D}_{wo}$, distributions of samples with and without the target property respectively. 

% We consider an adversary that has the ability to manipulate an upstream model, which will then be fine-tuned by the victim for the downstream tasks. The adversary also obtains varying degree of access (detailed discussion below) to the tuned (and released) downstream model.
% The adversary's goal is to infer some sensitive property about the training data used by the victim, such as whether images of specific individual are used in the training set of the downstream model~\footnote{This is different from the commonly studied membership inference attacks. In membership inference attack, for the given example, the attacker's goal is to infer if specific image of an individual is used in the training set of downstream model while in property inference, the attacker cares if any image of that individual is used in training.}.
%\dnote{can we give a clear example of what will be learned, that will fit with the experimental results?}.
% \dnote{diagram would be helpful}

%property \dnote{prefer to use distribution inference terminology} inference in the transfer learning scenario which involves an \emph{upstream model trainer} who releases backoored pre-trained models and \emph{downstream trainers} who reuse those pre-trained models for their own task in a transfer learning manner.

% We assume the upstream trainer has full control over the upstream training process and releases specially-crafted upstream models designed to enable a particular distribution inference attack on the victim.
% The downstream trainer trains the downstream model by reusing some layers of the released upstream models as the feature extractor for its own task.
% After the downstream training, the upstream trainer conducts the inference attack by either directly analyzing the downstream model parameters (white-box attack) or sending queries to the downstream models (black-box attack).
% The flow of information is given in Figure~\ref{fig:threat_model}.

\shortsection{Attacker's Knowledge}
We assume the attacker knows which layers of the pretrained model will be reused by the downstream trainer as the feature extractor. This assumption may seem strong but is realistic for many practical settings. Downstream fine-tuning usually modifies the final layers (or even just the classification layer/module) and keeps other parameters fixed~\cite{wang2018great, yao2019latent}. Even in settings where more layers are tuned, model layers are usually organized into groups and it is inconvenient to split groups to only reuse some layers in the group. For example, ResNet models~\cite{he2016deep} can have over a hundred layers, but are grouped into only four ResNet blocks. Hence,
% although there might be many layers in a model,
the number of feasible choices of layers from the upstream model that will be used as feature extractor is limited and constrained by the architecture of the pretrained model, which is controlled by the adversary in our threat model.

We consider three scenarios based on the level of access.
The weakest adversary, representing the most common practical scenario, is the \emph{black-box API access} adversary who only has access to the model through the ability to send queries to its API and receive confidence vectors as outputs.
% \dnote{does API provide confidence vector or just label?}
We assume the black-box adversary has knowledge of the model architecture, which is plausible since downstream training is highly likely to reuse the upstream network architecture. 

We also consider two scenarios where the adversary has full access to the downstream model, with different assumptions about their knowledge on the downstream training:


%versions of white-box attacks based on the adversary's knowledge of the downstream training process: 

%levels of knowledge and also the access to the downstream model:
%\dnote{I think the knowledge of the initialization is orthogonal to the level of exposure of the model (API or parameters); it doesn't make sense to consider the API-known initialization case, though, since less clear how this helps the API-only adversary. Still, we should describe these more clearly by separating the issues.}
%\dnote{rewrite to describe this one first, then explain why there are two types of white-box threat modesl to consider based on how much is known about the downstream training process}

%\textbf{Black-box API Access} --- the attacker does not know the model parameters and can only access the downstream model via API query access, but knows the general model architecture used. Knowledge of  model architecture is plausible, since downstream training is highly likely to reuse the upstream network architecture. This scenario is the most common one in practice, as the downstream trainer may deem its models as intellectual property (IP) and not expose model parameters. 

%\textbf{White-box Access} --- for adversaries with white-box access (i.e., known model architecture, parameters) to the downstream model, we further split them into two categories based on their knowledge about the downstream training process:
\begin{enumerate}
% We also consider attackers with varying degree of access to the downstream model and assess the vulnerabilities of downstream models in different practical applications. 

\item \emph{white-box access with unknown initialization} --- the adversary has full access to the trained downstream model but does not know the parameter initialization of $g_d(\cdot)$. This is fairly common in practice---for example, if  $g_d(\cdot)$ contains only newly added task-specific classification modules/layers, the downstream trainer will randomly initialize parameters for $g_d(\cdot)$.

\item \emph{white-box access with known initialization} --- the adversary also knows the initialization of the parameters of layers in $g_d(\cdot)$ that are reused (but will also be updated during downstream training) from the upstream models. In practice, the attacker only needs to know the initialization of  the first layer of $g_d(\cdot)$ (Section~\ref{sec:zero_activation_attack}).
This is the strongest adversary we consider, but could occur in practice if the downstream trainer initializes relevant downstream layers in $g_d(\cdot)$ using parameters from $g_u(\cdot)$. 

%This is more common than known-initialization as in many practical settings (e.g., only adding fully-connected layers in downstream classification), the downstream trainer will choose to randomly initialize parameters for $g_d(\cdot)$. 
%\ynote{This one has a weaker assumption, but this scenario is not necessarily more common than the first one}

\end{enumerate}

%Our attack consists of two phases: 1) training specially crafted pretrained models to amplify effectiveness of property inference, and 2) conducting property inference attacks on downstream models that are fine-tuned from the pretrained models. In next sections, we first introduce how we generate the specially crafted pretrained models in Section~\ref{sec:attack_design} and then introduce the property inference attacks on the fine-tuned downstream models in Section~\ref{sec:zero_activation_inference}.

\section{Method}
% \label{sec:method}

\begin{figure*}[t]
\centering
\includegraphics[width=\linewidth]{figures/framework_v3.pdf}
\vspace{-5mm}
\caption{An overview of our 3D-CLR framework. First, we learn a 3D compact scene representation from multi-view images using neural fields (I). Second, we use CLIP-LSeg model to get per-pixel 2D features (II). We utilize a 3D-2D alignment loss to assign features to the 3D compact representation (III). By calculating the dot-product attention between the 3D per-point features and CLIP language embeddings, we could get the concept grounding in 3D (IV). Finally, the reasoning process is performed via a set of neural reasoning operators, such as \textsc{Filter}, \textsc{Get\_Instance} and \textsc{Count\_Relation} (V). Relation operators are learned via relation networks.}
\vspace{-5mm}
\label{fig:framework}
\end{figure*}

Fig.~\ref{fig:framework} illustrates an overview of our framework. Specifically, our framework consists of three steps.  First, we learn a 3D compact representation from multi-view images using neural field. And then we propose to leverage pre-trained 2D vision-and-language model to ground concepts on 3D space. This is achieved by 1) generating 2D pixel features using CLIP-LSeg; 2) aligning the features of 3D voxel grid and 2D pixel features from CLIP- LSeg~\cite{li2022language}; 3) dot-product attention between the 3D features and CLIP language features~\cite{li2022language}. Finally, to perform visual reasoning, we propose neural reasoning operators, which execute the question step by step on the 3D compact representation and outputs a final answer. For example, we use \textsc{Filter} operators to ground semantic concepts on the 3D representation, \textsc{Get\_Instance} to get all instances of a semantic class, and \textsc{Count\_Relation} to count how many pairs of the two semantic classes have the queried relation.
% \gc{metion all the neural operators.}
% Works from linguistic and cognitive science suggest that semantic concepts are diverse and open-vocabulary, while relational concepts describing 3D objects' relationships can be very limited and thus can be considered a close-class vocabulary \cite{Landau1993WhatA, Hayward1995SpatialLA}. Therefore, it's unrealistic to learn the embeddings of all the concepts in the question-answering pairs, while it's more natural to learn the relation embeddings. Inspired by this, we propose to leverage 2D pretrained vision-language model (\textit{i.e.,} CLIP) for open-vocabulary semantic concept learning, while proposing a neural relation module network for relational reasoning. 

\subsection{Learning 3D Compact Scene Representations}

% Since 3D-related reasoning works on 3D compact representations rather than 2D images, we first propose to use a neural field to extract 3D representations from multi-view images. The next step is to learn the 3D features for visual reasoning. However, 3D assets are limited in diversity and scale, posing challenges for training large-scale 3D foundation models, while there's much progress on large-scale 2D pretrained models which provide decent features\cite{Radford2021LearningTV, Ramesh2021ZeroShotTG}. Since neural field maps a 2D pixel to several 3D points along the ray, it's natural to get 3D features for 2D per-pixel features. We apply CLIP-LSeg\cite{Li2022LanguagedrivenSS} to learn per-xel 2D features, and use an alignment loss to align 3D features with 2D features.

% \paragraph{3D Compact Representation from neural field.} 
Neural radiance fields  \cite{mildenhall2020nerf} are capable of learning a 3D representation that can reconstruct a volumetric 3D scene representation from a set of images. Voxel-based methods \cite{Garbin2021FastNeRFHN, Hedman2021BakingNR, Yu2021PlenOctreesFR, Sun2022DirectVG} speed up the learning process by explicitly storing the scene properties (\textit{e.g.}, density, color and feature) in its voxel grids. We leverage Direct Voxel Grid Optimization (DVGO) \cite{Sun2022DirectVG} as our backbone for 3D compact representation for its fast speed. DVGO stores the learned density and color properties in its grid cells. The rendering of multi-view images is by interpolating through the voxel grids to get the density and color for each sampled point along each sampled ray, and integrating the colors based on the rendering alpha weights calculated from densities according to quadrature rule \cite{Max1995OpticalMF}. The model is trained by minimizing the L2 loss between the rendered multi-view images and the ground-truth multi-view images. By extracting the density voxel grid, we can get the 3D compact representation (\textit{e.g.,} By visualizing points with density greater than 0.5, we can get the 3D representation as shown in Fig. \ref{fig:framework} I. ) 

\subsection{3D Semantic Concept Grounding}
Once we extract the 3D compact representation of the scene, we need to ground the semantic concepts for reasoning from language. 
Recent work from \cite{hong20223d} has proposed to ground concepts from paired 3D assets and question-answers. Though promising results have been achieved on synthetic data, it is not feasible for open-vocabulary 3D reasoning in real-world data, since it is hard to collect large-scale 3D vision-and-language paired data.  To address this challenge, our idea is to leverage  pre-trained 2D vision and language model \cite{Radford2021LearningTV, Ramesh2021ZeroShotTG} for 3D concept grounding in real-world scenes.  But how can we map 2D concepts into 3D neural field representations? Note that 3D compact representations can be learned from 2D multi-view images and that each 2D pixel actually corresponds to several 3D points along the ray. Therefore, it's possible to get 3D features from 2D per-pixel features. Inspired by this, we first add a feature voxel grid representation to DVGO, in addition to density and color, to represent 3D features. 
% it's natural to utilize 2D VLMs to ground semantic concepts on the 3D representations. 
 We then apply CLIP-LSeg\cite{li2022language} to learn per-pixel 2D features, which can be attended to by CLIP concept embeddings. We use an alignment loss to align 3D features with 2D features so that we can perform concept grounding on the 3D representations.
% Since 3D voxel grids and 2D pixels are aligned via alpha compositing, we add one L1 loss to force the features of 3D voxel grids to align with the 2D LSeg pixels based on the alpha values. 

\noindent\textbf{2D Feature Extraction.}
To get per-pixel features that can be attended by concept embeddings, we use the features from language-driven semantic segmentation (CLIP-LSeg) \cite{li2022language}, which learns 2D per-pixel features from a pre-trained vision-language model (\textit{i.e.,} \cite{Radford2021LearningTV}). Specifically, it
uses the text encoder from CLIP, trains an image encoder to produce an embedding vector for each pixel, and calculates the scores of word-pixel correlation by dot-product. By outputting the semantic class with the maximum score of each pixel, CLIP-LSeg is able to perform zero-shot 2D semantic segmentation.

\noindent\textbf{3D-2D Alignment.}
In addition to density and color, we also store a 512-dim feature in each grid cell in the compact representation. To align the 3D per-point features with 2D per-pixel features, we calculate an L1 loss between each pixel and each 3D point sampled on the ray of the pixel. The overall L1 loss along a ray is the weighted sum of all the pixel-point alignment losses, with weights same as the rendering weights: $\mathcal{L}_{\text {feature}}=\sum_{i=1}^K w_i(\|\boldsymbol{f_i}-F(\boldsymbol{r})\|),$
where $\boldsymbol{r}$ is a ray corresponding to a 2D pixel, $F(\boldsymbol{r})$ is the 2D feature from CLIP-LSeg, $K$ is the total number of sampled points along the ray and $\boldsymbol{f_i}$ is the feature of point $i$ by interpolating through the feature voxel grid, $w_i$ is the rendering weight.
% \gc{add equations.} 

\noindent\textbf{Concept Grounding through Attention.}  Since our feature voxel grid representation is learnt from CLIP-LSeg, by calculating the dot-product attention $<\boldsymbol{f}, \boldsymbol{v}> $ between per-point 3D feature $\boldsymbol{f}$ and the CLIP concept embeddings $\boldsymbol{v}$, we can get zero-shot view-independent concept grounding and semantic segmentations in the 3D representation, as is presented in Fig. \ref{fig:framework} IV. 
% \gc{add equations.}

\subsection{Neural Reasoning Operators}
Finally, we use the grounded semantic concepts for 3D reasoning from language. We first transform questions into a sequence of operators that can be executed on the 3D representation for reasoning. We adopt a LSTM-based semantic parser   \cite{Yi2018NeuralSymbolicVD} for that. As \cite{Mao2019TheNC, hong20223d}, we further devise a set of operators which can be executed on the 3D representation.  Please refer to \textbf{Appendix} for a full list of operators.

\noindent\textbf{Filter Operators.}  We filter all the grid cells with a certain semantic concept.

\noindent\textbf{Get\_Instance Operators.} We implement this by utilizing DBSCAN \cite{Ester1996ADA}, an unsupervised algorithm which assigns clusters to a set of points. Specifically, given a set of points in the 3D space, it can group together the points that are closely packed together for instance segmentation.

\noindent\textbf{Relation Operators.} We cannot directly execute the relation on the 3D representation as we have not grounded relations. Thus, we represent each relation using a distinct neural module (which is practical as the vocabulary of relations is limited \cite{Landau1993WhatA}). We first concatenate the voxel grid representations of all the referred objects and feed them into the relation network.
% \yd{Do we do something afterwards -- we first concatenate, then what?} 
The relation network consists of three 3D convolutional layers and then three 3D deconvolutional layers. A score is output by the relation network indicating whether the objects have the relationship or not. Since vanilla 3D CNNs are very slow, we use Sparse Convolution \cite{spconv2022} instead. Based on the relations asked in the questions, different relation modules are chosen. 

% \subsection{Learning 3D Compact Representation}
% In recent years, neural field models(\textit{e.g.,} \cite{mildenhall2020nerf}) have gained much popularity since they can reconstruct a volumetric 3D scene representation from a set of images. Recent works \cite{Garbin2021FastNeRFHN, Hedman2021BakingNR, Yu2021PlenOctreesFR, Sun2022DirectVG} have pushed it further by using classic voxel-grids to explicitly store the scene properties (\textit{e.g.}, density, color and feature) for rendering, which allows for real-time rendering. Since concept grounding and relation learning are expected to work on the per-point features in the 3D space \cite{hong20223d} of thousands of scenes, it's more suitable to use voxel-grid-based methods since they store explicit properties in each point which can be directly used for reasoning, and super-fast convergence makes it feasible to train thousands of scenes. Specifically, we use the fine reconstruction process of Direct Voxel Grid Optimization \cite{Sun2022DirectVG} as our backbone for 3D compact representation for its fast speed. 

% A compact voxel-grid representation models the modalities of interest (\textit{e.g.,} density, color or feature) explicitly in its grid cells. To query the properties at any given 3D point, interpolation is used:
% \begin{equation}
% \operatorname{interp}(\boldsymbol{x}, \boldsymbol{V}):\left(\mathbb{R}^3, \mathbb{R}^{C \times N_x \times N_y \times N_z}\right) \rightarrow \mathbb{R}^C
% \end{equation}
% where $\boldsymbol{x}$ is the queried 3D point,  $\boldsymbol{V}$ is the voxel grid, and $C$ is
% the dimension of one of the modalities, and $N_x, N_Y, N_z$ is the number of voxels. We first predict the density of a specified point by interpolating the density grid. This is crucial for the geometric reconstruction of the scene.  
% \begin{equation}
% \sigma=\operatorname{interp}\left(\boldsymbol{x}, \boldsymbol{V}^{(\text {density })}\right)
% \end{equation}
% where $\sigma$ is the volume density at position $\boldsymbol{x}$. For the modeling of color emission, we use an explicit-implicit hybrid representation where  a shallow MLP is placed after the color voxel grid interpolation process:
% \begin{equation}
% \boldsymbol{c}=\operatorname{MLP}^{(\mathrm{rgb})}\left(\operatorname{interp}\left(\boldsymbol{x}, \boldsymbol{V}^{(\mathrm{color})}\right), \boldsymbol{x}, \boldsymbol{d}\right)
% \end{equation}
% where $\boldsymbol{c}$ is the view-dependent color emission at position $\boldsymbol{x}$ viewing from direction $\boldsymbol{d}$.

% To render the color $\hat{C}(\boldsymbol{r})$ of ray $r$, K points are sampled on ray $r$ with densities and colors $\left\{\left(\sigma_i, \boldsymbol{c}_i\right)\right\}_{i=1}^K$. The K results are accumulated by the quadrature rule by Max \cite{Max1995OpticalMF}:
% \begin{align}
% \hat{C}(\mathbf{r})=\sum_{i=1}^K T_i\left(1-\exp \left(-\sigma_i \delta_i\right)\right) \mathbf{c}_i, 
% &\\
% T_i=\exp \left(-\sum_{j=1}^{i-1} \sigma_j \delta_j\right)
% \end{align}
% where $\delta_i=t_{i+1}-t_i$ is the distance between adjacent points along a ray, and $\alpha_i=1-\exp \left(-\sigma_i \delta_i\right)$ is the alpha value for traditional alpha compositing.

% The backbone is trained by minimizing the mean
% square error between the rendered and observed color. 

% \begin{equation}
% \mathcal{L}_{\text {color }}=\|\hat{C}(\boldsymbol{r})-C(\boldsymbol{r})\|_2^2
% \end{equation}

% By extracting the density values of the voxel grid $\boldsymbol{V}^{(\text {density })} \in \mathbb{R}^{1 \times N_x \times N_y \times N_z}$, we can get the compact 3D representation of the scene, as shown in the middle of Figure 2.

% We refer the readers to \cite{Sun2022DirectVG} for more details about the Direct Voxel Grid Optimization.


% \subsection{3D Semantic Concept Grounding}
% In \cite{hong20223d}, a Neural Descriptor Field (NDF) \cite{simeonov2021neural} which gives a feature vector for each 3D coordinate a feature vector is used for concept grounding by aligning the feature vector with the learned concept embeddings. Drawing inspiration from this, we also propose to use a feature voxel-grid  (in addition to density voxel grid and color voxel grid) used for concept grounding. The compact 3D feature representation is composed of one feature voxel-grid representation plus one view-independent shallow MLP: 

% \begin{equation}
% \boldsymbol{f}=\operatorname{MLP}^{(\mathrm{feature})}\left(\operatorname{interp}\left(\boldsymbol{x}, \boldsymbol{V}^{(\mathrm{feature})}\right), \boldsymbol{x}, \boldsymbol{d}\right)
% \end{equation}

% However, the drawback of \cite{hong20223d} is that the embeddings of concepts are learnt from sratch, which is unrealistic in the open-vocabulary reasoning in real-world data. Furthermore, compared to 2D data, 3D assets are limited in diversity and scale, posing challenges for training large vision-language models (VLMs) on 3D-and-language data. Therefore, there's no large-scale 3D VLMs that can be directly used for concept grounding. On the contrary, there's much progress on large-scale 2D VLMs \cite{Radford2021LearningTV, Ramesh2021ZeroShotTG} thanks to the countless image-caption data on the internet. Since we obtain 3D compact representations from 2D multi-view images, it's natural to utilize 2D VLMs to ground semantic concepts on the 3D representations. Based on the CLIP model \cite{Radford2021LearningTV}, LSeg\cite{Li2022LanguagedrivenSS} manages to ground semantic concepts on each 2D pixel (and thus each ray $r$). We denote the feature of ray $r$ as $F(\boldsymbol{r})$.
% Since 3D voxel grids and 2D pixels are aligned via alpha compositing, we add one L1 loss to force the features of 3D voxel grids to align with the 2D LSeg pixels based on the alpha values. Specifically,

% \begin{equation}
% \mathcal{L}_{\text {feature}}=\sum_{i=1}^K T_i\left(1-\exp \left(-\sigma_i \delta_i\right)\right)(\|\boldsymbol{f}-F(\boldsymbol{r})\|)
% \end{equation}

% Assuming we have a set of concepts $P$, the similarities between a concept $\boldsymbol{p} \in P$ and a feature $\boldsymbol{f}$ is calculated as $\langle \boldsymbol{f}, \boldsymbol{p} \rangle$. We define a \textsc{Filter} operator. Specifically, the 3D compact representation for a semantic class $p$ after filtering out that class is:

% \begin{equation}
% \boldsymbol{V}_{\boldsymbol{p}} =  min(\langle\boldsymbol{V}^{(\mathrm{feature})}, \boldsymbol{p}\rangle, \boldsymbol{V}^{(\mathrm{density})}) 
% \end{equation}

% In practice, we only set the values of voxel grids with densities < 0.5 to 0, since we find that those points are irrelevant to the 3D geometry of the scene.

% To get each instance of the objects of the same category, we use DBSCAN \cite{Ester1996ADA} to implement the \textsc{Get\_Instance} operator which assigns clusters to all true values of $\boldsymbol{V}_{\boldsymbol{p}}$. The DBSCAN takes the 3D coordinates as input.





\section{Experimental Design} \label{sec:exp_setup}
This section explains our experimental setup. We present results from our experiments to measure the effectiveness of different attacks in Section~\ref{sec:eval_zero_activation_attack}. 

% We provide details of our experiment setup, followed by results for the baseline setting (no manipulation) and models manipulated with our proposed method.

%We provide details of our task setup (Section~\ref{sec:exp_setup}). We then present the results for the baseline settings where the upstream models are trained normally (no manipulation) (Section~\ref{sec:eval_baseline}) followed by manipulated upstream models (Section~\ref{sec:eval_zero_activation_attack}), demonstrating the effectiveness of our attack. 

% \dnote{if we are short on space, most of these details should be moved to appendix, but the description of the tasks should be included in the introduction (at least one example to make it clear what scenarios are considered)}


\iffalse
We then  show our proposed attack  works well even when the attacker does not know the actual size of the downstream training set used by the victim (Section~\ref{sec:unknown_training_size}).
%considering the attacker might not know the size of the downstream training set used by the victim when preparing the shadow models, we show that the inference is robust to this factor. 
At last, we provide the results of ablation studies (Section~\ref{sec:study_distribution_aug}) 
\fi


%\subsection{Experimental Design}


% We focus on the inference of the inclusion of specific persons in the downstream training, and design three transfer learning tasks to evaluate our attacks. 
% \begin{table*}[htbp!]
%   \centering
%   \small
%     \begin{tabular}{c|cccc}
%     \toprule
%   Downstream Task & Upstream Task & Target Propety & Upstream model & Downstream Model\\
%     \midrule
%     Gender recognition & Face Recognition & Specific Person & MobileNetV2~\cite{sandler2018mobilenetv2} &  Final Classification Module\\
%     Smile detection & ImageNet Classification~\cite{deng2009imagenet} & Senior / Specific Person  & ResNet-34~\cite{he2016deep} &  $4^{th}$ block + Classification Layer\\
%     Age prediction & ImageNet Classification~\cite{deng2009imagenet} & Asian / Specific Person & ResNet-18~\cite{he2016deep} &   $4^{th}$ block + Classification Layer\\
    
%     \bottomrule
%     \end{tabular}%
%   \caption{The tasks, inference targets, and model architectures considered in this paper.}
%   \label{tab:experimental_setup}
% \end{table*} 

\shortsection{Tasks and Models} We consider three transfer learning tasks in our experiments: \emph{gender recognition}, \emph{smile detection}, and \emph{age prediction}. These tasks are commonly studied in the transfer learning literature~\cite{akhand2020human, dornaika2019age, guo2018smile, nga2020transfer, wang2018great, xia2017detecting, yao2019latent}.  In the gender recognition task, the victim trains downstream models for gender recognition reusing the feature extraction module of pre-trained (upstream) MobileNetV2~\cite{sandler2018mobilenetv2} models of face recognition as the feature extractor. The upstream face recognition models classify images of 50 people randomly sampled from the VGGFace2 dataset~\cite{cao2018vggface2}, and the feature extraction module in a MobileNetV2 model contains all the layers before the final classification module. For the smile detection and age prediction (classify as ``young", ``middle-aged" or ``senior") tasks, the victim reuses the layers before the fourth block of ResNet~\cite{he2016deep} classifiers (ResNet-34 for smile detection and ResNet-18 for age prediction) trained on ImageNet~\cite{deng2009imagenet} as the feature extractors. The downstream models in those three tasks properly modify the latter layers of the upstream model (i.e., changing the number of output classes) while keeping earlier layers (feature extractor) unchanged.

\shortsection{Upstream and Downstream Training}
For all the scenarios, when training the upstream models, we consider the property inference task of determining whether images of specific individuals are present in the downstream training set. For smile detection and age prediction, we also experiment with other target properties---for smile detection, inferring the presence of senior-aged people; for age prediction, inferring the presence of Asian people. Appendix~\ref{sec:extra_dataset_details} provides more details about the upstream training,
% including the selection of and the number of samples in the upstream training set
%More details about the upstream training including the injection of samples with the target property into the upstream training set and distribution augmentation are available in Appendix~\ref{sec:extra_dataset_details}.

We conduct the downstream training on VGGFace2 with the attribute labels provided by MAADFace~\cite{terhorst2021maad, terhorst2019reliable}. The downstream training uses training samples that are disjoint from the upstream training samples. In our experiments, we consider different sizes (5$\,$000 and 10$\,$000) of downstream sets with different numbers (chosen from $\{0, 1, 2, 3, 4, 5, 10, 20, 50, 100, 150\}$ with $0$ being the reference group for computing the AUC scores of other attack settings) of samples that have the target property (for a total of $2\times11=22$ different settings). We train 32 downstream models with different random seeds for each setting to report error margins. 
Appendix~\ref{sec:details-of-downstream-training} gives more details of downstream training and the %adversary's 
training of meta-classifiers.

\iffalse
\shortsection{Tasks and Model Preparation}

\begin{table*}[ht]
  \centering
  \small
    \scalebox{1}{
    \begin{tabular}{ccccc}
    \toprule
  \textbf{Downstream Task} & \textbf{Upstream Task} & \textbf{Target Property} & \textbf{Upstream Model} & \textbf{Downstream Model}\\
    \midrule
    Gender recognition & Face Recognition & Specific Person & MobileNetV2~\cite{sandler2018mobilenetv2} &  Final Classification Module\\
    \hline
    \multirow{2}{*}{Smile detection} & \multirow{2}{*}{ImageNet Classification~\cite{deng2009imagenet}} & Specific Person & \multirow{2}{*}{ResNet-34~\cite{he2016deep}} & \multirow{2}{*}{$4^{th}$ block + Classification Layer} \\
    & & Senior\\
    % Smile detection & ImageNet Classification~\cite{deng2009imagenet} & Senior & ResNet-34~\cite{he2016deep} &  $4^{th}$ block + Classification Layer\\
    % Smile detection & ImageNet Classification~\cite{deng2009imagenet} & Specific Person  & ResNet-34~\cite{he2016deep} &  $4^{th}$ block + Classification Layer\\
    \hline
    \multirow{2}{*}{Age prediction} & \multirow{2}{*}{ImageNet Classification~\cite{deng2009imagenet}} & Specific Person & \multirow{2}{*}{ResNet-18~\cite{he2016deep}} & \multirow{2}{*}{$4^{th}$ block + Classification Layer} \\
    % Age prediction & ImageNet Classification~\cite{deng2009imagenet} & Asian & ResNet-18~\cite{he2016deep} &   $4^{th}$ block + Classification Layer\\
    % Age prediction & ImageNet Classification~\cite{deng2009imagenet} & Specific Person & ResNet-18~\cite{he2016deep} &   $4^{th}$ block + Classification Layer\\
    & & Asian \\
    \bottomrule
    \end{tabular}%
    }
  \caption{The transfer learning tasks, inference targets, and model architectures considered in this paper.}
  \label{tab:experimental_setup}
\end{table*}


Table~\ref{tab:experimental_setup} summarizes the tasks, target properties, and model architectures of the three transfer learning scenarios we consider in our experiments. The first scenario is where the victim trains (downstream) models for gender recognition reusing the feature extraction module of pre-trained (upstream) models of face recognition. The second and third scenarios are for smile detection and age prediction (classify into categories of ``young", ``middle-aged" and ``senior") respectively, and the victim reuses some earlier layers of ImageNet classifiers as the feature extractors. We choose these tasks, as they are the commonly studied settings in the transfer learning literature~\cite{akhand2020human, dornaika2019age, guo2018smile, nga2020transfer, wang2018great, xia2017detecting, yao2019latent}. 
%\dnote{these seem like quite strange and unusual tasks - need to provide some justification for these choices, and explain the more realistic scenarios they are meant to be proxies for}
%We choose these tasks, as they are commonly studied settings in property inference literature~\cite{ganju2018property, melis2019exploiting, suri2022formalizing, zhang2021leakage}. \anote{TODO: Talk about how/why they are proxies.} \ynote{first, I don't think they are unusual tasks. Second, if we admit that they are just proxies for more realistic scenarios, why do not we directly conduct our experiments with those more realistic scenarios? I think we can  cite existing transfer learning papers to justify the considered transfer learning scenarios. And if the inference targets are also strange, we can cite property inference papers.}
% [Property inference] We choose these tasks, as they are commonly studied settings in property inference literature~\cite{ganju2018property, melis2019exploiting, suri2022formalizing, zhang2021leakage}.
For all the scenarios, the goal of the attacker is to infer whether images of specific individuals are present in the downstream training set. For smile detection and age prediction, we also consider additional target properties: for smile detection, the attacker aims to infer the presence of senior-aged people; for age prediction, the attacker aims to infer the presence of Asian people. 
%\dnote{confusing to map these to Table 1. Should have a row for each setting, instead of trying to combine them. Also, why are the models different (ResNet-34/ResNet-18 for the two ImageNet tasks? If the goal is to understand impact of upstream model choice, should be trying the same tasks with different models; if the goal is to understand impact of the task, should be using the same upstream model for both tasks? At least, need to explain why things are different this way.} \ynote{We just want to try as many models as possible.}\dnote{okay, but that would be trying all combinations of things - can't be picking and choosing which combinations to try without an explanation of why, and can't make conclusions about what impacts the results without having clear compairsons} \ynote{The training is too expensive, we need to train thousands of models for each task. I am not able to try  3 (\#models) * 5 (\# inference targets) settings. Just random choice, not picking better performing ones. We did not say things like which arch is better or worse.}
%smile detection, and age prediction (involves classes: young, middle aged, and senior people) reusing the knowledge of pre-trained (upstream) models of face recognition (details will be covered later), ImageNet classification, and ImageNet classification models respectively. The attacker aims at inferring whether or not specific individuals are in the downstream set for the all the tasks, and for the smile detection and age prediction, we additionally consider inferring the existence of senior people and Asian people, respectively.

Following common practice in transfer learning, downstream models (``Downstream Model" column in Table~\ref{tab:experimental_setup}) are obtained by reusing and properly modifying (i.e., changing the number of output classes) the latter layers of the upstream model (``Upstream Model" column in Table~\ref{tab:experimental_setup}) while keeping earlier layers (``feature extractor'') of the upstream models unchanged. Note that, since the upstream and downstream models have different numbers of output classes, we 
%cannot simply reuse the parameters of the (final) classification layers/module of the upstream model for the downstream task but need 
also need to reinitialize the modified latter layers and then update during downstream training. The attacker manipulates the ``feature extractor" to boost the effectiveness of property inference. 

\shortsection{Upstream Training} For the gender recognition task, the upstream models are MobileNetV2 trained for facial recognition and classify images of 50 people randomly sampled from the VGGFace2 dataset~\cite{cao2018vggface2}. For smile detection and age prediction, the upstream models are all ImageNet~\cite{deng2009imagenet}\footnote{In the training of the upstream models for inferring the presence of Senior people and Asian people, we delete the facial images inside ImageNet using the annotations provided by Yang et. al.~\cite{yang2021imagenetfaces} because there are no attribute labels indicating whether the persons in those images are senior or are Asian.} classifiers of architecture ResNet-34 and ResNet-18 respectively. As for the inference of the existence of specific individuals, we choose the person who has the most samples in VGGFace2 as the target for both the gender recognition and the age prediction tasks, and choose the person who has the most samples of smile labels (labels provided by MAADFace~\cite{terhorst2021maad}) as the target for the smile detection task. We choose the target property in this manner mainly for convenience in conducting experiments, as the upstream model training, victim model training, and shadow model training (for meta-classifier-based property inference) (ideally) require no overlaps between their training data. More details about the upstream training including the injection of samples with the target property and distribution augmentation are available in Appendix~\ref{sec:extra_dataset_details}.

\fi


% \subsection{Transfer learning tasks}
%We focus on the inference of the inclusion of specific persons in the downstream training, and design three transfer learning tasks to evaluate our attacks. For each of the following tasks, we include the model architecture for the upstream model and the parts of the model reused by the downstream-trainer (with proper adjustments on the number of output classes) in Table~\ref{tab:experimental_setup}. The feature extractor for each of the tasks is thus the part of the model before the downstream classification model. \snote{if we are out of space, it is better to move this architecture }. More details on the datasets are given in Appendix~\ref{sec:extra_dataset_details}.

% The upstream models are trained on a fraction of VGGFace2~\cite{cao2018vggface2} or the whole ImageNet dataset~\cite{deng2009imagenet}.
% And since VGGFace2 provides over 3 million facial images from 9131 subjects and there are also lots of attributes available for those facial images~\cite{terhorst2021maad}, we choose the classification of facial attributes as the downstream tasks, and the downstream training is based on VGGFace2 using the attribute label provided by MAADFace~\cite{terhorst2021maad}.

% \begin{table}[htbp]
%   \centering
%     \begin{tabular}{c|cc}
%     \toprule
%     Task & Upstream Model $g_d(f(.))$ & Downstream Classification Model $g_d(.)$\\
%     \midrule
%     Gender recognition & MobileNetV2~\cite{sandler2018mobilenetv2} & Final Classification Module\\
%     Smile detection & ResNet-34~\cite{he2016deep} & Fourth ResNet block + Classification Layer\\
%     Age prediction & ResNet-18~\cite{he2016deep} & Fourth ResNet block + Classification Layer\\
%     \bottomrule
%     \end{tabular}%
%   \caption{Model architectures for the upstream model, and the corresponding downstream classification models for the tasks described in Section ~\ref{sec:exp_setup}.}
%   \label{tab:experimental_setup}
% \end{table} 

%\shortsection{Gender recognition} The upstream models are trained for facial recognition, classifying images of 50 people randomly-sampled from VGGFace2. To avoid overlap, we also ensure that any images of these 50 people do not appear in downstream training which also uses subsets of VGGFace2. The downstream trainer

%\todo{Shift to Appendix: For each person, we randomly choose 400 samples for training and 100 for testing.}
% To avoid overlap, we also ensure that any images of these 50 people do not appear in downstream training.
% In the model training, we use the MobileNetV2 implementation provided by PyTorch.
% The downstream trainer uses all layers in the MobileNetV2 except the final classification module as the feature extractor, and the downstream classification model is the final classification module with the number of output classes changed, and the initial values of the downstream parameters are reinitialized.

% ,and the samples of that people are all considered as having the target property.
% Since the target person is not in the randomly chosen upstream set, to achieve the attack, we add 342 samples of the target person into the upstream set. 


%\shortsection{Smile Detection}
%The upstream model is
% a ResNet-34~\cite{he2016deep} model,
%trained for object classification on ImageNet~\cite{deng2009imagenet} .
% We choose the ResNet-34~\cite{he2016deep} for the model training and still use the implementation provided by PyTorch.
% The downstream trainer uses all the layers before the fourth ResNet block as the feature extractor and take the fourth ResNet block as well as the final classification layer as the downstream model
%The downstream trainer initializes the last ResNet block with the parameters from the released upstream model.
% Most settings are the same as those in the gender recognition task. And the differences are: first, the upstream training set is ImageNet; second,
% \todo{Shift to Appendix: Since the people with the most samples does not have enough samples with the smile attribute labeled (some samples are labeled as undefined by MAADFace), we use the people with the most samples that have labels about the smile attribute as the target person. The number of samples of the target property injected into the upstream set is 261, and the number of samples for the distribution augmentation is 1305 ($5\times261$).} 

%\shortsection{Age Prediction}
%The upstream model in this case is also trained on ImageNet for object classification.
% The downstream trainers builds the downstream model in the same way as that in the smile detection task.
% Compared to the gender recognition task, the differences of the settings are: first, the upstream training set is ImageNet; second, s


% For the gender recognition task, we select the people who has the most samples in VGGFace2 as the target person, and the samples of that people are all considered as having the target property. For the upstream training of this task, we randomly choose 50 people from VGGFace2, and for each of those people, we randomly choose 400 samples as the training data and 100 samples as the test data. And those 50 people will not appear in the downstream training.  Since the target person is not in the randomly chosen upstream set, to achieve the attack, we add 342 samples of the target person into the upstream set. Also, for the object augmentation, we add 1710 samples (which is 5 times the number of the samples of the target person added into the upstream set) of other person that do not appear in the upstream and also will not appear the the downstream training as the augmentation data.

% Shifted to Appendix
% \shortsection{Detailed}
% % 1. How to select the target person (may be important to the experiments but not that interesting to the reader)
% % 2. How to choice the samples for distribution aug (surely not important and less interesting)
% \todo{Shift to Appendix: We select people with the most samples in VGGFace2 as the target person.}

% The upstream 
% For each person, we randomly choose 400 samples for training and 100 for testing.To avoid overlap, we also ensure that any images of these 50 people do not appear in downstream training. \todo{Shift to Appendix: For the distribution augmentation of the upstream training, we add 1710 samples ($5\times 342$) to the upstream set, and those added samples are from individuals that are not in the original upstream training set nor in the downstream training set. \anote{Not sure what this means}}

% \todo{Shift to Appendix: Smile since the people with the most samples does not have enough samples with the smile attribute labeled (some samples are labeled as undefined by MAADFace), we use the people with the most samples that have labels about the smile attribute as the target person. The number of samples of the target property injected into the upstream set is 261, and the number of samples for the distribution augmentation is 1305 ($5\times261$).} 

% \todo{Shift to Appendix: Since there are not enough samples that are clearly labeled for age prediction (a lot of undefined labels), we reduce the number of samples without the target property in the candidate sets to 165915.}

% \begin{figure*}[ht]
%     \centering
%     \includegraphics[width=1\linewidth]{fig/attack/bn_baseline_inference_5000_10000.pdf}  
% \caption{Inference AUC scores when the upstream model is trained benignly. For the meta-classifier-based inferences, we report average AUC values and standard deviation over 5 runs of meta-classifiers with different random seeds. The downstream training sets have 5,000 samples in the results in the first row and 10,000 samples for the second row. Since there are no activation manipulations, we can only use the inference methods that do not consider target parameters (confidence score test, white-box meta classifier, and the black-box meta classifier). For smile detection and age prediction, the inference targets are senior people and Asian people respectively; inference of specific individuals show a similar trend (Figure~\ref{fig:baseline_results_t_individual}).
% % \dnote{bars showing standard error over K executions?} \ynote{5 meta classifiers trained with different random seeds
% %\anote{Cyan blue color may be a bit hard to read. We should also try and make the lines different (with dashes or something) to make it color-blind friendly. Same comment for the other figures as well.}
% }
% \label{fig:baseline_results}
% \end{figure*}

%\dnote{can Table 1 include the datasets also? they are different ones from upstream and downstream, should be in the table (but can condense text in the table to have room for this, and avoid including text that is the same for all scenarios ("Downstream Model") to make room to put more interesting information in the table} \ynote{the downstream set are all subsets of VGGFace2; not sure how to include the information of the subsets in Table1}

\iffalse
\shortsection{Downstream Training}
VGGFace2 provides over 3 million facial images from 9,131 subjects, with multiple attributes available for each image~\cite{terhorst2021maad}. We choose the classification of facial attributes (i.e., gender, smile, and age) as the downstream tasks, with attribute labels provided by MAADFace~\cite{terhorst2021maad, terhorst2019reliable} for downstream training.

To generate the downstream training set, we first prepare over %\dnote{don't understand the "over" here - isn't it a fixed number that we control?} \ynote{for some settings, we do not have enough labeled downstream training samples, the maximum size is 200 000}
120,000 (the maximum number is 200,000) randomly selected samples without the target property and over 250 (the maximum number is 1,000) samples with the target property and form the downstream candidate set. Details of the candidate set are in Appendix~\ref{sec:extra_dataset_details}.
Then, a downstream training set of size $n$ is generated by randomly sampling from this candidate set while also specifying the number of samples with target property as $n_t$. For experiments in this section, we consider settings where $n=5,000$ or $10,000$, and $n_t$ takes value from $\{0, 1, 2, 3, 4, 5, 10, 20, 50, 100, 150\}$ (this gives  $2\times11=22$ different settings). We train a total of 32 downstream models with different random seeds for each setting to report error margins. 
% \dnote{? do you mean stable, or we use these to report error margins?} \ynote{Stable. Error margins are reported based on the repeats of the inferences} results.

To train the meta-classifier attacks, the attacker needs to train many downstream shadow models and thus, we also prepare a separate downstream candidate set with the same size as the victim's downstream candidate set but without any overlaps on the data. This simulates the most difficult and realistic scenario for the attacker. We also ensure that no samples in the two downstream candidate sets appear in the upstream training set, which again makes the attack more difficult. To simulate the victim's downstream training, we assume the attacker also uses a downstream training set of size $n$, but has no overlap with the actual victim's downstream training set. In Appendix~\ref{sec:unknown_training_size}, we relax this assumption and show our attack retains its effectiveness even when the size of the victim's downstream training dataset is unknown to the adversary. %, and this is consistent with the results in the existing work~\cite{suri2022formalizing}.
%We assume the attacker knows the downstream training set size $n$ of the victim (this assumption is relaxed in Appendix~\ref{sec:unknown_training_size}, but there is almost no impact on attack performance), but does not know the exact value of $n_t$. \anote{I still feel we should focus on having similar performance, rather than knowledge of the exact dataset size. It will confuse readers and is not very relevant to the meta-classifier performance.}
% \dnote{move this to the section that does these experiments: }
% Then, for each setting with fixed $n$, the attacker trains 320 downstream models (256 for training, 64 for validation) for each of the distributions (with and without target property). The number of training samples with the target property for each model is randomly selected from the range $[1,170]$, which simulates the scenario where the value of $n_t$ of the victim downstream model cannot be accurately guessed.

To the meta-classifier training, for each setting with fixed $n$, the attacker trains 320 downstream models (256 for training, 64 for validation) for each of the distributions (with and without target property). The number of training samples with the target property for each model is randomly selected from the range $[1,170]$, which simulates the scenario where the value of $n_t$ of the victim downstream model cannot be accurately guessed.

\fi

%same number of randomly selected samples as the attacker's candidate set. Note that attacker's downstream candidate set have no overlap with the downstream trainer's candidate set, and also the samples in the two downstream candidate sets will not appear in the upstream training. \anote{A bit confusing for readers that may not have context, may want to paraphrase.}
% In each downstream training, the downstream trainer randomly select a given number of samples as the training set. Specifically, we
%We consider cases where the total number of samples is 5000 and 10000 separately and the number of training samples with the target property are 0, 1, 2, 3, 4, 5, 10, 20, 50, 100, and 150 separately. We train 32 models with different random seeds for each setting.
% For the attacker's shadow model training, in the main study we assume the attacker uses the same downstream training size as that used by the downstream trainer. \anote{For a meta-classifier, should it really matter that the training sizes are the same? I don't think so, and mentioning this here explicitly may confuse readers that are not familiar with the meta-classifier approach.}
%For each downstream size setting, the attacker train 320 models (256 for training and 64 for validating) without the target property and train other 320 models with the number of samples with the target property randomly selected from $[1, 170]$ (This range is slightly bigger than the range $[1, 150]$ used by the downstream trainer).
% We also look at the case where the attacker trains downstream shadow models using a training size different from that used by the downstream trainer in Section~\ref{sec:unknown_training_size}, and the results show that the inference is robust to this factor. \anote{Why should we expect it to be different? As long as the shadow models are similar in performance, meta-classifiers should be unaffected by how many samples were used to train the shadow models. It might be more interesting to look at the performance of the shadow models and show that they need not be very high performant (if that is the case), instead of talking about the explicit dataset size.} \ynote{we need to set a value for the size of the dataset, so what do you suggest on this?}

\begin{figure*}[ht]
    \centering
    \includegraphics[trim={0.2cm 0cm 0.2cm 0cm}, clip,width=.8\linewidth]{fig/attack/bn_zero_activation_inference_with_baseline_10000.pdf}
\vspace{-0.3cm}
\caption{Inference AUC scores when the upstream model is trained  %\dnote{not about the attack goals here, how is it trained?} 
with the attack method described in Section~\ref{sec:attack_design}.
Baseline scores (\emph{Baseline}) are the maximum AUC scores %(of the three inference methods) 
of the baseline experiments where the upstream models are not manipulated. % in  Section~\ref{sec:eval_baseline}. 
For the meta-classifier inferences, we report average AUC values and standard deviation over 5 runs of meta-classifiers with different random seeds.
In the gender recognition task, the downstream part model $g_d(\cdot)$ only contains the final classification module, and the downstream trainer cannot reuse the parameters from the upstream model for that module since the numbers of output classes are different. Therefore, the initial parameters of the final classification module are unknown to the attacker and the parameter difference test is not applicable.
%The inference targets for the smile detection and age prediction are senior people and Asian people respectively. The inference of specific individuals for those two tasks are similarly successfully and found in Figure~\ref{fig:zero_activation_attack_results_t_individual} in the appendix. 
The inference of specific individuals for smile detection and age prediction are similarly successfully (Figure~\ref{fig:zero_activation_attack_results_t_individual} in the appendix).  %and found in Figure~\ref{fig:zero_activation_attack_results_t_individual} in the appendix. 
The downstream training sets contain 10$\,$000 samples and inference results of 5$\,$000 samples are similar and given in Figure~\ref{fig:zero_activation_attack_results_5000} in the appendix. %\anote{This information about 10K v/s 5K in Appendix is repeated in all figure captions. Maybe we can just mention it once somewhere in the prose?}\snote{yes, only mention it once.}
}
\label{fig:zero_activation_attack_results}
\vspace{-0.5cm}
\end{figure*}
%\dnote{The attacks in Fig 1 should be ordered in some sensible way. I would suggest by increasing threat model adversary strength, and then, by better attack in each category: baseline, confidence score, black-box meta classifier, white-box metaclassifiers, variance test, difference test.}
%\dnote{I don't understand the explanation what the difference test is missing from Gender recognition - why isn't it possible to avoid the reinitialization for this one, but done for the others?}
% \dnote{I also don't understand what the target properties are for the 3 scenarios (should make this more clear in the axis labels, and how they were selected). Would make sense to me to do:
% Gender Recognition / Property: senior
% Gender Recognition / Property: Asian
% Smile Detection / Property: senior
% Smile Detection / Property: Asian
% Age Prediction / property: Asian (I guess senior doesn't apply here since it is related to task)
% this is 5 scenarios, but talk about 6 or 8, still don't understand what they are, or why these ones are in the figure. Really need the table to show all the scenarios considered and summarize the results. I think there are also some scenarios where property is a specific individual, but not sure where these are included in results.}


% For the smile detection task, most settings are the same as those in the gender recognition task. And the differences are: first, the upstream training set is ImageNet; second, since the people with the most samples does not have enough samples with the smile attribute labeled (Some samples are labeled as undefined by MAADFace), we use the people with the most samples that have labels about the smile attribute as the target person; last, the number of samples of the target property injected into the upstream set is 261, and the number of samples for the object augmentation is 1305 ($5\times261$). 

% As for the age prediction task, compared to the gender recognition task, the differences of the settings are: first, the upstream training set is ImageNet; second, since there are not enough samples that are clearly labeled for age prediction (a lot of undefined labels), we reduce the number of samples without the target property in the candidate sets to 165915.

% \shortsection{Metrics} We report AUC scores for distinguishing between models trained with and without the target property.
% % discriminating the 32 models trained without the target property from the 32 models trained without the target property to show the performance of the inference.
% For meta-classifier related inferences, we will report average AUC values over 5 runs of meta-classifiers with different random seeds, along with standard deviation.

\shortsection{Attack Evaluation Metric}
% To avoid arbitrarily choosing the threshold, we 
We use the Area Under Curve (AUC) score for evaluating attack effectiveness in distinguishing released downstream models (by the victim) with and without the target property.
%A random-guess adversary would thus achieve an inference AUC score of 0.5. %For experiments with poor inference performance, we observe AUC scores to be normally distributed around 0.5, with high variance.


\section{Evaluation of Attack Effectiveness}\label{sec:emp_eval} \label{sec:eval_zero_activation_attack}

%\autoref{fig:zero_activation_attack_results} summarizes our experimental results. Baseline results (solid dark) in Figure~\ref{fig:zero_activation_attack_results} are chosen as the maximum of AUC scores from the three inference methods in \autoref{fig:baseline_results}. 
%\dnote{why is this pointing to a later figure? should be able to cut (move to appendix) the who baseline section 6.2, and just explain in a sentence what the baseline results are enough for readers to be convinced they are valid, and refer to appendix for details}

  \autoref{fig:zero_activation_attack_results} summarizes our results. The solid dark lines (\emph{baseline} lines) in the figure show the inference AUC scores when the upstream models are trained normally (we report the best results of all tested attacks). More details of the baseline experiments can be found in Appendix~\ref{sec:eval_baseline}. Hyperparameter settings for the experiments can be found in Appendix~\ref{sec:hyper-parameter-setup-zero-activation} and the results are insensitive to the selection to hyperparameters.
  
  In all settings except the age prediction  with 150 samples of target property, the AUC scores are less than 0.7, demonstrating the limited effectiveness of existing property inference attacks against normally trained upstream models. 
%For these experiments, since there are no manipulated activations and parameters, we can only use the inference methods that are not directly related to the manipulation (i.e., confidence score test, black-box meta-classifiers, and white-box meta-classifiers) and report the maximum of the AUC scores of these three inference methods in the solid dark lines (Details are in Appendix~\ref{sec:eval_baseline}). 
%We observe that all the attacks have inference AUC scores less than 0.7 except the age prediction setting when 150 samples are with the target property (AUC score is 0.77).  These results demonstrate the limited effectiveness of existing methods applicable to normally trained upstream models.
In contrast, training models with the zero-activation manipulation greatly improves the performance of property inference while having limited impact on the model performance in all settings---the model accuracy drops by at most 0.9\% (see \cref{sec:impact_to_upstream_accuracy} for detailed results on the impact of the activation manipulation to the upstream and downstream accuracies).
% Avoid using footnotes, except for humor. O was going to move this to the caption, but it seems to already be there. \footnote{In the gender recognition task, the downstream trainer reinitializes the classification module. Thus, the attacker does not know the initial target parameters and cannot use the parameter difference test.} 
Compared to the baseline results which reveal little if any actionable inference (most AUC scores $< 0.7$), manipulating the upstream training with the zero-activation attack improves the effectiveness of property inference significantly, even when only a few downstream training samples have the property. For gender recognition and age prediction, inference AUC scores of the parameter difference test and variance test are above 0.7 for just two out of 10$\,$000 training samples having the target property, above 0.9 for 10 training samples, and exceed 0.95 for $\geq20$ training samples. The one exception also has AUC scores exceed 0.9 for $\geq20$ training samples. %\dnote{hard to decode this - just add the table, and then will be easier to explain what you are pointing out}


\shortersection{Black-box attacks}
The black-box meta-classifier achieves inference AUC scores above 0.9 when $\geq50$ out of 10$\,$000 training samples have the target property. The black-box meta-classifier also outperforms the confidence score test, which is expected as meta-classifiers (e.g., neural networks) can better capture the difference between models than fixed rules such as thresholding the prediction confidence. 
%; for the two settings of smile detection, the AUC scores of the former exceed 0.85 when $\geq 20$ samples have the target property, while the AUC scores are always lower than 0.6 for the latter. 
%The reason of superior performance of the meta-classifier approach is, it can better capture the difference between models trained with and without the target property from more angles while confidence score test only relies on the model prediction scores. 
% \dnote{don't know what you mean by "angles" here, or if this sentence is adding anything - do we really know the reason, or this is just describing the attacks again?} \dnote{the main point this section should be making is that the black-box attacks are very effective. It is hard to see this on Fig 1, since they are grouped in a funny way.} 

\shortersection{White-box attacks}
Our white-box methods (the parameter difference test and the variance test) also achieve AUC scores $>0.9$ when $\geq20$ training samples are with the target property.
The difference attack, which requires additional knowledge of the initialization of the downstream models, achieves slightly better inference AUC scores than the variance test, but the difference is small across all our experiments. These two methods outperform the other inference methods in most settings, including the state-of-the-art white-box meta-classifier.
%and of course all the black-box attacks.

\shortersection{White-box meta-classifier vs. Black-box meta-classifier}
%Interestingly, the black-box meta-classifier attack works better than the white-box meta-classifier attack on some settings.
For smile detection and age prediction, 
%when there are two or more downstream training samples are with the target property, 
the black-box meta-classifier surprisingly achieves higher AUC scores than the white-box meta-classifier attack. A possible reason for this is that the white-box attack mainly uses the fully-connected layers~\cite{ganju2018property,suri2022formalizing} and hence, performs worse when the updatable downstream module also contains convolutional layers (adapting this attack to convolutional networks was not very successful). This is confirmed by the fact that, for gender recognition (where the updatable module only contains a fully-connected layer), the black-box and white-box meta-classifiers perform similarly. %\snote{we know the meta-classifier approach is adapted to conv layers, but need to explain in deeper level why this is not the case.} \ynote{how about we just cite the two papers, the original white-box meta-classifier and the one that adpat it to conv layers}

\shortersection{Attacks of AUC scores $< 0.5$}
When the performance of an inference attack is poor, it is expected to have AUC scores near 0.5 (close to random guessing). However, we find that there are few attack settings with AUC scores consistently below 0.5.
%but in Figure~\ref{fig:zero_activation_attack_results} we have unintuitive observation that the AUC scores of some attacks in Figure~\ref{fig:baseline_results} and Figure~\ref{fig:zero_activation_attack_results}, some attacks have AUC scores (almost) consistently below 0.5. 
%(e.g., %the confidence score test for the smile detection task with 5,000 downstream samples in Figure~\ref{fig:zero_activation_attack_results_5000} in the appendix, and the confidence score test and black-box meta-classifier for the gender recognition with 10$\,$000 downstream samples in Figure~\ref{fig:baseline_results} in the appendix).  
Appendix~\ref{sec:auc<0.5} discusses those anomalies and surmises that they are caused by the limitations of original inference methods designed for normal pretrained models when facing challenging inference tasks.





%\dnote{moved this down for now to get to the results - should just start with Figure 1 here, and move these details to appendix, and just mention what is necessary where it is needed to understand the results}


\iffalse
\shortsection{Hyperparameter Setup}
We evaluate the performance of the zero-activation attack (Section~\ref{sec:zero_activation_attack}) when coupled with different inference methods (Section~\ref{sec:zero_activation_inference}).
We set $\alpha$ and $\beta$ to 1, treating all loss terms equally. We tried different settings on $\alpha$ and $\beta$, as well as methods that automatically set them~\cite{sener2018multi}, but no significant improvements are observed, so we only report the results for the simplest choices. 
% We set the hyper-parameters $\alpha$ and $\beta$ in Equation~\ref{eq:zero_activation_loss} to 1 and we find that treating all loss terms equally is sufficient to produce effective attacks for all settings.
%\anote{Could move these details to the Appendix}
We also tested different values for $\lambda$ and $\boldsymbol m$, but did not observe significant differences in the attack effectiveness, suggesting our attack is not sensitive to hyper-parameters.
Details of experiments on different combinations of $\lambda$ and $\boldsymbol m$ are deferred to Appendix~\ref{sec:hyper_params}. For the results here, we select $\lambda$ values that are big enough while ensuring the upstream model accuracy is not impacted significantly ($\lambda = 10$ for smile detection and age prediction, and $\lambda = 5$ for gender recognition). For $\boldsymbol m$, for gender recognition, we select the first 16 activations of the total 256 activations. For smile detection and age prediction, since the first layer of downstream model is convolutional, we can only select activations at the granularity of channels, and we choose to manipulate the first channel of the total 256 channels. 
\iffalse
For these experiments, we also use distribution augmentation described in Section~\ref{sec:attack_design} in the upstream training; ablation studies (Section~\ref{sec:study_distribution_aug}) suggest it is crucial for performance.
\fi
\fi
% \section{Defenses and Stealthier Attacks} 

\section{Stealthier Manipulation}
\label{sec:possible_defense}
\iffalse
We assume the victim/defender knows the attack method used for training the pretrained models, but not the specific selected secreting activations, which is usually a secret to the individual adversary. We also assume the number of samples with target property is limited in the downstream training set, as such properties are less likely to be revealed by the normally trained models and are practically more interesting for the attackers. As mentioned in \autoref{sec:threat_model}, we also assume the victim/defender is also unaware of the property targeted by the adversary, but knows the target property has a limited presence in its downstream training set. 
\fi

% \begin{figure*}[ht] %[htbp]
%     \centering
%     % \includegraphics[width=.85\linewidth]{fig/defense/bn_clustering.pdf} 
%     \includegraphics[trim={0.2cm 0cm 0.2cm 0cm}, clip,width=.8\linewidth]{fig/defense/bn_summarize_clustering.pdf} 
% \caption{Percentage of samples with the target property detected by the best anomaly detection method. 
% Similar to Hayase et al.,~\cite{hayase2021spectre}, we filter out $n \times 1.5$ samples with anomaly detection, where $n$ is the number of samples in downstream training data with the target property. \emph{Detection Percentage} is the number of samples with the target property being filtered out divided by $n$ and report the \emph{maximum} of the detection percentage of the three anomaly detection methods (detailed results are in Figures~\ref{fig:anomaly_detection_zero_activation_attack},~\ref{fig:anomaly_detection_stealthier_attack} in the appendix).
% %values are averaged (with standard deviation) over 5 runs of anomaly detection.
% The `5K' lines report detection results on settings with 5,000 total samples, while `10K' lines report for 10,000 total samples. 
% % Inference targets for smile detection and age prediction are senior people and Asian people respectively; results for specific individuals follow similar trends (Figure~\ref{fig:anomaly_detection_stealthier_attack_t_individual} in the appendix).
% Detection results for inferring specific individuals for smile detection and age prediction are similar (Figure~\ref{fig:anomaly_detection_stealthier_attack_t_individual} in the appendix).
% \label{fig:anomaly_detection_zero_activation_attack_stealthier_attack} }
% \vspace{-0.3cm}
% \end{figure*}


The attack described in Section~\ref{sec:attack_design} introduces obvious artifacts in the pretrained model, which can be utilized for detection by a downstream model trainer aware of the risks posed by our attacks. We first present two detection methods
% that can easily detect the original zero-activation attack 
(Section~\ref{sec:detect_pretrained}) and then demonstrate how to make the model manipulation stealthier to evade detection while still preserving the inference effectiveness (Section~\ref{sec:stealthier-design-methods} and Section~\ref{sec:stealthier-design-results}).
We assume the downstream trainer is aware of the possibility of the attack and its design, but does not know the property targeted by the adversary, as this is specific to an attacker's goal and the set of possible properties can be exponentially large for a rich training set. 


%\subsection{Possible Defenses} \label{sec:possible_defense}

%\dnote{need to condense this a lot, and make some room to actually include results in the paper body}

\subsection{Detecting Manipulated Pretrained Models}\label{sec:detect_pretrained}
We present two detection methods that use the distributional difference between activations of samples with and without property. 
%\dnote{I think we are only considering ways to detect a bad pretrained model, not other types of defenses?}

%design nature of the attack, and is a proof-of-concept defense, as the actual application of this defense may require many engineering efforts and domain knowledge (e.g., setting proper threshold for individual tasks) and is out of the scope of this paper. \dnote{this is a very strange thing to say - is it really just that we didn't have time to test it? Is this talking about the "Activation Distribution Checking" paragraph or something else?} 
%The second type of defense is based on anomaly detection and can be directly used as a practical defense.
%\dnote{I don't understand what the "two types" of defenses are. The most obvious kinds of defenses are (1) ones that check the pretrained model for manipulation, and (2) ones that do something in the tuning/deployment process to mitigate the property inference. I'm not sure what the "two types" we are considering are though, so should state at a high level the general defense types, and then the specific defenses we consider and evaluate.}

\shortsection{Checking the Distribution of Activations}
Since the distributional difference between activations of samples with and without target property is significant, this defense focuses on spotting this difference to identify manipulated models. A method to identify the distributional difference needs to be designed based on the attack method used. For the original zero-activation attacks in Section~\ref{sec:zero_activation_attack}, since the secreting activations of samples without property are all 0, the defender can feed random training samples to the pretrained models and check if there are abnormally many 0s. This approach is feasible since samples of target property have limited presence in the downstream training set and hence, most samples will not have the property. 
Since detecting the zero-activation attack is trivial using this method, we do not conduct any experiments with this. %perform actual experiments and assume a stealthier attack is needed. 

% The design of more involved attacks that no longer rely on generating abnormally many 0s and also the corresponding distribution checking detection for the involved attacks is given in~\cref{sec:stealthier-design-methods}.

%the defense may trivially detect the manipulated pretrained models, as they will produce abnormally many 0s if the attacker feeds some random training samples without the target property to the models. The likelihood of finding random samples without the property is high because samples of target property have limited presence in the downstream training set. Such defense becomes more interesting when the attacker no longer relies on generating zero activations for samples without target property and more details is in~\autoref{sec:stealthier_design}. 
%If the attack method is made public, the downstream trainer can directly inspect outputs of the feature extractor to see if the distribution of the activations is unusual. For example, for the zero-activation attack, the victim can check for an abnormal number of zeros in the outputs of the feature extractor to identify the  attack. For the zero activation defense, we believe this defense can be effective as manipulated upstream model will produce many more abnormal zeros in the activations, compared to normally trained upstream models.
%\dnote{if this is all we have to say about this, I don't think we can consider it a "Defense" that was evaluated. I thought you had done some experiments on this?}


\shortsection{Anomaly Detection} 
%In general, adversaries are more interested in inferring properties with limited presence in the downstream tasks, as those are less likely to be revealed by normally trained models. 
%Since the target property has limited presence in the downstream training set, a trivial yet effective defense would thus be to remove all samples with the target property from downstream data while having negligible impact on downstream model performance.
Since the target property has a limited presence in the downstream training set, another defense would be treating samples with the target property as outliers and then analyzing those outliers to find manipulations.
% Therefore, an effective defense is to identify and remove the small number of samples with the target property from the downstream set and train models on the rest.
% When the target property is known to the downstream trainer, then the defense is trivial, but in
%In practice, it is impossible for the victim to know the target property beforehand as it is specific to an attacker's goal. 
%\dnote{?? I don't see why this is impossible - we are just assuming that the victim does not know the property the adversary might care about, but there are exponentially (?) many properties the victim would care about an adversary learning?}
%Although the target property is unknown to the defender, we can still design a successful defense by leveraging expected distributional differences between samples with and without a property that some adversary may want to target. 
% Anomaly detection methods~\cite{jain1999data,abdi2010principal,hayase2021spectre} aim to identify the small fraction of outliers that are distributed differently from other normal data points.
Existing anomaly detection methods~\cite{jain1999data,abdi2010principal,hayase2021spectre} can be adapted to detect manipulated pretrained models in our setting because: 1) the number of samples with the property is of small fraction and 2) their activation distribution is significantly different (i.e., outliers) from the distribution for samples without the property. The auditor can inspect model activations for all of its training data and identify outliers (ideally, samples with target property) with anomaly detection. The auditor can then inspect identified outliers and may find commonalities to identify the potential target property. For instance, they may find that a small fraction of the training data produce unusual model activations, and then notice that most of that data has a particular property such as belonging to a specific individual or group. %In this paper, we consider three anomaly detection methods: PCA~\cite{abdi2010principal}, K-means~\cite{jain1999data} and Spectre~\cite{hayase2021spectre}, which is the current state-of-the-art. Our experiments suggest that these defenses are indeed effective (Appendix~\ref{sec:defense_details}).

We consider three common anomaly detection methods: K-means~\cite{jain1999data}, PCA~\cite{abdi2010principal} and Spectre~\cite{hayase2021spectre} (where Spectre is the current state-of-the-art) and we report the detection results from the three defenses. Appendix~\ref{sec:defense_details} gives details of these methods. The detection results on the zero-activation attack are given in %Figure~\ref{fig:anomaly_detection_zero_activation_attack_stealthier_attack}  
Figure~\ref{fig:anomaly_detection_zero_activation_attack} in the appendix. Anomaly detection is very effective at identifying the samples with target property. For example, for the gender recognition and smile detection tasks, the detection rate is over 80\% in most cases. 
%Zero-activation attacks are easily detected because the attack signatures of target samples are too strong after upstream manipulation and
These results motivate the design of stealthier attacks which we describe next.
%introduce in Section~\ref{sec:stealthier-design-methods}.
%The results show that conducting anomaly detection can filter out majority of samples with the target property in the downstream set and hence, lower the effectiveness of the inference. For example, the Spectre defense can filter out 80\% of the samples with the target property in most cases for gender recognition and smile detection, and 60\% for age prediction. Anomaly detection detects zero activation attacks because the attack mainly focuses on improving attack effectiveness by increasing the distinction between samples with and without property, which makes the attack signature of samples with property much stronger. 


\iffalse
K-means leverages k-means clustering technique to identify outliers while PCA leverages principal component analysis to identify the outliers. Spectre is an improved version of PCA and works much better than PCA when the attack signature is weak (i.e., the distributional difference is small)~\cite{hayase2021spectre}.

When conducting the anomaly detection, following the common setup in Hayase et al.,~\cite{hayase2021spectre}, we filter out $1.5 n_t$ ($n_t$ is the number of samples with target property) samples, simulating the scenario where the defender does not know the exact $n_t$, but is able to roughly estimate its value and attempt to filter out most of them.
\fi

% \begin{figure*}[htbp]
%     \centering
%     \includegraphics[width=.85\linewidth]{fig/defense/bn_clustering_zero_activation.pdf} 
% \caption{Percentage of samples with the target property detected by the anomaly detection for the zero-activation attack. Total number of filtered samples are $1.5$ times the number of actual target samples and the detection percentage is the fraction of actual target samples that are filtered out. We repeat the experiments for 5 times and report the average value and standard deviation.
% The `5K', `10K' lines report for settings of downstream training set of 5000 and 10 000 total samples.
% \label{fig:anomaly_detection_zero_activation_attack} }
% \end{figure*}

%\shortsection{Results of Anomaly Detection} 
%\label{sec:defense-zero-activation}
%We show the detection performance in Figure~\ref{fig:anomaly_detection_zero_activation_attack}. 
%The results show that conducting anomaly detection can filter out majority of samples with the target property in the downstream set and hence, lower the effectiveness of the inference. For example, the Spectre defense can filter out 80\% of the samples with the target property in most cases for gender recognition and smile detection, and 60\% for age prediction. Anomaly detection detects zero activation attacks because the attack mainly focuses on improving attack effectiveness by increasing the distinction between samples with and without property, which makes the attack signature of samples with property much stronger. 

% These results suggest that
%Since the original zero-activation attack can be easily defended, more investigation is needed to find out whether more sophisticated attacks that exist and evade the proposed detections. 
%Next, we propose stealthy attack designs (Section~\ref{sec:stealthier-design-methods}) and show that they can circumvent detection while achieving high inference AUC scores (Section~\ref{sec:stealthier-design-results}).


% \begin{figure*}[ht] %[htbp]
%     \centering
%     % \includegraphics[width=.85\linewidth]{fig/defense/bn_clustering.pdf} 
%     \includegraphics[width=.85\linewidth]{fig/defense/bn_summarize_clustering.pdf} 
% \caption{Percentage of samples with the target property detected by the anomaly detection. 
% Similar to ~\cite{hayase2021spectre}, we filter out $n \times 1.5$ samples with anomaly detection, where $n$ is the number of samples in downstream training data with the target property. We define the number of samples with the target property filtered out divided by $n$ as the \emph{Detection Percentage} and  report the \emph{maximum} of the detection percentage of the three anomaly detection methods (detailed results are in Figures~\ref{fig:anomaly_detection_zero_activation_attack},~\ref{fig:anomaly_detection_stealthier_attack} in the appendix).
% %values are averaged (with standard deviation) over 5 runs of anomaly detection.
% The `5K' lines report detection results on settings with 5,000 total samples, while `10K' lines report for 10,000 total samples. Inference targets for smile detection and age prediction are senior people and Asian people respectively; results for specific individuals follow similar trends (Figure~\ref{fig:anomaly_detection_stealthier_attack_t_individual} in the appendix).
% \label{fig:anomaly_detection_zero_activation_attack_stealthier_attack} }
% %\vspace{-1em}
% \end{figure*}


\subsection{Stealthier Model Manipulation} \label{sec:stealthier-design-methods}
%\anote{Can be shortened- taking up a lot of space in main paper.}
%\dnote{I'm still confused about the "two types of attacks" - in parts it seems like there are two detection methods we are considering, and include here, but setup said only one is considered - I don't have enough understand what is going on here. If there are two detection methods, we should state this clearly, and keep it consistent throughout.}

\begin{figure*}[ht]
    \centering
    \includegraphics[trim={0.2cm 0cm 0.2cm 0cm}, clip,width=.80\linewidth]{fig/attack/bn_stealthier_attack_10000.pdf} 
\vspace{-0.3cm}
\caption{Inference AUC scores of the stealthier design. Since the secreting activations are no longer zero, the inference methods based on difference or variance tests are no longer applicable. The inference results of specific individuals for smile detection and age prediction also show similar improvement compared to the baseline settings (Figure~\ref{fig:stealthier_attack_results_t_individual} in the appendix).
%Inference targets for the smile detection and age prediction are senior people and Asian people respectively; inference of specific individuals also shows improvement compared to the baseline settings (Figure~\ref{fig:stealthier_attack_results_t_individual}; Appendix). 
The downstream training sets contain 10$\,$000 samples and inference results results of 5$\,$000 samples are similar and given in Figure~\ref{fig:stealthier_attack_results_5000} in the appendix.
\label{fig:stealthier_attack_results_10000}}
\vspace{-0.5cm}
\end{figure*}

% \shortsection{Evading Activation Distribution Checking} 
To evade the defense that checks the distribution of activations, we modify our zero-activation attack to ensure: (1) secreting activations for samples without the property are also non-zero (bypassing simple defense of checking abnormal zeros); (2) secreting activations of samples with and without target property are still distinct (the attack is still effective); (3) that distinction between activations should not be captured by anomaly detection methods (evading anomaly detection); (4) the actual distribution of activations that matches the attacker's goal cannot be easily guessed by the defender (handling cases when the defender actively searches other patterns in the distribution of activations).


For (1) and (2), we adapt the loss in  Equation~\ref{eq:x_w_wo_activation} as \vspace{-0.2cm}
\begin{equation}
\begin{cases}
 \;\alpha \cdot \max(\Vert f(\boldsymbol x) \circ  \boldsymbol m\Vert - \Vert f(\boldsymbol x) \circ \neg \boldsymbol m \Vert, 0)  &\text{if $y_t=0$}\\
 \;\beta \cdot \max(\lambda \cdot \Vert f(\boldsymbol x) \circ \neg \boldsymbol m  \Vert - \Vert f(\boldsymbol x) \circ \boldsymbol m \Vert, 0) &\text{if $y_t = 1$}
\end{cases}
\label{eq:x_w_wo_activation_stealthier}
\vspace{-0.2cm}
\end{equation} 
where $\lambda \geq 1$. (1): The case of $y_t=0$ is redefined to bypass the detection of abnormal zeros. Minimizing this new loss ensures that samples without the target property will have secreting activations ($f(\boldsymbol x) \circ  \boldsymbol m$) with (close-to-normal) non-zero values. 
%(1): to bypass the detection of abnormal zeros, we first redefine the loss in Equation~\ref{eq:x_w_wo_activation} for the case of $y_t=0$ as $\alpha \cdot \max(\Vert f(\boldsymbol x) \circ  \boldsymbol m\Vert - \Vert f(\boldsymbol x) \circ \neg \boldsymbol m \Vert, 0)$. Minimizing this new loss ensures that, for samples without target property, the secreting activations ($f(\boldsymbol x) \circ  \boldsymbol m$) now also have (close-to-normal) non-zero values. 
(2): to ensure the property is still detectable, 
we actively increase the difference between the secreting activations of samples with and without property.
We observe that, for upstream models with reasonable performance on the main task, non-secreting activations ($f(\boldsymbol x) \circ \neg \boldsymbol m$) have similar amplitude regardless of the fed samples containing target property. 
Therefore, for samples with target property, as long as we ensure the secreting activations have a larger amplitude than that of non-secreting activations, there will be a distinction between secreting activations of samples with and without property. We do this by assigning larger values to $\lambda$ (e.g., $\lambda \geq 1$, instead of the original $\lambda > 0$) for the second line of Equation~\ref{eq:x_w_wo_activation_stealthier} to induce sharper distinction between samples with and without property and enable higher inference performance. 

To prevent detection by anomaly detectors (requirement (3) above), $\lambda$ should be set to balance the attack effectiveness and stealthiness rightly. By choosing proper values for $\lambda$, our attack is able to evade anomaly detection methods in most settings. However, in some settings (mostly in gender recognition tasks), state-of-the-art anomaly detection (Spectre) can still identify most of the samples with target property. %The main reason for this detectability is that for these settings the attack signature of samples with target property is still strong and is easily distinguishable from samples without the property~\cite{hayase2021spectre}.
To counter this, we add an additional regularization term (weighted by parameter $\gamma$) to the overall loss function $l(\boldsymbol x, y, y_t)$ in \autoref{eq:zero_activation_loss} that further improves attack stealthiness while still maintaining relatively high attack effectiveness. Specifically, we first obtain the corresponding covariance matrices of the activations of samples with the target property ($\boldsymbol{cov_w}$), activations of all samples with and without the target property ($\boldsymbol{cov_{w,wo}}$), and activations of samples without the target property ($\boldsymbol{cov_{wo}}$) respectively. Then, we encourage $mean(\boldsymbol{cov_w)} = mean(\boldsymbol{cov_{w,wo}}) = mean(\boldsymbol{cov_{wo}})$ and $var(\boldsymbol{cov_w}) = var(\boldsymbol{cov_{w,wo}}) = var(\boldsymbol{cov_{wo}})$ (both $mean(\cdot)$ and $var(\cdot)$ treat the whole covariance matrix as a flattened array and return scalar values) for the three covariance matrices by minimizing their differences in their mean and variance. Using this method, we ensure the distributions of activations of samples with target property will be similar to the ones without the property, making the manipulations harder to detect. We use this approach for all the experiments.
%For consistency, we use this modified approach for all of our experiments (including settings where proper value of $\lambda$ can also already evades detection).
%We also add a regularization term to the loss, leveraging the covariance matrix of the samples with and without the property, to further ensure the distribution of activations of samples with and without target property are similar (Details are deferred to Appendix~\ref{sec:further-evade-anomaly-detection}). 
%Choosing an appropriate value of $\lambda$ will also help to evade the anomaly detection methods in most cases, but with a few exceptions. Handling those exceptions will be discussed in detail below. 
To ensure the distributional pattern related to the attacker goal cannot be easily guessed (requirement (4)), we generate $\boldsymbol m$ randomly (instead of picking first $\|\boldsymbol m\|$ activations in Section~\ref{sec:eval_zero_activation_attack}). This makes the brute-force search of possible patterns computationally infeasible (details in Appendix~\ref{sec:adaptive_activation_distribution_checking}). 

%already makes the attack stealthy enough to evade the possible adaptive detection methods that check activation distribution (Details are in Appendix~\ref{sec:adaptive_activation_distribution_checking}). 

% \begin{figure*}[htbp]
%     \centering
%     \includegraphics[width=.85\linewidth]{fig/attack/bn_stealthier_attack_10000.pdf} 
% \caption{Inference AUC scores of the stealthier design. Since the secreting activations are no longer zero, the inference methods based on difference or variance tests are no longer applicable. 
% Inference targets for the smile detection and age prediction are senior people and Asian people respectively; inference of specific individuals also shows improvement compared to the baseline settings (Figure~\ref{fig:stealthier_attack_results_t_individual}; Appendix). The downstream training sets have 10,000 samples in the results; results for 5,000 samples show similar trends and are in Figure~\ref{fig:stealthier_attack_results_5000} in the appendix.
% \label{fig:stealthier_attack_results_10000}}
% \end{figure*}


\iffalse
\shortsection{Evading Anomaly Detection} Using the attack design above (i. e., choosing proper value of $\lambda$ that balances attack effectiveness and stealthiness), our attack is able to evade anomaly detection methods in most settings. However, in some settings (mostly in gender recognition tasks), state-of-the-art anomaly detection (Spectre) can still identify most of the samples with target property easily. The main reason for this detectability is that for these settings the attack signature of samples with target property is still strong and is easily distinguishable from samples without the property~\cite{hayase2021spectre}. Therefore, we add an additional term (weighted parameter $\gamma$) to the overall loss function $l(\boldsymbol x, y, y_t)$ in \autoref{eq:zero_activation_loss} to further reduce the attack signature while still maintaining relatively high attack effectiveness. Specifically, we first obtain the corresponding covariance matrices of the (all) activations of only samples with the target property ($\boldsymbol{cov_w}$), activations of samples with and without the target property ($\boldsymbol{cov_{w,wo}}$), and activations of only samples without the target property($\boldsymbol{cov_{wo}}$) respectively. Then, we encourage $mean(\boldsymbol{cov_w)} = mean(\boldsymbol{cov_{w,wo}}) = mean(\boldsymbol{cov_{wo}})$ and $var(\boldsymbol{cov_w}) = var(\boldsymbol{cov_{w,wo}}) = var(\boldsymbol{cov_{wo}})$ (both $mean(\cdot), var(\cdot)$ treats the whole covariance matrix as a flattened array and returns a scalar value) of the three covariance matrices by minimizing their differences in their mean and variance. Using this method, we ensure the distributions of activations of samples with target property will be similar to the ones without property and hence, is harder to detect. For consistency, we use this modified approach for all of our experiments (including settings where the proper value of $\lambda$ can also already evade detection).
\fi

%aims to make the covariance matrix of samples  with target property have similar distribution (e.g. similar average value and variance) as the covariance matrix of samples without target property.
%Minimizing this term ensures the distribution of activations of samples with and without target property are similar. Once combined with the other loss terms (i.e., $l_{normal}, l_t$), the attacker now seeks to find the minimal distributional difference while still achieving the attacker goal. 
%\snote{better to formulate what is written here}. \ynote{if we do so, will have a complicated equation. This part may not be interesting enough to occupy a lot of space}

\iffalse
further evade anomaly detection-based defense methods described in Section~\ref{sec:possible_defense}, the adversary can incorporate adversarial training to evade defenses. We consider the adapted Spectre (Section~\ref{sec:possible_defense}), and add a training goal to make the covariance matrix of activations of samples with and without the target property have similar values. %Note that PCA-based methods rely on the covariance matrix \anote{(How) is this relevant here?}. 
And the overall loss can be rewritten as:
\begin{equation} \label{eq:stealthier_loss}
    % l = \alpha \cdot l_{normal} + \beta \cdot l_{{reg_s}_w} + \gamma \cdot l_{{reg_s}_{wo}} + \delta \cdot l_{regs_{adv}}
    l(\boldsymbol x, y, y_t) = l_{normal}(\boldsymbol x, y) + \alpha \cdot l_{{reg}_w}(\boldsymbol x, y_t) + \beta \cdot l_{{reg}_{wo}}(\boldsymbol x, y_t) + \gamma \cdot l_{reg_{adv},}(\boldsymbol x, y)
\end{equation}
where $l_{normal}$ is the task loss, $l_{{reg}_w}$ and $l_{{reg}_{wo}}$ are defined in  Equation~\ref{eq:x_w_activation} and Equation~\ref{eq:stealthier_activation_wo} respectively, $l_{{reg}_{adv}}$ is the adversarial training goal, and $\alpha$, $\beta$, and $\gamma$ are hyper-parameters which will be all set as 1 for the same reason as Equation~\ref{eq:zero_activation_loss}.
\fi
% As for the model inference, we can use the white-box meta classifier, the black-box meta classifier, and the confidence score test described in Section~\ref{sec:zero_activation_inference}. 



\subsection{Experiments with Stealthy Attacks}\label{sec:stealthier-design-results}

\iffalse
\shortsection{Experimental Setup}
For $\boldsymbol m$, we randomly select 16 activations out of a total of 512 for gender recognition and also select 196 activations out of a total of 50,176 for smile detection and age prediction.
In practice, the total number of channels in convolutional kernels is not very large and therefore, the defender may still be able to brute-force the manipulated activations if $
\boldsymbol m$ is chosen only at the channel level. Thus, we also choose to select secreting activations directly for tasks where the first layer of the downstream model is convolutional, which may reduce some of the attack effectiveness.
%Another small change is, if the first layer of downstream model is convolutional, we still generate $\boldsymbol m$ for the secreting activations (which may lead to some slight loss in attack effectiveness) instead of generating it for the channels. We do this because we are dealing with an active defender who knows the attack method and generating $\boldsymbol m$ this way can significantly increase the difficulty of defender guessing the secreting activations (more details in the analysis of stealthiness below). 
For $\lambda$, we prefer a larger value for better inference effectiveness while still evading anomaly detection. Therefore, we performed a linear search starting from 1 and incrementing it by 0.5, and terminating when the attack can no longer evade the mentioned anomaly detection methods. With this strategy, we set $\lambda=2$ for gender recognition, $\lambda=1.5$ for smile detection and age detection when the inference targets are senior people and Asian people respectively, and $\lambda=1$ for smile detection and age detection when the inference targets are specific individuals. $\alpha$, $\beta$, and $\gamma$ are all set to be 1 in the experiments. 
\fi

%and the detection results and the inference results are reported below.

%To evaluate the performance of \textit{average value based detection}, we measure the detection rate, which is the fraction of actual secreting activations in identified potential activations.  For the \textit{intersection based method}, since the size of final returned secreting activations can vary (due to intersection over multiple inputs) for different settings, we evaluate the defense performance by reporting their F1-score (viewing actual target as positive class and others as negative). When running these two detections, we consider an idealized scenario for the defender, where all the randomly sampled inputs are without target property and so, their secreting activations are even smaller for manipulated models and is easier to be detected by the defender. 

\iffalse
%in  Equation~\ref{eq:x_w_activation} and Equation~\ref{eq:stealthier_activation_wo} 
such that $|\boldsymbol m|$ is the same as that in Section~\ref{sec:eval_zero_activation_attack}; the only difference is that the activation selection here is conducted randomly. As for the hyper-parameter $\lambda$, since it
% directly controls the extent of the difference between activations of samples with and without the target property and is
directly relates to stealthiness, we set it through a linear search process starting from 1 and increasing it by 0.5 until the attack cannot pass anomaly detection.
% described in Section~\ref{sec:possible_defense}.
Using the search, we set $\lambda$ as 2 for gender recognition and as 1 for smile detection and age detection.  
% The rest settings are the same as those described in Section~\ref{sec:eval_zero_activation_attack}
\fi

% \shortsection{Detection Results of Anomaly Detection}
\shortsection{Detection Evasion}
Figure~\ref{fig:anomaly_detection_stealthier_attack} (in the appendix) summarizes the results of our experiments to detect the stealthy upstream models (Appendix~\ref{sec:experimental-setup-stealthier-attacks} provides details on these experiments). We find that the anomaly detection methods 
%mentioned in Section~\ref{sec:possible_defense} 
are ineffective against our stealthier attack--- $<10\%$ of samples with the target property are detected across all settings with the exception of a detection rate  $<20\%$ (still low) for smile detection when %the defender deploys PCA and 
the total number of samples is 5$\,$000 and 100 or 150 of them are with the target property. We also made several attempts to approximately identify (instead of brute-force search) possible attack patterns in the activations but none of these succeeded in uncovering the stealthy attacks (details are in Appendix~\ref{sec:adaptive_activation_distribution_checking}).


%if it is possible In Appendix~\ref{sec:adaptive_activation_distribution_checking}, we also find that the possible adaptive activation distribution checking methods proposed by us cannot detect our attack either. 
%In addition, the actual distribution that matches the attacker goal can also hard to be identified as the brute-force search spaces are extremely large and hence is infeasible for computationally bounded defenders. Theoretically, even with the exact knowledge on the size of the secreting activations, defenders have to try a total of $\binom{512}{16}$ ($>8e29$) forms of $\boldsymbol m$ (i.e., $\|\boldsymbol m\|_1=16$ with total of 512 activations) for gender recognition experiments and $\binom{50176}{196}$ for smile detection and age prediction to possibly identify the selected secreting activations. 
%For \dnote{don't understand this one to express simply} (3), since $\boldsymbol m$ is generated randomly and the secreting activations are also optimized to have close-to-normal values, the defender needs to first identify the manipulated activations by brute-force search to check for the possible distributional differences that indicate the model has been manipulated to have targeted activations since distributions of non-secreting activations are similar for samples with and without the property. Therefore, as long as we choose $\boldsymbol m$ with proper size, the brute-force search space for the defender can be extremely large and computationally infeasible. For instance, identifying 16 manipulated activations out of a total of 512 (gender recognition experiments) corresponds to $\binom{512}{16}$ ($> 8.4 \times 10^{29}$) combinations.




% \snote{important to mention what the defender will do after ranking all the secreting activations. I tentatively left it as a proof-of-concept experiment. activation checking defense is relying on fraction of identified secreting activations by feeding samples without property. anomaly detection uses fraction of identified target samples and the defender will filter out the identified samples from the training set. \snote{ideally, we should do this part and add to the paper.}}

%Specifically, since the property should be rare in the downstream set to make it interesting, most samples in the downstream set are without the target property. And in our loss definition, the average values of the secreting activations for those samples should not be larger than those of the non-secreting activations. Therefore, the downstream model trainer can compute the values of the activations using different inputs, and then find the activations that are of small values for most inputs as the possible secreting activations. Based on this principle, we have two designs for finding the possible secreting activations. The first one is to average the outputs of each activation over different inputs and then find the ones that have smaller average values as the possible secreting activations (\emph{average value based detection}); the second method is to find the activations that output smaller vales for each samples and then calculate the intersection of those activations as the secreting activations (\emph{intersection based detection}). To simulate the worst case for the attacker, we only use the samples without the target property to generate activations values in the experiments. The results show that those attempts can hardly pinpoint secreting activations. The detection rate of secreting activations of the first method is less than 11.3\% for gender recognition and is less than 1.5\% for smile detection and age prediction; the F1-score of the target activation detection of the second method is less than 0.009 for the second method. More details of the experiments and results are in Section~\ref{sec: details_distribution_checking} in the appendix. \anote{Re-word the paragraph above slightly if possible- a bit confusing in the first read.}


% \begin{figure*}[htbp]
%     \centering
%     \includegraphics[width=1\linewidth]{fig/defense/bn_clustering.pdf} 
% \caption{Percentage of samples with the target property detected by the anomaly detection. Similar to ~\cite{hayase2021spectre}, we filter out $n \times 1.5$ samples with anomaly detection, where $n$ is the number of samples in downstream training data with the target property. We report the number of samples with the target property filtered out divided by $n$ as the \emph{Detection Percentage}; values are averaged (with standard deviation) over 5 runs of anomaly detection.
% The `5K' lines report detection results on the settings with 5,000 total samples, while the `10K' lines report for 10,000 total samples. Inference targets for smile detection and age prediction are senior people and Asian people respectively; results for the inference of specific individuals follow similar trends (Figure~\ref{fig:anomaly_detection_stealthier_attack_t_individual}).
% \label{fig:anomaly_detection_stealthier_attack} }
% \end{figure*}

% \begin{figure*}[htbp]
%     \centering
%     \includegraphics[width=1\linewidth]{fig/attack/bn_stealthier_attack_5000_10000.pdf} 
% \caption{Inference AUC scores of the stealthy design. Since we do not introduce zeros for secreting activations, we cannot use the difference or variance tests, but are left with meta-classifiers and the confidence score test. The downstream training sets have 5,000 samples in the results in the first row and 10,000 samples for the second row. Inference targets for the smile detection and age prediction are senior people and Asian people respectively; inference of specific individuals also shows improvement compared to the baseline settings (Figure~\ref{fig:stealthier_attack_results_t_individual}).
% \label{fig:stealthier_attack_results_10000}}
% \end{figure*}

\shortsection{Inference Results}
%Model task accuracy drops by at most 0.9\% (details in Table~\ref{sec:impact_to_upstream_accuracy} in the appendix). Thus, adding a stealthiness attacker goal does not harm upstream task performance significantly.
% Next we show the actual inference results when the defender deploys the two defenses mentioned in Section~\ref{sec:possible_defense} in.
From Figure~\ref{fig:stealthier_attack_results_10000}, we can see that activation manipulation still leads to significantly improved inference results compared to the baselines with normally trained upstream models. For example, for gender recognition, when $\geq 50$ downstream training samples have the target property, inference AUC scores exceed 0.95, which is a huge improvement compared to the baseline attack where all AUC scores are less than 0.6, and similar trends follow for smile detection (with over 100 samples with property, AUC improves from $<0.6$ to $>0.78$) and age prediction (with over 100 samples with property, AUC improves from $<0.77$ to $>0.9$).
% For smile detection, when over 50 out of 5,000 training samples are with the property, AUC scores can be greater than 0.85, whereas the baseline settings have AUC scores all less than 0.76. 
%For age prediction, when over 100 out of 5,000 or 10,000 samples are with the target property, AUC scores of the black-box meta classifier can exceed $0.9$, while the baseline setting have AUC scores all $<0.82$.
Comparing the results for the stealthier attacks to the results that do not consider defenses in Figure~\ref{fig:zero_activation_attack_results}, we observe that the attack effectiveness declines as expected since we are now trading-off attack effectiveness for stealthiness.
Training models with the attack goal poses negligible impact on the model performance (accuracy drop $ < 0.9\%$, see Appendix~\ref{sec:impact_to_upstream_accuracy}).
%Those results suggest that our proposed attacks are still a practical threat even under anomaly detection based defenses . 

% \shortsection{Upstream Task Accuracy} Model task accuracy drops by at most 0.9\% (details in Table~\ref{sec:impact_to_upstream_accuracy} in the appendix). Thus, adding a stealthiness attacker goal does not harm upstream task performance significantly.
% \dnote{doesn't belong here, just mention this at the beginning where it belongs}

%\anote{Too many details below- we should present the results and what we learn from them with maybe some numbers. Having too many numbers may overwhelm the reader and bury our actual message- should shift unnecessary details to the Appendix.}
% Figure~\ref{fig:stealthier_attack_results} shows the inference results. Although the inference scores are not as high as those of the zero-activation attack reported in Section~\ref{sec:eval_zero_activation_attack}, compared to the results of the baseline settings, our stealthier attack still improves the baseline attack by a large margin in general. For example, for the gender recognition task, when over 50 out of 5000/10000 samples are with the property, the inference AUC scores exceed 0.9 and are close to 1, which is a huge improvement compared to the baseline settings where all AUC scores are less than 0.7; for the smile detection task, when over 100 samples are with the property, the inference AUC scores of the black-box meta classifiers are all greater than 0.71, whereas the baseline settings have AUC scores all less than 0.61; for the age prediction task, when over 50 out of 5000 samples are with the target property, the AUC scores of the black-box meta classifier exceed 0.78, while the baseline setting have AUC scores less than 0.65.


% \begin{figure*}[htbp]
%     \centering
%     \includegraphics[width=1\linewidth]{fig/attack/bn_stealthier_attack_5000_10000.pdf} 
% \caption{Inference AUC scores of the stealthy design. Since we do not introduce zeros for secreting activations, we cannot use the difference or variance tests, but are left with meta-classifiers and the confidence score test. The downstream training sets have 5,000 samples in the results in the first row and 10,000 samples for the second row. Inference targets for the smile detection and age prediction are senior people and Asian people respectively; inference of specific individuals also shows improvement compared to the baseline settings (Figure~\ref{fig:stealthier_attack_results_t_individual}).
% \label{fig:stealthier_attack_results_10000}}
% \end{figure*}

\iffalse
\shortsection{Stealthiness analysis} %\todo{Odd to have this empty}
\emph{Distribution Checking:} Since there is no indicator like abnormal zeros as in the zero-activation attack, to check if the activation distribution matches the attack goal, the downstream trainer needs to first pinpoint the manipulated activations in the feature extractor's outputs through brute-force-like ways. This process is not feasible in terms of computational resources--- the number of combinations is $\binom{512}{16}$ ($>8e29$) in finding out 16 out of 512 activations (gender recognition), and is $\binom{50176}{196}$\footnote{The calculator told us the result is infinite.} in finding out 196 out of 50176 activations (smile detection and age prediction).
\fi

This work presented a simulation approach that centers around finding iteratively an approximation of the evolution of the algebraic variables in the power system \glspl{DAE}. The approximation of the dynamic state evolutions by NNs, instead of classical explicit numerical integration schemes, allows larger time-steps to be realized while being fast to execute. This work aimed at providing a proof of concept, it is foreseeable that future work on this method shares many typical questions with established \gls{DAE} solvers, hence, by applying various existing techniques the computational performance and scalability of the approach should improve significantly.

%\dnote{some figures are appearing inside the references, should either be in appendix, or before references}
%%%%%%%%% REFERENCES
{\small
\bibliographystyle{ieee_fullname}
\bibliography{reference}
}
\clearpage
\appendix
\onecolumn
\addcontentsline{toc}{section}{Appendices}

\clearpage

\section{Detailed Local Data Distribution}
\label{sec:dataset_summary}

We adopt a Latent Dirichlet Allocation (LDA) strategy for Non-IID setting \cite{fedma, moon}, where each client $k$ is assigned the partition of classes by sampling $\mathbf{p}_k \sim Dir(\alpha \cdot \mathds{1})$, where $\mathds{1} \in \, \mathbb{R}^C$.
$\alpha$ is a concentration parameter that controls the local heterogeneity level. The smaller $\alpha$, the more heterogeneous data distribution.
Since we consider a fairness issue in the FAL framework, the total number of samples should be equally partitioned for all clients.
Therefore, we made a doubly stochastic matrix $P = [\tilde{\mathbf{p}}_1, \dots, \tilde{\mathbf{p}}_K]^\top$ by scaling $\mathbf{p}_k$ to $\tilde{ \mathbf{p}}_k$, when the number of client and class are same (i.e., $P$ is a square matrix).
Note that we set the sum of columns and rows to the proper values for a non-square matrix.
We visualized the examples of CIFAR-10 when the clients $K=10$ in Figure \ref{fig:data_dist}.

\begin{figure}[h!]
\centering
\begin{minipage}{0.94\linewidth}
    \begin{subfigure}[b]{0.32\linewidth}
    \includegraphics[width=\linewidth]{figure/data_distribution/rho1/dir0.1_rho1.pdf}
    \caption{\small {$\rho$ = 1 and $\alpha$ = 0.1}}
    \end{subfigure}
    \hfill
    \begin{subfigure}[b]{0.32\linewidth}
    \includegraphics[width=\linewidth]{figure/data_distribution/rho1/dir1_rho1.pdf}
    \caption{\small {$\rho$ = 1 and $\alpha$ = 1.0}}
    \end{subfigure}
    \hfill
    \begin{subfigure}[b]{0.32\linewidth}
    \includegraphics[width=\linewidth]{figure/data_distribution/rho1/dir10000_rho1.pdf}
    \caption{\small {$\rho$ = 1 and $\alpha$ = $\infty$}}
    \end{subfigure}
    \begin{subfigure}[b]{0.32\linewidth}
    \centering
    \includegraphics[width=\linewidth]{figure/data_distribution/rho5/dir0.1_rho5.pdf}
    \caption{\small {$\rho$ = 5 and $\alpha$ = 0.1}}
    \end{subfigure}
    \hfill
    \begin{subfigure}[b]{0.32\linewidth}
    \centering
    \includegraphics[width=\linewidth]{figure/data_distribution/rho5/dir1_rho5.pdf}
    \caption{\small {$\rho$ = 5 and $\alpha$ = 1.0}}
    \end{subfigure}
    \hfill
    \begin{subfigure}[b]{0.32\linewidth}
    \centering
    \includegraphics[width=\linewidth]{figure/data_distribution/rho5/dir10000_rho5.pdf}
    \caption{\small {$\rho$ = 5 and $\alpha$ = $\infty$}}
    \end{subfigure}
    \begin{subfigure}[b]{0.32\linewidth}
    \centering
    \includegraphics[width=\linewidth]{figure/data_distribution/rho10/dir0.1_rho10.pdf}
    \caption{\small {$\rho$ = 10 and $\alpha$ = 0.1}}
    \end{subfigure}
    \hfill
    \begin{subfigure}[b]{0.32\linewidth}
    \centering
    \includegraphics[width=\linewidth]{figure/data_distribution/rho10/dir1_rho10.pdf}
    \caption{\small {$\rho$ = 10 and $\alpha$ = 1.0}}
    \end{subfigure}
    \hfill
    \begin{subfigure}[b]{0.32\linewidth}
    \centering
    \includegraphics[width=\linewidth]{figure/data_distribution/rho10/dir10000_rho10.pdf}
    \caption{\small {$\rho$ = 10 and $\alpha$ = $\infty$}}
    \end{subfigure}
    \begin{subfigure}[b]{0.32\linewidth}
    \centering
    \includegraphics[width=\linewidth]{figure/data_distribution/rho20/dir0.1_rho20.pdf}
    \caption{\small {$\rho$ = 20 and $\alpha$ = 0.1}}
    \end{subfigure}
    \hfill
    \begin{subfigure}[b]{0.32\linewidth}
    \centering
    \includegraphics[width=\linewidth]{figure/data_distribution/rho20/dir1_rho20.pdf}
    \caption{\small {$\rho$ = 20 and $\alpha$ = 1.0}}
    \end{subfigure}
    \hfill
    \begin{subfigure}[b]{0.32\linewidth}
    \centering
    \includegraphics[width=\linewidth]{figure/data_distribution/rho20/dir10000_rho20.pdf}
    \caption{\small {$\rho$ = 20 and $\alpha$ = $\infty$}}
    \end{subfigure}
\end{minipage}
\caption{Visualization of the client data distribution on CIFAR-10. Each color represents a different class. The higher $\rho$ denotes the more global imbalanced distribution. The higher $\alpha$ denotes the more locally balanced data.}
\label{fig:data_dist}
\end{figure}

\clearpage

\section{Detailed Analysis Results}
\label{sec:detail_analysis}


\subsection{Detailed Matrices for Data Counts and Accuracy}
\label{sec:detail_analysis_matrices}

We summarized the detailed matrices for the combinations of $\rho$ = $\{$1, 5, 10, 20$\}$ and $\alpha$ = $\{$0.1, 1.0, $\infty \}$.

\begin{figure*}[!h]
\centering
\begin{minipage}{0.8\linewidth}
    \centering
    \begin{subfigure}[b]{0.32\linewidth}
    \centering
    \includegraphics[width=\linewidth]{figure/analysis/obs2/appendix/rho1_alpha01_total.pdf}
    \caption{$\rho$ = 1 and $\alpha$ = 0.1}         
    \end{subfigure}
    \hfill
    \centering
    \begin{subfigure}[b]{0.32\linewidth}
    \centering
    \includegraphics[width=\linewidth]{figure/analysis/obs2/appendix/rho1_alpha1_total_v.pdf}
    \caption{$\alpha$ = 1.0}       
    \end{subfigure}
    \hfill
    \begin{subfigure}[b]{0.32\linewidth}
    \centering
    \includegraphics[width=\linewidth]{figure/analysis/obs2/appendix/rho1_alpha_inf_total.pdf}
    \caption{$\alpha$ = $\infty$}     
    \end{subfigure}
    \vspace*{-0.2cm}
\end{minipage}
\caption{Matrices of data count (top) and class-wise accuracy (down) when $\rho$ = 1.}
\vspace*{-0.2cm}
\label{fig:cnt_acc_rho1}
\end{figure*}

\vspace{-10pt}
\begin{figure*}[!h]
\centering
\begin{minipage}{0.8\linewidth}
    \centering
    \begin{subfigure}[b]{0.32\linewidth}
    \centering
    \includegraphics[width=\linewidth]{figure/analysis/obs2/appendix/rho5_alpha01_total.pdf}
    \caption{$\alpha$ = 0.1}         
    \end{subfigure}
    \hfill
    \centering
    \begin{subfigure}[b]{0.32\linewidth}
    \centering
    \includegraphics[width=\linewidth]{figure/analysis/obs2/appendix/rho5_alpha1_total.pdf}
    \caption{$\alpha$ = 1.0}       
    \end{subfigure}
    \hfill
    \begin{subfigure}[b]{0.32\linewidth}
    \centering
    \includegraphics[width=\linewidth]{figure/analysis/obs2/appendix/rho5_alpha_inf_total.pdf}
    \caption{$\alpha$ = $\infty$}     
    \end{subfigure}
    \vspace*{-0.2cm}
\end{minipage}
\caption{Matrices of data count (top) and class-wise accuracy (down) when $\rho$ = 5.}
\vspace*{-0.2cm}
\label{fig:cnt_acc_rho5}
\end{figure*}

\newpage

\begin{figure*}[!h]
\centering
\begin{minipage}{0.8\linewidth}
    \centering
    \begin{subfigure}[b]{0.32\linewidth}
    \centering
    \includegraphics[width=\linewidth]{figure/analysis/obs2/appendix/rho10_alpha01_total.pdf}
    \caption{$\alpha$ = 0.1}         
    \end{subfigure}
    \hfill
    \centering
    \begin{subfigure}[b]{0.32\linewidth}
    \centering
    \includegraphics[width=\linewidth]{figure/analysis/obs2/appendix/rho10_alpha1_total.pdf}
    \caption{$\alpha$ = 1.0}       
    \end{subfigure}
    \hfill
    \begin{subfigure}[b]{0.32\linewidth}
    \centering
    \includegraphics[width=\linewidth]{figure/analysis/obs2/appendix/rho10_alpha_inf_total.pdf}
    \caption{$\alpha$ = $\infty$}     
    \end{subfigure}
    \vspace*{-0.2cm}
\end{minipage}
\caption{Matrices of data count (top) and class-wise accuracy (down) when $\rho$ = 10.}
\vspace*{-0.2cm}
\label{fig:cnt_acc_rho10}
\end{figure*}


\begin{figure*}[!h]
\centering
\begin{minipage}{0.8\linewidth}
    \centering
    \begin{subfigure}[b]{0.32\linewidth}
    \centering
    \includegraphics[width=\linewidth]{figure/analysis/obs2/appendix/rho20_alpha_01_total.pdf}
    \caption{$\alpha$ = 0.1}         
    \end{subfigure}
    \hfill
    \centering
    \begin{subfigure}[b]{0.32\linewidth}
    \centering
    \includegraphics[width=\linewidth]{figure/analysis/obs2/appendix/rho20_alpha_1_total.pdf}
    \caption{$\alpha$ = 1.0}       
    \end{subfigure}
    \hfill
    \begin{subfigure}[b]{0.32\linewidth}
    \centering
    \includegraphics[width=\linewidth]{figure/analysis/obs2/appendix/rho20_alpha_inf_total.pdf}
    \caption{$\alpha$ = $\infty$}     
    \end{subfigure}
    \vspace*{-0.2cm}
\end{minipage}
\caption{Matrices of data count (top) and class-wise accuracy (down) when $\rho$ = 20.}
\vspace*{-0.2cm}
\label{fig:cnt_acc_rho20}
\end{figure*}

\newpage
\subsection{Detailed Earth Mover Distance}
\label{sec:detail_analysis_emd}
Table \ref{tab:detail_emd} \ summarizes the detailed local and global EMD for the combinations of $\rho$ = $\{$1, 5, 10, 20$\}$ and $\alpha$ = $\{$0.1, 1.0, $\infty \}$.

\begin{table}[H]
\centering
\small
\renewcommand*{\arraystretch}{0.85}
\addtolength{\tabcolsep}{1pt}
\resizebox{0.8\linewidth}{!}{
\begin{tabular}{cc|c|ccccc|ccccc}
\toprule
\multirow{2}{*}{$\rho$} &
  \multirow{2}{*}{$\alpha$} &
  \multirow{2}{*}{model} &
  \multicolumn{5}{c|}{Local EMD $(\downarrow)$} &
  \multicolumn{5}{c}{Global EMD $(\downarrow)$} \\
\cmidrule(l{2pt}r{2pt}){4-8} \cmidrule(l{2pt}r{2pt}){8-13}
                   &                           &   & 10\%  & 20\%  & 30\%  & 40\%  & 50\%  & 10\%  & 20\%  & 30\%  & 40\%  & 50\%  \\
                   \midrule
\multirow{2}{*}{1} & \multirow{2}{*}{0.1}      & G & 0.632 & 0.638 & 0.641 & 0.643 & 0.646 & 0.019 & 0.064 & 0.086 & 0.095 & 0.091 \\
                   &                           & L & 0.632 & 0.597 & 0.592 & 0.595 & 0.601 & 0.019 & 0.050 & 0.050 & 0.046 & 0.055 \\ \midrule
\multirow{2}{*}{1} & \multirow{2}{*}{1.0}      & G & 0.297 & 0.297 & 0.300 & 0.300 & 0.300 & 0.017 & 0.066 & 0.079 & 0.084 & 0.083 \\
                   &                           & L & 0.297 & 0.248 & 0.232 & 0.235 & 0.241 & 0.017 & 0.053 & 0.065 & 0.068 & 0.074 \\ \midrule
\multirow{2}{*}{1} & \multirow{2}{*}{$\infty$} & G & 0.049 & 0.077 & 0.070 & 0.065 & 0.061 & 0.014 & 0.070 & 0.066 & 0.063 & 0.060 \\
                   &                           & L & 0.049 & 0.042 & 0.054 & 0.059 & 0.066 & 0.014 & 0.025 & 0.044 & 0.053 & 0.062 \\  \midrule
% Rho = 5

\multirow{2}{*}{5} & \multirow{2}{*}{0.1}      & G & 0.662 & 0.663 & 0.666 & 0.666 & 0.669 & 0.211 & 0.201 & 0.196 & 0.194 & 0.195 \\ 
                   &                           & L & 0.662 & 0.628 & 0.627 & 0.628 & 0.634 & 0.211 & 0.232 & 0.232 & 0.236 & 0.228 \\ \midrule
\multirow{2}{*}{5} & \multirow{2}{*}{1.0}      & G & 0.402 & 0.391 & 0.387 & 0.388 & 0.389 & 0.206 & 0.188 & 0.180 & 0.173 & 0.169 \\
                   &                           & L & 0.402 & 0.309 & 0.306 & 0.306 & 0.341 & 0.206 & 0.200 & 0.201 & 0.196 & 0.196 \\ \midrule
\multirow{2}{*}{5} & \multirow{2}{*}{$\infty$} & G & 0.213 & 0.190 & 0.178 & 0.168 & 0.165 & 0.206 & 0.185 & 0.174 & 0.162 & 0.163 \\
                   &                           & L & 0.213 & 0.179 & 0.176 & 0.180 & 0.180 & 0.206 & 0.176 & 0.173 & 0.178 & 0.180 \\ \midrule

\multirow{2}{*}{10} & \multirow{2}{*}{0.1}      & G & 0.692 & 0.685 & 0.687 & 0.685 & 0.685 & 0.280 & 0.268 & 0.267 & 0.265 & 0.267 \\
                    &                           & L & 0.692 & 0.652 & 0.650 & 0.654 & 0.660 & 0.280 & 0.270 & 0.277 & 0.282 & 0.281 \\ \midrule
\multirow{2}{*}{10} & \multirow{2}{*}{1.0}      & G & 0.491 & 0.463 & 0.459 & 0.456 & 0.455 & 0.297 & 0.263 & 0.247 & 0.244 & 0.242 \\
                    &                           & L & 0.491 & 0.408 & 0.402 & 0.405 & 0.415 & 0.297 & 0.256 & 0.257 & 0.255 & 0.255 \\ \midrule
\multirow{2}{*}{10} & \multirow{2}{*}{$\infty$} & G & 0.315 & 0.240 & 0.229 & 0.223 & 0.222 & 0.303 & 0.237 & 0.226 & 0.222 & 0.221 \\
                    &                           & L & 0.315 & 0.238 & 0.237 & 0.239 & 0.240 & 0.303 & 0.237 & 0.234 & 0.237 & 0.239 \\ \midrule

\multirow{2}{*}{20} & \multirow{2}{*}{0.1}      & G & 0.692 & 0.680 & 0.676 & 0.674 & 0.677 & 0.377 & 0.300 & 0.294 & 0.294 & 0.298 \\
                    &                           & L & 0.692 & 0.641 & 0.633 & 0.636 & 0.644 & 0.377 & 0.304 & 0.326 & 0.321 & 0.323 \\ \midrule
\multirow{2}{*}{20} & \multirow{2}{*}{1.0}      & G & 0.481 & 0.455 & 0.450 & 0.448 & 0.448 & 0.374 & 0.311 & 0.300 & 0.295 & 0.292 \\
                    &                           & L & 0.481 & 0.448 & 0.437 & 0.431 & 0.437 & 0.374 & 0.354 & 0.342 & 0.303 & 0.304 \\ \midrule
\multirow{2}{*}{20} & \multirow{2}{*}{$\infty$} & G & 0.371 & 0.298 & 0.284 & 0.274 & 0.276 & 0.368 & 0.294 & 0.282 & 0.271 & 0.272 \\
                    &                           & L & 0.371 & 0.313 & 0.293 & 0.290 & 0.289 & 0.368 & 0.309 & 0.287 & 0.288 & 0.289 \\
                   
                   
                   
                   \bottomrule
\end{tabular}}
    \caption{Local and global EMD on CIFAR-10 for 12 combinations of $\rho$ = $\{$1, 5, 10, 20$\}$ and $\alpha$ = $\{$0.1, 1.0, $\infty \}$.}
    \label{tab:detail_emd}
\end{table}

\clearpage
\section{Pseudo Algorithm of LoGo}
\label{sec:pseudo_algorithm}


Algorithm\,\ref{alg:logo} is the overall pipeline of the FAL framework.
Specifically, we summarize the detailed pseudocode of our \algname{} algorithm.

\begin{algorithm}[h]
\small
\caption{FAL framework with \algname{} algorithm}
\label{alg:logo}
\textbf{Input}: initialized parameter $\Theta$; unlabeled data $U^{\scaleto{1}{4pt}}$; sampling strategy $\mathcal{A}$; labeling budget $B$; clients number $K$; AL round $R$; \\
\textbf{Output}: trained parameter $\Theta^{\scaleto{R*}{4pt}}$ \\
\\
\textbf{\# Alternating AL and FL Procedure}
\begin{algorithmic}[1]
\FOR{$k=1, \dots, K$}
\STATE Randomly sample $L_{\scaleto{k}{4pt}}^{\scaleto{1}{4pt}}= \{ x_{\scaleto{1}{4pt}}, \dots, x_{\scaleto{B}{4pt}} \}$ from $U_{\scaleto{k}{4pt}}^{\scaleto{1}{4pt}}$, and $U_k^{2} = U_k^1 \setminus L_k^1$
\STATE Get the labeled set $D_{\scaleto{k}{4pt}}^{\scaleto{1}{4pt}}$\, from the oracles
\ENDFOR
\STATE $\Theta^{\scaleto{1*}{4pt}}=$\,\texttt{FedAvg}\,($\Theta$, $D^{\scaleto{1}{4pt}}, K$) \\
\, \\
\FOR{$r=2, \dots, R$}
\FOR{$k=1, \dots, K$}
\STATE $D^{r}_k, \,U_k^{r+1}=$\,\,\texttt{LoGo}\,($\Theta^{(r-1)*}$, $D^{r-1}_{\scaleto{k}{4pt}}, U_k^r$)
\ENDFOR
\STATE $\Theta^{r*}=$\,\texttt{FedAvg}\,($\Theta$, $D^{r}, K$)
\ENDFOR
\end{algorithmic}
\, \\
\textbf{Function}\,\,\texttt{LoGo}:
\begin{algorithmic}[1] 
\STATE \textbf{\# Macro Step}
\STATE 
Train a local-only model $\Theta^{(r-1)}_{k*}$ from the scratch only using $D_k^{r-1}$
\STATE  For each $x\in U_k^r$, calculate the gradient embedding $g_{\hat{y}}^x$  by Eq.\,\eqref{eq:gradient}
\STATE Cluster $U_k^r$ into $B$ clusters($\mathcal{C}_1,...,\mathcal{C}_B$) by Eq.\,\eqref{eq:kmeans} \\
\, \\
\STATE \textbf{\# Micro Step}
\STATE 
$L_k^r= \emptyset $
\FOR{$\mathcal{C}_{\scaleto{i}{4pt}}=\mathcal{C}_{\scaleto{1}{4pt}}, \dots, \mathcal{C}_{\scaleto{B}{4pt}}$}
\STATE $L_k^r = L_k^r \cup \{ \mathcal{A}(\mathcal{C}_{\scaleto{i}{4pt}}, \Theta^{(r-1)*}, 1) \}$
\STATE $D_k^r = D_k^{r-1} \cup D_k^r$\, and\, $U_{\scaleto{k}{4pt}}^{\scaleto{r+1}{4pt}} = U_{\scaleto{k}{4pt}}^{\scaleto{r}{3pt}} \setminus L_k^r$
\ENDFOR 
\STATE \textbf{return}  $D_k^r$, \,$U_k^{r+1}$
\end{algorithmic}
\, \\
\textbf{Function}\,\,\texttt{FedAvg}:
\begin{algorithmic}[1]
\FOR{$\,FL\,\,round$\,}
\STATE Distribute $\Theta$ to the all client
\FOR{$k = 1, \dots, K$}
\STATE Train $\Theta_k$ on $D_k^r$ by minimizing $\mathbb{E}_{D_k^r}[\ell(x,y; \Theta_k)]$
\ENDFOR
\STATE $\Theta = (\sum_k \Theta_k) / K$
\ENDFOR
\STATE \textbf{return} $\Theta$
\end{algorithmic}

\end{algorithm}

\clearpage
\section{Experimental Settings}
\label{sec:exp_settings}

\subsection{Datasets}
We mainly experimented on two natural image datasets (CIFAR-10\footnote{https://www.cs.toronto.edu/~kriz/cifar.html}, SVHN\footnote{http://ufldl.stanford.edu/housenumbers}) and three medical image datasets\footnote{https://medmnist.com/} (PathMNIST, DermaMNIST, OrganAMNIST). 
Table \ref{tab:dataset_summray} provides a summary of the five datasets.
For the details of partitioning data to each client, please refer to Appendix\,\ref{sec:dataset_summary}.

\begin{table*}[t]
\caption{Details of the evaluated datasets.}
\label{tab:dataset-summary}
\vskip 0.15in
\centering
\begin{tabular}{@{}lllllll@{}}
\toprule
             & \begin{tabular}[c]{@{}l@{}}Number of\\ Series\end{tabular} & \begin{tabular}[c]{@{}l@{}}Time Steps\\ per Series\end{tabular} & Period  & Sampling & \begin{tabular}[c]{@{}l@{}}Number of\\ Features\end{tabular} & Data Split [\%] \\ \midrule
\solarSmall  & 8                                                          & 2,000                                                            & 01--03 2018    & 60m      & 8                                                            & 60/15/25   \\ 
\solarOneY   & 50                                                         & 8,760                                                            & 2019    & 60m      & 8                                                            & 60/15/25   \\ 
\solarThreeY & 50                                                         & 26,304                                                           & 2018--20 & 60m      & 8                                                            & 34/33/33   \\ \midrule
\airTen      & 12                                                         & 35,064                                                           & 2013--17 & 60m      & 11                                                           & 34/33/33   \\ 
\airTwenty   & 12                                                         & 35,064                                                           & 2013--17 & 60m      & 11                                                           & 34/33/33   \\ \midrule
\sapflux     & 24                                                         & \begin{tabular}[c]{@{}l@{}}15,000--\\ 20,000\end{tabular}          & 2008--16 & varying  & 10                                                           & 34/33/33   \\ \midrule
\hydro       & 531                                                        & 9,862                                                          & 1981--2008 & 24h      & 41                                                           & 34/33/33   \\ \bottomrule
\end{tabular}
\end{table*}


\subsection{Implementation Details}
For the FL training pipeline, we set the number of FL rounds to 100 and local update epochs to 5.
We used a SGD optimizer with the initial learning rate of 0.01 and the momentum of 0.9.
The learning rate was decayed by 0.1 at half and three-quarters of federated learning rounds to ensure convergence, and we used a random horizontal flipping as data augmentation.
For training local-only models, we trained the model using the aforementioned settings for 50 epochs. However, the training was terminated if the training accuracy reached 99\%.
It should be noted that we averaged the classification accuracy of the last 5 epochs in each round and repeated all experiments with four different seeds.
All algorithms were implemented using PyTorch 1.11.0 and executed using NVIDIA RTX 3080 GPUs.




\subsection{Experimental Categories}
A total of six categories were considered in the evaluation:
\begin{enumerate}
\item {`Query selector'} of whether to use a local-only or global model with the six compared strategies. 
\item  {`Heterogeneity level'} of varying degree of class imbalance. We adopt a Latent Dirichlet Allocation (LDA) \cite{moon} strategy. 
For example, the smaller $\alpha$, the more heterogeneous the data distribution. 
\item  {`Imbalance ratio'} of used datasets. 
We classified five datasets for evaluation based on the imbalance ratio $\rho$.
CIFAR-10 and PathMNIST belong to a low imbalance ratio ($\rho<2$), and SVHN, DermaMNIST, and OrganAMNIST belong to a high imbalance ratio ($\rho\geq 2$).
\item  {`Model architecture.'} We employed four layers of convolution neural network for a base architecture and also experimented with ResNet-18 \cite{resnet} and MobileNet \cite{mobilenet}. 
\item  `{Budget size}' for labeling. We tested small (1\%), medium (5\%), and large (20\%) budget sizes for each round. 
\item  {`Model initialization'} of either learning from scratch (random) or from the checkpoint of the previous AL round (continue). \looseness=-1
\end{enumerate}

\newpage
\subsection{Combination of experimental settings}
We compared our algorithms and baselines in 38 comprehensive experimental settings, which are the combinations of the aforementioned six categories.
All the experimental combinations we performed are summarized in Table \ref{tab:setting_summary}.

\begin{table*}[h!]
\centering
\small
\renewcommand*{\arraystretch}{1}
\addtolength{\tabcolsep}{1pt}
\resizebox{0.75\linewidth}{!}{
\begin{tabular}{cccccc}
\toprule
Query Selelctor   & Dir($\alpha$) & Data Type     & Model Arch.   & Budget Size & Model Init. \\ 
\midrule
Global          & 0.1 & CIFAR-10     & 4CNN   & 5\%         & Random     \\ 
Global          & 0.1 & SVHN        & 4CNN   & 5\%         & Random     \\ 
Global          & 0.1 & PathMNIST   & 4CNN   & 5\%         & Random     \\ 
Global        & 0.1  & OrganAMNIST & 4CNN & 5\%         & Random     \\ 
Global        & 0.1  & DermaMNIST & 4CNN   & 5\%         & Random     \\ 
Global        & 1   & CIFAR-10      & 4CNN   & 5\%         & Random     \\ 
Global        & 1   & SVHN         & 4CNN    & 5\%         & Random     \\ 
Global        & $\infty$   & CIFAR-10     & 4CNN   & 5\%         & Random     \\ 
Global        & $\infty$   & SVHN        & 4CNN   & 5\%         & Random     \\ 
Global        & 0.1  & CIFAR-10      & 4CNN  & 5\%         & Continue   \\ 
Global        & 0.1  & SVHN         & 4CNN   & 5\%         & Continue   \\ 
Global        & 0.1  & CIFAR-10      & ResNet-18 & 5\%         & Random     \\ 
Global      & 0.1    & SVHN       & ResNet-18  & 5\%         & Random     \\ 
Global        & 0.1  & CIFAR-10      & MobileNet  & 5\%         & Random     \\ 
Global        & 0.1  & SVHN         & MobileNet  & 5\%         & Random     \\ 
Global        & 0.1  & CIFAR-10      & 4CNN   & 1\%         & Random     \\ 
Global        & 0.1  & SVHN         & 4CNN   & 1\%         & Random     \\ 
Global        & 0.1  & CIFAR-10      & 4CNN   & 20\%        & Random     \\ 
Global        & 0.1  & SVHN         & 4CNN   & 20\%        & Random     \\ 
Local-only    & 0.1  & CIFAR-10      & 4CNN  & 5\%         & Random     \\ 
Local-only    & 0.1  & SVHN         & 4CNN   & 5\%         & Random     \\ 
Local-only    & 0.1  & PathMNIST    & 4CNN   & 5\%         & Random     \\ 
Local-only    & 0.1  & OrganAMNIST  & 4CNN   & 5\%         & Random     \\ 
Local-only   & 0.1   & DermaMNIST   & 4CNN & 5\%         & Random     \\ 
Local-only      & 1  & CIFAR-10    & 4CNN    & 5\%         & Random     \\ 
Local-only     & 1   & SVHN       & 4CNN   & 5\%         & Random     \\ 
Local-only     & $\infty$  & CIFAR-10     & 4CNN   & 5\%         & Random     \\ 
Local-only     & $\infty$  & SVHN       & 4CNN  & 5\%         & Random     \\ 
Local-only      & 0.1 & CIFAR-10    & 4CNN   & 5\%         & Continue   \\ 
Local-only     & 0.1  & SVHN        & 4CNN  & 5\%         & Continue   \\ 
Local-only      & 0.1 & CIFAR-10    & ResNet-18    & 5\%         & Random     \\ 
Local-only     & 0.1  & SVHN        & ResNet-18 & 5\%         & Random     \\ 
Local-only     & 0.1  & CIFAR-10     & MobileNet  & 5\%         & Random     \\ 
Local-only     & 0.1  & SVHN         & MobileNet  & 5\%         & Random     \\ 
Local-only     & 0.1  & CIFAR-10      & 4CNN  & 1\%         & Random     \\ 
Local-only     & 0.1  & SVHN         & 4CNN   & 1\%         & Random     \\ 
Local-only     & 0.1  & CIFAR-10      & 4CNN   & 20\%        & Random     \\ 
Local-only     & 0.1  & SVHN        & 4CNN   & 20\%        & Random     \\ 
\bottomrule
\end{tabular}}
\caption{Summary of the entire experimental combinations.}
\label{tab:setting_summary}
\end{table*}



\clearpage
\section{Computational Cost of Query Selection}
\label{sec:computational_cost}
In Table\,\ref{tab:time_cost}, we measured the wallclock time for various combinations of the algorithm, query selector, and labeling ratio.
We confirmed that as the percentage of labeled data increases, the time required to measure the importance score with the global model decreases due to the reduced amount of unlabeled data.
Conversly, the local-only model takes more time as it requires training on a larger number of labeled samples.
Our LoGo algorithm shows a comparable computational cost to the baselines that use the local-only model\,(L) for query selection.
Note that we used a simple Entropy sampling within LoGo algorithm to measure the uncertainty, and the only possible bottleneck is \textit{k}-means clustering in the Macro step.


\begin{table}[h!]
\centering
\addtolength{\tabcolsep}{-1pt}
\renewcommand*{\arraystretch}{1.05}
\resizebox{0.7\linewidth}{!}{
\begin{tabular}{l|cc|cc|cc|cc|cc|c}
\toprule
                   & \multicolumn{2}{c|}{\!\!\!Entropy\!\!\!} & \multicolumn{2}{c|}{Coreset} & \multicolumn{2}{c|}{BADGE} & \multicolumn{2}{c|}{GCNAL} & \multicolumn{2}{c|}{ALFA-Mix} & LoGo \\
 \cmidrule(l{2pt}r{2pt}){2-3} \cmidrule(l{2pt}r{2pt}){4-5} \cmidrule(l{2pt}r{2pt}){6-7} \cmidrule(l{2pt}r{2pt}){8-9} \cmidrule(l{2pt}r{2pt}){10-11} \cmidrule(l{2pt}r{2pt}){12-12}
\multirow{-2.5}{*}{Query ratio} & G & L & G & L & G & L & G & L & G & L & G,\,L \\ \midrule
5\%\,$\rightarrow$\,10\% & 5.99 \!  & \! 8.85 & 7.32 & 10.24 & 14.43 \!  & \! 17.36 & 8.20 & 11.13 & 13.88 \! & 20.87 & 17.10 \\ \hline
40\%\,$\rightarrow$\,45\% & 4.17 \!  & \! 33.59 & 7.02 & 33.99 & 10.01 \!  & \! 39.11 & 8.11 & 35.46 & 11.94 \! & 41.99 & 37.42 \\ \hline
75\%\,$\rightarrow$\,80\% & 3.95 \!  & \! 59.57 & 6.72 & 58.98 & 3.95 \!  & \! 62.62 & 7.71 & 60.26 & 10.46 \! & 65.16 & 56.81 \\
\bottomrule
\end{tabular}
}
\caption{Computational cost on CIFAR-10 with 4 layers of CNN. We averaged the query selection time\,(sec.) of all 10 clients, measured on a RTX 3090 GPU.}
\label{tab:time_cost}
\end{table}

\section{LoGo with Various FL Methods}
\label{sec:various_fl_algo}
We have further experimented with two federated learning algorithms, FedProx\cite{fedprox} and SCAFFOLD\cite{scaffold}, in conjunction with AL strategies.
Specifically, we compared our LoGo with baselines that demonstrated Top-1 or Top-2 performance more than once in Table\,\ref{tab:acc_comparision}. The experimental configurations are same to those used in Table\,\ref{tab:acc_comparision}.
As summarized in Table\,\ref{tab:compare_fedprox_scaffold}, LoGo consistently outperforms the baselines for both federated learning algorithms.
This observation suggests that LoGo is an orthogonal selection algorithm that can be integrated with any federated learning algorithm, having potential to improve the performance in various applications.

\begin{table}[h]
    \small
    \centering
    \renewcommand*{\arraystretch}{0.97}
    \resizebox{0.7\linewidth}{!}{
    \begin{tabular}{l|l|c|ccc|ccc}
        \toprule
         & &  & \multicolumn{3}{c|}{CIFAR-10} & \multicolumn{3}{c}{SVHN} \\
         \cmidrule(l{2pt}r{2pt}){4-6} \cmidrule(l{2pt}r{2pt}){7-9}
        \multirow{-2.5}{*}{FL\,algo.} & \multirow{-2.5}{*}{Method} & \multirow{-2.5}{*}{Model} & 20\% & 40\% & 60\% & 20\% & 30\% & 40\% \\
        \midrule
         \multirow{7}{*}{FedProx} & & G & 62.89 & 67.52 & 70.38 & 82.22 & 84.34 & 85.42  \\
         & \multirow{-2}{*}{Entropy} & L  & \underline{65.72} & \underline{70.57} & \underline{72.42} &82.08 & 83.73 & 85.30  \\
         \cline{2-9}
         & & G & 64.16 & 68.62 & 70.82 & \underline{83.09} & \textbf{84.65} & 85.84  \\
         & \multirow{-2}{*}{BADGE} & L  & 65.54 & 70.56 & 72.30 &81.99 & 84.17 & 85.17  \\
         \cline{2-9}
         & & G & 63.77 & 68.34 & 70.78 & 82.63 & 84.48 & \underline{85.94}  \\
         & \multirow{-2}{*}{ALFA-Mix} & L  & 63.44 & 67.83 & 70.31 &80.71 & 82.81 & 84.22  \\
         \cline{2-9}
          & \textbf{LoGo} & G, L & \textbf{65.79} & \textbf{70.61} & \textbf{72.61} & \textbf{83.12} & \underline{84.61} & \textbf{86.09}  \\
         \midrule
         \multirow{7}{*}{SCAFFOLD} & & G & 65.58 & 70.37 & 72.52 & 82.75 & 85.69 & 86.48  \\
         & \multirow{-2}{*}{Entropy} & L  & 67.96 & \underline{72.67} & \underline{74.06} & 83.24 & 84.30 & 85.82  \\
         \cline{2-9}
         & & G & 66.33 & 70.68 & 72.79 & 83.80 & 84.72 & \textbf{86.93}  \\
         & \multirow{-2}{*}{BADGE} & L  & \underline{68.27} & 72.52 & 73.79 & 83.40 & 84.61 & 86.16  \\
         \cline{2-9}
         & & G & 66.11 & 70.50 & 72.55 & \underline{84.11} & \textbf{85.72} & 86.14  \\
         & \multirow{-2}{*}{ALFA-Mix} & L  & 66.11 & 70.00 & 71.91 & 82.15 & 82.89 & 84.74  \\
         \cline{2-9}
          & \textbf{LoGo} & G, L & \textbf{68.33} & \textbf{72.77} & \textbf{74.48} & \textbf{84.29} & \underline{85.70} & \underline{86.73}  \\
         \bottomrule
    \end{tabular}}
    \caption{Classification accuracy on two benchmarks with FedProx\,($\mu$\,=\,0.01) and SCAFFOLD. We compared to three overwhelming baselines and averaged three random seeds. \textbf{Bold} and \underline{underline} mean Top-1 and Top-2, respectively.}
    \label{tab:compare_fedprox_scaffold}
    \vspace{-0.37cm}
\end{table}



\clearpage
\section{Detailed Experimental Results}
\label{sec:exp_detail_results}

In this Section, \ref{sec:detail_comp_matrix} summarizes all the comparison matrices results based on six categories: query selector, heterogeneity level, imbalance ratio, model architecture, budget size, and model initialization in Figure\,\ref{fig:bar_graph}. Figure\,\ref{fig:comp_selector}--\ref{fig:comp_model_init} are breakdowns of the matrix in Figure\,\ref{fig:comparision_matrix} into six categories.
\ref{sec:detail_comp_performance} provides comprehensive line plots for 38 experimental settings.
It can be seen that \algname{} overwhelms the baselines in most cases at both each category and detailed experimental setting level.


\subsection{Detailed Penalty Comparision Matrix}
\label{sec:detail_comp_matrix}

A maximum value of each matrix corresponds to Table\,\ref{tab:setting_summary}, and the bar plots in Figure\,\ref{fig:bar_graph} are calculated from these matrices.

\begin{figure*}[htb]
\centering
    \begin{subfigure}[b]{0.3\linewidth}
    \raggedleft
    \includegraphics[width=\linewidth]{figure/experiments/comparison/query_selector/global_comp.pdf}
    \caption{Global}
    \end{subfigure}
    \hspace{10pt}
    \begin{subfigure}[b]{0.3\linewidth}
    \raggedright
    \includegraphics[width=\linewidth]{figure/experiments/comparison/query_selector/local.pdf}
    \caption{Local-only}
    \end{subfigure}
    \caption{Pairwise penalty matrix for a query selector category. The maximum value of both matrices is 19.}
    \label{fig:comp_selector}
\end{figure*}

\vspace{-15pt}
\begin{figure*}[htb]
    \centering
    \begin{subfigure}[b]{0.3\linewidth}
    \includegraphics[width=\linewidth]{figure/experiments/comparison/dir/dir0_1.pdf}
    \caption{$\alpha = 0.1$}
    \end{subfigure}
    \hfill
    \begin{subfigure}[b]{0.3\linewidth}
    \includegraphics[width=\linewidth]{figure/experiments/comparison/dir/dir1_0.pdf}
    \caption{$\alpha = 1.0$}
    \end{subfigure}
    \hfill
    \begin{subfigure}[b]{0.3\linewidth}
    \includegraphics[width=\linewidth]{figure/experiments/comparison/dir/dir10_0.pdf}
    \caption{$\alpha = \infty$}
    \end{subfigure}
    \caption{Pairwise penalty matrix for a heterogeneity level category. The maximum value of three matrices is 30, 4, and 4.}
    \label{fig:comp_hetero}
\end{figure*}


\vspace{-15pt}
\begin{figure*}[htb]
    \centering
    \begin{subfigure}[b]{0.3\linewidth}
    \raggedleft
    \includegraphics[width=\linewidth]{figure/experiments/comparison/rho/rho_under2.pdf}
    \caption{$\rho <$ 2}
    \end{subfigure}
    \hspace{5pt}
    \begin{subfigure}[b]{0.3\linewidth}
    \raggedright
    \includegraphics[width=\linewidth]{figure/experiments/comparison/rho/rho_over2.pdf}
    \caption{$\rho \ge$ 2}
    \end{subfigure}
    \caption{Pairwise penalty matrix for imbalance ratio category. The maximum value of two matrices is 18 and 20, respectively.}
    \label{fig:comp_data_type}
\end{figure*}


\vspace{-15pt}
\begin{figure*}[!t]
    \centering
    \begin{subfigure}[b]{0.3\linewidth}
    \includegraphics[width=\linewidth]{figure/experiments/comparison/architecture/4cnn.pdf}
    \caption{Four Convolutional Neural Network}
    \end{subfigure}
    \hfill
    \begin{subfigure}[b]{0.3\linewidth}
    \includegraphics[width=\linewidth]{figure/experiments/comparison/architecture/resnet.pdf}
    \caption{ResNet-18}
    \end{subfigure}
    \hfill
    \begin{subfigure}[b]{0.3\linewidth}
    \includegraphics[width=\linewidth]{figure/experiments/comparison/architecture/mobilenet.pdf}
    \caption{MobileNet}
    \end{subfigure}
    \caption{Pairwise penalty matrix for a model architecture category. The maximum value of three matrices is 30, 4, and 4.}
    \label{fig:comp_model_arch}
\end{figure*}

\vspace{-15pt}
\begin{figure*}[!t]
    \centering
    \begin{subfigure}[b]{0.3\linewidth}
    \includegraphics[width=\linewidth]{figure/experiments/comparison/budget_size/budget_1.pdf}
    \caption{Budget 1\%}
    \end{subfigure}
    \hfill
    \begin{subfigure}[b]{0.3\linewidth}
    \includegraphics[width=\linewidth]{figure/experiments/comparison/budget_size/budget_5.pdf}
    \caption{Budget 5\%}
    \end{subfigure}
    \hfill
    \begin{subfigure}[b]{0.3\linewidth}
    \includegraphics[width=\linewidth]{figure/experiments/comparison/budget_size/budget_20.pdf}
    \caption{Budget 20\%}
    \end{subfigure}
    \caption{Pairwise penalty matrix for a budget size category. The maximum value of three matrices is 4, 30, and 4, respectively.}
    \label{fig:comp_budget_size}
\end{figure*}

% \vspace{-60pt}
\vspace{-15pt}
\begin{figure*}[!t]
    \centering
    \begin{subfigure}[b]{0.3\linewidth}
    \raggedleft
    \includegraphics[width=\linewidth]{figure/experiments/comparison/model_init/random.pdf}
    \caption{Random initialization}
    \end{subfigure}
    \hspace{10pt}
    \begin{subfigure}[b]{0.3\linewidth}
    \raggedright
    \includegraphics[width=\linewidth]{figure/experiments/comparison/model_init/continue.pdf}
    \caption{Continue initialization}
    \end{subfigure}
    \caption{Pairwise penalty matrix for a model initialization category. The maximum value of two matrices is 34 and 4.}
    \label{fig:comp_model_init}
\end{figure*}



\clearpage

\subsection{Detailed Performance Comparision}
\label{sec:detail_comp_performance}

For the line plots, we note that `Random' and `Ours' are independent of the query selector type.

\begin{figure*}[h!]
    \centering
    \includegraphics[width=0.8\linewidth]{figure/experiments/appendix/cifar10_total.pdf}
    \caption{Test accuracy on CIFAR-10, four layers of CNN, $\alpha=0.1$,  medium budget size\,(5\%), and random initialization. }
    \label{fig:app_cifar10}
\end{figure*}

\vspace{-15pt}
\begin{figure*}[h!]
    \centering
    \includegraphics[width=0.8\linewidth]{figure/experiments/appendix/svhn_total.pdf}
    \caption{Test accuracy on SVHN, four layers of CNN, $\alpha=0.1$,  medium budget size\,(5\%), and random initialization.}
    \label{fig:app_svhn}
\end{figure*}

\vspace{-15pt}
\begin{figure*}[!h]
    \centering
    \includegraphics[width=0.8\linewidth]{figure/experiments/appendix/path_total.pdf}
    \caption{Test accuracy on PathMNIST, four layers of CNN, $\alpha=0.1$,  medium budget size\,(5\%), and random initialization.}
    \label{fig:app_pathmnist}
\end{figure*}

\vspace{-15pt}
\begin{figure*}[!h]
    \centering
    \includegraphics[width=0.8\linewidth]{figure/experiments/appendix/derma_total.pdf}
    \caption{Test accuracy on DermaMNIST, four layers of CNN, $\alpha=0.1$,  medium budget size\,(5\%), and random initialization.}
    \label{fig:app_dermamnist}
\end{figure*}


\vspace{-15pt}
\begin{figure*}[!h]
    \centering
    \includegraphics[width=0.8\linewidth]{figure/experiments/appendix/organ_total.pdf}
    \caption{Test accuracy on OrganAMNIST, four layers of CNN, $\alpha=0.1$,  medium budget size\,(5\%), and random initialization.}
    \label{fig:app_organmnist}
\end{figure*}


\vspace{-15pt}
\begin{figure*}[!h]
    \centering
    \includegraphics[width=0.8\linewidth]{figure/experiments/appendix/cifar10_total_dir1.pdf}
    \caption{Test accuracy on CIFAR-10, four layers of CNN, \textbf{$\alpha=1.0$},  medium budget size\,(5\%), and random initialization.}
    \label{fig:app_cifar_dir1}
\end{figure*}

\vspace{-15pt}
\begin{figure*}[!h]
    \centering
    \includegraphics[width=0.8\linewidth]{figure/experiments/appendix/svhn_total_dir1.pdf}
    \caption{Test accuracy on SVHN, four layers of CNN, \textbf{$\alpha=1.0$},  medium budget size\,(5\%), and random initialization.}
    \label{fig:app_svhn_dir1}
\end{figure*}

\vspace{-15pt}
\begin{figure*}[!h]
    \centering
    \includegraphics[width=0.8\linewidth]{figure/experiments/appendix/cifar10_total_dir10.pdf}
    \caption{Test accuracy on CIFAR-10, four layers of CNN, \textbf{$\alpha=\infty$},  medium budget size\,(5\%), and random initialization.}
    \label{fig:app_cifar_dir10}
\end{figure*}



\vspace{-15pt}
\begin{figure*}[!h]
    \centering
    \includegraphics[width=0.8\linewidth]{figure/experiments/appendix/svhn_total_dir10.pdf}
    \caption{Test accuracy on SVHN, four layers of CNN, \textbf{$\alpha=\infty$},  medium budget size\,(5\%), and random initialization.}
    \label{fig:app_svhn_dir10}
\end{figure*}



\vspace{-15pt}
\begin{figure*}[!h]
    \centering
    \includegraphics[width=0.8\linewidth]{figure/experiments/appendix/cifar10_total_mobilenet.pdf}
    \caption{Test accuracy on CIFAR-10, MobileNet, $\alpha=0.1$,  medium budget size\,(5\%), and random initialization.}
    \label{fig:app_cifar10_mobile}
\end{figure*}



\vspace{-15pt}
\begin{figure*}[!h]
    \centering
    \includegraphics[width=0.8\linewidth]{figure/experiments/appendix/svhn_total_mobilenet.pdf}
    \caption{Test accuracy on SVHN, MobileNet, $\alpha=0.1$,  medium budget size\,(5\%), and random initialization.}
    \label{fig:app_svhn_mobile}
\end{figure*}



\vspace{-15pt}
\begin{figure*}[!h]
    \centering
    \includegraphics[width=0.8\linewidth]{figure/experiments/appendix/cifar10_resnet_total.pdf}
    \caption{Test accuracy on CIFAR-10, ResNet-18, $\alpha=0.1$,  medium budget size\,(5\%), and random initialization.}
    \label{fig:app_cifar10_resnet}
\end{figure*}



\vspace{-15pt}
\begin{figure*}[!h]
    \centering
    \includegraphics[width=0.8\linewidth]{figure/experiments/appendix/svhn_resnet_total.pdf}
    \caption{Test accuracy on SVHN, ResNet-18, $\alpha=0.1$,  medium budget size\,(5\%), and random initialization.}
    \label{fig:app_svhn_resnet}
\end{figure*}


\vspace{-15pt}
\begin{figure*}[!h]
    \centering
    \includegraphics[width=0.8\linewidth]{figure/experiments/appendix/cifar10_budget1_total.pdf}
    \caption{Test accuracy on CIFAR-10, four layers of CNN, $\alpha=0.1$,  small budget size\,(1\%), and random initialization.}
    \label{fig:app_cifar10_budget1}
\end{figure*}


\vspace{-15pt}
\begin{figure*}[!h]
    \centering
    \includegraphics[width=0.8\linewidth]{figure/experiments/appendix/svhn_budget1_total.pdf}
    \caption{Test accuracy on SVHN, four layers of CNN, $\alpha=0.1$,  small budget size\,(1\%), and random initialization.}
    \label{fig:app_svhn_budget1}
\end{figure*}


\vspace{-15pt}
\begin{figure*}[!h]
    \centering
    \includegraphics[width=0.8\linewidth]{figure/experiments/appendix/cifar10_total_budget20.pdf}
    \caption{Test accuracy on CIFAR-10, four layers of CNN, $\alpha=0.1$,  large budget size\,(20\%), and random initialization.}
    \label{fig:app_cifar10_budget20}
\end{figure*}


\vspace{-15pt}
\begin{figure*}[!h]
    \centering
    \includegraphics[width=0.8\linewidth]{figure/experiments/appendix/svhn_total_budget20.pdf}
    \caption{Test accuracy on SVHN, four layers of CNN, $\alpha=0.1$,  large budget size\,(20\%), and random initialization.}
    \label{fig:app_svhn_budget20}
\end{figure*}

\newpage
\vspace{-15pt}
\begin{figure*}[!h]
    \centering
    \includegraphics[width=0.8\linewidth]{figure/experiments/appendix/cifar10_total_continue.pdf}
    \caption{Test accuracy on CIFAR-10, four layers of CNN, $\alpha=0.1$,  medium budget size\,(5\%), and continue initialization.}
    \label{fig:app_cifar10_cont}
\end{figure*}



\vspace{-15pt}
\begin{figure*}[t]
    \centering
    \includegraphics[width=0.8\linewidth]{figure/experiments/appendix/svhn_total_continue.pdf}
    \caption{Test accuracy on SVHN, four layers of CNN, $\alpha=0.1$,  medium budget size\,(5\%), and continue initialization.}
    \label{fig:app_svhn_cont}
\end{figure*}



\end{document}
