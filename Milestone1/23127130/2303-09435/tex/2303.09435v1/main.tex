% This must be in the first 5 lines to tell arXiv to use pdfLaTeX, which is strongly recommended.
\pdfoutput=1
% In particular, the hyperref package requires pdfLaTeX in order to break URLs across lines.

\documentclass[11pt]{article}

% Remove the "review" option to generate the final version.
%\usepackage[review]{ACL2023}
\usepackage{ACL2023}

% Standard package includes
\usepackage{times}
\usepackage{latexsym}

% For proper rendering and hyphenation of words containing Latin characters (including in bib files)
\usepackage[T1]{fontenc}
% For Vietnamese characters
% \usepackage[T5]{fontenc}
% See https://www.latex-project.org/help/documentation/encguide.pdf for other character sets

% This assumes your files are encoded as UTF8
\usepackage[utf8]{inputenc}

% This is not strictly necessary, and may be commented out.
% However, it will improve the layout of the manuscript,
% and will typically save some space.
\usepackage{microtype}

% This is also not strictly necessary, and may be commented out.
% However, it will improve the aesthetics of text in
% the typewriter font.
\usepackage{inconsolata}


% If the title and author information does not fit in the area allocated, uncomment the following
%
%\setlength\titlebox{10cm}
%
% and set <dim> to something 5cm or larger.

%%%%%%%%%%%%%%%%%%%%%%%%%%%%%%%%%%
\usepackage{graphicx}
\usepackage{amsfonts}
\usepackage{amsmath}
\usepackage{bigdelim}
\usepackage{diagbox}
\usepackage{amsthm}
\usepackage{makecell}
\usepackage{mathtools}
\usepackage{booktabs}
\usepackage[shortlabels]{enumitem}
\graphicspath{ {figs/} }

\theoremstyle{remark}
\newtheorem*{question}{Question}

\newcommand{\tk}[1]{\textcolor{blue}{{#1}}}
\newcommand{\sy}[1]{\textcolor{red}{{#1}}}
\newcommand{\mg}[1]{\textcolor{purple}{{#1}}}
\newcommand{\lh}[1]{\textcolor{green}{{#1}}}
\newcommand{\lc}[1]{\textcolor{green}{{#1}}}

% Rounded color box
\definecolor{light_blue}{HTML}{cfdfff}
\usepackage[most]{tcolorbox}
\tcbset{on line, 
        boxsep=1pt, left=0pt,right=0pt,top=0pt,bottom=0pt,
        colframe=white,colback=light_blue,  
        highlight math style={enhanced}
        }

\newcommand{\quash}[1]{}  %Anything in \quash is ignored
\newcommand{\gpt}{\textsc{GPT-2}}
\newcommand{\bert}{\textsc{BERT}}
\newcommand{\bertlarge}{\textsc{BERT-large}}
\newcommand{\mask}{\texttt{[MASK]}}
\newcommand{\cls}{\texttt{[CLS]}}
\newcommand{\sep}{\texttt{[SEP]}}
\newcommand{\mat}{\texttt{mat}}
\newcommand{\id}{\texttt{id}}
\newcommand{\matl}{\texttt{mat}_{\ell \rightarrow \ell'}}
\newcommand{\matattnl}{\texttt{mat\_attn}_{\ell \rightarrow \ell'}}
\newcommand{\matffl}{\texttt{mat\_ffn}_{\ell \rightarrow \ell'}}
\newcommand{\matlnl}{\texttt{mat\_ln1\_ln2}_{\ell \rightarrow \ell'}}
\newcommand{\idl}{\texttt{id}_{\ell \rightarrow \ell'}}
\newcommand{\matlL}{\texttt{mat}_{\ell \rightarrow L}}
\newcommand{\matattnlL}{\texttt{mat\_attn}_{\ell \rightarrow L}}
\newcommand{\matfflL}{\texttt{mat\_ffn}_{\ell \rightarrow L}}
\newcommand{\matlnlL}{\texttt{mat\_ln1\_ln2}_{\ell \rightarrow L}}
\newcommand{\idlL}{\texttt{id}_{\ell \rightarrow L}}

\definecolor{blue(munsell)}{rgb}{0.0, 0.5, 0.69}
%%%%%%%%%%%%%%%%%%%%%%%%%%%%%%%%%%

\title{Jump to Conclusions: Short-Cutting Transformers\\With Linear Transformations}

% Author information can be set in various styles:
% For several authors from the same institution:
% \author{Author 1 \and ... \and Author n \\
%         Address line \\ ... \\ Address line}
% if the names do not fit well on one line use
%         Author 1 \\ {\bf Author 2} \\ ... \\ {\bf Author n} \\
% For authors from different institutions:
% \author{Author 1 \\ Address line \\  ... \\ Address line
%         \And  ... \And
%         Author n \\ Address line \\ ... \\ Address line}
% To start a seperate ``row'' of authors use \AND, as in
% \author{Author 1 \\ Address line \\  ... \\ Address line
%         \AND
%         Author 2 \\ Address line \\ ... \\ Address line \And
%         Author 3 \\ Address line \\ ... \\ Address line}

\author{Alexander Yom Din$^{1}$ ~~~~~ Taelin Karidi$^{1}$ ~~~~~ Leshem Choshen$^{1}$ ~~~~~
Mor Geva$^{2}$ 
\vspace{0.2cm} \\
$^1$Hebrew University of Jerusalem ~~~ $^2$Google Research \\
\small{\texttt{\{alexander.yomdin, taelin.karidi, leshem.choshen\}@mail.huji.ac.il}}, \small{\texttt{pipek@google.com}}}

\quash{
\author{Alexander Yom Din \\
  Hebrew University of Jerusalem \\ \texttt{alexander.yomdin@mail.huji.ac.il} \\\And
  Taelin Karidi \\
  Hebrew University of Jerusalem \\
  \texttt{taelin.karidi@mail.huji.ac.il} \\\And
  Leshem Choshen \\
  Hebrew University of Jerusalem \\ \texttt{leshem.choshen@mail.huji.ac.il} \\\And
  Mor Geva \\
  Google Research \\
  \texttt{pipek@google.com} \\}
}

\begin{document}
\maketitle



\begin{abstract}
% \vspace{-1em}
The diffusion-based generative models have achieved remarkable success in text-based image generation. However, since it contains enormous randomness in generation progress, it is still challenging to apply such models for real-world visual content editing, especially in videos. 
In this paper, we propose \texttt{FateZero}, a zero-shot text-based editing method on real-world videos without per-prompt training or use-specific mask. 
\RM{Specifically, different from a pipeline of two independent inversion and then generation stages, we find the intermediate attention maps during inversions store better structure and motion information. We thus reform them to temporally casual attention and replace them in the generation progress. To further reduce the unnecessary semantic leakage of source video and enhance the editing quality, we then remix the temporally casual attentions via the cross-attention features of the source prompt as the mask.}
To edit videos consistently, we propose several techniques based on the pre-trained models. Firstly, in contrast to the straightforward DDIM inversion technique, our approach captures intermediate attention maps during inversion, which effectively retain both structural and motion information. These maps are directly fused in the editing process rather than generated during denoising. To further minimize semantic leakage of the source video, we then fuse self-attentions with a blending mask obtained by cross-attention features from the source prompt. Furthermore, we have implemented a reform of the self-attention mechanism in denoising UNet by introducing spatial-temporal attention to ensure frame consistency.
Yet succinct, our method is the first one to show the ability of zero-shot text-driven video style and local attribute editing from the trained text-to-image model. We also have a better zero-shot shape-aware editing ability based on the text-to-video model~\cite{tuneavideo}. \RM{Besides video, our unified method also achieves state-of-the-art performance in zero-shot image editing.\chenyang{Need exp or remove the zero-shot image}} Extensive experiments demonstrate our superior temporal consistency and editing capability than previous works.
% The code will be released.
% \chenyang{emphasize: our observation at inversion time} \xiaodong{replacing the bold part to the actual pipeline: \textbf{Specifically, we work on replacing and mixing the attention maps between the inversion and generation since the self-attention map keeps the structure of the original natural image and the cross-attention is semantic-related, after remixing, we replace them in the corresponding generation steps for denoising.}}
% \footnote{Since there is no general video diffusion model is publicly available, we use one-shot video generation method~(Tune-A-Video~\cite{tuneavideo}) as the pretrained video diffusion model for zero-shot video editing\xiaodong{can be removed if we actually zero-shot on video}.}.
\end{abstract}
\section{Introduction}

The ability to reason about plans is critical for performing long-horizon tasks \citep{erol1996hierarchical, sohn2018hierarchical, sharma-etal-2022-skill}, compositional generalization \citep{corona-etal-2021-modular} and generalization to unseen tasks and environments \citep{shridhar2020alfred}.
Consider a simple long-horizon planning scenario where a robot is tasked with preparing a meal and serving it on the table. 
This presents a non-trivial planning problem since the agent needs to understand the sequence of operations required to perform the task and search for the relevant objects in the unfamiliar environment by interacting with various objects. %



Large language models have been recently shown to possess commonsense knowledge about the world such as object affordances and physical dynamics \citep{ouyang2022training,chowdhery2022palm}.
Early approaches considered text based environments and fine-tuned PLMs to predict actions given the history of past observations and actions \citep{jansen-2020-visually,micheli-fleuret-2021-language,yao-etal-2020-keep}.
Recent work has used this ability to reason about plans from text instructions in simulated household environments with simplifying assumptions such as text-only environment observations or feedback \citep{huang2022language,ahn2022can,li2022pre,logeswaran-etal-2022-shot}.


We focus on \emph{visually grounded planning} with PLMs --- the ability to adapt plans based on interaction and visual feedback from the environment.
While PLMs have strong planning commonsense priors, predictions from a PLM may not be directly realizable in the environment since the observation and action spaces are unknown.
This requires \emph{grounding} the PLM in the environment and adapting it to observe visual feedback, which is highly non-trivial.
Some prior works assume the availability of a pre-trained affordance function \citep{ahn2022can} or a success detector \citep{mirchandani2021ella}.
Notably, SayCan \citep{ahn2022can} completely decouples the PLM from observation information by selecting actions that have both high affordability (through a pre-trained affordance model) and high PLM likelihood.
Although this partially addresses the grounding problem, the use of visual feedback for action affordance alone is limited.
Often an agent must choose one of many affordable actions using information from observations.
For example, a driving agent should re-navigate and possibly turn around when encountering a ``road closed'' sign, but both turning around and driving forward are indistinguishable to SayCan because they are both affordable and the PLM is blind to observations.

Another workaround explored in prior work is translating the information in the visual observations to text using a pre-trained captioning system \citep{shridhar2021alfworld,huang2022language}.
However, it can be difficult to faithfully describe an image in words and information is lost in this inherently noisy process, which limits the information available to the planner.



Recent work shows that PLMs can be adapted for various natural language tasks by inserting tunable embeddings or soft prompts at the input of the PLM (also called prompt tuning or prefix tuning)~\citep{li-liang-2021-prefix,lester-etal-2021-power}.
This approach also extends to multi-modal understanding tasks such as image captioning \citep{mokady2021clipcap} and VQA \citep{tsimpoukelli2021multimodal} where images are encoded as soft prompts and finetuned for the target task.
Transformer based architectures have also been successfully applied to offline Reinforcement Learning in recent work \citep{chen2021decision,janner2021offline,li2022pre,reid2022can}.

Taking inspiration from these works, we propose the simple approach of embedding visual observations (`visual prompts') and \textit{directly inserting them as PLM input embeddings}.
The visual encoder and PLM are jointly trained for the target task, an approach we call \textbf{\oursfull}~(\ours).
By teaching the PLM to use observations for planning in an end to end manner, we remove the dependency on external data such as captions and affordability information that was used in prior work.
We show that this simple approach performs better than prior PLM-based planning approaches on two embodied planning benchmarks based on ALFWorld~\citep{shridhar2021alfworld} and Virtualhome~\cite{puig2018virtualhome}.



\section{Related Work}

%Here we summarize prior work on transfer learning and property inference.

%\shortsection{Transfer Learning}
%%Transfer learning reuses features learned by pre-trained models for new tasks, with the pretext that inherent similarities in the generic features will be useful for the downstream tasks and hence reducing their cost of downstream training. Specifically, the downstream model trainer will use a pre-trained upstream model as the starting point for the downstream training, with inclusion of (or replacement with) the task-specific classification layer/module. The downstream model is then trained by either updating all layers of the model (including ones reused from upstream model) or freezing some earlier layers of the reused parts as the ``feature extractor'' and only updating the rest. The latter approach is more popular as the reused feature extractors can already learn useful feature representations and the training cost is also much lower and affordable for individuals with limited computational resources. We study the vulnerability of the latter transfer learning approach in this paper. 


%\shortsection{Transfer Learning} 
Several works have demonstrated risks associated with transfer learning across a variety of attack goals. Wang et al.~\cite{wang2018great} and Yao et al.~\cite{yao2019latent} consider manipulating the upstream model such that the fine-tuned downstream models contain backdoors, misclassifying test inputs that contain predefined backdoor triggers. These transfer manipulations are tailored to their particular attack goals and cannot be applied for the property inference goal considered in this paper. Zou et al.~\cite{zou2020privacy} study the threat of membership inference attacks on transfer learning, but with normally trained upstream models.  
%\dnote{its clear that the goals are different for these attacks, but how similar are the methods?} \ynote{similarity of the methods? more details about the methods? do not know what is expected here}
%In contrast, we investigate the possibility of boosting the effectiveness of property inference by manipulating the upstream model training. % Schuster et al.~\cite{schuster2020humpty} show that the attacker can modify the corpus on which the word embedding is trained such that the downstream NLP models which use that embedding will behave abnormally.

%\shortsection{Property Inference}
The risk of property inference was introduced by Ateniese et al.~\cite{ateniese2015hacking}, % introduces the threat of inferring properties of the training data from pre-trained models, 
and several subsequent works have developed property inference (also known as distribution inference) attacks~\cite{Wang2022GroupPI, suri2022formalizing, Jurez2022BlackBoxAF, Hartmann2022DistributionIR}.
% Ganju et al.~\cite{ganju2018property} and Suri and Evans~\cite{suri2022formalizing} 
These works study property inference against normally trained models, and they launch attacks using a variety of black-box and white-box attacks. All the white-box attacks use meta-classifiers, which take the permutation-invariant representation~\cite{ganju2018property} of the model parameters as the features. We use the state-of-the-art white-box attack~\cite{suri2022formalizing} in our experiments.
%We will use the state-of-the-art white-box method proposed by Ganju et al.~\cite{ganju2018property} and later extended by suri et al.~\cite{suri2022formalizing} in this paper.
%\dnote{do we use these attacks?} 
Melis et al.~\cite{melis2019exploiting} and Zhang et al.~\cite{zhang2021leakage} focus on property inference in distributed training scenarios. In their settings, the attacker is a participant in the global model training and conducts property inference using meta-classifiers that are trained on model outputs or gradients. Similarly, Suri et al.~\cite{suri2022subject} focus on federated learning settings where the attacker is a participant (or the central server) that utilizes black-box attacks for inferring membership of data from particular subjects. %\dnote{if we use black-box attacks, explain which ones, or how ours are related to previous ones} 
For our experiments, We improve the black-box meta-classifier proposed by Zhang et al.~\cite{zhang2021leakage} using the ``query tuning'' technique in Xu et al.~\cite{xu2019detecting}. 

The closest works to ours are Chase et al.~\cite{saeed} and Chaudhari et al.~\cite{Chaudhari2022SNAPEE}, which both consider a scenario where the attacker can manipulate some of the training data of the model to induce a model that significantly increases property inference risk.
% \dnote{it enables precise property inference attacks?}.
These works assume an adversary with the ability to poison the victim's training data, while the adversary in our scenario has no access to the victim's training data, and therefore, their methods are not applicable.
% \dnote{example how different from ours, and why the methods are not applicable}
%Thus, their methods are not applicable to our transfer learning scenario.
%Their methods rely on inducing certain behavior correlated with the properties to be inferred, and thus are not applicable to our transfer learning scenario. \anote{Still a bit unclear why that is the case.}
%
There are also works similar to ours that leverage ``adversarial initializations'' for attack purposes.
% \cite{grosse2019adversarial, boenisch2021curious, wen2022fishing, fowl2021robbing}.
Grosse et al.~\cite{grosse2019adversarial} focus on scenarios where the attacker can control the parameter initialization of a model, and demonstrate that the attacker can use special initializations to damage the performance of the trained model. %This attack is orthogonal to ours.
Other works \cite{boenisch2021curious, wen2022fishing, fowl2021robbing} show that the malicious central server in a federated learning protocol can reconstruct some training samples via falsifying the global model in some training rounds and then analyzing the submitted gradients. These kinds of attacks do not apply to our transfer-learning scenario since the attacker cannot access the downstream gradients, and can only manipulate the upstream training.

\iffalse %%%%%%%%%%%%%%%%%%%%%%%%%%%%%%%%

In this section, we provide the background and also the summary of prior attacks on transfer learning (Section~\ref{sec:transfer_learning}) and property inference (Section~\ref{sec:property_inference}). Then, we introduce the closely related manipulation attacks against machine learning models to boost different privacy risks in Section~\ref{sec:active_inference_attacks}.

%\anote{Do we really need a dedicated section for this? It's barely 2 paragraphs right now.}

%\dnote{the most closely related work to ours are works that attempt to amplify inference attacks by poisoning models, the two most relevant I know of are \url{https://www.computer.org/csdl/proceedings-article/sp/2022/131600b569/1CIO8nmuota} and \url{https://arxiv.org/abs/2204.00032}, but need to look thoroughly for others. We should definitely be describing this and relating it to our work, probably in the introduction. Most of what is here is Background, but should be clear what this section is for (not muddling background and related work)}

\subsection{Transfer Learning} \label{sec:transfer_learning}
Transfer learning reuses features learned by pre-trained models for new tasks, with the pretext that inherent similarities in generic features can be useful for downstream tasks, thus reducing the cost of downstream training. Specifically, the downstream model trainer uses a pre-trained upstream model as the starting point for downstream training, with the inclusion (or replacement) of task-specific classification layers/modules. The downstream model is then trained by either updating all layers of the model (including ones reused from the upstream model) or freezing some earlier layers of the reused parts as the ``feature extractor'' and only updating the rest. The latter approach is more popular as the reused feature extractors can already learn useful feature representations and the training cost is also much lower and affordable for individuals with limited computational resources. We study the vulnerability of the latter transfer learning approach in this paper. 
%mainly in two ways:  1) all the layers (including ones reused from ) and tune the full model; the other one is to freeze some earlier layers of the model as the feature extractor and only tune the rest later layers. The second update strategy could achieve better efficiency since the frozen layers can already produce meaningful feature representations~\cite{wang2018great,yao2019latent}, and we will study the transfer learning using this strategy. 

Recently, various attacks have been proposed for the transfer learning setting, but with different attack goals from ours. Wang et al.~\cite{wang2018great} generate adversarial examples against black-box student models that transfer knowledge from publicly available teacher models without repeated queries. Yao et al.~\cite{yao2019latent} propose to manipulate the upstream model such that the downstream models derived from the upstream model contain backdoors, which would misclassify test inputs that contain some predefined backdoor triggers. Zou et al.~\cite{zou2020privacy} study the threat of membership inference attacks on transfer learning and the upstream models are trained normally. In contrast, we investigate the possibility of boosting the effectiveness of property inference by manipulating the upstream model training. Schuster et al.~\cite{schuster2020humpty} show that the attacker can modify the corpus on which the word embedding is trained such that the downstream NLP models which use that embedding will behave abnormally.

%This additionally allows model trainers to achieve satisfactory performance with limited training samples, leading to reduced computational costs. The most common approach reuses parameters in the earlier layers of the pre-trained model, either by fixing them as the feature extractor or just using them for initialization, to conduct downstream training.

\subsection{Property Inference} \label{sec:property_inference}

\shortsection{Property Inference Attacks} In property inference attacks, the adversary aims to infer some sensitive properties of some data, given a model trained on it. For example, the adversary may be interested in sensitive properties like the presence of people of a specific race in the dataset~\cite{ateniese2015hacking, melis2019exploiting}), or even be curious about the 
the statistics of the training set (e.g, the ratio of people with a specific gender~\cite{saeed, ganju2018property, suri2022formalizing, zhang2021leakage}).


Ateniese et al.~\cite{ateniese2015hacking} were the first to identify the threat of inferring properties of the training data from pre-trained models. Ganju et al.~\cite{ganju2018property} and Suri and Evans~\cite{suri2022formalizing} 
study property inference against normally trained models, and they launch attacks using white-box meta-classifiers, which utilize the permutation-invariance representation~\cite{ganju2018property} of the model parameters, while other works focus on distributed training~\cite{zhang2021leakage} where the attacker is a participant in the global model training and conducts property inference using meta-classifiers trained on model outputs. Similarly, Suri et al.~\cite{suri2022subject} focus on federated learning, where the attacker is a participant (or the central server) that utilizes black-box attacks for inferring membership of data from particular subjects. Chase et al.~\cite{saeed} propose an active property inference attack for data poisoning scenarios, which we will cover and compare to in Section~\ref{sec:active_inference_attacks}.

%The closest work to ours are by Chase et al.~\cite{saeed} and Tramer et al.~\cite{tramer2022truth}. In their work, the attacker can manipulate some of the training data of the model such that a model trained (from scratch) on the poisoned data has an increased inference risk. However, their methods are not applicable to the transfer learning scenario. 
%In this work, we will focus on the property inference in transfer learning scenarios in which the attacker releases the upstream model and infer sensitive properties of the downstream models tuned from that upstream model.
% 

\shortsection{Defenses}
Defending against property inference attacks is an open problem. There are no studies in the current literature on active adversaries, and only a couple on passive ones. Ma et. al.~\cite{ma2021nosnoop} propose a defense against property inference attacks on data batches in the  collaborative learning setting. However, adversaries in the transfer-learning setting do not have access to batch-wise gradients of the downstream trainer. Chen and Ohrimenko~\cite{chen2022protecting} utilize mechanisms that add carefully-crafted noise to features to provide theoretical guarantees against inference adversaries, but focus on query-based access to the underlying dataset, not a machine learning model trained on it. These existing defenses thus do not apply to our threat model.

%propose a framework that reduces property inference to Boolean functions of individual members, posing the ratio of members satisfying the given function in a dataset as the property. These property inference attacks have since then been proposed as distribution inference attacks~\cite{suri2022formalizing}, presenting such attacks as inferring properties of the distributions used to sample datasets, differentiating them from exact inference attacks like dataset inference~\cite{maini2021dataset}. Nearly all property inference attacks use meta-classifiers to perform inference: training models on versions of datasets with and without the target property, followed by training a meta-classifier on top of these classifiers's model representations. These representations can take several forms: using model weights themselves with permutation-invariance~\cite{ganju2018property}, or model activations or logits for a generated set of query points~\cite{xu2019detecting}. However, the capability of such approaches is limited: the most that these attacks have been shown to work is medium-sized convolutional networks on the CelebA dataset~\cite{suri2022formalizing}.


\subsection{Active Privacy Attacks} \label{sec:active_inference_attacks}
% Perhaps the closely related works to ours as ones that proactively enhance the effectiveness of privacy attacks by manipulating the model training process in certain ways~\cite{saeed, melis2019exploiting, nasr2019comprehensive, tramer2022truth}. 
%shown that the adversary can, by using proactive ways, achieve stronger attacks that infer private information from deep learning systems~\cite{nasr2019comprehensive, melis2019exploiting, tramer2022truth, saeed}. In this section, we introduce the ones that are close to ours.

In the decentralized federated learning training, by submitting specially crafted gradients to the central server, malicious agents can increase membership inference risk~\cite{nasr2019comprehensive} and property inference risks~\cite{melis2019exploiting} of other benign agents' training data. However, these attacks do not apply to transfer learning scenario, as the attacker cannot control model gradients of downstream training. In the centralized setting, researchers propose attacks to poison the victim's training data such that the impacts of attribute inference and membership inference~\cite{tramer2022truth} and property inference~\cite{saeed} attacks are amplified on the poisoned model.
The ability to poison the victim's data is a threat model orthogonal to ours, since we have no access to the victim's downstream data. While there is scope to combine such approaches for stronger attacks (albeit with stronger access assumptions), we choose to focus on the scenario with no read/write access to the victim's data.

\fi %%%%%%%%%%%%%%%%%%%%%%%%%%%%%%%%

\section{Linear Shortcut Across Blocks}
\label{sec:layer_jump}

To use a hidden representation from layer $\ell<L$ as a final representation, we propose to cast it using linear regression, while skipping the computation in-between these layers. More generally, this approach can be applied to cast any $\ell$-th hidden representation to any subsequent layer $\ell'>\ell$.


\subsection{Method}
\label{subsec:methodology_linear_shortcut}

Given a source layer $\ell$ and a target layer $\ell'$ such that $0 \leq \ell < \ell' \leq L$, our goal is to learn a mapping
%$A_{\ell', \ell} \in \mathbb{R}^{d_h \times d_h}$
from hidden representations at layer $\ell$ to those at layer $\ell'$. To this end, we first collect a set of corresponding hidden representation pairs $(h^\ell, h^{\ell'})$. Concretely, we run a set $\mathcal{T}$ of input sequences through the model, and for each input $s$, we extract the hidden representations $h_{i_s}^{\ell}, h_{i_s}^{\ell'}$, where $i_s$ is a random position in $s$.
Next, we learn a matrix $A_{\ell', \ell} \in \mathbb{R}^{d_h \times d_h}$ by fitting linear regression over $\mathcal{T}$, i.e., $A_{\ell', \ell}$ is a numerical minimizer for:
$$ A \mapsto \sum_{s \in \mathcal{T}} || A \cdot h_{i_s}^\ell - h_{i_s}^{\ell'} ||^2,$$ 
and define the mapping of a representation $h$ from layer $\ell$ to layer $\ell'$ as:
\begin{equation}
\label{eq:linear_jump}
    \matl{} (h) \coloneqq A_{\ell', \ell} \cdot h.
\end{equation}


\subsection{Baseline}
\label{subsec:baseline}

We evaluate 
% our method against 
the prevalent approach of ``reading'' hidden representations directly, without any transformation. 
Namely, the propagation of a hidden representation from layer $\ell$ to layer $\ell'$ is given by the identity function, dubbed \id{}:

$$ \idl{} (h) \coloneqq h.$$

% Notably, 
This baseline 
assumes that representations at different layers operate in the same linear space.

\subsection{Quality of Fit}
\label{subsec:experiments_r2}

We first evaluate our method by measuring how well the learned linear mappings approximate the representations at the target layer. To this end, we calculate the (coordinate-averaged) $r^2$-score of our mapping's outputs with respect to the representations obtained from a full inference pass, and compare to the same for the \id{} baseline.


\paragraph{Models.}

We use \gpt{} \cite{radford2019language}, a decoder-only auto-regressive LM, with $L = 48$, $d_h = 1600$, and \bert{} \cite{devlin-etal-2019-bert}, an encoder-only model trained with masked language modeling, with $L=24$, $d_h=1024$.
% \footnote{\label{footnote:hf}We use models and data from Huggingface \cite{wolf-etal-2020-transformers,lhoest-etal-2021-datasets}.}
%For masked token prediction, we use a masked LM head pre-trained for our \bert{} model.

% \footnote{Specifically, we use the Huggingface Transformers \cite{wolf-etal-2020-transformers} implementations of all these models.}

%\sy{We use \gpt{} \cite{radford2019language}, a decoder-only auto-regressive LM, coming in four scales; $\texttt{gpt2}$ ($L = 12$, $d_h = 768$), $\texttt{gpt2-medium}$ ($L = 24$, $d_h = 1024$), $\texttt{gpt2-large}$ ($L = 36$, $d_h = 1280$) and $\texttt{gpt2-xl}$ ($L = 48$, $d_h = 1600$). Also, we use \bert{} \cite{devlin-etal-2019-bert}, an encoder-only model trained with masked language modeling, coming in two scales;  \texttt{bert-base-uncased} ($L=12$, $d_h=768$) and \texttt{bert-large-uncased} ($L=24$, $d_h=1024$). For masked token prediction, we use masked LM heads pre-trained for our models. Specifically, we use the Huggingface Transformers \cite{wolf-etal-2020-transformers} implementations of all these models. The plots presented in this section are for $48$-layered \gpt{} and $24$-layered \bert{}.}

%\sy{We use \gpt{} \cite{radford2019language}, a decoder-only auto-regressive LM, in the Huggingface \cite{wolf-etal-2020-transformers} implementation\footnote{\url{https://huggingface.co/gpt2}}, coming in four scales; $\texttt{gpt2}$ ($L = 12$, $d_h = 768$), $\texttt{gpt2-medium}$ ($L = 24$, $d_h = 1024$), $\texttt{gpt2-large}$ ($L = 36$, $d_h = 1280$) and $\texttt{gpt2-xl}$ ($L = 48$, $d_h = 1600$). Also, we use \bert{} \cite{devlin-etal-2019-bert}, an encoder-only model trained with masked language modeling, in the Hugginface implementation, coming in two scales;  \texttt{bert-base-uncased}\footnote{\url{https://huggingface.co/bert-base-uncased}} ($L=12$, $d_h=768$) and \texttt{bert-large-uncased}\footnote{\url{https://huggingface.co/bert-large-uncased}} ($L=24$, $d_h=1024$). For masked token prediction, we use the \texttt{BertForMaskedLM} heads from Huggingface, pretrained for these models. The plots presented in this section are for $48$-layered \gpt{} and $24$-layered \bert{}.}

\paragraph{Data.}
We sample random sentences from Wikipedia,
% \footref{footnote:hf} 
collecting 9,000 (resp. 3,000) sentences for the training set $\mathcal{T}$ (resp. validation set $\mathcal{V}$).\footnote{We use sentences rather than full documents to simplify the analysis.}
%\sy{We use two data sources to evaluate our method. One is Wikiepdia \cite{lhoest-etal-2021-datasets}\footnote{\url{https://huggingface.co/datasets/wikipedia}}; we use \texttt{spaCy}\footnote{\url{https://spacy.io/}} to divide documents into sentences\footnote{We use sentences rather than full documents to simplify the analysis.}\footnote{We pick randomly a Wikipedia document and then pick randomly a sentence ending in a newline character in it. \sy{[maybe this footnote is not needed?]}}, collecting 9,000 (resp. 3,000) random sentences for the training set $\mathcal{T}$ (resp. validation set $\mathcal{V}$). The second is a news article sentences dataset, the 10K English 2020 news sentences corpus
% \footnote{\url{https://downloads.wortschatz-leipzig.de/corpora/eng_news_2020_10K.tar.gz}} from the Leipzig Corpora Collection \cite{goldhahn-etal-2012-building}, which we randomly divide into a training set $\mathcal{T}$ consisting of 9,000 examples and a validation set $\mathcal{V}$ consisting of 1,000 examples.
% We truncate sentences to the maximal token length allowed by the model \mg{do we ever need to truncate? a sentence has about 10 words and the max. input len is thousands} \sy{[I surely did not need to in Leipzig, but discovered (via a transformers runtime warning) that I do need to for some (probably a minority) of the Wikipedia sentences. This probably has to do with that it is not really ``sentences" necessarily, for example, I noticed that it has some listings or something like that (bulleted items)... So some minority might get very long I guess...]}.
For each example $s$, we select a random position $i_s$ and extract the hidden representations $h_{i_s}^{\ell}$ at that position from all the layers.
For \bert{}, we first replace the input token at position $i_s$ with a \mask{} token, as our motivation is interpreting predictions, which are obtained via masked tokens in \bert{} (see \S\ref{subsec:BERT}).
Thus, in this case, the hidden representations we consider
%in the case of \bert{}
are of \mask{} tokens only.
%As we observed highly similar results for the two data sources across all our experiments, throughout the paper we will mainly report results for Wikipedia (except for \S\ref{sec:robustness}, where we cross-validate).


\begin{figure}[t]
\includegraphics[scale=0.2]{figs/r2_scores_48.pdf}
% \includegraphics[width=\columnwidth]{figs/r2_scores_48.pdf}
\caption{The coordinate-averaged $r^2$-score of $\matl{}$ (left) and $\idl{}$ (right) (\gpt{}).}
\label{fig:r2_scores}
\end{figure}


\begin{figure}[t]
\setlength{\belowcaptionskip}{-10pt}
\includegraphics[scale=0.2]{figs/bertmask_r2_scores_24.pdf}
% \includegraphics[width=\columnwidth]{figs/bertmask_r2_scores_24.pdf}
\caption{The coordinate-averaged $r^2$-score of $\matl{}$ (left) and $\idl{}$ (right) (\bert{}).}
\label{fig:bertmask_r2_scores}
\end{figure}



\paragraph{Evaluation.}
For every pair of layers $\ell, \ell'$, such that $0 \leq \ell < \ell' \leq L$, we use the training set $\mathcal{T}$ to fit linear regression as described in \S\ref{subsec:methodology_linear_shortcut}, and obtain a mapping $\matl{}$. 
Next, we evaluate the quality of $\matl{}$ as well as of $\idl{}$ using the $r^2$-coefficient, uniformly averaged over all coordinates. Concretely, we compute the $r^2$-coefficient of each of the predicted representations $\matl{} (h_{i_s}^{\ell})$ and $\idl{} (h_{i_s}^{\ell})$ versus the true representations $h_{i_s}^{\ell'}$
over all $s \in \mathcal{V}$.
%as we vary $s \in \mathcal{V}$.
%for every $s \in \mathcal{V}$.



\paragraph{Results.}
Results for \gpt{} and \bert{} are presented in Figs.~\ref{fig:r2_scores} and~\ref{fig:bertmask_r2_scores}, respectively.
In both models, \mat{} consistently yields better approximations than \id{}, as it obtains higher $r^2$-scores (in blue) across the network. 
This gap between \mat{} and \id{} is especially evident in \bert{}, where \id{} completely fails to map the representations between most layers, suggesting that hidden representations are modified  substantially by every transformer block.
Overall, this highlights the shortcoming of existing practices to inspect representations in the same linear space, and the gains from using our method to approximate future layers.
% in the network.
\section{Linear Shortcut for Language Modeling}
\label{sec:prediction}

We saw that our method approximates future hidden representations substantially better than a naive propagation. 
In this section, we will show that this improvement also translates to better predictive abilities from earlier layers. Specifically, we will use our method to estimate how often intermediate representations encode the final prediction, in the context of two fundamental LM tasks; next token prediction and masked token prediction.

\paragraph{Evaluation Metrics.}
Let $h, h' \in \mathbb{R}^{d_h}$ be a final representation and a substitute final representation obtained by some mapping, and denote by $\delta (h), \delta (h') \in \mathbb{R}^{d_v}$ their corresponding output probability distributions (obtained through projection to the output vocabulary -- see details below). 
We measure the prediction quality of $h'$ with respect to $h$ using two metrics:
\begin{itemize}
[leftmargin=*,topsep=1pt,parsep=1pt]
    \item \textbf{Precision@$k$} ($\uparrow$ is better): This checks whether the token with the highest probability according to $\delta(h')$ appears in the top-$k$ tokens according to $\delta(h)$. Namely, we sort $\delta(h)$ and assign a score of $1$ if $\arg\max(\delta(h'))$ appears in the top-$k$ tokens by $\delta(h)$, and $0$ otherwise.
    
    \item \textbf{Surprisal} ($\downarrow$ is better): We measure the minus log-probability according to $\delta(h)$, of the highest-probability token according to $\delta(h')$. Intuitively, low values mean that the model sees the substitute result as probable and hence not surprising.
\end{itemize}

\noindent We report the average Precision@$k$ and Surprisal over the validation set $\mathcal{V}$.



\subsection{Next Token Prediction}
\label{subsec:next_token_prediction_task}

Auto-regressive LMs output for every position a probability distribution over the vocabulary for the next token. Specifically, the output distribution for every position $i$ is given by $\delta (h_i^L)$, where:
\begin{equation}\label{eq:output_distribution}
    \delta (h) = \texttt{softmax} ( E^\top \cdot h) \in \mathbb{R}^{d_v}
\end{equation}
For some LMs, including \gpt{}, a layer normalization $\texttt{ln\_f}$ is applied to the final layer representation before this conversion (i.e., computing $\delta (\texttt{ln\_f}(h))$ rather than $\delta (h)$).

Recall that our goal is to measure how well this distribution can be estimated from intermediate representations, i.e. estimating $\delta (h_i^L)$ from $\delta (h_i^\ell)$ where $\ell<L$. To this end, we first run examples from the validation set through the model, while extracting for each example $s$ the hidden representation of a random position $i_s$ at every layer. Next, we apply our mappings $\matlL{}$ and the $\idlL{}$ baseline to cast the hidden representations of every layer $\ell$ to final layer substitutes (see \S\ref{sec:layer_jump}). Last, for each layer, we convert its corresponding final-layer substitute to an output distribution (Eq.~\ref{eq:output_distribution}) and compute the average Precision@$k$ (for $k=1,5,10$) and Surprisal scores with respect to the final output distribution, over the validation set.

\paragraph{Results.}
Figs.~\ref{fig:pre} and~\ref{fig:surp} show the average Precision@$k$ and Surprisal scores per layer in $48$-layered \gpt{}, respectively (the plots for the other \gpt{} models are presented in \S\ref{sec:app_scale}). Across all layers, \mat{} outperforms \id{} in terms of both scores, often by a large margin (e.g. till layer $44$ the Precision@$1$ achieved by \mat{} is bigger than that of $\id{}$ by more than $0.2$). 
This shows that linear mappings enable not just better estimation of final layer representations, but also of the predictions they induce. Moreover, the relatively high Precision@$k$ scores of \mat{} in early layers ($0.62$-$0.82$ for $k=10$, $0.52$-$0.74$ for $k=5$, and $0.28$-$0.45$ for $k=1$) suggest that early representations already encode a good estimation of the final prediction. Also, the substantially lower Surprisal scores of \mat{} compared to \id{} imply that our method allows for a more representative reading into the layer-wise prediction-formation of the model than allowed through direct projection to the vocabulary.

\begin{figure}[t]
\centering
\includegraphics[scale=0.4]{figs/pre_48.pdf}
\caption{Precision@$k$ ($k = 1,5, 10$) of $\matlL{}$ and $\idlL{}$ for next token prediction in $48$-layered \gpt{}.}
\label{fig:pre}
\end{figure}

\begin{figure}[t]
\centering
\includegraphics[scale=0.35]{figs/surp_48.pdf}
\caption{Surprisal for $\matlL$ and the baseline $\idlL{}$ ($48$-layered \gpt{} next token prediction task). A 95\% confidence interval surrounds the lines.}
\label{fig:surp}
\end{figure}

\subsection{Masked Token Prediction}
\label{subsec:BERT}

We now conduct the same experiment for the task of masked language modeling, where the model predicts a probability distribution of a masked token in the input rather than the token that follows the input. Unlike next token prediction, where the output distribution is computed from representations of varying input tokens, in masked token prediction the output is always obtained from representations of the same input token (i.e. \texttt{[MASK]}).

For this experiment, we use \bert{}, on top of which we use a pretrained masked language model head $\delta$; given a token sequence $s$, a \mask{} token inside it and its final representation $h$, $\delta (h) \in \mathbb{R}^{d_v}$
 is a probability distribution over tokens giving the model's assessment
 of the likelihood of tokens to be fitting in place of the \mask{} token in $s$.


\begin{figure}[t]
\centering
\includegraphics[scale=0.4]{figs/bertmask_pre_24.pdf}
\caption{Precision@$k$ ($k = 1,5, 10$) for  $\matlL{}$ and the baseline $\idlL{}$ ($24$-layered \bert{} masked token prediction task).}
\label{fig:bertmask_pre}
\end{figure}

\begin{figure}[t]
\centering
\includegraphics[scale=0.35]{figs/bertmask_surp_24.pdf}
\caption{Surprisal for $\matlL{}$ and the baseline $\idlL{}$ ($24$-layered \bert{} masked token prediction task). A 95\% confidence interval surrounds the lines.}
\label{fig:bertmask_surp}
\end{figure}

\paragraph{Results.}
Figs.~\ref{fig:bertmask_pre} and~\ref{fig:bertmask_surp} present the average Precision@$k$ and Surprisal scores per layer in $24$-layered \bert{} (the plots for the $12$-layered \bert{} model are presented in \S\ref{sec:app_scale}), overall showing trends similar to those observed for next token prediction in \gpt{} (\S\ref{subsec:next_token_prediction_task}). This is despite the differences between the two tasks and the considerable architectural differences between \bert{} and \gpt{}.
Notably, the superiority of \mat{} over \id{} in this setting is even more prominent; 
while \mat{}'s precision is between $0.2-0.6$ in the first ten layers (Fig.~\ref{fig:bertmask_pre}), \id{}'s precision for all values of $k$ is close to zero, again strongly indicating that our method allows for better reading into early layer hidden representations. 
More generally, \mat{} improves the Precision@$1$ of \id{} by more than $17\%$ at most layers, and unveils that a substantial amount of predictions ($>25\%$ starting from layer $3$) appear already in the very first layers.
Interestingly, the (rough) divide between the first half of layers and last half of layers for $\id{}$ in Figs.~\ref{fig:bertmask_pre},~\ref{fig:bertmask_surp} seems to align with the two-hump shape of the blue region for $\mat{}$ in Fig.~\ref{fig:bertmask_r2_scores}.

\paragraph{Analysis.}
We manually compare the predictions of our mapping $\matlL{}$ with $\idlL{}$, for a $24$-layered \bert{} model.  Concretely, we select 50 random sentences from the Leipzig dataset. Next, for each layer $\ell$, we manually analyze how many of the top-$5$ tokens according to $\matlL{}$ and $\idlL{}$ fit into context. We consider a token to fit into context if it is grammatically plausible within the sentence (see Tab.~\ref{tab:manual} for concrete examples).
In the resulting $1250$ instances (i.e. $50$ sentences $\times$ $25$ representations), we observe a substantially higher plausibility rate of $85.36\%$ for \mat{} compared to $52.8\%$ for \id{}. In fact, only in less than $4.3\%$ of the instances there are more plausible tokens among the top-$5$ tokens according to \id{} than among the top-$5$ tokens according to \mat{}, further supporting the Surprisal results above.

\begin{table*}
\footnotesize
\setlength{\belowcaptionskip}{-15pt}
\begin{tabular}{p{0.3\linewidth}ccccc}
& $\texttt{id}_{4 \rightarrow 24}$ & $\texttt{mat}_{4 \rightarrow 24}$ & $\texttt{id}_{12 \rightarrow 24}$ & $\texttt{mat}_{12 \rightarrow 24}$ & $\texttt{id}_{24 \rightarrow 24}$ \\ \midrule
\multirow{5}{=}{aldridge had shoulder surgery in \mask{}.} & fellowship & \tcbox{time} & cyclist & \tcbox{2009} & \tcbox{september} \\
& employment & \tcbox{it} & emergencies & \tcbox{2008} & \tcbox{november} \\
& agreement & her & seniors & \tcbox{2010} & \tcbox{december} \\
& \#\#ostal & them & cycling & \tcbox{2006} & \tcbox{august} \\
& \#\#com & work & \tcbox{pennsylvania} & \tcbox{2007} & \tcbox{july} \\ \midrule
\multirow{5}{=}{on your next view you will be asked to \mask{} continue reading.} & \#\#com & be & be & be & \tcbox{please} \\
& accreditation & get & undergo & \tcbox{please} & \tcbox{simply} \\ 
& $	\copyright$ & go & spartans & help & \tcbox{also} \\ 
& fellowship & \tcbox{help} & seniors & \tcbox{simply} & \tcbox{again} \\ 
& summer & have & * & say & \tcbox{immediately} \\ \bottomrule
\end{tabular}
\caption{Examples of top-$5$ predictions at layers $4$, $12$ and $24$, under the mappings $\matlL{}$ and $\idlL{}$, for a $24$-layered \bert{} model. Grammatically plausible predictions (according to a human annotator) are marked in \tcbox{blue}. Note that at layer $24$ the predictions of $\matlL{}$ and $\idlL{}$ are the same (by definition).} 
\label{tab:manual}
\end{table*}

\section{Implication to Early Exiting}
\label{sec:applications}

%The fact that it is often possible to approximate
The possibility of approximating
the final prediction already in the early layers has important implications for efficiency; applying our linear mapping instead of executing transformer blocks of quadratic time complexity, could save a substantial portion of the computation. In this section, we demonstrate this in the context of early exiting.

When 
% performing transformer model inference under 
using an early exit strategy \cite{schwartz-etal-2020-right, xin-etal-2020-deebert, schuster2022confident}, one aims at deciding dynamically at which layer to stop the computation and ``read'' the prediction from the hidden representation of that layer.
More precisely, under a confidence measure paradigm, one decides to stop the computation for a position $i$ at layer $\ell$ based on a confidence criterion, that is derived from casting the hidden representation $h_i^\ell$ as a final-layer representation and converting it to an output probability distribution. Specifically, following \citet{schuster2022confident}, a decision to exit is made if the difference between the highest and the second highest probabilities is bigger than $$ 0.9 \cdot \lambda + 0.1 \cdot {\rm exp} (-4 i / N),$$
where $N$ is the average length of the input until position $i_s$ for $s \in \mathcal{V}$, and $\lambda$ is a hyper-parameter.

\begin{figure}[t]
\setlength{\belowcaptionskip}{-10pt}
\centering
\includegraphics[width=\columnwidth]{figs/ee_gpt2bert.pdf}
\caption{Precision@$1$ with early exit and ``fixed exit'', applied to the $24$-layer \gpt{} for next token prediction (left) and the $24$-layer \bert{} for masked token prediction (right). Varying the confidence parameter $\lambda$, the $x$-coordinate is the average number of layers processed before an early exit decision is reached.}
\label{fig:ee_gpt2bert}
\end{figure}

\quash{
\begin{figure}[t]
\setlength{\belowcaptionskip}{-10pt}
\centering
\includegraphics[scale=0.35]{figs/ee_pre1_24.pdf}
\caption{Precision@$1$ for the various early exit methods, and previous ``fixed exit'' methods for comparison ($24$-layer \gpt{} next token prediction task). Varying the confidence parameter $\lambda$, the $x$-coordinate is the average number of layers processed before an early exit decision is reached.}
\label{fig:ee_pre1}
\end{figure}
}

\paragraph{Experiment.}
We assess the utility of our mapping $\matlL{}$ for early exit as a plug-and-play replacement for $\idlL{}$, through which intermediate representations are cast into final-layer representations.
We use \gpt{} for the next token prediction and \bert{} for masked token prediction (both with 24 layers).
We run each of the models over the validation set examples, while varying the confidence parameter $\lambda$ and using either $\idlL{}$ or $\matlL{}$ for casting intermediate representations.
Furthermore, we compare these early exit variants to the ``fixed exit'' strategy from \S\ref{sec:prediction}, where the computation is stopped after a pre-defined number of layers rather than relying on a dynamic decision.
We evaluate each variant in terms of both prediction's accuracy, using the Precision@$1$ metric (see \S\ref{sec:prediction}), and efficiency, measured as the average number of transformer layers processed during inference.


\paragraph{Results.}
%Figs.~\ref{fig:ee_pre1} and~\ref{fig:bertmask_ee_pre1}
Fig.~\ref{fig:ee_gpt2bert}
plots the average Precision@$1$ score against the average number of layers processed, for $24$-layer \gpt{} and $24$-layer \bert{}. For both models, under an early exit strategy our mapping \mat{} again provides a substantial improvement over \id{}.
For example, aiming at $95\%$ average precision, \mat{} saves $\sim3.3$ ($13.8$\%) layers in \gpt{} compared to only $\sim1.4$ ($5.9$\%) layers by \id{}, and $\sim4.8$ ($20$\%) layers in \bert{} versus $\sim3.5$ ($14.6$\%) layers by \id{}.
These results highlight the potential gains prominent early exit methods can obtain by using our method.
Notably, in both models and for each of the mapping methods, early exit obtains better results than fixed layer exit, as expected. 

\quash{
\begin{figure}[t]
\setlength{\belowcaptionskip}{-10pt}
\centering
\includegraphics[scale=0.35]{figs/bertmask_ee_pre1_24.pdf}
\caption{Precision@$1$ for the various early exit methods, and previous ``fixed exit'' methods for comparison ($24$-layer \bert{} masked token prediction task). Varying the confidence parameter $\lambda$, the $x$-coordinate is the average number of layers processed before an early exit decision is reached.}
\label{fig:bertmask_ee_pre1}
\end{figure}
}
\section{Linear Shortcut Across Sub-Modules}
\label{sec:submodules}

% Our experiments show that
% , despite the commonly-applied simplification by interpretability works, transformer layers do not operate in the same linear space and 
% there is a major gap in approximating future representations using an identity mapping (\S\ref{sec:layer_jump}, \S\ref{sec:prediction}).
% Here, 
In this section, we investigate whether discrepancies across layers result from specific sub-modules or are a general behaviour of all sub-modules in the network.  
This is done by extending our approach to test how well particular components in transformer blocks can be linearly approximated. 


\paragraph{Method.}

Consider \gpt{} for definiteness, then:
% we have 
$$ \texttt{b}_{\ell} = \texttt{b}_{\ell}^{\texttt{ffn}} \circ \texttt{b}_{\ell}^{\texttt{attn}}$$ 
% with
\begin{equation}\label{eq:attn} \texttt{b}^{\texttt{attn}}_{\ell} (H) = \texttt{attn}_{\ell} (\texttt{ln1}_{\ell} (H)) + H,\end{equation} 
where $\texttt{attn}_{\ell}$ is
%a multi-head self-attention
a MHSA
layer and \texttt{ln1} is a layer normalization (LN), and 
$$ \texttt{b}^{\texttt{ffn}}_{\ell} (H) = \texttt{ffn}_{\ell} (\texttt{ln2}_{\ell} (H)) + H,$$  
where $\texttt{ffn}_{\ell}$ is
%a feed-forward network
an FFN
layer and $\texttt{ln2}$ is a LN.
\quash{
Given a block $\texttt{b}_\ell$ and one of its sub-modules $\texttt{ln1}_\ell, \ \texttt{attn}_\ell, \ \texttt{ln2}_\ell$, or $\texttt{ffn}_\ell$, we fit linear regression approximating the output of the sub-module given its input and then use it in order to define mappings, as we now describe.
}
Given a block $\texttt{b}_\ell$ and one of its sub-modules $\texttt{ln1}_\ell, \ \texttt{attn}_\ell, \ \texttt{ln2}_\ell$, or $\texttt{ffn}_\ell$, we fit linear regression approximating the output of the sub-module given its input, and then use it to define mappings $\matattnl{}$, $\matlnl{}$ and $\matffl{}$.
%We provide the definition of $\matattnl{}$ below, and that of the other two in App. \ref{sec:app_submodule_skip_description}.
We provide the formal definitions of these mappings in App. \ref{sec:app_submodule_skip_description}.
\iffalse
\paragraph{$\matattnl{}$.}
%Illustrating this on $\texttt{attn}_\ell$ for definiteness,
For an input $s$, let $v^\ell_{i_s}$ be the vector at position $i_s$ in the output of $\texttt{attn}_\ell (\texttt{ln1}_\ell (H^{\ell - 1}))$. We denote by $A_\ell^{\texttt{attn}} \in \mathbb{R}^{d_h \times d_h}$ the matrix numerically minimizing 
$$ A \mapsto \sum_{s \in \mathcal{T}} || A \cdot \texttt{ln1}_\ell (h^{\ell-1}_{i_s}) - v^\ell_{i_s}||^2,$$
and define an attention sub-module replacement (Eq.~\ref{eq:attn}) by $$
\texttt{b}^{\overline{\texttt{attn}}}_\ell (h) \coloneqq A_{\ell}^{\texttt{attn}} \cdot \texttt{ln1}_\ell (h) + h. $$
We then define a mapping between two layers ${\ell \rightarrow \ell'}$ by:
$$ \matattnl{} (h) \coloneqq $$
$$ \texttt{b}^{\texttt{ffn}}_{\ell'} ( \texttt{b}^{\overline{\texttt{attn}}}_{\ell'} ( \ldots (\texttt{b}^{\texttt{ffn}}_{\ell+1} ( \texttt{b}^{\overline{\texttt{attn}}}_{\ell+1} (h)))\ldots)).$$ 
Namely, when applying each $\ell''$-th block, $\ell < \ell'' \leq \ell'$, we replace its attention sub-module $\texttt{attn}_{\ell''}$ by its linear approximation.
%In an analogous way, we consider the mappings $\matffl{}$ and $\matlnl{}$, where in the latter we perform the linear shortcut both for \texttt{ln1} and for \texttt{ln2} (see~\S\ref{sec:app_submodule_skip_description} for precise descriptions).
Importantly, unlike the original attention module, the approximation $\texttt{b}^{\overline{\texttt{attn}}}_\ell$ operates on each position independently, and therefore applying $\matattnl{}$ disables any contextualization between the layers $\ell$ and $\ell'$. Note that this is not the case for $\matffl{}$ and $\matlnl{}$, which retain the self-attention sub-modules and operate contextually.
\fi

\paragraph{Evaluation.}


We analyze the $24$-layered \gpt{}, and proceed completely analogously to \S\ref{subsec:next_token_prediction_task}, evaluating the Precision@$1$ and Surprisal metrics for the mappings $\matattnlL{}$, $\matfflL{}$ and $\matlnlL{}$.

\begin{figure}[t]
\setlength{\belowcaptionskip}{-0pt}
\centering
%\includegraphics[scale=0.2]
\includegraphics[width=\columnwidth]{figs/parts_presurp_24.pdf}
\caption{Precision@$1$ and Surprisal for the various sub-module linear mappings, and $\matlL{}$ for comparison ($24$-layer \gpt{} next token prediction task). A 95\% confidence interval surrounds the Surprisal lines.}
\label{fig:parts_presurp}
\end{figure}

\quash{
\begin{figure}[t]
\centering
\includegraphics[scale=0.4]{figs/parts_pre1_24.pdf}
\caption{Precision@$1$ for the various sub-module linear shortcut mappings, and the mapping $\matlL{}$ for comparison (\gpt{} next token prediction task).}
\label{fig:parts_pre1}
\end{figure}

\begin{figure}[t]
\centering
\includegraphics[scale=0.35]{figs/parts_surp_24.pdf}
\caption{Surprisal for the various sub-module linear shortcut mappings, and the mapping $\matlL{}$ for comparison (\gpt{} next token prediction task). A 95\% confidence interval surrounds the lines.}
\label{fig:parts_surp}
\end{figure}
}

\paragraph{Results.}
Fig.~\ref{fig:parts_presurp} shows the average Precision@$1$ and Surprisal scores per layer.
From a certain layer (\textasciitilde$7$), all sub-module mappings achieve better results than the full-block mapping $\matlL{}$. Thus, it is not just the cumulative effect of all the sub-modules in the transformer block that is amenable to linear approximation, but also individual sub-modules can be linearly approximated. 
Furthermore, the linear approximation of attention sub-modules is less harmful than that of the FFN or LN sub-modules. 
% Hypothetically, 
A possible reason is that the linear replacement of FFN or LN ``erodes'' the self-attention computation after a few layers. 
Moreover, the good performance of $\matattnlL{}$ suggests that contextualization often exhausts itself in early layers; speculatively, it is only in more delicate cases that the self-attention of late layers adds important information. Last, remark the sharp ascent of the scores for layer normalization in layers $5$-$8$, for which we do not currently see a particular reason. To conclude, we see that the possibility of linear approximation permeates
%the various
transformer components.


\section{Related Work}

Recently, there was a lot of interest in utilizing intermediate representations in transformer-based LMs, both for interpretability and for efficiency.

In the direction of interpretability, one seeks to understand the prediction construction process of the model \cite{tenney-etal-2019-bert, voita-etal-2019-bottom}.

More recent works use mechanistic interpretability and view the inference pass as a residual stream of information \cite{dar2022analyzing,geva-etal-2022-transformer}. Additionally, there are works on probing, attempting to understand what features are stored in the hidden representations \cite{adi2017finegrained, conneau-etal-2018-cram,liu-etal-2019-linguistic}. Our work is different in that it attempts to convert intermediate representations into a final-layer form, which is interpretable by design.

In the direction of efficiency, there is the thread of work on early exit, where computation is cut at a dynamically-decided earlier stage \cite{schwartz-etal-2020-right,xin-etal-2020-deebert,schuster2022confident}. Other works utilize a fixed early stage network to parallelize inference \citep{leviathan2022fast, chen2023accelerating}. However, intermediate representations are directly propagated in these works, which we show is substantially worse than our approach. Moreover, our method requires training considerably less parameters than methods such as \citet{schuster-etal-2021-consistent}, that learn a different output softmax for each intermediate layer.  

More broadly, skipping transformer layers and analyzing the linearity properties of transformer components have been discussed in prior works \cite{Zhao2021of,mickus-etal-2022-dissect,wang-etal-2022-skipbert,lamparth2023analyzing}.


\section{Conclusion and Future Work}

We present a simple and effective method for enhancing utilization of hidden representations in transformer-based LMs, that uses 
pre-fitted context-free and token-uniform linear mappings.
Through a series of experiments on different data sources, model architectures and scales, we show that our method consistently outperforms the prevalent practice of interpreting representations in the final-layer space of the model, yielding better approximations of succeeding representations and the predictions they induce, thus allowing a more faithful interpretation of the model's prediction-formation.
We demonstrate the practicality of our method for improving computation efficiency, saving a substantial amount of compute on top of prominent early exiting approaches. 
Also, by extending our method to sub-modules, 
% more specifically the attention sub-modules, 
we observe that replacing a part of the transformer inference by a non-contextual linear computation often results in a small deterioration of the prediction.
This opens new research directions for improving model efficiency,
% and parallelizability.
% including breaking the computation into several parallelizable tasks.
including breaking the computation into parallel tasks.

\section*{Limitations}

Although we see in this work that there is more linear structure to transformer inference than could be explained solely by the residual connection, we do not elucidate a reason for that. We also do not try to formulate formal criteria according to which to judge, in principle, the quality of ways of short-cutting transformer inference in-between layers. In addition, our experiments cover only English data.


%\section*{Ethics Statement}
%Scientific work published at ACL 2023 must comply with the ACL Ethics Policy.\footnote{\url{https://www.aclweb.org/portal/content/acl-code-ethics}} We encourage all authors to include an explicit ethics statement on the broader impact of the work, or other ethical considerations after the conclusion but before the references. The ethics statement will not count toward the page limit (8 pages for long, 4 pages for short papers).

\section*{Acknowledgements}

We thank Tal Schuster for constructive comments.

% Entries for the entire Anthology, followed by custom entries
\bibliography{anthology,custom}
\bibliographystyle{acl_natbib}

\appendix

\section{Descriptions of $\matattn{}$, $\matff{}$ and $\matln{}$}
\label{sec:app_submodule_skip_description}

Here we detail the definitions of the mappings $\matattnl{}$, $\matffl{}$ and $\matlnl{}$ utilized in \S\ref{sec:submodules}.

\paragraph{Description of $\matattnl{}$.}
%Illustrating this on $\texttt{attn}_\ell$ for definiteness,
For an input $s$, let $v^\ell_{i_s}$ be the vector at position $i_s$ in the output of $\texttt{attn}_\ell (\texttt{ln1}_\ell (H^{\ell - 1}))$. We denote by $A_\ell^{\texttt{attn}} \in \mathbb{R}^{d_h \times d_h}$ the matrix numerically minimizing 
$$ A \mapsto \sum_{s \in \mathcal{T}} || A \cdot \texttt{ln1}_\ell (h^{\ell-1}_{i_s}) - v^\ell_{i_s}||^2,$$
and define an attention sub-module replacement (Eq.~\ref{eq:attn}) by $$
\texttt{b}^{\overline{\texttt{attn}}}_\ell (h) \coloneqq A_{\ell}^{\texttt{attn}} \cdot \texttt{ln1}_\ell (h) + h. $$
We then define a mapping between two layers ${\ell \rightarrow \ell'}$ by:
$$ \matattnl{} (h) \coloneqq $$
$$ \texttt{b}^{\texttt{ffn}}_{\ell'} ( \texttt{b}^{\overline{\texttt{attn}}}_{\ell'} ( \ldots (\texttt{b}^{\texttt{ffn}}_{\ell+1} ( \texttt{b}^{\overline{\texttt{attn}}}_{\ell+1} (h)))\ldots)).$$ 
Namely, when applying each $\ell''$-th block, $\ell < \ell'' \leq \ell'$, we replace its attention sub-module $\texttt{attn}_{\ell''}$ by its linear approximation.
%In an analogous way, we consider the mappings $\matffl{}$ and $\matlnl{}$, where in the latter we perform the linear shortcut both for \texttt{ln1} and for \texttt{ln2} (see~\S\ref{sec:app_submodule_skip_description} for precise descriptions).
Importantly, unlike the original attention module, the approximation $\texttt{b}^{\overline{\texttt{attn}}}_\ell$ operates on each position independently, and therefore applying $\matattnl{}$ disables any contextualization between the layers $\ell$ and $\ell'$. Note that this is not the case for $\matffl{}$ and $\matlnl{}$, which retain the self-attention sub-modules and operate contextually.

\paragraph{Description of $\matffl{}$.}
Let $v^\ell_{i_s}$ be the vector at position $i_s$ in the output of $\texttt{ln2}_{\ell} (\texttt{b}_\ell^{\texttt{attn}} (H^{\ell - 1}))$, for a given input $s$. We denote by $A_\ell^{\texttt{ffn}} \in \mathbb{R}^{d_h \times d_h}$ the matrix numerically minimizing 
$$ A \mapsto \sum_{s \in \mathcal{T}} || A \cdot v^{\ell}_{i_s} - \texttt{ffn}_{\ell} (v^\ell_{i_s})||^2,$$
and define a replacement of the feed-forward sub-module $\texttt{b}_{\ell}^{\texttt{ffn}}$ by $$ \texttt{b}^{\overline{\texttt{ffn}}}_\ell (H) \coloneqq A_{\ell}^{\texttt{ffn}} \cdot \texttt{ln2}_\ell (H) + H.$$
We then define a mapping between two layers ${\ell \rightarrow \ell'}$ by:
$$ \matffl{} (H) \coloneqq $$
$$ \texttt{b}^{\overline{\texttt{ffn}}}_{\ell'} ( \texttt{b}^{\texttt{attn}}_{\ell'} ( \ldots (\texttt{b}^{\overline{\texttt{ffn}}}_{\ell+1} ( \texttt{b}^{\texttt{attn}}_{\ell+1} (H))\ldots)).$$

\paragraph{Description of $\matlnl{}$.}
Let $v^\ell_{i_s}$ be the vector at position $i_s$ in the output of $\texttt{b}^{\texttt{attn}}_{\ell} (H^{\ell - 1})$, for a given input $s$. We denote by $A_\ell^{\texttt{ln1}} \in \mathbb{R}^{d_h \times d_h}$ the matrix numerically minimizing 
$$ A \mapsto \sum_{s \in \mathcal{T}} || A \cdot h^{\ell}_{i_s} - \texttt{ln1}_{\ell} (h^\ell_{i_s})||^2$$ and we denote by $A_\ell^{\texttt{ln2}} \in \mathbb{R}^{d_h \times d_h}$ the matrix numerically minimizing $$ A \mapsto \sum_{s \in \mathcal{T}} || A \cdot v^{\ell}_{i_s} - \texttt{ln2}_{\ell} (v^\ell_{i_s})||^2.$$ We define a replacement of the block $\texttt{b}^{\texttt{attn}}_{\ell}$ by \begin{equation} \texttt{b}^{\overline{\texttt{ln1}}}_\ell (H) \coloneqq \texttt{attn}_{\ell} (A_{\ell}^{\texttt{ln1}} \cdot H) + H\end{equation} and we define a replacement of the block $\texttt{b}^{\texttt{ffn}}_{\ell}$ by \begin{equation} \texttt{b}^{\overline{\texttt{ln2}}}_\ell (H) \coloneqq \texttt{ffn}_{\ell} (A_{\ell}^{\texttt{ln2}} \cdot H) + H.\end{equation}
We then define a mapping between two layers ${\ell \rightarrow \ell'}$ by:
$$ \matlnl{} (H) \coloneqq $$
$$ \texttt{b}^{\overline{\texttt{ln2}}}_{\ell'} ( \texttt{b}^{\overline{\texttt{ln1}}}_{\ell'} ( \ldots (\texttt{b}^{\overline{\texttt{ln2}}}_{\ell+1} ( \texttt{b}^{\overline{\texttt{ln1}}}_{\ell+1} (H))\ldots)).$$


\end{document}
