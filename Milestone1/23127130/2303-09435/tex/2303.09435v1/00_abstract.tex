\begin{abstract}
Transformer-based language models (LMs) create hidden representations of their inputs at every layer, but only use final-layer representations for prediction. This obscures the internal decision-making process of the model and the utility of its intermediate representations. One way to elucidate this is to cast the hidden representations as final representations, bypassing the transformer computation in-between.
In this work, we suggest a simple method for such casting, by using linear transformations. We show that our approach produces more accurate approximations than the prevailing practice of inspecting hidden representations from all layers in the space of the final layer. Moreover, in the context of language modeling, our method allows ``peeking'' into early layer representations of \gpt{} and \bert{}, showing that often LMs already predict the final output in early layers.
We then demonstrate the practicality of our method to recent early exit strategies, showing that when aiming, for example, at retention of 95\% accuracy, our approach saves additional 7.9\% layers for \gpt{} and 5.4\% layers for \bert{}, on top of the savings of the original approach.
Last, we extend our method to linearly approximate sub-modules, finding that attention is most tolerant to this change.\footnote{Our code and learned mappings are publicly available at \url{https://github.com/sashayd/mat}.}
\end{abstract}
