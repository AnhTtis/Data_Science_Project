\section{Regulatory Risks}
\label{sec:regrisk}
At least one car is generally responsible in an accident \cite{shalev2017}. For example, during a following scenario the back vehicle is hold liable, if it failed to keep safe distances to the leading vehicle. %required to always keep safe distances to the front vehicle. Proper responses can thus be applied when the leading behavior changes. If the follower fails to do so, the responsibility for the crash lies with him. 
%
%\noindent Shalev-Shwartz \\
In contrast, for car pairs frontally driving against each other, both are seen at fault. Then, priorities around intersections with traversing paths allow to shift the responsibility on the driver who had to yield. A requirement for these longitudinal and lateral circumstances is that the superior entity (leading or prioritised car) did not brake or accelerate unreasonably. Otherwise in law, the share of the blame and costs is again divided among the involved parties.
%An important research question for autonomous vehicles is how to mathematically implement this responsibility system.

%In previous research, we constructed, for the ego car
To implement asymmetry in interactions, we formerly treated situations as discrete awareness or non-awareness entity combinations \cite{damerow2015}. By iterating over each and superposing the inherent risks, an optimal trajectory was constructed. However for ROPT, a more computationally efficient way is to only focus on the likely situations based on priorities. ROPT thus a) categorizes the path relation between vehicles plus matches them to legal right-of-ways (e.g. front-before-back, right-before-left) and b) modifies appropriately the behavior-relevant prediction model of other cars (i.e., altering the influence on own risk and calculating different trajectories). 

\vspace{0.15cm}
\subsection{Order Assignment}
\label{sec:orderas}
A generic driving scene of two traffic participants (TP) with $i=1,2$ is illustrated in Figure $\ref{fig:interaction}$. As a starting point, we trail corridors having widths $c_w$ from their current longitudinal position $l_1$ and $l_2$ until the trajectory end. Subsequently, the zone of interaction is given where both corridors interfere. We project start and end points to each path and get separate boundaries $I_{s,1}$, $I_{e,1}$ for TP1 and $I_{s,2}$ and $I_{e,2}$ for TP2. %\footnote{From now on, the second subscript denotes the participant number $i$ and not velocity segment order $n$.}
 
In the longitudinal case, one or both TP's are in the interaction zone at moment $t$. Comparing the positions $l_i$ allows to assign TP2 being in front or in the back to TP1. In total, we can write
\begin{equation}
l_1 \in [I_{s,1}, I_{e,1}] \wedge l_2 \in [I_{s,2}, I_{e,2}] \rightarrow
\end{equation}
\begin{equation}
\text{front: } l_1 < l_2 \text{, back: } l_1 > l_2. \nonumber
\end{equation}
For the lateral case, the trajectories meet in the future. When we look at the difference angle $\Delta \gamma$ of the interaction start $I_{s,1}$ and $I_{s,2}$, TP2 is to the right or left depending on its value in compliance with
%\noindent -Lateral case: trajectories intersect/merge in prediction horizon \\ 
%-depending on difference angle at start points of interaction zone, other car left or right
\begin{equation}
\angle I_{s,1} I_{s,2} = \gamma_{s,1} - \gamma_{s,2} = \Delta \gamma_s,
%\Delta \gamma_b = \gamma_{b,1} - \gamma_{b,2} 
\end{equation}
\begin{equation}
\text{right: } \Delta \gamma_s \in (0, \pi) \text{, left: } \Delta \gamma_s \in (\pi, 2\pi).
%\angle l_{b,1} l_{b,2} = \Delta \gamma_b
%\Delta \gamma_b = \gamma_{b,1} - \gamma_{b,2} 
\end{equation}

Possible interaction types for TP1 driving fixed from the bottom to the top on X-intersections are also summarized in Figure $\ref{fig:interaction}$. Besides TP's driving on the same path, the trajectory of TP2 can intersect, be curved before or after and merge with trajectory of TP1. For front-before-back, TP2 is superior in front and inferior in back relations. Analogously, right-before-left determines TP2 as superior for right and inferior for left contexts. In other countries with left-before-right, the order assignment is switched.

\subsection{Prediction under Priority}
\label{sec:predprio}
\begin{figure}[t!]
      \centering
      \resizebox{\linewidth}{!}{
      \import{img/}{interaction_examples.pdf_tex}}
      %\caption{Left: Collision risk prediction with survival analysis. Right: Interaction parameters.}
      \caption{Individual steps for regulatory risk estimation on the basis of spatial path corridors.} 
      \vspace{0.1cm}
      \label{fig:interaction}
\end{figure}
\begin{figure}[t!]
      \centering
      %\resizebox{\linewidth}{!}{   
      %\import{img/}{interaction_examples3.pdf_tex}}
      \resizebox{\linewidth}{!}{
      \import{img/}{pred_vel_in_rules.pdf_tex}}
      \caption{Left: Change in collision risk over future times. Right: Acceleration and deceleration assumptions for other entity. \vspace{-0.3cm}}
      \vspace{0.2cm}
      \label{fig:modpred}
\end{figure}
%\vspace{-0.15cm}
\subsubsection{Awareness Discounting} %Reduce prediction horizon (not entirely true, because we still predict comfort and utility with full prediction horizon)
On crowded public roads, we concentrate on the main cars around which have right-of-way. The remaining cars are solely considered if they come critically close.
In this sense, ROPT discounts the collision risk $\tau_{\text{coll},j}^{-1}$ of inferior obstacles with a monotonically decreasing function. Regarding longitudinal interactions, our
sigmoid function $\alpha_{\text{lon}}(s)$ is described with the slope $k_{\text{lon}}$ and midpoint $s_{\text{lon}}$ which leads to
%\vspace{-0.2cm}
\vspace{0.05cm}
%\noindent -when ego superior entity: discount influence of other car in collision risk with monotonically decreasing function, here logistic function \\
\begin{equation}
\alpha_{\text{lon}}(s) = 1 - \frac{1}{1+\text{exp}\{k_{\text{lon}}(s-s_{\text{lon}}) \}} \hspace{0.01cm},
\end{equation}
%\vspace{-0.5cm}
\vspace{0.01cm}
\begin{equation}
\tau_{\text{coll},j}^{* \hspace{0.04cm} -1}(\textbf{z}_{t:t+s}) = \alpha_{\text{lon},j}(s) \tau_{\text{coll},j}^{-1}(\textbf{z}_{t:t+s}). 
\label{eq:riskmod}
\end{equation}
\vspace{-0.3cm}

\noindent The equations for $\alpha_{\text{lat}}(s)$ are the same, whereby the chances that the other vehicle perceives us is lower in intersection scenarios and parameters $k_{\text{lat}}$ and $s_{\text{lat}}$ are set higher (compare Figure \ref{fig:modpred} on the left).
%-formulas for $\alpha_{\text{lat}}(s)$ are the same with parameters $k_{lat}$ and $s_{lat}$

%-chances that other car see me are larger in longitudinal following and thus the parameters can be set to lower values, whereby for intersection scenario it has to be higher prediction horizon \\
\subsubsection{Delayed Acceleration Patterns}
\noindent Without priority knowledge, vehicles are extrapolated with constant velocity from Section \ref{sec:fram}. In addition to decreasing the awareness, ROPT predicts delayed accelerations in the lateral situation as well.
If the other car is superior, the ego planner assumes first constant velocity $s_0$ long, an acceleration phase $(s_a, a_a)$ and ultimately steady velocity up to $s_h$. The case differentiation follows as
\begin{equation}
v(s)=\begin{cases}
  v_0,  & \text{for }s = [0, s_0), \\
  v_0 + a_a(s - s_0), & \text{for }s = [s_0, s_a], \\
  v_0 + a_a(s_a - s_0), & \text{for } s = (s_a,s_h].
\end{cases}
\label{eq:pattern}
\end{equation}
\vspace{-0.1cm}

\noindent Here, the strength of $a_a$ depends on the active velocity $v_0$ (i.e., we apply $a_a\hspace{-0.09cm}=\hspace{-0.09cm}0$ for $v_0\hspace{-0.09cm}=\hspace{-0.09cm}v_{\text{max}}$ and linear growth to $a_a\hspace{-0.04cm}=\hspace{-0.04cm}a_{\text{max}}$ when $v_0\hspace{-0.04cm}=\hspace{-0.04cm}0$). 
This is based on the fact that applied accelerations around intersections are statistically stronger from standstill. By comparison if the other car has superior relations, ROPT uses a longer deceleration phase $(s_d, a_d)$ with unchanged $a_d$. %The strength \footnote{We add that when the car has $v_j\approx0$, it accelerates more than when $v_j \gg 0$.} 

In a last step, we clip the velocities $v_{j}(s)$ to be higher than $0$ and lower than the maximal curve velocity $v_{c,j}(s)$ and allowed limit $v_{\text{max}}$ with
\vspace{-0.15cm}
\begin{equation}
v_j(s) = \text{max}(v_j(s), 0),
\end{equation}
\begin{equation}
v_j(s) = \text{min}(v_j(s), v_{\text{c},j}(s), v_{\text{max}}). 
\end{equation}
 %\rightarrow v_{max} road constraint
The altered velocity patterns from Figure \ref{fig:modpred} on the right lead to better predictions of other vehicles, given that they behave according to the traffic rules. Because of the delay $s_0$ in combination with the start of unawareness $s_{\text{lon}}$ and $s_{\text{lat}}$, ROPT is in short-times even robust against moderately wrong assumptions. %(i.e. does not obey to the traffic rule). 
Each should be set that no crash happens for any acceleration or deceleration maneuver. More detailed predictions can be achieved by considering environment conditions (just accelerating in interaction zone and coming to halt at stop line), participant types (e.g. motorbike or truck) and occuring situation class (highway versus inner city). 
