%reference powell, th and pet
\vspace{0.38cm}
\section{Experiments}
\label{sec:exp}
We want to show in our simulations that ROPT can handle a wide array of interactions which typically occur at intersection crossings and that the planned solution is compliant with priority rules.   %As it turns out, the regulatory risks lead to an asymmetry in the solutions of front vs. back and left vs. right while keeping safety and comfort at desirable levels. 
For this reason, we first analyze in Section \ref{sec:analyticalvar} one vehicle pair during dynamic followings before or after crossroads as well as during passing behaviors within intersection areas.  %  longitudinal plus intersection scenarios where the actions of the other car are varied. The longitudinal scenario represents thereby dynamic following as it occurs predominantly before and after intersections, whereas the intersection scenario is particularly suited to investigate passing behavior. 
We hereby show quantitatively the effect of the altered prediction models from ROPT. Second in Section \ref{sec:randomvar}, we randomize the possible paths for the two cars in test statistics to establish the robustness of ROPT in terms of risk and comfort. As it turns out, the optimization compensates non-priority-compliant other behavior with adequately elevated jerks.  
\subsection{Analytical Variation of Other Behavior}
\label{sec:analyticalvar} % the two
Both regarded basic scenarios are pictured in Figure \ref{fig:genericsituations}: longitudinally driving behind a leading TP to the front and an uncontrolled intersection having a second TP to the right. We also reproduce the cases that TP2 is in the back or approaching from the left. In each case, we vary for TP2 the initial velocity $v_{f,2}$ in between $0$ and $\unit[15]{m/\text{sec}}$. %\footnote{From now on, the second subscript in $v$ denotes the participant number $i=1,2$ and not snake segment $n=1\small{-}\hspace{0.05cm}4$.}
After $\unit[1]{\text{sec}}$, a deceleration/acceleration $a_{f,2}$ is applied in the range from $-3$ to $\unit[3]{m/\text{sec}^2}$ for the duration of $\unit[3]{\text{sec}}$. The challenge for ROPT is then to adapt TP1 (ego car, green) to the fixed actions of TP2 (other car, red) 
%to the fixed actions of TP2 
while considering the regulations front-before-back and right-before-left. Concerning the longitudinal environment, ROPT starts at a distance $d_0\hspace{-0.03cm} = \hspace{-0.03cm} \unit[50]{m}$ to TP2 and with equal speed $v_{f,1}\hspace{-0.05cm}=\hspace{-0.05cm}v_{f,2}$. A soft road limit of $v_{\text{max}} = \unit[20]{m/\text{sec}}$ is also valid. For the intersection instance, beginning offsets until the path corridors overlap are chosen as $d_{I,1}\hspace{-0.03cm}=\hspace{-0.03cm}d_{I,2}\hspace{-0.03cm}=\hspace{-0.03cm}\unit[40]{m}$ and the velocity parameters of ROPT (i.e., $v_{f,1}$ and desired velocity $v_{d,1}$) are set to $\unit[10]{m/\text{sec}}$.   %With these parameter settings, the 3-second deceleration process of the other car always finishes before the stop line of the intersection is reached.
%density of simulation runs is 0.5 for v and a

%control the velocity $v_1$ in TP1

%put variables for other car (v_0 and a) and visualize pet and th
% on top of plots other priority and ego priority, in caption other car to the right, left, etc. (or other superior, other inferior)
\begin{figure}[t!]
      \centering
      \resizebox{\linewidth}{!}{
      \import{img/}{generic_scenarios.pdf_tex}}
      \caption{Left: Initial plus final conditions in a following scenario under front-before-back priority (case ego following). Right: Two possible scene evolutions for an intersection with right-before-left (case other priority).} % (case ego following), (case other priority) The ego car (green) is controlled by ROPT and represents TP1. Depicted in red, the competing other car is TP2.} %the competing other car (red) represents TP2.} %and the competing other car (red) represents TP2.} %}%\footnotemark} 
      %The ego car (green) is controlled by ROPT and represents TP1. Depicted in red, the competing other car is TP2.
      \vspace{-0.05cm}%} %}
      \label{fig:genericsituations}
\end{figure}

%\footnotetext{From now on, the second subscript in $v$ denotes the participant number $i$ and not segment order $n$.}

In the evaluation, we particularly look at the indicators Time Headway ($\text{TH}$) \cite{transpres2010} %eventually
and Post-Encroachment Time ($\text{PET}$) \cite{brian1978} which depend on the kinematics of the vehicles  
\begin{align}
\text{TH} = \frac{-\Delta l}{v_1} \ \text{with} \ \Delta l = l_1 - l_2, \label{eq:th} \\
\text{PET} = -\Delta t \ \text{with} \ \Delta t = t_1 - t_2. 
\end{align}
%  when the other car is in front of the ego car, for the inverted case the other car TH is used.} 
The events $t_1$ and $t_2$ indicate in $\text{PET}$ when the ego entity leaves and when the obstacle enters the interaction zone, respectively. On that account, ROPT can either pass in front with $\text{PET} >0$ or behind with $\text{PET}<0$. 
For $\text{TH}$, we extract the stable value $\text{TH}_{\text{stable}}$ once a constant longitudinal distance $\Delta l$ is maintained.\footnote{Equation (\ref{eq:th}) counts if the ego car follows another vehicle. For the inverted case, the indices in TH are swapped.} To complete the utility assessment of ROPT, we eventually capture the lower boundary $v_{\text{low},1}$ and upper boundary $v_{\text{up},1}$ from the executed velocity course $v_1$. 

\begin{figure}[t!]
      \centering
      %\includegraphics[width=1.0\linewidth]{./img/results_figures/other_front.png}  
      %\includegraphics[width=1.0\linewidth]{./img/results_figures/other_back.png}  
      \resizebox{\linewidth}{!}{
      \import{img/results_figures/}{th_stable_and_v_boundaries.pdf_tex}}

      \caption{Indicators of ROPT behavior (minimal and maximal velocity) and its interplay with other car (i.e., stable headway) for range of fixed other actions (varying inital speed and acceleration). The priority-dependant awareness horizons lead to lower distances for back vehicles.}%higher distances for front vehicles.} 
      \vspace{-0.05cm}
      \label{fig:frontbeforeback}
\end{figure}

\subsubsection{Dynamic Following}
%First, explain what the Figures show, and how they have to be read (interpreted). Then, explain what ``good" behavior would look like (risk/utility/comfort) -> crucial that the reader understands this
An agent controlled with symmetric risk calculations for front and back would react very sensitively to following cars, e.g. in the case of tailgaiting. 
Due to the longitudinal risk discounting with Equation (\ref{eq:riskmod}), it is now harder for the back vehicle to push ROPT in front. The inferior entity is however not entirely ignored, since non-reaction can result in partial legal blame.  %motivate why it makes sense to not ignore the other vehicle in the back compleletly (braking abruptly can be dangerous and is forbidden /will lead to partial blame)
The contour plots within Figure \ref{fig:frontbeforeback} sort measured $\text{TH}_{\text{stable}}$ and extrema of $v_1$ into colored bins for every other trajectory point ($v_{f,2}$,~$a_{f,2}$).\footnote{The grid step size amounts to $\Delta_{x,y} = 0.5$ with linear interpolations.} %for the x- and y-direction
As can be seen at the bottom row, we allow small but sufficient $\text{TH}_{\text{stable}}$ until $\unit[1]{\text{sec}}$ to the back. If $a_{f,2}$ is positive, higher $v_{f,2}$ lead to decreasing $\text{TH}_{\text{stable}}$. At the same time, the final maximum velocity $v_{\text{up},1}$ matches the accelerating follower with $v_{f,2}\hspace{-0.01cm}+\hspace{-0.01cm}\unit[3]{\text{sec}}\cdot a_{f,\text{2}} \hspace{-0.03cm}<\hspace{-0.03cm}\unit[22.5]{\text{m/sec}}$. %As $v_{\text{max}}$ is a soft constraint, ROPT is allowed to apply higher values in $v_{1}$ once collision risk is immanent. %\footnote{For $v_{f,2}=$ and $a_{f,2}$} 
For negative $a_{f,2}$, ROPT is not influenced by the decelerating obstacle (i.e., $\text{TH}_{\text{stable}}$ from $\unit[3]{\text{sec}}$ upwards) and delivers steady velocity (i.e., $v_{\text{up},1}~\hspace{-0.13cm}=~\hspace{-0.13cm}v_{f,1}$ applies). %(i.e. $\text{TH}_{\text{stable}}>\unit[5]{\text{sec}}$)

In contrast, front vehicles with priority yield more proactive ego behaviors. When the leader brakes down, ROPT uses unaltered collision risks and converges to moderate $\text{TH}_{\text{stable}}\approx \unit[2]{\text{sec}}$ for large $v_{f,2}$ and $|a_{f,2}|$ (compare top row of Figure \ref{fig:frontbeforeback}). Is the other trajectory a stopping trajectory, the minimum end speed $v_{\text{low},1}$ becomes $\unit[0]{m/\text{sec}}$ in ROPT and thus $\text{TH}_{\text{stable}}$ exceeds $\unit[5]{\text{sec}}$. Unlimited $\text{TH}_{\text{stable}}$ are moreover also carried on, when TP2 is moving away with $a_{f,2}>0$. ROPT is therefore able to retain the varying beginning velocity, more specifically $v_{\text{low},1}=[\unit[0]{\text{m/sec}},\unit[15]{\text{m/sec}}]$.      

\subsubsection{Intersection Passing} %as mentioned before
Simple heuristical go/no-go decisions cannot ensure optimal driving cost tradeoffs for lateral priorities. Another entity on the right might be far away or decelerating so that the ego utility is neglected. More importantly, inferior cars that do not respect the right-of-way create arbitrary risk or discomfort peaks. 
Via the delayed acceleration patterns from Section \ref{sec:predprio}, ROPT is capable to continuously weigh benefits with risks for passing a rule-based intersection first or second. Figure \ref{fig:leftbeforeright} visualizes the isolines of $\text{PET}$, $v_{\text{low},1}$ or $v_{\text{up},1}$ for the same parameter variations in $v_{f,2}\hspace{-0.03cm}$ and $\hspace{-0.03cm}a_{f,2}$. While there is more area of $\text{PET}\hspace{-0.03cm}<\hspace{-0.07cm}\unit[-2]{\text{sec}}$ (i.e., ROPT driving second) when the car comes from the right, the condition $\text{PET}\hspace{-0.03cm}>\hspace{-0.03cm}0$ dominates (i.e., ROPT crossing first) for vehicles to the left.
%While there is more area of $\text{PET}\hspace{-0.05cm}>\hspace{-0.05cm}0$ (i.e. ROPT crossing first) when the car comes from the left, the condition $\text{PET}\hspace{-0.05cm}<\hspace{-0.05cm}\unit[-2]{\text{sec}}$ dominates (i.e. ROPT driving second) for vehicles to the right. 
The transition from positive to negative values is on average at $v_{f,2}\hspace{-0.05cm}=\hspace{-0.05cm}\unit[7]{\text{m/sec}}$ in the former and about $v_{f,2}=\unit[10]{\text{m/sec}}$ for the latter case. With smaller $a_{f,2}$, the sign change happens at greater $v_{f,2}$.
%The transition from positive to negative values is on average at $v_{f,2}=\unit[10]{\text{m/sec}}$ in the former and about $v_{f,2}=\unit[7]{\text{m/sec}}$ for the latter case. With smaller $a_{f,2}$, the sign change happens at greater $v_{f,2}$.


The reason can be well observed in the ego velocity course $v_1$. %An inferior ROPT is inclined to decelerate as far as 
A prioritized ROPT is inclined to accelerate with $v_{\text{up},1}$ as far as $\unit[17.5]{\text{m/sec}}$, because it assumes the obstacle to stop. If the encountered vehicle disobeys (e.g. $a_{f,2} \approx  \unit[3]{\text{m/sec}}$ and $v_{f,2}~\hspace{-0.1cm}\approx~\hspace{-0.1cm} \unit[7]{\text{m/sec}}$), ROPT will at some point halt and give way. %safely which creates the highest $j_1$. 
These situations are still safe but create the highest jerk (refer to Section \ref{sec:randomvar}). Vice versa once TP2 has priority, ROPT brakes frequently having $v_{\text{low},1}$ under $\unit[2.5]{\text{m/sec}}$. At the same time, accelerating back to desired $v_{d,1}$ takes more time with $\text{PET}=[\unit[-3]{\text{sec}},\unit[-20]{\text{sec}}]$. The velocity growth prediction of TP2 leads to cautious ego behavior. %ROPT also needs more time to accelerate back to $v_{d,f}$ with $\text{PET}$. 
Here, overtaking is still established in small initial other speeds $v_{f,2}$. For $v_{f,2}\rightarrow0$, the other car does not even interfere with the ego trajectory and $\text{PET}>\unit[10]{\text{sec}}$ holds.  

\begin{figure}[t!]
      \centering
      \resizebox{\linewidth}{!}{
      \import{img/results_figures/}{pet_and_v_boundaries.pdf_tex}} %
      \caption{Results for ROPT with altered velocity extrapolations. Top: Other car approaching from right. Bottom: Other car coming from left. On an intersection, ROPT accelerates more frequently when having priority.} %brakes more frequently for prioritized vehicles.}
      %\caption{Post-Encroachment Time in 2-car intersection scenario with right vs. left priority. Top row: Other car approaching from right. Bottom row: Other car approaching from left. On an intersection, ROPT either drives more often after the other prioritized car ($\text{PET}<0$) or takes its precedence ($\text{PET}>0$).} %with normal comfort ranges %The comfort is never disregarded (low $j_1$)
      \label{fig:leftbeforeright}
\end{figure}


\subsection{Randomized Intersection Geometries} %Other Paths and Intersections
\label{sec:randomvar}

For our large-scale experiment, the unsignaled intersection is hereafter extended with statistical conditions for the simulation. Altogether, we randomize path geometries, agents' starting states and the priority compliance of the other participant. This enables us to discuss hazards and jerk caused or rather avoided by ROPT from car-to-car passings. 

\vspace{0.04cm}
\subsubsection{Simulation Setup}
While we reduce the driving limit to fixed $v_{\text{max}}\hspace{-0.07cm}=\hspace{-0.07cm}\unit[8.5]{m/\text{sec}}$, each individual run has different angles between the four roads and random lane widths. Moreover, the start and destination roads for the ego and other vehicle are stochastic.\footnote{A prerequisit is that the start roads are distinct and their paths intersect. The situation will therefore always correspond to the basic lateral types depicted in Figure \ref{fig:interaction}.} Both cars subsequently start with sampled velocities $v_{f,1}$ and $v_{f,2}$ from $\unit[3.0]{m/\text{sec}}$ until $\unit[8.5]{m/\text{sec}}$ having a set distance $d_{I,1} \hspace{-0.02cm}=\hspace{-0.02cm}d_{I,2}\hspace{-0.02cm}=\hspace{-0.02cm}\unit[45]{m}$ to the intersection edge. Here, the desired cruising velocity $v_{d,1}$ for ROPT is always equal to $v_{\text{max}}$ and $v_{d,2}$ of the other participant is also randomized including higher speeds $\leq \hspace{-0.01cm} \unit[10]{m/\text{sec}}$. 
We finally vary the compliancy of the second car. In $\unit[50]{\%}$ of the experiments, TP2 is inattentive and ignores ROPT as long as the center-to-center distance $d_{2}$ is above $\unit[10]{m}$. This results into particularly challenging situations if ROPT has priority and assumes the obstacle to yield.

Opposed to Section \ref{sec:analyticalvar}, the simulation applies a full multi-agent planning. ROPT steers as above the ego car. In addition, the other car is controlled dynamically: 
it posseses the same cost function to evaluate trajectories (see Section \ref{sec:trajeval}), but exploits a simpler mechanism for creating candidate trajectories (i.e., no full optimization from Section \ref{sec:trajgen}). In each time step, the other vehicle directly constructs $21$ differing acceleration/deceleration profiles and selects the best one among them.

After each run, the unfolded driving scene is evaluated. For computing risk levels, we introduce a measure termed two-dimensional headway $\text{TH}_{\text{2D}}$ which expands TH to account for lateral distances. With the help of constant velocity extrapolation, $\text{TH}_{\text{2D}}$ essentially indicates the  time when vehicle pairs will occupy or have occupied the same space. %within the next $\text{TH}_{\text{2D}}$ seconds. %2DTH is a spatio-temporal measure indicating if two vehicles will occupy of have occupied the same space within a $T$ seconds window, using constant velocity extrapolation.
In detail, $\text{TH}_{\text{2D}}$ is obtained by first taking the bounding box for each agent consisting of four corners at the current step $t$. In the following calculation, we enlarge this box with the length of $v_t \cdot \frac{T}{2}$ in both directions along their path (whereby $v_t$ represents the velocity of the participant at $t$ and $T$ is the extrapolation interval). This means that  the resulting shape can bent around corners and is not convex.
We lastly define $\text{TH}_{\text{2D}}$ to be the minimum $T$ once the shapes of two vehicles overlap. Alongside our $\text{TH}_{\text{2D}}$, the maximal value of the ego jerk course $r_{\text{max},1}$ is likewise gathered. As a reference, most passengers rate a jerk until $\unit[3]{m/\text{sec}^3}$ as 
acceptable \cite{powell2015} and in emergency trajectories jerks above $\unit[6]{m/\text{sec}^3}$ are common \cite{bagdadi2009}. To neglect comfort reduction because of high frequency motion, we filter beforehand peaks in $r_1$ with rolling means and a window factor of $W=\unit[0.5]{\text{sec}}$.\footnote{Note that the moving average filter is not used within ROPT and only reduces outliers from $r_{\text{max},1}$ for the evaluation.}%\footnote{Downstreaming a trajectory controller after ROPT will act as a low-pass filter in the same way.}

\begin{figure}[t!]
      \centering
      \resizebox{\linewidth}{!}{
      \import{img/results_figures/}{randomized_runs.pdf_tex}} %
      \caption{Robustness of ROPT in diverse stochastic conditions during intersection crossings, e.g. priority violation of other car. Left: Cumulative histogram for two-dimensional headway. Right: Probability histogram for maximum ego jerk (note the log scale).} %(note the log scale).} %in log scale.
      %\caption{Post-Encroachment Time in 2-car intersection scenario with right vs. left priority. Top row: Other car approaching from right. Bottom row: Other car approaching from left. On an intersection, ROPT either drives more often after the other prioritized car ($\text{PET}<0$) or takes its precedence ($\text{PET}>0$).} %with normal comfort ranges %The comfort is never disregarded (low $j_1$)
      \label{fig:statistics}
\end{figure}

\vspace{0.04cm}
\subsubsection{Robustness Discussion}
More than 2000 simulations are executed involving the described settings. Figure \ref{fig:statistics} outlines the measured statistics for $\text{TH}_{\text{2D}}$ and $r_{\text{max},1}$. We initially focus on the ex-post risks.
The left side of Figure \ref{fig:statistics} renders cumulative distributions $A_{\text{runs}}$ for $\text{TH}_{\text{2D}}$. Regardless of TP2 following priority (top) or violating (bottom) priority, $\text{TH}_{\text{2D}}$ is larger than $\unit[1]{\text{sec}}$ in at least $\unit[85]{\%}$ of runs and $>\hspace{-0.04cm}\unit[0.5]{\text{sec}}$ in all runs.
In the cases when $\unit[0.5]{\text{sec}}\hspace{-0.02cm}<\hspace{-0.04cm}\text{TH}_{\text{2D}} \leq \unit[1]{\text{sec}}$, ROPT rightfully exerts its priority and the other vehicle crosses right behind. Decreasing values of $\text{TH}_{\text{2D}}$ are a consequence of the parametrization for TP2. It has higher escape rates $\tau_0$, which in effect lead to shorter prediction horizons and more aggressive planning. However, the main observation is that the trajectories of ROPT are always safe. ROPT must have compensated the incompliance of the other car and we thus look now more closely into the behavior of ROPT. % -- we will therefore look more closely to the ego vehicle in the next paragraph.

Probability distributions $Q_{\text{runs}}$ of the maximum jerk $r_{\text{max},1}$ encountered by ROPT are given in Figure \ref{fig:statistics} on the right.
If the other vehicle obeys right-of-way, $r_{\text{max},1}$ is below $\unit[2]{m/\text{sec}^3}$ in almost any situation (approximately $\unit[99]{\%}$), i.e. the ride feels comfortable. ROPT is robust against the intersection geometry or differences in taken starting and desired speeds from TP2. Rising $r_{\text{max},1}$ solely appear when TP2 has a counteracting behavior with non-compliancy to right-of-way. In such instances, ROPT has to compensate others' negligence by accepting larger $r_{\text{max},1}$. Usually, it reacts by either clearing the intersection earlier with accelerating away or by making a full brake to let TP2 pass in front. The latter can produce in $<\unit[1]{\%}$ highest $r_{\text{max},1}$ with up to $\unit[13]{m/\text{sec}^3}$. Nevertheless even for inattentive other participants, the jerk of ROPT is low to moderate for $\unit[90]{\%}$ of the situations and ROPT is able to smoothly adjust its behavior.
