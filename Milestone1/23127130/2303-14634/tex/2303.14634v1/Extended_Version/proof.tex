\section{Proof of Proposition \ref{main-body-mw}}
\label{bigproof}
Our proof follows the same procedure described over several chapters in \cite{neely} and is divided into two parts.

First, we consider a class of schedulers that make decisions based solely on the current bandwidth demands $\mathbf{W^e}(t)$ called $\mathbf{W^e}(t)$-only schedulers. Using \cite[Theorem 4.5]{neely}, we show that there exists a set of schedulers of this class that satisfies the premise of Proposition \ref{equiv2} for ergodic markovian demands.

Second, by following the steps in \cite[p.34-p.36]{neely}, we show that the Max-Weight scheduler strongly stabilizes the deficits $d_i(t)$ whenever that set of $\mathbf{W^e}(t)$-only schedulers stabilizes them. Thus, we show that the Max-Weight scheduler also satisfies the premise of Proposition \ref{equiv2} for ergodic markovian demands. 

The step-by-step details of the first part and the second part of this proof are in Appendix \ref{firstpart} and Appendix \ref{secondpart} respectively.

\subsection{$\mathbf{W^e(t)}$-only schedulers}
\label{firstpart}
Here, we show that there exists a set of $\mathbf{W^e}(t)$-only schedulers that satisfies the premise of Proposition \ref{equiv2} when $\mathbf{W}(t)$ follows an ergodic MC. Suppose $\mathbf{W}(t)$ follows an ergodic MC, and let $\pi_i^w$ denote the probability of state $w \in \mathcal{S}_i$, where $\mathcal{S}_i$ the state space of MC $i$. Since each MC $i$ is ergodic, it follows \cite[Corollary 9.29]{queuebook}:
\begin{equation}
 \liminf\limits_{T \to \infty}\frac{1}{T}\sum\limits_{t=1}^{T}\mathbbm{1}_{W_i(t) \leq W_i^r}=\sum_{w \in \mathcal{S}_i : w \leq W_i^r}\pi_i^w, \: w.p.1, \: \forall i.
\label{time-averages}
\end{equation}

From (\ref{time-averages}), it follows that assumption (\ref{assumption}) holds. Moreover, since the MC of $\mathbf{W}(t)$ is ergodic, its limiting distribution is the stationary distribution \cite[Theorem 9.4 and Theorem 8.6]{queuebook}. Therefore, regardless of the initial distribution of the MC, after some period of time, the process $\mathbf{W}(t)$ becomes identically distributed (id). 

Assuming that this transition period is small enough, we consider that the process once it has reached its stationary distribution. Thus, $\Pr(W_i(t)=w)=\pi_i^w, \,\forall t \in \mathbb{N}$, where $\pi_i$ is the stationary distribution of NS $i$ and $\pi_i^w$ is the probability of state $w$. Hence, the vector $\mathbf{W^e}(t)$ is also id with $\Pr(W_i^e(t)=w-W^L_i)=\pi_i^w, \, \forall w \in \mathcal{S}_i$. Let $\mathcal{S}_i^e \triangleq \{w^e: w^e+W^L_i \in \mathcal{S}_i\}$, $\mathcal{S}^e \triangleq \mathcal{S}_1^e \times ... \times \mathcal{S}_N^e$ and $p_i^{w^e} \triangleq \pi_i^{w^e+W^L_i}$, $\forall w^e \in \mathcal{S}_i^e$. Then, $\Pr(\mathbf{W^e}(t)=\mathbf{w^e})=\prod\limits_{i \leq N}p_i^{w^e_i} \triangleq p_{\mathbf{W^e}}(\mathbf{w^e})$, where $\mathbf{w^e} \in \mathcal{S}^e$.

Now, consider the class of $\mathbf{W^e}(t)$-only schedulers defined by probability distributions $p_{\mathbf{u}|\mathbf{w^e}}(\mathbf{u},\mathbf{w^e}) \triangleq \Pr(\mathbf{u}(t)=\mathbf{u}| \mathbf{W^e}(t)=\mathbf{w^e})$, i.e., if $\mathbf{W^e}(t)=\mathbf{w^e}$, then $\mathbf{u}(t)=\mathbf{u}$ with probability $p_{\mathbf{U}|\mathbf{W^e}}(\mathbf{u},\mathbf{w^e})$ which does not depend on time $t$. Note that $p_{\mathbf{U}|\mathbf{W^e}}(\mathbf{u},\mathbf{w^e})$ must satisfy:
\begin{equation}
\sum\limits_{\mathbf{u} \in \mathcal{U}(\mathbf{w^e},W^c)}p_{\mathbf{U}|\mathbf{W^e}}(\mathbf{u},\mathbf{w^e})=1, \: \forall \mathbf{w^e} \in \mathcal{S}^e.
\end{equation}
Next, we wish to compute the limit of the time average of $u_i(t)\mathbbm{1}_{W_i^e(t) > 0}$. To this end, first note that:
\begin{equation}
\begin{aligned}
& \E[u_i(t)\mathbbm{1}_{W_i^e(t) > 0}]= \bar{u_i} \triangleq\\
& \sum\limits_{\mathbf{w^e} \in \mathcal{S}^e, \mathbf{u} \in \mathcal{U}(\mathbf{w^e},W^c)}p_{\mathbf{U}|\mathbf{W^e}}(\mathbf{u},\mathbf{w^e})p_{\mathbf{W^e}}(\mathbf{w^e})u_i\mathbbm{1}_{W_i^e > 0}, \: \forall t.
\end{aligned}
\label{expectation}
\end{equation}
Since $u_i(t)\mathbbm{1}_{W_i^e(t) > 0}$ is modulated by the ergodic MC that $\{\mathbf{W^e}(t)\}_{t \in \mathbb{N}}$ follows, it holds \cite[p.76]{neely}:
\begin{equation}
\begin{aligned}
\liminf\limits_{T \to \infty}\frac{1}{T}\sum\limits_{t=1}^{T}u_i(t)\mathbbm{1}_{W_i^e(t) >0}=\frac{\E[\sum\limits_{t=1}^{T_1}u_i(t)\mathbbm{1}_{W_i^e(t) > 0}]}{\E[T_1]}, \: w.p.1,
\end{aligned}
\label{limit1}
\end{equation}
where $T_1$ is the first recurrence time to some arbitrary state $s \in \mathcal{S}^e$. Since $u_i(t)\mathbbm{1}_{W_i^e(t) > 0}$ is a bounded function of a finite-state and ergodic MC, the extension of Wald's equation for MCs in \cite{Moustakides} can be applied. Since we observe our MC after convergence to the stationary distribution, then the extension in \cite{Moustakides} results to the standard Wald's equation. Thus:
\begin{equation}
\liminf\limits_{T \to \infty}\frac{1}{T}\sum\limits_{t=1}^{T}u_i(t)\mathbbm{1}_{W_i^e(t) >0}=\bar{u_i}, \: w.p.1.
\label{limit2}
\end{equation}
Therefore, under an $\mathbf{W^e}(t)$-only scheduler, the long-term time-average of $u_i(t)\mathbbm{1}_{W_i^e(t) > 0}$ is equal to its one-slot expected value which is constant since $u_i(t)\mathbbm{1}_{W_i^e(t) > 0}$ is id. Now, consider the following constraints:
\begin{equation}
\begin{aligned}
& \liminf\limits_{T \to \infty}\frac{1}{T}\sum\limits_{t=1}^{T}\E[u_i(t)\mathbbm{1}_{W_i^e(t) >0}] \geq P_i^H - P_i^M,  \; \forall i,\\
& \mathbf{u}(t) \in \mathcal{U}(\mathbf{W^e}(t),W^c), \forall t.
\end{aligned}
\label{meanrate}
\end{equation}
Note that (\ref{meanrate}) is identical to the feasibility region of (\ref{schedprob4}) with the exception of the expected value operator $\E[\cdot]$ in the limit. Due to $(\ref{limit2})$, it can be shown that if there does not exist a $\mathbf{W^e}(t)$-only scheduler that satisfies (\ref{meanrate}), no scheduler satisfies (\ref{meanrate}) including non-causal schedulers \cite[Theorem 4.5]{neely}. 

Thus, Proposition \ref{equiv2} applies for a set of $\mathbf{W^e}(t)$-only schedulers, where $\mathcal{F}_{W^c}$ is defined by (\ref{meanrate}). Ideally, however we would like to show that it also applies for $\mathcal{F}_{W^c}=\mathcal{G}_{W^c}$. To this end, note that due to Fatou's lemma, it holds:
\begin{equation}
\begin{aligned}
& \liminf\limits_{T \to \infty}\frac{1}{T}\sum\limits_{t=1}^{T}u_i(t)\mathbbm{1}_{W_i^e(t) >0} \geq P_i^H - P_i^M, \: w.p.1 \\
\Rightarrow  & \liminf\limits_{T \to \infty}\frac{1}{T}\sum\limits_{t=1}^{T}\E[u_i(t)\mathbbm{1}_{W_i^e(t) >0}] \geq P_i^H - P_i^M, \: w.p.1.
\end{aligned}
\label{fatou}
\end{equation}
Using the above, we can show the following lemma.
\begin{lemma}
There exists $\mathcal{C'}\triangleq \{\mathbf{u}_{W^c}\}_{W^c \in \mathbb{R}}$ s.t. $\forall W^c \in \mathbb{R}$, if $\mathbf{u}_{W^c} \notin \mathcal{G}_{W^c}$, then $\mathcal{G}_{W^c}=\emptyset$, where each $\mathbf{u}_{W^c}$ is a $\mathbf{W^e}(t)$-only scheduler.

\noindent Proof:

We prove the contrapositive. Suppose $\exists \mathbf{v} \in \mathcal{G}_{W_c}$. Due to (\ref{fatou}), scheduler $\mathbf{v}$ satisfies (\ref{meanrate}). Thus, due to \cite[Theorem 4.5]{neely}, $\exists \mathbf{u}_{W^c}$ that satisfies (\ref{meanrate}) where $\mathbf{u}_{W^c}$ is a $\mathbf{W^e}(t)$-only scheduler. Thus, since (\ref{fatou}) holds with equivalence for $\mathbf{W^e}(t)$-only schedulers, $\mathbf{u}_{W^c} \in \mathcal{G}_{W_c}$ where $\mathbf{u}_{W^c}$ is $\mathbf{W^e}(t)$-only.

\label{adequacy}
\end{lemma}

Therefore, due to Proposition \ref{equiv2}, it is adequate to consider the set $\mathcal{C'}$ of $\mathbf{W^e}(t)$-only schedulers. Note that Lemma \ref{adequacy} shows the existence of set $\mathcal{C'}$ but it does not describe a method to find it. To find each $\mathbf{W^e(t)}$-only scheduler $\mathbf{u}_{W^c} \in \mathcal{C'}$, we need to solve the following problem for each $W^c \in \mathbb{R}$:
\begin{align}
& \underset{\epsilon_{W^c}, p_{\mathbf{U}|\mathbf{W^e}}}{\rm{maximize}} \qquad \epsilon_{W^c} \nonumber\\
& \text{s.t.:} \: \sum\limits_{\mathbf{u} \in \mathcal{U}(\mathbf{w^e},W^c)}p_{\mathbf{U}|\mathbf{W^e}}(\mathbf{u},\mathbf{w^e})=1, \: \forall \mathbf{w^e} \in \mathcal{S}^e, \nonumber \\
& \phantom{\text{s.t.:} \:} p_{\mathbf{U}|\mathbf{W^e}}(\mathbf{u},\mathbf{w^e}) \geq 0, \: \forall \mathbf{w^e} \in \mathcal{S}^e,\: \forall \mathbf{u} \in \mathcal{U}(\mathbf{w^e},W^c), \nonumber\\
& \phantom{\text{s.t.:} \:} \sum\limits_{\mathbf{w^e} \in \mathcal{S}^e, \mathbf{u} \in \mathcal{U}(\mathbf{w^e},W^c)}p_{\mathbf{U}|\mathbf{W^e}}(\mathbf{u},\mathbf{w^e})p_{\mathbf{W^e}}(\mathbf{w^e})u_i\mathbbm{1}_{W_i^e > 0} \nonumber \\
& \phantom{\text{s.t.:} \: \sum\limits_{\mathbf{w^e} \in \mathcal{S}^e, \mathbf{u} \in \mathcal{U}(\mathbf{w^e},W^c)}} \geq P_i^H - P_i^M + \epsilon_{W^c}, \: \forall i.
\label{findc}
\end{align}

In (\ref{findc}), the first two constraints ensure that the conditional distribution that defines the $\mathbf{W^e}(t)$-only scheduler is valid and the third constraint relates to the time-average constraint. Let $\epsilon^*_{W^c}$ denote the optimal value of (\ref{findc}). The optimal solutions of (\ref{findc}) for all $W^c \in \mathbb{R}$ compose set $\mathcal{C'}$. Clearly, if $\epsilon^*_{W^c}<0$, the long-term average constraint cannot be met by any $\mathbf{W^e}(t)$-only scheduler which means that no other scheduler can meet it as stated in Lemma \ref{adequacy}. Thus, if $\epsilon^*_{W^c}<0$, then $\mathcal{G}_{W^c} = \emptyset$.

\subsection{Max-Weight Scheduler}
\label{secondpart}
Our objective now is to show that the Max-Weight scheduler strongly stabilizes the deficits whenever $\epsilon^*_{W^c}<0$. If that is true, then it immediately follows from Lemma \ref{adequacy} and (\ref{rate-stability}) that the Max-Weight scheduler also satisfies the premise of Proposition \ref{equiv2} for ergodic markovian demands. To do so, we consider the following Lyapunov function:
\begin{equation}
L(\mathbf{d}(t)) \triangleq \frac{1}{2}\mathbf{1}^\top\mathbf{d}(t).
\label{lyapunov}
\end{equation}
Note that if $L(\mathbf{d}(t))$ is small then all deficits are small. Next, consider the conditional Lyapunov drift:
\begin{equation}
\Delta(\mathbf{d}(t)) \triangleq \E[L(\mathbf{d}(t+1))-L(\mathbf{d}(t))| \mathbf{d}(t)].
\label{drift}
\end{equation}
Note that if the conditional Lyapunov drift is small, then the deficits between two consecutive slots do not differ significantly. Following the same steps as in \cite[p.33]{neely}, we can show:
\begin{equation}
\begin{aligned}
\Delta(\mathbf{d}(t)) \leq  B + & \sum\limits_{i=1}^Nd_i(t)(P^H_i-P^M_i)\\
&-\E \left[\sum\limits_{i=1}^Nd_i(t)u_i(t)\mathbbm{1}_{W_i^e(t) > 0}|\mathbf{d}(t)\right],
\end{aligned}
\label{driftbound}
\end{equation}
where $B=1 + \sum_{i=1}^N(P^H_i-P^M_i)^2/N$. Note that since a small conditional drift implies "stable" deficits, we wish to reduce the conditional Lyapunov drift. To this end, we maximize the expectation in (\ref{driftbound}) by maximizing its argument. 

Thus, we need to derive a scheduler that observes the deficits $\mathbf{d}(t)$ and the bandwidth demands $\mathbf{W^e}(t)$, and maximizes the argument of the expectation in (\ref{driftbound}). This scheduler is precisely the Max-Weight scheduler defined in (\ref{max-weight}). Since the Max-Weight scheduler maximizes the expectation in (\ref{driftbound}), it holds:
\begin{equation}
\begin{aligned}
& \Delta(\mathbf{d}^*(t)) \leq B + \sum\limits_{i=1}^Nd_i^*(t)(P^H_i-P^M_i) \\
& -\E \left[\sum\limits_{i=1}^Nd^*_i(t)u_i(t)\mathbbm{1}_{W_i^e(t) > 0}|\mathbf{d}^*(t)\right], \: \forall \mathbf{u}(t) \in \mathcal{U}_{(\mathbf{w^e},W^c)},
\end{aligned}
\label{maxweightbound}
\end{equation}
where $\mathbf{d}^*(t)$ are the deficits under the Max-Weight scheduler. Since bound (\ref{maxweightbound}) holds for every possible scheduling decision $ \mathbf{u}(t) \in \mathcal{U}_{(\mathbf{w^e},W^c)}$, we can apply it for the $\mathbf{W^e}(t)$-only scheduler that solves (\ref{findc}). In that case, since $u_i(t)\mathbbm{1}_{W_i^e(t) > 0}$ is independent of deficits $\mathbf{d}^*(t)$ and  $\E[u_i(t)\mathbbm{1}_{W_i^e(t) > 0}]=P^H_i-P^M_i+\epsilon^*_{W^c}$, it holds:
\begin{equation}
\Delta(\mathbf{d}^*(t)) \leq B - \epsilon^*_{W^c}\sum\limits_{i=1}^Nd_i^*(t).
\label{newbound}
\end{equation}
By taking the expectation of (\ref{newbound}) and following the steps in \cite[p.36]{neely}, it holds:
\begin{equation}
\epsilon^*_{W^c}\limsup\limits_{T \to \infty}\frac{1}{T}\sum\limits_{t=1}^T\E[d^*_i(t)] \leq B, \: \forall i.
\label{almost-final}
\end{equation}

Therefore, if $\epsilon^*_{W^c}>0$, then all deficits $\mathbf{d}(t)$ are strongly stable. Due to (\ref{rate-stability}), it follows that the Max-Weight scheduler satisfies the time-average constraint whenever there exists a $\mathbf{W^e}(t)$-only scheduler that satisfies it. Let $\mathbf{u}'_{W^c}$ be the $\mathbf{W^e}(t)$-only scheduler defined by the solution of (\ref{findc}). Equivalently, if $\mathbf{u}'_{W^c} \in \mathcal{G}_{W^c}$, then $\mathbf{u}^*_{W^c} \in \mathcal{G}_{W^c}$. 

Given the above and Lemma \ref{adequacy}, it immediately follows that the Max-Weight scheduler satisfies the premise of Proposition \ref{equiv2} for $\mathcal{F}_{W^c}=\mathcal{G}_{W^c}$ when $\mathbf{W}(t)$ follows an ergodic MC. Therefore, our proof is finally complete.
