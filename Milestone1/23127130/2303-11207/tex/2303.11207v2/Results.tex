\section{Results}
\label{sec:results}

\subsection{The toric code}
\label{sec:TC_results}

We now focus our attention on the toric code Hamiltonian which is the simplest model that hosts a $Z_2$ topological order~\cite{kitaevAnyonsExactlySolved2006,Hamma2005} and has a non-zero TEE equal to $\gamma = \ln(2)$. The Hamiltonian is defined in terms of spin-$1/2$ degrees of freedom located on the edges of a square lattice (see Fig.~\ref{fig:mapping}(a)) and is given by
\begin{equation*}
    \hat{H} = -\sum_{p} \prod_{i \in p} \hat{\sigma}_i^{z} - \sum_{v} \prod_{i \in v} \hat{\sigma}_i^{x},
\end{equation*}
where $\hat{\sigma}_i^{x,z}$ are Pauli matrices. Additionally, the first summation is on the plaquettes and the second summation is on the vertices of the lattice~\cite{Hamma2005}. Note that the lattice in Fig.~\ref{fig:mapping}(a) can be seen as a square lattice with a unit cell containing two spins. In our simulations, we use an $L \times L \times 2$ array of spins where $L$ is the number of plaquettes on each side of the underlying square lattice. It is possible  to study the toric code with a 2D RNN defined on a primitive square lattice by merging the two spin degrees of freedom of the unit cell of the toric code into a single ``patch" followed by an enlargement of the local Hilbert space dimension in the RNN from $2$ to $4$. This idea is illustrated in Fig.~\ref{fig:mapping}(a) and is similar in spirit to how the local Hilbert space is enlarged in DMRG to study quasi-1D systems~\cite{Milsted2019}. We provide additional details about the mapping in App.~\ref{app:GRU}. 
\begin{figure}
    \centering
    \includegraphics[width = 0.95\linewidth]{figs/mapping_toric_kagome_to_square.pdf}
    \caption{Mapping of 2D toric code lattice and kagome lattice to a square lattice that can be handled by a 2D RNN wave function. For the toric code lattice in panel (a), every two sites inside the dashed green ellipses are merged. For the kagome lattice in panel (c), every three sites in a unit cell enclosed by the dashed green circles are combined.}
    \label{fig:mapping}
\end{figure}

To extract the TEE from our ansatz, we variationally optimize the 2D RNN wave function targetting the ground state of this model for multiple system sizes on a square lattice with periodic boundary conditions. After the optimization, we compute the TEE using system size extrapolation and using the Kitaev-Preskill scheme provided in Sec.~\ref{sec:TEE}. More details about the regions chosen for this construction are provided in App.~\ref{app:KP}. To avoid local minima during the variational optimization, we perform an initial annealing phase as described in Sec.~\ref{sec:annealing} (see additional details in App.~\ref{app:hyperparams}).

The results shown in Fig.~\ref{fig:renyi2_tc}(a) suggest that our 2D RNN wave function can describe states with an area law scaling in 2D. Linearized versions of the RNN wave function have been recently shown to display an entanglement area law~\cite{Wu2022}. For $L = 10$ (not included in the extrapolations in Fig.~\ref{fig:renyi2_tc}(a)), it is challenging to evaluate $S_2$ accurately as the expectation value of the swap operator is proportional to $\exp{(-S_2)}$, which becomes very small and is hard resolve accurately via sampling the RNN wave function. The improved ratio trick is an interesting alternative for enhancing the accuracy of our estimates~\cite{EE2010, Torlai_2018}. The use of conditional sampling is also another possibility for enhancing the accuracy of our measurements~\cite{EE2020}.

The extrapolation confirms the existence of a non-zero TEE whose value is close to $\gamma' = \ln(2)$ within error bars. Note that the sub-region we use to compute the TEE is half of the torus, namely a cylinder with two disconnected boundaries~\footnote{Note that this choice allows minimizing the boundary size as opposed to a square region in the bulk. This feature is desirable since the swap operator used to estimate the second Renyi entropy~\cite{RNNWF} becomes very small, and thus more sensitive to statistical errors when the boundary increases for a quantum system satisfying the area law.}. As shown in Ref.~\cite{Quasiparticle2012}, the use of this geometry means that the expected TEE becomes state-dependent and given by
\begin{equation}
    \gamma' = 2 \gamma + \ln \left( \sum_i \frac{p_i^2}{d_i^2} \right )
    \label{eq:state_dep_TEE}
\end{equation}
for the second Renyi entropy. Here $d_i \geq 1$ is the quantum dimension of an $i$-th quasi-particle. For the toric code, we have abelian anyons with $d_i = 1$ for $i = 1,2,3,4$. Additionally $p_i = |\alpha_i|^2$ is the overlap of the computed ground state $\ket{\Psi}$ with the $i$-th minimally entangled state (MES) $\ket{\Xi_i}$ where
\begin{equation*}
    \ket{\Psi} = \sum_{i} \alpha_i \ket{\Xi_i}.
\end{equation*}
The observations above and the numerical result $\gamma_{\text{RNN}} \approx \ln(2)$ suggest that the RNN wave functions optimized via gradient descent and annealing find a superposition of MES, as opposed to DMRG which preferentially collapses to a single MES for relatively low bond dimensions. For relatively large bond dimensions a superposition of MES can be recovered in a DMRG simulation~\cite{TEE2012, Jiang2013}. The analysis provided in App.~\ref{app:RNN_MES} demonstrates that our optimized RNN ansatz finds a uniform superposition of two MES which increases the entanglement in the state with respect to a single MES. Thus using Eq.~\eqref{eq:state_dep_TEE}, we expect $\gamma' = 2 \ln(2) + \ln\left(\frac{1}{4} + \frac{1}{4} \right) = \ln(2)$, which is consistent with our numerical observations. 

We note that the exact autoregressive sampling procedure plays a key role in the ability of our RNN ansatz to sample a superposition of different topological sectors when this superposition is encoded in our ansatz. For wave functions representing the ground state of the toric code used in combination with Markov-chain Monte Carlo methods, the probability of sampling different topological sectors of the state is exponentially suppressed even if the exact wave function ansatz encodes different topological sectors. This observation can be illustrated using an exact convolutional neural network construction of the toric code ground state which contains an equal superposition of different topological sectors~\cite{Carrasquilla2017}. Although in principle such representation contains all topological sectors, its form is not amenable to exact sampling and uses Markov chains so that upon sampling with local moves the system chooses a fixed topological sector. 

To further verify that our 2D RNN wave function can extract the correct TEE of the 2D toric code, we compute the TEE using the Preskill-Kitaev construction, that has contractible surfaces, and for which the TEE does not depend on the topological sector superposition~\cite{Quasiparticle2012,TEE2017} (see App.~\ref{app:KP} for details about the construction). The results reported in Fig.~\ref{fig:renyi2_tc}(b) demonstrate an excellent agreement between the TEE extracted by our RNN and the expected theoretical value for the toric code.

\begin{figure}
    \centering
    \includegraphics[width = \linewidth]{figs/ToricCode_plots.pdf}
    \caption{Entanglement properties of the 2D toric code. (a) Second Renyi entropy scaling is computed using our RNN wave function on the 2D toric code for different lengths $L$ where the total system size is given as $L \times L \times 2$. (b) TEE computed with the Kitaev-Preskill construction (see App.~\ref{app:KP}). The values found by the RNN are very close to $\ln(2)$. Error bars correspond to one standard deviation and are smaller than the symbol size.}
    \label{fig:renyi2_tc}
\end{figure}

\subsection{Bose-Hubbard model on kagome lattice}

We now turn our attention to a hard-core Bose-Hubbard model on the Kagome lattice, which has been shown to host topological order~\cite{PhysRevLett.97.207204,TEE2011,Zhao_2022}. The Hamiltonian of this model is given by
\begin{equation}
    \hat{H} = -t\sum_{\langle i,j \rangle} \left( b_i^{\dagger} b_j + b_i b_j^{\dagger} \right) + V\sum_{\hexagon} n_{\hexagon}^2,
    \label{eq:bhsl}
\end{equation}
where $b_i$ ($b_i^{\dagger}$) is the annilihation (creation) operator. Furthermore, $t$ is the kinetic strength, $V$ is a tunable interaction strength and $n_{\hexagon} = \sum_{i \in \hexagon} (n_i - 1/2)$. The first term corresponds to a kinetic term that favors hopping between nearest neighbors, whereas the second term promotes an occupation of three hard-core bosons in each hexagon of the kagome lattice. In our setup, we choose $V$ in units of the kinetic term strength $t$.

The atom configurations of this model correspond to an $L \times L \times 3$ array of binary degrees of freedom where $L$ is the size of each side of the kagome lattice. Following an analogous approach to the toric code, we combine three sites of the unit cell of the kagome lattice as input to the 2D RNN cell, as illustrated in Fig.~\ref{fig:mapping}(b). This allows us to map our kagome lattice with a local Hilbert space of $2$ to a square lattice with an enlarged Hilbert space of size $2^3 = 8$.

The model is known to host a $Z_2$ spin-liquid phase for $V \gtrsim 7$~\cite{TEE2011,wangTopologicalSpinLiquid2017,Zhao_2022}. To confirm this finding, we estimate $\gamma$ for the system sizes $6 \times 6 \times 3$ and $8 \times 8 \times 3$. We use the Kitaev-Preskill construction~\cite{KitaevPreskill2006}. The details of the construction of the regions $A, B$ and $C$ are provided in App.~\ref{app:KP}. As the Hamiltonian in Eq.~\ref{eq:bhsl} has a $U(1)$ symmetry associated with the conservation of bosons in the system, we impose this symmetry on our RNN wave function~\cite{RNNWF}. We also supplement the VMC optimization with annealing to overcome local minima as previously done for the 2D toric code (see App.~\ref{app:VMC}). For the system size $8 \times 8 \times 3$, the RNN ansatz parameters were initialized using the optimized parameters of the system size $6\times6\times3$ (see details about the hyperparameters in App.~\ref{app:hyperparams}). This pre-training technique has been motivated in Refs.~\cite{roth2020iterative, RNNAnnealing, luo2021gauge}.

The results are provided in Fig.~\ref{fig:BoseHubbard}. The computed TEEs for $L = 6,8$ show a saturation of $\gamma_{\rm RNN}$ for large values of the interaction strength $V$. We observe that the saturation values of $\gamma_{\rm RNN}$ are in good agreement with the expected TEE $\gamma = \ln(2)$ of a $Z_2$ spin-liquid~\cite{TEE2011}. Additionally, the negative values of $\gamma_{\rm RNN}$ observed for $V \leq 6$ in the superfluid phase~\cite{TEE2011} may be related to the presence of Goldstone modes that manifest themselves as corrections to the area law in the entanglement entropy and can be seen as a negative contribution to the TEE~\cite{Bohdan2015}. We note that the QMC methods are capable of obtaining a consistent value with the exact TEE for this model at $V = 8$ for very large system sizes~\cite{Zhao_2022} using finite-size extrapolation. This observation suggests that our RNN ansatz is still limited by finite-size effects at $V=8$ (see Fig.~\ref{fig:BoseHubbard}) for which the TEE is not yet saturated to $\ln{2}$. Other sources of error in our calculation may be due to inaccuracies in the variational calculations and statistical errors due to the sampling. However, we note that our variational calculation is performed at zero temperature, which makes our calculations insensitive to temperature effects as opposed to QMC~\cite{TEE2011}. 

\begin{figure}
    \centering
    \includegraphics[width =\linewidth]{figs/TEE_BoseHubbard_PreskillKitaev.pdf}
    \caption{A plot of the topological entanglement entropy against the interaction strength $V$ (in units of $t$) of Hard-core Bose-Hubbard model on kagome lattice for system sizes $N = L \times L \times 3$ where $L = 6,8$. The calculations were performed using the Kitaev-Preskill construction (see App.~\ref{app:KP}). The continuous black horizontal line corresponds to a zero TEE, and the dashed blue horizontal line for a $\ln(2)$ TEE.}
    \label{fig:BoseHubbard}
\end{figure}

