\section{Conclusions and Outlooks}

We have demonstrated a successful application of neural network wave functions to the task of detecting topological order in quantum systems. In particular we use RNNs borrowed from natural language processing as ansatz wave functions.
RNNs enjoy the autoregressive property which allows to sample uncorrelated configurations. They are also capable of estimating second Renyi entropies using the swap trick~\cite{RNNWF} with which we computed TEEs using finite-size scaling and Kitaev-Preskill constructions~\cite{KitaevPreskill2006}. Furthermore, the structural flexibility of the RNN offers the possibility to handle a wide variety of geometries including periodic boundary conditions in any spatial dimension which alleviate boundary effects on the TEE.

We have empirically demonstrated that 2D RNN wave functions support the 2D area law and can find a non-zero TEE for the toric code and for the hard-core Bose-Hubbard model on the Kagome lattice. We also find that RNNs favor coherent superpositions of minimally-entangled states over minimally-entangled states themselves. The success of our numerical experiments hinges on the combination of the exact sampling strategy used to compute observables, the structural properties of the RNN wave function, and the use of annealing as a strategy to overcome local minima during the optimization procedure. 

The accuracy improvement of our findings can be achieved through the use of more advanced versions of RNNs and autoregressive models in general~\cite{RNNAnnealing, Wu2022}, or even a hybrid approach that combines QMC and RNNs~\cite{Bennewitz2021NeuralEM, Czischek_2022}. Similarly, the incorporation of lattice symmetries provides a strategy to enhance the accuracy of our calculations~\cite{RNNWF, RNNAnnealing, Nomura2021}. Although our results match the anticipated behaviour of the toric code and Bose-Hubbard spin liquid models, we highlight that the RNN wave function may be susceptible to spurious contributions to the TEE~\cite{kimUniversalLowerBound2023} and we have not addressed this issue in our work.

Finally, our methods can be applied to study other systems displaying topological order, such as the Rydberg atoms array~\cite{RubyDMRG2021, RydbergSimulator2021, giudici2022dynamical}, either through variational methods or in combination with experimental data. To experimentally study topological order, it is possible to use quantum state tomography with RNNs~\cite{Carrasquilla2019}. This involves using experimental data to reconstruct the state seen in the experiment followed by an estimation of the TEE using the methods outlined in our work. Overall, our findings suggest that RNN wave function ansatzes have promising potential for discovering new phases of matter with topological order.
