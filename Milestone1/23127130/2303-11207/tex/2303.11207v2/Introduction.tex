\section{Introduction}

Landau symmetry breaking theory provides a fundamental description of a wide range phases of matter and their phase transitions through the use of local order parameters~\cite{sachdev_2011}. Despite the fact that a great deal of our theoretical and experimental investigations of interacting quantum many-body systems have been developed with the aim of studying local order parameters, it is well-known that the most intriguing strongly correlated phases of matter may not be easily characterized through these observables. 
Instead, several states of matter seen in modern theoretical and experimental studies are characterized using non-local order parameters that rely on the phases' topological properties~\cite{Wen_2013,RydbergSimulator2021,TEE2021}. Topological order, in particular, refers to a type of order characterized by the emergence of quasi-particle anyonic excitations, topological invariants, and long-range entanglement, which typically do not appear in traditional forms of order. As a result of these properties, topologically ordered phases have been suggested as an important building block for the development of a protected qubit resistant to perturbations and errors~\cite{Dennis_2002, kitaevAnyonsExactlySolved2006,kitaev2009topological}. Interestingly, such qubits have been devised recently at the experimental level~\cite{Acharya2023}. 
    
While most manifestations of topological order are dynamical in nature---e.g. anyon statistics, ground state degeneracy, and edge excitations~\cite{LevinWen2006}--topological order can also be characterized directly in terms of the ground state wave function and its entanglement. In particular, a probe for topological order is the topological entanglement entropy (TEE)~\cite{KitaevPreskill2006, LevinWen2006}, which offers a characterization of the global entanglement pattern of topological ground states not present in conventionally ordered systems. Notably, the TEE is readily accessible for large classes of topological orders~\cite{LevinWen2006,Hamma2004}, in numerical simulations based on quantum Monte Carlo (QMC)~\cite{TEE2011,TEE2017, Zhao_2022} and density matrix renormalization group (DMRG)~\cite{TEE2012, Jiang2013}, as well as in experimental realizations of topological order based on gate-based quantum computers~\cite{TEE2021}.  
Machine learning (ML) techniques offer an alternative approach study quantum many-body systems and have proved useful for a wide array of tasks including the classification of phases of matter~\cite{Carrasquilla2017, Broecker2017, PhysRevX.7.031038, Miles_2021}, quantum state tomography~\cite{Torlai2018, Carrasquilla2019}, finding ground states of quantum systems~\cite{androsiuk1993,LAGARIS19971,Carleo2017, zi2018, Di_Luo, Pfau_2020, Hermann_2020, choo_fermionicnqs2020, RNNWF, roth2020iterative}, studying open quantum systems~\cite{Vicentini2019,Luo2022}, and simulating quantum circuits~\cite{Jonsson2018,Medvidovic2021, Carrasquilla_2021}, among many others~\cite{dunjkoMachineLearningArtificial2018,RevModPhys.91.045002,Carrasquilla2020,dawidModernApplicationsMachine2022}. In particular, neural networks representations of quantum many-body states have been shown to be able of expressing topological order using, e.g., restricted Boltzmann machines~\cite{Deng_2017,Glasser_2018, chenEquivalenceRestrictedBoltzmann2018,PhysRevB.99.155136}, convolutional neural networks~\cite{Carrasquilla2017} and autoregressive neural networks~\cite{luo2021gauge}. Here we use recurrent neural networks (RNN)~\cite{hochreiter1997long,graves2012supervised,lipton2015} as an ansatz wave function~\cite{RNNWF, roth2020iterative} to investigate topological order in 2D through the estimation of the TEE. We focus on two model Hamiltonians exhibiting topological order, namely Kitaev's toric code~\cite{kitaevAnyonsExactlySolved2006,kitaev2009topological} and a Bose-Hubbard model on the kagome lattice previously shown to host a gapped quantum spin liquid with non-trivial emergent $\mathbb{Z}_2$ gauge symmetry~\cite{PhysRevLett.97.207204,TEE2011,Zhao_2022}. In our study, we use Kitaev-Preskill constructions~\cite{KitaevPreskill2006}, and finite size-scaling analysis of the entanglement entropy to extract the TEE. We find convincing evidence that RNNs are capable of expressing ground states of Hamiltonians displaying topological order. We also find evidence that the RNN wave function is naturally biased toward finding superpositions of minimally entangled states, as reflected in the calculations of entanglement entropy and Wilson loop operators for the toric code. Overall, our results indicate that RNNs can represent phases of matter beyond the conventional Landau symmetry-breaking paradigm.
