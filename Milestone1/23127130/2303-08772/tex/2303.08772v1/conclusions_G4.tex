\vspace{-2mm}
\section{Conclusion} \label{sec:conclusion}

In this paper, we introduced the Optimistic Online Learning for Reservation (OOLR) algorithm that allows the SP to make reservations under uncertainty while incorporating predictions about the future gradient. We then proposed to combine this decision model with a prediction model, thus creating the OOLRgrad solution with better performance than the classical FTRL solution.

%We plan next to extend the model to the case the NO is unable to fulfill the SP request in its entirety.

%One interesting direction would be to consider varying constraints for the SP reservations, which means the feasible set $\Delta$ boundaries would change from slot to slot.
%The potential of network slicing markets can be only unleashed if service providers are able to reserve resources effectively. To that end, we proposed a set of slice reservation policies, based on the theory of online convex optimization, which enable the SP to learn how to reserve resources optimally. Our policies are robust to arbitrary changes of the resource prices, oblivious to lack of this information when the reservations are made, and can achieve optimal slice orchestration even when the SP needs are unknown and time-varying. These key elements build a practical and general slicing framework with performance and budget guarantees.

\section{Acknowledgments}

The research leading to this work is funded, in part, by Science Foundation Ireland (SFI), the National Natural Science Foundation of China (NSFC), and the European Commission under the SFI-NSFC Partnership Programme Grant Number 17/NSFC/5224, SFI grant 13/RC/2077 P2, and the Grant Number 101017109 (DAEMON).


%This work was supported by the European Commission through Grant No. 856709 (5Growth) and Grant No. 101017109 (DAEMON); and by SFI through Grant No. SFI 17/CDA/4760 and Grant No. 17/NSFC/5224.
