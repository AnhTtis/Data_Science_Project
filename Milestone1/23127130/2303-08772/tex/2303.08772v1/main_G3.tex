
%\documentclass[conference]{IEEEtran}
%\documentclass[10pt, conference, letterpaper]{IEEEtran}
%% \documentclass[conference,onecolumn]{IEEEtran}
%%\hyphenation{op-tical net-works semi-conduc-tor}
%\IEEEoverridecommandlockouts

\documentclass[conference]{IEEEtran}
\IEEEoverridecommandlockouts


\usepackage{amsmath}
\usepackage{bm}
\usepackage{mdframed}
\usepackage{amsthm}
\usepackage{mathtools}
\usepackage{comment}
\newtheorem{thm}{Theorem}
\usepackage[utf8]{inputenc}
\usepackage{graphicx}
\usepackage[colorinlistoftodos]{todonotes}
\usepackage{amssymb}
\usepackage{longtable}
%\usepackage[colorlinks=true, allcolors=blue]{hyperref}
\usepackage{nohyperref}
\hypersetup{bookmarks=false}
\usepackage[noadjust]{cite}
\allowdisplaybreaks
\usepackage{lineno}
\usepackage{lipsum}
\usepackage{multicol}
\usepackage{subcaption}
\usepackage{enumitem}
\newtheorem{theorem}{Theorem}
\newtheorem{lemma}{Lemma}
\newtheorem{corollary}{Corollary}
\newtheorem{assumption}{Assumption}
\usepackage{xcolor}
%\addtolength{\topmargin}{+0.1cm}

%\linenumbers
%\modulolinenumbers[5]
\usepackage[linesnumbered,ruled,vlined]{algorithm2e}
\renewcommand{\baselinestretch}{0.97}
\newcommand{\gi}[1]{{\color{red}\textbf{}{G: #1}}}

\newcommand{\fm}[1]{{\color{purple}\textbf{}{FM: #1}}}

\def\ll{\lambda}
\def\grad{\nabla}
\DeclarePairedDelimiter{\norm}{\lVert}{\rVert}
\DeclarePairedDelimiter\abs{\lvert}{\rvert}

\SetKwInput{KwInput}{Input}                % Set the Input
\SetKwInput{KwOutput}{Output}              % set the Output
\SetKwInput{KwInitialize}{Initialize}                % Initialization

\renewcommand{\baselinestretch}{0.97} 

\begin{document}
% Enable bibliography control
\bstctlcite{IEEEexample:BSTcontrol}
 \title{Reservation of Virtualized Resources with Optimistic Online Learning}


\vspace{-4mm}
\author{\IEEEauthorblockN{Jean-Baptiste Monteil\IEEEauthorrefmark{1}, George Iosifidis\IEEEauthorrefmark{2}, Ivana Dusparic\IEEEauthorrefmark{1}}%Jernej Hribar,\IEEEauthorrefmark{3} Luiz DaSilva\IEEEauthorrefmark{4}}\\
%	\vspace{-4.75mm}
%	\IEEEauthorblockA{
		\IEEEauthorrefmark{1}School of Computer Science and Statistics, Trinity College Dublin\\
		\IEEEauthorrefmark{2}Delft University of Technology, Netherlands}%\\
		%\IEEEauthorrefmark{3}CONNECT Centre, Trinity College Dublin\\
		%\IEEEauthorrefmark{4}Department of Electrical and Computer Engineering, Virginia Tech, USA}
%		\thanks{{This work was supported by Science Foundation Ireland (SFI) under Grant No. 17/CDA/4760. This work has been partially supported by EC H2020 5GPPP 5Growth project (Grant 856709). This publication has emanated from research supported in part by a research grant from SFI and is co-funded under the European Regional Development Fund under Grant Number 13/RC/2077}}%


\maketitle
%\thispagestyle{plain}
%\pagestyle{plain}

\vspace{-2mm}
\begin{abstract}
The virtualization of wireless networks enables new services to access network resources made available by the Network Operator (NO) through a Network Slicing market. The different service providers (SPs) have the opportunity to lease the network resources from the NO to constitute slices that address the demand of their specific network service. The goal of any SP is to maximize its service utility and minimize costs from leasing resources while facing uncertainties of the prices of the resources and the users' demand. In this paper, we propose a solution that allows the SP to decide its online reservation policy, which aims to maximize its service utility and minimize its cost of reservation simultaneously.  %This paper looks at how one SP can design an online reservation policy based on any feedback and prediction it receives from the NO.
We design the Optimistic Online Learning for Reservation (OOLR) solution, a decision algorithm built upon the Follow-the-Regularized Leader (FTRL), that incorporates key predictions to assist the decision-making process. Our solution achieves a $\mathcal{O}(\sqrt{T})$ regret bound where $T$ represents the horizon. We integrate a prediction model into the OOLR solution and we demonstrate through numerical results the efficacy of the combined models' solution against the FTRL baseline.



%The softwarization of wireless networks paves the way to reservation-based models, like in cloud computing systems, where service providers (SPs) lease sliced resources from network operators. In our setting, an SP reserves resources owned by the NO and then orchestrates those resources into the slice that accommodates the demand. In this paper, we look how an SP can acquire resources available on the spot market, to maximize its performance, while satisfying its budget constraint, in dynamic settings. We consider different kind of uncertainties: the demand coming from the users, the pricing of spot resources,  the utility of the different resources and we make no assumptions about them. Using the constrained-OCO framework with time-varying constraints, we propose the Online Learning for Reservation (OLR) solution that achieves sublinear fit and regret in the case of non-stationary traces, and that has theoretical guarantees. We offer two extensions of practical importance, which improve the performance of our ORL solution and place it in a realistic setting. 
\end{abstract}
\IEEEpeerreviewmaketitle

%\begin{IEEEkeywords} 5G Networks, Cloud-RAN, Virtualized RAN, Flexible Functional Split, Optimization \end{IEEEkeywords}
\begin{IEEEkeywords}Online convex optimization, network slicing markets, virtualization, resource reservation, SP utility maximization, FTRL algorithm.
\end{IEEEkeywords}

%%%%%%%%%%%%%%%%%%%%%%%%%%%%%%%% ================== MAIN CONTENT ==================
% Introduction
\section{Introduction}

%\textbf{Motivation}. 

\textbf{Motivation}. The virtualization of wireless networks has gained significant interest in recent studies, cf. \cite{5g-ppp, mano-nfv}. This new technology enables the development of the Network Slicing framework, where service providers (SPs) can lease virtualized network resources from the Network Operator (NO) to address the demand of their specific network service \cite{foukas-commag}. %, xavi-commag2018}. 
Network Slicing promises to boost the utilization efficiency of the network resources by accommodating multiple and diverse SPs on the NO's infrastructure. This in turn brings new challenges: on the one hand the NO must accommodate heterogeneous slices on its network to satisfy diverse requirements of the SPs; on the other hand the SPs must request network resources or slice requirements in a smart and proactive way by anticipating their future demand.

The players are expected to operate in a real-time market, where the SPs can lease both computing and storage resources while the NO offers both in-advance reservation and on-the-fly spot opportunities. The modeling of such slicing market draws ideas from cloud marketplaces \cite{amazon-reserved, amazon-spot}, where the Cloud Provider allows customers to bid for resources in the on-demand and spot markets \cite{carlee-infocom18, carlee-sigcomm15}. This market will allow the NO to proactively schedule the slice configuration based on the information coming from the in-advance reservation requests, but also offer the available spot resources dynamically, leading to slice re-configuration and boosting network utilization.

In this context, one SP competes with other SPs for the network resources in the on-demand and spot markets and must request/bid for the resources while ignorant of their prices. % and facing uncertainty about its own demand. %\cite{hossain-survey2018}. 
We expect the NO to reveal those prices after the SP request. %, once the resources are leased to the SP and the payment is effective. %We expect the NO reveals the prices along with the SP demand % the deserved traffic of the SP after the SP request. 
Therefore, the SP must decide its requests dynamically without the information of the resource pricing and its own future demand. Additionally, we expect the prices to vary according to non-stationary patterns, as they might depend on multiple underlying factors, such as the other SPs requests, the NO internal needs, etc. We highlight here the necessity for the SP to build a decision model robust to uncertainty, while being able to use its own historical demand and the NO's feedback about historical prices.

\textbf{Related Work}. %paper using prediction
By anticipating the resource utilization, the NOs can enhance their resource management decisions regarding resource provisioning or allocation.
In \cite{DeepCog}, network traffic information is leveraged to plan the capacity needed for each slice in a multi-tenant framework. Using a data-driven approach including C-RAN, MEC and core networks, the solution outperforms other state-of-the-art deep learning solutions \cite{infocom17}, \cite{mobihoc18}. The approach in \cite{oliveiraTNSM} employed an adaptive forecasting model of the elastic demand for network resources to perform slice allocation in Internet Access Services. The authors in \cite{XaviINFOCOM17} and \cite{XaviTrans19} predicted the required resources by tenants for the future time window to perform slice requests admission and schedule the users' traffic within each slice. %The prediction module can adapt its predictions by changing a forecasting error probability according to the feedback sent by the slice scheduler. 
In \cite{cui-iccc20} cellular traffic prediction helps the allocation policy for the vehicular network slice. \cite{jb-icc20} uses historical traffic to design the SP resource reservation policy. %Similarly, we use prediction to assist the SP reservation policy.
Unlike these approaches, our solution does not need offline training and provides performance guarantees against all types of traces.

Recent works consider the SP resource provisioning problem. The paper \cite{reyhanian} developed a two-time scale approach for the activation and the re-configuration of the slices while considering the reservation of both RAN and backhaul resources. \cite{zhang-tcom2018} focuses on wireless spectrum considering two reservation schemes (in advance and on demand). In \cite{vincent-TVT2020}, the authors develop a two-stage approach for the resource reservation and the intra-slice resource allocation. These works presume a stationary environment where user statistics do not change and/or cost of resources are supposed constant. This paper differs from our previous works \cite{JB2021, JB2022}, as we now use prediction to support the reservation model.

%The focus of this paper is to tackle exactly the problem faced by the SP. We propose a solution that allows the SP to decide its online reservation policy, which aims to maximize its service utility and minimize its cost of reservation simultaneously. We develop an optimistic online learning algorithm (OOLR) able to incorporate predictions to assist the reservation decision of the SP. %Our solution provides guarantees of performance even for arbitrarily bad predictions which may arise in volatile settings. We complement our decision model with an accurate, robust, and low-complexity prediction model. The combined solution achieves a $\mathcal{O}(T^{3/4})$ regret.

%We develop an online optimization algorithm so that the SP learns how to proactively reserve resources in an online manner. Our proposal falls under the scope of Online Convex Optimization, which we detail in the following subsection.

\textbf{Methodology and Contributions}.
%OCO
The problem of learning how to bid in an online manner while facing uncertainty fits to the Online Convex Optimization (OCO) framework, introduced by Zinkevich \cite{zinkevich}. In OCO, the learner tries to minimize its total loss with respect to the best static solution:
\begin{align}
    R(T) = \sum_{t=1}^T f_t(\bm z_t) -  \min_{\bm z \in \mathcal{Z}} \sum_{t=1}^T f_t(\bm z), \label{eq:static_regret}
\end{align}
by deciding the reservation vector $\bm z_t$ at each round $t$, without knowing the convex loss $f_t$. We say the online policy $\{\bm z_t\}_{t=1}^T$ has \emph{no-regret} if the achieved regret is sublinear, i.e. $R(T) = o(T)$, in other words $\lim_{T\rightarrow \infty} R(T)/T = 0$. We build our Optimistic Online Learning for Reservation (OOLR) solution upon the Follow-The-Regularized Leader (FTRL) algorithm \cite{ftrl}. % which ensures a $\mathcal{O}(\sqrt{T})$ regret. 
We develop an \emph{optimistic} version of the FTRL, first introduced by Rakhlin and Sridharan \cite{sridharan}, where the decision relies on an adaptive proximal regularizer term and the optimistic term of the next gradient prediction $\grad \hat f_{t+1}(\hat{\boldsymbol{z}}_{t+1})$. With perfect predictions, the regret of our decisions reduces to $\mathcal{O}(1)$, synonymous of negative regret. With arbitrarily bad predictions (of the order of $T$), the regret bound is $\mathcal{O}(\sqrt{T})$.


%predictions
As the SP accumulates historical data about prices and demand, there exists the possibility to extract predictions for the next slot values based on previous window of the traces by using auto-regressive methods. Holt-Winters, Auto-Regressive Integrated Moving Average (ARIMA) or Neural Networks have been applied in \cite{oliveiraTNSM} and in \cite{XaviINFOCOM17}. Albeit accurate, these methods do not provide performance guarantees. % and therefore do not improve the $\mathcal{O}(\sqrt{T})$ performance bound we observe for arbitrarily bad predictions. 
The ARMA-OGD algorithm presented in \cite{anava} is an accurate, robust and computationally low prediction model. It generates the predictions through an auto-regressive process, where the lag coefficients are updated using the online gradient descent (OGD) method, which has low time complexity. It also provides regret guarantees against the best Auto-Regressive Moving Average (ARMA) predictor with full hindsight of the future. %The integration of this prediction module into our main decision algorithm (OOLR) improves the regret bound to $\mathcal{O}(T^{1/4})$.


%In the line of our previous works \cite{JB2021, JB2022}, we consider a hybrid slicing market where the SP can reserve different types of resources, which will compose the reserved slice. %, both in-advance and on-the-spot. 
%The NO pricing scheme is unknown to the SP, and the prices of the resources are revealed to the SP only after it makes the bid. Hence, the SP has to reserve the network resources without knowing the demand from its users and the prices of the network resources. %The goal of the SP is to maximize the slice performance and to minimize the overall cost of reservation. 
%The SP can leverage the feedback received from the NO to design accurate predictions that will assist the online reservation policy, which end goal is to maximize the slice performance and minimize the cost of reservation.

%The SP receives feedback from the NO about its demand and the resource pricing after its reservation being made. As time passes, the SP accumulates historical data of those varying and potentially non-stationary signals. The latter information paves the way to the design of a prediction algorithm able to feed accurate predictions of the different signals into the reservation-based decision model.

%We model the reservation problem faced by the SP as a learning problem, where the goal is to design a \emph{no-regret online reservation policy}. We fall under the scope of the standard Online Convex Optimization (OCO) problem introduced in the seminal paper of Zinkevich \cite{zinkevich}, which aims to find the online policy $\{\bm z_t\}_{t=1}^T$ that minimizes the regret with respect to the best static solution:
%\begin{align}
 %   R(T) = \sum_{t=1}^T f_t(\bm z_t) -  \min_{\bm z \in \mathcal{Z}} \sum_{t=1}^T f_t(\bm z). \label{eq:static_regret}
%\end{align}
%At each slot $t$, the learner (the SP in our case) must decide $\bm z_t$ (the reservation vector) from a convex set $\mathcal{Z}$ without knowing the convex loss $f_t$. 
%We say the online policy $\{\bm z_t\}_{t=1}^T$ has \emph{no-regret} if the achieved regret is sublinear, i.e. $R(T) = o(T)$, in other words $\lim_{T\rightarrow \infty} R(T)/T = 0$. 

%We build our OOLR solution upon the FTRL algorithm \cite{ftrl}. % which ensures a $\mathcal{O}(\sqrt{T})$ regret. 
%We develop an \emph{optimistic} version of the FTRL, first introduced by Mohri and Yang \cite{mohri}, where the decision relies on an adaptive proximal regularizer term and the optimistic term of the next gradient prediction $\grad \hat f_{t+1}(\hat{\boldsymbol{z}}_{t+1})$. 
%We modify the latter to unleash the use of predictions that will reduce the uncertainty inherent to the unknown parameters at slot $t$. 
%We opt for an accurate, robust and computationally low prediction model which we find in the work of Anava et al \cite{anava}: the ARMA-OGD algorithm. %The latter generates the predictions through an AR($q$) process, where the $q$ lag coefficients are updated using the online gradient descent (OGD) method, which has low time complexity. It also provides regret guarantees against the best ARMA predictor with full hindsight of the future $\{f_i\}_{i=t+1}^T$. The two combined models provide the SP with a unique solution which is competitive and presents strong guarantees.




The contribution can be stated as follows:
\begin{itemize}[leftmargin=4mm]
\item we formulate an optimization problem for the SP where it aims to maximize the leased slice utility and minimize the reservation cost in the long-term; 
\item to solve the reservation problem faced by the SP, we develop an online learning solution (OOLR) which incorporates the \emph{optimistic} prediction of the next slot gradient;
\item we provide regret bound guarantees of $\mathcal{O}(\sqrt{T})$ for arbitrarily bad predictions and $\mathcal{O}(1)$ for perfect predictions;
\item we implement a prediction module to assist our OOLR decision algorithm and we demonstrate good performance of the combined solution, named OOLRgrad; 
\item against real world data and non-stationary traces, our OOLRgrad solution outperforms the FTRL baseline. We extend our model to the situation the NO only fulfills part of the SP reservation request due to capacity constraints. 
%\item 


%\item We use the multi-resource reservation model introduced in \cite{JB2021} and \cite{JB2022} and we reformulate the learning problem into an unconstrained OCO problem;
%\item for our specific reservation problem, we propose a new optimistic online learning algorithm with adaptive regularizer (OOLR) which achieves negative regret in the best case, sublinear regret in the worst case;
%\item we develop a prediction algorithm able to feed accurate input into the OOLR algorithm which consequently enhance the quality of decisions and improve the regret performance.
\end{itemize}









%In this framework, the NO, which is the owner of the network, must accommodate heterogeneous slices corresponding to diverse types of services \cite{transWC2018, VT2020}.  Most papers have proposed solutions for assisting the NO in the resource allocation task that comes along with the accommodation of slices. 


%Fewer works have focused on how the SP should request slices or network resources. In \cite{zhang-tcom2018}, the authors consider a hybrid reservation/spot market where the SP reservations are decided by solving a stochastic problem, and a mixed time scale reservation model was studied in \cite{vincent-TVT2020}. %Our previous work \cite{jb-icc20} employed demand and price predictions (via neural networks) to assist the SP's reservations, while \cite{tony-tnsm19} focused on slice reconfiguration costs.
%In \cite{paschos-infocom19}, the authors consider a reservation model, where a customer at each slot requests different types of cloud resources to minimize its total cost while satisfying the average constraint on the demand. The authors propose a primal-dual type of algorithm and their online policy is asymptotically feasible and achieves $o(T)$ regret against the $K$-slot static benchmark (static reservation that only satisfies the average constraint every $K$ slots), with $K=o(T^{1-\epsilon})$. Unlike our approach, the costs in the objective are known and fixed.
%In \cite{zhang2017}, the authors design a reservation model of long-term and short-term VM instances, then used for VNF deployment, and create a complete online solution - composed of VNF demand prediction and reservation decisions - with performance guarantees.


%softwarization of resources
%The increasing softwarization of wireless networks coupled with the proliferation of over-the-top service providers (SPs) which rely on network operators' infrastructure (NOs), have spurred numerous studies for  {network slicing} solutions, cf. \cite{foukas-commag}, \cite{xavi-commag2018}. For instance, researchers have proposed embedding algorithms for assisting NOs to accommodate heterogeneous slices \cite{paschos-mag17}; and pricing mechanisms to maximize the operators' revenue from selling slices to different SPs \cite{hossain-survey2018}. These schemes are expected to operate in near-real time and enable the fine-grained (re-)allocation of network resources; hence, boosting their utilization efficiency. Yet, an aspect that has received less attention is how the SPs should request slices.% in this dynamic environment. 

%slicing markets, ~ cloud computing market
%The envisioned network slicing markets draw ideas from pertinent cloud computing marketplaces \cite{google-pricing, google-spot, amazon-reserved, amazon-spot}, where cloud providers adapt dynamically their offered prices, and SPs complement their reservations with on-the-fly bidding in spot markets. This flexibility compounds the slice-reservation task of each SP which has to request resources  without knowing the needs of its users; to decide between (lower-cost) advance reservation and (higher-cost) dynamic reservation; and to anticipate the future slice prices that, in turn, depend on the NO's internal needs and the requests of other SPs. If these decisions are ineffective, the SPs might over/under-reserve resources, which will lead to network under-utilization or induce prohibitively high servicing costs. And these effects can nullify the anticipated benefits of slicing.

% online optimization
%OCO is an efficient mathematical framework introduced by Zinkevich \cite{zinkevich}, to evaluate the performance -through the regret metric- of online optimization algorithms \cite{koppel, cao2018online, giannakis-TSP17}, i.e. algorithms that solve an optimization problem in an online manner, at each time $t \in \mathcal{T}=\{1,\ldots,T\}$, assuming we have a time-slotted system. 
%In online learning and more specifically in OCO problems \cite{shalev-book, hazan-book}, an agent has to take sequential decisions at each time $t$, that minimizes the objective, while satisfying the constraints (if the problem is constrained). The agent does not have access to the time-varying objective function $f_t(.)$ prior to the decision (nor the constraints). Thus, the difficulty arises in this situation as the agent must take the decisions under uncertainty. The design of an efficient algorithm, able to adapt to the dynamics of the problem, becomes essential. The regret metric reflects the efficiency of the algorithm: it measures the difference of the cumulative loss incurred by our decisions -over $T$ time slots- from the cumulative loss that experiences the benchmark, which has full hindsight of the problem. A sublinear regret, i.e. $R_T=o(T)$, is regarded as a good performance, as it implies that the online algorithm asymptotically performs no worse than the benchmark. In constrained OCO problems, we deem both the regret metric and the constraint violations metric -or fit- which is the accumulation of violations through the duration $T$. We observe a natural trade-off between the two metrics and our aim is to design algorithms that keep both a sublinear regret and a sublinear fit.

%paper focus

%The focus of this paper is to tackle exactly this problem by studying optimal slice reservation from the perspective of service providers. Our aim is to design online reservation policies which an SP can employ to maximize the performance of its service while not exceeding the average monetary budget it has committed for this purpose. This is a key step for unleashing the full potential of slicing markets. 

 
%\textbf{Related Work}.
%The design of slice markets is a relatively new research area. In \cite{toktam-icc17}, the authors proposed a mechanism for the NO to auction its sliced resources; \cite{paschos-infocom18} formulated slicing as a utility maximization problem; and  \cite{malandrino-infocom20} considered QoS metrics when serving the slices. Similarly, \cite{andres-conext18} studied the impact of slice overbooking; \cite{xavi-infocom20} employed predictive capacity allocation for improving the slice composition; and \cite{nguyen-jsac19} focused on dynamic slicing via reinforcement learning. Fewer works consider the problem from the SP's point of view. In \cite{zhang-tcom2018}, the authors consider a hybrid reservation/spot market where the SP reservations are decided by solving a stochastic problem, and a similar mixed time scale reservation model was studied in \cite{vincent-TVT2020}. Our previous work \cite{jb-icc20} employed demand and price predictions (via neural networks) to assist the SP's reservations, while \cite{tony-tnsm19} focused on slice reconfiguration costs. 

%\footnote{In cloud computing there are both reservation-based solutions and spot markets, e.g., see \cite{google-pricing} and \cite{google-spot}.}

%Similar reservation problems have been considered in the context of cloud computing, cf. \cite{carlee-sigcomm15}, using various reservation criteria, e.g., costs or task deadlines, \cite{carlee-infocom18}. However, all above reservation solutions make (often strong) assumptions about knowing the user demands and resource  prices, or assume that these parameters follow stationary stochastic processes. Our approach is fundamentally different, as we design reservation policies that do not require the SPs to know in advance their needs and the charged prices, nor we make any assumptions about the evolution of these parameters. In essence, we treat slice reservation as a learning problem for the SP, and rely on the theory of online convex optimization (OCO) \cite{hazan-book} to design algorithms that adapt the reservations to users' needs and the NO's (potentially arbitrary) pricing decisions. Our approach adapts recent  \emph{constrained} OCO algorithms, cf. \cite{giannakis-TSP17}, \cite{victor}, which are particularly robust and practical. 


% contributions
%\textbf{Contributions}. In detail, we consider a hybrid market with advance-reservation and spot-bidding options, where an SP can request resources from a network operator in the beginning of each period and update its reservation at each slot within every period. The NO is allowed to change arbitrarily both its reservation and spot prices, where the latter are made available to the SP only after the bidding is decided. Hence, in effect, the SP has to reserve slices without knowing the needs of its users or the total cost of its reservation. In this dynamic environment, the SP aims to maximize a general utility function of the slice resources, which reflects its service performance, while not violating an average budget constraint. 

%We formulate this process as a learning problem which allows the SP to implement a \emph{no-regret slice reservation policy}. This means that, as time evolves, the SP is guaranteed to achieve the same performance with that of an ideal benchmark policy that one could design only with hindsight, i.e., having access to all future demand and cost values. Our algorithm relies on a primal-dual online iteration \cite{giannakis-TSP17} which minimizes a Lagrangian function, and controls concurrently for  performance and budget costs. We extend our framework to allow the SP determine the \emph{composition} of its slices (as opposed only to its size), without even knowing what resource combination is performance-optimal. This is crucial when the qualitative features of user demand are volatile and/or unknown. And finally, we propose a mixed time-scale learning policy that exploits any price information that is revealed by the NO during each period, so as to improve the SP reservations.

%Our contributions can be thus summarized as follows:

%$\bullet$ We consider a general model, where the SP reserves slices in two time scales; determines the slice size and composition; and is oblivious to the user needs and NO prices.
	
%$\bullet$ We design an online learning framework for slice reservations, that ensures sample-path asymptotically-optimal performance while respecting the SP budget constraints.

 	
%$\bullet$ We perform a battery of numerical tests using stationary and non-stationary parameter patterns. The results verify the robustness and efficacy of our learning-based algorithms.

%paper organization
%The reminder of the paper is organized as follows. In section II, we review the related works. In section III, we present the system model and our formulation of the problem. We detail in section IV the online learning solution (OLR) and its performance analysis. We propose in section V and VI two extensions, namely the slice orchestration (OLR-SO) and the mixed time scale reservation models (OLR-MTS and OLR-SO-MTS). In section VII, we verify the guarantees of the OLR and OLR-SO solutions and show empirically the improvement brought by the mixed time scale models (OLR-MTS and OLR-SO-MTS) through our simulations. Lastly, we conclude our work in section VIII.
 

%notations
%\textbf{Notation}. We use bold typeface for vectors, $\bm a$, and vector transpose is denoted $\bm a^\top$. A sequence of vectors is denoted with braces, e.g., $\{\bm a_t\}$, and we use sub/superscripts to define a sequence of certain length, e.g., $\{\bm a_t\}_{t=1}^T$ is the sequence $\bm a_1, \bm a_2, \ldots, \bm a_T$. Sets are denoted with calligraphic capital letters, e.g., $\mathcal M$. The projection onto the non-negative orthant is denoted $[\cdot]_+$, and $\|\cdot\|$ is the $\ell_2$ norm.


%\textbf{Paper Organization}. The rest of this paper is organized as follows.  In Sec. \ref{sec:model} we introduce the system model and define the problem under consideration. Sec. \ref{sec:algo} presents the online reservation policy and proves its properties, and Sec. \ref{sec:extensions} discusses several generalizations of the model and problem. Finally, Sec. \ref{sec:sims} presents detailed numerical evaluations for a wealth of scenarios and we conclude in Sec. \ref{sec:conclusion}


%Literature review
%\section{Related Work}


%\subsection{Adaptive Online Learning}

%In \cite{mcmahan}, \cite{mohri},%, the authors raised a comprehensive survey of the online optimization methods using adaptive regularizers based on data. In the context of OCO, the survey takes a census of the regret bounds of the FTRL, primal-dual, mirror descent families of algorithm.


%paper using prediction
By anticipating the resource utilization, the NOs can enhance their resource management decisions regarding resource provisioning or allocation.
In \cite{DeepCog}, network traffic information is leveraged to plan the capacity needed for each slice in a multi-tenant framework. Using a data-driven approach including C-RAN, MEC and core networks, the solution outperforms other state-of-the-art deep learning solutions \cite{infocom17}, \cite{mobihoc18}. The approach in \cite{oliveiraTNSM} employed an adaptive forecasting model of the elastic demand for network resources to perform slice allocation in Internet Access Services. The authors in \cite{XaviINFOCOM17} and \cite{XaviTrans19} predicted the required resources by tenants for the future time window to perform slice requests admission and schedule the users' traffic within each slice. %The prediction module can adapt its predictions by changing a forecasting error probability according to the feedback sent by the slice scheduler. 
In \cite{cui-iccc20} cellular traffic prediction helps the allocation policy for the vehicular network slice.
Unlike these approaches, our solution does not need offline training and provides performance guarantees against all types of traces.

Fewer works consider the SP resource provisioning problem.
\cite{monteil-icc20} uses historical traffic to design the SP resource reservation policy. The paper \cite{reyhanian} developed a two-time scale approach for the activation and the re-configuration of the slices while considering the reservation of both RAN and backhaul resources. \cite{zhang-tcom2018} focuses on wireless spectrum considering two reservation schemes (in advance and on demand). In \cite{vincent-TVT2020}, the authors develop a two-stage approach for the resource reservation and the intra-slice resource allocation. These works presume a stationary environment where user statistics do not change and/or cost of resources are supposed constant. %The paper \cite{zhang2017} considers as well a mixed-time scale reservation model of long-term and short-term VM instances then used for VNF deployment. The authors propose three solutions for the VNF demand prediction, the short-term and long-term reservations of VMs. In \cite{paschos-infocom19}, the authors consider a reservation model, where a customer at each slot requests different types of cloud resources aiming to minimize a fixed and known cost function while satisfying a time-average unknown demand constraint. This model, however, is not suitable for the considered hybrid markets where the prices are volatile and hence the cost functions change dynamically.


%\subsection{Reservation-based network slicing}
%important reservation-based network slicing papers (not necessary)
%\cite{paschos-infocom18} studied the problem of allocating network and computing resources to a set of slices in order to maximize a system-wide utility function, namely to enforce fair resource allocation across slices.
%In \cite{malandrino-infocom20} the authors proposed a static optimization framework for embedding VNF chains (interpreted as slices) in a shared network. Their key contribution is the formulated problem which accounts for reliability, delay and other slice requirements. Albeit detailed and rigorous, this analysis considers the various system parameters and requests to be known. Similarly, \cite{andres-conext18} studied the impact of slice overbooking; \cite{xavi-infocom20} employed predictive capacity allocation for improving the slice composition; and \cite{nguyen-jsac19} focused on dynamic slicing via reinforcement learning.

%In \cite{nguyen-icc19}, the authors model the slice requests as a SMDP, with the network provider deciding whether to admit or not the new slice request. The solution consists of a deep Q-learning algorithm. The authors consider $3$ classes of slice in their simulations and Poisson arrival rates for the requests of such slices.
%The paper \cite{XaviInfocom19} falls under the scope of the Slice-as-a-Service framework to support on-demand slice requests that a set of tenants issue to one MNO (Mobile Network Operator).
%The MNO steers the slice requests towards a multi-queue system, where each queue stacks the bids for one specific slice type. The MNO can prioritize one queue based on its preferences.
%The model offers the possibility to choose one type of slice request over another, based on the MNO preferences.
 
%In \cite{zhang-tcom2018} and \cite{vincent-TVT20200}, the authors study a hybrid reservation and spot market where the SP reservations are decided by solving a stochastic problem.

%The authors then solve a stochastic convex problem to decide the SP reservations.







\begin{comment}
\subsection{Resource Allocation Policy}

The paper\cite{paschos-infocom18} studied the problem of allocating network and cloud computing resources in a set of slices in order to maximize a system utility function (achieving fair allocation across slices). Namely, there is a network $G=(N,L)$ and a set of slices $\mathcal S={1,2,\ldots, S}$, with $x_{sp}\geq0$ denoting the allocated rate to slice $s$ over path $p$, and $y_{sn}\geq 0$ is the amount of data processed for slice $s$ at in-network node $n$ (can be in the path or at the cloud). Assuming a linear relation between the traffic load and the computation load \cite{lee-ietf2016}, the resource allocation decisions can be succinctly given by variables $z_{snp}\geq 0$, which is the amount of demand from slice $s$ that is routed over path $p$ and processed at node $n$. The authors propose a static optimization problem where the objective is some type of convex function (a-fair functions) of the allocated rates to the different slices, and then they use consistency-pricing (or ADMM) which allows the different entities: slice requesters, network operator, cloud provider, to agree on the system operation.

\cite{fossatiACM} proposes a new multi-resource allocation framework that catches the inter-dependency of different resources (e.g. computing resource to traffic bit-rate dependency) to avoid wasting resources in a real-world setting, where some resources are congested (highly demanded) while others are not.
When compared to existing solutions that consider single-resource allocation without catching the dependencies between resources, the proposed solution shows huge savings of resources.

\cite{XaviACM19}




\subsection{Slice Admission policy}

The paper \cite{XaviInfocom19} falls under the scope of the Slice-as-a-Service framework to support on-demand slice requests that a set of tenants issue to one MNO (Mobile Network Operator).
The MNO steers the slice requests towards a multi-queue system, where each queue stacks the bids for one specific slice type.
The model offers the possibility to choose one type of slice request over another, based on the MNO preferences.
\cite{xavi-trans20} is the extension of \cite{XaviInfocom19}.




The authors in \cite{XaviTrans19} build their admission control solution upon a forecasting module of the required resources by tenants for the future time window.  %Traffic forecasting relies on the following assumptions: the demand is seasonal, hence the Holt-Winter forecasting model is relevant; the authors use the SLAW model to describe the user mobility; to design the tenant spatial distribution, they rely on the probabilistic latent variable model. 
The slice controller uses the traffic information to decide whether to admit new slice requests by solving a geometrical knapsack problem. After a thorough complexity analysis, the authors find the problem NP-hard and provide a heuristic solution with a satisfying performance ratio.
The slice scheduler minimizes the consumed resources within the slice while complying with SLAs requirements. The forecasting module can adapt its predictions by changing the forecasting error probability, according to the feedback sent by the slice scheduler.

In \cite{andres-conext18}, knapsack admission control.

In \cite{sciancalepore2018onets}, the tenants issue one slice request with the amount of resources and the time duration specified. The Network Slice Broker decides whether to admit new requests, ensuring that SLAs are being met. The authors formulate a Multi-Armed Bandit problem, taking account of the limited resource budget and the lock-up periods of tenants. Then, they design three online solutions, with two regret upper bounds, and finally give a proof-of-concept. The proof of regret for the ONETS algorithm relies on the assumption of i.i.d. exponential inter-arrival time of the requests.


In \cite{Raza19}, the InP can accept two kinds of slice, with strict latency constraints or non-strict latency constraints. The former has a higher revenue/penalty than the latter, which means it can lead to higher profit if the slice is setup properly, or to higher penalty, if the slice cannot be accommodated due to a lack of resources, at the radio, transport or cloud controller. The authors design a reinforcement learning solution, where an ANN model is trained to minimize the loss due to slice rejection and the loss caused by service degradation (when available resources are not sufficient to accommodate the slice). The solution shows good results when compared to baselines present in the literature, and the authors conduct sensitivity analysis, against the slice degradation penalty versus slice revenue factor, and the proportion of slice type.

In \cite{nguyen-icc19}, the authors model the arriving slice requests as a SMDP, with the network provider deciding whether to admit or not the new slice request. The solution consists of a deep Q-learning algorithm. The authors consider $3$ classes of slice in their simulations and Poisson arrival rates for the requests of such slices.

Most papers we came across deal with a finite number of slice types/classes, leading to Integer Linear Programming problems. In our case, we assume the SP can request any amount of resources that belongs to the NO capacity. Therefore, there exists an infinity of slice request types, as there exists an infinity of real values on $[0,\Delta]$ and so on. A multi-queue system for admission control cannot work unless we move backward and change the resource reservation system of the SP, limiting its requests to a discrete and finite set of values. Therefore, we prefer the Knapsack formulation.






\subsection{Reservation of virtualized resources}


The paper \cite{reyhanian} uses a two-time scale approach for the activation and the re-configuration of the slices while considering the reservation of both RAN and backhaul resources.

In \cite{DeepCog}, the authors leverage network traffic information to plan the capacity needed by each slice. Using a data-driven approach including C-RAN, MEC and core networks, their solution outperforms other state-of-the-art deep learning solutions \cite{infocom17}, \cite{mobihoc18}. 

The paper \cite{paschos-infocom18} studied the problem of allocating network and computing resources to a set of slices in order to maximize a system-wide utility function, namely to enforce fair resource allocation across slices.

In \cite{malandrino-infocom20} the authors proposed a static optimization framework for embedding VNF chains (interpreted as slices) in a shared network. Their key contribution is the formulated problem which accounts for reliability, delay and other slice requirements. Albeit detailed and rigorous, this analysis considers the various system parameters and requests to be known. Similarly, \cite{andres-conext18} studied the impact of slice overbooking; \cite{xavi-infocom20} employed predictive capacity allocation for improving the slice composition; and \cite{nguyen-jsac19} focused on dynamic slicing via reinforcement learning. Unlike these works, we make no assumptions regarding the availability or the statistical properties of the prices and user needs.



Fewer works consider the problem from the SP point of view. In \cite{zhang-tcom2018} and \cite{vincent-TVT2020}, the authors study a hybrid reservation and spot market where the SP reservations are decided by solving a stochastic problem. This, however, presumes a stationary environment, an assumption that is likely to fail when multiple SPs bid strategically and the NO adapts the prices accordingly. Our previous work \cite{jb-icc20} employed demand and price predictions (via neural networks) to assist the SP reservations; while \cite{tony-tnsm19} focused on slice reconfiguration costs. In \cite{paschos-infocom19}, the authors consider a reservation model, where a customer at each slot requests different types of cloud resources aiming to minimize a fixed and known cost function while satisfying a time-average unknown demand constraint. This model, however, is not suitable for the considered hybrid markets where the prices are volatile and hence the cost functions change dynamically.

\cite{DeepCog} uses deep learning to forecast capacity in a sliced network. The AI solution is a mix of 3-D CNN and MLP architectures, takes as input traffic snapshots at various base stations for one specific slice and outputs the desired capacity at the various data centers for the very same slice. The authors design a loss function able to leverage between SLA violations due to under-provisioning (under-estimation of the optimal capacity values at the data-centers) and unnecessary costs due to over-provisioning (over-estimation). Using a data-driven approach including C-RAN, MEC and core networks, the authors outperform other state-of-the-art deep learning solutions \cite{infocom17}, \cite{mobihoc18}. 

\subsection{Slicing Markets}

The design of markets for virtualized network resources is a relatively new research area. The survey in \cite{hossain-survey2018} provides an overview of auction theory-based slicing solutions and \cite{toktam-icc17} proposed mechanisms for the NO to auction its sliced resources. Unlike these works, we consider a dynamic pricing scheme that  is more practical as it does not require to run any type of auction. Importantly, our model is based on already-deployed and widely-used market models in cloud computing ecosystems, e.g., see \cite{google-spot, amazon-spot}.

Prior works that focus on such hybrid cloud market models have studied spot pricing models and devised intelligent bidding strategies for the buyers \cite{zafer-cloud2012, lumpe-ccgrid17, sharma-hotcloud2016, carlee-infocom18, carlee-sigcomm15 }. For instance, in \cite{carlee-infocom18} the users place bids to reserve cloud resources for executing certain long tasks, aiming to minimize their costs while ensuring task completion over successive bidding periods. The main idea  is to employ a hidden Markov model for tracking the evolution of spot prices; however, the analysis relies on the user needs complying to certain statistical assumptions. Similarly, in \cite{carlee-sigcomm15} an interesting bidding approach is considered where the users try to infer the pricing strategy of the cloud provider and bid accordingly in a spot market. Our work differs in that we make no assumption for the spot pricing model of the operator, and our reservation algorithm offers performance guarantees for any possible pricing scheme and demand pattern. This is crucial as in practice the operator might as well revise and adapt its pricing policy in the presence of strategic bidders.



\cite{Pla2021}, \cite{fisher}, 

\cite{Lieto19}

\subsection{Traffic slice scheduling}
\cite{XaviTrans19}, \cite{XaviINFOCOM17}



\subsection{Reservation of resources}

%\cite{paschos-infocom18} formulated slicing as a utility maximization problem;
The paper \cite{paschos-infocom18} studied the problem of allocating network and cloud computing resources in a set of slices in order to maximize a system utility function (achieving fair allocation across slices).
\cite{malandrino-infocom20} proposed a static optimization framework for embedding VNF chains (perceived as slices) in a shared network. The key contribution in this work is the formulated problem that accounts for reliability, delay, and other requirements of each slice. Albeit detailed and rigorous, this analysis has the key limitation, compared to our work, that it considers all system parameters and requests to be known.
%and  \cite{malandrino-infocom20} considered QoS metrics when serving the slices. 
Similarly, \cite{andres-conext18} studied the impact of slice overbooking; \cite{xavi-infocom20} employed predictive capacity allocation for improving the slice composition; and \cite{nguyen-jsac19} focused on dynamic slicing via reinforcement learning.

Fewer works consider the problem from the SP's point of view. In \cite{zhang-tcom2018}, the authors consider a hybrid reservation/spot market where the SP reservations are decided by solving a stochastic problem, and a similar mixed time scale reservation model was studied in \cite{vincent-TVT2020}. Our previous work \cite{jb-icc20} employed demand and price predictions (via neural networks) to assist the SP's reservations, while \cite{tony-tnsm19} focused on slice reconfiguration costs.
In \cite{paschos-infocom19}, the authors consider a reservation model, where a customer at each slot requests different types of cloud resources to minimize its total cost while satisfying the average constraint on the demand. The authors propose a primal-dual type of algorithm and their online policy is asymptotically feasible and achieves $o(T)$ regret against the $K$-slot static benchmark (static reservation that only satisfies the average constraint every $K$ slots), with $K=o(T^{1-\epsilon})$. Unlike our approach, the costs in the objective are known and fixed.
In \cite{zhang2017}, the authors design a reservation model of long-term and short-term VM instances, then used for VNF deployment, and create a complete online solution - composed of VNF demand prediction and reservation decisions - with performance guarantees.


\subsection{Slicing markets}



The design of slice markets is a relatively new research area. \cite{hossain-survey2018} provides an overview of auction theory and auction theory-based slicing solutions. In \cite{toktam-icc17}, the authors proposed a mechanism for the NO to auction its sliced resources. Unlike these works, we consider a dynamic pricing scheme, that does not require to organize and run any type of auction.%; hence it is more applicable, and follows similar solutions that have been successfully applied in cloud computing ecosystems. 

Prior works have studied spot pricing models, trying to infer the employed mechanism (e.g., by Amazon) and optimize accordingly the bidding strategy from the buyer's perspective, see \cite{carlee-sigcomm15, carlee-infocom18} and references therein. And there are various biding optimization criteria, e.g., considering costs, task dependency, or job deadlines; see overview in \cite{carlee-infocom18}. The work \cite{carlee-sigcomm15} considered only the spot market, while \cite{carlee-infocom18} analyzed a hybrid on-demand and spot market. Our work differs in that we make no assumption for the spot pricing model of the operator, and our reservation algorithm offers performance guarantees for any possible pricing scheme and resource availability pattern.



%Similar reservation problems have been considered in the context of cloud computing, cf. \cite{carlee-sigcomm15}, using various reservation criteria, e.g., costs or task deadlines, \cite{carlee-infocom18}, \cite{paschos-infocom19}.


%In \cite{carlee-infocom18}, users place bids to access cloud resources for job completion. The aim of a user is to select the resource instance at each slot which minimizes the expected payment needed to complete the job, leveraging future spot price predictions. The authors develop a hidden Markov model that views the spot price as a stochastic function of latent states, which depend on the arrival and departure rates of jobs in the bidding queue. Although the authors achieve dramatic cost reductions when compared to auto-regressive prediction models, their model assume arrival and departure rates are independent, time-invariant, exponential random variables.

%However, all above reservation solutions make (often strong) assumptions about knowing the user demands and resource prices, or assume that these parameters follow stationary stochastic processes. Our approach is fundamentally different, as we design reservation policies that do not require the SPs to know in advance their needs and the charged prices, nor we make any assumptions about the evolution of these parameters.

\subsection{Online optimization}

% ------------ Introduction content
%OCO is an efficient mathematical framework introduced by Zinkevich \cite{zinkevich}, to evaluate the performance -through the regret metric- of online optimization algorithms \cite{koppel, cao2018online, giannakis-TSP17}, i.e. algorithms that solve an optimization problem in an online manner, at each time $t \in \mathcal{T}=\{1,\ldots,T\}$, assuming we have a time-slotted system. 
%In online learning and more specifically in OCO problems \cite{shalev-book, hazan-book}, an agent has to take sequential decisions at each time $t$, that minimizes the objective, while satisfying the constraints (if the problem is constrained). The agent does not have access to the time-varying objective function $f_t(.)$ prior to the decision (nor the constraints). Thus, the difficulty arises in this situation as the agent must take a decision under uncertainty. The design of an efficient algorithm, able to adapt to the dynamics of the problem, becomes essential. The regret metric reflects the efficiency of the algorithm: it measures the difference of the cumulative loss incurred by our decisions -over $T$ time slots- from the cumulative loss that experiences the benchmark, which has full hindsight of the problem. A sublinear regret, i.e. $R_T=o(T)$, is regarded as a good performance, as it implies that the online algorithm asymptotically performs no worse than the benchmark. In constrained OCO problems, we deem both the regret metric and the constraint violations metric -or fit- which is the accumulation of violations through the duration $T$. We observe a natural trade-off between the two metrics and our aim is to design algorithms that keep both a sublinear regret and a sublinear fit.
% -----------------------------------

Part of our contributions is to design an online algorithm that achieves \emph{zero-regret} policy. Thus, we propose a summary of papers in the field of constrained OCO with similar guarantees on the regret and fit metrics \cite{mahdavi-jmlr2012, jennaton, koppel, giannakis-TSP17, cao2018online, liakopoulos2019cautious, victor, johansson}.

Mahdavi \emph{et al.} \cite{mahdavi-jmlr2012} solve the problem of finding an online policy which satisfies the long-term constraints $\forall i, \sum_t g_i(x_t)\leq0$ and compare it to the static benchmark sequence. Their proposed primal-dual method achieves a $\mathcal{O}(T^{1/2})$ static regret and $\mathcal{O}(T^{3/4})$ violation for each constraint $i$. Jennaton \emph{et al.} \cite{jennaton} essentially solve the same problem with more elaborate regularizer and steps and improve the results of \cite{mahdavi-jmlr2012}. The authors in \cite{koppel} design a distributed saddle-point algorithm -with primal and dual decisions- that achieves sublinear regret ($\mathcal{O}(\sqrt{T})$), against the static benchmark. In \cite{liakopoulos2019cautious}, the authors consider both time-varying objective and constraints functions and employ a Lagrangian method to achieve sublinear regret and constraint violation against a family of static benchmarks. In \cite{victor}, the authors consider the same problem and with a mirror-based method achieve sublinear regret and constraint violation against tighter comparators.

All these papers include fixed or time-varying constraints and compare their online policies to the static benchmark sequence. In our work, we aim to compare our policies to the dynamic benchmark sequence. From the papers \cite{cao2018online, johansson} we observe that it is possible to achieve both sublinear dynamic regret and fit, given sublinear accumulated variations of the dynamic benchmark sequence.
In \cite{giannakis-TSP17}, both objective $f_t$ and constraint $g_t$ functions are time-varying and revealed after the primal decision $\bm x_t$. The proposed method, a modified saddle-point algorithm, achieves sublinear dynamic regret and fit, given sublinear accumulated variations of the dynamic benchmark sequence and of the constraints.
In \cite{cao2018online}, the authors adapt \cite{giannakis-TSP17} to the bandit problem and achieve better bounds. In \cite{johansson}, the authors derive guarantees on the dynamic and static regret, with diverse sets of assumptions (with or without strongly-convex objective, Slater's condition).

%All these algorithms rely on primal-dual optimization, as the dual decision can be taken after receiving the feedback of the objective function $f_t(.)$ and of the constraint function $g_t(.)$ (or after receiving the bandit feedback).
\end{comment}



% System Model and Problem Statement
\vspace{2mm}
\section{Model and Problem Statement} \label{sec:model}

%The goal is to prove that the SP has interest to request resources proactively, with the online learning policy that we propose, which relies on traffic demand and resource pricing information. We aim to show that our online optimization approach is smarter than heuristics or has lower complexity than smart policies being model- or parameter-dependent solutions, or data-driven solutions.

%We consider thus a system model with multiple tenants requesting resources to one NO. The tenants belong to the set $\mathcal{I}$, with cardinality $|\mathcal{I}|$ ($=10$ for instance). We consider in the first scenario where only one tenant $i$ opts for our approach, and we implement other approaches for the other tenants. We evaluate the benefits drawn for tenant $i$. Then, we consider a hybrid situation where half tenants are using our approach. We evaluate the benefits for those tenants and for the NO. Finally, we look at the case where all the tenants use the OLR approach, and how it can benefit the NO.

%\begin{figure}
%    \centering
%    \includegraphics[width=0.49\textwidth]{images/journalTNSMfig.pdf}
%    \caption{A Network Operator (NO) leases different types of resources, e.g., wireless capacity, storage capacity and edge computing capacity, to different types of Service Providers (SPs) that offer over-the-top services to their users.}
%    \label{fig:model}
%\end{figure}


%multi-resource allocation framework
\textbf{Network and Market Model}. The key parameters of our model and solution are summarized in table \ref{table:notation} below.
We consider a slotted system $\{1,\ldots,T\}$.
A Network Operator (NO) sells virtualized resources to the service provider (SP), and we denote with $\mathcal{H}$ the set of $m=|\mathcal{H}|$ types of resources that comprise each slice. For instance, $\mathcal{H}$ may include bandwidth capacity, backhaul link capacity, edge computing and storage resources ($m=4$). The SP can reserve multiple kinds of resources which orchestration will enable the operation of the slice. We denote the in-advance reservation and spot reservation decisions at slot $t$ respectively as $\bm x_t = [x_1, \ldots,x_m]_t^\top$ and $\bm y_t = [y_1, \ldots,y_m]_t^\top$.
The optimal mix of resources composing the slice is unknown to the SP, as it depends on the type of request the SP receives from its users. Moreover, the benefit from each resource can be time-varying, e.g. bandwidth capacity can change due to varying channel conditions. %The contributions of the different resources to the utility of the slice are unknown and might even vary over time. 
The benefit from reservation $\bm x_t$ ($\bm y_t$) is quantified by the scalar $\bm x_t^\top \bm \theta_t$ ($\bm y_t^\top \bm \theta_t$), where the items of $\bm \theta_t \in \mathbb R^m$ are the individual contributions of each resource on the performance at slot $t$.

%We represent such variations by multiplicative factors $\theta_i$ which weigh the reservation values $x_i$. Therefore, at slot $t$, the actual reservation of the SP is $\bm x_t^\top\bm \theta_t$, where $\bm \theta_t = [\theta_1,\ldots,\theta_m]_t^\top$.


%The SP can request a certain amount for each type of resource, which are then composing the end-to-end network slice dedicated for the users of the service. The SP must decide the reservation vector $\bm x = [x_1, \ldots,x_m]^\top$. Based on its reservation $\bm x$, the slice obtained by the SP will have a given capacity, which we assume to be a linear combination of the leased resources which contributions to the capacity are transcribed in the vector $\bm \theta = [\theta_1,\ldots,\theta_m]^\top$. Therefore, the SP obtains a slice of capacity $\bm x ^\top \bm \theta$. This assumption is general and can be found in related works such as \cite{fossatiACM} which considers a multi-resource allocation framework. Moreover, we make no assumptions about the contributions of the resources which we deem unknown and changing from slot to slot, potentially in a non-stationary manner. Therefore, at slot $t$ the slice capacity is equal to $\bm x_t ^\top \bm \theta_t$, where $\bm x_t$ is the reservation vector for slot $t$, and $\bm \theta_t$ is the contribution vector.


%slice utility ~ service performance function
The utility stemming from such reservation scheme is non-linear. We model the slice utility of the SP as an increasing concave function of the acquired resources by using the logarithm function. For instance, the paper \cite{paschos-infocom18} provides the general form of $\alpha$-fair utility functions:
\begin{align}
    f(\bm z) = \left\{
    \begin{array}{ll}
        \frac{\bm z^{1-\alpha}}{1-\alpha} & \alpha \neq 1 \\
        \log(\bm z) & \alpha = 1
    \end{array}
\right.
\end{align}
where $\bm z$ is the reservation vector of slices. In \cite{srikant}, the utility from allocating bandwidth $x$ to a certain network flow $f$ is modeled as $a_f\log(x_f)$, where $a_f$ is a problem (and flow)-specific parameter. %Other reservation-based papers relate to the same kind of convex/concave objective function to minimize/maximize \cite{malandrino-infocom20, zhang-tcom2018, paschos-infocom19}. 
The logarithm function allows us to model as well the diminishing returns which naturally arise with the over-reservation of the network resources. For instance, the data rate is a logarithmic function of the spectrum; the additional revenue of the SP from more slice resources is typically diminishing. We model the slice utility function as a logarithmic concave function, weighted by the SP demand $a_t$, i.e. $a_t\log(1+\bm \theta_t^\top(\bm x_t + \bm y_t))$. %The SP demand $a_t$ is unknown at the beginning of slot $t$ and only revealed at the end of the slot. %cite

%network resources prices, market operation
The market operates in a hybrid model. At the beginning of each slot, the SP can lease network resources, plus additional resources on a spot market. We denote with $\bm p_t = [p_1,\ldots,p_m]^\top_t \in \mathbb R_+^m$ the unit price of the network resources; and we denote with $\bm q_t=[q_1,\ldots,q_m]^\top_t \in \mathbb R_+^m$ the unit price of the resources available in the spot market. The SP reservation policy consists of the reservation decision $\bm x_t$ and the spot decision $\bm y_t$. At the beginning of each slot $t$, the SP decides its $t$-slot reservation plan $(\bm x_t, \bm y_t)$, and pays the price $\bm p_t^\top \bm x_t + \bm q_t^\top \bm y_t$ at the end of the slot.%At the time the SP makes a bid, both reservation and spot unit prices are unknown and are only revealed \emph{after} the SP makes its decision. Moreover, the prices dynamically change following non-stationary patterns, which force the SP to use a reservation decision model robust to the price uncertainty. As prices are revealed along time to the SP, the latter acquires historical traces which render possible the prediction of the next slot values.% We will detail how to use this information within our decision algorithm.

%limits of capacity for the operator
The NO can impose upper limits on the requests of the SP. For instance, the reservation request for resource $i$ must belong to the set $\Gamma_i = [0,D_i]$, where $D_i$ is the limit imposed by the NO on resource $i$. Therefore, the SP request will belong to $\Gamma_1 \times\ldots\times \Gamma_m$, which we denote $\Delta$. Such limitations arise from natural capacity constraints of the network, in charge of multiple services and its own needs. In some cases, the NO can be unable to fulfill the SP request, especially when the network is congested due to high users' demand load and heavy SPs requests. The NO must guarantee a certain Service Level Agreement (SLA), which we relate to the respect of a certain threshold ratio of the requested amount resource. For instance, the NO must deliver at least $\alpha=80\%$ of the desired capacity for the resource. We envision this scenario as an extension and assume from now that the NO must comply with the whole request if it belongs to $\Gamma_i$.

%We envision this scenario in the supplementary file \cite{JB}, with a slight change to the main problem statement.
%possibility of failure to fulfill the SP request -> new scenario
%At this point, we envision two very distinct cases. First, as long as the request falls in the constraint set, the NO will fulfill. Secondly, depending on the other SPs' requests, the NO will fairly provide each SP with a certain percentage of its request. The fairness rule used by the NO is out of the scope of this paper. 

%SP reservation policy, strategy
%We now elucidate the SP reservation policy, which consists of the $t$-slot reservation decision $\bm x_t$ and the spot decision $\bm y_t$. At the beginning of each slot $t$, the SP decides its $t$-slot reservation plan $(\bm x_t, \bm y_t)$, and pays the price $\bm p_t^\top \bm x_t + \bm q_t^\top \bm y_t$ at the end of the slot. The SP's goal is to maximize the performance of its service, while avoiding excessive monetary costs. 

%The service performance is quantified with a concave \emph{utility} function, increasing on the resources and modulated by parameter $a_t$, that captures the aggregated demand in slot $t$. Note that the demand is unknown to the SP, and is revealed before the decision of the next slot $(\bm x_{t+1}, \bm y_{t+1})$.



\textbf{Problem statement}.
Putting the above together, the ideal reservation slice policy is the solution of the following convex program:

\begin{align}
(\mathbb P):\quad \max_{ \{\bm x_t,\{\bm y_t\}\}_{t=1}^T } & \sum_{t=1}^T \Big(V  a_t \log((\bm x_t + \bm y_t)^\top \bm \theta_t + 1) \notag\\
&- (\bm p_t^\top \bm x_t +\bm q_t^\top \bm y_t) \Big) \label{prob-obj} \\
%\text{s.t.} \quad & \sum_{t=1}^T\Big(x_tp_t+\!\!\!\!\sum_{k=(t-1)K+1}^{tK}\!\!\!\!\!y_kq_k \Big) \leq B, \label{prob-const1} \\
\text{s.t.}\quad   \bm y_t\in &\Delta, \quad \forall t=1,\ldots,T,  \label{prob-const2} \\
	 \bm x_t \in &\Delta, \quad \forall t=1, \ldots, T. \label{prob-const3}
\end{align}

In Objective (\ref{prob-obj}), we recognize the weighted sum of the slice performance (logarithmic term) and the payments (linear term). The latter term has a minus sign as the SP seeks to minimize its monetary cost. We sum over the number of slots $T$, as the goal is to maximize this weighted sum in the long-term.
Constraints (\ref{prob-const2}) and (\ref{prob-const3}) ensure the decisions belong to the constraint convex set $\Delta$. We define the hyper-parameter $V\geq1$ which balance the influence between the two terms (utility term and cost term). The bigger $V$, the more we favor the slice utility in the detriment of the cost of reservation.

%this part for the subsection
%If we envision the case where the SP only obtains part of its request, then we replace $\bm x_t$ by $A_t \bm x_t$ and $\bm y_t$ by $B_t \bm y_t$, where $A_t$ and $B_t$ are $m\times m$ diagonal matrices. If the SP obtains all its request, then $A_t = B_t = 1\!\!1_m$ (identity matrix).

$(\mathbb P)$ is a convex optimization problem but cannot be tackled directly due to the following challenges:
\begin{itemize}
    \item the users' demand $\{a_t\}$ is unknown, time-varying and non-stationary;
    \item the unit prices $\{\bm q_t\}$ and $\{\bm p_t\}$, are unknown, time-varying and non-stationary;
\end{itemize}

Due to these challenges, the convex problem $(\mathbb P)$ cannot be solved at $t=1$ for the next $T$ slots. Henceforth we define the loss function, at each slot $t$:
\begin{align}
    f_t(\bm x_t, \bm y_t) = - V a_t \log((\bm x_t + \bm y_t)^\top \bm \theta_t + 1)\\
    + (\bm p_t^\top \bm x_t +\bm q_t^\top \bm y_t)
\end{align}
The function $f_t$ is convex which allows us to use the OCO framework. Our goal is to decide at each slot $t$ the reservation plan $\bm z_t = (\bm x_t, \bm y_t)$ and achieve in the long term a sublinear static regret as defined in \eqref{eq:static_regret}.

\begin{table}
\caption{Key parameters and variables}
\scriptsize
	\centering%
	\begin{tabular}{|c|c|}
		\hline %
		\hline
		Symbol & Physical Meaning\\
            \hline %
		$m$ & Number of network resources composing a slice\\
            \hline
		$\bm x_t$ & Reservation in advance market in slot $t$ \\
            \hline
            $\bm y_t$ & Reservation in spot market in slot $t$ \\
		\hline %
            $\bm \theta_t$ & Contribution vector in slot $t$ \\
            \hline
		$a_t$ & User needs for the SP service in slot $t$\\
		\hline
		$\bm p_t$ & Unit price vector of the network resources at $t$ \\
		\hline
		$\bm q_t$ & Spot price vector for slot $t$\\
		\hline
		$T$ & Number of slots/horizon\\
		\hline
		  $D_i$ & Upper-bound imposed by the NO for reservation of resource $i$ \\
		\hline
		$\Gamma_i$ & $\Gamma=[0,D_i]$, feasible set for reservation of resource $i$\\
		\hline %
		$\Delta$ & Compact convex set $\Gamma_1\times\ldots\times\Gamma_m$ \\
            \hline
            $D$ & Diameter of $\Delta$ \\
		\hline
		$V$ & Calibration parameter \\
            \hline
            $\sigma$ & Regularization parameter, best choice  $\sigma=\sqrt{2}/D$ \\
            \hline
            $\grad \hat{f}_{t+1}(\hat{\boldsymbol{z}}_{t+1})$ & Gradient prediction known at $t$ \\
            \hline
            $\zeta$ & Prediction model average relative error rate \\
            \hline
            $\alpha$ & Minimum ratio the NO must provide for advance resources \\
            \hline
            $\beta$ & Minimum ratio the NO must provide for spot resources \\
%		\hline
%		$\Gamma_1, \ldots ,\Gamma_i, \ldots ,\Gamma_m$ & $\forall i \in [1,m], \Gamma_i=[0,D_i]$, feasible set for reserved instance $i$\\
		\hline %
		%$??$ & \gi{other variables that we would like to put here?}\\
		%\hline
		\hline
	\end{tabular}
	\label{table:notation} % is used to refer this table in the text
\end{table}

%Our goal is to design the online reservation policy $\{\bm z_t\}_{t=1}^T$, where we denote $\bm z_t = (\bm x_t, \bm y_t)\quad \forall t$, able to give sublinear regret guarantees against an optimal policy. We define the latter in the next subsection.



%\subsection{Estimation step}
%The OOLR algorithm requires estimating the next gradient function and the next point.
%Thus, we propose two methods to estimate the next point $\hat{\bm z}_{t+1}$: either we take $\hat{\bm z}_{t+1} = \bm z_t$, i.e. the previous reservation decision; or we use another optimization problem.
%The former case is relevant if the optimal solution to \eqref{eq:stepOOLR} stays relatively similar from one slot to the other. As $\hat{\bm z_t} = \bm z_{t-1}$ and $\bm z_{t-1} \sim \bm z_{t}$, the quadratic error on the gradient as defined in \eqref{eq:quadratic_error} will be relatively low.

%The latter case implies to make a decision for the estimation of the next point. We aim to solve a similar problem to the one defined in \eqref{eq:stepOOLR}, knowing that we cannot use the predicted gradient term, as it is precisely what we are deciding by estimating the next point. Therefore, we formulate the estimation step of our algorithm as:

%\begin{align}
 %   \hat{\bm z}_{t+1} = \arg \min_{\bm z \in \Delta^2} \Big\{ r_{1:t}(\bm z) +
  %  \Big(\sum_{s=1}^t \grad f_s(\bm z_s)\Big)\top \bm z\Big\} \label{eq:stepEstim}
%\end{align}

%As one can read in \eqref{eq:stepEstim}, we only use the accumulated past information to estimate the next point value, before making the actual decision in \eqref{eq:stepOOLR}. This estimation step is the classical Follow-the-Regularized-Leader method which ensures a $\mathcal{O}(\sqrt{T})$ regret bound against the static benchmark. %In volatile settings, we expect with this method a lesser accumulated quadratic error of the predictions, thus enhancing the regret performance.



%We must predict $\tilde c_t$, hence $\tilde a_t$, $\tilde q_t = [\tilde q_t^1,\ldots,\tilde q_t^m]^\top$, $\tilde \theta_t = [\tilde \theta_t^1,\ldots,\tilde \theta_t^m]^\top$, which correspond to the demand, the spot price of the $m$ resources, the contribution of the $m$ resources.

%Let's consider first the SP's demand $a_t$. We assume this information is revealed to the SP after each decision, hence the SP has access to an history of past data of this variable. We can apply a learning algorithm that performs -in average- no worse than the best ARMA model in hindsight \cite{anava} (i.e. with full knowledge of the future demand $a_t$).
%We assume the demand of our SP is not correlated with the spot price, as there is a sufficiently large number of SPs leasing resources, therefore the impact of only one SP on the spot price is negligible.% However, the ARMA comparator is reliable only if the trace $a_t$ presents the stationary feature (statistics of the trace do not vary with time $t$). It is possible that the demand might be non-stationary, depending on underlying hidden factors that cause drastic changes to the demand. In this case, we need another prediction model for $a_t$ than the one in \cite{anava}.

%Now let's consider the contribution vector $\bm \theta_t$, and let assume it does not vary (or very little) over time: $\bm \theta_t \simeq \bm \theta$. This is a realistic assumption, as the resource contribution should be relatively similar from one slot to the other. After a sufficient number $T$ of observed output $(\bm x_t + \bm y_t)^\top \bm \theta$, we can find $\bm \theta$ such that it is close enough to the actual contribution vector. We can use either a one-layer perceptron, a linear regression, or an online gradient descent which would lead to a $\mathcal{O}(\sqrt{T})$ accumulated error.

%Again, the spot price depends on latent factors such as the other SPs' demand, the NO availability, which are hidden to our SP. In \cite{carlee-infocom18}, they assume the underlying dynamics follow a time-invariant exponential distribution, and the bids in the spot market are uniformly distributed. Using the EM algorithm, they are able to predict the spot prices and subsequently to find the ideal spot instance to query for.

\begin{comment}
\subsection{Static Benchmark}

The efficacy of our learning policy is mainly characterized by the benchmark to which we compare it (i.e. there exists different kind of benchmark, which are more or less stringent); and by the convergence rate of the learning policy loss relative to the chosen benchmark. 

In our problem, we opt for a static benchmark, that consists of a unique reservation vector $\bm z^*$, optimally chosen over the period of evaluation, to minimize the loss.

More precisely, the optimal solution $\bm z^*$ is defined as:
\begin{align}
    \bm z^* = \arg \min_{\{\bm x, \bm y \in \Delta\}} \sum_{t=1}^T f_t(\bm x, \bm y),
\end{align}
where the period of evaluation includes the slots $t=1,\ldots ,T$.

The benchmark has the pros to know \emph{a priori} all the future prices $\{\bm p_t\}_{t=1}^T$, $\{\bm q_t\}_{t=1}^T$, the future demand $\{a_t\}_{t=1}^T$ and the future contribution vectors $\{\bm \theta_t\}_{t=1}^T$. It has the cons to reserve a unique vector, which is ideal \emph{on average} (over the period of evaluation), but \emph{sub-optimal} when considering the slots separately. Therefore, our algorithm can possibly outperform the benchmark, synonym of a negative regret. 

Given that in practice, the information of the benchmark is unavailable, our goal is to design an algorithm that finds the reservation vector $\bm z_t = (\bm x_t, \bm y_t)$ at each slot $t$, such that the overall loss achieved (for $t=1,\ldots ,T$) is of the same order of the overall loss achieved by $\bm z^*$. Formally, we define the \emph{static regret} as:
\begin{align}
    R(T) = \sum_{t=1}^T \Big(f_t(\bm z_t) - f_t(\bm z^*)\Big),
\end{align}
and the goal is to have $R(T)/T \rightarrow 0$ when $T\rightarrow \infty$.
\end{comment}

\begin{comment}
\subsection{Calibration of $\alpha$}

The vector function $f_t$ is a weighted sum of two terms, the utility term $-a_t \log((\bm x + \bm y)^\top\theta_t +1)$, which is convex with respect to $\bm z = (\bm x, \bm y)$; and the price consumption term, which is affine hence convex with respect to $\bm z$. With parameter $\alpha$, we can adjust the relative influence of the two terms, depending on the SP preferences.
For instance, if the SP wishes to have both quantities of the same order:
\begin{align}
   \alpha a_t\log(1+(\bm x + \bm y)^\top \bm \theta_t) \sim (1-\alpha) \bm p_t^\top \bm x_t + \bm q_t^\top \bm y_t, \notag
\end{align}

then $\alpha=1/(1+r)$, with:
\begin{align}
    r = \frac{a_t\log(1+(\bm x + \bm y)^\top \bm \theta_t)}{ p_t^\top \bm x_t + \bm q_t^\top \bm y_t}. \notag
\end{align}

We remark that the ratio $r$ of the two quantities depends on the slot $t$. However, if we consider a sufficient number of slots and take the average ratio $r$, then the calibration of $\alpha$ will be optimal \emph{on average}. We can think of it as the law of large numbers.
If the SP has a factor of preference $s$ for the utility (for instance $s=2$), then:
\begin{align}
   \alpha a_t\log(1+(\bm x + \bm y)^\top \bm \theta_t) \sim s(1-\alpha) \bm p_t^\top \bm x_t + \bm q_t^\top \bm y_t, \notag
\end{align}
then $\alpha = 1/(1+r)$, with:
\begin{align}
    r = \frac{a_t\log(1+(\bm x + \bm y)^\top \bm \theta_t)}{ s \bm p_t^\top \bm x_t + \bm q_t^\top \bm y_t}. \notag
\end{align}

\end{comment}










% Solution Algorithm

\section{Optimistic Online Learning for Reservation} \label{sec:algo}

\subsection{Algorithm}

Our approach is inspired from the \emph{Follow-the-Regularized-Leader} (FTRL) policy, whereby the learner aims to minimize the loss on all past slots plus a regularization term:
\begin{flalign}
\forall t, \bm z_{t+1} = \arg\min_{\bm z \in \Delta^2} \sum_{i=1}^t f_i(\bm z) + R(\bm z)
\end{flalign}
Due to the convexity of $f_t$, the following property holds:
\begin{flalign}
f_t(\bm z_t) - f_t(\bm z^*) \leq \grad f_t(\bm z_t)^\top (\bm z_t - \bm z^*)
\end{flalign}
which means that the regret against the functions $\{f_t\}$ is upper-bounded by the regret against their linearized form $\bar f_t(\bm z) = \grad f_t(\bm z_t)^\top \bm z$ \cite{mcmahan}. Consequently, the FTRL algorithm simplifies to:
\begin{flalign} \label{eq:ftrl}
\forall t, \bm z_{t+1} = \arg\min_{\bm z \in \Delta^2} \sum_{i=1}^t \grad f_i(\bm z_i)^\top \bm z + R(\bm z)
\end{flalign}

In our approach, we consider an additional gradient term, which is the \emph{optimistic} next slot gradient prediction $\grad \hat f_{t+1}( \hat{\boldsymbol{z}}_{t+1})$. 
In the FTRL, the regularization function is quadratic $R(\bm z) = \frac{1}{2\eta}||\bm z||^2$. In contrast, we design a sequence of proximal regularizers:
\begin{align}
    \forall t=1\ldots T, \quad r_t(\bm z) = \frac{\sigma_t}{2}||\bm z -\bm z_t||^2, \label{eq:reg}
\end{align}
with $||.||$ the Euclidean norm. The regularizer parameters are:
\begin{align}
    \sigma_t &= \sigma\Big( \sqrt{h_{1:t}} - \sqrt{h_{1:t-1}}\Big), \label{eq:acc_error}\\
    h_t &= ||\grad f_t(\bm z_t) - \grad \hat f_t(\hat{\boldsymbol{z}_t})  ||^2, \label{eq:quadratic_error}
\end{align}
where $\sigma\geq 0$, and $h_{1:t}=\sum_{i=1}^t h_i$.

All the above lead to the final form of our algorithm decision step:
\begin{align}
    \bm z_{t+1} = \arg& \min_{\bm z \in \Delta^2} \Big\{ r_{1:t}(\bm z) + \notag\\
    &\Big(\sum_{i=1}^t \grad f_i(\bm z_i) + \grad \hat f_{t+1}(\hat{\boldsymbol{z}}_{t+1})\Big)\top \bm z\Big\} \label{eq:stepOOLR}
\end{align}

\begin{algorithm}[t]
\SetAlgoRefName{OOLR}
\caption{Optimistic Online Learning for Reservation}
\DontPrintSemicolon
\KwInitialize{ \; $\bm z_1 \in \Delta^2, \sigma = 1$, $a_1$, $\bm q_1$, $f_1(\bm z_1)$ } 
	%
	%
\For{ $t=1,\ldots, T-1$ } 
{
Observe the new prediction of the gradient $\grad \hat{f}_{t+1}(\hat{\boldsymbol{z}}_{t+1})$ \;
%
Decide $\bm z_{t+1}$ by solving \eqref{eq:stepOOLR} \;
%
Observe the demand $a_{t+1}$, the reservation price $\bm p_{t+1}$, the spot price $\bm q_{t+1}$, the contributions $\bm \theta_{t+1}$ \;
%
Calculate $f_{t+1}(\bm z_{t+1})$ and $\grad f_{t+1}(\bm z_{t+1})$\;
%
Update $r_{1:t+1}(\bm z)$ according to \eqref{eq:reg} and \eqref{eq:acc_error} \;
}
%
\end{algorithm}


\subsection{Performance analysis}
We start with the necessary assumptions.
\begin{assumption}
The sets $\Gamma_i$, $i=1\ldots m$, are convex and compact, and it holds $|x|\leq D_i$, for any $x \in \Gamma_i$\footnote{Note that we can rename the set $\Gamma_1\times\ldots\times\Gamma_m$ as $\Delta$ and simply assume $\Delta$ is a compact convex set with diameter $D$. The design of the different sets $\Gamma_i$ allows us to choose a different reservation restriction for each resource type.}.
\end{assumption}

\begin{assumption}
The function $f_t$ is convex.
\end{assumption}

\begin{assumption}
$\{r_t\}_{t=1}^T$ is a sequence of proximal non-negative functions.
\end{assumption}

\begin{assumption}
Prediction $\grad \hat{f}_{t+1}(\hat{\boldsymbol{z}}_{t+1})$ is known at $t$.
\end{assumption}

%\begin{assumption}
%The function $r_{1:t}(\bm z)$ is $1$-strongly convex with respect to some norm $||.||_{(t)}$.%, that we will define in the sequel.
%\end{assumption}
%For simplicity, we denote:
%$$\left\{ \begin{array}{ll}
%     \grad f_t(\bm z_t) &= c_t  \\
%     \grad \hat{f}_t(\hat{\boldsymbol{z}}_t) &= \tilde c_t 
%\end{array}
%\right.$$


\begin{corollary} \label{corollary-oolr}
Under Assumptions 1-4, we derive from \cite[Theorem~1]{mohri} and \cite[Theorem~1]{mhaisen} the following regret bound:
\begin{align}
    \boxed{
    R(T) \leq \sqrt{\sum_{t=1}^T ||\grad f_t(\bm z_t)-\grad \hat{f}_t(\hat{\boldsymbol{z}}_t)||^2}(\frac{2}{\sigma}+\frac{\sigma}{2}2D^2)} \label{eq:regret_bound}
    \end{align}
\end{corollary}




\begin{proof}
First let's remark that the function $h_{0:t}:\bm z \rightarrow r_{0:t}(\bm z) + (c_{1:t}+\tilde c_{t+1})^\top \bm z$ is $1$-strongly convex, with respect to the norm $||.||_{(t)}$. It allows us to use \cite[Theorem~1]{mohri}, which yields regret:
\begin{align}
    R(T) \leq r_{1:T}(\bm z^*) + \sum_{t=1}^T ||c_t-\tilde c_t||^2_{(t),*} \quad \forall \bm z^* \in \Delta^2 \label{lemma}
\end{align}


Now, we define the norm $||x||_{(t)} = \sqrt{\sigma_{1:t}}||x||$, which has dual norm $||x||_{(t),*} = ||x||/\sqrt{\sigma_{1:t}}$.
We remark that $\sigma_{1:t}=\sigma\sqrt{h_{1:t}}$, and starting from \eqref{lemma}, we get:
\begin{align}
    R(T) &\leq \frac{\sigma}{2}\sum_{t=1}^T (\sqrt{h_{1:t}}-\sqrt{h}_{1:t-1})||\bm z^* - \bm z_t||^2
    + \sum_{t=1}^T \frac{h_t}{\sigma\sqrt{h_{1:t}}} \\ \notag
     &\leq \frac{\sigma}{2}\sum_{t=1}^T (\sqrt{h_{1:t}}-\sqrt{h}_{1:t-1})2D^2
    + \sum_{t=1}^T \frac{h_t}{\sigma\sqrt{h_{1:t}}} \notag
\end{align}

We use the first order definition of convexity on the square root function to get:
\begin{align}
    \sqrt{h_{1:t}} - \sqrt{h_{1:t-1}} &\leq    \notag \frac{1}{2\sqrt{h_{1:t}}}(h_{1:t} - h_{1:t-1})\\
    &= \frac{h_t}{2\sqrt{h_{1:t}}} \notag
\end{align}
Thus,
\begin{align}
    R(T) \leq \frac{\sigma}{4}\sum_{t=1}^T \frac{h_t}{\sqrt{h_{1:t}}}2D^2 + \sum_{t=1}^T \frac{h_t}{\sigma\sqrt{h_{1:t}}} \label{intermed}
\end{align}

From \cite[Lemma~3.5]{cesa}, we have:

\begin{align}
    \sum_{t=1}^T \frac{h_t}{\sqrt{h_{1:t}}}  \leq 2\sqrt{h_{1:t}}
\end{align}

Plugging this result into \eqref{intermed}, it yields:

\begin{align}
    R(T) \leq \sqrt{h_{1:t}} (\frac{2}{\sigma} + \frac{\sigma}{2}2D^2)
\end{align}

%We finish the proof by finding the following bound:
%%\begin{align}
 %   \sum_{t=1}^T ||\bm z^* - \bm z_t||^2 \leq \sum_{t=1}^T 2(D_1^2+\ldots+D_m^2) = 2D^2 T \notag
%%\end{align}
\end{proof}


\emph{Remark 1.} We observe that a certain value of $\sigma$ can minimize the upper-bound on the regret, but one has to know the diameter of the decision set $\sqrt{2}D$. The very value of $\sigma$ which minimizes the upper-bound is:
\begin{align}
    \sigma = \frac{\sqrt{2}}{D} \label{eq:sig}
\end{align}

We re-write the upper bound:

\begin{align}
    \boxed{
    R(T) \leq 2\sqrt{2}D\sqrt{\sum_{t=1}^T ||\grad f_t(\bm z_t)-\grad \hat{f}_t(\hat{\boldsymbol{z}}_t)||^2}
   } \label{eq:regret_bound2}
\end{align}

\emph{Remark 2.} The regret bound is in $\mathcal{O}(\sqrt{T})$ if predictions are arbitrarily bad i.e. $\sum_{t=1}^T ||\grad f_t(\bm z_t)-\grad \hat{f}_t(\hat{\boldsymbol{z}}_t)||^2 = \mathcal{O}(T)$, and becomes \emph{null} when the predictions are perfect, i.e. when $\forall t, \quad  \grad \hat{f}_t(\hat{\boldsymbol{z}}_t) = \grad f_t(\boldsymbol{z}_t)$.

\emph{Remark 3.} We implement an online learning prediction method that learns how to predict the gradient with the regret $\mathcal{O}(2mGM\sqrt{T})$, where $G$ and $M$ are key constant in \cite{anava}. % The integration of this algorithm into our main OOLR solution leads to the regret of $\mathcal{O}(T^{1/4})$. 
Other prediction methods could be applied to the prediction of the gradient; however, this online learning method offers sublinear regret guarantees against all types of traces, even non-stationary.

\emph{Conclusion.} We conclude that our OOLR algorithm brings the best of both worlds. Given arbitrarily bad predictions, it provides the same guarantee of sublinear regret as the FTRL algorithm, i.e. $\mathcal{O}(\sqrt{T})$. Associated with an accurate prediction model, it provides tighter guarantee of performance down to $\mathcal{O}(1)$ in the ideal case, i.e. when predictions are perfect $\sum_{t=1}^T ||\grad f_t(\bm z_t)-\grad \hat{f}_t(\hat{\boldsymbol{z}}_t)||^2 = \mathcal{O}(1)$.

\subsection{Complexity analysis}
We stress there that the computational cost and memory requirements of the OOLR algorithm are fairly low. We need to solve at each slot $t$, the problem \eqref{eq:stepOOLR}. We add the term $r_t(\bm z)$ to the previous regularizer $r_{1:t-1}(\bm z)$. We can just replace $r_{1:t-1}(\bm z)$ by $r_{1:t}(\bm z)$ in the same variable to limit storage cost. 
The gradient terms $\grad f_i(\bm z_i)$ are equal to:
\begin{align}
    \grad f_i(\bm z_i) &= \begin{bmatrix}
          -V \frac{a_i \theta_{i,1}}{1+ (\bm x_i + \bm y_i)^\top \bm \theta_t  } + p_{t,1} \\
            -V \frac{a_i \theta_{i,2}}{1+ (\bm x_i + \bm y_i)^\top \bm \theta_i  } + p_{i,2} \\
           \vdots \\
           -V \frac{a_i \theta_{i,m}}{1+ (\bm x_i + \bm y_i)^\top \bm \theta_i  } + p_{i,m} \\
           -V \frac{a_i \theta_{i,1}}{1+ (\bm x_i + \bm y_i)^\top \bm \theta_i  } + q_{i,1} \\
           -V \frac{a_i \theta_{i,2}}{1+ (\bm x_i + \bm y_i)^\top \bm \theta_i  } + q_{i,2} \\
           \vdots \\
           -V \frac{a_i \theta_{i,m}}{1+ (\bm x_i + \bm y_i)^\top \bm \theta_i  } + q_{i,m}
         \end{bmatrix}. \label{eq:gradient}
 \end{align}
 Thus, at each slot $i$, we need to store the vectors $\bm p_i$, $\bm q_i$, $\bm \theta_i$ of length $m$ and the scalar $a_i$. Therefore, memory requirements are of $3m+1 = \mathcal{O}(m)$. We can just replace the gradient term $\sum_{i=1}^{t-1} \grad f_i(\bm z_i)$ by $\sum_{i=1}^{t} \grad f_i(\bm z_i)$ in the same variable to limit storage cost. 
The computation of the gradient \eqref{eq:gradient} runs in $2m(m+4)$ operations. The computation of the regularizer term \eqref{eq:reg} runs in $4m$ operations for $||\bm z - \bm z_t||^2$ and $4m+2$ operations for $\sigma_t$. Therefore the running time for \eqref{eq:stepOOLR} is in $2m(m+4)+8m+2=\mathcal{O}(m^2)$.

The complexity of the OOLRgrad algorithm is higher as we must take account of the complexity of the prediction module ARMA-OGD from \cite{anava}. We apply the ARMA-OGD to each gradient item separately. For one item, the prediction consists of an online gradient descent update of the $q$ lag coefficients. Then, the prediction is the linear combination of the $q$ previous real values of the gradient weighted by the $q$ lag coefficients. Thus, the running time of ARMA-OGD applied to our specific case is in $\mathcal{O}(mq)$. The memory requirements of the ARMA-OGD are $2m2q$ as we must store the last $q$ observations of the real gradient and the $q$ lag coefficients, for each item of the gradient. Therefore, memory requirements are of $\mathcal{O}(mq)$.

We conclude that both OOLR and OOLRgrad have fairly low running time and memory requirements, given that the number of resources $m$ composing one slice is not too high, which is practically the case.


%\emph{Remark 4.} We cannot set the value of $\sigma$ like we did in \eqref{eq:sig} while ignoring the horizon $T$. If we ignore the latter, which is possible in some settings, we must set the value of $\sigma$ according to the doubling trick. This consists in dividing the horizon $T$ in $\log_2(T)$ periods, then iterate over the periods $k=0,1,\ldots,log_2(T)$, and set for the slots of period $k$ which are $t=2^k,2^{k}+1,\ldots,2^{k+1}-1$ the value:
%$$\sigma = \frac{\sqrt{2}}{D\sqrt{2^{k+1}}}.$$
%This method will conserve the $\mathcal{O}(T^{3/4})$ regret bound, which will worsen only by a constant multiplicative factor. % of $\sqrt{2}/(\sqrt{2}-1)$. 
%We refer the reader to the proof \cite[Section~2.3.1]{shalev-book}.





% Model & Algorithm extensions
%\section{Model Extension}

\subsection{Problem statement}
We envision the case where the NO fails to fulfill the SP request in its entirety but must still comply with at least a certain ratio of what has been requested. Formally, we multiply the elements of the request vector $\bm x_t$ with a vector $\bm \alpha_t$ whose items belong to the set $[\alpha,1]$, where the value of $\alpha$ is representative of the SLA (Service Level Agreement) the NO and the SP have agreed upon. Similarly, we multiply the request vector on the spot market $\bm y_t$ with a vector $\bm \beta_t$, whose items belong to $[\beta,1]$.

We redefine the problem as:
\begin{align}
(\mathbb P):\quad \max_{ \{\bm x_t,\{\bm y_t\}\}_{t=1}^T } & \sum_{t=1}^T \Big(V  a_t \log(( \Tilde{\bm x}_t + \Tilde{\bm y}_t)^\top \bm \theta_t + 1) \notag\\
&- (\bm p_t^\top \Tilde{\bm x}_t +\bm q_t^\top \Tilde{\bm y}_t) \Big) \label{prob-obj} \\
%\text{s.t.} \quad & \sum_{t=1}^T\Big(x_tp_t+\!\!\!\!\sum_{k=(t-1)K+1}^{tK}\!\!\!\!\!y_kq_k \Big) \leq B, \label{prob-const1} \\
\text{s.t.}\quad   \bm y_t\in &\Delta, \quad \forall t=1,\ldots,T,  \label{prob-const2} \\
	 \bm x_t \in &\Delta, \quad \forall t=1, \ldots, T. \label{prob-const3}\\
	 %\alpha \in &(0,1) \label{prob-const4}
\end{align}
where $\Tilde{\bm x}_t = \bm \alpha_t \odot \bm x_t$ and $\Tilde{\bm y}_t = \bm \beta_t \odot \bm y_t$. The notation $\odot$ corresponds to the element-wise multiplication of two vectors also called the Adamar product. The latter is a linear operator, as it is equivalent to the product $A \bm x_t$ ($B \bm y_t$), where $A$ ($B$) is a diagonal matrix whose elements are the items of the vector $\bm \alpha_t$ ($\bm \beta_t$).
Thus it conserves the convexity of the problem and we can still apply the same OOLR solution, that we rename OOLRext.

\subsection{Solution}



% Simulations
\section{Numerical evaluation}

\textbf{Experimental scenario}. We consider a Mobile Virtual Network Operator (MVNO) which aims to acquire network resources that constitute the end-to-end network slice dedicated to its specific network service. Confronted with unknown and evolving traces such as the users' demand, the prices, and contributions of the network resources, the MVNO will follow the online reservation strategy designed by our OOLR solution. We consider the base case where the MVNO faces the incoming demand at one Base Station (BS) and must reserve $m=3$ types of resources to deliver its network service, encompassing radio resources at the BS, backhaul link capacity, and computing resources at the core. This base case falls under the scope of our system model. 

To model user demand, we use a real-world data set that contains the aggregated traffic volumes seen across multiple BSs owned by a major MNO of Shanghai \cite{jb-icc20}. Traffic volumes
have been recorded over a one-month period, spanning from
Friday 1 August 2014 00:00 to Sunday 31 August 2014 23:50,
with each recording averaged over a period of 10 minutes.
Hence, there are 6 measurements per hour and a total of 4464
measurements for each BS over this period.

%The data set we use contains the aggregated traffic volumes
%seen across 20 base stations owned by a major MNO within
%the city of Shanghai. For each base station, traffic volumes
%have been recorded over a one-month period, spanning from
%Friday 1 August 2014 00:00 to Sunday 31 August 2014 23:50,
%with each recording averaged over a period of 10 minutes.
%Hence, there are 6 measurements per hour and a total of 4464
%%measurements for each base station over this period. 
%We model the SP demand with the traffic load observed at one base station in a residential area. 

We assume the network resources prices vary with non-stationary dynamics. We model such variations with an AR($1$) (Auto-Regressive with $1$ lag) process, the discrete-time equivalent of the Ornstein-Ulhenbeck (OU) process. This stochastic process is applied in financial mathematics to model stock prices. %\cite{OU}. 
We model the contribution parameters -items of vector $\bm \theta_t$- as varying and non-stationary. Each item follows a seasonal trend (sine wave), with an offset and added OU stochastic process.

We compare the OOLRgrad solution to the FTRL baseline. The latter consists of the update as defined in equation \eqref{eq:ftrl}. We introduce the parameter $\zeta$ to control the quality of different prediction models, where $\zeta$ is the average relative error rate of the prediction $\grad \hat{f}_{t+1}(\boldsymbol{\hat{z}}_{t+1})$ against the real value $\grad f_{t+1}(\boldsymbol{z}_{t+1})$. We set $\zeta=0, 0.3$ and $4$ to represent prediction models from perfect accuracy to arbitrarily bad. This allows us to introduce three OOLR baselines, with different levels of prediction accuracy.

\textbf{Prediction module}. The solution OOLR is \emph{optimistic} in the sense it allows the SP to use the predicted gradient term $\grad \hat{f}_{t+1}(\boldsymbol{\hat{z}}_{t+1})$ of the next slot. In \eqref{eq:regret_bound}, we concluded that accurate predictions can greatly enhance the performance, as the regret bound goes from $\mathcal{O}(\sqrt{T})$ when predictions are arbitrarily bad to $\mathcal{O}(1)$ when predictions are perfect. This observation paves the way to the introduction of a prediction module, in support of our OOLR decision algorithm. We aim to find an accurate, robust and computationally low model. The algorithm ARMA-OGD created by Anava et al. in \cite{anava} presents these three key advantages. It consists of learning the AR($q$) signal of the trace where the $q$ lag coefficients are updated online at each slot by the gradient descent method. The algorithm guarantees that the total loss is no more on average than the loss of the best ARMA predictor with full hindsight. 

First, we show in Fig. \ref{fig:ARMAOGD} that the model is accurate against two intricate signals. The SP demand is based on multiple latent factors, which makes the signal non-stationary and hard to predict. Yet, we observe the predicted signal is able to track the SP demand. The $2m$ gradient items are composed of multiple signals, namely the SP demand, the prices and contributions of the network resources. %, as the reader can observe in the following expression of the gradient:
%\begin{align}
%    \grad f_t(\bm z_t) &= \begin{bmatrix}
 %          -V \frac{a_t \theta_{t,1}}{1+ (\bm x_t + \bm y_t)^\top \bm \theta_t  } + p_{t,1} \\
  %          -V \frac{a_t \theta_{t,2}}{1+ (\bm x_t + \bm y_t)^\top \bm \theta_t  } + p_{t,2} \\
   %        \vdots \\
    %       -V \frac{a_t \theta_{t,m}}{1+ (\bm x_t + \bm y_t)^\top \bm \theta_t  } + p_{t,m} \\
     %      -V \frac{a_t \theta_{t,1}}{1+ (\bm x_t + \bm y_t)^\top \bm \theta_t  } + q_{t,1} \\
      %     -V \frac{a_t \theta_{t,2}}{1+ (\bm x_t + \bm y_t)^\top \bm \theta_t  } + q_{t,2} \\
       %    \vdots \\
        %   -V \frac{a_t \theta_{t,m}}{1+ (\bm x_t + \bm y_t)^\top \bm \theta_t  } + q_{t,m}
        % \end{bmatrix} \label{eq:gradient}
 % \end{align}
Yet again, the model is able to give an accurate predicted signal. Secondly, ARMA-OGD provides guarantees of performance against all types of traces, which ensures its robustness. The total squared loss of the model is a $\mathcal{O}(\sqrt{T})+ Res$, where $Res$ represents the residual squared loss of the best ARMA predictor with full hindsight of the target signal. We show in Fig. \ref{fig:squaredloss} the convergence of the average squared loss towards $Res$. Finally, the ARMA-OGD is based on the OGD update step, which is very low computationally and allows us to develop the algorithm alongside the OOLR solution. We insist here that the two combined solutions having both low time complexity allow the SP to take \emph{optimistic decisions in real time}. There exists other models which employ advanced techniques such as Neural Networks that would obtain better accuracy than the ARMA-OGD. Nevertheless, these models necessitate an offline training phase, do not provide guarantees of performance, and have higher time complexity.

%Algorithm OOLR allows to use predictions through the term $\grad \hat{f}_{t+1}(\boldsymbol{\hat{z}}_{t+1})^\top \bm z$. From \eqref{eq:regret_bound}, we know that accurate prediction can greatly enhance the performance of the decisions. Thus, we combine our OOLR decision algorithm with a prediction module, which is both low computationally and accurate. To this aim, we use the algorithm ARMA-OGD from Anava et al. \cite{anava}. This algorithm forecasts the next point of the signal by creating an AR($q$) process, where the $q$ lag coefficients are updated at each observation thanks to an online gradient descent step. The algorithm guarantees with minimal assumptions the performance is no worse than the best ARMA predictor with full hindsight.

%In Fig. \ref{fig:ARMAOGD}, we show the predicted signal of the ARMA-OGD method against the actual SP demand signal. Although the initialization is rather imprecise as the first lags coefficients are chosen randomly, we observe that the predicted signal matches the actual signal over time, which means the online gradient descent update of the $2$ lags coefficients allows the generated AR($2$) process to track the actual signal. 

%In Fig. \ref{fig:pred_demand}, we compare the performance of ARMA-OGD and the best ARMA model. First, we clarify that the ARMA model is fitted to the demand signal, which means it knows the signal \emph{beforehand}. On the other hand, the signal value is revealed to the ARMA-OGD method only after the prediction. We observe in Fig. \ref{fig:pred_demand} that the ARMA-OGD performance in terms of squared loss and relative error rate slowly converges towards the best ARMA model performance. The average relative error rate is close to the variance of the trace ($0.24$), as the generated AR($2$) process is unable to predict the noise feature. We observe that the ARMA($10$,$1$) can understand better the noise feature, thanks to its moving-average component, leading to a smaller average relative error rate.


\begin{figure}
    \centering
    \includegraphics[width=1.\linewidth]{images/ARMAOGD.pdf}
    \caption{\small{The x-axis encompasses the first week of August period. \emph{Upper part:} the predicted signal against the real-world MVNO demand signal. The y-axis values are normalized. \emph{Lower part:} the predicted signal against the first of the gradient $2m$ items.}}
    \label{fig:ARMAOGD}
    \vspace{-6mm}
\end{figure}

\begin{figure}
    \centering
    \includegraphics[width=1.\linewidth]{images/squaredloss.pdf}
    \caption{\small{We evaluate the accuracy of the models on the first week of August. \emph{Left side:} We observe the convergence of the average squared loss of the predicted gradient first item towards the best ARMA in hindsight. \emph{Right side:} We observe the convergence of the average squared loss of the predicted MVNO demand toward the best ARMA in hindsight.}}
    \label{fig:squaredloss}
    \vspace{-6mm}
\end{figure}




\textbf{Impact of the quality of predictions}. The SP can reserve $m=3$ kinds of resources. We assume the NO sets the upper bound constraint to $D_i=1,\forall i$. This means the SP reserves normalized values for each type of resource. %We recall from our system model that the slice capacity is a linear combination of those normalized values and the SP utility is a logarithmic function of the obtained slice capacity. 
 Our goal is to maximize the SP utility while avoiding excessive reservation cost. We balance between the two terms (utility and cost) using the hyper-parameter $V$. % As one term is approximately $V*2\log(1+X)$, the other term is $X$ and $X$ is around $1$, we set $V=2$ to have both terms of the same order.
 We calibrate $V=2$ to have both terms of the same order.

We call OOLRgrad the online decision algorithm OOLR because the prediction method ARMA-OGD is directly applied to the gradient items. %We omit to show the OOLRsignal version where the prediction method is applied to the feedback signals (demand, prices, etc.) as it does not provide any regret bound guarantee.
%We expect a regret bound of $\mathcal{O}(\sqrt{Res+\sqrt{T}})\sim\mathcal{O}(T^{1/4})$, at the condition the residual error $Res$ of the best ARMA fitted model is a $\mathcal{O}(\sqrt{T})$. 
We show in Fig. \ref{fig:static} against the static benchmark as defined in \eqref{eq:static_regret} the performance of the OOLRgrad solution, the classical FTRL algorithm with euclidean regularizer and the different OOLR models $\zeta=0, 0.3$ and $4$. We first observe the convergence of the average regret $R_T/T$ towards $0$ for the five models, which confirm the regret bound of $\mathcal{O}(\sqrt{T})$ even for arbitrarily bad predictions (represented by the OOLR $\zeta=4$ model). Secondly, we observe a negative regret for the other four models, which confirm the $\mathcal{O}(1)$ regret bound when the predictions are accurate and the accumulated error $\sum_{t=1}^T ||\grad f_t(\boldsymbol{z}_t) - \grad \hat{f}_t(\hat{\boldsymbol{z}}_t)||^2$ is close to $0$. Zooming in the last slots, we remark that our OOLRgrad solution based on the ARMA-OGD predictor shows better performance than the OOLR solution with a $70\%$ accurate predictor ($\zeta=0.3$) and is inferior to the OOLR with perfect predictor ($\zeta=0$). The OOLRgrad and the OOLR $\zeta=0, 0.3$ solutions outperform the FTRL baseline, which shows that the incorporation of accurate predictions enhances the performance. One needs to be cautious as arbitrarily bad predictions (OOLR $\zeta=4$) worsens the performance.  %, which is a thrilling result.
In Fig. \ref{fig:dynamic}, we show the performance of the same solutions against the optimal benchmark, defined by the dynamic sequence $\{\bm z_t^* \}$, where $\forall t$, $$\bm z_t^* = \arg\min_{\bm z \in \mathcal{Z}} f_t(\bm z).$$ Against such competitive benchmark, the regret cannot be sublinear and thus the convergence of $R_T/T$ towards $0$ is not achieved. Nevertheless, we observe that the OOLRgrad solution displays good performance when compared to the different baselines.

%Therefore, we can either predict the $2m$ items of the gradient or predict the $3m+1$ signals the SP receives (the demand $\{a_t\}_t$, the network resources prices $\{p_{t,i}\}_{t,i}$ and $\{q_{t,i}\}_{t,i}$,  the network resources contributions $\{\theta_{t,i}\}_{t,i}$). We consider the two approaches in the sequel. On the one hand, the predictions on the gradient itself imply to predict $2m$ different signals, instead of $3m+1$. On the other hand, the variance of the $2m$ signals is higher as they are composed of multiple signals.

%We introduce the parameter $\beta$ to control the quality of the predictions and we look at two specific values of $\beta$ which are $0.3$ and $4$, being the average relative error rate on the sequence of gradients $\grad f_1(\bm z_1), \ldots \grad f_T(\bm z_T)$, of the good predictor and the bad predictor, respectively.

%In Fig. \ref{fig:reg}, we observe 4 different solutions. The solutions $\beta=0.3$ and $\beta=4$ correspond to the OOLR with good predictor and bad predictor, respectively. In the OGDs solution, we directly predict the $3m+1$ signals, while in the OGDg solution, we predict the resulting $2m$ signals within the gradient. We observe that the $\beta=0.3$ solution is the best. The OGDs solution is second, with an average relative error rate of $0.8$ on the gradient. Finally, $\beta=4$ and OGDg performances are similarly worse due to similar relative error rates. We still obtain a negative regret for the two worst solutions, which confirms the regret bound stated in Section II. We observe as expected a degradation of the performance when the prediction quality worsens.
\begin{figure}
\begin{subfigure}{.24\textwidth}
    \centering
    \includegraphics[width=1.\linewidth]{images/new_regret.pdf}
    \vspace{-6mm}    
    \caption{Static benchmark}
    \label{fig:static}
\end{subfigure}
\begin{subfigure}{.24\textwidth}
    \centering
    \includegraphics[width=1.\linewidth]{images/new_dynamicregret.pdf}
    \vspace{-6mm}
    \caption{Optimal benchmark}
    \label{fig:dynamic}
\end{subfigure}
\vspace{-3mm}
\caption{\small{\emph{Evolution of $R_T/T$:} Horizon $T=1008$, $m=3$, $V=2$, $\bm D=[1,1,1]$, $D=\sqrt{3}$, $\sigma = \sqrt{2}/D$.}}
\label{fig:regret}
\end{figure}


%\begin{figure}
 %   \centering
  %  \includegraphics[width=1.\linewidth]{images/new_regret.pdf}
   % \caption{\small{\emph{Evolution of $R_T/T$:} Horizon $T=1008$, $m=3$, $V=2$, $\bm D=[1,1,1]$, $D=\sqrt{3}$, $\sigma = \sqrt{2}/D\sqrt{T}$.}}
    %\label{fig:static}
%\end{figure}

%\begin{figure}
 %   \centering
 %   \includegraphics[width=1.\linewidth]{images/new_dynamicregret.pdf}
 %   \caption{\small{\emph{Evolution of $R_T/T$:} Horizon $T=1008$, $m=3$, $V=2$, $\bm D=[1,1,1]$, $D=\sqrt{3}$, $\sigma = \sqrt{2}/D\sqrt{T}$.}}
 %   \label{fig:dynamic}
%\end{figure}

%\textbf{Doubling trick}.
%An essential feature of the OOLRgrad solution is the selection of the hyper-parameter $\sigma = \sqrt{2}/D\sqrt{T}$. Only this specific value of $\sigma$ gives a sublinear upper-bound on the regret. In the case the horizon $T$ is unknown, which might probably arise in real-world settings, it becomes necessary to find a good alternative to the selection of $\sigma$. As detailed previously in section \ref{sec:algo}, \emph{Remark 4}, the doubling trick allows us to select $\sigma = \sqrt{2}/D\sqrt{2^{k+1}}$ when $t\in[2^k,\ldots2^{k+1}-1]$. The regret bound only worsens by a factor of $\sqrt{2}/(\sqrt{2}-1)$, which keep the sublinear guarantee of our solution. We observe in Fig. \ref{fig:dt} that the doubling trick actually improves slightly the performance, while it provides the same regret guarantee.

%\begin{figure}
%\begin{subfigure}{.24\textwidth}
%    \centering
%    \includegraphics[width=1.\linewidth]{images/doublingtrick.pdf}
%    \vspace{-6mm}    
%    \caption{Doubling trick}
%    \label{fig:dt}
%\end{subfigure}
%\begin{subfigure}{.24\textwidth}
%    \centering
%    \includegraphics[width=1.\linewidth]{images/extensionSLAcomparison.pdf}
%    \vspace{-6mm}
%    \caption{Extension}
%    \label{fig:ex}
%\end{subfigure}
%\vspace{-1.5mm}
%\caption{\small{\emph{Evolution of $R_T/T$:} Horizon $T=1008$, $m=3$, $V=2$, $\bm D=[1,1,1]$, $D=\sqrt{3}$, $\sigma = \sqrt{2}/D$ when $T$ is known.}}
%\label{fig:extra}
%\vspace{-6mm}
%\end{figure}


%\begin{figure}
 %   \centering
 %   \includegraphics[width=1.\linewidth]{images/doublingtrick.pdf}
 %   \caption{\small{\emph{Evolution of $R_T/T$:} Horizon $T=1008$, $m=3$, $V=2$, $\bm D=[1,1,1]$, $D=\sqrt{3}$, $\sigma = \sqrt{2}/D\sqrt{T}$ when $T$ is known.}}
 %   \label{fig:DT}
%\end{figure}


\textbf{Extension}.
Now we evaluate the OOLRgrad solution in the scenario where the NO is unable to fulfill the SP request in its entirety.
We focus on a basic scenario in which the NO ensures a minimum ratio of $\alpha$ for in-advance resources -- in a more complex scenario the NO commits to a ratio of $\alpha_i$ for each resource $i$, where $\alpha_i$ are possibly different. Thus, for each resource $i$ at slot $t$, the SP expect to receive a ratio $\alpha_{i,t}$ that belongs to the set $[\alpha,1]$. We draw the $\{\alpha_{i,t}\}_t$ from the uniform distribution on $[\alpha,1]$.  We assume that the NO consistently deliver all requested spot resources, thus we keep $\beta=1$.
We observe in Fig. \ref{fig:SLA} the regret performance of the OOLRgrad solution for three different SLAs, which are $\alpha\in\{0.5, 0.8, 0.95\}$. We observe that the performance stays similar regardless of the SLA the SP has complied for, which implies our OOLRgrad solution is consistently applicable.

\begin{figure}
    \centering
    \includegraphics[width=7cm, height=5cm]{images/extensionSLAcomparison.pdf}
    \caption{\small{\emph{Evolution of $R_T/T$:} Horizon $T=1008$, $m=3$, $V=2$, $\bm D=[1,1,1]$, $D=\sqrt{3}$, $\sigma = \sqrt{2}/D$.}}
    \label{fig:SLA}
    \vspace{-5mm}
\end{figure}


%\subsection{Estimation step method of the next point}
%Which method to use, and in which scenario?
%\subsection{Doubling trick}
%Conclusions

\vspace{-2mm}
\section{Conclusion} \label{sec:conclusion}

In this paper, we introduced the Optimistic Online Learning for Reservation (OOLR) algorithm that allows the SP to make reservations under uncertainty while incorporating predictions about the future gradient. We then proposed to combine this decision model with a prediction model, thus creating the OOLRgrad solution with better performance than the classical FTRL solution.

%We plan next to extend the model to the case the NO is unable to fulfill the SP request in its entirety.

%One interesting direction would be to consider varying constraints for the SP reservations, which means the feasible set $\Delta$ boundaries would change from slot to slot.
%The potential of network slicing markets can be only unleashed if service providers are able to reserve resources effectively. To that end, we proposed a set of slice reservation policies, based on the theory of online convex optimization, which enable the SP to learn how to reserve resources optimally. Our policies are robust to arbitrary changes of the resource prices, oblivious to lack of this information when the reservations are made, and can achieve optimal slice orchestration even when the SP needs are unknown and time-varying. These key elements build a practical and general slicing framework with performance and budget guarantees.

\section{Acknowledgments}

The research leading to this work is funded, in part, by Science Foundation Ireland (SFI), the National Natural Science Foundation of China (NSFC), and the European Commission under the SFI-NSFC Partnership Programme Grant Number 17/NSFC/5224, SFI grant 13/RC/2077 P2, and the Grant Number 101017109 (DAEMON).


%This work was supported by the European Commission through Grant No. 856709 (5Growth) and Grant No. 101017109 (DAEMON); and by SFI through Grant No. SFI 17/CDA/4760 and Grant No. 17/NSFC/5224.


%appendix
%\section*{Appendix} 

We detail how we obtain the theoretical regret bound of the OOLRgrad solution.
First, we apply the ARMA-OGD on each gradient item and we obtain the regret bound for item $i$:
$$\sum_{t=1}^T (\grad_i f_t(\bm z_t) - \grad_i \tilde{f}_t(\bm \tilde{z}_t))^2 = Res + \mathcal{O}(GM\sqrt{T}),$$
where $Res$ is the residual error of the best ARMA prediction method with full hindsight, $G$ and $M$ are key constant in \cite{anava}.
Thus, we obtain the regret bound: $$\sum_{t=1}^T ||\grad f_t(\bm z_t) - \grad \tilde{f}_t(\bm \tilde{z}_t)||^2 = 2m Res + \mathcal{O}(2mGM\sqrt{T}).$$
Thus we rewrite the regret bound in \eqref{eq:regret_bound2} as:
$$R(T) = \mathcal{O}\Big(2D\sqrt{2T}\sqrt{2mRes+\mathcal{O}(2mGM\sqrt{T})}\Big).$$
Within the square root, we neglect $2mRes$ in front of $\mathcal{O}(2mGM\sqrt{T})$, hence we can simplify: $$R(T)=\mathcal{O}(4D\sqrt{mGM} T^{3/4}) = \mathcal{O}(T^{3/4}).$$

%Now we detail the expression of $G$:
%$$G = \sum_{i=1}^m \max^2\{Va\theta_i,p_i\}+\sum_{i=1}^m \max^2\{Va\theta_i,q_i\},$$
%where $a$, $\theta_i$, $p_i$ and $q_i$ are the upper-bounds on the different signals. $M$ comes directly from \cite{anava} and represents the maximum norm of the difference between two feasible lag vectors (vectors that contains the lag coefficients which generate the prediction). 








\begin{comment}
The expressions for $V_T$ and $R_T$ follow from Theorems 1 and  2 of \cite{giannakis-TSP17}, respectively, and the contribution of the Lemma is to quantify the different parameters and the upper bound of the dual variables, which is a key step in the analysis. 

First, note that $\|f_t(x_t, y_k)\|\leq  G\triangleq a\sqrt{K(K+1)}$, for all $t$ and $x_t, y_k\in \Gamma$. Moreover, recall that we showed in \eqref{eq:constraint-bound}  that the upper bound of the constraint is $|g_t(\bm z)|\leq \max\left\{ D(p+Kq)-B, B\right\}$. Replacing these quantities in \cite{giannakis-TSP17} we obtain the respective bounds of Lemma 1.

Now, let us focus on $\tilde \ll$. We first define $\Delta(\lambda_t) := (\lambda_{t+1}^2-\lambda_t^2)/2 $, and use  \cite[Lemma 1]{giannakis-TSP17} to upper bounded as follows:
\begin{align}
\Delta(\ll_{t+1}) &\leq \mu\ll_{t+1}(U_g - \epsilon) \notag \\
&+\mu \left(2a\sqrt{K(K+1)}D + \frac{D^2}{2\nu} + \frac{\mu M^2}{2}\right)
\end{align}
Next, we proceed to prove by contradiction the upper bound on $\lambda_t$. Let us assume that $t+2$ is the first period for which the dual upper bound $\tilde{\ll}$ does not hold. Therefore,
\begin{flalign}
\begin{aligned}
{\ll_{t+1}}\leq  \mu M \!+\! \frac{2a\sqrt{K(K+1)}D+D^2/(2\nu)+(\mu M^2)/2}{\epsilon-U_g} ,\notag
\end{aligned}
\end{flalign}
\begin{flalign}
\begin{aligned}
\text{and}\,\,{\ll_{t+2}} \!>\!  \mu M \!+ \frac{2a\sqrt{K(K\!+1)}D\!+D^2/(2\nu)+(\mu M^2)/2}{\epsilon-U_g}.\notag
\end{aligned}
\end{flalign}
Working on ${\ll_{t+1}}$, we getL
\begin{flalign}
  \begin{aligned} \label{tmp3}
  \abs{\ll_{t+1}} &= \abs{\ll_{t+2}-(\ll_{t+2}-\ll_{t+1})} \\
  & \geq \abs{\ll_{t+2}} - \abs{\ll_{t+2}-\ll_{t+1}} \\
  & = \abs{\ll_{t+2}} - \abs{[\ll_{t+1}+\mu g_{t+1}(\boldsymbol{z}_{t+1})]^+ - \ll_{t+1}} \\
  & = \abs{\ll_{t+2}} - \abs{\mu g_{t+1}(\boldsymbol{z}_{t+1})} \\
  & \geq \abs{\ll_{t+2}} - \mu M \\
  & > \frac{2a\sqrt{K(K+1)}D+D^2/(2\nu)+(\mu M^2)/2}{\epsilon-U_g}
  \end{aligned}
\end{flalign}
If we multiply both sides with $\mu(U_g - \epsilon)$, which is strictly negative due to Assumption 4, (\ref{tmp3}) is equivalent to
\begin{flalign}
  \begin{aligned}
  \mu(U_g -\epsilon){\ll_{t+1}} < -\mu\left(2a\sqrt{K(K+1)}D+\frac{D^2}{2\nu}+\frac{\mu M^2}{2}\right)\notag
  \end{aligned}
\end{flalign}
Passing all the terms on the left side, we deduce:
\begin{flalign}
  \begin{aligned}
  \Delta(\ll_{t+1}) < 0 \notag
  \end{aligned}
\end{flalign}
which yields $\abs{\ll_{t+2}}<\abs{\ll_{t+1}}$ and that contradicts our assumption. As we set $\ll_1 = 0$, then $\abs{\ll_2} \leq \mu M$. Thus, for every $t\geq1$, $\abs{\ll_t}\leq\Tilde{\ll}$ holds. 

\end{comment}

%\begin{flalign}
%  \begin{aligned}
%  \ll_{T+1} &= [\ll_T + \mu g_T(\boldsymbol{z}_T)]^+ \\
%  & \geq \ll_T + \mu g_T(\boldsymbol{z}_T) \geq \ll_1 + \sum_{t=1}^T \mu g_t(\boldsymbol{z}_t)
%  \end{aligned}
%\end{flalign}

%Then we have
%\begin{flalign}
%  \begin{aligned}
%  \sum_{t=1}^T g_t(\boldsymbol{z}_t) \leq \frac{\ll_{T+1}}{\mu}
%  \end{aligned}
%\end{flalign}
%as $\ll_1=0$. Non-negativity of $\ll_{T+1}$ implies 

%\begin{flalign}
%  \begin{aligned}
%  &[\sum_{t=1}^T g_t(\boldsymbol{z}_t)]^+ \leq %\frac{\ll_{T+1}}{\mu} \\
%  \iff & \abs{[\sum_{t=1}^T g_t(\boldsymbol{z}_t)]^+} \leq \frac{\abs{\ll_{T+1}}}{\mu} \\
%  \iff & V_T \leq \frac{\abs{\ll_{T+1}}}{\mu} \leq \frac{\abs{\Tilde{\ll}}}{\mu}
%  \end{aligned}
%\end{flalign}
%which completes the proof.

%Now we work through the proof on the upper bound of the regret $R_T$. It can be shown that $\mathcal{L}_t(\boldsymbol{z},\ll)$ is $1/\nu$-strongly convex with regard to $\boldsymbol{z}$, which implies that for any $\boldsymbol{x}, \boldsymbol{y}$, we have \cite[Chapter 2.1]{hazan-book}

%\begin{flalign}
%  \begin{aligned} \label{strongConvex}
%  \mathcal{L}_t(\boldsymbol{y}) \geq \mathcal{L}_t(\boldsymbol{x}) + \grad\mathcal{L}_t(\boldsymbol{x})^\top(\boldsymbol{y}-\boldsymbol{x}) + \frac{1}{2\nu}\norm{\boldsymbol{y}-\boldsymbol{x}}^2
%  \end{aligned}
%\end{flalign}
%Since $\boldsymbol{z}_{t+1}$ minimizes the problem $\min_{\bm z\in \mathcal Z} L_{t}(\bm z, \ll_{t+1})$, the optimality condition \cite[Theorem 2.2]{hazan-book} applies

%\begin{flalign}
%  \begin{aligned} \label{KKT}
%  \grad\mathcal{L}_t(\boldsymbol{z}_{t+1})^\top(\boldsymbol{y}-\boldsymbol{z}_{t+1}) \geq 0 \qquad \forall \boldsymbol{y} \in \mathcal{Z}
%  \end{aligned}
%\end{flalign}

%Setting $\boldsymbol{y}=\boldsymbol{z}_t^*$ and $\boldsymbol{z}=\boldsymbol{z}_{t+1}$ in (\ref{strongConvex}), we have, using (\ref{KKT}), 

%\begin{flalign}
%  \begin{aligned} \label{tmp4}
%  \mathcal{L}_t(\boldsymbol{z}_t^*) \geq \mathcal{L}_t(\boldsymbol{z}_{t+1}) + \frac{1}{2\nu}\norm{\boldsymbol{z}_t^* - \boldsymbol{z}_{t+1}}^2
%  \end{aligned}
%\end{flalign}

%To alleviate the notation, we write $\mathcal{L}_t(\boldsymbol{z})$ for $\mathcal{L}_t(\boldsymbol{z},\ll)$. Then, (\ref{tmp4}) leads to

%\begin{flalign}
%  \begin{aligned} \label{mainReg}
%  \mathcal{L}_t(\boldsymbol{z}_{t+1}) \leq \mathcal{L}_t(\boldsymbol{z}_t^*) - \frac{1}{2\nu}\norm{\boldsymbol{z}_t^* - \boldsymbol{z}_{t+1}}^2 \\
 % \iff f_t(\boldsymbol{z}_t)+\mathcal{L}_t(\boldsymbol{z}_{t+1}) \leq f_t(\boldsymbol{z}_t)+\mathcal{L}_t(\boldsymbol{z}_t^*) \\
 %  \qquad - \frac{1}{2\nu}\norm{\boldsymbol{z}_t^* - \boldsymbol{z}_{t+1}}^2 \\
 % = f_t(\boldsymbol{z}_t) + \grad f_t(\boldsymbol{z}_t)^\top(\boldsymbol{z}_t^*-\boldsymbol{z}_t) \\
 % + \ll_{t+1}g_t(\boldsymbol{z}_t^*) + \frac{1}{2\nu}(\norm{\boldsymbol{z}_t^*-\boldsymbol{z}_t}^2-\norm{\boldsymbol{z_t^*}-\boldsymbol{z}_{t+1}}^2) \\
 % \stackrel{\text{(a)}}{\leq} f_t(\boldsymbol{z}_t^*) + 0 + \frac{1}{2\nu}(\norm{\boldsymbol{z}_t^*-\boldsymbol{z}_t}^2-\norm{\boldsymbol{z_t^*}-\boldsymbol{z}_{t+1}}^2)
 % \end{aligned}
%\end{flalign}

%where (a) uses the convexity of $f$ and set $\ll_{t+1}g_t(\boldsymbol{z}_t^*)$ to $0$. Indeed, $\ll_{t+1}g_t(\boldsymbol{z}_t^*) \leq 0$ as $\ll_{t+1}\geq0$ and the per-slot optimal $\boldsymbol{z}_t^*$ is always feasible, i.e. $g_t(\boldsymbol{z}_t^*) \leq 0$.

%Next, we seek to bound the term $-\grad f_t(\boldsymbol{z}_t)^\top(\boldsymbol{z}_{t+1}-\boldsymbol{z}_t)$ by

%\begin{flalign}
%  \begin{aligned} \label{tmp5}
 % -\grad f_t(\boldsymbol{z}_t)^\top(\boldsymbol{z}_{t+1}-\boldsymbol{z}_t) \stackrel{\text{(b)}}{\leq} \norm{\grad f_t(\boldsymbol{z}_t)}\norm{\boldsymbol{z}_{t+1}-\boldsymbol{z}_t} \\
 % \stackrel{\text{(c)}}{\leq} %\frac{\norm{\grad %f_t(\boldsymbol{z}_t)}^2}{2\eta} + %\frac{\eta}{2}\norm{\boldsymbol{z}_{t+1}-\boldsymbol{z}_t}^2 \\ \stackrel{\text{(d)}}{\leq}  \frac{a^2K(K+1)}{2\eta}+\frac{\eta}{2}\norm{\boldsymbol{z}_{t+1}-\boldsymbol{z}_t}^2
 % \end{aligned}
%\end{flalign}

%where (b) uses Cauchy-Schwartz inequality, (c) is true for any arbitrary positive constant $\eta$, (d) uses the upper bound on the gradient in Assumption 3. Plugging (\ref{tmp5}) into (\ref{mainReg}), we can isolate $f_t(\boldsymbol{z}_t)+\ll_{t+1}g_t(\boldsymbol{z}_{t+1})$

%\begin{flalign}
%  \begin{aligned} \label{tmp6}
%  f_t(\boldsymbol{z}_t)+\ll_{t+1}g_t(\boldsymbol{z}_{t+1}) \leq f_t(\boldsymbol{z}_t^*) + (\frac{\eta}{2}-\frac{1}{2\nu})\norm{\boldsymbol{z}_{t+1}-\boldsymbol{z}_t}^2 \\
%  + \frac{1}{2\nu}(\norm{\boldsymbol{z}_t^*-\boldsymbol{z}_t}^2-\norm{\boldsymbol{z_t^*}-\boldsymbol{z}_{t+1}}^2) + \frac{a^2K(K+1)}{2\eta} \\
%  \stackrel{\text{(e)}}{=}f_t(\boldsymbol{z}_t^*) + \frac{1}{2\nu}(\norm{\boldsymbol{z}_t^*-\boldsymbol{z}_t}^2-\norm{\boldsymbol{z}_t^*-\boldsymbol{z}_{t+1}}^2) \\
%  + \frac{\nu a^2 K(K+1)}{2}
%  \end{aligned}
%\end{flalign}

%where (e) happens as we choose $\eta = 1/\nu$ so that $\eta/2 - 1/2\nu =0$. Using the dual drift bound in (\ref{lemma1}), we have

%\begin{flalign}
%  \begin{aligned} \label{tmp7}
%  \frac{\Delta(\ll_{t+1})}{\mu} + f_t(\boldsymbol{z}_t) \leq \ll_{t+1}(g_{t+1}(\boldsymbol{z}_{t+1})-g_t(\boldsymbol{z}_{t+1})) \\
%  + f_t(\boldsymbol{z}_t)+ \ll_{t+1}g_t(\boldsymbol{z}_{t+1})+\frac{\mu}{2}g_{t+1}(\boldsymbol{z}_{t+1})^2 \\
%  \stackrel{\text{(f)}}{\leq} \ll_{t+1}[g_{t+1}(\boldsymbol{z}_{t+1})-g_t(\boldsymbol{z}_{t+1})]^+ + f_t(\boldsymbol{z}_t^*) \\
%  + \frac{1}{2\nu}(\norm{\boldsymbol{z}_t^*-\boldsymbol{z}_t}^2-\norm{\boldsymbol{z}_t^*-\boldsymbol{z}_{t+1}}^2) + \frac{\nu a^2 K(K+1)}{2} \\
%  + \frac{\mu g_{t+1}(\boldsymbol{z}_{t+1})^2}{2} \\
%  \stackrel{\text{(g)}}{\leq} f_t(\boldsymbol{z}_t^*) + \frac{1}{2\nu}(\norm{\boldsymbol{z}_t^*-\boldsymbol{z}_t}^2-\norm{\boldsymbol{z}_t^*-\boldsymbol{z}_{t+1}}^2) \\
 % + \Tilde{\ll}U_g + \frac{\nu a^2 K(K+1)}{2} + \frac{\mu M^2}{2}
 % \end{aligned}
%\end{flalign}

%where (f) uses the non-negativity of $\ll_{t+1}$ and (\ref{tmp6}), (g) comes from the upper bound $\abs{g_{t+1}(\boldsymbol{z}_{t+1})} \leq U_g$ in Assumption 3, and from the dual upper bound $\ll_{t+1}\leq\Tilde{\ll}$.

%By interpolating the norm terms, we have

%\begin{flalign}
%  \begin{aligned} \label{tmp8}
%  \norm{\boldsymbol{z}_t^*-\boldsymbol{z}_t}^2 - \norm{\boldsymbol{z}_t^*-\boldsymbol{z}_{t+1}}^2 \\
%  = \norm{\boldsymbol{z}_t^*-\boldsymbol{z}_t}^2 - \norm{\boldsymbol{z}_t-\boldsymbol{z}_{t-1}^*}^2 +\norm{\boldsymbol{z}_t-\boldsymbol{z}_{t-1}^*}^2 \\
%  - \norm{\boldsymbol{z}_t^*-\boldsymbol{z}_{t+1}}^2 \\
%  = \norm{\boldsymbol{z}_t^*-\boldsymbol{z}_{t-1}^*}\norm{\boldsymbol{z}_t^* - 2\boldsymbol{z}_t + \boldsymbol{z}_{t-1}^*} + \norm{\boldsymbol{z}_t-\boldsymbol{z}_{t-1}^*}^2 \\
%  - \norm{\boldsymbol{z}_t^*-\boldsymbol{z}_{t+1}}^2 \\
%  \stackrel{\text{(h)}}{\leq} \norm{\boldsymbol{z}_t^*-\boldsymbol{z}_{t-1}^*}2D + \norm{\boldsymbol{z}_t-\boldsymbol{z}_{t-1}^*}^2 - \norm{\boldsymbol{z}_t^*-\boldsymbol{z}_{t+1}}^2
%  \end{aligned}
%\end{flalign}

%where (h) follows from the diameter of $\mathcal{Z}$ in Assumption 3. Plugging (\ref{tmp8}) into (\ref{tmp7}), we have

%\begin{flalign}
%  \begin{aligned} \label{tmp9}
%  \frac{\Delta(\ll_{t+1})}{\mu} + f_t(\boldsymbol{z}_t) \stackrel{\text{(i)}}{\leq} f_t(\boldsymbol{z}_t^*)+ \Tilde{\ll}U_g + \frac{\nu a^2 K(K+1)}{2} \\
%  + \frac{\mu M^2}{2} + \frac{1}{2\nu}(2DU_z  + \norm{\boldsymbol{z}_t-\boldsymbol{z}_{t-1}^*}^2 - \norm{\boldsymbol{z}_t^*-\boldsymbol{z}_{t+1}}^2)
%  \end{aligned}
%\end{flalign}

%where (i) comes from $\norm{\boldsymbol{z}_t^*-\boldsymbol{z}_{t-1}^*} \leq U_z$, from the definition of $U_z$ in Assumption 3.

%Summing up (\ref{tmp9}) over $t=1...T$, and rearranging the terms, we find

%\begin{flalign}
%  \begin{aligned}
%  \sum_{t=1}^T f_t(\boldsymbol{z}_t)-\sum_{t=1}^T f_t(\boldsymbol{z}_t^*) \leq \Tilde{\ll}U_g^T + \frac{\nu a^2T K(K+1)}{2} \\
%  + \frac{\mu T M^2}{2} + \frac{D U_z^T}{\nu} + \frac{1}{2\nu}\sum_{t=1}^T (\norm{\boldsymbol{z}_t-\boldsymbol{z}_{t-1}^*}^2 - \norm{\boldsymbol{z}_t^*-\boldsymbol{z}_{t+1}}^2) \\
%  - \sum_{t=1}^T \frac{\Delta(\ll_{t+1})}{\mu} \\
%  \stackrel{\text{(j)}}{=} \Tilde{\ll}U_g^T + \frac{\nu a^2T K(K+1)}{2 }+ \frac{\mu T M^2}{2} + \frac{D U_z^T}{\nu} \\
%  + \frac{1}{2\nu}(\norm{\boldsymbol{z}_1-\boldsymbol{z}_0^*}^2-\norm{\boldsymbol{z}_T^*-\boldsymbol{z}_{T+1}}^2) - \frac{1}{2\mu}(\ll_{T+2}^2-\ll_{2}^2) \\
%  \stackrel{\text{(k)}}{\leq} \Tilde{\ll}U_g^T + \frac{\nu a^2T K(K+1)}{2 }+ \frac{\mu T M^2}{2} + \frac{D U_z^T}{\nu} \\
%  + \frac{D^2}{2\nu} + \frac{\mu M^2}{2}
%  \end{aligned}
%\end{flalign}

%where (j) comes from the telescoping sums, (k) uses $\norm{\boldsymbol{z}_T^*-\boldsymbol{z}_{T+1}}^2 \geq 0$, $\ll_{T+2}^2 \geq 0$, $\norm{\boldsymbol{z}_1-\boldsymbol{z}_0^*}^2 \leq D^2$, $\ll_2^2 \leq \mu^2 M^2$.
%This completes the proof.






%%%%%%%%%%%%%%%%%%%%%%%%%%%%%%%%%%%%%%%% ============================================= %%%%%%%%%%%%%%%%%
\vspace{-2mm}

\bibliographystyle{IEEEtran}
%\bibliography{ref}

%\bibliographystyle{ieeetr}
\bibliography{ref.bib}

%\newpage
%
\chapter{Modelling Black Hole Signals with Gaussian Processes}

\section{Additional Graphical Tests for Identifying the Flux Distribution}
\label{dist_tests}

In \autoref{PP Plots} probability-probability (PP) plots and empirical cumulative distributions functions (ECDFs) are shown as graphical distribution tests for Gaussianity. It may be observed qualitatively that both X-ray band log count rates and UVW2 flux are well-modelled by a Gaussian distribution.


\begin{figure}[h!]
\centering
\subfigure[PP plot for X-ray log count rates]{\label{fig:4pt1}\includegraphics[width=0.49\textwidth]{Chapter3/Figures/xray_prob_plot.png}}
\subfigure[PP plot for UVW2 flux]{\label{fig:4pt2}\includegraphics[width=0.49\textwidth]{Chapter3/Figures/uv_prob_plot_mags_is_False.png}}
\subfigure[ECDF for X-ray log count rates]{\label{fig:4pt3}\includegraphics[width=0.49\textwidth]{Chapter3/Figures/xray_ecdf_with_ref.png}}
\subfigure[ECDF for UVW2 flux]{\label{fig:4pt4}\includegraphics[width=0.49\textwidth]{Chapter3/Figures/uv_ecdf_mags_is_False_with_ref.png}}  
\caption{PP plots and ECDFs for X-ray log count rates and UVW2 flux, graphical tests of Gaussianity. In the case of the PP plots, proximity to the line is an indicator of Gaussianity. In the case of the ECDF plots, resemblance to the cumulative distribution function of a Gaussian is indicative of Gaussianity. The  figures above were generated by Douglas Buisson.}
\label{PP Plots}
\end{figure}




\section{Spectral Properties of the Examined Kernels}
\label{kern_rat}

The autocorrelation functions, log autocorrelation functions and PSDs are illustrated for the Matérn, squared exponential and rational quadratic kernels in \autoref{kern_acfs}. The figures were generated by Douglas Buisson.

\begin{figure}[h!]
\centering
\subfigure[Kernel autocorrelation functions]{\label{fig:4k}\includegraphics[width=0.4\textwidth]{Chapter3/Figures/gp_kernel_acf.pdf}}
\subfigure[Kernel log autocorrelation functions]{\label{fig:3k}\includegraphics[width=0.4\textwidth]{Chapter3/Figures/gp_kernel_acf_log.pdf}}
\subfigure[Kernel PSDs]{\label{fig:5k}\includegraphics[width=0.4\textwidth]{Chapter3/Figures/gp_kernel_psds.pdf}}
\caption{Kernel autocorrelation functions and PSDs. The rational quadratic kernel is plotted for different values of the $\alpha$ parameter. The Matérn kernel plots in the PSD figure are offset by a factor of 10 for clarity. A PSD of $f^{-2}$ will match the high frequency part of the Matérn $\frac{1}{2}$ kernel and the rational quadratic is endowed with additional flexibility to model PSDs by virtue of its $\alpha$ parameter. Such characteristics may explain why these kernels are preferred in the simulation study.}
\label{kern_acfs}
\end{figure}



\end{document}
