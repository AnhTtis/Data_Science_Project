\section{Model \& Algorithm Extensions} \label{sec:extensions}
We consider two practical extensions which demonstrate the modeling power of our framework: \emph{(i)} when the SP decides the \emph{slice composition} where the benefit from each resource is time-varying; and \emph{(ii)} mixed-time scale bidding where the SP updates its per-slot reservations \emph{during each period} as it acquires information for the demand and spot prices. 

\textbf{Slice Orchestration (SO)}. The SP makes multi-dimensional reservations using $\bm x_t, \bm y_k\in \mathbb R^d$, where $d$ is the number of resources comprising the slice. The benefit from each reservation $\bm x_t$ is quantified by the scalar $\bm \theta_t^\top \bm x_t$ ($\bm \theta_t^\top \bm y_k$, respectively), where the elements of $\bm \theta_t\in \mathbb R^d$ measure the contribution of each resource on performance. And, we allow this vector to change with time, and be unknown when the reservations are decided. Consider, e.g., an SP that is unaware of the optimal computation, storage and bandwidth mix, as this depends on the type of user requests. Our analysis can be extended to handle this richer scenario. Namely,  we need to update the objective and cost functions defined in (\ref{eq:objective-function}), \eqref{eq:constraint-function} by:
\begin{align}
&f_t(\bm x_t, \{\bm y_k\}_{k=(t-1)K+1}^{tK}) = -\!\!\!\!\!\!\!\!\!\!\!\sum_{k=(t-1)K+1}^{tK}\!\!\!\!\!\!\!\!a_k \log(\bm \theta_t^\top \bm x_t + \bm \theta_t^\top \bm y_k +1), \label{eq:new-objective-function} \\
&g_t(\bm x_t, \{\bm y_k\}_{k=(t-1)K+1}^{tK}) =  \bm x_t^\top \bm p_t +\!\!\!\!\!\!\!\!\sum_{k=(t-1)K+1}^{tK}\!\!\!\!\!\!\!\! \bm y_k^\top \bm q_k - B. \label{eq:new-constraint-function}
\end{align}
Then, Algorithm OLR needs to be slightly modified by replacing the primal update (step 3) with a similar update that finds the multidimensional reservations $\bm x_t, \bm y_k\in \mathbb R^d$; change the step 4 so as to observe both the $\bm a_t$ and the $\bm \theta_t$ vectors; and perform the dual update \eqref{eq:dual-step} using $g_t(\bm x_t$, \{$\bm y_k\})$. %The new algorithm allows the SP to decide in each slot the \emph{size and composition} of its slice, without even knowing the maximum-performance mix of resources. 

%We collect all the reservation vectors of the period in one super-vector $\boldsymbol{Z} = (\bm x, \{\bm y\})$ and we define the Lagrangian, like in (\ref{eq:lagrange}):
%\begin{align}
%L_t(\bm Z, \ll) &= \grad f_t(\bm Z)^\top (\bm Z- \bm Z_t)+  \ll^\top g_t(\bm Z)  \notag \\
%					   &+ \frac{\norm{\bm Z-\bm Z_t}^2}{2\nu} -\frac{\ll^2}{2\mu}   \label{eq:new-lagrange}
%\end{align}
%
%where $\ll\in\mathbb R_+$ is the dual variable that we introduce by relaxing constraint (\ref{eq:new-constraint-function}). Then, the SP decides its reservation policy $\bm Z_t = (\bm x_t, \{\bm y\}_t)$, by performing a primal update as follows:
%\begin{equation}
%\bm Z_{t} = \arg \min_{\bm Z\in \mathcal Z} L_{t-1}(\bm Z, \ll_{t}),\label{eq:new-primal-step}
%\end{equation} 
%
%where $\mathcal{Z} = \Gamma^{K+1}$ and $\Gamma \subset \mathbb R^M$. We need to impose a resource availability limit for each resource. Thus, in the spirit of the parameter $D$ of Assumption 3, we use the vector notation $\bm d$. Concretely, our reservations $\bm x$ belong to $\Gamma$ if $\bm 0 \preceq \bm x \preceq \bm d$. At the end of the period, the SP has access to the current Lagrangian, and can accordingly update its dual variable by executing the dual update:
%\begin{equation}
%\ll_{t+1} = \arg \max_{\ll\geq 0} L_t(\bm Z_t, \ll). \label{eq:new-dual-step}
%\end{equation} 





\textbf{Mixed-time-scale reservation (MTS)}. Our second extension is a mixed time scale reservation model, where the SP can update the slot reservations $ \bm y_t$ of each period $t$, based on the demand $\bm a_t$ and spot prices $\bm q_t$ it observes during that period. The functions $f_k$ and $g_k$ for this slot-decision instance, are:
\begin{align}
f_k(y_k)=- a_k \log(x_t+y_k+1),\,\,\,\,\,\,
g_k(y_k) =  y_k q_k - B_{slot},  \notag % \label{eq:new-new-constraint-function}
\end{align}
where note that $x_t$ is a parameter here, as it has been fixed in the beginning of $t$, and we have defined $B_{slot}=(B-x_tp_t)/K$. Finally, the per-slot Lagrangian is:
\begin{align}
L_k(y, \hat \ll) &=\grad f_k(y_k) ( y-  y_k)+  {\hat \ll} g_k( y) \notag \\ &+ \frac{(y - y_k)^2}{2\hat\nu} 
%- \frac{{\hat \ll}^2}{2\hat \mu}
\label{eq:new-new-lagrange}
\end{align}
where $\hat \ll \!\in\!\mathbb R_+$ is the new dual variable. Then, the SP updates its reservation $y_k$ for each slot $k$, and its dual variable after observing $a_k $ and $q_k$, by executing:
\begin{equation}
y_k = \arg \min_{ y\in \Gamma} L_{k-1}( y, \hat \ll_{k}),\,\,\,\,
\hat \ll_{k+1} = [\hat{\ll_k} + \hat{\mu}\grad L_k(y_k,\hat{\ll})]_+. \notag
\end{equation}
$\hat \nu$ and $\hat \mu$ are the steps for the intra-period decisions. We set in the next section $\hat{\nu}=\nu$ and $\hat{\mu}=\mu$.
Algorithm OLR can be amended with these updates (for all slot $k$) after step 3. We do not provide theoretical guarantees but we do verify next the performance gains of this refined approach.


%\begin{algorithm}
%\SetAlgoRefName{OLR extension 2}
%\caption{Online Learning for Reservation extension 2}
%\DontPrintSemicolon
%\KwInitialize{ \; $\ll_1=0, x_0\in \Gamma, \bm y_0 \in \Gamma^K,\, \nu=\mu=T^{-1/3} $ } 
%	%
%	%
%\For{ $t=1,\ldots, T$ } 
%{
%Observe the $t$-period price $p_t$ \;
%Decide $(x_t, \bm y_t)$ by solving \eqref{eq:primal-step} \;
%% 
%Calculate $B_{slot} = (B-x_t*p_t)/K$ \;
%\For{ $k=1,\ldots, K$}
%{
%Decide $y_k$ by solving (\ref{eq:new-new-primal-step}) and replace $k$-th element of vector $\bm y_t$ by $y_k$ \;
%%
%Observe $ q_k$ and calculate $g_k(y_k)$ \;
%%
%Decide $\ll_{k+1}$ by solving (\ref{eq:new-new-dual-step})}
%%
%Observe $\bm a_t$ and calculate $f_t(x_t, \bm y_t)$ \;
%%
%Observe $\bm q_t$ and calculate $g_t(x_t, \bm y_t)$ \;
%%
%Decide $\ll_{t+1}$  by solving \eqref{eq:dual-step} \;
%%
%}
%\end{algorithm}
