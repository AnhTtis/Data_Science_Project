\section{Introduction}

%\textbf{Motivation}. 

\textbf{Motivation}. The virtualization of wireless networks has gained significant interest in recent studies, cf. \cite{5g-ppp, mano-nfv}. This new technology enables the development of the Network Slicing framework, where service providers (SPs) can lease virtualized network resources from the Network Operator (NO) to address the demand of their specific network service \cite{foukas-commag}. %, xavi-commag2018}. 
Network Slicing promises to boost the utilization efficiency of the network resources by accommodating multiple and diverse SPs on the NO's infrastructure. This in turn brings new challenges: on the one hand the NO must accommodate heterogeneous slices on its network to satisfy diverse requirements of the SPs; on the other hand the SPs must request network resources or slice requirements in a smart and proactive way by anticipating their future demand.

The players are expected to operate in a real-time market, where the SPs can lease both computing and storage resources while the NO offers both in-advance reservation and on-the-fly spot opportunities. The modeling of such slicing market draws ideas from cloud marketplaces \cite{amazon-reserved, amazon-spot}, where the Cloud Provider allows customers to bid for resources in the on-demand and spot markets \cite{carlee-infocom18, carlee-sigcomm15}. This market will allow the NO to proactively schedule the slice configuration based on the information coming from the in-advance reservation requests, but also offer the available spot resources dynamically, leading to slice re-configuration and boosting network utilization.

In this context, one SP competes with other SPs for the network resources in the on-demand and spot markets and must request/bid for the resources while ignorant of their prices. % and facing uncertainty about its own demand. %\cite{hossain-survey2018}. 
We expect the NO to reveal those prices after the SP request. %, once the resources are leased to the SP and the payment is effective. %We expect the NO reveals the prices along with the SP demand % the deserved traffic of the SP after the SP request. 
Therefore, the SP must decide its requests dynamically without the information of the resource pricing and its own future demand. Additionally, we expect the prices to vary according to non-stationary patterns, as they might depend on multiple underlying factors, such as the other SPs requests, the NO internal needs, etc. We highlight here the necessity for the SP to build a decision model robust to uncertainty, while being able to use its own historical demand and the NO's feedback about historical prices.

\textbf{Related Work}. %paper using prediction
By anticipating the resource utilization, the NOs can enhance their resource management decisions regarding resource provisioning or allocation.
In \cite{DeepCog}, network traffic information is leveraged to plan the capacity needed for each slice in a multi-tenant framework. Using a data-driven approach including C-RAN, MEC and core networks, the solution outperforms other state-of-the-art deep learning solutions \cite{infocom17}, \cite{mobihoc18}. The approach in \cite{oliveiraTNSM} employed an adaptive forecasting model of the elastic demand for network resources to perform slice allocation in Internet Access Services. The authors in \cite{XaviINFOCOM17} and \cite{XaviTrans19} predicted the required resources by tenants for the future time window to perform slice requests admission and schedule the users' traffic within each slice. %The prediction module can adapt its predictions by changing a forecasting error probability according to the feedback sent by the slice scheduler. 
In \cite{cui-iccc20} cellular traffic prediction helps the allocation policy for the vehicular network slice. \cite{jb-icc20} uses historical traffic to design the SP resource reservation policy. %Similarly, we use prediction to assist the SP reservation policy.
Unlike these approaches, our solution does not need offline training and provides performance guarantees against all types of traces.

Recent works consider the SP resource provisioning problem. The paper \cite{reyhanian} developed a two-time scale approach for the activation and the re-configuration of the slices while considering the reservation of both RAN and backhaul resources. \cite{zhang-tcom2018} focuses on wireless spectrum considering two reservation schemes (in advance and on demand). In \cite{vincent-TVT2020}, the authors develop a two-stage approach for the resource reservation and the intra-slice resource allocation. These works presume a stationary environment where user statistics do not change and/or cost of resources are supposed constant. This paper differs from our previous works \cite{JB2021, JB2022}, as we now use prediction to support the reservation model.

%The focus of this paper is to tackle exactly the problem faced by the SP. We propose a solution that allows the SP to decide its online reservation policy, which aims to maximize its service utility and minimize its cost of reservation simultaneously. We develop an optimistic online learning algorithm (OOLR) able to incorporate predictions to assist the reservation decision of the SP. %Our solution provides guarantees of performance even for arbitrarily bad predictions which may arise in volatile settings. We complement our decision model with an accurate, robust, and low-complexity prediction model. The combined solution achieves a $\mathcal{O}(T^{3/4})$ regret.

%We develop an online optimization algorithm so that the SP learns how to proactively reserve resources in an online manner. Our proposal falls under the scope of Online Convex Optimization, which we detail in the following subsection.

\textbf{Methodology and Contributions}.
%OCO
The problem of learning how to bid in an online manner while facing uncertainty fits to the Online Convex Optimization (OCO) framework, introduced by Zinkevich \cite{zinkevich}. In OCO, the learner tries to minimize its total loss with respect to the best static solution:
\begin{align}
    R(T) = \sum_{t=1}^T f_t(\bm z_t) -  \min_{\bm z \in \mathcal{Z}} \sum_{t=1}^T f_t(\bm z), \label{eq:static_regret}
\end{align}
by deciding the reservation vector $\bm z_t$ at each round $t$, without knowing the convex loss $f_t$. We say the online policy $\{\bm z_t\}_{t=1}^T$ has \emph{no-regret} if the achieved regret is sublinear, i.e. $R(T) = o(T)$, in other words $\lim_{T\rightarrow \infty} R(T)/T = 0$. We build our Optimistic Online Learning for Reservation (OOLR) solution upon the Follow-The-Regularized Leader (FTRL) algorithm \cite{ftrl}. % which ensures a $\mathcal{O}(\sqrt{T})$ regret. 
We develop an \emph{optimistic} version of the FTRL, first introduced by Rakhlin and Sridharan \cite{sridharan}, where the decision relies on an adaptive proximal regularizer term and the optimistic term of the next gradient prediction $\grad \hat f_{t+1}(\hat{\boldsymbol{z}}_{t+1})$. With perfect predictions, the regret of our decisions reduces to $\mathcal{O}(1)$, synonymous of negative regret. With arbitrarily bad predictions (of the order of $T$), the regret bound is $\mathcal{O}(\sqrt{T})$.


%predictions
As the SP accumulates historical data about prices and demand, there exists the possibility to extract predictions for the next slot values based on previous window of the traces by using auto-regressive methods. Holt-Winters, Auto-Regressive Integrated Moving Average (ARIMA) or Neural Networks have been applied in \cite{oliveiraTNSM} and in \cite{XaviINFOCOM17}. Albeit accurate, these methods do not provide performance guarantees. % and therefore do not improve the $\mathcal{O}(\sqrt{T})$ performance bound we observe for arbitrarily bad predictions. 
The ARMA-OGD algorithm presented in \cite{anava} is an accurate, robust and computationally low prediction model. It generates the predictions through an auto-regressive process, where the lag coefficients are updated using the online gradient descent (OGD) method, which has low time complexity. It also provides regret guarantees against the best Auto-Regressive Moving Average (ARMA) predictor with full hindsight of the future. %The integration of this prediction module into our main decision algorithm (OOLR) improves the regret bound to $\mathcal{O}(T^{1/4})$.


%In the line of our previous works \cite{JB2021, JB2022}, we consider a hybrid slicing market where the SP can reserve different types of resources, which will compose the reserved slice. %, both in-advance and on-the-spot. 
%The NO pricing scheme is unknown to the SP, and the prices of the resources are revealed to the SP only after it makes the bid. Hence, the SP has to reserve the network resources without knowing the demand from its users and the prices of the network resources. %The goal of the SP is to maximize the slice performance and to minimize the overall cost of reservation. 
%The SP can leverage the feedback received from the NO to design accurate predictions that will assist the online reservation policy, which end goal is to maximize the slice performance and minimize the cost of reservation.

%The SP receives feedback from the NO about its demand and the resource pricing after its reservation being made. As time passes, the SP accumulates historical data of those varying and potentially non-stationary signals. The latter information paves the way to the design of a prediction algorithm able to feed accurate predictions of the different signals into the reservation-based decision model.

%We model the reservation problem faced by the SP as a learning problem, where the goal is to design a \emph{no-regret online reservation policy}. We fall under the scope of the standard Online Convex Optimization (OCO) problem introduced in the seminal paper of Zinkevich \cite{zinkevich}, which aims to find the online policy $\{\bm z_t\}_{t=1}^T$ that minimizes the regret with respect to the best static solution:
%\begin{align}
 %   R(T) = \sum_{t=1}^T f_t(\bm z_t) -  \min_{\bm z \in \mathcal{Z}} \sum_{t=1}^T f_t(\bm z). \label{eq:static_regret}
%\end{align}
%At each slot $t$, the learner (the SP in our case) must decide $\bm z_t$ (the reservation vector) from a convex set $\mathcal{Z}$ without knowing the convex loss $f_t$. 
%We say the online policy $\{\bm z_t\}_{t=1}^T$ has \emph{no-regret} if the achieved regret is sublinear, i.e. $R(T) = o(T)$, in other words $\lim_{T\rightarrow \infty} R(T)/T = 0$. 

%We build our OOLR solution upon the FTRL algorithm \cite{ftrl}. % which ensures a $\mathcal{O}(\sqrt{T})$ regret. 
%We develop an \emph{optimistic} version of the FTRL, first introduced by Mohri and Yang \cite{mohri}, where the decision relies on an adaptive proximal regularizer term and the optimistic term of the next gradient prediction $\grad \hat f_{t+1}(\hat{\boldsymbol{z}}_{t+1})$. 
%We modify the latter to unleash the use of predictions that will reduce the uncertainty inherent to the unknown parameters at slot $t$. 
%We opt for an accurate, robust and computationally low prediction model which we find in the work of Anava et al \cite{anava}: the ARMA-OGD algorithm. %The latter generates the predictions through an AR($q$) process, where the $q$ lag coefficients are updated using the online gradient descent (OGD) method, which has low time complexity. It also provides regret guarantees against the best ARMA predictor with full hindsight of the future $\{f_i\}_{i=t+1}^T$. The two combined models provide the SP with a unique solution which is competitive and presents strong guarantees.




The contribution can be stated as follows:
\begin{itemize}[leftmargin=4mm]
\item we formulate an optimization problem for the SP where it aims to maximize the leased slice utility and minimize the reservation cost in the long-term; 
\item to solve the reservation problem faced by the SP, we develop an online learning solution (OOLR) which incorporates the \emph{optimistic} prediction of the next slot gradient;
\item we provide regret bound guarantees of $\mathcal{O}(\sqrt{T})$ for arbitrarily bad predictions and $\mathcal{O}(1)$ for perfect predictions;
\item we implement a prediction module to assist our OOLR decision algorithm and we demonstrate good performance of the combined solution, named OOLRgrad; 
\item against real world data and non-stationary traces, our OOLRgrad solution outperforms the FTRL baseline. We extend our model to the situation the NO only fulfills part of the SP reservation request due to capacity constraints. 
%\item 


%\item We use the multi-resource reservation model introduced in \cite{JB2021} and \cite{JB2022} and we reformulate the learning problem into an unconstrained OCO problem;
%\item for our specific reservation problem, we propose a new optimistic online learning algorithm with adaptive regularizer (OOLR) which achieves negative regret in the best case, sublinear regret in the worst case;
%\item we develop a prediction algorithm able to feed accurate input into the OOLR algorithm which consequently enhance the quality of decisions and improve the regret performance.
\end{itemize}









%In this framework, the NO, which is the owner of the network, must accommodate heterogeneous slices corresponding to diverse types of services \cite{transWC2018, VT2020}.  Most papers have proposed solutions for assisting the NO in the resource allocation task that comes along with the accommodation of slices. 


%Fewer works have focused on how the SP should request slices or network resources. In \cite{zhang-tcom2018}, the authors consider a hybrid reservation/spot market where the SP reservations are decided by solving a stochastic problem, and a mixed time scale reservation model was studied in \cite{vincent-TVT2020}. %Our previous work \cite{jb-icc20} employed demand and price predictions (via neural networks) to assist the SP's reservations, while \cite{tony-tnsm19} focused on slice reconfiguration costs.
%In \cite{paschos-infocom19}, the authors consider a reservation model, where a customer at each slot requests different types of cloud resources to minimize its total cost while satisfying the average constraint on the demand. The authors propose a primal-dual type of algorithm and their online policy is asymptotically feasible and achieves $o(T)$ regret against the $K$-slot static benchmark (static reservation that only satisfies the average constraint every $K$ slots), with $K=o(T^{1-\epsilon})$. Unlike our approach, the costs in the objective are known and fixed.
%In \cite{zhang2017}, the authors design a reservation model of long-term and short-term VM instances, then used for VNF deployment, and create a complete online solution - composed of VNF demand prediction and reservation decisions - with performance guarantees.


%softwarization of resources
%The increasing softwarization of wireless networks coupled with the proliferation of over-the-top service providers (SPs) which rely on network operators' infrastructure (NOs), have spurred numerous studies for  {network slicing} solutions, cf. \cite{foukas-commag}, \cite{xavi-commag2018}. For instance, researchers have proposed embedding algorithms for assisting NOs to accommodate heterogeneous slices \cite{paschos-mag17}; and pricing mechanisms to maximize the operators' revenue from selling slices to different SPs \cite{hossain-survey2018}. These schemes are expected to operate in near-real time and enable the fine-grained (re-)allocation of network resources; hence, boosting their utilization efficiency. Yet, an aspect that has received less attention is how the SPs should request slices.% in this dynamic environment. 

%slicing markets, ~ cloud computing market
%The envisioned network slicing markets draw ideas from pertinent cloud computing marketplaces \cite{google-pricing, google-spot, amazon-reserved, amazon-spot}, where cloud providers adapt dynamically their offered prices, and SPs complement their reservations with on-the-fly bidding in spot markets. This flexibility compounds the slice-reservation task of each SP which has to request resources  without knowing the needs of its users; to decide between (lower-cost) advance reservation and (higher-cost) dynamic reservation; and to anticipate the future slice prices that, in turn, depend on the NO's internal needs and the requests of other SPs. If these decisions are ineffective, the SPs might over/under-reserve resources, which will lead to network under-utilization or induce prohibitively high servicing costs. And these effects can nullify the anticipated benefits of slicing.

% online optimization
%OCO is an efficient mathematical framework introduced by Zinkevich \cite{zinkevich}, to evaluate the performance -through the regret metric- of online optimization algorithms \cite{koppel, cao2018online, giannakis-TSP17}, i.e. algorithms that solve an optimization problem in an online manner, at each time $t \in \mathcal{T}=\{1,\ldots,T\}$, assuming we have a time-slotted system. 
%In online learning and more specifically in OCO problems \cite{shalev-book, hazan-book}, an agent has to take sequential decisions at each time $t$, that minimizes the objective, while satisfying the constraints (if the problem is constrained). The agent does not have access to the time-varying objective function $f_t(.)$ prior to the decision (nor the constraints). Thus, the difficulty arises in this situation as the agent must take the decisions under uncertainty. The design of an efficient algorithm, able to adapt to the dynamics of the problem, becomes essential. The regret metric reflects the efficiency of the algorithm: it measures the difference of the cumulative loss incurred by our decisions -over $T$ time slots- from the cumulative loss that experiences the benchmark, which has full hindsight of the problem. A sublinear regret, i.e. $R_T=o(T)$, is regarded as a good performance, as it implies that the online algorithm asymptotically performs no worse than the benchmark. In constrained OCO problems, we deem both the regret metric and the constraint violations metric -or fit- which is the accumulation of violations through the duration $T$. We observe a natural trade-off between the two metrics and our aim is to design algorithms that keep both a sublinear regret and a sublinear fit.

%paper focus

%The focus of this paper is to tackle exactly this problem by studying optimal slice reservation from the perspective of service providers. Our aim is to design online reservation policies which an SP can employ to maximize the performance of its service while not exceeding the average monetary budget it has committed for this purpose. This is a key step for unleashing the full potential of slicing markets. 

 
%\textbf{Related Work}.
%The design of slice markets is a relatively new research area. In \cite{toktam-icc17}, the authors proposed a mechanism for the NO to auction its sliced resources; \cite{paschos-infocom18} formulated slicing as a utility maximization problem; and  \cite{malandrino-infocom20} considered QoS metrics when serving the slices. Similarly, \cite{andres-conext18} studied the impact of slice overbooking; \cite{xavi-infocom20} employed predictive capacity allocation for improving the slice composition; and \cite{nguyen-jsac19} focused on dynamic slicing via reinforcement learning. Fewer works consider the problem from the SP's point of view. In \cite{zhang-tcom2018}, the authors consider a hybrid reservation/spot market where the SP reservations are decided by solving a stochastic problem, and a similar mixed time scale reservation model was studied in \cite{vincent-TVT2020}. Our previous work \cite{jb-icc20} employed demand and price predictions (via neural networks) to assist the SP's reservations, while \cite{tony-tnsm19} focused on slice reconfiguration costs. 

%\footnote{In cloud computing there are both reservation-based solutions and spot markets, e.g., see \cite{google-pricing} and \cite{google-spot}.}

%Similar reservation problems have been considered in the context of cloud computing, cf. \cite{carlee-sigcomm15}, using various reservation criteria, e.g., costs or task deadlines, \cite{carlee-infocom18}. However, all above reservation solutions make (often strong) assumptions about knowing the user demands and resource  prices, or assume that these parameters follow stationary stochastic processes. Our approach is fundamentally different, as we design reservation policies that do not require the SPs to know in advance their needs and the charged prices, nor we make any assumptions about the evolution of these parameters. In essence, we treat slice reservation as a learning problem for the SP, and rely on the theory of online convex optimization (OCO) \cite{hazan-book} to design algorithms that adapt the reservations to users' needs and the NO's (potentially arbitrary) pricing decisions. Our approach adapts recent  \emph{constrained} OCO algorithms, cf. \cite{giannakis-TSP17}, \cite{victor}, which are particularly robust and practical. 


% contributions
%\textbf{Contributions}. In detail, we consider a hybrid market with advance-reservation and spot-bidding options, where an SP can request resources from a network operator in the beginning of each period and update its reservation at each slot within every period. The NO is allowed to change arbitrarily both its reservation and spot prices, where the latter are made available to the SP only after the bidding is decided. Hence, in effect, the SP has to reserve slices without knowing the needs of its users or the total cost of its reservation. In this dynamic environment, the SP aims to maximize a general utility function of the slice resources, which reflects its service performance, while not violating an average budget constraint. 

%We formulate this process as a learning problem which allows the SP to implement a \emph{no-regret slice reservation policy}. This means that, as time evolves, the SP is guaranteed to achieve the same performance with that of an ideal benchmark policy that one could design only with hindsight, i.e., having access to all future demand and cost values. Our algorithm relies on a primal-dual online iteration \cite{giannakis-TSP17} which minimizes a Lagrangian function, and controls concurrently for  performance and budget costs. We extend our framework to allow the SP determine the \emph{composition} of its slices (as opposed only to its size), without even knowing what resource combination is performance-optimal. This is crucial when the qualitative features of user demand are volatile and/or unknown. And finally, we propose a mixed time-scale learning policy that exploits any price information that is revealed by the NO during each period, so as to improve the SP reservations.

%Our contributions can be thus summarized as follows:

%$\bullet$ We consider a general model, where the SP reserves slices in two time scales; determines the slice size and composition; and is oblivious to the user needs and NO prices.
	
%$\bullet$ We design an online learning framework for slice reservations, that ensures sample-path asymptotically-optimal performance while respecting the SP budget constraints.

 	
%$\bullet$ We perform a battery of numerical tests using stationary and non-stationary parameter patterns. The results verify the robustness and efficacy of our learning-based algorithms.

%paper organization
%The reminder of the paper is organized as follows. In section II, we review the related works. In section III, we present the system model and our formulation of the problem. We detail in section IV the online learning solution (OLR) and its performance analysis. We propose in section V and VI two extensions, namely the slice orchestration (OLR-SO) and the mixed time scale reservation models (OLR-MTS and OLR-SO-MTS). In section VII, we verify the guarantees of the OLR and OLR-SO solutions and show empirically the improvement brought by the mixed time scale models (OLR-MTS and OLR-SO-MTS) through our simulations. Lastly, we conclude our work in section VIII.
 

%notations
%\textbf{Notation}. We use bold typeface for vectors, $\bm a$, and vector transpose is denoted $\bm a^\top$. A sequence of vectors is denoted with braces, e.g., $\{\bm a_t\}$, and we use sub/superscripts to define a sequence of certain length, e.g., $\{\bm a_t\}_{t=1}^T$ is the sequence $\bm a_1, \bm a_2, \ldots, \bm a_T$. Sets are denoted with calligraphic capital letters, e.g., $\mathcal M$. The projection onto the non-negative orthant is denoted $[\cdot]_+$, and $\|\cdot\|$ is the $\ell_2$ norm.


%\textbf{Paper Organization}. The rest of this paper is organized as follows.  In Sec. \ref{sec:model} we introduce the system model and define the problem under consideration. Sec. \ref{sec:algo} presents the online reservation policy and proves its properties, and Sec. \ref{sec:extensions} discusses several generalizations of the model and problem. Finally, Sec. \ref{sec:sims} presents detailed numerical evaluations for a wealth of scenarios and we conclude in Sec. \ref{sec:conclusion}
