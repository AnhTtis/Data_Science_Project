\vspace{2mm}
\section{Model and Problem Statement} \label{sec:model}

%The goal is to prove that the SP has interest to request resources proactively, with the online learning policy that we propose, which relies on traffic demand and resource pricing information. We aim to show that our online optimization approach is smarter than heuristics or has lower complexity than smart policies being model- or parameter-dependent solutions, or data-driven solutions.

%We consider thus a system model with multiple tenants requesting resources to one NO. The tenants belong to the set $\mathcal{I}$, with cardinality $|\mathcal{I}|$ ($=10$ for instance). We consider in the first scenario where only one tenant $i$ opts for our approach, and we implement other approaches for the other tenants. We evaluate the benefits drawn for tenant $i$. Then, we consider a hybrid situation where half tenants are using our approach. We evaluate the benefits for those tenants and for the NO. Finally, we look at the case where all the tenants use the OLR approach, and how it can benefit the NO.

%\begin{figure}
%    \centering
%    \includegraphics[width=0.49\textwidth]{images/journalTNSMfig.pdf}
%    \caption{A Network Operator (NO) leases different types of resources, e.g., wireless capacity, storage capacity and edge computing capacity, to different types of Service Providers (SPs) that offer over-the-top services to their users.}
%    \label{fig:model}
%\end{figure}


%multi-resource allocation framework
\textbf{Network and Market Model}. The key parameters of our model and solution are summarized in table \ref{table:notation} below.
We consider a slotted system $\{1,\ldots,T\}$.
A Network Operator (NO) sells virtualized resources to the service provider (SP), and we denote with $\mathcal{H}$ the set of $m=|\mathcal{H}|$ types of resources that comprise each slice. For instance, $\mathcal{H}$ may include bandwidth capacity, backhaul link capacity, edge computing and storage resources ($m=4$). The SP can reserve multiple kinds of resources which orchestration will enable the operation of the slice. We denote the in-advance reservation and spot reservation decisions at slot $t$ respectively as $\bm x_t = [x_1, \ldots,x_m]_t^\top$ and $\bm y_t = [y_1, \ldots,y_m]_t^\top$.
The optimal mix of resources composing the slice is unknown to the SP, as it depends on the type of request the SP receives from its users. Moreover, the benefit from each resource can be time-varying, e.g. bandwidth capacity can change due to varying channel conditions. %The contributions of the different resources to the utility of the slice are unknown and might even vary over time. 
The benefit from reservation $\bm x_t$ ($\bm y_t$) is quantified by the scalar $\bm x_t^\top \bm \theta_t$ ($\bm y_t^\top \bm \theta_t$), where the items of $\bm \theta_t \in \mathbb R^m$ are the individual contributions of each resource on the performance at slot $t$.

%We represent such variations by multiplicative factors $\theta_i$ which weigh the reservation values $x_i$. Therefore, at slot $t$, the actual reservation of the SP is $\bm x_t^\top\bm \theta_t$, where $\bm \theta_t = [\theta_1,\ldots,\theta_m]_t^\top$.


%The SP can request a certain amount for each type of resource, which are then composing the end-to-end network slice dedicated for the users of the service. The SP must decide the reservation vector $\bm x = [x_1, \ldots,x_m]^\top$. Based on its reservation $\bm x$, the slice obtained by the SP will have a given capacity, which we assume to be a linear combination of the leased resources which contributions to the capacity are transcribed in the vector $\bm \theta = [\theta_1,\ldots,\theta_m]^\top$. Therefore, the SP obtains a slice of capacity $\bm x ^\top \bm \theta$. This assumption is general and can be found in related works such as \cite{fossatiACM} which considers a multi-resource allocation framework. Moreover, we make no assumptions about the contributions of the resources which we deem unknown and changing from slot to slot, potentially in a non-stationary manner. Therefore, at slot $t$ the slice capacity is equal to $\bm x_t ^\top \bm \theta_t$, where $\bm x_t$ is the reservation vector for slot $t$, and $\bm \theta_t$ is the contribution vector.


%slice utility ~ service performance function
The utility stemming from such reservation scheme is non-linear. We model the slice utility of the SP as an increasing concave function of the acquired resources by using the logarithm function. For instance, the paper \cite{paschos-infocom18} provides the general form of $\alpha$-fair utility functions:
\begin{align}
    f(\bm z) = \left\{
    \begin{array}{ll}
        \frac{\bm z^{1-\alpha}}{1-\alpha} & \alpha \neq 1 \\
        \log(\bm z) & \alpha = 1
    \end{array}
\right.
\end{align}
where $\bm z$ is the reservation vector of slices. In \cite{srikant}, the utility from allocating bandwidth $x$ to a certain network flow $f$ is modeled as $a_f\log(x_f)$, where $a_f$ is a problem (and flow)-specific parameter. %Other reservation-based papers relate to the same kind of convex/concave objective function to minimize/maximize \cite{malandrino-infocom20, zhang-tcom2018, paschos-infocom19}. 
The logarithm function allows us to model as well the diminishing returns which naturally arise with the over-reservation of the network resources. For instance, the data rate is a logarithmic function of the spectrum; the additional revenue of the SP from more slice resources is typically diminishing. We model the slice utility function as a logarithmic concave function, weighted by the SP demand $a_t$, i.e. $a_t\log(1+\bm \theta_t^\top(\bm x_t + \bm y_t))$. %The SP demand $a_t$ is unknown at the beginning of slot $t$ and only revealed at the end of the slot. %cite

%network resources prices, market operation
The market operates in a hybrid model. At the beginning of each slot, the SP can lease network resources, plus additional resources on a spot market. We denote with $\bm p_t = [p_1,\ldots,p_m]^\top_t \in \mathbb R_+^m$ the unit price of the network resources; and we denote with $\bm q_t=[q_1,\ldots,q_m]^\top_t \in \mathbb R_+^m$ the unit price of the resources available in the spot market. The SP reservation policy consists of the reservation decision $\bm x_t$ and the spot decision $\bm y_t$. At the beginning of each slot $t$, the SP decides its $t$-slot reservation plan $(\bm x_t, \bm y_t)$, and pays the price $\bm p_t^\top \bm x_t + \bm q_t^\top \bm y_t$ at the end of the slot.%At the time the SP makes a bid, both reservation and spot unit prices are unknown and are only revealed \emph{after} the SP makes its decision. Moreover, the prices dynamically change following non-stationary patterns, which force the SP to use a reservation decision model robust to the price uncertainty. As prices are revealed along time to the SP, the latter acquires historical traces which render possible the prediction of the next slot values.% We will detail how to use this information within our decision algorithm.

%limits of capacity for the operator
The NO can impose upper limits on the requests of the SP. For instance, the reservation request for resource $i$ must belong to the set $\Gamma_i = [0,D_i]$, where $D_i$ is the limit imposed by the NO on resource $i$. Therefore, the SP request will belong to $\Gamma_1 \times\ldots\times \Gamma_m$, which we denote $\Delta$. Such limitations arise from natural capacity constraints of the network, in charge of multiple services and its own needs. In some cases, the NO can be unable to fulfill the SP request, especially when the network is congested due to high users' demand load and heavy SPs requests. The NO must guarantee a certain Service Level Agreement (SLA), which we relate to the respect of a certain threshold ratio of the requested amount resource. For instance, the NO must deliver at least $\alpha=80\%$ of the desired capacity for the resource. We envision this scenario as an extension and assume from now that the NO must comply with the whole request if it belongs to $\Gamma_i$.

%We envision this scenario in the supplementary file \cite{JB}, with a slight change to the main problem statement.
%possibility of failure to fulfill the SP request -> new scenario
%At this point, we envision two very distinct cases. First, as long as the request falls in the constraint set, the NO will fulfill. Secondly, depending on the other SPs' requests, the NO will fairly provide each SP with a certain percentage of its request. The fairness rule used by the NO is out of the scope of this paper. 

%SP reservation policy, strategy
%We now elucidate the SP reservation policy, which consists of the $t$-slot reservation decision $\bm x_t$ and the spot decision $\bm y_t$. At the beginning of each slot $t$, the SP decides its $t$-slot reservation plan $(\bm x_t, \bm y_t)$, and pays the price $\bm p_t^\top \bm x_t + \bm q_t^\top \bm y_t$ at the end of the slot. The SP's goal is to maximize the performance of its service, while avoiding excessive monetary costs. 

%The service performance is quantified with a concave \emph{utility} function, increasing on the resources and modulated by parameter $a_t$, that captures the aggregated demand in slot $t$. Note that the demand is unknown to the SP, and is revealed before the decision of the next slot $(\bm x_{t+1}, \bm y_{t+1})$.



\textbf{Problem statement}.
Putting the above together, the ideal reservation slice policy is the solution of the following convex program:

\begin{align}
(\mathbb P):\quad \max_{ \{\bm x_t,\{\bm y_t\}\}_{t=1}^T } & \sum_{t=1}^T \Big(V  a_t \log((\bm x_t + \bm y_t)^\top \bm \theta_t + 1) \notag\\
&- (\bm p_t^\top \bm x_t +\bm q_t^\top \bm y_t) \Big) \label{prob-obj} \\
%\text{s.t.} \quad & \sum_{t=1}^T\Big(x_tp_t+\!\!\!\!\sum_{k=(t-1)K+1}^{tK}\!\!\!\!\!y_kq_k \Big) \leq B, \label{prob-const1} \\
\text{s.t.}\quad   \bm y_t\in &\Delta, \quad \forall t=1,\ldots,T,  \label{prob-const2} \\
	 \bm x_t \in &\Delta, \quad \forall t=1, \ldots, T. \label{prob-const3}
\end{align}

In Objective (\ref{prob-obj}), we recognize the weighted sum of the slice performance (logarithmic term) and the payments (linear term). The latter term has a minus sign as the SP seeks to minimize its monetary cost. We sum over the number of slots $T$, as the goal is to maximize this weighted sum in the long-term.
Constraints (\ref{prob-const2}) and (\ref{prob-const3}) ensure the decisions belong to the constraint convex set $\Delta$. We define the hyper-parameter $V\geq1$ which balance the influence between the two terms (utility term and cost term). The bigger $V$, the more we favor the slice utility in the detriment of the cost of reservation.

%this part for the subsection
%If we envision the case where the SP only obtains part of its request, then we replace $\bm x_t$ by $A_t \bm x_t$ and $\bm y_t$ by $B_t \bm y_t$, where $A_t$ and $B_t$ are $m\times m$ diagonal matrices. If the SP obtains all its request, then $A_t = B_t = 1\!\!1_m$ (identity matrix).

$(\mathbb P)$ is a convex optimization problem but cannot be tackled directly due to the following challenges:
\begin{itemize}
    \item the users' demand $\{a_t\}$ is unknown, time-varying and non-stationary;
    \item the unit prices $\{\bm q_t\}$ and $\{\bm p_t\}$, are unknown, time-varying and non-stationary;
\end{itemize}

Due to these challenges, the convex problem $(\mathbb P)$ cannot be solved at $t=1$ for the next $T$ slots. Henceforth we define the loss function, at each slot $t$:
\begin{align}
    f_t(\bm x_t, \bm y_t) = - V a_t \log((\bm x_t + \bm y_t)^\top \bm \theta_t + 1)\\
    + (\bm p_t^\top \bm x_t +\bm q_t^\top \bm y_t)
\end{align}
The function $f_t$ is convex which allows us to use the OCO framework. Our goal is to decide at each slot $t$ the reservation plan $\bm z_t = (\bm x_t, \bm y_t)$ and achieve in the long term a sublinear static regret as defined in \eqref{eq:static_regret}.

\begin{table}
\caption{Key parameters and variables}
\scriptsize
	\centering%
	\begin{tabular}{|c|c|}
		\hline %
		\hline
		Symbol & Physical Meaning\\
            \hline %
		$m$ & Number of network resources composing a slice\\
            \hline
		$\bm x_t$ & Reservation in advance market in slot $t$ \\
            \hline
            $\bm y_t$ & Reservation in spot market in slot $t$ \\
		\hline %
            $\bm \theta_t$ & Contribution vector in slot $t$ \\
            \hline
		$a_t$ & User needs for the SP service in slot $t$\\
		\hline
		$\bm p_t$ & Unit price vector of the network resources at $t$ \\
		\hline
		$\bm q_t$ & Spot price vector for slot $t$\\
		\hline
		$T$ & Number of slots/horizon\\
		\hline
		  $D_i$ & Upper-bound imposed by the NO for reservation of resource $i$ \\
		\hline
		$\Gamma_i$ & $\Gamma=[0,D_i]$, feasible set for reservation of resource $i$\\
		\hline %
		$\Delta$ & Compact convex set $\Gamma_1\times\ldots\times\Gamma_m$ \\
            \hline
            $D$ & Diameter of $\Delta$ \\
		\hline
		$V$ & Calibration parameter \\
            \hline
            $\sigma$ & Regularization parameter, best choice  $\sigma=\sqrt{2}/D$ \\
            \hline
            $\grad \hat{f}_{t+1}(\hat{\boldsymbol{z}}_{t+1})$ & Gradient prediction known at $t$ \\
            \hline
            $\zeta$ & Prediction model average relative error rate \\
            \hline
            $\alpha$ & Minimum ratio the NO must provide for advance resources \\
            \hline
            $\beta$ & Minimum ratio the NO must provide for spot resources \\
%		\hline
%		$\Gamma_1, \ldots ,\Gamma_i, \ldots ,\Gamma_m$ & $\forall i \in [1,m], \Gamma_i=[0,D_i]$, feasible set for reserved instance $i$\\
		\hline %
		%$??$ & \gi{other variables that we would like to put here?}\\
		%\hline
		\hline
	\end{tabular}
	\label{table:notation} % is used to refer this table in the text
\end{table}

%Our goal is to design the online reservation policy $\{\bm z_t\}_{t=1}^T$, where we denote $\bm z_t = (\bm x_t, \bm y_t)\quad \forall t$, able to give sublinear regret guarantees against an optimal policy. We define the latter in the next subsection.



%\subsection{Estimation step}
%The OOLR algorithm requires estimating the next gradient function and the next point.
%Thus, we propose two methods to estimate the next point $\hat{\bm z}_{t+1}$: either we take $\hat{\bm z}_{t+1} = \bm z_t$, i.e. the previous reservation decision; or we use another optimization problem.
%The former case is relevant if the optimal solution to \eqref{eq:stepOOLR} stays relatively similar from one slot to the other. As $\hat{\bm z_t} = \bm z_{t-1}$ and $\bm z_{t-1} \sim \bm z_{t}$, the quadratic error on the gradient as defined in \eqref{eq:quadratic_error} will be relatively low.

%The latter case implies to make a decision for the estimation of the next point. We aim to solve a similar problem to the one defined in \eqref{eq:stepOOLR}, knowing that we cannot use the predicted gradient term, as it is precisely what we are deciding by estimating the next point. Therefore, we formulate the estimation step of our algorithm as:

%\begin{align}
 %   \hat{\bm z}_{t+1} = \arg \min_{\bm z \in \Delta^2} \Big\{ r_{1:t}(\bm z) +
  %  \Big(\sum_{s=1}^t \grad f_s(\bm z_s)\Big)\top \bm z\Big\} \label{eq:stepEstim}
%\end{align}

%As one can read in \eqref{eq:stepEstim}, we only use the accumulated past information to estimate the next point value, before making the actual decision in \eqref{eq:stepOOLR}. This estimation step is the classical Follow-the-Regularized-Leader method which ensures a $\mathcal{O}(\sqrt{T})$ regret bound against the static benchmark. %In volatile settings, we expect with this method a lesser accumulated quadratic error of the predictions, thus enhancing the regret performance.



%We must predict $\tilde c_t$, hence $\tilde a_t$, $\tilde q_t = [\tilde q_t^1,\ldots,\tilde q_t^m]^\top$, $\tilde \theta_t = [\tilde \theta_t^1,\ldots,\tilde \theta_t^m]^\top$, which correspond to the demand, the spot price of the $m$ resources, the contribution of the $m$ resources.

%Let's consider first the SP's demand $a_t$. We assume this information is revealed to the SP after each decision, hence the SP has access to an history of past data of this variable. We can apply a learning algorithm that performs -in average- no worse than the best ARMA model in hindsight \cite{anava} (i.e. with full knowledge of the future demand $a_t$).
%We assume the demand of our SP is not correlated with the spot price, as there is a sufficiently large number of SPs leasing resources, therefore the impact of only one SP on the spot price is negligible.% However, the ARMA comparator is reliable only if the trace $a_t$ presents the stationary feature (statistics of the trace do not vary with time $t$). It is possible that the demand might be non-stationary, depending on underlying hidden factors that cause drastic changes to the demand. In this case, we need another prediction model for $a_t$ than the one in \cite{anava}.

%Now let's consider the contribution vector $\bm \theta_t$, and let assume it does not vary (or very little) over time: $\bm \theta_t \simeq \bm \theta$. This is a realistic assumption, as the resource contribution should be relatively similar from one slot to the other. After a sufficient number $T$ of observed output $(\bm x_t + \bm y_t)^\top \bm \theta$, we can find $\bm \theta$ such that it is close enough to the actual contribution vector. We can use either a one-layer perceptron, a linear regression, or an online gradient descent which would lead to a $\mathcal{O}(\sqrt{T})$ accumulated error.

%Again, the spot price depends on latent factors such as the other SPs' demand, the NO availability, which are hidden to our SP. In \cite{carlee-infocom18}, they assume the underlying dynamics follow a time-invariant exponential distribution, and the bids in the spot market are uniformly distributed. Using the EM algorithm, they are able to predict the spot prices and subsequently to find the ideal spot instance to query for.

\begin{comment}
\subsection{Static Benchmark}

The efficacy of our learning policy is mainly characterized by the benchmark to which we compare it (i.e. there exists different kind of benchmark, which are more or less stringent); and by the convergence rate of the learning policy loss relative to the chosen benchmark. 

In our problem, we opt for a static benchmark, that consists of a unique reservation vector $\bm z^*$, optimally chosen over the period of evaluation, to minimize the loss.

More precisely, the optimal solution $\bm z^*$ is defined as:
\begin{align}
    \bm z^* = \arg \min_{\{\bm x, \bm y \in \Delta\}} \sum_{t=1}^T f_t(\bm x, \bm y),
\end{align}
where the period of evaluation includes the slots $t=1,\ldots ,T$.

The benchmark has the pros to know \emph{a priori} all the future prices $\{\bm p_t\}_{t=1}^T$, $\{\bm q_t\}_{t=1}^T$, the future demand $\{a_t\}_{t=1}^T$ and the future contribution vectors $\{\bm \theta_t\}_{t=1}^T$. It has the cons to reserve a unique vector, which is ideal \emph{on average} (over the period of evaluation), but \emph{sub-optimal} when considering the slots separately. Therefore, our algorithm can possibly outperform the benchmark, synonym of a negative regret. 

Given that in practice, the information of the benchmark is unavailable, our goal is to design an algorithm that finds the reservation vector $\bm z_t = (\bm x_t, \bm y_t)$ at each slot $t$, such that the overall loss achieved (for $t=1,\ldots ,T$) is of the same order of the overall loss achieved by $\bm z^*$. Formally, we define the \emph{static regret} as:
\begin{align}
    R(T) = \sum_{t=1}^T \Big(f_t(\bm z_t) - f_t(\bm z^*)\Big),
\end{align}
and the goal is to have $R(T)/T \rightarrow 0$ when $T\rightarrow \infty$.
\end{comment}

\begin{comment}
\subsection{Calibration of $\alpha$}

The vector function $f_t$ is a weighted sum of two terms, the utility term $-a_t \log((\bm x + \bm y)^\top\theta_t +1)$, which is convex with respect to $\bm z = (\bm x, \bm y)$; and the price consumption term, which is affine hence convex with respect to $\bm z$. With parameter $\alpha$, we can adjust the relative influence of the two terms, depending on the SP preferences.
For instance, if the SP wishes to have both quantities of the same order:
\begin{align}
   \alpha a_t\log(1+(\bm x + \bm y)^\top \bm \theta_t) \sim (1-\alpha) \bm p_t^\top \bm x_t + \bm q_t^\top \bm y_t, \notag
\end{align}

then $\alpha=1/(1+r)$, with:
\begin{align}
    r = \frac{a_t\log(1+(\bm x + \bm y)^\top \bm \theta_t)}{ p_t^\top \bm x_t + \bm q_t^\top \bm y_t}. \notag
\end{align}

We remark that the ratio $r$ of the two quantities depends on the slot $t$. However, if we consider a sufficient number of slots and take the average ratio $r$, then the calibration of $\alpha$ will be optimal \emph{on average}. We can think of it as the law of large numbers.
If the SP has a factor of preference $s$ for the utility (for instance $s=2$), then:
\begin{align}
   \alpha a_t\log(1+(\bm x + \bm y)^\top \bm \theta_t) \sim s(1-\alpha) \bm p_t^\top \bm x_t + \bm q_t^\top \bm y_t, \notag
\end{align}
then $\alpha = 1/(1+r)$, with:
\begin{align}
    r = \frac{a_t\log(1+(\bm x + \bm y)^\top \bm \theta_t)}{ s \bm p_t^\top \bm x_t + \bm q_t^\top \bm y_t}. \notag
\end{align}

\end{comment}





