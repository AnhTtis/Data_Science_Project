\section{Related Work}


%\subsection{Adaptive Online Learning}

%In \cite{mcmahan}, \cite{mohri},%, the authors raised a comprehensive survey of the online optimization methods using adaptive regularizers based on data. In the context of OCO, the survey takes a census of the regret bounds of the FTRL, primal-dual, mirror descent families of algorithm.


%paper using prediction
By anticipating the resource utilization, the NOs can enhance their resource management decisions regarding resource provisioning or allocation.
In \cite{DeepCog}, network traffic information is leveraged to plan the capacity needed for each slice in a multi-tenant framework. Using a data-driven approach including C-RAN, MEC and core networks, the solution outperforms other state-of-the-art deep learning solutions \cite{infocom17}, \cite{mobihoc18}. The approach in \cite{oliveiraTNSM} employed an adaptive forecasting model of the elastic demand for network resources to perform slice allocation in Internet Access Services. The authors in \cite{XaviINFOCOM17} and \cite{XaviTrans19} predicted the required resources by tenants for the future time window to perform slice requests admission and schedule the users' traffic within each slice. %The prediction module can adapt its predictions by changing a forecasting error probability according to the feedback sent by the slice scheduler. 
In \cite{cui-iccc20} cellular traffic prediction helps the allocation policy for the vehicular network slice.
Unlike these approaches, our solution does not need offline training and provides performance guarantees against all types of traces.

Fewer works consider the SP resource provisioning problem.
\cite{monteil-icc20} uses historical traffic to design the SP resource reservation policy. The paper \cite{reyhanian} developed a two-time scale approach for the activation and the re-configuration of the slices while considering the reservation of both RAN and backhaul resources. \cite{zhang-tcom2018} focuses on wireless spectrum considering two reservation schemes (in advance and on demand). In \cite{vincent-TVT2020}, the authors develop a two-stage approach for the resource reservation and the intra-slice resource allocation. These works presume a stationary environment where user statistics do not change and/or cost of resources are supposed constant. %The paper \cite{zhang2017} considers as well a mixed-time scale reservation model of long-term and short-term VM instances then used for VNF deployment. The authors propose three solutions for the VNF demand prediction, the short-term and long-term reservations of VMs. In \cite{paschos-infocom19}, the authors consider a reservation model, where a customer at each slot requests different types of cloud resources aiming to minimize a fixed and known cost function while satisfying a time-average unknown demand constraint. This model, however, is not suitable for the considered hybrid markets where the prices are volatile and hence the cost functions change dynamically.


%\subsection{Reservation-based network slicing}
%important reservation-based network slicing papers (not necessary)
%\cite{paschos-infocom18} studied the problem of allocating network and computing resources to a set of slices in order to maximize a system-wide utility function, namely to enforce fair resource allocation across slices.
%In \cite{malandrino-infocom20} the authors proposed a static optimization framework for embedding VNF chains (interpreted as slices) in a shared network. Their key contribution is the formulated problem which accounts for reliability, delay and other slice requirements. Albeit detailed and rigorous, this analysis considers the various system parameters and requests to be known. Similarly, \cite{andres-conext18} studied the impact of slice overbooking; \cite{xavi-infocom20} employed predictive capacity allocation for improving the slice composition; and \cite{nguyen-jsac19} focused on dynamic slicing via reinforcement learning.

%In \cite{nguyen-icc19}, the authors model the slice requests as a SMDP, with the network provider deciding whether to admit or not the new slice request. The solution consists of a deep Q-learning algorithm. The authors consider $3$ classes of slice in their simulations and Poisson arrival rates for the requests of such slices.
%The paper \cite{XaviInfocom19} falls under the scope of the Slice-as-a-Service framework to support on-demand slice requests that a set of tenants issue to one MNO (Mobile Network Operator).
%The MNO steers the slice requests towards a multi-queue system, where each queue stacks the bids for one specific slice type. The MNO can prioritize one queue based on its preferences.
%The model offers the possibility to choose one type of slice request over another, based on the MNO preferences.
 
%In \cite{zhang-tcom2018} and \cite{vincent-TVT20200}, the authors study a hybrid reservation and spot market where the SP reservations are decided by solving a stochastic problem.

%The authors then solve a stochastic convex problem to decide the SP reservations.







\begin{comment}
\subsection{Resource Allocation Policy}

The paper\cite{paschos-infocom18} studied the problem of allocating network and cloud computing resources in a set of slices in order to maximize a system utility function (achieving fair allocation across slices). Namely, there is a network $G=(N,L)$ and a set of slices $\mathcal S={1,2,\ldots, S}$, with $x_{sp}\geq0$ denoting the allocated rate to slice $s$ over path $p$, and $y_{sn}\geq 0$ is the amount of data processed for slice $s$ at in-network node $n$ (can be in the path or at the cloud). Assuming a linear relation between the traffic load and the computation load \cite{lee-ietf2016}, the resource allocation decisions can be succinctly given by variables $z_{snp}\geq 0$, which is the amount of demand from slice $s$ that is routed over path $p$ and processed at node $n$. The authors propose a static optimization problem where the objective is some type of convex function (a-fair functions) of the allocated rates to the different slices, and then they use consistency-pricing (or ADMM) which allows the different entities: slice requesters, network operator, cloud provider, to agree on the system operation.

\cite{fossatiACM} proposes a new multi-resource allocation framework that catches the inter-dependency of different resources (e.g. computing resource to traffic bit-rate dependency) to avoid wasting resources in a real-world setting, where some resources are congested (highly demanded) while others are not.
When compared to existing solutions that consider single-resource allocation without catching the dependencies between resources, the proposed solution shows huge savings of resources.

\cite{XaviACM19}




\subsection{Slice Admission policy}

The paper \cite{XaviInfocom19} falls under the scope of the Slice-as-a-Service framework to support on-demand slice requests that a set of tenants issue to one MNO (Mobile Network Operator).
The MNO steers the slice requests towards a multi-queue system, where each queue stacks the bids for one specific slice type.
The model offers the possibility to choose one type of slice request over another, based on the MNO preferences.
\cite{xavi-trans20} is the extension of \cite{XaviInfocom19}.




The authors in \cite{XaviTrans19} build their admission control solution upon a forecasting module of the required resources by tenants for the future time window.  %Traffic forecasting relies on the following assumptions: the demand is seasonal, hence the Holt-Winter forecasting model is relevant; the authors use the SLAW model to describe the user mobility; to design the tenant spatial distribution, they rely on the probabilistic latent variable model. 
The slice controller uses the traffic information to decide whether to admit new slice requests by solving a geometrical knapsack problem. After a thorough complexity analysis, the authors find the problem NP-hard and provide a heuristic solution with a satisfying performance ratio.
The slice scheduler minimizes the consumed resources within the slice while complying with SLAs requirements. The forecasting module can adapt its predictions by changing the forecasting error probability, according to the feedback sent by the slice scheduler.

In \cite{andres-conext18}, knapsack admission control.

In \cite{sciancalepore2018onets}, the tenants issue one slice request with the amount of resources and the time duration specified. The Network Slice Broker decides whether to admit new requests, ensuring that SLAs are being met. The authors formulate a Multi-Armed Bandit problem, taking account of the limited resource budget and the lock-up periods of tenants. Then, they design three online solutions, with two regret upper bounds, and finally give a proof-of-concept. The proof of regret for the ONETS algorithm relies on the assumption of i.i.d. exponential inter-arrival time of the requests.


In \cite{Raza19}, the InP can accept two kinds of slice, with strict latency constraints or non-strict latency constraints. The former has a higher revenue/penalty than the latter, which means it can lead to higher profit if the slice is setup properly, or to higher penalty, if the slice cannot be accommodated due to a lack of resources, at the radio, transport or cloud controller. The authors design a reinforcement learning solution, where an ANN model is trained to minimize the loss due to slice rejection and the loss caused by service degradation (when available resources are not sufficient to accommodate the slice). The solution shows good results when compared to baselines present in the literature, and the authors conduct sensitivity analysis, against the slice degradation penalty versus slice revenue factor, and the proportion of slice type.

In \cite{nguyen-icc19}, the authors model the arriving slice requests as a SMDP, with the network provider deciding whether to admit or not the new slice request. The solution consists of a deep Q-learning algorithm. The authors consider $3$ classes of slice in their simulations and Poisson arrival rates for the requests of such slices.

Most papers we came across deal with a finite number of slice types/classes, leading to Integer Linear Programming problems. In our case, we assume the SP can request any amount of resources that belongs to the NO capacity. Therefore, there exists an infinity of slice request types, as there exists an infinity of real values on $[0,\Delta]$ and so on. A multi-queue system for admission control cannot work unless we move backward and change the resource reservation system of the SP, limiting its requests to a discrete and finite set of values. Therefore, we prefer the Knapsack formulation.






\subsection{Reservation of virtualized resources}


The paper \cite{reyhanian} uses a two-time scale approach for the activation and the re-configuration of the slices while considering the reservation of both RAN and backhaul resources.

In \cite{DeepCog}, the authors leverage network traffic information to plan the capacity needed by each slice. Using a data-driven approach including C-RAN, MEC and core networks, their solution outperforms other state-of-the-art deep learning solutions \cite{infocom17}, \cite{mobihoc18}. 

The paper \cite{paschos-infocom18} studied the problem of allocating network and computing resources to a set of slices in order to maximize a system-wide utility function, namely to enforce fair resource allocation across slices.

In \cite{malandrino-infocom20} the authors proposed a static optimization framework for embedding VNF chains (interpreted as slices) in a shared network. Their key contribution is the formulated problem which accounts for reliability, delay and other slice requirements. Albeit detailed and rigorous, this analysis considers the various system parameters and requests to be known. Similarly, \cite{andres-conext18} studied the impact of slice overbooking; \cite{xavi-infocom20} employed predictive capacity allocation for improving the slice composition; and \cite{nguyen-jsac19} focused on dynamic slicing via reinforcement learning. Unlike these works, we make no assumptions regarding the availability or the statistical properties of the prices and user needs.



Fewer works consider the problem from the SP point of view. In \cite{zhang-tcom2018} and \cite{vincent-TVT2020}, the authors study a hybrid reservation and spot market where the SP reservations are decided by solving a stochastic problem. This, however, presumes a stationary environment, an assumption that is likely to fail when multiple SPs bid strategically and the NO adapts the prices accordingly. Our previous work \cite{jb-icc20} employed demand and price predictions (via neural networks) to assist the SP reservations; while \cite{tony-tnsm19} focused on slice reconfiguration costs. In \cite{paschos-infocom19}, the authors consider a reservation model, where a customer at each slot requests different types of cloud resources aiming to minimize a fixed and known cost function while satisfying a time-average unknown demand constraint. This model, however, is not suitable for the considered hybrid markets where the prices are volatile and hence the cost functions change dynamically.

\cite{DeepCog} uses deep learning to forecast capacity in a sliced network. The AI solution is a mix of 3-D CNN and MLP architectures, takes as input traffic snapshots at various base stations for one specific slice and outputs the desired capacity at the various data centers for the very same slice. The authors design a loss function able to leverage between SLA violations due to under-provisioning (under-estimation of the optimal capacity values at the data-centers) and unnecessary costs due to over-provisioning (over-estimation). Using a data-driven approach including C-RAN, MEC and core networks, the authors outperform other state-of-the-art deep learning solutions \cite{infocom17}, \cite{mobihoc18}. 

\subsection{Slicing Markets}

The design of markets for virtualized network resources is a relatively new research area. The survey in \cite{hossain-survey2018} provides an overview of auction theory-based slicing solutions and \cite{toktam-icc17} proposed mechanisms for the NO to auction its sliced resources. Unlike these works, we consider a dynamic pricing scheme that  is more practical as it does not require to run any type of auction. Importantly, our model is based on already-deployed and widely-used market models in cloud computing ecosystems, e.g., see \cite{google-spot, amazon-spot}.

Prior works that focus on such hybrid cloud market models have studied spot pricing models and devised intelligent bidding strategies for the buyers \cite{zafer-cloud2012, lumpe-ccgrid17, sharma-hotcloud2016, carlee-infocom18, carlee-sigcomm15 }. For instance, in \cite{carlee-infocom18} the users place bids to reserve cloud resources for executing certain long tasks, aiming to minimize their costs while ensuring task completion over successive bidding periods. The main idea  is to employ a hidden Markov model for tracking the evolution of spot prices; however, the analysis relies on the user needs complying to certain statistical assumptions. Similarly, in \cite{carlee-sigcomm15} an interesting bidding approach is considered where the users try to infer the pricing strategy of the cloud provider and bid accordingly in a spot market. Our work differs in that we make no assumption for the spot pricing model of the operator, and our reservation algorithm offers performance guarantees for any possible pricing scheme and demand pattern. This is crucial as in practice the operator might as well revise and adapt its pricing policy in the presence of strategic bidders.



\cite{Pla2021}, \cite{fisher}, 

\cite{Lieto19}

\subsection{Traffic slice scheduling}
\cite{XaviTrans19}, \cite{XaviINFOCOM17}



\subsection{Reservation of resources}

%\cite{paschos-infocom18} formulated slicing as a utility maximization problem;
The paper \cite{paschos-infocom18} studied the problem of allocating network and cloud computing resources in a set of slices in order to maximize a system utility function (achieving fair allocation across slices).
\cite{malandrino-infocom20} proposed a static optimization framework for embedding VNF chains (perceived as slices) in a shared network. The key contribution in this work is the formulated problem that accounts for reliability, delay, and other requirements of each slice. Albeit detailed and rigorous, this analysis has the key limitation, compared to our work, that it considers all system parameters and requests to be known.
%and  \cite{malandrino-infocom20} considered QoS metrics when serving the slices. 
Similarly, \cite{andres-conext18} studied the impact of slice overbooking; \cite{xavi-infocom20} employed predictive capacity allocation for improving the slice composition; and \cite{nguyen-jsac19} focused on dynamic slicing via reinforcement learning.

Fewer works consider the problem from the SP's point of view. In \cite{zhang-tcom2018}, the authors consider a hybrid reservation/spot market where the SP reservations are decided by solving a stochastic problem, and a similar mixed time scale reservation model was studied in \cite{vincent-TVT2020}. Our previous work \cite{jb-icc20} employed demand and price predictions (via neural networks) to assist the SP's reservations, while \cite{tony-tnsm19} focused on slice reconfiguration costs.
In \cite{paschos-infocom19}, the authors consider a reservation model, where a customer at each slot requests different types of cloud resources to minimize its total cost while satisfying the average constraint on the demand. The authors propose a primal-dual type of algorithm and their online policy is asymptotically feasible and achieves $o(T)$ regret against the $K$-slot static benchmark (static reservation that only satisfies the average constraint every $K$ slots), with $K=o(T^{1-\epsilon})$. Unlike our approach, the costs in the objective are known and fixed.
In \cite{zhang2017}, the authors design a reservation model of long-term and short-term VM instances, then used for VNF deployment, and create a complete online solution - composed of VNF demand prediction and reservation decisions - with performance guarantees.


\subsection{Slicing markets}



The design of slice markets is a relatively new research area. \cite{hossain-survey2018} provides an overview of auction theory and auction theory-based slicing solutions. In \cite{toktam-icc17}, the authors proposed a mechanism for the NO to auction its sliced resources. Unlike these works, we consider a dynamic pricing scheme, that does not require to organize and run any type of auction.%; hence it is more applicable, and follows similar solutions that have been successfully applied in cloud computing ecosystems. 

Prior works have studied spot pricing models, trying to infer the employed mechanism (e.g., by Amazon) and optimize accordingly the bidding strategy from the buyer's perspective, see \cite{carlee-sigcomm15, carlee-infocom18} and references therein. And there are various biding optimization criteria, e.g., considering costs, task dependency, or job deadlines; see overview in \cite{carlee-infocom18}. The work \cite{carlee-sigcomm15} considered only the spot market, while \cite{carlee-infocom18} analyzed a hybrid on-demand and spot market. Our work differs in that we make no assumption for the spot pricing model of the operator, and our reservation algorithm offers performance guarantees for any possible pricing scheme and resource availability pattern.



%Similar reservation problems have been considered in the context of cloud computing, cf. \cite{carlee-sigcomm15}, using various reservation criteria, e.g., costs or task deadlines, \cite{carlee-infocom18}, \cite{paschos-infocom19}.


%In \cite{carlee-infocom18}, users place bids to access cloud resources for job completion. The aim of a user is to select the resource instance at each slot which minimizes the expected payment needed to complete the job, leveraging future spot price predictions. The authors develop a hidden Markov model that views the spot price as a stochastic function of latent states, which depend on the arrival and departure rates of jobs in the bidding queue. Although the authors achieve dramatic cost reductions when compared to auto-regressive prediction models, their model assume arrival and departure rates are independent, time-invariant, exponential random variables.

%However, all above reservation solutions make (often strong) assumptions about knowing the user demands and resource prices, or assume that these parameters follow stationary stochastic processes. Our approach is fundamentally different, as we design reservation policies that do not require the SPs to know in advance their needs and the charged prices, nor we make any assumptions about the evolution of these parameters.

\subsection{Online optimization}

% ------------ Introduction content
%OCO is an efficient mathematical framework introduced by Zinkevich \cite{zinkevich}, to evaluate the performance -through the regret metric- of online optimization algorithms \cite{koppel, cao2018online, giannakis-TSP17}, i.e. algorithms that solve an optimization problem in an online manner, at each time $t \in \mathcal{T}=\{1,\ldots,T\}$, assuming we have a time-slotted system. 
%In online learning and more specifically in OCO problems \cite{shalev-book, hazan-book}, an agent has to take sequential decisions at each time $t$, that minimizes the objective, while satisfying the constraints (if the problem is constrained). The agent does not have access to the time-varying objective function $f_t(.)$ prior to the decision (nor the constraints). Thus, the difficulty arises in this situation as the agent must take a decision under uncertainty. The design of an efficient algorithm, able to adapt to the dynamics of the problem, becomes essential. The regret metric reflects the efficiency of the algorithm: it measures the difference of the cumulative loss incurred by our decisions -over $T$ time slots- from the cumulative loss that experiences the benchmark, which has full hindsight of the problem. A sublinear regret, i.e. $R_T=o(T)$, is regarded as a good performance, as it implies that the online algorithm asymptotically performs no worse than the benchmark. In constrained OCO problems, we deem both the regret metric and the constraint violations metric -or fit- which is the accumulation of violations through the duration $T$. We observe a natural trade-off between the two metrics and our aim is to design algorithms that keep both a sublinear regret and a sublinear fit.
% -----------------------------------

Part of our contributions is to design an online algorithm that achieves \emph{zero-regret} policy. Thus, we propose a summary of papers in the field of constrained OCO with similar guarantees on the regret and fit metrics \cite{mahdavi-jmlr2012, jennaton, koppel, giannakis-TSP17, cao2018online, liakopoulos2019cautious, victor, johansson}.

Mahdavi \emph{et al.} \cite{mahdavi-jmlr2012} solve the problem of finding an online policy which satisfies the long-term constraints $\forall i, \sum_t g_i(x_t)\leq0$ and compare it to the static benchmark sequence. Their proposed primal-dual method achieves a $\mathcal{O}(T^{1/2})$ static regret and $\mathcal{O}(T^{3/4})$ violation for each constraint $i$. Jennaton \emph{et al.} \cite{jennaton} essentially solve the same problem with more elaborate regularizer and steps and improve the results of \cite{mahdavi-jmlr2012}. The authors in \cite{koppel} design a distributed saddle-point algorithm -with primal and dual decisions- that achieves sublinear regret ($\mathcal{O}(\sqrt{T})$), against the static benchmark. In \cite{liakopoulos2019cautious}, the authors consider both time-varying objective and constraints functions and employ a Lagrangian method to achieve sublinear regret and constraint violation against a family of static benchmarks. In \cite{victor}, the authors consider the same problem and with a mirror-based method achieve sublinear regret and constraint violation against tighter comparators.

All these papers include fixed or time-varying constraints and compare their online policies to the static benchmark sequence. In our work, we aim to compare our policies to the dynamic benchmark sequence. From the papers \cite{cao2018online, johansson} we observe that it is possible to achieve both sublinear dynamic regret and fit, given sublinear accumulated variations of the dynamic benchmark sequence.
In \cite{giannakis-TSP17}, both objective $f_t$ and constraint $g_t$ functions are time-varying and revealed after the primal decision $\bm x_t$. The proposed method, a modified saddle-point algorithm, achieves sublinear dynamic regret and fit, given sublinear accumulated variations of the dynamic benchmark sequence and of the constraints.
In \cite{cao2018online}, the authors adapt \cite{giannakis-TSP17} to the bandit problem and achieve better bounds. In \cite{johansson}, the authors derive guarantees on the dynamic and static regret, with diverse sets of assumptions (with or without strongly-convex objective, Slater's condition).

%All these algorithms rely on primal-dual optimization, as the dual decision can be taken after receiving the feedback of the objective function $f_t(.)$ and of the constraint function $g_t(.)$ (or after receiving the bandit feedback).
\end{comment}

