\section*{Appendix} 

We detail how we obtain the theoretical regret bound of the OOLRgrad solution.
First, we apply the ARMA-OGD on each gradient item and we obtain the regret bound for item $i$:
$$\sum_{t=1}^T (\grad_i f_t(\bm z_t) - \grad_i \tilde{f}_t(\bm \tilde{z}_t))^2 = Res + \mathcal{O}(GM\sqrt{T}),$$
where $Res$ is the residual error of the best ARMA prediction method with full hindsight, $G$ and $M$ are key constant in \cite{anava}.
Thus, we obtain the regret bound: $$\sum_{t=1}^T ||\grad f_t(\bm z_t) - \grad \tilde{f}_t(\bm \tilde{z}_t)||^2 = 2m Res + \mathcal{O}(2mGM\sqrt{T}).$$
Thus we rewrite the regret bound in \eqref{eq:regret_bound2} as:
$$R(T) = \mathcal{O}\Big(2D\sqrt{2T}\sqrt{2mRes+\mathcal{O}(2mGM\sqrt{T})}\Big).$$
Within the square root, we neglect $2mRes$ in front of $\mathcal{O}(2mGM\sqrt{T})$, hence we can simplify: $$R(T)=\mathcal{O}(4D\sqrt{mGM} T^{3/4}) = \mathcal{O}(T^{3/4}).$$

%Now we detail the expression of $G$:
%$$G = \sum_{i=1}^m \max^2\{Va\theta_i,p_i\}+\sum_{i=1}^m \max^2\{Va\theta_i,q_i\},$$
%where $a$, $\theta_i$, $p_i$ and $q_i$ are the upper-bounds on the different signals. $M$ comes directly from \cite{anava} and represents the maximum norm of the difference between two feasible lag vectors (vectors that contains the lag coefficients which generate the prediction). 








\begin{comment}
The expressions for $V_T$ and $R_T$ follow from Theorems 1 and  2 of \cite{giannakis-TSP17}, respectively, and the contribution of the Lemma is to quantify the different parameters and the upper bound of the dual variables, which is a key step in the analysis. 

First, note that $\|f_t(x_t, y_k)\|\leq  G\triangleq a\sqrt{K(K+1)}$, for all $t$ and $x_t, y_k\in \Gamma$. Moreover, recall that we showed in \eqref{eq:constraint-bound}  that the upper bound of the constraint is $|g_t(\bm z)|\leq \max\left\{ D(p+Kq)-B, B\right\}$. Replacing these quantities in \cite{giannakis-TSP17} we obtain the respective bounds of Lemma 1.

Now, let us focus on $\tilde \ll$. We first define $\Delta(\lambda_t) := (\lambda_{t+1}^2-\lambda_t^2)/2 $, and use  \cite[Lemma 1]{giannakis-TSP17} to upper bounded as follows:
\begin{align}
\Delta(\ll_{t+1}) &\leq \mu\ll_{t+1}(U_g - \epsilon) \notag \\
&+\mu \left(2a\sqrt{K(K+1)}D + \frac{D^2}{2\nu} + \frac{\mu M^2}{2}\right)
\end{align}
Next, we proceed to prove by contradiction the upper bound on $\lambda_t$. Let us assume that $t+2$ is the first period for which the dual upper bound $\tilde{\ll}$ does not hold. Therefore,
\begin{flalign}
\begin{aligned}
{\ll_{t+1}}\leq  \mu M \!+\! \frac{2a\sqrt{K(K+1)}D+D^2/(2\nu)+(\mu M^2)/2}{\epsilon-U_g} ,\notag
\end{aligned}
\end{flalign}
\begin{flalign}
\begin{aligned}
\text{and}\,\,{\ll_{t+2}} \!>\!  \mu M \!+ \frac{2a\sqrt{K(K\!+1)}D\!+D^2/(2\nu)+(\mu M^2)/2}{\epsilon-U_g}.\notag
\end{aligned}
\end{flalign}
Working on ${\ll_{t+1}}$, we getL
\begin{flalign}
  \begin{aligned} \label{tmp3}
  \abs{\ll_{t+1}} &= \abs{\ll_{t+2}-(\ll_{t+2}-\ll_{t+1})} \\
  & \geq \abs{\ll_{t+2}} - \abs{\ll_{t+2}-\ll_{t+1}} \\
  & = \abs{\ll_{t+2}} - \abs{[\ll_{t+1}+\mu g_{t+1}(\boldsymbol{z}_{t+1})]^+ - \ll_{t+1}} \\
  & = \abs{\ll_{t+2}} - \abs{\mu g_{t+1}(\boldsymbol{z}_{t+1})} \\
  & \geq \abs{\ll_{t+2}} - \mu M \\
  & > \frac{2a\sqrt{K(K+1)}D+D^2/(2\nu)+(\mu M^2)/2}{\epsilon-U_g}
  \end{aligned}
\end{flalign}
If we multiply both sides with $\mu(U_g - \epsilon)$, which is strictly negative due to Assumption 4, (\ref{tmp3}) is equivalent to
\begin{flalign}
  \begin{aligned}
  \mu(U_g -\epsilon){\ll_{t+1}} < -\mu\left(2a\sqrt{K(K+1)}D+\frac{D^2}{2\nu}+\frac{\mu M^2}{2}\right)\notag
  \end{aligned}
\end{flalign}
Passing all the terms on the left side, we deduce:
\begin{flalign}
  \begin{aligned}
  \Delta(\ll_{t+1}) < 0 \notag
  \end{aligned}
\end{flalign}
which yields $\abs{\ll_{t+2}}<\abs{\ll_{t+1}}$ and that contradicts our assumption. As we set $\ll_1 = 0$, then $\abs{\ll_2} \leq \mu M$. Thus, for every $t\geq1$, $\abs{\ll_t}\leq\Tilde{\ll}$ holds. 

\end{comment}

%\begin{flalign}
%  \begin{aligned}
%  \ll_{T+1} &= [\ll_T + \mu g_T(\boldsymbol{z}_T)]^+ \\
%  & \geq \ll_T + \mu g_T(\boldsymbol{z}_T) \geq \ll_1 + \sum_{t=1}^T \mu g_t(\boldsymbol{z}_t)
%  \end{aligned}
%\end{flalign}

%Then we have
%\begin{flalign}
%  \begin{aligned}
%  \sum_{t=1}^T g_t(\boldsymbol{z}_t) \leq \frac{\ll_{T+1}}{\mu}
%  \end{aligned}
%\end{flalign}
%as $\ll_1=0$. Non-negativity of $\ll_{T+1}$ implies 

%\begin{flalign}
%  \begin{aligned}
%  &[\sum_{t=1}^T g_t(\boldsymbol{z}_t)]^+ \leq %\frac{\ll_{T+1}}{\mu} \\
%  \iff & \abs{[\sum_{t=1}^T g_t(\boldsymbol{z}_t)]^+} \leq \frac{\abs{\ll_{T+1}}}{\mu} \\
%  \iff & V_T \leq \frac{\abs{\ll_{T+1}}}{\mu} \leq \frac{\abs{\Tilde{\ll}}}{\mu}
%  \end{aligned}
%\end{flalign}
%which completes the proof.

%Now we work through the proof on the upper bound of the regret $R_T$. It can be shown that $\mathcal{L}_t(\boldsymbol{z},\ll)$ is $1/\nu$-strongly convex with regard to $\boldsymbol{z}$, which implies that for any $\boldsymbol{x}, \boldsymbol{y}$, we have \cite[Chapter 2.1]{hazan-book}

%\begin{flalign}
%  \begin{aligned} \label{strongConvex}
%  \mathcal{L}_t(\boldsymbol{y}) \geq \mathcal{L}_t(\boldsymbol{x}) + \grad\mathcal{L}_t(\boldsymbol{x})^\top(\boldsymbol{y}-\boldsymbol{x}) + \frac{1}{2\nu}\norm{\boldsymbol{y}-\boldsymbol{x}}^2
%  \end{aligned}
%\end{flalign}
%Since $\boldsymbol{z}_{t+1}$ minimizes the problem $\min_{\bm z\in \mathcal Z} L_{t}(\bm z, \ll_{t+1})$, the optimality condition \cite[Theorem 2.2]{hazan-book} applies

%\begin{flalign}
%  \begin{aligned} \label{KKT}
%  \grad\mathcal{L}_t(\boldsymbol{z}_{t+1})^\top(\boldsymbol{y}-\boldsymbol{z}_{t+1}) \geq 0 \qquad \forall \boldsymbol{y} \in \mathcal{Z}
%  \end{aligned}
%\end{flalign}

%Setting $\boldsymbol{y}=\boldsymbol{z}_t^*$ and $\boldsymbol{z}=\boldsymbol{z}_{t+1}$ in (\ref{strongConvex}), we have, using (\ref{KKT}), 

%\begin{flalign}
%  \begin{aligned} \label{tmp4}
%  \mathcal{L}_t(\boldsymbol{z}_t^*) \geq \mathcal{L}_t(\boldsymbol{z}_{t+1}) + \frac{1}{2\nu}\norm{\boldsymbol{z}_t^* - \boldsymbol{z}_{t+1}}^2
%  \end{aligned}
%\end{flalign}

%To alleviate the notation, we write $\mathcal{L}_t(\boldsymbol{z})$ for $\mathcal{L}_t(\boldsymbol{z},\ll)$. Then, (\ref{tmp4}) leads to

%\begin{flalign}
%  \begin{aligned} \label{mainReg}
%  \mathcal{L}_t(\boldsymbol{z}_{t+1}) \leq \mathcal{L}_t(\boldsymbol{z}_t^*) - \frac{1}{2\nu}\norm{\boldsymbol{z}_t^* - \boldsymbol{z}_{t+1}}^2 \\
 % \iff f_t(\boldsymbol{z}_t)+\mathcal{L}_t(\boldsymbol{z}_{t+1}) \leq f_t(\boldsymbol{z}_t)+\mathcal{L}_t(\boldsymbol{z}_t^*) \\
 %  \qquad - \frac{1}{2\nu}\norm{\boldsymbol{z}_t^* - \boldsymbol{z}_{t+1}}^2 \\
 % = f_t(\boldsymbol{z}_t) + \grad f_t(\boldsymbol{z}_t)^\top(\boldsymbol{z}_t^*-\boldsymbol{z}_t) \\
 % + \ll_{t+1}g_t(\boldsymbol{z}_t^*) + \frac{1}{2\nu}(\norm{\boldsymbol{z}_t^*-\boldsymbol{z}_t}^2-\norm{\boldsymbol{z_t^*}-\boldsymbol{z}_{t+1}}^2) \\
 % \stackrel{\text{(a)}}{\leq} f_t(\boldsymbol{z}_t^*) + 0 + \frac{1}{2\nu}(\norm{\boldsymbol{z}_t^*-\boldsymbol{z}_t}^2-\norm{\boldsymbol{z_t^*}-\boldsymbol{z}_{t+1}}^2)
 % \end{aligned}
%\end{flalign}

%where (a) uses the convexity of $f$ and set $\ll_{t+1}g_t(\boldsymbol{z}_t^*)$ to $0$. Indeed, $\ll_{t+1}g_t(\boldsymbol{z}_t^*) \leq 0$ as $\ll_{t+1}\geq0$ and the per-slot optimal $\boldsymbol{z}_t^*$ is always feasible, i.e. $g_t(\boldsymbol{z}_t^*) \leq 0$.

%Next, we seek to bound the term $-\grad f_t(\boldsymbol{z}_t)^\top(\boldsymbol{z}_{t+1}-\boldsymbol{z}_t)$ by

%\begin{flalign}
%  \begin{aligned} \label{tmp5}
 % -\grad f_t(\boldsymbol{z}_t)^\top(\boldsymbol{z}_{t+1}-\boldsymbol{z}_t) \stackrel{\text{(b)}}{\leq} \norm{\grad f_t(\boldsymbol{z}_t)}\norm{\boldsymbol{z}_{t+1}-\boldsymbol{z}_t} \\
 % \stackrel{\text{(c)}}{\leq} %\frac{\norm{\grad %f_t(\boldsymbol{z}_t)}^2}{2\eta} + %\frac{\eta}{2}\norm{\boldsymbol{z}_{t+1}-\boldsymbol{z}_t}^2 \\ \stackrel{\text{(d)}}{\leq}  \frac{a^2K(K+1)}{2\eta}+\frac{\eta}{2}\norm{\boldsymbol{z}_{t+1}-\boldsymbol{z}_t}^2
 % \end{aligned}
%\end{flalign}

%where (b) uses Cauchy-Schwartz inequality, (c) is true for any arbitrary positive constant $\eta$, (d) uses the upper bound on the gradient in Assumption 3. Plugging (\ref{tmp5}) into (\ref{mainReg}), we can isolate $f_t(\boldsymbol{z}_t)+\ll_{t+1}g_t(\boldsymbol{z}_{t+1})$

%\begin{flalign}
%  \begin{aligned} \label{tmp6}
%  f_t(\boldsymbol{z}_t)+\ll_{t+1}g_t(\boldsymbol{z}_{t+1}) \leq f_t(\boldsymbol{z}_t^*) + (\frac{\eta}{2}-\frac{1}{2\nu})\norm{\boldsymbol{z}_{t+1}-\boldsymbol{z}_t}^2 \\
%  + \frac{1}{2\nu}(\norm{\boldsymbol{z}_t^*-\boldsymbol{z}_t}^2-\norm{\boldsymbol{z_t^*}-\boldsymbol{z}_{t+1}}^2) + \frac{a^2K(K+1)}{2\eta} \\
%  \stackrel{\text{(e)}}{=}f_t(\boldsymbol{z}_t^*) + \frac{1}{2\nu}(\norm{\boldsymbol{z}_t^*-\boldsymbol{z}_t}^2-\norm{\boldsymbol{z}_t^*-\boldsymbol{z}_{t+1}}^2) \\
%  + \frac{\nu a^2 K(K+1)}{2}
%  \end{aligned}
%\end{flalign}

%where (e) happens as we choose $\eta = 1/\nu$ so that $\eta/2 - 1/2\nu =0$. Using the dual drift bound in (\ref{lemma1}), we have

%\begin{flalign}
%  \begin{aligned} \label{tmp7}
%  \frac{\Delta(\ll_{t+1})}{\mu} + f_t(\boldsymbol{z}_t) \leq \ll_{t+1}(g_{t+1}(\boldsymbol{z}_{t+1})-g_t(\boldsymbol{z}_{t+1})) \\
%  + f_t(\boldsymbol{z}_t)+ \ll_{t+1}g_t(\boldsymbol{z}_{t+1})+\frac{\mu}{2}g_{t+1}(\boldsymbol{z}_{t+1})^2 \\
%  \stackrel{\text{(f)}}{\leq} \ll_{t+1}[g_{t+1}(\boldsymbol{z}_{t+1})-g_t(\boldsymbol{z}_{t+1})]^+ + f_t(\boldsymbol{z}_t^*) \\
%  + \frac{1}{2\nu}(\norm{\boldsymbol{z}_t^*-\boldsymbol{z}_t}^2-\norm{\boldsymbol{z}_t^*-\boldsymbol{z}_{t+1}}^2) + \frac{\nu a^2 K(K+1)}{2} \\
%  + \frac{\mu g_{t+1}(\boldsymbol{z}_{t+1})^2}{2} \\
%  \stackrel{\text{(g)}}{\leq} f_t(\boldsymbol{z}_t^*) + \frac{1}{2\nu}(\norm{\boldsymbol{z}_t^*-\boldsymbol{z}_t}^2-\norm{\boldsymbol{z}_t^*-\boldsymbol{z}_{t+1}}^2) \\
 % + \Tilde{\ll}U_g + \frac{\nu a^2 K(K+1)}{2} + \frac{\mu M^2}{2}
 % \end{aligned}
%\end{flalign}

%where (f) uses the non-negativity of $\ll_{t+1}$ and (\ref{tmp6}), (g) comes from the upper bound $\abs{g_{t+1}(\boldsymbol{z}_{t+1})} \leq U_g$ in Assumption 3, and from the dual upper bound $\ll_{t+1}\leq\Tilde{\ll}$.

%By interpolating the norm terms, we have

%\begin{flalign}
%  \begin{aligned} \label{tmp8}
%  \norm{\boldsymbol{z}_t^*-\boldsymbol{z}_t}^2 - \norm{\boldsymbol{z}_t^*-\boldsymbol{z}_{t+1}}^2 \\
%  = \norm{\boldsymbol{z}_t^*-\boldsymbol{z}_t}^2 - \norm{\boldsymbol{z}_t-\boldsymbol{z}_{t-1}^*}^2 +\norm{\boldsymbol{z}_t-\boldsymbol{z}_{t-1}^*}^2 \\
%  - \norm{\boldsymbol{z}_t^*-\boldsymbol{z}_{t+1}}^2 \\
%  = \norm{\boldsymbol{z}_t^*-\boldsymbol{z}_{t-1}^*}\norm{\boldsymbol{z}_t^* - 2\boldsymbol{z}_t + \boldsymbol{z}_{t-1}^*} + \norm{\boldsymbol{z}_t-\boldsymbol{z}_{t-1}^*}^2 \\
%  - \norm{\boldsymbol{z}_t^*-\boldsymbol{z}_{t+1}}^2 \\
%  \stackrel{\text{(h)}}{\leq} \norm{\boldsymbol{z}_t^*-\boldsymbol{z}_{t-1}^*}2D + \norm{\boldsymbol{z}_t-\boldsymbol{z}_{t-1}^*}^2 - \norm{\boldsymbol{z}_t^*-\boldsymbol{z}_{t+1}}^2
%  \end{aligned}
%\end{flalign}

%where (h) follows from the diameter of $\mathcal{Z}$ in Assumption 3. Plugging (\ref{tmp8}) into (\ref{tmp7}), we have

%\begin{flalign}
%  \begin{aligned} \label{tmp9}
%  \frac{\Delta(\ll_{t+1})}{\mu} + f_t(\boldsymbol{z}_t) \stackrel{\text{(i)}}{\leq} f_t(\boldsymbol{z}_t^*)+ \Tilde{\ll}U_g + \frac{\nu a^2 K(K+1)}{2} \\
%  + \frac{\mu M^2}{2} + \frac{1}{2\nu}(2DU_z  + \norm{\boldsymbol{z}_t-\boldsymbol{z}_{t-1}^*}^2 - \norm{\boldsymbol{z}_t^*-\boldsymbol{z}_{t+1}}^2)
%  \end{aligned}
%\end{flalign}

%where (i) comes from $\norm{\boldsymbol{z}_t^*-\boldsymbol{z}_{t-1}^*} \leq U_z$, from the definition of $U_z$ in Assumption 3.

%Summing up (\ref{tmp9}) over $t=1...T$, and rearranging the terms, we find

%\begin{flalign}
%  \begin{aligned}
%  \sum_{t=1}^T f_t(\boldsymbol{z}_t)-\sum_{t=1}^T f_t(\boldsymbol{z}_t^*) \leq \Tilde{\ll}U_g^T + \frac{\nu a^2T K(K+1)}{2} \\
%  + \frac{\mu T M^2}{2} + \frac{D U_z^T}{\nu} + \frac{1}{2\nu}\sum_{t=1}^T (\norm{\boldsymbol{z}_t-\boldsymbol{z}_{t-1}^*}^2 - \norm{\boldsymbol{z}_t^*-\boldsymbol{z}_{t+1}}^2) \\
%  - \sum_{t=1}^T \frac{\Delta(\ll_{t+1})}{\mu} \\
%  \stackrel{\text{(j)}}{=} \Tilde{\ll}U_g^T + \frac{\nu a^2T K(K+1)}{2 }+ \frac{\mu T M^2}{2} + \frac{D U_z^T}{\nu} \\
%  + \frac{1}{2\nu}(\norm{\boldsymbol{z}_1-\boldsymbol{z}_0^*}^2-\norm{\boldsymbol{z}_T^*-\boldsymbol{z}_{T+1}}^2) - \frac{1}{2\mu}(\ll_{T+2}^2-\ll_{2}^2) \\
%  \stackrel{\text{(k)}}{\leq} \Tilde{\ll}U_g^T + \frac{\nu a^2T K(K+1)}{2 }+ \frac{\mu T M^2}{2} + \frac{D U_z^T}{\nu} \\
%  + \frac{D^2}{2\nu} + \frac{\mu M^2}{2}
%  \end{aligned}
%\end{flalign}

%where (j) comes from the telescoping sums, (k) uses $\norm{\boldsymbol{z}_T^*-\boldsymbol{z}_{T+1}}^2 \geq 0$, $\ll_{T+2}^2 \geq 0$, $\norm{\boldsymbol{z}_1-\boldsymbol{z}_0^*}^2 \leq D^2$, $\ll_2^2 \leq \mu^2 M^2$.
%This completes the proof.
