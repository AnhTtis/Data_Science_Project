\section{Introduction}
\begin{wrapfigure}[17]{r}{0.57\textwidth}\vspace{-17pt}
  \begin{center}
    \includegraphics[width=0.52\textwidth]{figs/teaser/teaser.png}\vspace{-10pt}
  \end{center}
  \caption{The VisCo prior (left) is incorporating viscosity and Coarea; Middle and right shows ablations on each.}\label{fig:teaser}
\end{wrapfigure}
Reconstructing 3D surfaces from sparse point clouds is a long standing problem in both computer vision and graphics \cite{berger2017survey}. Methods tackling this problem aim to estimate 3D surfaces given as input unordered point sets (point clouds), with or without corresponding normals. Surface representations can be divided to two groups: parametric and implicit. Parametric methods represents the surface using some parametric domain, while implicit methods represent the surface as some level-set, $\gS=\set{p\in\Real^3\vert f(p)=c}$, of a volumetric function $f:\Real^3\too \Real$. While parametric methods can easily sample the surface, implicit methods can readily adapt to topological changes of the reconstructed surface. Parametric methods include, \eg, meshes and spline surfaces, while implicit methods use, \eg, volumetric data structures such as voxel grids, Radial Basis Functions (RBFs), or (recently) neural networks. 

Implicit Neural Representation (INR) \cite{mescheder2019occupancy, park2019deepsdf,chen2019learning,atzmon2019sal, erler2020points2surf, gropp2020implicit,atzmon2021sald, peng2021shape} is categorized as an implicit method using a neural networks to define the implicit function $f$. INRs build upon the inherit inductive bias in neural networks and their optimization process to provide smooth yet flexible and expressive surface reconstructions. INRs have several disadvantages: First, the neural inductive bias is hard to control, often introducing undesired or unexplained surface behaviours. In fact, a considerable research effort is dedicated to fix/change/control this bias \cite{tancik2020fourier,sitzmann2020implicit,lipman2021phase,lindell2021bacon}. Second, INRs have an increased deployment cost, requiring many network evaluations for surface contouring, \eg, with marching cube based methods \cite{mescheder2019occupancy, park2019deepsdf}, or direct rendering \cite{yariv2020multiview, mildenhall2020nerf}. Lastly, although using high optimized solvers, INRs are still slow to train.  

The goal of this work it show that network-free grid-based implicit representations can achieve INR-level reconstructions when incorporating suitable priors. To that end, we present VisCo Grids: a grid-based surface reconstruction algorithm that incorporates well-defined geometric priors: Viscosity and Coarea. In short, VisCo, see Figure \ref{fig:teaser}, right. The viscosity loss, is replacing the Eikonal loss \cite{gropp2020implicit,sitzmann2020implicit} used in INRs for optimizing Signed Distance Functions (SDF). The Eikonal loss posses many bad minimal solution that are avoided in the INR setting due to the network's inductive bias, but are present in the grid parametrization, see \eg, Figure \ref{fig:teaser}, middle.  The viscosity loss, uses the notion of vanishing viscosity to regularize the Eikonal loss and provide well defined smooth solution that converges to the "correct" viscosity SDF solution. The viscosity loss provides smooth SDF solution but do not punish excessive or "ghost" surface parts, see \eg, Figure \ref{fig:teaser} (right). Therefore, a second useful prior is the coarea loss, directly controlling the surface's area, and encourages it to be smaller. The coarea loss is defined using a "squashing" function applied to the viscosity SDF making it approximately an indicator function, and then integrates its gradient norm over the domain. Integrating the gradient norm of a function is called the Total Variation loss \citep{Chambolle10anintroduction,lipman2021phase} and is measuring the perimeter of indicator functions, which in our case approximates $\mathrm{area}(\gS)$. VisCo grids (as other grid methods) have instant inference, and even with our current rather naive implementation are faster to train than INRs. Considerable training time improvement are expected with a more efficient implementation. 

We tested VisCo Grids on a standard 3D reconstruction dataset, and achieved comparable accuracy to the state-of-the-art INR methods. Through ablations, we demonstrate the properties and benefit in the VisCo prior. 
