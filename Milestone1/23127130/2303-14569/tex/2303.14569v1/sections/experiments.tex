\section{Experiments}


\rebb{In this section we extensively evaluate VisCo grids. First, we evaluate on two standard surface reconstruction benchmarks~\cite{williams2019deep,huang2022surface} (Sec.~\ref{sec:exp_rec}) against a large variety of state-of-the-art methods: Poisson Surface Reconstruction ~\cite{kazhdan2006poisson}, DGP~\cite{williams2019deep}, IGR~\cite{gropp2020implicit}, SIREN~\cite{sitzmann2020implicit}, FFN~\cite{tancik2020fourier}, NSP~\cite{williams2021neural}, PHASE~\cite{lipman2021phase}, GD~\cite{greedydelaunay}, BPA~\cite{BPA}, SPSR~\cite{kazhdan2013screened}, RIMLS~\cite{RIMLS}, SALD~\cite{sald}, IGR~\cite{gropp2020implicitgeometricregularizationforlearningshape}, OccNet~\cite{occupancy}, DeepSDF~\cite{deepsdf}, LIG~\cite{LIG}, Points2Surf~\cite{points2surf}, DSE~\cite{learningdelaunaysurface}, IMLSNet~\cite{liu2021DeepIMLS} and ParseNet~\cite{parsenet}.}
We then perform an ablation study (Sec.~\ref{sec:exp_ablation}), and conduct a detail examination of the main components of the model, namely the viscosity and coarea losses. Finally, we discuss the model’s ability to reconstruct 
sparse point clouds (Sec.~\ref{sec:sparse}) using scans from Stanford 3D Scanning Repository.


\subsection{Surface reconstruction benchmarks}
\label{sec:exp_rec}

\rebb{We next evaluated our model on two benchmarks: Surface Reconstruction Benchmark~\cite{williams2019deep} and Surface Reconstruction from Real-Scans~\cite{huang2022surface}. Each containing challenging object with complex shape and topology. Importantly, we use same hyper-parameters for all meshes of all benchmarks with no extensive hyper-parameter search. }


\paragraph{Surface Reconstruction Benchmark}
This benchmark~\cite{williams2019deep}
consists of 5 noisy range scans, each containing point cloud and normal data. 
We evaluate our method against current state of the art methods on this benchmark: Deep Geometric Prior (DGP)~\cite{williams2019deep}, Implicit Geometric Regularization (IGR)~\cite{gropp2020implicit}, SIREN~\cite{sitzmann2020implicit}, Fourier Feature Networks (FFN)~\cite{tancik2020fourier}, NSP~\cite{williams2021neural} and PHASE~\cite{lipman2021phase}. We additionally compare to the classical method of Poisson Surface Reconstruction ~\cite{kazhdan2006poisson}.
Quantitative results are summarized in Table~\ref{tab:dgp}. We report the Chamfer ($d_C$) and Hausdorff ($d_H$) distances between the reconstructed meshes and the ground-truth point clouds. Furthermore, we report their corresponding one sided distances ($d_H^\too$ and $d_C^\too$) between the reconstructed meshes and the input noisy point cloud. 
Representative qualitative results are shown in Figure~\ref{fig:recon1}. Note that we achieve comparable results to the current state-of-the-art INR methods.  



\rebb{
\paragraph{Surface Reconstruction from Real-Scans} This benchmark~\cite{huang2022surface} consists of 21 noisy range scans of real objects. We evaluate our method against: GD~\cite{greedydelaunay}, BPA~\cite{BPA}, SPSR~\cite{kazhdan2013screened}, RIMLS~\cite{RIMLS}, SALD~\cite{sald}, IGR~\cite{gropp2020implicitgeometricregularizationforlearningshape}, OccNet~\cite{occupancy}, DeepSDF~\cite{deepsdf}, LIG~\cite{LIG}, Points2Surf~\cite{points2surf}, DSE~\cite{learningdelaunaysurface}, IMLSNet~\cite{liu2021DeepIMLS} and ParseNet~\cite{parsenet}. Quantitative results are summarized in Table~\ref{tab:bench2}. We report Chamfer Distance ($d_C$), F-score, Normal Consistency Score (NCS)~\cite{occupancy}, and Neural Feature Similarity (NFS)~\cite{huang2022surface} distances between the reconstructed meshes and the ground-truth point clouds. Furthermore, we report the one sided distances ($d_H^\too$ and $d_C^\too$) between the reconstructed meshes and the input noisy point cloud. Representative qualitative results are shown in Figure~\ref{fig:recon2}. Note that we achieve 1-st to 3-rd place across all categories, with top F-score. 
}





\begin{figure}[ht!]
    \includegraphics[width=\textwidth]{figs/reconstr_short4.png}
    \caption{Qualitative results for surface reconstruction \cite{williams2019deep} compared to existing methods. Note how VisCo Grids achieve comparable level of details when compared to other baselines. 
    } \label{fig:recon1}
\end{figure}


\begin{table}[h!]\scriptsize	
    \centering

    \begin{tabular}{c|c|c|c|c|c|c|c|c|c|c} 
        \multicolumn{2}{c|}{} & Poisson & DGP & IGR & SIREN & FFN & NSP & PHASE & \textbf{Ours \tiny{(30 mins)}} & \textbf{Ours \tiny{(8 mins)}} \\ \midrule
            \bottomrule
            
            
        \multirow{4}{*}{Anchor} 
            & $d_C$ & 0.60 & 0.33 & 0.22 & 0.32 & 0.31 & 0.22 & \textbf{0.21} & \textbf{0.21}  & 0.28 \\
            & $d_H$ & 14.89 & 8.82 & 4.71 & 8.19 & 4.49 & 4.65 & 4.29 & \textbf{3.00} & 5.69 \\
            & $d_C^\too$ & 0.60 & 0.08 & 0.12 & 0.10 & 0.10 & 0.11 & 0.09 & 0.15 & 0.15\\
            & $d_H^\too$ & 14.89 & 2.79 & 1.32 & 2.43 & 0.10 & 1.11 & 1.23 & 1.07 & 1.15\\ \hline
        \multirow{4}{*}{Daratech}
            & $d_C$ & 0.44 & 0.20 & 0.25 & 0.21 & 0.34 & 0.21 & \textbf{0.18} & 0.26  & 0.25 \\
            & $d_H$ & 7.24 & 3.14 & 4.01 & 4.30 & 5.97 & 4.35 & \textbf{2.92} & 4.06  & 4.15 \\
            & $d_C^\too$ & 0.44 & 0.04 & 0.08 & 0.09 & 0.10 & 0.08 & 0.08 & 0.14 & 0.13  \\
            & $d_H^\too$ & 7.24 & 1.89 & 1.59 & 1.77 & 0.10 & 1.14 & 1.80 & 1.76 & 1.78 \\ \hline
        \multirow{4}{*}{DC}
            & $d_C$ & 0.27 &0.18 & 0.17 & 0.15 & 0.20 & \textbf{0.14} & 0.15 & 0.15  & 0.15  \\
            & $d_H$ & 3.10 & 4.31 & 2.22 & 2.18 & 2.87 & \textbf{1.35} & 2.52 & 2.22  & 2.23 \\
            & $d_C^\too$ & 0.27 & 0.04 & 0.09 & 0.06 & 0.10 & 0.06 & 0.05 & 0.09  & 0.09 \\
            & $d_H^\too$ & 3.10 & 2.53 & 2.61 & 2.76 & 0.12 & 2.75 & 2.78 & 2.76 & 2.78 \\ \hline
        \multirow{4}{*}{Gargoyle}
            & $d_C$ & 0.26 &0.21 & \textbf{0.16} & 0.17 & 0.22 & \textbf{0.16} & \textbf{0.16} & 0.17  & 0.17 \\
            & $d_H$ & 6.8 & 5.98 & 3.52 & 4.64 & 5.04 & 3.20 & \textbf{3.14} & 4.40  & 4.45 \\
            & $d_C^\too$ & 0.26 & 0.06 & 0.06 & 0.08 & 0.09 & 0.08 & 0.07 & 0.11 & 0.11 \\
            & $d_H^\too$ &  6.80 & 3.41 & 0.81 & 0.91 & 0.09 & 2.75 & 1.09 & 0.96 & 0.98 \\ \hline
        \multirow{4}{*}{Lord Quas}
            & $d_C$ & 0.20 & 0.14 & 0.12 & 0.17 & 0.35 & 0.12 & \textbf{0.11} & 0.12  & 0.13 \\
            & $d_H$ & 4.61 & 3.67 & 1.17 & 0.82 & 3.90 & \textbf{0.69} & 0.96 & 1.06  & 1.14\\
            & $d_C^\too$ & 0.20 & 0.04 & 0.07 & 0.12 & 0.06 & 0.05 & 0.04 & 0.07 & 0.07\\
            & $d_H^\too$ & 4.61 & 2.03 & 0.98 & 0.76 & 0.06 & 0.62 & 0.96 & 0.64 & 0.68\\ 
            \bottomrule
            
            
      
    \end{tabular} \vspace{10pt}
    \caption{Surface reconstruction results on the benchmark of \cite{williams2019deep}. We show reconstruction results for each model for our method at 256 grid resolution with 30 minute and 8 minute time budget. We also show results from comparative methods. Bold numbers signify top performance.  We report Chamfer and Hausdorff distances using ground truth scans ($d_C$, $d_H$) and input scans ($d_C^\too$, $d_H^\too$). Note that VisCo Grids achieve comparable results to SOTA INRs, and even matches it in terms of Chamfer distance in 3 out of 5 meshes.
    }
    \label{tab:dgp}
\end{table}


\begin{figure}[ht!]
    \includegraphics[width=\textwidth]{figs/reconstr_bench2.png}
    \caption{\rebb{Qualitative results for surface reconstruction of real objects \cite{huang2022surface} compared to existing methods. Note how VisCo Grids does not over-extend the surface in the bottom row example. The competing methods meshes were provided by the benchmark organizers.}
    } \label{fig:recon2}
\end{figure}




\begin{table}[h!]\scriptsize	
    \centering
    \begin{tabular}{c|l|c|c|c|c} 
        Prior & Method & $d_C$ $(\times10^{-2}) \downarrow$ & F-score $(\%) \uparrow$ & NCS $(\times10^{-2}) \uparrow$ & NFS $(\times10^{-2}) \uparrow$ \\ \midrule
         \multirow{2}{*}{Triangulation-based}   & GD \cite{greedydelaunay}                                              & 31.72                  & 87.51                  & 88.86                  & 82.20                        \\
                                                        & BPA \cite{BPA}                                                        & 40.37                  & 80.95                  & 87.56                  & 68.69                        \\\cline{1-2}
    \multirow{3}{*}{Smoothness}            & SPSR \cite{kazhdan2013screened}                                       & \textbf{31.05}         & \underline{87.74}         & \underline{{94.94}} & \textbf{89.38}               \\
                                                        & RIMLS \cite{RIMLS}                                                    & 32.80                  & 87.05                  & 91.97                  & 85.19   \\
                                               & \textbf{Ours}                                                     & \phantom{xxxxx}32.11 ($3^{rd}$)                 & \phantom{xxxxx}\textbf{88.52}   ($1^{st}$)                & \phantom{xxxxx}94.20  ($3^{rd}$)                 & \phantom{xxxxx}\underline{89.16} ($2^{rd}$)          
                                            \\\cline{1-2}
    \multirow{2}{*}{Modeling}                & SALD \cite{sald}                                                      & \underline{{31.13}} & {{87.72}} & 94.68                  & 86.86                        \\
                                                        & IGR \cite{gropp2020implicitgeometricregularizationforlearningshape}   & 32.70                  & 87.18                  & \textbf{95.99}         & {{89.10}}       \\\cline{1-2}
    \multirow{2}{*}{Learning Semantics}     & OccNet \cite{occupancy}                                               & 232.71                 & 17.11                  & 80.96                  & 39.70                        \\
                                                        & DeepSDF \cite{deepsdf}                                                & 263.92                 & 19.83                  & 77.95                  & 40.95                        \\\cline{1-2}
    \multirow{2}{*}{Local Learning}         & LIG \cite{LIG}                                                        & 48.75                  & 83.76                  & 92.57                  & 81.48                        \\
                                                        & Points2Surf \cite{points2surf}                                        & 48.93                  & 80.89                  & 89.52                  & 81.83                        \\\cline{1-2}
     \multirow{3}{*}{Hybird}                                                   & DSE \cite{learningdelaunaysurface}                                    & 32.16                  & 86.88                  & 87.20                  & 76.81                        \\
                      & IMLSNet \cite{liu2021DeepIMLS}                                        & 38.46                  & 82.44                  & 93.31                  & 85.30                        \\
                                                        & ParseNet \cite{parsenet}                                              & 149.96                 & 38.92                  & 81.51                  & 45.67                        \\\hline


    \end{tabular} \vspace{10pt}
    \caption{\rebb{Surface reconstruction results on the 20 real-scanned benchmark~\cite{huang2022surface} meshes. We report Chamfer Distance ($d_C$), F-score, Normal Consistency Score (NCS)~\cite{occupancy}, and Neural Feature Similarity (NFS)~\cite{huang2022surface}. Methods are grouped according to surface geometry priors, as originally defined in the benchmark. Our method achieves top F-score and 1-st to 3-rd place in all scores.}}
    \label{tab:bench2}
\end{table}




\subsection{Ablation study}
\label{sec:exp_ablation}



\begin{figure}[ht!]
    \includegraphics[width=\textwidth]{figs/ablation.png} \vspace{-20pt}
    \caption{Ablation for the main components of our method. Removing elements of our loss leads to subpar reconstructions. We can observe these artifacts in the level sets shown in this figure. Removing viscosity results in discontinuities in the final surface, while no coarea produces excess surface area.} \label{fig:abl}
    \vspace{-15pt}
\end{figure}


\begin{table}[h!]\scriptsize	
    \centering
    \begin{tabular}{c|c|c|c|c|c}
        \multicolumn{2}{c|}{} & Baseline & w/o normals & w/o viscosity & w/o coarea \\ \hline
        \multirow{4}{*}{Anchor}
            & $d_C$ & \textbf{0.21} & 0.61 & 0.55 & 0.72 \\
            & $d_H$ & \textbf{3.00} & 7.82 & 10.83 & 10.24 \\
            & $d_C^\too$ & 0.15 & 0.37 & 0.27 & 0.36 \\
            & $d_H^\too$ & 1.07 & 7.84 & 1.44 & 9.68 \\ \hline
        \multirow{4}{*}{Daratech}
            & $d_C$ & 0.26 & 0.24 & 0.24 & \textbf{0.23} \\
            & $d_H$ & 4.06 & 4.2 & 4.3 & \textbf{2.19} \\
            & $d_C^\too$ & 0.14 & 0.13 & 0.12 & 0.13 \\
            & $d_H^\too$ & 1.76 & 2.69 & 1.77 & 1.77 \\ \hline
        \multirow{4}{*}{DC}
            & $d_C$ & \textbf{0.15} & \textbf{0.15} & \textbf{0.15} & 0.34 \\
            & $d_H$ & \textbf{2.22} & 2.24 & 2.24 & 6.58 \\
            & $d_C^\too$ & 0.09 & 0.08 & 0.08 & 0.16 \\
            & $d_H^\too$ & 2.76 & 2.76 & 2.79 & 2.82 \\ \hline
        \multirow{4}{*}{Gargoyle}
            & $d_C$ & \textbf{0.17} & 0.58 & 0.47 & 0.59 \\
            & $d_H$ & \textbf{4.40} & 6.32 & 10.38 & 6.35 \\
            & $d_C^\too$ & 0.11 & 0.07 & 0.26 & 0.38 \\
            & $d_H^\too$ & 0.96 & 2.39 & 1.34 & 1.25 \\ \hline
        \multirow{4}{*}{Lord Quas}
            & $d_C$ & \textbf{0.12} & 0.12 & 0.12 & 0.58 \\
            & $d_H$ & 1.06 & 1.38 & \textbf{1.04} & 6.05 \\
            & $d_C^\too$ & 0.07 & 0.37 & 0.06 & 0.32 \\
            & $d_H^\too$ & 0.64 & 0.69 & 0.64 & 3.73 \\ \hline %
            
    \end{tabular} \vspace{5pt}
    \caption{Ablations study. We show the contribution of each component of VisCo Grids. Baseline is the full method. The remaining columns correspond to optimizing without normal loss, viscosity loss and coarea loss, respectively. We show results for each mesh of the benchmark \cite{williams2019deep}. The results justify the use of the different components in VisCo Grids.}
    \label{tab:ablations}
\end{table}



We provide an ablation study of the main components of our model in Table~\ref{tab:ablations} and Figure~\ref{fig:abl}. Specifically we compared with the following alternatives: i) Eikonal loss without the viscosity term that prevents undesirable non-SDF solutions, \ie, $\eps=0$ in \eqref{e:loss_viscosity_eikonal}; ii) removing the coarea loss enforcing minimal surface area, \ie, $\lambda_{\text{c}}=0$;  and iii) removing the normal loss, \ie, $\lambda_{\text{n}}=0$. Note that without coarea and viscosity the reconstruction tends to have holes and discontinuities near the surface boundaries. Only combination of all the components results in a good surface reconstruction.

\paragraph{Learning with viscosity.} We further provide a more in depth discussion of the proposed viscosity loss. Figure~\ref{fig:viscosity} compares reconstructions with different levels of the viscosity parameter, \ie, $\eps$ in \eqref{e:loss_viscosity_eikonal}. As can be inspected from this figure, viscosity affects the smoothness of the reconstructed surface. For a low viscosity parameter the zero level sets become noisy. This can be explained by the viscosity eikonal loss (\ie, \eqref{e:loss_viscosity_eikonal}) becoming numerically too close to the eikonal loss in \eqref{e:loss_eikonal} and the solution deviates from the viscosity SDF solution. This leads to artifacts across the surface, similar to the limit case ($\eps=0$) where only the eikonal loss is used, see the second column from the left. For a high viscosity parameter, and as expected with the addition of a non-vanishing Laplacian term, the surface becomes over-smoothed.

\begin{figure}[ht!]
\vspace{-20pt}
    \includegraphics[width=\textwidth]{figs/viscosity.png} \vspace{-10pt}
    \caption{Viscosity loss ablation. Setting a high viscosity loss parameter, $\eps$, leads to oversmoothing. In contrast, setting it too low leads to noise and discontinuities in the surfaces, similarly to removing it by setting $\eps=0$.} \vspace{-20pt} \label{fig:viscosity}
\end{figure}

\paragraph{Learning with coarea. } Similarly, we also provide an analysis of the proposed coarea loss. 
In Figure~\ref{fig:coarea} we show the effect of changing the parameter weight of the coarea loss, $\lambda_{\text{c}}$. As can be observed in the figure, a low coarea weight leads to the presence of excessive surface area in the reconstruction. In contrast, a high weight will strive to minimize the surface area. In a sense, the coarea serves a surface tension parameter; stronger tension will ignore points, weaker tension will overfit. 

\begin{figure}[ht!]
    \includegraphics[width=\textwidth]{figs/coarea.png} \vspace{-10pt}
    \caption{Coarea loss ablation. Coarea loss favors solution with low surface area. Here, note how larger coarea weight tends to fill in the cavities and close the gaps in the shape. In contrast, very low weight fails to drive the optimization procedure towards a solution with smaller surface area. } \label{fig:coarea}
    \vspace{-10pt}
\end{figure}

\begin{figure}[ht!]
\centering
    \includegraphics[width=0.8\textwidth]{figs/sparse4.png}\vspace{-10pt}
    \caption{VisCo reconstructions from sparse point cloud inputs. }\vspace{-10pt} \label{fig:sparse}
\end{figure}


\subsection{Reconstructing from sparse point clouds}
\label{sec:sparse}
We now evaluate our model ability to reconstruct surfaces in the challenging case of sparse input point clouds. For this experiment we use point clouds from the Stanford 3D Scanning Repository and downsample them at different levels: 1\%, 10\%, and 25\%. In Figure~\ref{fig:sparse} we visualize the VisCo reconstructions. Note that VisCo can reconstruct the shapes even with sparse input. This provides a further validation for the proposed geometrical priors.\vspace{-5pt}




