\documentclass[aps, prx,twocolumn,longbibliography,superscriptaddress,amsmath,amssymb]{revtex4}
\usepackage[latin9]{inputenc}
\usepackage{amssymb}
\usepackage{graphicx}
\usepackage{amsmath}
\usepackage{color}
\usepackage{mathrsfs}
\usepackage{float}
\usepackage{indentfirst}
\usepackage{mathrsfs}
\usepackage{float}
\usepackage{indentfirst}
\usepackage{textcomp}
\usepackage{comment}
\usepackage{mathtools}
\usepackage{natbib,hyperref}
\usepackage{soul}                                       

\begin{document}

\title{Supplementary materials for ``coexistence of superconductivity with partially filled stripes in the Hubbard model"}

\author{Hao Xu}
\thanks{These two authors contributed equally to this work.}
\affiliation{Department of Physics, College of William and Mary, Williamsburg, Virginia 23187, USA}


\author{Chia-Min Chung}
\thanks{These two authors contributed equally to this work.}
\affiliation{Department of Physics, National Sun Yat-sen University, Kaohsiung 80424, Taiwan}
\affiliation{Center for Theoretical and Computational Physics, National Sun Yat-Sen University, Kaohsiung 80424, Taiwan}
\affiliation{Physics Division, National Center for Theoretical Sciences, Taipei 10617, Taiwan}

\author{Mingpu Qin}
\affiliation{Key Laboratory of Artificial Structures and Quantum Control, School of Physics and Astronomy, 
Shanghai Jiao Tong University, Shanghai 200240, China}

\author{ Ulrich Schollw\"{o}ck}
\affiliation{Arnold Sommerfeld Center for Theoretical Physics, Ludwig-Maximilians-Universit\"{a}t M\"{u}nchen, 80333 Munich, Germany}
\affiliation{Munich Center for Quantum Science and Technology (MCQST), 80799 Munich, Germany}

\author{Steven R. White}
\affiliation{Department of Physics and Astronomy, University of California, Irvine, California 92697, USA}

\author{Shiwei Zhang}
\affiliation{Center for Computational Quantum Physics, Flatiron Institute, New York, NY 10010, USA}






\maketitle


\section{Partial particle-hole transformation of the Hubbard model}
When pairing fields are applied in the Hubbard model, the total particle number is not conserved. 
The usual ground-state AFQMC is formulated in the space of Slater determinant with a fixed electron number. While a more general solution is to reformulate AFQMC in 
Hartree-Fock-Bogoliubov (HFB) space \cite{HFB-AFQMC-Shi-2017},
the problem here can be solved without modifying the AFQMC codes, by applying a partial particle-hole transformation \cite{PhysRevX.10.031016} 
\begin{eqnarray}
\hat{c}_{i\uparrow} & \rightarrow & \hat{d}_{i\uparrow}, \quad \hat{c}_{i\uparrow}^{\dagger}  \rightarrow  \hat{d}_{i\uparrow}^{\dagger} \nonumber \\ 
\hat{c}_{i\downarrow} & \rightarrow & \hat{d}_{i\downarrow}^{\dagger}(-1)^{i}, \quad \hat{c}_{i\downarrow}^{\dagger} 
\rightarrow  \hat{d}_{i\downarrow}(-1)^{i}\,, 
\label{eq:partial_ph_trans}
\end{eqnarray}
where $i$ labels the lattice sites in the bipartite lattice.
With this transformation, the $t'$ Hubbard Hamiltonian in Eq.~(1) in the main text turns into
\begin{equation} \label{hubbard_h}
\begin{split}
\hat{H} = -t\sum_{\langle i,j\rangle\sigma}\hat{d}_{i\sigma}^{\dagger}\hat{d}_{j\sigma} - t'\sum_{\langle\langle i,j\rangle\rangle\sigma}s(\sigma)\hat{d}_{i\sigma}^{\dagger}\hat{d}_{j\sigma}  \\
+U\sum_{i}(\hat{m}_{i\uparrow}-\hat{m}_{i\downarrow}\hat{m}_{i\uparrow})
-\mu\sum_{i}(\hat{m}_{i\uparrow} + 1 - \hat{m}_{i\downarrow})\,, 
\end{split}
\end{equation}
where
$s(\uparrow)=+1$ and $s(\downarrow)=-1$,
and $\hat{m}_{i,\sigma}=\hat{d}_{i,\sigma}^{\dagger}\hat{d}_{i,\sigma}$. 
Note that the next near-neighbor
hopping changes sign for down spins after the transformation. The pairing operator $\hat{\Delta}_{ij}\ = (\hat{c}_{i\uparrow}\hat{c}_{j\downarrow}-\hat{c}_{i\downarrow}\hat{c}_{j\uparrow})/\sqrt{2}$ is transformed to: 
\begin{equation}
\hat{\Delta}_{ij}=((-1)^{j+1}\hat{d}_{j\downarrow}^{\dagger}\hat{d}_{i\uparrow}-(-1)^{i}\hat{d}_{i\downarrow}^{\dagger}\hat{d}_{j\uparrow}))/\sqrt{2}
\label{pairing_operator}
\end{equation}
which is now a spin-flip hopping term. The sign of $U$ is
reversed, meaning the interaction turns to attractive. 
Up and down electron now acquire effective chemical potentials $\mu - U$ and $-\mu$, respectively, 
which means $\langle \sum_{i}(\hat{m}_{i\uparrow}\rangle \ne \langle \sum_{i}(\hat{m}_{i\downarrow}\rangle$. 
After the transformation we have
 \begin{equation}
\sum_{i}(\hat{m}_{i\uparrow}+\hat{m}_{i\downarrow})=\sum_{i}(\hat{n}_{i\uparrow}+1-\hat{n}_{i\downarrow})=N_{s},
\end{equation}
 such that
 the total number of electrons equals the number of sites, 
 i.e., the system is at half-filling but
 with spin imbalance.
The random walkers (Slater determinants) are
now represented as $2N \times N_e$ matrix \cite{PhysRevB.94.085103} in the AFQMC calculation, and each orbital in the Slater determinant is now a spin-orbital with a mixture of up and down 
components. 



\section{Twist boundary conditions}

Twist boundary conditions (TBC) in $x$-direction means the wave-function satisfies: 
\begin{equation}
\psi ({\mathbf r}_1 + L {\hat e_x}, {\mathbf r}_2, \cdots,  {\mathbf r}_N) =  e^{i\theta_x} \psi({\mathbf r}_1, {\mathbf r}_2, \cdots, {\mathbf r}_N)\,.
\end{equation}
For convenience  we have used $L$ to denote $L_x$, the linear dimension of the periodic cell in 
$x$-direction.
(The definitions of TBC in other directions are similar).
For the two dimensional systems studied in this work, the phases for the two directions are independent of each other, and the phase factors from $x$- and $y$-directions are multiplicative. 
Thus, with no loss of generality, we will  only explicitly write out 
one dimension ($x$) below.
We will assume that the lattice sites are labeled  from $1$ to $L$.

Different gauges can be adopted to realize TBC. 
We discuss two common
choices here. In gauge A, an electron picks up a phase only when it crosses the boundary, while in
gauge B, the phase is split over all bonds evenly. 

\subsection{Gauge A}
In gauge A, if we apply the same twist 
for $\uparrow$ and $\downarrow$-spins in the repulsive Hubbard model, we have
\begin{eqnarray}
c^{\dagger}_{L+1,\sigma} &=& \exp(i\theta) c^{\dagger}_{1,\sigma}  \nonumber \\
c_{L+1,\sigma} &=& \exp(-i\theta) c_{1,\sigma}
\end{eqnarray}
So the hopping between the last and first site is modified as
\begin{equation}
-t\hat{c}_{1\sigma}^{\dagger}\hat{c}_{L\sigma}+h.c \rightarrow 
-t\exp(i\theta)\hat{c}_{1\sigma}^{\dagger}\hat{c}_{L\sigma}+h.c
\label{phase_1}
\end{equation}
while the other hopping terms remains unchanged. After the partial particle-hole transformation in Eq.~(\ref{eq:partial_ph_trans}), the spin up term is unchanged, but the phase for down spin changes to $-\theta$ as
\begin{equation}
-t\exp(-i\theta)\hat{d}_{1\downarrow}^{\dagger}\hat{d}_{L\downarrow}+h.c
\label{phase_2}
\end{equation}
The same is true for the $t'$ term.


With TBC,
we also need to modify the definition of the pairing operator ($\hat{\Delta}_{kj}\ = (\hat{c}_{k\uparrow}\hat{c}_{j\downarrow}-\hat{c}_{k\downarrow}\hat{c}_{j\uparrow})/\sqrt{2}$, for
bonds connecting nearest-neighbor sites, $\langle jk\rangle$) for the bond connecting the first and last site as
\begin{equation}
\hat{\Delta}_{1L}=\exp(-i\theta)(\hat{c}_{1\uparrow}\hat{c}_{L\downarrow}-\hat{c}_{1\downarrow}\hat{c}_{L\uparrow})/\sqrt{2}\,.
\label{phase_pairing}
\end{equation}
When applying the pairing 
field to 
calculate the pairing order, we need to include the phase in Eq.~(\ref{phase_pairing}) when twist boundary conditions are imposed. In the AFQMC calculation, the pairing operator in Eq.~(\ref{phase_pairing}) can be transformed, following Eq.~(\ref{eq:partial_ph_trans}), as
\begin{equation}
\hat{\Delta}_{1L}=\exp(-i\theta)(\hat{d}_{1\downarrow}^\dagger\hat{d}_{L\uparrow}-(-1)^L\hat{d}_{L\downarrow}^\dagger\hat{d}_{1\uparrow})/\sqrt{2}
\label{phase_pairing-U}
\end{equation}
Other pairing terms are transformed to the $-U$ case following Eq.~(\ref{pairing_operator})


\subsection{Gauge B}

We next consider gauge B in a similar setup to Gauge A. 
Now the phase is spread evenly over 
each bond and we have, for the repulsive model
\begin{eqnarray}
c_{j\sigma}^{\dagger}&\rightarrow& c_{j\sigma}^{\dagger}\exp(i(j-1)\frac{\theta}{L}) \nonumber \\ c_{j\sigma}&\rightarrow& c_{j\sigma}\exp(-i(j-1)\frac{\theta}{L})  
\end{eqnarray}
and
\begin{eqnarray}
c^{\dagger}_{L+1,\sigma} &=& \exp(i\theta) c^{\dagger}_{1,\sigma}  \nonumber \\
c_{L+1,\sigma} &=& \exp(-i\theta) c_{1,\sigma}
\end{eqnarray}
The nearest neighbor hopping term is then modified to 
\begin{equation}
-t\sum_{j}\exp(i\theta/L) \hat{c}_{j+1\sigma}^{\dagger}\hat{c}_{j\sigma} + h.c.
\end{equation}
The $t'$ term has similar form. 

The pairing operator $\hat{\Delta}_{kj}\ = (\hat{c}_{k\uparrow}\hat{c}_{j\downarrow}-\hat{c}_{k\downarrow}\hat{c}_{j\uparrow})/\sqrt{2}$ is modified as
\begin{equation}
  \hat{\Delta}_{kj}\ = (\hat{c}_{k\uparrow}\hat{c}_{j\downarrow}-\hat{c}_{k\downarrow}\hat{c}_{j\uparrow})\exp(-i(k+j-2)\theta/L)/\sqrt{2}
   \label{pair-2}
\end{equation}
For the bond connecting the first and last site, we have
\begin{equation}
  \hat{\Delta}_{L,1}\ = (\hat{c}_{L\uparrow}\hat{c}_{1\downarrow}-\hat{c}_{1\downarrow}\hat{c}_{L\uparrow})\exp(-i(2L-1)\theta/L)/\sqrt{2}
   \label{pair-2-b}
\end{equation}
We can then follow the particle-hole transformation in Eq.~(\ref{eq:partial_ph_trans}) 
to transform the definition of pairing order to the negative $U$ model, which is used in the AFQMC calculation.


\subsection{The equivalence of the two gauges}
The two gauges discussed above are equivalent and physical quantities should have the same values under them, 
which we have explicitly verified.
Since the interaction term is independent of the twist angle, it is convenient
to test the TBC implementation in 
non-interacting systems.
For example, 
in a $20\times4$ lattice with $t'=-0.2t$, $\mu=0.8$, and twist angle $\theta_x = 1.2994\pi, \theta_y = 0.6026\pi $, 
it is easily checked in all our codes  
that physical quantities, such as the energy per site ($-1.15861112$), average pairing order per bond ($0.01108939$), and the electron density ($0.82422077$), are all exactly the same under the two gauges. 

\section{Self-consistent constraint in
AFQMC} 
We apply magnetic and pairing pinning fields to probe the corresponding response in the studied systems. A self-consistent 
procedure in AFQMC allows us to apply a constraint to remove the sign problem. 
We describe the pinning field calculations and the self-consistency procedure below.

The magnetic pinning fields are typically 
applied in cylindrical cells, to one or both ends of the cylinder, not 
in the rest of the cell. 
We try different configurations of the magnetic pinning fields to probe the possible magnetic order or correlation.
The strength of the fields is fixed at $h_m = 0.25$, and limited to only the 
edge(s) of the cylinder.
For most of the systems, we applied anti-ferromagnetic pinning fields at the open edges of the studied cylinders ($(-1)^{(i_x + i_y)}h_m$ for $i_x = 1$ and $L_x$). In some cases, we also 
test a pinning field configuration with 
a $\pi$ phase to the pinning magnetic fields on the right edge as ($(-1)^{(i_x + i_y)}h_m$ for $i_x = 1$ and $(-1)^{(i_x + i_y + 1)}h_m$ for $i_x = L_x$), and compare the energies to determine the true ground state, the one with the lower energy. Note that it is important in this scheme to examine progressively larger (longer) systems, in order to remove the 
effect of the local pinning field.
Ref.~\cite{PhysRevResearch.4.013239} includes further details of our analysis 
method and how we extract information in the TDL.

\begin{figure}[t]
\centering{
	\includegraphics[width=0.98\linewidth] {iteration_example.pdf}
 }
	\caption{ Robustness of the self-consistent constraint in AFQMC. 
     An example is shown for
     the pairing order parameter 
     in the $1/4$ hole-doped Hubbard model on a $16 \times 4$ cylinder. 
     On the left panel, the self-consistent calculation starts with an initial value of 
     $\alpha = 0.1$. The calculation converges with a handful or iterations 
     to the exact result (DMRG, red) in this system, with a converged value
     $\alpha = 0.43$.
     On the right panel, the calculation is initialized with $\alpha=1.0$, and converges from the opposite direction to the same result. 
     }
	\label{iteration_example}
\end{figure}


To compute the pairing order parameter, 
we apply global pairing fields 
across the entire simulation cell,
similar to Ref.~\cite{PhysRevX.10.031016}.
To probe the $d$-wave pairing response
the applied pair-inducing fields 
on vertical and horizontal bonds have the same strength $h_d$ but opposite signs.
The Hamiltonian with pairing fields is
\begin{equation}
\hat{H}'(h_d) = \hat{H} +  \hat{H_d}(h_d)
\end{equation}
where 
\begin{equation}
\hat{H_d}(h_d) = -h_d \sum_{\langle i,j\rangle} b_{ij} \frac{\hat{\Delta}_{ij} + \hat{\Delta}^{\dagger}_{ij}}{2}\,, 
\end{equation}
where $b_{ij}=+1$ for 
a bond connecting two nearest-neighbors $i$ and $j$ in the $x$-direction and  $b_{ij}=-1$ if $\langle ij\rangle$ is in the $y$-direction.


The pairing order is calculated from the derivative of the ground state energy $E'(h_d)$ with respect to $h_d$, following the Hellmann-Feynman theorem.
Recall these calculations are performed 
in the particle-hole-transformed 
attractive Hubbard Hamiltonian in 
Eq.~(\ref{hubbard_h}).
We take the ground state of the non-interacting Hamiltonian ($U=0$ in Eq.~(\ref{hubbard_h})) with the pairing field $\alpha h_d$ as trial wave function,
where $\alpha$ is a parameter to be determined in the self-consistent iterations.  In the first step, we choose an arbitrary $\alpha$ and calculate the average value of pairing order with AFQMC
using the corresponding trial wave function as a constraint. We then tune the value of $\alpha$ 
by minimizing the difference between the 
pairing orders given by the
non-interacting wave-function
 and the previous iteration of AFQMC.
We then carry out the next iteration AFQMC calculation with the new trial wave-function. This process is repeated until $\alpha$ 
is converged. 
In each mean-field solution we tune the value of chemical potential to target the desired spin imbalance 
(i.e., the electron density in the 
repulsive model). 
In performing TABC, we determine the final 
value of $\alpha$ via averaging over different twist angles. The value is 
found to converge quickly, so a small 
set of pilot calculations can be performed 
first to obtain a good estimate. More computations can be added if further precision is needed.

In Fig.~\ref{iteration_example}, we show the computed pairing order in the self-consistent process for the $1/4$ hole doped $t'$ Hubbard model on a $16 \times 4$ cylinder. DMRG results are also shown, because for this narrow system, 
it provides a reference result which is essentially exact. 
In Fig.~\ref{iteration_example}, we start the self-consistent calculation with an initial value $\alpha = 0.1$ (the left panel). After 6 iterations,
the pairing order converges to the DMRG results. The converged value of $\alpha$ is $\alpha = 0.43$.
We also obtain the same converged pairing order and $\alpha$ value by starting the self-consistent process with
$\alpha = 1.0$ (the right panel), indicating the self-consistent calculation is independent of the initial value of $\alpha$. 



\section{Sensitivity of order to system sizes and boundary conditions}
\begin{figure}[t]
	\includegraphics[width=0.98\linewidth]{hsz2_holedoped.pdf}        
	\caption{ Strong sensitivity of the spin and charge orders to system sizes and boundary conditions. Here we show results for four different systems, all with the same bulk parameters ($\delta=1/8$, electron doped), but different in size and boundary conditions. Line cuts of the doped electron density (top panels) and staggered spin density (bottom) are shown, with combinations of two system sizes, $16\times4$ and $16\times6$, and periodic (PBC) and antiperiodic (APBC) boundary conditions. Antiferromagnetic pinning fields have been applied at the left and right edges of the open cylinders. 
 Two distinct types of states appear, N\'eel AFM with fairly uniform electron densities (left panels), and filled stripe states (right panels). 
 Good agreement is found between DMRG (filled symbols) and AFQMC (empty symbols); the differences are tied to the sizes and boundary conditions. 
   } 
	\label{size-effect}
\end{figure} 
In Fig.~\ref{size-effect} we show an example of the strong sensitivity of the ground states to system sizes and boundary conditions (BCs).  Two entirely different ground states are obtained for the same physical parameters from four different combinations of size/BCs, 
with an alternation between the effects of size versus BC. 
Consistent results are seen from both methods. 
This sort of sensitivity is also observed on the hole-doped side (see Fig.~\ref{stripe-hole}). 
To determine the order in the thermodynamic limit in these systems thus requires computations in significantly larger sizes than 
has been previously reached.
Below, we also show the presence of numerous low-lying states whose ordering in energy can be affected by size and BCs. In many cases these low-lying states can be tied to different stripe configurations.  In the case of Fig.~\ref{size-effect}, the system may be close to a phase boundary between the two types of states\cite{doi:10.1073/pnas.2109978118}. 
The twist-averaging procedure adopted here tends to average over the various states, which allows better extrapolation to the TDL 
compared to earlier approaches\cite{2019Sci...365.1424J}.



\section{Level crossing with applied pairing fields}

\begin{figure}[t]
\centering{
\includegraphics[width=0.49\linewidth]{compare.pdf}
\includegraphics[width=0.49\linewidth]{pin.pdf}
}
\caption{
Illustration of low-lying states and level-crossing.
The pairing order $\Delta_d$ (a) and energy per site (b) are shown as functions of the global pairing field strength $h_d$ for a $1/5$ hole doped system in a $20\times 4$ cylinder. The error bars are smaller than the symbol sizes.
In (c) and (d) the staggered spin densities are shown along the $x$ direction for the three low-lying states from DMRG, for $h_d\approx0.0075$, the smallest $h_d$ we consider. The AF magnetic pinning fields are applied at the left open edge.
} 
\label{fig:low_lying}
\end{figure} 

 In addition to the enhanced sensitivity of the ground state to the boundary conditions and system sizes in the $t'$ Hubbard model,
 the evolution of the ground state with the strength of the applied pairing field 
 is subtle, and
 creates another computational challenge. 
In Fig.~\ref{fig:low_lying} (a) and (b), as an example, we show the evolution of a few low-lying states as a function of 
$h_d$,
in a $20 \times 4$ system at $1/5$ hole doping.
As $h_d$ decreases, the low-lying states,
which are separated by tiny energy differences (note the small energy scale in b),
exhibit crossovers between several branches.
The different branches are characterized by different numbers of stripes in the states, as shown in Fig.~\ref{fig:low_lying} (c) and (d).
That $d$-wave pairing field induces strong level crossings 
is another indication of the intimate connection between the fluctuation of the stripe state and superconductivity in the system.
As mentioned, such level-crossings 
make the comparisons between DMRG and AFQMC more challenging,
as each calculation is often sensitive to 
even small variations in the calculational parameters. 
However this effect is reduced by employing twist averaging, in which all the low-lying states are sampled. As can be seen 
in Fig.~5 
in the main text, 
TABC effectively treats the crossovers 
as a function of $h_d$, which results
in smooth curves, and  
DMRG and AFQMC agree very well. 

\section{Supplemental data}

\subsection{Hole density for $1/8$ hole doped systems}
\begin{figure}[t]
        \centering{
	\includegraphics[width=0.98\linewidth]{hole_overall_2.pdf}
        }
	\caption{
 Hole density for the systems in Fig.~2 in the main text for $1/8$ hole doping. 
 The evolution of the stripe patterns is shown versus system size.
 The hole densities are shown as linecuts 
 along the length of the cylinders. 
 The length of the cylinder ($L_x$) is varied across 
 the three columns 
and the width ($L_y$) across rows. 
AFM pinning fields are applied at the two edges of the cylinder ($x=1$ and $x=L_x$), either in phase or with a $\pi$-phase shift
(marked by an asterisk);
the one with lower energy is shown.
The filling fraction $f$ of each stripe pattern is indicated, with NIPS denoting non integer-pair stripes. 
 DMRG results (red) are shown for width-4 and 6 systems, and AFQMC (black) are in good agreement with them.
} 
\label{stripe-hole}
\end{figure} 
In Fig.~\ref{stripe-hole}, we show the hole density for the systems in Fig.~2 in the main text. For systems with width 4 and 6, for which DMRG is available, 
we find good agreement between AFQMC amd DMRG results.
In width-6 systems, the discrepancies are somewhat larger in the density here compared to the spin 
in Fig.~2 in the main text. 
This is likely because of a combination of two factors. First AFQMC has shown in 
$t'=0$ Hubbard model a slight tendency to under-estimate the amplitude of the density fluctuations in 
stripes \citep{PhysRevResearch.4.013239, PhysRevB.94.235119}. Second, in some cases we have seen indications that the DMRG may not have reached full convergence in width-6 systems, even with the very large bond dimensions we were able to do.

\subsection{Additional data on pairing order parameter}
In this subsection, we include  
the finite size data for the pairing order, as well as the extrapolation process to obtain the spontaneous pairing order in the thermodynamic limit
which is plotted in Fig.~1 in the main text.

\subsubsection{Electron doped region}

In 
Figs.~\ref{pair-extra-1-8-ele}, \ref{pair-extra-1-5-ele}, and \ref{pair-extra-1-3-ele}
we present the data 
for pairing order in the electron-doped region, with $\delta = 1/8, 1/5$, and $1/3$ respectively. 
As discussed in the main text, 
we use fully periodic systems (i.e., 
torus simulations cells of $L_x\times L_y$) and perform TABC with quasi-random 
twists. We have verified that the 
results have converged with respect to 
$L_x$ to within our statistical error. 
When extrapolating the pairing orders with width of the system, we omit width-4 systems.
When extrapolating the TDL values versus $h_d$, we perform both linear and quadratic fits. 
In the linear fits, we use $h_d<0.05$,
when the data is clearly in the linear response regime. In the quadratic fits,
we include more data points,  
10-12 values of $h_d$. 
We find that the
resulting extrapolated $\Delta_d(h_d\rightarrow 0)$ values are often indistinguishable.
When there is a difference, the result from the better fit (smaller $\chi^2$) is used.
The final pairing order at TDL is $0.007(3),0.007(4)$, and $0.000(2)$ for $\delta = 1/8, 1/5$, and $1/3$, respectively,
as reported in Fig.~1 in the main text.


\begin{figure*}[t]
        \centering{
	\includegraphics[width=0.32\linewidth]{pairing_a1}
        \includegraphics[width=0.32\linewidth]{pairing_b1}
        \includegraphics[width=0.32\linewidth]{pairing_c1}
	}
        \caption{The extrapolation of pairing order for 1/8 electron doping. 
        (a) shows the pairing order
parameter versus pairing field for different system sizes. 
The data for each torus of $L_x\times L_y$ is obtained from TABC with quasi-random 
twists $(k_x,k_y)$. The number of twists
is a few dozens for smaller systems and
about a dozen 
for the larger systems. 
(b) shows extrapolation of the pairing order with the
width of the system for each fixed pairing field. In (c) the extrapolated value in (b) is plotted against
pairing field and then an extrapolation of the pairing field to $h_d\rightarrow 0$ gives the spontaneous pairing order at
thermodynamic limit. Both linear and quadratic fits give consistent results.
} 
	\label{pair-extra-1-8-ele}
\end{figure*} 



\begin{figure*}[t]
        \centering{
	\includegraphics[width=0.32\linewidth]{pairing_a2}
        \includegraphics[width=0.32\linewidth]{pairing_b2}
        \includegraphics[width=0.32\linewidth]{pairing_c2}
	}
        \caption{Similar as Fig.~\ref{pair-extra-1-8-ele} but for $1/5$ electron doping.} 
	\label{pair-extra-1-5-ele}
\end{figure*} 


\begin{figure*}[t]
        \centering{
	\includegraphics[width=0.32\linewidth]{pairing_a3}
        \includegraphics[width=0.32\linewidth]{pairing_b3}
        \includegraphics[width=0.32\linewidth]{pairing_c3}
	}
        \caption{Similar as Fig.~\ref{pair-extra-1-8-ele} but for $1/3$ electron doping.} 
	\label{pair-extra-1-3-ele}
\end{figure*} 



\subsubsection{Hole doped region}

In 
Figs.~\ref{pair-extra-1-8-hol}, \ref{pair-extra-1-5-hol}, \ref{pair-extra-1-4-hol}, and \ref{pair-extra-1-3-hol}
we present the data for 
pairing order in the hole-doped region, with $\delta = 1/8, 1/5, 1/4$, and $1/3$ respectively.
The final pairing order at TDL
for $\delta = 1/8, 1/5, 1/4$, and $1/3$ are $0.017(3),0.016(2),0.007(3)$, and $0.002(3)$, 
as shown in Fig.~1 in the main text.
\begin{figure*}[t]
        \centering{
	\includegraphics[width=0.32\linewidth]{pairing_a4}
        \includegraphics[width=0.32\linewidth]{pairing_b4}
        \includegraphics[width=0.32\linewidth]{pairing_c4}
	}
        \caption{Similar as Fig.~\ref{pair-extra-1-8-ele} but for $1/8$ hole doping.} 
	\label{pair-extra-1-8-hol}
\end{figure*} 

\begin{figure*}[t]
        \centering{
	\includegraphics[width=0.32\linewidth]{pairing_a7}
        \includegraphics[width=0.32\linewidth]{pairing_b7}
        \includegraphics[width=0.32\linewidth]{pairing_c7}
	}
        \caption{Similar as Fig.~\ref{pair-extra-1-8-ele} but for $1/5$ hole doping.} 
	\label{pair-extra-1-5-hol}
\end{figure*} 


\begin{figure*}[t]
        \centering{
	\includegraphics[width=0.32\linewidth]{pairing_a5}
        \includegraphics[width=0.32\linewidth]{pairing_b5}
        \includegraphics[width=0.32\linewidth]{pairing_c5}
	}
        \caption{Similar as Fig.~\ref{pair-extra-1-8-ele} but for $1/4$ hole doping.} 
	\label{pair-extra-1-4-hol}
\end{figure*} 



\begin{figure*}[t]
        \centering{
	\includegraphics[width=0.32\linewidth]{pairing_a6}
        \includegraphics[width=0.32\linewidth]{pairing_b6}
        \includegraphics[width=0.32\linewidth]{pairing_c6}
	}
        \caption{Similar as Fig.~\ref{pair-extra-1-8-ele} but for $1/3$ hole doping.} 
	\label{pair-extra-1-3-hol}
\end{figure*} 



\bibliography{supplement}

\end{document}
