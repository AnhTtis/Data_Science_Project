\documentclass[10pt,twocolumn,letterpaper]{article}

\usepackage{iccv}
\usepackage{times}
\usepackage{epsfig}
\usepackage{graphicx}
\usepackage{amsmath}
\usepackage{amssymb}
\usepackage{booktabs}
\usepackage{multirow}
\usepackage{caption}
\usepackage{subcaption}
\usepackage[table,xcdraw]{xcolor}
\newcommand{\bbox}{\text{bbox}}
\newcommand{\alphapck}{\alpha_\bbox}
\newcommand{\kcycle}{\text{k-CyPCK}}
\newcommand{\cycle}{\text{-CyPCK}}

\newcommand{\I}{\mathbf{I}}
\newcommand{\Ia}{\I^\text{a}}
\newcommand{\Ib}{\I^\text{b}}
\newcommand{\Iatob}{\I^\text{a $\rightarrow$ b}}
\newcommand{\F}{\mathbf{F}}
\newcommand{\Fa}{\F^\text{a}}
\newcommand{\Fb}{\F^\text{b}}
\newcommand{\f}{\mathbf{f}}
\newcommand{\fa}{\f^\text{a}}
\newcommand{\fb}{\f^\text{b}}
\newcommand{\p}{\mathbf{p}}
\newcommand{\pa}{\p^\text{a}}
\newcommand{\pb}{\p^\text{b}}
\newcommand{\A}{\boldsymbol{\Phi}_\text{align}}
\newcommand{\G}{\mathbf{G}}
\newcommand{\C}{\mathbf{C}}
\newcommand{\Ca}{\C^\text{a}}
\newcommand{\Cb}{\C^\text{b}}
\newcommand{\cc}{\mathbf{c}}
\newcommand{\cca}{\cc^\text{a}}
\newcommand{\ccb}{\cc^\text{b}}
\newcommand{\Irec}{\I_\text{Recon}}
\newcommand{\M}{\mathbf{M}}
\newcommand{\Mrec}{\M_\text{Recon}}
\newcommand{\loss}{\mathcal{L}}
\newcommand{\T}{\mathcal{T}}
\newcommand{\W}{\mathcal{W}}
\newcommand{\Id}{\mathcal{I}}

\usepackage{algorithm}
\usepackage{algorithmic}
\usepackage{rotating}

\newlength\savewidth\newcommand\shline{\noalign{\global\savewidth\arrayrulewidth
  \global\arrayrulewidth 1pt}\hline\noalign{\global\arrayrulewidth\savewidth}}
  
  \makeatletter
\def\@fnsymbol#1{\ensuremath{\ifcase#1\or \dagger\or \ddagger\or
   \mathsection\or \mathparagraph\or \|\or **\or \dagger\dagger
   \or \ddagger\ddagger \else\@ctrerr\fi}}
\makeatother

\definecolor{citecolor}{HTML}{0071bc}
\usepackage[pagebackref=true,breaklinks=true,letterpaper=true,colorlinks,  citecolor=citecolor,bookmarks=false]{hyperref}

\usepackage{xcolor}

\iccvfinalcopy
\def\httilde{\mbox{\tt\raisebox{-.5ex}{\symbol{126}}}}

% Pages are numbered in submission mode, and unnumbered in camera-ready
\ificcvfinal\pagestyle{empty}\fi

% added packages
% \usepackage{booktabs} % To thicken table lines
\usepackage{colortbl} % To add color to v-lines
\usepackage{graphicx} % To resize table
% \usepackage{subfigure} % To reference subfigures
\usepackage{enumitem}  % fancy enumerate
\usepackage{xspace}  % for adding space for abbreviations
% \usepackage[bottom]{footmisc}   % for placing tables/figures above footnote


% copy from cvpr template
% Add a period to the end of an abbreviation unless there's one
% already, then \xspace.
% \makeatletter
% \DeclareRobustCommand\onedot{\futurelet\@let@token\@onedot}
% \def\@onedot{\ifx\@let@token.\else.\null\fi\xspace}

% \def\eg{\emph{e.g}\onedot} \def\Eg{\emph{E.g}\onedot}
% \def\ie{\emph{i.e}\onedot} \def\Ie{\emph{I.e}\onedot}
% \def\cf{\emph{cf}\onedot} \def\Cf{\emph{Cf}\onedot}
% \def\etc{\emph{etc}\onedot} \def\vs{\emph{vs}\onedot}
% \def\wrt{w.r.t\onedot} \def\dof{d.o.f\onedot}
% \def\iid{i.i.d\onedot} \def\wolog{w.l.o.g\onedot}
% \def\etal{\emph{et al}\onedot}
% \makeatother



% general writing
% \interfootnotelinepenalty=10000
% % \newcommand{\train}{{\small\texttt{train}\xspace}}
% \newcommand{\val}{{\small\texttt{val}\xspace}}
% \newcommand{\test}{{\small\texttt{test}\xspace}}

\newcommand{\SPT}[0]{\textsc{SPT}}
\newcommand{\SPTc}[0]{\textsc{SPT-Conv}}
\newcommand{\SPTa}[0]{\textsc{SPT-Adapter}}
\newcommand{\SPTl}[0]{\textsc{SPT-LoRA}}
\newcommand{\vprompt}[0]{\textsc{Prompt}}
\newcommand{\deepprompt}[0]{\vprompt{}-\textsc{deep}}
\newcommand{\shallowprompt}[0]{\vprompt{}-\textsc{shallow}}

\newcommand{\partialft}[0]{\textsc{Partial}}
\newcommand{\linear}[0]{\textsc{Linear}}
\newcommand{\fullft}[0]{\textsc{Full}}
\newcommand{\sidetune}[0]{\textsc{Sidetune}}
\newcommand{\mlp}[0]{\textsc{Mlp}}
\newcommand{\bias}[0]{\textsc{Bias}}
\newcommand{\adapter}[0]{\textsc{Adapter}}
\newcommand{\adaptformer}[0]{\textsc{AdaptFormer}}
\newcommand{\lora}[0]{\textsc{LoRA}}
\newcommand{\noah}[0]{\textsc{NOAH}}
\newcommand{\promptbias}[0]{\textsc{VPT+Bias}}
\newcommand{\convpass}[0]{\textsc{Convpass}}
\newcommand{\ssf}[0]{\textsc{SSF}}



% FEATURES
\newcommand{\vit}[0]{ViT}
% \newcommand{\rn}[0]{ResNet}
\newcommand{\swin}[0]{Swin}
% \newcommand{\rnx}[0]{ConvNeXt}

\newcommand{\suplong}[0]{Supervised}
\newcommand{\supshort}[0]{Sup}
\newcommand{\moco}[0]{MoCo v3}
\newcommand{\moby}[0]{MoBY}
\newcommand{\mae}[0]{MAE}



% DATASETS
\newcommand{\imagenet}[0]{ImageNet}
\newcommand{\longcub}[0]{Caltech-UCSD Birds-200-2011}
\newcommand{\cub}[0]{CUB-200-2011}
\newcommand{\nabirds}[0]{NABirds}
\newcommand{\flowers}[0]{Oxford Flowers}
\newcommand{\cars}[0]{Stanford Cars}
\newcommand{\dogs}[0]{Stanford Dogs}
\newcommand{\vtab}[0]{VTAB-1k}


% Define some colors
\definecolor{tabvline}{HTML}{a8a495}
\definecolor{prompt_blue}{HTML}{1f78b4}
\definecolor{prompt_red}{HTML}{d45c43}

\definecolor{green_im}{rgb}{0.0, 0.5, 0.0}
% Table color formatting macros.
\newcommand{\ttbf}[1]{\textbf{\texttt{#1}}}
\newcommand{\band}{\rowcolor{gray!15}}
\newcommand{\graycell}{\cellcolor{gray!15}}

\newcommand{\gray}[1]{\textcolor{gray}{#1}}
\newcommand{\orangehl}[1]{\textcolor{RedOrange}{\textbf{#1}}}

\newcommand{\drop}[1]{\textcolor{gray}{\tiny{$\downarrow$#1}}}
\newcommand{\rise}[1]{\textcolor{gray}{\tiny{$\uparrow$#1}}}

\newcommand{\Drop}[1]{\textcolor{prompt_red}{\xspace\scriptsize{\bf $\downarrow$#1}}}
\newcommand{\Rise}[1]{\textcolor{green_im}{\xspace\scriptsize{\bf $\uparrow$#1}}}


% \newcommand{\para}[1]{\noindent\textbf{#1}}
\newcommand{\para}[1]{\subsubsection{#1}}



% Comments related
\newcommand{\draftcomment}[3]{\textcolor{#2}{\bf [#1: #3]}}


% change citecolor from green to light blue. include below line at the end of macro, before hyperref packages in main.tex
% \definecolor{citecolor}{RGB}{0, 113, 188}

% It is strongly recommended to use hyperref, especially for the review version.
% hyperref with option pagebackref eases the reviewers' job.
% Please disable hyperref *only* if you encounter grave issues, e.g. with the
% file validation for the camera-ready version.
%
% If you comment hyperref and then uncomment it, you should delete
% ReviewTempalte.aux before re-running LaTeX.
% (Or just hit 'q' on the first LaTeX run, let it finish, and you
%  should be clear).
% \usepackage[pagebackref,breaklinks,colorlinks]{hyperref}
% \usepackage[pagebackref,breaklinks,colorlinks,citecolor=citecolor]{hyperref}


% Support for easy cross-referencing after hyperref
% \usepackage[capitalize]{cleveref}
% \crefname{section}{Sec.}{Secs.}
% \Crefname{section}{Section}{Sections}
% \Crefname{table}{Table}{Tables}
% \crefname{table}{Tab.}{Tabs.}

% \newenvironment{packed_enum}{
% \begin{enumerate}
%   \setlength{\itemsep}{1pt}
%   \setlength{\parskip}{2pt}
%   \setlength{\parsep}{0pt}
% }{\end{enumerate}}

% \newenvironment{packed_item}{
% \begin{itemize}
%   \setlength{\itemsep}{1pt}
%   \setlength{\parskip}{2pt}
%   \setlength{\parsep}{0pt}
% }{\end{itemize}}


% % for equal contribution footnote
% \makeatletter
% \newcommand{\printfnsymbol}[1]{%
%   \textsuperscript{\@fnsymbol{#1}}%
% }
% \makeatother

\begin{document}
%%%%%%%%% TITLE
\title{Sensitivity-Aware Visual Parameter-Efficient Tuning}

\author{%
  Haoyu He$^{1}$ ~~ Jianfei Cai$^{1}$ ~~ Jing Zhang$^{2}$ \\  ~~ Dacheng Tao$^{2,3}$ ~~ Bohan Zhuang$^{1}\thanks{Corresponding author. E-mail: $\tt  bohan.zhuang@gmail.com$}$ \\ [0.25cm]
$^1$ Monash University \quad $^2$ The University of Sydney \quad $^3$ JD Explore Academy \\[0.1cm]
% 
}

\maketitle
% Remove page # from the first page of camera-ready.
\ificcvfinal\thispagestyle{empty}\fi

\begin{abstract}
Visual Parameter-Efficient Tuning (VPET) has become a powerful alternative for full fine-tuning so as to adapt pre-trained vision models to downstream tasks, which only tunes a small number of parameters while freezing the vast majority ones to ease storage burden and optimization difficulty. However, existing VPET methods introduce trainable parameters to the same positions across different tasks depending solely on human heuristics and neglect the domain gaps. To this end, we study where to introduce and how to allocate trainable parameters by proposing a novel \textbf{S}ensitivity-aware visual \textbf{P}arameter-efficient \textbf{T}uning (SPT) scheme, which adaptively allocates trainable parameters to task-specific important positions given a desired tunable parameter budget. Specifically, our SPT first quickly identifies the sensitive parameters that require tuning for a given task in a data-dependent way. Next, our SPT further boosts the representational capability for the weight matrices whose number of sensitive parameters exceeds a pre-defined threshold by utilizing any of the existing structured tuning methods, e.g., LoRA~\cite{hu2022lora} or Adapter~\cite{houlsby2019parameter}, 
to replace directly tuning the selected sensitive parameters (unstructured tuning) under the budget. Extensive experiments on a wide range of downstream recognition tasks show that our SPT is complementary to the existing VPET methods and largely boosts their performance, e.g., SPT improves Adapter with supervised pre-trained ViT-B/16 backbone by 4.2\% and 1.4\% mean Top-1 accuracy, reaching SOTA performance on FGVC and VTAB-1k benchmarks, respectively. Source code is at \url{https://github.com/ziplab/SPT}.
\end{abstract}

%%%%%%%%% BODY TEXT
\section{Introduction}


Recent years have witnessed the rise of human digitization~\cite{habermannDeepCapMonocularHuman2020,alexanderCREATINGPHOTOREALDIGITAL,pengNeuralBodyImplicit2021,alldieckDetailedHumanAvatars2018, rajANRArticulatedNeural2020}. This technology greatly impacts the entertainment, education, design, and engineering industry.
There is a well-developed industry solution for this task.
High-fidelity reconstruction of humans can be achieved either with full-body laser scans~\cite{saitoSCANimateWeaklySupervised2021}, dense synchronized multi-view cameras~\cite{xiangModelingClothingSeparate2021a,xiangDressingAvatarsDeep2022a}, or light stages~\cite{alexanderCREATINGPHOTOREALDIGITAL}.
However, these settings are expensive and tedious to deploy and consist of a complex processing pipeline, preventing the technology's democratization.

Another solution is to view the problem as inverse rendering and learn digital humans directly from custom-collected data.
Traditional approaches directly optimize explicit mesh representation~\cite{loperSMPLSkinnedMultiperson2015, fangRMPERegionalMultiperson2018, pavlakosExpressiveBodyCapture2019} which suffers from the problems of smooth geometry and coarse textures~\cite{prokudinSMPLpixNeuralAvatars2020,alldieckVideoBasedReconstruction2018}. Besides, they require professional artists to design human templates, rigging, and unwrapped UV coordinates.
Recently, with the help of volumetric-based implicit representations~\cite{mildenhallNeRFRepresentingScenes2020, parkDeepSDFLearningContinuous2019, meschederOccupancyNetworksLearning2019} and neural rendering~\cite{laineModularPrimitivesHighPerformance2020, liuSoftRasterizerDifferentiable2019, thiesDeferredNeuralRendering2019}, 
one can easily digitize a quality-plausible human avatar from video footage~\cite{jiangNeuManNeuralHuman2022,wengHumanNeRFFreeviewpointRendering}.
Particularly, volumetric-based implicit representations~\cite{mildenhallNeRFRepresentingScenes2020, pengNeuralBodyImplicit2021} can reconstruct scenes or objects with much higher fidelity against previous neural renderer~\cite{thiesDeferredNeuralRendering2019,prokudinSMPLpixNeuralAvatars2020}, and is more user-friendly as it does not need any human templates, pre-set rigging, or UV coordinates.
Captured visual footage and corresponding skeleton tracking are enough for training.
However, better reconstructions and more friendly usability are at the expense of the following factors.
1) \textbf{Inefficiency:}
They require longer optimization times (typically tens of hours or days) and inference slowly.
Volume rendering~\cite{mildenhallNeRFRepresentingScenes2020,lombardiNeuralVolumesLearning2019} formulates images by querying the densities and colors of millions of spatial coordinates. 
In the training stage, due to memory constraints, only a small fraction of points are sampled which leads to slow convergence speed.
2) \textbf{Entangled representations}:
The geometry, materials, and motion dynamics are entangled in the neural networks. 
Due to the implicit nature of neural nets, one can hardly edit one property without touching the others~\cite{yuanNeRFEditingGeometryEditing2022a,liuEditingConditionalRadiance2021}.
3) \textbf{Graphics incompatibility}:
Volume rendering is incompatible with the current popular graphic pipeline,
which renders triangular/quadrilateral meshes efficiently with the rasterization technique.
Many downstream applications require mesh rasterization in their workflow (\eg, editing~\cite{foundationBlenderOrgHome}, simulation~\cite{benderPositionBasedSimulationMethods2015}, real-time rendering~\cite{akenine2019real}, ray-tracing~\cite{waldRTXRayTracing}).
Although there are approaches~\cite{lorensenMarchingCubesHigh,labelleIsosurfaceStuffingFast2007} can convert volumetric fields into meshes, the gaps from discrete sampling degrade the output quality in terms of both meshes and textures.


To address these issues, we present \textbf{EMA}, a method based on \textbf{E}fficient \textbf{M}eshy neural fields to reconstruct animatable human \textbf{A}vatars.
Our method enjoys flexibility from implicit representations and efficiency from explicit meshes, yet still maintains high-fidelity reconstruction quality.
Given video sequences and the corresponding pose tracking, our method digitizes humans in terms of canonical triangular meshes, physically-based rendering (PBR) materials, and skinning weights \textit{w.r.t.} skeletons.
We jointly learn the above components via inverse rendering~\cite{laineModularPrimitivesHighPerformance2020,chenDIBRLearningPredict2021,chenLearningPredict3D2019} in an end-to-end manner.
Each of them is derived from a separate neural field, which relaxes the requirements of a preset human template, rigging, or UV coordinates.
Specifically, we predict a canonical mesh out of a signed distance field (SDF) by differentiable marching tetrahedra~\cite{shenDeepMarchingTetrahedra2021,gaoGET3DGenerativeModel,gaoLearningDeformableTetrahedral2020,munkbergExtractingTriangular3D2022}, then we extend the marching tetrahedra~\cite{shenDeepMarchingTetrahedra2021} for spatial-varying materials by utilizing a neural field to predict PBR materials \textit{on the mesh surfaces} after rasterization~\cite{munkbergExtractingTriangular3D2022,hasselgrenShapeLightMaterial2022,laineModularPrimitivesHighPerformance2020}.
To make the canonical mesh animatable, we take another neural field to model the forward linear blend skinning for the meshes. 
Given a posed skeleton, the canonical mesh is then transformed into the corresponding poses.
Finally, we shade the mesh with a rasterization-based differentiable renderer~\cite{laineModularPrimitivesHighPerformance2020} and train our models with a photo-metric loss.
After training, we export the mesh with materials and discard the neural fields.

\looseness=-1
There are several merits of our method design.
1) \textbf{Efficiency}:
Powered by efficient mesh rendering, our method can render in real-time.
Besides, the training speed is boosted as well, 
since we compute loss holistically on the whole image and the gradients only flow on the mesh surface. In contrast, volume rendering takes limited pixels for loss computation and back-propagates the gradients in the whole space.
Our method only needs about an hour of training and minutes of optimization are enough for plausible avatar reconstruction.
2) \textbf{Disentangled representations}:
Our shape, materials, and motion modules are disentangled naturally by design, which facilitates editing. 
Besides, Canonical meshes with forward skinning modeling handle the out-of-distribution poses better.
3) \textbf{Graphics compatibility}:
Our derived mesh representation is compatible with 
the prominent graphic pipeline, which leads to instant downstream applications (\eg, the shape and materials can be edited directly in design software~\cite{foundationBlenderOrgHome}).
To further improve reconstruction quality, we additionally optimize image-based environment lights and non-rigid motions.


We conduct extensive experiments on standards benchmarks H36M~\cite{ionescuHuman36MLarge2014b} and ZJU-MoCap~\cite{pengNeuralBodyImplicit2021}.
Our method achieves very competitive performance for novel view synthesis, generalizes better for novel poses, 
and significantly improves both training time and inference speed against previous arts.
Our research-oriented code reaches real-time inference speed (100+ FPS for rendering $512\times512$ images).
We in addition showcase applications including novel pose synthesis, material editing, and relighting.
\vspace{-4mm}
\section{Related Works}
\noindent\textbf{Sign Language Recognition.} Sign language recognition (SLR) is a fundamental task in the field of sign language understanding.
Feature extraction plays a key role in an SLR model.
% 
Most recent SLR works \cite{jiang2021sign, jiang2021skeleton, hu2021signbert, hu2021hand, li2020transferring, li2020word, joze2019ms, hu2021global, stmc, zuo22_interspeech, vac} adopt CNN-based architectures, \eg, I3D \cite{I3D} and R3D \cite{qiu2017learning}, to extract vision features from RGB videos.
In this work, we adopt S3D \cite{xie2018rethinking} as the backbone of our VKNet due to its excellent accuracy-speed trade-off.

However, RGB-based SLR models may suffer from the large variation of video backgrounds. 
As a complement, some SLR works \cite{jiang2021skeleton, jiang2021sign, hu2021hand, hu2021signbert, chentwo} explore to jointly model RGB videos and keypoints.
For example, SAM-SLR \cite{jiang2021skeleton} uses graph convolutional networks (GCNs) to model pre-extracted keypoints.
HMA \cite{hu2021hand} and SignBERT \cite{hu2021signbert} propose to decode 3D hand keypoints from RGB videos.
A common deficiency of these works is that they need a dedicated network to model keypoints.
In this work, we represent keypoints as a sequence of heatmaps~\cite{duan2022revisiting, chentwo} so that the keypoint encoder of our VKNet can share the identical architecture with the video encoder.

To enable mini-batch training, previous works \cite{jiang2021sign, jiang2021skeleton, hu2021signbert, hu2021hand, li2020transferring, li2020word} crop fixed-length clips from raw videos as model inputs.
However, the model may overfit to the training videos of fixed temporal receptive fields.
In contrast, our VKNet is trained on videos with varied temporal receptive fields to improve its generalization capability.



\noindent\textbf{Word Representation Learning.}
Word2vec \cite{word2vec} and GloVe \cite{glove} are two classical word representation learning frameworks in the field of NLP.
Based on word2vec, fastText \cite{mikolov2018advances} improves word representations with several modifications including the use of sub-word information \cite{bojanowski2017enriching} and position independent features \cite{mnih2013learning}.
Although some advanced language models, \eg, BERT \cite{kenton2019bert}, can also be used to extract word representations, they are computationally intensive and are not dedicated to word representation learning.
In this paper, we adopt the lightweight but effective fastText, which is also used in a recent sign language translation work \cite{yin2021simulslt}, to pre-compute gloss (word) representations.


\noindent\textbf{Vision-Language Models.}
Recently, a majority of vision-language models \cite{clip, align, yao2022filip, gu2022wukong} learn visual representations on large-scale image-text pairs.
Among them, CLIP \cite{clip} is the pioneer to jointly optimize an image encoder and a text encoder through a contrastive loss. 
% 
Besides, the pre-trained CLIP can be generalized to various downstream tasks, \eg, semantic segmentation \cite{xu2022groupvit, li2021language, xu2021simple}, object detection \cite{du2022learning, rao2022denseclip}, image classification~\cite{zhou2022learning,huang2022unsupervised}, and style transfer \cite{patashnik2021styleclip, kwon2022clipstyler}.
In this work, we exploit the implicit knowledge included in glosses (sign labels), which is distinct from previous works on vision-language modeling.


\noindent\textbf{Multi-label Classification.} Real-world objects may have multiple semantic meanings, which motivates research on multi-label classification \cite{ridnik2021asymmetric, ke2022hyperspherical, zhang2013review, rajeswar2022multi, kim2022large} requiring models to map inputs to multiple possible labels.
Although the VISigns may be associated with the multi-label classification problem, most widely-adopted SLR datasets \cite{li2020word, joze2019ms, hu2021global} are singly labeled.
In this work, we deal with the VISigns by incorporating language information included in glosses.

\vspace{-0.3em}
\section{Method}
\vspace{-0.3em}

Our sensitivity-aware visual parameter-efficient fine-tuning consists of two stages. In the first stage, SPT measures the task-specific sensitivity for the pre-trained parameters (Section~\ref{subsec:sensitivity}). Based on the parameter sensitivity and a given parameter budget, SPT then adaptively allocates trainable parameters to task-specific important positions (Section~\ref{subsec:SPT}).

\vspace{-0.3em}
\subsection{Task-specific Parameter Sensitivity}
\label{subsec:sensitivity}
\vspace{-0.3em}

Recent research has observed that pre-trained backbone parameters exhibit varying feature patterns~\cite{raghu2021vision,naseer2021intriguing} and criticality~\cite{zhang2019all,chatterji2019intriguing} at distinct positions. 
Moreover, when transferred to downstream tasks, their efficacy varies depending on how much pre-trained features are reused and how well they adapt to the specific domain gap~\cite{yosinski2014transferable,kumar2022finetuning,neyshabur2020being}. Motivated by these observations, we argue that not all parameters contribute equally to the performance across different tasks in PEFT and propose a new criterion to measure the sensitivity of the parameters in the pre-trained backbone for a given task.

Specifically, given the training dataset $\gD_t$ for the $t$-th task and the pre-trained model weights $\vw=\left\{w_1, w_2, \ldots, w_N\right\}\in \sR^N$ where $N$ is the total number of parameters, the objective for the task is to minimize the empirical risk: $\min_{\vw} E(\gD_t, \vw)$.
We denote the parameter sensitivity \bohan{set} as $\gS=\{s_1, \ldots, s_N\}$ and the sensitivity $s_n$ for parameter $w_n$ is measured by the empirical risk difference when tuning it:
\begin{equation}
\vspace{-0.3em}
    \begin{aligned}
        s_n = E(\gD_t, \vw)-E(\gD_t, \vw\mid w_n=w_n^*),
    \end{aligned}
\label{eq:sensitivity}
\end{equation}
where $w_n^*=\underset{w_n}{\rm argmin}(E(\gD_t, \vw))$. We can reparameterize the tuned parameters as  $w_n^*=w_n+\Delta_{w_n}$, where $\Delta_{w_n}$ denotes the update for $w_n$ after tuning. Here we individually measure the sensitivity of each parameter, which is reasonable given that most of the parameters are frozen during fine-tuning in PEFT. However, it is still computationally intensive to compute Eq.~(\ref{eq:sensitivity}) for two reasons. Firstly, getting the empirical risk for $N$ parameters requires forwarding the entire network $N$ times, which is time-consuming. Secondly, it is challenging to derive $\Delta_{w_n}$, as we have to tune each individual $w_n$ until convergence.

{\begin{algorithm}[t!]
\caption{\label{alg:tps} Computing task-specific parameter sensitivities}
\begin{algorithmic}
    \STATE \textbf{Input:} Pre-trained model with network parameters $\vw$, training set $\gD_t$ for the $t$-th task, and number of training samples $C$ used to calculate the parameter sensitivities
    \STATE \textbf{Output:} Sensitivity set $\gS=\{s_1, \ldots, s_N\}$
    \STATE Initialize $\gS=\{0\}^N$
    \FOR{$i\in\{1,\ldots,C\}$}
        \STATE Get the $i$-th training sample of $\gD_t$
	    \STATE Compute loss $E$
		\STATE Compute gradients $\vg$
		\FOR{$n\in\{1,\ldots,N\}$}
                \STATE Update sensitivity for the $n$-th parameter: $s_{n} = s_{n} + g_n^2$
		    \ENDFOR
    \ENDFOR
\end{algorithmic}
\end{algorithm}}


\begin{figure*}[t]
\begin{center}
    \includegraphics[width=\linewidth]{main_figure.pdf}
\end{center}\vspace{-2em}
\caption{Overview of our trainable parameter allocation strategy. With the parameter sensitivity \bohan{set} $\gS$, we first get the top-$\tau$ sensitive parameters. Instead of directly tuning these sensitive parameters, we also boost the representational capability by replacing unstructured tuning with structured tuning at sensitive weight matrices that have a large number of sensitive parameters, which can be implemented by an existing structured tuning method, \eg, LoRA~\cite{hu2022lora} and Adapter~\cite{houlsby2019parameter}. Red lines and blocks represent trainable parameters and modules, while blue lines represent frozen parameters.}
\label{fig:main}
\vspace{-1.5em}
\end{figure*}


To overcome the first barrier, we simplify the empirical loss by approximating $s_n$ in the vicinity of $\vw$ by its first-order Taylor expansion
\vspace{-0.3em}
\begin{equation}
\vspace{-0.5em}
    \begin{aligned}
        s_n^{(1)} = -g_n\Delta_{w_n},
    \end{aligned}
\label{eq:first-order}
\end{equation}
where the gradients $\vg=\partial E/\partial\vw$, and $g_n$ is the gradient of the $n$-th element of $\vg$. 
To address the second barrier, following~\cite{liu2018darts,cai2018proxylessnas}, we take the one-step unrolled weight as the surrogate for $w_n^*$ and approximate $\Delta_{w_n}$ in Eq.~(\ref{eq:first-order}) with a single step of gradient descent. We can accordingly get $s_n^{(1)} \approx g_n^2\epsilon$,
where $\epsilon$ is the learning rate. Since $\epsilon$ is the same for all parameters, we can eliminate it when comparing the sensitivity with the other parameters and finally get 
\vspace{-0.5em}
\begin{equation}
\vspace{-0.3em}
    \begin{aligned}
        s_n^{(1)} \approx g_n^2.
    \end{aligned}
\label{eq:first-order-simp}
\end{equation}
Therefore, the sensitivity of a parameter can be efficiently measured by its potential to reduce the loss on the target domain. Note that although our criterion draws inspiration from pruning work~\cite{molchanov2019importance}, it is distinct from it. \cite{molchanov2019importance} measures the parameter importance by the squared change in loss when removing them, \ie, $\left( E(\gD_t, \vw)-E(\gD_t, \vw\mid w_n=0) \right)^2$ and finally derives the parameter importance by $\left( g_n w_n \right)^2$, which is different from our formulations in Eqs.~(\ref{eq:sensitivity}) and~(\ref{eq:first-order-simp}).

In practice, we accumulate $\gS$ from a total number of $C$ training samples ahead of fine-tuning to generate accurate sensitivity as shown in Algorithm~\ref{alg:tps}, where $C$ is a pre-defined hyper-parameter. In Section~\ref{subsec:abl}, we show that employing only 400 training samples is sufficient for getting reasonable parameter sensitivity, which requires only 5.5 seconds with a single GPU for any VTAB-1k dataset with ViT-B/16 backbone~\cite{vit}.

\vspace{-0.3em}
\subsection{Adaptive Trainable Parameters Allocation}
\label{subsec:SPT}
\vspace{-0.2em}

Our next step is to allocate trainable parameters based on the obtained parameter sensitivity set $\gS$ and a desired parameter budget $\tau$. A straightforward solution is to directly tune the top-$\tau$ most sensitive unstructured connections (parameters) \rev{while keeping the rest frozen}, which we name unstructured tuning. Specifically, we select the top-$\tau$ most sensitive weight connections in $\gS$ to form the sensitive weight connection set $\gT$. Then, for \rev{a} weight matrix $\mW\in \sR^{d_{\rm in}\times d_{\rm out}}$, we can get a binary mask $\mM\in \sR^{d_{\rm in}\times d_{\rm out}}$ computed by
\vspace{-0.5em}
\begin{equation}
\vspace{-0.5em}
    {\begin{array}{ll}
    \small
    \begin{aligned}
    \mM^j =
    \left\{\begin{array}{ll} 
    1 ~~~~~ \mW^j \in \gT \\
    0 ~~~~~ \mW^j \notin \gT
    \end{array}\right.
    \end{aligned},
    \small
    \end{array}}
\label{eq:mask}
\end{equation}
where $\mW^j$ and $\mM^j$ are the $j$-th element in $\mW$ and $\mM$, respectively. Accordingly, we can train the sensitive parameters by gradient descent and the updated weight matrix can be formulated as $\mW'\leftarrow \mW - \epsilon\vg_{\mW}\odot\mM$, where $\vg_{\mW}$ is the gradient for $\mW$.

However, considering PEFT approaches generally limit the proportion of trainable parameters to less than 1\%, tuning only a small number of unstructured weight connections might not have enough representational capability to handle the downstream datasets with large domain gaps from the source pre-training data. Therefore, to improve the representational capability, we propose to replace unstructured tuning with structured tuning at the sensitive weight matrices that have a high number of sensitive parameters. To preserve the parameter budget, we can implement structured tuning with an existing efficient structured tuning PEFT method~\cite{hu2022lora,chen2022adaptformer,houlsby2019parameter,jie2022convolutional} that learns to directly adjust \rev{all hidden dimensions at once}. We depict an overview of our trainable parameter allocation strategy in Figure~\ref{fig:main}. For example, we can employ the low-rank reparameterization trick LoRA~\cite{hu2022lora} to the sensitive weight matrices \rev{and the one-step update for $\mW$ can be formulated as}
\vspace{-0.4em}
\begin{equation}
\vspace{-0.4em}
    {\begin{array}{ll}
    \small
    \begin{aligned}
    \mW' = \left\{\begin{array}{ll} 
    \mW + \mW_{\rm down}\mW_{\rm up} & ~~ \text { if } ~~ \sum_{j=0}^{d_{\rm in}\times d_{\rm out}} \mM^j \geq \sigma_{\rm opt} \\
    \mW - \epsilon\vg_{\mW}\odot\mM & ~~ {\rm otherwise}
    \end{array}\right.
    \end{aligned},
    \small
    \end{array}}
\label{eq:weight_updat}
\end{equation}
where $\mW_{\rm down}\in \sR^{d_{\rm in}\times r}$ and $\mW_{\rm up}\in \sR^{r\times d_{\rm out}}$ are two learnable low-rank matrices to approximate the update of $\mW$ and rank $r$ is a hyper-parameter where $r \ll {\rm min}(d_{\rm in},d_{\rm out})$. In this way, we perform structured tuning on $\mW$ when its number of sensitive parameters exceeds $\sigma_{\rm opt}$, whose value depends on the pre-defined type of structured tuning method. For example, since implementing structured tuning with LoRA requires $2\times d_{\rm in} \times d_{\rm out} \times r$ trainable parameters for each sensitive weight matrix, we set $\sigma_{\rm LoRA} \leftarrow 2\times d_{\rm in} \times d_{\rm out} \times r$ to ensure that the number of trainable parameters introduced by structured tuning is always equal to or lower than the number of sensitive parameters.

In this way, our SPT adaptively incorporates both structured and unstructured tuning granularities to enable higher flexibility and stronger representational power, simultaneously. In Section~\ref{subsec:abl}, we show that structured tuning is important for the downstream tasks with larger domain gaps and both unstructured and structured tuning contribute clearly to the superior performance of our SPT.
\begin{table*}[t!]
\begin{minipage}{0.175\linewidth}
\centering
% \hspace{1.8mm}
\captionof{table}{\small Datasets statistics \label{graph_datasets}}
\begin{tiny}
\begin{tabular}{c||c|c}
      {\bf Graph} & {\bf \#nodes} & {\bf \#edges} \\ \hline
      {\em FL} & 80\,513     & 5\,899\,882 \\
      {\em YT} & 1\,138\,499 & 2\,990\,443 \\
      {\em LJ} & 2\,238\,731 &14\,608\,137 \\
      {\em OR} & 3\,072\,441 & 117\,185\,083 \\
      {\em TW} & 41\,652\,230 & 1\,468\,365\,182 \\
\end{tabular}
\end{tiny}
\end{minipage}%
 \quad
 \begin{minipage}{.265\linewidth}
\centering
\tabcolsep=0.05cm

\captionof{table}{\small Avg. memory footprint (GB) of {\sf DistGER} and {\sf KnightKing} on each machine, where $\sigma$ is the standard deviation}
\label{Memory_usage}
\begin{tiny}
\newcommand{\tabincell}[2]{\begin{tabular}{@{}#1@{}}#2\end{tabular}}
  % \caption{\small {\color{blue} Avg. memory footprint (GB) of {\sf DistGER} and {\sf KnightKing} on each machine, where $\sigma$ is the standard deviation.}}
  \begin{tabular}{c|cc|cc}
    %\hline
    { }&\multicolumn{2}{c|}{\bfseries{ Sampling}}&\multicolumn{2}{c}{\bfseries{Training}}\\
    \hline
    {\bf{Graph}} &{\sf KnightKing} &{\sf DistGER} &{\sf KnightKing} &{\sf DistGER} \\
    \hline
     {\em FL} & 0.66($\pm$0.06)	&{\bf 0.41($\pm$0.02)}	&1.31($\pm$0.17) 	&{\bf 0.86($\pm$0.06)} 	\\

     {\em YT} &4.11($\pm$0.55)	&{\bf 1.36($\pm$0.23)} 	&4.73($\pm$0.72) 	&{\bf 4.26($\pm$0.63)} \\

     {\em LJ} & 7.65($\pm$0.82)	&{\bf 1.95($\pm$0.16)}	&6.38($\pm$0.97) 	&{\bf 5.49($\pm$0.85)} 	\\

     {\em CO} &10.98($\pm$1.03)	&{\bf 3.27($\pm$0.79)} 	&8.52($\pm$1.01) 	&{\bf 6.86($\pm$0.69)} 	\\

     {\em TW} & out-of-memory	&{\bf 20.18($\pm$3.62)} 	&out-of-memory 	& {\bf 67.16($\pm$5.18)} 	\\
  %\hline
\end{tabular}
\end{tiny}

\end{minipage}
\quad
\begin{minipage}{.25\linewidth}
    \centering
    \includegraphics[width= 1.85in, height = 1.2in]{./Figures/Dist_total_time_partition.eps}%
    \captionof{figure}
      {\small Efficiency: {\sf PBG} \cite{PBG_2019}, {\sf DistDGL} \cite{DistDGL_2020}, {\sf KnightKing} \cite{KnighKing_2019}, {\sf HuGE-D} (baseline), {\sf DistGER} (ours)
        \label{overall_performance}
      }
\end{minipage}%\hfill
\quad
\begin{minipage}{.25\linewidth}
    \centering
    \includegraphics[width= 1.85in, height = 1.2in]{./Figures/Dist_scalability_partition.eps}%
    \captionof{figure}
      {\small Scalability: {\sf PBG} \cite{PBG_2019}, {\sf DistDGL} \cite{DistDGL_2020}, {\sf KnightKing} \cite{KnighKing_2019}, {\sf HuGE-D} (baseline), {\sf DistGER} (ours)
        \label{Dist_scalability}
      }
\end{minipage}
\end{table*}


\section{Experimental Results}
\label{sec:experiments}
We evaluate the efficiency (\S \ref{sec:overall}) and scalability (\S \ref{sec:scalability}) of our proposed method, {\sf DistGER}
by comparing with {\sf HuGE-D} (baseline),
{\sf KnightKing} \cite{KnighKing_2019}, {\sf PyTorch-BigGraph} ({\sf PBG}) \cite{PBG_2019}, and {\sf Distributed DGL}
({\sf DistDGL}) \cite{DistDGL_2020}. We also compare the effectiveness (\S \ref{sec:effectiveness}) of generated embeddings
on link prediction.
% and multi-label classification tasks. 
Finally, we analyze efficiency due to individual
parts of {\sf DistGER} (\S \ref{sec:individual})
and the generality of {\sf DistGER} for other random walk-based embeddings (\S \ref{sec:generality}).
Our codes and datasets are at \cite{code}.
%
\subsection{Experimental Setup}
\label{sec:setup}
%
\spara{Environment.} We conduct experiments on a cluster of 8 machines with 2.60GHz Intel $^\circledR$ Xeon $^\circledR$ Gold 6240 CPU with 72 cores (hyper-threading)
in a dual-socket system, and each machine is equipped with 192GB DDR4 memory and connected by a 100Gbps network.
The machines run Ubuntu 16.04 with Linux kernel 4.15.0. We use GCC v9.4.0 for compiling {\sf DistGER}, {\sf KnightKing}, and {\sf HuGE-D},
and use Python v3.6.15 and torch v1.10.2 as the backend deep learning framework for {\sf Pytorch-BigGraph} and {\sf DistDGL}.

\spara{Datasets. } We employ five widely-used, real-world graphs
(Table~\ref{graph_datasets}): {\em Flickr} (FL) \cite{Flickr_Youtube_Graph},
{\em Youtube} (YT) \cite{Flickr_Youtube_Graph},
{\em LiveJournal} (LJ) \cite{BlogCatalog_Twitter_LiveJournal_Graph},
{\em Com-Orkut} (OR) \cite{com-orkut_2012}, and {\em Twitter} (TW) \cite{twitter_2010}.
The first two graphs are selected for multi-label node classification with distinct number of node labels 195 and 47, respectively, %in {\em Flickr} and {\em Youtube},
where labels in {\em Flickr} represent interest groups of users, and {\em Youtube}'s labels represent groups of viewers that enjoy common video genres. The last four graphs are used in link prediction. We also use synthetic graphs \cite{RMAT_2004} (up to 1 billion nodes, 10 billion edges) and a real-world {\em UK graph} \cite{BSVLTAG} (100M nodes, 3.7B edges) to assess the scalability of {\sf DistGER}.
Considering the default settings of popular random walk-based methods (e.g., Deepwalk, node2vec, HuGE), we use their undirected version.

\spara{Competitors.} We compare {\sf DistGER} against three state-of-the-art distributed graph embedding frameworks: the distributed random walk engine, {\sf KnightKing} {\scriptsize\url{https://github.com/KnightKingWalk/KnightKing}}
\cite{KnighKing_2019}; the distributed multi-relations based graph embedding system, {\sf PyTorch-BigGraph} ({\sf PBG})
{\scriptsize\url{https://github.com/facebookresearch/PyTorch-BigGraph}} \cite{PBG_2019} -- designed by Facebook; and
the distributed graph neural networks-based system, {\sf DistDGL} {\scriptsize\url{https://github.com/dmlc/dgl}}
\cite{DistDGL_2020} -- recently proposed by Amazon. We also implement {\sf HuGE-D}, a distributed version of
information-centric random walk-based graph embedding ({\sf HuGE} \cite{HuGE_2021}), on top of {\sf KnightKing},
served as our baseline. Since {\sf KnightKing} and {\sf HuGE-D} provide distributed support only for
random walk without that for embedding learning, we generate their node embeddings using
{\sf Pword2vec} {\scriptsize\url{https://github.com/IntelLabs/pWord2Vec}} \cite{Pword2vec_2019},
the most popular distributed {\sf Skip-Gram} system released by Intel.
%We find that {\sf pSGNScc} \cite{pSGNSCC_2017} (\S \ref{sec:learning})
%only provides a single-machine implementation, thus we do not include it in our distributed experiments.

\spara{Parameters.} For {\sf DistGER} and {\sf HuGE-D} random walks, we set
parameters $\mu$=0.995, $\delta$=0.001 based on information measurements (\S \ref{sec:preliminaries}),
while {\sf KnightKing} uses $L$=80 and $r$=10 that are routine configurations in the traditional
random walk-based graph embedding \cite{node2vec_2016, DeepWalk_2014, KnighKing_2019}. For {\sf DistGER}, {\sf KnightKing}, and {\sf HuGE-D} training,
we set the sliding window size $w$=10, number of negative samples $K$=5, and synchronization period=0.1 sec \cite{Pword2vec_2019},
and additionally, multi-windows number=2, $\gamma$=2 for {\sf DisrGER}.
%For {\sf Pytorch-BigGraph} ({\sf PBG}), we set the number of partitions to 16 following \cite{PBG_2019}, that is, using $2m$ partitions for the number of machines $m$ = 8
%in our case. %For {\sf DistDGL}, the deployed {\sf GaphSAGE} model uses three graph convolutional layers.
For fair comparison across all systems, %the efficiency performance of all systems involved in the experiments,
we set the embedding dimension $d$=128 that is commonly used \cite{HuGE_2021,node2vec_2016,DeepWalk_2014,Line_2015,Verse_2018,ProNE_2019},
and report the average running time for each epoch. For task effectiveness evaluations,
we find the best results from a grid search over learning rates from 0.001-0.1, \# epochs from 1-30,
and \# dimensions from 128-512.


%
\eat{
\begin{table}
\newcommand{\tabincell}[2]{\begin{tabular}{@{}#1@{}}#2\end{tabular}}
  \caption{\small Avg. memory footprint (GB) of {\sf DistGER} and {\sf KnightKing} on each machine, where $\sigma$ is the standard deviation.}
  \label{Memory_usage}
  \begin{center}
   \footnotesize
  \begin{tabular}{c|cc|cc}
    %\hline
    { }&\multicolumn{2}{c|}{\bfseries{ Sampling}}&\multicolumn{2}{c}{\bfseries{Training}}\\
    \hline
    {\bf{Graph}} &{\sf KnightKing} &{\sf DistGER} &{\sf KnightKing} &{\sf DistGER} \\
    \hline
     {\em Flickr} & 0.66($\pm$0.06)	&{\bf 0.41($\pm$0.02)}	&1.31($\pm$0.17) 	&{\bf 0.86($\pm$0.06)} 	\\

     {\em Youtube} &4.11($\pm$0.55)	&{\bf 1.36($\pm$0.23)} 	&4.73($\pm$0.72) 	&{\bf 4.26($\pm$0.63)} \\

     {\em LiveJournal} & 7.65($\pm$0.82)	&{\bf 1.95($\pm$0.16)}	&6.387($\pm$0.97) 	&{\bf 5.49($\pm$0.85)} 	\\

     {\em Com-Orkut} &10.98($\pm$1.03)	&{\bf 3.27($\pm$0.79)} 	&8.52($\pm$1.01) 	&{\bf 6.86($\pm$0.69)} 	\\

     {\em Twitter} & out-of-memory	&{\bf 37.1($\pm$5.28)} 	&out-of-memory 	& {\bf 79.5($\pm$7.27)} 	\\
  %\hline
\end{tabular}
\end{center}
\end{table}
%
}
%


%
\subsection{Efficiency and Memory Use w.r.t. Competitors}
\label{sec:overall}
%\begin{figure}
%  \centering
%  \includegraphics[width= 3 in]{Dist_total_time.eps}
%  \caption{\small Overall performance of PBG, DistDGL, KnightKing, HuGE-D and DistGER for generating embeddings on different read-word graphs, {\color{blue}for Twitter graph, DistDGL cannot finish in one day, and KnightKing fails to perform due to memory issue, where the y axis is in log-scale.}}
%  \label{overall_performance}
%\end{figure}
%
We report the end-to-end running times of {\sf PBG}, {\sf DistDGL}, {\sf KnightKing}, {\sf HuGE-D}, and {\sf DistGER}
on five real-world graphs with the cluster of 8 machines in Figure~\ref{overall_performance}.
The reported end-to-end time includes the running time of partitioning, random walks (for random walk-based frameworks), and training procedures.
%{\color{blue} Noted that the reported end-to-end time in our experiments excludes the partition time for all evaluated frameworks due to the all used partition schemes are executed as a preprocessing component, and we separately evaluate the partition efficiency in Section 6.5, thus the end-to-end time refers to the running time of random walk (only for random walk-based framework) and training procedure.}
{\sf DistGER} significantly outperforms the competitors
on all these graphs, achieving a speedup ranging from 2.33$\times$ to 129$\times$. %, by an average acceleration of $39.78 \times$.
Recall that {\sf DistGER} is a similar type of system as {\sf KnightKing} and {\sf HuGE-D},
and our key improvements are discussed in \S \ref{sec:DistGER} and in \S \ref{sec:learning}.
Analogously, Figure~\ref{overall_performance} exhibits that our system, %designs are more effective (see more evaluation details in \S 6.3),
{\sf DistGER} achieves an average speedup of 9.25$\times$ and 6.56$\times$ compared with {\sf KnightKing} and {\sf HuGE-D}.
Notice that we fail to run {\sf KnightKing} on the largest {\em Twitter} dataset
because its routine random walk strategy requires more main memory space.
%Although Huge-D achieves comparable performance,
The advantage of information-centric random walk in {\sf HuGE} is almost wiped out in {\sf HuGE-D}
due to on-the-fly information measurements and the higher communication costs in a distributed setting.
The multi-relation-based {\sf PBG} leverages a parameter server to synchronize embeddings between clients,
resulting in more load on the communication network. As a result, {\sf PBG} is on average
26.22$\times$ slower than {\sf DistGER}. For graph neural network-based system {\sf DistDGL},
due to the long running time of graph sampling (e.g., taking 80\% of the overhead for the {\sf GraphSAGE}),
it is highly inefficient than other systems. For the billion-edge {\em Twitter} graph, it does not terminate in 1 day.
%
%Considering the resource consumption that affects scalability,
{Table ~\ref{Memory_usage}} shows {\sf DistGER}'s average memory footprint on each machine of the 8-machine cluster. %from a cluster of 8 machines.
%and the standard deviation %($\sigma$) %of the results
%in 
%Table ~\ref{Memory_usage}. 
Compared
to %other methods, %with the 
same type of system
{\sf KnightKing}, 
% that is of the same system type, 
{\sf DistGER} requires less memory for sampling and training.


\subsection{Scalability w.r.t. Competitors}
\label{sec:scalability}
%
%\begin{figure}
%  \centering
%  \includegraphics[width= 3.2 in]{Dist_scalability.eps}
%  \caption{\small Scalability comparison on LiveJournal graph, where the y axis is in log-scale.}
% \label{Dist_scalability}
%\end{figure}
%
Figure~\ref{Dist_scalability} shows end-to-end running times of all competing
systems on the {\em LiveJournal} graph, as we increase \# machines
from 1 to 8 to evaluate scalability. {\sf DistGER} achieves better scalability than the other
four distributed systems.
%Due to space limitation, we omit results on other graph datasets,
%which exhibit similar trends.
{\sf PBG} leverages a parameter server and a shared network filesystem
to synchronize the parameters in the distributed model. %The edges are partitioned into $m^2$ buckets
%and training can be performed in parallel using up to $m/2$ machines. After one bucket completes
%the training, it needs to communicate with the parameter server.
When the number of machines increases, {\sf PBG} puts more load
on the communications network, resulting in poor scalability. Likewise, {\sf DistDGL}
is bounded by the synchronization overhead for gradient updates,
limiting its scalability.
%Since {\sf DistDGL} uses mini-batches for sampling, %features %for GraphSAGE,
%if the mini-batch samples cannot be generated on time, the trainer will be delayed on the forward pass, and all other
%machines need to wait before starting the backward pass. Thus, increasing the number of
%machines also affects the efficiency of backward pass. %Being the random walk-based distributed systems,
Both {\sf KnightKing} and {\sf HuGE-D} suffer from higher communication costs during random walks,
due to their only workload-balancing partitioning scheme (\S \ref{sec:dRand}, \S \ref{sec:individual}).
%Their scalability is relatively poor as the number of machines increases.
%{\sf KnightKing} partitions the graph by a workload-balancing scheme, inevitably introducing higher
%cross-machine communications due to the randomness inherent in the random walking procedure (\S \ref{sec:dRand}, \S \ref{sec:individual}).
%With more machines, the inefficiency of the partitioning scheme is further magnified.
Since {\sf HuGE-D} is implemented on top of {\sf KnigtKing},
it exhibits worse scalability due to high communication costs and on-the-fly information measurements in a distributed setting (\S \ref{sec:HUGED}).
%In contrast, its performance is much better than all the competitors in a single machine.
In comparison, {\sf DistGER} incorporates multi-proximity-aware streaming graph partitioning and incremental computations
to reduce both communication and computation costs, it also employs hotness-block based parameters synchronization
during training to dramatically reduce the pressure on network bandwidth. Hence, {\sf DistGER} achieves better scalability than other systems.
Due to space limitations, we omit {\sf DistGER}'s scalability results on other graphs, which exhibit similar trends. On {\em Twitter}, the end-to-end running times {\sf DistGER} on 1, 2, 4, and 8 machines are 3090s, 1739s, 1197s, and 746s, respectively,
while on {\em Com-Orkut}, the results are 304s, 204s, 149s, and 89s, respectively. 
The results show a good linear relationship.
% The results demonstrate a desired scalability with the increase of the machines.

\begin{table}
\quad
\begin{minipage}{0.46\linewidth}
    \centering
    \includegraphics[width= 1.6 in]{./Figures/Dist_scalability_datasize.eps}%
    \captionof{figure}
      {\small {Scalability of {\sf DistGER} on synthetic graphs, where Y-axis is in log-scale}}
      %The lines depict the running time required for random walk (blue line) and training (red line), respectively. Pentagrams show the time cost of six real-world graphs,
        \label{Dist_scalability_data}
      
\end{minipage}\hfill
\quad
\begin{minipage}{.46\linewidth}
    \centering
    \includegraphics[width= 1.6 in]{./Figures/Dist_time_auc.eps}%
    \captionof{figure}
      {\small {The influence of running time on embedding quality for {\sf DistGER} and competitors}}
        \label{Dist_time_auc}
\end{minipage}
\end{table}


To further assess the scalability of {\sf DistGER}, we generate synthetic graphs \cite{RMAT_2004} with a fixed node degree of 10 and the number of nodes from $10^5$ to $10^9$. Figure~\ref{Dist_scalability_data} presents the running times for random walks and training on these synthetic graphs using a cluster of 8 machines, suggesting that the running time increases linearly with the size of a graph, and {\sf DistGER} has the capability to handle even billion-node graphs. Moreover, the running times for six real-world graphs (including the {\em UK graph} with $|E|=3.7B$, $|V|=100M$, for which the competing systems do not terminate in 1 day or crash due to hardware and memory limitation) are inserted into the plot, which is consistent with the trend on synthetic data.

%
%
\subsection{Effectiveness w.r.t. Competitors}
\label{sec:effectiveness}
%
\spara{Link prediction.} To perform link prediction on a given graph $G$, following \cite{HuGE_2021,node2vec_2016,Verse_2018,NRP_2020},
we first uniformly at random remove 50\% edges as positive test edges, and the rest are used as positive training edges.
We also provide negative training and test edges by considering those node pairs between which no edge exists in $G$.
We ensure that the positive and negative set sizes are similar. %For a pair of nodes $(u, v)$, let $\varphi(u)$ and
%$\varphi(v)$ be the vectors learned by embedding methods.
The link prediction is conducted as a classification task
based on the similarity of $u$ and $v$, i.e., $\varphi(u)\cdot\varphi(v)$.
The effectiveness of link prediction is measured via the $AUC$ (Area Under Curve) score \cite{AUC_kdd} -- the higher the better.
We repeat this procedure 50 times to offset the randomness of edge removal and report the average $AUC$ in
Table~\ref{AUC_results}.
%shows $AUC$ for all the methods on five real-world graphs.
%, respectively, where a ``$-$'' indicates that the method fails due to the limitation of computing resources or because its running time exceeds 1 day.
{\sf DistGER} outperforms all competitors on these graphs, except for {\sf PBG} on {\em Com-Orkut}, where {\sf DistGER} ranks second.
On average, {\sf DistGER} has an 11.7\% higher $AUC$ score compared with the other three systems, thanks to our
information-centric random walks. {\sf PBG} is the best on {\em Com-Orkut} because this graph is much denser
and is friendly to the multi-relationship-based model in {\sf PBG}.
Figure~\ref{Dist_time_auc} exhibits accuracy-efficiency tradeoffs of {\sf DistGER} and competitors, i.e., their $AUC$ convergence curves w.r.t. increasing running times of random walks and training, over {\em LiveJournal}, further indicating
that {\sf DistGER} has better efficiency and effectiveness than the competitors.
%As a system of the same type, DistGER achieves better accuracies on all graphs than KnightKing which leverages the routine random walk configuration, thanks to its information-centric random walk strategies. We do not report the effectiveness of HuGE-D here because it uses the same random walk model as DistGER.
%
\begin{table}[h!]
\newcommand{\tabincell}[2]{\begin{tabular}{@{}#1@{}}#2\end{tabular}}
  \caption{\small $AUC$ scores of {\sf DistGER} and competitors for link prediction}
  \label{AUC_results}
  \begin{center}
  \footnotesize
  \begin{tabular}{cccccc}
%    \hline
    {Method}&\tabincell{c}{Youtube}&{LiveJournal}&\tabincell{c}{Com-Orkut}&{ Twitter}\\
    \hline
    {\sf PBG}        & 0.753           &0.882            &\bfseries{0.955} &0.912\\

    {\sf DistDGL}    &0.894            &0.718            &0.815            & running time $>$ 1 day \\

    {\sf KnightKing} &0.904            &0.963            & $0.918$         & out-of-memory\\

    {\sf DistGER}    &\bfseries{0.966} &\bfseries{0.976} &0.921            &\bfseries{0.919}\\
%  \hline
\end{tabular}
\end{center}
\end{table}

% \eat{
\spara{Multi-label node classification.}
This task predicts one or more labels for each graph node and has applications in %modern applications ranging from
text categorization \cite{zhang2006multilabel} and bioinformatics \cite{zhang2018ontological}.
We use embedding vectors and a one-vs-rest logistic regression classifier
with L2 regularization \cite{MLC_LIBLINEAR_2008}, %(using the LIBLINEAR library),
then evaluate the effectiveness by micro-averaged F1 ($Micro-F1$) and macro-averaged F1 ($Macro-F1$) \cite{WangC016}
scores, where $Micro-F1$ gives equal weight to each test instance and $Macro-F1$ assigns equal weight to each label category \cite{keikha2018community}.
%To train a classifier, nodes are uniformly at random split into training and test sets.
Following \cite{HuGE_2021,node2vec_2016,DeepWalk_2014,Line_2015,Verse_2018},
we select 10\% to 90\% training data ratio on {\em Flickr}, and 1\% to 9\% training ratio on {\em Youtube}.
%and the remaining nodes for testing.
We report the averaged $Macro-F1$ and $Micro-F1$ scores from 50 trials in Figure~\ref{Dist_MLC_mac_mic_F1}.
% shows the $Macro-F1$ and $Micro-F1$ scores achieved by each system as a
%function of the training ratio variation, respectively.
We find that {\sf DistGER} has better $Macro-F1$ and $Micro-F1$ scores
than existing frameworks, %on these graphs, %. In particular, compared with the KnightKing,
%DistGER consistently outperforms the other random walk-based systems on all graphs in $Macro-F1$ and $Micro-F1$ scores,
gaining 9.2\% and 3.3\% average improvements, respectively, due to its more effective information-centric random walks.
%Definition of $Macro-F1$ and $Micro-F1$ are as the following:

\begin{figure}[h!]
  \centering
  \includegraphics[width= 3.45 in]{./Figures/Dist_MLC_mac_mic_F1_1.eps}
  \caption{\small $Macro-F1$ (a1, b1) and $Micro-F1$ (a2, b2) scores for multi-label node classification. $X$-axis: training data ratio}
  \label{Dist_MLC_mac_mic_F1}
\end{figure}

% }
%\begin{equation}
%Precision = \frac{\sum\nolimits_{i}^{K}TP(i)}{\sum\nolimits_{i}^{K}(TP(i)+FP(i))}
%\end{equation}
%
%\begin{equation}
%Recall = \frac{\sum\nolimits_{i}^{K}TP(i)}{\sum\nolimits_{i}^{K}(TP(i)+FN(i))}
%\end{equation}
%
%\begin{equation}
%Micro-F1 = \frac{2\times Precision\times Recall}{Precision+Recall}
%\end{equation}
%
%\begin{equation}
%Macro-F1 = \frac{\sum\nolimits_{i}^{K}Micro-F1(i)}{|K|}
%\end{equation}
%
%where $TP(i)$, $FP(i)$ and $FN(i)$ are the number of true positives, false positives and false negatives in the instances which are predicted as $i$, respectively. Suppose $K$ is the overall label set, $Micro-F1$($i$) and $Macro-F1$ are the measure of $Micro-F1$ and $Macro-F1$ for the label $i$, respectively.
%\begin{table}
%\setlength{\abovecaptionskip}{0.cm}
%\setlength{\belowcaptionskip}{-0.cm}
%\newcommand{\tabincell}[2]{\begin{tabular}{@{}#1@{}}#2\end{tabular}}
%  \caption{$Macro-F1$ and $Micro-F1$ for multi-label classification on Flickr and Youtube graph, where train ratio is 0.5.}
%  \label{cluster_results}
%  \begin{center}
%  \small
%  \begin{tabular}{ccccc}
%    \hline
%    { }&\multicolumn{2}{c}{\bfseries{ \scriptsize Flickr}}&\multicolumn{2}{c}{\bfseries{\scriptsize Youtube}}\\
%    \hline
%    { }&Macro-F1 &Micro-F1&Macro-F1 &Micro-F1\\
%
%    \hline
%    \small PBG & 0.225	&0.387 	&0.295 	&0.406 	\\
%
%    \small DistDGL &0.205 	&0.378 	&0.283 	&0.403 	\\
%
%    \small KnightKing &0.239    &0.386 &0.285 	&0.402 	\\
%
%    \small DistGER  &\bfseries{0.277} &\bfseries{0.409}&\bfseries{0.298} &\bfseries{0.417}\\
%
%  \hline
%\end{tabular}
%\end{center}
%\end{table}
%

\begin{figure}
  \centering
  \includegraphics[width= 3.2 in]{./Figures/Dist_sampling_training_Mpad_efficiency.eps}
  \caption{\small {(a) Random walk efficiency, (b) training efficiency, (c) \# cross-machine messages, (d) random walk efficiency for {\sf MPGP} (ours) and workload-balancing scheme ({\sf KnightKing})}}
  \label{Dist_efficiency_sampling_training_MPGP}
\end{figure}

\begin{table*}[t!]
\begin{minipage}{0.275\linewidth}
\centering
\renewcommand\arraystretch{1.2}
\captionof{table}{\small Performance evaluation of partitioning for {\sf DistGER} and Competitors } %{\sf PBG} and {\sf DistDGL}
\label{Partition_sechme_overhead}
\begin{scriptsize}
\begin{tabular}{ccccc}
    \multicolumn{5}{c}{{\bfseries (a) Partitioning time for {\sf DistGER} and competitors }} \\
    \hline
    {\bf graph} & {\sf PBG} & {\sf DistDGL} & {\sf DistGER}\\
                &           & ({\sf METIS}) &  ({\sf MPGP}) \\
    \hline
    {\sf FL} & 383.28 s & 127.72 s & \bfseries{15.96 s} \\
    {\sf YT} & 349.15 s & 116.30 s & \bfseries{13.56 s} \\
    {\sf LJ} & 458.52 s & 425.19 s & \bfseries{36.42 s} \\
    {\sf OR} & 2662.62 s & 2761.25 s &\bfseries{294.68 s}\\
    {\sf TW} & 22 hour s & $>$ 1 day &\bfseries{9 hours}\\
    \hline
%    \multicolumn{5}{c}{} \\
    \multicolumn{5}{c}{{\bfseries (b) Evaluation of {\sf Parallel MPGP} }} \\
    \hline
    {\bf graph} & {\sf Streaming} & {\sf Partitioning} & {\sf Walking}\\
    \hline
  %  {\sf MPGP}   &DFS+deg  & 9 hours & \bfseries{575.22 s} \\
    \multirow{2}{*}{\sf LJ} &DFS+deg & 21.86 s & \bfseries{23.78 s} \\
           & BFS+deg & \bfseries{21.25 s} & 24.79 s \\
    \multirow{2}{*}{\sf OR} &DFS+deg & \bfseries{151.29 s} & 77.12 s \\
           & BFS+deg & 156.37 s & \bfseries{46.55 s} \\
    \multirow{2}{*}{\sf TW} &DFS+deg & \bfseries{1940.65} s & 683.81 s \\
           & BFS+deg & 2034.21 s & \bfseries{590.36 s}
\end{tabular}
\end{scriptsize}
\end{minipage}%\hfill
\quad
\begin{minipage}{.3\linewidth}
    \centering
    \includegraphics[width= 2.5in, height = 1.45 in]{./Figures/Dist_Mpad_streaming_vertex_time.eps}%
    \captionof{figure}
      {\small The distribution of local computations and cross-machine communications for different streaming orders on {\em LiveJournal}. The top table reports their running times for partitioning and random walks
        \label{Dist_MPaD_streaming}
      }
\end{minipage}%\hfill
\qquad
\begin{minipage}{.37\linewidth}
    \centering
    \includegraphics[width= 2.5 in, height = 1.45 in]{./Figures/Dist_generality_table_HuGE+.eps}%
    \captionof{figure}
      {\small Generality of {\sf DistGER} vs. {\sf KnightKing}. The bars show random walk efficiency ($-R$) and training efficiency ($-T$) for {\sf Deepwalk} ({\sf DW}), {\sf node2vec} ({\sf n2v}) and {\sf HuGE+}. The top table shows the ratio $\frac{\text{{\em AUC} for {\sf DistGER}}}{\text{{\em AUC} of {\sf KnightKing}}}$, with {\sf DW} and {\sf n2v}, task: link prediction
        \label{Dist_generality}
      }
\end{minipage}
\end{table*}


\subsection{Efficiency due to Individual Parts of DistGER}
\label{sec:individual}
\spara{Random walk and training efficiency.}
To evaluate the system design of {\sf DistGER} (\S \ref{sec:DistGER}, \S \ref{sec:learning}),
we first compare the efficiency of random walks and training with those of {\sf KnighKing} and {\sf HuGE-D}.
%For fair comparison, the running times that we reported for {\sf KnightKing} and {\sf HuGE-D} exclude the
%time of vocabulary table construction, since it is a serial process in {\sf Pwode2vec}, while {\sf DistGER}
%pipelines the construction during random walks.
For random walks (Figure~\ref{Dist_efficiency_sampling_training_MPGP}(a)),
{\sf DistGER} significantly outperforms {\sf KnightKing} and {\sf HuGE-D} on all our graph
datasets, achieving an average speedup of $3.32\times$ and $3.88\times$, respectively.
Although {\sf HuGE-D} implements information-oriented random walks on {\sf KnightKing},
due to additional computation and communication overheads during on-the-fly information
measurements (\S \ref{sec:HUGED}), its efficiency can be lower than that of {\sf KnightKing}.
We also notice that the random walk lengths ($L$) and the number of random walks ($r$) reduce (on average)
63.2\% and 18\%, respectively, in our information-oriented random walks, compared to {\sf KnightKing}'s
routine random walk configuration.
%which supports the traditional
%random walk methods. %To provide a straightforward adaptation for the information-oriented approach, DistGER leverages the incremental information-centric computation mechanism to mitigate the redundant computation and high communication cost in HuGE-D, then it achieves an average speedup of $3.32\times$ and $3.88\times$ in random walk procedure compared to KnightKing and HuGE-D.

Another benefit of information-centric random walks is that it generates concise and effective corpus to improve 
training efficiency. Compared to {\sf KnightKing}, {\sf DistGER} achieves $17.37\times$-$27.95\times$ acceleration
in training over all our graphs. Next, considering the same corpus size, we compare the training efficiency of {\sf Pword2vec} and {\sf DSGL}
(trainer in {\sf DistGER}). Figure~\ref{Dist_efficiency_sampling_training_MPGP}(b) shows that {\sf DSGL} achieves $4.31\times$ average speedup
compared to {\sf Pword2vec}. We also notice that the average throughput (number of nodes processed per second) for {\sf DSGL} is up to 49.5 million/s,
while that of {\sf Pword2vec} is only up to 16.1 million/s. These results indicate that our distributed {\sf Skip-Gram} learning model (\S \ref{sec:learning})
is more efficient than {\sf Pword2vec}.
%
%\begin{figure}
%  \centering
%  \includegraphics[width= 2.5 in]{Dist_Mpad_efficiency.eps}
%  \caption{\small (a) exhibits the number of cross-machine computation for DistGER on workload-balancing and MPGP partition scheme, respectively, and (b) shows the random walk time of DistGER on the two schemes, where y axis is in log-scale.}
%  \label{Dist_efficiency_MPaD}
%\end{figure}

\spara{Partitioning efficiency.} Considering the %large number cross-machine computing introduced by the
randomness inherent in random walks, the partitioning scheme is critical to overall efficiency. %of the distributed framework.
%To validate the efficiency of our multi-proximity-aware streaming graph partitioning (MPGP),
%we deploy the workload balancing scheme used in KnightKing and MPGP on DistGER,
%respectively,
%and report the number of cross-machine computations during the random walk procedure for the two schemes.
%We also present the efficiency performance of MPGP compared with the workload-balancing scheme.
For {\sf DistGER},
Figure~\ref{Dist_efficiency_sampling_training_MPGP}(c) exhibits that our multi-proximity-aware streaming graph partitioning ({\sf MPGP})
significantly reduces (avg. reduction $45\%$) the number of cross-machine messages than the workload-balancing partition of {\sf KnightKing}
on five graphs. Moreover, it improves the efficiency by 38.9\% for the random walking procedure
(Figure~\ref{Dist_efficiency_sampling_training_MPGP}(d)) over the same set of walks.
We report in Table~\ref{Partition_sechme_overhead}(a) the time required for graph partitioning in competing systems,
where {\sf DistDGL} uses the {\sf METIS} algorithm \cite{METIS_1998} for partitioning.
The results show that {\sf MPGP} performs partitioning with very little overhead in most cases, and
the partitioning efficiency is on average $25.1\times$ faster than competitors.
In Figure~\ref{Dist_MPaD_streaming}, we exhibit the distribution of local computations and cross-machine communications
on four machines for different streaming orders, and the top table reports their running times for partitioning and random walks.
For sequential {\sf MPGP}, we find that the {\sf DFS+degree}-based streaming order (\S \ref{sec:partition}) is more efficient than other streaming orders,
and it also strikes the best balance between cross-machine communications reduction and workload balancing.
Table~\ref{Partition_sechme_overhead}(b) exhibits the performance evaluation of {\sf parallel MPGP} on the small- ({\em LiveJournal}), medium- ({\em Com-Orkut}) and large-scale ({\em Twitter}) graphs. The results show that {\sf DFS+Degree} in {\sf parallel MPGP} is still the best or comparable in terms of partition time, due to the same reason as stated in our third optimization scheme (\S \ref{sec:partition}). On the other hand, {\sf BFS+Degree} in {\sf parallel MPGP} works the best in terms of random walk time due to preserving the locality of the graph structure (our fourth optimization scheme in \S \ref{sec:partition}).
%as using its streaming order to parallel partitioning can reduce the influence of relevance between each segment.
We ultimately recommend {\sf BFS+Degree} for {\sf parallel MPGP}, since it reduces the partition time greatly, while the random walk time is comparable to that obtained from sequential {\sf MPGP}.
%
%\begin{figure}
%  \centering
%  \includegraphics[width= 3 in]{Dist_Mpad_streaming_vertex_time.eps}
%  \caption{\small The distribution of local computations and cross-machine communications for different streaming orders on {\em LiveJournal}. The top table reports their running times for partitioning and random walks.}
%  \label{Dist_MPaD_streaming}
%\end{figure}
%
%\begin{table}
%\setlength{\abovecaptionskip}{0.cm}
%\setlength{\belowcaptionskip}{-0.cm}
%\newcommand{\tabincell}[2]{\begin{tabular}{@{}#1@{}}#2\end{tabular}}
%  \caption{\small Time execution time (seconds) of the partition scheme in PBG, DistDGL, and DistGER, ``$-$'' means the scheme fails under constrains of computation resource.}
%  \label{Partition_sechme_overhead}
%  \begin{center}
%  \small
%  \begin{tabular}{ccccc}
%    \hline
%    {Graph}&{PBG}&{DistDGL(METIS)}&{DistGER}\\
%    \hline
%    Flickr& 383.28 &127.72 &\bfseries{15.96} \\
%
%    Youtube& 349.15 &116.30&\bfseries{13.56} \\
%
%    LiveJournal& 458.52 &425.19 &\bfseries{36.42} \\
%
%    Com-Orkut& 2662.62 &2761.25 &\bfseries{294.68}\\
%
%    Twitter&78986.85 &$-$&\bfseries{35500.41}\\
    %\hline
%    \multicolumn{5}{l}{* HuGE+ generates the smallest corpus size for training among all methods tested.} \\
%
%  \hline
%\end{tabular}
%\end{center}
%\end{table}
%


\subsection{Generality of DistGER}
\label{sec:generality}
%\begin{figure}
%  \centering
%  \includegraphics[width= 3 in, height= 1.65 in]{Dist_generality_table.eps}
%  \caption{\small Generality comparison for DistGER and KnightKing, %on real-word graphs,
%  The bars display random walk (denoted as $-R$) and training efficiency (denoted as $-T$) for {\sf Deepwalk} (DW) and {\sf node2vec} (n2v), respectively.%, and the y axis is in log-scale.
%  Top table shows the ratio $\frac{\text{{\em AUC} for {\sf DistGER}}}{\text{{\em AUC} of {\sf KnightKing}}}$, both with Deepwalk and node2vec, respectively, considering link prediction.}
%  \label{Dist_generality}
%\end{figure}
%
%Since our proposed information-oriented random walk framework DistGER aims to address the redundant computations and high communication cost introduced by the effectiveness measurement of the generated walking information in distributed setting, it provides a good systematic support for the information-centric approach HuGE as shown by the previous experimental results. A natural question arises: can DistGER also support the traditional random-walk-based methods?
To demonstrate the generality of {\sf DistGER}, we deploy {\sf Deepwalk} \cite{DeepWalk_2014}, {\sf node2vec} \cite{node2vec_2016} and {
\sf HuGE+} \cite{HuGE+_2022}
on {\sf DistGER}. While the original {\sf Deepwalk} and {\sf node2vec} follow
traditional random walks, in {\sf DistGER} the walk length and the number of walks are decided via information-centric measurements.
Next, we also deploy both {\sf Deepwalk} and {\sf node2vec} on {\sf KnightKing} which supports the routine configuration random walk.
Figure~\ref{Dist_generality} illustrates that {\sf DistGER} reduces the random walks time by 41.1\% and 51.6\% on average for
{\sf Deepwalk} and {\sf node2vec}, respectively. For training, {\sf DistGER} is on average $17.7\times$ and $21.3\times$ faster than {\sf KnightKing}+{\sf Pword2vec}
for {\sf Deepwalk} and {\sf node2vec}, respectively.
Moreover, we also show the {\em AUC} ratio of {\sf DistGER} and {\sf KnightKing}, considering {\sf Deepwalk} and {\sf node2vec}, for link prediction.
% tasks, where performing multi-label classification on Flickr graph  and link prediction on other graphs, the accuracy metric for the two task are $Miro-F1$ and $AUC$ score, respectively, it can be found from
Our results depict that {\sf DistGER} has comparable (in most cases, higher) {\em AUC} scores, while it improves the efficiency significantly
even for traditional random walk-based graph embedding methods.
{\sf HuGE+} is an extension of {\sf HuGE}, and it uses the same {\sf HuGE} information-centric method to determine the walk length and the number of walks per node. Figure~\ref{Dist_generality} exhibits the compatibility of {\sf HuGE+} on {\sf DistGER} via its general API.


%-------------------------------------------------------------------------
\vspace{-0.3em}
\section{Conclusion}
\vspace{-0.2em}
In this paper, we have explored identifying and allocating trainable parameters to task-specific important positions for visual parameter-efficient tuning. Specifically, we have proposed a novel criterion to quickly measure the sensitivity of the pre-trained parameters for each specific task before fine-tuning. Based on the parameter sensitivity, we have proposed a trainable parameter allocation strategy that adaptively combines both unstructured and structured tuning under a desired trainable parameter budget, enabling high representational capability and flexibility. Finally, we have conducted extensive experiments on a total of 24 downstream recognition tasks with both plain and hierarchical vision Transformer backbones under different pre-training strategies to demonstrate the versatility and effectiveness of our proposed SPT. Notably, we have shown that our approach is complementary to the existing VPET methods and improves their performance significantly. In the future, we will explore adapting large vision models to more downstream tasks with SPT, 
\eg, dense prediction and vision-and-language tasks, and improve the training efficiency of SPT for on-device training~\cite{cai2020tinytl,lin2022device}. 


\bibliographystyle{abbrv}
{\small
\bibliography{reference}
}

\clearpage
\appendix
\onecolumn
\section*{Appendix}

We organize our supplementary material as follows. 
\begin{itemize}
    \item In Section~\ref{subsec:supp_contenders}, we introduce more details about the contenders.
    \item In Section~\ref{subsec:supp_pattern1}, we show more sensitivity patterns for ViT-B/16 with various pre-training strategies.
    \item In Section~\ref{subsec:supp_visual}, we show some dataset samples from \imagenet~\cite{krizhevsky2012imagenet} and \vtab{}~\cite{zhai2019vtab}.
    \item In Tables~\ref{tab:full_fgvc} and~\ref{tab:full_vtab}, we show per-task results for our SPT variants on FGVC and \vtab{} benchmarks, respectively.
    
\end{itemize}

\section{More Details of Contenders} 
\label{subsec:supp_contenders}

\begin{itemize}[leftmargin=2em]{

\item \fullft{}: fully tunes all the backbone and classification head parameters.
\vspace{-0.75em}
\item\linear{}: freezes all the backbone parameters and only tunes a linear classification head.
\vspace{-0.75em}
\item\bias{}~\cite{zaken2022bitfit}: freezes all the backbone parameters except for the bias terms and also tunes the linear classification head.
\vspace{-0.75em}
\item\partialft{}-$k$: freezes all the backbone parameters except for the last $k$ layers and also tunes the linear classification head as described in~\cite{jia2022vpt}.
\vspace{-0.75em}
\item \mlp{}-$k$: freezes all the backbone parameters and tunes the classification head which is implemented by a trainable $k$-layer multi-layer perceptron as described in~\cite{jia2022vpt}.
\vspace{-0.75em}
\item \shallowprompt{}~\cite{jia2022vpt}: freezes all the backbone parameters while introducing additional trainable prompts to the input space of the pretrained ViT.
\vspace{-0.75em}
\item \deepprompt{}~\cite{jia2022vpt}: freezes all the backbone parameters while appending additional trainable prompts to the sequence in the multi-head self-attention layer of each ViT block.
\vspace{-0.75em}
\item\adapter{}-$k$~\cite{houlsby2019parameter}: freezes all the backbone parameters while adding a down projection, a ReLU~\cite{hendrycks2016gaussian} non-linearity, and an up projection layer sequentially in the feed-forward network (FFN) of each visual Transformer block. 
We follow the training details of~\cite{zhang2022neural} to achieve better performance.
\vspace{-0.75em}
\item \lora{}-$k$~\cite{hu2022lora}: freezes all the backbone parameters while adding a concurrent branch including two low-rank matrices to the weight matrices in the multi-head self-attention layers to approximate efficiently updating them. 
The low-rank matrices can be merged into the backbone weights after fine-tuning. We follow the training details of~\cite{zhang2022neural} to achieve better performance.
\vspace{-0.75em}
\item \adaptformer{}~\cite{chen2022adaptformer}: freezes all the backbone parameters while adding a concurrent branch including a down projection, a ReLU~\cite{agarap2018deep} non-linearity, an up projection layer, and a pre-defined scaling factor to the FFN layer of each ViT block.
\vspace{-0.75em}
\item \noah{}~\cite{zhang2022neural}: searches for an optimal configuration with a once-for-all~\cite{cai2019once} network that includes trainable prompts, adapter modules, and LoRA modules, which requires a longer training schedule than the other VPET methods.
}
\end{itemize}

\section{More Parameter Sensitivity Patterns}
\label{subsec:supp_pattern1}
\rev{We show more parameter sensitivity patterns for ViT-B/16 with various pre-training strategies (i.e., MAE~\cite{he2022masked} and MoCo V3~\cite{chen2021empirical}) and datasets sampled from FGVC benchmark~\cite{jia2022vpt}. We visualize the proportions of the sensitive parameters under 0.4M trainable parameter budget. Visualizations of sampled VTAB-1k datasets with MAE and MoCo V3 pre-trained ViT-B/16 are shown in Figures~\ref{fig:sensitive_sup},~\ref{fig:sensitive_mae},~\ref{fig:sensitive_moco}. Visualizations of sampled FGVC datasets with supervised pre-trained ViT-B/16 are shown in Figure~\ref{fig:sens_fgvc}. We find our observations in the main paper are general: the proportions of the sensitive parameter exhibit: 1) dataset-specific varying patterns in terms of network depth; and 2) dataset-agnostic similar patterns in terms of operations. We empirically find} that the self-supervised pre-trained backbones have higher sensitivity variances than the supervised pre-trained one across the 19 downstream tasks. In particular, the variance of ViT-B/16 pre-trained with MAE~\cite{he2022masked} is twice as large as that of the supervised pre-trained ViT-B/16. We speculate that our SPT variants can better handle the large variances for self-supervised pre-trained backbones (Table 2 of the main paper) by identifying task-specific positions to introduce the trainable parameters.

\begin{figure}[htb]
\begin{center}
    \includegraphics[width=\linewidth]{sensitive_sup.pdf}
\end{center}
\caption{The distribution of sensitive parameters by blocks under 0.4M trainable parameter budget with supervised pre-trained ViT-B/16 backbone. We sample six tasks from VTAB-1k~\cite{zhai2019vtab}.
}
\label{fig:sensitive_sup}
\end{figure}

\begin{figure}[htb]
\begin{center}
    \includegraphics[width=\linewidth]{sensitive_mae.pdf}
\end{center}
\caption{The distribution of sensitive parameters by blocks under 0.4M trainable parameter budget with \mae{}~\cite{he2022masked} pre-trained ViT-B/16 backbone. We sample six tasks from VTAB-1k~\cite{zhai2019vtab}.}
\label{fig:sensitive_mae}
\end{figure}

\begin{figure}[tb]
\begin{center}
    \includegraphics[width=\linewidth]{sensitive_moco.pdf}
\end{center}
\caption{The distribution of sensitive parameters by blocks under 0.4M trainable parameter budget for \moco{}~\cite{chen2021empirical} pre-trained ViT-B/16 backbone. We sample six tasks from VTAB-1k~\cite{zhai2019vtab}.}
\label{fig:sensitive_moco}
\end{figure}

\begin{figure}[tb]
\begin{center}
    \includegraphics[width=0.8\linewidth]{rebuttal_fgvc_sensitivity.pdf}
\end{center}
\caption{Sensitivity patterns under 0.4M trainable parameters for Oxford Flowers~\cite{nilsback2008automated}, Stanford Cars~\cite{gebru2017cars}, and Stanford Dogs~\cite{Khosla_FGVC2011dogs}. We show the proportions of the sensitive
parameters for the query $\mW_{q}$, key $\mW_{k}$, value $\mW_{v}$, and $\mW_{o}$ weight matrices in the multi-head self-attention layer and two weight matrices $\mW_{fc1}$ and $\mW_{fc2}$ in the feed-forward network. 
}
\label{fig:sens_fgvc}
\end{figure}

\begin{figure}[htb]
\begin{center}
    \includegraphics[width=0.6\linewidth]{variance.pdf}
\end{center}
\caption{Comparisons of sensitivity variances across backbones with different pre-training strategies on \vtab{}.}
\label{fig:variance}
\end{figure}

\begin{figure}[htb]
\begin{center}
\includegraphics[width=0.6\linewidth]{natural_structured.pdf}
\end{center}
\caption{Dataset samples from \imagenet~\cite{krizhevsky2012imagenet} and \vtab{}~\cite{zhai2019vtab}. Samples from Natural tasks of \vtab{} ((a), (b), and (c)) are relatively more similar to the source \imagenet{} samples compared to the ones from Structured tasks of \vtab{} ((d), (e), and (f)).}
\label{fig:domain}
\end{figure}

\section{Dataset Samples for the Source and Target Domains}
\label{subsec:supp_visual}
We visualize some sampled images from the source domain (\imagenet~\cite{krizhevsky2012imagenet}) and the target domains (\vtab{}~\cite{zhai2019vtab}) in Figure~\ref{fig:domain}. We observe that the images from the Natural tasks of \vtab{} are relatively more similar to the source domain compared to those from the Structured tasks of \vtab{}, which aligns with our observation that Structured tasks have large domain gaps. As structured tuning
improves the performance of Structured datasets (Section 4.3 of the main paper), we speculate that
structured tuning facilitates mitigating such large domain gaps.

\begin{table}[t]

\scriptsize
\resizebox{\textwidth}{!}{%
    \begin{tabular}{lc| cccccc}
    \toprule
      &  Tuned / Total &\bf{\cub{}} 
  &\bf{\nabirds{}}
  &\bf{\flowers{}} &\bf{\dogs{}} &\bf{\cars{}}
  &\bf{Mean Acc.} \\
    \midrule
    \band \fullft{} & 100\% &87.3 &82.7 &98.8 &89.4 &84.5 &88.5\\
    \midrule
        \multicolumn{8}{c}{\bf{Addition-based methods}}\\
    \midrule
    \mlp{}-3 & 1.50\% &85.1 &77.3 &97.9 &84.9 &53.8 &79.8
    \\
    \shallowprompt{} & 0.31\% & 86.7 &78.8 &98.4 &\underline{90.7} &68.7 &84.6\\
    \deepprompt{} & 0.98\% & \underline{88.5} &\underline{84.2} &\underline{99.0} &90.2 &83.6 &89.1\\
    \adapter{}-8 & 0.39\% & 87.3 &\textbf{84.3} &98.4 &88.8 &68.4 &85.5\\
    \adapter{}-32 & 0.95\% & 87.2 &\textbf{84.3} &98.5 &89.6 &68.4 &85.6\\
    \adaptformer{} & 0.44\% & 84.7 &75.2 &97.9 &84.7 &83.1 &85.1\\
    \SPTa{} & 0.41\% & \textbf{89.1} &83.3 &\textbf{99.2} &90.5 &\underline{85.6} &\underline{89.5}\\
    \SPTa{} & 0.47\% & \textbf{89.1} &83.3 &\textbf{99.2} &\textbf{91.1} &\textbf{86.2} &\textbf{89.8}\\
    \midrule
     \multicolumn{8}{c}{\bf{Reparameterization-based methods}}\\
    \midrule
    \linear{} & 0.12\% & 85.3 &75.9 &97.9 &86.2 &51.3 &79.3\\
    \partialft{}-1 & 8.38\% &85.6 &77.8 &98.2 &85.5 &66.2 &82.6\\
    \bias{} & 0.13\% &\underline{88.4} &\textbf{84.2} &98.8 &\underline{91.2} &79.4 &88.4\\
    \lora{}-8 & 0.55\% &84.9 &79.0 &98.1 &88.1 &79.8 &86.0 \\
    \lora{}-16 & 0.90\% &85.6 &79.8 &98.9 &87.6 &72.0 &84.8 \\
    \SPTl{} & 0.41\% &\textbf{88.6} &82.8 &\underline{99.4} &\textbf{91.4} &\underline{84.5} &\underline{89.3} \\
    \SPTl{} & 0.60\% &\textbf{88.6} &\underline{83.4} &\textbf{99.5} &\textbf{91.4} &\textbf{87.3} &\textbf{90.1} \\
\bottomrule
    \end{tabular}}
    \caption{
    Per-task results on the FGVC benchmark from Table~1 of the main paper. ``Tuned / Total'' denotes the fraction of the trainable parameters. Top-1 accuracy (\%) is reported. The best result is in \textbf{bold}, and the second-best result is \underline{underlined}.
}\label{tab:full_fgvc}
\end{table}


\begin{sidewaystable}[t]
\scriptsize
\resizebox{\textwidth}{!}{%
    \begin{tabular}{lc | cccccccc | ccccc | ccccccccc}
    \toprule
    & & \multicolumn{8}{c|}{\textbf{Natural}} & \multicolumn{5}{c|}{\textbf{Specialized}} & \multicolumn{9}{c}{\textbf{Structured}} \\
    & \rotatebox{90}{Tuned / Total} & \rotatebox{90}{\bf{Cifar100}} & \rotatebox{90}{\bf{Caltech101}} & \rotatebox{90}{\bf{DTD}} & \rotatebox{90}{\bf{Flower102}} & \rotatebox{90}{\bf{Pets}} & \rotatebox{90}{\bf{SVHN}}  & \rotatebox{90}{\bf{Sun397}} & \rotatebox{90}{\bf{Mean Acc.}} & \rotatebox{90}{\bf{Camelyon}}  & \rotatebox{90}{\bf{EuroSAT}}   & \rotatebox{90}{\bf{Resisc45}}  & \rotatebox{90}{\bf{Retinopathy}} & \rotatebox{90}{\bf{Mean Acc.}} & \rotatebox{90}{\bf{Clevr-Count}} & \rotatebox{90}{\bf{Clevr-Dist}}  & \rotatebox{90}{\bf{DMLab}} & \rotatebox{90}{\bf{KITTI-Dist}}  & \rotatebox{90}{\bf{dSpr-Loc}} & \rotatebox{90}{\bf{dSpr-Ori}}   & \rotatebox{90}{\bf{sNORB-Azim}}  & \rotatebox{90}{\bf{sNORB-Ele}} & \rotatebox{90}{\bf{Mean Acc.}}   \\
    \midrule
\band \fullft{} & 100\% &68.9 &87.7 &64.3 &97.2 &86.9 &87.4 &38.8 &75.9 &79.7 &95.7 &84.2 &73.9 &83.4 &56.3 &58.6 &41.7 &65.5 &57.5 &46.7 &25.7 &29.1 &47.6

    \\\midrule
     \multicolumn{22}{c}{\bf{Addition-based methods}}
    \\\midrule
    \mlp{}-3 & 1.50\% &63.8 &84.7 &62.3 &97.4 &84.7 &32.5 &49.2 &67.8 &77.0 &88.0 &70.2 &56.1 &72.8 &47.8 &32.8 &32.3 &58.1 &12.9 &21.2 &15.2 &24.8 &30.6\\
    \shallowprompt{} & 0.31\%  & 77.7 &86.9 &62.6 &97.5 &87.3 &74.5 &51.2 &76.8 &78.2 &92.0 &75.6 &72.9 &79.7 &50.5 &58.6 &40.5 &67.1 &68.7 &36.1 &20.2 &34.1 &47.0\\
    \deepprompt{} & 0.98\% &78.8 &90.8 &65.8 &98.0 &88.3 &78.1 &49.6 &78.5 &81.8 &96.1 &83.4 &68.4 &82.4 &68.5 &60.0 &46.5 &72.8 &73.6 &47.9 &32.9 &37.8 &55.0\\
    \adapter{}-8  & 0.39\% & 69.2 & 90.1 & 68.0 & 98.8 & 89.9 & 82.8 & 54.3 & 79.0 & 84.0 & 94.9 & 81.9 & 75.5 & 84.1 & 80.9 & 65.3 & 48.6 & 78.3 & 74.8 & 48.5 & 29.9 & 41.6 & 58.5\\
    \adapter{}-32 & 0.71\% & 68.7 & 92.2 & 69.8 &98.9 & 90.3& 84.2& 53.0& 79.6& 83.2& 95.4& 83.2& 74.3 & 84.0 & 81.9 & 63.9& 48.7 & 80.6& 76.2& 47.6& 30.8& 36.4 & 58.3 \\
    \noah{} & 0.50\% & 69.6 & 92.7 & 70.2 & 99.1 & 90.4 & 86.1 & 53.7 & 80.2 & 84.4 & 95.4 & 83.9 & 75.8 & 84.9 & 82.8 & 68.9 & 49.9 & 81.7 & 81.8 & 48.3 & 32.8 & 44.2 & 61.3\\
    \SPTa{} & 0.30\% & 72.9 & 93.2 & 72.5 & 99.3 & 91.4 & 84.6 & 55.2 & 81.3 & 85.3 & 96.0 & 84.3 & 75.5 & 85.3 & 82.2 & 68.0 & 49.3 & 80.0 & 82.4 & 51.9 & 31.7 & 41.2 & 60.8\\
    \SPTa{} & 0.44\% & 72.9 & 93.2 & 72.5 & 99.3 & 91.4 & 88.8 & 55.8 & 82.0 & 86.2 & 96.1 & 85.5 & 75.5 & 85.8 & 83.0 & 68.0 & 51.9 & 81.2 & 82.4 & 51.9 & 31.7 & 41.2 & 61.4\\
    \midrule
     \multicolumn{22}{c}{\bf{Reparameterization-based methods}}
    \\\midrule
    \linear{} & 0.12\% & 63.4 &85.0 &63.2 &97.0 &86.3 &36.6 &51.0 &68.9 &78.5 &87.5 &68.6 &74.0 &77.2 &34.3 &30.6 &33.2 &55.4 &12.5 &20.0 &9.6 &19.2 &26.8\\
    \partialft{}-1 & 8.38\% &66.8 &85.9 &62.5 &97.3 &85.5 &37.6 &50.6 &69.4 &78.6 &89.8 &72.5 &73.3 &78.5 &41.5 &34.3 &33.9 &61.0 &31.3 &32.8 &16.3 &22.4 &34.2\\
    \bias{} & 0.13\% &72.8 &87.0 &59.2 &97.5 &85.3 &59.9 &51.4 &73.3 &78.7 &91.6 &72.9 &69.8 &78.3 &61.5 &55.6 &32.4 &55.9 &66.6 &40.0 &15.7 &25.1 &44.1\\
\lora{}-8 & 0.55\% & 67.1 & 91.4 & 69.4 & 98.8 & 90.4 & 85.3 & 54.0 &79.5 & 84.9 & 95.3 & 84.4 & 73.6 & 84.6 & 82.9 & 69.2 & 49.8 & 78.5 & 75.7 & 47.1 & 31.0 & 44.0 & 60.5 \\
\lora{}-16 & 0.90\% & 68.1 & 91.4 & 69.8 & 99.0 & 90.5 & 86.4 & 53.1 &79.8 & 85.1 & 95.8 & 84.7 & 74.2 & 84.9 & 83.0 & 66.9 & 50.4 & 81.4 & 80.2 & 46.6 & 32.2 & 41.1 & 60.2 \\
\SPTl{} & 0.31\% & 72.3 & 93.0 & 72.5 & 99.3 & 91.5 & 86.2  & 55.5 & 81.5 & 85.0 & 96.2 & 85.1 & 75.9 & 85.6 & 83.7 & 66.4 & 52.5 & 80.2 & 80.1 & 51.1 &  30.1 & 41.3 & 60.7 \\
\SPTl{} & 0.63\% & 73.5 & 93.3 & 72.5 & 99.3 & 91.5 & 87.9 & 55.5 & 81.9 & 85.7 & 96.2 & 85.9 & 75.9 & 85.9 & 84.4 & 67.6 & 52.5 & 82.0 & 81.0 & 51.1 &  30.2 & 41.3 & 61.3 \\
\bottomrule
    \end{tabular}}
    \caption{
    Per-task results on the \vtab{} benchmark from Table~1 of the main paper. ``Tuned / Total'' denotes the fraction of the trainable parameters. Top-1 accuracy (\%) is reported.
}\label{tab:full_vtab}
\end{sidewaystable}


\end{document}
