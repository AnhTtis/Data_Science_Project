\documentclass[12pt]{amsart}
\usepackage[margin=1.5in]{geometry}

\usepackage[utf8]{inputenc}
\usepackage{amsmath,amsfonts,amssymb,mathrsfs,amstext,amscd,latexsym,amsthm, mathtools}
\usepackage{hyperref}
\usepackage{enumerate}
\usepackage{comment}
\usepackage{xcolor}
\usepackage{pgfplots}


%\usepackage[style=numeric]{biblatex}
%\addbibresource{references.bib}

\usepackage{graphicx}
\usepackage{subcaption}

\def\R{\mathbb{R}}
\def\Z{\mathbb Z}
%\def\N{\mathbb{N}}
\def\e{\mathbf{e}}
\def\E{\mathbb{E}}
\DeclareMathOperator{\spa}{span}
\DeclareMathOperator{\tr}{tr}

\newtheorem{theorem}{Theorem}[section]
\newtheorem{lemma}[theorem]{Lemma}
\newtheorem{corollary}[theorem]{Corollary}
\newtheorem{proposition}[theorem]{Proposition}
\newtheorem*{definition}{Definition}
\newtheorem*{remark}{Remark}
\newtheorem*{contributions}{Contributions}

\DeclareMathOperator{\fff}{I}
\DeclareMathOperator{\sff}{II}
\DeclareMathOperator{\graph}{graph}

\title{Ancient pancake solutions to fully nonlinear curvature flows}
\author{Mat Langford}
\author{Sathyanarayanan Rengaswami}
\date{March 2023}

\usepackage{graphicx}

\pgfplotsset{compat=1.17}
\begin{document}

\begin{abstract}
We construct $O(1)\times O(n)$-invariant ancient ``pancake'' solutions %{\color{blue}(i.e. convex ancient solutions lying in a slab region)} 
to a large and natural class of fully nonlinear curvature flows. We then establish that each is unique within the class of $O(n)$-invariant ancient solutions to the corresponding flow which sweep out a slab by carrying out a fine asymptotic analysis for this class. This extends the main results of \cite{BLT} to a surprisingly general class of flows.
\end{abstract}

%\keywords{Curvature flows, ancient solutions, pancakes, fully nonlinear parabolic PDE}


\maketitle

\setcounter{tocdepth}{1}

\tableofcontents


%{\color{red}Some comments:

%-- We should distinguish $\phi$ (as a function of curvatures) from $\phi\circ\kappa$ (the speed) or make a note that we conflate the two.}







\section{Introduction}

Extrinsic geometric flows drive  hypersurfaces in their normal direction with speed determined pointwise by their extrinsic curvature. The most well-known example is the mean curvature flow, but several other interesting flows have been studied in the literature, and have found applications in, for example, geometry, materials science, general relativity, and image processing. %In order for the flow \eqref{eq:F} to be well-defined and parabolic (in a suitable gauge), the minimum requirements on the speed, as a function of the principal curvatures function, are that it should be permutation invariant and monotone increasing, respectively. %In order to admit parabolic scaling invariance, $\phi$ should be homogeneous. 
%We will be interested in flows that admit an additional invariance under parabolic rescaling, as well as very mild concavity and non-degeneracy conditions. We will give a more precise definition in \S \ref{sec:admissible speeds}. %We will be interested in \emph{compact flows}, i.e. when the timeslices are compact. We also only consider convex flows, meaning that each timeslice $\Sigma_t$ bounds a convex body (and hence is topologically the $n$-sphere $S^n$). The latter condition is preserved under the class of flows we consider. 
%A solution to such a flow is said to be \emph{ancient} if it is defined for all times in a backwards-infinite time interval $(-\infty,T)$, where $T \leq \infty$.

\emph{Ancient} solutions to nonlinear parabolic partial differential equations typically arise as blowup limits about singularities, and tend to exhibit rigidity properties% (cf. Appell's theorem for the heat equation)
. %A complete classification of such solutions appears to be a very difficult problem, and any such classification cannot be very simple \cite{...} but the added assumption of convexity makes the problem more tractable% (cf. positivity in Appell's theorem). %Indeed, the convex ancient solutions to curve shortening flow in the plane have been completely classified: the only ones up to ambient isometries and parabolic rescaling are the shrinking circle, the Grim Reaper, the Angenent oval, the stationary line, and the stationary line. In higher dimensions, however, such a classification has not been achieved, even for the mean curvature flow.
Our purpose here is to analyse ``ancient pancake'' solutions (i.e. compact convex ancient solutions which are confined to slab regions) to a large and natural class of extrinsic geometric flows. In particular, we establish the following classification theorem.

\begin{theorem}\label{thm:pancakes}
Given any non-degenerate admissible speed function, there exists an $O(1)\times O(n)$-invariant ancient solution $\{\partial\Omega_t\}_{t\in(-\infty,0)}$, $\Omega_t\underset{\text{convex}}{\subset}\R^{n+1}$, to the corresponding extrinsic geometric flow exhibiting the following behaviour:
\begin{enumerate}[(a)]
\item $\{\lambda\Omega_{\lambda^{-2}t}\}_{t\in(-\infty,0)}\to\{B^{n+1}_{\sqrt{-2\phi_1t}}\}_{t\in(-\infty,0)}$ locally uniformly in the smooth topology as $\lambda\to\infty$, where $\phi_1$ is the value the speed takes on the unit sphere.
\item $\cup_{t\in(-\infty,0)}\Omega_t=\{(x,y)\in \R\times\R^{n-1}:\vert x\vert<\frac{\pi}{2}\}$, and
\item $\{\partial\Omega_{t+\tau}-P(w,\tau)\}_{t\in(-\infty,-\tau)}\to \{\Gamma_t^n\}_{t\in(-\infty,0)}$ locally uniformly in the smooth topology as $\tau\to-\infty$ for every $w\in \{0\}\times S^{n-1}$, where $P(w,t)$ is the point on $\partial\Omega_t$ whose outer unit normal is $w$, and %$\{\Gamma^n_t\}_{t\in(-\infty,0)}$ is the standard Grim hyperplane with velocity $-z$: 
\[
\Gamma^n_t\coloneqq\{p=(x,\hat x)\in \left(-\tfrac{\pi}{2},\tfrac{\pi}{2}\right)\times\R^{n}:-p\cdot w=\log\sec x+t\}
\]
is the Grim hyperplane with velocity $w$. 
\end{enumerate}
Moreover, $\{\partial\Omega_t\}_{t\in(-\infty,0)}$ is unique amongst $O(n)$-invariant convex ancient solutions to the corresponding flow satisfying property (b). 
\end{theorem}

The class of speeds we consider (see \S \ref{sec:admissible speeds} for a definition and discussion) is large and natural in view of the behaviour described in Theorem \ref{thm:pancakes}. It includes of course the mean curvature, which is uniformly elliptic, but also many examples whose ellipticity degenerates at the boundary of the positive cone (such as power means %$P_r(z)=\left(\frac{1}{n} \Sigma_{i=1}^n z_i^r \right)^{1/r}$
in the principal curvatures and linear combinations of these with highly degenerate speeds such as the $n$-th root of the Gauss curvature). %Note that even if a speed is only ``asymptotically" 1-homogeneous i.e. $\phi_\lambda(z)=\lambda^{-1}\phi(\lambda z)$ converges to a flow speed when $\lambda \to \infty$, parabolic blowup limits give rise to a $1$-homogeneous flow. 


%We define a \emph{slab region} to be a subset of $\R^{n+1}$ that is a rigid motion of $ I\times\R^n$, where $I=(a,b)$ is an interval of finite width. A \emph{pancake solution} to a curvature flow is that is a compact, convex ancient solution that lies in a slab region. %These are important in mean curvature flow because of the following theorem of X-.J. Wang.
Ancient solutions confined to a slab (the region between two parallel hyperplanes) are natural in view of Wang's slab dichotomy \cite{Wang}, which states, for convex ancient mean curvature flows $\{\partial\Omega_t\}_{t\in(-\infty,\omega)}$, that if the region $\cup_{t\in(-\infty,0)}\Omega_t$ is a strict subset of $\R^{n+1}$, then it is a slab. 
%
%\begin{theorem}
%Any convex ancient solution to mean curvature flow along which $\sup_{\mathcal{M}_t}\vert{\sff}\vert$ is bounded on compact time intervals either sweeps out all of $\R^{n+1}$ or is contained in a slab region.
%\end{theorem}
%
%
%Wang \cite{Wang} also proved the existence of ancient pancake solutions to mean curvature flow by taking the limit of a sequence of solutions to the Dirichlet problem for the (Legendre transformed) level set flow. %This produces a variety of pancake solutions with a lot of symmetry, and the one with $O(1)\times O(n)$ symmetry is particularly interesting  as it is in some sense the most symmetric pancake solution. %By $O(n)$-symmetry we mean rotational symmetry with respect to some fixed axis, and by $O(1)$-symmetry, we mean reflection symmetry about a hyperplane orthogonal to the axis.
%
%
%For fully  nonlinear flows, the picture becomes far more complicated...
%The shrinking sphere solutions are examples of ``entire'' solutions (i.e. $\cup_{t<0}\Omega_t=\R^{n+1}$). 
When $n=1$, an ancient solution which sweeps out a slab region in the plane was constructed by Bourni \emph{et al.} in \cite{MR4375790} for flows by certain powers of the curvature.
On the other hand, while it is known, for a large class of flows, that the shrinking spheres are the only ``pinched'' ancient solutions \cite{MR4129354,MR3935256}, non-spherical $O(n)$-invariant ancient solutions (to a somewhat different but nontrivially overlapping class of flows) which sweep out all of space were constructed by Risa and Sinestrari \cite{RisaSinestrari} (see also \cite{MR4234099,RisaThesis}). When a suitable differential Harnack inequality is available \cite{AndrewsHarnack}, these ``ancient ovaloid'' solutions decompose into a pair of entire ``bowl-type'' translating solitons joined by a shrinking cylinder. (A comprehensive analysis of axially symmetric translating solitons for extrinsic geometric flows, which sheds light on these phenomena, was undertaken in \cite{Rengaswami}. See also \cite{MR1625754}.)

%In \cite{BLT}, an $O(1)\times O(n)$-invariant ancient pancake solution was obtained by taking the limit of a family of ``very old" solutions obtained by evolving rotations of ``very old'' timeslices of the Angenent oval. %The desired ancient solution is just an Arzel\`a--Ascoli type limit of these flows, which yields an axially symmetric pancake solution to mean curvature flow. 
%The advantage of this approach is that a lot of geometric information about this constructed solution becomes available, like the asymptotic expansion for the radial displacement of the solution as $t \to -\infty$. The resulting fine asymptotics turn out to be sufficient to show that this solution is unique amongst $O(n)$-invariant examples (up to rigid motions and parabolic rescaling).

When the speed is the mean curvature, we recover the main results of \cite{BLT}, and indeed we follow a similar approach here (though there are a number of new difficulties that arise, due to the nonlinearity of the speed function and the degeneration of ellipticity as convexity degenerates): we construct the ancient pancake by taking the limit of a family of ``very old" solutions obtained by evolving rotations of ``very old'' timeslices of the Angenent oval. A rough asymptotic analysis of arbitrary convex ancient solutions that have $O(n)$-symmetry (which exploits Andrews' differential Harnack inequality \cite{AndrewsHarnack}), shows that such solutions necessarily have the additional $O(1)$-symmetry as well. With some work, we are then able to establish a fine asymptotic expansion for the radial displacements of these solutions as $t \to -\infty$, which we are able to exploit to show that they are unique.

\section{Preliminaries}

\subsection{Admissible flow speeds}\label{sec:admissible speeds}

Given a smooth function $\phi:\Gamma\subset \R^n\to \R$, a family of hypersurfaces $\{\Sigma^n_t\}_{t\in I}$ of $\R^{n+1}$ evolves by the corresponding extrinsic curvature flow (which we call the $\phi$-\emph{flow}) if there is a smooth 1-parameter family of embeddings $F:M^n \times I \to \R^{n+1}$ such that $F(M,t)=\Sigma_t$ and
\begin{equation}\label{eq:F}
    \partial_t F=-\phi(\kappa_1,\dots,\kappa_n)\nu\,,
\end{equation}
where $\nu$ is a unit normal field and $\kappa_1\leq\ldots\leq\kappa_n$ are the principal curvatures with respect to $\nu$. %We call $\phi$ the \emph{flow speed}. We call solutions to \eqref{eq:F} \emph{$\phi$-flows.}

%We define here the speeds that we consider:

Since we are interested in convex solutions, we require the speed function to be defined in the positive cone $\Gamma=\Gamma_+ \coloneqq \{(\kappa_1, \ldots, \kappa_n): \kappa_1,\ldots,\kappa_n > 0\}$. %Its closure is $\overline{\Gamma}_+$.

\begin{definition}
A function\footnote{The regularity of $\phi$ can be relaxed to $C^2$, so long as suitable adjustments to regularity conclusions are made in what follows.} $\phi\in C^\infty(\Gamma_+)$ is an \emph{admissible speed} if it is
\begin{enumerate}[(a)]
\item symmetric: $\phi(z_{\sigma(1)},\ldots,z_{\sigma(n)})=\phi(z_1,\ldots,z_n)$ for all permutations $\sigma$ of the set \{1,\ldots,n\};
\item elliptic: $\frac{\partial\phi}{\partial z_i}>0$ for each $i=1,\dots,n$;\label{cond:parabolic}
\item 1-homogeneous: $\phi(\lambda z_1,\ldots,\lambda z_n)=\lambda\phi(z_1,\dots,z_n)$ for every $\lambda>0$;
\item inverse-concave: the function $(r_1,\ldots,r_n)\mapsto\phi(r_1^{-1},\ldots,r_n^{-1})^{-1}$ is concave.
\end{enumerate}
We call an admissible speed $\phi$ \emph{non-degenerate} if it is
\begin{enumerate}[(e)]
\item non-degenerate: the function $s\mapsto \phi(1,s,\dots,s)$ is of class $C^2([0,\infty))$ and satisfies $\phi(1,0,\dots,0)>0$.
\end{enumerate}
\end{definition}

Note that ellipticity of a 1-homogeneous function on $\Gamma_+$ is equivalent to ellipticity on the subset $\Gamma_+\cap S^n$. Moreover, any 1-homogeneous elliptic function on $\Gamma_+$ extends continuously to $\overline{\Gamma}_+$ \cite[Lemma 1]{MR3070558} %We include (\ref{cond:parabolic}) as a weaker condition than uniform parabolicity $\frac{\partial\phi}{\partial \kappa_i}>\delta>0$.
and is necessarily positive in $\Gamma_+$ (due to Euler's identity for homogeneous functions). Since we assume that $\phi(1,0,\ldots,0) \neq 0$, we may without loss of generality normalize the speed function so that $\phi(1,0,\ldots,0)=1$. We also set $\phi_1\coloneqq \phi(1,\ldots,1)$ and $\dot\phi_1\coloneqq \frac{d}{ds}\big\vert_{s=0}\{s\mapsto \phi(1,s,\dots,s)\}$; these three constants will play a major role in determining the asymptotic behaviour of the ancient solutions we consider.

Let us briefly discuss the purpose of conditions (a)-(e). The first two conditions are ``non-negotiable'' in that symmetry is needed to ensure that $\phi$ may be regarded as a smooth function of the second fundamental form $\sff$ (which in turn ensures that the composition of $\phi$ with the principal curvatures/second fundamental form is a smooth function on spacetime), while ellipticity is needed to ensure that \eqref{eq:F} may be interpreted as a parabolic partial differential equation. Note that, even for non-degenerate admissible speeds, the derivatives of $\phi$ with respect to the principal curvatures are allowed to degenerate at $\partial\Gamma_+$.

Homogeneity guarantees that \eqref{eq:F} is invariant under the parabolic rescaling $\Sigma_t\mapsto \lambda\Sigma_{\lambda^{-2}t}$. Note that, under the milder condition of \emph{asymptotic 1-homogeneity}, meaning that the limit
\[
(T\phi)(z)\coloneqq \lim_{\lambda\to\infty}\lambda^{-1}\phi(\lambda z)
\]
exists, blow-up limits (with respect to the rescaling $\Sigma_t\mapsto \lambda\Sigma_{\lambda^{-2}t}$) to the flow \eqref{eq:F} with speed $\phi$ evolve by \eqref{eq:F} with $\phi$ replaced by the 1-homogeneous speed $T\phi$. So our results are relevant also to certain flows by asymptotically 1-homogeneous speeds.

Inverse-concavity is typically invoked to guarantee the applicability of the Krylov--Safanov Harnack inequality; this is not needed here, however, due to the symmetry of the solutions we consider --- instead, we require it in order to make use of Andrews' differential Harnack inequality, which guarantees, e.g., that ancient solutions are asymptotically modelled on translating solitons.

The non-degeneracy condition ensures that $\phi$ is uniformly elliptic in the principal direction corresponding to the largest principal curvature, even as the other principal curvatures approach zero (the derivatives of $\phi$ in the remaining directions are still allowed to degenerate, however). Under the normalization $\phi(1,0,\dots,0)=1$, the non-degeneracy condition guarantees that the cylinder $\{\Gamma_t\times \R^{n-1}\}_{t\in I}$ is a solution to \eqref{eq:F} whenever $\{\Gamma_t\}_{t\in I}$ is a solution to curve shortening flow. This ensures that the ancient solutions we construct are asymptotically modelled on Grim hyperplanes.

%ML: Examples: when $n=2$, loads of examples are given in EGF. There are many convex speeds which uniformly parabolic on $\Gamma_+$. Given an open cone $\Gamma$ which contains $\overline\Gamma_+\cap S^n$, can we construct a concave example which is defined on $\Gamma$ and inverse-concave on $\Gamma_+$?

\subsubsection{Examples}
The class of admissible speeds is very large (see, for example, \cite[\S18.3]{EGF}). As mentioned in the introduction, the power means
\[
P_r(z_1,\dots,z_1)\coloneqq\left(\sum_{i=1}^nz_i^r\right)^{\frac{1}{r}}\,,\;\; r>0
\]
give rise to non-degenerate admissible speeds. Further examples are given by convex functions $f\in C^\infty(\Gamma)$ satisfying conditions (a)-(c) with $\Gamma_+\cap S^n\Subset \Gamma$. 
\begin{comment}
When $n=2$, we can express any smooth, symmetric, one-homogeneous function $\phi:\Gamma_+\to\R$ as
\[
\phi(z_1,z_2)=r(z_1,z_2)h(\vartheta(z_1,z_2))
\]
with $h:\in C^\infty(-\frac{\pi}{4},\frac{\pi}{4})$ an even function, where $(r,\vartheta)$ are counterclockwise oriented polar coordinates measured about the positive ray; that is,
\[
r(z_1,z_2)\doteqdot\sqrt{z_1^2+z_2^2}\,,\;\;\cos\vartheta(z_1,z_2)=\frac{z_1+z_2}{\sqrt{2}r(z_1,z_2)},\;\;\sin\vartheta(z_1,z_2)=\frac{z_1-z_2}{\sqrt{2}r(z_1,z_2)}.
\]
Ellipticity of $\phi$ is then equivalent to positivity of $h$ and the inequality
\[
\frac{h'(\vartheta)}{h(\vartheta)}\ge \frac{\cos\vartheta-\sin\vartheta}{\cos\vartheta+\sin\vartheta}\,,
\]
and inverse-concavity is equivalent to concavity of the function
\[
h_\ast(\vartheta)\doteqdot\frac{\cos(2\vartheta)}{h(\vartheta)}\,.
\]
The non-degeneracy condition requires that $h\in C^2([-\frac{\pi}{4},\frac{\pi}{4}])$ and $h(\frac{\pi}{4})>0$. It follows that the function
\[
h(\vartheta)=\exp\left(\int_0^\vartheta g(\sigma)\,d\sigma\right)
\]
defines an admissible speed whenever $g\in C^\infty(-\frac{\pi}{4},\frac{\pi}{4})$ is an odd function satisfying
\[
g(\vartheta)\ge \frac{\cos\vartheta-\sin\vartheta}{\cos\vartheta+\sin\vartheta}\;\;\text{and}\;\;\frac{d}{d\vartheta}\frac{1}{g(\vartheta)+2\tan(2\vartheta)}\le -1\,.
\]
\end{comment}

Note also that the class of (non-degenerate) admissible speeds is closed under natural compositions: if $\zeta_1,\dots,\zeta_k\in C^\infty(\Gamma_+^n)$ and $\phi\in C^\infty(\Gamma_+^k)$ are (non-degenerate) admissible speeds, then so is $\phi(\zeta_1,\dots,\zeta_k)$. 

%\bigskip

%\noindent [ML: In fact, if we agree to call an admissible speed $\phi$ \emph{$i$-non-degenerate} whenever $s\mapsto \phi(x_1,\dots,x_i,s,\dots,s)\in C^2([0,1))$ and $\phi(x_1,\dots,x_i,0,\dots,0)>0$ for each $(x_1,\dots,x_i)\in \Gamma_+^i$, then $\phi(\zeta_1,\dots,\zeta_k)$ is $j$ non-degenerate whenever $\phi$ is $i$-non-degenerate and at least $i$ of the $\zeta_i$ are $j$ non-degenerate.]

\subsection{The Angenent oval}

The Angenent oval (a.k.a the paperclip) is a solution to curve shortening flow that lies in the strip $\left( -\frac{\pi}{2},\frac{\pi}{2} \right)\times \R$. It is defined for $t \in (-\infty,0)$ and satisfies the implicit equation
\begin{equation}
    \cos x = e^t \cosh{y}\,.
\end{equation}

In the turning angle parametrization $(x(\theta,t),y(\theta,t))$, where $\theta$ is the angle made by the tangent vector and the $x$-axis when the curve is oriented counterclockwise, the $x,y$ coordinates are given explicitly by
\begin{equation*}
    x(\theta,t)= \arctan \left( \frac{\sin\theta}{\sqrt{\cos^2\theta+a^2(t)}}\right)
\end{equation*}
and
\begin{equation*}
    y(\theta,t)= -t + \log \left( \frac{\sqrt{\cos^2\theta+a^2(t)}-\cos \theta}{\sqrt{a^2(t)+1}}\right)\,,
\end{equation*}
where $a^2(t) \coloneqq \frac{1}{e^{-2t}-1}$.

We also define the horizontal and vertical displacements $h(t)$ and $\ell(t)$ by
\begin{equation*}
    h(t) \coloneqq x(\tfrac{\pi}{2},t)
\end{equation*}
and 
\begin{equation*}
    h(t) \coloneqq y(\pi,t)\,.
\end{equation*}
These displacements satisfy the estimates
\begin{eqnarray} \label{DispEst}
    \frac{\pi}{2}(1-e^{-t}) \leq h(t) \leq \frac{\pi}{2}\,,\\
    -t \leq \ell(t) \leq -t+ \log 2\,.
\end{eqnarray}

\subsection{$O(n)$-invariance}

Let us give $\R^{n+1}$ the coordinates $(x,y,z)\in \R\times\R\times\R^{n-1}$. By $O(n)$-invariance of a hypersurface $\Sigma^n \subset \R^{n+1}$, we will mean that it is invariant under the action of $O(n)$ on the $(y,z)$ hyperplane; %(i.e. axial symmetry about the $x$-axis). 
this of course means that $\Sigma^n=\{xe_1+y\eta : xe_1+ye_2\in \Sigma^n\cap \E^2, \eta \in S^n\cap\{0\}\times\R^{n}\}$, where $\E^2\coloneqq \R^2\times\{0\}\subset\R^{n+1}$. We will refer to the curve $\Sigma^n\cap \E^2$ as the \emph{profile curve} of $\Sigma^n$.

%More precisely, we mean the following: Define $\E^2 \coloneqq \{(x,y,z)\in \R\times\R\times\R^{n-1}:z=0\}$, $\mathbb{F}^n \coloneqq \{(x,y,z)\in\R\times\R\times\R^{n-1}:x=0\}$ and $\Sigma'=\Sigma^n \cap \E^2$. Let $e_1,e_2$ be unit vectors in the positive $x,y$ directions;identify the set of unit vectors in $\mathbb{F}^n$ with the unit sphere $S^{n-1}$. Then we say $\Sigma^n$ is $O(n)$-invariant if $\Sigma^n= \{xe_1+y \eta : xe_1+ye_2\in \Sigma', \eta \in S_n\}$

Conversely, given a smooth  convex curve $\Gamma\subset \R^2$ that is reflection symmetric about the $x$-axis, we can form an $O(n)$-invariant hypersurface by revolving $\Gamma\subset \R^2\times\R^{n-1}$ about the $x$-axis. This hypersurface has only two distinct principal curvatures, $\kappa$, the curvature of $\Gamma$ (multiplicity one), and $\lambda$, the ``rotational'' curvature (multiplicity $n-1$), which is given by
\[\lambda=
\begin{cases}
    \frac{\cos\theta}{y}, & \theta\neq \frac{\pi}{2},\frac{3\pi}{2}\\
    \kappa, & \theta=  \frac{\pi}{2},\frac{3\pi}{2}\,.
\end{cases}\]
Note that $\lambda$ is a smooth function on $S^1=\R/2\pi\Z$ and satisfies
\begin{equation}\label{lambdatheta}
    \kappa\lambda_\theta=-\lambda\tan\theta(\kappa-\lambda)\,.
\end{equation}

\subsubsection{Notation} 

%Returning to the function $\phi=\phi(\kappa_1,\ldots,\kappa_n)$ of the principal curvatures, we remark here that g
Given an $O(n)$-invariant hypersurface with profile curvature $\kappa$ and rotational curvatures $\lambda$, we shall, by an abuse of notation, write $\phi(\kappa,\lambda)$ to mean $ \phi(\kappa,\lambda,\ldots,\lambda)$. On an $O(n)$-invariant flow with profile curvature $\kappa(\theta,t)$ and rotational curvatures $\lambda(\theta,t)$, we also sometimes use $\phi(\theta,t)$ to mean $\phi(\kappa(\theta,t),\lambda(\theta,t))$; the distinction will be clear from context, however. %because when the first argument is a multiple of $\pi/2$, we are talking about $\phi(\theta,t)$. 
By $\phi_t$ and $\phi_\theta$, etc, we mean the $t$ and $\theta$ derivatives, respectively, of this function. By $\phi_\kappa$ and $\phi_\lambda$, etc, we mean the derivatives of the two-variable function $\phi(\kappa,\lambda)$ with respect to its first and second arguments, respectively.


\subsection{$O(n)$-invariance and curvature flows}

We assume in this section that $\phi\in C^\infty(\Gamma_+)$ is an admissible (but not necessarily non-degenerate) speed function.

We start by computing the evolution equations for $\kappa,\lambda,\phi$ and the enclosed area $A$.

\begin{lemma} \label{evoeq}
Along any $O(n)$-invariant solution to the flow \eqref{eq:F},
\begin{align}\label{eq:evolve kappa}
\kappa_t={}& \kappa^2 \phi_\kappa \kappa_{\theta \theta}+ \kappa^2 (\phi_{\kappa\kappa}\kappa_\theta^2 + 2\phi_{\kappa \lambda} \kappa_\theta \lambda_\theta + \phi_{\lambda \lambda} \lambda_\theta^2)\nonumber\\
{}&+\kappa \phi_\lambda (-\kappa_\theta \lambda_\theta-\lambda\tan\theta(\kappa_\theta-2\lambda_\theta))+\kappa(\phi_\kappa\kappa^2 + \phi_\lambda \lambda^2)\,,
\end{align}
\begin{equation}\label{eq:evolve lambda}\lambda_t= \kappa^2 \phi_\kappa \lambda_{\theta\theta}-\lambda(\phi+\phi_\kappa\kappa)\tan\theta \lambda_\theta +\lambda (\phi_\kappa \kappa^2 + \phi_\lambda \lambda^2)\,,
\end{equation}
\begin{equation}\label{eq:evolve phi}\phi_t= \phi_\kappa\kappa^2 \phi_{\theta\theta}-\phi_\lambda\lambda^2 \tan\theta \phi_\theta + \phi(\phi_\kappa \kappa^2 + \phi_\lambda \lambda^2)\,,
\end{equation}
and
\begin{equation}\label{eq:evolve A}
-\frac{dA}{dt}=\int_0^{2\pi} \phi\,d\theta\,.
\end{equation}
\end{lemma}
\begin{proof}
With respect to the Gauss map parametrization, the support function $\sigma(\cdot,t):S^n\to\R$ of the solution evolves as
\[\partial_t \sigma=-\phi\]
and %(since the support function of the profile curve of a convex $O(n)$-invariant hypersurface is the restriction of the support function of the hypersurface to the normal set of the profile curve) 
we can express the principal curvatures $\kappa,\lambda$ %in terms of the support function $\sigma$ of the profile curve using the equations
as
\begin{align*}
    \kappa^{-1}&=\sigma_{\theta\theta}+\sigma\\
    \lambda^{-1}&=\sigma-\sigma_\theta \tan\theta\,.
\end{align*}

Differentiating with respect to $t$ gives
\begin{align*}
   \kappa_t&=\kappa^2(\phi_{\theta\theta}+\phi)\,,\\
   \lambda_t&=\lambda^2(\phi-\phi_\theta \tan\theta).
\end{align*}
Formulae \eqref{eq:evolve kappa}, \eqref{eq:evolve lambda} are obtained now by using the chain rule and relating $\lambda_{\theta\theta}$ to lower order terms by differentiating \eqref{lambdatheta}.

The evolution equation for $\phi$ follows from those for $\kappa$ and $\lambda$ since
\[
\phi_t=\phi_\kappa\kappa_t+\phi_\lambda\lambda_t\,.
\]
%Alternatively, one may apply the well-known formula
%\[
%(\partial_t-\dot\phi^{ij}\nabla_i\nabla_j)\phi=\dot\phi^{ij}\sff^2_{ij}\,,
%\]
%where $\sff$ is the second fundamental form and $\dot\phi^{ij}$ is the derivative of $\phi$ with respect to $\sff_{ij}$.

The identity \eqref{eq:evolve A} is just the usual first variation of enclosed area.
\end{proof}

The following fundamental lemmas apply to all $O(n)$-invariant solutions.

\begin{lemma}\label{lem:KappaOverLambda}
The ratio $u \coloneqq \kappa / \lambda$ evolves under \eqref{eq:F} according to
\begin{align}\label{eq:KappaOverLambda}
u_t-{}&\kappa^2 \phi_\kappa u_{\theta \theta}-2 \phi_\kappa \kappa^2 (\lambda_\theta/\lambda)u_\theta\nonumber\\
{}&= \lambda^{-1}D^2\phi((\kappa_\theta,\lambda_\theta),(\kappa_\theta,\lambda_\theta))-\phi_\lambda \lambda^2 \tan\theta u_\theta -2\phi \tan^2\theta (\kappa-\lambda)\,.
\end{align}
\end{lemma}
\begin{proof}
This follows by direct calculation.
\end{proof}

\begin{lemma} \label{lem:u crit}
At a critical point of $u$, $D^2\phi((\kappa_\theta,\lambda_\theta),(\kappa_\theta,\lambda_\theta))=0$.
\end{lemma}
\begin{proof}
At a critical point of $u$, we have 
\begin{align*}
&u_\theta=0\\ 
\implies &\lambda \kappa_\theta-\kappa \lambda_\theta=0\\
\implies &\kappa_\theta/\lambda_\theta=\kappa/\lambda\,,
\end{align*}
which means that $(\kappa_\theta,\lambda_\theta)$ is a multiple of $(\kappa,\lambda)$. Now since $\phi$ is 1-homogeneous, we have that $D^2\phi((\kappa,\lambda),(\kappa,\lambda))=0$ and the claim follows.
\end{proof}

\begin{corollary}\label{cor:sturm}
The inequality $\kappa\ge\lambda$ is preserved along $O(n)$-invariant solutions to \eqref{eq:F}.
\end{corollary}
\begin{proof}
Given $\varepsilon>0$, consider the function $u^\varepsilon\coloneqq u-1+\varepsilon\mathrm{e}^{t}$, where $u\coloneqq \kappa/\lambda$. By hypothesis, $u^{\varepsilon}(\cdot,0)\ge \varepsilon>0$. We claim that $u_{\varepsilon}$ remains positive for all positive times. Suppose, to the contrary, that there is some $t_0>0$ and some $\theta_0\in S^1$ such that $u^{\varepsilon}(\theta_0,t_0)=0$ but $\min_{S^1}u^{\varepsilon}>0$ for all $t<t_0$. In particular, 
\[
0\ge u^\varepsilon_{t}\,,\;\ 0\ge -u^\varepsilon_{\theta\theta}\,,\;\;\text{and}\;\; 0=u^\varepsilon_{\theta}=u_{\theta}\,,
\]
at $(\theta_0,t_0)$, and hence, by \eqref{eq:KappaOverLambda} and Lemma \ref{lem:u crit},
\begin{align*}
0\ge{}&u^\varepsilon_t-\kappa^2 \phi_\kappa u^\varepsilon_{\theta \theta}-2 \phi_\kappa \kappa^2 (\lambda_\theta/\lambda)u^\varepsilon_\theta\\
={}&\varepsilon\mathrm{e}^{t_0}+2\varepsilon\mathrm{e}^{t_0}\phi\lambda\tan^2\theta\\
>{}&0
\end{align*}
at $(\theta_0,t_0)$, yielding a contradiction. So $u^\varepsilon$ remains positive and hence, taking $\varepsilon\to 0$, we conclude that $u-1$ remains non-negative.
\end{proof}

\begin{corollary} \label{lambdanondec}
    $\lambda(\cdot,t)$ is nondecreasing on $[\frac{\pi}{2},\pi]$ for each $t$.
\end{corollary}
\begin{proof}
    This is a combination of (\ref{lambdatheta}) and Corollary \ref{cor:sturm}
\end{proof}

\begin{corollary}\label{cor:pinching preserved}
The inequality $\kappa/\lambda\le C$ is preserved along $O(n)$-invariant solutions to \eqref{eq:F} which satisfy $\kappa\ge\lambda$.
\end{corollary}
\begin{proof}
Since the inequality $\kappa\ge\lambda$ ensures that the reaction term in \eqref{eq:KappaOverLambda} has the correct sign for preserving upper bounds, the claim follows from the maximum principle in a similar manner to Corollary \ref{cor:sturm}.
\end{proof}

A well-known argument originally due to Chou (formerly Tso) \cite{MR812353} provides an estimate for the speed for as long as the inradius remains bounded from below.
\begin{lemma}\label{lem:Chou}
If the support function $\sigma:S^n\times[0,t_0]\to\R$ of a solution $\{M_t\}_{t\in[0,t_0]}$ to \eqref{eq:F} satisfies
\[
2r\le\sigma\le R
\]
for all $t\in[0,t_0]$, then
\begin{equation}\label{eq:Chou estimate}
\phi\le \frac{R}{r}\max\left\{2\phi_1r^{-1},\max_{M_0}\phi\right\}
\end{equation}
and
\begin{equation}\label{eq:Chou interior estimate}
\phi\le C\left(2r^{-1}+t^{-\frac{1}{2}}\right)\,,
\end{equation}
where $C=C(n,\phi_1,\frac{R}{r})$ and we recall that $\phi_1\coloneqq \phi(1,\dots,1)$.
\end{lemma}
\begin{proof}
If we define the function $\phi_\ast:\Gamma_+\to\R$ by
\[
\phi_\ast(r)\coloneqq -\phi(r^{-1})\,,
\]
then, with respect to the Gauss map parametrization,
\[
\partial_t\sigma=-\phi=\phi_\ast(\rho_1,\dots,\rho_n)\,,
\]
where $\rho_i\coloneqq\kappa_i^{-1}$ are the principal radii. Note that, under the Gauss map parametrization, the principal radii are the eigenvalues of the tensor
\[
A\coloneqq \overline\nabla{}^2\sigma+\sigma \overline g\,,
\]
where $\overline g$ and $\overline\nabla$ are the standard metric and connection on $S^n$.

Consider the function $v\coloneqq\frac{\phi}{\sigma-r}$. If we denote by $\dot\phi^{ij}$ and $\dot\phi^i$ the derivatives of $\phi$ with respect to $\sff_{ij}$ and $\kappa_i$, respectively, and by $\dot\phi_\ast^{ij}$ and $\dot\phi_\ast^i$ the derivatives of $\phi_\ast$ with respect to $A_{ij}$ and $\rho_i$, respectively, then, at a new interior maximum of $v$, we find that
\begin{align*}
0\le{}& (\partial_t-\dot\phi_\ast^{ij}\overline\nabla_i\overline\nabla_j)\frac{\phi}{\sigma-r}\\
={}&\frac{\phi}{\sigma-r}\left(\frac{(\partial_t-\dot\phi_\ast^{ij}\overline\nabla_i\overline\nabla_j)\phi}{\phi}-\frac{(\partial_t-\dot\phi_\ast^{ij}\overline\nabla_i\overline\nabla_j)\sigma}{\sigma-r}\right)+2\dot\phi_\ast^{ij}\overline\nabla_i\frac{\phi}{\sigma-r}\overline\nabla_j\sigma\\
={}&\frac{\phi}{\sigma-r}\left(\tr(\dot\phi_\ast)+\frac{\phi+\dot\phi_\ast(A)-\sigma\tr(\dot\phi_\ast)}{\sigma-r}\right)\\
={}&\frac{\phi}{\sigma-r}\left(\dot\phi^{i}\kappa^2_i+\frac{2\phi-\dot\phi^{i}\kappa^2_i\sigma}{\sigma-r}\right)\,.
\end{align*}
That is,
\begin{align*}
\frac{\dot\phi^{i}\kappa^2_i}{\phi^2}\frac{\phi}{\sigma-r}\le\frac{2r^{-1}}{\sigma-r}\le 2r^{-2}\,.
\end{align*}
Since $\phi$ is inverse-concave, the first claim now follows from \cite[Lemma 5]{MR3070558}. 

To obtain the second claim, consider instead the ratio $\frac{t\phi}{\sigma-r}$.
\end{proof}

%{\color{red} [It would also suffice for $\phi_\lambda$ to be bounded from above. The only other thing we could try would be to look at $\frac{\kappa}{\sigma-r}$, but then we would likely need concavity of $\phi$ to control the $D^2\phi$ term. In any case, I think looking at inverse-concave speeds is fine.]}

\begin{proposition}\label{prop:round point}
Let $\Sigma$ be an $O(n)$-invariant bounded convex hypersurface of $\R^{n+1}$. If $\kappa\ge\lambda$ on $\Sigma$, then (for any admissible speed $\phi$) the solution to \eqref{eq:F} starting from $\Sigma$ contracts to a round point in finite time.
\end{proposition}
\begin{proof}
Since the profile curve $\gamma:S^1\times [0,T)\to\R^2$ of the solution satisfies the equation
\[
\left\langle\gamma_t(\theta,t),\nu(\theta,t)\right\rangle=-f(\theta,t,\kappa(\theta,t))\,,
\]
where
\[
f(\theta,t,k)\coloneqq \phi(k,\lambda(\theta,t))\,,
\]
the support function satisfies an inhomogeneous parabolic equation with smooth coefficients. By Corollary \ref{cor:pinching preserved}, this equation remains uniformly parabolic, so Lemma \ref{lem:Chou} and estimates for inhomogeneous uniformly parabolic equations in one space variable \cite[\S XI.6]{Lieberman} provide \emph{a priori} estimates in $C^{k,\alpha}$ for every $k$, depending only on $k$, $f$, $\max_{S^1\times\{0\}}{\kappa/\lambda}$ and $r_0$, on any time interval $[0,t_0]$ on which $\sigma$ is bounded from below by $r_0$. Moreover, by Corollary \ref{cor:pinching preserved} and Andrews' lemma \cite[Theorem 5.1]{Andrews94}, the ratio of maximum to minimum width of the profile curve remains uniformly bounded throughout the evolution. A standard blow-up argument (see \cite{Andrews94}) in conjunction with \eqref{eq:KappaOverLambda} now implies the claim.
\end{proof}

We note that Proposition \ref{prop:round point} follows from a more general theorem of McCoy--Mofarreh--Wheeler \cite[Theorem 6.1]{MR3338439} (cf. \cite[Theorem 1]{RisaSinestrari}). It may also be seen as a consequence of a result of Andrews and McCoy \cite{MR3528528} (cf. \cite{MR2729317}).

\section{Existence}

We shall construct our ancient pancake solutions by taking the limit of a sequence of old-but-not-ancient solutions. As in \cite{BLT}, suitable old-but-not-ancient solutions are obtained by evolving rotated timeslices of the Angenent oval.

\subsection{The approximating solutions}

Let $\gamma:S^1 \times (-\infty,0) \to \R^2$ be the turning angle parametrization of the Angenent oval and set $\Gamma_t \coloneqq \gamma(S^1,t)$. For each $R<0$, we define $\Sigma^R$  to be the hypersurface obtained by revolving $\Gamma_{-R}$ about the $x$-axis, i.e. 
\begin{equation*}
    \Sigma^R \coloneqq \{x_R(\theta)e_1+y_R(\theta)\varphi: \theta\in S^1, \varphi\in S^{n-1}\subset\{e_1\}^{\perp}\}\,,
\end{equation*}
where $x_R,y_R$ are the coordinate functions for $\Gamma_{-R}$, i.e., $\gamma(\theta,-R)=(x_R(\theta),y_R(\theta))$. Let $F_0^R:S^n\to \R^{n+1}$ be the Gauss map parametrization of $\Sigma_R$.

%Due to a result of Andrews and McCoy \cite{MR3528528} (cf. \cite{MR2729317}), any $O(n)$-invariant convex hypersurface contracts to a point in finite time under the flow \eqref{eq:F} when $\phi$ is one of our admissible speeds. (Note that concavity condition on $\phi$ is required.) 

Consider the maximal $O(n)$-invariant solution $F_R:S^n \times [-T_R,0) \to \R^{n+1}$ to the $\phi$-flow with initial data $F_R(\cdot,0)=F_0^R(\cdot)$. Since $\Sigma^R$ satisfies $\kappa\ge\lambda$ (see the proof of \cite[Lemma 4.1]{BLT}), Corollary \ref{cor:sturm} ensures that this inequality continues to hold for $t>-T_R$. Proposition \ref{prop:round point} then guarantees that the timeslices $\Sigma_t^R\coloneqq F_R(S^n,t)$ shrink to a round point as $t\to 0$.


Define $\Sigma_t^R \coloneqq F_R(S^n,t)$ and $\Gamma_t^R \coloneqq \Sigma_t^R \cap \E^2$. By rotating as necessary, we may assume that the unit tangent vector field $\tau_R$ satisfies $\tau_R(0)=e_1$. Due to to uniqueness of solutions, the reflection symmetry about the two axes of $\E^2$ are preserved under the flow. Consequently, the points $\gamma_R(\pi/2,t),\gamma_R(\pi,t)$ are the unique points of $\Gamma_t^R$ that lie on the positive axes. Their distances from the origin, $h_R(t) \coloneqq x(\gamma_R(\pi/2,t))$ and $\ell_R(t) \coloneqq y(\gamma_R(\pi,t))$, are referred to as the horizontal and vertical displacements of $\Gamma_t^R$ and they play an important role in our analysis.


We prove the existence of an ancient pancake solution by considering the family of flows $\{F_R\}_{R>0}$ and and taking a (subsequential) limit as $R \to \infty$. For this, we need estimates on the displacements $|F_R|$ and the second fundamental form $\sff$ that are uniform for sufficiently large $R$, so that the Arzel\`{a}-Ascoli theorem gives us the existence of a convergent subsequence. 

\subsection{Displacement and curvature estimates}
For estimates on $|F_R|$, we take advantage of the fact that the flow preserves convexity. Thus it will suffice to have estimates on the horizontal and vertical displacements $h_R(t) \coloneqq x(\gamma_R(\pi/2,t))=\langle\gamma_R(\pi/2,t),e_1\rangle$ and $\ell_R(t) \coloneqq y(\gamma_R(\pi,t))=\langle\gamma_R(\pi,t),e_2\rangle$.

%\begin{lemma} \label{sturm}
%$\kappa \geq \lambda$
%\end{lemma}
%\begin{proof}
%This follows from lemmas \ref{lem:KappaOverLambda} and \ref{u crit} via the maximum principle, since the initial data have $\min_{x \in S^1}u(x,0) \geq 1$
%\end{proof}

\begin{lemma}\label{lem:h ell  estimates}
    $\ell_R(t) \geq h_R(t)$
\end{lemma}
\begin{proof}
    Due to Corollary \ref{cor:sturm},
   \begin{align*}
    -\ell'(t)&= \phi(\kappa(\pi,t),\lambda(\pi,t))\\
    &\geq \phi(\lambda(\pi,t),\lambda(\pi,t))\\
    &\geq \phi(\lambda(\pi/2,t),\lambda(\pi/2,t))\\
    &=\phi(\kappa(\pi/2,t),\lambda(\pi/2,t))\\
    &=-h'(t)\,.
\end{align*}
The claim follows by integrating the above inequality from $t$ to $0$, at which time we know that $\ell(0)=h(0)=0$ due to Proposition \ref{prop:round point}. 
\end{proof}

Recall that we define the constant $\phi_1 \coloneqq \phi(1,1)$.

\begin{lemma}\label{lem:hl}
$-\frac{\pi}{2}t \leq h_R(t)\ell_R(t) \leq -\pi\phi_1 t$.
\end{lemma}
\begin{proof}
Due to Corollary \ref{cor:sturm}, we have that
\begin{equation}
\kappa=\phi(\kappa,0)\leq\phi(\kappa,\lambda)\leq\phi(\kappa,\kappa)=\kappa\phi_1\,.
\end{equation}

Integrating \ref{eq:evolve A} and using the above estimate, we get
\begin{equation}
-2\pi t \leq A_R(t) \leq -2\pi\phi_1 t\,.
\end{equation}
Lastly, by convexity of our solution, we may estimate the area from the inside and outside by the area of quadrilaterals through the points of maximal horizontal and vertical displacements to get
\begin{equation}
2h_R(t)\ell_R(t) \leq A_R(t) \leq 4h_R(t)\ell_R(t)\,.
\end{equation}
The desired estimate for $h_R \ell_R$ now follows.
\end{proof}

\begin{lemma}
$\frac{R}{2\phi_1}(1-e^{-R)} \leq T_R \leq (R+\log2)$
\end{lemma}
\begin{proof}
By estimates in (\ref{DispEst}) for Angenent ovals,
\begin{eqnarray}
\frac{\pi}{2}\left(1-e^{-R}\right) \leq h_R(-T_R) \leq \pi/2 \\
R \leq  \ell_R(-T_R) \leq R+ \log2
\end{eqnarray}
Now multiplying these inequalities together and using Lemma \ref{lem:hl} yields the desired inequality.
\end{proof}

\begin{lemma}\label{lem:phi min est}
$\phi^R_{\min}(t)= \phi^R(\tfrac{\pi}{2},t)\leq \frac{2h_R}{h_R^2+\ell_R^2}\phi_1$. %In particular, $-h_R'(t)=\phi^R(\frac{\pi}{2},t)\leq \frac{2\phi_1}{h_R}$
\end{lemma}
\begin{proof}
    A circle $\mathcal{C}$ centered on the $x$-axis and passing through $\gamma_R(\pi/2,t)$ and $\gamma_R(\pi,t)$ has radius $r=\frac{\ell_R^2+h_R^2}{2h_R}$ and center $(-(r-h_R),0)$, and hence any circle with radius $\rho<\frac{\ell_R^2+h_R^2}{2h_R}$ and center $(-(\rho-h_R),0)$ is tangent to $\Gamma_t$ at $\gamma(\pi/2,t)$ and the point $\gamma_R(\pi,t)$ lies to its outside. Due to the curvature of $\Gamma_t$ being minimized at $\gamma_R(\pi/2,t)$, it then follows that  $\kappa_R(\pi/2,t) \leq \frac{2h_R}{\ell_R^2+h_R^2}$ (See \cite[Claim 4.4.1]{BLT}.) This yields
    \[\phi^R\left(\tfrac{\pi}{2},t\right) = \phi(\kappa_R(\pi/2,t),\lambda_R(\pi/2,t))=\phi(\kappa_R(\pi/2,t),\kappa_R(\pi/2,t)) =\phi_1\kappa_R(\pi/2,t) \, ,\]
    from which the claim follows.
\end{proof}

\begin{comment}
\begin{lemma}\label{lem:l>-t}
        $\phi(\theta,t)\geq |\cos\theta|$. In particular, $\phi(\pi,t)\geq 1$. Therefore, $\ell(t)\geq -t$ for all $t>-T_R$. 
\end{lemma}
\begin{proof}
    By the geometry of the Angenent ovals, $\phi(\theta,-T_R)\geq |\cos\theta|$. The inequality continues to hold trivially at the poles $\theta=-\pi/2,\pi/2$. So we only need to show that it continues to hold in $(-\pi/2,\pi/2)$. By reflection symmetry, it will hold on the other half as well. So consider the function $w=\phi/\cos\theta$. Suppose $w$ develops a new interior minimum at some $(\theta_0,t_0)$. At this point, we have
    \[\phi_\theta=-\phi \tan\theta_0.\] Thus we would have
    \[0 \geq w_t-\kappa^2\phi_k w_{\theta\theta}=\phi_\lambda\lambda^2\phi\sec^3\theta_0 > 0.\]
    This leads to a contradiction. The rest of the claim follows by plugging in $\theta=\pi$, and integrating from $t$ to $0$.
\end{proof}
\end{comment}

\begin{lemma}\label{hEst}
$h_R(t) \geq \frac{\pi}{2}(1-e^{-R}) \exp\left(\frac{2\phi_1}{t}\right)$
\end{lemma}
\begin{proof}
Since $h_R\le\frac{\pi}{2}$, Lemmas \ref{lem:hl} and \ref{lem:phi min est} yield%and \ref{lem:l>-t}, 
\[
-h_R'(t)=\phi^R(\tfrac{\pi}{2},t)\leq \frac{2\phi_1h_R}{\ell(t)^2} \leq \frac{2\phi_1h_R}{(-t)^2}\,.
\]
Integrating from $T_R$ to $-t$, we obtain
\[
h_R(t) \geq h_R(T_R)\exp\left(\frac{2\phi_1}{T_R}+\frac{2\phi_1}{t}\right).
\]
The claim now follows from the estimate (\ref{DispEst}).
\end{proof}

\begin{lemma} \label{lEst}
$\ell_R(t)\leq \frac{-2\phi_1t}{1-\mathrm{e}^{-R}}\exp\left(\frac{2\phi_1}{-t}\right)$.
\end{lemma}
\begin{proof}
This follows by combining the estimates of Lemmas \ref{lem:hl} and \ref{hEst}.
\end{proof}

\subsection{The existence argument}

%Combining the lower bound for $h$ and the upper bound for $\ell$ with the interior estimate \eqref{eq:Chou interior estimate} yields a uniform-in-$R$ curvature estimate. Higher order estimates follow from estimates for inhomogeneous uniformly parabolic equations in one space variable as in Proposition \ref{prop:round point} (or, alternatively, from the Krylov--Safanov theory). We can then extract a limit solution along some sequence $R_j\to\infty$ by applying the Arzel\'a--Ascoli theorem.

Combining the above estimates, we may now extract a limit along some sequence $R_j\to-\infty$.


\begin{comment}
Now we seek (locally uniform) curvature estimates. We will start by obtaining an estimate on $\phi$. The following is Corollary 2.2 from \cite{Lynch} that applies to speeds that we consider:
\begin{lemma}\label{lynch}
Fix an admissible speed $\phi \in C^\infty$, Suppose $\{M_t=\partial \Omega_t\}_{t\in[-K^2r^2,0]}$ is a uniformly parabolic convex flow such that the ball $B(0,r)$ is contained in $\Omega_0$. Let $L>1$. Then we have 
\[
\sup_{B(0,Lr)\times[-K^2r^2,0]} \left\{(L^2r^2-|x^2|)(t+K^2r^2)^{1/2}\phi\right\} \leq C(1+K)L^5r^2\,,
\]
where $C$ may depend on $n,\phi,dist(\lambda/\phi,\partial\Gamma)$.
\end{lemma}

\color{red}could use Harnack instead \color{black}

\begin{lemma}
For sufficiently large $R$, there exists $r$ such that each $\Sigma^R$ lies outside $B(0,r)$
\end{lemma} 
\begin{proof}
Fix $R_0>0$. Firstly, due to convexity and the fact that $h(t)\leq \ell(t)$, $|\gamma(\theta,t)| \geq h(t)/\sqrt{2}>h(t)/2$. Now for $t\leq-R_0$, by (\ref{hEst}),
\[
h_R(t) \geq \frac{\pi}{2}(1-e^{-R})\exp{\frac{\phi_1}{-2t}} \geq \frac{\pi}{4}\exp{\frac{\phi_1}{-2t}} \geq \frac{\pi}{4}\exp{\frac{\phi_1}{2R_0}}.\qedhere\]
\end{proof}
This gives us locally uniform bounds for $\phi$ on compact subsets of spacetime of the form $\overline B(0,Lr) \times [-T,-R_0]$


{\color{red}ML: In order to get higher order estimates without uniform parabolicity, note that 
\[
1=\phi(1,0)=\phi_\kappa(1,0)\,,
\]
which implies that $\phi_\kappa$ is uniformly bounded (since it is bounded on the compact set $\{(1,s):s\in[0,1]\}$, $\lambda/\kappa\in(0,1]$, and $\phi_\kappa(\kappa,\lambda)=\phi_\kappa(1,\lambda/\kappa)$). So higher order estimates follow from estimates for inhomogeneous uniformly parabolic equations in one space variable as in Proposition \ref{prop:round point}.}
\end{comment}

\begin{theorem}
Given any non-degenerate admissible speed $\phi$, there is a sequence of approximating solutions which converges locally uniformly in the smooth topology to a smooth, compact, convex, locally uniformly convex, $O(1) \times O(n)$-invariant ancient solution to the $\phi$-flow which lies in the slab $\Omega \coloneqq \{(x,y,z) \in \R \times \R \times \R^{n-1}: |x| < \frac{\pi}{2}\}$ and in no smaller slab.
\end{theorem}
\begin{proof}
We need uniform-in-$R$ estimates for $F_R$ and its derivatives on compact subsets of time. A bound for $\vert F_R\vert$ follows from Lemma \ref{lEst}. A bound for the curvature follows from Lemma \ref{lem:Chou} due to the inradius bound (which is a consequence of Lemmas \ref{lem:h ell  estimates} and \ref{hEst} and convexity) and the circumradius bound (which is a consequence of Lemma \ref{lEst}). Since $\phi_\kappa$ is bounded uniformly from above and below, higher order estimates then follow from estimates for inhomogeneous uniformly parabolic equations in one space variable as in Proposition \ref{prop:round point}. Well-known arguments then provide uniform-in-$R$ estimates for $F_R$ in $C^k$ for all $k$, and also a lower bound for $DF_R$. 
\begin{comment}
    {\color{blue} Let $B_{2r_0}(0)$ be a ball centered at the origin and contained in $\Omega_{t_0}$. Let $\sigma:\R/2\pi\Z\times [-T_R,t_0]\to\R$ be the support function for the profile curve, defined by
\[
\sigma(\theta,t)\coloneqq \sup_{\theta\in S^1}\left\langle \gamma(\theta,t),\nu\right\rangle\,,
\]
where $\nu=(-\cos\theta,\sin\theta)$ is the normal vector at $\gamma(\theta,t)$.

It is well-known that $\partial_t \sigma = -\phi(\kappa,\lambda).$ Now, in terms of the support function, one can express the curvatures as 
\begin{align}
    \kappa &=(\sigma_{\theta\theta}+\sigma)^{-1}\\
    \lambda &= \begin{cases}
    (\sigma+\sigma_\theta \tan\theta)^{-1},  &\theta \notin \{\pi/2,3\pi/2\}\\
    \kappa,  &\theta \in\{\pi/2,3\pi/2\}
   \end{cases}
\end{align}
We have already seen that $\lambda$ is smooth.

So the evolution equation for the support function is
\[
\partial_t\sigma=-\phi\left((\sigma_{\theta\theta}+\sigma)^{-1},\lambda(\theta,t)\right)\,.
\]
To see that this is a uniformly parabolic equation, define 
\[
f(\theta,t,z,p,r)=-\phi(k(z,r),\lambda(\theta,t))\,,
\]
where $k(z,r)\coloneqq (r+z)^{-1}$. It follows that
\[\partial_r f=k^2(z,r)\phi_\kappa(k(z,r),\lambda(\theta,t))\,.\]
Due Euler's theorem for homogeneous functions and the normalization $\phi(1,0)=1$, we have $\phi_\kappa(1,0)=1$. Also, from (***), $\lambda/\kappa \in [0,1]$ regardless of the parameter $R$. Combining these facts with the parabolicity of $\phi$ inside the positive cone, we obtain uniform parabolicity.

We claim that $\sigma$ is bounded in $C^2$ uniformly in $R$ for sufficiently large $R$.} The $C^0$-estimate comes from Lemma \ref{lEst}. Since $\kappa\geq \lambda$, we have $\vert{\sff}\vert^2 \leq n \kappa^2 \leq n \phi^2$, from which we get $C^2$-estimates.
    
For higher order asymptotics, we will show that the flow of the profile curve remains uniformly parabolic (in spite of $\phi$ itself not being so), and then Krylov-Safonov theory gives us $C^{k+2,\alpha}$-regularity provided $\phi$ is $C^{k,\alpha}$-regular. Higher derivative estimates for $\sff$ come from standard bootstrapping arguments. We do this as follows. Note that in the evolution equations (***), the coefficient of the second spatial derivative is $\kappa^2 \phi_\kappa(\kappa,\lambda)$ which equals $\kappa^2 \phi_\kappa(1,\lambda/\kappa)$ due to $1$-homogeneity. So higher order estimates follow from estimates for inhomogeneous uniformly parabolic equations in one space variable as in Proposition \ref{prop:round point}.
\end{comment}
    
Thus, by the Arzel\`{a}--Ascoli theorem, there exists a sequence of flows $F_{R_j}$, with $R_j \to -\infty$, converging locally uniformly in the smooth topology to a smooth limit flow. The limit is certainly weakly convex since this is the case for the approximating solutions. Moreover, the upper bound for $\ell_R$ ensures that the limit is compact (and in particular does not degenerate into two parallel hyperplanes). Strict convexity then follows from applying the strong maximum principle to \eqref{eq:evolve kappa} (under slightly stronger conditions on the speed $\phi$, we could also apply \cite[Proposition A.2]{Lynch}). The limit solution inherits rotational and reflection symmetries from the approximating solutions. Convexity and the lower bound for $h_R$ ensure that the solution lies in no smaller slab.
\end{proof}

\section{Unique asymptotics and reflection symmetry}

We wish to show in Section \ref{sec:uniqueness} that the solution constructed above is unique in the class of $O(n)$-invariant convex ancient solutions in the slab $[-\frac{\pi}{2},\frac{\pi}{2}]\times \R^n$. We will use Alexandrov's moving plane method to show this. But to be able to do this, we need to know the asymptotics of any such solution; that is the subject of this section. 

Consider any convex ancient solution $F:S^n \times (-\infty,0) \to \R^{n+1}$ of $\phi$-flow that is $O(n)$-invariant with respect to some $e_1 \in S^n$ and lies in the slab $\Omega \coloneqq \{(x,y,z) \in\R\times\R\times\R^{n-1}:|x|< \pi/2\}$ and in no smaller slab. Define $\Sigma_t=\Sigma_t^n \coloneqq F(S^n,t)$. Given some unit vector $e_2 \perp e_1 $, define the plane $\E^2\coloneqq \spa\{e_1,e_2\}$, and parametrize the profile curve $\Gamma_t \coloneqq \E^2 \cap \Sigma_t$ with respect to turning angle by a curve $\gamma:S^1 \times (-\infty,0) \to \E^2$. Denote, as usual, the unit tangent and unit normal by $\tau, \nu$ respectively and the curvature and rotational curvature by $\kappa(\cdot,t), \lambda(\cdot,t)$  respectively. We continue to use $\phi(\cdot,t) \coloneqq \phi(\kappa(\cdot,t),\lambda(\cdot,t))$.

Finally, for each $t<0$ we define the vertical dispacement to be 
\[
\ell(t) \coloneqq \max_{\theta\in S^1} |\langle \gamma(\theta,t),e_2 \rangle|.
\]
Due to convexity and rotational symmetry,
\[
\ell(t)=\langle \gamma(\pi,t),e_2 \rangle=-\langle \gamma(0,t),e_2 \rangle.
\]
Note that we do not know \emph{a priori} that the curve is reflection-symmetric about the $y$-axis.

\subsection{Unique asymptotics}

The following two lemmas describe the shape of arbitrary solutions $\Sigma_t$ as $t\to\infty$. They are analogous \cite[Lemmas 5.1 and 5.2]{BLT}. The proofs are also identical to theirs, since they only rely on convexity, the strong maximum principle and the Harnack inequality, all of which hold in our case.

\begin{lemma} \label{parabolic}
For any sequence of times $t_i \to -\infty$ the sequence of flows $\Sigma_t^i \coloneqq \Sigma_{t+t_i}$ defined on $(-\infty,-t_i)$ converge locally uniformly in the smooth topology to the stationary $\phi$-flow defined by $\partial \Omega$.
\end{lemma}


Now recall that we have defined a normalization on our speed such that $\phi(1,0)=1$. Define $\alpha$ by
\begin{equation}
    \alpha^{-1}\coloneqq \lim_{t \to -\infty
}\phi(\pi,t)
\end{equation}
This limit is well-defined as $\phi$ is nondecreasing in $t$ due to the differential Harnack inequality \cite{AndrewsHarnack}. Let $P$ be the inverse of the Gauss map, i.e. for any unit vector $v$, $P(v,t)$ is the point on $\Sigma_t$ with normal $v$.

\begin{lemma} \label{tip}
For any sequence of times $t_i \to -\infty$ and any unit vector $v \perp e_1$ the sequence of flows $\Sigma_t^i \coloneqq \Sigma_{t+t_i}-P(v,t_i)$ converges locally uniformly in the smooth topology to the scaled Grim hyperplane $\alpha G^n_{\alpha^{-2}t}$ where $G^n_t$ defined by
\begin{equation}
    G^n_t \coloneqq \{\theta e_1 +(t-\log \cos \theta)v\}, \, t \in (-\infty,\infty)
\end{equation}
is the standard Grim hyperplane that translates in direction $v$ with unit speed.
\end{lemma}

Due to containment of the flow in the strip, it is clear that $\alpha \leq 1$. Now we show that $\alpha=1$. We obtain this from purely geometrical considerations by showing that if $\alpha<1$,  then the area $A(t)$ contained inside $\Gamma_t$ decreases too quickly (cf. \cite{MR4127403}).

\begin{lemma}\label{lemma:lEstArbit}
$\ell(t) \geq \alpha^{-1}t$.
\end{lemma}
\begin{proof}
It is a simple consequence of the fact that $-\ell'(t)=\phi(\pi,t)\geq \alpha^{-1}$ due to the Harnack inequality.
\end{proof}

\begin{lemma}\label{lambdaleqkappa}
$\lambda(\cdot,t) \leq \kappa(\cdot,t)$.
\end{lemma}
\begin{proof}
By rotational symmetry, $\lambda( \pm \pi/2,t)=\kappa( \pm \pi/2,t)$ and by Lemma \ref{tip}, 
\[\displaystyle \lim_{t\to -\infty} \frac{\kappa}{\lambda}(\theta,t)=\infty \]
for $\theta \in (-\frac{\pi}{2},\pi/2)$. The claim now follows from Lemmas \ref{lem:KappaOverLambda} and \ref{cor:sturm}  and the symmetry of $\gamma$.
\end{proof}

\begin{lemma}\label{lem:crudeareadecay}
$\displaystyle \lim_{t\to -\infty}A'(t)=-2\pi$.
\end{lemma}
\begin{proof}
We have
\begin{align*}
    |A'+2\pi|&=\left\vert-\int_{\Gamma_t}\phi\,ds+\int_{\Gamma_t}\kappa\,ds\right\vert\\
    &= \left\vert\int_{S^1}(\phi(\kappa,\lambda)-\kappa) \frac{d\theta}{\kappa}\right\vert\\
    &=\left\vert\int_{S^1}\left(\phi(1,\lambda/\kappa)-1\right) d\theta\right\vert
\end{align*}
By Lemma \ref{lambdaleqkappa}, $0\leq\phi(1,\lambda/\kappa)-1 \leq \phi(1,1)-1$ and $\lim_{t\to-\infty}\lambda/\kappa=0$ a.e. on $S^1$, so that the integrand tends to zero a.e. and the claim follows by Lebesgue dominated convergence.
\end{proof}

\begin{corollary}
    The width of the limiting Grim hyperplane is maximal, i.e. $\alpha=1$.
\end{corollary}
\begin{proof}
We claim that for any $\varepsilon>0$, there exists $t_\varepsilon<0$ such that for any $t<t_\varepsilon$,
\[
A(t)\leq -(2\pi +\varepsilon)t
\]
Indeed, by Lemma \ref{lem:crudeareadecay}, given any $\varepsilon>0$ there exists $t_\varepsilon'<0$ such that for all $t<t_\varepsilon'$
\begin{align*}
A(t)\le{}&A(t_\varepsilon)+(2\pi+\tfrac{\varepsilon}{2})(t_\varepsilon'-t)\\
={}&-(2\pi +\tfrac{\varepsilon}{2})t+\frac{A(t_\varepsilon')+(2\pi+\tfrac{\varepsilon}{2})t_\varepsilon'}{t}t\,.%\\
\end{align*}
so that we may set $t_\varepsilon\doteqdot \min\{t_\varepsilon',-\frac{2}{\varepsilon}(A(t_\varepsilon')+(2\pi+\tfrac{\varepsilon}{2})t_\varepsilon')\}$.

We bound the area from below by an enclosed trapezoid that we describe below. Let $C(t),D(t)\in \Gamma_t$ such that $y(C(t))=y(D(t))=0$. Without loss of generality, assume that $x(C(t))>x(D(t))$. Let $p(t)=\gamma(0,t)$ be the ``tip" of $\Gamma_t$. Now for any $\delta\in (0,1)$ we can find $t_\delta < 0 $ such that for all $t<t_\delta$, 
\[\pi-\delta \leq x(C(t))-x(D(t)) \leq \pi.\]
Since the tip region converges locally uniformly to the scaled Grim Reaper $\alpha G$, one can, (by taking $t_\delta$ more negative if need be), also find points $p^{\pm}(t)$ on $\Gamma_t$ and a constant $C_\delta$ such that
\[
y(p^-(t))=y(p^+(t))\,,
\]
\[
\pi\alpha-\delta \leq x(p^+(t))-x(p^-(t)) \leq \pi \alpha\,,
\]
and
\[
0 \leq y(p(t))-y(p^\pm(t))=\ell(t)-y(p^\pm(t)) \leq C_\delta\,.
\]
Now since the trapezoid formed by $C(t),D(t),p^\pm(t)$ is contained inside $\Gamma_t$ (by convexity), we estimate $\ell(t)$ using Lemma \ref{lemma:lEstArbit} to get 
\[(\pi\alpha-\delta+\pi-\delta)(\alpha^{-1}t-C_\delta) \leq A(t).\]
Combining these upper and lower bounds, one sees that for $t< \min\{t_\varepsilon,t_\delta
\}$, 
\[
\left[\pi \left(\alpha^{-1}-1\right)-2\delta \alpha^{-1}-\varepsilon \right](-t) \leq \pi(1+\alpha)C_\delta+2\delta C_\delta\]
By choosing $\varepsilon,\delta
$ small, one can make $\left[\pi \left(\alpha^{-1}-1\right)-2\delta\alpha^{-1}-\varepsilon \right]>0$, and allowing $t\to-\infty$ yields a contradiction unless $\alpha=1$. Thus we conclude that $\alpha=1$.
\end{proof}

\subsection{Reflection symmetry}
We now exploit the maximality of the width of the limiting Grim hyperplane to deduce the following reflection symmetry.
\begin{theorem}\label{reflection}
Let $\{\Sigma_t\}_{t\leq 0}$ be a convex, ancient $O(n)$-symmetric solution to $\phi$-flow that lies in the slab $\{|x_1|\leq \pi/2\}$ and in no smaller slab. Then it is necessarily reflection symmetric about the hyperplane $\{x_1=0\}$. 
\end{theorem}
\begin{proof}
The argument is a standard application of the Alexandrov reflection principle, which holds for parabolic flows, and hence the same as \cite[Theorem 6.2]{BLT}. 
\end{proof}


\section{Area and displacement estimates} \label{sec:estimates}
We continue to study arbitrary convex $O(n)$-invariant ancient solutions lying in a slab. The aim here is to provide improved estimates for the enclosed area $A(t)=\int_{-t}^{0}\int_{\Gamma_\tau}\phi\,ds\,d\tau$ and the vertical displacement $\ell(t)$ for such solutions.

Since $\phi$ is non-degenerate, we may write
\begin{align*}              
    \phi(\kappa,\lambda)&=\kappa\phi(1,\lambda/\kappa)\\     &=\kappa\left(\phi(1,0)+\phi_\lambda(1,0)(\lambda/\kappa)+\phi_{\lambda\lambda}(1,\xi)(\lambda/\kappa)^2\right)\\
    &=\kappa\phi(1,0)+\lambda \dot{\phi}_1+\phi_{\lambda\lambda}(1,\xi)\lambda^2/\kappa\\
    &\leq \kappa+\lambda \dot{\phi}_1+C\lambda^2/\kappa\,,
\end{align*}
and hence
\begin{equation}\label{eq:phiapprox}  
    \kappa+\lambda \dot{\phi}_1-C\lambda^2/\kappa \leq \phi(\kappa,\lambda) \leq \kappa+\lambda \dot{\phi}_1+C\lambda^2/\kappa\
\end{equation}
where $0\leq\xi\leq\lambda/\kappa$ is given by the mean value theorem, $\dot{\phi}_1 \coloneqq \phi_\lambda(0,1)$ and $C\coloneqq \sup_{\xi\in[0,1]}\vert\phi_{\lambda\lambda}(1,\xi)\vert$. Thus the area estimate will reduce to estimates of $\int_{\Gamma_t}\lambda ds$ and $\int_{\Gamma_t}(\lambda^2/\kappa) ds$.

Due to the reflection symmetry of such solutions (Theorem \ref{reflection}), we have that the horizontal displacement
\[h(t) \coloneqq \max_{\theta \in S^1} \langle \gamma(\theta,t),e_1 \rangle\]
satisfies
\[h(t)=\langle\gamma(\pi/2,t),e_1\rangle=-\langle\gamma(-\pi/2,t),e_1\rangle\,.\]

Moreover, due to Lemma \ref{lambdaleqkappa}, we have
\[\kappa(\theta,t)\geq\lambda(\theta,t)\geq\lambda(\pi/2,t)=\kappa(\pi/2,t)\,,\]
which then implies that the minimum value of $\phi$ occurs at the poles, i.e.
\begin{equation}
    \min_{\theta \in S^1} \phi(\theta,t) =\phi(\pm \pi/2,t)\,.
\end{equation}

We obtain the desired estimates by using a graphical representation for the solution. The part of the curve $\Gamma_t$ with $y\geq0$ can be written as a graph $(x,u(x,t))$ with $x\in[-h(t),h(t)]$. In this representation,
\begin{equation}
    \frac{1}{4}\int_{\Gamma_t}\lambda ds=\int_0^{h(t)}\frac{1}{u(x,t)}dx
\end{equation}
and
\begin{equation}\label{lambda^2}
    \frac{1}{4}\int_{\Gamma_t}\frac{\lambda^2}{\kappa} ds=\int_0^{h(t)}\frac{\lambda^2}{\kappa}\frac{dx}{|{\cos\theta}|}=\int_0^{h(t)}\frac{|{\cos\theta}|}{\kappa u(x,t)^2}dx
\end{equation}
coming from the fact that $ds=\frac{dx}{\vert{\cos\theta}\vert}$ and $\lambda= \frac{\vert{\cos\theta}\vert}{u}$.

Moreover, the function $u$ satisfies the graphical $\phi$-flow equation
\[\frac{du}{dt}=-\phi\sqrt{1+u_x^2}.\]
Since $\phi$ is nondecreasing due to the Harnack inequality, and since the solution converges to the Grim reaper, we have the basic estimate
\[-\frac{du}{dt}\geq\phi(1,0)=1.\]
Now, given $x\in[0,\frac{\pi}{2})$, we may denote by $T(x)$ the time when the profile curve passes through the point $(x,0)$, i.e. $h(T(x))=x$, and hence $u(x,T(x))=0$. Now, integrating the previous inequality from $T$ to $t$, we get for all $x\in[0,h(t))$ that
\[u(x,t)\geq -t+T(x)\,.\]

Observe that
\[\kappa(\pi/2,t) \leq 2\frac{h(t)-x}{u(x,t)^2}\,,\]
which when combined with $\phi(\kappa,\lambda)\leq \phi(\kappa,\kappa)=\phi(1,1)\kappa=\phi_1\kappa$ yields
\begin{equation}
    \phi_{\min}(t) \leq 2\phi_1 \frac{h(t)-x}{u(x,t)^2}\,.
\end{equation}
Putting $x=0$, we get
\begin{equation}
    -\frac{dh}{dt}=\phi_{\min} \leq \frac{2\phi_1 h}{(-t)^2}\,.
\end{equation}
which, upon integration, gives
\begin{equation}
    h(t) \geq \frac{\pi}{2}e^{\frac{2\phi_1}{t}} \geq \frac{\pi}{2}\left(1-\frac{2\phi_1}{-t}\right).
\end{equation}
Setting $t=T(x)$ now gives
\begin{equation}
    u(x,t)\geq -t+T(x) \geq -t-\frac{\phi_1\pi}{\frac{\pi}{2}-x}\,.
\end{equation}
Thus we have the following estimate:
\begin{lemma}\label{lem:uhphiasymp}
For each $k\in\mathbb{N}$ there exists a constant $c_k$ such that for all $t<0$ and all $x\in[0,h(t)]$,
\begin{enumerate}[(i)]
    \item $\phi_{\min}(t) \leq \frac{\phi_1 \pi c_k}{(-t)^{k+1}}$ 
    \item $h(t) \leq \frac{\pi}{2}\left(1-\frac{2\phi_1 c_k}{(-t)^{k}}\right)$
    \item $u(x,t) \geq-t- \left(\frac{\phi_1 \pi c_k}{\frac{\pi}{2}-x}\right)^{1/k}$
\end{enumerate}
\end{lemma}
\begin{proof}
In view of the above estimates, we we may proceed by induction, as in \cite[Lemma 7.1]{BLT}.
\end{proof}

This allows us to estimate the integral that we wish to estimate:

\begin{lemma}\label{lem:lambdaintegrals}
    For any $\varepsilon>0$,
    \begin{enumerate}[(i)]
        \item
        $\int_{\Gamma_t}\lambda ds \leq \frac{2\pi}{-t}+o\left(\frac{1}{(-t)^{2-\varepsilon}}\right)$
        \item $\int_{\Gamma_t}\frac{\lambda^2}{\kappa} ds \leq O\left(\frac{1}{(-t)^{2}}\right)$
    \end{enumerate}
\end{lemma}
\begin{proof}
The first estimate may be proved as in \cite[Claim 7.2.1]{BLT}. The second estimate is not required in \cite{BLT} because the mean curvature $H=\kappa+(n-1)\lambda$ is linear in $\kappa$ and $\lambda$. Nonetheless, a similar idea works. %; we will only remark on the differences necessary to make the argument in \cite[Claim 7.2.1]{BLT} work). 
Indeed, following \cite[Claim 7.2.1]{BLT}, we define $c(x,t)=\sqrt{\rho(t)^2-(x-(h(t)-\rho(t)))^2},$ where $\rho(t)=\frac{(-t)^2}{\pi}$, and set $\underline{x}(t)=\inf\{x\in[\pi/4,h(t)]:u(x,t)=c(x,t)\}$, and split the integral to be estimated as
\[\frac{1}{4}\int_{\Gamma_t}\frac{\lambda^2}{\kappa} ds=\int_0^{h(t)}\frac{\lambda^2}{\kappa}\frac{dx}{\vert{\cos\theta}\vert}=\int_0^{\underline x(t)}\frac{\lambda^2}{\kappa}\frac{dx}{\vert{\cos\theta}\vert}+\int_{\underline x(t)}^{h(t)}\frac{\lambda^2}{\kappa}\frac{dx}{\vert{\cos\theta}\vert}.\]
Since $\lambda/\kappa \leq 1$,
\begin{equation}\label{lamb^2pt1}
    \int_{\underline x(t)}^{h(t)}\frac{\lambda^2}{\kappa}\frac{dx}{\vert{\cos\theta}\vert}\leq \int_{\underline x(t)}^{h(t)}\frac{\lambda\,dx}{\vert{\cos\theta}\vert} =o(t^{-2})
\end{equation}
just as in \cite[Claim 7.2.1]{BLT}. The remaining term is estimated as follows. Let $k\in\mathbb{N}$ and choose $x_0(t)=\frac{\pi}{2}-\frac{\pi nc_k}{(-t)^k}$. If $\underline x(t)\leq x_0(t)$, then
\begin{equation*}
     \int_0^{\underline x(t)}\frac{\lambda^2}{\kappa}\frac{dx}{\vert{\cos\theta}\vert} \leq \int_0^{x_0(t)}\frac{\lambda^2}{\kappa}\frac{dx}{\vert{\cos\theta}\vert}\leq \int_0^{x_0(t)}\frac{1}{u(x,t)^2}dx
\end{equation*}
and if not,
\begin{align*}
    \int_0^{\underline x(t)}\frac{\lambda^2}{\kappa}\frac{dx}{\vert{\cos\theta}\vert} &\leq \int_0^{x_0(t)}\frac{\lambda^2}{\kappa}\frac{dx}{\vert{\cos\theta}\vert}+\int_{x_0(t)}^{\underline x(t)}\frac{\lambda^2}{\kappa}\frac{dx}{\vert{\cos\theta}\vert}\\
    &\leq \int_0^{x_0(t)}\frac{\lambda^2}{\kappa}\frac{dx}{\vert{\cos\theta}\vert}+\int_{x_0(t)}^{\underline x(t)}\lambda\frac{dx}{\vert{\cos\theta}\vert}\\
    &\leq \int_0^{x_0(t)}\frac{1}{u(x,t)^2}dx +\int_{x_0(t)}^{h(t)}\frac{dx}{c(x,t)}\,.
\end{align*}
Either way, Lemma \ref{lem:uhphiasymp} (cf. \cite[Claim 7.2.1]{BLT}) yields
\begin{align}\label{lamb^2pt2}
    \int_0^{\underline x(t)}\frac{\lambda^2}{\kappa}\frac{dx}{\vert{\cos\theta}\vert} 
    &\leq \frac{x_0(t)}{u(x_0(t),t)^2}+\int_{x_0(t)}^{h(t)}\frac{dx}{c(x,t)}\nonumber\\
    &\leq \frac{\pi}{2}\left(\frac{1}{-t}+o((-t)^{-1})\right)^2+o((-t)^{-2})\nonumber\\
    &\leq O((-t)^{-2})\,.
\end{align}
Putting equations (\ref{lamb^2pt1}) and (\ref{lamb^2pt2}) together yields
\[\frac{1}{4}\int_{\Gamma_t}\frac{\lambda^2}{\kappa} ds \leq O((-t)^{-2})
\]
as claimed.
\end{proof}

This leads us to the following area estimate:
\begin{corollary}
    \begin{equation}
        -t+\dot{\phi}_{1}\log(-t)-C\leq \frac{A(t)}{2\pi} \leq -t+\dot{\phi}_{1}\log(-t) +C
    \end{equation} 
\end{corollary}

\begin{proof}
The upper bound follows from integrating $A'$ in (\ref{eq:evolve A}), estimating $\phi$ using the upper bound in equation (\ref{eq:phiapprox}) and using Lemma \ref{lem:lambdaintegrals}. The lower bound will be inferred from the upper bound via an application of H\"{o}lder's inequality. Indeed,
\[h^2(t)\leq \int_0^{h(t)}u(x,t)dx \int_0^{h(t)}u(x,t)^{-1} dx=\frac{A(t)}{16}\int_{\Gamma_t}\lambda ds\,,\]
so that Lemma \ref{lem:uhphiasymp} gives
\[\int_{\Gamma_t}\lambda ds \geq \frac{2\pi\left(1-\frac{2\phi_1c_1}{-t}\right)^2}{-t+\dot{\phi}_{1}\log(-t) +C} \geq \frac{2\pi}{-t}-o\left(\frac{1}{(-t)^{2-\epsilon}}\right).\]
Thus, using the lower bound in \eqref{eq:phiapprox} gives us the desired lower bound for area.
\end{proof}   

Now the following refinement to the displacement estimate is obtained:
\begin{lemma}\label{lem:ooft^-eps}
    For any $\varepsilon\in (0,1)$,
    \begin{equation}
        \ell(t) \leq -t + o((-t)^{\varepsilon}).
    \end{equation}
\end{lemma}
\begin{proof}
    This follows from the same area estimation trick of \cite[Lemma 7.3]{BLT}.
\end{proof}

%{\color{red}Is the precise statement of the above area estimates (with $C=0$) a red herring? We just need $A(t)\sim -2\pi t\pm o((-t)^\varepsilon)$, right?}

\begin{corollary}\label{cor:improvedHestimate} For any $\varepsilon\in(0,1)$
\[
\phi(\theta, t)\ge \vert{\cos\theta}\vert\left(1+\frac{\dot{\phi}_1}{-t}-o\left(\frac{1}{(-t)^{2-\varepsilon}}\right)\right)\;\;\text{as}\;\; t\to-\infty\;\;\text{for all}\;\; \theta \in S^1\,,
\]
where $\dot\phi_1$ is defined by \eqref{eq:phiapprox}.
\end{corollary}
\begin{proof} By symmetry, it suffices to prove the claim for $\theta\in (-\pi/2,\pi/2)$ (note that for $\theta=\pm\pi/2$ the claim holds trivially). Consider the function $w:(-\frac{\pi}{2},\frac{\pi}{2})\times(-t,0)\to\R$ defined by $w(\theta,t):=f(t)\cos\theta$, where the function $f:(-\infty,0)\to\R_+$ will be determined momentarily. Observe that
\[
w_t=\phi_\kappa\kappa^2w_{\theta\theta}+|\mathrm{II}|_\phi^2w+\left(\frac{f'}{f}-\phi_\lambda\lambda^2\right)w\,,
\]
where $\vert \mathrm{II}\vert_\phi^2:=\phi_\kappa\kappa^2+\phi_\lambda\lambda^2$.
Recalling Lemma \eqref{evoeq}, we compute
\begin{align*}
\frac{w}{\phi}\left(\partial_t-\phi_\kappa\kappa^2\partial^2 _{\theta}\right)\frac{\phi}{w}={}&2\phi_\kappa\kappa^2\frac{w}{\phi}\left(\frac{\phi}{w}\right)_\theta\frac{w_\theta}{w}-\phi_\lambda\lambda^2\tan\theta \frac{\phi_\theta}{\phi}-\left(\frac{f'}{f}-\phi_\lambda\lambda^2\right)\,.
\end{align*}
Rewriting
\[
\frac{\phi_\theta}{\phi}=\frac{w}{\phi}\left(\frac{\phi}{w}\right)_\theta-\frac{w_\theta}{w}\quad\text{and}\quad
\frac{w_\theta}{w}=-\tan\theta
\]
we obtain
\begin{align}\label{eq:evolveH/w}
\frac{w}{\phi}\left(\partial_t-\phi_\kappa\kappa^2\partial^2 _{\theta}\right)\frac{\phi}{w}+{}&\tan\theta\frac{w}{\phi}\left(\frac{\phi}{w}\right)_\theta\left(\phi_\lambda\lambda^2+2\phi_\kappa\kappa^2\right)\nonumber\\
={}&\phi_\lambda\sec^2\theta\lambda^2-\frac{f'}{f}\nonumber\\
={}&\frac{\phi_\lambda}{y^2}-\frac{f'}{f}\nonumber\\
\ge{}&\frac{\dot{\phi}_1}{y^2}-C\frac{\lambda}{\kappa y^2}-\frac{f'}{f}\nonumber\\
\ge{}&\frac{\phi_\lambda}{\ell^2}-\frac{C}{\ell^3}-\frac{f'}{f}
\end{align}
since $\kappa\ge C\phi\ge \cos\theta$.

Now fix any $\varepsilon\in(0,1)$. By Lemma \ref{lem:ooft^-eps}, there is some $C_\varepsilon<\infty$ such that
\[
\frac{1}{\ell(t)^2}\geq\frac{1}{(-t)^2}-\frac{C_\varepsilon}{(-t)^{3-\varepsilon}}
\]
for all $t\in(-\infty,-1]$, say. Thus, if we set
\[
f(t):=\exp\left(\left[\frac{\dot\phi_1}{-t}-\frac{\Lambda}{(2-\varepsilon)(-t)^{2-\varepsilon}}\right]\right)\,,
\]
then we may choose $\Lambda$ so large that
\[
\frac{f'(t)}{f(t)}=\left[\frac{\dot\phi_1}{(-t)^2}-\frac{\Lambda}{(-t)^{3-\varepsilon}}\right]\leq \frac{\dot\phi_1}{\ell(t)^2}-\frac{C}{\ell^3}
\]
for all $t<-1$. So the maximum principle yields
\[
\min_{S^1\times\{t\}}\frac{\phi}{w}\geq\min_{S^1\times\{t_0\}}\frac{\phi}{w}\quad\text{for all}\quad -1>t>t_0\,.
\]
But the right hand side approaches 1 as $t\to-\infty$. The claim follows by estimating $\exp(\zeta)\geq 1+\zeta$. %[If $\phi$ is convex, then $\dot\phi_\lambda(\kappa,\lambda)=\dot\phi_\lambda(1,\lambda/\kappa)\ge \dot{\phi}_1$, so we may take $\gamma=\dot{\phi}_1$ in this case.]
\end{proof}

Integrating the lower speed bound yields a displacement estimate.
\begin{lemma}\label{lem:ellasymptotics}
The limit
\[
C:=\lim_{t\to-\infty}(\ell(t)+t-\dot{\phi}_1\log(-t))
\]
exists (in the extended real line $\R\cup\{\infty\}$).
\end{lemma}
\begin{proof}
Given any $\varepsilon\in(0,1)$ set $f(t):=\frac{C}{1-\varepsilon}\frac{1}{(-t)^{1-\varepsilon}}$ for some $C\in\R$. By Corollary \ref{cor:improvedHestimate}, we can choose $C$ so that
\[
\frac{d}{dt}(\ell+t-\dot{\phi}_1\log(-t)-f)\leq 0\,.
\]
The claim follows because $\lim_{t\to-\infty}f=0$.
\end{proof}

\section{Uniqueness} \label{sec:uniqueness}
In this section, we will show that the pancake solutions constructed in Section 4 are unique. We will do so in several steps. The essential idea, as in \cite{BLT}, is to exploit the constructed solution as a barrier using the Alexandrov reflection principle. First, we show that on the constructed solution the constant $C$ defined by
\[
C\coloneqq \lim_{t \to -\infty}\left(\ell(t)+t-\dot{\phi}_1\log(-t)\right)
\]
is finite. We will use methods developed in Section \ref{sec:estimates} to obtain area and displacement estimates, which will allow us to prove this. We require the following lemma.

%We claim that $\phi_R(\theta,t)>\vert{\cos\theta}\vert$ for $t\geq-T_R$ on the approximating solutions. 
\begin{lemma}\label{lem:phi ge cos theta}
On the approximating solutions, $\phi_R(\theta,t)\geq |{\cos\theta}|$.
%In particular, $\phi(\pi,t)\geq 1$. %Therefore, $\ell(t)\geq -t$ for all $t>-T_R$. 
\end{lemma}
\begin{proof}
First observe that $\phi_R(\theta,-T_R)\geq |{\cos\theta}|$ by construction, and the inequality is always (trivially) satisfied at the poles $\theta=-\pi/2,\pi/2$. So we need to show that it continues to hold in $(-\pi/2,\pi/2)$ (by reflection symmetry, it will hold on the other half as well). To that end, consider the function $w=\phi/\cos\theta$. Suppose that $w$ develops a new interior minimum at some $(\theta_0,t_0)$. At this point, we have
\[
\phi_\theta=-\phi \tan\theta_0
\]
and hence
\[
0 \geq w_t-\kappa^2\phi_k w_{\theta\theta}=\phi_\lambda\lambda^2\phi\sec^3\theta_0 > 0.
\]
This leads to a contradiction and the claim follows. %The rest of the claim follows by plugging in $\theta=\pi$, and integrating from $t$ to $0$.
\end{proof}

Equations \eqref{DispEst} now allow us to proceed as in \S \ref{sec:estimates} to obtain the area estimate 
\begin{equation}
    -t+ \Dot{\phi}_1\log(-t)+C \geq \frac{A(t)}{2\pi} \geq -t+ \Dot{\phi}_1\log(-t)-C
\end{equation}
for all $t\in[-T_R,1)$. Bounding the area of $\Gamma_R(t)$ by a rectangle of height $2\ell_R(t)$ and width $\pi$ gives the length estimate
\[\ell_R(t) \geq -t+ \Dot{\phi}_1\log(-t)-C.\]

Now we show that $\ell_R(t)$ is bounded similarly from above (uniformly with respect to $R$). Proceeding in the same way as Lemma \ref{lem:ooft^-eps}, we obtain that for any $\varepsilon\in (0,1)$,
\[\ell(t) \leq -t+C_\varepsilon(-t)^{\varepsilon}\,,\]
where $C_\varepsilon$ is independent of $R.$
Now, proceeding as in \cite{BLT}, we may choose $C<\infty$ such that
\[\frac{d}{dt}(\ell_R(t)+t-\Dot{\phi}_1\log(-t)-f) \leq 0,\]
where $f \coloneqq \frac{C}{1-\varepsilon}\frac{1}{(-t)^{1-\varepsilon}}.$ Now integration yields
\[\ell_R(t)+t-\Dot{\phi}_1\log(-t) \leq \ell_R(-T_R)-T_R-\Dot{\phi}_1\log(T_R)+f(t)-f(-T_R).\]
Now, since
\[\ell_R(-T_R) \leq T_R + \Dot{\phi}_1\log(-T_R)+C+\log 2\]
and $\lim_{t \to -\infty}f(t)=0$, we let $R \to \infty$ to see that $C$ is finite.

Now we sketch the proof of uniqueness (cf.  \cite[Proof of Theorem 1.2]{BLT} for details.)  Suppose $\gamma,\gamma':S^1\times(-\infty,0)\to\R^2$ are the turning angle parametrizations of the profile curves $\Gamma_t,\Gamma_t'$ respectively of the solution constructed in Section 3, and an arbitrary $O(n)$-invariant compact convex ancient solution to the $\phi$-flow that lies in the same slab and no smaller slab. Let $\Omega_t,\Omega'_t$ be the open sets contained inside the profile curves. Since both solutions contract to the origin at $t=0$, they must intersect for all previous times due to the avoidance principle. Now consider the two constants
\begin{align*}
    C&:=\lim_{t\to-\infty}(\ell(t)+t-\dot{\phi}_1\log(-t))\\
    C'&:=\lim_{t\to-\infty}(\ell'(t)+t-\dot{\phi}_1\log(-t))
\end{align*}
where $\ell(t),\ell'(t)$ are the respective vertical displacements of $\gamma,\gamma'$. We have just shown that $C$ is finite, and that $C'\in \R\cup\{\infty\}$. If $C \neq C'$, it follows by the strong maximum principle that $\Omega_t'\Subset \Omega_t$ or $\Omega_t\Subset \Omega_t'$, and hence $\Gamma_t,\Gamma_t'$ never intersect. Thus $C=C'$. This implies that by the same reasoning, for any $\tau>0$ we would have that $\Omega_t'\Subset\Omega_{t+\tau}$. Taking $\tau \to 0$ shows that $\bar\Omega_t' \subset \bar\Omega_t$. By symmetry,  $\bar\Omega_t \subset \bar\Omega_t'$. Thus the solutions coincide.

\bibliographystyle{plain}
\bibliography{references.bib}

\end{document}