%\def\realnumbers{\rm I\!R}
\def\realnumbers{\mathbb{R}}
\newcommand{\realvectors}[1]{\mathbb{R}^{#1}}
\newcommand{\realmatrices}[2]{\mathbb{R}^{#1\times#2}}
\newcommand{\symmetricmatrices}[1]{\mathbb{S}^{#1}}
\newcommand{\posdefsymmetricmatrices}[1]{\mathbb{S}^{#1}_{++}}
\newcommand{\possemdefsymmetricmatrices}[1]{\mathbb{S}^{#1}_{+}}
\newcommand{\lowertriangularmatrices}[1]{\mathbb{L}^{#1}}
\newcommand{\uppertriangularmatrices}[1]{\mathbb{U}^{#1}}
\newcommand{\lowertrapezoidalmatrices}[2]{\mathbb{L}^{#1\times#2}}
\newcommand{\uppertrapezoidalmatrices}[2]{\mathbb{U}^{#1\times#2}}
\newcommand{\posdefdiagonalmatrices}[1]{\mathbb{D}^{#1}_{++}}
\newcommand{\permutationmatrices}[1]{\mathbb{P}^{#1}}
\newcommand{\orthogonalmatrices}[1]{\mathbb{Q}^{#1}}

 \DeclareRobustCommand{\lander}[1]{ {\begingroup\sethlcolor{BurntOrange}\hl{(lander:) #1}\endgroup} }
 \DeclareRobustCommand{\wilm}[1]{ {\begingroup\sethlcolor{BurntOrange}\hl{(wilm:) #1}\endgroup} }
%\DeclareRobustCommand{\lander}[1]{}
%\DeclareRobustCommand{\wilm}[1]{}

\newcommand{\dol}[1]{\overline{\overline{#1}}}
\newcommand{\ol}[1]{\overline{#1}}
\newcommand{\wt}[1]{\widetilde{#1}}
\newcommand{\wh}[1]{\widehat{#1}}
\newcommand{\inert}[1]{\text{inertia}\left(#1\right)}
\newcommand{\rightprod}[3]{\overset{\curvearrowright}{\displaystyle \prod_{#1}^{#2}} #3}
\newcommand{\leftprod}[3]{\overset{\curvearrowleft}{\displaystyle \prod_{#1}^{#2}} #3}
\newcommand{\sumdp}[3]{\sum_{#1}^{#2} #3}
\newcommand{\N}{T}
\newcommand{\pkg}[1]{\textsc{#1}}
\newcommand\overmat[2]{%
  \makebox[0pt][l]{$\smash{\color{white}\overbrace{\phantom{%
    \begin{matrix}#2\end{matrix}}}^{#1}}$}#2}
\newcommand\bovermat[2]{%
  \makebox[0pt][l]{$\smash{\overbrace{\phantom{%
    \begin{matrix}#2\end{matrix}}}^{#1}}$}#2}
\newcommand\partialphantom{\vphantom{\frac{\partial e_{P,M}}{\partial w_{1,1}}}}
\newcommand*{\Scale}[2][4]{\scalebox{#1}{$#2$}}%
\newcommand*{\Resize}[2]{\resizebox{#1}{!}{$#2$}}%
\setlength{\abovedisplayskip}{3pt}
\setlength{\belowdisplayskip}{3pt}

\newcommand\tp[2][-4]{{#2}^{\mkern#1mu\top}}
\newcommand\invtp[2][-4]{{#2}^{\mkern#1mu-\top}}
\newcommand\inv[2][-4]{{#2}^{\mkern#1mu-1}}

\DeclareMathOperator*{\minimize}{minimize}

\DeclareRobustCommand{\cpluspluslogo}{\hbox{C\hspace{-0.5ex}
                       \protect\raisebox{0.5ex}
                       {\protect\scalebox{0.67}{++}}}}


% \newcommand{\transpose}{^\mathsf{T}}
\newcommand{\transpose}{^\prime}
%\newcommand{\mintranspose}{^{-\mathsf{T}}}
\newcommand{\mintranspose}{^{\prime^{-1}}} % TODO: check/improve this notation, do we want -1 to be a superscript of \prime or not.
\newcommand\scalemath[2]{\scalebox{#1}{\mbox{\ensuremath{\displaystyle #2}}}}
% \newcommand*{\Scale}[2][4]{\scalebox{#1}{$#2$}}%