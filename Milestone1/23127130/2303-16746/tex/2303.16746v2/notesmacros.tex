% notesmacros.tex, LaTeX2e-definities
% Herman Bruyninckx, 28 JAN 2005

\renewcommand{\topfraction}{0.90}
\renewcommand{\bottomfraction}{0.90}
\renewcommand{\textfraction}{0.10}
\renewcommand{\floatpagefraction}{1.00}
% \renewcommand{\topfigrule}{\hrulefill}
% \renewcommand{\botfigrule}{\hrulefill}
\setcounter{totalnumber}{10}

\def\nochapnum{\par
  \setcounter{chapter}{0}
  \setcounter{section}{0}
  \def\@chapapp{}
}


\def\chapnum{\par
  \setcounter{chapter}{0}
  \setcounter{section}{0}
  \def\@chapapp{Chapter}
}

% macro to reduce separation between items in lists:
\newcommand{\tight}{ \addtolength{ \itemsep} {-3mm} }

\newtheorem{definition}{Definition}
\newtheorem{theorem}{Theorem}
\newtheorem{fact}{Fact-to-Remember}
%\newenvironment{fact}{quotation}

% `fact' surrounded by box:
\newcommand{\factbox}[3]% first argument: label of fact
%                         second argument: title of fact
%                         third argument: text of fact
{
  \mbox{ }\\%add some space above the frame
  \centerline{%
    %  \parbox{0.9\textwidth}{%
    % \begin{quotation} \noindent #3 \end{quotation}
    \framebox[0.9\textwidth]{%
      \ifthenelse
      {\equal{#2}{}}
      {\parbox{0.8\columnwidth}{\begin{fact}\label{#1} \mbox{ } \\{#3}\end{fact}}}
      {\parbox{0.8\columnwidth}{\begin{fact}[#2]\label{#1}\mbox{ }\\ {#3} \end{fact}}}
    }
    \mbox{ }\\%add some space after frame
  }
}

\newcommand{\refframe}[1]{\{#1\}}

\newcommand{\mysymb}[2]{
  \noindent
  \begin{tabular}[t]{p{2cm}@{\hspace{1em}:\hspace{1em}}p{9cm}}
    \makebox[2cm][r]{$#1$} & #2\end{ta}\\
}

\newcommand{\ca}{c_{\alpha}}
\newcommand{\cb}{c_{\beta}}
\newcommand{\cg}{c_{\gamma}}
\newcommand{\sa}{s_{\alpha}}
\newcommand{\sbe}{s_{\beta}}
\newcommand{\sg}{s_{\gamma}}
\newcommand{\ct}{c_{\theta}}
%\newcommand{\st}{s_{\theta}}
\newcommand{\vt}{v_{\theta}}
\newcommand{\cth}{c_{\frac{\theta}{2}}}
\newcommand{\sth}{s_{\frac{\theta}{2}}}

\newcommand{\leftsideset}[3]{
\newlength{\bovlen}
\newlength{\onlen}
\newlength{\boxlen}
\settowidth{\bovlen}{\ensuremath{^{#2}}}
\settowidth{\onlen}{\ensuremath{_{#1}}}
\addtolength{\bovlen}{-\onlen}
\settowidth{\boxlen}{\ensuremath{^{#2}_{#1}#3}}
\makebox[\boxlen][r]{\ensuremath{
\vphantom{#3}^{\hspace{0.2em}#2\hspace{-0.2em}}_{\hspace{\bovlen}#1}
#3}}
}

\newcommand{\bd}{\begin{displaymath}}
    \newcommand{\ed}{\end{displaymath}}

\newcommand{\MO}{\emph{MO}}
\newcommand{\ENV}{\emph{ENV}}

\newcommand{\sina}{\ensuremath{s_{\alpha}}}
\newcommand{\cosa}{\ensuremath{c_{\alpha}}}
\newcommand{\sinb}{\ensuremath{s_{\beta}}}
\newcommand{\cosb}{\ensuremath{c_{\beta}}}

\newcommand{\weg}[1]{}

% equality sign with `delta' on top (for definitions):
\newcommand{\eqdef}{\stackrel{\scriptscriptstyle \Delta}{=}}

% differential operator:
%\newcommand{\d}{\,\mathrm{d}}

\newcommand{\namearr}[1]{\stackrel{#1}{\rightarrow}}

\newcommand{\plusp}{\mbox{$\scriptstyle \protect\raisebox{-0.8ex}[1ex][0ex]{$'$}+$}}
\newcommand{\minp}{\mbox{$\scriptstyle \protect\raisebox{-0.8ex}[1ex][0ex]{$'$}-$}}
\newcommand{\Rrot}{R^r}
\newcommand{\rplus}{r^{\scriptscriptstyle +}}
\newcommand{\Rplus}{R^{\scriptscriptstyle +}}
\newcommand{\rmin}{r^{\scriptscriptstyle -}}
\newcommand{\Rmin}{R^{\scriptscriptstyle -}}
\newcommand{\Jac}{\boldsymbol{J}}
\newcommand{\Jslide}{\boldsymbol{J}^{\textrm{\textit{slide}}}}
\newcommand{\Jslip}{\boldsymbol{J}^{\textrm{\textit{slip}}}}
\newcommand{\Jplus}{\boldsymbol{J}^{\scriptscriptstyle +}}
\newcommand{\Jmin}{\boldsymbol{J}^{\scriptscriptstyle -}}
\newcommand{\Jrot}{\boldsymbol{J}^r}
\newcommand{\rplusp}{r^{\scriptscriptstyle \protect\raisebox{-1ex}[1ex][0ex]{$'$}+}}
\newcommand{\Rplusp}{R^{\scriptscriptstyle \protect\raisebox{-1ex}[1ex][0ex]{$'$}+}}
\newcommand{\rminp}{r^{\scriptscriptstyle \protect\raisebox{-1ex}[1ex][0ex]{$'$}-}}
\newcommand{\Rminp}{R^{\scriptscriptstyle \protect\raisebox{-1ex}[1ex][0ex]{$'$}-}}
\newcommand{\Jplusp}{\boldsymbol{J}^{\scriptscriptstyle
\protect\raisebox{-1ex}[1ex][0ex]{$'$}+}}
\newcommand{\Jpluspee}{\boldsymbol{J}^{{\scriptscriptstyle
    \protect\raisebox{-1ex}[1ex][0ex]{$'$}+},ee}}
\newcommand{\Jminp}{\boldsymbol{J}^{\scriptscriptstyle
\protect\raisebox{-1ex}[1ex][0ex]{$'$}-}}
\newcommand{\Jminpee}{\boldsymbol{J}^{{\scriptscriptstyle
    \protect\raisebox{-1ex}[1ex][0ex]{$'$}-},ee}}
\newcommand{\qplus}{q^{\scriptscriptstyle +}}
\newcommand{\qmin}{q^{\scriptscriptstyle -}}
\newcommand{\qrot}{q^r}
\newcommand{\qplusp}{q^{\scriptscriptstyle \protect\raisebox{-1ex}[1ex][0ex]{$'$}+}}
\newcommand{\qminp}{q^{{\scriptscriptstyle \protect\raisebox{-1ex}[1ex][0ex]{$'$}-}}}
\newcommand{\dotq}{\dot{q}}
\newcommand{\ddotq}{\ddot{q}}
\newcommand{\dqrot}{\dot{q}^r}
\newcommand{\dqplus}{\dot{q}^{\scriptscriptstyle +}}
\newcommand{\dqplusp}{\dot{q}^{{\scriptscriptstyle
    \protect\raisebox{-1ex}[1ex][0ex]{$'$}+}}}
\newcommand{\dqmin}{\dot{q}^{\scriptscriptstyle -}}
\newcommand{\dqminp}{\dot{q}^{{\scriptscriptstyle
    \protect\raisebox{-1ex}[1ex][0ex]{$'$}-}}}
\newcommand{\Pplus}{P^{\scriptscriptstyle +}}
\newcommand{\Pmin}{P^{\scriptscriptstyle -}}
\newcommand{\Pplusp}{P^{\scriptscriptstyle \protect\raisebox{-1ex}[1ex][0ex]{$'$}+}}
\newcommand{\Pminp}{P^{\scriptscriptstyle \protect\raisebox{-1ex}[1ex][0ex]{$'$}-}}

% omega-plus:
\newcommand{\omplus}{\omega^{\scriptscriptstyle +}}
% omega-min:
\newcommand{\ommin}{\omega^{\scriptscriptstyle -}}

% Symbol for J with nicely centered bar:
\newcommand{\barJ}
{\vphantom{J}{\hskip.3em\bar{\hskip-.3em\boldsymbol{J}}}\vphantom{J}}


% coordinate column vector:
\newcommand{\vc}[1]{\boldsymbol{#1}}
% coordinate column vector with leading sub- and superscripts:
\newcommand{\vcss}[3]
{\vphantom{#1}_{#3}^{\hspace*{0.2ex}#2}%
{\hspace*{-0.1ex}\boldsymbol{#1}}\vphantom{#1}}

% derivative of coordinate column vector with leading sub- and superscripts:
\newcommand{\dvcss}[3]
{\vphantom{#1}_{#3}^{\hspace*{0.2ex}#2}%
{\hspace*{-0.1ex}\boldsymbol{\dot{#1}}}\vphantom{#1}}

% unit vectors on X, Y and Z axes:
\newcommand{\unitx}{\boldsymbol{e}_x}
\newcommand{\unity}{\boldsymbol{e}_y}
\newcommand{\unitz}{\boldsymbol{e}_z}

% matrix:
\newcommand{\mt}[1]{\boldsymbol{#1}}

% identity matrix in R^n:
\newcommand{\identm}[1]{\boldsymbol{\mathbf{\mathit{I}}}_{#1}}

% zero matrix in R^n:
\newcommand{\zerom}[1]{\boldsymbol{\mathbf{\mathit{0}}}_{#1}}

% (poor!) symbol for set of real numbers:
\newcommand{\Reals}{\mbox{$\textrm{I}\!\textrm{R}$}}


% Homogeneous transformation matrix, from frame {#1} to frame {#2}:
\newcommand{\htrans}[2]%
{\vphantom{T}_{#2}^{\hspace*{0.1ex}#1}%
{\hspace*{-0.1ex}\boldsymbol{T}}\vphantom{T}}


% symbol for time derivative of \htrans~:
\newcommand{\dotT}[2]%
{\vphantom{T}_{#2}^{\hspace*{0.1ex}#1}%
{\dot{\hspace*{-0.2ex}\boldsymbol{T}}}\vphantom{T}}


% symbol for second time derivative of \htrans~:
\newcommand{\ddotT}[2]%
{\vphantom{T}_{#2}^{\hspace*{0.1ex}#1}%
{\ddot{\hspace*{-0.2ex}\boldsymbol{T}}}\vphantom{T}}

% Infinitesimal homogeneous transformation matrix, from frame {#1} to
% frame {#2}:
\newcommand{\infhtrans}[2]%
{\vphantom{T}_{#2}^{\hspace*{0.1ex}#1}%
{\hspace*{-0.3ex}\boldsymbol{T}}_{\hspace*{-0.8ex}\Delta}\vphantom{T}}

% Estimated homogeneous transformation matrix, from frame {#1} to frame {#2}:
\newcommand{\esthtrans}[2]%
{\sideset{_{#2}^{\hspace*{0.1ex}#1}}{}{\vphantom{T}}%
  {\hspace*{-0.2ex}\mathop{\widehat{\boldsymbol{T}}}}\vphantom{T}}


% Rotation matrix, from frame {#1} to frame {#2}:
\newcommand{\rot}[2]%
{\vphantom{R}_{#2}^{\hspace*{0.1ex}#1}%
{\hspace*{-0.2ex}\boldsymbol{R}}\vphantom{R}}

% symbol for time derivative of \rot~:
\newcommand{\dotR}[2]%
{\vphantom{R}_{#2}^{\hspace*{0.1ex}#1}%
{\dot{\hspace*{-0.2ex}\boldsymbol{R}}}\vphantom{R}}

% Screw transformation matrix, from frame {#1} to frame {#2}:
\newcommand{\strans}[2]%
{\vphantom{S}_{#2}^{\hspace*{0.2ex}#1}%
{\hspace*{-0.3ex}\boldsymbol{S}}\vphantom{S}}

% symbol for time derivative of \strans~:
\newcommand{\dotS} [2]%
{\vphantom{S}_{#2}^{\hspace*{0.3ex}#1}%
{\dot{\hspace*{-0.3ex}\boldsymbol{S}}}\vphantom{S}}

% Infinitesimal screw transformation matrix, from frame {#1} to frame
% {#2}:
\newcommand{\infstrans}[2]%
{\vphantom{S}_{#2}^{\hspace*{0.2ex}#1}%
{\hspace*{-0.3ex}\boldsymbol{S}}_{\hspace*{-0.3ex}\Delta}\vphantom{S}}



% Screw projection matrix, from frame {#1} to frame {#2}:
\newcommand{\sproj}[2]%
{\vphantom{P}_{#2}^{\hspace*{0.2ex}#1}%
{\hspace*{-0.3ex}\boldsymbol{P}}\vphantom{P}}


% symbol for time derivative of \sproj~:
\newcommand{\dotP} [2]%
{\vphantom{P}_{#2}^{\hspace*{0.3ex}#1}%
{\dot{\hspace*{-0.3ex}\boldsymbol{P}}}\vphantom{P}}

% symbol for infinitesimal \sproj~:
\newcommand{\infsproj} {\boldsymbol{P}_{\hspace*{-0.8ex}\Delta}}

% Reference point transformation matrix, from origin of frame {#1} to
% origin of frame {#2}:
\newcommand{\ptrans}[2]%
{\vphantom{M}_{#2}^{\hspace*{0.2ex}#1}%
{\hspace*{-0.3ex}\boldsymbol{M}}\vphantom{M}}



% symbol for time derivative of \ptrans~:
\newcommand{\dotM}[2]%
{\dot{\hspace*{-0.4ex}\boldsymbol{M}}\vphantom{M}^{#2}_{\hspace*{-0.3ex}#1}}

% symbol for infinitesimal \ptrans~:
\newcommand{\infptrans}%
{\boldsymbol{M}_{\hspace*{-0.5ex}\Delta}}

% symbol for reciprocity operator:
\newcommand{\DELTA} {\widetilde{\boldsymbol{\Delta}}\vphantom{\Delta}}

% symbol for ``bad'' stiffness matrix:
\newcommand{\tK}
{\vphantom{K}{\hspace*{1.3ex}\widetilde{\vphantom{K}}}%
  \hspace*{-1.3ex}\boldsymbol{K}\vphantom{K}}
% symbol for ``bad'' damping matrix:
\newcommand{\tC} {\vphantom{C}\widetilde{\boldsymbol{C}}\vphantom{C}}
% symbol for ``bad'' inertia matrix:
\newcommand{\tI} {\vphantom{I}\widetilde{\boldsymbol{I}}\vphantom{I}}

% symbol for time derivative of J~:
\newcommand{\dotJ}
{\vphantom{J}{\hskip.3em\dot{\hskip-.3em\boldsymbol{J}}}\vphantom{J}}
% symbol for twist system:
\newcommand{\T}{\boldsymbol{T}}
% symbol for wrench system:
\newcommand{\W}{\boldsymbol{W}}

% symbol for six-vector:
\newcommand{\sixvc}[1]{\textbf{\textsf{#1}}}

% symbol for twist:
\newcommand{\twist}{\textbf{\textsf{t}}}
% symbol for relative twist:
% \newcommand{\reltwist}[4]%
% {\vphantom{\textsf{t}}_{#4}^{\hspace*{0.2ex}#1}%
%  {\hspace*{-0.3ex}{\textbf{\textsf{t}}}}{\vphantom{\textsf{t}}_{#3}^{\hspace*{0.2ex}#2}}}
\newcommand{\reltwist}[4]
{\vphantom{\textbf{\textsf{t}}}_{#4}^{#1}\textbf{\textsf{t}}_{#3}^{#2}\vphantom{\textbf{\textsf{t}}}}
% 
% {\vphantom{\textbf{\textsf{t}}}_{#3}^{#4}\textbf{\textsf{t}}_{#2}^{#1}\vphantom{\textbf{\textsf{t}}}}



% symbol for infinitesimal twist:
\newcommand{\inftwist}{\textbf{\textsf{t}}_{\Delta}}
% symbol for wrench:
\newcommand{\wrench}{\textbf{\textsf{w}}}
% symbol for angular velocity:
\newcommand{\avel}{\boldsymbol{\omega}}
% symbol for linear velocity:
\newcommand{\lvel}{\textbf{\textsf{v}}}
% symbol for force:
\newcommand{\force}{\textbf{\textsf{f}}}
% symbol for moment:
\newcommand{\moment}{\textbf{\textsf{m}}}

% large boldmath greek gamma:
\newcommand{\GAMMA}{\ensuremath{\boldsymbol{\Gamma}}}
% large boldmath greek tau:
\newcommand{\TAU}{{\textrm{\Large \ensuremath{\boldsymbol{\tau}}}}}
% large boldmath greek upsilon:
\newcommand{\UPSILON}{{\textrm{\Large \ensuremath{\boldsymbol{\upsilon}}}}}
% large boldmath greekepsilon:
\newcommand{\EPSILON}{{\textrm{\Large \ensuremath{\boldsymbol{\epsilon}}}}}


% pseudo-inverse of matrix:
\newcommand{\pinv}[1]{\boldsymbol{#1}^{\dag}}
% weighted pseudo-inverse of matrix #1 with #2 as weighting matrix;
\newcommand{\wpinv}[2]{\boldsymbol{#1}^{\dag}_{#2}}

% estimated value of #1
\newcommand{\es}[1]{\hat{#1}}
% error between actual and estimated value
\newcommand{\er}[1]{\tilde{#1}}

% unit vectors along coordinate axes of a reference 
\newcommand{\eX}{\vc{e}^{{\scriptscriptstyle X}}}
\newcommand{\eY}{\vc{e}^{{\scriptscriptstyle Y}}}
\newcommand{\eZ}{\vc{e}^{{\scriptscriptstyle Z}}}

% nabla symbol with subscript
\newcommand{\mynabla}[1]{\mbox{\boldmath $\nabla_{\hspace*{-0.6ex}#1}$}}

% macro for printing current time: (Thierry.Bouche@ujf-grenoble.fr)
\newcount\m \newcount\n
\def\hours{\n=\time \divide\n 60
  \m=-\n \multiply\m 60 \advance\m \time
  \twodigits\n\ :\ \twodigits\m}
\def\twodigits#1{\ifnum #1<10 0\fi \number#1}

%macro for typesetting "C++":
\newcommand{\cpp}
{\mbox{C\hspace{-.05em}\protect\raisebox{.4ex}{\tiny\bf ++}}}
