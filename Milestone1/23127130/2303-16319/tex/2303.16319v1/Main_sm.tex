\documentclass[%
%  reprint,
aps,prb,preprint,
superscriptaddress,
% citeautoscript,
nobibnotes,
% bibnotes,
amsmath,amssymb,
floatfix,
]{revtex4-2}

\usepackage{xcolor}
\usepackage{graphicx}% Include figure files
\usepackage{dcolumn}% Align table columns on decimal point
\usepackage{bm}% bold math
\usepackage{array}
\usepackage{threeparttable}


\usepackage{caption}
\captionsetup[figure]{labelfont={bf},labelformat={default},labelsep=period,name={Fig.}}

\renewcommand{\thefigure}{S\arabic{figure}}
\renewcommand{\thetable}{S\arabic{table}}


% \newcommand{\blue}[1]{\textcolor{blue}{#1}}

\begin{document}

\title{Supplementary Information for: \\ Engineering the formation of spin-defects from first principles}

\author{Cunzhi Zhang}
\affiliation{Pritzker School of Molecular Engineering, University of Chicago, Chicago, IL
60637, USA}
\author{Francois Gygi}
\affiliation{Department of Computer Science, University of California Davis, Davis, CA 95616, USA}
\author {Giulia Galli}
\email{gagalli@uchicago.edu}
\affiliation{Pritzker School of Molecular Engineering, University of Chicago, Chicago, IL
60637, USA}
\affiliation{Department of Chemistry, University of Chicago, Chicago, IL 60637, USA}
\affiliation{Materials Science Division and Center for Molecular Engineering, Argonne National Laboratory, Lemont, IL 60439, USA}

\date{\today}

\maketitle
\clearpage

% Turned on when combined
% \nocite{*}



% \begin{table}[bp!]
% \centering

% \caption{The defect processes considered, specifying initial state, final state, process (M: migration; P: pairing; I: Isomerization) and label. Int. refers to intermediate.
% % Initial and final states are not shown for VV rotation \& migration, as both are VV.
% }

% \begin{tabular}{ |m{1.5cm}|m{3.3cm}|m{3.3cm}|m{3.3cm}|m{3.3cm}| }
%  \hline
%  \multicolumn{5}{|c|}{ \textcolor{cyan}{MV dynamics} } \\
%  \hline
%   Initial & V$_{\rm C}$ & V$_{\rm Si}$ & V$_{\rm Si}$ & CAV\\
%  \hline
%   Final   & V$_{\rm C}$ & V$_{\rm Si}$ & CAV  & V$_{\rm Si}$\\
%  \hline
%   Process & \blue{M} & \blue{M}   & \blue{I} & \blue{I} \\
%  \hline
%   Label   & V$_{\rm C}$ migration &	V$_{\rm Si}$ migration & V$_{\rm Si}$ $\rightarrow$ CAV & CAV $\rightarrow$ V$_{\rm Si}$ \\
%  \hline
%  \multicolumn{5}{|c|}{ \textcolor{cyan}{V-V pairing to VV} } \\
%  \hline
%   Initial & V-V$^2$ & V-V$^2$ &	V-V$^3$ & V-V$^3$ \\
%  \hline
%   Final   &	VV      & VV      &	VV      & VV      \\
%  \hline
%   Process &	\blue{P} via \blue{M}-V$_{\rm C}$ & \blue{P} via \blue{M}-V$_{\rm Si}$ & \blue{P} via \blue{M}-V$_{\rm C}$ &	\blue{P} via \blue{M}-V$_{\rm Si}$ \\
%  \hline
%   Label & 
%   V-V$^2$ $\rightarrow$ VV @ V$_{\rm C}$ &
%   V-V$^2$ $\rightarrow$ VV @ V$_{\rm Si}$ &
%   V-V$^3$ $\rightarrow$ VV @ V$_{\rm C}$ &	
%   V-V$^3$ $\rightarrow$ VV @ V$_{\rm Si}$ \\
%  \hline
%  \multicolumn{5}{|c|}{ \textcolor{cyan}{VV migration} } \\
%  \hline
%   Initial & VV      & VV      &        &  \\
%  \hline
%   Final   &	VV      & VV      &	       &  \\
%  \hline
%   Process &	\blue{M} via \blue{Int.} VCV      & \blue{M} via \blue{Int.} V-V$^3$      &	       &  \\
%  \hline
%   Label   & Migration @ VCV  & Migration @ V-V$^3$ &  & \\
%  \hline
  
% \end{tabular}

% % \begin{tabular}{ |m{1.5cm}|m{2.78cm}|m{2.78cm}|m{2.78cm}|m{2.78cm}|m{2.78cm}| }
% %  \hline
% %  \multicolumn{6}{|c|}{ \textcolor{cyan}{VV rotation \& migration} } \\
% %  \hline
% %   Process &	\blue{R} via \blue{M}-V$_{\rm C}$ & \blue{R} via \blue{M}-V$_{\rm Si}$ & \blue{R} via \blue{Int.} VCV & \blue{M} via \blue{Int.} VCV & \blue{M} via \blue{Int.} V-V$^3$ \\
% %  \hline
% %   Label   &	Rot. @ V$_{\rm C}$ & Rot. @ V$_{\rm Si}$ & Rot. @ VCV &	Mig. @ VCV & Mig. @ V-V$^3$ \\
% %  \hline
% % \end{tabular}

% \begin{threeparttable}
% \begin{tablenotes}\footnotesize
%     % \item[a] Int. refers to intermediate.
%     % \item[b] Barrier of CAV dissociation, CAV $\rightarrow$ C$_{\rm Si}$ + V$_{\rm C}$, is estimated as the sum of binding energy and V$_{\rm C}$ migration barrier.
%     % \item[c] These pathways are motivated by previous results and our chemical intuitions.
%     \item[a] Orientation of defect is ignored, since high $T$ enables the explorations of multiple configurations; we assume transition state is insensitive to orientation.
% \end{tablenotes}
% \end{threeparttable}

% \end{table}



\section*{Note 1: Defect formation energies}
\begin{figure*}[hp!]
\centering
  \includegraphics[width=0.45\textwidth]{Fig_S1.pdf}
  \caption{Formation energy of defects in 3C-SiC obtained under C-rich conditions, obtained using DFT and the DDH functional.  The geometry of these defects can be found in Fig. 1. See Methods in the manuscript.}
  \label{fig:ef_SM}
\end{figure*}

 
\clearpage
\section*{Note 2: Collective variables and free energy barriers}

\begin{figure*}[hp!]
\centering
  \includegraphics[width=0.9\textwidth]{Fig_S2.pdf}
  \caption{Collective variables (CV) for three paths investigated in our simulations. Number 0 denotes the moving atom; numbers 1-8 denote the gate atoms; projection vectors are also indicated. The three paths are: A: V-V$^3$ $\rightarrow$ VV @ V$\rm _C$; B: V-V$^3$ $\rightarrow$ VV @ V$_{\rm Si}$; C: first step in VV migration (see Fig. 1 in the manuscript).}
  \label{fig:CV_SM}
\end{figure*}
We obtained the free energy surface at 1,500 K using the adaptive biasing force method. In our enhanced sampling calculations, we considered three pathways, as shown in Fig. \ref{fig:CV_SM}, where we highlight the moving atom, gate atoms and projection vectors used to define the collective variable (see Methods in the manuscript). 
% We note that the performance {\bf ?} of the CV is not sensitive to the choice of gate atoms. 
For each pathway, we computed the free energy barrier $G_{\rm b}$ and entropy change $\Delta S$ from the initial to the transition state, which we denote  as forward (F) direction; we then computed the $G_{\rm b}$ and $\Delta S$ from the final to the transition state, which we denote as backward (B) direction. 
We summarize the computed values of $G_{\rm b}$ and $\Delta S$ in Table \ref{tab:barrier_SM}. 

For the three pathways studied here at 1,500 K, $G_{\rm b}$ decreases  by $\sim$ (0.11, 0.38) eV, relative to the value at 0 K. Using the harmonic approximation and classical statistics (see Note 4), we estimate $\Delta S$ in the range of $\sim$ (0.85, 2.90) $k_{\rm B}$. In particular, in going from the stable VV configuration to the transition state, the $G_{\rm b}$ decreases by $\sim$ 0.3 eV and the corresponding $\Delta S$ is $\sim$ 2.3 $k_{\rm B}$; the latter value is  used  to estimate the activation temperature (see Note 4). 


We note that the values of $\Delta S$ were computed at the PBE level of theory ($\Delta S^{\rm PBE}$).
Using the harmonic approximation and classical statistics (see Note 4),
$\Delta S$ is determined from phonon frequencies. Assuming error cancellations between total energies computed at the PBE and DDH levels of theory, it is also reasonable to assume that $\Delta S^{\rm PBE} \sim \Delta S^{\rm DDH}$.

\begin{table}[hp!]
\centering
\caption{Free energy barriers $G_{\rm b}$ (eV),  free energy barrier change $\Delta G_{\rm b} = G_{\rm b} @\ 0\ {\rm K} - G_{\rm b} @\ 1,500\ {\rm K} $ 
(eV) and entropy change  $\Delta S = \Delta G_{\rm b} / T$ ($k_{\rm B}$), where $k_{\rm B}$ is the Boltzmann constant and $T$ = 1,500 K.}

\begin{tabular}{ |m{2.8cm}|m{2.cm}|m{2.cm}|m{2.cm}|m{2.cm}|m{3.5cm}| }
 \hline
 \multicolumn{6}{|c|}{  $G_{\rm b}$ } \\ \hline
 Pathways & 
 \multicolumn{2}{|l|}{ V-V$^3$ $\rightarrow$ VV @ V$\rm _C$ } 
 &
 \multicolumn{2}{|l|}{ V-V$^3$ $\rightarrow$ VV @ V$_{\rm Si}$ 
 } & 
 \multicolumn{1}{|l|}{ VV migration$^{\rm a}$ } \\ \hline
  Direction$^{\rm b}$ & F & B & F & B & F (=B) \\ \hline
  $G_{\rm b}$ @ 0 K & 2.35 & 4.34 & 1.29 & 3.28 & 3.07 \\  \hline
  $G_{\rm b}$ @ 1,500 K & 2.24 & 4.06 & 1.14 & 2.98 & 2.70 \\ \hline
  \multicolumn{6}{|c|}{  $\Delta G_{\rm b}$} \\ \hline
  $\Delta G_{\rm b}$ & 0.11 & 0.28 & 0.15 & 0.30 & 0.38 \\  \hline
  \multicolumn{6}{|c|}{  $\Delta S$ } \\  \hline
  $\Delta S$ & 0.85 & 2.17 & 1.16 & 2.32 & 2.90 \\ \hline
  
\end{tabular}

\begin{threeparttable}
\begin{tablenotes}\footnotesize
    \item[a] Only the first step in VV migration path is simulated, as shown in Fig. \ref{fig:CV_SM}C.
    \item[b] F refers to forward direction from the initial to the transition state; B refers to backward direction from the final to the transition state.
\end{tablenotes}
\end{threeparttable}

\label{tab:barrier_SM}
\end{table}

\clearpage

\section*{Note 3: Calculation of effective barriers}

During transformations occurring at high temperature ($T$), point defects may be in several charge ($q$) and spin ($s$) states different from the  thermodynamically stable ones.  Hence we should consider different energy barriers $E_{\rm b}(q, s)$. Moreover, the transition between different $q$ and $s$ states may occur at elevated $T$ due, e.g., to vibrational effects. For these reasons, we used effective barriers $E_{\rm b,\ EFF}$\cite{gerst2004PRB_EFF,bruneval2011PRB_EFF} to describe atomic processes occurring at high $T$, instead of simple barriers $E_{\rm b}$.  

% We found that the  energy splitting  between different spin states is usually small relative to the  energy barriers $\sim$ several eV.
% ({\bf small relative to what?}).

In our NEB calculations, we first  determined the most stable $s$ state for each image at a given $q$; the corresponding total energies and atomic forces were then used to determine the minimum energy path and $E_{\rm b}$. Hence the final
$E_{\rm b}$ obtained in this way is only a function of $q$, with the effect of the $s$ degree of freedom (DOF) included implicitly.
Our treatment of the spin DOFs assumes an instantaneous equilibration of spin states during defects' transformations.
Although we could not estimate the timescale of $s$ transitions via spin-orbital-coupling or spin-phonon interactions at $\sim$ 1,000 K, we found that 
in general considering different spin states affects only slightly the computed $E_{\rm b}$ (see Methods in the manuscript).
 
We computed $E_{\rm b,\ EFF}$ based on $E_{\rm b}$ and defect formation energies, considering only the charge DOF (see Computational strategy in the manuscript). We assumed the charge state $q$ to be  preserved during defect transformations, due to the short lifetime of barrier crossing over the transition state, and we considered $q$ transitions  at the initial and final states of a given path.

In the case of the dissociation of complex defects, involving multiple steps, $E_{\rm b,\ EFF}$ was estimated from the  binding energy and diffusion barriers. For instance, $E_{\rm b,\ EFF}$ for the CAV dissociation process was obtained as the sum of the  binding energy ( C$_{\rm Si}$ \& V$_{\rm C}$) and the  $E_{\rm b,\ EFF}$ of V$_{\rm C}$ migration.




% we assumed the $q$ to be  preserved during defect transformations.{\bf ?}
% \begin{equation}
%    E_{\rm b,\ EFF} (E_{\rm F}) = \underset{q \in \{ 2, 1, 0, -1, -2 \} }{\text{min}}
%    \big\{ \Delta E_{\rm f}(q, E_{\rm F})
%    + E_{\rm b}(q) \big\}
% \end{equation}
% where $\Delta E_{\rm f}(q, E_{\rm F})$ is formation energy difference for a given $q$, with respect to the most stable one at given $E_{\rm F}$; $E_{\rm b}(q)$ is barrier at $q$ obtained from NEB. 
For $E_{\rm b,\ EFF}$ to be accurate, the transition between different $q$ states needs to be fast 
relative to the transformation of defects into different configurations.
We estimated the $q$ transition, via carrier capture or emission, is indeed fast at high $T$, as indicated below. 
% We also note that for processes with low $E_{\rm b}$, e.g. interstitial jump, the defect transformations can be fast and comparable to the $q$ or $s$ transition; in these cases, the use of $E_{\rm b,\ EFF}$ can be problematic.   {\bf ?} 

Under equilibrium condition, the $q$ transition rate ($g$)~\cite{bruneval2011PRB_EFF,bourgoin1983_Q} can be estimated as:  
\begin{equation}
   g(T) = \sigma \langle v \rangle N \gamma 
   \exp( - \frac{ \Delta E }{ k_{\rm B}T } )
\end{equation}
where $\sigma$ is the capture cross section; $\langle v \rangle$ is the average thermal velocity of carriers; $N$ is the effective density of states (DOS); $\gamma$ is the degeneracy factor; $\Delta E$ is the energy difference between defect levels and the closest band edge; $k_{\rm B}$ is the Boltzmann constant.

Given $N(T)$ $\sim T^{3/2}$ \cite{kimoto2014book_SiC, neamen2012_fourth}, we estimated $N (1,000\ \rm K)$ $\sim$ $N (300\ \rm K) \times (10/3)^{3/2}$. $N (300\ \rm K)$ was measured experimentally \cite{kimoto2014book_SiC}. 
Given $\langle v \rangle (T) $ $\sim$ $(3 k_{\rm B}T / m_{0})^{0.5}$, where $m_{0}$ is the mass of stationary electron; we estimated $\langle v \rangle (1,000\ \rm K)$ $\sim$ $2 \times 10^7$ cm/s. Based on experiments, for most of deep levels in SiC, $\sigma$ are in the range of (10$^{-17}$, 10$^{-14}$) cm$^2$ \cite{nakane2021JAP_cross,hazdra2019PSS_cross,danno2007JAP_cross,tunhuma2018JAP_cross}.
We assumed $\gamma$ = 1 and $\Delta E$ $\sim$ 1 eV, approximately  half the band gap of 3C-SiC. Then, we obtained the timescale, $1/g(1,000\ \rm K)$, in the range of $\sim$ (10$^{-8}$, 10$^{-5}$) s. For comparison, for defect transformations at 1,000 K with barrier $\sim$ 3 eV, the timescale is estimated to be $\sim$ 10 s.




\clearpage
\section*{Note 4: Calculations of the activation temperature}

According to the harmonic transition state theory\cite{viney1957JPCS_HTST,li2007MRS_HTST}, the jump frequency $\Gamma$ can be calculated as:
\begin{equation}
  \Gamma = \Gamma_{0} \exp( - G_{\rm b} / k_{\rm B}T )
\end{equation} 
where $\Gamma_{0}$ is the attempt frequency and  $G_{\rm b}$ the free energy barrier of a given process. 

During defect transformations, we assumed the system volume to be constant, hence:
\begin{equation}
  G_{\rm b} = \Delta U - T\Delta S
\end{equation} 
where $\Delta U$ is the change in internal energy from the initial to the transition state of a given path; $\Delta S$ is the change in entropy from the initial to the transition state of the path.  Note that we computed $\Delta S$ for three paths only, due to the computational cost,  and found values varying within $\sim$ (0.85, 2.90) $k_{\rm B}$ (see Note 2).  Based on the harmonic approximation and classical statistics, the change in kinetic energy is 0 eV, and the change in potential energy is $E_{\rm b}$ (barrier at 0 K), enforced by the equi-partition theorem. In addition, $\Delta S$ is constant, as determined by calculations of phonon frequencies\cite{viney1957JPCS_HTST,li2007MRS_HTST}. Therefore, we obtain:
\begin{equation}
\label{eq:gamma_SM}
  \Gamma \approx \Gamma_{0} \exp( \frac{\Delta S}{k_{\rm B}} )
  \exp( - E_{\rm b} / k_{\rm B}T )
\end{equation} 

The activation temperature $T_{a}$ is defined as the $T$ above which a process is thermally activated. Based on Eq. \ref{eq:gamma_SM}, $T_{a}$ can be written as: 
\begin{equation}
\label{eq:ta_SM}
  T_{\rm a} = \Big[ k_{\rm B} 
  \ln( \Gamma_{0} \exp( \frac{\Delta S}{k_{\rm B}}) 
  / \Gamma ) \Big]^{-1} 
  \times E_{\rm b,\ EFF} 
%   \approx 331 \times E_{\rm b}\ ({\rm K})
\end{equation} 

Similar to previous studies, we approximated $\Gamma_{0}$ to be 1.6 $\times$ 10$^{13}$ Hz \cite{rauls2003PRB_TS}; jump frequency $\Gamma$ to be 0.1 Hz \cite{kyrts2017PRB_1Hz,froda2021PRM_1min}; $\Delta S$ to be 2.3 $k_{\rm B}$. This value corresponds to our estimate of  the $\Delta S$ from the stable VV configuration to the transition state (see Note 2).
We obtained a prefactor (inverse of the quantity within square brackets in Eq. \ref{eq:ta_SM} ) of 331 K/eV. 
% We also noted that anharmonic effects are partially included into $\Delta S$, as phonon renormalization at 1500 K is captured by enhanced sampling MD. 
A simple sensitivity analysis shows that such prefactor is relatively insensitive to the choice of $\Gamma$ and $\Delta S$. For example, by varying $\Gamma$ in the range of (0.01, 1) Hz (with all other parameters fixed), the prefactor changes by $<$ $\sim$ 7 \%; by varying $\Delta S$ in the range of (0.85, 2.90) $k_{\rm B}$, the prefactor changes by $<$ $\sim$ 4 \%. 

We systematically investigated the thermal expansion and entropic effects on computed energy barriers. We found that considering lattice expansion at 1,500 K leads to a minor change  of $E_{\rm b}$ (which is on the order of several eV) of approximately  $\sim \pm$ 0.1 eV for most processes,  with the exception of  small carbon-clusters' formation for which  differences in energy barriers are $\sim \pm$ 0.3 eV. 
We found that due to entropic effects, our computed free energy barriers are lowered by $\sim$ (0.11, 0.38) eV at 1,500 K, relative to those obtained at 0 K (see Note 2), consistent  with estimates based on the harmonic approximation~\cite{rauls2003PRB_TS}. 
As a result, we estimate that the variation of $T_{\rm a}$ due to thermal expansion and entropic effects is less than 10 $\%$ .


\clearpage
\section*{Note 5: Calculations of the Fermi level}

After irradiation or implantation of SiC samples, multiple defects can be created including interstitials, antisites, substitutionals and vacancies. 
% Given that: 1) interstitials are annealed out at low $T$; 2) antisites and substitutions are mostly immobile, only vacancies are relevant at high $T$ annealing for VV creation processes.  
% We note that antisites in SiC are generally neutral within the band gap\cite{yan2020JAP_VSi}, while substitutional atoms can be charged. 
In order to obtain an accurate value of $E_{\rm F}$ we need to consider both external doping and the charge state of the defects created in the sample. 

\begin{figure*}[!htp]
\centering
  \includegraphics[width=0.99\textwidth]{Fig_S3.pdf}
  \caption{Fermi level as a function of $T$ in 3C-SiC. The Fermi level is referred to the top of the valence band. We consider the doping or defects density in the range of (10$^{14}$, 10$^{19}$) cm$^{-3}$. A: presence of $n$- or $p$-doping only. B: presence of carbon vacancy (V$_{\rm C}$) only. C: presence of silicon vacancy (V$_{\rm Si}$) only. D: presence of carbon antisite vacancy  (CAV) only. E: presence of V$_{\rm C}$ and V$_{\rm Si}$ of the same amount. F: presence of V$_{\rm C}$ and CAV of the same amount. The value of mid-gap for 3C-SiC is indicated by a grey dashed line; the green-region for $E_{\rm F}$ $>$ 1.46 eV indicates the suitable conditions for the VV creation.}
  \label{fig:fermi_SM}
\end{figure*}

\begin{figure*}[!htp]
\centering
  \includegraphics[width=0.5\textwidth]{Fig_S4.pdf}
  \caption{Fermi level as a function of $T$ in 4H-SiC.
  The Fermi level is referred to the top of the valence band. We consider the $n$- or $p$-doping density in the range of (10$^{14}$, 10$^{19}$) cm$^{-3}$. The value of mid-gap for 4H-SiC is indicated by a grey dashed line; the green-region for $E_{\rm F}$ $>$ 1.46 eV indicates the suitable conditions for the VV creation.
  }
  \label{fig:fermi_4H_SM}
\end{figure*}

In this study, we took into account several vacancies: V$_{\rm C}$, V$_{\rm Si}$ and CAV, which are relevant to the VV creation processes. We determined $E_{\rm F}$ using the following equations \cite{neamen2012_fourth,ma2011PRB_compensation,bucker2019CPC_compensation}:
\begin{equation}
\label{eq:ef_SM}
  \begin{cases}
    a) & n_{0}p_{0} = N_{\rm C} N_{\rm V} 
      \exp( - E_{\rm g} / k_{\rm B}T ) \\
    b) & n_{0} + N_{\rm a} = p_{0} + N_{\rm d} + \sum_{ {\rm X \in (V_C, V_{Si}, CAV) } }
         \sum_{q} q \times N( {\rm X} ^ {q} )\\
    % b) & n_{0} + N({\rm V_{Si}^{\ -}}) + N_{\rm a} = p_{0} +
    % N({ \rm CAV^{+1} }) + 2 \times N({ \rm V_{C}^{\ +2} }) + 
    % N({ \rm V_{C}^{\ +1} }) + N_{\rm d}\ \text{(e.g.)}\\
    c) & N( {\rm X} ^ {q} ) \propto N_{\rm X} \textsl{g}_{ {\rm X} ^ {q} } 
    \exp( - E_{\rm f} ({\rm X}^{q}) / k_{\rm B}T ) \\
    d) & E_{\rm F} - E_{\rm V} = k_{\rm B}T 
    \ln(N_{\rm V} / p_{0})
  \end{cases}   
\end{equation} 
% $N_{\rm V}=2 (\frac{2\pi m_{p} k_{\rm B} T}{ h^{2} }) ^ {3/2}$
where $n_0$ is the electron density; $p_0$ is the hole density; $N_{\rm C}$ is the effective conduction band DOS; $N_{\rm V}$ is the effective valence band DOS; $E_{\rm g}$ is the band gap; $N_{\rm a}$ is the acceptor density; $N_{\rm d}$ is the donor density; X$^q$ stands for defect X in charge state $q$; $N( {\rm X}^q )$ is the density of X$^q$; $N_{\rm X}$ is the total density of defect X; $\textsl{g}$ is the degeneracy factor; taken as 1; $E_{\rm f}$ is the defect formation energy (see Note 1); $E_{\rm V}$ is the valence band maximum (VBM) energy. 

Eq. \ref{eq:ef_SM}-$b$ expresses the charge neutrality condition, incorporating the effects of defects. These equations need to be solved self-consistently. 
Here, we performed a line-search by step-wisely increasing $E_{\rm F}$ from VBM to the conduction band minimum; we determined $E_{\rm F}$ as the value, which makes Eq. \ref{eq:ef_SM}-$b$ satisfied with a minimal error.
The electronic properties of SiC were obtained from the Appendix C of the book \cite{kimoto2014book_SiC}. For simplicity, we ignored the $T$ dependence of these parameters: 1) $E_{\rm g}$ at 300 K was used; 2) $E_{\rm f}$ diagram at 0 K was used (Fig. \ref{fig:ef_SM}). We deduced the DOS effective masses based on the measured $N_{\rm C}$ and $N_{\rm V}$ at 300 K, which were then used to compute $N_{\rm C}$ and $N_{\rm V}$ at various $T$. 
Some of our results are shown in Fig. \ref{fig:fermi_SM} and \ref{fig:fermi_4H_SM}. We note our $E_{\rm F}$ may be over-estimated, since band gap decreases at high $T$.

Overall, the calculation of $E_{\rm F}$ here should be taken as a qualitative estimate, as we: 1) ignored the $T$ dependence of electronic properties, e.g. band gap of SiC; 2) ignored the effects of other defects, in addition to V$_{\rm C}$, V$_{\rm Si}$ and CAV. More accurate treatment is beyond the scope of this work. 

% \bibliography{Ref_main,Ref_SM}
\bibliography{Ref_SM}

\bibliographystyle{apsrev4-2}

\end{document}
