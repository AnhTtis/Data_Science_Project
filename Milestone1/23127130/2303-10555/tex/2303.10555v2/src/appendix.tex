




\begin{figure}[h!]
\centering
\includegraphics[width=\linewidth]{imgs/obj_detector/n_points.pdf}
\vspace{-0.25in}
\caption{The number of points for the targeted vehicle at each distance to the victim ego vehicle.}
\label{fig:n_points}
\vspace{-0.1in}
\end{figure}

\begin{figure}[h!]
\centering
\includegraphics[width=\linewidth]{imgs/methodology/high-frequency_removal_attack.pdf}
\caption{Attack mechanism difference between the saturating attack and the newly-identified HFR attack (\S\ref{sec:high_freq_attack}).
}
\label{fig:highfreq_vs_saturating}
\end{figure}

\begin{figure}[h!]
\centering
\includegraphics[width=\linewidth]{imgs/spoofing/spoofing_livox.pdf}
\caption{Point cloud of Livox Horizon in benign and attack scenarios. The point cloud is totally randomized by the attack and the pattern of the rectangle area (our lab room) in the left figure completely disappears as shown in the right figure.}
\label{fig:livox_spoofing}
\end{figure}

\begin{figure}[h!]
\centering
\includegraphics[width=\linewidth]{imgs/obj_detector/synthesized_scenarios.pdf}
\caption{Targeted experiment scenario from KITTI~\cite{Geiger2012CVPR}.}
\label{fig:synth_scenario}
\end{figure}




\begin{figure}[h!]
\centering
\includegraphics[width=\linewidth]{imgs/simulator/sim_setup.pdf}
\caption{Illustration of the evaluation scenario. AD starts driving from 200 m away and it can always successfully stop before the target sedan in the scenarios without attack.}
\label{fig:overview_sim}
\end{figure}

\newpage


\vspace{0.3in}
\appendices

 

\vspace{-0.2in}
\nsection{Detailed Explanations on Synchronized LiDAR Spoofing Attacks}
\label{appndix:sync_attack}

Fig.~\ref{fig:overview_sync} illustrates the synchronized attacks on VLP-16. The attack mechanism is common on both the injection attacks~\cite{shin2017illusion, cao2019adversarial, jiachen2020towards, hallyburton2022security} and the removal attacks~\cite{zhongyuan2021object, cao2023you} since their difference is whether they move points at target locations or move points into undetectable area. The attack procedure consists of 3 steps:~\circled{1} PD first receive the legitimate laser from the target LiDAR to know when the LiDAR will scan the point that the attacker want to change; ~\circled{2} FG plans when to fire lasers based on the information from PD and the pre-defined scan pattern of the target LiDAR;~\circled{3} the laser is fired based on the plan from FG through the gate driver and LD. Therefore, to achieve the CPI attack capabilities, the attacker must know exactly where LiDAR is scanning and the scan schedule must be predefined or predictable. The timing randomization breaks the assumption by randomizing the scan schedule.


\begin{figure}[h!]
\centering
\vspace{0.05in}
\includegraphics[width=\linewidth]{imgs/methodology/sync_attack_detail.pdf}
\caption{Illustration of synchronized attacks on VLP-16. VLP-16 scans each azimuth (every 0.1$^{\circ}$) one by one. At an azimuth, it fires 16 lasers vertically based on the pre-defined scan pattern. Once attackers can identify its state by PD, they can know \textit{when} to fire a malicious laser based on the pattern.}
\label{fig:overview_sync}
\end{figure}


\vspace{-0.05in}
\nsection{Details of LiDAR Spoofer Used in Experiments}
\label{appendix:spoofer}

Our spoofer operates in 4 steps (shown in Fig.~\ref{fig:spoofer1}): \circled{1} The photodetector (PD) receives a laser pulse from LiDAR; \circled{2} the transimpedance amplifier (TIA) amplifies the pulse to 100 mV to feed it into the function generator (FG); \circled{3} FG synchronizes with the LiDAR spanning pattern based on the received pulse and generates laser pulses according to the points that the attacker wants to inject; \circled{4} The gate driver (GD) drives the laser diode (LD) which transmits a laser pulse to the LiDAR.


We use the same PD (S6775~\cite{S6775}) and TIA (TL082~\cite{TL082}) as the prior work~\cite{cao2023you}. For other components, we used the equivalent or enhanced devices: FG is Agilent 81160A~\cite{81160A}, GD is EPC9126HC~\cite{EPC9126HC}, and LD is SPL PL90\_3~\cite{SPLPL90_3}. In particular, we utilize a more functional FG (Agilent 81160A) than the one used in previous studies (Tektronix AFG320~\cite{AFG320}) to better facilitate the exploration of the CPI attack capability.

\newpart{
Furthermore, the improved optical design significantly affects the number and angle coverage of spoofable points as discussed in~\S\ref{sec:optical_setup}. Fig.~\ref{fig:optics_exp} illustrates the 3 types of laser beam shapes:  converged, diverged, and collimated. Among them, the collimated beam is the most ideal for LiDAR spoofing attacks because it can deliver the laser to the target with a minimum loss. To precisely calibrate the lens setup, we develop a device that can adjust the distance between the LD and the lens as designed. As shown in Fig.~\ref{fig:spoofer1}, the lens is connected to the frame with a hollow screw so that we can adjust it precisely. 
}

\begin{figure}[h!]
\centering
\includegraphics[width=\linewidth]{imgs/methodology/optics_exp.pdf}
\vspace{-0.1in}
\caption{
\newpart{
Illustration of the 3 types of laser beam shapes: converged, diverged, and collimated.
}
}
\label{fig:optics_exp}
\end{figure}



\nsection{Case Study Results on Specific LiDARs}
\label{appendix:specific-lidars}
\vspace{0.05in}

\nsubsection{The Use of Rare Laser Wavelength in OS1-32}

For OS1-32~\cite{OS1-32}, we find that we are not able to inject many points since we only have a 905 nm wavelength (SPL PL90\_3), which is also the setup for all prior works~\cite{shin2017illusion, cao2019adversarial, jiachen2020towards, hallyburton2022security, cao2023you}. %
The spoofing with 905 nm wavelength is not effective on OS1-32 that uses an 865 nm wavelength laser. 
We tried our best to purchase 865 nm laser diodes, but we find that high-power 865 nm laser diodes are actually not generally publicly available for personal use. We also tried low-power 860 nm laser diodes such as OPV380 and RLD85PZJ4, but they cannot work more than 4 W, which is too low to inject any points (our 905 nm diodes work at 90W to achieve successful injections).
From some perspective, this may imply that developing/using a LiDAR with a relatively rare wavelength may have a ``immunization'' effect against LiDAR spoofing attacks in practice since it may make it harder for attackers to acquire the corresponding attack laser diodes.
At this point, only a very small portion of LiDARs has this property; a recent survey~\cite{lidar_survey2021} reports that 905 nm wavelength laser is the most commonly used and 865 nm wavelength laser does not even appear in their graph due to its rarity.

Nevertheless, such a potential ``immunization'' effect would lose effectiveness if more LiDARs adopt 865 nm wavelength as this may boost the availability of 865 nm laser diodes on market. Also we find that not all LiDARs using a rare wavelength can directly have such an ``immunization'' effect. For example, although Realsense L515 is using 860 nm instead of 905 nm, we find that it is still directly attackable with the 905 nm attack laser since it does not have a band-pass filter to exclude light other than its wavelength.



\nsubsection{Relay Attack on Leddar Pixell}

Flash LiDAR fires a wide diverging laser beam and needs to receive them at the same time. This mechanism is exploitable by the relay attack (\S\ref{sec:async_attack}), which is not effective on the scanning LiDARs because PD can only receive a very limited azimuthal range, and the spoofed points will always be located far from the spoofer as discussed in~\cite{shin2017illusion}. 
However, we find that the relay %
attack can be effective on the flash LiDAR as a removal attack. As its laser covers a wide range, the attacker can also disturb a wider range. If the attacker's laser is strong enough, they can remove the original points by moving them to farther points than the spoofer.
Fig.~\ref{fig:relay_attack} shows the results of the relay attack on Leddar Pixell~\cite{pixell}. The entire row of detection is moved to very far position.
Note that Leddar Pixell has a special design that consists of multiple rows of scanning. If it is a typical flash LiDAR, the relay attack can cause an effect on more rows. %


\begin{observation}{RQ2}
Compared to traditional scanning LiDARs, relay attack may be more effective on flash LiDARs as an object removal attack.
\end{observation}

\begin{figure}[h!]
\centering
\includegraphics[width=\linewidth]{imgs/spoofing/relay_attack_leddar.png}
\caption{Relay Attack on Leddar Pixell~\cite{pixell}. The attack moves the detected area to a much farther location. %
}
\label{fig:relay_attack}
\end{figure}

\nsubsection{Zero-Distance Sensing of XT32}

As listed in Table~\ref{tbl:target_lidars}, XT32~\cite{XT32} is capable to measure the distance to LiDAR down to 0 m. The zero-distance sensing is not available in most LiDARs due to its technical challenges: detecting a very short time of laser flight is difficult to distinguish from noise due to unideal hardware calibration as discussed in~\cite{cao2023you}. However, it is technically realizable, which thus directly breaks the design assumption of the latest PRA attack~\cite{cao2023you} (i.e., needs enough MOT, \S\ref{sec:sync_removal_attack}).


\begin{observation}{RQ1}
LiDARs with zero-distance sensing do exist, which directly breaks the design assumption of the latest PRA attack~\cite{cao2023you}.
\end{observation}

\nsubsection{Extra Wide Vertical FOV of Helios 5515} \label{sec:case_helios}


As listed in Table~\ref{tbl:target_lidars}, Helios~\cite{Helios} has an extra wide vertical field-of-view (FOV), 70$^{\circ}$, which is far wider than the normal range of 30-40$^{\circ}$ (Table~\ref{tbl:target_lidars}).
It results in the lower attack success rate $\mathcal{R}=19.4\%$ even though the number of points is large, 3,203 points.
This is because our spoofer cannot cover all altitudes of Helios. However, it does not mean that Helios is more robust against attacks because the attacker does not need to attack the entire vertical FOV. Fig.~\ref{fig:helios_hfa} shows an example of the HFR attack effect on Helios. As shown, the HFR attack is successful in the middle of the vertical FOV, which can allow attackers to hide objects there. %

\begin{figure}[t!]
    \begin{minipage}{.6\linewidth}
        \centering
        \includegraphics[width=\linewidth]{imgs/spoofing/helios_hfa.pdf}
        \caption{HFR attack effect on Helios~\cite{Helios}. Our spoofer cannot cover the entire FOV, but can still be effective in the center area.}
        \label{fig:helios_hfa}
   \end{minipage}\hspace{0.03in}
    \begin{minipage}{.37\linewidth}
        \centering
        \vspace{-0.1in}
        \includegraphics[width=\linewidth]{imgs/spoofing/xt32_100pt_injection.png}
        \caption{Spoofed points on XT32~\cite{XT32}.}
        \label{fig:xt32_100pt_injection}
    \end{minipage}
\end{figure}





\nsubsection{Simultaneous Firing on OS1-32 and VLS-128}

Fig.~\ref{fig:sim_firing} shows the spoofed points on OS1-32~\cite{OS1-32} and VLS-128~\cite{VLS128}.%
As shown, OS1-32 fires and scans 32 lasers vertically, and thus we can only move the depth of each vertical line with 32 points simultaneously.
VLS-128 shoots 8 lasers based on a predefined pattern, and thus we can only simultaneously change the depth of each group consisting of 8 points.
As we do not have the capability to selectively return a laser to each simultaneous laser, we are not able to achieve the CPI attack capability as discussed in~\S\ref{sec:sec_enchance_feats}.

\begin{figure}[h!]
\centering
\includegraphics[width=\linewidth]{imgs/spoofing/sim_firing_os1_vls128.pdf}
\caption{Spoofing results for OS1-32 and VLS-128.
}
\label{fig:sim_firing}
\end{figure}

\nsection{Taxonomy of 3D Object Detectors}
\label{appndix:obj_detector}


\textbf{Voxel-based Methods.}
Voxel-based method is a very early, but still dominant approach~\cite{lang2019pointpillars, zhou2018voxelnet, shi2020points, yan2018second}. To deal with the irregular structure of point clouds, this approach aggregates points into 3D voxels to make the CNN work effectively. 
PointPillars~\cite{lang2019pointpillars} is the most widely used in autonomous driving systems such as Baidu Apollo~\cite{apollo} and Autoware~\cite{autoware} because it can achieve higher throughput by constructing voxels as pillars perpendicular to the grounds although the z-axis resolution becomes coarse. In~\S\ref{sec:od_injection}, we find that the voxel-based method is slightly less robust than the other methods to the injection attacks as it cannot recognize the detailed geometry of points.
This design allows to achieve of higher throughput and is widely used in autonomous driving systems such as Baidu Apollo~\cite{apollo} and Autoware~\cite{autoware} since the z-axis of an object is typically unimportant in ground vehicles. Part-A$^2$~\cite{shi2020points} achieves a higher attack success rate than the above methods by leveraging multi-scale voxelization and anchor-based strategies. 


\textbf{Point-based Methods.}
Point-based methods directly handle point cloud without voxelization. To efficiently handle point clouds, this approach utilizes permutation-invariant operators. PointRCNN~\cite{shi2019pointrcnn} is a two-stage method inspired by FastRCNN~\cite{girshick2015fast} in 2D object detection. PointRCNN generate 3D proposals and point features with PointNet backborns~\cite{qi2017pointnet++}.  3DSSD~\cite{yang20203dssd} is a single stage method inspired by SSD~\cite{liu2016ssd} in 2D object detection. By leaving only representative points in the downsampling, 3DSSD removes the feature propagation layer and achieves higher throughput than two-stage method.





\textbf{Point Voxel-based Methods.}
Point voxel-based method~\cite{shi2020pv, chen2019fast} is a hybrid approach of the voxel-based and point-based  methods. PV-RCNN~\cite{shi2020pv} is a two-stage method that uses the voxel-based method in the first stage (3D proposal generation) and the point-based method in the second stage (regional refinement). In~\S\ref{sec:od_injection}, PV-RCNN~\cite{shi2020pv} shows the highest robustness to the injection attacks. The second stage with PointNet backbone, which is not in 3DSSD, enables to obtain more detailed point geometry.




\nsection{Impact of Pulse Frequency and Laser Drive Voltage on HFR Attack} \label{appndix:hfr_freq}
For HFR attack, the higher the attack pulse frequency is, the more effective the point removal attack capability should be since this can make it more likely for the attack pulse to (1) hit the laser receiving window on the victim LiDAR side to affect the legitimate point measurement; and (2) push the attack-induced random position measurements to be within MOT (and thus become undetectable).
However, highly-frequent laser firing makes the temperature of LD high and thus results in the degradation of laser intensity, which affects the spoofing capability.
We thus experimentally study this trade-off. As shown in Fig.~\ref{fig:relationship_freq_pts}, when we increase the frequency, the number of removed points peaks at $\sim$1 MHz and decreases after that. Thus, we use 1 MHz as the default attack laser frequency in our other experiments.

Meanwhile, the attack laser intensity at the firing time also practically affects the attack effectiveness, since a lower one is not able to ensure that the attack laser received at the victim is stronger than the legitimate one. To understand this, we also vary the attack laser drive voltage in our experiments. Since VLP-16 has $\sim$30V laser drive voltage, we vary the attack laser drive voltage from 40 to 80 V (80V is the maximum possible one in our setup). As shown in Fig.~\ref{fig:relationship_freq_pts}, the number of removed points monotonically decreases with the voltage value. Thus, we use 80V as the default voltage in our experiments.

\begin{figure}[t!]
\centering
\includegraphics[width=0.8\linewidth]{imgs/spoofing/relationship_freq_pts2.pdf}
\vspace{-0.1in}
\caption{The relationship between the attack pulse frequency and the removed points by HFR attack. 
}
\label{fig:relationship_freq_pts}
\end{figure}

        







\nsection{Criteria to Count the Number of Removed and Injected Points} \label{appndix:count_method}
To quantitatively evaluate the attack performance, we design a systematic method to count the number of removed and injected points by attacks. We conduct all indoor experiments 5 m away facing the wall of the room, i.e., the legitimate reflection should be from 5 m away.
For point injection attacks, we judge whether a point is spoofed or not based on its intensity. When the intensity scale is 0 to 255, The intensity of reflections from the walls of the room is typically below 70. On the other hand, the attack laser has more than 200 as we directly shoot lasers at the LiDAR without any reflections.
We thus used 80 as the threshold for indoor experiments to differentiate whether legitimate or injected points. For the outdoor experiments, we selected an adequate threshold from 80 to 150 based on the maximum intensity in a benign point cloud.
For point removal attacks, we also utilize the intensity of the point. We first identify the attack-induced spoofed points with the same threshold of intensity. Next, we subtract the remaining points in the attacked point cloud from the benign point cloud; now the remaining points in the benign point cloud are thus the removed benign points by the attack. To calculate the point removal percentage of each azimuth, we split the area into pies corresponding to each azimuth and apply the same calculation.
Based on our measurement, our method can capture 96\% of injected or removed points, i.e., 4\% of injected or removed points could be missed.


