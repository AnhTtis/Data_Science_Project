\nsection{Conclusion} \label{sec:conclusion}

In this work, we conduct the first large-scale measurement study on LiDAR spoofing attack capabilities on object detectors with 9 popular LiDARs, covering both first- and new-gen LiDARs, and 3 major types of object detectors trained on 5 different datasets. To facilitate the measurements, we (1) identify spoofer improvements that significantly improve the latest spoofing capability, (2) identify a new synchronized object removal attack (HFR)), and (3) perform novel mathematical modeling for both object injection and removal attacks. Through this study, we are able to uncover a total of 15 novel findings, including not only completely new ones due to the measurement angle novelty, but also many that can directly challenge the latest understandings in this problem space. We discuss defenses based on the findings. We hope that our findings can inspire and facilitate future security research on LiDAR spoofing, especially those targeting safety-critical application domains such as autonomous driving.




    

