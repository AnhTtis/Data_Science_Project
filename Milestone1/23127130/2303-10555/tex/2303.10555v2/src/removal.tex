\nsection{Object Removal Attack Measurements} \label{sec:removal_attack}

After the measurements on object injection attacks, in this section we measure the other important class of LiDAR spoofing attacks: object removal attacks (\S\ref{sec:attack_taxonomy}). 

\nsubsection{LiDAR-Level Measurements (RQ1, RQ2)}\label{sec:lidar_removal}

Table~\ref{tbl:hfa} shows the LiDAR-level attack capability measurement results for the PRA attack and the newly-identified HFR attack (\S\ref{sec:high_freq_attack}) on different LiDARs. To count the removed points, starting from the attacked point cloud, we first identify the attack-induced spoofed points using the same method as in \S\ref{sec:lidar_injection} and remove them. Next, we subtract the remaining points in the attacked point cloud from the benign point cloud; now the remaining points in the benign point cloud are thus the removed benign points by the attack (detailed in Appendix~\ref{appndix:count_method}).
We select 1 MHz as the pulse frequency and 80V as the laser drive voltage for HFR (detailed in Appendix~\ref{appndix:hfr_freq}).



As shown in Table~\ref{tbl:hfa}, with our spoofer improvements in~\S\ref{sec:spoofer_design}, PRA can now achieve over 6,600 removed points on VLP-16, which is 65.5\% more than the original paper~\cite{cao2023you}. However, such strong attack capability is limited to the first-gen LiDARs; for the new-gen ones, PRA is not applicable to any of them since its requirement of synchronization is directly foiled by common next-gen LiDAR features such as timing randomization and pulse fingerprinting (\S\ref{sec:sec_enchance_feats}). 

\begin{observation}{RQ1}
Due to the requirement of synchronization, the basic design assumption of the latest object removal attack is generally broken for next-gen LiDARs due to common features such as timing randomization and pulse fingerprinting. 
\end{observation}

\begin{figure}[t!]
\centering
\vspace{0.1in}
\includegraphics[width=\linewidth]{imgs/spoofing/HFR_attack_indoor_demo.pdf}
\caption{HFR attack effect on VLP-32c. Patterns of a person and the majority of the room wall are completely removed. 
}
\label{fig:removal_attack}
\end{figure}

Since there is no need of synchronization, as shown in Table~\ref{tbl:hfa}, the newly-identified HFR attack can be generally applied to all LiDARs in the table, no matter if the targeted LiDAR is first- or new-gen. Specifically, for the first-gen ones, HFR is able to achieve comparable point removal capability to PRA, e.g., removing $>$5,300 points in $>$85$^{\circ}$. Fig.~\ref{fig:removal_attack} shows an example of the HFR attack effect on VLP-32c. As shown, the point cloud patterns of a person and the majority of the room wall are completely removed, with only some points with random noise patterns left.
Compared to PRA, the point removal success rate of HFR is slightly slower (72-78\% versus 80-97\% in PRA), which is expected since PRA has more precise control of points with synchronization. 

For next-gen LiDARs with timing randomization, HFR is still highly-effective since the high-frequency lasers can still hit all legitimate measurements within the affected azimuthal range despite the randomized timing of legitimate laser firing, showing the capability of removing 4k-206k points (for OS1-32 it is only 28 due to the lack of laser diodes with matching wavelength, explained in~\S\ref{sec:sec_enchance_feats}). For Helios, the point removal success rate is lower mainly due to its extra-wide vertical FOV (Appendix~\ref{sec:case_helios}); in the vertical range critical to real-world attack scenarios (e.g., 30-40$^{\circ}$ in the center), the success rate is similar to others (Fig.~\ref{fig:hfa_asr}). For next-gen LiDAR with pulse fingerprinting (XT32), the point removal capability is significantly reduced due to the difficulty for the attack pulses to coincide with the fingerprinting interval. However, as we found in~\S\ref{sec:sec_enchance_feats}, there are still chances to randomly remove $\sim$100 points by using a pulse frequency that is most likely to coincide with the fingerprinting interval.


\begin{observation}{RQ2}
\label{finding:hfr}
For new-gen LiDARs, although they are no longer generally vulnerable to the latest practical attacks such as PRA, they unfortunately still remain generally vulnerable to object removal attacks, with a similar level of practical attack capabilities as PRA, due to the possibility of new asynchronized removal attack designs such as HFR.
\end{observation}










\begin{table}[t!]
\centering
\footnotesize
\setlength{\tabcolsep}{1.2pt}
\setlength{\aboverulesep}{0pt}
\setlength{\belowrulesep}{0pt}
\renewcommand{\arraystretch}{0.9}
\caption{
LiDAR-level attack capability measurements for object removal attacks. PRA~\cite{cao2023you} is only applicable to the first-gen LiDARs. Symbols are the same as in Table~\ref{tbl:distance}.
}
\label{tbl:hfa}
\begin{tabular}{cc|cc|cccc|c}
\toprule
 &       & \multicolumn{2}{c|}{\multirow{2}{*}{First-Gen}} & \multicolumn{5}{c}{New-Gen}  \\  \cline{5-9}
 &       &  & & \multicolumn{4}{c|}{w/ Timing Randomization}    & w/ Fingerprint  \\  \cline{3-9}
 &       & VLP-16  & VLP-32c & OS1-32 & Helios    & Horizon  & L515    & XT32  \\  
 \hline  \hline
\multirow{3}{*}{\begin{tabular}[c]{@{}c@{}}PRA\\ \cite{cao2023you} \\ \end{tabular}} & $\mathcal{N}$     & 6,621    & 9,711    & N/A      & N/A       & N/A       & N/A & N/A     \\
                     & $\mathcal{R}$     & 96.9\% & 82.9\% & N/A      & N/A       & N/A       & N/A  & N/A    \\
                     & $\theta$ & 85.4$^{\circ}$ & 73.2$^{\circ}$ & N/A      & N/A       & N/A       & N/A & N/A     \\ \hline
\multirow{3}{*}{\begin{tabular}[c]{@{}c@{}}HFR\\(\S\ref{sec:high_freq_attack})\end{tabular}} & $\mathcal{N}$     & 5,358    & 8,778    & 28 & 4,108    & 19.2k    & 206k   & 113   \\
                     & $\mathcal{R}$     & 78.1\%  & 72.2\% & 43.8\%  & 24.8\%  & 79.9\%   & 91.3\%   & 2.1\% \\
                     & $\theta$ & 85.8$^{\circ}$ & 76.0$^{\circ}$ & 0.72$^{\circ}$  & 103.4$^{\circ}$  & 81.7$^{\circ}$ & 70.0$^{\circ}$ & 34.2$^{\circ}$ \\ \toprule
\end{tabular}
\raggedright

* N/A: Attack is not applicable to the LiDAR\\
\end{table}





\nsubsection{Object Detector-Level Measurements (RQ3)} \vspace{0.1in}

\nsubsubsection{Modeling of the Spoofing Attack Capability in Point Removal} \label{sec:od_removal_modeling}
Similar to the object injection attack side, we first perform a mathematical modeling of the spoofing attack capabilities in point removal in order to enable large-scale object detection-level attack capability measurements. For removal attacks, the point removal goal is the same for all point measurements that can be hit by the attack laser (e.g., move to within MOT for PRA, or place the point to a random position for HFR). Thus, the main factor that can affect the removal capability is whether the point is at an azimuth angle that can be effectively hit by the attack laser. For example, Fig.~\ref{fig:hfa_asr} shows the point removal percentage for the azimuth angles under attack. As shown, for VLP-16 under both PRA and HFR, the points in the center of the attack laser-hit azimuth range are almost 100\% removed regardless of altitudes and distances, while the removal percentages symmetrically decrease from the center to the side.





Based on these observations, we model the attack capability for point removal $\mathcal{P}_R$ as follows:
\vspace{-0.06in}
\begin{align}\footnotesize
    \mathcal{P}_R (x_{ij}) :=
    \begin{cases}
    \xi \cdot g(x_{ij}) & {\rm if } \  \operatorname{Bernoulli}(p_j) = 1 \\
    x_{ij} & {\rm othewise } 
  \end{cases}
     \label{eq:attack_modeling_removal}
\end{align}\vspace{-0.0in}\normalsize, where $x_{ij} \in \mathbb{R}^{3}$ is its point at $i$-th altitude and $j$-th azimuth. For each point, we decide whether it will be removed or not based on a Bernoulli trial with a per-azimuth probability $p_j$, which can be set using the measured point removal percentage as in Fig.~\ref{fig:hfa_asr}. $\xi$ models the removal effects for both HFR and PRA. For HFR, 
$\xi$ is the error of the HFR attack, which distributes the original points to random locations along with the laser direction $g(x_{ij})$. 
Thus, $\xi$ can be written as 
$\xi := \mathcal{U} \left ( 0, \frac{c}{2} \cdot \frac{1}{f} \right )$, where $\mathcal{U}(a, b)$ is the uniform distribution with the minimum $a$ and maximum $b$ values, $c$ is the speed of light, $f$ is the frequency of HFR attack pulses. This is derived from the fact that the maximum ToF is capped by the time interval of the high-frequency attack pulses in HFR. For PRA, $\xi$ is set to 0 since the point removal principle of PRA is to move the points toward the origin as much as possible to locate them within MOT (\S\ref{sec:attack_taxonomy}). 
After applying Eq.~\ref{eq:attack_modeling_removal}, if $\mathcal{P}_R (x_{ij})$ is within the MOT or outside of the maximum detection range (Table~\ref{tbl:target_lidars}), point $x_{ij}$ is removed from the point cloud. To the best of our knowledge, this is the first mathematical modeling of the point removal capability for LiDAR spoofing attacks.



\begin{figure}[t!]
\centering
\vspace{-0.1in}
\includegraphics[width=\linewidth]{imgs/simulator/angle_asr.pdf}
\vspace{-0.3in}
\caption{Point removal percentage of PRA~\cite{cao2023you} and the HFR attack (\S\ref{sec:high_freq_attack}) for the azimuth angles under attack.%
}
\label{fig:hfa_asr}
\end{figure}


\vspace{0.1in}
\nsubsubsection{Model-Level Attack Capability Measurement}
\label{sec:od_removal}
~




\textbf{Experimental Setup.}\label{sec:model_level_scenario_removal}
We use the same evaluation scenarios as in~\S\ref{sec:model_level_scenario} (i.e., 15 scenarios varying the distance between the victim to the vehicle object from 0m to 14m). We consider the object removal as successful if the IoU between any detected objects and the ground truth object is 0. When applying $\mathcal{P}_R(.)$, we set $p_j$ using the measured point removal percentages for different LiDARs as in Fig.~\ref{fig:hfa_asr}. Note that for Helios, since its vertical FOV is much wider than others (Appendix~\ref{sec:case_helios}), we only calculate the point removal percentage for the FOV range related to our attack scenario (i.e., 33$^{\circ}$, which is enough to cover the height of the front vehicle to remove).






\label{sec:od_removal_eval}
\textbf{Results.} Fig.~\ref{fig:removal_noise_data} shows the object removal attack success rates for object detectors with different architectures and datasets. As shown, for the majority of the cases, HFR can reach similar success rates as PRA, which thus further extends our Finding~\ref{finding:hfr} to the object detector level. As a validation of such attack effectiveness in the physical world, Fig.~\ref{fig:hfa_real_car} shows the object removal attack effect against real vehicles using the HFR attack. As shown, the HFR attack is found to cause  5 front vehicles to become undetected by the Apollo model~\cite{apollo}, with 100\% success rate for over 10 seconds. In the figure, we can see the spatial features of the vehicles were completely destroyed (with some random points left) and thus no objects were detectable in the attacked region.

For new-gen LiDAR features, similar to the object injection side, timing randomization (Helios in Fig.~\ref{fig:removal_noise_data}) can significantly reduce the attack success rate in the majority of the cases (by 35\% on average from VLP-16). However, pulse fingerprinting shows much higher defense effectiveness compared to the object injection side: with the pulse fingerprinting strength level of XT32 (i.e., $\sim$100 randomly removed points), the average attack success rate is significantly reduced by 63\% on average from VLP-16, while such a reduction is 3\% on average on the object injection side (Fig.~\ref{fig:obj_fingerprinting}). 
\newpart{This is due to the trade-off of the object detector’s sensitivity to object injection and removal attacks. Since fingerprinting only allows a random subset of 100 points to be injected, the object detectors that are more vulnerable to object injection are those that are very sensitive to even a heavily-occluded point cloud pattern with a very small number of object points.}

\begin{observation}{RQ3}
Both timing randomization and pulse fingerprinting show high defense capabilities against object removal attacks in general; the defense capability from pulse fingerprinting is especially strong when compared with that against object injection attacks.
\end{observation}

In addition, similar to our findings on the object injection attack side, the model robustness to object removal attacks shows a high diversity across different model architectures and different training datasets, which is especially prominent across training datasets. Interestingly, the model robustness properties are generally the \textit{opposite} across object injection and removal attacks: for example, the models trained on Waymo and by Apollo are found to be much more robust against object injection attacks when compared to those trained on Lyft and nuScenes (\S\ref{sec:od_injection}), while the former ones become much less robust than the latter ones for object removal attacks. This might be because different dataset has different balances of the trade-off between false positives and more false negatives for the corner cases. As shown in Fig.~\ref{fig:obj_fingerprinting}, models trained on Lyft and nuScenes can have 100\% detection rate of a vehicle even when the point cloud is randomly downsampled to as few as 10 points. This can allow the trained models to have a very low false negative rate even for highly-occluded legitimate vehicle point cloud patterns, which can thus make the models highly robust to object removal attacks, but this also makes them highly vulnerable to object injection attacks.

\begin{observation}{RQ3}
The model robustness to object injection and removal attacks can be highly diverse when trained on different datasets. The choice of training datasets can incur large trade-offs between the model robustness to object injection attacks and that to object removal attacks.
\end{observation}









\begin{figure}[t!]
\centering
\includegraphics[width=\linewidth]{imgs/obj_detector/removal_noise_data_bar2.pdf}
\vspace{-0.2in}
\caption{
Object removal attack success rates of HFR and PRA for (a) 5 different models trained on the KITTI dataset and (b) PointPillars trained on 5 different datasets. %
}
\label{fig:removal_noise_data} \label{fig:removal_noise_arch}
\end{figure}



\label{sec:case_study}


\begin{figure}[t!]
\centering
\includegraphics[width=\linewidth]{imgs/case_study/hfa_real_car2.pdf}
\caption{Object removal attack effect against real vehicles using the HFR attack. The 5 front vehicles become undetected with a 100\% success rate for over 10 seconds (100 frames in total) by PointPillars~\cite{lang2019pointpillars} in Apollo~\cite{apollo}.
}
\label{fig:hfa_real_car}
\end{figure}



\nsubsubsection{System-Level Attack Capability Measurement}
\label{sec:od_removal_system}
Due to the downstream components such as object tracking in real-world applications such as autonomous driving (AD), object detector-level object removal effect may not directly imply system-level attack effect such as vehicle crashes~\cite{jia2020fooling, shen2022sok}. Thus, in this section we further measure the attack capabilities of different object removal attacks at the system level, with the focus on the representative application: AD.

 

\textbf{Experimental Setup.} \label{sec:system_level_scenario}
We use a common setup widely used in previous work~\cite{cao2019adversarial, jiachen2020towards, hallyburton2022security, cao2023you}. For the AD system, we use Baidu Apollo 7.0~\cite{apollo}. For the driving simulator, we use LGSVL~\cite{lgsvl}. Both systems are known as industry-grade software.
Fig.~\ref{fig:overview_sim} in Appendix illustrates the experiment scenario. 
We place a sedan vehicle as a target object 200 m away from the victim AD and make the victim AD drive toward the target. We add up to 1 m random perturbation both laterally and longitudinally to (1) the starting point of the victim and (2) the target object. Before reaching the target object, the AD vehicle reaches and keeps 40 km/h. 
We set an attack start distance $\mathcal{D}$ and only start to apply the simulated removal attack effect when the distance from the victim AD vehicle to the front vehicle is $\leq\mathcal{D}$.
We use the collision rate over 10 trials as the evaluation metric, i.e., how many times the victim vehicle collides with the sedan out of 10 trials.






\textbf{Results.}
Table~\ref{tbl:sim_hfa} shows the collision rate over 10 trials for different LiDARs. As shown, although HFR has relatively weaker attack capabilities than PRA at raw point cloud (\S\ref{sec:lidar_removal}) and object detector (\S\ref{sec:model_level_scenario_removal}) levels due to the lack of synchronization, their attack capabilities at the system level are almost the same at every attack start distances $\mathcal{D}$, and reach 100\% collision rate when $\mathcal{D}$ is over 18 m. However, due to the requirement of synchronization, PRA is only applicable to first-gen LiDARs such as VLP-16, while HRA can be generally applied to new-gen ones such as Helios with similar attack effectiveness. Fig.~\ref{fig:sim_attack} shows an example attack trial under the HFR attack with $\mathcal{D}$ = 15 m on Helios. The target sedan can be momentarily detected before the collision, but the detection disappears soon and the sedan becomes undetected until and also after the collision.

For next-gen LiDAR with pulse fingerprinting (XT32), although HFR can still have $\leq$32\% model-level attack success rate (\S\ref{sec:od_removal_modeling}), it cannot achieve any system-level attack success at all (0\%), which is likely because the object removal rate is not high enough to fool object tracking~\cite{jia2020fooling}. This suggests that pulse fingerprinting can be quite effective in defending against object removal attacks at the system level. videos of these experiments can be found at our website~\cite{project_page}.









\begin{observation}{RQ3}
Although the HFR attack has weaker attack capabilities than PRA at point cloud and object detector levels, the system-level attack effect is almost the same. Since the HFR attack does not require synchronization, this means that real-world LiDAR applications such as autonomous driving, no matter if using first- or new-gen LiDARs, are facing practical threats from object removal attacks. Meanwhile, pulse fingerprinting can be quite effective in defending against the removal attack effect at the system level.
\end{observation}
\vspace{-0.13in}






\begin{figure}[t!]
\centering
\includegraphics[width=\linewidth]{imgs/simulator/sim_attack_case.pdf}
\caption{An example HFR attack trial with $\mathcal{D}$ = 15 m on Helios. The remaining points are occasionally detected as an object but are not sufficient to avoid a collision.}
\label{fig:sim_attack}
\vspace{-0.05in}
\end{figure}


\begin{table}[t!]
\centering
\footnotesize
\caption{Vehicle collision rates over 10 trials using PRA and HFR for different LiDARs with varying attack start distances. \textbf{Bold} and \underline{underline} highlight %
100\% and 0\% collision rates.}
\label{tbl:sim_hfa}
\setlength{\tabcolsep}{3.4pt}
\renewcommand{\arraystretch}{0.8}
\begin{tabular}{cccccccccc}
\toprule
      &       & Benign & 10m & 15m & 16m & 17m  & 18m  & 19m  & 20m  \\ \hline
PRA & VLP-16 & \underline{0/10}   & \underline{0/10} & 5/10 & 8/10 & 9/10 & \textbf{10/10} & \textbf{10/10} & \textbf{10/10} \\\hline
   & VLP-16 & \underline{0/10}   & \underline{0/10} & 6/10 & 7/10 & 8/10  & \textbf{10/10} & \textbf{10/10} & \textbf{10/10} \\
HFR & VLP-32c & \underline{0/10}   & 1/10 & 9/10 & 8/10 & \textbf{10/10} & \textbf{10/10} & \textbf{10/10} & \textbf{10/10} \\
   & XT32 & \underline{0/10}   & \underline{0/10} & \underline{0/10} & \underline{0/10} & \underline{0/10}  & \underline{0/10}  & \underline{0/10}  & \underline{0/10}  \\ 
   & Helios & 	\underline{0/10} & 	\underline{0/10} & 	6/10 & 	5/10 & 	\textbf{10/10} & 	\textbf{10/10} & 	\textbf{10/10} & 	\textbf{10/10}\\
\toprule
\end{tabular}
\end{table}

