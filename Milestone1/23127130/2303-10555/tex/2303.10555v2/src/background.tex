\nsection{Background and Related Works} \label{sec:background}
\vspace{0.1in}


\nsubsection{LiDAR Basics and Recent Trend}
\label{sec:lidar_basic}
\label{sec:lidar_tax}
LiDAR is an active sensor that can obtain high-resolution 3D point cloud data of the surrounding environment. The most common type of LiDAR today is direct time-of-flight (dToF) LiDAR, which fires laser pulses to the environment and uses their reflections to actively measure (or ``scan'') the surrounding objects. Specifically, for each laser pulse, it uses the time difference between the firing and reflection to obtain the 3D position of a ``point'' on the object surface. By performing laser firing to different horizontal angles (\textit{azimuth}) and vertical angles (\textit{altitude}), the point measurements thus form a ``point cloud'' (Fig.~\ref{fig:arbitrary_point}). Such LiDARs typically use laser peak power (e.g., $\sim$10 W) significantly higher than sunlight even after a long flight, which allows them to detect objects more than 200 meters away even under high ambient lights.










\textbf{Recent Trend: New-Gen LiDARs.} Early Velodyne LiDARs such as VLP-16~\cite{VLP16} and VLP-32c~\cite{VLP32c} are commonly known as the first-generation (\textit{first-gen}) LiDARs~\cite{yoshioka2022tutorial}, which naively integrate the classic point-wise laser ranging systems. The advent of first-gen LiDARs has significantly boosted critical real-world applications such as autonomous driving (AD), but its complex mechanical design increases costs and limits scalability.
To overcome the limitations, the new-generation (\textit{new-gen}) LiDARs~\cite{yoshioka2022tutorial} mount all components, such as the photodetector and the readout circuitry, on a single chip (called a system-on-chip (SoC) approach). This not only reduces the cost and improves the scalability of the system, but also allows new designs of laser firing and receiving.  %
For example, microelectromechanical systems (MEMS) LiDAR~\cite{wang2020mems} move a mirror to scan a wide azimuth range instead of mechanical rotation. Flash LiDARs~\cite{roriz2021automotive, ousterflash} fire a broad laser that covers the entire field of view (FOV) and calculates the distance by compensating its weak return laser by accumulation. 
Moreover, the SoC approach allows more complex signal processing, such as a large number of simultaneous laser firings~\cite{OS1-32}, laser timing randomization~\cite{OS1-32, Helios}, and fingerprinting~\cite{XT32}, to be robust against challenging environments, e.g., multiple LiDARs operating adjacent to each other.

All in all, the recent new-gen LiDARs substantially improved the electronics and the scanning mechanisms even though they still have the same basic design: laser firing and receiving. However, none of the prior works on LiDAR spoofing attacks have studied the security property of such new-gen LiDARs; \textit{all} of them are performed on only the first-gen rotational ones, with a predominate focus on in fact \textit{only 1 specific LiDAR model}: VLP-16~\cite{shin2017illusion, cao2019adversarial, jiachen2020towards, hallyburton2022security, cao2023you}. The major design differences between first- and new-gen LiDARs are likely to cause significant differences in their security characteristics, which thus motivates this study.









\nsubsection{Object Detection on Point Cloud} \label{sec:3d_obj}

In real-world applications such as autonomous driving (AD), object detection is no doubt one of the most critical roles for LiDAR. As in other computer vision areas, 3D object detection on point cloud is substantially benefited by the recent progress of DNN. However, traditional DNN architectures such as CNN cannot be directly applied due to the irregular and unordered structure of point clouds~\cite{li2018pointcnn}. To handle such data complexity, 3 major types of 3D object detection methods are widely adopted: \textit{voxel-based}, \textit{point-based}, and \textit{point voxel-based} methods~\cite{qian2022object}. More details are in Appendix~\ref{appndix:obj_detector}.

\nsubsection{LiDAR Spoofing Attacks against Object Detection} \label{sec:lidar_spoofing_attack}

Due to the basic sensing mechanism of laser firing and receiving, LiDARs are fundamentally vulnerable to malicious laser shooting, or \textit{LiDAR spoofing attacks}.
Specifically, such attacks work by using an external attack device (``spoofer'') to fire laser pulses back to the victim LiDAR in order to manipulate the time measurements of the laser-receiving events and thus the corresponding 3D position measurements (\S\ref{sec:lidar_basic}). Such point position manipulations can thus be strategically used to achieve attack effects on the 3D object detector side, more specifically (1) \textit{Object injection}, e.g., by moving the positions of a set of points to form a 3D pattern to cause a false-positive detection of a non-existing object; and (2) \textit{Object removal}, e.g., by moving the points originally on an object to elsewhere to cause false-negative detection of such an object.






\nsubsubsection{Attack Taxonomy}
\label{sec:attack_taxonomy} 



\begin{table}[t!]
\centering
\footnotesize
\caption{Taxonomy of existing LiDAR spoofing attacks against 3D object detection. Sync. and Async. refer to synchronized and asynchronized spoofing techniques (\S\ref{sec:lidar_spoofing_attack}).}
\label{tbl:tax_attacks}
\setlength{\tabcolsep}{5.6pt}
\renewcommand{\arraystretch}{0.9}
\begin{tabular}{ccc}
\toprule
Spoofing & \multicolumn{2}{c}{Object Detection Model-Level Attack Effect} \\\cline{2-3} 
Technique & Object Injection & Object Removal \\\hline
\begin{tabular}[c]{@{}l@{}}\ \ \ \ \ Sync.\\(White-box)\end{tabular}   & 
\begin{tabular}[c]{@{}l@{}}\hspace{3em}Adv-LiDAR~\cite{cao2019adversarial},\\ Occlusion~\cite{jiachen2020towards}, Frustum~\cite{hallyburton2022security}\end{tabular}
        & \hspace{0.5em}PRA~\cite{cao2023you}, ORA~\cite{petit2015remote}       \\
\hdashline
\begin{tabular}[c]{@{}l@{}}\ \ \ \ Async.\\(Black-box)\end{tabular}  & Relay$^{*}$~\cite{petit2015remote}      & \begin{tabular}[c]{@{}l@{}}\hspace{2em}Saturating$^{\dag}$~\cite{shin2017illusion},\\ \hspace{2em}{HFR (\S\ref{sec:high_freq_attack})}\end{tabular}           \\
 \toprule
\end{tabular}
\raggedright

$^{*}$ Cannot inject fake objects (1) closer than the spoofer itself, and (2) in a different angle towards the victim than the spoofer itself.\\
$^{\dag}$ Only show removal of a 41$\times$42 cm$^2$ metal plate in short duration (4 sec)
\end{table}
Table~\ref{tbl:tax_attacks} shows a taxonomy of existing LiDAR spoofing attacks on object detection based on the LiDAR spoofing technique (specifically, the requirement of \textit{synchronization}) and the targeted object detector-level attack effect.
\textit{Synchronization} means to synchronize the malicious laser firing timing with the victim LiDAR scanning (i.e., laser firing) timing, which can enable precise control of the attack laser-receiving timing and thus the corresponding positions of the spoofed points. 
Fig.~\ref{fig:sync_and_async_attack} illustrates the synchronized and asynchronized spoofing techniques with the latest attack capabilities. As shown, to achieve synchronization, the attacker needs to use an extra device (photodetector, or PD, in Fig.~\ref{fig:sync_and_async_attack}) to first learn the current state of the victim LiDAR scanning in the real time, and then use the victim LiDAR's scanning pattern to derive future victim laser-firing timings for synchronized attack laser-firing. More detailed explanations are in Appendix~\ref{appndix:sync_attack}. This process requires precise and predictable knowledge of the victim LiDAR's scanning pattern beforehand, which thus can be viewed as a \textit{white-box} LiDAR attack. Those that do not assume synchronization (i.e., \textit{asynchronized} attacks) can be applied without knowing anything about the LiDAR internal scanning logic, which can be viewed as \textit{black-box} attacks.


\begin{figure}[t!]
\centering
\includegraphics[width=\linewidth]{imgs/methodology/sync_and_async_attack_ndss.pdf}
\caption{
Illustration of the synchronized and asynchronized LiDAR spoofing techniques with the latest attack capabilities. Synchronized spoofing needs white-box knowledge of the victim LiDAR scanning patterns and an extra device (Photodetector, or PD) for synchronization (\S\ref{sec:lidar_spoofing_attack}), while asynchronized spoofing does not need these (i.e., black-box LiDAR attack).
}
\label{fig:sync_and_async_attack}
\end{figure}


\textbf{Spoofing Attacks for Object Injection.} 
The very first LiDAR spoofing attack was discovered in 2015~\cite{petit2015remote}, which found that shooting lasers to a victim LiDAR can inject $\geq$200 points and achieve fake object injection at the object detector side. This attack works by relaying the laser received from the victim LiDAR, which thus does not require synchronization. However, this also makes it impossible to inject objects (1) in closer positions than the spoofer itself, and (2) in a different angle towards the victim than the spoofer. These limitations make it not very useful in real-world attack scenarios since at the time of spoofing, the attackers would hope that the most threatening object to the victim vehicle is the induced fake objects instead of the attacker herself~\cite{shin2017illusion}.

To address such limitations, synchronized spoofing~\cite{shin2017illusion} is proposed to use the \textit{synchronization} process described above and also illustrated in Fig.~\ref{fig:sync_and_async_attack}.
The early-stage attack designs can inject only 10 spoofed points~\cite{shin2017illusion}, but the following works progressively increase the number of spoofed points to 60~\cite{cao2019adversarial} and 200~\cite{jiachen2020towards}, and show that the 60 and 200 spoofed points are enough to cause fake object injection on latest object detectors. After the success, the attack capability of injecting 200 points under the Chosen-Pattern Injection (CPI) assumption (detailed later in \S\ref{sec:cpi}) becomes the \textit{de facto} threat model in the following works~\cite{zhongyuan2021object, hallyburton2022security, hau2021shadow}. However, none of them have (1) demonstrated and quantified the CPI attack capability, and (2) studied their attack effect on new-gen LiDARs, which have new features such as laser timing randomization and pulse fingerprinting that can directly challenge their basic design assumption --- the requirement of synchronization. In this paper, we fill both of these critical research gaps.



\textbf{Spoofing Attacks for Object Removal.} 
With the success to achieve object injection, more recent works start to explore using LiDAR spoofing for object removal. Specifically, saturating attack~\cite{shin2017illusion} is the first to show the object removal effect. In this attack, instead of firing pulsed lasers like the prior works above, it fires a strong \textit{continuous} laser to the victim LiDAR to indirectly cause measurement errors of laser-receiving events. However, due to the requirement of maintaining continuous high-power laser, it is physically difficult for the attack laser beam to (1) achieve a large receiver area coverage at the victim LiDAR side with high intensity, and (2) maintain the attack effect for a long time. For example, the demonstrated object removal effect is only about removing a 41$\times$42 cm$^2$ metal plate, lasting $<$4 seconds, which thus makes it highly limited in real-world attack scenarios such as when attacking AD. 
\newpart{For example, the saturating attack (0.8 W average power) needs roughly 10 times higher average power than the HFR attack (80 W peak power) (\S\ref{sec:high_freq_attack}) to achieve similar attack effectiveness, but such a powerful diode is not publically available so far.}


To address such limitations, similar to the object injection side, more recent works start to leverage synchronized spoofing techniques. Specifically, the ORA attack (object removal attack)~\cite{zhongyuan2021object} found that using the synchronized spoofing capability of injecting 200 points under the CPI assumption as mentioned above, attackers can fool 3D object detectors by strategically injecting spoofed points inside the target object's bounding box, since the point cloud of legitimate objects should have points mostly on the object surface instead of inside. However, such removal effect still depends on the object detector-side vulnerabilities. Most recently, the physical removal attack (PRA)~\cite{cao2023you} found that the object removal effect can be more generally achieved by moving all points on a victim object to within the minimum operational threshold (MOT) of the victim LiDAR, which is common filtering mechanism to automatically discard points below a certain distance. This filtering is implemented in most LiDARs because the LiDAR measurement at a very close distance is generally inaccurate since the ToF can be too short to distinguish from errors due to inadequate calibration. This work shows the capability of removing $\sim$4k points and thus can remove critical road objects such as pedestrians, which is thus the current state-of-the-art in object removal attack capability.

However, similar to the object injection attack side, none of these prior works have considered the new-gen LiDARs that may come with features that can break some of their fundamental design assumptions. In this paper, we find that (1) PRA can no longer be applied to new-gen LiDARs due to the requirement of synchronization (\S\ref{sec:lidar_removal}), and (2) the assumption of non-zero MOT for PRA may not necessarily hold (e.g., for XT32~\cite{XT32}, \S\ref{sec:specific-lidars}). We further identify a new object removal attack adapted from the saturation attack, which does not require synchronization but can still achieve practical object removal effects (can remove $>$5k points in 10$\times$10 m$^2$ area, which is enough for 5 cars as shown later in~\S\ref{sec:od_removal}).














\label{sec:async_attack}







\label{sec:sync_injection_attack}




\label{sec:sync_removal_attack}



\nsubsubsection{Chosen-Pattern Injection (CPI) Attack Capability}
\label{sec:cpi}
So far, all prior works using synchronized spoofing for object injection explicitly or implicitly assume that at the actual attack time the attacker is able to accurately inject a specific spoofed point cloud pattern chosen by the attacker beforehand, e.g., via various kinds of off-line optimization/identification processes~\cite{cao2019adversarial, jiachen2020towards, hallyburton2022security}. In this paper, we thus explicitly define this as the \textit{Chosen Pattern Injection (CPI)} attack capability.

Such an attack capability is theoretically achievable since the synchronized LiDAR spoofing technique should by design be capable of precisely controlling the position of each spoofed point. However, so far no prior works have systematically studied how achievable this is in practice, not to mention to provide a systematic quantification of the pattern control capability, which is necessary for achieving valid security analysis on the object detector side. In this paper, we thus perform the first measurement study to fill this critical gap, with the coverage of both first- and new-gen LiDARs.











\nsubsection{Threat Model} \label{sec:threat_model}

We follow the same threat model as in prior works~\cite{cao2019adversarial, jiachen2020towards, hallyburton2022security}, i.e., the attacker fires malicious lasers from their spoofer to the victim LiDAR (\S\ref{sec:lidar_spoofing_attack}, Fig.~\ref{fig:sync_and_async_attack}).
As described in~\cite{hallyburton2022security}, the spoofer device can be at a front vehicle, vehicle in the next lane, or a roadside in AD scenarios. 


