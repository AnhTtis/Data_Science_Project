\nsection{Measurement Study Setup
} \label{sec:methodology}
\vspace{0.1in}




\nsubsection{Targeted LiDARs}
Table~\ref{tbl:target_lidars} summarizes the LiDARs involved in this measurement study. As shown, we are able to cover 9 popular LiDARs in total, including 3 first-gen ones, 6 new-gen ones, an operating range from 9 m to 300 m, 3 scanning types (Rotating, MEMS, and Flash), and 3 security-related features (simultaneous laser firing, timing randomization, and fingerprinting). This makes this work the largest-scale measurement study on this topic so far in terms of both LiDAR numbers (none of the prior works studied over 1 LiDAR model, while we study 9) and LiDAR types (all prior works concentrate on first-gen ones, and we are the first to study the new-gen ones).


\nsubsection{Targeted Object Detection Models and Datasets}
\label{sec:target_lidar_od}
Our study includes 5 popular DNN-based 3D object detectors in total covering all the 3 major model types (\S\ref{sec:3d_obj}): voxel-based (PointPillars~\cite{lang2019pointpillars}, SECOND~\cite{yan2018second}, and Part-A$^2$~\cite{shi2020points}), point-based (3DSSD~\cite{yang20203dssd}), and point voxel-based (PV-RCNN~\cite{shi2020pv}). In order to study the impact of the training dataset, we also prepare 5 different versions of PointPillars~\cite{lang2019pointpillars} trained on 5 different datasets (KITTI~\cite{Geiger2012CVPR}, Waymo~\cite{Sun_2020_CVPR}, nuScenes~\cite{nuscenes}, Lyft~\cite{lyft} datasets), and the dataset used in Apollo 6.0~\cite{apollo}. This thus makes our measurement also the largest in terms of the coverage of object detection model architectures and training datasets. We directly use pre-training ones from MMdetection3D~\cite{mmdet3d2020}, OpenPCDet~\cite{openpcdet2020}, or the Apollo repository~\cite{apollo}.




\begin{table*}[t!]
\centering
\footnotesize
\caption{The 9 LiDARs involved in our measurement study. 1st-G and New-G: First- and new-generation. FOV: Field of View.}
\label{tbl:target_lidars}
\setlength{\tabcolsep}{1.2pt}
\renewcommand{\arraystretch}{0.75}
\begin{tabular}{llccccccccc}
\toprule
                  &                     & \multicolumn{3}{c}{Velodyne}         & Leddar & Ouster & Intel & Livox & Hesai & Robosense \\ \cline{3-11} 
                  &                     & \includegraphics[width=0.3in]{imgs/LiDARs/vlp16.png}           & \includegraphics[width=0.25in]{imgs/LiDARs/vlp32c.png}           
                  & \includegraphics[width=0.25in]{imgs/LiDARs/vls128.png}     & \includegraphics[width=0.25in]{imgs/LiDARs/pixell.png}
                  & \includegraphics[width=0.2in]{imgs/LiDARs/os1-32.png}      & \includegraphics[width=0.2in]{imgs/LiDARs/L515.jpg}
                  & \includegraphics[width=0.25in]{imgs/LiDARs/horizon.png}    & \includegraphics[width=0.25in]{imgs/LiDARs/xt32.png}       
                  & \includegraphics[width=0.2in]{imgs/LiDARs/Helios.png}
                  \\
                  & & VLP-16~\cite{VLP16}     & VLP-32c~\cite{VLP32c}    & VLS-128~\cite{VLS128}    & Pixell~\cite{pixell} & OS1-32~\cite{OS1-32} & Realsense L515~\cite{L515} & Horizon~\cite{livox_horizon} & XT32~\cite{XT32}  &  Helios 5515~\cite{Helios}     \\ \hline
\multirow{10}{*}{\vspace{0.15in}\rotatebox{90}{\notsotiny{General Specs}}} & Gen. (year)       & 1st-G \notsotiny{(2016)}   &  1st-G \notsotiny{(2017)}   & 1st-G \notsotiny{(2017)}   & New-G \notsotiny{(2019)} & New-G \notsotiny{(2019)} & New-G \notsotiny{(2019)} & New-G \notsotiny{(2020)}  &  New-G \notsotiny{(2020)}  & New-G \notsotiny{(2021)}               \\
                  & Scanning Type       & Rotating      & Rotating      & Rotating      & Flash         & Rotating       & MEMS          & MEMS           & Rotating      & Rotating      \\
                  & Wavelength          & 905 nm        & 905 nm        & 905 nm        & 905 nm        & 865 nm         & 860 nm        & 905 nm         &  905 nm       & 905 nm        \\
                  & Vertical FOV        & 30$^{\circ}$  & 40$^{\circ}$  & 40$^{\circ}$  & 16$^{\circ}$  & 45$^{\circ}$   & 55$^{\circ}$  & 25.1$^{\circ}$ & 31$^{\circ}$  & 70$^{\circ}$  \\
                  & Horizontal FOV      & 360$^{\circ}$ & 360$^{\circ}$ & 360$^{\circ}$ & 180$^{\circ}$ & 360$^{\circ}$  & 70$^{\circ}$  & 81.7$^{\circ}$ & 360$^{\circ}$ & 360$^{\circ}$ \\
                  & Max. Range {[}m{]}  & 100           & 200           & 300           & 56            & 120            & 9             & 260            & 120           & 150           \\
                  & Min. Range {[}m{]}  & 1             & 1             & 0.5           & 0.1           & 0.3            & 0.25          & 0.5            & 0             & 0.2           \\
                  & Vertical Channel    & 16            & 32            & 128           & 8             & 32             & -             & -              & 32            & 32            \\
                  \hline
\multirow{3}{*}{\vspace{-0.08in}\rotatebox[origin=c]{90}{
 \notsotiny{Security}
}}                & Simul. Firing       & 1             & 2             & 8             & 3             & 32             & 1             & 1              & 1             & 1             \\
                  & Timing Random.      &               &               &               & \CheckmarkBold& \CheckmarkBold & \CheckmarkBold& \CheckmarkBold &               & \CheckmarkBold\\
                  & Fingerprinting      &               &               &               &               &                &               &                & \CheckmarkBold&               \\ \toprule
\end{tabular}
\end{table*}



\nsubsection{Targeted LiDAR Spoofing Attacks}\label{sec:target_attacks}

Our study targets spoofing attacks with the latest attack capabilities on both the object injection and removal sides (\S\ref{sec:attack_taxonomy}). Specifically, for object injection, we reproduce the latest synchronized spoofing technique~\cite{cao2019adversarial,jiachen2020towards,cao2023you}, and for object removal, we reproduce the latest physical removal attack (PRA)~\cite{cao2023you}. In addition, since the latest removal attack cannot be applied when synchronization is not possible (which we found to be the case for new-gen LiDARs in general, detailed in~\S\ref{sec:inj_other_lidars}), we further identify and include in our measurement a new asynchronized spoofing technique for object removal, which is adapted from the saturation attack (\S\ref{sec:attack_taxonomy}) but with much more practical object removal capabilities:

\textbf{High-Frequency Removal (HFR): New Asynchronized Spoofing Technique for Object Removal.} \label{sec:high_freq_attack}
As described in~\S\ref{sec:attack_taxonomy}, due to the requirement of maintaining continuous high-power laser, the saturation attack is fundamentally limited in the removal area size and duration (e.g., 41$\times$42 cm$^2$ in $<$4 seconds~\cite{shin2017illusion}), making it highly limited in real-world attack scenarios. Later works address this by using synchronization, but this also makes them inapplicable to new-gen LiDARs (\S\ref{sec:inj_other_lidars}). To address this, we identify a new adaptation of the saturation attack, which still does not require synchronization, but instead of using high-power continuous laser, it uses \textit{high-frequency pulsed laser}. This new attack is thus called \textit{high-frequency removal (HFR) attack}. The key idea is to fire a large number of attack laser pulses to the victim LiDAR at a frequency that is higher than the laser-firing frequency of the victim LiDAR. This allows the attack laser to hit every laser-firing event of the victim LiDAR in the scanning range hit by the spoofer, which can thus achieve the spoofing effect for every points in that range without any knowledge of the victim scanning pattern (i.e., without requiring synchronization). Fig.~\ref{fig:sync_and_async_attack} illustrates the spoofing mechanism, and Fig.~\ref{fig:highfreq_vs_saturating} in Appendix illustrates the difference to the saturation attack. Due to the lack of synchronization, the receiving timing of the attack laser is random, and thus the spoofing effect will be moving each legitimate surface point of the targeted objects to a random position or undetectable area of the victim LiDAR (e.g., within MOT). As shown later in~\S\ref{sec:od_removal}, this can completely destruct the point cloud patterns at the original object position and thus cause the object removal effects.













\nsubsection{LiDAR Spoofer Setup}
\label{sec:optical_setup}
\label{sec:arbitrary_point_injection}
\label{sec:spoofer_design}

Fig.~\ref{fig:spoofer1} shows an overview of the LiDAR spoofer, its optics design, and the setups of the indoor and outdoor experiments to support our measurement study of the targeted spoofing attacks (\S\ref{sec:target_attacks}). We generally follow the setup adopted in the latest works~\cite{cao2019adversarial, jiachen2020towards, hallyburton2022security, cao2023you}, so we leave the details to Appendix~\ref{appendix:spoofer}. 

\textbf{Spoofer Improvements.} When reproducing the prior works, we also implemented several improvements on top of the latest spoofer setup to more properly study the spoofing attack capabilities. For instance, we find that inadequate optical design can significantly affect the number and angle coverage of spoofable points due to undesired diffusion and convergence of the attack laser beam. Specifically, if the laser beam is expanding, the intensity rapidly decays with distance, which thus limits the number of spoofable points; if the laser beam is converging, the diameter of the beam decreases with the distance, which thus makes it difficult to aim and cover a wide receiver angle. Ideally, the laser beam should be collimated without diffusion and convergence to achieve the target LiDAR with a minimum loss. 
To achieve this, we use a 25.4 mm focal length plano-convex lens with 1 inch (25.4 mm) diameter to cover the laser emitted by SPL90\_3~\cite{SPLPL90_3}, which has a maximum beam divergence angle of 25$^{\circ}$. 
Since the diameter of the laser at the focal point is $\tan{25^\circ}\times 25.4 \times 2 = 23.7$ mm, we can cover it with the 1 inch (25.4 mm) lens. 
To precisely calibrate the lens setup, we develop a device that can use a hollow screw to adjust the distance between the laser diode (LD in Fig.~\ref{fig:spoofer1}) and the lens.
Technically, this allows the spoofer to maintain a stably large number and angle coverage of spoofable points from hundreds of meters away. 

Besides the optics side, we also notice that inadequate electronic implementations in the spoofer setup can introduce inaccurate laser detection timing (from LiDAR to PD in Fig.~\ref{fig:spoofer1}) and also long and unstable delays in the signal processing of the spoofer devices, which can thus significantly affect the attack capability in precisely controlling the placement of a spoofed point at a chosen 3D position. To address this, we improve the amplifier for the PD to increase the laser detection accuracy, and also improve the function generator (FG) setup to allow more precise nanosecond-level configuration and calibration. As shown in Fig.~\ref{fig:arbitrary_point} and detailed later in~\S\ref{sec:inj_vlp16}, these improvements significantly increase the latest spoofing attack capabilities in terms of the spoofed point number, angle coverage, pattern control, and robustness over distance. Following the best practices advocated in~\cite{shen2022sok}, we release the raw hardware design file (CAD file) and the bill of materials on our website~\cite{project_page} to make it easier for future works to reproduce and build upon (most recent works did not do this~\cite{cao2019adversarial,jiachen2020towards,cao2023you}).


















\begin{figure}[t!]
\centering
\includegraphics[width=\linewidth]{imgs/methodology/spoofer_design_caption_with_optics.pdf}
\caption{Overview of our LiDAR spoofer setup, the optics design, and the setup of the indoor and outdoor experiments. PD: Photodetector. TIA: Transimpedance amplifier. FG: Function generator. LD: Laser diode. More details in Appendix~\ref{appendix:spoofer}.
}
\label{fig:spoofer1}
\end{figure}













