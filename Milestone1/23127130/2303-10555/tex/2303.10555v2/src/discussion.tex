\nsection{Discussions} \label{sec:discussion}
\nsubsection{Defense Discussions}
\nsubsubsection{Sensor-Level Defenses}
\label{sec:sensor-level-defense}
Through large-scale measurements, we identify the defense capability of popular security-related features in new-gen LiDARs on the spoofing attack capabilities.
Table~\ref{tbl:defence} summarizes our observations regarding their defense effectiveness and limitations \newpart{against object injection and removal attacks}.



\begin{table}[t!]
\centering
\footnotesize
\caption{Defense effectiveness and limitations of security-related features in the new-gen LiDARs \newpart{against object injection and removal attacks}. \textbf{Bold} means desired properties; \underline{Underline} means undesire ones.
}
\label{tbl:defence}
\setlength{\tabcolsep}{3pt}
\setlength{\aboverulesep}{0pt}
\setlength{\belowrulesep}{0pt}
\renewcommand{\arraystretch}{0.9}
\begin{tabular}{ccc|ccc}
\toprule 
                  & \multicolumn{2}{c|}{Effectiveness} & \multicolumn{3}{c}{Limitations}        \\ \cline{2-6} 
Features           & Injection             & Removal           & Eye safety & Latency         & Range$\downarrow$* \\ \hline
Timing Random.    & \textbf{High}                & \textbf{High}                 & \textbf{No risk}    & \textbf{Low impact}      & \textbf{None}     \\
Pulse Fingerprint & Mid                & \textbf{High}                 & \underline{High risk} & Mid impact & \underline{High}  \\ 
Simul. Firing     & \underline{Low}                & \underline{None}                  & \textbf{Low risk}   & \textbf{Low impact}      & \textbf{Low}    \\\toprule
\end{tabular}
\raggedright

* Range$\downarrow$: Degradation of the effective sensing range of LiDAR
\end{table}




\textbf{Defenses against Object Injection Attacks.} 
As shown in \S\ref{sec:od_injection_eval}, the timing randomization feature has high mitigation capabilities, particularly on several models such as the industry-grade Apollo model. 
\newpart{Even with low randomization entropy, the timing randomization shows high defense capabilities. Meanwhile, higher randomization entropy has a higher defense capability. We suggest that the magnitude of the randomization be set as high as possible, e.g., std. dev. $\sigma\geq0.75$ %
}. 
Pulse fingerprinting also has mitigation capabilities but the complexity of fingerprinting in new-gen LiDARs today is not enough to effectively defend against injection attacks.
\newpart{Finally, simultaneous firing also has a defense effect as it can prevent the CPI attack capability (\S\ref{sec:sec_enchance_feats}). However, this alone is unlikely to affect the object injection attack goal in general since the attacker is still able to inject an object point cloud with at least the same number of points as the chosen pattern (but just cannot control the pattern very precisely).}


\textbf{Defenses against Object Removal Attacks.}
As discussed in~\S\ref{sec:od_removal_eval}, both timing randomization and pulse fingerprinting show high mitigation capability against removal attacks, especially pulse fingerprinting in XT32 shows high defense capability even against the HFR attack.
\newpart{The object detectors are typically tuned to avoid false negatives rather than false positives, especially for AD scenarios. This trade-off benefits pulse fingerprinting for defending against object removal attacks because the majority of object points are not compromised.
However, this tuning is a backfire to defend against object injection attacks because it allows a hundred injected points to be detected as an object.
}
However,  the current XT32-level pulse fingerprinting is still not enough to prevent all attacks as the HFR attack still has 32\% attack success rate on the Apollo model (\S\ref{sec:od_removal_eval}).



To improve this further, more complex fingerprint encoding is needed. However, this is fairly non-trivial due to the dilemma between eye safety and detection range.
More specifically, the most direct way to increase the fingerprint coding complexity is to increase the number of pulses for each distance measurement. However,  the laser power per unit time is capped to ensure eye safety. For example, if we shoot $N$ pulses for each point measurement, the power for each pulse should be $1/N$, and it will roughly degrade the effective sensing range by $1/\sqrt{N}$. To address the dilemma, future work can explore: (1) possible fingerprint coding designs both with high complexity and fewer pulses, and (2) use a wavelength to which the human eye is highly resistant, such as 1550 nm wavelength. \newpart{For the simultaneous firing feature, for object removal it does not have any defense capability since it just makes the attack effects on some sub-group of the points in the chosen pattern equal instead of having a removal effect of them.}


\newpart{\textbf{Summary.} Overall, the timing randomization and pulse fingerprinting features show promising defense effectiveness on at least one of the object detector-side attack goals. Particularly, timing randomization has high effectiveness on both attack goals while suffering from the least limitations in eye safety, latency, and sensing range, despite the simplicity of the method.
Thus, we strongly recommend implementing it in future LiDARs as a highly cost-effective measure to improve their resiliency against LiDAR spoofing attacks.}



\nsubsubsection{Software-Level Defenses}

Recently, several software-level defenses have been proposed against injection attacks. CARLO~\cite{jiachen2020towards}, SVF~\cite{jiachen2020towards}, and Shadow-Catcher\cite{hau2021shadow} leverage an assumption that the spoofer cannot directly spoof a point cloud pattern with sufficient number of points and precision to make it indistinguishable from a benign one.
However, we find that this assumption does not hold since our improved spoofer can achieve the CPI attack capability with a much larger number of points that is sufficient to directly spoof the complete point cloud pattern of a near-front vehicle (\S\ref{sec:inj_vlp16}).
Fortunately, VLP-16~\cite{VLP16} is the only LiDAR for which such CPI attack capability is feasible (\S\ref{sec:inj_other_lidars}). Thus, future research can focus more on more recent LiDARs, especially the new-gen ones, to explore software-level defense design possibilities. 
For object removal attacks, we still do not have effective defenses since PRA~\cite{cao2023you} is discovered very recently.
A potential software-level defense for the HFR attack is to detect a unique characteristic of the HFR attack that will cause a randomized point distribution pattern (like salt-and-pepper noise) in the area under attack, which is caused by the attack design (\S\ref{sec:high_freq_attack}) and can rarely occur in benign cases.





\newpart{
\nsubsection{Limitations of Our Study} \label{appendix:limitation}

\nsubsubsection{Aiming at Driving AD Vehicles}
In this study, we do not discuss the deployability of LiDAR spoofing attacks against real-world applications such as AD, especially when the victim AD vehicle is moving in high speed. We focus more on investigating the security properties of different types of LiDARs, rather than the system-level security analysis of autonomous driving in the real world. In a future study, we plan to demonstrate attacks against driving vehicles. %
A recent study~\cite{cao2023you} has demonstrated a system capable of victim LiDAR tracking and spoofer aiming to address this problem.
Therefore, we expect that targeting a driving vehicle is feasible. %

\nsubsubsection{LiDAR Model Coverage}
While we cover popular LiDARs as many as possible with our best efforts, it is infeasible to cover all public LiDARs due to our budget and their supply capacity. For example, we cannot cover 1550 nm LiDARs such as AEye~\cite{Aeye} and Luminar~\cite{Luminar} which utilize unique technology, e.g. adaptive scanning and FMCW ToF, and thus potentially have different spoofing capabilities.
We also cannot cover private LiDARs such as those in Waymo's robotaxi~\cite{waymoone} since they are not publicly available on the market.

\nsubsection{Considerations for Safe Experiments} \label{appendix:safety}
All experiments were conducted in a controlled environment and we wore safety goggles for extra eye safety. Note that the unit-area peak power of our laser is actually weaker than prior works~\cite{cao2023you} as we use a 50\% larger aperture lens. %


}


