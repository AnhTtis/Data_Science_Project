




\documentclass[conference]{IEEEtran}

\pagestyle{plain}
















\ifCLASSINFOpdf
\else
\fi

















































\let\proof\relax
\let\endproof\relax
\let\amalg\relax
\let\labelindent\relax

\usepackage{epsfig,amsmath,amsfonts,multirow,graphicx,makecell,caption,soul,csquotes,color,wrapfig,subcaption,mathtools,bm,spverbatim,booktabs,xcolor,color,amsthm,tcolorbox,wrapfig,enumitem}
\usepackage[e]{esvect}
\usepackage{graphics}
\usepackage{marvosym,listings,etoolbox}
\usepackage{adjustbox}
\usepackage[space]{cite}
\usepackage{subcaption}
\DeclareCaptionSubType * [alph]{table}
\captionsetup[subtable]{labelformat=simple, labelsep=space}
\usepackage{multicol}
\usepackage{lipsum}

\let\Cross\relax
\usepackage{bbding}

\usepackage{array}
\usepackage{longtable}
\usepackage{colortab}
\usepackage{colortbl}
\usepackage{arydshln}


\renewcommand\thesubtable{(\alph{subtable})}

\captionsetup[sub]{labelformat=simple}
\makeatletter
\renewcommand\p@subfigure{\thefigure\,}
\renewcommand\thesubfigure{(\alph{subfigure})}
\makeatother

\usepackage{fancyhdr}
\renewcommand{\headrulewidth}{0pt}
\fancypagestyle{empty}{\fancyfoot[C]{\vspace*{-1.8\baselineskip}\thepage}}
\fancypagestyle{plain}{\fancyfoot[C]{\vspace*{-1.8\baselineskip}\thepage}}


\newcommand{\alfred}[1]{\textcolor{red}{[Alfred: #1]}}

\usepackage[first=0,last=9]{lcg}
\usepackage{colortbl}
\definecolor{Gray}{gray}{0.8}

\usepackage{url}
\def\UrlBreaks{\do\/\do-}
\renewcommand{\UrlBreaks}{\do\/\do-\do:\do\a\do\b\do\c\do\d\do\e\do\f\do\g\do\h\do\i\do\j\do\k\do\l\do\m\do\n\do\o\do\p\do\q\do\r\do\s\do\t\do\u\do\v\do\w\do\x\do\y\do\z\do\A\do\B\do\C\do\D\do\E\do\F\do\G\do\H\do\I\do\J\do\K\do\L\do\M\do\N\do\O\do\P\do\Q\do\R\do\S\do\T\do\U\do\V\do\W\do\X\do\Y\do\Z}
\usepackage{hyperref}
\usepackage{xurl}


\setlength\extrarowheight{2pt}



\captionsetup[table]{format=plain,labelformat=simple,labelsep=period}
\def\tablename{Table}
\def\figurename{Figure}






\makeatletter
\newif\if@restonecol
\makeatother
\let\algorithm\relax
\let\endalgorithm\relax
\usepackage[boxed, ruled, vlined, linesnumbered]{algorithm2e}
\SetKwRepeat{Do}{do}{while}




\setlength{\textfloatsep}{0.25\baselineskip}
\setlength{\floatsep}{0.25\floatsep}
\setlength{\dblfloatsep}{0.25\dblfloatsep}
\setlength{\dbltextfloatsep}{0.25\dbltextfloatsep}
\setlength{\intextsep}{0.25\intextsep}


\setlength{\belowcaptionskip}{3pt}
\setlength{\abovecaptionskip}{3pt}


\newenvironment{changemargin}[2]{\begin{list}{}{
	\setlength{\topsep}{0pt}\setlength{\leftmargin}{0pt}
	\setlength{\rightmargin}{0pt}
	\setlength{\listparindent}{\parindent}
	\setlength{\itemindent}{\parindent}
	\setlength{\parsep}{0pt plus 1pt}
	\addtolength{\leftmargin}{#1}\addtolength{\rightmargin}{#2}
	}\item}
	{\end{list}}

\newenvironment{myitemize}{
	\begin{changemargin}{-3pt}{-0cm}
	\vspace{-10pt}
	\hspace{-5pt}
	\begin{itemize}
	\setlength{\itemsep}{3pt}}
	{\end{itemize}
	\vspace{2pt}
	\end{changemargin}}

\newenvironment{mydescription}{
	\begin{changemargin}{-8pt}{-0cm}
	\vspace{-13pt}
	\hspace{5pt}
	\begin{description}
	\setlength{\itemsep}{-1pt}}
	{\end{description}
	\end{changemargin}}

\newenvironment{myenumerate}{
	\begin{changemargin}{-8pt}{-0cm}
	\vspace{-13pt}
	\hspace{5pt}
	\begin{enumerate}
	\setlength{\itemsep}{1pt}}
	{\end{enumerate}
	\end{changemargin}}

\newcommand{\mypara}[1]{\vspace{-6pt}\paragraph*{#1}}



\newtheorem{definition}{Definition}
\newtheorem{lemma}{Lemma}
\newtheorem{theorem}{Theorem}
\newtheorem{prop}{Proposition}






\renewcommand{\sectionautorefname}{\S}
\newcommand{\msec}[1]{\S\ref{#1}}
\newcommand{\meq}[1]{Eq.\,(\ref{#1})}
\newcommand{\mcite}[1]{~\cite{#1}}
\newcommand{\mref}[1]{\,\ref{#1}}

\newcommand\matt{{\scaleobj{0.8}{\top}}}
\newcommand\mati{{\scaleobj{0.8}{-1}}}
\newcommand\szero{{\scaleobj{0.8}{0}}}




\usepackage{xr}

\makeatletter
\newcommand*{\addFileDependency}[1]{%
  \typeout{(#1)}
  \@addtofilelist{#1}
  \IfFileExists{#1}{}{\typeout{No file #1.}}
}
\makeatother

\newcommand*{\myexternaldocument}[1]{%
    \externaldocument{#1}%
    \addFileDependency{#1.tex}%
    \addFileDependency{#1.aux}%
}


\DeclareRobustCommand{\stirling}{\genfrac\{\}{0pt}{}}
\newcommand{\com}[2]{C_{#1}^{#2}}


\usepackage{scalerel}[2016/12/29]

\usepackage{diagbox}

\usepackage{array}
\newcolumntype{P}[1]{>{\centering\arraybackslash}p{#1}}



\newcommand{\bx}{{x_\circ}}
\newcommand{\by}{{c_\szero}}
\newcommand{\bbm}{{m_\circ}}

\newcommand{\ax}{{x_*}}
\newcommand{\ay}{c_t}
\newcommand{\tm}{m_t}
\newcommand{\am}{{m_*}}

\newcommand{\perc}{\mathrm{perceptual}}
\newcommand{\norm}[1]{\left\lVert#1\right\rVert}
\DeclareMathAlphabet\mathbfcal{OMS}{cmsy}{b}{n}

\newcommand{\etal}{\textit{et al.}}


\iffalse
%\iftrue

\newcommand{\ncaption}[1]{\caption{#1}}
\newcommand{\nsection}[1]{\section{#1}}
\newcommand{\nsubsection}[1]{\subsection{#1}}
\newcommand{\nsubsubsection}[1]{\subsubsection{#1}}

\else

%\newcommand{\ncaption}[1]{\vspace{-0.27cm}\caption{#1}{\vspace{-0.37cm}}}
%\newcommand{\ncaption}[1]{\caption{#1}}
\newcommand{\nsection}[1]{\vspace{-0.05in}\section{#1} \vspace{-0.03in}}
\newcommand{\nsubsection}[1]{\vspace{-0.04in}\subsection{#1}\vspace{-0.03in}}
\newcommand{\nsubsubsection}[1]{\vspace{-0.01in}\subsubsection{#1}\vspace{-0.03in}}

\fi



\usepackage{mdframed}
\mdfdefinestyle{rebuttalstyle}{innerleftmargin=3pt, innerrightmargin=3pt, innertopmargin=3pt, innerbottommargin=3pt}

\newcounter{response}[section]
\newenvironment{response}{\refstepcounter{response}
\vspace{-1mm}
\begin{mdframed}[style=rebuttalstyle]
\noindent \textbf{Response~\theresponse:}\normalfont
}
{
\end{mdframed}
\vspace{-1mm}
}

\newcounter{revision}[section]
\newenvironment{revision}{\refstepcounter{revision}
\vspace{-1mm}
\begin{mdframed}[style=rebuttalstyle]
\noindent \textbf{Revision~\therevision:}\normalfont
}
{
\end{mdframed}
\vspace{-1mm}
}

\newcounter{link}[section]
\newenvironment{link}{
\vspace{-1mm}
\begin{mdframed}[style=rebuttalstyle]
\noindent \normalfont
}
{
\end{mdframed}
\vspace{-2mm}
}


\newcounter{comments}[section]
\newenvironment{comments}{\refstepcounter{comments}
\vspace{-1mm}
\begin{mdframed}[style=rebuttalstyle]
\noindent 
}
{
\end{mdframed}
\vspace{-1mm}
}

\newcounter{observation} %
\newenvironment{observation}[1]{\refstepcounter{observation}
\vspace{-1mm}
\begin{mdframed}[style=rebuttalstyle]
\noindent \textbf{Finding~\theobservation} (#1): \itshape
}
{
\end{mdframed}
\vspace{-0.15mm}
}

\definecolor{gray}{rgb}{0.7,0.7,0.7}

\newcommand{\cut}[1]{}
\newcommand{\takami}[1]{\textcolor{purple}{[Takami: #1]}}
\newcommand{\newpart}[1]{#1}

\newcommand{\DemoWeb}{\textcolor{blue}{\textbf{\url{https://sites.google.com/view/cav-sec/new-gen-lidar-sec}}}}

\newcommand{\deleted}[1]{}
\newcommand{\ken}[1]{\textcolor{magenta}{[Ken: #1]}}
\newcommand{\ryo}[1]{\textcolor{magenta}{[Ryo: #1]}}
\newcommand{\circled}[1]{\raisebox{.5pt}{\textcircled{\raisebox{-.9pt} {{\small #1}}}}}
\makeatletter
\newcommand\notsotiny{\@setfontsize\notsotiny{7}{8}}
\makeatother

\newcommand\appendixxx{\par
  \setcounter{section}{0}%
  \setcounter{subsection}{0}%
  }

\usepackage{lipsum}

\newcommand\blfootnote[1]{%
  \begingroup
  \renewcommand\thefootnote{}\footnote{#1}%
  \addtocounter{footnote}{-1}%
  \endgroup
}



\hyphenation{op-tical net-works semi-conduc-tor}


\begin{document}
\title{
LiDAR Spoofing Meets the New-Gen: \\ Capability Improvements, Broken Assumptions, and New Attack Strategies}

\author{\IEEEauthorblockN{
Takami Sato\IEEEauthorrefmark{1}\IEEEauthorrefmark{2}\thanks{\IEEEauthorrefmark{1}co-first authors},
Yuki Hayakawa\IEEEauthorrefmark{1}\IEEEauthorrefmark{3},
Ryo Suzuki\IEEEauthorrefmark{1}\IEEEauthorrefmark{3}, 
Yohsuke Shiiki\IEEEauthorrefmark{1}\IEEEauthorrefmark{3},
Kentaro Yoshioka\IEEEauthorrefmark{3}, 
Qi Alfred Chen\IEEEauthorrefmark{2}}
\IEEEauthorblockA{\IEEEauthorrefmark{2}University of California, Irvine, Department of Computer Science\\
\IEEEauthorrefmark{3}Keio University, Department of Electronics and Electrical Engineering
}
\IEEEauthorblockA{
\IEEEauthorrefmark{2}\{takamis, alfchen\}@uci.edu, \IEEEauthorrefmark{3}\{hykwyuk, suzuki.ryo, 
kyoshioka47\}@keio.jp,  \IEEEauthorrefmark{3}shiiki@iskr.elec.keio.ac.jp
}
}





\IEEEoverridecommandlockouts
\makeatletter\def\@IEEEpubidpullup{5\baselineskip}\makeatother
\IEEEpubid{\parbox{\columnwidth}{
    Network and Distributed System Security (NDSS) Symposium 2024\\
    26 February - 1 March 2024, San Diego, CA, USA\\
    ISBN 1-891562-93-2\\
    https://dx.doi.org/10.14722/ndss.2024.23350\\
    www.ndss-symposium.org
}
\hspace{\columnsep}\makebox[\columnwidth]{}}


\maketitle




Over the past few years, there has been a significant amount of research focused on studying the ReLU activation function, with the aim of achieving neural network convergence through over-parametrization. However, recent developments in the field of Large Language Models (LLMs) have sparked interest in the use of exponential activation functions, specifically in the attention mechanism.

Mathematically, we define the neural function $F: \R^{d \times m} \times  \mathbb{R}^d \rightarrow \mathbb{R}$ using an exponential activation function. Given a set of data points with labels $\{(x_1, y_1), (x_2, y_2), \dots, (x_n, y_n)\} \subset \mathbb{R}^d \times \mathbb{R}$ where $n$ denotes the number of the data. Here $F(W(t),x)$ can be expressed as $F(W(t),x) := \sum_{r=1}^m a_r \exp(\langle w_r, x \rangle)$, where $m$ represents the number of neurons, and $w_r(t)$ are weights at time $t$. It's standard in literature that $a_r$ are the fixed weights and it's never changed during the training. We initialize the weights $W(0) \in \mathbb{R}^{d \times m}$ with random Gaussian distributions, such that $w_r(0) \sim \mathcal{N}(0, I_d)$ and initialize $a_r$ from random sign distribution for each $r \in [m]$.

Using the gradient descent algorithm, we can find a weight $W(T)$ such that $\| F(W(T), X) - y \|_2 \leq \epsilon$ holds with probability $1-\delta$, where $\epsilon \in (0,0.1)$ and $m = \Omega(n^{2+o(1)}\log(n/\delta))$. To optimize the over-parametrization bound $m$, we employ several tight analysis techniques from previous studies [Song and Yang arXiv 2019, Munteanu, Omlor, Song and Woodruff ICML 2022]. 

 

\section{Introduction}
\label{sec:introduction}
% \begin{itemize}
%     % Diffusion of FL
%     \item {\st{Diffusion of FL}}
%     % Security threats to FL
%     \item {\st{Security threats to FL with particular focus on model poisoning}}
%     % Limitations of existing countermeasures
%     \item {\st{Current countermeasures (e.g., KRUM) and their limitations}}
%     % Proposed method and its advantages
%     \item {\st{Intuitive description of the proposed method and its difference (i.e., advantages) w.r.t. state of the art}}
%     % Main contributions
%     \item {\st{Summary of the main contributions of this work}}
%     % Paper's structure and organization
%     \item {\st{Paper's structure and organization}}
% \end{itemize}

% Diffusion of FL
Recently, {\em federated learning} (FL) has emerged as the leading paradigm for training distributed, large-scale, and privacy-preserving machine learning (ML) systems~\cite{mcmahan2017googleai,mcmahan2017aistats}. 
The core idea of FL is to allow multiple edge clients to collaboratively train a shared, global model without disclosing their local private training data.
%Specifically, an FL system consists of a central server and many edge clients; 
A typical FL round involves the following steps: {\em(i)} the server randomly picks some clients and sends them the current, global model; {\em(ii)} each selected client locally trains its model with its own private data; then, it sends the resulting local model to the server;\footnote{Whenever we refer to global/local model, we mean global/local model {\em parameters}.} {\em(iii)} the server updates the global model by computing an \emph{aggregation function}, usually the average (FedAvg), on the local models received from clients.
% \begin{enumerate}
%     \item[{\em(i)}] the server sends the current, global model to the clients and appoints some of them for training;
%     \item[{\em(ii)}] each selected client locally trains its copy of the global model with its own private data; then, it sends the resulting local model back to the server;\footnote{Whenever we refer to global/local model, we mean global/local model {\em parameters}.}
%     \item[{\em(iii)}] the server updates the global model by computing an \emph{aggregation function} on the local models received from clients (by default, the average, also referred to as FedAvg~\cite{mcmahan2017aistats}).
% \end{enumerate}
This process goes on until the global model converges. %(e.g., after a certain number of rounds or other similar stopping criteria).
%\\
% The advantages of FL over the traditional, centralized learning paradigm are undoubtedly clear in terms of flexibility/scalability (clients can join/disconnect from the FL network dynamically), network communications (only model weights\footnote{We will use \textit{parameters} and \textit{weights} interchangeably.} are exchanged between clients and server), and privacy (each client's private training data is kept local at the client's end and not uploaded to the server).
\\
% Security threats to FL
%However, the growing adoption of FL also raises security concerns~\cite{costa2022covert}, particularly about its confidentiality, integrity, and availability.
Although its advantages over standard ML, FL also raises security concerns~\cite{costa2022covert}. %, particularly about its confidentiality, integrity, and availability~\cite{costa2022covert}.
% OLD, LONG VERSION
% Indeed, some work deals with privacy leakage that may expose the local data of some clients~\cite{melis2019sp}. 
% A large body of work, instead, investigates attacks that usually aim to detriment the predictive accuracy of the learned global model. For instance, \emph{data poisoning} attacks achieve this goal by letting an adversary pollute the training set of some corrupt FL clients with maliciously crafted examples~\cite{jagielski2018sp}.
% Similarly, in \emph{model poisoning} the attacker attempts to tweak the global model weights~\cite{bhagoji2019pmlr} by directly perturbing the local model's weights of some infected FL clients before these are sent to the central server for aggregation, usually via so-called Byzantine attacks. 
% It turns out that Byzantine model poisoning attacks severely impact standard FedAvg; therefore, more robust aggregation functions must be designed to make FL systems secure.
Here, we focus on \emph{untargeted model poisoning} attacks~\cite{bhagoji2019pmlr}, where an adversary attempts to tweak the global model weights %\footnote{We will use the terms \textit{parameters} and \textit{weights} interchangeably.} 
by directly perturbing the local model's parameters of some infected clients before these are sent to the central server for aggregation.
In doing so, the adversary aims to jeopardize the global model \textit{indiscriminately} at inference time.
Such model poisoning attacks severely impact standard FedAvg; therefore, more robust aggregation functions must be designed to secure FL systems.
\\
% In this paper, we focus on designing a novel robust aggregation scheme at the server's end to contrast the effect of Byzantine model poisoning attacks.
%
% Current countermeasures and their limitations
%Several countermeasures have been proposed in the literature to combat model poisoning attacks on FL systems.
% Some methods use simple statistics more robust than plain average to smooth the impact of malicious updates (e.g., Trimmed Mean and FedMedian~\cite{yin2018icml}). 
% Other defenses implement outlier detection techniques to discard malicious updates from the aggregation performed at the server's end. Those are either based on heuristics (e.g., Krum/Multi-Krum~\cite{blanchard2017nips} and Bulyan~\cite{mhamdi2018pmlr}) or data-driven approaches (e.g., K-means clustering~\cite{shen2016acm} or DnC via spectral analysis~\cite{shejwalkar2021ndss}). 
% Finally, some strategies rely on a centralized ``source of trust'' to spot potential malicious updates (e.g., FLTrust~\cite{cao2020fltrust}).
% Several countermeasures have been proposed in the literature to combat model poisoning attacks on FL systems, i.e., to discard possible malicious local updates from the aggregation performed at the server's end. 
% These techniques range from simple statistics more robust than plain average (e.g., Trimmed Mean and FedMedian~\cite{yin2018icml}) to outlier detection heuristics (e.g., Krum/Multi-Krum~\cite{blanchard2017nips} and Bulyan~\cite{mhamdi2018pmlr}) or data-driven approaches (e.g., spectral analysis via K-means clustering~\cite{shen2016acm} or spectral analysis), or methods based on ``source of trust'' (e.g., FLTrust~\cite{cao2020fltrust}).
% OLD, LONG VERSION
%Several countermeasures have been proposed in the literature to combat Byzantine model poisoning attacks on FL systems.
% Descriptive statistics
% For example, Trimmed Mean and FedMedian aggregate local model updates using more robust statistics than standard average~\cite{yin2018icml}.
%
% % Heuristics for outlier detection
% Many existing Byzantine-resilient strategies implement some outlier detection heuristics to discard the model updates sent by potentially malicious clients from the input of the aggregation function.
% One of the most popular heuristics is Krum~\cite{blanchard2017nips}.
% This strategy tries to mitigate the impact of Byzantine attacks by selecting as a global model the local model with the smallest sum of Euclidean distances to {\em all} the other local models.
% Although powerful, Krum requires the server to know (or, at least, estimate) the number of malicious FL clients upfront, which is generally impossible in a realistic attack scenario. %
% Moreover, Krum may become ineffective for complex, high-dimensional model parameter spaces due to the curse of dimensionality.
% Bulyan~\cite{mhamdi2018pmlr} tries to overcome this issue by combining Krum with a variant of Trimmed Mean.
% % Data-driven outlier detection
% Other strategies use data-driven outlier detection techniques -- e.g., via K-means clustering~\cite{shen2016acm} -- to spot potential malicious local model updates. 
% %For instance, Shen et al. propose to cluster local model updates with K-means and thus identify outliers.
%
% % Other techniques
% As far as the server is concerned, any local model received can be from a potential malicious client. 
% FLTrust~\cite{cao2020fltrust} assumes the server acts as a client, i.e., trains a local model on an additional {\em trustworthy} dataset at the server's end and compares it against all the local models from other clients. 
% This way, the server can rely on some ``source of trust'' when discarding potentially malicious clients.
%\\
% Limitations of existing Byzantine-resilient strategies
Unfortunately, existing defense mechanisms either rely on simple heuristics (e.g., Trimmed Mean and FedMedian by~\cite{yin2018icml}) or need strong and unrealistic assumptions to work effectively (e.g., foreknowledge or estimation of the number of malicious clients in the FL system, as for Krum/Multi-Krum~\cite{blanchard2017nips} and Bulyan~\cite{mhamdi2018pmlr}, which, however, cannot exceed a fixed threshold).
Furthermore, outlier detection methods using K-means clustering~\cite{shen2016acm} or spectral analysis like DnC~\cite{shejwalkar2021ndss} do not directly consider the temporal evolution of local model updates received.
Finally, strategies like FLTrust~\cite{cao2020fltrust} require the server to collect its own dataset and act as a proper client, thereby altering the standard FL protocol.
\\
% OLD, LONG VERSION
% Overall, existing Byzantine-resilient strategies are either simple heuristics (e.g., FedMedian) or, if they are more complex, they rely on strong and unrealistic assumptions to work effectively (e.g., knowing the number of malicious clients in the FL system in advance, as for Krum and alike).
% Furthermore, data-driven outlier detection methods do not consider the temporary evolution of local model updates received (e.g., K-means clustering). 
% Finally, strategies like FLTrust requires the server to collect its own dataset and act as a proper client, thereby altering the standard FL protocol.
%
% Description of the proposed method
This work introduces a novel pre-aggregation \textit{filter} robust to untargeted model poisoning attacks. Notably, this filter $(i)$ operates without requiring prior knowledge or constraints on the number of malicious clients and $(ii)$ inherently integrates temporal dependencies. 
The FL server can employ this filter as a preprocessing step before applying \textit{any} aggregation function, be it standard like FedAvg or robust like Krum or Bulyan.
Specifically, we formulate the problem of identifying corrupted updates as a multidimensional (i.e., matrix-valued) time series anomaly detection task. 
The key idea is that legitimate local updates, resulting from well-calibrated iterative procedures like stochastic gradient descent (SGD) with an appropriate learning rate, show \textit{higher predictability} compared to malicious updates. This hypothesis stems from the fact that the sequence of gradients (thus, model parameters) observed during legitimate training exhibit regular patterns, as validated in Section~\ref{subsec:intuition}. %until convergence. 
%This regularity may be more pronounced for smooth convex loss functions, but it can still be captured within an appropriate time window, even for more complex and convoluted loss surfaces. 
%We provide evidence of this claim in Appendix~B, where we show that the average mutual information (i.e., ``predictability''), calculated over pairs of legitimate model updates sent at different FL rounds, is significantly higher than the corresponding computation for a malicious client.
\\
Inspired by the matrix autoregressive (MAR) framework for multidimensional time series forecasting~\cite{chen2021je}, we propose the FLANDERS ({\em \textbf{F}ederated \textbf{L}earning meets \textbf{AN}omaly \textbf{DE}tection for a \textbf{R}obust and \textbf{S}ecure}) filter.
The main advantages of FLANDERS over existing strategies like FLDetector~\cite{zhao2020multivariate} are its resilience to large-scale attacks, where $50\%$ or more FL participants are hostile, and the capability of working under realistic non-iid scenarios.
We attribute such a capability to two key factors: $(i)$ FLANDERS works without knowing a priori the ratio of corrupted clients, and $(ii)$ it embodies temporal dependencies between intra- and inter-client updates, quickly recognizing local model drifts caused by evil players. Below, we summarize our main contributions:

\begin{itemize}
\item[{\em(i)}]
We provide empirical evidence that the sequence of models sent by legitimate clients is more predictable than those of malicious participants performing untargeted model poisoning attacks.
\\
\item[{\em(ii)}] 
We introduce FLANDERS, the first pre-aggregation filter for FL robust to untargeted model poisoning based on multidimensional time series anomaly detection.
\\
\item[{\em(iii)}] 
We integrate FLANDERS into Flower,\footnote{\scriptsize{\url{https://flower.dev/}}} a popular FL simulation framework for reproducibility.
\\
\item[{\em(iv)}] 
We show that FLANDERS improves the robustness of the existing aggregation methods under multiple settings: different datasets, client's data distribution (non-iid), models, and attack scenarios.
\\
\item[{\em(v)}] 
We publicly release all the implementation code of FLANDERS along with our experiments.\footnote{\scriptsize{\url{https://anonymous.4open.science/r/flanders_exp-7EEB}}}
\end{itemize}

% Paper's structure and organization
The remainder of the paper is structured as follows. %some related work and the current state-of-the-art solutions to security issues that FL entails. 
Section~\ref{sec:background} covers background and preliminaries. 
In Section~\ref{sec:related}, we discuss related work.
Section~\ref{sec:problem} and Section~\ref{sec:method} describe the problem formulation and the method proposed. % to tackle it. 
Section~\ref{sec:experiments} gathers experimental results. %, and Section~\ref{sec:limitations} discusses some limitations of this work.
Finally, we conclude in Section~\ref{sec:conclusion}.
 %discusses the limitations of this work and draws future research directions.
%reports conclusions and draws perspectives for future research directions.

%%%%%%% OLD %%%%%%%
%to overcome the resilience of Byzantine failures in distributed Stochastic Gradient Descent computations. 
% The strength of Krum is its time complexity, which is linear in the gradient dimension. 
% However, the robustness of the approach is guaranteed for gradient-based learning applications only when the majority of the clients are not compromised. 
% Besides, the aggregation mechanism of Krum, as well as that of similar methods, is robust from a coarse-grained perspective and does not provide solutions to errors and perturbations that may occur at inference time.
%A related approach to~\cite{blanchard2017nips} is the work of Su et al.~\cite{su2016dc}. Here, the authors propose an iterated approximate agreement to tackle a multi-layer scenario attacked by Byzantine agents. 
%However, the method works efficiently on the sole discrete context and it is inapplicable to continuous state environments.
%\gabri{Maybe, we should just talk about the main limitations of existing countermeasures without digging into their details (or, we can just mention Krum as this is the most popular one). I will move the description of all these methods to the Related Work section.}
\section{Background on Network Calculus}
\label{sec: background}


\begin{figure*}[tbh]
\centering
\begin{subfigure}[b]{0.3\textwidth}
    \centering
    \includegraphics[width=\linewidth]{images/in-out.png}
    \caption{Arrival and departure data and their relation with delay $d(t)$ and backlog $b(t)$. For a FIFO system, the delay is the horizontal distance between $R(t)$ and $R^*(t)$ but some other multiplexing techniques may shift the data to a later priority, causing a longer delay.}
    \label{fig: data in-out}
\end{subfigure}
\hfill
\begin{subfigure}[b]{0.35\textwidth}
    \centering
    \includegraphics[width=\linewidth]{images/arrival-service.png}
    \caption{Characteristics of an arrival curve and a service curve. From any point of observation, the arriving data never exceeds its arrival curve; the departure data is also never less than the service curve with respect to the data arrival.}
    \label{fig: arrival-service curves}
\end{subfigure}
\hfill
\begin{subfigure}[b]{0.33\textwidth}
    \centering
    \includegraphics[width=\linewidth]{images/bound.png}
    \caption{Delay and backlog bounds of a system. Backlog is the maximum vertical distance between $\alpha(t)$ and $\beta(t)$; FIFO delay is their maximum horizontal distance; but for arbitrary multiplexing, the delay guarantee is when the system clears its buffer, thus it's the intersection of $\alpha(t)$ and $\beta(t)$.}
    \label{fig: system bounds}
\end{subfigure}
\caption{Network calculus framework. We let $R(t)$ and $R^*(t)$ be the arrival and departure data flow of a system; $\alpha(t)$ be the piecewise linear concave arrival curve and $\beta(t)$ be the piecewise linear convex service curve of a system.}
% \hossein{Better to show piece-wise linear concave arrival curve and piece-wise linear convex service curve instead of token-bucket and rate-latency.}}
\end{figure*}

We recall some of the network calculus essentials for a better understanding of the framework used in Saihu. In the following context, we use the following notation: $\mbb{R}^+$ is the set of non-negative real numbers; $[x]_+$ denotes $\max(0, x)$

The data flow is by convention modeled as a left-continuous wide-sense increasing function $R(t): \mbb{R}^+ \mapsto \mbb{R}^+$ with respect to time $t$~\cite{ncbook2001leboudec}. 

A system $\mcal{S}$ receives arrival data described as a cumulative function $R(t)$ and delivers departure data as another cumulative function $R^*(t)$. Figure~\ref{fig: data in-out} illustrates such a system $\mcal{S}$. The benefit of representing a system like this is that we can observe system backlog and delay with such a model. 

\begin{definition}[Backlog and Delay~\cite{ncbook2001leboudec}]
    The backlog of a system at time~$t$ is
    \begin{equation}
        b(t) = R(t) - R^*(t)
    \end{equation}
    
    The virtual delay of a FIFO system at time $t$ is
    \begin{equation}
        d_{FIFO}(t) = \inf \lbp \tau \geq 0 : R(t) \leq R^*(t+\tau) \rbp
    \end{equation}
\end{definition}



The backlog of a system can be viewed as the vertical distance between $R$ and $R^*$. The FIFO (\textit{First-in First-out}) delay is the horizontal distance between $R$ and $R^*$. One may obtain other delay values if the multiplexing technique is not FIFO.

% \begin{figure}
%     \centering
%     \includegraphics[width=0.9\linewidth]{images/in-out.png}
%     \caption{In/out data flow; delay and backlog}
%     \label{fig: data in-out}
% \end{figure}

Since we are interested in the system guarantee instead of a single instance of data flow, we would like to have general bounds to the arrival and departure data flows. Therefore, we define \textit{arrival curve} and \textit{service curve} as the bounds of arrival and departure data flows.

\begin{definition}[Arrival Curve~\cite{ncbook2001leboudec}]
    Given a wide-sense increasing function $\alpha: \mbb{R}^+ \mapsto \mbb{R}^+$, we say that a flow $R(t)$ is $\alpha$-constrained if and only if for all $s \leq t$:
    \begin{equation}
        R(t) - R(s) \leq \alpha(t-s)
    \end{equation}
    We say $R(t)$ has $\alpha$ as an arrival curve.
\end{definition}

\begin{definition}[Service Curve~\cite{ncbook2001leboudec}]
    Given a wide-sense increasing function $\beta: \mbb{R}^+ \mapsto \mbb{R}^+$ and $\beta(0) = 0$. A system $\mcal{S}$ having $R(t)$ and $R^*(t)$ as its arrival and departure flows. We say $\mcal{S}$ offers a service curve $\beta$ if and only if
    \begin{equation}
        R^*(t) \geq (R \otimes \beta)(t) =: \inf_{s \leq t} \lbp R(s) + \beta(t-s) \rbp
    \end{equation}
    where $\otimes$ denotes the min-plus convolution
\end{definition}

Figure~\ref{fig: arrival-service curves} illustrates the arrival and service curves. Any segment of arrival flow $R(t)$ is constrained by arrival curve $\alpha$ and the output curve $R^*(t)$ is always no less than the curve $R\otimes\beta$. As a result, an arrival curve upper bounds the incoming traffic, and a service curve lower bounds the outgoing traffic.

% \begin{figure}
%     \centering
%     \includegraphics[width=\linewidth]{images/arrival-service.png}
%     \caption{Arrival/Service curve}
%     \label{fig: arrival-service curves}
% \end{figure}

We consider 2 special types of curves throughout this paper, \textit{token-bucket} (or sometimes called \textit{leaky-bucket}) curve and \textit{rate-Latency} curve.

\begin{definition}[Token-bucket and Rate-latency~\cite{ncbook2001leboudec}]
    A token-bucket curve $\gamma_{r,b}$ with arrival rate $r$ and burst $b$ is defined as
    \begin{equation}
        \gamma_{r,b}(t) = b + rt
    \end{equation}

    A rate-latency curve $\beta_{R,T}$ with service rate $R$ and latency $T$ is defined as
    \begin{equation}
        \beta_{R,T}(t) = R \lb t - T \rb_+
    \end{equation}
\end{definition}

A token-bucket curve is determined by a burst $b$ and an arrival rate~$r$. Burst represents the maximum possible data volume that can arrive simultaneously, and arrival rate represents the maximum long-term data rate~\cite{bouillard2022tradeoff}.
A rate-latency curve is determined by a latency~$T$ and a service rate~$R$. Latency represents the time a server needs before starting to process the incoming data, and service rate represents the minimum rate to process data after the initial latency.

With the help of arrival and service curves, we can derive delay and backlog bounds for a system $\mcal{S}$ illustrated in Figure~\ref{fig: system bounds}. Suppose a system $\mcal{S}$ has arrival curve $\alpha$ and service curve~$\beta$, its worst-case backlog $b^*$ is the maximum vertical distance between~$\alpha$ and~$\beta$. Similarly, depending on the multiplexing technique applied to the system, its worst-case delay bound $d^*$ is the maximum horizontal distance between $\alpha$ and $\beta$ if $\mcal{S}$ is a FIFO system. If we don't have any information about its multiplexing technique, referred to as arbitrary multiplexing, the best we can say is that when $\alpha$ and $\beta$ intersect each other, where all data has been delivered out of the system. Consequently, the worst-case delay bound for arbitrary multiplexing is the time required for $\mcal{S}$ to clear its buffer.

% \begin{figure}
%     \centering
%     \includegraphics[width=\linewidth]{images/bound.png}
%     \caption{System delay/backlog bounds}
%     \label{fig: system bounds}
% \end{figure}

While a service curve captures the slowest possible output speed of a system, a link's transmission capacity limits the speed as well. Hence, we model this phenomenon using a \textit{greedy shaper} with a sub-additive function $\sigma: \mbb{R}^+ \mapsto \mbb{R}^+$ concatenated with a server. We consider a concatenation as shown in Figure \ref{fig: system}. By convention we assume $\sigma(0) = 0$ and $\beta(t) \leq \sigma(t), \forall t \in \mbb{R}^+$, meaning that the buffer is cleared at the beginning and the service never exceed its physical limitation. With the above definition, such greedy shaper conserves the service provided by the system due to theorem \ref{thm: shaping}.

\begin{figure}[thb]
    \centering
    \includegraphics[width=0.7\linewidth]{images/system.png}
    \caption{Shaping of departure data. A flow that has an arrival curve $\alpha$ feeds into a server with an arrival data flow $R(t)$. The server having service curve $\beta$ takes $R(t)$ and gives a departure data flow $R^*(t)$ to a shaper with shaping function $\sigma$. The shaper takes $R^*(t)$ and shape the data flow as another departure $D(t)$.}
    \label{fig: system}
\end{figure}


\begin{theorem}[Shaping conserves service \cite{ncbook2001leboudec}]
\label{thm: shaping}
Following the system shown in Figure \ref{fig: system}, we have
\begin{equation}
     D = R^* \otimes \sigma \geq \lp R \otimes \beta \rp \otimes \sigma = R \otimes \lp \beta \otimes \sigma \rp = R \otimes \beta
\end{equation}
\end{theorem}

In the following context, we model the shaping function $\sigma$ as a token-bucket curve $\gamma_{C,L}$ with transmission capacity $C$ and the packet size $L$ to capture the link capacity and packetization~\cite{bouillard2022tradeoff}.

\section{Method}
\label{s:method}

We consider the 3D euclidean space $\Real^3$ with points $p=(x,y,z)\in\Real^3$. We discretize the unit cube $\gC=[0,1]^3$ with a 3D voxel grid $\gG=\set{p_I}$, with nodes $p_I$ indexed by $I=(i,j,k)$, $i,j,k\in [n]=\set{1,\ldots,n}$, \ie, $p_I=(x_{ijk},y_{ijk},z_{ijk})$. We denote by $h=n^{-1}$, and by $N=n^3$ the total number of nodes.   
We represent our reconstructed surface as a zero level of a scalar function $f$ defined over the cube $\gC$. $f$ is defined by prescribing its values at the grid's nodes $f_I\in\Real$ and trilinear interpolating in each voxel. We will denote by $f(p)$ the interpolated value at point $p$. 

Given an input point cloud consisting of $m$ points $q_k\in\Real^3$ with or without (unit norm) normals $n_k\in \Real^3$, $k\in [m]$, our goal is to compute $f$ so that its zero level set approximates the unknown surface, \ie, 
\begin{equation}
    \gS = \set{p\in\gC \ \vert \ f(p)=0}.
\end{equation}
Our approach to compute $f$ is to minimize a loss function of the form
\begin{equation}
    \gL = \gL_{\text{data}} + \gL_{\text{prior}}
\end{equation}
where 
\begin{equation}\label{e:loss_data}
    \gL_{\text{data}} = \frac{\lambda_{\text{p}}}{m}\sum_{k=1}^m \abs{f(q_k)}^2 + \frac{\lambda_{\text{n}}}{m}\sum_{k=1}^m \norm{\nabla f(q_k) - n_k}^2
\end{equation}
where $\norm{\cdot}$ is the standard euclidean norm in $\Real^3$, $\nabla f(p) \in \Real^3$ is the gradient of $f$ sampled at point $p$. Note that $\nabla f$ is defined in interior of voxels, which is generically where the input points $q_k$ resides. $\gL_{\text{data}}$ is the standard data loss encouraging the zero level to pass through the input points $q_k$, and its normals (defined by gradients of $f$) to coincide with input normals $n_k$. 

The prior, $\gL_{\text{prior}}$, is the main contribution of this work, where we combine two novel losses,
\begin{equation}
    \gL_{\text{prior}} = \lambda_{\text{v}} \gL_{\text{viscosity}} + \lambda_{\text{c}} \gL_{\text{coarea}}
\end{equation}
Intuitively, the viscosity loss optimizes for a smooth Signed Distance Function (SDF) solutions, avoiding auxiliary bad minima of the Eikonal equation, while the coarea loss strives to minimize the area of the zero level surface. Our loss has $4$ hyper-parameters $\lambda_{\text{p}},\lambda_{\text{n}},\lambda_{\text{v}},\lambda_{\text{c}}$. We provide more details on these priors next. 


\subsection{Viscosity Loss}\label{ss:viscosity_loss}
The goal of the viscosity loss is to make $f$ approximate an SDF over $\gC$. Given boundary conditions asking $f$ to vanish on some closed compact surface $\gS$, the SDF solves the Eikonal equation PDE, \ie, $\norm{\nabla f(p)}=1$, in a certain well defined sense (viscosity). This motivated some previous work to directly optimize the Eikonal loss \citep{gropp2020implicit,sitzmann2020implicit}
\begin{equation}\label{e:loss_eikonal}
    \gL_{\text{eikonal}} = \int_\gC \Big (\norm{\nabla f(p)}-1\Big )^2 dp
\end{equation}
\begin{wrapfigure}[14]{r}{0.28\textwidth}\vspace{-15pt}
  \begin{center}
    \includegraphics[width=0.25\textwidth]{figs/illustrations/eikonl_1d.png}
  \end{center}
  \caption{Two global minimizers of the Eikonal loss over a grid in 1D. Top solution is not an SDF. }\label{fig:eikonal_1d}
\end{wrapfigure}
Unfortunately, the Eikonal loss has many undesirable minima which are not SDFs. Figure \ref{fig:eikonal_1d} shows a 1D example: both depicted solutions (denoted $f$) vanish at the input points $q_1,q_2$ (black points) and globally minimize the Eikonal loss over the grid (grid points are shown in blue). The INR works mentioned above use neural networks for representing $f$ which injects an inductive bias avoiding these bad minima, however on grids, minimizing \eqref{e:loss_eikonal} cannot avoid these solutions. See, \eg, middle column in Figure \ref{fig:teaser}. 

More classical Eikonal solvers do work with grids however use mostly fast marching or sweeping methods \citep{osher1988fronts,sethian1996fast,zhao2005fast,chacon2012fast}. Namely, use a special discretization of the Eikonal equation favoring the viscosity  solution of the Eikonal \cite{rouy1992viscosity}, and update node values according to a moving front \cite{sethian1996fast}. Since this discretization is up-wind (will only propagate values in one direction) and requires choosing the maximal among its solution, its success in adaptation to a loss is not clear. 

We use a different approach to build a loss favoring SDF solutions over grids motivated by the vanishing viscosity method \cite{crandall1983viscosity}. Namely, adding to the Eikonal PDE a small perturbation of the Laplacian of $f$ (denoted by $\Delta f$), \ie, $\norm{\nabla f(p)}-1 - \eps\Delta f(p)=0$, makes the PDE semi-linear elliptic \citep{calder2018lecture}, and hence with suitable boundary conditions it is uniquely solvable inside $\gS$ with a smooth solution, approaching the viscosity positive distance function to the boundary as $\eps\too 0$. Similarly, for $1-\norm{\nabla f(p)} - \eps \Delta f(p)=0$ the solution approaches the negative distance function inside the domain. 
Motivated by the vanishing viscosity principle we suggest the following viscosity loss:
\begin{equation}\label{e:loss_viscosity_eikonal}
\gL_{\text{viscosity}} = \int_\gC \Big((\norm{\nabla f (p)}-1)\mathrm{sign}(f(p)) - \eps \Delta f(p)\Big)^2 dp
\end{equation}
We discretize this loss over the grid $\gG$ by replacing the first order derivatives and second order derivatives with symmetric finite  differences, \ie,
\begin{align*}
    D_x f_I=D_x f_{i,j,k} = \frac{f_{i+1,j,k}-f_{i-1,j,k}}{2h}, \quad D^2_x f_I = D^2_x f_{i,j,k}=\frac{f_{i+1,j,k}-2f_{i,j,k}+f_{i-1,j,k}}{h^2}
\end{align*}
and similarly for $D_y$ and $D_z$. We use these discrete operators to approximate the gradient $\widehat{\nabla} f(p_I) = (D_x f_I, D_y f_I, D_z f_I)$ and Laplacian $\widehat{\Delta}f(p_I) = D_x^2f_I + D_y^2 f_I + D_z^2 f_I$. The discretized viscosity loss now takes the form
\begin{equation}
    \widehat{\gL}_{\text{viscosity}} = \frac{1}{N}\sum_{I} \Big((\|\widehat{\nabla} f (p_I)\|-1)\mathrm{sign}(f(p_I)) - \eps \widehat{\Delta} f(p_I)\Big)^2
\end{equation}



\subsection{Coarea loss}\label{ss:coarea_loss}
The coarea loss is approximating the area of the zero level set, and therefore incorporating it in the optimization pushes the reconstructed surface to be economic in area. 

First, similarly to  \citep{yariv2021volume} we use the centered Laplace CDF
\begin{equation}
   \Psi\beta(s)= \begin{cases}
   \frac{1}{2}\exp\parr{\frac{s}{\beta}} & s\leq 0 \\ 1-\frac{1}{2}\exp\parr{-\frac{s}{\beta}} & s\geq  0
   \end{cases}
\end{equation} to transform the SDF $f$ to a smooth approximation of the indicator function:
\begin{equation}
    \chi_\beta(p)=\Psi\beta (-f(p))
\end{equation}
As $\beta\too 0$, $\chi_\beta$ converges to an indicator function leading to $1$ inside $\gS$ and $0$ outside. The coarea loss is defined as 
\begin{equation}
    \gL_{\text{coarea}} = \int_\gC \norm{\nabla \chi_\beta (p)} dp
\end{equation}
To understand why this loss approximates the area of $\gS$ we can use the coarea formula \citep{rindler2018calculus}:
\begin{equation}\label{e:coarea}
    \int_\gC \norm{\nabla \chi_\beta(p)}dp = \int_{-\infty}^{\infty} \mathrm{area}(\chi_\beta^{-1}(s))ds,
\end{equation}
where $\chi_\beta^{-1}(s)=\set{p\ \vert \ \chi_\beta(p)=s}$ is the preimage of the value $s$. Since $\chi_x(p)\in [0,1]$ the r.h.s.~integral can be restricted to the interval $[0,1]$, and therefore the coarea loss averages the area of the level sets of $\chi_\beta$. Next,  $$\chi_\beta^{-1}(s)= \set{p\ \vert \ \Psi\beta (-f(p)) = s } = \{p\ \vert \ f(p) = -\Psi\beta^{-1} (s) \} = f^{-1}(-\Psi\beta^{-1} (s)),$$
\begin{wrapfigure}[11]{r}{0.28\textwidth}\vspace{-20pt}
  \begin{center}
  \includegraphics[width=0.25\textwidth]{figs/semi.png}
  \end{center}
  \caption{Reconstruction of a semisphere point cloud (white dots) without (left) and with (right) coarea loss. }\label{fig:coarea_semisphere}
\end{wrapfigure}

which shows that the level set $s\in (0,1)$ of $\chi_\beta$ is the level set $-\Psi\beta^{-1}(s)$ of the SDF $f$. As $\beta\too 0$, $-\Psi\beta^{-1}(s)\too 0$ for all $s\in (0,1)$ (and uniformly in $(\eps,1-\eps)$ for fixed $\eps>0$). Therefore the average of the level set area (\ie, the r.h.s.~of \eqref{e:coarea}) converges to the area of $f^{-1}(0)=\gS$. Figure \ref{fig:teaser} (right) shows how removing the coarea loss introduces an extraneous zero level set, and hence results in an undesired surface part. Figure \ref{fig:coarea_semisphere} shows a comparison of a reconstruction of semisphere with and without coarea. In the experiments section we provide more ablation tests with the coarea and viscosity losses.

To discretize the coarea loss we let $w_I$ denote the centers of grid's voxels, and note that $\nabla \chi_\beta(w_I) = \Phi_\beta(-f(w_I))\nabla f(w_I)$, where 
\begin{equation*}
    \Phi_\beta(s) = \frac{1}{2\beta}\exp\parr{\frac{\abs{s}}{\beta}}
\end{equation*}
is the PDF of the Laplace distribution, and $\nabla f(w_I)$ is computed as a linear combination of the voxel's corner values $f_{I_1},\ldots,f_{I_8}$, see more details in the Appendix. We end up with the discretized loss:
\begin{equation}
    \widehat{\gL}_{\text{coarea}} = \frac{1}{N}\sum_{I}\Phi_\beta(-f(w_I))\norm{\nabla f(w_I)}
\end{equation}
This loss is usually incorporated with a small hyper-parameter $\lambda_{\text{c}}$ with the purpose of eliminating redundant surface parts.


\nsection{Object Injection Attack Measurements} \label{sec:injection_attack}

With the measurement setup above, in this paper we perform the first large-scale measurement study for both classes of state-of-the-art LiDAR spoofing attacks: object injection attack and object removal attack, which will be the focus of this and the next section respectively. For each attack class, we first perform the attack capability measurements at the LiDAR point cloud level, and then model such attack capabilities for the subsequent measurements at the object detector level. 



\nsubsection{LiDAR-Level Measurements (RQ1, RQ2)}
\label{sec:lidar_injection}

In this section, we measure the object injection attack capabilities at the LiDAR point cloud level, which is measured by the spoofing attack's capability in injecting spoofed points. We perform this measurement using an attack laser with sufficiently high intensity so that the attack-induced spoofed points can be easily differentiated from the benign ones. To answer RQ1, we first re-visit the existing state-of-the-art spoofing attacks for point injection on the VLP-16 LiDAR~\cite{VLP16}, which is predominantly used as the \textit{only} evaluation target in prior works~\cite{shin2017illusion, cao2019adversarial, jiachen2020towards, cao2023you, hallyburton2022security}. After that, we then conduct the measurement study on new-gen LiDARs to answer RQ2. 




\nsubsubsection{Re-Visiting Design Assumptions Made in Prior Works with VLP-16 (RQ1)} \label{sec:inj_vlp16}

Prior works have shown that the number of spoofed points in VLP-16 increases as the quality of the spoofer device improves, ranging from 60~\cite{cao2019adversarial} to $\sim$4k~\cite{cao2023you}. 
As shown in Table~\ref{tbl:distance}, with our spoofer improvements (\S\ref{sec:spoofer_design}), such attack capabilities are further improved significantly,
which can now inject $>$6,131 points indoors and $>$6,514 points outdoors. This is at least 50\% more than those in the latest prior work~\cite{cao2023you}, which capped at $\sim$4,000. We noticed a concurrent work that may have also made contributions in improving the spoofing capability~\cite{jin2022pla}. The full paper is not available at the time of our writing, but from the abstract it seems that (1) the spoofing capability is still capped at 4200 points, which is still at least 50\% fewer than ours; and (2) the study scale is much smaller than ours in terms of LiDAR number (only 2, versus 9 for us) and likely also LiDAR types and generation coverage.


Meanwhile, we also observe that the number of spoofed points and angles in the outdoor setup is generally larger than those in the indoor setup. We consider this reasonable since the legitimate laser reflections are fewer in the outdoor environment (e.g., no wall reflections), leaving more room for spoofing.  However, this actually contradicts those reported results in the latest prior work~\cite{cao2023you}: as shown in Table~\ref{tbl:distance}, the spoofed points outdoors are significantly fewer than those indoors (1.8k versus 4k) and they attribute this to the lighting condition differences (e.g., they also report that the number of spoofed points decreases from $\sim$2,500 at night to $\sim$1,600 at day.)
We consider that our results are achieved by our more careful optical setup as described in~\S\ref{sec:optical_setup} since the spoofing laser intensity is much stronger than the sunlight and should not be decayed in a short 10 m flight. 
We suspect that the optical setup of the previous work is not well calibrated and the laser beam diverges. 


\begin{observation}{RQ1}
With electronic and optical setups, the LiDAR spoofing attack can inject $>$6,000 points ranging $>$80$^{\circ}$, which is a significantly higher number of points and wider range than in previous studies.
Furthermore, previous observations that spoofing distance and lighting conditions can greatly affect the LiDAR spoofing capability~\cite{cao2023you} no longer hold if implemented using more careful optics.
\end{observation}



\nsubsubsection{CPI Attack Capability Measurements} \label{sec:cpi_capability}
As shown in Fig.~\ref{fig:arbitrary_point}, with our spoofing improvements we are able to demonstrate strong spoofed point pattern control capability for the first time. Considering that the CPI attack capability is a common design-level assumption made in prior works (\S\ref{sec:cpi}), we thus perform the first systematic quantification of it to allow more rigorous attack capability analysis on the object detector side.
Note that some prior works tried to include a modeling of such CPI attack capability in their object detector-side analysis~\cite{hallyburton2022security}, but their modeling is based on their intuitions (e.g., simply add vertical and horizontal noises~\cite{hallyburton2022security}) instead of the LiDAR sensing mechanisms and the spoofer designs. As shown later in~\S\ref{sec:od_injection_eval}, such erroneous spoofing accuracy modeling can cause significant differences in the object detector-side attack results. 


Based on our attack reproduction and spoofing inaccuracy cause investigations, we find that there are two types of errors in controlling the position control of each spoofed point $x_{ij}$ at $i$-th altitude and $j$-th azimuth (illustrated in Fig~\ref{fig:inter_inner}): (1) \textit{inner-frame error} $\delta^{\rm inner}_{ij}$: the inaccuracy in spoofing a point at a chosen 3D position within an individual frame; and (2) \textit{inter-frame error} $\delta^{\rm inter}$: the inaccuracy in maintaining the spoofed position of the same point in the chosen pattern across consecutive frames. Rooted in the spoofing design (Fig.~\ref{fig:spoofer1}), the inner-frame error is due to the inaccurate signal bursting of FG to the gate driver, while the inter-frame one is due to inaccurate timing of triggering the FG from TIA. With our experimental setup, we measure both and find that $\delta^{\rm inner}_{ij}$ and $\delta^{\rm inter}$ are $\sim$10 cm as in Fig~\ref{fig:inner_frame_error} and $\sim$35 cm respectively.





\begin{figure}[t!]
\centering
\vspace{-0.1in}
\includegraphics[width=\linewidth]{imgs/spoofing/inner_frame_noise.pdf}
\vspace{-0.3in}
\caption{Standard deviations of inner-frame error on VLP-16.}
\label{fig:inner_frame_error}
\end{figure}

\begin{observation}{RQ1}
The CPI attack capability, which is commonly assumed but never demonstrated in prior works~\cite{cao2019adversarial, jiachen2020towards, cao2023you, hallyburton2022security, zhongyuan2021object}, can be achievable in practice with a well-calibrated spoofer, while by design such pattern control capability is subject to inner- and inter-frame errors.
\end{observation}


\textbf{Assumption that object detector-level vulnerability exploitation is needed:} Combining these spoofing capability advances above (significantly improved spoofable points, angle, and CPI attack capability), this actually means that \textit{an attacker does not always need to exploit object detector-level vulnerabilities to achieve a near-front road object injection}, which is actually the major design assumption for all attack works so far on the object injection attack side~\cite{cao2019adversarial, jiachen2020towards, hallyburton2022security}.
Specifically, as clearly stated in prior works~\cite{cao2019adversarial, jiachen2020towards}, a valid point cloud for a front-near vehicle needs $\sim$2,000 points and 15$^{\circ}$ azimuth ($\theta$) coverage, but their spoofers can only achieve $<$200 points and $<$8$^{\circ}$, which is why object detector-level vulnerability exploitation (and thus those novel spoofing pattern optimization/identification) are needed. \newpart{These results indicate that the exploitation of the object detector-level vulnerability is not the necessary condition although it may help to design more sophisticated attacks.}



\begin{observation}{RQ1}
\newpart{
Now with the CPI attack capability demonstrated with sufficiently larger spoofable point number and angle, an attacker actually has sufficient capabilities to directly inject a near-front vehicle pattern. This finding has significant implications for the current line of research, since this  (1) may not need to exploit object detector-level vulnerabilities on the attack sides~\cite{cao2019adversarial, jiachen2020towards, hallyburton2022security}; and 
(2) may suggest that model-level defenses that try to leverage the inability of spoofers in directly injecting indistinguishable patterns as benign cases (e.g., \cite{jiachen2020towards}) can be ineffective.
}
\end{observation}

\begin{table}[t!]
\centering
\footnotesize
\caption{Evaluation results of the synchronized injection attack on VLP-16. 
$\mathcal{N}$: Number of injected points by spoofing. $\theta$: Azimuthal range of the point injection. $\mathcal{R}$: Point injection success rate within $\theta$.
The distance $d$ is between the spoofer and the LiDAR. Number in parenthesis: $\mathcal{N}$ from latest prior work~\cite{cao2023you}. ``-'': Data not available.
}
\label{tbl:distance}
\footnotesize
\setlength{\tabcolsep}{4.5pt}
\renewcommand{\arraystretch}{0.75}
\begin{tabular}{cccccccc}
\toprule
     & \multicolumn{3}{c}{Indoor} &  & \multicolumn{3}{c}{Outdoor (Daytime: 70 lux)} \\ \cline{2-4} \cline{6-8} 
$d$     & $\mathcal{N}$     &  $\mathcal{R}$     & $\theta$         &  & $\mathcal{N}$     &  $\mathcal{R}$     & $\theta$      \\ \cline{1-4} \cline{6-8} 
2 m  & 6,523 (-)  & 98.5\% & 82.7$^{\circ}$  &  & 7,705 (-)  &  94.9\%     & 100.5$^{\circ}$        \\
(2.5 m)  & ($<$4k)  &  &   &  & (-)  &      &        \\
4 m  & 6,386 (-)  & 96.9\% & 82.5$^{\circ}$ &  &  7,950 ({$<$1.8k})  &   96.9\%     & 101.5$^{\circ}$        \\
6 m  & 6,575 (-)  & 98.6\% & 83.4$^{\circ}$  &  & 7,357 ({$<$1.5k}) &   87.2\%    & 99.6$^{\circ}$        \\
8 m  & 6,213 (-)  & 93.8\% & 82.8$^{\circ}$  &  & 6,702 ({$<$1k}) &   97.7\%     & 83.4$^{\circ}$         \\
10 m & 6,131 (-) & 93.2\% & 82.1$^{\circ}$  &  & 6,514 ({$<$1k}) &    93.3\%     &  84.2$^{\circ}$     \\ \toprule
\end{tabular}
\vspace{-0in}
\end{table}


\nsubsubsection{Measurements on New-Gen LiDARs}  \label{sec:inj_other_lidars}

Table~\ref{tbl:inject_attack} shows the measurement results of the point injection capabilities on not only first-gen LiDARs such as VLP-16 but also new-gen ones. The first thing we find is that the latest spoofing attack for object injection, synchronized spoofing  (\S\ref{sec:attack_taxonomy}), can only be applied to the first-gen ones; for the new-gen ones, 
since they all have either timing randomization or pulse fingerprinting, synchronization (and thus CPI) becomes virtually impossible since timing randomization by design prevents the prediction of the future laser firing timing from the victim LiDAR, while pulse fingerprinting by design prevents the prediction of the future laser pulse pattern of each measurement. To still measure their vulnerabilities to spoofing attacks, we try to inject as many spoofed points as possible with a random attack with high-frequency pulses. 
The random attack is similar to the newly-identified HFR attack (\S\ref{sec:high_freq_attack}), but the frequency is tuned to achieve the largest number of injected points.

As shown, compared to the first-gen LiDARs, the security-related features in the new-gen LiDARs result in huge differences in the point injection capabilities: $>$19k injected points for Horizon, $\sim$3.2k for Helios, 100-350 on Realsense L515 and XT32, and only 28 on OS1-32. We will closely investigate such differences in the next section. Despite these variances, one observation is in common: VLP-16 is actually the \textit{only} LiDAR among all these 7 ones that can achieve the CPI attack capability --- it is not possible by design for the 6 new-gen ones due to timing randomization and pulse fingerprinting, while it is also not achievable to the other first-gen LiDAR (VLP-32c) due to simultaneous firing (\S\ref{sec:sec_enchance_feats}).





\begin{observation}{RQ1}
VLP-16 is actually the only LiDAR for which the CPI attack capability is feasible, which is the key design assumption made in all prior works on object injection attack side~\cite{cao2019adversarial, jiachen2020towards, hallyburton2022security}. This directly challenges the validity of all these existing designs against the more general and recent set of LiDARs.
\end{observation}



\nsubsubsection{Impacts from Security-Related Features}  \label{sec:sec_enchance_feats}
~

\begin{table}[t!]
\centering
\footnotesize
\caption{Measurements of the point injection capabilities on different LiDARs. For the first-gen LiDARs, the attack is synchronized spoofing (\S\ref{sec:attack_taxonomy}). For the new-gen ones, we inject as many points as possible with random firing (1 MHz). Symbols are the same as in Table~\ref{tbl:distance}. %
}
\label{tbl:inject_attack}
\setlength{\tabcolsep}{3.1pt}
\setlength{\aboverulesep}{0pt}
\setlength{\belowrulesep}{0pt}
\renewcommand{\arraystretch}{0.9}
\begin{tabular}{c|cc|cccc|c}
\toprule
       & \multicolumn{2}{c|}{\multirow{2}{*}{First-Gen}} & \multicolumn{5}{c}{New-Gen}  \\  \cline{4-8}
       &  & & \multicolumn{4}{c|}{w/ Timing Randomization}    & w/ Fingerprint  \\  \cline{2-8}
 &       VLP-16  & VLP-32c & OS1-32 & Helios    & Horizon  & L515    & XT32  \\  
 \hline  \hline$\mathcal{N}$     & 6,523    & 9,711    & 28     & 3,203   & 19,182    & 321   & 113    \\
$\mathcal{R}$     & 98.50\% & 82.90\% & 43.80\%   & 19.4\%& 79.90\%  & 0.1\%  & 2.10\%  \\
$\mathcal{\theta}$ & 82.7$^{\circ}$ & 73.2$^{\circ}$ & 0.72$^{\circ}$  & 34.2$^{\circ}$ & 103.4$^{\circ}$ & 81.7$^{\circ}$ & 70$^{\circ}$\\ \toprule
\end{tabular}
\raggedright
\vspace{0.05in}
\end{table}

\noindent\textbf{Simultaneous Laser Firing.} \label{sec:sim_fireing}
As listed in Table~\ref{tbl:target_lidars}, many LiDARs fire multiple laser pulses simultaneously. Velodyne LiDAR is likely adopting a modular approach that doubles units when increasing altitudes. Hence, the number of simultaneous laser firings is also doubled: VLP-16~\cite{VLP16} fires 1 laser during each measurement, and VLS-32~\cite{VLP32c} fires 2 lasers simultaneously. This design makes the CPI attack infeasible because we cannot selectively return the laser to each simultaneous laser due to the large diameter of the spoofing laser. For example, on VLP-32c we can always inject spoofed points pair by pair, and the injected pair will always have the same distance to LiDAR. Although simultaneous laser firing may help attackers since it can affect more points by a single attack laser pulse, the CPI becomes no longer feasible. 




\noindent\textbf{Timing Randomization.}
New-gen LiDARs typically have a feature that randomizes their laser shooting timing at each firing to mitigate interference from other LiDARs as listed in Table~\ref{tbl:target_lidars}. The timing randomization makes the CPI attack capability virtually impossible because the attacker can no longer synchronize with the laser firing pattern. However, interestingly, we find that the attacker may still be able to inject some points if the randomization is not strong enough. To quantify the impacts of the timing randomization, we measure the laser firing interval (not available from data sheets) with a PD and an oscilloscope. Table~\ref{tbl:randomization} illustrates the histogram, standard deviation, and maximum value of laser firing intervals of each LiDAR with timing randomization. 
We also categorize and fit the observed firing intervals into the uniform or Gaussian distribution based on the shape of its histogram.
We convert time difference to distance with the following formula: $\frac{\Delta t \times c}{2}$, where $\Delta t$ is the timing difference and $c$ is the speed of light. Fig.~\ref{fig:livox_spoofing} in Appendix shows the point clouds of Horizon~\cite{livox_horizon}, which can illustrate the impacts of timing randomization on spoofing capability. As shown, when under attack, the whole point cloud becomes highly randomized, and the object shapes (e.g., our lab room as shown in the benign case) completely disappeared. We further evaluate the significance of the errors on object detectors in~\S\ref{sec:impact_noise}. 
 




\begin{table}[t!]
\centering
\footnotesize
\setlength{\tabcolsep}{2pt}
\renewcommand{\arraystretch}{0.75}
\caption{Distribution of laser firing intervals for LiDAR with timing randomization. Std. $\sigma$: Standard deviation of the error caused by the timing randomization in meters. Max. $\Delta$: Maximum error caused by the timing randomization in meters. 
}
\label{tbl:randomization}
\begin{tabular}{lccccc}
\toprule
                        & OS1-32~\cite{OS1-32}        & Horizon~\cite{livox_horizon}       &  L515~\cite{L515} & Pixell~\cite{pixell}  &  Helios~\cite{Helios}    \\ \cline{2-6} 
                        & \includegraphics[width=0.5in]{imgs/spoofing/randomization/os1-random-hist.pdf}              
                        & \includegraphics[width=0.5in]{imgs/spoofing/randomization/horizon-random-hist.pdf}               
                        & \includegraphics[width=0.5in]{imgs/spoofing/randomization/L515-random-hist.pdf}               
                        & \includegraphics[width=0.5in]{imgs/spoofing/randomization/pixell-random-hist.pdf}
                        &
                        \includegraphics[width=0.5in]{imgs/spoofing/randomization/helios-random-hist.pdf}
                        \\ \hline
Dist. {[}$\mu$s{]} & $\mathcal{U}_{1.4, 1.8}$ & $\mathcal{U}_{4.0, 4.3}$ & $\mathcal{N}_{51, 0.025}$   & $\mathcal{U}_{4.5, 5.8}$ & $\mathcal{N}_{1.6, 0.005}$\\
Std. $\sigma$      & 33.3 m          & 26.0 m            & 7.5 m           & 110.4 m     &    1.5 m   \\
Max. $\Delta$      & 57.7 m          & 45.0 m            & 20.1 m           & 191.3 m     &   5.3 m      \\ \toprule
\end{tabular}
$\mathcal{U}_{{\rm min, max}}$ - Uniform distribution, $\mathcal{N}_{{\rm mean, std}}$ - Gaussian distribution
\vspace{0.1in}
\end{table}

\noindent\textbf{Pulse Fingerprinting.} We identify that Hesai XT32~\cite{XT32} can foil synchronization (and CPI attack capability) even without timing randomization due to its pulse fingerprinting. While we cannot be entirely sure due to the lack of official documentation, XT32 likely encodes its fingerprinting in the interval of the pair: As shown in Fig.~\ref{fig:hesai_fingerprint}, the XT32 pulse always forms a pair of pulses (two closely-connected spikes) corresponding to a single distance measurement.
However, we also find that the fingerprinting itself cannot perfectly defend against spoofing attacks because random spoofing can sometimes coincide with the fingerprinting interval, which can lead to up to 113 spoofed points as in Table~\ref{tbl:inject_attack}.
This could be because such pulse fingerprinting is originally developed for anti-interference purposes (e.g., to allow multiple LiDARs to operate at close range) instead of security and thus does not have enough randomness/entropy. Furthermore, it is not trivial to design more complex fingerprinting while ensuring eye safety as we will discuss in~\S\ref{sec:discussion}.
This means that if in each attack laser-firing event the attacker also fires a pair of pulses with the interval that can most likely coincide the fingerprinting interval (e.g., the one we found that can spoof 113 points), there can still be a random subset (e.g., 113) of the points in the attacker-chosen point cloud pattern that can be spoofed with the CPI attack capability. 
Thus, later in~\S\ref{sec:od_injection}, we model this effect on the spoofing capability as a random downsampling from a chosen pattern.


\begin{observation}{RQ2}
The current pulse fingerprinting is not complex enough to perfectly prevent spoofing attacks, likely because it is currently designed only for anti-interference purposes instead of security. However, it is not trivial to design a complex fingerprinting while ensuring eye safety.
\end{observation}


\begin{figure}[t!]
    \begin{minipage}{.63\linewidth}
\centering
\includegraphics[width=\linewidth]{imgs/spoofing/inter_inner_errors.pdf}
\vspace{-0.1in}
\caption{Illustration of inner- and inter-frame errors. Inner-frame error causes spoofing inaccuracy along with the ray direction within a frame. Inter-frame error causes the entire pattern to vibrate across consecutive frames.
}
\label{fig:inter_inner}
    \end{minipage}
    \hspace{0.01in}
    \begin{minipage}{.33\linewidth}
        \vspace{-0.1in}
        \centering
        \includegraphics[width=\linewidth]{imgs/spoofing/xt32_pulse.pdf}
        \caption{Examples of the receiving pulse shape of XT32~\cite{XT32}.
        }
        \label{fig:hesai_fingerprint}
   \end{minipage}
\end{figure}



\nsubsubsection{Case Study on Specific LiDARs}
\label{sec:specific-lidars}
Using our measurement setup, we also found 5 uniquely-interesting characteristics of specific LiDARs: the use of rare wavelength in OS1-32, relay attack on Leddar Pixell, zero-distance sensing of XT32, extra-wide vertical FOV of Helios 5515, and simultaneous laser firing on OS1-32 and VLS-128, which led to 2 new findings that are different from existing understandings in the community. Details are in Appendix~\ref{appendix:specific-lidars}.



\nsubsection{Object Detector-Level Measurements (RQ3)} 
\label{sec:od_injection}

\nsubsubsection{Modeling of the Spoofing Attack Capability in Point Injection} \label{sec:od_injection_modeling}
Similar to prior works in this problem space~\cite{cao2019adversarial, jiachen2020towards, hallyburton2022security}, to enable large-scale measurements of the attack capabilities on the object detector side, we need to mathematically model the point injection capabilities. Prior efforts in such modeling did not systematically consider (1) the modeling of the CPI attack capability (\S\ref{sec:cpi}), and (2) common new-gen LiDAR features that can fundamentally affect the object injection attack capabilities such as timing randomization and pulse fingerprinting. Leveraging this measurement study, we can thus develop a new modeling that addresses both fronts: 

\vspace{-0.2in}
\footnotesize
\begin{align}
    \mathcal{P}_I (x_{ij}) = 
     x_{ij} + (\delta_{ij}^{\rm rand} + \delta^{\rm inner}_{ij} + \delta^{\rm inter}) \cdot g(x_{ij}), \ x_{ij} \in \mathcal{C}_n \subset \mathcal{C} 
     \label{eq:attack_cap}
\end{align}
\normalsize
, where $\mathcal{C}$ is a point cloud (i.e., chosen pattern) that the attacker originally wants to inject (e.g., the point cloud of a vehicle); 
$\mathcal{C}_n$ is a point cloud randomly downsampled to $n$ points from $\mathcal{C}$ to model the impact from pulse fingerprinting (\S\ref{sec:sec_enchance_feats});
$x_{ij} \in \mathbb{R}^{3}$ is a point injected by the attack at $i$-th altitude and $j$-th azimuth; $g: \mathbb{R}^{3} \rightarrow \mathbb{R}^{3}$ is a function to obtain a movable unit direction of point $x$. Due to the physics of LiDAR, each point can only move along with the laser direction. As the LiDAR typically locates at the origin of the point cloud, $g(x_{ij})$ can be written as $\frac{x}{||x||_{2}}$ in this case.
To model the spoofing inaccuracy and the effect of the timing randomization, we add 3 types of errors: $\delta_{ij}^{\rm rand}$, $\delta^{\rm inner}_{ij}$, and $\delta^{\rm inter}$. Randomization error $\delta_{ij}^{\rm rand}$ is the error introduced by the timing randomization. It follows a certain distribution based on the target LiDAR as measured in Table~\ref{tbl:randomization}. 
We use $\delta_{ij}^{\rm rand} \sim \mathcal{N}(0, \sigma$) for the Gaussian distribution and $\delta_{ij}^{\rm rand} \sim \mathcal{U}(-\frac{\max - \min}{2}, \frac{\max - \min}{2})$ for the uniform distribution. $\delta_{ij}^{\rm inner}$ and $\delta^{\rm inter}$ are the inner-frame and inter-frame errors, respectively. Based on the measurements in~\S\ref{sec:cpi_capability}, we sample $\delta_{ij}^{\rm inner}$ from $\mathcal{N}(0, 10 \ {\rm cm})$ and sample $\delta^{\rm inter}$ from $\mathcal{N}(0, 35 \ {\rm cm})$. Note that $\delta^{\rm inter} \in \mathbb{R}$ does not depend on each point $x_{ij}$, i.e., one scalar value is sampled per case and applied to all points in the case. To the best of our knowledge, this is the first modeling
of the point injection capability that can cover both first- and new-gen LiDAR features.

\newpart{Note that this modeling does not cover the impacts from the simultaneous laser firing feature (\S\ref{sec:sim_fireing}). This is because although such a feature can make the CPI attack capability infeasible, (1) it cannot prevent the point injection itself; (2) details of the simultaneous firing pattern are not well documented; and (3) the more recent LiDARs (e.g., the ones after 2019 in Table~\ref{tbl:target_lidars}) do not have such a feature. We thus leave its modeling to future work.}











\nsubsubsection{Experimental Setups} \label{sec:model_level_scenario}
We perform the experiments based on a scenario in the KITTI dataset.
Fig.~\ref{fig:synth_scenario} in Appendix %
depicts the generated scenario: we place a 3D vehicle object in front of the victim and move the corresponding points to the object's surface, following the same methodology used in~\cite{jiachen2020towards, cao2021invisible}. 
We generate 15 scenarios varying the distance between the victim to the vehicle object from 0 m to 14 m (0 m means the victim's nose touches the vehicle object's tail). 
Fig.~\ref{fig:n_points} in Appendix shows the number of points on the injected vehicle object.
In addition to the original point clouds, we evaluate point clouds downsampled from the original one to 10, 50, 100, and 200 points (i.e., $n = 10, 50, 100,$ and $200$ in Eq.~\ref{eq:attack_cap}) as a modeling of the fingerprinting effect  (\S\ref{sec:sec_enchance_feats}).
We judge the object injection as successful if there exists a detected object overlapping with the ground truth area of the injected object, i.e., the IoU between the ground truth and detected object is more than zero. 



\vspace{0.05in}
\nsubsubsection{Results} \label{sec:od_injection_eval}  \label{sec:impact_model} \label{sec:impact_data} \label{sec:impact_noise}
~

\textbf{Impact of Error Modeling on Spoofing Accuracy:}
We measure the impacts of 3 types of different modeling: without errors, our modeling with inner- and inter-frame errors (\S\ref{sec:cpi_capability}), and the error modeling used in the Frustum attack~\cite{hallyburton2022security}, which is the latest modeling effort in prior works and simply adds Gaussian errors along with the vehicle's Cartesian coordinates (forward, left, and up) following ($\mathcal{N}(1, 0.1)$, $\mathcal{N}(0, 0.5)$, $\mathcal{N}(1, 0.2)$) in meters, respectively.
Fig.~\ref{fig:obj_impact_errors} shows the attack success rates against our targeted models with different architectures and different training datasets (\S\ref{sec:target_lidar_od}).
As shown, the different error modeling can generally cause significant differences in the attack success rates, e.g., the results without errors and with the naive modeling in~\cite{hallyburton2022security} can differ 96\% and 70\% respectively on average to those using our modeling. Specifically, the latest modeling effort in~\cite{hallyburton2022security} significantly over-estimates the errors along their forward and up coordinates, which causes the model-level success rates to be generally lower than those with our more systematic modeling based on actual experiments (\S\ref{sec:cpi_capability}).




\begin{observation}{RQ1} 
Error modeling on the spoofing accuracy can significantly affect the object detector-level attack results. The latest prior efforts~\cite{hallyburton2022security} largely overestimate the errors as compared to our systematically modeled and quantified ones via experiments. 
\end{observation}
\vspace{-0.1in}

\begin{figure}[t!]
\centering
\includegraphics[width=\linewidth]{imgs/obj_detector/obj_comp_noises.pdf}
\vspace{-0.2in}
\caption{
Object injection attack success rates under 3 different types of noises on (a) multiple models trained on the KITTI dataset and (b) PointPillars trained on different datasets.
}
\label{fig:obj_impact_errors}
\end{figure}

\textbf{Impact of Pulse Fingerprinting:}
Fig.~\ref{fig:obj_fingerprinting} shows the attack success rates under different down-sampling levels $n$ as a modelling of different pulse fingerprinting levels (\S\ref{sec:sec_enchance_feats}). 
To perform control experiments of the fingerprinting impact factor, we do not apply inter- and inner-frame errors in this experiment. 
Generally, fingerprinting with higher complexity (i.e., lower $n$) shows higher defense capability. However, when $n=100$, which can represent the fingerprinting complexity level today (XT32 as measured in~\S\ref{sec:sec_enchance_feats}), the defense effectiveness is actually still very minimal (reduce the attack success rate only by 3\% on average). The defense effectiveness only becomes significant when $n$ reaches as low as 10, while such higher effectiveness is still model architecture and training dataset dependent.

\begin{observation}{RQ3}
If with enough complexity, pulse fingerprinting can indeed show high defense capability to object injection attacks (although can be object detector model architecture and training dataset dependent). Unfortunately, the complexity of pulse fingerprinting today may not be enough to achieve such high defense capability. 
\label{finding:pulsefp}
\end{observation}
\vspace{-0.1in}


\begin{figure}[t!]
\centering
\includegraphics[width=\linewidth]{imgs/obj_detector/obj_fingerprinting.pdf}
\vspace{-0.2in}
\caption{
Object injection attack success rates under different down-sampling levels $n$ on (a) multiple models trained on the KITTI dataset and (b) PointPillars trained on different datasets.
}
\label{fig:obj_fingerprinting}
\end{figure}

\textbf{Impact of Timing Randomization:}
Table~\ref{tbl:inj_rand_arch} and~\ref{tbl:inj_rand_data} show the attack success rates under different timing randomization levels based on our measurements in~\S\ref{sec:sec_enchance_feats}. As shown, timing randomization can in general significantly reduce the attack success rates, e.g., the average attack success rates with randomization are dramatically reduced to at most 34\% for most models, while those without randomization are at least 88\%. We also measured the attack success rates with both timing randomization and pulse fingerprinting, with the fingerprinting level set at the observed level today ($n=100$, based on XT32 measurements in~\S\ref{sec:sec_enchance_feats}). However, consistent with Finding~\ref{finding:pulsefp}, we are not able to observe significant changes in the attack success rate reduction.

\begin{observation}{RQ3}
Timing randomization, even with those with low randomization entropy that is realized for anti-interference purposes instead of security today, can have significant defense capabilities against object injection attacks. Today's fingerprinting complexity level again may not be able to significantly help boost the defense capabilities when used together with timing randomization.
\end{observation}


Meanwhile, we further notice that the defense effectiveness with timing randomization can differ significantly when applied to models trained on different datasets. For example, as shown in Table~\ref{tbl:inj_rand_data}, the attack success rates are reduced by only 10\% and even 0\% on average across all randomization levels for the models trained on Lyft and nuScenes respectively, while the success rates can be reduced by as high as 84\% and 88.6\% for the same model design trained on Waymo dataset and the private dataset from Apollo. In particular, the models trained on Lyft and nuScenes are particularly vulnerable to object injection attacks since as shown in Fig.~\ref{fig:obj_fingerprinting}, these models can detect the vehicle even when the vehicle point cloud is randomly downsampled to as few as only 10 points. This suggests that the training dataset choice can play a significant role in determining the model vulnerability to spoofing attacks. Later in~\S\ref{sec:od_removal}, we will continue investigating this aspect together with the object removal attack measurement results.





\begin{table}[t!]
\centering
\footnotesize
\setlength{\tabcolsep}{1.3pt}
\setlength{\aboverulesep}{0pt}
\setlength{\belowrulesep}{0pt}
\renewcommand{\arraystretch}{0.75}
\caption{Object injection attack success rates under different randomization levels based on our measurements in~\S\ref{sec:sec_enchance_feats} on \textit{different model architectures}. $\emptyset$: No randomization. Avg.: The average results across all these randomization levels.
}
\label{tbl:inj_rand_arch}
\begin{tabular}{ccccccc}
\toprule
LiDAR   & Rand. model {[}m{]}         & PointPillars & SECOND & PartA$^{2}$ & 3DSSD & PV-RCNN \\\hline\hline
VLP-16  & $\emptyset$                 & \underline{100\%}        & \underline{100\%}  & 80\%   & 93\%  & 97\%    \\\hline\hline 
Helios  & $\mathcal{N}(0, 1.5)$       & 2\%          & 54\%   & 41\%   & 7\%   & 24\%    \\
L515    & $\mathcal{N}(0, 7.5)$       & \textbf{0\%}          & 24\%   & 14\%   & 7\%   & \textbf{0\%}     \\
Horizon & $\mathcal{U}(-45, 45)$      & 39\%         & 35\%   & 21\%   & 30\%  & 17\%    \\
OS1-32  & $\mathcal{U}(-58, 58)$ & 47\%         & 38\%   & 21\%   & 28\%  & 23\%    \\
Pixell  & $\mathcal{U}(-191, 191)$   & 60\%         & 21\%   & 20\%   & 8\%   & 43\%    \\ \hline
        & Avg.                         &   30\%         & 34\%   & 23\%   & 16\%  & 21\%  \\\hline\hline
  \multicolumn{3}{l}{With fingerprinting effect $n=100$:}        &     &     &    &     \\
                    & Avg.     & 38\%         & 20\%   & 16\%   & 18\%  & 41\%   \\\toprule
\end{tabular}
\end{table}


\begin{table}[t!]
\centering
\footnotesize
\setlength{\tabcolsep}{3.5pt}
\setlength{\aboverulesep}{0pt}
\setlength{\belowrulesep}{0pt}
\renewcommand{\arraystretch}{0.75}
\caption{Object injection attack success rates under different randomization levels based on our measurements in~\S\ref{sec:sec_enchance_feats} on \textit{models trained on different datasets}. $\emptyset$: No randomization.  Avg.: The average results across all these randomization levels.
}
\label{tbl:inj_rand_data}
\begin{tabular}{ccccccc}
\toprule
LiDAR   & Rand. model {[}m{]}         & KITTI & Lyft  & nuScenes & Waymo & Apollo \\ \hline\hline
VLP-16  & $\emptyset$                 & \underline{100\%} & \underline{100\%} & \underline{100\%}    & \underline{100\%} & 88\%   \\\hline\hline 
Helios  & $\mathcal{N}(0, 1.5)$       & 2\%   & \underline{100\%} & \underline{100\%}    & 26\%  & 48\%   \\
L515    & $\mathcal{N}(0, 7.5)$       & \textbf{0\%}   & 60\%  & \underline{100\%}    & \textbf{0\%}   & \textbf{0\%}    \\
Horizon & $\mathcal{U}(-45, 45)$      & 39\%  & 96\%  & \underline{100\%}    & 7\%   & \textbf{0\%}    \\
OS1-32  & $\mathcal{U}(- 58, 58)$ & 47\%  & 99\%  & \underline{100\%}    & 12\%  & \textbf{0\%}    \\
Pixell  & $\mathcal{U}(- 191, 191)$   & 60\%  & 96\%  & \underline{100\%}    & 36\%  & 1\%    \\ \hline
        & Avg.                         & 30\%  & 90\%  & \underline{100\%}    & 16\%  & 10\%   \\ \hline\hline
  \multicolumn{3}{l}{With fingerprinting effect $n=100$:}        &     &     &    &     \\
        & Avg.                         & 38\%  & 85\%  & 92\%    & 33\%  & 5\%   \\ \toprule
\end{tabular}
\end{table}




\nsection{Object Removal Attack Measurements} \label{sec:removal_attack}

After the measurements on object injection attacks, in this section we measure the other important class of LiDAR spoofing attacks: object removal attacks (\S\ref{sec:attack_taxonomy}). 

\nsubsection{LiDAR-Level Measurements (RQ1, RQ2)}\label{sec:lidar_removal}

Table~\ref{tbl:hfa} shows the LiDAR-level attack capability measurement results for the PRA attack and the newly-identified HFR attack (\S\ref{sec:high_freq_attack}) on different LiDARs. To count the removed points, starting from the attacked point cloud, we first identify the attack-induced spoofed points using the same method as in \S\ref{sec:lidar_injection} and remove them. Next, we subtract the remaining points in the attacked point cloud from the benign point cloud; now the remaining points in the benign point cloud are thus the removed benign points by the attack (detailed in Appendix~\ref{appndix:count_method}).
We select 1 MHz as the pulse frequency and 80V as the laser drive voltage for HFR (detailed in Appendix~\ref{appndix:hfr_freq}).



As shown in Table~\ref{tbl:hfa}, with our spoofer improvements in~\S\ref{sec:spoofer_design}, PRA can now achieve over 6,600 removed points on VLP-16, which is 65.5\% more than the original paper~\cite{cao2023you}. However, such strong attack capability is limited to the first-gen LiDARs; for the new-gen ones, PRA is not applicable to any of them since its requirement of synchronization is directly foiled by common next-gen LiDAR features such as timing randomization and pulse fingerprinting (\S\ref{sec:sec_enchance_feats}). 

\begin{observation}{RQ1}
Due to the requirement of synchronization, the basic design assumption of the latest object removal attack is generally broken for next-gen LiDARs due to common features such as timing randomization and pulse fingerprinting. 
\end{observation}

\begin{figure}[t!]
\centering
\vspace{0.1in}
\includegraphics[width=\linewidth]{imgs/spoofing/HFR_attack_indoor_demo.pdf}
\caption{HFR attack effect on VLP-32c. Patterns of a person and the majority of the room wall are completely removed. 
}
\label{fig:removal_attack}
\end{figure}

Since there is no need of synchronization, as shown in Table~\ref{tbl:hfa}, the newly-identified HFR attack can be generally applied to all LiDARs in the table, no matter if the targeted LiDAR is first- or new-gen. Specifically, for the first-gen ones, HFR is able to achieve comparable point removal capability to PRA, e.g., removing $>$5,300 points in $>$85$^{\circ}$. Fig.~\ref{fig:removal_attack} shows an example of the HFR attack effect on VLP-32c. As shown, the point cloud patterns of a person and the majority of the room wall are completely removed, with only some points with random noise patterns left.
Compared to PRA, the point removal success rate of HFR is slightly slower (72-78\% versus 80-97\% in PRA), which is expected since PRA has more precise control of points with synchronization. 

For next-gen LiDARs with timing randomization, HFR is still highly-effective since the high-frequency lasers can still hit all legitimate measurements within the affected azimuthal range despite the randomized timing of legitimate laser firing, showing the capability of removing 4k-206k points (for OS1-32 it is only 28 due to the lack of laser diodes with matching wavelength, explained in~\S\ref{sec:sec_enchance_feats}). For Helios, the point removal success rate is lower mainly due to its extra-wide vertical FOV (Appendix~\ref{sec:case_helios}); in the vertical range critical to real-world attack scenarios (e.g., 30-40$^{\circ}$ in the center), the success rate is similar to others (Fig.~\ref{fig:hfa_asr}). For next-gen LiDAR with pulse fingerprinting (XT32), the point removal capability is significantly reduced due to the difficulty for the attack pulses to coincide with the fingerprinting interval. However, as we found in~\S\ref{sec:sec_enchance_feats}, there are still chances to randomly remove $\sim$100 points by using a pulse frequency that is most likely to coincide with the fingerprinting interval.


\begin{observation}{RQ2}
\label{finding:hfr}
For new-gen LiDARs, although they are no longer generally vulnerable to the latest practical attacks such as PRA, they unfortunately still remain generally vulnerable to object removal attacks, with a similar level of practical attack capabilities as PRA, due to the possibility of new asynchronized removal attack designs such as HFR.
\end{observation}










\begin{table}[t!]
\centering
\footnotesize
\setlength{\tabcolsep}{1.2pt}
\setlength{\aboverulesep}{0pt}
\setlength{\belowrulesep}{0pt}
\renewcommand{\arraystretch}{0.9}
\caption{
LiDAR-level attack capability measurements for object removal attacks. PRA~\cite{cao2023you} is only applicable to the first-gen LiDARs. Symbols are the same as in Table~\ref{tbl:distance}.
}
\label{tbl:hfa}
\begin{tabular}{cc|cc|cccc|c}
\toprule
 &       & \multicolumn{2}{c|}{\multirow{2}{*}{First-Gen}} & \multicolumn{5}{c}{New-Gen}  \\  \cline{5-9}
 &       &  & & \multicolumn{4}{c|}{w/ Timing Randomization}    & w/ Fingerprint  \\  \cline{3-9}
 &       & VLP-16  & VLP-32c & OS1-32 & Helios    & Horizon  & L515    & XT32  \\  
 \hline  \hline
\multirow{3}{*}{\begin{tabular}[c]{@{}c@{}}PRA\\ \cite{cao2023you} \\ \end{tabular}} & $\mathcal{N}$     & 6,621    & 9,711    & N/A      & N/A       & N/A       & N/A & N/A     \\
                     & $\mathcal{R}$     & 96.9\% & 82.9\% & N/A      & N/A       & N/A       & N/A  & N/A    \\
                     & $\theta$ & 85.4$^{\circ}$ & 73.2$^{\circ}$ & N/A      & N/A       & N/A       & N/A & N/A     \\ \hline
\multirow{3}{*}{\begin{tabular}[c]{@{}c@{}}HFR\\(\S\ref{sec:high_freq_attack})\end{tabular}} & $\mathcal{N}$     & 5,358    & 8,778    & 28 & 4,108    & 19.2k    & 206k   & 113   \\
                     & $\mathcal{R}$     & 78.1\%  & 72.2\% & 43.8\%  & 24.8\%  & 79.9\%   & 91.3\%   & 2.1\% \\
                     & $\theta$ & 85.8$^{\circ}$ & 76.0$^{\circ}$ & 0.72$^{\circ}$  & 103.4$^{\circ}$  & 81.7$^{\circ}$ & 70.0$^{\circ}$ & 34.2$^{\circ}$ \\ \toprule
\end{tabular}
\raggedright

* N/A: Attack is not applicable to the LiDAR\\
\end{table}





\nsubsection{Object Detector-Level Measurements (RQ3)} \vspace{0.1in}

\nsubsubsection{Modeling of the Spoofing Attack Capability in Point Removal} \label{sec:od_removal_modeling}
Similar to the object injection attack side, we first perform a mathematical modeling of the spoofing attack capabilities in point removal in order to enable large-scale object detection-level attack capability measurements. For removal attacks, the point removal goal is the same for all point measurements that can be hit by the attack laser (e.g., move to within MOT for PRA, or place the point to a random position for HFR). Thus, the main factor that can affect the removal capability is whether the point is at an azimuth angle that can be effectively hit by the attack laser. For example, Fig.~\ref{fig:hfa_asr} shows the point removal percentage for the azimuth angles under attack. As shown, for VLP-16 under both PRA and HFR, the points in the center of the attack laser-hit azimuth range are almost 100\% removed regardless of altitudes and distances, while the removal percentages symmetrically decrease from the center to the side.





Based on these observations, we model the attack capability for point removal $\mathcal{P}_R$ as follows:
\vspace{-0.06in}
\begin{align}\footnotesize
    \mathcal{P}_R (x_{ij}) :=
    \begin{cases}
    \xi \cdot g(x_{ij}) & {\rm if } \  \operatorname{Bernoulli}(p_j) = 1 \\
    x_{ij} & {\rm othewise } 
  \end{cases}
     \label{eq:attack_modeling_removal}
\end{align}\vspace{-0.0in}\normalsize, where $x_{ij} \in \mathbb{R}^{3}$ is its point at $i$-th altitude and $j$-th azimuth. For each point, we decide whether it will be removed or not based on a Bernoulli trial with a per-azimuth probability $p_j$, which can be set using the measured point removal percentage as in Fig.~\ref{fig:hfa_asr}. $\xi$ models the removal effects for both HFR and PRA. For HFR, 
$\xi$ is the error of the HFR attack, which distributes the original points to random locations along with the laser direction $g(x_{ij})$. 
Thus, $\xi$ can be written as 
$\xi := \mathcal{U} \left ( 0, \frac{c}{2} \cdot \frac{1}{f} \right )$, where $\mathcal{U}(a, b)$ is the uniform distribution with the minimum $a$ and maximum $b$ values, $c$ is the speed of light, $f$ is the frequency of HFR attack pulses. This is derived from the fact that the maximum ToF is capped by the time interval of the high-frequency attack pulses in HFR. For PRA, $\xi$ is set to 0 since the point removal principle of PRA is to move the points toward the origin as much as possible to locate them within MOT (\S\ref{sec:attack_taxonomy}). 
After applying Eq.~\ref{eq:attack_modeling_removal}, if $\mathcal{P}_R (x_{ij})$ is within the MOT or outside of the maximum detection range (Table~\ref{tbl:target_lidars}), point $x_{ij}$ is removed from the point cloud. To the best of our knowledge, this is the first mathematical modeling of the point removal capability for LiDAR spoofing attacks.



\begin{figure}[t!]
\centering
\vspace{-0.1in}
\includegraphics[width=\linewidth]{imgs/simulator/angle_asr.pdf}
\vspace{-0.3in}
\caption{Point removal percentage of PRA~\cite{cao2023you} and the HFR attack (\S\ref{sec:high_freq_attack}) for the azimuth angles under attack.%
}
\label{fig:hfa_asr}
\end{figure}


\vspace{0.1in}
\nsubsubsection{Model-Level Attack Capability Measurement}
\label{sec:od_removal}
~




\textbf{Experimental Setup.}\label{sec:model_level_scenario_removal}
We use the same evaluation scenarios as in~\S\ref{sec:model_level_scenario} (i.e., 15 scenarios varying the distance between the victim to the vehicle object from 0m to 14m). We consider the object removal as successful if the IoU between any detected objects and the ground truth object is 0. When applying $\mathcal{P}_R(.)$, we set $p_j$ using the measured point removal percentages for different LiDARs as in Fig.~\ref{fig:hfa_asr}. Note that for Helios, since its vertical FOV is much wider than others (Appendix~\ref{sec:case_helios}), we only calculate the point removal percentage for the FOV range related to our attack scenario (i.e., 33$^{\circ}$, which is enough to cover the height of the front vehicle to remove).






\label{sec:od_removal_eval}
\textbf{Results.} Fig.~\ref{fig:removal_noise_data} shows the object removal attack success rates for object detectors with different architectures and datasets. As shown, for the majority of the cases, HFR can reach similar success rates as PRA, which thus further extends our Finding~\ref{finding:hfr} to the object detector level. As a validation of such attack effectiveness in the physical world, Fig.~\ref{fig:hfa_real_car} shows the object removal attack effect against real vehicles using the HFR attack. As shown, the HFR attack is found to cause  5 front vehicles to become undetected by the Apollo model~\cite{apollo}, with 100\% success rate for over 10 seconds. In the figure, we can see the spatial features of the vehicles were completely destroyed (with some random points left) and thus no objects were detectable in the attacked region.

For new-gen LiDAR features, similar to the object injection side, timing randomization (Helios in Fig.~\ref{fig:removal_noise_data}) can significantly reduce the attack success rate in the majority of the cases (by 35\% on average from VLP-16). However, pulse fingerprinting shows much higher defense effectiveness compared to the object injection side: with the pulse fingerprinting strength level of XT32 (i.e., $\sim$100 randomly removed points), the average attack success rate is significantly reduced by 63\% on average from VLP-16, while such a reduction is 3\% on average on the object injection side (Fig.~\ref{fig:obj_fingerprinting}). 
\newpart{This is due to the trade-off of the object detector’s sensitivity to object injection and removal attacks. Since fingerprinting only allows a random subset of 100 points to be injected, the object detectors that are more vulnerable to object injection are those that are very sensitive to even a heavily-occluded point cloud pattern with a very small number of object points.}

\begin{observation}{RQ3}
Both timing randomization and pulse fingerprinting show high defense capabilities against object removal attacks in general; the defense capability from pulse fingerprinting is especially strong when compared with that against object injection attacks.
\end{observation}

In addition, similar to our findings on the object injection attack side, the model robustness to object removal attacks shows a high diversity across different model architectures and different training datasets, which is especially prominent across training datasets. Interestingly, the model robustness properties are generally the \textit{opposite} across object injection and removal attacks: for example, the models trained on Waymo and by Apollo are found to be much more robust against object injection attacks when compared to those trained on Lyft and nuScenes (\S\ref{sec:od_injection}), while the former ones become much less robust than the latter ones for object removal attacks. This might be because different dataset has different balances of the trade-off between false positives and more false negatives for the corner cases. As shown in Fig.~\ref{fig:obj_fingerprinting}, models trained on Lyft and nuScenes can have 100\% detection rate of a vehicle even when the point cloud is randomly downsampled to as few as 10 points. This can allow the trained models to have a very low false negative rate even for highly-occluded legitimate vehicle point cloud patterns, which can thus make the models highly robust to object removal attacks, but this also makes them highly vulnerable to object injection attacks.

\begin{observation}{RQ3}
The model robustness to object injection and removal attacks can be highly diverse when trained on different datasets. The choice of training datasets can incur large trade-offs between the model robustness to object injection attacks and that to object removal attacks.
\end{observation}









\begin{figure}[t!]
\centering
\includegraphics[width=\linewidth]{imgs/obj_detector/removal_noise_data_bar2.pdf}
\vspace{-0.2in}
\caption{
Object removal attack success rates of HFR and PRA for (a) 5 different models trained on the KITTI dataset and (b) PointPillars trained on 5 different datasets. %
}
\label{fig:removal_noise_data} \label{fig:removal_noise_arch}
\end{figure}



\label{sec:case_study}


\begin{figure}[t!]
\centering
\includegraphics[width=\linewidth]{imgs/case_study/hfa_real_car2.pdf}
\caption{Object removal attack effect against real vehicles using the HFR attack. The 5 front vehicles become undetected with a 100\% success rate for over 10 seconds (100 frames in total) by PointPillars~\cite{lang2019pointpillars} in Apollo~\cite{apollo}.
}
\label{fig:hfa_real_car}
\end{figure}



\nsubsubsection{System-Level Attack Capability Measurement}
\label{sec:od_removal_system}
Due to the downstream components such as object tracking in real-world applications such as autonomous driving (AD), object detector-level object removal effect may not directly imply system-level attack effect such as vehicle crashes~\cite{jia2020fooling, shen2022sok}. Thus, in this section we further measure the attack capabilities of different object removal attacks at the system level, with the focus on the representative application: AD.

 

\textbf{Experimental Setup.} \label{sec:system_level_scenario}
We use a common setup widely used in previous work~\cite{cao2019adversarial, jiachen2020towards, hallyburton2022security, cao2023you}. For the AD system, we use Baidu Apollo 7.0~\cite{apollo}. For the driving simulator, we use LGSVL~\cite{lgsvl}. Both systems are known as industry-grade software.
Fig.~\ref{fig:overview_sim} in Appendix illustrates the experiment scenario. 
We place a sedan vehicle as a target object 200 m away from the victim AD and make the victim AD drive toward the target. We add up to 1 m random perturbation both laterally and longitudinally to (1) the starting point of the victim and (2) the target object. Before reaching the target object, the AD vehicle reaches and keeps 40 km/h. 
We set an attack start distance $\mathcal{D}$ and only start to apply the simulated removal attack effect when the distance from the victim AD vehicle to the front vehicle is $\leq\mathcal{D}$.
We use the collision rate over 10 trials as the evaluation metric, i.e., how many times the victim vehicle collides with the sedan out of 10 trials.






\textbf{Results.}
Table~\ref{tbl:sim_hfa} shows the collision rate over 10 trials for different LiDARs. As shown, although HFR has relatively weaker attack capabilities than PRA at raw point cloud (\S\ref{sec:lidar_removal}) and object detector (\S\ref{sec:model_level_scenario_removal}) levels due to the lack of synchronization, their attack capabilities at the system level are almost the same at every attack start distances $\mathcal{D}$, and reach 100\% collision rate when $\mathcal{D}$ is over 18 m. However, due to the requirement of synchronization, PRA is only applicable to first-gen LiDARs such as VLP-16, while HRA can be generally applied to new-gen ones such as Helios with similar attack effectiveness. Fig.~\ref{fig:sim_attack} shows an example attack trial under the HFR attack with $\mathcal{D}$ = 15 m on Helios. The target sedan can be momentarily detected before the collision, but the detection disappears soon and the sedan becomes undetected until and also after the collision.

For next-gen LiDAR with pulse fingerprinting (XT32), although HFR can still have $\leq$32\% model-level attack success rate (\S\ref{sec:od_removal_modeling}), it cannot achieve any system-level attack success at all (0\%), which is likely because the object removal rate is not high enough to fool object tracking~\cite{jia2020fooling}. This suggests that pulse fingerprinting can be quite effective in defending against object removal attacks at the system level. videos of these experiments can be found at our website~\cite{project_page}.









\begin{observation}{RQ3}
Although the HFR attack has weaker attack capabilities than PRA at point cloud and object detector levels, the system-level attack effect is almost the same. Since the HFR attack does not require synchronization, this means that real-world LiDAR applications such as autonomous driving, no matter if using first- or new-gen LiDARs, are facing practical threats from object removal attacks. Meanwhile, pulse fingerprinting can be quite effective in defending against the removal attack effect at the system level.
\end{observation}
\vspace{-0.13in}






\begin{figure}[t!]
\centering
\includegraphics[width=\linewidth]{imgs/simulator/sim_attack_case.pdf}
\caption{An example HFR attack trial with $\mathcal{D}$ = 15 m on Helios. The remaining points are occasionally detected as an object but are not sufficient to avoid a collision.}
\label{fig:sim_attack}
\vspace{-0.05in}
\end{figure}


\begin{table}[t!]
\centering
\footnotesize
\caption{Vehicle collision rates over 10 trials using PRA and HFR for different LiDARs with varying attack start distances. \textbf{Bold} and \underline{underline} highlight %
100\% and 0\% collision rates.}
\label{tbl:sim_hfa}
\setlength{\tabcolsep}{3.4pt}
\renewcommand{\arraystretch}{0.8}
\begin{tabular}{cccccccccc}
\toprule
      &       & Benign & 10m & 15m & 16m & 17m  & 18m  & 19m  & 20m  \\ \hline
PRA & VLP-16 & \underline{0/10}   & \underline{0/10} & 5/10 & 8/10 & 9/10 & \textbf{10/10} & \textbf{10/10} & \textbf{10/10} \\\hline
   & VLP-16 & \underline{0/10}   & \underline{0/10} & 6/10 & 7/10 & 8/10  & \textbf{10/10} & \textbf{10/10} & \textbf{10/10} \\
HFR & VLP-32c & \underline{0/10}   & 1/10 & 9/10 & 8/10 & \textbf{10/10} & \textbf{10/10} & \textbf{10/10} & \textbf{10/10} \\
   & XT32 & \underline{0/10}   & \underline{0/10} & \underline{0/10} & \underline{0/10} & \underline{0/10}  & \underline{0/10}  & \underline{0/10}  & \underline{0/10}  \\ 
   & Helios & 	\underline{0/10} & 	\underline{0/10} & 	6/10 & 	5/10 & 	\textbf{10/10} & 	\textbf{10/10} & 	\textbf{10/10} & 	\textbf{10/10}\\
\toprule
\end{tabular}
\end{table}


We provide some comments on the growth conditions which constituted the majority of our analysis in sections \ref{sec:Hmixing} and \ref{sec:Hsigma}. In the simplest cases of Lemma \ref{lemma:unstableGrowth}, growth was established in an analogous fashion to the old one-step expansion condition (\ref{eq:oldOneStepExpansion}), finding the relevant Jacobians $M_j$ and checking that their expansion factors $K(M_j)$ satisfy
\begin{equation}
    \label{eq:discussionOneStep}
    \sum_j \frac{1}{K(M_j)} <1.
\end{equation}
For the more complicated cases, the inductive method used to establish growth near the accumulation points in Lemma \ref{lemma:unstableGrowth} and the weakened one-step expansion condition (\ref{eq:oneStep}) both address the same fundamental issue: the splitting of unstable curves by singularities into an unbounded number of small components. They circumvent this obstacle in rather different ways, however. While (\ref{eq:oneStep}) generalises (\ref{eq:discussionOneStep}) to ensure an growth of unstable curves `on average' (see \cite{chernov_statistical_2009} for a precise statement), our inductive method is a more direct adaptation of (\ref{eq:discussionOneStep}), using it to generate contradictory geometric conditions which a hypothetical non-growing unstable curve must satisfy. It may be possible to prove Theorem \ref{sec:Hmixing} using (\ref{eq:oneStep}) as the basis for growth. Since we required (\ref{eq:oneStep}) anyway for proving Theorem \ref{thm:HsigmaExp}, this could potentially condense our analysis, but only to a minor extent. A convenience of the method used in section \ref{sec:Hmixing} is that, by way of the `simple intersection' property, it naturally gives geometric information on the images of manifolds, useful for proving the property \textbf{(M)} of Theorem \ref{thm:katok-strelcyn}.

We expect that essentially analogous analysis can be applied to establish mixing properties in a wide class of piecewise linear non-uniformly hyperbolic maps, including those (like the OTM) which sit on the boundary of ergodicity and beyond. While we have relied on the precise partition structure of $H_\sigma$, its fundamental feature (self-similar sequences of elements $A^k$, sharing boundaries with its neighbours $A^{k-1},A^{k+1}$ and accumulating onto some point $p$) is quite typical to return map systems. See, for example, those of various stadium billiards \cite{chernov_chaotic_2006,chernov_improved_2008,chernov_statistical_2009} and LTMs \cite{springham_polynomial_2014}. Indeed, the same method can be used to prove the Bernoulli property for non-monotonic LTMs \cite{myers_hill_mixing_2022}, where monotonicity of the manifold images cannot be assumed and the classical argument \cite{sturman_mathematical_2006} fails. The OTM is the pointwise limit of these maps as the boundary shrinks to null measure. It further has utility in proving growth conditions for maps which are uniformly hyperbolic but possess regions $A_j$ where the hyperbolicity is very weak, signified by $K(M_j) \approx 1$, so that (\ref{eq:discussionOneStep}) fails. Typically this leads to suboptimal bounds on mixing windows, see e.g. \cite{wojtkowski_model_1981,przytycki_ergodicity_1983,myers_hill_family_2022}. The map $H_{(\eta,\eta)}$ for $\eta \approx 1/2$ is another example, possessing weak hyperbolicity over $A_2, A_3$. Letting $\varepsilon = |\eta-1/2|>0$, there is an upper bound $N = N(\varepsilon)$ on escape times from the intersections $A_2\cap \sigma, A_3 \cap \sigma$. The growth lemma then follows by applying the inductive step roughly $N$ times and can be established for arbitrarily small $\varepsilon$, opening the door to establishing optimal mixing windows.

The above gives two examples of piecewise linear perturbations to $H$ where mixing with respect to Lebesgue is preserved and our methods can be applied. Nonlinear perturbations to the shear profiles complicate the analysis in several ways. Firstly as the map's Jacobians takes on a broader range of values, cone invariance becomes an increasingly harder condition to establish. Cones must be widened, giving looser bounds on expansion factors, which may already be weak due to new regions of weaker stretching. This, together with the change from polygonal to curvilinear return time partition elements and nonlinear local manifolds, adds some complexity to showing growth conditions. This does not rule out certain (small) nonlinear perturbations however. There is some leeway in the inequalities which govern cone invariance and growth of local manifolds, the latter of which is not too dissimilar from the piecewise linear setting (see Lemmas \ref{lemma:piecewiseApprox}, \ref{lemma:componentLength}). Certain small perturbations would not alter the \emph{topological} structure of the return time partition, i.e. which elements share boundaries, the key information needed for setting up the induction. Finally while the partition elements would no longer be polygonal, only coarse geometric information is required for verifying each inductive step. Following the above, a potential perturbation could be to replace the linear portions of each shear by a cubic, perturbing the tent profile
\[  f(t) = \begin{cases} 2t & 0 \leq t \leq 1/2, \\ 2(1-t) & 1/2 \leq t \leq 1 ,\end{cases} \]
of the OTM shears to
\[  f_a(t) = \begin{cases} \frac{1}{8} t \left(16 - a + 6at - 8at^{2} \right) & 0 \leq t \leq 1/2, \\ \frac{1}{8}\left(1-t\right)\left( 16 - a + 6a\left(1-t\right) - 8a\left(1-t\right)^{2}\right)  & 1/2 \leq t \leq 1, \end{cases}   \]
for $a>0$. For small enough $a$ the gradient range $f'(t)$ is restricted to small neighbourhoods of $\{ 2, -2\}$ and the escape time partition retains a similar structure. We illustrate this in Figure \ref{fig:perturbations}, showing escapes from the square $S_3$ under the map $G \circ F$, equivalent to escapes from the perturbed $A_3$ under the $G \circ F$, but with a cleaner geometry for comparison. When $a$ is too large the analogy to the OTM breaks down. At $a=16$ the map is twice differentiable everywhere and features a new source of slowed mixing, the Jacobian is the identity at the corner points $x,y \in \{  0, 1/2 \}$ giving locally parabolic behaviour (visible in the escape time partition). 

\begin{figure}
    \centering
    \includegraphics[width=0.24 \linewidth]{0.png}
    \includegraphics[width=0.24 \linewidth]{4.png}
    \includegraphics[width=0.24 \linewidth]{8.png}
    \includegraphics[width=0.24 \linewidth]{16.png}
    \caption{Partition of escape times from $S_3$ under the mapping $F \circ G$ for $a= 0,4,8,16$. }
    \label{fig:perturbations}
\end{figure}
\section{Conclusion}\label{sec:conclusion}
In this work, we focus on addressing the fundamental challenge of OOD detection tasks, which is how to fully understand the semantic discrepancy between the ID/OOD samples. We reveal that the key to success in the realistic SCOOD task is to allocate as many ID samples in the unlabeled set correctly as possible. To this end, we propose a novel uncertainty-aware optimal transport scheme that introduces class-specific energy scores as guidance for effective label assignment. Experimental results show that our method achieves better performance than previous state-of-the-art methods on SCOOD benchmarks.

\textbf{Limitations.} In addition to temperature scaling, other techniques such as feature clipping applied in ReAct~\cite{sun2021react} also enhance the performance of energy score, so how to obtain an OOD score that best fits the SCOOD task can be further explored. Moreover, a setting highly related to SCOOD has been proposed in \cite{katz2022training} and formulated as a constrained optimization problem. We will also theoretically analyze these practical OOD settings in our feature work.

% \section*{Acknowledgments}
\textbf{Acknowledgments.} 
This work is supported by National Key R\&D Program of China under Grant 2020AAA0105701, National Natural Science Foundation of China (NSFC) under Grants 61872327, Major Special Science and Technology Project of Anhui, National Natural Science Foundation of China (62033012) and Ant Group through Ant Research Intern Program.



\vspace{-0.06in}
\section*{Acknowledgements}
\vspace{-0.06in}
This research was supported in part by the NSF CNS-1932464, CNS-1929771, CNS-2145493, USDOT UTC Grant 69A3552047138, JST SPRING JPMJSP2123, JST PRESTO JPMJPR22PA, and JSPS KAKENHI 21K20413.
\vspace{-0.06in}

{\small
\bibliographystyle{IEEEtran}
\bibliography{main.bib}
}


\section{Appendix for Proofs}

\paragraph{Proof of Theorem \ref{thm:main}.}

\begin{proof}
\label{proof:main}
Our proof has two steps. In Step 1, we will show that SimCLR is equivalent to minimizing the cross entropy loss defined in Eqn.~(\ref{eqn:cross-entropy}). 
In Step 2, we will show  that minimizing the cross-entropy loss 
is equivalent to spectral clustering on $\bfpi$. 
Combining the two steps together, we have proved our theorem. 

\textbf{Step 1: } SimCLR is equivalent to minimizing the cross entropy loss.

The cross-entropy loss takes expectation over 
$\bfW_\bfX\sim \mathbb{P}(\cdot ; \bfpi)$, 
which means $\bfW_\bfX$ has exactly one non-zero entry in each row $i$. By Lemma~\ref{lem:multinomial}, we know every row $i$ of $\bfW_\bfX$ is independent of other rows. Moreover, 
$\bfW_{\bfX,i}\sim \mathcal{M}(1, \bfpi_i/\sum_j \bfpi_{i,j})=\mathcal{M}(1, \bfpi_i)$, because $\bfpi_i$ itself is a probability distribution.
Similarly, we know $\bfW_\bfZ$ also has the row-independent property by sampling over $\mathbb{P}(\cdot;\bfK_\bfZ)$.
Therefore, by Lemma~\ref{lem:cross_split}, we know Eqn.~(\ref{eqn:cross-entropy}) is equivalent to:
\[
 -\sum_{i=1}^n \mathbb{E}_{\bfW_{\bfX,i}}[\log \mathbb{P}(\bfW_{\bfZ,i}=\bfW_{\bfX,i};\bfK_\bfZ)],
\]

This expression takes expectation over $\bfW_{\bfX,i}$ for the given row $i$. Notice that 
$\bfW_{\bfX,i}$ has exactly one non-zero entry, which equals $1$ (same for $\bfW_{\bfZ,i}$). 
As a result
we expand the above expression to be:
\begin{equation}
 -\sum_{i=1}^n \sum_{j\neq i} \Pr(\bfW_{\bfX,i,j}=1)\log \Pr(\bfW_{\bfZ,i,j}=1).
\label{eqn:detailed-expansion}    
\end{equation}


By Lemma~\ref{lem:multinomial}, $\Pr(\bfW_{\bfZ,i,j}=1)=\bfK_{\bfZ,i,j}/\|\bfK_{\bfZ,i}\|_1$ for $j\neq i$. Recall that $\bfK_\bfZ=(k(\bfZ_i-\bfZ_j))_{(i,j)\in[n]^2}$, which means 
$\bfK_{\bfZ,i,j}/\|\bfK_{\bfZ,i}\|_1=\frac{\exp(-\|\bfZ_i-\bfZ_j\|^2/{2\tau})}{\sum_{k\neq i}
\exp(-\|\bfZ_i-\bfZ_k\|^2/{2\tau})
}$ for $j\neq i$, when $k$ is the Gaussian kernel with variance $\tau$. 

Notice that $\bfZ_i=f(\bfX_i)$, so we know
\begin{equation}
-\log \Pr(\bfW_{\bfZ,i,j}=1)=
-\log \frac{\exp(-\|f(\bfX_i)-f(\bfX_j)\|^2/{2\tau})}{\sum_{k\neq i}
\exp(-\|f(\bfX_i)-f(\bfX_k)\|^2/{2\tau}),
}
\label{eqn:infonce-equivalence}    
\end{equation}


The right hand side is exactly the InfoNCE loss defined in Eqn.~(\ref{eqn:infonce}).
Inserting Eqn.~(\ref{eqn:infonce-equivalence}) into Eqn.~(\ref{eqn:detailed-expansion}), we get the SimCLR algorithm, which first samples augmentation pairs $(i,j)$ with $\Pr(\bfW_{\bfX,i,j}=1)$ for each row $i$, and then optimize the InfoNCE loss. 

\textbf{Step 2: } minimizing the cross entropy loss 
is equivalent to spectral clustering on $\bfpi$.


By Lemma~\ref{lem:convert_to_spectral}, we may further convert the loss to 
\begin{equation}
\label{eqn:main-theorem-repul-attr}
\min_{\bfZ}
-\sum_{(i,j)\in [n]^2} \mathbf{P}_{i,j}
\log k (\bfZ_i-\bfZ_j)+\log \mathbf{R}(\bfZ).
\end{equation}
Since $k$ is the Gaussian kernel, this reduces to \[
\min_\bfZ \mathrm{tr}(\bfZ^\top \mathbf{L}(\bfpi) \bfZ)
+\log \mathbf{R}(\bfZ),
\]

where we use the fact that $\mathbb{E}_{\bfW_\bfX\sim \mathbb{P}(\cdot; \bfpi)}[\mathbf{L}(\bfW_\bfX)]
=\mathbf{L}(\bfpi)
$, because the Laplacian operator is linear and $
\mathbb{E}_{\bfW_\bfX\sim \mathbb{P}(\cdot; \bfpi)}(\bfW_\bfX)=\bfpi
$.
\end{proof}

\paragraph{Proof of Theorem \ref{thm:clip}.}
\begin{proof}
Since $\bfW_\bfX\sim \mathbb{P}(\cdot;\bfpi_{\mathbf{A}, \mathbf{B}})$, we know 
$\bfW_\bfX$ has exactly one non-zero entry in each row, denoting the pair that got sampled. 
A notable difference compared to the previous proof is we now have $n_\mathcal{A}+n_\mathcal{B}$ objects in our graph. CLIP deals with this by taking a mini-batch of size $2N$, 
such that $n_\mathcal{A}=n_\mathcal{B}=N$, and adding the $2N$ InfoNCE losses together. We label the objects in $\mathcal{A}$ as $[n_\mathcal{A}]$, and the objects in $\mathcal{B}$ as $\{n_\mathcal{A}+1, \cdots, n_\mathcal{A}+n_\mathcal{B}\}$. 

Notice that $\bfpi_{\mathbf{A}, \mathbf{B}}$ is a bipartite graph, so the edges of objects in $\mathcal{A}$ will only connect to object in $\mathcal{B}$ and vice versa. We can define the similarity matrix in $\cZ$ as $\bfK_\bfZ$, 
where $\bfK_\bfZ(i, j+n_\mathcal{A})=\bfK_\bfZ(j+n_\mathcal{A},i)= k(\bfZ_i-\bfZ_j)$ for $i\in [n_\mathcal{A}], j\in [n_\mathcal{B}]$, and otherwise we set $\bfK_\bfZ(i,j)=0$. 
The rest is same as the previous proof. 
\end{proof}

\paragraph{Proof of Theorem \ref{thm:exponential}.}

\begin{proof}
\label{proof:exponential}
Since the objective function consists of a linear term combined with an entropy regularization, which is a strongly concave function, the maximization problem is a convex optimization problem. Owing to the implicit constraints provided by the entropy function, the problem is equivalent to having only the equality constraint. We then introduce the Lagrangian multiplier $\lambda$ and obtain the following relaxed problem:

$$
\widetilde{E}(\boldsymbol{\alpha})=\psi_{1}-\sum_{i=1}^n \alpha_{i} \psi_{i}+\tau \sum_{i=1}^n \alpha_{i}\log \alpha_{i}+\lambda\left(\boldsymbol{\alpha}^{\top} \mathbf{1}_n-1\right).
$$

As the relaxed problem is unconstrained, taking the derivative with respect to $\alpha_{i}$ yields

$$
\frac{\partial \widetilde{E}(\boldsymbol{\alpha})}{\partial \alpha_{i}}=-\psi_{i}+\tau\left(\log \alpha_{i}+\alpha_{i} \frac{1}{\alpha_{i}}\right)+\lambda=0.
$$

Solving the above equation implies that $\alpha_{i}$ takes the form
$
\alpha_{i}=\exp \left(\frac{1}{\tau} \psi_{i}\right) \exp \left(\frac{-\lambda}{\tau}-1\right).
$ Since $\alpha_{i}$ lies on the probability simplex, the optimal $\alpha_{i}$ is explicitly given by
$
\alpha^{*}_{i}=\frac{\exp \left(\frac{1}{\tau} \psi_{i}\right)}{\sum_{i^{\prime}=1}^n \exp \left(\frac{1}{\tau} \psi_{i^{\prime}}\right)} .
$ Substituting the optimal point into the objective function, we obtain
$$
\begin{aligned}
E\left(\boldsymbol{\alpha}^*\right)  &=\psi_1-\sum_{i=1}^n \frac{\exp \left(\frac{1}{\tau} \psi_{i}\right)}{\sum_{i^{\prime}=1}^n \exp \left(\frac{1}{\tau} \psi_{i^{\prime}}\right)} \psi_{i}+\tau \sum_{i=1}^n \frac{\exp \left(\frac{1}{\tau} \psi_{i}\right)}{\sum_{i^{\prime}=1}^n \exp \left(\frac{1}{\tau} \psi_{i^{\prime}}\right)}\log \frac{\exp \left(\frac{1}{\tau} \psi_{i}\right)}{\sum_{i^{\prime}=1}^n \exp \left(\frac{1}{\tau} \psi_{i^{\prime}}\right)} \\
& =\psi_1 - \tau \log \left(\sum_{i=1}^n \exp \left(\frac{1}{\tau} \psi_{i}\right)\right).
\end{aligned}
$$
Thus, the Lagrangian dual function is given by
\begin{equation*}
-E\left(\boldsymbol{\alpha}^*\right)= -\tau \log \frac{\exp \left(\frac{1}{\tau} \psi_{1}\right)}{\sum_{i=1}^n \exp \left(\frac{1}{\tau} \psi_{i}\right)}.\qedhere
\end{equation*}
\end{proof}



\section{More on Experiments} \label{section: experiment_details}

\paragraph{CIFAR-10 and CIFAR-100} CIFAR-10 ~\citep{krizhevsky2009learning} and CIFAR-100 ~\citep{krizhevsky2009learning} are well-known classic image classification datasets. Both CIFAR-10 and CIFAR-100 contain a total of 60k $32 \times 32$ labeled images of different classes, with 50k for training and 10k for testing. CIFAR-10 is similar to CIFAR-100, except there are 10 different classes in CIFAR-10 and 100 classes in CIFAR-100.

\paragraph{TinyImageNet} TinyImageNet ~\citep{le2015tiny} is a subset of ImageNet ~\citep{deng2009imagenet}. There are 200 different object classes in TinyImageNet, with 500 training images, 50 validation images, and 50 test images for each class. All the images in TinyImageNet are colored and labeled with a size of $64 \times 64$.

\textbf{Pseudo-code.} Algorithm \ref{alg:Training Procedure} presents the pseudo-code for our empirical training procedure.

\begin{algorithm}[!htbp]
\caption{Training Procedure}
\label{alg:Training Procedure}
\begin{algorithmic}[1]
\REQUIRE trainable encoder network $f$, batch size $N$, augmentation strategy \textit{aug}, loss function $L$ with hyperparameters \textit{args}
\FOR {sampled minibatch ${x_i}_{i=1}^N$}
\FORALL{$i \in { 1, ..., N }$}
\STATE draw two augmentations $t_i = \textit{aug}\left(x_i\right) $, $t_i' = \textit{aug}\left(x_i\right) $
\STATE $z_i = f\left(t_i\right)$, $z_i' = f\left(t_i'\right)$
\ENDFOR
\STATE compute loss $\mathcal{L} = L(N, z, z', \textit{args})$
\STATE update encoder network $f$ to minimize $\mathcal{L}$
\ENDFOR
\STATE \textbf{Return} encoder network $f$
\end{algorithmic}
\end{algorithm}

We also provide the pseudo-code for our core loss function used in the training procedure in Algorithm \ref{alg:Core loss}. The pseudo-code is almost identical to SimCLR's loss function, with the exception of an extra parameter $\gamma$.

\begin{algorithm}[!htbp]
\caption{Core loss function $\mathcal{C}$}
\label{alg:Core loss}
\begin{algorithmic}[1]
\REQUIRE batch size $N$, two encoded minibatches $z_1, z_2$, $\gamma$, temperature $\tau$
\STATE $z = \textit{concat}\left(z_1, z_2\right)$
\FOR {$i \in {1, ..., 2N }, j \in {1, ..., 2N}$ }
\STATE $s_{i,j} = \Vert z_i - z_j \Vert_2^{\gamma}$
\ENDFOR
\STATE \textbf{define} $l(i, j)$ \textbf{as} $l(i, j) = - \log \frac{exp\left(s_{i,j}/\tau \right)}{\sum_{k=1}^{2N} \mathbf{1}{[k \ne i]} exp\left(s{i, j} / \tau \right)} $
\STATE \textbf{Return} $\frac{1}{2N} \sum_{k=1}^N\left[l(i, i+N) + l(i+N, i)\right]$
\end{algorithmic}
\end{algorithm}

Utilizing the core loss function $\mathcal{C}$, we can define all kernel loss functions used in our experiments in Table \ref{table: loss definition}. For all $z_i \in z$ with even dimensions $n$, we define $z_{L_i} = z_i\left[0:n/2\right]$ and $z_{R_i} = z_i\left[n/2:n\right]$.

\begin{table}[ht]
\centering
\begin{tabular}{{@{}l|l@{}}}
Kernel  &  Loss function \\ \midrule
Laplacian & $\mathcal{C}\left(N, z, z', \gamma=1, \tau\right)$\\ \midrule
Sum       & $\lambda * \mathcal{C}\left(N, z, z', \gamma=1, \tau_1\right) + (1-\lambda) * \mathcal{C}\left(N, z, z', \gamma=2, \tau_2\right)$  \\ \midrule
Concatenation Sum&$\lambda * \mathcal{C}\left(N, z_L, z'_L, \gamma=1, \tau_1\right) + (1-\lambda) * \mathcal{C}\left(N, z_R, z'_R, \gamma=2, \tau_2\right)$\\ \midrule
$\gamma = 0.5$ & $\mathcal{C}\left(N, z, z', \gamma=0.5, \tau\right)$          \\ 

\end{tabular}

\caption{Definition of kernel loss functions in our experiments}
\label {table: loss definition}
\end{table}

\textbf{Baselines.} We reproduce the SimCLR algorithm using PyTorch Lightning~\citep{PytorchLightning}.

\textbf{Encoder details.}
The encoder $f$ consists of a backbone network and a projection network. We employ ResNet50~\citep{ResNet} as the backbone and a 2-layer MLP (connected by a batch normalization~\citep{ioffe2015batch} layer and a ReLU \cite{nair2010rectified} layer) with hidden dimensions 2048 and output dimensions 128 (or 256 in the concatenation kernel case).

\textbf{Encoder hyperparameter tuning.}
For each encoder training case, we randomly sample 500 hyperparameter groups (sample details are shown in Table \ref{table: Hyperparameter sample}) and train these samples simultaneously using Ray Tune ~\citep{RayTune}, with the ASHA scheduler~\citep{li2018massively}. Ultimately, the hyperparameter group that maximizes the online validation accuracy (integrated in PyTorch Lightning) within 5000 validation steps is chosen for the given encoder training case.

\begin{table}[ht]
\centering

\begin{tabular}{@{}l|l|l@{}}
\midrule
Hyperparameter  & Sample Range & Sample Strategy \\ \midrule
start learning rate & $\left[10^{-2}, 10\right]$ & log uniform \\ \midrule
$\lambda$       & $\left[0, 1\right]$ & uniform \\ \midrule
$\tau$, $\tau_1$, $\tau_2$ & $\left[0, 1\right]$ & log uniform \\ \midrule
\end{tabular}

\caption{Hyperparameters sample strategy}
\label {table: Hyperparameter sample}
\end{table}

\textbf{Encoder training.} 
We train each encoder using the LARS optimizer~\citep{LARSOptimizer}, LambdaLR Scheduler in PyTorch, momentum 0.9, weight decay $10^{-6}$, batch size 256, and the aforementioned hyperparameters for 400 epochs on a single A-100 GPU.

\textbf{Image transformation.} The image transformation strategy, including augmentation, is identical to the default transformation strategy provided by PyTorch Lightning.

\textbf{Linear evaluation.}
The linear head is trained using the SGD optimizer with a cosine learning rate scheduler, batch size 64, and weight decay $10^{-6}$ for 100 epochs. The learning rate starts at $0.3$ and ends at $0$.

\textbf{Moco Experiments.} We also tested our method based on MoCo~\citep{he2019moco}. The results are summarized in Table \ref{tab:results-moco}. Here we choose ResNet18~\citep{ResNet} as the backbone and set a temperature of $0.1$ as default. For our simple sum kernel, we set $\lambda=0.8$. The results show that our method outperforms the original MoCo method.

\begin{table}[thb]
\centering
\caption{MoCo Experiment Results on CIFAR-10 and CIFAR-100.}
\label{tab:results-moco}
\resizebox{\textwidth}{!}{%
\begin{tabular}{@{}c|ccc|ccc@{}}
\toprule
\multirow{3}{*}{Method} & \multicolumn{3}{c|}{CIFAR-10} & \multicolumn{3}{c}{CIFAR-100} \\ \cmidrule(lr){2-4} \cmidrule(lr){5-7} 
                        & 200 epochs & 400 epochs    & 1000 epochs   & 200 epochs & 400 epochs & 1000 epochs         \\ \midrule
MoCo (repro.)         & $76.41 \pm 0.12$    & $80.01 \pm 0.15$          & $84.45 \pm 0.08$    & $\mathbf{47.02 \pm 0.11}$ & $52.50 \pm 0.07$ & $57.62 \pm 0.15$            \\
\midrule
Laplacian Kernel        & ${78.09 \pm 0.10}$    & $\mathbf{83.85 \pm 0.09}$          & $\mathbf{88.34 \pm 0.16}$    & $46.12 \pm 0.22$   & $53.44 \pm 0.17$ & $59.10 \pm 0.14$        \\
Simple Sum Kernel & $\mathbf{78.12 \pm 0.15}$   & $83.23 \pm 0.18$ & $87.50 \pm 0.20$ & $46.65 \pm 0.06$ & $\mathbf{53.62 \pm 0.19}$ & $\mathbf{59.83 \pm 0.12}$\\
\bottomrule
\end{tabular}
}
\end{table}



\section{More Experiments on Synthetic Data}


Consider a scenario with $n$ clusters, each containing $k$ vertices. Let the probability of vertices $u$ and $v$ from the same cluster belonging to $\bfpi$ be $p$. Conversely, for vertices $u$ and $v$ from different clusters, let the probability of belonging to $\pi$ be $q$. We generate the graph $\bfpi$ randomly, based on $p$ and $q$. We experiment with values of $k=100$ and $n=6$ for ease of visualization, embedding all points in a two-dimensional space. Each vertex's initial position originates from a normal distribution. In each iteration, we sample a subgraph of $\bfpi$ uniformly, ensuring each vertex has an out-degree of $1$. We then optimize the corresponding vectors using InfoNCE loss with an SGD optimizer and iterate until convergence. Our experimental setup consists of an SGD learning rate of $1$, an InfoNCE loss temperature of $0.5$, and a batch size of $50$. We evaluate two scenarios with different $p$ and $q$ values: $p=1$, $q=0$, and $p=0.75$, $q=0.2$. The results of these experiments are visualized in Figure \ref{fig:vis-spectral-cluster}. The obtained embeddings exhibit the hallmark pattern of spectral clustering of graph $\bfpi$.

\begin{figure}[!tb]
\centering
\subfigure{
\includegraphics[width=1\textwidth]{Figures/cluster_pi.png}
\label{fig:vis-cluster}
}
\subfigure{
\includegraphics[width=1\textwidth]{Figures/noised_cluster_pi.png}
\label{fig:vis-noised-cluster}
}
\caption{Visualizations of the optimization process using InfoNCE Loss on the vectors corresponding to $\bfpi$. Points of identical color belong to the same cluster within $\bfpi$. To showcase the internal structure of $\bfpi$, we randomly select 10 vertices from each cluster to display the edge distribution of $\bfpi$.}
\label{fig:vis-spectral-cluster}
\end{figure}




\end{document}
