\nsubsection{Real Vehicle Removal with HFR attack} \label{sec:case_study}

Although our results in~\S\ref{sec:vul_to_removal} were able to show end-to-end attack capabilities of our HFR attack in causing vehicle collisions, the front-vehicle removal attack effect was simulated and not shown in the physical world. 
In this section, we thus further test this aspect in the physical world. Fig.~\ref{fig:hfa_real_car} shows the point clouds in the benign and attack scenarios. We target VLP-16~\cite{VLP16} LiDAR with the dual return mode. We detect objects with the PointPillars~\cite{lang2019pointpillars} model in Baidu Apollo 6.0~\cite{apollo}.
As shown, our HFR attack is found to successfully remove 5 front vehicles at $\sim$5 meters away, out of which all can be correctly detected in the benign scenario. Such an attack effect is found consistent across all the 100 continuous frames we collected, leading to a 100\% attack success rate over 10 seconds. In the figure, we can see the spatial features of the objects were completely eliminated (with some random points left) and thus no objects were detected in the attacked region.  
Videos are available on our website: \textcolor{blue}{\textbf{\url{https://sites.google.com/view/lidar-study/home}}}.


\begin{figure}[t!]
\centering
\includegraphics[width=\linewidth]{imgs/case_study/hfa_real_car2.pdf}
\caption{Front-vehicle removal attack effect against real vehicles using our HFR attack. The 5 front vehicles become undetected with a 100\% success rate over 10 seconds (100 frames in total) by PointPillars~\cite{lang2019pointpillars} in Apollo~\cite{apollo}.
}
\label{fig:hfa_real_car}
\vspace{-0.07in}
\end{figure}


\begin{observation}{RQ2}
Our HFR attack is indeed able to achieve a reliable front-vehicle removal attack effect in the physical world. 
\end{observation}