\nsection{Defense Discussion} \label{sec:discussion}
\vspace{0.05in}
\nsubsection{Sensor-Level Defenses}\label{sec:sensor-level-defense}
In this work, we perform various experiments to understand the impacts of popular security-related features in next-gen LiDARs on the spoofing attack capabilities.
Table~\ref{tbl:defense} summarizes our observations regarding their defense effectiveness and limitations. As shown, all features have trade-offs, with timing randomization and pulse fingerprinting being relatively more promising from the security perspective:

\begin{table}[t!]
\centering
\caption{\newpart{Defense effectiveness and limitations of security-related features in the next-gen LiDARs. \textbf{Bold} means desired properties; \underline{Underline} means undesire ones.}}
\label{tbl:defense}
\setlength{\tabcolsep}{1.5pt}
\renewcommand{\arraystretch}{0.75}
\begin{tabular}{ccc|ccc}
\hline 
                  & \multicolumn{2}{c|}{Effectiveness} & \multicolumn{3}{c}{Limitations}        \\ \cline{2-6} 
Features           & For HFR             & For Others           & Eye safety & Latency         & Range$\downarrow$* \\ \hline
Simul. Firing     & \underline{None}                & \underline{Low}                  & \textbf{Low risk}   & \textbf{Low impact}      & \textbf{Low}    \\
Timing Random.    & \underline{None}                & \textbf{High}                 & \textbf{No risk}    & \textbf{Low impact}      & \textbf{None}     \\
Pulse Fingerprint & \textbf{High}                & Mid                 & \underline{High risk} & Mid impact & \underline{High}  \\
Rare Wavelength   & Mid            & Mid             & \textbf{No risk}    & \textbf{No impact}       & \textbf{Low}    \\ \hline
\end{tabular}
\raggedright

* Range$\downarrow$: Degradation of the effective sensing range of LiDAR
\end{table}

\textbf{Defend against injection attacks.} 
As shown in~\S\ref{sec:vul_to_injection}, the combination of the timing randomization ($\sigma$) and the pulse fingerprinting (random downsampling) can have high defense capability against injection attacks for industry-grade object detectors. Nevertheless, as shown in~\S\ref{sec:impact_noise}, such a defense strategy can only be effective when the randomization degree is high enough (e.g., $\geq$15 m). However, as discovered in our study, this is actually not the case for some LiDARs today, e.g., NextG\circled{3} (\S\ref{sec:impact_noise}), potentially since timing randomization was initially implemented for anti-interference purposes instead of security. As shown in Table~\ref{tbl:defense}, increasing the randomization degree has much fewer drawbacks than improving pulse fingerprinting (will detail later). Thus, for defending against injection attacks, the most cost-effective solution is for manufacturers to increase the level of timing randomization as much as possible. 

\textbf{Defend against removal attacks.}
As shown in~\S\ref{sec:vul_to_removal}, the results for the LiDAR with pulse fingerprinting (NextG\circled{2}) show much higher defense effectiveness against the HFR attack compared to other LiDARs, likely because a random removal of $\sim$100 points out of a vehicle point cloud ($>$2,000 points) can hardly affect the detection of the 3D pattern in general, especially considering that 100 points randomly sampled from a vehicle point cloud can be enough to trigger the vehicle detection for the object detectors today (\S\ref{sec:impact_data}). Nevertheless, in~Table~\ref{tbl:sim_hfa} we can still notice collision possibilities (1/10 at 17m and 20m) even with pulse fingerprinting, likley because pulse fingerprinting is also initially implemented for anti-interference purposes instead of security. To further improve this, more complex fingerprint coding is needed to make the spoofable points even fewer (than $\sim$100 for NextG\circled{2}). However, this can be quite non-trivial due to a dilemma between eye-safety and sensing range.


\textbf{Dilemma between eye safety and effective LiDAR sensing range in pulse fingerprinting.}
The most direct way to increase the fingerprint coding complexity is to increase the number of pulses for each distance measurement. However,  the laser power per unit time is capped to ensure eye safety. For example, if we shoot $N$ pulses for each point measurement, the power for each pulse should be $1/N$, and it will roughly degrade the effective sensing range by $1/\sqrt{N}$. To address the dilemma, we need to explore: (1) possible fingerprint coding designs both with high complexity and fewer pulses, and (2) use a wavelength to which the human eye is highly resistant, such as 1550 nm wavelength, which is adopted in the latest LiDARs~\cite{Aeye, Luminar}.

\textbf{HFR attack-specific defense discussion.} A unique pattern of our HFR attack is to keep hitting LiDAR with laser pulses of high frequency. Thus, this may be leveraged for attack detection, e.g., by manufacturing the LiDAR to detect periodic, high-frequency, and also high-intensity receiving laser pulses, which can be quite unusual in the benign cases. However, it is unclear how low the false alarm rate can be in practice, which can be especially critical if we expect to further recover the original point cloud by removing such detected high-frequency attack laser pulses.


\nsubsection{Software-Level Defenses for HFR Attack}
\textbf{HFR attack-specific defense discussion.} As shown in Fig.~\ref{fig:hfa_real_car}, at the point cloud level, a unique characteristic of the HFR attack is that it will cause a randomized point distribution pattern (like salt-and-pepper noise) in the area under attack, which is caused by the attack design (\S\ref{sec:high_freq_attack}) and can rarely occur in benign cases. Thus, a potential software-level defense direction is to detect such randomized point distribution patterns. However, even with this attack detection, it is still impossible to recover the original point cloud at the software level, which thus still allows this attack to have a LiDAR blinding effect. Considering that the HFR attack does not require synchronization and thus can work regardless of the victim LiDAR model and generation, even when just causing such a blinding effect, the security/safety implications in practice are still more severe than ever, especially when compared to the latest prior removal attack PRA~\cite{cao2021invisible} that requires synchronization and thus can no longer be impactful for next-gen LiDARs.

\nsubsection{Limitations and Safety Considerations} \label{sec:limitation}

We further discuss 2 potential limitations of our study: the aiming problem for attacking a driving AD vehicle, and the LiDAR model coverage limitations. Details are in Appendix~\ref{appendix:limitation}. Our considerations for safe laser experiments are also discussed in Appendix~\ref{appendix:safety}.

