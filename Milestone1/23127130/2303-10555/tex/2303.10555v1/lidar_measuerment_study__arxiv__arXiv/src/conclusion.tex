\nsection{Conclusion} \label{sec:conclusion}

In this work, we conduct the first large-scale measurement study on LiDAR spoofing attack capabilities on object detectors and identify 12 novel research observations \newpart{to answer the 3 novel RQs in~\S\ref{sec:intro}}. We first significantly improved the LiDAR spoofing capability, which allows us to be the first to clearly demonstrate and quantify the commonly-assumed CPI attack capability. We further find that the CPI attack capability is only feasible on VLP-16, which directly challenges the validity of all the existing injection attack designs against more recent LiDARs.
To this end, we identify the HFR attack, the first removal attack that can attack a more general and recent set of LiDARs, which shows high effectiveness in both simulated and physical-world experiments.
Based on the newly-obtained attack capabilities, we further evaluate and analyze their impacts on major types of object detectors.
We also discuss the defense side.
We hope that our findings can inspire and facilitate future security research on LiDAR spoofing, especially those targeting safety-critical application domains such as autonomous driving.
