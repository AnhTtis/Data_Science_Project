\nsection{Evaluation on LiDAR Spoofing Attacks} \label{sec:spoofing}

In this section, we conduct a large-scale security analysis on 9 popular and state-of-the-art LiDARs against LiDAR spoofing attacks in the physical world. 

\nsubsection{Experiment Setup}

Table~\ref{tbl:target_lidars} shows the LiDARs under evaluation in this study. Due to security reasons, we anonymize the next-gen LiDARs (NextG\circled{1}-\circled{6}) in this version.
The 9 LiDARs except for NextG\circled{5} are recognized as applicable for AD. We collect as many LiDARs as possible with our best effort. We cover the first and next-gen LiDARs, wide operating ranges from 9 m to 300 m, 3 scanning types (Rotating, MEMS, and Flash), and 3 security-related features (simultaneous laser firing, timing randomization, and fingerprinting). Our spoofer design and setup are discussed in~\S\ref{sec:spoofer_design}. The criteria to count the number of removed and injected points are explained in Appendix~\ref{appndix:count_method}.


\begin{table*}[t!]
\centering
\footnotesize
\caption{The 9 LiDAR models evaluated in this paper. Next-gen LiDARs are anonymized due to security reasons. 1st and NextG: first- and next-gen. FOV: field of view.}
\label{tbl:target_lidars}
\setlength{\tabcolsep}{2.5pt}
\renewcommand{\arraystretch}{0.75}
\begin{tabular}{llccccccccc}
\toprule

& & VLP-16~\cite{VLP16}     & VLP-32c~\cite{VLP32c}    & VLS-128~\cite{VLS128}    & NextG\circled{1}   & NextG\circled{2}  &  NextG\circled{3}     & NextG\circled{4}       & NextG\circled{5} & NextG\circled{6} \\ \hline
\multirow{10}{*}{\vspace{0.15in}\rotatebox{90}{\notsotiny{General Specs}}} & Gen. (year)       & 1st \notsotiny{(2016)}   &  1st \notsotiny{(2017)}   & 1st \notsotiny{(2017)}   & NextG \notsotiny{(2019)}  & NextG \notsotiny{(2020)}    & NextG \notsotiny{(2021)}  &  NextG \notsotiny{(2020)}   & NextG \notsotiny{(2019)}           & NextG \notsotiny{(2019)}  \\
& Scanning Type       & Rotating   & Rotating   & Rotating   & Rotating & Rotating       & Rotating  & MEMS     & MEMS           & Flash    \\
                  & Wavelength & 905 nm     & 905 nm     & 905 nm     & 865 nm   & 905 nm       &  905 nm    & 905 nm        & 860 nm         & 905 nm   \\
                  & Vertical FOV   & 30$^{\circ}$  & 40$^{\circ}$  & 40$^{\circ}$  & 45$^{\circ}$  & 31$^{\circ}$  &  70$^{\circ}$   & 25.1$^{\circ}$ & 55$^{\circ}$   & 16$^{\circ}$ \\
                  & Horizontal FOV & 360$^{\circ}$ & 360$^{\circ}$ & 360$^{\circ}$ & 360$^{\circ}$ & 360$^{\circ}$  &  360$^{\circ}$ & 81.7$^{\circ}$ & 70$^{\circ}$   & 180$^{\circ}$ \\
                  & Max. Range {[}m{]}       & 100 & 200 & 300 & 120 & 120 & 150  & 260  & 9     & 56 \\
                  & Min. Range {[}m{]}       & 1   & 1   & 0.5  & 0.3 & 0   & 0.2  & 0.5   & 0.25 & 0.1\\
                  & Vertical Channel         & 16  & 32  & 128 & 32  & 32    & 32 & -    & -   & 8  \\
                  \hline
\multirow{3}{*}{\vspace{-0.08in}\rotatebox[origin=c]{90}{
 \notsotiny{Security}
}} & Simul. Firing    & 1   & 2   & 8   & 32   &  1 & 1   & 1  & 1  & 3 \\
                  & Timing Random.     &     &     &     & \CheckmarkBold   &     & \CheckmarkBold   & \CheckmarkBold    & \CheckmarkBold   & \CheckmarkBold \\
                  & Fingerprinting           &     &     &     &     & \CheckmarkBold   &      &      &     & \\ \toprule
\end{tabular}
\vspace{-0.03in}
\end{table*}

\nsubsection{Evaluation of Injection Attacks}
\vspace{0.2in}
\nsubsubsection{Comparison with Prior Work on VLP-16} \label{sec:inj_vlp16}

We first conduct a comparison experiment with the state-of-the-art LiDAR spoofing attacks. We use LiDAR VLP-16~\cite{VLP16}, which is dominantly used as the \textit{only} evaluation target in the prior works~\cite{shin2017illusion, cao2019adversarial, jiachen2020towards, cao2023you, hallyburton2022security}. 
In the prior works, the number of spoofed points keeps increasing from 60~\cite{cao2019adversarial} to $\sim$4k~\cite{cao2023you} along with the improvement of the spoofer device. 
We further improve the spoofer design with more careful optics and more functional FG (\S\ref{sec:spoofer_design}) and achieve a significant improvement on the spoofing capability.
Table~\ref{tbl:distance} shows the results of the synchronized injection attack on VLP-16. The attack capability is measured by the number of points injected by the spoofing ($\mathcal{N}$), the attacked azimuth range ($\theta$), and the spoofing success rate in the azimuth range ($\mathcal{R}$). As shown, our spoofer can inject $>$6,131 points indoor and $>$6,514 points outdoor. This number of spoofed points is at least 50\% more than those in the latest prior work~\cite{cao2023you}, which capped at $\sim$4,000. 

Meanwhile, we also observe that the number of spoofed points and angles in the outdoor setup is generally larger than those in the indoor setup. We consider this reasonable since the legitimate laser reflections are fewer in the outdoor environment (e.g., no wall reflections), leaving more room for spoofing. 
However, this actually contradicts the reported results in the latest prior work~\cite{cao2023you}: as shown in Table~\ref{tbl:distance}, the spoofable points outdoor are significantly fewer than those indoor (1.8k versus 4k) and they attribute this to the lighting condition differences (e.g., they also report that the number of spoofed points decreases from $\sim$2,500 at night to $\sim$1,600 at day.)
We consider that our results are achieved by our careful optical setup as discussed in~\S\ref{sec:optical_setup}, since the spoofing laser intensity is much stronger than the sunlight and should not be decayed in a short 10 m flight. 
We suspect that the optical setup of the previous work is not well calibrated and the laser beam diverges. The intensity of the beam decays rapidly as the square of the beam diameter and it will be less than the intensity of the legitimate beam. Ideally, the spoofing laser should be forming a cylinder without divergence and convergence while not generally infeasible. Our optical setup is also not perfect, resulting in a slight decrease in spoofing points at longer distances.

\begin{observation}{RQ1}
Our spoofer significantly improved the number of spoofed points and angle coverage. Previous observations that spoofing distance and lighting conditions can greatly affect the LiDAR spoofing capability~\cite{cao2023you} no longer hold with our design with more careful optics.
\end{observation}


\textbf{Quantification of CPI Attack Capability:} 
Our improved spoofer also enables us to achieve the CPI attack capability as shown in Fig.~\ref{fig:arbitrary_point}. 
To accurately model the CPI attack capability, we are the first to quantify the spoofing accuracy for the CPI. While prior work~\cite{hallyburton2022security} assumes a certain level of noise in spoofed points, the noise levels are based on their intuition and do not have any validations.
We observe that there are 2 types of errors that can affect the CPI attack capability for a given spoofed point $x_{ij}$ at $i$-th altitude and $j$-th azimuth: (1) \textit{inner-frame} error $\delta^{\rm inner}_{ij}$: the inaccuracy to spoof a point at a chosen 3D position in an individual frame; and (2) \textit{inter-frame} error $\delta^{\rm inter}$: the inaccuracy to maintain the spoofed position of the same point in the chosen pattern across consecutive frames. In other words, inner-frame error will cause a spoofed point cloud pattern to be less similar to the chosen one in an individual frame, while inter-frame error will cause the spoofed pattern to "shake" over time. Since the attacker needs to maintain the chosen pattern across consecutive frames to achieve end-to-end attack impact, the quantification of the CPI attack capability should include both errors (i.e., $\delta^{\rm inner}_{ij} + \delta^{\rm inter}$). Using our experimental setup, we measure both and find that $\delta^{\rm inner}_{ij}$ and $\delta^{\rm inter}$ are $\sim$10cm and $\sim$35cm respectively. With such quantification, we can thus for the first time evaluate the object detector-side vulnerabilities with a more scientifically-valid spoofing capability modeling, which will be detailed later in~\S\ref{sec:impact_noise}.


\begin{figure}[t!]
\centering
\vspace{-0.1in}
\includegraphics[width=\linewidth]{imgs/spoofing/inner_frame_noise.pdf}
\vspace{-0.3in}
\caption{Standard deviation of the inner-frame error at each azimuth of a chosen pattern for VLP-16.}
\label{fig:inner_frame_error}
\end{figure}

\begin{observation}{RQ1}
Our improved spoofer is the first to demonstrate and quantify the CPI attack capability, which is commonly assumed but never clearly demonstrated in prior works~\cite{cao2019adversarial, jiachen2020towards, cao2023you, hallyburton2022security, zhongyuan2021object}.
\end{observation}

\begin{table}[t!]
\centering
\footnotesize
\caption{Evaluation results of the synchronized injection attack on VLP-16. 
$\mathcal{N}$ is the maximum number of points injected by spoofing. $\theta$ is the attacked azimuthal range. $\mathcal{R}$ is the spoofing success rate in the azimuthal range.
The distance $d$ is between the spoofer and the LiDAR. The information in parenthesis ($\cdot$) are measurements of $\mathcal{N}$ from latest prior work~\cite{cao2023you}; ``-'' means data not available.
}
\label{tbl:distance}
\setlength{\tabcolsep}{3.8pt}
\renewcommand{\arraystretch}{0.7}
\begin{tabular}{cccccccc}
\toprule
     & \multicolumn{3}{c}{Indoor} &  & \multicolumn{3}{c}{Outdoor (Daytime: 70 lux)} \\ \cline{2-4} \cline{6-8} 
$d$     & $\mathcal{N}$     &  $\mathcal{R}$     & $\theta$         &  & $\mathcal{N}$     &  $\mathcal{R}$     & $\theta$      \\ \cline{1-4} \cline{6-8} 
2 m  & 6,523 (-)  & 98.5\% & 82.7$^{\circ}$  &  & 7,705 (-)  &  94.9\%     & 100.5$^{\circ}$        \\
(2.5 m)  & ($<$4k)  &  &   &  & (-)  &      &        \\
4 m  & 6,386 (-)  & 96.9\% & 82.5$^{\circ}$ &  &  7,950 ({$<$1.8k})  &   96.9\%     & 101.5$^{\circ}$        \\
6 m  & 6,575 (-)  & 98.6\% & 83.4$^{\circ}$  &  & 7,357 ({$<$1.5k}) &   87.2\%    & 99.6$^{\circ}$        \\
8 m  & 6,213 (-)  & 93.8\% & 82.8$^{\circ}$  &  & 6,702 ({$<$1k}) &   97.7\%     & 83.4$^{\circ}$         \\
10 m & 6,131 (-) & 93.2\% & 82.1$^{\circ}$  &  & 6,514 ({$<$1k}) &    93.3\%     &  84.2$^{\circ}$     \\ \toprule
\end{tabular}
\vspace{-0in}
\end{table}


\nsubsubsection{Evaluation on Other LiDARs} \label{sec:inj_other_lidars}

Table~\ref{tbl:inject_attack} shows the results of the injection attack on different LiDARs. We launch the synchronized injection attack for VLP-16~\cite{VLP16} and VLP-32c~\cite{VLP32c}. 
For the other LiDARs, since they all have either timing randomization or pulse fingerprinting, we can no longer synchronize with the laser firing pattern. To still measure their vulnerabilities to spoofing attacks, we try our best to inject as many points as possible with a random attack with high-frequency pulses. 
The random attack is similar to our HFR attack, but the frequency is tuned to achieve the largest number of injected points.
As shown, compared to the 1st generation LiDARs, the 4 representative security-related features in the next-gen LiDARs result in a huge difference in the attack capabilities: $>$19k injected points for NextG\circled{4}, $\sim$3.2k for NextG\circled{3}, 100-350 on NextG\circled{5} and NextG\circled{2}, and only 28 on NextG\circled{1}. We will closely investigate such differences in the next section. Despite these variances, one observation is in common: VLP-16 is the \textit{only} LiDAR among all these 7 ones that can achieve the CPI attack capability, since it holds a relatively-old first-gen design and thus lacks recent security-related features.

\begin{observation}{RQ1}
VLP-16 is actually the only LiDAR for which the CPI attack capability is feasible, which is the key design assumption made in all prior works on object injection attack side~\cite{cao2019adversarial, jiachen2020towards, hallyburton2022security}. This directly challenges the validity of all these existing designs against the more general and recent set of LiDARs, suggesting a need to systematically revisit/re-evaluate whether and how much our current understanding of this security problem space can still hold in general and in the near future.
\end{observation}



\nsubsubsection{Impacts from Security-Related Features}  \label{sec:sec_enchance_feats}
\vspace{0.05in}
\verb||


\begin{table}[t!]
\centering
\caption{Evaluation results of the injection attack on different LiDARs. For VLP-16 and VLP-32c, the attack is the synchronized injection attack. For the other LiDARs, we inject points with random firing (1 MHz). Symbols are the same as in Table~\ref{tbl:distance}.}
\label{tbl:inject_attack}
\setlength{\tabcolsep}{1pt}
\renewcommand{\arraystretch}{0.75}
\begin{tabular}{cccccccc}
\toprule
      & VLP-16  & VLP-32c & NextG\circled{1} & NextG\circled{2}  & NextG\circled{3}$*$   & NextG\circled{4}  & NextG\circled{5} \\
Sync.     & \CheckmarkBold    & \CheckmarkBold    &       &     &     &      \\ \hline
$\mathcal{N}$     & 6,523    & 9,711    & 28      & 113 & 3,203   & 19,182    & 321     \\
$\mathcal{R}$     & 98.50\% & 82.90\% & 43.80\%   & 2.10\%  & 19.4\%& 79.90\%  & 0.1\%  \\
$\mathcal{\theta}$ & 82.7$^{\circ}$ & 73.2$^{\circ}$ & 0.72$^{\circ}$  & 34.2$^{\circ}$ & 103.4$^{\circ}$ & 81.7$^{\circ}$ & 70$^{\circ}$\\ \toprule
\end{tabular}
\raggedright

$*$: NextG\circled{3} is measured at 0.3 m due to the restriction when we rent it.
\vspace{0.05in}
\end{table}

\noindent\textbf{Simultaneous Laser Firing.} \label{sec:sim_fireing}
As listed in Table~\ref{tbl:target_lidars}, many LiDARs fire multiple laser pulses simultaneously. Velodyne LiDAR is likely adopting a modular approach that doubles units when increasing altitudes. Hence, the number of simultaneous laser firings is also doubled: while the VLP-16~\cite{VLP16} fires 1 laser during each measurement, VLS-128~\cite{VLS128} fires 8 lasers simultaneously. This design makes the CPI attack infeasible because we cannot selectively return the laser to each simultaneous laser due to the large diameter of the spoofing laser. For example, on VLP-32c~\cite{VLP32c}, we can always inject pairs of points, but the injected pair will always have the same distance to LiDAR. Although simultaneous laser firing may help attackers since it can affect more points by a laser, the CPI is infeasible due to the constraint. 

\noindent\textbf{Timing Randomization.}
The next-gen LiDARs typically randomize their laser shooting timing at each firing to mitigate multi-LiDAR interference as listed in Table~\ref{tbl:target_lidars}. The timing randomization makes the CPI attack capability virtually impossible because the attacker can no longer synchronize with the laser firing pattern. However, interestingly, we find that the attacker may still be able to inject some points if the randomization is not strong enough. To quantify the impacts of the timing randomization, we measure the laser firing interval (not available from their data sheets) with a PD and an oscilloscope. Table~\ref{tbl:randomization} summarizes the distribution of laser firing intervals for LiDAR with timing randomization. We first categorize and fit the observed firing intervals into the uniform or Gaussian distribution based on the shape of its histogram. We then calculate the average and maximum absolute errors in the firing interval based on the distribution. We use the difference between the observed maximum and minimum intervals for the maximum error of the Gaussian distribution. Finally, we use the following formula to convert the time difference to distance: $\frac{\Delta t \times c}{2}$, where $\Delta t$ is the timing difference and $c$ is the speed of light. Fig.~\ref{fig:livox_spoofing} in Appendix shows the point clouds of NextG\circled{4}, which can illustrate the impacts of timing randomization on spoofing capability. As shown, when under attack, the whole point cloud becomes highly randomized, and the object shapes (e.g., our lab room as shown in the benign case) completely disappeared. We further evaluate the significance of the errors on object detectors in~\S\ref{sec:impact_noise}. 



\begin{table}[t!]
\centering
\footnotesize
\setlength{\tabcolsep}{1pt}
\renewcommand{\arraystretch}{0.75}
\caption{Distribution of laser firing intervals for LiDAR with timing randomization. Avg. $\Delta$ and Max. $\Delta$ are the expected and maximum time differences, respectively.}
\label{tbl:randomization}
\begin{tabular}{lccccc}
\toprule
                        & NextG\circled{1}        & NextG\circled{4}       &  NextG\circled{5} & NextG\circled{6}  & NextG\circled{3}    \\ \cline{2-6} 
                        & \includegraphics[width=0.5in]{imgs/spoofing/randomization/nextg1-random-hist.pdf}              
                        & \includegraphics[width=0.5in]{imgs/spoofing/randomization/nextg4-random-hist.pdf}               
                        & \includegraphics[width=0.5in]{imgs/spoofing/randomization/nextg5-random-hist.pdf}               
                        & \includegraphics[width=0.5in]{imgs/spoofing/randomization/nextg6-random-hist.pdf}
                        &
                        \includegraphics[width=0.5in]{imgs/spoofing/randomization/nextg3-random-hist.pdf}
                        \\ \hline
Dist. {[}$\mu$s{]} & $\mathcal{U}_{1.4, 1.8}$ & $\mathcal{U}_{4.0, 4.3}$ & $\mathcal{N}_{51, 0.025}$   & $\mathcal{U}_{4.5, 5.8}$ & $\mathcal{N}_{1.6, 0.005}$\\
Avg. $\Delta$      & 19.2 m          & 15.0 m            & 19.9 m           & 63.7 m     &    0.9 m   \\
Max. $\Delta$      & 57.7 m          & 45.0 m            & 20.1 m           & 191.3 m     &   5.3 m      \\ \toprule
\end{tabular}
$\mathcal{U}_{{\rm min, max}}$ - Uniform distribution, $\mathcal{N}_{{\rm mean, std}}$ - Gaussian distribution
\end{table}

\noindent\textbf{Pulse Fingerprinting.} We find that NextG\circled{2} can foil the CPI attack capability even without timing randomization. %
Fig.~\ref{fig:nextg2_fingerprint} shows the NextG\circled{2} pulse shape we measured. One pair of pulses (two closely-connected spikes) corresponds to a single distance measurement.
We suspect that the fingerprint is encoded by the interval of the pair, although we cannot be entirely sure due to the lack of official documentation.


However, we find that the fingerprinting itself cannot perfectly defend against spoofing attacks. Specifically, we find that by trying different attack laser-firing frequencies, certain frequencies can sometimes coincide with the fingerprinting interval, which can lead to up to 113 spoofed points as shown in Table~\ref{tbl:inject_attack}.
This could be because such pulse fingerprinting was originally developed for anti-interference purposes (e.g., to allow multiple LiDARs to operate at close range) instead of security and thus does not have enough randomness/entropy.  \newpart{This means that if in each attack laser-firing event the attacker also fires a pair of pulses with the interval that can most likely coincide the fingerprinting interval (e.g., the one we found that can spoof 113 points), there can still be a random subset (e.g., 113) of the points in the attacker-chosen point cloud pattern that can be spoofed with the CPI attack capability. 

Thus, later in~\S\ref{sec:vul_to_injection}, we model this effect on the spoofing capability as a random downsampling of a chosen pattern.
}

\begin{observation}{RQ2}
The current pulse fingerprinting is not complex enough to perfectly prevent spoofing attacks, likely because it is currently designed only for anti-interference purposes instead of security. However, it is not be trivial to design a complex fingerprinting while ensuring eye safety.
\end{observation}


\begin{figure}[t!]
    \begin{minipage}{.36\linewidth}
        \centering
        \includegraphics[width=\linewidth]{imgs/spoofing/nextg2_pulse.pdf}
        \caption{Examples of the receiving pulse shape of NextG\circled{2}.
        }
        \label{fig:nextg2_fingerprint}
   \end{minipage}\hspace{0.05in}
    \begin{minipage}{.63\linewidth}
        \centering
        \vspace{0.075in}
        \includegraphics[width=\linewidth]{imgs/spoofing/relationship_freq_pts2.pdf}
        \vspace{-0.2in}
        \caption{The relationship between the attack pulse frequency and the removed points by HFR attack.
       }
        \label{fig:relationship_freq_pts}
    \end{minipage}
\end{figure}

\noindent\textbf{Laser Wavelength.}

We find that the spoofing with 905 nm wavelength is not effective on NextG\circled{1} since NextG\circled{1} uses an 865 nm wavelength laser. We tried our best to purchase 865 nm laser diodes, but we find that the high-power ones are actually not generally publicly available for personal use. Thus, using rare wavelengths may have a mitigation effect on spoofing attacks in practice. Details are in Appendix~\ref{appendix:wavelength}.


\begin{observation}{RQ2}
Developing/using LiDARs with relatively rare wavelengths (e.g., 865 nm) may have a mitigation effect on LiDAR spoofing attacks in practice, but such a mitigation effect can be temporary and subject to LiDAR implementation details.
\end{observation}

\nsubsection{Evaluation of High-Frequency Removal Attack}\label{sec:eval_hfa}

Fig.~\ref{fig:removal_attack} shows the attack results of the PRA~\cite{cao2023you} and our HFR attacks. As shown, the person and the majority of the room wall are removed by the attacks. 
For our HFR attack, there are points like a salt-and-pepper noise in the removed area.
This is because as the key design feature, the HFR attack is asynchronized and thus achieves removal by moving points to a random location or undetectable area. 
Table~\ref{tbl:hfa} lists the results of the PRA attack and our newly-identified HFR attack on multiple LiDARs. As shown, PRA can remove the largest number of points on VLP-16 and VLP-32c, which is the same number as the synchronized injection attack in Table~\ref{tbl:inject_attack} because the only difference between a synchronized injection attack and a removal attack is whether the point is moved into the MOT or not (\S\ref{sec:sync_removal_attack}). 
However, due to the reliance on synchronization, PRA is only applicable to the first-gen LiDARs; for the next-gen LiDARs, synchronization is directly foiled by common features such as time randomization. 


On the other hand, our HFR attack can still be effective on next-gen LiDARs with time randomization, since it does not depend on the synchronization with the fixed scanning pattern. As shown, the HFR attack is thus successful on NextG\circled{4}, NextG\circled{5}, and NextG\circled{3} and can remove $\geq$4,108 points in $\geq$70$^{\circ}$ azimuth range.
For the first-gen LiDARs, VLP-16 and VLP-32c, we observe HFR attack has slight attack capability degradation from PRA (e.g., 20\% fewer spoofing points for VLP-16), since PRA has more precise control of points by synchronizing with the LiDAR scan pattern.
However, we find that such a slight decrease in removal capability does not have significant impacts from end-to-end attack effect perspective as shown later in~\S\ref{sec:vul_to_removal}.


\begin{observation}{RQ2}
Compared to the latest removal attack, our newly-identified HFR attack is able to attack a much more general set of LiDARs, especially the more recent generation LiDARs with timing randomization.
\end{observation}

\noindent\textbf{Impact of Pulse Frequency and Laser Drive Voltage:}
The attack effectiveness of HFR mainly depends on the attack laser frequency; the higher it is, the more effective the attack should be (S\ref{sec:high_freq_attack}).
However, we find that highly-frequent laser firing makes the temperature of LD high and thus results in the degradation of laser intensity, which may affect the spoofing capability.
We thus experimentally evaluate this as shown in Fig.~\ref{fig:relationship_freq_pts}. As shown, when we increase the frequency, the number of removed points peaks at $\sim$1 MHz and decreases after that. Thus, we use 1 MHz as the default attack laser frequency in our other experiments.

Meanwhile, the attack laser intensity at the firing time may also practically affect the attack effectiveness, since lower one may not be able to ensure that the attack laser received at the victim is stronger than the legitimate one. To understand this, we also vary the attack laser drive voltage in our experiments. Since VLP-16 has $\sim$30V laser drive voltage, we vary the attack laser drive voltage from 40 to 80 V, where 80V is the maximum possible one in our setup. As shown in Fig.~\ref{fig:relationship_freq_pts}, the number of removed points monotonically decrease with the voltage value. Thus, we use 80V as the default voltage in our other experiments.


\begin{table}[t!]
\centering
\footnotesize
\setlength{\tabcolsep}{0.01pt}
\renewcommand{\arraystretch}{0.75}
\caption{
Evaluation results of the removal attacks (PRA~\cite{cao2023you} and our HFR attack) on different LiDARs. PRA is only feasible on the first-gen LiDARs (VLP-16 and VLP-32c) due to its reliance on synchronization, which is foiled by common next-gen LiDAR features such as timing randomization. Symbol meanings are the same as in Table~\ref{tbl:distance}.
}
\label{tbl:hfa}
\begin{tabular}{ccccccccc}
\toprule
                     &       & VLP-16  & VLP-32c & NextG\circled{1}    & NextG\circled{2} & NextG\circled{3}    & NextG\circled{4}  & NextG\circled{5}  \\ \hline
\multirow{3}{*}{\begin{tabular}[c]{@{}c@{}}PRA~\cite{cao2023you}\\w/ our\\ spoofer\end{tabular}} & $\mathcal{N}$     & 6,621    & 9,711    & N/A      & N/A       & N/A       & N/A & N/A     \\
                     & $\mathcal{R}$     & 96.9\% & 82.9\% & N/A      & N/A       & N/A       & N/A  & N/A    \\
                     & $\theta$ & 85.4$^{\circ}$ & 73.2$^{\circ}$ & N/A      & N/A       & N/A       & N/A & N/A     \\ \hline
\multirow{3}{*}{\begin{tabular}[c]{@{}c@{}}HFR\\(ours)\end{tabular}} & $\mathcal{N}$     & 5,358    & 8,778    & 28   & 113 & 4,108    & 19,182    & 205,730   \\
                     & $\mathcal{R}$     & 78.1\%  & 72.2\% & 43.8\%   & 2.1\% & 24.8\%  & 79.9\%   & 91.3\% \\
                     & $\theta$ & 85.8$^{\circ}$ & 76.0$^{\circ}$ & 0.72$^{\circ}$  & 34.2$^{\circ}$ & 103.4$^{\circ}$  & 81.7$^{\circ}$ & 70.0$^{\circ}$\\ \toprule
\end{tabular}
\raggedright

NextG\circled{3} is measured at 0.3 m due to the restriction when we rent it.\\
N/A: Attack is not applicable on the LiDAR
\end{table}


\begin{figure}[t!]
\centering
\includegraphics[width=\linewidth]{imgs/spoofing/Comparison_removal_attacks3.pdf}
\caption{Example results from removal attacks. A person and the majority of the room wall are totally removed by PRA~\cite{cao2023you} and our HFR attack.}
\label{fig:removal_attack}
\end{figure}

\nsubsection{Case Study on Specific LiDARs}
\label{sec:specific-lidars}
Using our measurement setup, we also found 4 unique characteristics of specific LiDARs: relay attack on NextG\circled{6}, zero-distance sensing of NextG\circled{2}, wide vertical FOV of NextG\circled{3}, and simultaneous Laser Firing on NextG\circled{1} and VLS-128, which led to 2 new observations that are different from existing understandings in the community. Details of the analysis and results are in Appendix~\ref{appendix:specific-lidars}.


\nsubsection{Summary} \label{sec:summary_spoofing}

In this section, we are the first to demonstrate and quantify the CPI attack capability, which is a key design assumption made in all prior works on object injection attack side~\cite{cao2019adversarial, jiachen2020towards, hallyburton2022security}. We find that the CPI attack is realizable on VLP-16 with the 10 cm inner-frame noise and the 35 cm inter-frame noise, but also find that the VLP-16 is actually the only LiDAR for which we can achieve the CPI attack capability, due to the security-related features adopted in more recent LiDARs.
To ensure more general, representative, and state-of-the-art research findings and designs, we would like to advocate that further research along this line should focus more on more recent generation LiDARs with the security-related features, especially the timing randomization and the pulse fingerprinting, which show high defensive capabilities.

