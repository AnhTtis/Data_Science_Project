\nsection{Attack Experiment Setup and Methodology Improvements} \label{sec:methodology}


Fig.~\ref{fig:spoofer1} shows an overview of our LiDAR spoofer, its optics design, and the setups of the indoor and outdoor experiments, which will be used for the measurement studies in later sections. We generally follow the common setup adopted in the prior works~\cite{cao2019adversarial, jiachen2020towards, hallyburton2022security, cao2023you}. Our spoofer operates in 4 steps: \circled{1} The photodetector (PD) receives a laser pulse from LiDAR; \circled{2} the transimpedance amplifier (TIA) amplifies the pulse to 100 mV to feed it into the function generator (FG), which ignores a pulse less than 100 mV; \circled{3} FG synchronizes with the LiDAR spanning pattern based on the received pulse and generates laser pulses according to the points that the attacker wants to inject; \circled{4} The gate driver (GD) drives the laser diode (LD) which transmits a laser pulse to the LiDAR. We use the same PD (S6775~\cite{S6775}) and TIA (TL082~\cite{TL082}) as the prior work~\cite{cao2023you}. For other components, we used the equivalent or enhanced devices: FG is Agilent 81160A~\cite{81160A}, GD is EPC9126HC~\cite{EPC9126HC}, and LD is SPL PL90\_3~\cite{SPLPL90_3}. In particular, we utilize a more functional FG (Agilent 81160A) than the one used in previous studies (Tektronix AFG320~\cite{AFG320}) to better facilitate the exploration of the CPI attack capability.


\nsubsection{Our Improvements on Spoofer Design}\label{sec:optical_setup}\label{sec:arbitrary_point_injection}\label{sec:spoofer_design}


As described~\S\ref{sec:cpi}, the CPI attack capability should be achievable in theory since the synchronized attack methodology should by design be able to precisely control the position of each spoofed point. However, we find that existing spoofers are not capable of clearly demonstrating it (Fig.~\ref{fig:arbitrary_point}) mainly because (1) the number and angle coverage of the points they can spoof ($\leq$200 points) is too small to make it possible to directly spoof the point cloud pattern of the object they intend to spoof, e.g., a front-near vehicle ($>$2,000 points)~\cite{cao2019adversarial, jiachen2020towards, hallyburton2022security}; and (2) the spoofer electronics are not precise enough to control a spoofed point to place at a chosen 3D position. Based on our experiments, we identify that the limitations are due to their inadequate optical design and electronics setups, respectively.

\textbf{Improvement on Optics.}
The inadequate optical design causes undesired diffusion and convergence of the laser beam. This degrades the laser power per unit area before reaching the target LiDAR, which thus causes the limitation on the number and angle coverage of spoofable points. 
For example, if the laser beam is expanding (e.g., no lens), the laser intensity rapidly decays with distance. On the other hand, if the laser beam is converging by the lens, it makes the attack unstable because the diameter of the beam decreases along with the distance, making the attacker difficult to aim at the target LiDAR. Ideally, the laser beam should be collimated without diffusion and convergence to achieve the target LiDAR with a minimum loss. 
To form a collimated beam, we use a 25.4 mm focal length plano-convex lens with 1 inch (25.4 mm) diameter to cover the laser emitted by SPL90\_3~\cite{SPLPL90_3}, which has a maximum beam divergence angle of 25$^{\circ}$. 
Since the diameter of the laser at the focal point is $\tan{25^\circ}\times 25.4 \times 2 = 23.7$ mm, we can cover it with the 1 inch (25.4 mm) lens. 
To precisely calibrate the lens setup, we develop a device that can adjust the distance between the LD and the lens as designed. As shown in Fig.~\ref{fig:spoofer1}, the lens is connected to the frame with a hollow screw so that we can adjust it precisely. Technically, this allows our spoofer to maintain a larger number and angle coverage of spoofable points from hundreds of meters away since the intensity of decay by air is not so substantial. Detailed results are shown later in~\S\ref{sec:inj_vlp16}.


\textbf{Improvement on Electronics.}
The inadequate electronics introduce inaccurate laser detection timing in step \circled{1} and also long and unstable delays in the signal processing of the spoofer devices, which thus makes it difficult to precisely control the timing of the attack laser firing. To address this, we improve the amplifier for the PD to increase the laser detection accuracy, and also improve FG setup to allow more precise nanosecond-level configuration and calibration.
Due to these improvements, we find that our spoofer is the first to clearly demonstrate the CPI attack capability and can launch robust attacks at a long range and against high ambient light, which are shown in Fig.~\ref{fig:arbitrary_point} and detailed later in~\S\ref{sec:inj_vlp16}.


\begin{figure}[t!]
\centering
\includegraphics[width=\linewidth]{imgs/methodology/spoofer_design_caption_with_optics.pdf}
\caption{Overview of our LiDAR spoofer setup, the optics design, and the setup of the indoor and outdoor experiments.
}
\label{fig:spoofer1}
\end{figure}

\cut{
\begin{figure}[t!]
\centering
\includegraphics[width=\linewidth]{imgs/methodology/spoofer_setup.png}
\caption{Illustration of our LiDAR spoofer design. \takami{image will be updated}
}
\label{fig:spoofer2}
\end{figure}


\begin{figure}[t!]
\centering
\includegraphics[width=\linewidth]{imgs/methodology/spoofing_long.png}
\caption{Illustration of our LiDAR spoofer design. \takami{image will be updated}
}
\label{fig:spoofer3}
\end{figure}
}


\nsubsection{New Asynchronized Removal Attack: High-Frequency Removal (HFR) Attack} \label{sec:high_freq_attack}

\newpart{
As mentioned in~\S\ref{sec:intro}, to measure the vulnerability status of next-gen LiDARs, we need powerful and practical \textit{asynchronized} attacks since synchronized ones are directly foiled by common security-related features in next-gen LiDARs (\S\ref{sec:inj_other_lidars}).
In this work, we identify a new type of asynchronized removal attack called high-frequency removal (HFR) attack, which is illustrated in Fig.~\ref{fig:sync_and_async_attack} and Fig.~\ref{fig:highfreq_vs_saturating} in Appendix. The key idea of it is to fire a large number of attack laser pulses to the victim LiDAR at a very high \textit{frequency}, which, more specifically, is higher than the laser-firing frequency of the victim LiDAR. This allows the attack laser to hit every laser-firing event of the victim LiDAR in the scanning range hit by the spoofer, which can thus achieve the spoofing effect for every points in that range without any knowledge or synchronization with the victim scanning pattern (i.e., the black-box LiDAR attack model defined in~\S\ref{sec:lidar_spoofing_attack}). However, due to the lack of synchronization, the receiving timing of the attack laser is random, and thus the spoofing effect will be moving each legitimate surface point of target objects to a random position or undetectable area of the victim LiDAR (e.g., within MOT). This can completely destruct the point cloud patterns at the original object position, which can thus cause the object removal effects. The attack effectiveness of the HFR attack mainly depends on how high the attack laser pulse frequency can be; the theoretical attack success rate of the HFR attack can thus be mathematically derived based on the laser frequency as in Appendix~\ref{appndix:attack_formula}.
}


\textbf{Comparison with prior removal attacks.} Among all spoofing attacks with object removal effect, the latest is PRA, a \textit{synchronized} attack (\S\ref{sec:attack_taxonomy}). Although it can remove $\sim$4,000 points, it requires synchronization, and thus compared to HFR, it is by design (1) less deployable due to the white-box attack assumption (\S\ref{sec:attack_taxonomy}): for HFR, the attack can work without knowing which LiDAR model the victim uses, and can omit the PD part in the spoofer (Fig.~\ref{fig:sync_and_async_attack}); and (2) not generalizable to next-gen LiDARs since the common security-related features (e.g., timing randomization, pulse fingerprinting) can directly foil synchronization (\S\ref{sec:inj_other_lidars}).

On the \textit{asynchronized} removal attack side, the state-of-the-art is the saturating attack~\cite{shin2017illusion} (\S\ref{sec:attack_taxonomy}). The fundamental difference is that instead of using \textit{pulsed} lasers to directly manipulate the laser-receiving event timing, the saturating attack works by using a \textit{continuous} laser to increase the ambient noise level to indirectly cause random detection errors of laser-receiving events, which can thus cause random point injection and removal effects as illustrated in Fig.~\ref{fig:highfreq_vs_saturating} in Appendix. However, due to the requirement of maintaining continuous high-power laser, it is physically difficult to (1) achieve a large attack laser beam coverage at the victim LiDAR side with sufficiently high intensity, and (2) maintain the attack effect. These thus cause fundamental limitations in the attack capability and practicality, especially when compared to HFR. For example, the demonstrated removal attack effect is only about removing points in a 41$\times$42 cm$^2$ area, lasting $<$4 seconds. On the other hand, our HFR attack leveraging pulsed lasers can remove points in a 10$\times$10 m$^2$ area, without any limit on such attack effect duration, which is thus much more powerful and practical, especially for AD settings. Detailed results are shown later in~\S\ref{sec:eval_hfa}.

