\nsection{Background and Related Works} \label{sec:background}
\vspace{0.1in}

\newpart{
\nsubsection{Basics of LiDAR Sensing} \label{sec:lidar_basic}
LiDAR is an active sensor that can obtain 3D point cloud data of the surrounding environment. By firing laser pulses to the environment and receiving their reflections, direct time-of-flight (dToF) LiDARs are most commonly used in AD systems since with the laser peak power (typically $\sim$10 W) significantly higher than sunlight even after a long flight, they can detect objects more than 200 m away even under high ambient lights. In dToF LiDARs, for each fired laser, by measuring the laser reflection after hitting an object, it can use the time difference between the firing and reflection to calculate the 3D position of a ``point'' on the surface of such an object. With such active measurement (or ``scanning'') by firing lasers to different horizontal angles (\textit{azimuth}) and vertical angles (\textit{altitude}), each object is thus perceived as a group of such object surface points, called ``point cloud'' (Fig.~\ref{fig:arbitrary_point}), and is typically processed by a DNN for accurate object detection.}
 Among dToF LiDARs, early Velodyne LiDARs such as VLP-16~\cite{VLP16} and VLP-32c~\cite{VLP32c} are commonly known as the first-generation (\textit{first-gen}) LiDARs~\cite{yoshioka2022tutorial}, which naively integrate the classic point-wise laser ranging systems.
To achieve 360$^{\circ}$ measurements, they mechanically rotate to scan each azimuth with a small step such as 0.1$^{\circ}$.


\nsubsection{Technology Trends in Recent LiDARs} \label{sec:lidar_tax}

The advent of first-gen LiDAR has greatly improved AD perception, but its complex mechanical design increases costs and limits scalability.
To overcome the limitations, the next-generation (\textit{next-gen}) LiDARs~\cite{yoshioka2022tutorial} mount all components, such as the photodetector and the readout circuitry, on a single chip known as a system-on-chip (SoC) approach. This approach not only reduces the cost and improves the scalability of the system, but also allows new designs of laser firing and receiving.
For example, microelectromechanical systems (MEMS) LiDAR~\cite{wang2020mems} move a mirror to scan a wide azimuth range instead of mechanical rotation. Flash LiDARs~\cite{roriz2021automotive, ousterflash} fire a broad laser that covers the entire field of view (FOV) and calculates the distance by compensating its weak return laser by accumulation. 
Moreover, the SoC approach allows more complex signal processing, such as a large number of simultaneous laser firings, laser timing randomization, and fingerprinting, to be robust against challenging environments, e.g., multiple LiDARs operating adjacent to each other.

Hence, the recent next-gen LiDARs substantially improved the electronics and the scanning mechanisms even though they still have the same basic design: laser firing and receiving. However, none of the prior works on LiDAR spoofing attacks has evaluated the security property of such next-gen LiDARs; \textit{all} of them are only evaluated on the first-gen rotational ones, with a predominate focus on in fact \textit{only 1 specific model}: VLP-16~\cite{shin2017illusion, cao2019adversarial, jiachen2020towards, hallyburton2022security, cao2023you}. The major design differences between first- and next-gen LiDARs are likely to cause significant differences in their security characteristics, which thus motivates this study.


\nsubsection{LiDAR Spoofing Attacks} \label{sec:lidar_spoofing_attack}
LiDAR spoofing attack works by using an external attack device (``spoofer'') to fire laser pulses back to the victim LiDAR in order to manipulate the time measurements of the laser-receiving events and thus the corresponding 3D position measurements (\S\ref{sec:lidar_basic}). In the resulting point cloud, the points induced by the receiving of the laser pulses fired by the attacker are thus called ``spoofed points''. Such point spoofing can thus cause object misdetection on the downstream object detector side, e.g., by spoofing the measurements of the points originally on an object to locate elsewhere to cause the object to be undetected (\textit{object removal}), or by spoofing a cluster of points in proximity to cause the detection of a non-existing object (\textit{object injection}).

\nsubsubsection{Attack Taxonomy}\label{sec:attack_taxonomy} Table~\ref{tbl:tax_attacks} shows a taxonomy of LiDAR spoofing attacks based on (1) the requirement of \textit{synchronization} with the LiDAR scanning pattern (row); and (2) the \textit{attack effect}: object injection or removal (column). The spoofing mechanisms are illustrated in Fig.~\ref{fig:sync_and_async_attack}. \textit{Synchronization} means to synchronize the malicious laser firing timing with the victim LiDAR scanning (i.e., laser firing) timing, which can enable precise control of the attack laser-receiving timing and thus the corresponding positions of the spoofed points. To achieve this, the attacker needs to use an extra device (PD in Fig.~\ref{fig:sync_and_async_attack}) to first learn the current state of the victim LiDAR scanning in the real time, and then use the victim LiDAR's scanning pattern to derive future victim laser-firing timings for synchronized attack laser-firing. More detailed explanations are in Appendix~\ref{appndix:sync_attack}. This process requires precise knowledge of the victim LiDAR's scanning pattern beforehand, which is thus referred to as \textit{white-box} attacks in this paper. Those that do not assume synchronization (i.e., \textit{asynchronized} attacks) can be applied without knowing anything about the LiDAR internal scanning logic, which is thus referred to as \textit{black-box} attacks.


\begin{figure*}[t!]
\centering
\includegraphics[width=\linewidth]{imgs/methodology/sync_and_async_attack.pdf}
\caption{\newpart{
Illustration of 4 LiDAR spoofing attack types. Synchronized attacks need white-box knowledge of the victim LiDAR scanning patterns and an extra device (PD) for synchronization (\S\ref{sec:lidar_spoofing_attack}). 
Asynchronized attacks do not need these (i.e., black-box LiDAR attack), and thus are both more deployable (can work without knowing the victim LiDAR model) and generalizable (to next-gen LiDARs since synchronization is directly foiled by their security-related features). Our HFR attack is the first asynchronized removal attack on par with synchronized removal attacks. PD: Photodetector. Pts: Points.
}
}
\label{fig:sync_and_async_attack}
\vspace{-0.1in}
\end{figure*}

\begin{table}[t!]
\centering
\caption{Taxonomy of existing LiDAR spoofing attacks. Rows correspond to whether the attack requires the synchronization with the LiDAR scanning pattern. Columns correspond to attack effects: object injection or removal.}
\label{tbl:tax_attacks}
\setlength{\tabcolsep}{3pt}
\renewcommand{\arraystretch}{0.75}
\begin{tabular}{cll}
\toprule
               & Object Injection Attack & Object Removal Attack \\ \hline
\begin{tabular}[c]{@{}l@{}}\ \ \ \ Async.\\(Black-box)\end{tabular}  & Relay~\cite{petit2015remote}, Saturating~\cite{shin2017illusion}      & \begin{tabular}[c]{@{}l@{}}\hspace{2em}Saturating~\cite{shin2017illusion},\\ \hspace{2em}\textbf{HFR$*$ (ours)}\end{tabular}           \\
\hdashline
\begin{tabular}[c]{@{}l@{}}\ \ \ \ \ Sync.\\(White-box)\end{tabular}   & 
\begin{tabular}[c]{@{}l@{}}\hspace{3em}Adv-LiDAR$*$~\cite{cao2019adversarial},\\ Occlusion$*$~\cite{jiachen2020towards}, Frustum$*$~\cite{hallyburton2022security}\end{tabular}
        & \hspace{0.5em}PRA$*$~\cite{cao2023you}, ORA~\cite{petit2015remote}       \\
 \toprule
\end{tabular}
\raggedright

$*$ Attack effectiveness against AD has been considered.
\end{table}


\textbf{Asynchronized Injection and Removal Attack.}
\label{sec:async_attack}
Relay attack~\cite{petit2015remote} is an asynchronized injection attack, which can inject spoofed points by relaying the laser received from the target LiDAR. It was shown capable of injecting $\geq$200 points, but it can only spoof points in farther positions than the spoofer since it needs to first receive a laser pulse before it can send the same pulse back.
With a delay in electronics and signal processing, the injected points are typically distributed $\geq 40$ meters behind the spoofer.

Saturating attack~\cite{shin2017illusion} is another asynchronized attack that fires a continuous infrared (IR) laser instead of pulsed ones to cause misdetections of laser-receiving events. It is shown that such an attack can inject dozens of randomly placed spoofed points and diminish a 41$\times$42 cm$^2$ metal plate. In this work, we are able to identify a new asynchronized attack design that is much more powerful and practical (\S\ref{sec:high_freq_attack}).


\textbf{Synchronized Injection Attack.}
\label{sec:sync_injection_attack}
The synchronized injection attack~\cite{shin2017illusion} is proposed to use the \textit{synchronization} process described above to overcome the limitations of the relay attack that it can only inject spoofed points farther than the spoofer~\cite{petit2015remote}, 
The early-stage attack designs can inject only 10 spoofed points~\cite{shin2017illusion}, but the following works progressively increase the number of spoofed points to 60~\cite{cao2019adversarial} and 200~\cite{jiachen2020towards} and demonstrate that the 60 and 200 spoofed points are enough to cause a false positive, i.e., injecting a fake object. After the success, the attack capability of injecting 200 points under the CPI assumption (\S\ref{sec:cpi}) becomes the \textit{de facto} threat model in the following works~\cite{zhongyuan2021object, hallyburton2022security, hau2021shadow}. However, none of them have clearly demonstrated and quantified the CPI attack capability. In this work, we thus re-visit the validity of this assumption with multiple spoofer design improvement (\S\ref{sec:arbitrary_point_injection}).

\textbf{Synchronized Removal Attack.}
\label{sec:sync_removal_attack}
To remove legitimate objects from detection, 2 synchronized removal attacks have been proposed so far. The first attack is the physical removal attack (PRA)~\cite{cao2023you} which removes a target object by leveraging the same methodology as the synchronized object injection attack~\cite{cao2019adversarial}. They utilize LiDAR's minimum operational threshold (MOT), common preliminary filtering to automatically discard points below a certain distance. This filtering is implemented in most LiDARs because the LiDAR measurement at a very close distance is generally inaccurate since the ToF can be too short to distinguish from errors due to inadequate calibration. However, in this paper we find that (1) PRA can no longer be applied to next-gen LiDARs due to the reliance on synchronization by design (\S\ref{sec:eval_hfa}), and (2) the assumption of non-zero MOT may not necessarily hold (e.g., for NextG\circled{2}, \S\ref{sec:specific-lidars}). 
The second attack is the object removal attack (ORA)~\cite{zhongyuan2021object}, which fools 3D object detectors by injecting spoofed points inside the target object's bounding box, since the point cloud of legitimate objects should have points mostly on the object surface instead of inside. This attack also assumes the 200-point CPI attack capability for spoofing.



\nsubsubsection{Chosen-Pattern Injection (CPI) Attack Capability}
\label{sec:cpi}
So far, all prior works on the synchronized injection attack side explicitly or implicitly assume that at the actual attack time the attacker is able to effectively inject a specific spoofed point cloud pattern carefully chosen by the attacker beforehand, e.g., via various kinds of off-line optimization/identification processes~\cite{cao2019adversarial, jiachen2020towards, hallyburton2022security}. In this paper, we thus explicitly define it as a type of LiDAR spoofing attack capabilities called \textit{Chosen Pattern Injection}, \textit{CPI} for short.

Such an attack capability is theoretically achievable since the synchronized LiDAR spoofing methodology should by design be capable of precisely controlling the position of each injected spoofed point. However, so far no prior works have clearly demonstrated it experimentally as shown in Fig.~\ref{fig:arbitrary_point}, not to mention to provide a systematic quantification of the pattern control precision, which is necessary for achieving valid downstream task security analysis (e.g., for object detectors). This inevitably raises the doubt that whether such an attack capability is indeed feasible or not in practice; if not, the basic attack design assumption made by all these prior works actually becomes invalid, which can directly challenge the validity and meaningfulness of their current research findings. To fill this critical research gap, we re-visit the validity of this assumption with multiple spoofer design improvements as will be detailed later in \S\ref{sec:arbitrary_point_injection}.



\nsubsection{3D Object Detection on Point Cloud} \label{sec:3d_obj}
\vspace{-0.01in}
As in other computer vision areas, 3D object detection on point cloud is substantially benefited by the recent progress of DNN. However, traditional DNN architectures such as CNN cannot be directly applied due to the irregular %
structure of point clouds~\cite{li2018pointcnn}. To handle such data complexity, 3 major types of 3D object detection methods are widely adopted: \textit{voxel-based}, \textit{point-based}, and \textit{voxel point-based} methods~\cite{qian2022object}. More details are in Appendix~\ref{appndix:obj_detector}.


\nsubsection{Threat Model} \label{sec:threat_model}

We follow the same threat model as in prior works~\cite{cao2019adversarial, jiachen2020towards, hallyburton2022security}, i.e., the attacker fires malicious lasers from their spoofer to the victim LiDAR (\S\ref{sec:lidar_spoofing_attack}, Fig.~\ref{fig:sync_and_async_attack}).
As described in~\cite{hallyburton2022security}, the spoofer device can be at a front vehicle, vehicle in the next lane, or a roadside in AD scenarios. 
