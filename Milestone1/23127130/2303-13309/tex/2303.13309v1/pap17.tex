%% bare_jrnl.tex
%% V1.4b
%% 2015/08/26
%% by Michael Shell
%% see http://www.michaelshell.org/
%% for current contact information.
%%
%% This is a skeleton file demonstrating the use of IEEEtran.cls
%% (requires IEEEtran.cls version 1.8b or later) with an IEEE
%% journal paper.
%%
%% Support sites:
%% http://www.michaelshell.org/tex/ieeetran/
%% http://www.ctan.org/pkg/ieeetran
%% and
%% http://www.ieee.org/

%%*************************************************************************
%% Legal Notice:
%% This code is offered as-is without any warranty either expressed or
%% implied; without even the implied warranty of MERCHANTABILITY or
%% FITNESS FOR A PARTICULAR PURPOSE! 
%% User assumes all risk.
%% In no event shall the IEEE or any contributor to this code be liable for
%% any damages or losses, including, but not limited to, incidental,
%% consequential, or any other damages, resulting from the use or misuse
%% of any information contained here.
%%
%% All comments are the opinions of their respective authors and are not
%% necessarily endorsed by the IEEE.
%%
%% This work is distributed under the LaTeX Project Public License (LPPL)
%% ( http://www.latex-project.org/ ) version 1.3, and may be freely used,
%% distributed and modified. A copy of the LPPL, version 1.3, is included
%% in the base LaTeX documentation of all distributions of LaTeX released
%% 2003/12/01 or later.
%% Retain all contribution notices and credits.
%% ** Modified files should be clearly indicated as such, including  **
%% ** renaming them and changing author support contact information. **
%%*************************************************************************


% *** Authors should verify (and, if needed, correct) their LaTeX system  ***
% *** with the testflow diagnostic prior to trusting their LaTeX platform ***
% *** with production work. The IEEE's font choices and paper sizes can   ***
% *** trigger bugs that do not appear when using other class files.       ***                          ***
% The testflow support page is at:
% http://www.michaelshell.org/tex/testflow/



\documentclass[journal,twoside]{IEEEtran}
%
% If IEEEtran.cls has not been installed into the LaTeX system files,
% manually specify the path to it like:
% \documentclass[journal]{../sty/IEEEtran}





% Some very useful LaTeX packages include:
% (uncomment the ones you want to load)


% *** MISC UTILITY PACKAGES ***
%
%\usepackage{ifpdf}
% Heiko Oberdiek's ifpdf.sty is very useful if you need conditional
% compilation based on whether the output is pdf or dvi.
% usage:
% \ifpdf
%   % pdf code
% \else
%   % dvi code
% \fi
% The latest version of ifpdf.sty can be obtained from:
% http://www.ctan.org/pkg/ifpdf
% Also, note that IEEEtran.cls V1.7 and later provides a builtin
% \ifCLASSINFOpdf conditional that works the same way.
% When switching from latex to pdflatex and vice-versa, the compiler may
% have to be run twice to clear warning/error messages.






% *** CITATION PACKAGES ***
%
\usepackage{cite}
% cite.sty was written by Donald Arseneau
% V1.6 and later of IEEEtran pre-defines the format of the cite.sty package
% \cite{} output to follow that of the IEEE. Loading the cite package will
% result in citation numbers being automatically sorted and properly
% "compressed/ranged". e.g., [1], [9], [2], [7], [5], [6] without using
% cite.sty will become [1], [2], [5]--[7], [9] using cite.sty. cite.sty's
% \cite will automatically add leading space, if needed. Use cite.sty's
% noadjust option (cite.sty V3.8 and later) if you want to turn this off
% such as if a citation ever needs to be enclosed in parenthesis.
% cite.sty is already installed on most LaTeX systems. Be sure and use
% version 5.0 (2009-03-20) and later if using hyperref.sty.
% The latest version can be obtained at:
% http://www.ctan.org/pkg/cite
% The documentation is contained in the cite.sty file itself.






% *** GRAPHICS RELATED PACKAGES ***
%
%\ifCLASSINFOpdf
  % \usepackage[pdftex]{graphicx}
  % declare the path(s) where your graphic files are
  % \graphicspath{{../pdf/}{../jpeg/}}
  % and their extensions so you won't have to specify these with
  % every instance of \includegraphics
  % \DeclareGraphicsExtensions{.pdf,.jpeg,.png}
%\else
  % or other class option (dvipsone, dvipdf, if not using dvips). graphicx
  % will default to the driver specified in the system graphics.cfg if no
  % driver is specified.
  % \usepackage[dvips]{graphicx}
  % declare the path(s) where your graphic files are
  % \graphicspath{{../eps/}}
  % and their extensions so you won't have to specify these with
  % every instance of \includegraphics
  % \DeclareGraphicsExtensions{.eps}
%\fi
% graphicx was written by David Carlisle and Sebastian Rahtz. It is
% required if you want graphics, photos, etc. graphicx.sty is already
% installed on most LaTeX systems. The latest version and documentation
% can be obtained at: 
% http://www.ctan.org/pkg/graphicx
% Another good source of documentation is "Using Imported Graphics in
% LaTeX2e" by Keith Reckdahl which can be found at:
% http://www.ctan.org/pkg/epslatex
%
% latex, and pdflatex in dvi mode, support graphics in encapsulated
% postscript (.eps) format. pdflatex in pdf mode supports graphics
% in .pdf, .jpeg, .png and .mps (metapost) formats. Users should ensure
% that all non-photo figures use a vector format (.eps, .pdf, .mps) and
% not a bitmapped formats (.jpeg, .png). The IEEE frowns on bitmapped formats
% which can result in "jaggedy"/blurry rendering of lines and letters as
% well as large increases in file sizes.
%
% You can find documentation about the pdfTeX application at:
% http://www.tug.org/applications/pdftex

\usepackage{graphicx}
\usepackage{xcolor}


% *** MATH PACKAGES ***
%
\usepackage{amsmath}
\usepackage{mathrsfs}
\usepackage{amsfonts}
% A popular package from the American Mathematical Society that provides
% many useful and powerful commands for dealing with mathematics.
%
% Note that the amsmath package sets \interdisplaylinepenalty to 10000
% thus preventing page breaks from occurring within multiline equations. Use:
%\interdisplaylinepenalty=2500
% after loading amsmath to restore such page breaks as IEEEtran.cls normally
% does. amsmath.sty is already installed on most LaTeX systems. The latest
% version and documentation can be obtained at:
% http://www.ctan.org/pkg/amsmath





% *** SPECIALIZED LIST PACKAGES ***
%
%\usepackage{algorithmic}
% algorithmic.sty was written by Peter Williams and Rogerio Brito.
% This package provides an algorithmic environment fo describing algorithms.
% You can use the algorithmic environment in-text or within a figure
% environment to provide for a floating algorithm. Do NOT use the algorithm
% floating environment provided by algorithm.sty (by the same authors) or
% algorithm2e.sty (by Christophe Fiorio) as the IEEE does not use dedicated
% algorithm float types and packages that provide these will not provide
% correct IEEE style captions. The latest version and documentation of
% algorithmic.sty can be obtained at:
% http://www.ctan.org/pkg/algorithms
% Also of interest may be the (relatively newer and more customizable)
% algorithmicx.sty package by Szasz Janos:
% http://www.ctan.org/pkg/algorithmicx




% *** ALIGNMENT PACKAGES ***
%
%\usepackage{array}
% Frank Mittelbach's and David Carlisle's array.sty patches and improves
% the standard LaTeX2e array and tabular environments to provide better
% appearance and additional user controls. As the default LaTeX2e table
% generation code is lacking to the point of almost being broken with
% respect to the quality of the end results, all users are strongly
% advised to use an enhanced (at the very least that provided by array.sty)
% set of table tools. array.sty is already installed on most systems. The
% latest version and documentation can be obtained at:
% http://www.ctan.org/pkg/array


% IEEEtran contains the IEEEeqnarray family of commands that can be used to
% generate multiline equations as well as matrices, tables, etc., of high
% quality.




% *** SUBFIGURE PACKAGES ***
%\ifCLASSOPTIONcompsoc
%  \usepackage[caption=false,font=normalsize,labelfont=sf,textfont=sf]{subfig}
%\else
%  \usepackage[caption=false,font=footnotesize]{subfig}
%\fi
% subfig.sty, written by Steven Douglas Cochran, is the modern replacement
% for subfigure.sty, the latter of which is no longer maintained and is
% incompatible with some LaTeX packages including fixltx2e. However,
% subfig.sty requires and automatically loads Axel Sommerfeldt's caption.sty
% which will override IEEEtran.cls' handling of captions and this will result
% in non-IEEE style figure/table captions. To prevent this problem, be sure
% and invoke subfig.sty's "caption=false" package option (available since
% subfig.sty version 1.3, 2005/06/28) as this is will preserve IEEEtran.cls
% handling of captions.
% Note that the Computer Society format requires a larger sans serif font
% than the serif footnote size font used in traditional IEEE formatting
% and thus the need to invoke different subfig.sty package options depending
% on whether compsoc mode has been enabled.
%
% The latest version and documentation of subfig.sty can be obtained at:
% http://www.ctan.org/pkg/subfig




% *** FLOAT PACKAGES ***
%
%\usepackage{fixltx2e}
% fixltx2e, the successor to the earlier fix2col.sty, was written by
% Frank Mittelbach and David Carlisle. This package corrects a few problems
% in the LaTeX2e kernel, the most notable of which is that in current
% LaTeX2e releases, the ordering of single and double column floats is not
% guaranteed to be preserved. Thus, an unpatched LaTeX2e can allow a
% single column figure to be placed prior to an earlier double column
% figure.
% Be aware that LaTeX2e kernels dated 2015 and later have fixltx2e.sty's
% corrections already built into the system in which case a warning will
% be issued if an attempt is made to load fixltx2e.sty as it is no longer
% needed.
% The latest version and documentation can be found at:
% http://www.ctan.org/pkg/fixltx2e


%\usepackage{stfloats}
% stfloats.sty was written by Sigitas Tolusis. This package gives LaTeX2e
% the ability to do double column floats at the bottom of the page as well
% as the top. (e.g., "\begin{figure*}[!b]" is not normally possible in
% LaTeX2e). It also provides a command:
%\fnbelowfloat
% to enable the placement of footnotes below bottom floats (the standard
% LaTeX2e kernel puts them above bottom floats). This is an invasive package
% which rewrites many portions of the LaTeX2e float routines. It may not work
% with other packages that modify the LaTeX2e float routines. The latest
% version and documentation can be obtained at:
% http://www.ctan.org/pkg/stfloats
% Do not use the stfloats baselinefloat ability as the IEEE does not allow
% \baselineskip to stretch. Authors submitting work to the IEEE should note
% that the IEEE rarely uses double column equations and that authors should try
% to avoid such use. Do not be tempted to use the cuted.sty or midfloat.sty
% packages (also by Sigitas Tolusis) as the IEEE does not format its papers in
% such ways.
% Do not attempt to use stfloats with fixltx2e as they are incompatible.
% Instead, use Morten Hogholm'a dblfloatfix which combines the features
% of both fixltx2e and stfloats:
%
% \usepackage{dblfloatfix}
% The latest version can be found at:
% http://www.ctan.org/pkg/dblfloatfix




%\ifCLASSOPTIONcaptionsoff
%  \usepackage[nomarkers]{endfloat}
% \let\MYoriglatexcaption\caption
% \renewcommand{\caption}[2][\relax]{\MYoriglatexcaption[#2]{#2}}
%\fi
% endfloat.sty was written by James Darrell McCauley, Jeff Goldberg and 
% Axel Sommerfeldt. This package may be useful when used in conjunction with 
% IEEEtran.cls'  captionsoff option. Some IEEE journals/societies require that
% submissions have lists of figures/tables at the end of the paper and that
% figures/tables without any captions are placed on a page by themselves at
% the end of the document. If needed, the draftcls IEEEtran class option or
% \CLASSINPUTbaselinestretch interface can be used to increase the line
% spacing as well. Be sure and use the nomarkers option of endfloat to
% prevent endfloat from "marking" where the figures would have been placed
% in the text. The two hack lines of code above are a slight modification of
% that suggested by in the endfloat docs (section 8.4.1) to ensure that
% the full captions always appear in the list of figures/tables - even if
% the user used the short optional argument of \caption[]{}.
% IEEE papers do not typically make use of \caption[]'s optional argument,
% so this should not be an issue. A similar trick can be used to disable
% captions of packages such as subfig.sty that lack options to turn off
% the subcaptions:
% For subfig.sty:
% \let\MYorigsubfloat\subfloat
% \renewcommand{\subfloat}[2][\relax]{\MYorigsubfloat[]{#2}}
% However, the above trick will not work if both optional arguments of
% the \subfloat command are used. Furthermore, there needs to be a
% description of each subfigure *somewhere* and endfloat does not add
% subfigure captions to its list of figures. Thus, the best approach is to
% avoid the use of subfigure captions (many IEEE journals avoid them anyway)
% and instead reference/explain all the subfigures within the main caption.
% The latest version of endfloat.sty and its documentation can obtained at:
% http://www.ctan.org/pkg/endfloat
%
% The IEEEtran \ifCLASSOPTIONcaptionsoff conditional can also be used
% later in the document, say, to conditionally put the References on a 
% page by themselves.




% *** PDF, URL AND HYPERLINK PACKAGES ***
%
%\usepackage{url}
% url.sty was written by Donald Arseneau. It provides better support for
% handling and breaking URLs. url.sty is already installed on most LaTeX
% systems. The latest version and documentation can be obtained at:
% http://www.ctan.org/pkg/url
% Basically, \url{my_url_here}.




% *** Do not adjust lengths that control margins, column widths, etc. ***
% *** Do not use packages that alter fonts (such as pslatex).         ***
% There should be no need to do such things with IEEEtran.cls V1.6 and later.
% (Unless specifically asked to do so by the journal or conference you plan
% to submit to, of course. )
\usepackage{hyperref}
\hypersetup{
    colorlinks=true,
    linkcolor=blue,
    filecolor=magenta,      
    urlcolor=cyan
}

% correct bad hyphenation here
%\hyphenation{op-tical net-works semi-conduc-tor}
\usepackage[detect-all]{siunitx}
\usepackage{lmodern}
\usepackage[english]{babel}
\newtheorem{theorem}{Proposition}
\allowdisplaybreaks

%\usepackage{csquotes}
%\usepackage[backend=biber,
%isbn=false,%
%style=numeric,%
%sorting=ydnt,%
%giveninits=true,% initials of first names
%maxbibnames=20,%
%defernumbers,%
%labeldateparts,% for academic cv
%locallabelwidth% for academic cv
%]{biblatex}
%\renewcommand*{\bibfont}{\footnotesize}
%\addbibresource{mybib.bib}
%\addbibresource{mybib1.bib}
%\addbibresource{mybib2.bib}
%\addbibresource{mybib3.bib}
%\addbibresource{mybib4.bib}
%\addbibresource{mybib5.bib}

\begin{document}
%
% paper title
% Titles are generally capitalized except for words such as a, an, and, as,
% at, but, by, for, in, nor, of, on, or, the, to and up, which are usually
% not capitalized unless they are the first or last word of the title.
% Linebreaks \\ can be used within to get better formatting as desired.
% Do not put math or special symbols in the title.
\title{New Results on Single User Massive MIMO}
%
%
% author names and IEEE memberships
% note positions of commas and nonbreaking spaces ( ~ ) LaTeX will not break
% a structure at a ~ so this keeps an author's name from being broken across
% two lines.
% use \thanks{} to gain access to the first footnote area
% a separate \thanks must be used for each paragraph as LaTeX2e's \thanks
% was not built to handle multiple paragraphs
%

\author{Kasturi Vasudevan$^1$, %~\IEEEmembership{Member,~IEEE,}
        Surendra Kota$^1$, %~\IEEEmembership{Fellow,~OSA,}
        Lov Kumar$^1$ and~Himanshu Bhusan Mishra$^2$
        %,~\IEEEmembership{Life~Fellow,~IEEE}
        % <-this % stops a space
\thanks{$^1$Department of Electrical Engineering Indian Institute of Technology Kanpur
208016 India e-mail: \{vasu, skota, lovkr20\}@iitk.ac.in, $^2$Department of Electronics
Engineering Indian Institute of Technology (Indian School of Mines) Dhanbad 826004
India email: himanshu@iitism.ac.in.}% <-this % stops a space
%\thanks{J. Doe and J. Doe are with Anonymous University.}% <-this % stops a space
%\thanks{Manuscript received April 19, 2005; revised August 26, 2015.}
}

% note the % following the last \IEEEmembership and also \thanks - 
% these prevent an unwanted space from occurring between the last author name
% and the end of the author line. i.e., if you had this:
% 
% \author{....lastname \thanks{...} \thanks{...} }
%                     ^------------^------------^----Do not want these spaces!
%
% a space would be appended to the last name and could cause every name on that
% line to be shifted left slightly. This is one of those "LaTeX things". For
% instance, "\textbf{A} \textbf{B}" will typeset as "A B" not "AB". To get
% "AB" then you have to do: "\textbf{A}\textbf{B}"
% \thanks is no different in this regard, so shield the last } of each \thanks
% that ends a line with a % and do not let a space in before the next \thanks.
% Spaces after \IEEEmembership other than the last one are OK (and needed) as
% you are supposed to have spaces between the names. For what it is worth,
% this is a minor point as most people would not even notice if the said evil
% space somehow managed to creep in.



% The paper headers
\markboth{Book Chapter Intech Open}%
{Vasudevan \MakeLowercase{\textit{et al.}}: New Results on SU-MMIMO}
% The only time the second header will appear is for the odd numbered pages
% after the title page when using the twoside option.
% 
% *** Note that you probably will NOT want to include the author's ***
% *** name in the headers of peer review papers.                   ***
% You can use \ifCLASSOPTIONpeerreview for conditional compilation here if
% you desire.




% If you want to put a publisher's ID mark on the page you can do it like
% this:
%\IEEEpubid{0000--0000/00\$00.00~\copyright~2015 IEEE}
% Remember, if you use this you must call \IEEEpubidadjcol in the second
% column for its text to clear the IEEEpubid mark.



% use for special paper notices
%\IEEEspecialpapernotice{(Invited Paper)}




% make the title area
\maketitle

% As a general rule, do not put math, special symbols or citations
% in the abstract or keywords.
%----------------------------------------------------------------------------------------
\begin{abstract}
Achieving high bit rates is the main goal of wireless technologies like 5G and
beyond. This translates to obtaining high spectral efficiencies using large number
of antennas at the transmitter and receiver (single user massive multiple input
multiple output or SU-MMIMO). It is possible to have a large number of antennas in the
mobile handset at mm-wave frequencies in the range $30 - 300$ GHz due to the
small antenna size.
In this work, we investigate the bit-error-rate (BER) performance of SU-MMIMO in two
scenarios (a) using serially concatenated turbo code (SCTC) in uncorrelated channel and (b)
parallel concatenated turbo code (PCTC) in correlated channel. Computer simulation results
indicate that the BER is quite insensitive to re-transmissions and wide variations in the
number of transmit and receive antennas. Moreover, we have obtained a BER of $10^{-5}$
at an average signal-to-interference plus noise ratio (SINR) per bit of just 1.25 dB with
512 transmit and receive antennas ($512\times 512$ SU-MMIMO system) with a spectral
efficiency of 256 bits/transmission or 256 bits/sec/Hz in an uncorrelated channel.
Similar BER results have been obtained for SU-MMIMO using PCTC in correlated channel.
A semi-analytic approach to estimating the BER of a turbo code has been derived.
\end{abstract}
%----------------------------------------------------------------------------------------
% Note that keywords are not normally used for peerreview papers.
\begin{IEEEkeywords}
Single user massive multiple input multiple output (SU-MMIMO), Rayleigh fading, serially
concatenated turbo code (SCTC), parallel concatenated turbo code (PCTC), spectral
efficiency (SE), signal-to-interference plus noise ratio (SINR) per bit, spatial
multiplexing, bit-error-rate (BER).
\end{IEEEkeywords}
%----------------------------------------------------------------------------------------


% For peer review papers, you can put extra information on the cover
% page as needed:
% \ifCLASSOPTIONpeerreview
% \begin{center} \bfseries EDICS Category: 3-BBND \end{center}
% \fi
%
% For peerreview papers, this IEEEtran command inserts a page break and
% creates the second title. It will be ignored for other modes.
\IEEEpeerreviewmaketitle



\section{Introduction}
% The very first letter is a 2 line initial drop letter followed
% by the rest of the first word in caps.
% 
% form to use if the first word consists of a single letter:
% \IEEEPARstart{A}{demo} file is ....
% 
% form to use if you need the single drop letter followed by
% normal text (unknown if ever used by the IEEE):
% \IEEEPARstart{A}{}demo file is ....
% 
% Some journals put the first two words in caps:
% \IEEEPARstart{T}{his demo} file is ....
% 
% Here we have the typical use of a "T" for an initial drop letter
% and "HIS" in caps to complete the first word.
\IEEEPARstart{A}{s} wireless technologies evolve beyond 5G \cite{9144301,9757375,9939166},
there is a growing need to
attain peak data rates of about gigabits per second per user, which is required for
high definition video, remote surgery, autonomous vehicles, gaming and so on, while
at the same time
consuming minimum transmit power. This can only be achieved by using multiple antennas
at the transmitter and receiver \cite{9864042,10002368,10006402,10045774,10049425},
small constellations like quadrature shift keying (QPSK)
and powerful error correcting codes like turbo or low density parity check (LDPC) codes.
Having a large number of antennas in the mobile handset is feasible in mm-wave frequencies
\cite{8901159,9310258,9705623,9946839}
($30 - 300$ GHz) due to the small antenna size. The main concern about mm wave
communications has been its rather high attenuation in outdoor environments with
rain and snow
\cite{8761087}. Therefore, at least in the initial stages, mm wave could be deployed
indoors. The second issue relates to the poor penetration characteristics of mm wave
through walls, doors, windows and other materials. This points towards to usage of mm wave
\cite{8901159} in a single room, say a big auditorium or underground parking and so on.
Reconfigurable intelligent surface (RIS)
\cite{9087848,9424177,9721205,10.1007/978-981-16-7423-5_102} could be used to boost
the propagation of mm waves, both indoors and outdoors.
% You must have at least 2 lines in the paragraph with the drop letter
% (should never be an issue)

Most of the massive MIMO systems discussed in the literature are multi-user (MU)
\cite{6798744,6777306,6808541,6971234,6987288,7160668,7342925,7439790,MMIMO_Khwandah_2021},
that is, the base station has a large number of antennas and the mobile handset has only
a single antenna ($N_t=1$). A large number of users are served simultaneously by the
base station. A comparison between MU-MMIMO and SU-MMIMO is given in
Table~\ref{Tbl:MU_SU_MMIMO_Comp} \cite{KV_MMWS2021,Vasu_MMIMO_INGR_2022}.
%*******************************************************************************
\begin{table}[tbhp]
\centering
%\begin{center}
\caption{Comparison of MU-MMIMO and SU-MMIMO.}
\input{mmimo_comp1.pstex_t}
\label{Tbl:MU_SU_MMIMO_Comp}
%\end{center}
\end{table}
%*******************************************************************************
The base station in MU-MMIMO uses beamforming to improve the signal-to-noise
ratio at the mobile handset. On the other hand, SU-MMIMO uses spatial
multiplexing to improve the spectral efficiency in the downlink and uplink.
The comparison between beamforming and spatial multiplexing is given in
Table~\ref{Tbl:Beam_vs_SM} \cite{KV_MMWS2021,Vasu_MMIMO_INGR_2022}.
%*******************************************************************************
\begin{table}[tbhp]
\centering
%\begin{center}
\caption{Comparison of beamforming and spatial multiplexing.}
\input{beam_vs_sm.pstex_t}
\label{Tbl:Beam_vs_SM}
%\end{center}
\end{table}
%*******************************************************************************
The total transmit power of SU-MMIMO using uncoded QPSK versus MU-MMIMO using
$M$-ary QAM is shown in Table~\ref{Tbl:MMIMO_QPSK_Mary}. The minimum Euclidean
distance between symbols of all constellations is taken to be 2.
%*******************************************************************************
\begin{table*}[tbhp]
\centering
%\begin{center}
\caption{SU-MMIMO using QPSK vs MU-MMIMO using $M$-ary.}
\input{mmimo_qpsk_mary.pstex_t}
\label{Tbl:MMIMO_QPSK_Mary}
%\end{center}
\end{table*}
%*******************************************************************************
The peak-to-average power ratio (PAPR) for SU-MMIMO using QPSK is compared with
MU-MMIMO using $M$-ary QAM in Table~\ref{Tbl:MMIMO_PAPR1} \cite{KV_MMWS2021}.
Of course in the case of frequency selective fading channels, OFDM needs to be used,
which would result in PAPR greater than 0 dB even for QPSK signalling.
%*******************************************************************************
\begin{table}[tbhp]
\centering
%\begin{center}
\caption{PAPR of SU-MMIMO using QPSK vs MU-MMIMO using $M$-ary.}
\input{mmimo_papr1.pstex_t}
\label{Tbl:MMIMO_PAPR1}
%\end{center}
\end{table}
%*******************************************************************************
It is clear from Tables~\ref{Tbl:MU_SU_MMIMO_Comp} -- \ref{Tbl:MMIMO_PAPR1} that
technologies that use SU-MMIMO have a lot to gain. Moreover, since all transmit
antennas use the same carrier frequency, there is no increase in bandwidth.

SU-MMIMO with equal number of transmit and receive antennas is given in
\cite{KV_OpSigPJ2019,73ddc0ea-7d42-4fdd-969d-da08c8e4d0c0}. The probability of
erasure in MIMO-OFDM is presented in \cite{KV_SSID2020}. A practical SU-MMIMO
receiver with estimated channel, carrier frequency offset and timing is described
in \cite{Vasu_intech:2019,d4bbbdf0-7468-4727-9ebe-76d5e6160b64}. SU-MMIMO with
unequal number of transmit and receive antennas and precoding is discussed in
\cite{KV_ARCI2021,da1844bd-7ee2-4d99-b53b-2339010e03b0} and the case without
precoding in \cite{KV_Oct_2021,4466bded-2b5a-454b-8f3d-a6dd0b74831d}. All the earlier
research on SU-MMIMO involved the use of a parallel concatenated turbo code (PCTC) and
uncorrelated channel. In this work, we investigate the performance of SU-MMIMO using
(a) serial concatenated turbo code (SCTC) in uncorrelated channel and (b) PCTC in
correlated channel. Throughout this article we assume that the channel is known
perfectly at the receiver. Perfect carrier and timing synchronization is also
assumed.

This work is organized as follows. Section~\ref{Sec:SU_MMIMO_SCTC} discusses
SU-MMIMO with SCTC in uncorrelated channel, the procedure for bit-error-rate
(BER) estimation and computer simulation results.
Section~\ref{Sec:SU_MMIMO_PCTC_Corr_Chan} deals with SU-MMIMO using PCTC in
correlated channel with and without precoding along with computer simulation
results. Section~\ref{Sec:Conclude} presents the conclusions and scope for
future work.
%*******************************************************************************
\section{SU-MMIMO with SCTC}
\label{Sec:SU_MMIMO_SCTC}
\subsection{System Model}
\label{SSec:SU_MMIMO_SCTC_Model}
Consider the block diagram in Figure~\ref{Fig:System_SCTC}
\cite{Vasu07,KV_Oct_2021}. The input bits $a_i$, $1\le i \le L_{d1}$ is passed
through an outer rate-$1/2$ recursive systematic convolutional (RSC) encoder to
obtain the coded bit stream $b_i$, $1\le i \le L_d$, where
%*******************************************************************************
\begin{equation}
\label{Eq:SU_MMIMO_SCTC_Eq1}
L_d = 2 L_{d1}.
\end{equation}
%*******************************************************************************
Now $b_i$ is input to an interleaver to generate $c_i$, $1\le i \le L_d$. Next
$c_i$ is passed through an inner rate-$1/2$ RSC encoder and mapped to symbols
$S_i$, $1\le i \le L_d$, in a quadrature phase shift keyed (QPSK) constellation
having symbol coordinates $\pm 1\pm\,\mathrm{j}$, where $\mathrm{j}=\sqrt{-1}$.
Throughout this article we assume that bit ``0'' maps to $+1$ and bit ``1''
maps to $-1$.
%*******************************************************************************
\begin{figure*}[tbhp]
\centering
%\begin{center}
\input{system_sctc.pstex_t}
\caption{SU-MMIMO with serially concatenated turbo code.}
\label{Fig:System_SCTC}
%\end{center}
\end{figure*}
%*******************************************************************************
The set of $L_d$ QPSK symbols constitute a ``frame'' and are transmitted using
$N_t$ antennas. We assume that
%*******************************************************************************
\begin{equation}
\label{Eq:SU_MMIMO_SCTC_Eq2}
\frac{L_d}{N_t} = \mbox{an integer}
\end{equation}
%*******************************************************************************
so that all symbols in the frame are transmitted using $N_t$ antennas. The set
of QPSK symbols transmitted simultaneously using $N_t$ antennas constitute a
``block''. The generator matrix for both the inner and outer rate-$1/2$ RSC
encoder is given by
%*******************************************************************************
\begin{equation}
\label{Eq:SU_MMIMO_SCTC_Eq2_1}
\mathbf{G}(D) = \left[
                \begin{array}{cc}
                 1 & \frac{1+D^2}{1+D+D^2}
                \end{array}
                \right].
\end{equation}
%*******************************************************************************
Hence, both encoders have $S_E=4$ states in the trellis.
Assuming uncorrelated Rayleigh flat fading, the received signal for the $k^{th}$
re-transmission ($0\le k\le N_{rt}-1$, $k$ is an integer) is given by (2) of
\cite{KV_Oct_2021}, which is repeated here for convenience
%*******************************************************************************
\begin{equation}
\label{Eq:SU_MMIMO_SCTC_Eq3}
\tilde{\mathbf{R}}_k = \tilde{\mathbf{H}}_k
                       \mathbf{S} +
                       \tilde{\mathbf{W}}_k
\end{equation}
%*******************************************************************************
where $\mathbf{S}\in \mathbb{C}^{N_t\times 1}$ whose elements are drawn
from the QPSK constellation, $\tilde{\mathbf{H}}_k\in \mathbb{C}^{N_r\times N_t}$
whose elements are mutually independent and $\mathscr{CN}(0,\, 2\sigma^2_H)$ and
and $\tilde{\mathbf{W}}_k\in \mathbb{C}^{N_r\times 1}$ is the additive 
white Gaussian noise (AWGN) vector whose elements are mutually independent
and $\mathscr{CN}(0,\, 2\sigma^2_W)$. Note that $\sigma^2_H,\, \sigma^2_W$
denote the variance per dimension (real part or imaginary part) and $N_r$ is
the number of receive antennas. We assume that $\tilde{\mathbf{H}}_k$ and
$\tilde{\mathbf{W}}_k$ are independent across blocks and re-transmissions,
hence (4) in  \cite{KV_OpSigPJ2019} is valid with $N$ replaced by $N_t$.
Recall that (see also (16) of \cite{KV_Oct_2021})
%*******************************************************************************
\begin{equation}
\label{Eq:SU_MMIMO_SCTC_Eq3_1}
N_{\mathrm{tot}} = N_t + N_r.
\end{equation}
%*******************************************************************************
Following the procedure
given in Section~4 of \cite{KV_Oct_2021} we get (see (36) of \cite{KV_Oct_2021})
%*******************************************************************************
\begin{equation}
\label{Eq:SU_MMIMO_SCTC_Eq4}
\tilde{Y}_i  =  F_i S_i + \tilde{U}_i \qquad \mbox{for $1\le i \le N_t$}.
\end{equation}
%*******************************************************************************
After concatenation over blocks, $\tilde{Y}_i$ in (\ref{Eq:SU_MMIMO_SCTC_Eq4})
for $1\le i \le L_d$ is sent to the turbo decoder (see also the sentence after
(25) in \cite{KV_OpSigPJ2019}). For the sake of consistency with earlier work
\cite{Vasu07}, we re-index $i$ as $0\le i \le L_d-1$ and use the
same index $i$ for $a_i$, $b_i$, $c_i$ and $Y_i$ without any ambiguity.
In the next subsection, we
discuss the turbo decoding (BCJR) algorithm \cite{BCJR74,Vasu_Book10} for the
inner code.
%*******************************************************************************
\subsection{BCJR for the Inner Code}
\label{SSec:BCJR_Inner_Code}
Let $\mathscr{D}_n$ denote the set of states that diverge from state $n$ in the
trellis \cite{Vasu07,Vasu_Book10}. Similarly, let $\mathscr{C}_n$ denote the set
of states that converge to state $n$. Let $\alpha_{i,\, n}$ denote the forward
sum-of-products (SOP) at time $i$, $0 \le i \le L_d - 2$, at state $n$,
$0 \le n \le S_E - 1$. Then the forward SOP can be recursively computed
as follows (see also (30) of \cite{Vasu07}):
%*******************************************************************************
\begin{align}
\label{Eq:SU_MMIMO_SCTC_Eq5}
\alpha_{i+1,\, n}' & = \sum_{m\in \mathscr{C}_n}
                       \alpha_{i,\, m}
                       \gamma_{i,\, m,\, n}
                        P(c_{i,\, m,\, n})    \nonumber  \\
\alpha_{0,\, n}    & =  1                     \nonumber  \\
\alpha_{i+1,\, n}  & = \left.
                       \alpha_{i+1,\, n}'
                       \middle/
                       \left(
                       \sum_{n=0}^{S_E-1}
                       \alpha_{i+1,\, n}'
                       \right)
                       \right.
\end{align}
%*******************************************************************************
where $P(c_{i, m, n})$ denotes the \textit{a priori} probability of the
systematic bit corresponding to the transition from encoder state $m$
to $n$, at time $i$ (this is set to 0.5 at the beginning of the first iteration).
The last equation in (\ref{Eq:SU_MMIMO_SCTC_Eq5}) is required to prevent
numerical instabilities \cite{Vasu_Book10}. We have
%*******************************************************************************
\begin{equation}
\label{Eq:SU_MMIMO_SCTC_Eq6}
\gamma_{i,\, m,\, n}   = \exp
                         \left(
                         -
                         \frac{\left(\tilde{Y}_i-S_{m,\, n}\right)^2}
                              {2\sigma^2_U}
                         \right)
\end{equation}
%*******************************************************************************
where $\tilde{Y}_i$ is given by (\ref{Eq:SU_MMIMO_SCTC_Eq4}), $S_{m,\, n}$ is the
QPSK symbol corresponding to the transition from encoder state $m$ to $n$ and
$\sigma^2_U$ is given by (38) of \cite{KV_Oct_2021} which is repeated here for
convenience:
%*******************************************************************************
\begin{align}
\label{Eq:SU_MMIMO_SCTC_Eq7}
 E
\left[
\left|
\tilde{U}_i
\right|^2
\right] & = \frac{8 \sigma_H^4 N_r (N_t-1) + 4\sigma_W^2\sigma^2_H N_r}
                 {N_{rt}}         \nonumber  \\
        &   \stackrel{\Delta}{=}
            \sigma^2_U.
\end{align}
%*******************************************************************************
Robust turbo decoding (see section 4.2 of \cite{Vasudevan2015}) can be employed
to compute $\gamma_{i,\, m,\, n}$ in (\ref{Eq:SU_MMIMO_SCTC_Eq6}).
Similarly, let $\beta_{i,\, m}$ denote the backward SOP at time $i$,
$1\le i \le L_d -1$, at state $m$, $0\le m \le S_E -1$. Then the
backward SOP can be recursively computed as (see also (33) of \cite{Vasu07}):
%*******************************************************************************
\begin{align}
\label{Eq:SU_MMIMO_SCTC_Eq8}
\beta_{i,\, m}'  & = \sum_{n\in \mathscr{D}_m}
                     \beta_{i+1,\, n}
                     \gamma_{i,\, m,\, n}
                      P(c_{i,\, m,\, n})    \nonumber  \\
\beta_{L_d,\, m} & =  1                     \nonumber  \\
\beta_{i,\, m}   & = \left.
                     \beta_{i,\, m}'
                     \middle/
                     \left(
                     \sum_{m=0}^{S_E-1}
                     \beta_{i,\, m}'
                     \right)
                     \right.
\end{align}
%*******************************************************************************
Let $\rho^+(n)$ denote the state that is reached from encoder state $n$
when the input symbol is $+1$. Similarly let $\rho^-(n)$ denote the
state that can be reached from encoder state $n$ when the input symbol
is $-1$. Then for $0\le i \le L_d-1$ we compute
%*******************************************************************************
\begin{align}
\label{Eq:SU_MMIMO_SCTC_Eq9}
C_{i+}                  & = \sum_{n=0}^{S_E-1}
                            \alpha_{i,\, n}
                            \gamma_{i,\, n,\,\rho^+(n)}
                            \beta_{i+1,\,\rho^+(n)}          \nonumber  \\
C_{i-}                  & = \sum_{n=0}^{S_E-1}
                            \alpha_{i,\, n}
                            \gamma_{i,\, n,\,\rho^-(n)}
                            \beta_{i+1,\,\rho^-(n)}.
\end{align}
%*******************************************************************************
Finally, the extrinsic information that is fed to the BCJR algorithm for the
outer code is computed as, for $0\le i \le L_d-1$, (see (36) of \cite{Vasu07}):
%*******************************************************************************
\begin{align}
\label{Eq:SU_MMIMO_SCTC_Eq10}
 E
\left(
 c_i = +1
\right) & =  C_{i+}/(C_{i+}+C_{i-})                \nonumber  \\
  E
\left(
 c_i = -1
\right) & =  C_{i-}/(C_{i+}+C_{i-}).
\end{align}
%*******************************************************************************
Next, we describe the BCJR for the outer code.
%*******************************************************************************
\subsection{BCJR for the Outer Code}
\label{SSec:BCJR_Outer_Code}
Let $\alpha_{i,\, n}$ denote the forward SOP at time $i$, $0\le i \le L_{d1}-2$,
at state $n$, $0\le n \le S_E -1$. Then the forward SOP is recursively computed
as follows:
%*******************************************************************************
\begin{align}
\label{Eq:SU_MMIMO_SCTC_Eq11}
\alpha_{i+1,\, n}' & = \sum_{m\in \mathscr{C}_n}
                       \alpha_{i,\, m}
                       \gamma_{\mathrm{sys},\, i,\, m,\, n}
                       \gamma_{\mathrm{par},\, i,\, m,\, n}
                        P(a_{i,\, m,\, n})    \nonumber  \\
\alpha_{0,\, n}    & =  1                     \nonumber  \\
\alpha_{i+1,\, n}  & = \left.
                       \alpha_{i+1,\, n}'
                       \middle/
                       \left(
                       \sum_{n=0}^{S_E-1}
                       \alpha_{i+1,\, n}'
                       \right)
                       \right.
\end{align}
%*******************************************************************************
where $P(a_{i,\, m,\, n})$ denotes the \textit{a priori} probability of the
systematic bit corresponding to the transition from state $m$ to
state $n$, at time $i$. In the absence of any other information, we assume
$P(a_{i,\, m,\, n})=0.5$ \cite{Singer02}. We also have for $0\le i \le L_{d1}-1$
(similar to (38) of \cite{Vasu07})
%*******************************************************************************
\begin{align}
\label{Eq:SU_MMIMO_SCTC_Eq12}
\gamma_{\mathrm{sys},\, i,\, m,\, n}
& = \left
    \{
    \begin{array}{ll}
     E\left(c_{\pi(2i)}=+1\right) &  \mbox{if $\mathscr{H}_1$}\\
     E\left(c_{\pi(2i)}=-1\right) &  \mbox{if $\mathscr{H}_2$}
    \end{array}
    \right.                                     \nonumber  \\
\gamma_{\mathrm{par},\, i,\, m,\, n}
& = \left
    \{
    \begin{array}{ll}
     E\left(c_{\pi(2i+1)}=+1\right) &  \mbox{if $\mathscr{H}_3$}\\
     E\left(c_{\pi(2i+1)}=-1\right) &  \mbox{if $\mathscr{H}_4$}
    \end{array}
    \right.
\end{align}
%*******************************************************************************
where $\pi(\cdot)$ denotes the interleaver map and
%*******************************************************************************
\begin{align}
\label{Eq:SU_MMIMO_SCTC_Eq13}
\mathscr{H}_1 & : \mbox{systematic bit from state $m$ to $n$ is $+1$} \nonumber  \\
\mathscr{H}_2 & : \mbox{systematic bit from state $m$ to $n$ is $-1$} \nonumber  \\
\mathscr{H}_3 & : \mbox{parity bit from state $m$ to $n$ is $+1$}     \nonumber  \\
\mathscr{H}_4 & : \mbox{parity bit from state $m$ to $n$ is $-1$}.
\end{align}
%*******************************************************************************
Observe that in (\ref{Eq:SU_MMIMO_SCTC_Eq12}) and (\ref{Eq:SU_MMIMO_SCTC_Eq13})
it is assumed that after the parallel-to-serial conversion in
Figure~\ref{Fig:System_SCTC}, $b_{2i}$ corresponds to the systematic (data) bits
and $b_{2i+1}$ corresponds to the parity bits for $0\le i\le L_{d1}-1$.

Similarly, let $\beta_{i,\, m}$ denote the backward SOP at time $i$,
$1\le i \le L_{d1}-1$, at state $m$, $0\le m \le S_E-1$. Then the backward
SOP can be recursively computed as:
%*******************************************************************************
\begin{align}
\label{Eq:SU_MMIMO_SCTC_Eq14}
\beta_{i,\, m}'  & = \sum_{n\in \mathscr{D}_m}
                     \beta_{i+1,\, n}
                     \gamma_{\mathrm{sys},\, i,\, m,\, n}
                     \gamma_{\mathrm{par},\, i,\, m,\, n}
                      P(a_{i,\, m,\, n})    \nonumber  \\
\beta_{L_{d1},\, m}
                 & =  1                     \nonumber  \\
\beta_{i,\, m}   & = \left.
                     \beta_{i,\, m}'
                     \middle/
                     \left(
                     \sum_{m=0}^{S_E-1}
                     \beta_{i,\, m}'
                     \right).
                     \right.
\end{align}
%*******************************************************************************
Next, for $0\le i\le L_{d1}-1$ we compute
%*******************************************************************************
\begin{align}
\label{Eq:SU_MMIMO_SCTC_Eq15}
B_{2i+}                 & = \sum_{n=0}^{S_E-1}
                            \alpha_{i,\, n}
                            \gamma_{\mathrm{par},\, i,\, n,\,\rho^+(n)}
                            \beta_{i+1,\,\rho^+(n)}          \nonumber  \\
B_{2i-}                 & = \sum_{n=0}^{S_E-1}
                            \alpha_{i,\, n}
                            \gamma_{\mathrm{par},\, i,\, n,\,\rho^-(n)}
                            \beta_{i+1,\,\rho^-(n)}.
\end{align}
%*******************************************************************************
Let $\mu^+(n)$ and $\mu^-(n)$ denote the states that are
reached from state $n$ when the parity bit is $+1$ and $-1$, respectively.
Similarly for $0\le i \le L_{d1}-1$ compute
%*******************************************************************************
\begin{align}
\label{Eq:SU_MMIMO_SCTC_Eq16}
B_{2i+1+}               & = \sum_{n=0}^{S_E-1}
                            \alpha_{i,\, n}
                            \gamma_{\mathrm{sys},\, i,\, n,\,\mu^+(n)}
                            \beta_{i+1,\,\mu^+(n)}          \nonumber  \\
B_{2i+1-}               & = \sum_{n=0}^{S_E-1}
                            \alpha_{i,\, n}
                            \gamma_{\mathrm{sys},\, i,\, n,\,\mu^-(n)}
                            \beta_{i+1,\,\mu^-(n)}.
\end{align}
%*******************************************************************************
The extrinsic information that is sent to the inner decoder for
$0\le i\le L_d-1$ is computed as
%*******************************************************************************
\begin{align}
\label{Eq:SU_MMIMO_SCTC_Eq17}
 E
\left(
 b_i = +1
\right) & =  B_{i+}/(B_{i+}+B_{i-})                \nonumber  \\
  E
\left(
 b_i = -1
\right) & =  B_{i-}/(B_{i+}+B_{i-})
\end{align}
%*******************************************************************************
where $B_{i+},\, B_{i-}$ are given by (\ref{Eq:SU_MMIMO_SCTC_Eq15}) or
(\ref{Eq:SU_MMIMO_SCTC_Eq16}) depending on whether $i$ is even or odd
respectively. Note that $P(c_{i,\, m,\, n})$ for $0\le i \le L_d-1$ in
(\ref{Eq:SU_MMIMO_SCTC_Eq5}) and (\ref{Eq:SU_MMIMO_SCTC_Eq8}) is equal to
%*******************************************************************************
\begin{equation}
\label{Eq:SU_MMIMO_SCTC_Eq18}
 P
\left(
 c_{i,\, m,\, n}
\right) = \left
          \{
          \begin{array}{ll}
           E
          \left(
           b_{\pi^{-1}(i)} = +1
          \right) &  \mbox{if $\mathscr{H}_1$}\\
           E
          \left(
           b_{\pi^{-1}(i)} = -1
          \right) &  \mbox{if $\mathscr{H}_2$}
          \end{array}
          \right.
\end{equation}
%*******************************************************************************
where $\pi^{-1}(\cdot)$ denotes the inverse interleaver map.
Note that $c_{i,\, m,\, n}$ are the systematic (data) bits for the inner
encoder.

After the convergence of the BCJR algorithm in the last iteration, the final
\textit{a posteriori} probabilities of $a_i$ for $0\le i\le L_{d1}-1$ is given
by
%*******************************************************************************
\begin{align}
\label{Eq:SU_MMIMO_SCTC_Eq19}
 P
\left(
 a_i = +1
\right) & =  E
            \left(
             b_{2i} = +1
            \right)
             E
            \left(
             c_{\pi(2i)} = +1
            \right)                       \nonumber  \\
 P
\left(
 a_i = -1
\right) & =  E
            \left(
             b_{2i} = -1
            \right)
             E
            \left(
             c_{\pi(2i)} = -1
            \right)
\end{align}
%*******************************************************************************
where $E\left(c_i=\pm 1\right)$ and $E\left(b_i=\pm 1\right)$ are given by
(\ref{Eq:SU_MMIMO_SCTC_Eq10}) and (\ref{Eq:SU_MMIMO_SCTC_Eq17}) respectively.
Finally note that for $0\le i \le L_{d1} - 1$
%*******************************************************************************
\begin{align}
\label{Eq:SU_MMIMO_SCTC_Eq19_1}
a_i = b_{2i} = c_{\pi(2i)}.
\end{align}
%*******************************************************************************
In the next section we present the estimation of the bit-error-rate (BER) of the
SCTC.
%*******************************************************************************
\subsection{Estimation of BER}
\label{SSec:Est_BER}
%*******************************************************************************
\begin{figure*}[tbhp]
\centering
%\begin{center}
\input{dmt55_i_avg_hist.pstex_t}
\caption{Normalized histogram for $N_{\mathrm{tot}}=1024$, $N_t=512$, $N_{rt}=2$
         (a) $L_{d1}=1024$, $\mathrm{SNR}_{\mathrm{av},\, b}=1.25$ dB, $F=10^5$
             frames
         (b) $L_{d1}=50176$, $\mathrm{SNR}_{\mathrm{av},\, b}=0.3$ dB, $F=2000$
             frames
         (c) Expanded view of (a) around $r_{3,\, i}=0$ and
         (d) $L_{d1}=50176$, $\mathrm{SNR}_{\mathrm{av},\, b}=0.5$ dB, $F=2000$
             frames.}
\label{Fig:DMT55_i_Avg_Hist}
%\end{center}
\end{figure*}
%*******************************************************************************
%*******************************************************************************
\begin{figure*}[tbhp]
\centering
%\begin{center}
\input{dmt55_i_hist_max_fr2.pstex_t}
\caption{Normalized histogram over two frames ($F=2$) for
         $N_{\mathrm{tot}}=1024$, $N_t=512$, $N_{rt}=2$
         (a) $L_{d1}=1024$, $\mathrm{SNR}_{\mathrm{av},\, b}=1.25$ dB and
         (b) $L_{d1}=50176$, $\mathrm{SNR}_{\mathrm{av},\, b}=0.5$ dB.}
\label{Fig:DMT55_i_Hist_Max_Fr2}
%\end{center}
\end{figure*}
%*******************************************************************************
The estimation of BER of SCTC is based on the following propositions:
%*******************************************************************************
\begin{theorem}
\textit{The extrinsic information as computed in (\ref{Eq:SU_MMIMO_SCTC_Eq10}) and
(\ref{Eq:SU_MMIMO_SCTC_Eq17}) lies in the range $[0,\, 1]$ (0 and 1 included).
The extrinsic information in the range $(0,\, 1)$, 0 and 1 excluded,
is Gaussian distributed \cite{tenBrink01} for each frame.}
\end{theorem}
%*******************************************************************************
This is illustrated in Figure~\ref{Fig:DMT55_i_Avg_Hist} for different values
of the frame length $L_{d1}$, over many frames ($F$). We find that for large
values of $L_{d1}$, the histogram better approximates the Gaussian characteristic. 
It may be noted that the extrinsic information at the output of one decoder
is equal to the \textit{a priori} probabilities for the other decoder.
%*******************************************************************************
\begin{theorem}
\textit{After convergence of the BCJR algorithm in the final iteration, the
extrinsic information at a decoder output has the same mean and
variance as that of the \textit{a priori} probability at its input.}
\end{theorem}
%*******************************************************************************
%*******************************************************************************
\begin{theorem}
\textit{The mean and variance of the Gaussian distribution may vary from
frame to frame.}
\end{theorem}
%*******************************************************************************
This is illustrated in Figure~\ref{Fig:DMT55_i_Hist_Max_Fr2} over two frames,
that is, $F=2$.

Based on \textit{Propositions 1 \& 2} and (\ref{Eq:SU_MMIMO_SCTC_Eq19_1}),
after convergence of the BCJR algorithm,
we can write for $0\le i \le L_{d1} - 1$
%*******************************************************************************
\begin{align}
\label{Eq:SU_MMIMO_SCTC_Eq20}
 E
\left(
 b_{2i}=+1
\right) & = \frac{1}{\sigma\sqrt{2\pi}}
            \mathrm{e}^{-(r_{1,\, i}-A)^2/(2\sigma^2)}     \nonumber  \\
 E
\left(
 c_{\pi(2i)}=+1
\right) & = \frac{1}{\sigma\sqrt{2\pi}}
            \mathrm{e}^{-(r_{2,\, i}-A)^2/(2\sigma^2)}
\end{align}
%*******************************************************************************
where it is assumed that bit ``0'' maps to $A$ and bit ``1'' maps to $-A$ and
%*******************************************************************************
\begin{align}
\label{Eq:SU_MMIMO_SCTC_Eq21}
r_{1,\, i} & = \pm A + w_{1,\, i}                     \nonumber  \\
r_{2,\, i} & = \pm A + w_{2,\, i}
\end{align}
%*******************************************************************************
where $w_{1,\, i},\, w_{2,\, i}$ denote real-valued samples of zero-mean
additive white Gaussian noise (AWGN) with variance $\sigma^2$. Similarly
we have
%*******************************************************************************
\begin{align}
\label{Eq:SU_MMIMO_SCTC_Eq22}
 E
\left(
 b_{2i}=-1
\right) & = \frac{1}{\sigma\sqrt{2\pi}}
            \mathrm{e}^{-(r_{1,\, i}+A)^2/(2\sigma^2)}     \nonumber  \\
 E
\left(
 c_{\pi(2i)}=-1
\right) & = \frac{1}{\sigma\sqrt{2\pi}}
            \mathrm{e}^{-(r_{2,\, i}+A)^2/(2\sigma^2)}.
\end{align}
%*******************************************************************************
Clearly
%*******************************************************************************
\begin{align}
\label{Eq:SU_MMIMO_SCTC_Eq23}
\ln
\left(
\frac{
 E
\left(
 b_{2i}=+1
\right)}{E\left(b_{2i}=-1\right)}
\right)
        & = \frac{2A}{\sigma^2} r_{1,\, i}             \nonumber  \\
\ln
\left(
\frac{
 E
\left(
 c_{\pi(2i)}=+1
\right)}{E\left(c_{\pi(2i)}=-1\right)}
\right)
        & = \frac{2A}{\sigma^2} r_{2,\, i}.
\end{align}
%*******************************************************************************
From (\ref{Eq:SU_MMIMO_SCTC_Eq19}) and (\ref{Eq:SU_MMIMO_SCTC_Eq23}) we have
for $0\le i \le L_{d1}-1$
%*******************************************************************************
\begin{align}
\label{Eq:SU_MMIMO_SCTC_Eq24}
\ln
\left(
\frac{
 P
\left(
 a_i=+1
\right)}{P\left(a_i=-1\right)}
\right) & = \frac{2A}{\sigma^2}
            \left(
             r_{1,\, i} + r_{2,\, i}
            \right)                    \nonumber  \\
        &   \stackrel{\Delta}{=}
            \frac{2A}{\sigma^2}
             r_{3,\, i}.
\end{align}
%*******************************************************************************
Consider the average
%*******************************************************************************
\begin{align}
\label{Eq:SU_MMIMO_SCTC_Eq25}
\mathscr{Y}
  & = \frac{2A}{\sigma^2L_{d2}}
      \sum_{i=0}^{L_{d2}-1}
       a_i
       r_{3,\, i}                              \nonumber  \\
  & = \frac{4A^2}{\sigma^2} + \mathscr{Z}
\end{align}
%*******************************************************************************
where
%*******************************************************************************
\begin{align}
\label{Eq:SU_MMIMO_SCTC_Eq26}
\mathscr{Z}
& =   \frac{2A}{\sigma^2L_{d2}}
      \sum_{i=0}^{L_{d2}-1}
       a_i
      \left(
       w_{1,\, i} + w_{2,\, i}
      \right)                   \nonumber  \\
L_{d2}
& \le  L_{d1}.
\end{align}
%*******************************************************************************
Note that the average in
(\ref{Eq:SU_MMIMO_SCTC_Eq25}) is done over less than $L_{d1}$ terms to avoid
situations like
%*******************************************************************************
\begin{equation}
\label{Eq:SU_MMIMO_SCTC_Eq26_1}
 P
\left(
 a_i=\pm 1
\right) = \mbox{1 or 0}.
\end{equation}
%*******************************************************************************
In fact, only those time instants $i$ have been considered in the
summation of (\ref{Eq:SU_MMIMO_SCTC_Eq25}) for which
%*******************************************************************************
\begin{equation}
\label{Eq:SU_MMIMO_SCTC_Eq26_2}
 P
\left(
 a_i=\pm 1
\right)  > \mathrm{e}^{-500}.
\end{equation}
%*******************************************************************************
Now
%*******************************************************************************
\begin{align}
\label{Eq:SU_MMIMO_SCTC_Eq27}
 E
\left[
\mathscr{Z}
\right]
   & =  0                               \nonumber  \\
 E
\left[
\mathscr{Z}^2
\right]
   & = \frac{4A^2}{\sigma^4L_{d2}^2}
       \sum_{i=0}^{L_{d2}-1}
        2
       \sigma^2                         \nonumber  \\
   & = \frac{8A^2}{\sigma^2L_{d2}}
\end{align}
%*******************************************************************************
where we have used the fact that $w_{1,\, i},\, w_{2,\, i}$ are independent.
Now, we know that the probability of error for the BPSK signal in
(\ref{Eq:SU_MMIMO_SCTC_Eq24}), that is
%*******************************************************************************
\begin{equation}
\label{Eq:SU_MMIMO_SCTC_Eq28}
r_{3,\, i} = r_{1,\, i}+r_{2,\, i} = \pm 2A + w_{1,\, i} + w_{2,\, i}
\end{equation}
%*******************************************************************************
is equal to \cite{Vasu_Book10}
%*******************************************************************************
\begin{equation}
\label{Eq:SU_MMIMO_SCTC_Eq29}
P(e) = \frac{1}{2}
       \mbox{erfc}
       \left(
       \sqrt{
       \frac{A^2}{\sigma^2}}
       \right).
\end{equation}
%*******************************************************************************
Therefore from (\ref{Eq:SU_MMIMO_SCTC_Eq25}), (\ref{Eq:SU_MMIMO_SCTC_Eq27}) and
(\ref{Eq:SU_MMIMO_SCTC_Eq29}) we have
%*******************************************************************************
\begin{equation}
\label{Eq:SU_MMIMO_SCTC_Eq30}
P_f(e)
       \approx
       \frac{1}{2}
       \mbox{erfc}
       \left(
       \sqrt{
       \frac{\left|\mathscr{Y}\right|}{4}}
       \right)
\end{equation}
%*******************************************************************************
where $P_f(e)$ denotes the probability of bit error for frame ``$f$'' and
%*******************************************************************************
\begin{equation}
\label{Eq:SU_MMIMO_SCTC_Eq31}
 E
\left[
\mathscr{Z}^2
\right] \rightarrow 0 \qquad \mbox{for $L_{d2}\gg 1$.}
\end{equation}
%*******************************************************************************
Observe that it is necessary to take the absolute value of $\mathscr{Y}$ in
(\ref{Eq:SU_MMIMO_SCTC_Eq30}) since there is a possibility that it can be
negative.
The average probability of bit error over $F$ frames is given by
%*******************************************************************************
\begin{equation}
\label{Eq:SU_MMIMO_SCTC_Eq32}
P(e) = \frac{1}{F}
       \sum_{f=0}^{F-1}
        P_f(e).
\end{equation}
%*******************************************************************************
In the next section we present computer simulation results
for SU-MMIMO using SCTC in uncorrelated channel.
%*******************************************************************************
\subsection{Simulation Results}
\label{SSec:SCTC_Uncorr_Results}
%*******************************************************************************
\begin{table}[tbhp]
\centering
%\begin{center}
\caption{Simulation parameters for results in
         Figures~\ref{Fig:Ber_NUM_ANT1024_Uncorr_DMT55i} --
                 \ref{Fig:Ber_NUM_ANT2_Uncorr_DMT55i}}
\input{sim_param_sctc_dmt55i.pstex_t}
\label{Tbl:Sim_Param_SCTC_DMT55i}
%\end{center}
\end{table}
%*******************************************************************************
%*******************************************************************************
\begin{figure*}[tbhp]
\centering
%\begin{center}
\input{ber_num_ant1024_uncorr_dmt55i.pstex_t}
\caption{Simulation results for $N_{\mathrm{tot}}=1024$.}
\label{Fig:Ber_NUM_ANT1024_Uncorr_DMT55i}
%\end{center}
\end{figure*}
%*******************************************************************************
%*******************************************************************************
\begin{figure*}[tbhp]
\centering
%\begin{center}
\input{ber_num_ant32_uncorr_dmt55i.pstex_t}
\caption{Simulation results for $N_{\mathrm{tot}}=32$.}
\label{Fig:Ber_NUM_ANT32_Uncorr_DMT55i}
%\end{center}
\end{figure*}
%*******************************************************************************
%*******************************************************************************
\begin{figure}[tbhp]
\centering
%\begin{center}
\input{ber_num_ant2_uncorr_dmt55i.pstex_t}
\caption{Simulation results for $N_{\mathrm{tot}}=2$, $N_t=1$.}
\label{Fig:Ber_NUM_ANT2_Uncorr_DMT55i}
%\end{center}
\end{figure}
%*******************************************************************************
%*******************************************************************************
%\begin{figure}[tbhp]
%\centering
%\begin{center}
%\input{ber_num_ant1024_uncorr_dmt55i_min_fr50000.pstex_t}
%\caption{Simulation results for $N_{\mathrm{tot}}=1024$, $N_t=512$,
%         $L_{d1}=50176$.}
%\label{Fig:Ber_NUM_ANT2_Uncorr_DMT55i_Min_Fr50000}
%\end{center}
%\end{figure}
%*******************************************************************************
The simulation parameters are given in Table~\ref{Tbl:Sim_Param_SCTC_DMT55i}.
We can make the following observations from
Figures~\ref{Fig:Ber_NUM_ANT1024_Uncorr_DMT55i} --
        \ref{Fig:Ber_NUM_ANT2_Uncorr_DMT55i} \cite{KV_Oct_2021}:
%*******************************************************************************
\begin{itemize}
    \item The theoretical prediction of BER closely matches with simulations.
    \item For  $N_{\mathrm{tot}}=32,\, 1024$, the BER is quite insensitive to
          wide variations in the total number
          of antennas $N_{\mathrm{tot}}$, transmit antennas $N_t$ and
          retransmissions $N_{rt}$.
    \item For $N_{\mathrm{tot}}=2$, the BER improves significantly with
          increasing retransmissions.
\end{itemize}
%*******************************************************************************
In Figure~\ref{Fig:Ber_NUM_ANT1024_Uncorr_DMT55i}(c) we observe that there
is more than 1 dB improvement in SINR compared to
Figures~\ref{Fig:Ber_NUM_ANT1024_Uncorr_DMT55i}(a, b),
\ref{Fig:Ber_NUM_ANT32_Uncorr_DMT55i} and
\ref{Fig:Ber_NUM_ANT2_Uncorr_DMT55i}. However, large values of $L_{d1}$ may
introduce more latency which is contrary to the requirements of 5G and beyond.
In the next section we present SU-MMIMO using PCTC in correlated channel.
%*******************************************************************************
\section{SU-MMIMO using PCTC in Correlated Channel}
\label{Sec:SU_MMIMO_PCTC_Corr_Chan}
\subsection{System Model}
\label{SSec:SU_MMIMO_PCTC_Corr_Chan_Sys_Model}
The block diagram of the system is identical to Figure~2 in \cite{KV_Oct_2021}
and the received signal is given by (\ref{Eq:SU_MMIMO_SCTC_Eq3}).
Note that in (\ref{Eq:SU_MMIMO_SCTC_Eq3}), the channel autocorrelation matrix is given by
%----------------------------------------------------------------------------------------
\begin{align}
\label{Eq:EE677A_Asgn1_Eq1}
 \mathbf{R}_{\tilde{\mathbf{H}}\tilde{\mathbf{H}}}
& = \frac{1}{2}
     E
    \left[
    \tilde{\mathbf{H}}_k^H\tilde{\mathbf{H}}_k
    \right]                                                   \nonumber  \\
& =  N_r
    \mathbf{I}_{N_t}
\end{align}
%----------------------------------------------------------------------------------------
where the superscript ``$H$'' denotes Hermitian and $\mathbf{I}_{N_t}$ denotes the
$N_t\times N_t$ identity matrix. In this section, we investigate the situation where
$\mathbf{R}_{\tilde{\mathbf{H}}\tilde{\mathbf{H}}}$ is not an identity matrix, but is
a valid autocorrelation matrix \cite{Vasu_Book10}.
As mentioned in \cite{KV_Oct_2021}, the elements of $\tilde{\mathbf{H}}_k$ -- given by
$\tilde{H}_{k,\, i,\, j}$ for the $k^{th}$ re-transmission, $i^{th}$ row, $j^{th}$
column of $\tilde{\mathbf{H}}_k$ -- are zero-mean, complex Gaussian random variables with
variance per dimension equal to $\sigma^2_H$. The in-phase and quadrature components of
$\tilde{H}_{k,\, i,\, j}$ -- denoted by $H_{k,\, i,\, j,\, I}$ and
$H_{k,\, i,\, j,\, Q}$ respectively -- are statistically independent. Moreover,
we assume that the rows of $\tilde{\mathbf{H}}_k$ are statistically independent.
Following the procedure in \cite{KV_Oct_2021} for the case without precoding, we
now find the expression for the average SINR per bit before and after averaging over
re-transmissions ($k$). All symbols and notations have the usual meaning, as
given in \cite{KV_Oct_2021}.
%----------------------------------------------------------------------------------------
\subsection{SINR Analysis}
\label{SSec:SU_MMIMO_PCTC_Corr_Chan_SINR_Ana}
The $i^{th}$ element of $\tilde{\mathbf{H}}_k^H\tilde{\mathbf{R}}_k$ is given by
(25) of \cite{KV_Oct_2021} which is repeated here for convenience
%----------------------------------------------------------------------------------------
\begin{equation}
\label{Eq:EE677A_Asgn1_Sol_Eq1}
\tilde{Y}_{k,\, i} = \tilde{F}_{k,\, i,\, i} S_i +
                     \tilde{I}_{k,\, i} +
                     \tilde{V}_{k,\, i}
                     \quad \mbox{for $1\leq i\leq N_t$}
\end{equation}
%----------------------------------------------------------------------------------------
where
%*******************************************************************************
\begin{align}
\label{Eq:EE677A_Asgn1_Sol_Eq2}
\tilde{V}_{k,\, i}      & = \sum_{j=1}^{N_r}
                            \tilde{H}_{k,\, j,\, i}^*
                            \tilde W_{k,\, j}                 \nonumber  \\ 
\tilde{I}_{k,\, i}      & = \sum_{\substack{j=1\\j\neq i}}^{N_t}
                            \tilde{F}_{k,\, i,\, j} S_j       \nonumber  \\
\tilde{F}_{k,\, i,\, j} & = \sum_{l=1}^{N_r}
                            \tilde{H}_{k,\, l,\, i}^*
                            \tilde{H}_{k,\, l,\, j}.
\end{align}
%*******************************************************************************
We have
%*******************************************************************************
\begin{align}
\label{Eq:EE677A_Asgn1_Sol_Eq3}
 E
\left[
\tilde{F}_{k,\, i,\, i}^2
\right] & =  E
            \left[
            \sum_{l=1}^{N_r}
            \left|
            \tilde{H}_{k,\, l,\, i}
            \right|^2
            \sum_{m=1}^{N_r}
            \left|
            \tilde{H}_{k,\, m,\, i}
            \right|^2
            \right]                         \nonumber  \\
        & =  E
            \left[
            \sum_{l=1}^{N_r}
            \left(
             H_{k,\, l,\, i,\, I}^2 +
             H_{k,\, l,\, i,\, Q}^2
            \right)
            \right.                         \nonumber  \\
        &   \left.
            \qquad
            \sum_{m=1}^{N_r}
            \left(
             H_{k,\, m,\, i,\, I}^2 +
             H_{k,\, m,\, i,\, Q}^2
            \right)
            \right]                         \nonumber  \\
        & =  4\sigma_H^4 N_r(N_r+1)
\end{align}
%*******************************************************************************
which is identical to (27) in \cite{KV_Oct_2021} and we have used the following
properties         
%----------------------------------------------------------------------------------------
\begin{enumerate}
    \item The in-phase and quadrature components of $\tilde{H}_{k,\, i,\, j}$ are
          independent.
    \item The rows of $\tilde{\mathbf{H}}_k$ are independent.
    \item For zero-mean, real-valued Gaussian random variable $X$ with variance equal to
          $\sigma^2_X$, $E\left[X^4\right]=3\sigma_X^4$.
\end{enumerate}
%----------------------------------------------------------------------------------------
The interference power is
%*******************************************************************************
\begin{align}
\label{Eq:EE677A_Asgn1_Sol_Eq4}
 E
\left[
\left|
\tilde{I}_{k,\, i}
\right|^2
\right] & =  E
            \left[
            \sum_{\substack{j=1\\j\neq i}}^{N_t}
            \tilde{F}_{k,\, i,\, j} S_j
            \sum_{\substack{l=1\\l\neq i}}^{N_t}
            \tilde{F}_{k,\, i,\, l}^* S_l^*
            \right]                                          \nonumber  \\
        & = \sum_{\substack{j=1\\j\neq i}}^{N_t}
            \sum_{\substack{l=1\\l\neq i}}^{N_t}
             E
            \left[
            \tilde{F}_{k,\, i,\, j}
            \tilde{F}_{k,\, i,\, l}^*
            \right]
             E
            \left[
             S_j S_l^*
            \right]                                          \nonumber  \\
        & =  P_{\mathrm{av}}
            \sum_{\substack{j=1\\j\neq i}}^{N_t}
             E
            \left[
            \left| 
            \tilde{F}_{k,\, i,\, j}
            \right|^2
            \right].
\end{align}
%*******************************************************************************
where we have used (9) in \cite{KV_Oct_2021}. Similarly the noise power is
%*******************************************************************************
\begin{align}
\label{Eq:EE677A_Asgn1_Sol_Eq5}
 E
\left[
\left|
\tilde{V}_{k,\, i}
\right|^2
\right] & =  E
            \left[
            \sum_{j=1}^{N_r}
            \tilde{H}_{k,\, j,\, i}^* \tilde{W}_{k,\, j}
            \sum_{m=1}^{N_r}
            \tilde{H}_{k,\, m,\, i} \tilde{W}_{k,\, m}^*
            \right]                                           \nonumber  \\
        & = \sum_{j=1}^{N_r}
            \sum_{m=1}^{N_r}
             E
            \left[ 
            \tilde{H}_{k,\, j,\, i}^*
            \tilde{H}_{k,\, m,\, i}
            \right]
             E
            \left[
            \tilde{W}_{k,\, m}^* \tilde{W}_{k,\, j}
            \right]                                           \nonumber  \\
            & = \sum_{j=1}^{N_r}
            \sum_{m=1}^{N_r}
             2
            \sigma^2_H
            \delta_K(j-m)
             2
            \sigma^2_W(j-m)                                   \nonumber  \\
        & =  4 N_r
            \sigma^2_H
            \sigma^2_W
\end{align}
%*******************************************************************************
which is identical to (29) in \cite{KV_Oct_2021} and we have used the following
properties:
%*******************************************************************************
\begin{enumerate}
    \item Rows of $\tilde{\mathbf{H}}_k$ are independent.
    \item Sifting property of the Kronecker delta function.
    \item Noise and channel coefficients are independent.
\end{enumerate}
%*******************************************************************************
Now in (\ref{Eq:EE677A_Asgn1_Sol_Eq4})
%*******************************************************************************
\begin{align}
\label{Eq:EE677A_Asgn1_Sol_Eq6}
 E
\left[
\left|
\tilde{F}_{k,\, i,\, j}
\right|^2
\right] & =  E
            \left[
            \sum_{l=1}^{N_r}
            \tilde{H}_{k,\, l,\, i}^*
            \tilde{H}_{k,\, l,\, j}
            \sum_{m=1}^{N_r}
            \tilde{H}_{k,\, m,\, i}
            \tilde{H}_{k,\, m,\, j}^*
            \right]                                            \nonumber  \\
        & = \sum_{l=1}^{N_r}
             E
            \Biggl[
            \tilde{H}_{k,\, l,\, i}^*
            \tilde{H}_{k,\, l,\, j}
            \Biggl(
            \tilde{H}_{k,\, l,\, i}
            \tilde{H}_{k,\, l,\, j}^*
            \Biggr.
            \Biggr.                                            \nonumber  \\
        &   \qquad
            \Biggl.
            \Biggl. +
            \sum_{\substack{m=1\\m\ne l}}^{N_r}
            \tilde{H}_{k,\, m,\, i}
            \tilde{H}_{k,\, m,\, j}^*
            \Biggr)
            \Biggr]                                            \nonumber  \\           
        & = \sum_{l=1}^{N_r}
             E
            \Biggl[
            \left|
            \tilde{H}_{k,\, l,\, i}
            \right|^2
            \left|
            \tilde{H}_{k,\, l,\, j}
            \right|^2
            \Biggr.                                            \nonumber  \\
        &   \qquad    +
            \Biggl.
            \Biggl(
            \sum_{\substack{m=1\\m\ne l}}^{N_r}
            \tilde{H}_{k,\, l,\, i}^*
            \tilde{H}_{k,\, l,\, j}
            \tilde{H}_{k,\, m,\, i}
            \tilde{H}_{k,\, m,\, j}^*
            \Biggr)
            \Biggr].
\end{align}
%*******************************************************************************       
Now the first summation in (\ref{Eq:EE677A_Asgn1_Sol_Eq6}) is equal to
%*******************************************************************************
\begin{align}
\label{Eq:EE677A_Asgn1_Sol_Eq7}
E_1 & =  E
        \left[
        \left|
        \tilde{H}_{k,\, l,\, i}
        \right|^2
        \left|
        \tilde{H}_{k,\, l,\, j}
        \right|^2
        \right]                                              \nonumber  \\
    & =  E
        \left[
        \left(
         H_{k,\, l,\, i,\, I}^2 +
         H_{k,\, l,\, i,\, Q}^2
        \right)
        \left(
         H_{k,\, l,\, j,\, I}^2 +
         H_{k,\, l,\, j,\, Q}^2
        \right)
        \right]                                              \nonumber  \\
    & =  4
        \sigma^4_H + 4 R_{\tilde{H}\tilde{H},\, j-i}^2
\end{align}
%*******************************************************************************
where we have used the property that for real-valued, zero-mean Gaussian
random variables $X_i$, $1\le i \le 4$ \cite{Papoulis91,Vasu_AC_PS}
%*******************************************************************************
\begin{equation}
\label{Eq:EE677A_Asgn1_Sol_Eq8}
 E
\left[
 X_1 X_2 X_3 X_4
\right] = C_{12}C_{34} + C_{13}C_{24} + C_{14}C_{23}
\end{equation}
%*******************************************************************************
where
%*******************************************************************************
\begin{equation}
\label{Eq:EE677A_Asgn1_Sol_Eq9}
 C_{ij} =
 E
\left[
 X_i X_j
\right] \qquad \mbox{for $1\le i,\, j\le 4$}
\end{equation}
%*******************************************************************************
and
%*******************************************************************************
\begin{align}
\label{Eq:EE677A_Asgn1_Sol_Eq10}
 R_{\tilde{H}\tilde{H},\, j-i}
& =
 E
\left[
 H_{k,\, l,\, i,\, I}
 H_{k,\, l,\, j,\, I}
\right]                                           \nonumber  \\
& =
 E
\left[
 H_{k,\, l,\, i,\, Q}
 H_{k,\, l,\, j,\, Q}
\right]                                           \nonumber  \\
& =
\frac{1}{2}
 E
\left[
\tilde{H}_{k,\, l,\, i}^*
\tilde{H}_{k,\, l,\, j}
\right]                                           \nonumber  \\
& =
 R_{\tilde{H}\tilde{H},\, i-j}
\end{align}
%*******************************************************************************
is the real-valued autocorrelation of $\tilde{H}_{k,\, m,\, n}$ and we have
made the assumption that the in-phase and quadrature components of
$\tilde{H}_{k,\, m,\, n}$ are independent. The second
summation in (\ref{Eq:EE677A_Asgn1_Sol_Eq6}) can be written as
%*******************************************************************************
\begin{align}
\label{Eq:EE677A_Asgn1_Sol_Eq11}
E_2
& =
\sum_{\substack{m=1\\m\ne l}}^{N_r}
 E
\left[
\tilde{H}_{k,\, l,\, i}^*
\tilde{H}_{k,\, l,\, j}
\tilde{H}_{k,\, m,\, i}
\tilde{H}_{k,\, m,\, j}^*
\right]                                                    \nonumber  \\
& =
\sum_{\substack{m=1\\m\ne l}}^{N_r}
 E
\left[
\tilde{H}_{k,\, l,\, i}^*
\tilde{H}_{k,\, l,\, j}
\right]
 E
\left[
\tilde{H}_{k,\, m,\, i}
\tilde{H}_{k,\, m,\, j}^*
\right]                                                    \nonumber  \\
& =
\sum_{\substack{m=1\\m\ne l}}^{N_r}
 4R_{\tilde{H}\tilde{H},\, j-i}^2                          \nonumber  \\
& =
 4(N_r-1)R_{\tilde{H}\tilde{H},\, j-i}^2
\end{align}
%*******************************************************************************
where we have used the property that the rows of $\tilde{\mathbf{H}}_k$ are
independent. Therefore (\ref{Eq:EE677A_Asgn1_Sol_Eq6}) becomes
%*******************************************************************************
\begin{align}
\label{Eq:EE677A_Asgn1_Sol_Eq12}
 E
\left[
\left|
\tilde{F}_{k,\, i,\, j}
\right|^2
\right] & =  N_r (E_1 +E_2)                           \nonumber  \\
        & =  4N_r
            \left[
            \sigma_H^4 + R_{\tilde{H}\tilde{H},\, j-i}^2 +
            \left(
             N_r-1
            \right)
             R_{\tilde{H}\tilde{H},\, j-i}^2
            \right]                                   \nonumber  \\
        & =  4N_r
            \left[
            \sigma_H^4 + N_r R_{\tilde{H}\tilde{H},\, j-i}^2
            \right].
\end{align}
%*******************************************************************************
The total power of interference plus noise is
%*******************************************************************************
\begin{align}
\label{Eq:EE677A_Asgn1_Sol_Eq13}
 E
\left[
\left|
\tilde{I}_{k,\, i} + \tilde{V}_{k,\, i}
\right|^2
\right] & =  E
            \left[
            \left|
            \tilde{I}_{k,\, i}
            \right|^2
            \right] +
             E
            \left[
            \left|
            \tilde{V}_{k,\, i}
            \right|^2
            \right]                                    \nonumber  \\
        & =  4 P_{\mathrm{av}}
             N_r
            \sum_{\substack{j=1\\j\ne i}}^{N_t}
            \left[
            \sigma^4_H + N_r R_{\tilde{H}\tilde{H},\, j-i}^2
            \right]                                    \nonumber  \\
        &   \qquad +
             4 N_r\sigma^2_H\sigma^2_W
\end{align}
%*******************************************************************************
where we have made the assumption that noise and symbols are independent.
The average SINR per bit for the $i^{th}$ transmit antenna is similar to (31)
of \cite{KV_Oct_2021} which is repeated here for convenience
%*******************************************************************************
\begin{equation}
\label{Eq:EE677A_Asgn1_Sol_Eq14}
\mathrm{SINR}_{\mathrm{av},\, b,\, i}
          = \frac{
             E
            \left[
            \left|
            \tilde{F}_{k,\, i,\, i} S_i
            \right|^2
            \right]
            \times 2N_{rt}}
            {
             E
            \left[
            \left|
            \tilde{I}_{k,\, i} + \tilde{V}_{k,\, i}
            \right|^2
            \right]
            }                       \qquad  \mbox{for $1\le i \le N_t$}
\end{equation}
%*******************************************************************************
into which (\ref{Eq:EE677A_Asgn1_Sol_Eq3}) and (\ref{Eq:EE677A_Asgn1_Sol_Eq13})
have to be substituted. The upper bound on the average SINR per bit for
the $i^{th}$ transmit antenna is obtained by
setting $\sigma^2_W=0$ in (\ref{Eq:EE677A_Asgn1_Sol_Eq13}),
(\ref{Eq:EE677A_Asgn1_Sol_Eq14}) and is given by, for $1\le i \le N_t$
%*******************************************************************************
\begin{equation}
\label{Eq:SU_MMIMO_SCTC_Eq33}
\mathrm{SINR}_{\mathrm{av},\, b,\, \mathrm{UB},\, i}
          = \frac{
            \sigma_H^4
            \left(
             1 + N_r
            \right)
            \times 2N_{rt}}
            {
            \sum_{\substack{j=1\\j\ne i}}^{N_t}
            \left[
            \sigma^4_H + N_r R_{\tilde{H}\tilde{H},\, j-i}^2
            \right]
            }.
\end{equation}
%*******************************************************************************
Observe that in contrast to (31) and (32) in \cite{KV_Oct_2021}, the average
SINR per bit and its upper bound depend on the transmit antenna. Let us now
compute the average SINR per bit after averaging over retransmissions. The
received signal after averaging over retransmissions is given by
(\ref{Eq:SU_MMIMO_SCTC_Eq4}) with (see also (20) of \cite{KV_Oct_2021})
%*******************************************************************************
\begin{align}
\label{Eq:SU_MMIMO_SCTC_Eq34}
F_i         & = \frac{1}{N_{rt}}
                \sum_{k=0}^{N_{rt}-1}
                \tilde{F}_{k,\, i,\, i}                        \nonumber  \\
\tilde{U}_i & = \frac{1}{N_{rt}}
                \sum_{k=0}^{N_{rt}-1}
                \left(
                \tilde{I}_{k,\, i} +
                \tilde{V}_{k,\, i}
                \right)                                        \nonumber  \\
            & = \frac{1}{N_{rt}}
                \sum_{k=0}^{N_{rt}-1}
                \tilde{U}_{k,\, i}'   \qquad \mbox{(say)}
\end{align}
%*******************************************************************************
where $\tilde{F}_{k,\, i,\, i}$, $\tilde{I}_{k,\, i}$ and $\tilde{V}_{k,\, i}$
are given in (\ref{Eq:EE677A_Asgn1_Sol_Eq1}). The power of the signal component
of (\ref{Eq:SU_MMIMO_SCTC_Eq4}) is
%*******************************************************************************
\begin{align}
\label{Eq:SU_MMIMO_SCTC_Eq35}
 E
\left[
\left|
 S_i
\right|^2
 F_i^2
\right] & =  P_{\mathrm{av}}
             E
            \left[
             F_i^2
            \right]                                     \nonumber  \\
        & = \frac{P_{\mathrm{av}}}{N_{rt}^2}
             E
            \left[
            \sum_{k=0}^{N_{rt}-1}
            \tilde{F}_{k,\, i,\, i}
            \sum_{l=0}^{N_{rt}-1}
            \tilde{F}_{l,\, i,\, i}
            \right]                                     \nonumber  \\
        & = \frac{P_{\mathrm{av}}}{N_{rt}^2}
            \sum_{k=0}^{N_{rt}-1}
            \Biggl[
            \sum_{\substack{l=0\\l\ne k}}^{N_{rt}-1}
             E[\tilde{F}_{k,\, i,\, i}]
             E[\tilde{F}_{l,\, i,\, i}]
            \Biggr.                                     \nonumber  \\
        &   \qquad +
            \Biggl.
             E
            \left[
            \left|
            \tilde{F}_{k,\, i,\, i}
            \right|^2
            \right]
            \Biggr]
\end{align}
%*******************************************************************************
where we have used the fact that the channel is independent across
retransmissions, therefore
%*******************************************************************************
\begin{equation}
\label{Eq:SU_MMIMO_SCTC_Eq36}
E[\tilde{F}_{k,\, i,\, i} \tilde{F}_{l,\, i,\, i}] =
E[\tilde{F}_{k,\, i,\, i}]
E[\tilde{F}_{l,\, i,\, i}] \qquad \mbox{for $k\ne l$}.
\end{equation}
%*******************************************************************************
Now
%*******************************************************************************
\begin{align}
\label{Eq:SU_MMIMO_SCTC_Eq37}
E[\tilde{F}_{k,\, i,\, i}]
                   & =  E
                       \left[
                       \sum_{l=1}^{N_r}
                       \left|
                       \tilde{H}_{k,\, l,\, i}
                       \right|^2
                       \right]                          \nonumber  \\
                   & =  2 N_r\sigma_H^2.
\end{align}
%*******************************************************************************
Substituting (\ref{Eq:EE677A_Asgn1_Sol_Eq3}) and (\ref{Eq:SU_MMIMO_SCTC_Eq37})
in (\ref{Eq:SU_MMIMO_SCTC_Eq35}) we get
%*******************************************************************************
\begin{equation}
\label{Eq:SU_MMIMO_SCTC_Eq38}
 E
\left[
\left|
 S_i
\right|^2
 F_i^2
\right] = \frac{4N_r P_{\mathrm{av}} \sigma_H^4}{N_{rt}}
          \left(
           1 + N_r N_{rt}
          \right).
\end{equation}
%*******************************************************************************
The power of the interference component in (\ref{Eq:SU_MMIMO_SCTC_Eq4}) and
(\ref{Eq:SU_MMIMO_SCTC_Eq34}) is
%*******************************************************************************
\begin{align}
\label{Eq:SU_MMIMO_SCTC_Eq39}
 E
\left[
\left|
\tilde{U}_i
\right|^2
\right] & =
          \frac{1}{N_{rt}^2}
           E
          \left[
          \sum_{k=0}^{N_{rt}-1}
          \left(
          \tilde{I}_{k,\, i} +
          \tilde{V}_{k,\, i}
          \right)
          \sum_{l=0}^{N_{rt}-1}
          \left(
          \tilde{I}_{l,\, i}^* +
          \tilde{V}_{l,\, i}^*
          \right)
          \right]                                  \nonumber  \\
        & =
          \frac{1}{N_{rt}^2}
          \sum_{k=0}^{N_{rt}-1}
          \sum_{l=0}^{N_{rt}-1}
           E
          \left[
          \tilde{I}_{k,\, i}
          \tilde{I}_{l,\, i}^*
          \right] +
           E
          \left[
          \tilde{V}_{k,\, i}
          \tilde{V}_{l,\, i}^*
          \right]
\end{align}
%*******************************************************************************
where we have used the following properties from (\ref{Eq:EE677A_Asgn1_Sol_Eq2})
%*******************************************************************************
\begin{align}
\label{Eq:SU_MMIMO_SCTC_Eq40}
 E
\left[
\tilde{I}_{k,\, i}
\right] & =  E
            \left[
            \tilde{V}_{k,\, i}
            \right]                       \nonumber  \\
        & =  0                            \nonumber  \\
 E
\left[
\tilde{I}_{k,\, i}
\tilde{V}_{l,\, i}^*
\right] & =  E
            \left[
            \tilde{V}_{k,\, i}
            \tilde{I}_{l,\, i}^*
            \right]                       \nonumber  \\
        & =  0    \qquad   \mbox{for all $k,\, l$}
\end{align}
%*******************************************************************************
since $S_j$ and $\tilde{W}_{k,\, j}$ are mutually independent with zero-mean.
Now
%*******************************************************************************
\begin{align}
\label{Eq:SU_MMIMO_SCTC_Eq41}
 E
\left[
\tilde{I}_{k,\, i}
\tilde{I}_{l,\, i}^*
\right] & =  E
            \left[
            \sum_{\substack{j=1\\j\ne i}}^{N_t}
            \tilde{F}_{k,\, i,\, j} S_j
            \sum_{\substack{n=1\\n\ne i}}^{N_t}
            \tilde{F}_{l,\, i,\, n}^* S_n^*
            \right]                                 \nonumber  \\
        & = \sum_{\substack{j=1\\j\ne i}}^{N_t}
            \sum_{\substack{n=1\\n\ne i}}^{N_t}
             E
            \left[
            \tilde{F}_{k,\, i,\, j}
            \tilde{F}_{l,\, i,\, n}^*
            \right]
             E
            \left[
             S_j S_n^*
            \right]                                 \nonumber  \\
        & = \sum_{\substack{j=1\\j\ne i}}^{N_t}
            \sum_{\substack{n=1\\n\ne i}}^{N_t}
             E
            \left[
            \tilde{F}_{k,\, i,\, j}
            \tilde{F}_{l,\, i,\, n}^*
            \right]
             P_{\mathrm{av}}
            \delta_K(j-n)                           \nonumber  \\
        & =  P_{\mathrm{av}}
            \sum_{\substack{j=1\\j\ne i}}^{N_t}
             E
            \left[
            \tilde{F}_{k,\, i,\, j}
            \tilde{F}_{l,\, i,\, j}^*
            \right]
\end{align}
%*******************************************************************************
where we have used the property that the symbols are uncorrelated and
$\delta_K(\cdot)$ is the Kronecker delta function \cite{Vasu_Book10}. When
$k=l$, (\ref{Eq:SU_MMIMO_SCTC_Eq41}) is given by (\ref{Eq:EE677A_Asgn1_Sol_Eq4})
and (\ref{Eq:EE677A_Asgn1_Sol_Eq12}). When $k\ne l$,
(\ref{Eq:SU_MMIMO_SCTC_Eq41}) is given by
%*******************************************************************************
\begin{align}
\label{Eq:SU_MMIMO_SCTC_Eq42}
 E
\left[
\tilde{I}_{k,\, i}
\tilde{I}_{l,\, i}^*
\right] & =  P_{\mathrm{av}}
            \sum_{\substack{j=1\\j\ne i}}^{N_t}
             E
            \left[
            \tilde{F}_{k,\, i,\, j}
            \right]
             E
            \left[
            \tilde{F}_{l,\, i,\, j}^*
            \right]                                     \nonumber  \\
        & =  P_{\mathrm{av}}
            \sum_{\substack{j=1\\j\ne i}}^{N_t}
             4
             N_r^2
             R_{\tilde{H}\tilde{H},\, j-i}^2
\end{align}
%*******************************************************************************
where we have used (\ref{Eq:EE677A_Asgn1_Sol_Eq2}) and
(\ref{Eq:EE677A_Asgn1_Sol_Eq10}). Similarly, we have
%*******************************************************************************
\begin{equation}
\label{Eq:SU_MMIMO_SCTC_Eq43}
 E
\left[
\tilde{V}_{k,\, i}
\tilde{V}_{l,\, i}^*
\right] = 4 N_r \sigma_H^2 \sigma^2_W \delta_K(k-l)
\end{equation}
%*******************************************************************************
where we have used (\ref{Eq:EE677A_Asgn1_Sol_Eq5}). Substituting
(\ref{Eq:EE677A_Asgn1_Sol_Eq4}), (\ref{Eq:EE677A_Asgn1_Sol_Eq12}),
(\ref{Eq:SU_MMIMO_SCTC_Eq42}) and (\ref{Eq:SU_MMIMO_SCTC_Eq43}) in
(\ref{Eq:SU_MMIMO_SCTC_Eq39}) we get
%*******************************************************************************
\begin{align}
\label{Eq:SU_MMIMO_SCTC_Eq44}
 E
\left[
\left|
\tilde{U}_i
\right|^2
\right] & = \frac{1}{N_{rt}^2}
            \left[
             4 P_{\mathrm{av}} N_r N_{rt}
            \sum_{\substack{j=1\\j\ne i}}^{N_t}
            \left(
            \sigma_H^4 + N_r R_{\tilde{H}\tilde{H},\, j-i}^2
            \right)
            \right.                                       \nonumber  \\
        &   \qquad +
            \left.
             4 P_{\mathrm{av}} N_r^2 N_{rt} (N_{rt}-1)
            \sum_{\substack{j=1\\j\ne i}}^{N_t}
             R_{\tilde{H}\tilde{H},\, j-i}^2
            \right]                                       \nonumber  \\
        &   \qquad +
            \frac{4N_r}{N_{rt}}
            \sigma_H^2
            \sigma^2_W                                    \nonumber  \\
        & = \frac{1}{N_{rt}}
            \left[
             4 P_{\mathrm{av}} N_r
            \sum_{\substack{j=1\\j\ne i}}^{N_t}
            \left(
            \sigma_H^4 + N_r R_{\tilde{H}\tilde{H},\, j-i}^2
            \right)
            \right.                                       \nonumber  \\
        &   \qquad +
            \left.
             4 P_{\mathrm{av}} N_r^2 (N_{rt}-1)
            \sum_{\substack{j=1\\j\ne i}}^{N_t}
             R_{\tilde{H}\tilde{H},\, j-i}^2
            \right]                                       \nonumber  \\
        &   \qquad +
            \frac{4N_r}{N_{rt}}
            \sigma_H^2
            \sigma^2_W.
\end{align}
%*******************************************************************************
The average SINR per bit for the $i^{th}$ transmit antenna, after averaging
over retransmissions (also referred to as ``combining'' \cite{KV_Oct_2021})
is given by
%*******************************************************************************
\begin{equation}
\label{Eq:SU_MMIMO_SCTC_Eq45}
\mathrm{SINR}_{\mathrm{av},\, b,\, C,\, i} =
\frac{2P_{\mathrm{av}}E\left[F_i^2\right]}
     {E\left[\left|\tilde{U}_i\right|^2\right]}
\end{equation}
%*******************************************************************************
into which (\ref{Eq:SU_MMIMO_SCTC_Eq38}) and (\ref{Eq:SU_MMIMO_SCTC_Eq44})
have to be substituted.
The upper bound on the average SINR per bit after ``combining'' for
the $i^{th}$ transmit antenna is given by
%*******************************************************************************
\begin{equation}
\label{Eq:SU_MMIMO_SCTC_Eq45_1}
\mathrm{SINR}_{\mathrm{av},\, b,\, C,\, \mathrm{UB},\, i} =
\left.
\mathrm{SINR}_{\mathrm{av},\, b,\, C,\, i}
\right|_{\sigma_W^2=0}.
\end{equation}
%*******************************************************************************
%*******************************************************************************
\begin{figure*}[tbhp]
\centering
%\begin{center}
\input{sinr_ub_ba_numant1024.pstex_t}
\caption{Plot of $\mathrm{SINR}_{\mathrm{av},\, b,\,\mathrm{UB},\, i}$
         for $N_{\mathrm{tot}}=1024$, $N_{rt}=2$. (a) Back view. (b) Sideview.
         (c) Front view.}
\label{Fig:SINR_UB_BA_Numant1024}
%\end{center}
\end{figure*}
%*******************************************************************************
%*******************************************************************************
\begin{figure*}[tbhp]
\centering
%\begin{center}
\input{sinr_ub_aa_numant1024.pstex_t}
\caption{Plot of $\mathrm{SINR}_{\mathrm{av},\, b,\, C,\,\mathrm{UB},\, i}$
         for $N_{\mathrm{tot}}=1024$, $N_{rt}=2$. (a) Back view. (b) Side view.
         (c) Front view.}
\label{Fig:SINR_UB_AA_Numant1024}
%\end{center}
\end{figure*}
%*******************************************************************************
The plots of the average SINR per bit for the $i^{th}$ transmit antenna
before and after ``combining'' are shown in
Figures~\ref{Fig:SINR_UB_BA_Numant1024} and
\ref{Fig:SINR_UB_AA_Numant1024} respectively for $N_{\mathrm{tot}}=1024$ and
$N_{rt}=2$. The channel correlation is given by
%*******************************************************************************
\begin{equation}
\label{Eq:SU_MMIMO_SCTC_Eq46}
R_{\tilde{H}\tilde{H},\, j-i} = 0.9^{|j-i|} \sigma^2_H
\end{equation}
%*******************************************************************************
in (\ref{Eq:EE677A_Asgn1_Sol_Eq10}), which is obtained by passing samples of
white Gaussian noise through a unit-energy, first-order infinite impulse
response (IIR) lowpass filter with $a=-0.9$ (see (30) of \cite{Vasu_SPJ04}).

We observe in Figures~\ref{Fig:SINR_UB_BA_Numant1024} and
\ref{Fig:SINR_UB_AA_Numant1024} that
%*******************************************************************************
\begin{enumerate}
    \item The upper bound on the average SINR per bit decreases rapidly
          with increasing transmit
          antennas $N_t$ and falls below 0 dB for $N_t>5$
          (see Figures~\ref{Fig:SINR_UB_BA_Numant1024}(b) and
          \ref{Fig:SINR_UB_AA_Numant1024}(b)). Since the spectral
          efficiency of the system is $N_t/(2N_{rt})$ bits/sec/Hz (see (33) of
          \cite{KV_Oct_2021}), the system
          would be of no practical use, since the BER would be close to 0.5
          for $N_t>5$.
    \item The upper bound on the average SINR per bit after ``combining''
          is \textit{less} than
          that before ``combining''. Therefore retransmissions are
          ineffective.
\end{enumerate}
%*******************************************************************************
In view of the above observation, it becomes necessary to design a better
receiver using precoding. This is presented in the next section.
%*******************************************************************************
\subsection{Precoding}
\label{SSec:Precoding}
Similar to (\ref{Eq:SU_MMIMO_SCTC_Eq3}), consider the modified received
signal given by
%*******************************************************************************
\begin{equation}
\label{Eq:SU_MMIMO_PCTC_Pre_Eq1}
\tilde{\mathbf{R}}_k = \tilde{\mathbf{H}}_k
                       \tilde{\mathbf{B}}
                       \mathbf{S} +
                       \tilde{\mathbf{W}}_k
\end{equation}
%*******************************************************************************
where
%*******************************************************************************
\begin{align}
\label{Eq:SU_MMIMO_PCTC_Pre_Eq2}
\tilde{\mathbf{B}}
& = \left[
    \begin{array}{cccc}
     1                         &  0     & \cdots                 &  0\\
    \tilde{a}_{1,\, 1}         &  1     & \cdots                 &  0\\
    \vdots                     & \cdots & \cdots                 & \vdots\\
    \tilde{a}_{N_t-1,\, N_t-1} & \cdots & \tilde{a}_{N_t-1,\, 1} & 1
    \end{array}
    \right]^T       \nonumber  \\
&   \stackrel{\Delta}{=}
    \tilde{\mathbf{A}}^T
\end{align}
%*******************************************************************************
where $(\cdot)^T$ denotes transpose.
In (\ref{Eq:SU_MMIMO_PCTC_Pre_Eq2}), $\tilde{\mathbf{A}}$ is an $N_t\times N_t$
lower triangular
matrix with diagonal elements equal to unity and $\tilde{a}_{i,\, j}$ denotes
the $j^{th}$ coefficient of the optimum $i^{th}$-order forward prediction filter
\cite{Vasu_Book10} and $\tilde{\mathbf{B}}$ is the precoding matrix. Let
%*******************************************************************************
\begin{align}
\label{Eq:SU_MMIMO_PCTC_Pre_Eq3}
\tilde{\mathbf{Y}}_k & = \tilde{\mathbf{B}}^H
                         \tilde{\mathbf{H}}_k^H
                         \tilde{\mathbf{R}}_k          \nonumber  \\
                     & = \tilde{\mathbf{B}}^H
                         \tilde{\mathbf{H}}_k^H
                         \tilde{\mathbf{H}}_k
                         \tilde{\mathbf{B}}
                         \mathbf{S} +
                         \tilde{\mathbf{B}}^H
                         \tilde{\mathbf{H}}_k^H
                         \tilde{\mathbf{W}}_k.
\end{align}
%*******************************************************************************
Define
%*******************************************************************************
\begin{align}
\label{Eq:SU_MMIMO_PCTC_Pre_Eq4}
\tilde{\mathbf{Z}}_k
& = \tilde{\mathbf{H}}_k
    \tilde{\mathbf{B}}               \nonumber  \\
& = \left[
    \begin{array}{ccc}
    \tilde{Z}_{k,\,1,\, 1}    & \cdots & \tilde{Z}_{k,\, 1,\, N_t} \\
    \vdots                    & \cdots & \vdots \\
    \tilde{Z}_{k,\, N_r,\, 1} & \cdots & \tilde{Z}_{k,\, N_r,\, N_t}
    \end{array}
    \right].
\end{align}
%*******************************************************************************
Now \cite{Vasu_Book10}
%*******************************************************************************
\begin{align}
\label{Eq:SU_MMIMO_PCTC_Pre_Eq5}
\frac{1}{2}
 E
\left[
\tilde{\mathbf{Z}}_k^H
\tilde{\mathbf{Z}}_k
\right]
& = N_r
\left[
\begin{array}{cccc}
\sigma_{Z,\, 1}^2  &  0                & \cdots &  0\\
 0                 & \sigma_{Z,\, 2}^2 & \cdots &  0\\
\vdots             & \cdots            & \cdots & \vdots \\
 0                 & \cdots            &  0     & \sigma_{Z,\, N_t}^2
\end{array}
\right]                 \nonumber  \\
& \stackrel{\Delta}{=}
\tilde{\mathbf{R}}_{\tilde{\mathbf{Z}}\tilde{\mathbf{Z}}}
\end{align}
%*******************************************************************************
is an $N_t\times N_t$ diagonal matrix and $\sigma_{Z,\, i}^2$ denotes the
variance per dimension of the optimum $(i-1)^{th}$-order forward prediction
filter. Note that \cite{Vasu_Book10}
%*******************************************************************************
\begin{align}
\label{Eq:SU_MMIMO_PCTC_Pre_Eq6}
\sigma_{Z,\, 1}^2 & = \sigma^2_H                   \nonumber  \\
\sigma_{Z,\, i}^2 & \ge \sigma_{Z,\, j}^2 \qquad \mbox{for $i< j$.}
\end{align}
%*******************************************************************************
Let
%*******************************************************************************
\begin{align}
\label{Eq:SU_MMIMO_PCTC_Pre_Eq7}
\tilde{\mathbf{V}}_k & = \tilde{\mathbf{Z}}_k^H
                         \tilde{\mathbf{W}}_k             \nonumber  \\
                     & = \left[
                         \begin{array}{ccc}
                         \tilde{V}_{k,\, 1} & \cdots & \tilde{V}_{k,\, N_t}
                         \end{array}
                         \right]^T
\end{align}
%*******************************************************************************
which is an $N_t\times 1$ vector. Now
%*******************************************************************************
\begin{align}
\label{Eq:SU_MMIMO_PCTC_Pre_Eq8}
 E
\left[
\tilde{V}_{k,\, i}
\tilde{V}_{k,\, m}^*
\right] & =  E
            \left[
            \sum_{j=1}^{N_r}
            \tilde{Z}_{k,\, j,\, i}^*
            \tilde{W}_{k,\, j}
            \sum_{l=1}^{N_r}
            \tilde{Z}_{k,\, l,\, m}
            \tilde{W}_{k,\, l}^*
            \right]                             \nonumber  \\
        & = \sum_{j=1}^{N_r}
            \sum_{l=1}^{N_r}
             E
            \left[
            \tilde{Z}_{k,\, l,\, m}
            \tilde{Z}_{k,\, j,\, i}^*
            \right]
             E
            \left[
            \tilde{W}_{k,\, j}
            \tilde{W}_{k,\, l}^*
            \right]                             \nonumber  \\
        & = \sum_{j=1}^{N_r}
            \sum_{l=1}^{N_r}
             2
            \sigma^2_{Z,\, i}
            \delta_K(i-m)
            \delta_K(j-l)                       \nonumber  \\
        &   \qquad \times
             2
            \sigma^2_W
            \delta_K(j-l)                       \nonumber  \\
        & =  4 N_r
            \sigma^2_{Z,\, i}
            \sigma^2_W
            \delta_K(i-m)
\end{align}
%*******************************************************************************
where we have used (\ref{Eq:SU_MMIMO_PCTC_Pre_Eq5}). Let
%*******************************************************************************
\begin{equation}
\label{Eq:SU_MMIMO_PCTC_Pre_Eq9}
\tilde{\mathbf{F}}_k = \tilde{\mathbf{Z}}_k^H
                       \tilde{\mathbf{Z}}_k
\end{equation}
%*******************************************************************************
which is an $N_t\times N_t$ matrix.
Substituting (\ref{Eq:SU_MMIMO_PCTC_Pre_Eq9}) in
(\ref{Eq:SU_MMIMO_PCTC_Pre_Eq3}) we get
%*******************************************************************************
\begin{equation}
\label{Eq:SU_MMIMO_PCTC_Pre_Eq10}
\tilde{\mathbf{Y}}_k = \tilde{\mathbf{F}}_k
                       \mathbf{S} +
                       \tilde{\mathbf{V}}_k.
\end{equation}
%*******************************************************************************
Similar to (\ref{Eq:EE677A_Asgn1_Sol_Eq1}), the $i^{th}$ element of
$\tilde{\mathbf{Y}}_k$ in (\ref{Eq:SU_MMIMO_PCTC_Pre_Eq10}) is given by
%*******************************************************************************
\begin{equation}
\label{Eq:SU_MMIMO_PCTC_Pre_Eq11}
\tilde{Y}_{k,\, i} = \tilde{F}_{k,\, i,\, i} S_i +
                     \tilde{I}_{k,\, i} +
                     \tilde{V}_{k,\, i}
                     \quad \mbox{for $1\leq i\leq N_t$}
\end{equation}
%*******************************************************************************
where
%*******************************************************************************
\begin{align}
\label{Eq:SU_MMIMO_PCTC_Pre_Eq12}
\tilde{V}_{k,\, i}      & = \sum_{j=1}^{N_r}
                            \tilde{Z}_{k,\, j,\, i}^*
                            \tilde W_{k,\, j}                 \nonumber  \\ 
\tilde{I}_{k,\, i}      & = \sum_{\substack{j=1\\j\neq i}}^{N_t}
                            \tilde{F}_{k,\, i,\, j} S_j       \nonumber  \\
\tilde{F}_{k,\, i,\, j} & = \sum_{l=1}^{N_r}
                            \tilde{Z}_{k,\, l,\, i}^*
                            \tilde{Z}_{k,\, l,\, j}.
\end{align}
%*******************************************************************************
Note that from (\ref{Eq:SU_MMIMO_PCTC_Pre_Eq5}) and
(\ref{Eq:SU_MMIMO_PCTC_Pre_Eq9}) we have
%*******************************************************************************
\begin{equation}
\label{Eq:SU_MMIMO_PCTC_Pre_Eq13}
 E
\left[
\tilde{F}_{k,\, i,\, i}
\right] = 2 N_r\sigma^2_{Z,\, i}.
\end{equation}
%*******************************************************************************
Now
%*******************************************************************************
\begin{align}
\label{Eq:SU_MMIMO_PCTC_Pre_Eq14}
 E
\left[
\tilde{F}_{k,\, i,\, i}^2
\right] & =  E
            \left[
            \sum_{l=1}^{N_r}
            \left|
            \tilde{Z}_{k,\, l,\, i}
            \right|^2
            \sum_{m=1}^{N_r}
            \left|
            \tilde{Z}_{k,\, m,\, i}
            \right|^2
            \right]                                    \nonumber  \\
        & = \sum_{l=1}^{N_r}
            \left|
            \tilde{Z}_{k,\, l,\, i}
            \right|^4                                  \nonumber  \\
        &   \qquad +
            \sum_{\substack{m=1\\m\ne l}}^{N_r}
             E
            \left[
            \left|
            \tilde{Z}_{k,\, l,\, i}
            \right|^2
            \right]
             E
            \left[
            \left|
            \tilde{Z}_{k,\, m,\, i}
            \right|^2
            \right]                                    \nonumber  \\
        & =  4N_r(N_r+1)
            \sigma_{Z,\, i}^4.
\end{align}
%*******************************************************************************
Similarly
%*******************************************************************************
\begin{align}
\label{Eq:SU_MMIMO_PCTC_Pre_Eq15}
 E
\left[
\left|
\tilde{I}_{k,\, i}
\right|^2
\right] & =  E
            \left[
            \sum_{\substack{j=1\\j\ne i}}^{N_t}
            \tilde{F}_{k,\, i,\, j} S_j
            \sum_{\substack{l=1\\l\ne i}}^{N_t}
            \tilde{F}_{k,\, i,\, l}^* S_l^*
            \right]                                    \nonumber  \\
        & =  P_{\mathrm{av}}
            \sum_{\substack{j=1\\j\ne i}}^{N_t}
             E
            \left[
            \left|
            \tilde{F}_{k,\, i,\, j}
            \right|^2
            \right].
\end{align}
%*******************************************************************************
Now
%*******************************************************************************
\begin{align}
\label{Eq:SU_MMIMO_PCTC_Pre_Eq16}
 E
\left[
\left|
\tilde{F}_{k,\, i,\, j}
\right|^2
\right] & =  E
            \left[
            \sum_{l=1}^{N_r}
            \tilde{Z}_{k,\, l,\, i}^*
            \tilde{Z}_{k,\, l,\, j}
            \sum_{m=1}^{N_r}
            \tilde{Z}_{k,\, m,\, i}
            \tilde{Z}_{k,\, m,\, j}^*
            \right]                                    \nonumber  \\
        & = \sum_{l=1}^{N_r}
            \sum_{m=1}^{N_r}
             4
            \sigma_{Z,\, i}^2
            \sigma_{Z,\, j}^2
            \delta_K(l-m)                              \nonumber  \\
        & =  4N_r
            \sigma_{Z,\, i}^2
            \sigma_{Z,\, j}^2
\end{align}
%*******************************************************************************
where we have used (\ref{Eq:SU_MMIMO_PCTC_Pre_Eq5}). Substituting
(\ref{Eq:SU_MMIMO_PCTC_Pre_Eq16}) in (\ref{Eq:SU_MMIMO_PCTC_Pre_Eq15}) we get
%*******************************************************************************
\begin{equation}
\label{Eq:SU_MMIMO_PCTC_Pre_Eq17}
 E
\left[
\left|
\tilde{I}_{k,\, i}
\right|^2
\right]   =  4P_{\mathrm{av}} N_r
            \sigma_{Z,\, i}^2
            \sum_{\substack{j=1\\j\ne i}}^{N_t}
            \sigma_{Z,\, j}^2.
\end{equation}
%*******************************************************************************
Note that
%*******************************************************************************
\begin{equation}
\label{Eq:SU_MMIMO_PCTC_Pre_Eq18}
 E
\left[
\left|
\tilde{I}_{k,\, i} +
\tilde{V}_{k,\, i}
\right|^2
\right] =
 E
\left[
\left|
\tilde{I}_{k,\, i}
\right|^2
\right] +
 E
\left[
\left|
\tilde{V}_{k,\, i}
\right|^2
\right].
\end{equation}
%*******************************************************************************
The average SINR per bit for the $i^{th}$ transmit antenna is given by
(\ref{Eq:EE677A_Asgn1_Sol_Eq14}) and is equal to
%*******************************************************************************
\begin{align}
\label{Eq:SU_MMIMO_PCTC_Pre_Eq19}
\mathrm{SINR}_{\mathrm{av},\, b,\, i}
        & = \frac{
             E
            \left[
            \left|
            \tilde{F}_{k,\, i,\, i} S_i
            \right|^2
            \right]
            \times 2N_{rt}}
            {
             E
            \left[
            \left|
            \tilde{I}_{k,\, i} + \tilde{V}_{k,\, i}
            \right|^2
            \right]
            }                                              \nonumber  \\
        & = \frac{P_{\mathrm{av}}(N_r+1)\sigma_{Z,\, i}^2\times 2N_{rt}}
                 {P_{\mathrm{av}}
                 \sum_{\substack{j=1\\j\ne i}}^{N_t}
                 \sigma_{Z,\, j}^2+\sigma_W^2}
\end{align}
%*******************************************************************************
where we have used (\ref{Eq:SU_MMIMO_PCTC_Pre_Eq8}),
(\ref{Eq:SU_MMIMO_PCTC_Pre_Eq14}) and
(\ref{Eq:SU_MMIMO_PCTC_Pre_Eq17}). The upper bound on the average
SINR per bit for the $i^{th}$ transmit antenna is obtained by setting
$\sigma_W^2=0$ in (\ref{Eq:SU_MMIMO_PCTC_Pre_Eq19}) and is equal to
%*******************************************************************************
\begin{equation}
\label{Eq:SU_MMIMO_PCTC_Pre_Eq20}
\mathrm{SINR}_{\mathrm{av},\, b,\, \mathrm{UB},\, i}
 = \frac{(N_r+1)\sigma_{Z,\, i}^2\times 2N_{rt}}
        {
        \sum_{\substack{j=1\\j\ne i}}^{N_t}
        \sigma_{Z,\, j}^2}
\end{equation}
%*******************************************************************************
which is illustrated in Figure~\ref{Fig:SINR_Pre_UB_BA_Numant1024} for
$N_{\mathrm{tot}}=1024$ and $N_{rt}=2$.
The value of the upper bound on the average SINR per bit for $N_t=i=50$
is 18.6 dB. The channel correlation is given by (\ref{Eq:SU_MMIMO_SCTC_Eq46}).
%*******************************************************************************
\begin{figure*}[tbhp]
\centering
%\begin{center}
\input{sinr_pre_ub_ba_numant1024.pstex_t}
\caption{Plot of $\mathrm{SINR}_{\mathrm{av},\, b,\,\mathrm{UB},\, i}$
         for $N_{\mathrm{tot}}=1024$, $N_{rt}=2$ after precoding. (a) Back view.
         (b) Sideview. (c) Front view.}
\label{Fig:SINR_Pre_UB_BA_Numant1024}
%\end{center}
\end{figure*}
%*******************************************************************************
%*******************************************************************************
\begin{figure*}[tbhp]
\centering
%\begin{center}
\input{sinr_pre_ub_aa_numant1024.pstex_t}
\caption{Plot of $\mathrm{SINR}_{\mathrm{av},\, b,\, C,\,\mathrm{UB},\, i}$
         for $N_{\mathrm{tot}}=1024$, $N_{rt}=2$ after precoding. (a) Back view.
         (b) Side view. (c) Front view.}
\label{Fig:SINR_Pre_UB_AA_Numant1024}
%\end{center}
\end{figure*}
%*******************************************************************************
Note that a first-order prediction filter completely decorrelates the channel
with \cite{Vasu_Book10}
%*******************************************************************************
\begin{align}
\label{Eq:SU_MMIMO_PCTC_Pre_Eq21}
\tilde{a}_{i,\, 1} & = -0.9 \qquad \mbox{for $1\le i\le N_t-1$}  \nonumber  \\
\tilde{a}_{i,\, j} & =  0   \qquad \mbox{for $2\le i\le N_t-1$, $2\le j\le i$}.
\end{align}
%*******************************************************************************
We also have \cite{Vasu_Book10}
%*******************************************************************************
\begin{align}
\label{Eq:SU_MMIMO_PCTC_Pre_Eq22}
\sigma_{Z,\, i}^2 & = \sigma_{Z,\, 2}^2     \nonumber  \\
                  & = \left(
                       1-|-0.9|^2
                      \right)
                      \sigma_{Z,\, 1}^2     \nonumber  \\
                  & =  0.19
                      \sigma_{Z,\, 1}^2     \qquad \mbox{for $i>2$.}
\end{align}
%*******************************************************************************
Therefore we see in Figure~\ref{Fig:SINR_Pre_UB_BA_Numant1024} that the
first transmit antenna $i=1$ has a high
$\mathrm{SINR}_{\mathrm{av},\, b,\,\mathrm{UB},\, i}$ due to low interference
power from remaining transmit antennas, whereas for $i\ne 1$ the
$\mathrm{SINR}_{\mathrm{av},\, b,\,\mathrm{UB},\, i}$ is low due to high
interference power from the first transmit antenna ($i=1$). The received signal
after ``combining'' is given by (\ref{Eq:SU_MMIMO_SCTC_Eq4}) and
(\ref{Eq:SU_MMIMO_SCTC_Eq34}).
Note that from (\ref{Eq:SU_MMIMO_SCTC_Eq34}) and (\ref{Eq:SU_MMIMO_PCTC_Pre_Eq12})
%*******************************************************************************
\begin{align}
\label{Eq:SU_MMIMO_PCTC_Pre_Eq22_1}
 E
\left[
 F_i^2
\right] & = \frac{1}{N_{rt}^2}
             E
            \left[
            \sum_{k=0}^{N_{rt}-1}
            \tilde{F}_{k,\, i,\, i}
            \sum_{l=0}^{N_{rt}-1}
            \tilde{F}_{l,\, i,\, i}
            \right]                                \nonumber  \\
        & = \frac{1}{N_{rt}^2}
            \sum_{k=0}^{N_{rt}-1}
             E
            \left[
            \left|
            \tilde{F}_{k,\, i,\, i}
            \right|^2
            \right] +
            \sum_{\substack{l=0\\l\ne k}}^{N_{rt}-1}
             E
            \left[
            \tilde{F}_{k,\, i,\, i}
            \tilde{F}_{l,\, i,\, i}
            \right]                                \nonumber  \\
        & = \frac{4N_r\sigma_{Z,\, i}^4}{N_{rt}^2}
            \sum_{k=0}^{N_{rt}-1}
            (N_r+1) +
            (N_{rt}-1) N_r                         \nonumber  \\
        & = \frac{4N_r\sigma_{Z,\, i}^4}{N_{rt}}
            (1+N_r N_{rt})
\end{align}
%*******************************************************************************
where we have used (\ref{Eq:SU_MMIMO_SCTC_Eq36}),
(\ref{Eq:SU_MMIMO_PCTC_Pre_Eq13}) and
(\ref{Eq:SU_MMIMO_PCTC_Pre_Eq14}). Similarly from
(\ref{Eq:SU_MMIMO_SCTC_Eq34}), (\ref{Eq:SU_MMIMO_PCTC_Pre_Eq8}),
(\ref{Eq:SU_MMIMO_PCTC_Pre_Eq17}) and (\ref{Eq:SU_MMIMO_PCTC_Pre_Eq18}) we
have
%*******************************************************************************
\begin{align}
\label{Eq:SU_MMIMO_PCTC_Pre_Eq22_2}
 E
\left[
\left|
\tilde{U}_i
\right|^2
\right] & = \frac{1}{N_{rt}^2}
             E
            \left[
            \sum_{k=0}^{N_{rt}-1}
            \tilde{U}_{k,\, i}'
            \sum_{l=0}^{N_{rt}-1}
            \left(
            \tilde{U}_{l,\, i}'
            \right)^*
            \right]                               \nonumber  \\
        & = \frac{1}{N_{rt}^2}
            \sum_{k=0}^{N_{rt}-1}
            \sum_{l=0}^{N_{rt}-1}
             E
            \left[
            \tilde{U}_{k,\, i}'
            \left(
            \tilde{U}_{l,\, i}'
            \right)^*
            \right]                               \nonumber  \\
        & = \frac{1}{N_{rt}^2}
            \sum_{k=0}^{N_{rt}-1}
            \sum_{l=0}^{N_{rt}-1}
             E
            \left[
            \left|
            \tilde{U}_{k,\, i}'
            \right|^2
            \right]
            \delta_K(k-l)                         \nonumber  \\
        & = \frac{1}{N_{rt}}
             E
            \left[
            \left|
            \tilde{U}_{k,\, i}'
            \right|^2
            \right]                               \nonumber  \\
        & = \frac{1}{N_{rt}}
            \left[
             E
            \left[
            \left|
            \tilde{I}_{k,\, i}
            \right|^2
            \right] +
             E
            \left[
            \left|
            \tilde{V}_{k,\, i}
            \right|^2
            \right]
            \right]                               \nonumber  \\
        & = \frac{4N_r\sigma_{Z,\, i}^2}{N_{rt}}
            \left[
             P_{\mathrm{av}}
            \sum_{\substack{j=1\\j\ne i}}^{N_t}
            \sigma_{Z,\, j}^2 +
            \sigma_W^2
            \right].
\end{align}
%*******************************************************************************
Substituting (\ref{Eq:SU_MMIMO_PCTC_Pre_Eq22_1}) and
(\ref{Eq:SU_MMIMO_PCTC_Pre_Eq22_2}) in (\ref{Eq:SU_MMIMO_SCTC_Eq45})
we have, after simplification, for $1\le i \le N_t$
%*******************************************************************************
\begin{align}
\label{Eq:SU_MMIMO_PCTC_Pre_Eq23}
\mathrm{SINR}_{\mathrm{av},\, b,\, C,\, i}
& =
\frac{2P_{\mathrm{av}}E\left[F_i^2\right]}
     {E\left[\left|\tilde{U}_i\right|^2\right]}    \nonumber  \\
& =
\frac{(N_r N_{rt}+1)\sigma_{Z,\, i}^2\times 2P_{\mathrm{av}}}
     {
      P_{\mathrm{av}}
     \sum_{\substack{j=1\\j\ne i}}^{N_t}
     \sigma_{Z,\, j}^2+\sigma_W^2}.
\end{align}
%*******************************************************************************
The upper bound on the average SINR per bit for the $i^{th}$ transmit antenna
is obtained by substituting (\ref{Eq:SU_MMIMO_PCTC_Pre_Eq23}) in
(\ref{Eq:SU_MMIMO_SCTC_Eq45_1}) and is equal to
%*******************************************************************************
\begin{align}
\label{Eq:SU_MMIMO_PCTC_Pre_Eq24}
\mathrm{SINR}_{\mathrm{av},\, b,\, C,\, \mathrm{UB},\, i}
& =  \frac{(N_r N_{rt}+1)\sigma_{Z,\, i}^2\times 2}
     {
     \sum_{\substack{j=1\\j\ne i}}^{N_t}
     \sigma_{Z,\, j}^2}                   \nonumber  \\
&    \approx
     \mathrm{SINR}_{\mathrm{av},\, b,\, \mathrm{UB},\, i}
\end{align}
%*******************************************************************************
for $1\le i \le N_t$, $N_r \gg 1$. This is illustrated in 
Figure~\ref{Fig:SINR_Pre_UB_AA_Numant1024} for $N_{\mathrm{tot}}=1024$ and
$N_{rt}=2$. We again observe that the first transmit antenna ($i=1$) has a high
upper bound on the average SINR per bit, after ``combining'', compared to the
remaining transmit antennas.
The value of the upper bound on the average SINR per bit  after ``combining''
for $N_t=i=50$, $N_{\mathrm{tot}}=1024$ is 18.6 dB.
After concatenation, $\tilde{Y}_i$ for $0\le i\le L_d-1$,
in (\ref{Eq:SU_MMIMO_SCTC_Eq4}) and (\ref{Eq:SU_MMIMO_SCTC_Eq34}) is
given to the turbo decoder \cite{Vasu_Book10,KV_OpSigPJ2019}.
Let (see (26) of \cite{KV_OpSigPJ2019}):
%*******************************************************************************
\begin{align}
\label{Eq:SU_MMIMO_PCTC_Pre_Eq25}
\tilde{\mathbf{Y}}_1 & = \left[
                         \begin{array}{ccc}
                         \tilde{Y}_0         & \ldots & \tilde{Y}_{L_{d1}-1}
                         \end{array}
                         \right]                      \nonumber  \\
\tilde{\mathbf{Y}}_2 & = \left[
                         \begin{array}{ccc}
                         \tilde{Y}_{L_{d1}}  & \ldots & \tilde{Y}_{L_{d}-1}
                         \end{array}
                         \right].
\end{align}
%*******************************************************************************
Then \cite{Vasu_Book10,KV_OpSigPJ2019}
%*******************************************************************************
\begin{align}
\label{Eq:SU_MMIMO_PCTC_Pre_Eq26}
\gamma_{1,\, i,\, m,\, n} & =
                            \exp
                            \left[
                             -
                            \,
                            \frac{
                            \left|
                            \tilde Y_i - F_i S_{m,\, n}
                            \right|^2}{2\sigma^2_{U,\, i}}
                            \right]         \nonumber  \\
\gamma_{2,\, i,\, m,\, n} & =
                            \exp
                            \left[
                             -
                            \,
                            \frac{
                            \left|
                            \tilde Y_{i1} - F_{i1} S_{m,\, n}
                            \right|^2}{2\sigma^2_{U,\, i}}
                            \right]
\end{align}
%*******************************************************************************
where
%*******************************************************************************
\begin{equation}
\label{Eq:SU_MMIMO_PCTC_Pre_Eq27}
i1 = i + L_{d1}  \qquad \mbox{for $0 \le i \le L_{d1}-1$}.
\end{equation}
%*******************************************************************************
The rest of the turbo decoding algorithm is similar to that discussed
in \cite{Vasu_Book10,KV_OpSigPJ2019} and will not be repeated here.
In the next subsection we present the computer simulation results for correlated
channel with precoding and PCTC.
%*******************************************************************************
\subsection{Simulation Results}
\label{SSec:PCTC_Corr_Results}
%*******************************************************************************
\begin{figure*}[tbhp]
\centering
%\begin{center}
\input{ber_num_ant1024_corr_dmt51_2_i.pstex_t}
\caption{Simulation results with precoding for $N_{\mathrm{tot}}=1024$.}
\label{Fig:Ber_NUM_ANT1024_Corr_DMT51_2_i}
%\end{center}
\end{figure*}
%*******************************************************************************
The channel correlation is given by (\ref{Eq:SU_MMIMO_SCTC_Eq46}).
The BER results for $N_{\mathrm{tot}}=1024$ with precoding are depicted in
Figure~\ref{Fig:Ber_NUM_ANT1024_Corr_DMT51_2_i}. Incidentally, the value of the
upper bound on the average SINR per bit before and after ``combining'' for
$N_t=i=512$, $N_{\mathrm{tot}}=1024$ is 6 dB.
%*******************************************************************************
\begin{figure*}[tbhp]
\centering
%\begin{center}
\input{ber_num_ant32_corr_dmt51_2_i.pstex_t}
\caption{Simulation results with precoding for $N_{\mathrm{tot}}=32$.}
\label{Fig:Ber_NUM_ANT32_Corr_DMT51_2_i}
%\end{center}
\end{figure*}
%*******************************************************************************
The BER results for $N_{\mathrm{tot}}=32$ with precoding are depicted in
Figure~\ref{Fig:Ber_NUM_ANT32_Corr_DMT51_2_i}. Note that since the average
SINR per bit depends on the transmit antenna, the \textit{minimum} average
SINR per bit is indicated along the $x$-axis of
Figures~\ref{Fig:Ber_NUM_ANT1024_Corr_DMT51_2_i} and
\ref{Fig:Ber_NUM_ANT32_Corr_DMT51_2_i}. We also observe from
Figures~\ref{Fig:Ber_NUM_ANT1024_Corr_DMT51_2_i}(a, b) and
\ref{Fig:Ber_NUM_ANT32_Corr_DMT51_2_i} that there is a large difference
between theory and simulations. This is probably because, the
average SINR per bit is not identical for all transmit antennas. In particular,
we observe from Figures~\ref{Fig:SINR_Pre_UB_BA_Numant1024} and
\ref{Fig:SINR_Pre_UB_AA_Numant1024} that the first transmit antenna
has a large average SINR per bit compared to the remaining antennas.
However, in Figures~\ref{Fig:Ber_NUM_ANT1024_Corr_DMT51_2_i}(c, d) there is
a close match between theory and simulations. This could be attributed to
having a large number of blocks in a frame, as given by
(\ref{Eq:SU_MMIMO_SCTC_Eq2}), resulting in better statistical properties. 
Even though the number of blocks is large in
\ref{Fig:Ber_NUM_ANT32_Corr_DMT51_2_i}, the number of transmit antennas is small,
resulting in inferior statistical properties. In order to improve the accuracy
of the BER estimate for $N_{\mathrm{tot}}=32$, we propose to transmit
``dummy data'' from the first transmit antenna and ``actual data'' from the
remaining antennas. The BER results shown in
Figure~\ref{Fig:Ber_NUM_ANT32_Corr_DMT51_3_i} indicates a good match between
theory and practice. However, comparison of
Figures~\ref{Fig:Ber_NUM_ANT1024_Corr_DMT51_2_i} and
\ref{Fig:Ber_NUM_ANT1024_Corr_DMT51_3_i} demonstrates that ``dummy data'' is
ineffective for large number of transmit antennas.
%*******************************************************************************
\begin{figure*}[tbhp]
\centering
%\begin{center}
\input{ber_num_ant1024_corr_dmt51_3_i.pstex_t}
\caption{Simulation results with precoding and dummy data for
         $N_{\mathrm{tot}}=1024$.}
\label{Fig:Ber_NUM_ANT1024_Corr_DMT51_3_i}
%\end{center}
\end{figure*}
%*******************************************************************************
%*******************************************************************************
\begin{figure*}[tbhp]
\centering
%\begin{center}
\input{ber_num_ant32_corr_dmt51_3_i.pstex_t}
\caption{Simulation results with precoding and dummy data
         for $N_{\mathrm{tot}}=32$.}
\label{Fig:Ber_NUM_ANT32_Corr_DMT51_3_i}
%\end{center}
\end{figure*}
%*******************************************************************************



%Subsection text here.

% needed in second column of first page if using \IEEEpubid
%\IEEEpubidadjcol

%\subsubsection{Subsubsection Heading Here}
%Subsubsection text here.


% An example of a floating figure using the graphicx package.
% Note that \label must occur AFTER (or within) \caption.
% For figures, \caption should occur after the \includegraphics.
% Note that IEEEtran v1.7 and later has special internal code that
% is designed to preserve the operation of \label within \caption
% even when the captionsoff option is in effect. However, because
% of issues like this, it may be the safest practice to put all your
% \label just after \caption rather than within \caption{}.
%
% Reminder: the "draftcls" or "draftclsnofoot", not "draft", class
% option should be used if it is desired that the figures are to be
% displayed while in draft mode.
%
%\begin{figure}[!t]
%\centering
%\includegraphics[width=2.5in]{myfigure}
% where an .eps filename suffix will be assumed under latex, 
% and a .pdf suffix will be assumed for pdflatex; or what has been declared
% via \DeclareGraphicsExtensions.
%\caption{Simulation results for the network.}
%\label{fig_sim}
%\end{figure}

% Note that the IEEE typically puts floats only at the top, even when this
% results in a large percentage of a column being occupied by floats.


% An example of a double column floating figure using two subfigures.
% (The subfig.sty package must be loaded for this to work.)
% The subfigure \label commands are set within each subfloat command,
% and the \label for the overall figure must come after \caption.
% \hfil is used as a separator to get equal spacing.
% Watch out that the combined width of all the subfigures on a 
% line do not exceed the text width or a line break will occur.
%
%\begin{figure*}[!t]
%\centering
%\subfloat[Case I]{\includegraphics[width=2.5in]{box}%
%\label{fig_first_case}}
%\hfil
%\subfloat[Case II]{\includegraphics[width=2.5in]{box}%
%\label{fig_second_case}}
%\caption{Simulation results for the network.}
%\label{fig_sim}
%\end{figure*}
%
% Note that often IEEE papers with subfigures do not employ subfigure
% captions (using the optional argument to \subfloat[]), but instead will
% reference/describe all of them (a), (b), etc., within the main caption.
% Be aware that for subfig.sty to generate the (a), (b), etc., subfigure
% labels, the optional argument to \subfloat must be present. If a
% subcaption is not desired, just leave its contents blank,
% e.g., \subfloat[].


% An example of a floating table. Note that, for IEEE style tables, the
% \caption command should come BEFORE the table and, given that table
% captions serve much like titles, are usually capitalized except for words
% such as a, an, and, as, at, but, by, for, in, nor, of, on, or, the, to
% and up, which are usually not capitalized unless they are the first or
% last word of the caption. Table text will default to \footnotesize as
% the IEEE normally uses this smaller font for tables.
% The \label must come after \caption as always.
%
%\begin{table}[!t]
%% increase table row spacing, adjust to taste
%\renewcommand{\arraystretch}{1.3}
% if using array.sty, it might be a good idea to tweak the value of
% \extrarowheight as needed to properly center the text within the cells
%\caption{An Example of a Table}
%\label{table_example}
%\centering
%% Some packages, such as MDW tools, offer better commands for making tables
%% than the plain LaTeX2e tabular which is used here.
%\begin{tabular}{|c||c|}
%\hline
%One & Two\\
%\hline
%Three & Four\\
%\hline
%\end{tabular}
%\end{table}


% Note that the IEEE does not put floats in the very first column
% - or typically anywhere on the first page for that matter. Also,
% in-text middle ("here") positioning is typically not used, but it
% is allowed and encouraged for Computer Society conferences (but
% not Computer Society journals). Most IEEE journals/conferences use
% top floats exclusively. 
% Note that, LaTeX2e, unlike IEEE journals/conferences, places
% footnotes above bottom floats. This can be corrected via the
% \fnbelowfloat command of the stfloats package.




\section{Conclusions \& Future Work}
\label{Sec:Conclude}
This article presents the advantages of single-user massive multiple
input multiple output (SU-MMIMO) over multi-user (MU) MMIMO systems. The
bit-error-rate (BER) performance of SU-MMIMO using serially concatenated
turbo codes (SCTC) over uncorrelated channel is presented. A semi-analytic
approach to estimating the BER of a turbo code is derived. A detailed
signal-to-interference-plus-noise ratio analysis for SU-MMIMO over
correlated channel is presented. The BER performance of SU-MMIMO with
parallel concatenated turbo code (PCTC) over correlated channel is studied.
Future work could involve estimating the MMIMO channel, since the
present work assumes perfect knowledge of the channel.





% if have a single appendix:
%\appendix[Proof of the Zonklar Equations]
% or
%\appendix  % for no appendix heading
% do not use \section anymore after \appendix, only \section*
% is possibly needed

% use appendices with more than one appendix
% then use \section to start each appendix
% you must declare a \section before using any
% \subsection or using \label (\appendices by itself
% starts a section numbered zero.)
%


%\appendices
%\section{Proof of the First Zonklar Equation}
%Appendix one text goes here.

% you can choose not to have a title for an appendix
% if you want by leaving the argument blank
%\section{}
%Appendix two text goes here.


% use section* for acknowledgment
%\section*{Acknowledgment}


%The authors would like to thank...


% Can use something like this to put references on a page
% by themselves when using endfloat and the captionsoff option.
\ifCLASSOPTIONcaptionsoff
  \newpage
\fi



% trigger a \newpage just before the given reference
% number - used to balance the columns on the last page
% adjust value as needed - may need to be readjusted if
% the document is modified later
%\IEEEtriggeratref{8}
% The "triggered" command can be changed if desired:
%\IEEEtriggercmd{\enlargethispage{-5in}}

% references section

% can use a bibliography generated by BibTeX as a .bbl file
% BibTeX documentation can be easily obtained at:
% http://mirror.ctan.org/biblio/bibtex/contrib/doc/
% The IEEEtran BibTeX style support page is at:
% http://www.michaelshell.org/tex/ieeetran/bibtex/
%\bibliographystyle{IEEEtran}
% argument is your BibTeX string definitions and bibliography database(s)
%\bibliography{IEEEabrv,../bib/paper}
%
% <OR> manually copy in the resultant .bbl file
% set second argument of \begin to the number of references
% (used to reserve space for the reference number labels box)
%\begin{thebibliography}{1}

%\bibitem{IEEEhowto:kopka}
%H.~Kopka and P.~W. Daly, \emph{A Guide to \LaTeX}, 3rd~ed.\hskip 1em plus
%  0.5em minus 0.4em\relax Harlow, England: Addison-Wesley, 1999.
%
%\end{thebibliography}

% biography section
% 
% If you have an EPS/PDF photo (graphicx package needed) extra braces are
% needed around the contents of the optional argument to biography to prevent
% the LaTeX parser from getting confused when it sees the complicated
% \includegraphics command within an optional argument. (You could create
% your own custom macro containing the \includegraphics command to make things
% simpler here.)
%\begin{IEEEbiography}[{\includegraphics[width=1in,height=1.25in,clip,keepaspectratio]{mshell}}]{Michael Shell}
% or if you just want to reserve a space for a photo:

%\begin{IEEEbiography}{Michael Shell}
%Biography text here.
%\end{IEEEbiography}

% if you will not have a photo at all:
%\begin{IEEEbiographynophoto}{John Doe}
%Biography text here.
%\end{IEEEbiographynophoto}

% insert where needed to balance the two columns on the last page with
% biographies
%\newpage

%\begin{IEEEbiographynophoto}{Jane Doe}
%Biography text here.
%\end{IEEEbiographynophoto}

% You can push biographies down or up by placing
% a \vfill before or after them. The appropriate
% use of \vfill depends on what kind of text is
% on the last page and whether or not the columns
% are being equalized.

%\vfill

% Can be used to pull up biographies so that the bottom of the last one
% is flush with the other column.
%\enlargethispage{-5in}



% that's all folks
\begin{footnotesize}

\bibliographystyle{IEEEtran}
\bibliography{mybib,mybib1,mybib2,mybib3,mybib4,mybib5}
%\printbibliography
\end{footnotesize}
\end{document}


