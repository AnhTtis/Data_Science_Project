
 %%%%%%%%%%%%%%%%%%%%%%%%%%%%%%%%%%%%%%%%%%%%%%%%%%%%%%%%%%%%%%%%%%%%%
%% This is a (brief) model paper using the achemso class
%% The document class accepts keyval options, which should include
%% the target journal and optionally the manuscript type. 
%%%%%%%%%%%%%%%%%%%%%%%%%%%%%%%%%%%%%%%%%%%%%%%%%%%%%%%%%%%%%%%%%%%%%
\documentclass[journal=jacsat,manuscript=article]{achemso}

%%%%%%%%%%%%%%%%%%%%%%%%%%%%%%%%%%%%%%%%%%%%%%%%%%%%%%%%%%%%%%%%%%%%%
%% Place any additional packages needed here.  Only include packages
%% which are essential, to avoid problems later. Do NOT use any
%% packages which require e-TeX (for example etoolbox): the e-TeX
%% extensions are not currently available on the ACS conversion
%% servers.
%%%%%%%%%%%%%%%%%%%%%%%%%%%%%%%%%%%%%%%%%%%%%%%%%%%%%%%%%%%%%%%%%%%%%
\usepackage[version=3]{mhchem} % Formula subscripts using \ce{}
\usepackage{amsmath}
\usepackage{amssymb}
\usepackage{amsfonts}
\usepackage{upgreek}
\usepackage{dcolumn}
\usepackage{caption}
\usepackage[labelfont=bf]{caption}
\usepackage[flushleft]{threeparttable}
\usepackage[dvipsnames]{xcolor}
\makeatletter
\setlength\acs@tocentry@height{4.75cm}
\setlength\acs@tocentry@width{8.5cm}
\makeatother
\usepackage{gensymb}
\usepackage{multirow}
\usepackage{xr}
\usepackage{geometry}
\usepackage{pdflscape}
\usepackage{pdfpages}
% \usepackage{orcidlink}
\makeatletter
\newcommand*{\addFileDependency}[1]{% argument=file name and extension
  \typeout{(#1)}% latexmk will find this if $recorder=0 (however, in that case, it will ignore #1 if it is a .aux or .pdf file etc and it exists! if it doesn't exist, it will appear in the list of dependents regardless)
  \@addtofilelist{#1}% if you want it to appear in \listfiles, not really necessary and latexmk doesn't use this
  \IfFileExists{#1}{}{\typeout{No file #1.}}% latexmk will find this message if #1 doesn't exist (yet)
}
\makeatother

\newcommand*{\myexternaldocument}[1]{%
    \externaldocument{#1}%
    \addFileDependency{#1.tex}%
    \addFileDependency{#1.aux}%
}
\newcommand\wordcount{
    \immediate\write18{texcount -sub=section \jobname.tex  | grep "Section" | sed -e 's/+.*//' | sed -n \thesection p > 'count.txt'}
(\input{count.txt}words)}
\myexternaldocument{Supporting-Info}
% \usepackage{biblatex}
% \addbibresource{achemso-demo.bib}
%%%%%%%%%%%%%%%%%%%%%%%%%%%%%%%%%%%%%%%%%%%%%%%%%%%%%%%%%%%%%%%%%%%%%
%% If issues arise when submitting your manuscript, you may want to
%% un-comment the next line.  This provides information on the
%% version of every file you have used.
%%%%%%%%%%%%%%%%%%%%%%%%%%%%%%%%%%%%%%%%%%%%%%%%%%%%%%%%%%%%%%%%%%%%%
%%\listfiles

%%%%%%%%%%%%%%%%%%%%%%%%%%%%%%%%%%%%%%%%%%%%%%%%%%%%%%%%%%%%%%%%%%%%%
%% Place any additional macros here.  Please use \newcommand* where
%% possible, and avoid layout-changing macros (which are not used
%% when typesetting).
%%%%%%%%%%%%%%%%%%%%%%%%%%%%%%%%%%%%%%%%%%%%%%%%%%%%%%%%%%%%%%%%%%%%%
\newcommand*\mycommand[1]{\texttt{\emph{#1}}}
\newcommand{\spc}[1]{$^\textit{#1}$}
\newcommand{\orcid}[1]{#1\orcidlink{#1}}
%%%%%%%%%%%%%%%%%%%%%%%%%%%%%%%%%%%%%%%%%%%%%%%%%%%%%%%%%%%%%%%%%%%%%
%% Meta-data block
%% ---------------
%% Each author should be given as a separate \author command.
%%
%% Corresponding authors should have an e-mail given after the author
%% name as an \email command. Phone and fax numbers can be given
%% using \phone and \fax, respectively; this information is optional.
%%
%% The affiliation of authors is given after the authors; each
%% \affiliation command applies to all preceding authors not already
%% assigned an affiliation.
%%
%% The affiliation takes an option argument for the short name.  This
%% will typically be something like "University of Somewhere".
%%
%% The \altaffiliation macro should be used for new address, etc.
%% On the other hand, \alsoaffiliation is used on a per author basis
%% when authors are associated with multiple institutions.
%%%%%%%%%%%%%%%%%%%%%%%%%%%%%%%%%%%%%%%%%%%%%%%%%%%%%%%%%%%%%%%%%%%%%
\author{Yulong~Zheng} 
\affiliation[GT-chem]
{School of Chemistry and Biochemistry, Georgia Institute of Technology, 901 Atlantic Drive, Atlanta, Georgia 30332, United~States}

\author{Rahul~Venkatesh}
\affiliation[GT-chbe]
{School of Chemical and Biomolecular Engineering, Georgia Institute of Technology, 311 Ferst Drive NW, Atlanta, Georgia 30332, United~States}

\author{Connor~P.~Callaway}
\affiliation[UKY]
{Department of Chemistry and Center for Applied Energy Research, University of Kentucky, Lexington, Kentucky 40506, United~States}

\author{Campbell~Viersen}
\affiliation[GT-chem]
{School of Chemistry and Biochemistry, Georgia Institute of Technology, 901 Atlantic Drive, Atlanta, Georgia 30332, United~States}

\author{Kehinde~H.~Fagbohungbe}
\affiliation[UKY]
{Department of Chemistry and Center for Applied Energy Research, University of Kentucky, Lexington, Kentucky 40506, United~States}

\author{Aaron~L.~Liu}
\affiliation[GT-chbe]
{School of Chemical and Biomolecular Engineering, Georgia Institute of Technology, 311 Ferst Drive NW, Atlanta, Georgia 30332, United~States}

% \author{Christopher~J.~Takacs}
% \affiliation[Stanford]
% {Stanford Synchrotron Radiation Lightsource, SLAC National Accelerator Laboratory, Menlo Park, California 94025, United~States}

\author{Chad~Risko}
\affiliation[UKY]
{Department of Chemistry and Center for Applied Energy Research, University of Kentucky, Lexington, Kentucky 40506, United~States}

\author{Elsa~Reichmanis}
\affiliation[Leigh]
{Department of Chemical \& Biomolecular Engineering, Lehigh University, 124 East Morton Street, Bethlehem, Pennsylvania 18015, United~States}

\author{Carlos~Silva-Acu\~na}
\email{carlos.silva@gatech.edu}
\affiliation[GT-chem]
{School of Chemistry and Biochemistry, Georgia Institute of Technology, 901 Atlantic Drive, Atlanta, Georgia 30332, United~States}
\alsoaffiliation[GT-phys]
{School of Physics, Georgia Institute of Technology, 837 State Street, Atlanta, Georgia 30332, United~States}
\alsoaffiliation[GT-MSE]
{School of Materials Science and Engineering, Georgia Institute of Technology, North Avenue, Atlanta, Georgia 30332, United~States}

%%%%%%%%%%%%%%%%%%%%%%%%%%%%%%%%%%%%%%%%%%%%%%%%%%%%%%%%%%%%%%%%%%%%%
%% The document title should be given as usual. Some journals require
%% a running title from the author: this should be supplied as an
%% optional argument to \title.
%%%%%%%%%%%%%%%%%%%%%%%%%%%%%%%%%%%%%%%%%%%%%%%%%%%%%%%%%%%%%%%%%%%%%
\title%[An \textsf{achemso} demo]
%    {Intrachain Exciton Dispersion in Conjugated Polymer Entanglement Limit} %I don't like this title because "entanglement limit" will be confusing for the pchem community. How about something like this:
    % {Exciton dispersion in a push-pull conjugated polymer: The effects of chain entanglement}
    %{Exciton Delocalization Revealing Polymer Chain Planarization in a Push-Pull Conjugated Polymer}
    {Chain Conformation and Exciton Delocalization in a Push-Pull Conjugated Polymer}


%%%%%%%%%%%%%%%%%%%%%%%%%%%%%%%%%%%%%%%%%%%%%%%%%%%%%%%%%%%%%%%%%%%%%
%% Some journals require a list of abbreviations or keywords to be
%% supplied. These should be set up here, and will be printed after
%% the title and author information, if needed.
%%%%%%%%%%%%%%%%%%%%%%%%%%%%%%%%%%%%%%%%%%%%%%%%%%%%%%%%%%%%%%%%%%%%%
%\abbreviations{IR,NMR,UV}
%\keywords{Conjugated polymers, charge transfer, transient absorption spectroscopy, interchain interaction.}


%%%%%%%%%%%%%%%%%%%%%%%%%%%%%%%%%%%%%%%%%%%%%%%%%%%%%%%%%%%%%%%%%%%%%
%% The manuscript does not need to include \maketitle, which is
%% executed automatically.
%%%%%%%%%%%%%%%%%%%%%%%%%%%%%%%%%%%%%%%%%%%%%%%%%%%%%%%%%%%%%%%%%%%%%

\begin{document}

%%%%%%%%%%%%%%%%%%%%%%%%%%%%%%%%%%%%%%%%%%%%%%%%%%%%%%%%%%%%%%%%%%%%%
%% The "tocentry" environment can be used to create an entry for the
%% graphical table of contents. It is given here as some journals
%% require that it is printed as part of the abstract page. It will
%% be automatically moved as appropriate.
%%%%%%%%%%%%%%%%%%%%%%%%%%%%%%%%%%%%%%%%%%%%%%%%%%%%%%%%%%%%%%%%%%%%%


%%%%%%%%%%%%%%%%%%%%%%%%%%%%%%%%%%%%%%%%%%%%%%%%%%%%%%%%%%%%%%%%%%%%%
%% The abstract environment will automatically gobble the contents
%% if an abstract is not used by the target journal.
%%%%%%%%%%%%%%%%%%%%%%%%%%%%%%%%%%%%%%%%%%%%%%%%%%%%%%%%%%%%%%%%%%%%%
% \twocolumn

\begin{tocentry}
    \includegraphics[width=8.47cm, height=4.76cm]{TOC_graphic.pdf}%
\end{tocentry}

\newpage
\begin{abstract}
      Linear and nonlinear optical lineshapes reveal details of excitonic structure in semiconductor polymers. 
      %Exciton delocalization is demonstrated to be enhanced in a push-pull polymer when the concentration of the solution surpasses the critical overlap point, which reveals increasingly planarized polymer chain backbones. 
      %Inter- and intrachain exciton dispersion are two competing factors %in determining 
      %that determine the photophysics %al features 
      %of semiconductor polymers. Here, w
      We implement absorption, photoluminescence, and
      transient absorption spectroscopies in DPP-DTT, an % type of 
      electron push-pull %donor-acceptor 
      copolymer, %different spectroscopic techniques are combined to study the spectral features of 
      to explore the relationship between their spectral lineshapes and chain conformation, deduced from resonance Raman spectroscopy and %&ts corresponding 
      from %density-function theory %assignments
      \textit{ab initio} calculations. % which is processed and prepared into thin films at varying solution concentrations. 
      The viscosity of precursor polymer solutions before film casting displays a transition that suggests gel formation above a critical concentration. 
      Upon crossing this viscosity deflection concentration, the lineshape analysis  of the absorption spectra within a photophysical aggregate model reveals a gradual increase in interchain excitonic coupling. We also observe a red-shifted and line-narrowed steady-state photoluminescence spectrum, along with increasing resonance Raman intensity in the stretching and torsional modes of the dithienothiphene unit, %, as confirmed by density functional theory, 
       which %could indicate 
      suggests a longer exciton coherence length along the polymer-chain backbone. Furthermore, we %also 
      observe a change of lineshape in the photoinduced absorption component of the transient absorption spectrum. The derivative-like lineshape may originate from two possibilities: a new excited-state absorption, or from optical Stark effect, both of which are consistent with the emergence of high-energy shoulder as seen in both photoluminescence and absorption spectra. Therefore, we conclude that the exciton is more dispersed along the polymer chain backbone with increasing concentrations, leading to the hypothesis that the polymer chain order is enhanced when the push-pull polymers are processed at higher concentrations. Thus, tuning the microscopic chain conformation by concentration %could 
      would be another factor of interest when considering the polymer assembly pathways for pursuing large-area and high-performance organic optoelectronic devices.
\end{abstract}

%%%%%%%%%%%%%%%%%%%%%%%%%%%%%%%%%%%%%%%%%%%%%%%%%%%%%%%%%%%%%%%%%%%%%
%% Start the main part of the manuscript here.
%%%%%%%%%%%%%%%%%%%%%%%%%%%%%%%%%%%%%%%%%%%%%%%%%%%%%%%%%%%%%%%%%%%%%
% \twocolumn
\newpage
\section{INTRODUCTION}

The most fundamental aspect of the materials science of semiconductor polymers concerns the relationship between solid-state microstructure, macromolecular conformation, and the electronic and optical properties of these materials. From a molecular perspective, the optical properties of organic semiconductors are governed by Frenkel excitons~\cite{frenkel1931transformation}, and this relationship entails the dependence of the exciton optical transition density on the conformation of the polymer backbone, and on the nature of Coulomb coupling between chain segments, both intra- and interchain~\cite{spano2006excitons}. Photophysical aggregate models developed by Spano have been spectacularly successful in quantifying excitonic coupling and the nature of the disordered energy landscape from linear spectral lineshapes~\cite{spano2005modeling,spano2006absorption,Clark2007PhysRevLett,spano2009determining,clark2009determining,hestand2015interference,hestand2017molecular,hestand2018expanded}. These models have been extended to push-pull polymers~\cite{zhong2020unusual,balooch2020vibronic,chang2021hj}, in which the role of charge-transfer interactions are evident. In this article, we examine absorption, photoluminescence (PL), and transient absorption optical lineshapes in films of poly[2,5-(2-octyldodecyl)-3,6-diketopyrrolopyrrole-\textit{alt}-5,5-(2,5-di(thien-2-yl)thieno-[3,2-b]-thiophene)] (DPP-DTT), a push-pull polymer designed for applications in thin-film transistors~\cite{kang2013record, li2012stable}. The films are cast from solutions of various concentrations; we find that the optical lineshapes of the films show a strong dependence on the precursor concentration, and correlate the spectral shapes with chain conformation derived from resonance Raman measurements and the corresponding \textit{ab initio} calcuations. We conclude that when solutions are cast from gel-like solutions, excitons are more highly delocalized along the polymer backbone, driven by more planar, less torsionally disordered backbones.

A subtle control of the polymer aggregation could tune the polymer chain conformations, which determines the arrangement of the chromophores. It is crucial to build the correlations between the microscopic conformations and photophysical properties. Although previous studies have shown that photophysical and electronic properties of conjugated homopolymers, like poly-(3-hexyl)thiophene (P3HT) or polyphenylenevinylenes, are greatly influenced by the long-range order,\cite{clark2009determining, paquin2013two, paquin2015multi, koehler2012order} where the exciton dissociation and charge migrations are impacted by the polymeric aggregate fractions, microstructural conformations, defects and etc.\cite{koch2013impact, brinkmann2007effect, brinkmann2009molecular}, the photophysical aggregates and nonaggregates of short-range order could also play a role in the photogenerated charges.\cite{paquin2011charge, reid2012influence} Furthermore, in P3HT-like derivatives, the exciton coherence lengths, which indicates the spatial span of the exciton wave, are varied with different molecular weights.\cite{paquin2013two} Specifically, the intrachain exciton extending more units along the chain backbone in the high molecular-weight P3HTs give rise to longer PL lifetime, compared to their low molecular-weight counterparts.\cite{paquin2015multi}

In comparison to %the
conjugated homopolymers, the photophysical properties and charge transport behavior might depend on the short-range order more subtly in electron push-pull or donor-acceptor (DA) copolymers.\cite{noriega2013general, sirringhaus201425th, guo2014relationship, steyrleuthner2012aggregation} Previous work showed that the electron push-pull nature enhances exciton-exciton annihilation, which gives rise to long-lived bound charge pairs.\cite{gelinas2013recombination} Besides, intra- and interchain charge-transfer interactions, representing the wavefunction overlaps between the electron-sufficient and -deficient moiety along and across the polymer chains, could be subject to the polymer chain backbone orders and $\pi$-$\pi$ stacking, respectively.\cite{balooch2020vibronic, pop2016solid, chaudhari2017donor, noriega2013general} Furthermore, the charge-transfer character in the electron push-pull polymers renders a permanent dipole and induces a strong overlap of the electron cloud between push and pull chromophores, which breaks the Kasha approximation of only transition dipole moments interacting.\cite{mcrae1958enhancement} Therefore, a different energetic landscape from that of homopolymers polymers might be expected in electron push-pull materials.

The impact of inter- and intrachain interactions on the photophysical responses of conjugated polymers has been studied through many ways, such as embedding the diluted semiconductor polymers in inert matrices, or surrounding polymer chains with molecular crowns.\cite{yan1994spatially, cacialli2002cyclodextrin, petrozza2008control, mroz2009laser, cabanillas2011pump, mroz2013amplified} In either way, the interchain excitonic interactions are considered to be reduced. However, most studies introduced new components which might lead to ambiguity regarding polymer chain order, microstructures, dielectric environment and unwanted bath-system interactions. Induced gel formation, on the other hand, only exploited either molecular weight- or concentration-dependence, which are intrinsic properties of polymers in their solution and solid state.\cite{wang2022unraveling, venkatesh2023overlap} Here, we contribute a detailed understanding of the impact of photophysical aggregation on excition delocalization as well as chain planarization through gel formation process in push-pull polymers.

\section{RESULTS}

\begin{figure}[h]
    \centering
    \includegraphics[width=8cm]{Figure1.pdf}
    \caption{(a) Molecular structure of DPP-DTT. (b) Normalized absorption spectra (open circle), and steady-state PL (solid circle) of thin films deposited from different solution concentrations. We note that the abosption spectra were previously published in Ref.~\citenum{venkatesh2021data}. The PL peaks are fitted with a single Gaussian distribution (red dash line), while the absorption vibronic replica are simulated with modified Franck-Condon progression (solid black line). The spectra shift for PL and absorption are denoted with the dashed black lines with increasing concentrations. A pronounced red shift can be readily observed in the PL spectra, meanwhile the absorption spectra also display a small blue shift of around 15\,meV (see Table~S1 in SI for quantitative results.) }
    \label{fig1}
\end{figure}

\begin{figure}[h]
    \centering
    \includegraphics[width=8cm]{Figure2.pdf}
    \caption{(a) Viscosity measurement on DPP-DTT-chlorobenzene solutions performed at $56^{\circ}$C, reproduced from Ref\citenum{venkatesh2021data}. The two-regime behavior is visualized by the two linear lines with different slopes (b) The $A_{0-0}/A_{0-1}$ ratio (black open squares) and the effective interchain exciton bandwidth (red solid squares) acquired from FC simulations as a function of concentrations. The error bars associated with exciton bandwidths are also indicated. (c) The ratio of the optical absorption area of the aggregate and nonaggregate (open hexagon). (d) The peak positions (black open triangles) and widths (purple solid triangles) acquired from Gaussian distributions.  (e) The integral ratio (blue open pentagons) of the shaded area in PIA and GSB, as shown in Fig.~\ref{fig4}. The zero cross points (ZCP) are denoted in blue solid pentagons. (f) The ratio of resonance Raman peak at 1411 and 1366\,cm$^{-1}$, with all spectra simultaneously normalized at peak 1525\,cm$^{-1}$.The dashed line is the guide for eye of the critical concentration point.}
    \label{fig2}
\end{figure}


Using the %\textit{p}-channel 
electron push-pull polymer, DPP-DTT, as the material of interest displayed in Fig.~\ref{fig1}a, thin films were prepared from solutions below and above the viscosity deflecting concentration, $c^*$ as shown in Fig.~\ref{fig2}a. The film preparation method was described previously.\cite{venkatesh2021data} To account for the perturbation of the Coulombic interactions within the excitation band, the absorption spectra were fit to a Franck-Condon (FC) progression modified by the contributions of exciton bandwidth, where the effective Huang-Rhys parameter was set to be 0.73 (Fig.~\ref{fig1}b).\cite{chang2021hj} With an increase in concentration, the ratio of $A_{0-0}$ and $A_{0-1}$ absorption peaks decreases, which corresponds to an increase in interchain exciton bandwidth from 16 to 54\,meV as shown in Fig.~\ref{fig2}b, accompanied with a small blue shift around 15\,meV of the $A_{0-0}$ peak (see Table~S1 and Fig.~\ref{fig1}b). The magnitude of the exciton bandwidth falls well under the weakly-coupled HJ-aggregate limit.\cite{clark2009determining} A direct consequence of larger excitonic interaction in the H-aggregate is an increasing Stokes shift when examining the steady-state photoluminescence (PL) and absorption measurements simultaneously (Fig.~\ref{fig1}b). Interestingly, in addition to the redshift of the PL peaks, a trend of linewidth narrowing is also observed as shown in Fig.~\ref{fig2}d, where the major peak can be simply fitted with a standard Gaussian distribution. Although enhanced interchain excitonic interaction could lead to the redshifting behavior in the emission of H-aggregate, it will not result in drastic line narrowing in the PL line shapes. Based on the model developed by Knapp\cite{knapp1984lineshapes}, Knoester\cite{knoester1993nonlinear} and Spano \cite{spano2009determining, hestand2018expanded}, the line narrowing effects seen in aggregate PL spectra can be explained by the motional narrowing effect, where the distribution of static disorder is averaged out due to the fast-moving excitons. Specifically, the narrowing effect could be attributed to a few physical factors; an increase in the magnitude of static disorder, (inhomogeneous broadening linewidth, $\sigma_d$), growing aggregate size, (number of chromophores in each aggregate, $N$), and/or shorter exciton coherence length across chains, in the weakly-coupled H aggregate.\cite{spano2009determining, paquin2013two, hestand2018expanded} We interpret these observations as a weakly varying width of total disorder with processing concentrations. 
% Especially, the linewidths of the absorption peaks remain approximately constant through the concentration range studied here.
The relevant length scales for the photophysical aggregate are highly microscopic (of order nearest molecular neighbor), while X-ray crystallography samples length scales are relevant to crystalline order. Nevertheless, we can compare this information with that derived from grazing-incidence wide-angle X-ray scattering (GIWAXS) measurements. Recent measurements on this material\cite{venkatesh2023overlap} demonstrated invariant d-spacing values for $\pi-\pi$ stacking of 3.7\,\r{A} and lamellar spacing of $19.6\pm0.2$\,\r{A}, consistent with other DPP-based copolymers as shown in previous literature.\cite{li2019impact, wang2022unraveling, chaudhari2017donor} Furthermore, consistent full width at half maximum (FWHM) linewidths for each (010) scattering peak when comparing all samples, indicating similar paracrystalline static disorder among all samples.\cite{peng2022understanding} To compare the aggregate sizes of the thin films for different concentrations, differential scanning calorimetry (DSC) was performed as shown in Fig.~S3. A melting temperature of $376 \pm 1$\,$^{\circ}$C is found among all samples, and a consistent curve shape for the melting peak is observed. Combining both GIWAXS and DSC measurements, we deduce that the aggregate size and the lattice disorder are probably not the dominant factors for the drastic red shift as observed in the steady-state PL spectra. Since the intra- and interchain exciton delocalization are countering each other\cite{yamagata2012interplay, paquin2013two}, the motional narrowing effect could be ascribed to the variation of the exciton dispersion due to the change of polymer chain order in the HJ-type photophysical aggregates, as will be demonstrated later on.

Another distinctive feature from the PL spectra is the emerging side peak on the high energy shoulder when the concentration crosses $c^*$. As only separated from the dominant Gaussian peak by 100\,meV, we exclude the possibility of it being a vibronic satellite of the typically dominant 160-180\,meV modes. A similar trend is also observed within the absorption spectra (Supplemental Fig.~S1); the deviation from the aggregate spectra on the high-energy side (from 1.43 to 1.68\,eV) is partially ascribed to the absorption of the non-aggregates.\cite{clark2009determining, li2019impact} To more clearly demonstrate the component ratio of the non-aggregates relative to the aggregates, the relative optical contribution from both species is calculated and shown in Fig.~\ref{fig2}c. A constant ratio is observed before c$^*$ while the relative contribution from the non-aggregates starts to increase drastically after the critical concentration. Therefore, the new PL side peaks can be correspondingly assigned to the emission of such photophysical non-aggregates. 



\begin{figure}
    \centering
    \includegraphics[width = 8cm]{Figure3.pdf}
    \caption{Transient absorption spectral maps showing the differential transmission signals of (a) 3, (b) 4, (c) 6, and (d) 8\,g/L samples pumped with 760-nm pulsed beam under 5-$\mu$J/cm$^2$ fluence. The transmission PIA and GSB signals are colored in blue and red, respectively. The dash line lies in the zero-amplitude cross points for each sample, where a clear shift is observed.}
    \label{fig3}
    
\end{figure}

The contributions of the non-aggregate were probed via transient absorption (TA) spectroscopy across the viscosity deflecting concentration. In Fig.~\ref{fig3} we display TA spectra for 3, 4, 6 and 8\,g/L films, with the pump pulse centered at 760\,nm, and with fluence under 5\,$\mu$J/cm$^2$. (Measurement for films prepared from 5\,g/L solutions are presented in Fig.~S5 in the Supplementary Information.) The four TA plots show two domains where the red and blue represent the ground state bleaching (GSB) and photoinduced absorption (PIA), respectively (Fig.~\ref{fig4}). All subplots show a dominant GSB signal and the high absorption coefficient is commonly observed due to the long persistence length of the conjugated copolymers.\cite{vezie2016exploring} The lifetimes of GSB are around 18\,ps (see SI Fig.~S6). The PIA signals, on the other hand, show a comparable lifetime to that of the GSB, which suggests that the high-lying excited states relax to the lowest excited vibronic state on a comparable timescale. 

To avoid arbitrary spectral drifting at early times, the spectra were averaged by taking the temporal cuts from 0.15 to 1\,ps as displayed in Fig.~\ref{fig5}. Within the polymer films deposited from higher concentrations, a new excited-state feature emerges in the PIA region. The zero-cross points, defined as photon energies where the differential transmission signal is zero, indicate counteracting contributions between positive signal (GSB) and negative signal (PIA). The zero cross points (black circles) with respect to the concentration are plotted in Fig.~\ref{fig2}e. The zero-cross points are observed to have a slight decrease up to c$^*$ followed by a drastic increase. The measurement of 5-g/L film shows a clear small peak around 1.33\,eV (932\,nm) due to stimulated emission (SE), which also gives rise to the red shift of the zero-cross points in addition to GSB. As noted, the zero-cross points for samples of 4 and 6\,g/L might also have contributions from SE, even though they are much weaker than 5-g/L sample. The relative integral ratio of the PIA feature and the GSB of the excitons shows a similar trend by comparing the integral of the absorption in these two regions (shown as the shaded area in Fig.~\ref{fig5}a). A direct absorption from the high-energy excited states in the non-aggregates could contribute to this enhanced PIA signature, which is consistent with a growing content of non-aggregates in samples prepared from high concentrations as mentioned earlier. It is also worth pointing out that the new PIA features resemble the derivative-like Stark effect features\cite{weiser1992stark,liess1997electroabsorption}, as the differential spectra are perturbed by the induced electric field from the accumulating photogenerated charges at the interfaces of aggregates and non-aggregates.\cite{paquin2011charge} The degenerate states are lifted by the induced electric field to give such new feature, especially when comparing the differential line shapes with the linear transmission spectra. Below c$^*$, the GSB transition very much follows the linear transmission, while above c$^*$, the differential spectra have a sharper transition. The two above mentioned factors could both contribute to the new PIA features. 


\begin{center}
    \centering
    \includegraphics[width = 8cm]{Figure4.pdf}
    \captionof{figure}{Normalized transient absorption spectra integrated from 0.15 to 1.0 ps measured for samples of 3, 4, 5, 6, and 8 g/L. The cutoff from 1.57 eV to 1.71 eV is due to the leak of pump beams. The shaded area is integrated to estimate the ratio between GSB and PIA. The differential transmission spectra are overlapped with the linear transmission spectra (dashed line) converted from Figure \ref{fig1}.}
    \label{fig4}
\end{center}

\begin{center}
    \centering
    \includegraphics[width=8cm]{Figure5.pdf}
     \captionof{figure}{(a) and (b) show the calculated resonance Raman spectra with geometry 1 (black) and geometry 2 (blue), respectively; both are normalized against the geometry 1 peak intensity at 1525 cm$^{-1}$. The spectra are normalized by their relative calculated maximum intensity. (c) and (d) demonstrated the experimental Resonance Raman spectra displayed within both low and high Raman shift ranges. All experimental spectra are simultaneously normalized at 1525 cm$^{-1}$, which is a C=C stretching mode localized in the DPP unit. The peaks shaded in grey indicate a decrease in Raman intensity, while the peaks shaded in purple indicate increasing Raman intensities with concentrations.}
    \label{fig5}
\end{center}

To further understand the polymer chain order and local conformations, we performed resonance Raman spectroscopy to study the on-chain vibrational modes, which are coupled to the electronic transitions. As a more torsionally ordered polymer chain backbone could sustain a longer exciton coherence length,\cite{paquin2013two} a more delocalized electron density would weaken the resonance Raman intensity.\cite{wood2015natures, paquin2015multi} Here, the high-energy band in DPP-DTT at 488\,nm was excited, which contributes to a delocalized $\pi-\pi^*$ transition as demonstrated by Wood \textit{et al}.\cite{wood2015natures} The Raman spectra at high and low frequency range are shown in Fig.~\ref{fig5}c and d, respectively. The most significant change is observed at 1410\,cm$^{-1}$ within the high-frequency range from 1200 to 1600\,cm$^{-1}$ displayed in Fig.~\ref{fig5}c, whereas the more substantial variations occur in the low-frequency range as shown in Fig.~\ref{fig5}d.

The Raman-active modes are assigned based on density functional theory (DFT) %and time-dependent DFT
calculations. These calculations were performed on a DPP-DTT trimer, truncated with a fourth DPP-thiophene (T) unit. The additional DPP-T unit is utilized to allow for a symmetric charge distribution. The calculated Raman spectra are displayed in Fig.~\ref{fig5}a and b. As demonstrated by Chaudhari\textit{et al.}, two different geometries might coexist in the ensemble, both contributing to the Raman cross section. In geometry 1 (G1), the oxygen atoms of the DPP unit are oriented close to the sulfur atoms of the neighboring T units, while in geometry 2 (G2), they are instead oriented near the hydrogen atoms of the neighboring T. The optimized structures of G1 and G2 are shown in Fig.~S7. Based on DFT calculations, the ground-state energy of G1 is lower than that of G2 by approximately 41.2\,kJ\,mol$^{-1}$. The calculated Raman spectra of both geometries are shown in Fig.~\ref{fig5}; both spectra are normalized to the peak at 1525\,cm$^{-1}$ in G1. In the following discussion, all Raman shifts refer to experimental results unless specified otherwise. A complete assignment of the Raman modes is presented in Table~\ref{tbl:Raman}.

To allow us to compare the change of Raman intensities quantitatively among different samples, a vibrational mode that is not significantly influenced by the torsional order of the polymer backbone (i.e., a local or intraunit vibrational mode) is used as a benchmark. Herein, such mode is chosen to be the local C=C stretching in the highly rigid DPP unit at 1525\,cm$^{-1}$, coupled with a local, asymmetric DTT ring deformation, as shown in the vector diagram in Fig.~\ref{fig6}a. Such localized C=C stretching mode on the DPP unit is also observed in a related DPP-based copolymer.\cite{wood2015natures} Another localized stretching mode at 1366\,cm$^{-1}$ associated with the two bridgehead carbon atoms on the DPP units, which shows constant intensity among all samples, supporting our rationalization. In contrast, the greatest changes in intensity are observed at 1416\,cm$^{-1}$, where the DTT ring demonstrates strong deformation as shown in Fig.~\ref{fig6}b. Interestingly, the corresponding simulated Raman peaks are calculated to be doubly degenerate, with one mode being the ring deformation of the second and third DTT unit and the other mode being that of the first and fourth DTT unit. Such 'globally' delocalized DTT ring deformation directly contributes to a more delocalized exciton wave along the polymer chain backbone, leading to a weaker vibronic coupling strength and decrease of the Raman intensity. A more quantitative demonstration is displayed in Fig.~\ref{fig2}f, where the intensity ratio of Raman peaks at 1410 and 1366\,cm$^{-1}$ is shown to decrease, with clear two-regime behavior. 

The other two globally delocalized vibrational modes, at 706 and 467\,cm$^{-1}$, show a similar decreasing trend in intensity, where the former mode is a gentle ring breathing motion among all DPP and DTT units and the latter is a degenerate ring deformation mode in all DTT units (Fig.~\ref{fig6}f). Aside from the stretching modes, the dynamically disordered thienothiophene unit also experiences a strong torsional motion as observed at 494\,cm$^{-1}$ (Fig.~\ref{fig6}e). Besides the two different trends and their according vibrational modes, an intriguing increasing trend is observed at 810\,cm$^{-1}$, colored in purple in Fig.~\ref{fig5}c. This mode corresponds to the global symmetric C-N stretch in the DPP units (Fig.~\ref{fig6}d), which localizes the exciton wave. Therefore, when the polymer chain backbone becomes more planarized, which sustains a more delocalized exciton, the vibrational mode at 810\,cm$^{-1}$ is not only able to collect the electron density, but also redistribute that density in an almost perpendicular direction relative to the polymer chain. The final result is an increase in intensity for higher concentration samples.

Such observation of the Raman intensity variations aligns well with the observation of the PL motional narrowing effect, as mentioned above, indicating the polymer chain planarization in samples prepared higher than the gel formation concentration. Besides, the detailed analysis of the absorption and PL spectra show the increased exciton bandwidth, indicating stronger interchain interaction. Another surprising finding of higher optical contributions from the non-aggregates with increasing concentrations surfaces when comparing the lineshape and oscillator strength from the linear absorption and nonlinear TA measurements. The strong correlations between the viscosity and spectroscopic results can now be well established (Fig.~\ref{fig2}).


\newgeometry{margin=1cm}
\begin{landscape}
\begin{table}[h]
    \begin{threeparttable}
      \caption{Comparison between the experimental and simulated resonance Raman modes. Simulation results correspond to DPP-DTT geometry 1.}
      \label{tbl:Raman}
      \begin{tabular}{c c cc c}
        \hline
        \hline
        Exp. Raman shift & Simulated Raman &\multicolumn{2}{c}{Norm. intensity} &\multirow{2}{*}{Qualitative assignment}\\ \cline{3-3}\cline{4-4}
        (cm$^{-1}$) & shift (cm$^{-1}$) & 2 (g/L) & 8 (g/L) &\\
        \hline
        \multirow{2}{*}{1525} & 1531 & \multirow{2}{*}{1\spc{a}}  & \multirow{2}{*}{1\spc{a}}& L: 1,4-DPP $\nu_{b-C_t=C_s}$, L: asym. $\nu_{C_s=C_s}$ on 1,3-DTT \\
        & 1525 & & & L: 2,3-DPP $\nu_{b-C_t=C_s}$, L: sym. ring deformation on 2-DTT\\
        1491 & 1493 & 0.507$\pm$0.014 & 0.363$\pm$0.009 & D: gentle DTT and DPP rings deformation, asym. m-Th ring deformation\\
        % \multirow{2}{*}{1441} & 1450 & \multirow{2}{*}{0.245$\pm$0.012} & \multirow{2}{*}{0.211$\pm$0.005}& \\
        % & 1449 & & &\\\\
        1410 & 1431.9, 1432.0\spc{b} & 1.507$\pm$0.007 &0.841$\pm$0.023 & D: 1,2,3-DTT rings deformation due to $\nu_{b-C_t=b-C_t}$\\
        \multirow{2}{*}{1366} & 1383 & \multirow{2}{*}{0.355$\pm$0.004} & \multirow{2}{*}{0.371$\pm$0.009}& L: 2,3-DPP $\nu_{b-C_t=b-C_t}$\\
        & 1381 & & & L: 4-DPP $\nu_{b-C_t=b-C_t}$\\
        818 & 802 & 0.103$\pm$0.011& 0.135$\pm$0.001& D: sym. $\nu_{C_p-N}$ on 1,2,3,4-DPP units \\
        % 780 & 755 & & & \\
        706 & 698 & 0.190$\pm$0.010 & 0.141$\pm$0.004&\multirow{2}{*}{D: gentle ring breathing of 1,2,3-DTT unit and 2,3-DPP unit.}\\
        686 & 678 & 0.095$\pm$0.008 & 0.096$\pm$0.001&\\
        643 & --- & 0.072$\pm$0.003 & 0.038$\pm$0.001&---\\
        \multirow{2}{*}{494} & 482 & \multirow{2}{*}{0.045$\pm$0.010} &\multirow{2}{*}{0.048$\pm$0.004} & \multirow{2}{*}{D: torsion of 1,2,3-DTT units and ending m-Th units}\\
        & 471 & & & \\
        467 & 455, 456\spc{b} & 0.085$\pm$0.010 & 0.044$\pm$0.004& D: ring deformation of 1,2,3-DTT units due to the stretching of S atoms\\
        \hline
        \hline
      \end{tabular}
      \begin{tablenotes}
        \small
        \item (i)\spc{a}The whole spectrum is normalized at 1525 cm$^{-1}$; \spc{b}Essentially degenerate modes;\\
        (ii) L: local, D: delocalized; 1, 2, 3 and 4 label the order for DPP or DTT unit displayed in Figure \ref{fig6};\\(iii)C$_t$, C$_s$, C$_p$ are ternary, secondary and primary carbon atom, respectively; b-C: bridgehead carbon of polycyclic rings; m-Th: monomeric thiophene ring.
      \end{tablenotes}
  \end{threeparttable}
  
\end{table}
\end{landscape}
\restoregeometry

\begin{center}
    \centering
    \includegraphics[width = \textwidth]{Figure6.pdf}
     \captionof{figure}{Vector diagrams for the Raman modes of a symmetrically truncated DPP-DTT trimer in geometry 1. (Raman modes for geometry 2 are shown in Figure~S8.) Alkyl side chains are substituted with methyl groups for computational efficiency. Atomic color scheme: gray = carbon, white = hydrogen, yellow = sulfur, blue = nitrogen, red = oxygen. The blue and red numbers at the top indicate the indices of the DPP and DTT units, respectively. Atomic displacements are indicated by the arrows.}
    \label{fig6}
\end{center}


\section{DISCUSSION}

Electron push-pull polymers, unlike conjugated homopolymers,  exhibit strong charge-transfer character between the electron sufficient and deficient unit within chains. The J-type spectral features are likely enhanced by the in-phase electron and hole integrals, and are prevalently shown in other types of electron push-pull polymers.\cite{banerji2012breaking, li2018low, rolczynski2012ultrafast, tautz2012structural, vezie2016exploring} As further demonstrated by Chang \textit{et al}., the dominant interchain charge transfer interaction in DPP-based systems could lead to abnormal red-shifted H-type aggregate behavior, when the electron and hole transfer integral is out-of-phase, which has been shown experimentally.\cite{li2019impact} While it can be established that the sign of charge-transfer integrals is sensitive to the donor/acceptor stacking arrangement, i.e., $\pi-\pi$ stacking geometry and distance,\cite{yamagata2012designing} the investigation of the stacking arrangement on a molecular/atomic level requires rigorous two-dimensional solid-state nuclear magnetic resonance spectroscopic studies, which to date have been limited for push-pull polymers especially in thin films.\cite{lo2018every, seifrid2020insight, chaudhari2017donor, melnyk2017macroscopic} Of particular relevance, Chaudhari \textit{et al}. showed that DPP-DTT polymer aggregates adopt a geometry where the donor and acceptor units between chains are alternating stacked when the films are prepared by spin coating. However, they also point out that segregated donor-on-donor or acceptor-on-acceptor stacking arrangement might have slightly lower energy compared to the slip-on geometry. Such stacking order could be achieved when certain processing conditions are met. 

Under the influence of solution concentration, the resonance Raman spectra displayed the most changes in the low-frequency range, where most of the contributions originate from the stretching and torsional modes of the DTT unit. Such conformational disorder stems from the rotational invariance of the mesogenic groups (i.e. thienothiophene component), and the favorable lowest-energy backbone conformation could be adopted easily.\cite{mcculloch2006liquid} Indeed, previous DFT calculations performed on DPP-DTT have shown that the different conformations of the DTT unit (e.g. different twisting angles between the thienothiophene and monomeric thiophene rings) have more minor energy difference, compared to that of the DPP unit, implying that the disorder source possibly originates from the former.\cite{chaudhari2017donor} In contrast, the DPP units are highly planarized and rigid, verified by the localized vibrational modes seen in Fig.~\ref{fig5}d in the high-frequency region. Note that whether the Raman intensities increase, decrease, or remain constant depends on the nature of each vibrational mode; some of the thienothiophene ring deformations along the chain backbone will aid in dispersing the exciton wave, while specific DPP ring deformation modes will lead to exciton wave localization, perpendicular to the polymer chain backbone. This direction dependency reflects the tensor aspects of the Raman polarizability. The large repeat unit will have not only diagonal elements but also off-diagonal elements in the polarizability expression.

Current studies do not allow us to quantitatively determine the spatial correlation of site energies, $\beta$, which can be quantified by the $I_{0-0}/I_{0-1}$ PL ratio\cite{paquin2013two}, specifically, due to limitations in the PL detection range. However, as mentioned earlier, the 0-0 peaks become more red-shifted and narrowed with increasing concentration. Furthermore, from the resonance Raman spectra, we are able to conclude that a more dispersed exciton wave is achieved along the polymer chain backbone with increasing concentration. Therefore, it is reasonable to deduce that the polymer chain backbone is more planarized when the processing concentrations increase. Combining both observations, we formulate the hypothesis that the more dispersive exciton wave along the polymer chain leads to a greater extent of spatial correlations of energies, which was described in the J-aggregates as shown by Knapp\cite{knapp1984lineshapes} and Knoester\cite{knoester1993nonlinear}. It is worth noting that the spatial correlation function of site energies should be two-dimensional, both along the polymer chain backbone and across the chains in the $\pi$-stack, which was demonstrated in P3HT by Spano \textit{et al.}\cite{spano2014h, paquin2013two, hestand2018expanded}. Specifically, in the high $M_w$ P3HT, the extent of exciton coherences along (across) the chains are higher (lower) than that of low $M_w$ P3HTs.\cite{paquin2013two} To accurately account for the exciton coherences of DPP-DTT, rigorous calculations of the two-dimensional correlation functions and a larger range of PL measurements are needed\cite{paquin2013two}.


\section{CONCLUSIONS}

We demonstrate that the exciton in push-pull polymers is more delocalized in films cast from solutions in which the concentration surpasses the viscosity threshold for gelation. We hypothesize that this phenomenon is attributed to enhanced chain backbone planarization, as indicated by resonance Raman spectroscopy and DFT calculations. Analysis of the absorption and steady-state PL spectra is consistent with a more highly delocalized exciton along the chain backbone, and more chromophores uncoupled to photophysical aggregates are also formed above the gel formation concentration. The contributions to the transient absorption spectra at the timescale of a few picoseconds are likely two-fold: derivative-like spectral line shapes due to accumulating photogenerated charges at the aggregate/non-aggregate interfaces\cite{paquin2015multi} and direct excited-state absorption from the non-aggregate chromophores. %Herein, w
We demonstrate the importance of understanding the short-range polymer chain order and excitonic interactions. Manipulating such short-range length scales could further our understanding in preaggregate assembly and favorably accelerate the development of next-generation organic optoelectronics.


\section{EXPERIMENTAL METHODS}

\textbf{Sample Preparation.} DPP-DTT (M$_\mathrm{w}$ = 290,000 g mol$^{-1}$, dispersity = 2) was purchased from Ossila Limited. For sample preparation, a stock solution of 10 g/L DPP-DTT in chlorobenzene (anhydrous, Sigma-Aldrich) was prepared by heating at 100 $\degree$C for around 4 hours, followed by heating at 60 $\degree$C overnight. Then solutions with lower concentrations are diluted from the stock solutions. The DPP-DTT thin films are prepared by wire-bar coating the solutions of different concentrations on fused-silica substrates at 56 $\degree$C, followed by annealing for 10 min.  

% \textbf{Viscosity Measurement.} The viscosity of the DPP-DTT-Chlorobenzene solutions was measured using a microVISC viscometer (RheoSense, Inc). The chamber can be heated up to 56 $\degree$C.

\textbf{Vis-NIR Absorption Spectroscopy.} The Vis-NIR absorption measurements were performed using the Cary 5000 UV-Vis-NIR spectroscopy. 

\textbf{Photoluminescence Spectroscopy} The steady-state photoluminescence spectroscopy is performed using the inVia Renishaw Spectrometer in the back-scattering configuration. The samples are illuminated by a 785 nm red laser.
	

\textbf{Ultrafast Transient Absorption Spectroscopy.} The transient absorption measurements are performed using an ultrafast laser system (Pharos Model PH1-20-02-10, Light Conversion). Tunable wavelengths are generated using a laser fundamental of 1030 nm at 100 kHz repetition rate. The integrated transient absorption was measured in a commercial setup (Light Conversion Hera). The wavelengths of the pump can be tuned from 360 to 2600 nm by feeding 10W laser output to a commercial optical parametric amplifier (Orpheus, Light Conversion, Lithuania) while probe beam is generated by sending 2 W to a sapphire crystal to obtain a single-filament white-light continuum in the spectral range of ~490-1060 nm. The probe beam was collected by an imaging spectrograph (Shamrock 193i, Andor Technology Ltd., U.K.) coupled with a multichannel detector (256 pixels, 200-1100 nm wavelength range) after transmitting through the sample. All the samples were measured in a homemade vacuum chamber.

\textbf{Resonance Raman Spectroscopy.} The resonance Raman spectra are measured using the inVia Renishaw Raman spectrometer, where the samples are excited with 488 nm laser with a back-scattering configuration.

\textbf{Quantum Chemistry Calculations.} Density functional theory (DFT) and time-dependent DFT (TDDFT) calculations were performed at the LC-$\omega$HPBE/6-311G(d) level of theory using the Gaussian 16 Rev. A.03 software suite.\cite{gaussian16} Empirical gap tuning was performed for the two oligomer geometries following the method of Sun et al.,\cite{sun2016tuning, henderson2009tuning, vreven2006qmmm, vydrov2006functionals, vydrov2007functionals} obtaining converged range-separation parameters of $\omega_{1}$ = 0.1295 (for G1) and $\omega_{2}$ = 0.1216 (G2). Following optimization, vibrational frequency analysis was performed on the oligomer, with vibrational scaling factors of 0.995 (G1) and 0.968 (G2) applied to the calculated Raman frequencies. The Raman activities were converted to intensities consistent with prior literature;\cite{polavarapu1990raman, keresztury1993raman} further details are available in the Supplementary Information. A TDDFT calculation was performed to obtain the excitation wavelength, 439 nm (22780 cm$^{-1}$), corresponding to the wavelength used in the associated experimental spectroscopy.


%%%%%%%%%%%%%%%%%%%%%%%%%%%%%%%%%%%%%%%%%%%%%%%%%%%%%%%%%%%%%%%%%%%%%
%% The "Acknowledgement" section can be given in all manuscript
%% classes.  This should be given within the "acknowledgement."
%% environment, which will make the correct section or running title.
%%%%%%%%%%%%%%%%%%%%%%%%%%%%%%%%%%%%%%%%%%%%%%%%%%%%%%%%%%%%%%%%%%%%%
\begin{acknowledgement}
C.S.A.\ acknowledges support from the National Science Foundation (Grant DMR-1729737) and from the School of Chemistry and Biochemistry and the College of Science of the Georgia Institute of Technology for startup support. E.R. appreciates support associated with Carl Robert Anderson Chair funds at Lehigh University. The authors also acknowledge the National Science Foundation Grant Nos. 1922111 and 1922174, DMREF: Collaborative Research: Achieving Multicomponent Active Materials through Synergistic Combinatorial, Informatics-enabled Materials Discovery for support. This work was performed in part at the Georgia Tech Institute for Electronics and Nanotechnology, a member of the National Nanotechnology Coordinated Infrastructure (NNCI), which is supported by the National Science Foundation (ECCS-2025462). We further acknowledge the University of Kentucky Center for Computational Sciences and Information Technology Services Research Computing for their fantastic support and collaboration and use of the Lipscomb Compute Cluster and associated research computing resources.  The authors thank Hongmo Li, Mark Weber, Marlow Durbin, and Dr.\,Natalie Stingelin for fruitful discussions on solution-state polymer conformations. 


% Use of the Stanford Synchrotron Radiation Lightsource, SLAC National Accelerator Laboratory, is supported by the U.S. Department of Energy, Office of Science, Office of Basic Energy Sciences under Contract No. DE-AC02-76SF00515.

\end{acknowledgement}


% \begin{ORCID}
% \item Yulong Zheng: \orcid{0000-0001-5136-1971}
% \end{ORCID}

%%%%%%%%%%%%%%%%%%%%%%%%%%%%%%%%%%%%%%%%%%%%%%%%%%%%%%%%%%%%%%%%%%%%%
%% The same is true for Supporting Information, which should use the
%% suppinfo environment.
%%%%%%%%%%%%%%%%%%%%%%%%%%%%%%%%%%%%%%%%%%%%%%%%%%%%%%%%%%%%%%%%%%%%%
\begin{suppinfo}

The Supporting Information is available free of charge at (will be inserted by editor...)

    Complete modified FC analysis of the absorption spectra, Gaussian fit of the steady-state PL spectra, DSC thermograms, TA maps measured under the lowest and highest fluences, GSB and ESA signal magnitude dependences upon fluences, double exponential decay fit for the kinetics traces.



\end{suppinfo}

%%%%%%%%%%%%%%%%%%%%%%%%%%%%%%%%%%%%%%%%%%%%%%%%%%%%%%%%%%%%%%%%%%%%%
%% The appropriate \bibliography command should be placed here.
%% Notice that the class file automatically sets \bibliographystyle
%% and also names the section correctly.
%%%%%%%%%%%%%%%%%%%%%%%%%%%%%%%%%%%%%%%%%%%%%%%%%%%%%%%%%%%%%%%%%%%%%
%\bibliography{DPPDTT}

\providecommand{\latin}[1]{#1}
\makeatletter
\providecommand{\doi}
  {\begingroup\let\do\@makeother\dospecials
  \catcode`\{=1 \catcode`\}=2 \doi@aux}
\providecommand{\doi@aux}[1]{\endgroup\texttt{#1}}
\makeatother
\providecommand*\mcitethebibliography{\thebibliography}
\csname @ifundefined\endcsname{endmcitethebibliography}
  {\let\endmcitethebibliography\endthebibliography}{}
\begin{mcitethebibliography}{68}
\providecommand*\natexlab[1]{#1}
\providecommand*\mciteSetBstSublistMode[1]{}
\providecommand*\mciteSetBstMaxWidthForm[2]{}
\providecommand*\mciteBstWouldAddEndPuncttrue
  {\def\EndOfBibitem{\unskip.}}
\providecommand*\mciteBstWouldAddEndPunctfalse
  {\let\EndOfBibitem\relax}
\providecommand*\mciteSetBstMidEndSepPunct[3]{}
\providecommand*\mciteSetBstSublistLabelBeginEnd[3]{}
\providecommand*\EndOfBibitem{}
\mciteSetBstSublistMode{f}
\mciteSetBstMaxWidthForm{subitem}{(\alph{mcitesubitemcount})}
\mciteSetBstSublistLabelBeginEnd
  {\mcitemaxwidthsubitemform\space}
  {\relax}
  {\relax}

\bibitem[Frenkel(1931)]{frenkel1931transformation}
Frenkel,~J. On the transformation of light into heat in solids. I. \emph{Phys.
  Rev.} \textbf{1931}, \emph{37}, 17\relax
\mciteBstWouldAddEndPuncttrue
\mciteSetBstMidEndSepPunct{\mcitedefaultmidpunct}
{\mcitedefaultendpunct}{\mcitedefaultseppunct}\relax
\EndOfBibitem
\bibitem[Spano(2006)]{spano2006excitons}
Spano,~F.~C. Excitons in conjugated oligomer aggregates, films, and crystals.
  \emph{Annu. Rev. Phys. Chem.} \textbf{2006}, \emph{57}, 217--243\relax
\mciteBstWouldAddEndPuncttrue
\mciteSetBstMidEndSepPunct{\mcitedefaultmidpunct}
{\mcitedefaultendpunct}{\mcitedefaultseppunct}\relax
\EndOfBibitem
\bibitem[Spano(2005)]{spano2005modeling}
Spano,~F.~C. Modeling disorder in polymer aggregates: The optical spectroscopy
  of regioregular poly (3-hexylthiophene) thin films. \emph{The Journal of
  chemical physics} \textbf{2005}, \emph{122}, 234701\relax
\mciteBstWouldAddEndPuncttrue
\mciteSetBstMidEndSepPunct{\mcitedefaultmidpunct}
{\mcitedefaultendpunct}{\mcitedefaultseppunct}\relax
\EndOfBibitem
\bibitem[Spano(2006)]{spano2006absorption}
Spano,~F.~C. Absorption in regio-regular poly (3-hexyl) thiophene thin films:
  Fermi resonances, interband coupling and disorder. \emph{Chemical Physics}
  \textbf{2006}, \emph{325}, 22--35\relax
\mciteBstWouldAddEndPuncttrue
\mciteSetBstMidEndSepPunct{\mcitedefaultmidpunct}
{\mcitedefaultendpunct}{\mcitedefaultseppunct}\relax
\EndOfBibitem
\bibitem[Clark \latin{et~al.}(2007)Clark, Silva, Friend, and
  Spano]{Clark2007PhysRevLett}
Clark,~J.; Silva,~C.; Friend,~R.~H.; Spano,~F.~C. Role of Intermolecular
  Coupling in the Photophysics of Disordered Organic Semiconductors: Aggregate
  Emission in Regioregular Polythiophene. \emph{Phys. Rev. Lett.}
  \textbf{2007}, \emph{98}, 206406\relax
\mciteBstWouldAddEndPuncttrue
\mciteSetBstMidEndSepPunct{\mcitedefaultmidpunct}
{\mcitedefaultendpunct}{\mcitedefaultseppunct}\relax
\EndOfBibitem
\bibitem[Spano \latin{et~al.}(2009)Spano, Clark, Silva, and
  Friend]{spano2009determining}
Spano,~F.~C.; Clark,~J.; Silva,~C.; Friend,~R.~H. Determining exciton coherence
  from the photoluminescence spectral line shape in poly (3-hexylthiophene)
  thin films. \emph{The Journal of chemical physics} \textbf{2009}, \emph{130},
  074904\relax
\mciteBstWouldAddEndPuncttrue
\mciteSetBstMidEndSepPunct{\mcitedefaultmidpunct}
{\mcitedefaultendpunct}{\mcitedefaultseppunct}\relax
\EndOfBibitem
\bibitem[Clark \latin{et~al.}(2009)Clark, Chang, Spano, Friend, and
  Silva]{clark2009determining}
Clark,~J.; Chang,~J.-F.; Spano,~F.~C.; Friend,~R.~H.; Silva,~C. Determining
  exciton bandwidth and film microstructure in polythiophene films using linear
  absorption spectroscopy. \emph{Applied Physics Letters} \textbf{2009},
  \emph{94}, 117\relax
\mciteBstWouldAddEndPuncttrue
\mciteSetBstMidEndSepPunct{\mcitedefaultmidpunct}
{\mcitedefaultendpunct}{\mcitedefaultseppunct}\relax
\EndOfBibitem
\bibitem[Hestand and Spano(2015)Hestand, and Spano]{hestand2015interference}
Hestand,~N.~J.; Spano,~F.~C. Interference between Coulombic and CT-mediated
  couplings in molecular aggregates: H-to J-aggregate transformation in
  perylene-based $\pi$-stacks. \emph{The Journal of chemical physics}
  \textbf{2015}, \emph{143}, 244707\relax
\mciteBstWouldAddEndPuncttrue
\mciteSetBstMidEndSepPunct{\mcitedefaultmidpunct}
{\mcitedefaultendpunct}{\mcitedefaultseppunct}\relax
\EndOfBibitem
\bibitem[Hestand and Spano(2017)Hestand, and Spano]{hestand2017molecular}
Hestand,~N.~J.; Spano,~F.~C. Molecular aggregate photophysics beyond the Kasha
  model: novel design principles for organic materials. \emph{Accounts of
  chemical research} \textbf{2017}, \emph{50}, 341--350\relax
\mciteBstWouldAddEndPuncttrue
\mciteSetBstMidEndSepPunct{\mcitedefaultmidpunct}
{\mcitedefaultendpunct}{\mcitedefaultseppunct}\relax
\EndOfBibitem
\bibitem[Hestand and Spano(2018)Hestand, and Spano]{hestand2018expanded}
Hestand,~N.~J.; Spano,~F.~C. Expanded theory of H-and J-molecular aggregates:
  the effects of vibronic coupling and intermolecular charge transfer.
  \emph{Chemical Reviews} \textbf{2018}, \emph{118}, 7069--7163\relax
\mciteBstWouldAddEndPuncttrue
\mciteSetBstMidEndSepPunct{\mcitedefaultmidpunct}
{\mcitedefaultendpunct}{\mcitedefaultseppunct}\relax
\EndOfBibitem
\bibitem[Zhong \latin{et~al.}(2020)Zhong, Bialas, and Spano]{zhong2020unusual}
Zhong,~C.; Bialas,~D.; Spano,~F.~C. Unusual Non-Kasha Photophysical Behavior of
  Aggregates of Push--Pull Donor--Acceptor Chromophores. \emph{The Journal of
  Physical Chemistry C} \textbf{2020}, \emph{124}, 2146--2159\relax
\mciteBstWouldAddEndPuncttrue
\mciteSetBstMidEndSepPunct{\mcitedefaultmidpunct}
{\mcitedefaultendpunct}{\mcitedefaultseppunct}\relax
\EndOfBibitem
\bibitem[Balooch~Qarai \latin{et~al.}(2020)Balooch~Qarai, Chang, and
  Spano]{balooch2020vibronic}
Balooch~Qarai,~M.; Chang,~X.; Spano,~F. Vibronic exciton model for low bandgap
  donor--acceptor polymers. \emph{The Journal of Chemical Physics}
  \textbf{2020}, \emph{153}, 244901\relax
\mciteBstWouldAddEndPuncttrue
\mciteSetBstMidEndSepPunct{\mcitedefaultmidpunct}
{\mcitedefaultendpunct}{\mcitedefaultseppunct}\relax
\EndOfBibitem
\bibitem[Chang \latin{et~al.}(2021)Chang, Balooch~Qarai, and
  Spano]{chang2021hj}
Chang,~X.; Balooch~Qarai,~M.; Spano,~F.~C. HJ-aggregates of
  donor--acceptor--donor oligomers and polymers. \emph{The Journal of Chemical
  Physics} \textbf{2021}, \emph{155}, 034905\relax
\mciteBstWouldAddEndPuncttrue
\mciteSetBstMidEndSepPunct{\mcitedefaultmidpunct}
{\mcitedefaultendpunct}{\mcitedefaultseppunct}\relax
\EndOfBibitem
\bibitem[Kang \latin{et~al.}(2013)Kang, Yun, Chung, Kwon, and
  Kim]{kang2013record}
Kang,~I.; Yun,~H.-J.; Chung,~D.~S.; Kwon,~S.-K.; Kim,~Y.-H. Record high hole
  mobility in polymer semiconductors via side-chain engineering. \emph{Journal
  of the American Chemical Society} \textbf{2013}, \emph{135},
  14896--14899\relax
\mciteBstWouldAddEndPuncttrue
\mciteSetBstMidEndSepPunct{\mcitedefaultmidpunct}
{\mcitedefaultendpunct}{\mcitedefaultseppunct}\relax
\EndOfBibitem
\bibitem[Li \latin{et~al.}(2012)Li, Zhao, Tan, Guo, Di, Yu, Liu, Lin, Lim,
  Zhou, \latin{et~al.} others]{li2012stable}
Li,~J.; Zhao,~Y.; Tan,~H.~S.; Guo,~Y.; Di,~C.-A.; Yu,~G.; Liu,~Y.; Lin,~M.;
  Lim,~S.~H.; Zhou,~Y., \latin{et~al.}  A stable solution-processed polymer
  semiconductor with record high-mobility for printed transistors.
  \emph{Scientific Reports} \textbf{2012}, \emph{2}, 1--9\relax
\mciteBstWouldAddEndPuncttrue
\mciteSetBstMidEndSepPunct{\mcitedefaultmidpunct}
{\mcitedefaultendpunct}{\mcitedefaultseppunct}\relax
\EndOfBibitem
\bibitem[Paquin \latin{et~al.}(2013)Paquin, Yamagata, Hestand, Sakowicz,
  B{\'e}rub{\'e}, C{\^o}t{\'e}, Reynolds, Haque, Stingelin, Spano,
  \latin{et~al.} others]{paquin2013two}
Paquin,~F.; Yamagata,~H.; Hestand,~N.~J.; Sakowicz,~M.; B{\'e}rub{\'e},~N.;
  C{\^o}t{\'e},~M.; Reynolds,~L.~X.; Haque,~S.~A.; Stingelin,~N.; Spano,~F.~C.,
  \latin{et~al.}  Two-dimensional spatial coherence of excitons in
  semicrystalline polymeric semiconductors: Effect of molecular weight.
  \emph{Physical Review B} \textbf{2013}, \emph{88}, 155202\relax
\mciteBstWouldAddEndPuncttrue
\mciteSetBstMidEndSepPunct{\mcitedefaultmidpunct}
{\mcitedefaultendpunct}{\mcitedefaultseppunct}\relax
\EndOfBibitem
\bibitem[Paquin \latin{et~al.}(2015)Paquin, Rivnay, Salleo, Stingelin, and
  Silva-Acu{\~n}a]{paquin2015multi}
Paquin,~F.; Rivnay,~J.; Salleo,~A.; Stingelin,~N.; Silva-Acu{\~n}a,~C.
  Multi-phase microstructures drive exciton dissociation in neat
  semicrystalline polymeric semiconductors. \emph{Journal of Materials
  Chemistry C} \textbf{2015}, \emph{3}, 10715--10722\relax
\mciteBstWouldAddEndPuncttrue
\mciteSetBstMidEndSepPunct{\mcitedefaultmidpunct}
{\mcitedefaultendpunct}{\mcitedefaultseppunct}\relax
\EndOfBibitem
\bibitem[K\"{o}hler \latin{et~al.}(2012)K\"{o}hler, Hoffmann, and
  B\"{a}ssler]{koehler2012order}
K\"{o}hler,~A.; Hoffmann,~S.~T.; B\"{a}ssler,~H. An order--disorder transition
  in the conjugated polymer MEH-PPV. \emph{Journal of the American Chemical
  Society} \textbf{2012}, \emph{134}, 11594--11601\relax
\mciteBstWouldAddEndPuncttrue
\mciteSetBstMidEndSepPunct{\mcitedefaultmidpunct}
{\mcitedefaultendpunct}{\mcitedefaultseppunct}\relax
\EndOfBibitem
\bibitem[Koch \latin{et~al.}(2013)Koch, Rivnay, Foster, M{\"u}ller, Downing,
  Buchaca-Domingo, Westacott, Yu, Yuan, Baklar, \latin{et~al.}
  others]{koch2013impact}
Koch,~F. P.~V.; Rivnay,~J.; Foster,~S.; M{\"u}ller,~C.; Downing,~J.~M.;
  Buchaca-Domingo,~E.; Westacott,~P.; Yu,~L.; Yuan,~M.; Baklar,~M.,
  \latin{et~al.}  The impact of molecular weight on microstructure and charge
  transport in semicrystalline polymer semiconductors--poly (3-hexylthiophene),
  a model study. \emph{Progress in Polymer Science} \textbf{2013}, \emph{38},
  1978--1989\relax
\mciteBstWouldAddEndPuncttrue
\mciteSetBstMidEndSepPunct{\mcitedefaultmidpunct}
{\mcitedefaultendpunct}{\mcitedefaultseppunct}\relax
\EndOfBibitem
\bibitem[Brinkmann and Rannou(2007)Brinkmann, and Rannou]{brinkmann2007effect}
Brinkmann,~M.; Rannou,~P. Effect of molecular weight on the structure and
  morphology of oriented thin films of regioregular poly (3-hexylthiophene)
  grown by directional epitaxial solidification. \emph{Advanced Functional
  Materials} \textbf{2007}, \emph{17}, 101--108\relax
\mciteBstWouldAddEndPuncttrue
\mciteSetBstMidEndSepPunct{\mcitedefaultmidpunct}
{\mcitedefaultendpunct}{\mcitedefaultseppunct}\relax
\EndOfBibitem
\bibitem[Brinkmann and Rannou(2009)Brinkmann, and
  Rannou]{brinkmann2009molecular}
Brinkmann,~M.; Rannou,~P. Molecular weight dependence of chain packing and
  semicrystalline structure in oriented films of regioregular poly
  (3-hexylthiophene) revealed by high-resolution transmission electron
  microscopy. \emph{Macromolecules} \textbf{2009}, \emph{42}, 1125--1130\relax
\mciteBstWouldAddEndPuncttrue
\mciteSetBstMidEndSepPunct{\mcitedefaultmidpunct}
{\mcitedefaultendpunct}{\mcitedefaultseppunct}\relax
\EndOfBibitem
\bibitem[Paquin \latin{et~al.}(2011)Paquin, Latini, Sakowicz, Karsenti, Wang,
  Beljonne, Stingelin, and Silva]{paquin2011charge}
Paquin,~F.; Latini,~G.; Sakowicz,~M.; Karsenti,~P.-L.; Wang,~L.; Beljonne,~D.;
  Stingelin,~N.; Silva,~C. Charge separation in semicrystalline polymeric
  semiconductors by photoexcitation: is the mechanism intrinsic or extrinsic?
  \emph{Physical Review Letters} \textbf{2011}, \emph{106}, 197401\relax
\mciteBstWouldAddEndPuncttrue
\mciteSetBstMidEndSepPunct{\mcitedefaultmidpunct}
{\mcitedefaultendpunct}{\mcitedefaultseppunct}\relax
\EndOfBibitem
\bibitem[Reid \latin{et~al.}(2012)Reid, Malik, Latini, Dayal, Kopidakis, Silva,
  Stingelin, and Rumbles]{reid2012influence}
Reid,~O.~G.; Malik,~J. A.~N.; Latini,~G.; Dayal,~S.; Kopidakis,~N.; Silva,~C.;
  Stingelin,~N.; Rumbles,~G. The influence of solid-state microstructure on the
  origin and yield of long-lived photogenerated charge in neat semiconducting
  polymers. \emph{Journal of Polymer Science Part B: Polymer Physics}
  \textbf{2012}, \emph{50}, 27--37\relax
\mciteBstWouldAddEndPuncttrue
\mciteSetBstMidEndSepPunct{\mcitedefaultmidpunct}
{\mcitedefaultendpunct}{\mcitedefaultseppunct}\relax
\EndOfBibitem
\bibitem[Noriega \latin{et~al.}(2013)Noriega, Rivnay, Vandewal, Koch,
  Stingelin, Smith, Toney, and Salleo]{noriega2013general}
Noriega,~R.; Rivnay,~J.; Vandewal,~K.; Koch,~F.~P.; Stingelin,~N.; Smith,~P.;
  Toney,~M.~F.; Salleo,~A. A general relationship between disorder, aggregation
  and charge transport in conjugated polymers. \emph{Nature Materials}
  \textbf{2013}, \emph{12}, 1038--1044\relax
\mciteBstWouldAddEndPuncttrue
\mciteSetBstMidEndSepPunct{\mcitedefaultmidpunct}
{\mcitedefaultendpunct}{\mcitedefaultseppunct}\relax
\EndOfBibitem
\bibitem[Sirringhaus(2014)]{sirringhaus201425th}
Sirringhaus,~H. 25th anniversary article: organic field-effect transistors: the
  path beyond amorphous silicon. \emph{Advanced Materials} \textbf{2014},
  \emph{26}, 1319--1335\relax
\mciteBstWouldAddEndPuncttrue
\mciteSetBstMidEndSepPunct{\mcitedefaultmidpunct}
{\mcitedefaultendpunct}{\mcitedefaultseppunct}\relax
\EndOfBibitem
\bibitem[Guo \latin{et~al.}(2014)Guo, Lee, Schaller, Zuo, Lee, Luo, Gao, and
  Huang]{guo2014relationship}
Guo,~Z.; Lee,~D.; Schaller,~R.~D.; Zuo,~X.; Lee,~B.; Luo,~T.; Gao,~H.;
  Huang,~L. Relationship between interchain interaction, exciton
  delocalization, and charge separation in low-bandgap copolymer blends.
  \emph{Journal of the American Chemical Society} \textbf{2014}, \emph{136},
  10024--10032\relax
\mciteBstWouldAddEndPuncttrue
\mciteSetBstMidEndSepPunct{\mcitedefaultmidpunct}
{\mcitedefaultendpunct}{\mcitedefaultseppunct}\relax
\EndOfBibitem
\bibitem[Steyrleuthner \latin{et~al.}(2012)Steyrleuthner, Schubert, Howard,
  Klaumu\"{u}nzer, Schilling, Chen, Saalfrank, Laquai, Facchetti, and
  Neher]{steyrleuthner2012aggregation}
Steyrleuthner,~R.; Schubert,~M.; Howard,~I.; Klaumu\"{u}nzer,~B.;
  Schilling,~K.; Chen,~Z.; Saalfrank,~P.; Laquai,~F.; Facchetti,~A.; Neher,~D.
  Aggregation in a high-mobility n-type low-bandgap copolymer with implications
  on semicrystalline morphology. \emph{Journal of the American Chemical
  Society} \textbf{2012}, \emph{134}, 18303--18317\relax
\mciteBstWouldAddEndPuncttrue
\mciteSetBstMidEndSepPunct{\mcitedefaultmidpunct}
{\mcitedefaultendpunct}{\mcitedefaultseppunct}\relax
\EndOfBibitem
\bibitem[G{\'e}linas \latin{et~al.}(2013)G{\'e}linas, Kirkpatrick, Howard,
  Johnson, Wilson, Pace, Friend, and Silva]{gelinas2013recombination}
G{\'e}linas,~S.; Kirkpatrick,~J.; Howard,~I.~A.; Johnson,~K.; Wilson,~M.~W.;
  Pace,~G.; Friend,~R.~H.; Silva,~C. Recombination dynamics of charge pairs in
  a push--pull polyfluorene-derivative. \emph{The Journal of Physical Chemistry
  B} \textbf{2013}, \emph{117}, 4649--4653\relax
\mciteBstWouldAddEndPuncttrue
\mciteSetBstMidEndSepPunct{\mcitedefaultmidpunct}
{\mcitedefaultendpunct}{\mcitedefaultseppunct}\relax
\EndOfBibitem
\bibitem[Pop \latin{et~al.}(2016)Pop, Lewis, and Amabilino]{pop2016solid}
Pop,~F.; Lewis,~W.; Amabilino,~D.~B. Solid state supramolecular structure of
  diketopyrrolopyrrole chromophores: correlating stacking geometry with visible
  light absorption. \emph{CrystEngComm} \textbf{2016}, \emph{18},
  8933--8943\relax
\mciteBstWouldAddEndPuncttrue
\mciteSetBstMidEndSepPunct{\mcitedefaultmidpunct}
{\mcitedefaultendpunct}{\mcitedefaultseppunct}\relax
\EndOfBibitem
\bibitem[Chaudhari \latin{et~al.}(2017)Chaudhari, Griffin, Broch, Lesage,
  Lemaur, Dudenko, Olivier, Sirringhaus, Emsley, and Grey]{chaudhari2017donor}
Chaudhari,~S.~R.; Griffin,~J.~M.; Broch,~K.; Lesage,~A.; Lemaur,~V.;
  Dudenko,~D.; Olivier,~Y.; Sirringhaus,~H.; Emsley,~L.; Grey,~C.~P.
  Donor--acceptor stacking arrangements in bulk and thin-film high-mobility
  conjugated polymers characterized using molecular modelling and MAS and
  surface-enhanced solid-state NMR spectroscopy. \emph{Chemical Science}
  \textbf{2017}, \emph{8}, 3126--3136\relax
\mciteBstWouldAddEndPuncttrue
\mciteSetBstMidEndSepPunct{\mcitedefaultmidpunct}
{\mcitedefaultendpunct}{\mcitedefaultseppunct}\relax
\EndOfBibitem
\bibitem[McRae and Kasha(1958)McRae, and Kasha]{mcrae1958enhancement}
McRae,~E.~G.; Kasha,~M. Enhancement of phosphorescence ability upon aggregation
  of dye molecules. \emph{The Journal of Chemical Physics} \textbf{1958},
  \emph{28}, 721--722\relax
\mciteBstWouldAddEndPuncttrue
\mciteSetBstMidEndSepPunct{\mcitedefaultmidpunct}
{\mcitedefaultendpunct}{\mcitedefaultseppunct}\relax
\EndOfBibitem
\bibitem[Yan \latin{et~al.}(1994)Yan, Rothberg, Papadimitrakopoulos, Galvin,
  and Miller]{yan1994spatially}
Yan,~M.; Rothberg,~L.; Papadimitrakopoulos,~F.; Galvin,~M.; Miller,~T.
  Spatially indirect excitons as primary photoexcitations in conjugated
  polymers. \emph{Physical Review Letters} \textbf{1994}, \emph{72}, 1104\relax
\mciteBstWouldAddEndPuncttrue
\mciteSetBstMidEndSepPunct{\mcitedefaultmidpunct}
{\mcitedefaultendpunct}{\mcitedefaultseppunct}\relax
\EndOfBibitem
\bibitem[Cacialli \latin{et~al.}(2002)Cacialli, Wilson, Michels, Daniel, Silva,
  Friend, Severin, Samor{\`\i}, Rabe, O'Connell, \latin{et~al.}
  others]{cacialli2002cyclodextrin}
Cacialli,~F.; Wilson,~J.~S.; Michels,~J.~J.; Daniel,~C.; Silva,~C.;
  Friend,~R.~H.; Severin,~N.; Samor{\`\i},~P.; Rabe,~J.~P.; O'Connell,~M.~J.,
  \latin{et~al.}  Cyclodextrin-threaded conjugated polyrotaxanes as insulated
  molecular wires with reduced interstrand interactions. \emph{Nature
  Materials} \textbf{2002}, \emph{1}, 160--164\relax
\mciteBstWouldAddEndPuncttrue
\mciteSetBstMidEndSepPunct{\mcitedefaultmidpunct}
{\mcitedefaultendpunct}{\mcitedefaultseppunct}\relax
\EndOfBibitem
\bibitem[Petrozza \latin{et~al.}(2008)Petrozza, Brovelli, Michels, Anderson,
  Friend, Silva, and Cacialli]{petrozza2008control}
Petrozza,~A.; Brovelli,~S.; Michels,~J.~J.; Anderson,~H.~L.; Friend,~R.~H.;
  Silva,~C.; Cacialli,~F. Control of Rapid Formation of Interchain Excited
  States in Sugar-Threaded Supramolecular Wires. \emph{Advanced Materials}
  \textbf{2008}, \emph{20}, 3218--3223\relax
\mciteBstWouldAddEndPuncttrue
\mciteSetBstMidEndSepPunct{\mcitedefaultmidpunct}
{\mcitedefaultendpunct}{\mcitedefaultseppunct}\relax
\EndOfBibitem
\bibitem[Mr{\'o}z \latin{et~al.}(2009)Mr{\'o}z, Perissinotto, Virgili, Gigli,
  Salerno, Frampton, Sforazzini, Anderson, and Lanzani]{mroz2009laser}
Mr{\'o}z,~M.~M.; Perissinotto,~S.; Virgili,~T.; Gigli,~G.; Salerno,~M.;
  Frampton,~M.~J.; Sforazzini,~G.; Anderson,~H.~L.; Lanzani,~G. Laser action
  from a sugar-threaded polyrotaxane. \emph{Applied Physics Letters}
  \textbf{2009}, \emph{95}, 031108\relax
\mciteBstWouldAddEndPuncttrue
\mciteSetBstMidEndSepPunct{\mcitedefaultmidpunct}
{\mcitedefaultendpunct}{\mcitedefaultseppunct}\relax
\EndOfBibitem
\bibitem[Cabanillas-Gonzalez \latin{et~al.}(2011)Cabanillas-Gonzalez, Grancini,
  and Lanzani]{cabanillas2011pump}
Cabanillas-Gonzalez,~J.; Grancini,~G.; Lanzani,~G. Pump-probe spectroscopy in
  organic semiconductors: monitoring fundamental processes of relevance in
  optoelectronics. \emph{Advanced Materials} \textbf{2011}, \emph{23},
  5468--5485\relax
\mciteBstWouldAddEndPuncttrue
\mciteSetBstMidEndSepPunct{\mcitedefaultmidpunct}
{\mcitedefaultendpunct}{\mcitedefaultseppunct}\relax
\EndOfBibitem
\bibitem[Mr{\'o}z \latin{et~al.}(2013)Mr{\'o}z, Sforazzini, Zhong, Wong,
  Anderson, Lanzani, and Cabanillas-Gonzalez]{mroz2013amplified}
Mr{\'o}z,~M.~M.; Sforazzini,~G.; Zhong,~Y.; Wong,~K.~S.; Anderson,~H.~L.;
  Lanzani,~G.; Cabanillas-Gonzalez,~J. Amplified spontaneous emission in
  conjugated polyrotaxanes under quasi-cw pumping. \emph{Advanced Materials}
  \textbf{2013}, \emph{25}, 4347--4351\relax
\mciteBstWouldAddEndPuncttrue
\mciteSetBstMidEndSepPunct{\mcitedefaultmidpunct}
{\mcitedefaultendpunct}{\mcitedefaultseppunct}\relax
\EndOfBibitem
\bibitem[Wang \latin{et~al.}(2022)Wang, Gao, He, Shi, Deng, Han, Ye, and
  Geng]{wang2022unraveling}
Wang,~Z.; Gao,~M.; He,~C.; Shi,~W.; Deng,~Y.; Han,~Y.; Ye,~L.; Geng,~Y.
  Unraveling the Molar Mass Dependence of Shearing-Induced Aggregation
  Structure of a High-Mobility Polymer Semiconductor. \emph{Advanced Materials}
  \textbf{2022}, 2108255\relax
\mciteBstWouldAddEndPuncttrue
\mciteSetBstMidEndSepPunct{\mcitedefaultmidpunct}
{\mcitedefaultendpunct}{\mcitedefaultseppunct}\relax
\EndOfBibitem
\bibitem[Venkatesh \latin{et~al.}(2023)Venkatesh, Zheng, Liu, Zhao, Silva,
  Takacs, Grover, Meredith, and Reichmanis]{venkatesh2023overlap}
Venkatesh,~R.; Zheng,~Y.; Liu,~A.~L.; Zhao,~H.; Silva,~C.; Takacs,~C.~J.;
  Grover,~M.; Meredith,~C.; Reichmanis,~E. Overlap concentration generates
  optimum device performance for DPP-based conjugated polymers. \emph{Organic
  Electronics} \textbf{2023}, 106779\relax
\mciteBstWouldAddEndPuncttrue
\mciteSetBstMidEndSepPunct{\mcitedefaultmidpunct}
{\mcitedefaultendpunct}{\mcitedefaultseppunct}\relax
\EndOfBibitem
\bibitem[Venkatesh \latin{et~al.}(2021)Venkatesh, Zheng, Viersen, Liu, Silva,
  Grover, and Reichmanis]{venkatesh2021data}
Venkatesh,~R.; Zheng,~Y.; Viersen,~C.; Liu,~A.; Silva,~C.; Grover,~M.;
  Reichmanis,~E. Data Science Guided Experiments Identify Conjugated Polymer
  Solution Concentration as a Key Parameter in Device Performance. \emph{ACS
  Materials Letters} \textbf{2021}, \emph{3}, 1321--1327\relax
\mciteBstWouldAddEndPuncttrue
\mciteSetBstMidEndSepPunct{\mcitedefaultmidpunct}
{\mcitedefaultendpunct}{\mcitedefaultseppunct}\relax
\EndOfBibitem
\bibitem[Knapp(1984)]{knapp1984lineshapes}
Knapp,~E. Lineshapes of molecular aggregates, exchange narrowing and intersite
  correlation. \emph{Chemical Physics} \textbf{1984}, \emph{85}, 73--82\relax
\mciteBstWouldAddEndPuncttrue
\mciteSetBstMidEndSepPunct{\mcitedefaultmidpunct}
{\mcitedefaultendpunct}{\mcitedefaultseppunct}\relax
\EndOfBibitem
\bibitem[Knoester(1993)]{knoester1993nonlinear}
Knoester,~J. Nonlinear optical line shapes of disordered molecular aggregates:
  motional narrowing and the effect of intersite correlations. \emph{The
  Journal of chemical physics} \textbf{1993}, \emph{99}, 8466--8479\relax
\mciteBstWouldAddEndPuncttrue
\mciteSetBstMidEndSepPunct{\mcitedefaultmidpunct}
{\mcitedefaultendpunct}{\mcitedefaultseppunct}\relax
\EndOfBibitem
\bibitem[Li \latin{et~al.}(2019)Li, Balawi, Leenaers, Ning, Heintges,
  Marszalek, Pisula, Wienk, Meskers, Yi, \latin{et~al.} others]{li2019impact}
Li,~M.; Balawi,~A.~H.; Leenaers,~P.~J.; Ning,~L.; Heintges,~G.~H.;
  Marszalek,~T.; Pisula,~W.; Wienk,~M.~M.; Meskers,~S.~C.; Yi,~Y.,
  \latin{et~al.}  Impact of polymorphism on the optoelectronic properties of a
  low-bandgap semiconducting polymer. \emph{Nature communications}
  \textbf{2019}, \emph{10}, 1--11\relax
\mciteBstWouldAddEndPuncttrue
\mciteSetBstMidEndSepPunct{\mcitedefaultmidpunct}
{\mcitedefaultendpunct}{\mcitedefaultseppunct}\relax
\EndOfBibitem
\bibitem[Peng \latin{et~al.}(2022)Peng, Ye, and Ade]{peng2022understanding}
Peng,~Z.; Ye,~L.; Ade,~H. Understanding, quantifying, and controlling the
  molecular ordering of semiconducting polymers: from novices to experts and
  amorphous to perfect crystals. \emph{Materials Horizons} \textbf{2022},
  \relax
\mciteBstWouldAddEndPunctfalse
\mciteSetBstMidEndSepPunct{\mcitedefaultmidpunct}
{}{\mcitedefaultseppunct}\relax
\EndOfBibitem
\bibitem[Yamagata and Spano(2012)Yamagata, and Spano]{yamagata2012interplay}
Yamagata,~H.; Spano,~F.~C. Interplay between intrachain and interchain
  interactions in semiconducting polymer assemblies: The HJ-aggregate model.
  \emph{The Journal of chemical physics} \textbf{2012}, \emph{136},
  184901\relax
\mciteBstWouldAddEndPuncttrue
\mciteSetBstMidEndSepPunct{\mcitedefaultmidpunct}
{\mcitedefaultendpunct}{\mcitedefaultseppunct}\relax
\EndOfBibitem
\bibitem[Vezie \latin{et~al.}(2016)Vezie, Few, Meager, Pieridou, D{\"o}rling,
  Ashraf, Go{\~n}i, Bronstein, McCulloch, Hayes, \latin{et~al.}
  others]{vezie2016exploring}
Vezie,~M.~S.; Few,~S.; Meager,~I.; Pieridou,~G.; D{\"o}rling,~B.;
  Ashraf,~R.~S.; Go{\~n}i,~A.~R.; Bronstein,~H.; McCulloch,~I.; Hayes,~S.~C.,
  \latin{et~al.}  Exploring the origin of high optical absorption in conjugated
  polymers. \emph{Nature Materials} \textbf{2016}, \emph{15}, 746--753\relax
\mciteBstWouldAddEndPuncttrue
\mciteSetBstMidEndSepPunct{\mcitedefaultmidpunct}
{\mcitedefaultendpunct}{\mcitedefaultseppunct}\relax
\EndOfBibitem
\bibitem[Weiser(1992)]{weiser1992stark}
Weiser,~G. Stark effect of one-dimensional Wannier excitons in polydiacetylene
  single crystals. \emph{Physical Review B} \textbf{1992}, \emph{45},
  14076\relax
\mciteBstWouldAddEndPuncttrue
\mciteSetBstMidEndSepPunct{\mcitedefaultmidpunct}
{\mcitedefaultendpunct}{\mcitedefaultseppunct}\relax
\EndOfBibitem
\bibitem[Liess \latin{et~al.}(1997)Liess, Jeglinski, Vardeny, Ozaki, Yoshino,
  Ding, and Barton]{liess1997electroabsorption}
Liess,~M.; Jeglinski,~S.; Vardeny,~Z.; Ozaki,~M.; Yoshino,~K.; Ding,~Y.;
  Barton,~T. Electroabsorption spectroscopy of luminescent and nonluminescent
  $\pi$-conjugated polymers. \emph{Physical Review B} \textbf{1997}, \emph{56},
  15712\relax
\mciteBstWouldAddEndPuncttrue
\mciteSetBstMidEndSepPunct{\mcitedefaultmidpunct}
{\mcitedefaultendpunct}{\mcitedefaultseppunct}\relax
\EndOfBibitem
\bibitem[Wood \latin{et~al.}(2015)Wood, Wade, Shahid, Collado-Fregoso, Bradley,
  Durrant, Heeney, and Kim]{wood2015natures}
Wood,~S.; Wade,~J.; Shahid,~M.; Collado-Fregoso,~E.; Bradley,~D.~D.;
  Durrant,~J.~R.; Heeney,~M.; Kim,~J.-S. Natures of optical absorption
  transitions and excitation energy dependent photostability of
  diketopyrrolopyrrole (DPP)-based photovoltaic copolymers. \emph{Energy \&
  Environmental Science} \textbf{2015}, \emph{8}, 3222--3232\relax
\mciteBstWouldAddEndPuncttrue
\mciteSetBstMidEndSepPunct{\mcitedefaultmidpunct}
{\mcitedefaultendpunct}{\mcitedefaultseppunct}\relax
\EndOfBibitem
\bibitem[Banerji \latin{et~al.}(2012)Banerji, Gagnon, Morgantini, Valouch,
  Mohebbi, Seo, Leclerc, and Heeger]{banerji2012breaking}
Banerji,~N.; Gagnon,~E.; Morgantini,~P.-Y.; Valouch,~S.; Mohebbi,~A.~R.;
  Seo,~J.-H.; Leclerc,~M.; Heeger,~A.~J. Breaking down the problem: optical
  transitions, electronic structure, and photoconductivity in conjugated
  polymer PCDTBT and in its separate building blocks. \emph{The Journal of
  Physical Chemistry C} \textbf{2012}, \emph{116}, 11456--11469\relax
\mciteBstWouldAddEndPuncttrue
\mciteSetBstMidEndSepPunct{\mcitedefaultmidpunct}
{\mcitedefaultendpunct}{\mcitedefaultseppunct}\relax
\EndOfBibitem
\bibitem[Li \latin{et~al.}(2018)Li, Tatum, Onorato, Zhang, and
  Luscombe]{li2018low}
Li,~Y.; Tatum,~W.~K.; Onorato,~J.~W.; Zhang,~Y.; Luscombe,~C.~K. Low elastic
  modulus and high charge mobility of low-crystallinity
  indacenodithiophene-based semiconducting polymers for potential applications
  in stretchable electronics. \emph{Macromolecules} \textbf{2018}, \emph{51},
  6352--6358\relax
\mciteBstWouldAddEndPuncttrue
\mciteSetBstMidEndSepPunct{\mcitedefaultmidpunct}
{\mcitedefaultendpunct}{\mcitedefaultseppunct}\relax
\EndOfBibitem
\bibitem[Rolczynski \latin{et~al.}(2012)Rolczynski, Szarko, Son, Liang, Yu, and
  Chen]{rolczynski2012ultrafast}
Rolczynski,~B.~S.; Szarko,~J.~M.; Son,~H.~J.; Liang,~Y.; Yu,~L.; Chen,~L.~X.
  Ultrafast intramolecular exciton splitting dynamics in isolated low-band-gap
  polymers and their implications in photovoltaic materials design.
  \emph{Journal of the American Chemical Society} \textbf{2012}, \emph{134},
  4142--4152\relax
\mciteBstWouldAddEndPuncttrue
\mciteSetBstMidEndSepPunct{\mcitedefaultmidpunct}
{\mcitedefaultendpunct}{\mcitedefaultseppunct}\relax
\EndOfBibitem
\bibitem[Tautz \latin{et~al.}(2012)Tautz, Da~Como, Limmer, Feldmann, Egelhaaf,
  Von~Hauff, Lemaur, Beljonne, Yilmaz, Dumsch, \latin{et~al.}
  others]{tautz2012structural}
Tautz,~R.; Da~Como,~E.; Limmer,~T.; Feldmann,~J.; Egelhaaf,~H.-J.;
  Von~Hauff,~E.; Lemaur,~V.; Beljonne,~D.; Yilmaz,~S.; Dumsch,~I.,
  \latin{et~al.}  Structural correlations in the generation of polaron pairs in
  low-bandgap polymers for photovoltaics. \emph{Nature Communications}
  \textbf{2012}, \emph{3}, 1--8\relax
\mciteBstWouldAddEndPuncttrue
\mciteSetBstMidEndSepPunct{\mcitedefaultmidpunct}
{\mcitedefaultendpunct}{\mcitedefaultseppunct}\relax
\EndOfBibitem
\bibitem[Yamagata \latin{et~al.}(2012)Yamagata, Pochas, and
  Spano]{yamagata2012designing}
Yamagata,~H.; Pochas,~C.~M.; Spano,~F.~C. Designing J-and H-aggregates through
  wave function overlap engineering: applications to poly (3-hexylthiophene).
  \emph{The Journal of Physical Chemistry B} \textbf{2012}, \emph{116},
  14494--14503\relax
\mciteBstWouldAddEndPuncttrue
\mciteSetBstMidEndSepPunct{\mcitedefaultmidpunct}
{\mcitedefaultendpunct}{\mcitedefaultseppunct}\relax
\EndOfBibitem
\bibitem[Lo \latin{et~al.}(2018)Lo, Gautam, Selter, Zheng, Oosterhout,
  Constantinou, Knitsch, Wolfe, Yi, Br{\'e}das, \latin{et~al.}
  others]{lo2018every}
Lo,~C.~K.; Gautam,~B.~R.; Selter,~P.; Zheng,~Z.; Oosterhout,~S.~D.;
  Constantinou,~I.; Knitsch,~R.; Wolfe,~R.~M.; Yi,~X.; Br{\'e}das,~J.-L.,
  \latin{et~al.}  Every atom counts: Elucidating the fundamental impact of
  structural change in conjugated polymers for organic photovoltaics.
  \emph{Chemistry of Materials} \textbf{2018}, \emph{30}, 2995--3009\relax
\mciteBstWouldAddEndPuncttrue
\mciteSetBstMidEndSepPunct{\mcitedefaultmidpunct}
{\mcitedefaultendpunct}{\mcitedefaultseppunct}\relax
\EndOfBibitem
\bibitem[Seifrid \latin{et~al.}(2020)Seifrid, Reddy, Chmelka, and
  Bazan]{seifrid2020insight}
Seifrid,~M.; Reddy,~G.; Chmelka,~B.~F.; Bazan,~G.~C. Insight into the
  structures and dynamics of organic semiconductors through solid-state NMR
  spectroscopy. \emph{Nature Reviews Materials} \textbf{2020}, \emph{5},
  910--930\relax
\mciteBstWouldAddEndPuncttrue
\mciteSetBstMidEndSepPunct{\mcitedefaultmidpunct}
{\mcitedefaultendpunct}{\mcitedefaultseppunct}\relax
\EndOfBibitem
\bibitem[Melnyk \latin{et~al.}(2017)Melnyk, Junk, McGehee, Chmelka, Hansen, and
  Andrienko]{melnyk2017macroscopic}
Melnyk,~A.; Junk,~M.~J.; McGehee,~M.~D.; Chmelka,~B.~F.; Hansen,~M.~R.;
  Andrienko,~D. Macroscopic structural compositions of $\pi$-conjugated
  polymers: combined insights from solid-state NMR and molecular dynamics
  simulations. \emph{The Journal of Physical Chemistry Letters} \textbf{2017},
  \emph{8}, 4155--4160\relax
\mciteBstWouldAddEndPuncttrue
\mciteSetBstMidEndSepPunct{\mcitedefaultmidpunct}
{\mcitedefaultendpunct}{\mcitedefaultseppunct}\relax
\EndOfBibitem
\bibitem[McCulloch \latin{et~al.}(2006)McCulloch, Heeney, Bailey, Genevicius,
  MacDonald, Shkunov, Sparrowe, Tierney, Wagner, Zhang, \latin{et~al.}
  others]{mcculloch2006liquid}
McCulloch,~I.; Heeney,~M.; Bailey,~C.; Genevicius,~K.; MacDonald,~I.;
  Shkunov,~M.; Sparrowe,~D.; Tierney,~S.; Wagner,~R.; Zhang,~W., \latin{et~al.}
   Liquid-crystalline semiconducting polymers with high charge-carrier
  mobility. \emph{Nature materials} \textbf{2006}, \emph{5}, 328--333\relax
\mciteBstWouldAddEndPuncttrue
\mciteSetBstMidEndSepPunct{\mcitedefaultmidpunct}
{\mcitedefaultendpunct}{\mcitedefaultseppunct}\relax
\EndOfBibitem
\bibitem[Spano and Silva(2014)Spano, and Silva]{spano2014h}
Spano,~F.~C.; Silva,~C. H-and J-aggregate behavior in polymeric semiconductors.
  \emph{Annual Review of Physical Chemistry} \textbf{2014}, \emph{65},
  477--500\relax
\mciteBstWouldAddEndPuncttrue
\mciteSetBstMidEndSepPunct{\mcitedefaultmidpunct}
{\mcitedefaultendpunct}{\mcitedefaultseppunct}\relax
\EndOfBibitem
\bibitem[Frisch \latin{et~al.}(2016)Frisch, Trucks, Schlegel, Scuseria, Robb,
  Cheeseman, Scalmani, Barone, Petersson, Nakatsuji, Li, Caricato, Marenich,
  Bloino, Janesko, Gomperts, Mennucci, Hratchian, Ortiz, Izmaylov, Sonnenberg,
  Williams-Young, Ding, Lipparini, Egidi, Goings, Peng, Petrone, Henderson,
  Ranasinghe, Zakrzewski, Gao, Rega, Zheng, Liang, Hada, Ehara, Toyota, Fukuda,
  Hasegawa, Ishida, Nakajima, Honda, Kitao, Nakai, Vreven, Throssell,
  Montgomery, Peralta, Ogliaro, Bearpark, Heyd, Brothers, Kudin, Staroverov,
  Keith, Kobayashi, Normand, Raghavachari, Rendell, Burant, Iyengar, Tomasi,
  Cossi, Millam, Klene, Adamo, Cammi, Ochterski, Martin, Morokuma, Farkas,
  Foresman, and Fox]{gaussian16}
Frisch,~M.~J. \latin{et~al.}  Gaussian˜16 {R}evision {A}.03. 2016; Gaussian
  Inc. Wallingford CT\relax
\mciteBstWouldAddEndPuncttrue
\mciteSetBstMidEndSepPunct{\mcitedefaultmidpunct}
{\mcitedefaultendpunct}{\mcitedefaultseppunct}\relax
\EndOfBibitem
\bibitem[Sun \latin{et~al.}(2016)Sun, Ryno, Zhong, Ravva, Sun, Körzdörfer,
  and Brédas]{sun2016tuning}
Sun,~H.; Ryno,~S.; Zhong,~C.; Ravva,~M.~K.; Sun,~Z.; Körzdörfer,~T.;
  Brédas,~J.-L. Ionization Energies, Electron Affinities, and Polarization
  Energies of Organic Molecular Crystals: Quantitative Estimations from a
  Polarizable Continuum Model (PCM)-Tuned Range-Separated Density Functional
  Approach. \emph{Journal of Chemical Theory and Computation} \textbf{2016},
  \emph{12}, 2906--2916\relax
\mciteBstWouldAddEndPuncttrue
\mciteSetBstMidEndSepPunct{\mcitedefaultmidpunct}
{\mcitedefaultendpunct}{\mcitedefaultseppunct}\relax
\EndOfBibitem
\bibitem[Henderson \latin{et~al.}(2009)Henderson, Izmaylov, Scalmani, and
  Scuseria]{henderson2009tuning}
Henderson,~T.~M.; Izmaylov,~A.~F.; Scalmani,~G.; Scuseria,~G.~E. Can
  short-range hybrids describe long-range-dependent properties? \emph{The
  Journal of Chemical Physics} \textbf{2009}, \emph{131}, 044108\relax
\mciteBstWouldAddEndPuncttrue
\mciteSetBstMidEndSepPunct{\mcitedefaultmidpunct}
{\mcitedefaultendpunct}{\mcitedefaultseppunct}\relax
\EndOfBibitem
\bibitem[Vreven \latin{et~al.}(2006)Vreven, Frisch, Kudin, Schlegel, and
  Morokuma]{vreven2006qmmm}
Vreven,~T.; Frisch,~M.~J.; Kudin,~K.~N.; Schlegel,~H.~B.; Morokuma,~K. Geometry
  optimization with QM/MM methods II: Explicit quadratic coupling.
  \emph{Molecular Physics} \textbf{2006}, \emph{104}, 701--714\relax
\mciteBstWouldAddEndPuncttrue
\mciteSetBstMidEndSepPunct{\mcitedefaultmidpunct}
{\mcitedefaultendpunct}{\mcitedefaultseppunct}\relax
\EndOfBibitem
\bibitem[Vydrov and Scuseria(2006)Vydrov, and Scuseria]{vydrov2006functionals}
Vydrov,~O.~A.; Scuseria,~G.~E. Assessment of a long-range corrected hybrid
  functional. \emph{The Journal of Chemical Physics} \textbf{2006}, \emph{125},
  234109\relax
\mciteBstWouldAddEndPuncttrue
\mciteSetBstMidEndSepPunct{\mcitedefaultmidpunct}
{\mcitedefaultendpunct}{\mcitedefaultseppunct}\relax
\EndOfBibitem
\bibitem[Vydrov \latin{et~al.}(2007)Vydrov, Scuseria, and
  Perdew]{vydrov2007functionals}
Vydrov,~O.~A.; Scuseria,~G.~E.; Perdew,~J.~P. Tests of functionals for systems
  with fractional electron number. \emph{The Journal of Chemical Physics}
  \textbf{2007}, \emph{126}, 154109\relax
\mciteBstWouldAddEndPuncttrue
\mciteSetBstMidEndSepPunct{\mcitedefaultmidpunct}
{\mcitedefaultendpunct}{\mcitedefaultseppunct}\relax
\EndOfBibitem
\bibitem[Polavarapu(1990)]{polavarapu1990raman}
Polavarapu,~P.~L. Ab initio vibrational Raman and Raman optical activity
  spectra. \emph{The Journal of Physical Chemistry} \textbf{1990}, \emph{94},
  8106--8112\relax
\mciteBstWouldAddEndPuncttrue
\mciteSetBstMidEndSepPunct{\mcitedefaultmidpunct}
{\mcitedefaultendpunct}{\mcitedefaultseppunct}\relax
\EndOfBibitem
\bibitem[Keresztury \latin{et~al.}(1993)Keresztury, Holly, Besenyei, Varga,
  Wang, and Durig]{keresztury1993raman}
Keresztury,~G.; Holly,~S.; Besenyei,~G.; Varga,~J.; Wang,~A.; Durig,~J.
  Vibrational spectra of monothiocarbamates-II. IR and Raman spectra,
  vibrational assignment, conformational analysis and ab initio calculations of
  S-methyl-N,N-dimethylthiocarbamate. \emph{Spectrochimica Acta Part A:
  Molecular Spectroscopy} \textbf{1993}, \emph{49}, 2007--2026\relax
\mciteBstWouldAddEndPuncttrue
\mciteSetBstMidEndSepPunct{\mcitedefaultmidpunct}
{\mcitedefaultendpunct}{\mcitedefaultseppunct}\relax
\EndOfBibitem
\end{mcitethebibliography}


\newpage

\includepdf[pages=1-11]{SI_arxiv_v1}


\end{document}


