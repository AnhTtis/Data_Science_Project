 
\documentclass[%
aps,amsmath,amssymb,preprint,%
%reprint,%
]{revtex4-2}
\usepackage{graphicx} 
\usepackage{dcolumn} 
\usepackage{bm} 
\usepackage[utf8]{inputenc}
\usepackage[T1]{fontenc}
\usepackage{mathptmx}
\usepackage{etoolbox}
\usepackage{physics}
\usepackage{hyperref}

\begin{document}

\title{Total Angular Momentum Conservation in Born-Oppenheimer Molecular Dynamics}

\date{\today}

\begin{abstract}
    We prove both analytically and numerically that the total angular momentum of a molecular system with an odd number of electrons undergoing adiabatic motion is conserved only when Berry forces are taken into account.   This finding sheds light on the nature of Berry forces for molecular systems with spin-orbit coupling and highlights how Born-Oppenheimer molecular dynamics simulations can successfully capture the entanglement of spin and nuclear degrees of freedom  as modulated by electronic interactions.
\end{abstract}

\maketitle 
 
\section{Introduction}
The Born-Oppenheimer (BO) representation lies in the heart of two of the central problems in chemistry and physics -- electronic structure theory and molecular dynamics. Under BO representation, the total molecular wavefunction is separated into two parts by mass difference:  the electronic part is treated by solving the electronic structure problems at different fixed nuclear geometries and the slow nuclear motion is propagated on a single or many coupled electronic potential energy surfaces. As first pointed out by Mead and Truhlar in molecular systems and then derived by Berry more generally, such separation can lead to a nontrivial geometric phase which can be accounted for by introducing a gauge potential into the BO Hamiltonian. 

Many intriguing phenomenon have emerged from such gauge structure. One example is the molecular geometric phase effect which the nuclear wavefunction will experience a sign-change during cyclic motion near a conical intersection (i.e., where potential energy surfaces cross). This molecular geometric phase effect is a purely topological effect and has been observed in various experiments.  Another kind of examples are the pseudo-electromagnetic field applied onto the nuclear wavefunction. In general, the electric contribution is of little importance and we focus more on observable effects arise from the magnetic contribution.    
Recently, several prominent solid-state experiments and theories such as phonon Hall effect and the phonon contribution in Einstein-de Haas effect have demonstrated the possibility that the pseudo-magnetic field can be deeply involved. One proposed mechanism is that, in the presence of the pseudo-magnetic field, the phonon modes of molecular systems can carry a non-zero angular momentum and thus lead to various interesting results mentioned above. 

Despite the concept of chiral phonon is receiving more and more attentions, the simple analog in molecular systems is not fully appreciated. To the best of our knowledge, the relation between the pseudo-magnetic gauge field and the nuclear angular momentum conservation (polarization) in molecular systems have not been discussed extensively. To understand the problem in detail, we decompose the total angular momentum into its electronic orbital, spin and nuclear components,
\begin{equation}
    {\bm J}_{\rm tot} = {\bm J}_{\rm nuc} + {\bm J}_{\rm orb} + {\bm J}_{\rm spin}.
\end{equation}
Due to the rotational symmetry of the system, the total angular momentum $\bm J_{\rm tot}$ must be conserved. However, either of these three components is not necessarily a good quantum number and therefore the angular momentum can transfer between spin, electron and nuclei in the presence of spin-electronic-rovibrational couplings.  

The purpose of this paper is to assess the role of the pseudo-magnetic gauge field in the angular momentum transfer process.  
In the following, we first derive the necessary equations of motion for propagating classical Born-Oppenheimer molecular dynamics (BOMD) with pseudo-magnetic gauge field. Then, as a proof of concept, we perform real-time {\em ab initio} BOMD simulations with pseudo-magnetic field of an open-shell molecular system with spin-orbit coupling (SOC). 
We show that the inclusion of pseudo-magnetic gauge field is crucial to the conservation of total angular momentum and conversion between individual components. Our results highlight the potential impact of the pseudo-magnetic field effects in molecular dynamics and could result in more efficient and effective designs in the field of molecular spintronics and molecular motors.

\section{Born-Oppenheimer molecular dynamics} 
For a generic molecular Hamiltonian with spin-orbit interaction:
\begin{equation} 
\hat H = \hat T_{\rm n} + \hat H_{\rm el},
\end{equation} 
\begin{equation} \label{eq:Hel}
 \hat H_{\rm el} = \hat T_{\rm e} + \hat V_{\rm ee} + \hat V_{\rm en} + \hat V_{\rm nn} + \hat V_{\rm SO},
\end{equation} 
the total molecular wavefunction is separated into two parts \begin{equation} \ket{\Psi({\bm R}, {\bm r})} = \sum_j \Phi_j({\bm R}) \ket{\phi_j({\bm R}, {\bm r})}, 
\end{equation}
where the electronic wavefunction is solved from time-indepedent electronic Schr\"odinger equation: 
\begin{equation} \label{eq:HelS}\hat H_{\rm el}\ket{\phi_j({\bm R}, {\bm r})} =  E_j ({\bm R}, {\bm r})\ket{\phi_j({\bm R}, {\bm r})} 
\end{equation} 
at fixed nuclear coordinates $\bm R$. The nuclear wavefunction is propagated according to the Schr\"odinger equation 
\begin{equation}
\sum_k H_{jk}^{\rm BO} \ket{\Phi_k} = i\hbar\frac {\partial} {\partial t} \ket {\Phi_j} 
\end{equation}  
with the Born-Oppenheimer Hamiltonian:
\begin{equation} \label{eq:fullS}
\hat H_{jk}^{\rm BO} =  \sum_l \frac {({\bm \hat P}\delta_{jl} - {\bm A}_{jl})({\bm \hat P}\delta_{lk} - {\bm A}_{lk})} {2M} + \tilde E_{jk}    
\end{equation} 
\begin{equation}
 \tilde E_{jk} = E_j\delta_{jk}  + \frac 1 {2M}\left( {\bra{\nabla\phi_j}\ket{\nabla\phi_k}} - \sum_l   {\bm A}_{jl}\cdot  {\bm A}_{lk} \right)
\end{equation} 
where $j,k$ label electronic eigenstates, ${\bm \hat P} = -i\hbar \bm \nabla$ is the nuclear momentum operator, ${\bm A}_{jk} = i\hbar\bra{\phi_j} \nabla \ket{\phi_l}$ defines the Mead-Berry gauge potential, and the last term in the second equation is the Born-Huang correction term. 
The above equations are all formally exact except one must truncate the electronic structure problem in Eq.\ref{eq:HelS} to finite dimensions. 
However, propagating coupled nuclear-electronic dynamics through Eq. \ref{eq:fullS} is not applicable for more than a few nuclear degrees of freedom (DoF) in practice. Several approximations must be made, for example, one can treat the nuclear DoFs classically and keep electronic 
DoFs quantum mechanical which leads to a class of so-called "mixed-quantum-classical" methods (e.g., Ehrenfest method and fewest swithes surface hopping method). Another important approximation which we will 
assume in this paper is the adiabatic approximation that the nuclei move slowly and the system will stay on a single electronic eigenstate. In such a case, the effective Hamiltonian governs system evolution simplifies to:
\begin{equation} \label{eq:HBO1}
\hat H^{\rm BO}_j =  \frac {({\bm \hat P} - {\bm A}_{jj})^2}  {2M} + \tilde E_{jj}
\end{equation} 
We define the nuclear kinetic momentum operator by ${\bm \hat  \pi} = {\bm \hat P} - {\bm A}_{jj}$. According to the Heisenberg equation of motion of effective adiabatic BO dynamics, we obtain 
\begin{equation}\label{eq:EOMR}
\frac {d  \hat R^{I\alpha}} {dt} = \frac i {\hbar} \left [ \hat H^{\rm BO}_j,  \hat R^{I\alpha} \right] = \frac {{\hat \pi^{I\alpha}}} {M^I},
\end{equation}
\begin{equation}
\begin{aligned} \label{eq:EOMpi}
\frac {d{\hat \pi}^{I\alpha}} {dt} &= \frac i {\hbar} \left [ \hat H^{\rm BO}_j, { \hat \pi^{I\alpha}} \right] \\ &= -\nabla_{I\alpha} \tilde E_{jj} + \frac 1 2 \sum_{J,\beta} \left( \Omega_{jj}^{I\alpha J \beta} {\hat\pi}^{J\beta} +  {\hat\pi}^{J\beta} \Omega_{jj}^{I\alpha J \beta}  \right).
\end{aligned}
\end{equation}
Here the index $I\alpha$ represents the Cartesian coordinate $\alpha = x,y,z$ of the $I$-th atom. Eq. \ref{eq:EOMpi} can be separated into two parts, the first term is the usual BO force $F^{\rm BO}$ and the second term is the pseudo-magnetic Berry force $F^{\rm Berry}$ expressed in terms of a molecular Berry curvature tensor $\Omega_{jj}^{I\alpha J \beta} = \nabla_{I\alpha} A_{jj}^{J\beta} - \nabla_{J\beta} A_{jj}^{I\alpha}$. 
At this point, if we employ the quantum-classical approximation, i.e., 
replace all nuclear operators by classical variables, we are ready to propagate molecular dynamics on any single BO energy surface with the pseudo-magnetic Berry force. However, there are several points must be addressed first. 

First, the above equations of motion are only valid under adiabatic limit which requires the BO energy surface $\tilde E_j$ is well-separated from all other surfaces. However, one can rewrite the Berry curvature as follow:
\begin{equation}
    \Omega_{jj}^{I\alpha J \beta} = -2 {\rm Im} \sum_{k\neq j} \frac {\bra{\phi_j} \nabla_{I\alpha} H_{\rm el} \ket{\phi_k}\bra{\phi_k} \nabla_{J\beta}H_{\rm el} \ket{\phi_j}} {(E_j - E_k)^2}. 
\end{equation}
The Berry curvature is inversely proportional to the square of energy gap between the state of interest $\phi_j$ and other states. This implies the pseudo-magnetic force is essentially a nonadiabatic effect. In order to fully understand the dynamics, one often needs to go beyond adiabatic approximation and use nonadiabatic dynamics methods. One potential candidates of running {\em ab initio} trajectory based nonadiabatic dynamics is the phase-space surface-hopping method, see Ref. XXX. This problem is beyond the scope of the current paper, so we will assume the system has an intermediate nonadiabaticity that electronic transition is insignificant but pseudo-magnetic force is still non-negligible.  

Second, we have not discussed any issue about electronic degeneracy. For molecular systems without external probe, the problem can be divided into two cases: (i) When the molecule has an even number of electrons, the true electronic ground state wavefunction must be real-valued due to time reversal symmetry, hence the Berry curvature must be zero. Meanwhile, approximate electronic structure method like generalized Hartree-Fock (GHF) or non-collinear density functional theory (DFT) with spin-orbit coupling (SOC) can sometimes give time reversal symmetry broken complex-valued electronic states and non-zero Berry curvatures. Although the approximate complex-valued wavefunction overestimates the true ground state energy, the non-zero Berry curvature is shown to be meaningful that it can capture at least part of the nonadiabatic dynamical effect. Again, it is a limitation of BO representation, and one seeking for full description must go beyond single surface picture.  One simplest example is the singlet-triplet intersystem crossing model as discussed in Ref. XXX.  
(ii) Another case is that when the molecule has an odd number of electron,  Kramers' theorem ensures the doubly degeneracy of all the electronic eigenstates over the entire nuclear configuration space. Even for a single eigenenergy Eqs.\ref{eq:HBO1} - \ref{eq:EOMpi} are not enough, instead a non-abelian $U(2)$ gauge theory must be considered. 
\begin{equation} \label{eq:HBO2}
H^{\rm BO}_{j,\mu\nu} =  \frac {({\bm P} \delta_{\mu\nu} - {\bm A}_{j\mu j\nu})^2}  {2M} + \tilde E_{j\nu j\mu}
\end{equation} 
where the two-fold degenerate eigenstates corresponding to the $j$-th eigenenergy are indexed by $\mu$ and $\nu$. The Hamiltonian described in Eq.\ref{eq:HBO2} is gauge-covariant, i.e., the nuclear dynamics governed by it is independent of the choice of gauge of electronic eigenstates. One practical matter is that one need to carefully choose a gauge frame that varies with nuclear configuration in a smooth manner. For a region where the surface is simply connected, i.e., no conical intersection is encircled, one can choose an arbitrary gauge frame at a given nuclear configuration and keep the wavefunction to be continuously varying along the path. 
\footnote{In fact, the gauge frame needs to be considered separately for two kinds of motion, the change of orientation and the change of shape, see Ref. XXX for a detailed discussion.} 
However, in the context of trajectory-based dynamics, the choice of gauge frame is important. Single surface dynamics methods are only valid if the off-diagonal matrix elements of Mead-Berry potential are zero 
\begin{equation} \label{eq:off0}
    {\bm A}_{j\mu j\nu} = 0, \quad  \text{if} \quad \mu \neq \nu, 
\end{equation}
so that the effective Hamiltonian is diagonal and degenerate surfaces are decoupled.  
(XXX this may not always be possible, but p dot A can). 

Third, as a compromise between accuracy and computation time, for general {\em ab initio} electronic structure calculations (e.g., HF, DFT) for propagating dynamics, the electronic eigenstate is often approximated by a single Slater determinant. For GHF or non-collinear DFT, even though the symmetry broken ground state and its time reversal pair are both solution of the electronic structure problem, one do not have freedom to choose the gauge because the linear combination of such states will no longer be a single Slater determinant and therefore not a solution anymore. Surprisingly, we find the GHF+SOC wavefunctions for doublet systems satisfy the desired condition Eq. \ref{eq:off0} for propagating BO molecular dynamics. In another word, the gauge selected by GHF+SOC method by enforcing the single Slater determinant constraint is dynamically meaningful. Therefore, we will use BO molecular dynamics with GHF+SOC method for a molecule with an odd number of electrons to demonstrate the interplay between pseudo-magnetic Berry force and angular momentum conservation.   

\section{Angular momentum conservation} 
Let us next review the conservation laws in the BO representation for classical nuclei. The total momentum and angular momentum of the molecular system on a single BO potential energy surface is given by
\begin{equation}
{\bm P}_{\rm tot} = {\bm \pi}_{\rm n} + \left<{\bm \hat P}_{\rm e}\right> = {\bm \pi}_{\rm n} + \bra{\phi_j({\bm R}(t))} {{\bm \hat P}_{\rm e}} \ket{\phi_j({\bm R}(t))}
\end{equation}
\begin{equation}
{\bm J}_{\rm tot} = {\bm J}_{\rm n}  + \left<{\bm \hat J}_{\rm e}\right> = {\bm R}_{\rm n} \times {\bm \pi}_{\rm n}  + \bra{\phi_j({\bm R}(t))} {{\bm \hat J}_{\rm e}} \ket{\phi_j({\bm R}(t))}.
\end{equation}
In classical BOMD, since the quantum nuclear wavefunction is simulated by classical trajectories, the canonical momentum is not measurable and replaced by the kinetic momentum, and electronic properties are measured at a given nuclear geometry. For any electronic wavefunction, it should satisfy transnational and rotational invariance: 
\begin{equation} \label{eq:transl} 
    \left( \hat {\bm P}_{\rm e}  + \hat {\bm P}_{\rm n}  \right) \ket{\phi_j({\bm R}(t))} = 0,
\end{equation}
\begin{equation}
    \left( \hat {\bm J}_{\rm n} + \hat {\bm J}_{\rm e } \right)\ket{\phi_j({\bm R}(t))}  = 0
\end{equation}
where $\hat {\bm J}_{\rm e} = \hat {\bm J}_{\rm orb}  + \hat {\bm J}_{\rm spin} $. From total momentum conservation,  
\begin{equation}
    \begin{aligned}
    \sum_{I} \frac{d{P}^{I\alpha}_{\rm tot}}{dt} &=  \sum_{I} \left( \frac{d{\pi}^{I\alpha}_{\rm n}}{dt} - \frac{d\bra{\phi_j} \hat {P}^{I\alpha}_{\rm n} \ket{\phi_j}  } {dt} \right)
    & =  \sum_{I} \frac {d \pi^{I\alpha}_{\rm n}} {dt} = 0.
    \end{aligned}
\end{equation} 
The second equation holds true only if
\begin{equation}
 \sum_{I}  \frac{d\bra{\phi_j} \hat  P^{I\alpha}_{\rm n} \ket{\phi_j}  } {dt} =
 \sum_{I} \frac {dA^{I\alpha}_{jj}} {dt} =  \sum_{I,J,k,\beta}\dot R_{\rm n}^{J\beta} {\rm Im} \left( A_{jk}^{I\alpha}A_{kj}^{J\beta} 
 \right) = 0,
\end{equation}
which is guaranteed repeatedly  in electronic structure calculations by considering a electronic transition factor (ETF), so classical nuclear momentum will always be conserved. Similarly from total angular momentum conservation, 
\begin{equation}
    \begin{aligned}
    \sum_{I} \frac{d{J}^{I\alpha}_{\rm tot}}{dt} &=  \sum_{I} \left( \frac{d{J}^{I\alpha}_{\rm n}}{dt} - \frac{d\bra{\phi_j} \hat {J}^{I\alpha}_{\rm n} \ket{\phi_j}  } {dt} \right) \\ 
    &=    \sum_{I} \frac{d{J}^{I\alpha}_{\rm n}}{dt} - \sum_{I} \sum_{\beta,\gamma \neq \alpha}   \epsilon_{\alpha\beta\gamma}  R_{\rm n}^{I\beta} \frac{d\bra{\phi_j} \hat {P}^{I\gamma}_{\rm n} \ket{\phi_j}  } {dt}    \\ 
     &=    \sum_{I} \frac{d{J}^{I\alpha}_{\rm n}}{dt} - \sum_{I,J,k,\delta}   \sum_{\beta,\gamma \neq \alpha}   \epsilon_{\alpha\beta\gamma}  R_{\rm n}^{I\beta}   \dot R_{\rm n}^{J\delta} {\rm Im}   
    \left( A_{jk}^{J\delta}A_{kj}^{I\gamma}  \right) = 0. 
    \end{aligned}
\end{equation} 
In general, the angular momentum for nuclei is not necessarily conserved since the electronic orbital and spin angular momentum can vary at different nuclear configurations on a time reversal-broken potential energy surface.  
% Therefore,
% \begin{equation}
% \begin{aligned}
% \frac{d{\bm P}_{\rm tot}}{dt} &=\frac{d{\bm \pi}_{\rm n}}{dt} - \frac{d\bra{\phi_j({\bm R}(t))} \hat {\bm \pi}_{\rm n} \ket{\phi_j({\bm R}(t))}  } {dt}   \\
% & =  \frac{d{\bm P}_{\rm n}}{dt} 
% -   \left<\frac{d\phi_j(R(t))}{dt} \middle| \hat {\bm P}_{\rm n} \middle| \phi_j(R(t)) \right> 
% - \left<\phi_j(R(t)) \middle | \hat {\bm P}_{\rm n}  \middle| \frac{d\phi_j(R(t))}{dt}\right> \\
% &=  \frac{d{\bm P}_{\rm n}}{dt}  - 
% \sum_{\alpha} \dot{R}_{\alpha} \left( \left<\frac{\partial\phi_j}{\partial R_{\alpha}} \middle| \hat {\bm P}_{\rm n}  \middle| \phi_j \right>
% +
% \left<\phi_j \middle| \hat {\bm P}_{\rm n}  \middle| \frac{\partial  \phi_j}{\partial R_{\alpha}}\right> \right) \\
% &=  \frac{d{\bm P}_{\rm n}}{dt}  - 
% \sum_{\alpha,\beta} \dot{R}_{\alpha} \left( \left<\frac{\partial\phi_j}{\partial R_{\alpha}} \middle|   \frac{\partial  \phi_j}{\partial R_{\beta}}\right>  
%  +   \left<\frac{\partial\phi_j}{\partial R_{\beta}} \middle|   \frac{\partial  \phi_j}{\partial R_{\alpha}}\right> \right)  \\
%  &=  \frac{d{\bm P}_{\rm n}}{dt}  - 
% \sum_k \sum_{\alpha,\beta} \dot{R}_{\alpha} \left( \left<\frac{\partial\phi_j}{\partial R_{\alpha}} \middle|  \phi_k \right> \left<\phi_k \middle | \frac{\partial  \phi_j}{\partial R_{\beta}}\right>  
%  +   \left<\frac{\partial\phi_j}{\partial R_{\beta}} \middle|  \phi_k \right> \left<\phi_k \middle | \frac{\partial  \phi_j}{\partial R_{\alpha}}\right>   \right) \\
%   &=  \frac{d{\bm P}_{\rm n}}{dt} +  \sum_k \sum_{\alpha,\beta} \dot{R}_{\alpha} \left( d_{jk}^{\alpha}d_{kj}^{\beta} + d_{jk}^{\beta}d_{kj}^{\alpha} \right) \\
%     &=  \frac{d{\bm P}_{\rm n}}{dt} +  \sum_k \sum_{\alpha,\beta} \dot{R}_{\alpha} \left( d_{jk}^{\alpha}d_{kj}^{\beta} + d_{jk}^{\beta}d_{kj}^{\alpha} \right)
% \end{aligned} 
% \end{equation}
 
As an demonstrative example, we propagate BOMD on one of the ground doublet states of a methoxy radical. The potential energy surface is computed by GHF+SOC method with 6-31G(d,p) basis set and one electron Breit-Pauli form of spin-orbit interaction is used. The details of nuclear Berry force computation is given in Appendix A . All electronic structure methods are implemented in a local branch of Q-Chem 6.0.  
\begin{figure} 
\includegraphics{Picture1.png} 
\caption{\label{fig:geometry} Initial geometry of TS1 of methoxy radical and the directions of the two initial velocity vectors (a) $v_1$ and (b) $v_2$.  XXX here better to show the spin direction as well} 
\end{figure}
The dynamics is initiated at the transition state 1 (TS1) of the methoxy radical optimized at the unrestricted Hartree-Fock (UHF) level and was conducted with a step size of 5 a.u. (0.12 fs). The initial geometry and velocity which is related to a  post-transition state bifurcation (PTSB) reaction are shown in Fig. \ref{fig:geometry}.  


In Fig.\ref{fig:angmom}, we show the real-time simulation results of the time-dependent total angular momentum as well as individual nuclear, electronic and spin components calculated with and without Berry force.   
As illustrated by Fig.\ref{fig:angmom} (a), the total angular momentum is conserved when considering the Berry force. The nuclear angular momentum is nearly equal and opposite to the spin angular momentum, while the electron orbital contribution is negligible.  In Fig.\ref{fig:angmom}(b), the results without Berry force show that the spin angular momentum changes at the same order of magnitude as in the case with Berry force, but the nuclear component is close to 0 which clearly violate the total angular momentum conservation.  

\begin{figure} 
\includegraphics[width=\textwidth]{Picture2.png} 
\caption{\label{fig:angmom}
Real-time evaluated angular momentum contributions along the trajectory with the initial velocity v1 (a) with Berry force and (b) without Berry force. In (a), with Berry force, nuclear and spin angular momentum transfer from each other, resulting in a conserved total angular momentum. In (b), without Berry force, the total angular momentum is not conserved.}
\end{figure}

Comparing Fig.\ref{fig:angmom} (a) and (b) shows that the change in electronic angular momentum during propagation can be compensated by the change of nuclear angular momentum when considering the Berry force in nuclear dynamics, resulting in the conservation of total angular momentum. This can be viewed as angular momentum transfer between electrons and nuclei in the BO representation.  Also, one may notice that the changes in spin angular momentum for dynamics with/without Berry force are very similar. This suggests that the pseudo-magnetic field has a small effect on nuclear motion, which may not even be observable in molecular systems experimentally, as the change in spin angular momentum in doublet systems will not exceed $1 \hbar$ at last. However, this is not the case since in reality there could be multiple reaction channels on potential energy surfaces. For the presenting methoxy radical, there are two possible PTSB pathways and different initial spin states can lead to different final nuclear geometries.  
 
In conclusion, we have used the {\em ab initio} BOMD to illustrate the importance of pseudo-magnetic Berry force in the total angular momentum conservation. The microscopic conversion mechanism between nuclear and electronic angular momentum reveals many interesting possibilities to design and control chemical reactions with the spin DoF. Future performing {\em ab initio} nonadiabatic molecular dynamics that fully considers the gauge potential holds great promise and will provide deeper insights into the field of spin chemistry. 

\section{Appendix A}
The classical nuclear equation of motion was propagated through a second-order Verlet type propagator.
\begin{equation}
\label{eqn:eom_r}
        \dot{ R} ^{I\alpha} = v^{I\alpha} 
\end{equation}
\begin{equation}
\label{eqn:eom_v}
        M^{I}\dot{v}^{I\alpha} =  F^{\rm BO,I\alpha} +  F^{\rm BF,I\alpha} 
\end{equation}
By integrating Eq. \ref{eqn:eom_v} over a small time interval $dt$ between two times $t_{n}$ and $t_{n+1}=t_{n}+dt$:
\begin{equation}
\label{eqn:integrate_v}
\begin{split}
      M^{I}\int^{t_{n+1}}_{t_n}\dot{\mathbf{v}}^{I\alpha}dt' &= \int^{t_{n+1}}_{t_n} \Big( F^{\rm BO,I\alpha} +\sum_{J\beta} \Omega^{I\alpha,J\beta}{v}^{J\beta} \Big) dt' 
\end{split}
\end{equation}
Assuming $\Omega^{I\alpha,J\beta}_{n+1} =\Omega^{I\alpha,J\beta}_{n}$ and , we can write
\begin{equation}
    \label{eqn:vn+1}
    \begin{split}
        M^{I}(v^{I\alpha}_{n+1}- v^{I\alpha}_{n}) &=  \int^{t_{n+1}}_{t_n} F^{\rm BO,I\alpha} dt' + \sum_{J\beta} \Omega^{I\alpha,J\beta}_{n+1} (R^{J\beta}_{n+1}-R^{J\beta}_{n}) \\
        & \approx \frac{dt}{2} (F^{\rm BO,I\alpha}_{n+1}+ F^{\rm BO,I\alpha}_{n}) + \sum_{J\beta} \Omega^{I\alpha,J\beta}_{n+1} (R^{J\beta}_{n+1}-R^{J\beta}_{n})
    \end{split}
\end{equation}
The integration of Eq. \ref{eqn:eom_r} can be approximated as
\begin{equation}
    \label{eqn:Rn+1}
    \begin{split} 
    (R^{I\alpha}_{n+1}-R^{I\alpha}_{n}) \approx \frac{dt}{2} (v^{I\alpha}_{n+1}+ v^{I\alpha}_{n}) \\
    \end{split}
\end{equation}
Plug Eq. \ref{eqn:integrate_v} into Eq. \ref{eqn:Rn+1} and approximate the integral to be correct up to second order in dt:
\begin{equation}
    \label{eqn:Rn+1}
    \begin{split} 
    R^{I\alpha}_{n+1} = R^{I\alpha}_{n} +  v^{I\alpha}_{n} dt + \frac{dt^2}{2M^{I}} ( F^{\rm BO,I\alpha}_{n}+ \sum_{J\beta}\Omega^{I\alpha,J\beta}_{n}v^{J\beta}_{n}) \\
    \end{split}
\end{equation}

\section{Appendix B}
The GHF electronic wavefunction $\ket{\phi^{\rm GHF}({\bm R}, {\bm r})}$ is a single slater determinant consisted of spinor orbitals $\phi_{i}({\bm R}, {\bm r})$, expanded into a linear combination of atomic orbitals $\phi_{i}({\bm R}, {\bm r})= \sum_{\mu} c^{\tau}_{\mu i}\chi_{\mu}({\bm R}, {\bm r}) $. Here, the indices $ijkl$, $\mu\nu\sigma\lambda$, and $\tau\kappa\eta\xi$ are used for spinor orbitals, atomic orbitals, and spins, respectively. The orbital coefficients $c^{\tau}_{\mu i}$ are solved self-consistently to variationally minimize $\bra{\phi^{\rm GHF}}\hat H_{\rm el}\ket{\phi^{\rm GHF}}$ ($\hat H_{\rm el}$ defined in Eq. \ref{eq:Hel}) with $\bra{\phi_{i}}\ket{\phi_{j}}=\delta_{ij}$.  According to Ref. XXX and XXX, the Berry curvature matrix elements in atomic orbital basis are given as
\begin{equation}
\label{eqn:bc}
\begin{split}
    \Omega^{I \alpha J \beta} =&-2\hbar \mathrm{Im}\langle\nabla_{I \alpha} \phi^{\rm GHF} |\nabla_{J \beta}\phi^{\rm GHF}\rangle \\
=&-2 \hbar \operatorname{Im}\Big[\sum_{i}\left\langle\phi_{i}^{(I \alpha)} \mid \phi_{i}^{(J \beta)}\right\rangle-\sum_{i j}\left\langle\phi_{i}^{(I \alpha)} \mid \phi_{j}\right\rangle\left\langle\phi_{j} \mid \phi_{i}^{(J \beta)}\right\rangle\\
&+\sum_{i a}\left\langle\phi_{a} \mid \phi_{i}^{(J \beta)}\right\rangle U_{a i}^{I \alpha *} +\sum_{i a}\left\langle\phi_{i}^{(I \alpha)} \mid \phi_{a}\right\rangle U_{a i}^{J \beta}+\sum_{i a} U_{a i}^{I \alpha *} U_{a i}^{J \beta} \Big]
\end{split}
\end{equation}
where the parenthesis notation ${(I \alpha)}$ refers to taking explicit nuclear derivative only, i.e. holding the orbital coefficients constant. The U matrix element $U^{I\alpha}_{ni}$ is defined as $\frac{\partial c_{\mu i}^\tau}{\partial R_{I\alpha}}=c_{\mu i}^{\tau, I\alpha}=\sum_n c_{\mu n}^\tau U_{n i}^{I\alpha}$, where the sum over $n$ includes all spinor orbitals. The U matrix is usually solved through CPSCF equations.
\begin{equation}
    \begin{bmatrix}
\mathbf{A} &  \mathbf{B}\\
\mathbf{B}^{*} & \mathbf{A}^{*} 
\end{bmatrix}
\begin{bmatrix}
\mathbf{U}  \\
\mathbf{U}^{*} 
\end{bmatrix}
=
\begin{bmatrix}
\mathbf{R}\\
\mathbf{R}^{*}
\end{bmatrix}
\end{equation}
where the orbital hessian components are given as $A_{ai,bj} = F_{a b} \delta_{i j}-F_{j i} \delta_{a b}+(a i \| j b) $ and $B_{ai,bj} = (a i \| b j)$. The right-hand sides are given as $R^{J\beta}_{ai} =-F_{a i}^{(x)}+\sum_{j} S_{a j}^{(J\beta)} F_{j i}+\sum_{k l} S_{kl}^{(J\beta)} (a i \| l k)$.
$F$ is the GHF+SOC Fock matrix and $S$ is the overlap matrix. 

With Berry curvature $\Omega^{I \alpha J \beta}$ and a given velocity vector $\textbf{v}$, Berry force can be calculated during dynamics $F^{\rm{BO,} I\alpha} = \sum_{J \beta}\Omega^{I \alpha J \beta}v^{J \beta}$. To save computational cost, instead of solving for the U matrix explicitly, we compute $X_{ai}=\sum_{J \beta} U_{a i}^{J \beta} v^{J \beta} $ directly in a z-vector way by first summing over the $J \beta$ degree of freedom in the right-hand sides after multiplying with the velocity vector, $\sum_{J \beta} R_{a i}^{J \beta} v^{J \beta} $. After solving for $X_{ai}$, we compute the imaginary component of $\sum_{ia}U_{a i}^{I \alpha *}X_{ai} $ directly also in a z-vector way by first solving for $\begin{bmatrix}
\mathbf{L_{1}} &
\mathbf{L_{2}}
\end{bmatrix}$,
\begin{equation*}
\begin{split}
\begin{bmatrix}
-\mathbf{X}^{*}  &
\mathbf{X} 
\end{bmatrix}
\begin{bmatrix}
\mathbf{U}  \\
\mathbf{U}^{*} 
\end{bmatrix} &=   \begin{bmatrix}
-\mathbf{X}^{*}   &
\mathbf{X}
\end{bmatrix}  \begin{bmatrix}
\mathbf{A} &  \mathbf{B}\\
\mathbf{B}^{*} & \mathbf{A}^{*} 
\end{bmatrix}^{-1}\begin{bmatrix}
\mathbf{R}\\
\mathbf{R}^{*}
\end{bmatrix}\\
\begin{bmatrix}
-\mathbf{X}^{*}  &
\mathbf{X} 
\end{bmatrix}
\begin{bmatrix}
\mathbf{U}  \\
\mathbf{U}^{*} 
\end{bmatrix} &=   \begin{bmatrix}
\mathbf{L_{1}} &
\mathbf{L_{2}}
\end{bmatrix}\begin{bmatrix}
\mathbf{R}\\
\mathbf{R}^{*}
\end{bmatrix}
\end{split}
\end{equation*}


This reduces the computational cost from solving the CPSCF equations 3*$N_{\rm atoms}$ times to 2 times.
    

\end{document}                                     
 