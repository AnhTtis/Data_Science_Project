\begin{widetext}

\section{Further details on the numerics}\label{app:num}

\subsection{Definition of vector potential}
In this section we provide a definition of the vector potential $A$ for the square lattice Hofstadter model on a torus. Some of the following details have previously appeared in Ref.~\cite{zhang2022fractional}. We first define our vector potential on a $L\times L$ torus with $L$ even. Considering even $L$ is sufficient for our calculations on the torus. An example of the gauge choice is shown in Fig.~\ref{fig:vector_potential}. We can use a similar gauge to define $A$ on an open disc. 

\begin{figure}[t]
    \centering
    \includegraphics[width=6.5cm]{vector_potential.png}
    \caption{$L\times L=6\times 6$ square lattice with periodic boundary condition indicated by the dotted bonds. We insert $2\pi m$ flux uniformly through the torus, where $m = \frac{\phi}{2\pi} L^2$. Each blue arrow represents a hopping phase $A_{ij}=\frac{2\pi m}{L^2}$; each red arrow represents a hopping phase $A_{ij}=\frac{2\pi m}{L}$. $\OO_1,\OO_2$ are the rotation centers. The $3\times 3$ partial rotation region is defined around $\OO_2$ and is colored blue.}
    \label{fig:vector_potential}
\end{figure}

\subsection{Choice of global rotation operator}
In systems with $\text{U}(1)$ and point group rotation symmetries, we have the ambiguity that any rotation symmetry operator can be multiplied by a phase $e^{i \theta \hat{N}}$ to give another valid symmetry operator. It is necessary to fix this ambiguity in order to uniquely define a rotation operator. Below we illustrate how to do this on the infinite plane, and then on the torus.

\subsubsection{On the infinite plane}

Ref.~\cite{zhang2022fractional} showed how to canonically define one such operator ($\Cmop$ in the notation of this paper) on the infinite plane, which assigns zero excess flux around a disclination of angle $2\pi/\MO$ created at $\OO$ using $\Cmop$. Since the defect Hamiltonian depends on the specific choice of $\text{U}(1)$ gauge transformation defining $\Cmop$, considering a different operator $e^{i \chi \hat{N}} \Cmop$ modifies the flux around the disclination by $\chi \mod 2\pi$, and therefore we can physically distinguish it from $\Cmop$. See Ref.~\cite{zhang2022fractional} for the explicit construction of a disclination Hamiltonian with fourfold rotational symmetry.

Let us specialize to the case $\MO=4$ and $(\tilde{C}^+_{\MO})^4 = +1$. Assume each plaquette on the infinite plane has flux $\phi \mod 2\pi$.
If $\OO = \beta$ (a vertex), the four possible operators assign equal flux $\phi \mod 2\pi$ to all plaquettes in the lattice with a $\pi/2$ disclination, except one that lies next to the disclination. 
In this plaquette the assigned flux is $\phi + k\frac{\pi}{2} \mod 2\pi$, where $k = 0,1,2,3$. Thus there is a canonical choice of $\tilde{C}^+_{M_{\beta}}$, corresponding to $k=0$ (meaning zero excess flux). When $\OO = \alpha$ (a plaquette center), there is a triangular plaquette at a $\pi/2$ disclination which is assigned a flux $\frac{3\phi}{4} + k\frac{\pi}{2} \mod 2\pi$. The operator $\tilde{C}^+_{M_{\alpha}}$ for which $k=0$ is also canonical, because it assigns a flux at the disclination that is proportional to $\phi$.

To extract the desired topological invariants in this paper, it turns out to be insufficient to use the operators $\Cmop$ alone, because they only generate groups of order $\MO$. Thus we also define $\Cmom := e^{i \frac{\pi}{\MO} \hat{N}} \Cmop$, which is of order $2\MO$ and satisfies $(\Cmom)^{\MO} = (-1)^F$. The two sets of operators, together denoted $\Cmopm$, are indeed sufficient for our purpose. Arbitrary $\text{U}(1)$ rotations can be considered by defining $\tilde{C}_{\MO,\chi}^{\pm} = e^{i \chi \frac{2\pi}{\MO} \hat{N}} \Cmopm$. Note that $\tilde{C}_{\MO,\chi}^{+},\tilde{C}_{\MO,\chi}^{-}$ insert excess flux $2\pi \chi/\MO$ at a $2\pi/\MO$ disclination, compared to $\Cmop$ and $\Cmom$ respectively.


\subsubsection{On the torus}
Defining a unique rotation operator on a torus with fourfold rotational symmetry is slightly more subtle than on the infinite plane, because any order 4 rotation preserves 2 points $\OO_1,\OO_2$. If the side length $L$ of the torus is even, the two fixed points are either both at $\alpha$ or both at $\beta$. If $L$ is odd, we have one fixed point each at $\alpha$ and $\beta$. 

$\Cmop$ can be defined for $\OO = \OO_1$ or $\OO_2$ by restricting to a disc $D$ centered around $\OO$, and creating a $\pi/2$ disclination within $D$ centered at $\OO$, using $\Cmop|_D$. As in the case of the infinite plane discussed above, we canonically define $\Cmop$ such that $\Cmop|_D$ creates a disclination with zero excess flux. We then define $\Cmom := e^{i \frac{\pi}{\MO} \hat{N}}\Cmop$ and $\tilde{C}_{\MO,\chi}^{\pm} = e^{i \chi \frac{2\pi}{\MO} \hat{N}} \Cmopm$. 



\subsection{Partial rotation on a torus}


Given the global rotation operators $\tilde{C}_{M_{\OO},\chi}^{\pm}$, we wish to restrict them to a region $D$, hence obtaining $\tilde{C}_{M_{\OO},\chi}^{\pm}|_D$. 
Note that the chosen operator inserts excess flux $2\pi\chi/\MO$ in a $2\pi/\MO$ disclination centered at $\OO$ relative to $\Cmopm|_D$. We find that the result for $l^{\pm}_{D,\OO,\chi}$ follows Eq.~\eqref{eq:angular_momentum}.



\subsection{Partial rotation on an open disc}


We can also perform the partial rotation calculation on an open disc. This can be done by first defining the vector potential similarly as in Fig.~\ref{fig:vector_potential}, using a gauge which inserts $\phi$ flux per plaquette in some region. As shown in Fig.~\ref{fig:general_setup}, It is \textit{not} necessary that the vector potential has support on the whole disc: the flux needs to be inserted only in an invariant subregion around $\OO$. Besides, $D$ \textit{does not} have to be defined around the center of the subregion with flux, as long as $D$ is enclosed by this subregion. The center of $D$, which we define as $\OO$, can be at any maximal Wyckoff point.

On the torus, we have a global rotation symmetry operator $\Cmopm$, and we define $\Cmopm|_D$ to be the restriction of $\Cmopm$ to region $D$. On the disc, this cannot be done as there might not be a global symmetry operator to start with. Therefore, we first choose $D$ and then define an operator $\Cmopm|_D$, which commutes with the Hamiltonian restricted to $D$. We find that choosing linear size $L_D$ of $D$ to be too large ($L_D \ge L-1$) or too small
($L_D \le 2$) gives unquantized $K^{\pm}_{\OO}$, which is tested this for $L \le 30$. As before,
we need to impose a constraint on $\tilde{C}_{4,\OO}|_D$ to fix the ambiguity by a global $\text{U}(1)$ phase $e^{i \theta \hat{N}}$. This can be done by fixing the excess flux inserted around a disclination created using the $\Cmopm|_D$ operator as explained in the previous section. We define $\tilde{C}_{\MO,\chi}^{+}$ (respectively $\tilde{C}_{\MO,\chi}^{-}$) so that it inserts an excess flux of $2\pi\chi/\MO$ at a $2\pi/\MO$ disclination, relative to $\Cmop, \Cmom$.

Note that, we obtain the same result, Eq.~\eqref{eq:angular_momentum}, on the torus or the open disc, as long as we choose $\OO$ at the same maximal Wyckoff position, and choose a partial rotation operator that inserts the same flux around a disclination at $\OO$.


\begin{figure}[t]
    \centering
    \includegraphics[width=8cm]{general_setup.png}
    \caption{Calculation on an open disc. Red shading indicates the sub-region where the flux is inserted. Blue shading inside the red region indicates different choices of maximal Wyckoff point as the partial rotation center.}
    \label{fig:general_setup}
\end{figure}




\subsection{Details on the jumps of $K_{\OO}^+$}

\begin{figure}[t]
    \centering
    \includegraphics[width=7cm]{K_b_raw.png}
    \caption{Raw data for $K_\beta$, calculated on a $30\times 30$ open disc. Partial rotation region $D$ is a $15\times 15$ square. $K_{\beta}$ jumps by 2 within a single lobe whenever the color switches between bright and dim shades.}
    \label{fig:jumps}
\end{figure}


As stated in the main text, if we use the operators $\Cmop$, $K^+_{\OO}$ sometimes jumps by $\MO/2$ inside a single Hofstadter lobe, as seen in Fig.~\ref{fig:jumps}. We define $\Theta_{\OO}^{+}\equiv K^+_{\OO} \mod \MO/2$ in order to eliminate these jumps. In general, the position of these jumps will depend on the system size and the size of the partial rotation region, but the value of these jumps is always $\MO/2$. We can understand these jumps as follows. The following arguments are equivalent to those given in \cite{herzogarbeitman2022interacting} using the many-body real space invariants defined there.

Suppose $\MO=4$ in a system where all orbitals are filled ($\nu = 1$). First consider $K^+_{\OO}$. Let $K_{\OO}^{+}|_{\nu=1}$ be the value of $K^+_{\OO}$ at $\nu=1$. Now suppose we minimally enlarge the partial rotation region $D$ by enclosing $4$ more sites which rotate into each other Under $\Cmop$. The states at these 4 sites form an orbit of size 4 under $\Cmop$ and contribute extra phases of $\{1, e^{i2\pi/4},e^{i4\pi/4},e^{i6\pi/4}\}$, which are fourth roots of 1. The expectation value of $\Cmop$ will gain a total extra phase of $e^{i\pi}$ (which is the product of the 4 phases), and thus $K_{\OO}^{+}|_{\nu=1}$ changes by 2 mod 4. But a change of the size of the partial rotation region should be regarded as trivial. Therefore only $K^+_{\OO} \mod 2$ is an invariant quantity. 

On the other hand, considering $\Cmom$ in the same argument, the states at the 4 extra sites in the enlarged region contribute phases of $\{e^{i\pi/4}, e^{i3\pi/4},e^{i5\pi/4},e^{i7\pi/4}\}$, which are fourth roots of -1. The total phase contribution from these four sites is 1, and thus $K_{\OO}^-$ is invariant upon changing $D$. 

Since changing $D$ is trivial, the above example suggests that $\Theta_{\OO}^+$ is an invariant mod $\MO/2$, and $\Theta_{\OO}^-$ is an invariant mod $\MO$. In the latter case we indeed verify numerically that there are no jumps of $K^-_{\OO}$ inside a given Hofstadter lobe.


\subsection{Periodicities of $\Theta_{\OO}^{\pm}$}

\begin{figure}[t]
    \centering
    \includegraphics[width=8cm]{periodicity.png}
    \caption{Symmetric gauge defined with respect to \textbf{(a)} $\alpha$; \textbf{(b)} $\beta$. Each blue arrow represents the vector potential $\phi/4$.}
    \label{fig:periodicity}
\end{figure}

We find that as a function of the background flux per unit cell in the Hofstadter model, $\Theta^{\pm}_{\beta}$ has periodicity $2\pi$, while $\Theta^{\pm}_{\alpha}$ has periodicity $8\pi$. We can straightforwardly see why these invariants need not be $2\pi$ periodic in $\phi$. Consider a system on the infinite plane, with flux $\phi$ per unit cell. For a given origin $\OO$, it is most convenient to define the vector potential in symmetric gauge around $\OO$; in this case the operator that trivially rotates points without any $\text{U}(1)$ gauge transformation is also a symmetry operator that commutes with the Hamiltonian.

As shown in Fig.~\ref{fig:periodicity}, if $\OO = \beta$, the links are assigned a vector potential which is an integer multiple of $\pi/2$, while if $\OO = \beta$, the links are assigned a vector potential which is an integer multiple of $\pi/4$. Therefore $H$ is $4\pi$ periodic in $\phi$ when $\OO = \beta$, and $8\pi$ periodic when $\OO = \alpha$. The actual periodicity of $\Theta^{\pm}_{\beta},\Theta^{\pm}_{\alpha}$ in $\phi$ must therefore be divisible by $4\pi$ and $8\pi$ respectively. The fact that $\Theta^{\pm}_{\beta}$ has a smaller periodicity $2\pi$ while $\Theta^{\pm}_{\alpha}$ has the maximal allowed periodicity of $8\pi$ is a more subtle result that we see empirically, but do not have a simple justification for on a lattice without defects. 

However, we can consider the flux inserted at a $\pi/2$ disclination created using $\tilde{C}_{M_{\beta}}^{\pm}$ and $\tilde{C}_{M_{\alpha}}^{\pm}$ respectively. As seen previously, for $\OO = \beta$ the flux is always expressed in terms of integer multiples of $\phi$ (which is $2\pi$ periodic), while for $\OO = \alpha$ the flux is expressed in terms of the quantity $3\phi/4$ (which is $8\pi$ periodic). Therefore defects of $\tilde{C}_{M_{\beta}}^{\pm}$ and $\tilde{C}_{M_{\alpha}}^{\pm}$ do have the same periodicities in $\phi$ as $\Theta^{\pm}_{\beta},\Theta^{\pm}_{\alpha}$. This gives additional justification for the observed periodicities.

\subsection{Obtaining empirical formulas for $\Theta_{\OO}^+$}

In this section, we explain how to obtain empirical formula for $\Theta_{\alpha}^+$ 
 and $\Theta_{\beta}^+$. The result is in Eqs.~\eqref{eq:theta_alpha}, \eqref{eq:theta_beta}.

In Fig.~\ref{fig:all} and ~\ref{fig:theta_plus_raw} we plot the raw Hofstadter butterflies for $\Theta_{\OO}^{\pm}$. For a fixed Chern number $C$, the different Hofstadter lobes are separated by the so called Farey sequence of order $2|C|$, which consists of ordered irreducible fractions $\frac{p}{q}$ with $0<p \le q\le 2|C|$. For example, the Farey sequence of order 4 is $\{\frac{1}{4},\frac{1}{3},\frac{1}{2},\frac{2}{3},\frac{3}{4},\frac{1}{1}\}$. If we track the lobes at a fixed $C$ as $\phi/2\pi$ is increased from 0 to 1, $\Theta_{\OO}^{\pm}$ may change its value between lobes which meet at the Farey seq of order $|C|$. (A similar behaviour was observed for the discrete shift $\mathscr{S}_{\OO}^+$ in Ref.~\cite{zhang2022fractional}.)  


\begin{figure*}[t]
    \centering
    \includegraphics[width=17.5cm]{all_thetaplus.png}
    \caption{Raw data for $\Theta_{\OO}$. $\Theta_\alpha^+$, $\Theta_\beta^+$, and $\Theta_{\gamma}^+$ are calculated on a $30\times 30$ open disc. $\Theta_{\gamma}^+=0 \mod 1$ everywhere in the Hofstadter butterfly, and is therefore not plotted here. The diameter of the partial rotation region $D$ is taken to be roughly half the system size.}
    \label{fig:theta_plus_raw}
\end{figure*}
 
The jumps for $\Theta_{\alpha}^+$ and $\Theta_{\beta}^+$ are tabulated in Figs.~\ref{fig:fareya}, \ref{fig:fareyb} and can be categorized into two contributions: A jump of 1 at every odd denominator $q$ for $q<|C|$; and a \textit{possible} jump of 1 when $q$ divides $C$. When $C>0$ we find that after summing the contribution from each jump point,
\begin{align}
    \Theta^+_{\alpha,\text{diff}} &=   \left(\sum_{\substack{\frac{p}{q}<\frac{\phi}{2\pi}\\\text{odd }q
    %, p\in \mathbb{N}^*, gcd(p,q)=1
    }} \left\lfloor \frac{C+q}{2q} \right\rfloor \right) + \begin{cases}\left\lfloor\frac{C\phi}{2\pi}\right\rfloor \text{if } C \mod 4 =2\\
    \left\lfloor\frac{C\phi}{4\pi}\right\rfloor \text{if } C \mod 4 =3\\
    0 \text{ if } C \mod 4 =0\\ 
    \left\lfloor\frac{1}{2}+\frac{C\phi}{4\pi}\right\rfloor \text{if } C \mod 4 =1
    \end{cases}\label{eq:theta_alpha}\\
    \Theta^+_{\beta,\text{diff}} &=  \left(\sum_{\substack{\frac{p}{q}<\frac{\phi}{2\pi}\\\text{odd }q
    %, p\in \mathbb{N}^*, gcd(p,q)=1
    }} \left\lfloor \frac{C+q}{2q} \right\rfloor \right)  + \begin{cases}\left\lfloor\frac{C\phi}{2\pi}\right\rfloor \text{if } C \mod 4 =1\\
    \left\lfloor\frac{C\phi}{4\pi}\right\rfloor \text{if } C \mod 4 =2\\
    0 \text{ if } C \mod 4 =3\\
    \left\lfloor\frac{1}{2}+\frac{C\phi}{4\pi}\right\rfloor \text{if } C \mod 4 =0
    \end{cases}\label{eq:theta_beta}\\
    \Theta^+_{\gamma,\text{diff}} &= 0.
\end{align}
These equations are all taken mod $\MO/2$. $\Theta^+_{\alpha}$ in the $C<0$ lobes can be obtained by the symmetry transformation $\Theta^+_{\alpha}(\mu,\phi)=\Theta^+_{\alpha}(-\mu,\phi) \mod 2$; $\Theta^+_{\beta}$ in the $C<0$ lobes can be obtained by the symmetry transformation $\Theta^+_{\beta}(\mu,\phi)=-\Theta^+_{\beta}(-\mu,\phi)\mod 2$. We have not found the analogous formulas for $\Theta^{-}_{\OO}$, as the jump patterns are different and much more complicated in this case. 


\begin{figure}[t]
    \centering
    \includegraphics[width=8cm]{fareya.png}
    \caption{Jumps in $\Theta_{\alpha}^+$ for fixed $C$, as $\frac{\phi}{2\pi}$ increases from 0 to $\frac{1}{2}$.}
    \label{fig:fareya}
\end{figure}
\begin{figure}[t]
    \centering
    \includegraphics[width=8cm]{fareyb.png}
    \caption{Jumps in $\Theta_{\beta}^+$ for fixed $C$, as $\frac{\phi}{2\pi}$ increases form 0 to $\frac{1}{2}$}
    \label{fig:fareyb}
\end{figure}



\section{Numerical results for global rotations}\label{app:global}

In this section we present some preliminary numerical results for the invariants of the Hofstadter model obtained from the eigenvalue under global rotations, $\bra{\Psi}\Cmopm\ket{\Psi}$, on the ground state $\ket{\Psi}$ on an even length torus. 

Define $\Cmopm$ as in the previous section. We define a set of invariants $\Phi_{\OO}^{\pm}$ as follows:
\begin{equation}
    \bra{\Psi}\Cmopm\ket{\Psi}  = e^{i l^{\pm}_{\OO,\text{global}}},
\end{equation}
and we empirically find that
\begin{align}
    l^{\pm}_{\OO_2,\text{global}} &= \frac{Cm^2}{2} + \mathscr{S}^{\pm}_{\OO_2}m + K_{\OO_2,\text{global}}^{\pm}\mod 4 \\
    l^{\pm}_{\OO_1,\text{global}} &= K_{\OO_1,\text{global}}^{\pm}\mod 4.
\end{align}

If we fix a given Hofstadter lobe, the quantities $K_{\OO,\text{global}}^+$ jump by multiples of $\MO/2$ as the total length of the torus is changed. (Recall that for partial rotations in a region $D$, the quantities $K_{\OO}^+$ similarly jump by multiples of $\MO/2$ for the same system, as the size of $D$ is changed.) However there is no change in the value of $K_{\OO,\text{global}}^-$. So we define
\begin{align}
    \Phi_{\OO}^{+} &= K^+_{\OO,\text{global}} \mod \MO/2 \nonumber \\
    \Phi_{\OO}^{-} &= K^-_{\OO,\text{global}} \mod \MO.
\end{align}
The raw Hofstadter butterflies for $\{\Phi_{\OO}^{\pm}\}$ are shown in Fig.~\ref{fig:fullrotation}. We see that for either $\OO = \OO_1$ or $\OO = \OO_2$,
\begin{align}
    \Phi^+_{\alpha} = \Phi^+_{\beta} = \Phi^+_{\gamma} &= \Phi^-_{\gamma} = -C \mod 2 \label{eq:cmod2} \\ 
    \Phi^-_{\alpha} = \Phi^-_{\beta} &= -C \mod 4. \label{eq:cmod4}
\end{align} Ref.~\cite{herzogarbeitman2022interacting} also studied global rotations on an even length torus in terms of a many-body real space invariant which appears closely related to $\Phi_{\OO}^{\pm}$, and also found that this invariant only depends on $C$. In the above cases, these invariants do not give any additional information beyond the Chern number. It is not clear how much more information can be obtained by also considering odd length tori. 

\begin{figure}[t]
    \centering
    \includegraphics[width=10cm]{fullrotation.png}
    \caption{Raw data for $\Phi^-_{\alpha}$ and $\Phi^+_{\gamma}$, calculated on a $30\times 30$ torus. They follow Eq.~\ref{eq:cmod4} and Eq.~\ref{eq:cmod2} respectively.}
    \label{fig:fullrotation}
\end{figure}


\section{CFT calculation}
\label{app:CFT}
In this appendix, we derive the expression Eq.~\eqref{eq:mainresultCFT} for the partial rotation by evaluating the CFT character,
\begin{equation}
\begin{split}
&\bra{\Psi}\tilde{C}^\pm_{M}|_D\ket{\Psi} = \frac{\mathrm{Tr}[e^{iQ_M\frac{\pi}{M}}e^{i\tilde{P}\frac{L}{M}}e^{-\frac{\xi}{v} H}]}{\mathrm{Tr}[e^{-\frac{\xi}{v} H}]} \\
&= e^{-\frac{2\pi i}{24M}c_-}\frac{\sum_{a=1,\psi}\chi_a(\frac{i\xi}{L}-\frac{1}{M};[\mathrm{AP},0],[\mathrm{AP},1])}{\sum_{a=1,\psi}\chi_a(\frac{i\xi}{L};[\mathrm{AP},0],[\mathrm{AP},0])},
\label{eq:rotcharacter}
\end{split}
\end{equation}
where we set $M$ to be even. For ease of notation, we will use $M$ in this appendix and suppress the $\OO$ subscript, but we assume that some $\OO$ has been fixed. While comparing with numerics we take $M = \MO=4$ for $\OO = \alpha, \beta$ and $M=2$ for $\OO = \gamma$.
Here, $\chi_a(\tau;[s,j],[s',j'])$ with $s,s'\in\{\mathrm{AP},\mathrm{P}\}$, $j,j'\in\Z_M$ is the CFT character that corresponds to the partition function on a torus equipped with a spin structure and a background $\Z_M$ gauge field. For example,
\begin{align}
    \chi_a(\tau;[\mathrm{AP},j],[\mathrm{AP},j'])=\mathrm{Tr}_{a,[\mathrm{AP},j]}[e^{iQ_M\frac{j'\pi}{M}}e^{2\pi i \tau (L_0-\frac{c_-}{24})}]
\end{align}
where $a$ labels the quasiparticle within the AP (antiperiodic) sector, with the twisted boundary condition corresponding to $j \in \Z_n$ along the spatial cycle. 

In order to evaluate the above CFT character, we need to study different cases of the action of the rotation operator $\tilde{C}^\pm_{M}$, i.e., whether $(\tilde{C}^+_{M})^M=+ 1$ or $(\tilde{C}^-_{M})^M=(-1)^F$. This choice of the symmetry action amounts to considering the twisted or untwisted spin structure respectively for the $\Z_M$ gauge field coupled to the CFT. That is, the rotation symmetry $\tilde{C}_{M}^{\pm}$ at long wavelengths is expressed as a combination of an internal $\Z_M$ symmetry and the translation symmetry of the CFT: $\tilde{C}_M^\pm=e^{iQ_M\frac{\pi}{M}}e^{i\tilde{P}\frac{L}{n}}$. Note that the symmetry action of $\tilde{C}_M^+$ (resp.~$\tilde{C}_M^-$) is realized by an internal symmetry satisfying $e^{iQ_M\pi}=(-1)^F$ (resp.~$e^{iQ_M\pi}=1$). The former case corresponds to a $\Z_{2M}^f$ symmetry where the $\Z_M$ is nontrivially extended by the fermion parity $(-1)^F$, which corresponds to the twisted spin structure, or equivalently a spin$^{\Z_{2M}}$ structure. Meanwhile, the latter corresponds to the untwisted spin structure together with a flat $\Z_M$ gauge field.

The above identification of the spatial symmetry $\tilde{C}_M^+   (\tilde{C}_M^-)$ with an internal symmetry $\Z_{2M}^f (\Z_M \times \Z_2^f)$ can be thought of a consequence of the crystalline equivalence principle \cite{Thorngren2018,manjunath2022mzm}. As we will see below, these two choices of the symmetry action give rise to different modular properties of the CFT characters, and hence distinct values of $\bra{\Psi}\tilde{C}^\pm_{M}|_D\ket{\Psi}$.

\subsection{The case with $\tilde{C}_M^+$}
Let us first take the rotation symmetry $\tilde{C}_M^+$, which has an equivalent internal $\Z_{2M}^f$ symmetry satisfying $e^{iQ_M\pi}=(-1)^F$.
The CFT character on the edge can be evaluated by the modular $S,T$ transformations as follows (note that $M$ is even): 
\begin{align}
\begin{split}
\chi_a\left(\frac{i\xi}{L}-\frac{1}{M};[\mathrm{AP},0],[\mathrm{AP},1]\right)&= S_{ab}\chi_b\left(-\frac{1}{\frac{i\xi}{L}-\frac{1}{M}};[\mathrm{AP},1],[\mathrm{AP},0]\right) \\
&= (ST^M)_{ab}\chi_b\left(\frac{-iM\frac{\xi}{L}}{\frac{i\xi}{L}+\frac{1}{M}};[\mathrm{AP},1],[\mathrm{P},0]\right),
\end{split}
\end{align}
where we used the fact that $T$ exchanges Spin$^{\Z_{2M}}$ structure as follows:
\begin{align}
T:\quad
\begin{cases}
     ([\mathrm{AP},j],[\mathrm{AP},j']) & \to ([\mathrm{AP},j],[\mathrm{P}+\frac{[j]_M +[j']_M-[j+j']_M}{M},[j+j']_M), \\
    ([\mathrm{AP},j],[\mathrm{P},j']) & \to ([\mathrm{AP},j],[\mathrm{AP}+\frac{[j]_M +[j']_M-[j+j']_M}{M},[j+j']_M). \\
    \end{cases}
\end{align}
Here, $[]_{M}$ denotes the mod $M$ operation, and the spin structure changes under $\Z_2$ action as $\mathrm{AP}+1=\mathrm{P}$, $\mathrm{P}+1=\mathrm{AP}$.

To evaluate the action of the $S,T$ modular matrices in the presence of a background $\Z_{2M}^f$ gauge field, we note that the modular transformation in the defect Hilbert space of the CFT can be determined from the modular $S,T$ matrices of the $G_b$-crossed braided fusion category that describes the bosonic shadow of the invertible phase of interest with bosonic global symmetry $G_b$ in the bulk~\cite{barkeshli2019}. In our case, the bosonic symmetry group is $G_b = \Z_M$. Using the data of the $\Z_M$-crossed braided fusion category, the $T$-matrix element is expressed as~\cite{barkeshli2019}
\begin{align}
    T^{(\mathbf{g},\mathbf{h})}_{a_{\mathbf{g}},b_{\mathbf{g}}} = e^{-\frac{2\pi i}{24}c_-}\cdot \theta_{a_{\mathrm{g}}}\cdot \eta_a(\mathbf{g},\mathbf{h})\cdot\delta_{a_{\mathbf{g}},b_{\mathbf{g}}}
\end{align}
where $\mathbf{g},\mathbf{h}\in \Z_M$, and $\eta_a(\mathbf{g},\mathbf{h})$ is a phase that describes the symmetry fractionalization of the bulk topological order~\cite{barkeshli2019}.
The character is then further rewritten as
\begin{align}
   \begin{split}
&\chi_a\left(\frac{i\xi}{L}-\frac{1}{M};[\mathrm{AP},0],[\mathrm{AP},1]\right)\\
&= \sum_{b\in\mathcal{C}_{1}}(ST^M)_{ab}\chi_b\left(\frac{-iM\frac{\xi}{L}}{\frac{i\xi}{L}+\frac{1}{M}};[\mathrm{AP},1],[\mathrm{AP},0]\right) \\
&= e^{-\frac{2\pi iM}{24}c_-}\sum_{b\in\mathcal{C}_{1}} S_{ab} \theta_b^n \prod_{j=0}^{M-1} \eta_b(1,j)\times \chi_b\left(\frac{-iM\frac{\xi}{L}}{\frac{i\xi}{L}+\frac{1}{M}};[\mathrm{AP},1],[\mathrm{P},0]\right) \\
&= e^{-\frac{2\pi iM}{24}c_-}\sum_{b\in\mathcal{C}_{{1}}}\sum_{c\in\mathcal{C}_{\mathrm{P}}} S_{ab} \theta_b^M \prod_{j=0}^{M-1} \eta_b\left(1,j\right)\times S_{bc} \chi_c\left(\frac{iL}{M^2\xi}+\frac{1}{M};[\mathrm{P},0],[\mathrm{AP},1]\right),
\end{split} 
\end{align}
and
\begin{align}
    \begin{split}
       &\chi_a\left(\frac{i\xi}{L};[\mathrm{AP},0],[\mathrm{AP},0]\right)= \sum_{b\in\mathcal{C}_0}S_{ab} \chi_b\left(\frac{iL}{\xi};[\mathrm{AP},0],[\mathrm{AP},0]\right),
    \end{split}
    \label{eq:trivialcharacter}
\end{align}
where $\mathcal{C}_0$ (resp.~$\mathcal{C}_{1}$) is the untwisted (resp.~twisted by $1\in\Z_M$) sector in $G_b$-crossed extension of super-modular category that consists of defects in the $[\mathrm{AP},0]$ sector, namely the identity particle and the fermion (resp. the $[\mathrm{AP},1]$ defects, namely the elementary $\Z_M$ fluxes). Meanwhile, $\mathcal{C}_{\mathrm{P}}$ denotes the defects in the $[\mathrm{P},0]$ sector that are labelled by fermion parity.

The character in Eq.~\eqref{eq:trivialcharacter} can further be approximated as
\begin{align}
\begin{split}
    \chi_b\left(\frac{iL}{\xi};[\mathrm{AP},0],[\mathrm{AP},0]\right) &\approx e^{-\frac{2\pi L}{\xi}(h_b-\frac{c_-}{24})}~.
    \end{split}
    \label{eq:characterapproxfermion}
\end{align}
The phase of the partial rotation can then be written as
\begin{align}
\begin{split}
 \bra{\Psi}\tilde{C}_{M}^+|_D \ket{\Psi}  & \propto {e^{-2\pi i(M+\frac{1}{M})\frac{c_-}{24}}}\sum_{b\in\mathcal{C}_{1}}\sum_{c\in\mathcal{C}_{\mathrm{P}}} (S_{1b}+S_{\psi b}) \theta_b^n \prod_{j=0}^{M-1} \eta_b\left(1,j\right)\times S_{bc} \chi_c\left(\frac{iL}{M^2\xi}+\frac{1}{M};[\mathrm{P},0],[\mathrm{AP},1]\right).
   \end{split}
   \end{align}
   While the above expression requires the sum over quasiparticles in the twisted sector $b\in\mathcal{C}_1$, it is more illuminating to express it in terms of the anyons in the untwisted sector. This is done by using the consistency equations and Verlinde formula of the $\Z_M$-crossed braided fusion category, given by~\cite{barkeshli2019}
\begin{align}
\begin{split}
    \theta_{b_{1}} &= \theta_{b_0}\theta_{0_{1}}\cdot (R^{b_0, 0_{1}}R^{0_{1}, b_0}), \\
    \eta_{b_1}\left(1,j\right) &= \eta_{b_0}\left(1,j\right)\eta_{0_1}\left(1,j\right), \\
    S_{a,b_{1}} &= \frac{1}{d_a}S_{a,b_0}(R^{a, 0_{-1}}R^{0_{-1}, a}),
\end{split}
\end{align} 
where $0_j\in \mathcal{C}_j$ denotes a $j\in \Z_M$ defect.
From the general solution outlined in Ref.~\cite{barkeshli2019},
one can work in the gauge where $R^{b,0_{1}}=1$, $R^{c,0_1}=1$ for $b\in\mathcal{C}_0$, $c\in\mathcal{C}_{\mathrm{P}}$. We then obtain
\begin{align}
\begin{split}
    \bra{\Psi}\tilde{C}^+_{M}|_D \ket{\Psi}  \propto &e^{-\frac{2\pi i}{24}(M+\frac{1}{M})c_-} \theta_{0_{1}}^M\prod_{j=0}^{M-1}\eta_{0_1}\left(1,j\right)\sum_{b\in\mathcal{C}_0}\sum_{c\in\mathcal{C}_{\mathrm{P}}} d_b \theta_b^M \prod_{j=0}^{M-1} \eta_b\left(1,j\right)
   \times S_{bc} \chi_c\left(\frac{iL}{M^2\xi}+\frac{1}{M};[\mathrm{P},0],[\mathrm{AP},1]\right).
   \end{split}
   \end{align}
Here we observe invariants given by the following combination of symmetry fractionalization data:
\begin{align}
    \mathcal{I}_M^{\pm}:= \theta_{0_{1}}^M\prod_{j=0}^{M-1}\eta_{0_1}\left(1,j\right),
\end{align}
where the $\pm$ superscript is used depending on whether we consider $\Tilde{C}_M^+$ or $\tilde{C}_M^-$, and
\begin{align}
    e^{i\pi Q_b}:= \prod_{j=0}^{M-1} \eta_b\left(1,j\right).
\end{align}
They can be interpreted as a $\Z_M$ analog of the Hall conductivity, and a fractional $\Z_M$ charge respectively \cite{manjunath2020FQH}.
For $\frac{L}{\xi}\gg 1$, we approximate the CFT character in terms of the highest weight state $\ket{h_b}$: 
\begin{align}
\begin{split}
    \chi_c\left(\frac{iL}{M^2\xi}+\frac{1}{M};[\mathrm{P},0],[\mathrm{AP},1]\right) &\approx e^{\frac{2\pi i}{M}(h_c-\frac{c_-}{24})} e^{-\frac{2\pi L}{M^2\xi}(h_c-\frac{c_-}{24})}~.
    \end{split}
\end{align}
We then have
\begin{align}
\begin{split}
   \bra{\Psi}\tilde{C}^+_{M}|_D \ket{\Psi}   \propto &e^{-\frac{2\pi i}{24}(M+\frac{2}{M})c_-} e^{\frac{2\pi i}{M}h_v}\mathcal{I}^{+}_M\sum_{b\in\mathcal{C}_0} d_b\theta_b^M e^{i\pi Q_b} S_{bv},
   \end{split}
   \end{align}
where $v\in\mathcal{C}_{\mathrm{P}}$ is the quasiparticle in the periodic sector (fermion parity flux) with the lowest value of spin.
For the Chern insulator with chiral central charge $c_-$, there are two such defects $v, v \times \psi$ with equal topological spin $c_-/8$. The result with $\mathcal{C}_0=\{1,\psi\}$ is then given by
\begin{align}
\begin{split}
  \bra{\Psi}\tilde{C}^+_{M}|_D \ket{\Psi}  \propto &e^{-\frac{2\pi i}{24}(M-\frac{1}{M})c_-} \mathcal{I}^+_M.
   \end{split}
   \label{eq:partialrotresult_CMM1}
   \end{align}

\subsection{The case with $\tilde C_M^-$}   
Here we consider the case where we take $\tilde C_M^-$ symmetry, which corresponds to the internal $\Z_M$ symmetry satisfying $e^{iQ_M\pi}=1$.
The CFT character on the edge can be evaluated by the modular $S,T$ transformation as (note that $M$ is even)
\begin{align}
\begin{split}
\chi_a\left(\frac{i\xi}{L}-\frac{1}{M};[\mathrm{AP},0],[\mathrm{AP},1]\right)&= S_{ab}\chi_b\left(-\frac{1}{\frac{i\xi}{L}-\frac{1}{M}};[\mathrm{AP},1],[\mathrm{AP},0]\right) \\
&= (ST^M)_{ab}\chi_b\left(\frac{-iM\frac{\xi}{L}}{\frac{i\xi}{L}+\frac{1}{M}};[\mathrm{AP},1],[\mathrm{AP},0]\right),
\end{split}
\end{align}
where we used $T$ exchanges spin$\times{\Z_{M}}$ structure as
\begin{align}
T:
\begin{cases}
    ([\mathrm{AP},j],[\mathrm{AP},j'])\to ([\mathrm{AP},j],[\mathrm{P},[j+j']_M]), \\
    ([\mathrm{AP},j],[\mathrm{P},j'])\to ([\mathrm{AP},j],[\mathrm{AP},[j+j']_{M}]). \\
    \end{cases}
\end{align}
Using a similar discussion as the previous subsection, one can write the phase of the partial rotation as
\begin{align}
\begin{split}
   \bra{\Psi}\tilde{C}^-_{M}|_D \ket{\Psi} & \propto {e^{-2\pi i\frac{c_-}{24} (M +\frac{1}{M})}}\mathcal{I}^-_M\sum_{b,c\in\mathcal{C}_0}d_b\theta_b^M \prod_{j=0}^{M-1} \eta_b\left(1,j\right)\times S_{bc} \chi_c\left(\frac{iL}{M^2\xi}+\frac{1}{M};[\mathrm{AP},0],[\mathrm{AP},1]\right),
   \end{split}
   \end{align}
   where we work on the gauge $R^{b,0_{\frac{\pi}{n}}}=1$ for $b\in\mathcal{C}_0$.
   The dominant contribution for the sum over anyons $c$ in the trivial sector comes from $c=1$, where the CFT character is approximated as
   \begin{align}
\begin{split}
    \chi_1\left(\frac{iL}{M^2\xi}+\frac{1}{M};[\mathrm{AP},0],[\mathrm{AP},1]\right) &\approx  e^{-\frac{2\pi i}{M}\frac{c_-}{24}} e^{\frac{2\pi L}{M^2\xi}\frac{c_-}{24}}~.
    \end{split}
\end{align}
We then obtain
\begin{align}
\begin{split}
  \bra{\Psi}\tilde{C}_{M}^-|_D \ket{\Psi} \propto &e^{-\frac{2\pi i}{24}(M+\frac{2}{M})c_-} \mathcal{I}^-_M\sum_{b\in\mathcal{C}_0} d^2_b\theta_b^M e^{i\pi Q_b}~.
   \end{split}
   \end{align}
For Chern insulators, in which $\mathcal{C}_0=\{1,\psi\}$, the partial rotation is simply given by
\begin{align}
\begin{split}
   \bra{\Psi}\tilde{C}_{M}^-|_D \ket{\Psi}  \propto &e^{-\frac{2\pi i}{24}(M+\frac{2}{M})c_-} \mathcal{I}^-_M.
   \end{split}
   \label{eq:partialrotresult_CMM(-1)F}   \end{align}

\subsection{Calculation of $\mathcal{I}^{\pm}_M$}

In Eq.~\eqref{eq:partialrotresult_CMM1},~\eqref{eq:partialrotresult_CMM(-1)F}, we observed that the partial rotation is proportional to the phase $\mathcal{I}^{\pm}_M$ that should be determined by $\tilde C^{\pm}_M$ symmetry action on the state. Here, we derive the general expression for $\mathcal{I}^{\pm}_M$ in terms of the data of a given fermionic invertible phase with $G_b = \Z_M$ symmetry and chiral central charge $c_-$.

The bosonic shadow of this phase is a 16-fold way $G_b$-enriched topological phase, also with chiral central charge $c_-$. In our setup, the symmetry does not permute anyons. Let the symmetry fractionalization data be given by
\begin{align}
    \omega_2({\bf g},{\bf h}) &= k_{\pm} \frac{[{\bf g}]_M+[{\bf h}]_M-[{\bf gh}]_M}{M} \mod 2 \\
    n_2({\bf g},{\bf h}) &= k_s \frac{[{\bf g}]_M+[{\bf h}]_M-[{\bf gh}]_M}{M} \mod 2.
\end{align}

$\omega_2$ describes the extension of the internal symmetry $\Z_M$ by fermion parity. From the fermionic crystalline equivalence principle, $k_+ = 1 \mod 2$ while $k_- = 0 \mod 2$. For a longer discussion of this point, see for example Ref.~\cite{manjunath2022mzm}. The two cases can be combined by defining $k_{\pm} = (1 \pm 1) \frac{1}{2}$.

If ${\bf g}_0$ is the generator of $G_b$, we wish to compute
\begin{equation}
     \mathcal{I}^{\pm}_M:= \theta_{0_{{\bf g}_0}}^M\prod_{j=0}^{M-1}\eta_{0_{{\bf g}_0}}\left({\bf g}_0,{\bf g}_0^j\right).
\end{equation}

We use the general solution outlined in Ref.~\cite{barkeshli2019} combined with the general theory of invertible fermionic phases in Ref.~\cite{barkeshli2021invertible}, which provides explicit expressions for the $\eta$ and $\theta$ symbols in terms of $n_2, \omega_2, c_-$. In particular, for this symmetry we can choose a gauge in which $ \theta_{0_{{\bf g}_0}} = 1$ and
\begin{equation}
    \eta_{0_{\bf k}}\left({\bf g},{\bf h}\right) = \nu_3^{-1}({\bf g},{\bf h},{\bf k})
\end{equation}
where
\begin{align}
     d\nu_3 &= e^{-2\pi i \mathcal{O}_4}, \\
     \mathcal{O}_4 &= \frac{1}{2} n_2(n_2 + \omega_2) + \frac{c_-}{8} \omega_2^2 \mod 1.
\end{align}
After plugging in the functional forms of $\omega_2, n_2$ and integrating, we get the following solution for $\nu_3$:
\begin{equation}
\begin{split}
    \nu_3({\bf g},{\bf h},{\bf k}) = &\exp\left(-2\pi i\left[\frac{k_s(k_{\pm}+k_s)}{2}+\frac{c_- k_{\pm}^2}{8}+k_3\right]\frac{[{\bf g}]_M}{M}\frac{[{\bf h}]_M+[{\bf k}]_M-[{\bf hk}]_M}{M}\right).
    \end{split}
\end{equation}
Here $k_3$ is a bosonic SPT index. Now a computation gives the desired result
\begin{equation}\label{eq:IM+-}
    \mathcal{I}^{\pm}_M = \exp\left(\frac{2\pi i}{M}\left[\frac{k_s(k_{\pm}+k_s)}{2}+\frac{c_- k_{\pm}^2}{8}+k_3\right]\right).
\end{equation}
Combining Eqs.~\eqref{eq:partialrotresult_CMM1},~\eqref{eq:partialrotresult_CMM(-1)F} with Eq.~\eqref{eq:IM+-}, and restoring the $\OO$ subscripts, we obtain Eq.~\eqref{eq:mainresultCFT} in the main text. Note that there is a phase proportional to $c_-$ coming from $\mathcal{I}^+_M$ but not $\mathcal{I}_M^-$. 

\section{Computation of $\ell_{s,\OO,\text{LL}}$}\label{app:LL}

Here we state the effective response theory for a continuum system of $C$ filled Landau levels with symmetry $\text{U}(1) \times \text{SO}(2)$ (where $\text{SO}(2)$ signifies spatial rotations), and find its relationship to the continuum limit of the Hofstadter model.


In terms of a $\text{U}(1)$ gauge field $A$ and an $\text{SO}(2)$ gauge field $\omega$ (which can be identified with the components of the spin connection on the underlying spatial manifold), the continuum response theory in the case where a $2\pi$ spatial rotation acts trivially on fermions is
\begin{equation}\label{eq:Leff_continuum}
    \mathcal{L}_{eff}=\sum_{n=1}^C( \frac{1}{4\pi} (A+s_n\omega) \wedge d(A+s_n\omega)-\frac{1}{48\pi}\omega \wedge d\omega)
\end{equation}
where $s_n=\frac{2n-1}{2}$ is the orbital spin per particle in the n-th Landau level. The Chern number of each filled Landau level is 1. The last term represents the gravitational anomaly. Expanding the above Lagrangian, we obtain

\begin{equation}
    \mathcal{L}_{eff}= \frac{1}{4\pi}A \wedge dA+ \frac{1}{2\pi}\frac{C^2}{2}A \wedge d\omega + \frac{1}{4\pi}\frac{2C^3-C}{6}\omega \wedge  d\omega.
\end{equation}
The coefficient of $\frac{1}{2\pi} A \wedge d\omega$ is identified with the Wen-Zee shift $\mathscr{S}^+_{\text{LL}}$. It has no origin dependence, and takes the value $\mathscr{S}^+_{\text{LL}} = C^2/2$. The coefficient $\tilde{\ell}^+_{s,\text{LL}}$ of the term $\frac{1}{4\pi}\omega \wedge d\omega$ also has no origin dependence, and takes the values 
\begin{equation}\label{eq:tildels_theory}
    \tilde{\ell}_{s,\text{LL}}^+=\frac{2C^3-C}{6}=\{\frac{1}{6},\frac{7}{3},\frac{17}{2},\frac{62}{3},\frac{245}{6},\dots\}
\end{equation}
We define $\ell_{\text{LL}} = \tilde{\ell}_s + C/12$, which is the coefficient of $\frac{1}{4\pi} \omega \wedge d \omega$ if we ignore the framing anomaly. 

In the continuum limit of the Hofstadter model, assuming that a $2\pi$ rotation acts trivially on fermions (that is, the rotation operator is given by $\Cmop$), the analogous response coefficient is denoted $\ell^+_{s,\OO,\text{LL}} = \ell^+_{s,\text{LL}} \mod \MO/2$; it acquires an origin dependence and is quantized mod $\MO/2$ where $\MO$ is the order of rotations that preserve $\OO$. Here we assume $\MO$ is even. For the square lattice,
\begin{equation}\label{eq:ells+oLL}
    \ell^+_{s,\OO,\text{LL}}=\frac{2C^3-C}{6}+\frac{C}{12} =\{\frac{1}{4},\frac{5}{2},\frac{3}{4},1,\frac{5}{4}\dots\}\mod \MO/2.
\end{equation}

To derive the expression for $\ell^-_{s,\OO,\text{LL}}$ when we instead use the operators $\Cmom$, we can simply replace $A \rightarrow A + \omega/2$ in Eq.~\eqref{eq:Leff_continuum} (this corresponds to choosing a reference rotation operator $\Cmom$ that inserts $\pi/\MO$ flux at a disclination of angle $2\pi/\MO$) and again compute the $\frac{1}{4\pi} \omega \wedge d\omega$ coefficient, after subtracting off the framing anomaly. The final result is
\begin{equation}\label{eq:ells-oLL}
    \ell^-_{s,\OO,\text{LL}}=\frac{2C^3-C}{6}+\frac{C^2}{2}+\frac{C}{4}+\frac{C}{12}=\{1,1,2,2,3,3,0,0,1,\dots\}\mod \MO.
\end{equation}
The above equations reproduce Eqs.~\eqref{eq:Theta+LL},~\eqref{eq:Theta-LL} appear in the main text.


\end{widetext}