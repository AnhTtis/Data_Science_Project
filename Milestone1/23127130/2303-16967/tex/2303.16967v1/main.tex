\def\year{2023}\relax
%File: formatting-instructions-latex-2021.tex
%release 2021.2
\documentclass[letterpaper]{article} % DO NOT CHANGE THIS
\usepackage{aaai23}  % DO NOT CHANGE THIS
\usepackage{times}  % DO NOT CHANGE THIS
\usepackage{helvet} % DO NOT CHANGE THIS
\usepackage[hyphens]{url}  % DO NOT CHANGE THIS
\usepackage{graphicx} % DO NOT CHANGE THIS
\usepackage{amsmath}
\usepackage{amssymb}
% \usepackage{gensymb}
\urlstyle{rm} % DO NOT CHANGE THIS
\def\UrlFont{\rm}  % DO NOT CHANGE THIS
\usepackage{natbib}  % DO NOT CHANGE THIS AND DO NOT ADD ANY OPTIONS TO IT
\usepackage{caption} % DO NOT CHANGE THIS AND DO NOT ADD ANY OPTIONS TO IT

\usepackage{booktabs}
\usepackage{multirow}
\usepackage{algorithm}
% \usepackage{algpseudocode}
\usepackage{xcolor}
\usepackage{tabularx}

\frenchspacing  % DO NOT CHANGE THIS
\setlength{\pdfpagewidth}{8.5in}  % DO NOT CHANGE THIS
\setlength{\pdfpageheight}{11in}  % DO NOT CHANGE THIS
\newcommand{\update}[1]{\textcolor{blue}{#1}}
\newcommand{\remove}[1]{\textcolor{red}{#1}}
\newcommand{\todo}[1]{\textcolor{green}{TODO: #1}}
%\nocopyright
%PDF Info Is REQUIRED.
% For /Author, add all authors within the parentheses, separated by commas. No accents or commands.
% For /Title, add Title in Mixed Case. No accents or commands. Retain the parentheses.
\pdfinfo{
/TemplateVersion (2023.1)
/Title (Heuristic Search For Physics-Based Problems: Angry Birds in PDDL+)
/Author (Wiktor Piotrowski, Yoni Sher, Sachin Grover, Roni Stern, Shiwali Mohan)
} %Leave this
% /Title ()
% Put your actual complete title (no codes, scripts, shortcuts, or LaTeX commands) within the parentheses in mixed case
% Leave the space between \Title and the beginning parenthesis alone
% /Author ()
% Put your actual complete list of authors (no codes, scripts, shortcuts, or LaTeX commands) within the parentheses in mixed case.
% Each author should be only by a comma. If the name contains accents, remove them. If there are any LaTeX commands,
% remove them.

% DISALLOWED PACKAGES
% \usepackage{authblk} -- This package is specifically forbidden
% \usepackage{balance} -- This package is specifically forbidden
% \usepackage{color (if used in text)
% \usepackage{CJK} -- This package is specifically forbidden
% \usepackage{float} -- This package is specifically forbidden
% \usepackage{flushend} -- This package is specifically forbidden
% \usepackage{fontenc} -- This package is specifically forbidden
% \usepackage{fullpage} -- This package is specifically forbidden
% \usepackage{geometry} -- This package is specifically forbidden
% \usepackage{grffile} -- This package is specifically forbidden
% \usepackage{hyperref} -- This package is specifically forbidden
% \usepackage{navigator} -- This package is specifically forbidden
% (or any other package that embeds links such as navigator or hyperref)
% \indentfirst} -- This package is specifically forbidden
% \layout} -- This package is specifically forbidden
% \multicol} -- This package is specifically forbidden
% \nameref} -- This package is specifically forbidden
% \usepackage{savetrees} -- This package is specifically forbidden
% \usepackage{setspace} -- This package is specifically forbidden
% \usepackage{stfloats} -- This package is specifically forbidden
% \usepackage{tabu} -- This package is specifically forbidden
% \usepackage{titlesec} -- This package is specifically forbidden
% \usepackage{tocbibind} -- This package is specifically forbidden
% \usepackage{ulem} -- This package is specifically forbidden
% \usepackage{wrapfig} -- This package is specifically forbidden
% DISALLOWED COMMANDS
% \nocopyright -- Your paper will not be published if you use this command
% \addtolength -- This command may not be used
% \balance -- This command may not be used
% \baselinestretch -- Your paper will not be published if you use this command
% \clearpage -- No page breaks of any kind may be used for the final version of your paper
% \columnsep -- This command may not be used
% \newpage -- No page breaks of any kind may be used for the final version of your paper
% \pagebreak -- No page breaks of any kind may be used for the final version of your paperr
% \pagestyle -- This command may not be used
% \tiny -- This is not an acceptable font size.
% \vspace{- -- No negative value may be used in proximity of a caption, figure, table, section, subsection, subsubsection, or reference
% \vskip{- -- No negative value may be used to alter spacing above or below a caption, figure, table, section, subsection, subsubsection, or reference

\setcounter{secnumdepth}{0} %May be changed to 1 or 2 if section numbers are desired.

% The file aaai21.sty is the style file for AAAI Press
% proceedings, working notes, and technical reports.
%

% Title

% Your title must be in mixed case, not sentence case.
% That means all verbs (including short verbs like be, is, using,and go),
% nouns, adverbs, adjectives should be capitalized, including both words in hyphenated terms, while
% articles, conjunctions, and prepositions are lower case unless they
% directly follow a colon or long dash

\title{Heuristic Search For Physics-Based Problems: Angry Birds in PDDL+}
% \author{Submission \#1388}
\author{
    %Authors
    % All authors must be in the same font size and format.
    Wiktor Piotrowski\textsuperscript{\rm 1},
    Yoni Sher\textsuperscript{\rm 1},
    Sachin Grover\textsuperscript{\rm 1},
    Roni Stern\textsuperscript{\rm 1,2},
    Shiwali Mohan\textsuperscript{\rm 1}
}
\affiliations{
    %Afiliations
    \textsuperscript{\rm 1} Palo Alto Research Center, CA, USA\\
    \textsuperscript{\rm 2} Ben-Gurion University of the Negev, Beer-Sheva, Israel\\
    \{wiktorpi,sgrover,smohan\}@parc.com,sternron@bgu.ac.il,yoni.sandsher@gmail.com\\
}
\begin{document}

\maketitle

\begin{abstract}
% We describe the first approach to solving Angry Birds, a well-established AI challenge problem, using combinatorial search and a domain-independent planner. Specifically, we model Angry Birds using PDDL+, a planning language for mixed discrete/continuous domains, and use this model to solve Angry Birds levels. This paper presents our PDDL+ model for this domain, identifies key design decisions that reduce the problem complexity, and compares the performance of our approach to dedicated model-specific methods for this domain. The results show that our performance is on par with the model-specific approaches, suggesting the applicability of domain-independent planning to this benchmark AI challenge. To gain further insight, we also present our own minimialistic domain-specific heuristics and evaluate them against the domain-independent approach. 

This paper studies how a domain-independent planner and combinatorial search can be employed to play Angry Birds, a well established AI challenge problem. To model the game, we use PDDL+, a planning language for mixed discrete/continuous domains that supports durative processes and exogenous events. The paper describes the model and identifies key design decisions that reduce the problem complexity. In addition, we propose several domain-specific enhancements including heuristics and a search technique similar to preferred operators. Together, they alleviate the complexity of combinatorial search. We evaluate our approach by comparing its performance with dedicated domain-specific solvers on a range of Angry Birds levels. The results show that our performance is on par with these domain-specific approaches in most levels, even without using our domain-specific search enhancements. 




%Solving Angry Birds poses a modeling challenge and a combinatorial search challenge. 
%
% [Original abstract]
% We describe the first approach to solving Angry Birds, a well-established AI challenge problem, using combinatorial search. 
% Solving the Angry Birds game with a domain-independent planner poses a modeling challenge and a combinatorial search challenge. 
% To address the modeling challenge, we chose to model Angry Birds using PDDL+, a planning language for mixed discrete/continuous domains, as use this model to solve Angry Birds levels. 
% We describe in details the reasons for this choice, present our PDDL+ model for this domain, and identifies key design decisions that reduce the problem complexity. 
% A comparison to dedicated domain-specific methods for this domain shows that on some types of Angry Birds levels, our domain-independent planner is able to perform is on par and even better than the state of the art. 
% with the model-specific approaches, suggesting the applicability of domain-independent planning to this benchmark AI challenge. To gain further insight, we also present our own minimialistic domain-specific heuristics and evaluate them against the domain-independent approach. 
\end{abstract}


\section{Introduction}

%$Automated Planning in mixed discrete/continuous domains is an important hurdle to developing solutions for real world problems. 

% Angry birds is hard and "important" 
Angry Birds, a wildly popular mobile game, is an open challenge problem for AI \cite{renz2019ai} that requires reasoning about sequential actions in a continuous world with discrete exogenous events. Different versions of the game have proven to be NP-Hard, PSPACE-complete, and EXPTIME-hard~\cite{stephenson2020computational}, and the reigning world champion is still a human. 
The game's simple layout and mechanics teamed with human players' innate spatio-temporal reasoning and forward state prediction makes for a challenging task. However, where humans excel, AI agents struggle. Small errors in initial assumptions can result in scores and states drastically different from the player's predictions. Angry Birds requires a holistic understanding of each individual level and its relevant characteristics. Understanding the relevance of all features and the sum of all of its parts is not a common trait of AI approaches. Angry Birds is a difficult and fascinating challenge for autonomous agents, solving it would prove a significant milestone in AI. 



% Our approach
In this work, we present the first successful game playing agent for Angry Birds that uses a domain-independent planner and combinatorial search.
Most existing planning languages, such a STRIPS \cite{fikes1971strips}, PDDL \cite{mcdermott1998pddl}, and PDDL2.1 \cite{fox2003pddl2}, lack features such as exogenous activity to capture the Angry Birds games. In this paper, we study how the game mechanics can be encoded using PDDL+~\cite{fox2006modelling}, a rich planning language that is designed for mixed discrete/continuous domains. 
% It enables reasoning about ballistic flight of birds and the consequences of their collisions with objects.
PDDL+ enables modeling the physics-based dynamics such as effect of gravity, collision between objects, and trajectory of the bird using equations of motion and their computational approximations (e.g., Bhaskara's sine approximation formula).

%e also employ time discretization techniques to model the behavior of the agent and use the predicted trajectory as a heuristic for solving the problem. Time discretization causes combinatorial search challenges which are alleviated by dividing the problem into smaller pieces to ensure a solution can be calculated in reasonable amount of time.
% We discuss complete design decisions in the paper in much detail.

Our first contribution is a model of Angry Birds using PDDL+, which includes several important modeling choices that ensure solution in reasonable time. We employ time discretization techniques to reason about continuous aspects of the domain which presents the challenge of combinatorial search. The second contribution of this work is the development of several search enhancements designed specifically to constrain solution space in the domain. These enhancements include domain-specific heuristics and a ``preferred states'' mechanism similar to the preferred operators technique~\cite{richter2009preferred}. Third contribution is through a comprehensive evaluation against some of the state-of-the-art agents for Angry Birds ~\cite{borovicka2014datalab,wang2017description}.
Our evaluation demonstrates that our techniques can solve a greater diversity of Angry Bird levels compared to other agents thus, showcasing that proposed domain-independent search strategies and domain-specific heuristics make search more efficient.
Please note, this is a preliminary exploratory work incorporating the initial modeling and searching techniques for discrete-continuous domains, such as basic features of birds of point and shoot. Since, submitting this work we have expanded the capabilities to incorporate some of the special features, such as, second tap to initialize special power of birds, that will be part of the forthcoming submissions.
% In comparison to existing AI agents for angry birds ~\cite{borovicka2014datalab,wang2017description}, a key advantage of our agent is the improved score using some of the novel heuristic-based search techniques.

% third contribution of this work is a comprehensive evaluation of our agent against existing agents on a diverse set of Angry Birds levels. The results of this evaluation clearly show that our agent is able to outperform existing agents on several types of levels.





% Our agent relies on a rich domain model, encoded in PDDL+ , which makes them 



% Existing AI agents for this game are based on encoding domain-specific strategies and selecting from them~\cite{borovicka2014datalab,wang2017description}. Our agent relies on a rich domain model, encoded in PDDL+ , which makes them 

% Our agent 

% Our approach 

% apply a domain-independent planner with domain-specific heuristics to solve this problem. 


% Existing AI agents for this game are based on encoding domain-specific strategies and selecting from them~\cite{borovicka2014datalab,wang2017description}. 
% In this work, we apply a domain-independent planner with domain-specific heuristics to solve this problem. 
% Most existing domain-independent planning languages, such a STRIPS \cite{fikes1971strips}, PDDL \cite{mcdermott1998pddl}, and PDDL2.1 \cite{fox2003pddl2}, lack features to capture Angry Birds. 
% Therefore, we modeled the problem using PDDL+ \cite{fox2006modelling}, a planning language that is designed for mixed discrete/continuous domains. PDDL+ enables reasoning about ballistic flight of birds and the consequences of their collisions with objects.

 


% In this work, we make the following contributions: 
% %(1) the first domain-independent approach for Angry Birds,[Roni: we can't saying  ``domain-independent approach for angry birds'' is a bit self-contradictory. I rephrased] 
% (1) present the first successful application of domain-independent planning to solve Angry Birds,  (2) demonstrate its performance is on par with existing domain-specific approaches, 
% (3) discuss the design choices around PDDL+ modeling needed to ensure computing solutions within timeouts. 


% Roni: Ideally say more about the design choices

The paper is organized as follows. We begin by providing relevant background on the Angry Birds game and PDDL+. This is followed by an analysis of different planning formalisms and their limitations for this domain. Next, we present the design of our PDDL+ model for Angry Birds. Then, we describe how this PDDL+ model is used within a game playing agent, and propose several domain-specific search enhancements. 
Finally, we close with an experimental evaluation of our approach on a variety of Science Birds problems comparing our results with state-of-the-art baseline agents from previous competitions.
For the convenience of the readers we have provided the complete domain, a problem file and the solution generated as part of the supplementary material with the submission.
%Roni: if space is an issue, this paragraph can be removed. It is not needed in a conference paper.


\section{Background}
\begin{figure}
\begin{center}
\includegraphics[trim = 0 100 0 150, clip, width=0.4\textwidth]{images/level.PNG}
\end{center}
\caption{Trivial Angry Birds level.}
\label{fig:easy-level}
\end{figure}
\begin{figure}
\begin{center}
\includegraphics[trim = 0 20 0 20, clip, width=0.4\textwidth]{images/hard-level.jpg}
\end{center}
\caption{Difficult Angry Birds level.}
\label{fig:hard-level}
\end{figure}


\setlength{\belowcaptionskip}{-3pt}

% \update{
PDDL+~\cite{fox2006modelling} is an extension of the well-known Planning Domain Description Language (PDDL)~\cite{mcdermott1998pddl}, that maps the planning constructs to a Hybrid Automata~\cite{henzinger2000theory} for improved expressiveness with numeric fluents and metrics, instantaneous actions, exogenous {\em events} and durative changes called {\em processes}.
% Processes can be modeled as an instantaneous event at the beginning and an instantaneous event at the end.
PDDL+ is designed to model and plan for problems requiring both discrete mode switches, continuous flows with exogenous environmental changes.
It builds on the expressiveness of its predecessors modeling languages, encapsulating the entire set of features from PDDL2.1~\cite{fox2003pddl2} and adding timed-initial literals from PDDL2.2~\cite{edelkamp2004pddl2}.
% }

% \update{
In PDDL+ the domain $\mathcal{D}$ for the game environment is a tuple $\langle F, R, A, E, P \rangle$ where
$F$ is a set of discrete state variables;
$R$ is a set of numeric state variables;
$A$ is a set of actions the agent can execute;
$E$ is a set of instantaneous events triggered in the environment directly or indirectly due to actions of the agent;
$P$ is a set of durative process that may be active in the environment or triggered by the agent.
A state of the environment $s$ is a complete assignment of values to the state variables $F \cup R$. Please note, that the domain of numeric state variables does not include $\bot$ value, i.e., none of the variables are undefined in the state as the environment and the transitions are completely deterministic.
A planning problem $\Pi = \langle \mathcal{D}, s_I, G \rangle$, where $s_I$ is the initial state, and $s_I \in S$ the set of possible states in the environment; $G$ is the goal condition for the environment. 
% }

% \begin{figure}
% \begin{center}
% \includegraphics[trim = 0 10 0 10, clip, width=0.35\textwidth]{images/angry-birds-making-of-birds.jpg}
% \end{center}
% \caption{Basic birds available to the player in Angry Birds.}
% \label{fig:birds}
% \end{figure}
% \begin{figure}
% \begin{center}
% \includegraphics[trim = 0 20 0 20, clip, width=0.35\textwidth]{images/pigs.jpeg}
% \end{center}
% \caption{Basic pigs in Angry Birds.}
% \label{fig:pigs}
% \end{figure}

% Next, we provide necessary background on the Angry Birds game and the PDDL+ language. 
% \update{
\subsubsection{The Angry Birds Game} consists of several different levels. Every level has some allocated numbers of birds that are launched using a slingshot in an attempt to hit pigs sitting inside structures built using {\em platforms} and {\em blocks}.
Pigs can be killed by a direct hit from a bird, or indirectly by falling blocks, explosions, or falling from a height.
Every level consists of different arrangement of structures and a number of pigs to be killed.
Platforms are indestructible floating objects, while blocks can be destroyed or toppled over. Blocks come in various shapes, sizes, and materials. 
For example, blocks made of ice break easily, stone blocks are harder to break, and TNT crates (shown in fig.~\ref{fig:hard-level}) explodes on impact, destroying close objects and launching others in the air.
% High-level game objective and structure. Birds and pigs.
% The objective in Angry Birds is to destroy pigs by launching birds at them from a slingshot. 
% Each level has an allotted number of birds that can be launched, and
% launched birds obey, in general, the laws of motion and gravity. 
Some birds have special abilities, activated by tapping on the screen during flight. 
While these special abilities are needed to pass some levels, we do not model them in this paper for simplicity. 
% }
% For the sake of simplicity, we only consider one type of pig in this paper, but the full game includes multiple types of pigs, which differ visually and in the amount of force needed to kill them. 
% Blocks and platforms
% Angry birds levels usually include structures built from \emph{platforms} and \emph{blocks}. 

%Together, the blocks and platforms are often used to protect the pigs and force the player to opt for more elaborate strategy to win any given level. 

% \update{
Aim of the game is to kill the pigs using minimum number of birds and maximize the game score. A well-known strategy of angry birds is to hit the weakest point of the structure, to maximize the impact on the pigs and kill them quickly. The composition of each level can drastically affect the choice of strategy for playing the game. 
For example, Figure \ref{fig:easy-level} shows a simple level with red birds and a single pig protected by a weak structure. 
To pass this level, it is sufficient to shoot a bird directly towards the pig, causing the structure to collapse on it. 
In contrast, Figure \ref{fig:hard-level} shows a more difficult level where the only viable strategy is executing precise shots that set off an explosive chain reaction. 
Thus, an intelligent agent playing Angry Birds needs to be able to predict future states of the world that extend beyond the immediate and direct consequences of the agent's actions. 
As the game is solved in discrete and continuous state space, it has become the game of choice for a long running yearly competition, AI Birds~\cite{renz2015aibirds}, organized by Australian National University at the IJCAI conference.
% }

% \subsubsection{The PDDL+ Language}

% % High-level: PDDL+ objective and foundation
% PDDL+~\cite{fox2006modelling} is an extension of the well-known Planning Domain Description Language (PDDL)~\cite{mcdermott1998pddl}, formally defined as a mapping of planning constructs to Hybrid Automata~\cite{henzinger2000theory}. 
% It is designed to enable modeling and solving planning problems set in systems exhibiting both discrete mode switches and continuous flows. 
% PDDL+ builds on the expressiveness of its predecessors modeling languages, encapsulating the entire set of features from PDDL2.1~\cite{fox2003pddl2} as well as the inclusion of timed-initial literals (TILs) from PDDL2.2~\cite{edelkamp2004pddl2}. 

% % Evens and processes 
% PDDL+ includes new constructs for defining exogenous activity in the domain: \textit{discrete events} and \textit{continuous processes}. Events represent the system's mode switches, which instantaneously change the dynamics of the modeled system. Processes evolve the system over time dictated by a set of ordinary differential equations. Processes and events can be thought of as the world's version of durative and instantaneous actions, respectively. Both are classed as ``must-happen'' constructs, i.e., their effects are applied as soon as preconditions are satisfied. 
% Thus, the agent can only interact with processes and events indirectly to mitigate or support their effects.
% % PDDL+ is a known thing, has been used in other contexts
% The expressiveness of PDDL+ enables natural definitions of phenomena in many domains including Urban Traffic Control~\cite{vallati-et-al:aaai-2016},  Chemical Batch Plant~\cite{della2010pddl+}, Minecraft~\cite{roberts2017automated}, and Atmospheric Re-entry~\cite{piotrowski2018heuristics}






\section{Modeling Angry Birds}
To create a domain for Angry Birds, we need to model the flight of the bird, collisions between structures, explosions, and structure collapse after collisions and explosions. In this section we look at these aspects in some detail. First, we begin with a brief discussion on why we chose PDDL+ as our modeling language, then describe the PDDL+ components used to model different objects in the game, and finally the PDDL+ components used to model the game dynamics. 
% The PDDL+ domain can be downloaded from \url{https://file.io/m7z8R1BoiCVR}

% \setlength{\belowcaptionskip}{-10pt}
\begin{table}[h]
\centering
\small
% \scriptsize
% \resizebox{\columnwidth}{!}{%
\begin{tabular}{c|c|c|c|}
\cline{2-4}
 & \textbf{PDDL+} & \textbf{PDDL2.1} & \textbf{PDDL2.2} \\ \hline
\multicolumn{1}{|c|}{Bird Flight}              & $\checkmark$ & $\checkmark$ &  \\ \hline
\multicolumn{1}{|c|}{Collisions}               & $\checkmark$ &   & $\checkmark$   \\ \hline
\multicolumn{1}{|c|}{Explosions}               & $\checkmark$ &   & $\checkmark$   \\ \hline
\multicolumn{1}{|c|}{Structure collapse}       & $\checkmark$ &   & $\checkmark$   \\ \hline
\end{tabular}%
% }
\caption{Language support for crucial Angry Birds features. \textbf{FSTRIPS+} similar to \textbf{PDDL2.1} is also able to model only the Bird Flight.}
\label{tab:lang-comparison}
\end{table}

% \remove{
% Since the Angry Birds game includes both discrete and continuous state variables, a modeling language richer than classical PDDL must be used. 
% PDDL2.1~\cite{fox2003pddl2} supports numeric variables, explicit representation of time and duration, plan metrics, and both linear and non-linear continuous action effects. 
% While PDDL2.1 can model the non-linear dynamics of flight as a durative action, it cannot naturally model events such as a collision between a bird and a block or a bird and a pig.
% }

% \remove{
% PDDL2.2~\cite{edelkamp2004pddl2} extends the features of its predecessor by including \emph{timed-initial literals} (TILs), which are unconditional events occurring at a pre-specified time in the plan, and \emph{derived predicates}, which are propositional variables derived from \textit{if-then} rules rather than actions. The latter provide some limited support for exogenous activity, yet it does not allow modeling continuous process  such as the flight of a bird in a natural way. 
% }

% \remove{
% Functional STRIPS (FSTRIPS)~\cite{ramirez2017numerical} is another modeling language for hybrid systems. It lacks full support for instantaneous exogenous activity, and is only able to define them as global constraints that invalidate the solution if triggered. Angry Birds contains many cases where exogenous happenings have positive or neutral effects (w.r.t. finding a solution), and are a natural and integral part of the system (e.g. collisions). The FSTRIPS extension does not have the expressiveness to define such cases, so can not model Angry Birds with sufficient accuracy. 
% }

% \remove{
% Table~\ref{tab:lang-comparison} summarizes the above discussion regarding relevant modeling languages and their ability to model crucial features of Angry Birds. As can be seen, only PDDL+ has the capacity to model and reason with all elements and dynamics of the game. Therefore, we chose PDDL+ for modeling Angry Birds. Our Angry Birds PDDL+ model relies on knowledge obtained from Angry Birds where possible. Beyond parameters we could reliably source, we approximate the behavior and parameters to the best of our knowledge via experiments and observations.
% }

\begin{table*}[!ht]
    \centering
    \begin{tabular}{c|l|c|l}
                                & Predicate & Domain & Comments \\ \toprule
    \multirow{5}{*}{Birds}      & \texttt{bird\_type} & $\mathbb{Z}$ & Different types of birds such as 0 is for the Red bird \\
                                    & \texttt{x\_bird, y\_bird} & $\mathbb{R}$ & XY location of the bird \\
                                    & \texttt{v\_bird, vx\_bird, vy\_bird} & $\mathbb{R}$ & Cumulative and XY component of velocity of of the bird \\
                                    & \texttt{m\_bird} & $\mathbb{R}^+$ & Mass of the bird. \\
                                    & \texttt{bird\_id} & $\mathbb{Z}$ & Sequence in which the birds will be fired. \\ \midrule
    \multirow{2}{*}{Pigs}      & \texttt{x\_pig, y\_pig} & $\mathbb{R}$ & XY location of the pig \\
                                    & \texttt{r\_pig, m\_pig} & $\mathbb{R}^+$ & Radius \& Mass of the pig \\ \midrule
    \multirow{3}{*}{Blocks}    & \texttt{x\_block, y\_block} & $\mathbb{R}$ & XY location of the block \\
                                    & \texttt{block\_height, block\_width} & $\mathbb{R}^+$ & Height and Width of the block \\
                                    & \texttt{block\_mass} & $\mathbb{R}^+$ & Mass of the block \\ \midrule
                                    % & \texttt{block\_life, block_stability} &  & \\
    \multirow{3}{*}{Platforms} & \texttt{x\_platform, y\_platform} & $\mathbb{R}$ & XY location of the platform \\
                                    & \texttt{platform\_width} & $\mathbb{R}^+$ & Width of the platform.\\
                                    & \texttt{platform\_height} & $\mathbb{R}^+$ & Height of the platform.\\
    \bottomrule
    \end{tabular}
    \caption{Modeling different objects in Angry Birds.}
    \label{tab:model_predicates}
\end{table*}

% \todo{Add a small explanation of table and highlight undecidability of the problem and connect it to this paragraph.}
% \subsection{Why PDDL+?}
Different aspects of Angry Birds can be modelled with different levels of expressiveness. For example, PDDL2.1 \cite{fox2003pddl2} can be used to model the flight of the bird using numeric fluents similar to an FSTRIPS+ \cite{ramirez2017numerical}. Table \ref{tab:lang-comparison} provides a comparison of PDDL2.1, PDDL2.2 \cite{edelkamp2004pddl2}, FSTRIPS+, and PDDL+, and what parts of the game can be modelled using them. PDDL2.2 provides a close competition for modeling, however, PDDL+ provides support for all parts of the game.

PDDL+ planning problems are difficult to solve due to immense search spaces, and complex system dynamics. Indeed, planning in hybrid domains is challenging because, apart from the state space explosion caused by discrete state variables, the continuous variables cause the reachability problem to become undecidable~\cite{alur95algorithmic}. 
Thus, clever design choices when modeling in PDDL+ are often crucial to solving PDDL+ problems, arguably more so than for any other planning domain definition languages. %We discuss these choices in our PDDL+ model below. 


% Understanding the challenges of planning with such feature-rich models enabled us to make informed design decisions and preemptively mitigate any risks associated with PDDL+. 
%  Next, we describe the contents and dynamics of the Angry Birds PDDL+ domain, highlighting how even complex and composite features can be efficiently encoded in the model and still yield solvable problems. 




% Though there have been many attempts to define other feature-rich, models, PDDL and its versions is accepted as the de facto standard planning language. One notable example is an extension to 




% a bird in f

% via derived predicates. 

%  (except continuous effects)


% The biggest gap between PDDL2.1/2.2 and realistic scenarios is the lack of support for modeling external influence. In Angry Birds, the agent needs to either utilize or mitigate the effects of activity beyond its control, such as gravity or explosions. In PDDL2.1, all change originates from the agent, whereas PDDL2.2 included only limited support for exogenous activity via derived predicates. 

% Less expressive languages have been in use for decades and a multitude of efficient approaches have been developed over the years. The simple structure of the models and a limited set of features have allowed research to concentrate on improving solvers and tackling large-scale problems. These languages are mostly propositional-only, and include STRIPS \cite{fikes1971strips}, ADL \cite{pednault1987formulating}, and PDDL \cite{mcdermott1998pddl}. However, these planning languages are unable to accurately represent real-world systems, only planning in restricted sub-problems and delegating reasoning to external solvers or semantic attachments}
% %On the other hand, more expressive languages enable incorporating a rich set of features into the model without the need for external methods. 
% One of the biggest leaps in expressiveness was made by the introduction of PDDL2.1~\cite{fox2003pddl2} which included support for numeric variables, explicit representation of time and duration, plan metrics, and continuous action effects (both linear and non-linear). It was quickly superseded by PDDL2.2~\cite{edelkamp2004pddl2} which encapsulates its predecessor's features (except continuous effects), and extended it by including timed-initial literals (TILs; unconditional events occurring at a pre-specified time in the plan) and derived predicates (propositional variables derived from \textit{if-then} rules rather than action). Both of these are powerful planning languages that standardized the modeling of important system features that edged AI planning closer to accurately representing real-world problems. However, the biggest gap between PDDL2.1/2.2 and realistic scenarios is the lack of support for modeling external influence. In Angry Birds, the agent needs to either utilize or mitigate the effects of activity beyond its control, such as gravity or explosions. In PDDL2.1, all change originates from the agent, whereas PDDL2.2 included only limited support for exogenous activity via derived predicates. 

%  (except continuous effects)

% Though there have been many attempts to define other feature-rich, models, PDDL and its versions is accepted as the de facto standard planning language. One notable example is an extension to Functional STRIPS (FSTRIPS) described by \cite{ramirez2017numerical}. It allows modeling of hybrid systems but lacked full support for instantaneous exogenous activity, and is only able to define them as global constraints that invalidate the solution if triggered. Angry Birdscontains many cases where exogenous happenings have positive or neutral effects (w.r.t. finding a solution), and are a natural and integral part of the system (e.g. collisions). The FSTRIPS extension does not have the expressiveness to define such cases, so can not model Angry Birds with sufficient accuracy. 


\subsection{Angry Birds Objects in PDDL+}

%Angry Birds is a difficult scenario to define as a planning domain. It contains a variety of distinct object types and complex interweaving dynamics\footnote{Our Angry Birds PDDL+ model relies on knowledge obtained from Angry Birds where possible. Beyond parameters we could reliably source, we approximate the behavior and parameters to the best of our knowledge via experiments and observations.}. However, PDDL+ facilitates the modeling of such constructs and their implementation.

The objects in the Angry Birds domain are separated into four types: \textit{birds, pigs, blocks, and platforms}. %, each one is a crucial component of the game and requires explicit representation. 
Note that we did not list the slingshot, which is where the birds are launched from, as an object in the domain. 
Instead of explicitly modeling the slingshot, we assign the coordinates of the at-rest slingshot to every bird that is about to be launched.
Avoiding modeling unnecessary objects helps in maintaining a light-weight domain definition. 
Table \ref{tab:model_predicates} describes different predicates that were used to model birds, pigs, blocks and platforms. The state is defined using the XY location of each object and their specific properties, e.g., mass or radius.

% \todo{Add formal details about modeling Birds and objects using PDDL+. Can be done using a separate table. and talking in a small paragraph. Then this complete section can be replaced by the table and the discussion.}


% \noindent\textbf{Birds} The only objects directly controlled by the player are the birds. Each bird object requires a set of numeric functions to track their state, including XY coordinates, velocities, mass, bounce count (denoting the number of times the bird collided with any other object), and an ID number to keep the order in which the birds are launched. This set of functions is supplemented by a single predicate for each bird stating whether it has been released from the slingshot. 

% \noindent\textbf{Pigs} %Pigs are simply targets in our PDDL+ model, they 
% A pig is represented by its XY position, radius, and mass functions, as well as a single predicate stating whether it is alive. Pigs are mostly static entities in Angry Birds and their PDDL+ representation is a straightforward translation.

% \noindent\textbf{Blocks} 
%We use a minimal PDDL+ representation for blocks. 
% \noindent\textbf{Blocks} Each block is defined by its basic characteristics, namely the XY coordinates, as well as the width and length values. Blocks are also assigned values for mass, life-points, and stability (force needed to topple it), calculated based on the shape and size of the block, material (ice, wood, or stone), and its height over the ground (the taller a structure, the lower its stability). Finally, each block is assigned a predicate dictating whether it is explosive, to account for TNT crates. While there are different types of blocks in the Angry Birds game, we  differentiate between them only by the above attributes to keep the entire block ontology unified, and reduce unnecessary bloating of the domain. 

% \noindent\textbf{Platforms} These Angry Birds elements are entirely static and indestructible, and modeled in PDDL+ by XY coordinates, width, and height functions.

% \smallskip
% The model also includes variables for global parameters and auxiliary operating parameters. The former includes important variables used globally in the system dynamics calculations, such as gravity, launch angle, and initial velocity. Auxiliary operating parameters are variables which play a supporting role such as denoting the currently active bird.

% To facilitate planning efficiency, we aimed to minimize the number of predicates and functions per object type required in the domain. This design decision is motivated by performance cost of planning in Angry Birds levels containing a large number of distinct objects. As described above, each object is defined in PDDL+ by a set of state variables. Densely populated levels can cause the resulting PDDL+ states to drastically increase in size (often containing hundreds of variables each), which, in turn, can lead to significant drop in planner performance.


% The number of predicates and functions per object type in the domain is purposely limited to facilitate efficient planning by preventing the domain and state size from exploding. This design decision was motivated by mitigating the performance cost of planning in Angry Birds levels, which often contain a multitude of distinct objects. As described above, each object is defined in PDDL+ by a set of state variables. Densely populated levels can cause states to drastically increase in size (often containing hundreds of variables each), which, in turn, can lead to significant drop in planner performance.

\subsection{Angry Birds Dynamics in PDDL+}

The PDDL+ domain of Angry Birds features a variety of dynamics which dictate the change and evolution of the system. This, however, creates planning problems with vast search spaces and large branching factors, making them computationally difficult to solve. Thus, mitigating the issues of solvability and efficiency was at the forefront of the domain's development at each step of the process, and is evident in the resulting PDDL+ model's composition.

\subsubsection{Launching a Bird}
% Angry Birds only features two types of actions for the player. %/agent. 
% The first is launching a bird from the slingshot at a chosen angle, and the second is activating the special tap actions (if available). We do not include levels requiring the special tap actions for this paper, but plan to add it in future work. 
Our PDDL+ model of Angry Birds contains a single action -- launching a bird from the slingshot at a chosen angle. 
Launching of the bird is split into three separate phases: selecting the launch velocity, finding the launch angle, and releasing the bird. 
%This activity requires in-depth reasoning about the locations of the targeted structures, their strength and stability, and the force required to destroy or collapse them, in addition to the locations of pigs and other objects of interest (e.g. TNT crates). Encoding all of this information into actions would require complex constructs for each phase, and would result in concurrent and entangled behavior to account for both the angle and initial launch velocity. To mitigate the complexity of the launch action, we leverage the expressiveness of PDDL+ which allows us to delegate much of the activity away from the agent's control. First, we fix the launch velocity to its maximum possible value, removing the need for the agent deliberating over the power adjustment.
In our model, we fix the launch velocity to its maximum possible value to decrease the solution search space. This decision is motivated by the fact that maximum velocity shots provide widest range of targets and the final impact velocity of the bird is directly proportional to damage it causes to the objects. Moreover, the same level of reachability can be modeled by changing the angle of slingshot alone instead of modulating the speed and angle together.
%This significantly simplifies the search phase and avoids the complexity associated with the interwoven nature of launch angle and velocity that the agent would have to otherwise reason with simultaneously. 


%, removing the need for the agent deliberating over the power adjustment. This decision is motivated by the fact that maximum velocity shots are the most common as they provide widest range of targets and impact velocity is proportional to damage. This significantly simplifies the search phase and avoids the complexity associated with the interwoven nature of launch angle and velocity that the agent would have to otherwise reason with simultaneously. 

Next, we address the construction of the action responsible for selecting the launch angle. In a naive approach, one could encode a set of instantaneous actions that increase or decrease the angle, supplemented by an action releasing the bird from the slingshot.
%Another formalism that could be used to encode the launch action is to model it as a durative action with continuous effects (from PDDL2.1), increasing the angle over time.
%However, leveraging the advantages of PDDL+ allows us to only require a single instantaneous action releasing the bird from the slingshot. 
In this approach, the search space will include many cycles, corresponding to increasing and decreasing the launch angle by equal amounts. 
Instead, we pair the release action with a supporting process that continuously adjusts the angle as soon as a new bird is placed on the slingshot, ready to be launched. 
This follows the Theory of Waiting~\cite{mcdermott2003reasoning}, which sees the agent idly waiting until the world evolves into a favorable state in which to execute actions. 
This encoding reduces the number of decision points, cycles, and branching factor, mitigating state space explosion.

Concretely, the model includes an \emph{increase\_angle} process for increasing the launch angle and a \emph{release\_bird} action (fig.~\ref{fig:action-pa-twang}) whose effects assign values to the vertical and horizontal velocity variables based on the angle. The velocities are then used to model the ballistic flight of the birds. 
Once a bird is launched, another process is triggered. This \textit{flying} process (fig.~\ref{fig:process-flying}) models ballistic flight of the active bird, updating its velocity and location over time, according to the governing equations of motion. 
Note that this process represents the domain's non-linear behavior, which is a major obstacle in PDDL+ planning as many planners struggle to handle complex dynamics.

% Unfortunately, trigonometric functions are not included in PDDL+, thus our model relies on Bhaskara's small-angle approximations for sine (\ref{eq:sin}) and cosine (\ref{eq:cos}), respectively (in the latter case, $\theta$ is expressed in radians). 
% \begin{equation} \label{eq:sin}
% \sin \theta \degree \approx \frac{4 \theta(180 - \theta)}{40500 - \theta(180 - \theta)}
% \end{equation}
% \begin{equation} \label{eq:cos}
% \cos \theta \approx 1 - \frac{\theta^2}{2}
% \end{equation}

% \setlength{\belowcaptionskip}{-15pt}

\begin{figure}
\begin{center}
% \begingroup
    \fontsize{8pt}{10pt}\selectfont
\begin{verbatim}
 (:action release_bird
    :parameters (?b - bird)
    :precondition (and
        (= (active_bird) (bird_id ?b))
        (not (angle_adjusted))
        (not (bird_released ?b)) )
    :effect (and
        (assign (vy_bird ?b) 
            (* (v_bird ?b) SIN(angle)))
        (assign (vx_bird ?b) 
            (* (v_bird ?b) COS(angle)))
        (bird_released ?b) (angle_adjusted)) )
\end{verbatim}
% \endgroup
\caption{PDDL+ action for launching a bird. Since trigonometric functions are not included in PDDL+, model relies on Bhaskara's small-angle approximations for SIN and COS.}
\label{fig:action-pa-twang}
\end{center}
\end{figure}

% \noindent As far as the birds' special actions are concerned, we defer their implementation to the next phase of our work. While most of their consequences are straightforward to implement, significant deliberation about their encoding is needed to maintain the model's efficiency. In the current version of the PDDL+ model, we only consider the basic behavior of birds without their special powers\footnote{While the tap actions are not yet implemented, in the Angry Birds environment where we execute our plans, the black bird's ability to explode is also triggered upon collision with any object after launch. At present, we keep the PDDL+ model oblivious to this ability, arguing that the model reflects the worst case scenario of launching an exploding bird. In practice, we notice that launching black birds yields higher returns in terms of points and destroyed targets than anticipated but does not perform worse than predicted during the planning phase.}. 

% The Angry Birds domain differs from more typical PDDL+ applications in that the vast majority of system dynamics is defined via exogenous activity. The agent's role is significantly diminished, simultaneously preserving the accuracy of the model and ensuring efficiency. 

% \noindent\textbf{Processes}
% Continuous behaviors in Angry Birds are encapsulated using processes in the PDDL+ model. The first process is a linear increase in the current angle value, prior to launching a bird, based on a pre-defined rate of change (i.e. angle\_rate). 
% % The increase\_angle process (fig.~\ref{fig:process-launch}) is triggered as soon as the next bird in the queue is available for launch (i.e. at time-point t=0, or when the previously launched bird has expired). 
% The process for increasing the angle is triggered as soon as the next bird in the queue is available for launch (i.e. at time-point t=0, or when the previously launched bird has expired). 

% % \setlength{\belowcaptionskip}{-15pt}
% \begin{figure}
% \begin{center}
% % \begingroup
%     \fontsize{8pt}{10pt}\selectfont
% \begin{verbatim}
%  (:process increasing_angle
%     :parameters (?b - bird)
%     :precondition (and
%         (not (angle_adjusted))
%         (= (active_bird) (bird_id ?b))
%         (not (bird_released ?b))
%         (< (angle) (max_angle))
%         (>= (angle) 0) )
%     :effect (and
%         (increase (angle) 
%             (* #t (angle_rate)))) )
% \end{verbatim}
% % \endgroup
% \caption{Process increasing the angle before launch.}
% \label{fig:process-launch}
% \end{center}
% \end{figure}

%\subsubsection{Bird Flying}
% Once a bird is launched, another process is triggered. This \textit{flying} process (fig.~\ref{fig:process-flying}) models ballistic flight of the active bird, updating its velocity and location over time, according to the governing equations of motion. 
% Note that this process represents the domain's non-linear behavior, which is a major obstacle in PDDL+ planning as many planners struggle to handle complex dynamics.
%This is a crucial component of the system dynamics and the focal point of the Angry Birds domain. 
%All post-launch environment changes must occur while this process is active. 
%The flying process represents the domain's non-linear behavior, which is a major obstacle in PDDL+ planning as many planners struggle to handle complex dynamics. This situation is further complicated by the effects of multiple events which fire while the process is active and can drastically change its behavior with each triggered activity. Crucially, this process could not be encoded using the predecessors of PDDL+ as they either cannot accurately represent the system dynamics or require the agent's command to begin and end continuous activity. 

% \setlength{\belowcaptionskip}{-15pt}


\subsubsection{Events}
%\noindent\textbf{Events} 
%In the grand scheme of our Angry Birds model composition, 
All the objects interact with the environment and other objects through events. Below, we discuss three different classes of events for modeling collisions, motion of the blocks, and auxiliary events at the start and end of each shot. These events model complex behaviors of destruction, explosion and structure collapse when hit by birds. In our planning model, all the post launch system dynamics are caused by events interacting with the effects of the flying process.

\begin{figure}
\begin{center}
% \begingroup
    \fontsize{8pt}{10pt}\selectfont
\begin{verbatim}
 (:process flying
    :parameters (?b - bird)
    :precondition (and
        (bird_released ?b)
        (= (active_bird) (bird_id ?b))
        (> (y_bird ?b) 0) )
    :effect (and
        (decrease (vy_bird ?b) 
            (* #t (gravity) ))
        (increase (y_bird ?b) 
            (* #t (vy_bird ?b)))
        (increase (x_bird ?b) 
            (* #t (vx_bird ?b)))) )
\end{verbatim}
% \endgroup
\caption{PDDL+ process of a bird's flight.}
\label{fig:process-flying}
\end{center}
\end{figure}

Collisions are the first class of events that are highlighted in an Angry Birds problem. The main aim of the game is to kill pigs by directly colliding with them, causing nearby explosions, or destroying blocks such that pigs are killed by collapsing structures or falling from heights. Each of those cases is a crucial tactic in Angry Birds, and any accurate model of the game must encompass such mechanics. 

As an example event, direct interactions between a bird and a pig are modeled based on elastic sphere collisions in two dimensions. Pigs are very fragile and disappear upon almost any contact from another object. Thus, after a collision, only the bird's resulting velocity needs to be calculated, according to the angle-free elastic collision formula:
\begin{equation} \label{eq:collision}
\vec{v}_{b}' = \vec{v}_{b} - \frac{2m_{p}}{m_{b}+m_{p}} \frac{\langle \vec{v}_{b} - \vec{v}_{p},\vec{x}_{b} - \vec{x}_{p}\rangle}{||\vec{x}_{b} - \vec{x}_{p}||^2}(\vec{x}_{b} - \vec{x}_{p})
\end{equation}
where $\vec{v}_b$ and $\vec{v}_p$ are the velocity of the bird and pig, respectively, $\vec{x}_b$ and $\vec{x}_p$ are their respective positions, $m_b$ and $m_p$ are their respective masses, and the angle brackets denote an inner product of two vectors. 
Translating this equation into a PDDL+ event allows us to model behavior beyond the obvious solutions (i.e., aiming directly at an exposed pig). It enables the planner to find highly innovative solutions such as deliberately and directly killing two pigs with one bird (fig.~\ref{fig:double-hit}).
Similarly, we model events for collisions with the grounds, which also enables finding trick shots in which the bird bounces off the ground to hit a pig or a block.

The ground in Angry Birds is often overlooked when playing the game but it can prove useful when aiming at difficult targets. As such, an event (seen in fig~\ref{fig:ground-event} is defined in the domain that influences the bird's trajectory after bouncing off the ground, based on energy lost during the collision (i.e., ground damper). Figure~\ref{fig:bounce-hit} shows how a ground-bounce event helps in killing a pig when a direct hit is impossible.

% \setlength{\belowcaptionskip}{0pt}
\begin{figure}[!ht]
\begin{center}
\begingroup
    \fontsize{8pt}{10pt}\selectfont
\begin{verbatim}
 (:event collision_ground
  :parameters (?b - bird)
  :precondition (and
     (= (active_bird) (bird_id ?b))
     (<= (y_bird ?b) 0)  )
  :effect (and
     (assign (y_bird ?b) 1)
     (assign (vy_bird ?b) 
     (* (* (vy_bird ?b) -1)(ground_damper)))
     (assign (bounce_count ?b) 
     (+ (bounce_count ?b) 1))) )
 \end{verbatim}
\endgroup
 \caption{Event modeling a bird bouncing off the ground.}
\label{fig:ground-event}
 \end{center}
\end{figure}


% \setlength{\belowcaptionskip}{-10pt}
\begin{figure*}
\begin{center}
\includegraphics[width=0.23\textwidth]{images/double1.png}
\includegraphics[width=0.23\textwidth]{images/double2.png}
\includegraphics[width=0.23\textwidth]{images/double3.png}
\includegraphics[width=0.23\textwidth]{images/double4.png}
\end{center}
\caption{Two pigs-one bird plan in execution.}
\label{fig:double-hit}
\end{figure*}

\begin{figure*}
\begin{center}
\includegraphics[width=0.23\textwidth]{images/under1.png}
\includegraphics[width=0.23\textwidth]{images/under2.png}
\includegraphics[width=0.23\textwidth]{images/under3.png}
\includegraphics[width=0.23\textwidth]{images/under4.png}
\end{center}
\caption{Solution requiring bouncing off the ground and collapsing a structure to kill the target pig.}
\label{fig:bounce-hit}
\end{figure*}


\smallskip
Most interactions in a level occurs between birds and blocks in the scene. Blocks form structures which act as a protection barrier for targeted pigs. Thus, modeling collisions with blocks is vital to playing Angry Birds. However, similarly to pig collisions, we opt to only define the event's effects in terms of the impact it has on the bird's velocity and the blocks stability and life values. There are two events, modeling bird-block collisions with the distinction based on whether the block is stable enough to withstand the impact without being knocked out of place or destroyed entirely. In such cases, the bird bounces back off the block, otherwise it continues moving forward with a diminished velocity. 

Modeling the motion of blocks after impact and chains of block-block collisions would be prohibitively difficult, requiring a process for tracking each individual block's change in position and rotation over time, as well as a set of events to account for secondary interactions. 
These additions would only marginally improve the domain's accuracy, though the planner would experience a drastic drop in performance, having to keep track of dozens of simultaneous non-linear processes in every state after a collision. Instead, we model one central block-block interaction, namely dislodging or destroying blocks supporting larger structures. In such cases, an event is triggered for every supported block, reducing its stability value to 0, repositioning the block to the ground, and adjusting its life value to account for fall damage. Analogously, we defined another event for killing pigs which sit atop collapsing block structures. %Figure~\ref{fig:bounce-hit} showcases the execution of a plan which includes a ground-bounce event, followed by the collapse of a tall structure to kill an otherwise unreachable target pig.

%We do model one central block-block interaction, namely dislodging or destroying blocks supporting larger structures. In such cases, an event is triggered for every supported block, reducing its stability value to 0, repositioning the block to the ground, and adjusting its life value to account for fall damage. AnalBob and Betty Beyster Buogously, we defined another event for killing pigs which sit atop of collapsing block structures. Figure~\ref{fig:bounce-hit} showcases the execution of a plan which includes a ground-bounce event, followed by the collapse of a tall structure to kill an otherwise unreachable target pig.

% We decided that modeling the motion of blocks after impact and chains of block-block collisions would be prohibitively difficult. Such endeavor would require a significant implementation effort, defining a process for tracking each individual block's change in position and rotation over time, as well as a set of events to account for secondary interactions. These additions would only marginally improve the domain's accuracy, though the planner would experience a drastic drop in performance, having to keep track of dozens of simultaneous non-linear processes in every state after a collision.

% We do model one central block-block interaction, namely dislodging or destroying blocks supporting larger structures. In such cases, an event is triggered for every supported block, reducing its stability value to 0, repositioning the block to the ground, and adjusting its life value to account for fall damage. Analogously, we defined another event for killing pigs which sit atop of collapsing block structures. Figure~\ref{fig:bounce-hit} showcases the execution of a plan which includes a ground-bounce event, followed by the collapse of a tall structure to kill an otherwise unreachable target pig.

The two final block-related events concern exploding TNT crates. Any impact upon the crate causes an explosion destroying all objects in its immediate vicinity and displacing objects further away. In our PDDL+ model, we incorporate two simple events modeling the explosive destruction of pigs and blocks near a TNT crate. As previously indicated, we omit movement of objects caused by explosions. 

% The described events account for a significant portion of interactions during gameplay where birds are used to destroy multi-block structures to expose or kill targeted pigs, when a direct hit is infeasible or impossible.

Platforms are indestructible objects which the agent needs to reason with to avoid poor results. However, almost no meaningful interactions occur between birds and platforms, and thus our domain contains an event modeling this type of collision but only to discourage the agent from shooting at platforms. The event deters the agent by nullifying the bird's velocity and expiring it without any impact on the level.

Finally, we encode a supporting event which models the end of life for an active bird. In the game, a bird expires when it moves beyond the limits of the scene or when it stops moving. Modeling the exact timing of the end of a bird's life is quite difficult. Instead, we define the birds usefulness in terms of the number of bounces off other objects. Each bounce partially reduces the bird's momentum, and its collision impact becomes negligible after three bounces. Therefore, we define a termination event for the active bird once it has bounced three times. The result of this event is that the next bird is placed on the slingshot for launch.

\textbf{Goals}
While the goal of Angry Birds is to destroy all the pigs, we consider a simpler formulation for individual planning problems. The simplified problem splits the original scenario into single-bird episodes with the goal of killing at least one pig. During evaluations we start with this simplified approach and use a direct trajectory calculator as a contingency strategy when the planning phase fails to find a solution.
Table \ref{table:domain_stats} gives a breakdown of the number of events, processes and actions in a domain, and the number of objects in a typical level. As noted above, Angry Birds is an atypical domain as the agent only has one action (release bird). 

\begin{table}
    \centering
    \small
    \begin{tabular}{l|r|rrr}
    \toprule
        Level & Objects & Actions & Events & Processes \\
        % & &\multicolumn{3}{c}{Grounded (Domain)} \\
        \midrule
        22 & 7-11   & 5(1)  & 19-114(17)        & 3(3) \\
        25 & 8-12   & 5(1)  & 20-88(17)         & 3(3) \\
        36 & 8-12   & 5(1)  & 30-114(17)        & 3(3) \\
        45 & 9-13   & 5(1)  & 37-162(17)        & 3(3) \\
        46 & 11-15  & 9(1)  & 59-199(17)        & 5(3) \\
        53 & 10-14  & 5(1)  & 28-163(17)        & 3(3) \\
        54 & 8-12   & 5(1)  & 30-114(17)        & 3(3) \\
        57 & 11-15  & 5(1)  & 65-220(17)        & 3(3) \\
        55 & 29-138 & 13(1) & 430-13772(17)     & 7(3) \\
         \bottomrule
    \end{tabular}
    \caption{Number of objects and happenings in the grounded domain (and in the domain template). The main computational load is checking whether events have occurred. }
    \label{table:domain_stats}

\end{table}


\section{Solving Angry Birds with a PDDL+ Planner}
In this section, we describe how we used the PDDL+ modeling of Angry Birds specified above to design an Angry Birds playing agent called \emph{Hydra}. We detail how Hydra interacts with the game, the PDDL+ planner it uses, and search techniques we implemented to improve its efficiency. 

% \begin{figure}

% \begin{algorithmic}
% \State $ open \gets initial\_state $
% \While{$ open \neq \emptyset$ } 
% \State $ s \gets select\_next(open) $
% \If{is-goal(s)}
% \State $ return path to s $
% \EndIf
% \State $ \mathcal{A} \gets available\_actions(s) \cup time\_passing $
% \State $ \mathcal{S} \gets \{ T(s, a) \,\, \textbf{for} \,\, a \,\,\textbf{in}\,\, \mathcal{A}\} $
% \State $ open \gets open \cup \mathcal{S} $
% \EndWhile
% \end{algorithmic}
% \caption{Planner main loop. $ select-next $ chooses a node to expand according to the active algorithm, e.g. BFS, DFS or GBFS. $ time-passing $ is a psuedo-action that advances time in small, descrete steps. The step size is a parameter of the planning package. }
% \label{alg:nyx_pseudocode}
% \end{figure}

\subsection{From Angry Birds to PDDL+}

The game playing API represents an Angry Birds level as a list of labeled objects and their locations. Hydra supplements this information with background knowledge of the game and its objects, such as the mass of the different birds. Then, it automatically translates all the collective relevant information of the current level into a PDDL+ problem file. 

Next, it uses a PDDL+ planner to generate a plan that kills at least one pig using one bird. If no plan is found in 30s, we execute a default non-planning action, namely a direct shot at a random pig. 
If a plan has been found, Hydra performs the plan. After performing an action, Hydra observes the updated state of the game, again via the game playing API. Then, it generates a corresponding PDDL+ problem as before, and generates a new plan from the current state. This continues until the level ends, either passing the level or failing to do so before exhausting all the available birds. 

\subsubsection{PDDL+ Planners}
Any state-of-the-art PDDL+ planners can be used for solving the Angry Birds game, such as, SMTPlan+~\cite{cashmore2016compilation}, UPMurphi~\cite{della2009upmurphi}, DiNo~\cite{piotrowski2016heuristic} and ENHSP~\cite{scala2016interval}.
However, during experimentation we faced several challenges due to costly domain-independent heuristics and lack of ability to incorporate other heuristics for improved results in the game.
Thus, we used our own domain-independent PDDL+ planner in Hydra, which is written in Python and has been used successfully incorporated for several other domains.
Now we will discuss some of the search algorithms that are supported by our planner and the two heuristics that we evaluated for the domain.

% In general, any planner capable of supporting the full range of the PDDL+ language can be used in Hydra. However, our PDDL+ modeling of Angry Birds requires a planner that supports a wide range of model features and non-linear system dynamics, which some of the existing planners do not support. 
% For example, the SMTPlan+~\cite{cashmore2016compilation} planner is limited to only dealing with polynomial non-linearity, and does not scale well with the number of events and processes. 
% UPMurphi~\cite{della2009upmurphi} is a highly optimized PDDL+ planner written in C++, which solves hybrid planning probindependentlems via a uniform discretization approach. It can reason with the entire set of PDDL+ features and non-linear dynamics. We have implemented a preliminary version of Hydra with UPMurphi with some degree of success. Yet, the performance of UPMurphi is limited in this domain, since it does not naturally support adding heuristics to the search. 
% DiNo~\cite{piotrowski2016heuristic} is a PDDL+ planner that builds on the strengths of UPMurphi and extends it by implementing in it a domain-independent heuristic. 
% However, DiNo's heuristic is computationally expensive and does not perform well on the Angry Birds domain, defeating its purpose. Extending DiNo and UPMurphi to allow adding new heuristics have proven to be technically challenging. 
% ENHSP~\cite{scala2016interval} is an efficient discretization-based planner for numeric domains equipped with a set of domain-independent heuristics. On paper, of the existing planners, ENHSP is best suited to tackle the Angry Birds problems as it can reason with all PDDL+ features and  non-linear dynamics. However, when tested, ENHSP does not solve Angry Birds problems, instead declaring them as unsolvable. Furthermore, ENHSP is not optimized for grounding, and can struggles with domains that have large numbers of grounded variables (characteristic of Angry Birds, especially for complex levels)\footnote{While not the focus of this paper, we will further investigate the applicability of ENHSP to PDDL+ Angry Birds in future work.}.
% Therefore, we used our own domain-independent PDDL+ planner in Hydra, which is written in Python and has been used successfully in several other domains. Algorithm \ref{alg:nyx_pseudocode} gives simplified pseudo-code for our planner's main loop. 

% Additionally, we incorporate custom heuristics which are not supported by other state-of-the-art PDDL+ planners available at present. 

\subsection{Search and Heuristics}
Our domain-independent PDDL+ planner supports several search algorithms including depth-first search (DFS), breadth-first search (BFS), and Greedy Best-First Search (GBFS)\footnote{Our planner also support A*~\cite{hart1968formal}, but since we do not aim to find optimal solutions, there is no benefit in using A*. Nevertheless, we performed an initial evaluation of using A* in Hydra, but the results were subpar.}.
GBFS requires heuristic function to evaluate explored states. We propose two heuristics for our domain. %are described below. 



\subsubsection*{The Score Heuristic} While the agent aims to \emph{solve} levels in the game, i.e., kill all pigs, the game itself associates a score to each state. Specifically, destroying pigs and blocks increases the score of the current level. The score heuristic, denoted $H_S$, uses exactly this score as a heuristic.
The agent needs to analyze the opportune angles at which to release the slingshot and evaluate distant effects. The heuristic is calculated based on the number of pigs and blocks the bird 
will hit multiplied by a constant value.
% The heuristic is admissible as only first hit is calculated to count the number of objects that will be hit by the bird.
% While intuitively appealing, this heuristic turns out to be ineffective since the PDDL+ model of the game has only one action (release the slingshot at an opportune angle), and the action has distant effects. The score is not updated as a direct consequence of the action, but rather it is only updated far in the future once the bird collides with other objects.

%, so there is little to no benefit for using score as a heuristic. 

% scores 
% actions are selected in Angry Birds early, significant

% for several reasons. 
% First, Angry Birds has very few actions --- only the choice of angle at which to launch each bird. These decision points are far removed from the score, so there is little to no benefit for using score as a heuristic. 

\subsubsection*{The Proximity Heuristic} The second heuristic we used calculates the active bird's proximity to the nearest pig, as a function of flight time and direction. This heuristic, denoted $H_P$, is computed as follows. 
\begin{eqnarray}
    \vec{d} = \vec{x}_{target} - \vec{x}_{bird} && \text{vector from bird to pig}   \\
    \hat{d} = \frac{\vec{d}}{\lVert \vec{d} \rVert} & &\text{unit vector of same}   \\
    H_P(s) = \frac{\rVert \vec{d} \rVert}{\vec{v}_{bird} \cdot \hat{d}} && \text{flight time to pig} 
\end{eqnarray}
Where $ \vec{x}_{target}, \vec{x}_{bird} $ and $ \vec{v}_{bird} $ are the coordinate vectors of the pig, the bird and the velocity of the bird respectively, $ \lVert \vec{x} \rVert $ is the Euclidean norm, and $ \cdot $ is the inner product. 
Using this heuristic results in giving preference to states where the bird is not only closer to a pig, but also heading towards it, reducing the effort spent on distant trajectories. As a consequence, this heuristic also gives preference to low trajectories.
% \update{
The heuristic is admissible as the straight-line distance to the target underestimating the flight time. 
% }

% \todo{discussion about the heuristic being admissible}


\subsection{Helpful States}
Orthogonal to the choice of heuristics, our system incorporates domain-specific strategies into the search process.
This mechanism, that we refer to as \emph{helpful states}, is based on the ``helpful operators'' mechanism from classical planning~\cite{richter2009preferred}. 
Helpful operators in classical planning is a mechanism that prioritizes the use of a subset of the operators while planning. 
Porting this mechanism to our case requires some adaptation, since the number of actions in our domain is very limited.

Our \emph{helpful states} mechanism accepts some definition of what a preferred state is, and then prioritize expanding such states. In our implementation we defined a preferred state as one in which the active bird is on a trajectory that is expected to hit a pig or a TNT block. Checking if the active bird is on a trajectory to hit an object is done by using the equation for ballistic motion to create a function $ y_t = f_i(x_t) $ for each potential target $i$ (pig or TNT block). Given a state, we can calculate whether the bird's $y$ is close to the desired value, and if so, mark it as a preferred state.
The search strategy can also be understood as doing Monte-Carlo rollout \cite{chaslot2008progressive} only on a smaller set of states, instead of performing a complete rollout on all the search states.

Throughout the search we maintain two open lists, one that only includes preferred states, and one that includes all states. During the search, we alternate between expanding states from the open list with the preferred states and the regular open list. 
To ensure that preferred states are reached early, we also mark all states where the bird has not yet been launched as preferred. This ensures the planner reasons with the full range of possible launch angles. 
% , and defines states where a bird is on a direct trajectory to a pig or a TNT block as preferential\footnote{This is done by using the equation for ballistic motion to create a function $ y_t = f_i(x_t) $ for each potential target $ i $ (pig or TNT block. Thus, given a state, we can calculate whether the bird's $y$ is close to the desired value, and if so mark it as a preferred state. }. This heuristic is also able to mark angles leading to said trajectories as preferred\footnote{Using the initial velocity $ v_{x_0} $ from the same calculation.}, further accelerating the search. This heuristic enables HYDRA to solve more of the complex levels, but it's real strength is in reducing planning time and nodes expanded.  (see table \ref{table:planning_time}). 
Note that this ``preferred states'' mechanism is fairly general, and allows different definitions of preferred states, and multiple open lists with different priorities. This is a topic for future research. 

% with defines a criteria by which to identify a helpful state. 

% the agent can add to a plan, that are given priority by alternating between choosing any action the agent can perform and only limiting to actions from the preferred list. 

% , and defines states where a bird is on a direct trajectory to a pig or a TNT block as preferential\footnote{This is done by using the equation for ballistic motion to create a function $ y_t = f_i(x_t) $ for each potential target $ i $ (pig or TNT block. Thus, given a state, we can calculate whether the bird's $y$ is close to the desired value, and if so mark it as a preferred state. }. This heuristic is also able to mark angles leading to said trajectories as preferred\footnote{Using the initial velocity $ v_{x_0} $ from the same calculation.}, further accelerating the search. This heuristic enables HYDRA to solve more of the complex levels, but it's real strength is in reducing planning time and nodes expanded.  (see table \ref{table:planning_time}). 

% [[Roni: @Yoni, this is very informal. We need a mathematical formula / pseudo code to explain how this was computed exactly. ]]

% \section{Search Enhancements}

% A major benefit in modeling Angry Birds in PDDL+ is that it enables using any domain-independent PDDL+ planner to obtain solutions. In our case, we used a proprietary PDDL+ planner, which we use in multiple domains. 


% \subsection{Heuristics used by HYDRA}
% We evaluated three different heuristic functions to guide the planning process. The first is simple the score: a higher score indicates more pigs and blocks destroyed, and therefore a better state. This turns out not to be an effective guide for several reasons. Firstly, Angry Birds has very few actions - only the choice of angle at which to launch each bird. These decision points are far removed from the score, so there is little to no benefit for using score as a heuristic. 
% The second heuristic we used calculates the active bird's proximity to the nearest pig, as a function of flight time and direction. This heuristic prefers states where the bird is not only closer to a pig, but also heading towards it, reducing the effort spent on distant trajectories. As a consequence, this heuristic also gives preference to low trajectories. 
% %The third heuristic we used is a composite: the first component is simple the heuristic mentioned immediately above (proximity to pig). The second component is adds a score for pigs (fewer pigs remaining is better), and an approximation of the number of birds required to expose pigs that are behind structures. We do not expect this heuristic to have any effect in simple levels (which is does not), and in complex levels our computation resources were insufficient to decide if it is helpful. % This heuristic not included in data currently shown in paper
% Our final heuristic is based on the notion of ``helpful operators'' (\ref{helmert2006fast}), and defines states where a bird is on a direct trajectory to a pig or a TNT block as preferential\footnote{This is done by using the equation for ballistic motion to create a function $ y_t = f_i(x_t) $ for each potential target $ i $ (pig or TNT block. Thus, given a state, we can calculate whether the bird's $y$ is close to the desired value, and if so mark it as a preferred state. }. This heuristic is also able to mark angles leading to said trajectories as preferred\footnote{Using the initial velocity $ v_{x_0} $ from the same calculation.}, further accelerating the search. This heuristic enables HYDRA to solve more of the complex levels, but it's real strength is in reducing planning time and nodes expanded.  (see table \ref{table:planning_time}). 

\section{Experiments}

\begin{table*}[t]
    \begin{center}
    \begin{tabular}{p{0.29\textwidth}p{0.29\textwidth}p{0.29\textwidth}}
         \includegraphics[width=0.29\textwidth]{images/222.png} & \includegraphics[width=0.29\textwidth]{images/225.png} &
         \includegraphics[width=0.29\textwidth]{images/236.png} \\
          22: A level requiring a single low-trajectory shot. & 25: A level requiring a single high-trajectory shot. & 36: In this level, the TNT is too far away to kill the pig.  \\
         \includegraphics[width=0.29\textwidth]{images/245.png} & 
         \includegraphics[width=0.29\textwidth]{images/246.png} & 
         \includegraphics[width=0.29\textwidth]{images/253.png} \\
          45: To kill both pigs with one bird, must hit TNT.  & 46: the TNT is a decoy, must hit each pig with a bird.  & 53: Must hit TNT to win, but it is a difficult shot.  \\
          \includegraphics[width=0.29\textwidth]{images/254.png} & 
          \includegraphics[width=0.29\textwidth]{images/257.png} & 
          \includegraphics[width=0.29\textwidth]{images/555.png} \\
           54: Bird must bounce off far platform wall to hit pig.  & 57: TNT provides a solution to an `impenetrable' box.  & 55: A complex level. 
    \end{tabular}
    \captionof{figure}{A screenshot example of each type of level layout}
    % \caption{An screenshot example of each type of level layout. }
    \label{fig:levels}
    \end{center}
\end{table*}

%The yearly Angry Birds competition~\cite{renz2015aibirds} at IJCAI is ran by researchers from ANU yielded a number of game-playing AI agents. 
We conducted a set of experiments to evaluate the efficacy of various planning-related design decisions explored in this paper to develop Hydra. Particularly, in experiment 1, we studied how different search perform when playing Angry Birds using the PDDL+ model introduced in this paper. Next, in experiment 2, we measured the impact of different heuristics have on controlling the search.  These experiments were conducting on a benchmark set of Angry Birds problems. We also present results from three other domain-specific Angry Birds agents on the benchmark set as a way to situate Hydra's performance. 

%We experimentally compared Hydra to three different agents across a range of game levels. 
%In addition, we compared Hydra with different search algorithms, namely BFS, DFS, and GBFS; different heuristics, namely $H_S$ and $H_P$; and with our helpful states mechanism. 
%The main metric we consider is the number of levels \emph{passed}. For clarity, finding a plan is not sufficient on its own, a level is \emph{passed} only if the resulting plan succeeds when executed in the Angry Birds simulation environment. For the different Hydra agent variants, we also report on runtime, and number of expanded nodes.
% , and number of levels \emph{solved}. We say that a level is solved if our planner was able to find a solution to the corresponding PDDL+ problem within the 30 seconds time limit.
%The results of this comparison are presented in this section. 
\subsubsection{Benchmark Levels}
Our benchmark set of problems contains \emph{simple levels} that contain a single bird and a relatively small number of other objects, and \emph{complex levels} that contain multiple birds and more than 50 objects. The evaluation is designed to test the agents in situations requiring varied strategies. The simple levels require high accuracy shots %and a limited amount of scene understanding and reasoning. 
while the complex levels require a high level of physics reasoning about interaction with blocks, but can often be passed by simple curated rules.

In total, we experimented with 9 types of levels, which are numbered 22, 25, 36, 45, 46, 53, 54, 57, and 55. % @ Yoni and Wiktor: I cut out the most significant digit of the level ID to save space int he table. So, level 22 is actaully level 222, and level 55 is actually level 555. 
Every evaluated agent attempted to solve 25 randomly generated levels of each type. 
An example layout for each type of level can be found in figure \ref{fig:levels}. 
%[[Roni: @Yoni, please add here the text on the 9 levels and the figures]]
Note that the first 8 level types are simple levels and the last one is a complex level.
%[[Roni: @Yoni/Wiktor, add some bla about the levels, not too much, and not related to how well the agents work on them (yet)]]

% , and number of levels \emph{solved}. We say that a level is solved if our planner was able to find a solution to the corresponding PDDL+ problem within the 30 seconds time limit.

% \subsubsection{Examples of levels and analysis}
% \paragraph{Low trajectory simple level} Figure \ref{fig:simple_lvl_low} shows an example of a simple level requiring a low trajectory. On these levels, almost all agents have a high success rate. This is due to an encoded preference for low trajectories, as an effective heuristic for ``natural looking'' levels. The naive ANU agent is an outlier here, as it randomly chooses between high and low trajectories, and does not have high accuracy. 

% \paragraph{High trajectory simple level} Figure \ref{fig:simple_lvl_high} shows an exaple of a simple level requiring a high trajectory. Here, HYDRA significantly out-performs other agents, as it is the only one equipped to actively reason about obstacles. While the Datalab and Eagle's wing agents' heuristics sometimes mark the direct (low) trajectory as ineffective and resort to the high shot, the baseline agent does not, and also does not have the accuracy to make a direct hit on a pig from a high angle. 

% \paragraph{Levels requiring simple physics reasoning} Figure \ref{fig:tnt_level} shows a level requiring a certain degree of reasoning about the interactions between Angry Birds objects. With a single bird and two pigs, the only solution to the level is ignoring the direct shots and hitting the TNT. In these levels, Datalab and Eagle's wing's heuristics lead them astray. Both agents choose to fire directly at the pigs, since they are exposed. This heuristic makes sense in complex levels with many birds, but in this case leads to failure. The naive agent was able to pass two levels due to the inaccuracy that cost is the other high trajectory levels - when aiming at the closer pig, it sometimes hits the TNT block by accident. 

% \paragraph{Complex levels} Figure \ref{fig:complex_lvl} shows an example of a highly-populated level. These levels exhibit much higher variability than the other templates, as a consequence of having more blocks, more pigs, and more birds. In these levels, exact reasoning about object interactions is computationally prohibitive, preventing HYDRA from finding solutions in nearly all cases. However, agents with heuristics designed for such levels can solve them with a high degree of success. Accuracy is not paramount, as any impact on a structure containing a pig is likely to bring it down, pig and all. The choice of which bird to aim at each pig, as an approximate function of the surrounding obstacles, is much more important than the choice of high or low trajectory. 



% Some simple levels require shooting birds in a low trajectory while others require shooting birds in a high trajectory. In most simple levels only a single bird exists. 

%We distinguish between simple levels that require shooting birds in a low trajectory and simple levels that require shooting birds in a high trajectory. 
% We evaluated all agents on two types of levels: \emph{simple levels}, which are levels that contain a relatively small number of objects, and \emph{complex levels}, which are levels that contain more than 50 objects in them. Furthermore, we separated the 

% We evaluated all agents on three types of levels: 
% \begin{itemize}
%     \item \textbf{Type 1.} Simple levels requiring a low trajectory to hit the pig, 
%     \item \textbf{Type 2.} Simple levels requiring a high trajectory to hit the pig, and
%     \item \textbf{Type 3.} Complex levels with many blocks, pigs, and birds. 
% \end{itemize} %simple levels requiring a low trajectory to hit the pig, simple levels requiring a high trajectory shot, and highly populated levels. 




\subsubsection{Competing Agents}
We selected a baseline agent by ANU and two former champions of the IJCAI competition. 
\begin{itemize}
    % \item \textbf{DataLab}~\cite{borovicka2014datalab} is the 2014 \& 2015 champion developed by researchers from Czech Technical University. The agent's actions are based on predicted utility of pre-defined strategies (destroy pigs, target TNT, target round blocks, destroy structures).
    \item \textbf{ANU Baseline}~\cite{stephenson2017creating} is a baseline agent developed by the IJCAI AI Birds competition organizers. This baseline agent targets a randomly selected pig with each active bird, and aims to shoot at the selected pig while disregarding all other objects in the scene.
    %and has predefined windows for tap actions per bird type. The Naive agent only considers the locations of pigs, disregarding all other objects in the scene.
    \item \textbf{DataLab}~\cite{borovicka2014datalab} is the 2014 \& 2015 champion of the IJCAI Angry Birds competition. DataLab's selects from 4 predefined strategies (destroy pigs, target TNT, target round blocks, destroy structures), based on predicted utilities.     
    % \item \textbf{Eagle's Wing}~\cite{wang2017description} is a triple champion (2016-18) developed by University of Waterloo. Eagle's Wing selects from 5 predefined strategies (pigshooter, TNT, most blocks, high round objects, and bottom building blocks), based on predicted utility. Eagle's Wing supplements these strategies with a trained xgbost model to optimize its performance.
    \item \textbf{Eagle's Wing}~\cite{wang2017description} is the 2016-2018 champion of the ICJAI Angry Birds competition. Eagle's Wing selects from 5 predefined strategies (pig-shooter, TNT, most blocks, high round objects, and bottom building blocks), based on predicted utility. Eagle's Wing supplements these strategies with a trained XGBoost~\cite{chen2016xgboost} model to optimize its performance.
    % \item \textbf{Naive Agent}~\cite{stephenson2017creating} was developed by ANU as a baseline for the competition. It targets a randomly selected pig with each active bird, and has predefined windows for tap actions per bird type. The Naive agent only considers the locations of pigs, disregarding all other objects in the scene.
    \item \textbf{Hydra}: We evaluated different variations of Hydra that use different search strategies, breadth-first search (Hy.,BFS), depth-first search (Hy.,DFS), and greedy-best first search (Hy.,GBFS). Three variations of Hy.,GBFS were tested each leveraging different heuristics discussed in this paper: score heuristic ($H_S$), proximity heuristic ($H_P$), and helpful states ($HS$).  
    
\end{itemize}
All agents are designed to work with the ANU Science Birds framework~\cite{renz2015aibirds}. 





%as all agents have been developed to work with its API. The API provides a list of labeled objects and their locations. Our agent supplements this information with our own deduced knowledge of the game and its objects. These are then merged and automatically translated into a PDDL+ problem file, encoding the collective relevant information about the current level. Using a 30s timeout, we attempt to generate a plan that kills at least one pig using one bird. If no plan is found in 30s, we execute a default non-planning action: a direct shot at a random pig. 


% \setlength{\belowcaptionskip}{-12pt}




\subsubsection{Results 1}
\begin{table}
\small
\centering
% \resizebox{\columnwidth}{!}{
\begin{tabular}{@{ }l>{\centering\arraybackslash}p{0.03\columnwidth}>{\centering\arraybackslash}p{0.03\columnwidth}>{\centering\arraybackslash}p{0.03\columnwidth}>{\centering\arraybackslash}p{0.03\columnwidth}>{\centering\arraybackslash}p{0.03\columnwidth}>{\centering\arraybackslash}p{0.03\columnwidth}>{\centering\arraybackslash}p{0.03\columnwidth}>{\centering\arraybackslash}p{0.03\columnwidth}>{\centering\arraybackslash}p{0.03\columnwidth}@{ }}
\toprule
                    & \multicolumn{9}{c}{Levels}                 \\ 
Agent               & 22 & 25 & 36 & 45 & 46 & 53 & 54 & 57 & 55 \\ \midrule
Hy., BFS            & \textbf{21} & \textbf{21} & 18 & \textbf{25} & 18 & 15 & 11 & \textbf{25} & 4  \\
Hy., DFS            & \textbf{21} & 16 & 16 & 21 & 10 & 13 & 11 & 17 & 2  \\
Hy., GBFS($H_S$)    & \textbf{21} & \textbf{21} & 20 & \textbf{25} & 14 & 15 & 14 & \textbf{25} & 1  \\
Hy., GBFS($H_P$)    & \textbf{21} & \textbf{21} & 24 & \textbf{25} & \textbf{25} & \textbf{25} & \textbf{16} & \textbf{25} & 4  \\
Hy., GBFS($HS$)     & \textbf{21} & \textbf{21} & \textbf{25} & \textbf{25} & 24 & \textbf{25} & \textbf{16} & \textbf{25} & 9  \\
\midrule
ANU Baseline        & 18 & 0  & 0  & 16 & 0  & 2  & 0  & 11 & \textbf{17} \\
Datalab             & 15 & 8  & 0  & 24 & 3  & 0  & 0  & 22 & 16 \\
Eaglewings          & 2  & 6  & 0  & 23 & 2  & 0  & 1  & 21 & 15 \\ \bottomrule
\end{tabular}
% }
\caption{Evaluation results for each agent, reporting on number of levels passed.} %Note that since the levels are randomly generated, some might be unsolveable.}
\label{tab:solved-results}
\end{table}

First, we measure the number of levels \emph{passed} by each agent. This metric ensure that we study not only that the agents can plan but are able to find a plan that can be executed in the environment for successful goal achievement. Table~\ref{tab:solved-results} shows the number of levels passed from each of the level templates in our benchmark, by each agent.

The results show several clear trends. First, all Hydra agents outperformed the other agents in all the simple levels. In some cases, e.g., level type 36, all GBFS variants of HYDRA passed at least 20 out of 25 levels of this type while none of the non-Hydra agents were able to pass any levels. On the other hand, the baseline agents outperformed all Hydra agents in the complex levels. 

Hydra's heuristics guide the planner through the physics of the problem, and are independent of level structure. As a result, Hydra is accurate and versatile, and it can reason with any level composition. Hydra achieves best performance in simple levels requiring highly accurate shots. 


In contrast, non-Hydra agents were developed for playing human-designed levels, which are readily solved by a small number of fixed strategies focused on destroying complex structures. Low-accuracy strategies fare well on densely-populated levels (i.e., level 55 in the evaluation). Agents developed for levels with hand-crafted choices of strategy, and requiring only low-accuracy shots, do not perform well when the underlying assumptions for those strategies are broken (i.e., randomly generated level compositions). 

% To explain the poor behavior of the non-Hydra agents on the simple level, note that these agents were designed for solving human-designed levels, which are readily solved by a small number of fixed strategies. Such puzzles also do not require high accuracy (as humans are not, in general, very accurate compared to computers). As such, agents developed for low-accuracy levels with hand-crafted choices of strategy do not perform well when the underlying assumptions for those strategies are broken. In contrast, our heuristics guide the planner through the physics of the problem, and are independent of levels structure. 
% Nevertheless, our agent cannot solve the complex levels since many objects in a level require higher computational resources to reason about them. This suggests that smarter search and heuristics may allow our agent to solve also the complex levels in the future. 
%This claim is born out by levels with more objects (two birds, and complex levels), where heuristics to reduce the search space significantly improve our success rate. 


\begin{table}[tbh]
\resizebox{1\columnwidth}{!}{
\begin{tabular}{@{}lrrrrrrrrr@{}}
\toprule
Level & 22    & 25    & 36  & 45  & 46  & 53  & 54    & 57  & 55 \\
\midrule
Nodes/s & 1,610 & 1,295 & 930 & 624 & 732 & 747 & 1,202 & 634 & 36 \\
\bottomrule
\end{tabular}
}
\caption{Rate of node expansion, in nodes per second, by Hydra using GBFS with helpful states (GBFS (HS)), for each of level type in our benchmark (top row).} 
\label{tab:nodes-per-sec}
\end{table}
Hydra does not perform well in complex levels since the sheer number of objects present require higher computational resources to reason about them. However, this suggests that smarter search and heuristics may allow our agent to also solve the complex levels - an extension we are considering for future work. Consider Table~\ref{tab:nodes-per-sec} that lists the number of nodes expanded per second when using Hydra with GBFS and the helpful states technique. As can be seen, node generation in the complex level type (type 55) is much slower, going from 1,610 nodes/s in simple level type 22 to only 36 nodes/s for the complex level type 55. Solving complex levels with domain-independent planning requires stronger heuristics and perhaps other search techniques. 


\begin{table}
\centering
\begin{tabular}{@{}rrr|rrr@{}}
\toprule
\multicolumn{1}{c}{}      & \multicolumn{1}{c}{}    & \multicolumn{1}{c}{}    & \multicolumn{3}{c}{GBFS}                                                 \\
\multicolumn{1}{c}{Level} & \multicolumn{1}{c}{BFS} & \multicolumn{1}{c}{DFS} & \multicolumn{1}{c}{$H_S$} & \multicolumn{1}{c}{$H_P$} & \multicolumn{1}{c}{$HS$} \\
\midrule
22                        & 11,468                  & 20,000                  & 11,495                 & 10,081                 & 8,445                  \\
25                        & 36,681                  & 46,108                  & 35,432                 & 21,382                 & 9,668                  \\
36                        & 32,111                  & 42,482                  & 32,753                 & 18,383                 & 1,970                  \\
45                        & 5,257                   & 21,341                  & 5,279                  & 2,264                  & 1,826                  \\
46                        & 31,517                  & 40,871                  & 31,317                 & 15,485                 & 2,674                  \\
53                        & 26,340                  & 36,199                  & 29,366                 & 14,909                 & 2,760                  \\
54                        & 34,564                  & 42,100                  & 35,115                 & 23,578                 & 15,526                 \\
57                        & 6,126                   & 17,974                  & 7,298                  & 2,378                  & 1,753                  \\
55                        & 1,208                   & 1,489                   & 1,454                  & 970                    & 707               \\ \bottomrule    
\end{tabular}
\caption{Average total number of nodes expanded by the different Hydra agents for different level types.}
\vspace{-0.1cm}
\label{tab:expanded}
\end{table}

\subsubsection{Results 2}
Next, we study how various heuristics implemented in Hydra impact plan search. To measure search efficacy, we recorded the number of nodes expanded during planning. Table~\ref{tab:expanded} shows the average number of nodes expanded by each of the Hydra agents we considered. Here we see that all of our implemented heuristics are effective as they reduce the number of nodes expanded during search. 

While intuitively appealing, the score heuristic ($H_S$) turns out to be ineffective since the PDDL+ model of the game has only one action (release the slingshot at an opportune angle), and the action has distant effects. The score is not a direct consequence of the action, but rather it is computed far in the future once the bird collides with other objects.

Helpful states mechanism is the most effective; in some cases it expands an order fewer nodes before finding a solution (see the row corresponding to level 53 for example). Returning to the results in Table \ref{tab:solved-results}, we see that search efficiency significantly impacts Hydra's performance. GBFS with various heuristics can solve more levels that BFS and DFS. Between the different search enhancements, the helpful states technique is clearly beneficial in most cases. 



% \begin{table}[ht!]
% \centering
% \small
% \begin{tabular}{|c|c|c|c|c|}
% \hline
% \textbf{Agent} & \textbf{low trajectory} & \textbf{high trajectory} & \textbf{Two birds} & \textbf{complex} \\ \hline 
% Naive (ANU)            & 18 & 0  & 0  & \textbf{17}            \\ \hline
% DataLab                & 15 & 8  & 3  & 16            \\ \hline
% Eagle's  Wing          & 2  & 6  & 2  & 15            \\ \hline
% \multicolumn{4}{|c|}{HYDRA} \\ \hline
% BFS                    & \textbf{21} & \textbf{21} & 18          & 4  \\ \hline
% DFS                    & \textbf{21} & 16          & 10          & 2  \\ \hline
% score heuristic        & \textbf{21} & \textbf{21} & 14          & 1  \\ \hline
% proximity heuristic    & \textbf{21} & \textbf{21} & \textbf{25} & 4  \\ \hline
% helpful operators      & \textbf{21} & \textbf{21} & 24          & 9  \\ \hline

% \end{tabular}
% \caption{Evaluation results for each algorithm (levels solved by type). Note that since the levels are randomly generated, some might be unsolveable. }
% \label{tab:results}
% \end{table}

%We evaluated all agents on two types of levels: \emph{simple levels}, which are levels that contain a relatively small number of objects, and \emph{complex levels}, which are levels that contain more than 50 objects in them. Furthermore, we separated the 

% \subsubsection{Comparison between agents}
% Comparing the performance of all agents, we are struck by a curious result. The naive agents does surprisingly well compared to the SoTA, most notably on simple low-trajectory levels, but also on complex levels. This is less surprising then it would seem when we examine the agents' design. DataLab and Eagle's wing were developed to solve human-oriented levels, which are redily solved by a small number of fixed strategies. Such puzzles also do not require high accuracy (as humans are not, in general, very accurate compared to computers). As such, agents developed for low-accuracy levels with hand-crafted choices of strategy do not perform well when the underlying assumptions for those stategies are broken. In contrast, our heuristics guide the planner through the physics of the problem, and are independent of levels structure. 



% \begin{figure}
% \centering
%     \begin{tabular}{|c|c|c|}
%     \hline
%         \textbf{Agent}      & \textbf{Planning time (s)} | \textbf{Nodes expanded} \\ \hline
%         BFS                 & 16.4 & 31517  \\ \hline
%         DFS                 & 21.3 & 40870  \\ \hline
%         score heuristic     & 18.5 & 31317  \\ \hline
%         proximity heuristic & 13.5 & 15484  \\ \hline
%         helpful operators   &  7.5 & 2674   \\ \hline
%     \end{tabular}
%     \caption{Average planning time for a single high-trajectory level, by algorithm .}
%     \label{table:planning_time}
% \end{figure}



% For the evaluation, each agent attempted to solve a set of 25 random Angry Birds levels, developed using ANU's level generator\footnote{We used existing levels generated by ANU, which can be found at~\url{https://gitlab.com/aibirds/sciencebirdsframework/}}. As can be seen from the overall results (Tab.\ref{tab:results}), the Naive agent performed best overall solving 49 problems, followed closely by our HYDRA approach on 44 problems solved. Surprisingly, both Datalab and Eagle's Wing agents significantly underperformed on the evaluation test set, solving 26 and 16 problems, respectively. Fig.~\ref{fig:scores-plot} plots each agent's scores for each level in the evaluation, for readability purposes we sorted the level ordering (x axis) by HYDRA scores. The HYDRA agent performs best in terms of point scored, achieving an average score of 47515.7 points per level. Surprisingly, Datalab achieved an average score of 46667.8, outscoring the Naive agent (45112.4 points), even though the former solves significantly fewer levels. These results support the claim that our domain-independent planning approach is on par with existing domain-dependent approaches, even without encoding the birds' special powers.

% \setlength{\belowcaptionskip}{-5pt}

% \begin{figure}[htbp!]
% \begin{center}
% \includegraphics[trim= 50 45 50 82, clip, width=0.44\textwidth]{images/plot_novelty0_log.pdf}
% \end{center}
% \caption{Points per level (sorted by HYDRA scores).}
% \label{fig:scores-plot}
% \end{figure}

% The speculative cause for Eagle's Wing and DataLab's low performance might be attributed to the fact that the AI competition tested on hand-designed Angry Birds levels with specific winning strategies devised by their creators. Both agents rely on human devised strategies which proved to work well on the original Angry Birds levels but struggle to solve levels created by an automated generator, lacking depth to effectively reason with level composition. 

% To determine if computational resources played a significant role in our performance, we report the number of shots performed as a result of different planning problems. Out of 358 possible shots in the evaluation only 34 used the relaxed simplification, and 30 employed the default non-planning action strategy. This indicates that one could improve performance through either a more accurate PDDL+ encoding of Angry Birds, or PDDL+ planner heuristics to tackle the full goal planning problems. We intend to pursue both courses of action in future phases of our research.

% One observation from the evaluation is that ANU's level generator produces compositions that rarely require advanced reasoning about the level's structure. Many of the generated levels can be  solved by only targeting pigs, as proved by the results of our experiments where the Naive agent outperformed sophisticated strategies of Eagle's Wing and Datalab. In future experiments we will aim to compose a set of test levels which require more diverse reasoning and strategies to achieve high scores.

\section{Conclusion}
In this work, we presented a novel approach based on domain-independent planning for Angry Birds, a popular mobile games and a challenging AI testbed~\cite{renz2019ai}. 
To capture the complex dynamics of the system, we modeled the game using PDDL+, a rich planning language developed for hybrid systems supporting exogenous activity in planning domains via discrete events and continuous processes. 
We discuss how expressiveness of PDDL+ can capture complex, physics-based dynamics of Angry Birds. We implemented this PDDL+ model within a game playing planning agent, and explored several domain-specific search enhancements, such as novel heuristics and the ``helpful-states'' search strategy. Such improvements were useful to efficiently solve the challenging Angry Birds domain, despite a PDDL+ model focused on reducing system complexity and only incorporating a single action.
% \update{The agent has two actions, either to shoot or to wait at every time step. However, despite smaller action space the domain is challenging due to delayed consequences of hitting the target, and thus causing back-tracking during search.
% We were able to handle these challenges using domain-specific heuristics and incorporating ``helpful-states'' search strategy.}
Our evaluation showed promising results, with our agent able to pass more levels than multiple champions of the Angry Birds AI competition, in most types of levels we considered. It highlights that domain-independent planning and combinatorial search is a viable approach for solving this AI benchmark. 

Despite promising results, our current agent can struggle with complex levels composed of several objects on the scene. Additionally, our implementation does not consider the full range of the Angry Birds game, ignoring birds with special abilities and various types of pigs. Future work will address both limitations: develop better heuristics and search algorithms and extending our domain model to include birds' special powers. Future work will also compare Hydra with emerging AI agents such as BamBirds and Agent X\cite{lutalo2022agent}, which improve shot selection by better balancing between exploration and exploitation.

Our work demonstrates that PDDL+ is useful for capturing the dynamics of challenging, real-world domains. 
However, our experience with Angry Birds demonstrates that solving planning problems in rich PDDL+ domains is a challenge for existing planners. The development of solutions for PDDL+ by the community has been slow. One of the reasons for this is that PDDL+ benchmark domains have been largely re-purposed from existing domains, which result in problems that could be addressed by PDDL2.1 techniques. 
%This leads to very few PDDL+ planners being developed and consequently little demand for PDDL+ domains. 
This work also serves as a call-to-arms for research into PDDL+ planners and heuristics, both domain-independent and model-specific.


% We presented a novel PDDL+ planning model of Angry Birds, one of the most popular mobile games and a challenging AI testbed. It requires a highly expressive modeling language to capture  dynamics of the system. We determined the best way to accurately model the game as a planning domain is with PDDL+, an advanced planning language developed for hybrid systems. PDDL+ introduced a standardized method for including exogenous activity in planning domains via discrete events and continuous processes, which are critical to modeling Angry Birds' fundamental features. Next, we showed how leveraging the expressiveness of PDDL+ can mitigate its inherent risks and complexity while maintaining planning efficiency. Our evaluation showed promising results, with our agent (HYDRA) reaching the overall highest average score per level and significantly outperforming multiple champions of the Angry Birds AI competition. 

% By representing Angry Birds as a planning domain, we have proven that PDDL+ is capable of modeling a challenging AI problem. We argue that PDDL+ is currently severely underused due to modeling complexity and large search spaces. However, we have shown that the expressiveness of PDDL+, in fact, facilitates building of efficient models, and the search space can be further reduced using appropriate heuristics. This allows the solving of real-world problems with continuous variables. We plan to further extend our domain by adding birds' special powers, and improving its accuracy. Simultaneously, we aim to develop domain-independent heuristics for more efficient PDDL+ planning in this and other domains. 


% \newpage

\section{Acknowledgements}

This work was supported by the DARPA SAIL-ON program under contract HR001120C0040. The
views and conclusions in this document are those of the authors and should not be interpreted as
representing the official policies, either expressly or implied, of the Defense Advanced Research
Projects Agency or the U.S. Government.

\bibliography{library}
\end{document}
