\documentclass[journal]{IEEEtran}
\usepackage{amsmath,amsfonts}
\usepackage{algorithmic}
\usepackage{algorithm}
\usepackage{array}
\usepackage{colortbl}
\usepackage{color}
\usepackage[table,xcdraw]{xcolor}
% \usepackage[caption=false,font=normalsize,labelfont=sf,textfont=sf]{subfig}
\usepackage{textcomp}
\usepackage{stfloats}
\usepackage{url}
\usepackage{verbatim}
\usepackage{newclude}
\usepackage{graphicx}
\usepackage{cite}
\usepackage{makecell}

\usepackage[caption=false, font=footnotesize, labelfont=rm, textfont=rm]{subfig}
\hyphenation{op-tical net-works semi-conduc-tor IEEE-Xplore}
% updated with editorial comments 8/9/2021

\begin{document}

\title{Motion Planning for Autonomous Driving:

The State of the Art and Future Perspectives}

\author{

Siyu Teng,  Xuemin Hu, Peng Deng, Bai Li, Yuchen Li, Yunfeng Ai, Dongsheng Yang, 

Lingxi Li, ~\IEEEmembership{Senior Member,~IEEE,} Long Chen, Zhe Xuanyuan, Fenghua Zhu, ~\IEEEmembership{Senior Member,~IEEE,}


        % <-this % stops a space
\thanks{This work was supported in part by National Natural Science Foundation of China under Grant 62273135; Natural Science Foundation of Hubei Province in China under Grant 2021CFB460; the Guangdong Provincial Key Laboratory of Interdisciplinary Research and Application for Data Science, BNU-HKBU United International College, project code 2022B1212010006. Guangdong Higher Education Upgrading Plan with UIC research grant R0400001-22 and R201902. (Siyu Teng and Xuemin Hu contributed equally to this work). (Corresponding authors: Zhe Xuanyuan and Fenghua Zhu).


Siyu Teng and Yuchen Li are with BNU-HKBU United International College, Zhuhai, 519087, China and Hong Kong Baptist University, Kowloon, Hong Kong, 999077, China (e-mail: siyuteng@ieee.org).

%Xuemin Hu and Peng Deng are with the School of Computer Science and Information Engineering, Hubei University, Wuhan 430062, China (e-mail: huxuemin2012@hubu.edu.cn; jimmydengpeng@gmail.com).
Xuemin Hu and Peng Deng are with the School of Computer Science and Information Engineering, Hubei University, Wuhan 430062, China (e-mail: huxuemin2012@hubu.edu.cn).

Bai Li is with the College of Mechanical and Vehicle Engineering, Hunan University, Changsha, 410082, China.

% Zhe Xuanyuan is with BNU-HKBU United International College, Zhuhai, 519087, China (e-mail: zhexuanyuan@uic.edu.cn).


%Yunfeng Ai is with University of Chinese Academy of Sciences, Beijing, 100049, China (e-mail: aiyunfeng@ucas.ac.cn).
Yunfeng Ai is with University of Chinese Academy of Sciences, Beijing, 100049, China.

Dongsheng Yang is with the School of Public Management/Emergency Management, Jinan University, Guangzhou 510632, China.

Lingxi Li is with the Purdue School of Engineering and Technology, Indiana University-Purdue University Indianapolis (IUPUI), Indianapolis, USA.

Zhe Xuanyuan is with BNU-HKBU United International College, Zhuhai, 519087, China.

Long Chen and Fenghua Zhu are with Institute of Automation, Chinese Academy of Sciences,Beijing, China, 100190, and Long Chen is also with Waytous Ltd. (e-mail:fenghua.zhu@ia.ac.cn).

}% <-this % stops a space
\thanks{Manuscript received April 19, 2021; revised August 16, 2021.}}

% The paper headers
\markboth{IEEE TRANSACTIONS ON INTELLIGENT VEHICLES}%
{Shell \MakeLowercase{\textit{et al.}}: A Sample Article Using IEEEtran.cls for IEEE Journals and Transactions}

% \IEEEpubid{0000--0000/00\$00.00~\copyright~2021 IEEE}
% Remember, if you use this you must call \IEEEpubidadjcol in the second
% column for its text to clear the IEEEpubid mark.

\maketitle

\begin{abstract}

Thanks to the augmented convenience, safety advantages, and potential commercial value, Intelligent vehicles (IVs) have attracted wide attention throughout the world. Although a few autonomous driving unicorns assert that IVs will be commercially deployable by 2025, their implementation is still restricted to small-scale validation due to various issues, among which precise computation of control commands or trajectories by planning methods remains a prerequisite for IVs. This paper aims to review state-of-the-art planning methods, including pipeline planning and end-to-end planning methods. In terms of pipeline methods, a survey of selecting algorithms is provided along with a discussion of the expansion and optimization mechanisms, whereas in end-to-end methods, the training approaches and verification scenarios of driving tasks are points of concern. Experimental platforms are reviewed to facilitate readers in selecting suitable training and validation methods. Finally, the current challenges and future directions are discussed. The side-by-side comparison presented in this survey not only helps to gain insights into the strengths and limitations of the reviewed methods but also assists with system-level design choices.

\end{abstract}

\begin{IEEEkeywords}
Pipeline planning, end-to-end planning, imitation learning, reinforcement learning, parallel learning.
\end{IEEEkeywords}
\begin{figure} [t]
    \centering
	  \subfloat[Pipeline framework    \label{1a_123}]{
       \includegraphics[width=0.98\linewidth]{Pictures/pipeline.png}}

   \hfill


    
	  \subfloat[End-to-end framework]{
        \includegraphics[width=0.98\linewidth]{Pictures/end-to-end.png}}
    \hfill
	  \caption{Modular and end-to-end frameworks surveyed in \cite{surveye2e}. The modular framework for autonomous driving consists of many interconnected modules, while the end-to-end method treats the entire framework as one learnable learning task.}
	  \label{fig:Total} 
\end{figure}

\section{Introduction}



\IEEEPARstart{I}{ntelligent} vehicles (IVs) have gained considerable attention from government, industry, academia, and the general public due to their potential to revolutionize transportation, facilitated by advances in artificial intelligence and computer hardware \cite{Surveyofsurvey}. The deployment of IVs in the environmental landscape has the great potential to reduce road accidents and traffic congestion so as to improve our mobility in overcrowded megacities \cite{wenshuowang}. Although phenomenal contributions achieved by plenty of the leading names in this area, IVs are still out of reach except in limited trial programs due to key concerns about their reliability and safety. In order to achieve a higher level of awareness of the surrounding environments, and increased safety, efficiency, and capabilities, IVs are always equipped with different types of sensors. Despite multifarious sensors, IVs are constrained by the ability to inadequately detecting in adverse scenarios. Thus ensuring the safety, robustness, and generalization of planning methods is a pivotal problem for the implementation of autonomous driving \cite{OS}.


%Apart from technical challenges, adverse weather environments, such as rain, fog, and snow, also present significant obstacles to the safety, robustness, and generalization of planning methods.

%Planning methods are an essential component of the autonomous driving framework, however, various scenarios and adverse weather conditions such as rain, fog, and snow pose substantial challenges for safe and reliable driverless technology 


% 家在其他地方
% Along with multiple sensors, artificial intelligence algorithms, simulation platforms, large datasets, and public policies play major roles in the development of autonomous driving with higher levels of intelligence and mobility. 


% Artificial algorithms efficiently process the vast amount of multisensory data to train and validate the family of machine learning models that underpin autonomous driving systems. 


% These systems make sense of the scenarios and the objects in the environment and dictate the paths that the vehicles ultimately take. 

% This paper presents a comprehensive analysis of the general planning method for autonomous driving. Broadly speaking, the planning methods for autonomous driving can be categorized into two distinct classifications, as shown in Fig. \ref{fig:Total}: pipeline and end-to-end \cite{surveye2e}. 
This paper presents a comprehensive analysis of the general planning method for autonomous driving. Generally speaking, the planning methods for autonomous driving can be classified into two categories, i.e., pipeline and end-to-end.


%\textbf{The pipeline planning method is commonly coupled with perception, location, and control to form the pipeline framework for autonomous driving. The pipeline framework enables developing teams to concentrate on well-defined sub-tasks and independently make improvements across the whole stack, keeping the system operational as long as the intermediate outputs are kept functional.}


The pipeline planning method, also known as the rule-based planning approach, is a well-established method and extensively adopted by the industry. As shown in Fig. \ref{1a_123}, this method is a subset of the pipeline framework and needs to be coupled with other methods to accomplish autonomous driving tasks, such as perception \cite{TIV___1}, localization, and control. As a major advantage, the pipeline framework is interpretable, the defective module can be identified when a malfunction or unexpected system behavior occurs. To limit the scope of Section-\ref{ppl}, we only concentrate on aspects of the planning method within the pipeline framework. The pipeline planning method consists of two primary components: global route planning generating a road-level path from the origin to the destination, and local behavior and (or) trajectory planning generating a short-term trajectory. Although the pipeline planning method is widely used in industry, the restriction of the method is that it requires a large volume of computation resources and many manual heuristic functions \cite{planning_k}. In this study, we focus on the expansion and optimization mechanisms of the pipeline planning method.

The end-to-end planning method, also known as the learning-based approach, is a new kind of approach and has become a tendency in autonomous vehicle research. In this method, the entire driving pipeline is regarded as a single machine-learning task which transfers raw perception data to control commands. The driving model can learn knowledge via imitation learning, explore driving policy via reinforcement learning, and continuously self-optimize via parallel learning. Despite its appealing concept, it is difficult or even impossible to find out the reasons when the model misbehaves. In this case, our study concentrates on the network structure, training technique, and deployment tasks of the end-to-end model.



In Section-\ref{ppl}, we begin with reviewing pipeline planning methods, including global route planning and local behavior/trajectory planning, where we especially discuss the expansion and optimization mechanisms. The reviewed end-to-end planning methods, including imitation learning, reinforcement learning, and parallel learning are listed in Section-\ref{e2e}. The network architect, generalizability and robustness, and validation \& verification methods are explored in detail. In addition, large datasets, simulation platforms, and physical platforms play auxiliary roles in the development of autonomous driving with higher levels of intelligence and mobility, thus, we summarize other contents of autonomous driving in Section-\ref{experiments}, including the dataset, simulation platforms, and physical platforms. Finally, we review the current challenges and future directions of autonomous driving in Section-\ref{future}.

% The results of the large-scale tests in physical platforms support our previous conclusion that autonomous driving still has a long way to go for large-scale commercial use.

The contributions of this survey are outlined as follows:

% This survey represents the first comprehensive review of all planning methods in IVs, encompassing both pipeline planning and end-to-end planning approaches.

% This survey provides a thorough analysis and summary of the latest datasets, simulation platforms, and semi-open real-world testing scenarios.

% This survey presents a summary of the current open challenges and future research directions.


\begin{itemize}
\item[$\bullet$] This survey presents the first comprehensive review of all planning methods in IVs, encompassing both pipeline planning and end-to-end planning approaches.

\item[$\bullet$] This survey provides a thorough analysis and summary of the latest datasets, simulation platforms, and semi-open real-world testing scenarios.

\item[$\bullet$] This survey gives a summary of the current open challenges and lays out future research directions.
\end{itemize}


\include*{SubSection/Pipeline}

\section{End-to-End Planning Methods \label{e2e}}

End-to-end stands for the direct mapping from raw sensor data into trajectory points or control signals. Because of its ability to extract task-specific policies, it has been utilized with great success in a variety of fields. Compared with the pipeline method, there is no external gap between the perception and control modules and seldom human-customized heuristics are embedded, so the end-to-end method deals with vehicle-environment interactions more efficiently \cite{zhu2021survey}. End-to-end has a higher ceiling, with the potential to achieve expert performance in the autonomous driving field. In this section, we divide the end-to-end method into three categories:  imitation learning, reinforcement learning, and parallel learning.

\include*{SubSection/ImitationLearning}

\include*{SubSection/ReinforcementLearning}

\include*{SubSection/ParallelPlanning}

\include*{SubSection/ExperimentPlatform}

\section{Challenges and Future Directions \label{future}}
Considerable progress has been achieved in autonomous driving, as evidenced by its successful validation on semi-open roads in various cities. However, its complete commercial deployment is yet to be realized due to numerous obstacles and impending challenges that need to be surmounted.
\subsection{Challenges}

The challenges in IVs are summarised below:

    \textbf{Limitations of perception.} Most autonomous driving frameworks heavily rely on perception results, but most sensors are limited by their inherent constraints. Vision sensors are susceptible to the effects of field of view and weather, and are less effective in back-lighting as well as strong light exposure. Perception results often suffer from partial perception problems, and thus potential dangers that are obscured by obstacles may be ignored. These drawbacks pose security challenges for autonomous driving.

    \textbf{Limitations of planning.}  Both pipeline and end-to-end planning have intrinsic limitations, and ensuring the production of high-quality outputs under uncertain and complex scenarios is an indispensable research objective.

  \textbf{Limitations of safety.} Hacking incidents for autonomous driving systems are increasing, and even minor disruptions have the possibility to trigger significant deviations in decision-making. Therefore, the successful deployment of autonomous driving technology on a massive scale necessitates robust measures to counter adversarial attacks.
  

\textbf{Limitations of datasets.} Simulation datasets facilitate model training, and the well-trained model in simulation environments is often not directly transferable to reality. Therefore, bridging the gap between virtual and real data is an imperative research avenue \cite{Proceeding}.






\subsection{Future Directions}
Currently, planning methods are hard to handle all complex scenarios, and the models are also restricted by safety, generalization, and interpretability. Future research trends include:
\begin{itemize}

  \item [1)] 
    Confronting the constraints of perception. Many researchers try to incorporate cognition into the perception layer. By leveraging human cognitive abilities, it is plausible to overcome autonomous driving challenges.
  \item [2)]
    Tackling the problem of the uninterpretability of end-to-end methods. Many researchers enhance interpretability by generating interpretable intermediate representations in the latent layer. The exploration of employing these representations for end-to-end methods represents a research direction for IVs.

  \item [3)]

    Managing the problem of hacker attacks on IVs, present defenses are proven inadequate against SOTA attacks, and the development of robust defense techniques against such attacks carries important research implications.
  
  \item [4)]
    Facing challenges for decision-making in complex scenarios. Integrating human cognitive abilities into autonomous driving and a comprehensive understanding of scenario features is an effective way to overcome the present limitations.


    \item [5)]
    Giving the challenges posed by the robustness and generalizability of planning methods, the well-trained large model in ChatCPT \cite{Chat1} shows surpassing human-level capability in solving complex problems. This holds true in the domain of autonomous driving as well, where a promising future direction concerns rationalizing the application of large models.
    \item[6)]
    
    Confronting with the challenge of datasets migration from virtual to real, the description principle \cite{miao2023parallel} of parallel system theory may serve as an efficacious solution. By coupling the two types of data using the description principle, a feedback loop is generated, which enables circular self-optimization.
    

\end{itemize}

%\vfill
\bibliographystyle{IEEEtran}
\bibliography{Mylib.bib}


\begin{IEEEbiography}[{\includegraphics[width=1in,height=1.25in,clip,keepaspectratio]{BIO/SiyuTeng.jpg}}] {Siyu Teng} received M.S. degree from Jilin University in 2021. Now he is a PhD Student at Department of Computer Science, Faculty of Science, Hong Kong Baptist University. His main interests are parallel planning, end-to-end autonomous driving and interpretable deep learning.
\end{IEEEbiography}

\begin{IEEEbiography}[{\includegraphics[width=1in,height=1.25in,clip,keepaspectratio]{BIO/XueminHU.jpg}}]{Xuemin Hu} is currently an Associate Professor with School of Artificial Intelligence, Hubei University, Wuhan, China. He received the B.S. degree from Huazhong University of Science and Technology and the Ph.D. degree from Wuhan University in 2007 and in 2012, respectively. He was a visiting scholar in the University of Rhode Island, Kingston, RI, US from November 2015 to May 2016. His areas of interest include computer vision, machine learning, motion planning, and autonomous driving.
\end{IEEEbiography}

\begin{IEEEbiography}[{\includegraphics[width=1in,height=1.25in,clip,keepaspectratio]{BIO/PengDeng.png}}]{Peng Deng} received the B.E. degree in vehicle engineering from China Agricultural University, Beijing, China. He is currently pursuing the M.S. degree with the School of Artificial Intelligence, Hubei University, Wuhan, China. His areas of interest include reinforcement learning and autonomous driving.
\end{IEEEbiography}


\begin{IEEEbiography}[{\includegraphics[width=1in,height=1.25in,clip,keepaspectratio]{BIO/baili}}]{Bai Li} (SM’13–M’18) received his B.S. degree in 2013 from Beihang University, China, and the Ph.D. degree in 2018 from the College of Control Science and Engineering, Zhejiang University, China. He is currently an associate professor in the College of Mechanical and Vehicle Engineering, Hunan University, Changsha, China. Before joining Hunan University, he was a research engineer of JD.com Inc., Beijing, China from 2018 to 2020. Prof. Li was the first author of more than 60 journal/conference papers and two books in numerical optimization, optimal control, and trajectory planning. He was a recipient of the International Federation of Automatic Control (IFAC) 2014–2016 Best Journal Paper Prize. He is an Associate Editor of IEEE TRANSACTIONS ON INTELLIGENT VEHICLES.
\end{IEEEbiography}

\begin{IEEEbiography}[{\includegraphics[width=1in,height=1.25in,clip,keepaspectratio]{BIO/yuchenli}}]{Yuchen Li} received the B.E. degree from the University of Science and Technology Beijing in 2016, and the M.E. degrees from Beihang University in 2020. He is pursuing the Ph.D. degree in Hong Kong Baptist University. He is an intern at Waytous. His research interest covers computer vision, 3D object detection, and autonomous driving.
\end{IEEEbiography}



\begin{IEEEbiography}[{\includegraphics[width=1in,height=1.25in,clip,keepaspectratio]{BIO/YunfengAI.png}}]{Yunfeng Ai} received the Ph.D. degree from the University of Chinese Academy of Sciences, Beijing, China in 2006. He is Associate Professor at University of Chinese Academy of Sciences. He was a research fellow at Carnegie Mellon University. His current research interest covers computer vision, machine learning, robots, and autonomous driving.
\end{IEEEbiography}

\begin{IEEEbiography}[{\includegraphics[width=1in,height=1.25in,clip,keepaspectratio]{BIO/dongshengyang.png}}]{Dongsheng Yang} received the Ph.D. degree in information system engineering from the National University of Defense Technology, Changsha, China, in 2004. He is currently a Professor with the School of Public Management/Emergency Management (The Laboratory for Military- Civilian Integration Emergency Command and Control), Jinan University, Guangzhou, China. His research interests include intelligent emergency response of complex systems, multiscale emergency command and control mode and mechanism, and parallel intelligent technology of emergency management.
\end{IEEEbiography}



\begin{IEEEbiography}[{\includegraphics[width=1in,height=1.25in,clip,keepaspectratio]{BIO/LingxiLi.jpg}}]{Lingxi Li}  (S’04-M’08-SM’13) is currently a full professor in the Department of Electrical and Computer Engineering at Purdue School of Engineering and Technology, Indiana University-Purdue University Indianapolis (IUPUI), USA. Dr. Li received his Ph.D. degree in Electrical and Computer Engineering from the University of Illinois at Urbana-Champaign in 2008. Dr. Li’s current research focuses on modeling, analysis, control, and optimization of complex systems, connected and automated vehicles, intelligent transportation systems, digital twins and parallel intelligence, and human-machine interaction. He has authored/co-authored one book and over 130 research articles in refereed journals and conferences. Dr. Li was the recipient of five best paper awards, 2021 IEEE ITSS outstanding application award, 2017 outstanding research contributions award, 2012 T-ITS outstanding editorial service award, and several university research/teaching awards. He is currently serving as an associate editor for five international journals and has served as the General Chair, Program Chair, Program Co-Chair, etc., for 20+ international conferences. 
\end{IEEEbiography}

\begin{IEEEbiography}[{\includegraphics[width=1in,height=1.25in,clip,keepaspectratio]{BIO/LongCHEN.png}}]{Long Chen} (Senior Member, IEEE) received the Ph.D. degree from Wuhan University in 2013, he is currently a Professor with State Key Laboratory of Management and Control for Complex Systems, Institute of Automation, Chinese Academy of Sciences, Beijing, China. His research interests include autonomous driving, robotics, and artificial intelligence, where he has contributed more than 100 publications. He serves as an Associate Editor for the IEEE Transaction on Intelligent Transportation Systems, the IEEE/CAA Journal of Automatica Sinica, the IEEE Transaction on Intelligent Vehicle and the IEEE Technical Committee on Cyber-Physical Systems.
\end{IEEEbiography}

\begin{IEEEbiography}[{\includegraphics[width=1in,height=1.25in,clip,keepaspectratio]{BIO/ZheXUANYUAN.jpg}}]{Zhe XuanYuan} received the B.S. degree in electronic engineering from Peking University, Beijing, China, in 2005, and the Ph.D. degree in electronic and computer engineering from the Hong Kong University of Science and Technology, Hong Kong, in 2012. He is now an Associate Professor of Data Science with Beijing Normal University-Hong Kong Baptist University United International College, Zhuhai, China. His research interests include robot mapping and navigation, autonomous driving, and vehicular networks.
\end{IEEEbiography}

\begin{IEEEbiography}[{\includegraphics[width=1in,height=1.25in,clip,keepaspectratio]{BIO/fenghuazhu.jpg}}]{Fenghua Zhu} (Senior Member, IEEE) received the Ph.D. degree in control theory and control engineering from the Institute of Automation, Chinese Academy of Sciences, Beijing, China, in 2008. He is currently an Associate Professor with the State Key Laboratory of Multimodal Artificial Intelligence Systems, China. His research interests include artificial transportation systems and parallel transportation management systems.
\end{IEEEbiography}


\end{document}


