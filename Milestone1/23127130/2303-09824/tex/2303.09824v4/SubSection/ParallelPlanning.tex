\subsection{Parallel Learning}

%Currently, autonomous driving methods are restricted by a shortage of data, inefficient learning, and poor robustness. Pipeline planning methods leverage too many human-customized heuristics, which leads to inefficient computation and low generalization. End-to-end planning methods require a large volume and diverse distribution of expert trajectories or frequent interactions with scenarios, which in the real world is both very expensive and major security concerns, especially for corner cases in extreme scenarios.

Planning methods in autonomous driving are constrained by several challenges. Pipeline planning methods couple numerous human-customized heuristics, which leads to inefficient computation and low generalization. Imitation learning (IL) methods require considerable volume and diverse distribution of expert trajectories, while reinforcement learning (RL) methods demand significant computational resources. Consequently, the presence of these limitations impedes the widespread implementation of autonomous driving.

\begin{figure}[b]
    \centering
    \includegraphics[width=0.98\linewidth]{Pictures/ACP2.png}
    \caption{The framework of parallel system theory proposed in \cite{ParallelACP}.}
    \label{fig:ACP}
\end{figure}

\begin{table*}[b]
\centering 
\label{tab:parallelmethods}
\caption{The survey about the parallel system theory and its sources and derived algorithms.}
\begin{tabular}{c c m{13cm}}
\hline
\rowcolor[HTML]{EFEFEF} 
Method                 & Year  & \multicolumn{1}{c}{\cellcolor[HTML]{EFEFEF}Detail}                                                                                                                                                                       \\ \hline
CPS                    & 1990s & Proposing a multi-dimensional intelligent technology framework, based on big data, internet of things, and large computing, the organic integration and deep collaboration of computing, communication and control (3C). \\
CPSS \cite{Cpss}                  & 2000  & Integrating social signals and relationships into CPS, leveraging the human, data and information of the social network to break through the various limitations of the real world.                                      \\
Parallel System Theory \cite{ParallelACP} & 2004  & Integrating artificial societies (A), computational experiments (C) and parallel execution (P), and provide effective tools for control and management of complex systems.                                               \\
Parallel Learning \cite{ParallelLearning}      & 2017  & Proposing a new framework of machine learning theory, parallel learning, which incorporates and inherits many elements from various existing machine learning theories.                                                  \\
Parallel Vision \cite{parallelvsion}        & 2017  & Introducing the parallel system theory into the computer vision area and constructing a novel research method for perception and understanding of complex driving scenarios.                                             \\
Parallel Driving \cite{ParallelDriving}       & 2019  & Constructing an advanced and unified framework for autonomous driving that includes operation management, online condition management and emergency disengagement.                                                       \\
Parallel Planning \cite{ParallelPlanning2}      & 2019  & Constructing a deep planning method that integrates a convolutional neural network and a Long short-term memory module to improve the generalization and robustness of planning models in intelligent vehicles.           \\
Parallel Testing \cite{paralleltest}       & 2019  & Proposing a closed-loop testing framework, which implements more challenging scenarios to accelerate evaluation and development of autonomous vehicles.                                                                  \\ \hline
\end{tabular}
\end{table*}


% With the development of various technologies in control, sensing, and communication, human society has been continuously enhancing the autonomy of machines.

In response to the various problems in planning methods, virtual-real interaction provides a proven solution \cite{JAS_FeiYue}. Cyber-physical-systems (CPS) based intelligent control can facilitate interactions and integration between physical and cyberspaces but are not considering human and social factors in systems. In reply, many researchers have added social factors and artificial information to the CPS to form the cyber-physical-social systems (CPSS). In CPSS, the 'C' stands for two dimensions: the information system in the real world and the virtual artificial system defined by software. The 'P' refers to the traditional real system. The 'S' includes not only the human social system but also the artificial system based on the real world.

CPSS enables virtual and real systems to interact, feedback, and promote each other. The real system provides valuable datasets for the construction and calibration of the artificial system, while the artificial system directs and supports the operation of the real system, thus achieving self-evolution. Due to the advantages of virtual-real interaction, CPSS provides a new verification method for end-to-end autonomous driving.



Based on CPSS, Fei-Yue Wang \cite{ParallelACP} proposes the concept of parallel system theory in 2004, as shown in Fig. \ref{fig:ACP}, the core concept of which is the ACP method, an organic combination of artificial societies (A), computational experiments (C) and parallel execution (P). Over the past two decades, the research system of parallel system theory has been enriched and improved by a large number of implementations in practice \cite{ITSM}, such as parallel intelligence \cite{parallelintelligence}, parallel control \cite{parallelcontrol, Lu_ParallelChontrol}, parallel management \cite{ParallelFactory}, parallel transportation \cite{paralleltransportation}, parallel driving \cite{ParallelDriving, Paralleljas}, parallel tracking \cite{Lu_parallelTracking}, parallel testing \cite{paralleltest}, parallel vision \cite{parallelvsion} and so on. The survey about the methods proposed in this section is shown in Table \ref{tab:parallelmethods}.

%\textbf{As shown in Fig. \ref{fig:ParallelTree}, in parallel system theory,} real systems provide valuable data for the construction and calibration of artificial systems, while artificial systems guide and support the operation of real systems, thus achieving self-improvement. 

%Artificial societies are often utilized to model complex systems, computational experiments are used to calculate and analyze the corresponding knowledge and policies, and parallel executions are implemented to control and manage the interaction between real and artificial systems.

% \begin{figure}[b]
%     \centering
%     \includegraphics[width=0.98\linewidth]{Pictures/ParallelTree.png}
%     \caption{With ACP method as the soil, and CPSS and parallel intelligence as the trunk, a parallel tree is formed that encompasses all elements in the world.}
%     \label{fig:ParallelTree}
% \end{figure}







%leads to a problem - inefficient learning. When dealing with large data from complex systems, the high dimensionality of the system often makes the exploration of the policy extremely difficult.}

In order to further expand the learning capabilities of neural networks, and to address the challenges of IL and RL, Li et al. \cite{ParallelLearning} propose a basic framework for parallel learning based on the parallel system theory as shown in Fig. \ref{fig:ParallelLearning}. In the action phase, parallel learning \cite{ParallelLearning} follows the RL paradigm, employing state transfer to represent the movement of the model, learning from big data, and storing the learned policy in the state-transition function. Notably, parallel learning capitalizes on computational experimentation to refine the policy. Through feature extraction methods, small knowledge can be applied to specific scenarios or tasks, and used for parallel control. Here, ``small" refers to specific and intelligent knowledge for the particular problem, rather than denoting the magnitude of knowledge.

\begin{figure}
    \centering
    \includegraphics[width=0.88\linewidth]{Pictures/Parallel_Learning.png}
    \caption{ The theoretical framework of parallel learning proposed in \cite{ParallelLearning}. (The part above the dashed line focuses on big data preprocessing using artificial systems; the part beneath the dashed line focuses on computational experiments. The thin arrows represent either data generation or data learning; the thick arrows present interactions between data and actions.)}
    \label{fig:ParallelLearning}
\end{figure}


% 放在后边用
% Wang et al. \cite{ParallelDriving} propose parallel driving method, which is combined the parallel learning based on the ACP method with autonomous driving.


An innovative training approach based on parallel learning \cite{ParallelLearning} presents an alternative solution for problem-solving in fully end-to-end autonomous stacks. As shown in Fig. \ref{fig:ParallelPlanning}, Wang et al. \cite{ParallelDriving2}  introduce a parallel driving framework, a unified approach for ITS and IVs. The framework directly bridges expert trajectories and control commands to compute the most optimal policy for specific scenarios. Plenty of expert trajectories are collected from real scenarios, and a neural network is employed to learn all these trajectories, inputs and outputs of this network are destination state and control signals. From the viewpoint of parallel learning, this is a self-labeling process, and the process significantly alleviates the data hunger of end-to-end methods.

% The leitmotif of parallel driving is to fine-tune the policy of a real intelligent vehicle by simulating and interacting with the artificial system to construct complex autonomous driving scenarios. In the artificial world, the policy in a virtual vehicle interacts with other virtual vehicles and physical intelligent vehicles to improve its generalization and robustness. Various computational experiments are used to improve the modeling accuracy of the virtual systems in the artificial world, to obtain policies for different scenarios, and to clarify the ways and means of interaction between virtual vehicles and real vehicles.


\begin{figure}[b]
    \centering
    \includegraphics[width=0.88\linewidth]{Pictures/ParallelDriving2.png}
    \caption{The theoretical framework of the parallel driving proposed in \cite{ParallelDriving2}.}
    \label{fig:ParallelPlanning}
\end{figure}



In order to handle the integrated data from the artificial system and computational experiment, a new theory is proposed, named parallel reinforcement learning (PRL), which combines the parallel learning and deep reinforcement learning approaches. Liu et al. \cite{ParallelDriving} integrate digital quadruplets with parallel driving. This framework defines the physical vehicle, the descriptive vehicle, the predictive vehicle, and the prescriptive vehicle. Based on the description of digital quadruplets, three virtual vehicles can be defined as three ``guardian angels" for the physical vehicle, playing different roles to make the IVs safer and more reliable in complex scenarios.
% \begin{figure}
%     \centering
%     \includegraphics[width=0.98\linewidth]{Pictures/ParallelDrivingFramework.png}
%     \caption{The theoretical framework of the parallel reinforcement learning (PRL) proposed in \cite{ParallelDriving2}.}
%     \label{fig:ParallelDriving}
% \end{figure}

\begin{figure}[t]
    \centering
    \includegraphics[width=0.8\linewidth]{Pictures/ParallelPlanning2.png}
    \caption{Hybrid model of combining the variational auto-encoder (VAE) and the generative adversarial network (GAN) for predicting and generating potential emergency image sequences proposed in \cite{ParallelPlanning2}.}
    \label{fig:ParallelDrivining}
\end{figure}


Planning is one of the most significant components of autonomous driving. As a concrete implementation of parallel driving, Chen et al. \cite{ParallelDriving, ParallelDriving2} propose a parallel planning framework for end-to-end planning, which constructs two customized approaches to solve emergency planning problems in specific scenarios. For the data-insufficient problem, parallel planning leverages artificial traffic scenarios to generate expert trajectories based on the pretrained knowledge from reality, as shown in Fig. \ref{fig:ParallelDrivining}. For the non-robustness problem, parallel planning utilizes a variational auto-encoder (VAE) and a generative adversarial network (GAN) to learn from virtual emergencies generated in artificial traffic scenes. For the learning inefficient problem, parallel planning learning policy from both virtual and real scenarios, and the final decision is determined by analysis of real observations. Parallel planning is able to make rational decisions without a heavy calculation burden when an emergency occurs.

The parallel system theory provides an effective tool for the control and management of complex systems, especially in the autonomous control field, parallel driving effectively alleviates the shortage of data, inefficient learning, and poor robustness for end-to-end planning models.
