\documentclass[journal]{IEEEtran}
\usepackage{amsmath,amsfonts}
\usepackage{algorithmic}
\usepackage{algorithm}
\usepackage{array}
\usepackage{colortbl}
\usepackage{color}
\usepackage[table,xcdraw]{xcolor}
% \usepackage[caption=false,font=normalsize,labelfont=sf,textfont=sf]{subfig}
\usepackage{textcomp}
\usepackage{stfloats}
\usepackage{url}
\usepackage{verbatim}
\usepackage{newclude}
\usepackage{graphicx}
\usepackage{cite}
\usepackage{makecell}
\usepackage{bm}

\usepackage[caption=false, font=footnotesize, labelfont=rm, textfont=rm]{subfig}
\hyphenation{op-tical net-works semi-conduc-tor IEEE-Xplore}
% updated with editorial comments 8/9/2021

\begin{document}

\title{Motion Planning for Autonomous Driving:

The State of the Art and Future Perspectives}

\author{

Siyu Teng,  Xuemin Hu, Peng Deng, Bai Li, Yuchen Li, Yunfeng Ai, Dongsheng Yang, 

Lingxi Li,  Zhe Xuanyuan, Fenghua Zhu, ~\IEEEmembership{Senior Member,~IEEE,} Long Chen, ~\IEEEmembership{Senior Member,~IEEE,} 


        % <-this % stops a space
\thanks{This work was supported in part by National Natural Science Foundation of China under Grant 62273135 and 62103139; Natural Science Foundation of Hubei Province in China under Grant 2021CFB460; 2022 Opening Foundation of State Key Laboratory of Management and Control for Complex Systems under Grant E2S9021119; the Guangdong Provincial Key Laboratory of Interdisciplinary Research and Application for Data Science, BNU-HKBU United International College 2022B1212010006. Guangdong Higher Education Upgrading Plan with UIC research grant R0400001-22 and R201902. (Siyu Teng and Xuemin Hu contributed equally to this work). (Corresponding authors: Zhe Xuanyuan, Fenghua Zhu and Long Chen).


Siyu Teng and Yuchen Li are with BNU-HKBU United International College, Zhuhai, 519087, China and Hong Kong Baptist University, Kowloon, Hong Kong, 999077, China (e-mail: siyuteng@ieee.org).

%Xuemin Hu and Peng Deng are with the School of Computer Science and Information Engineering, Hubei University, Wuhan 430062, China (e-mail: huxuemin2012@hubu.edu.cn; jimmydengpeng@gmail.com).
Xuemin Hu and Peng Deng are with the School of Computer Science and Information Engineering, Hubei University, Wuhan 430062, China.

Bai Li is with the State Key Laboratory of Advanced Design and Manufacturing for Vehicle Body, and also with the College of Mechanical and Vehicle Engineering, Hunan University, Changsha 410082, China.

% Zhe Xuanyuan is with BNU-HKBU United International College, Zhuhai, 519087, China (e-mail: zhexuanyuan@uic.edu.cn).


%Yunfeng Ai is with University of Chinese Academy of Sciences, Beijing, 100049, China (e-mail: aiyunfeng@ucas.ac.cn).
Yunfeng Ai is with University of Chinese Academy of Sciences, Beijing, 100049, China.

Dongsheng Yang is with the School of Public Management/Emergency Management, Jinan University, Guangzhou 510632, China.

Lingxi Li is with the Purdue School of Engineering and Technology, Indiana University-Purdue University Indianapolis (IUPUI), Indianapolis, USA.

Zhe Xuanyuan is with the Guangdong provincial key lab of IRADS, BNU-HKBU United International College, Zhuhai, 519087, China.

Fenghua Zhu and Long Chen are with Institute of Automation, Chinese Academy of Sciences,Beijing, China, 100190, and Long Chen is also with Waytous Ltd. (e-mail:fenghua.zhu@ia.ac.cn; long.chen@ia.ac.cn).

}% <-this % stops a space
\thanks{Manuscript received April 19, 2021; revised August 16, 2021.}}

% The paper headers
\markboth{IEEE TRANSACTIONS ON INTELLIGENT VEHICLES}%
{Shell \MakeLowercase{\textit{et al.}}: A Sample Article Using IEEEtran.cls for IEEE Journals and Transactions}

% \IEEEpubid{0000--0000/00\$00.00~\copyright~2021 IEEE}
% Remember, if you use this you must call \IEEEpubidadjcol in the second
% column for its text to clear the IEEEpubid mark.

\maketitle

\begin{abstract}

% Thanks to the augmented convenience, safety advantages, and potential commercial value, Intelligent vehicles (IVs) have attracted wide attention throughout the world. Although a few autonomous driving unicorns assert that IVs will be commercially deployable by 2025, their implementation is still restricted to small-scale validation due to various issues, among which precise computation of control commands or trajectories by planning methods remains a prerequisite for IVs. This paper aims to review state-of-the-art planning methods, including pipeline planning and end-to-end planning methods. In terms of pipeline methods, a survey of selecting algorithms is provided along with a discussion of the expansion and optimization mechanisms, whereas in end-to-end methods, the training approaches and verification scenarios of driving tasks are points of concern. Experimental platforms are reviewed to facilitate readers in selecting suitable training and validation methods. Finally, the current challenges and future directions are discussed. The side-by-side comparison presented in this survey not only helps to gain insights into the strengths and limitations of the reviewed methods but also assists with system-level design choices.


Intelligent vehicles (IVs) have gained worldwide attention due to their increased convenience, safety advantages, and potential commercial value. Despite predictions of commercial deployment by 2025, implementation remains limited to small-scale validation, with precise tracking controllers and motion planners being essential prerequisites for IVs. This paper reviews state-of-the-art motion planning methods for IVs, including pipeline planning and end-to-end planning methods. The study examines the selection, expansion, and optimization operations in a pipeline method, while it investigates training approaches and validation scenarios for driving tasks in end-to-end methods. Experimental platforms are reviewed to assist readers in choosing suitable training and validation strategies. A side-by-side comparison of the methods is provided to highlight their strengths and limitations, aiding system-level design choices. Current challenges and future perspectives are also discussed in this survey.
\end{abstract}

\begin{IEEEkeywords}
Motion planning, pipeline planning, end-to-end planning, imitation learning, reinforcement learning, parallel learning.
\end{IEEEkeywords}
\begin{figure} [t]
    \centering
	  \subfloat[Pipeline framework    \label{1a_123}]{
       \includegraphics[width=0.98\linewidth]{Pictures/pipeline.png}}

   \hfill


    
	  \subfloat[End-to-end framework \label{1b_123}]{
        \includegraphics[width=0.98\linewidth]{Pictures/end-to-end.png}}
    \hfill
	  \caption{Pipeline and end-to-end frameworks surveyed in \cite{surveye2e}. The pipeline framework for autonomous driving consists of many interconnected modules, while the end-to-end method treats the entire framework as one learnable learning task.}
	  \label{fig:Total} 
\end{figure}

\section{Introduction}



% \IEEEPARstart{I}{ntelligent} vehicles (IVs) have gained considerable attention from government, industry, academia, and the general public due to their potential to revolutionize transportation, facilitated by advances in artificial intelligence and computer hardware \cite{Surveyofsurvey}. The deployment of IVs in the environmental landscape has the great potential to reduce road accidents and traffic congestion so as to improve our mobility in overcrowded megacities \cite{wenshuowang}. Although phenomenal contributions achieved by plenty of the leading names in this area, IVs are still out of reach except in limited trial programs due to key concerns about their reliability and safety. In order to achieve a higher level of awareness of the surrounding environments, and increased safety, efficiency, and capabilities, IVs are always equipped with different types of sensors. Despite multifarious sensors, IVs are constrained by the ability to inadequately detecting in adverse scenarios. Thus ensuring the safety, robustness, and generalization of planning methods is a pivotal problem for the implementation of autonomous driving \cite{OS}.

\IEEEPARstart{I}{ntelligent} vehicles (IVs) have attracted significant interest from governments, industries, academia, and the public, owing to their potential to transform transportation through advances in artificial intelligence and computer hardware \cite{Surveyofsurvey}. The deployment of IVs holds great promise for reducing road accidents and alleviating traffic congestion, thereby improving mobility in densely populated urban areas \cite{wenshuowang}. Despite remarkable contributions by leading experts in the field, IVs remain primarily confined to limited trial programs due to concerns about their reliability and safety. To enhance situational awareness and improve safety, efficiency, and overall capabilities, IVs are equipped with a variety of sensors. However, even with an array of sensors, an IV stills face challenges in adequately detecting and responding to complex scenarios. Consequently, ensuring the safety, robustness, and adaptability of planning methods becomes crucial for the successful implementation of autonomous driving \cite{OS}.


%Apart from technical challenges, adverse weather environments, such as rain, fog, and snow, also present significant obstacles to the safety, robustness, and generalization of planning methods.

%Planning methods are an essential component of the autonomous driving framework, however, various scenarios and adverse weather conditions such as rain, fog, and snow pose substantial challenges for safe and reliable driverless technology 


% 家在其他地方
% Along with multiple sensors, artificial intelligence algorithms, simulation platforms, large datasets, and public policies play major roles in the development of autonomous driving with higher levels of intelligence and mobility. 


% Artificial algorithms efficiently process the vast amount of multisensory data to train and validate the family of machine learning models that underpin autonomous driving systems. 


% These systems make sense of the scenarios and the objects in the environment and dictate the paths that the vehicles ultimately take. 

% This paper presents a comprehensive analysis of the general planning method for autonomous driving. Broadly speaking, the planning methods for autonomous driving can be categorized into two distinct classifications, as shown in Fig. \ref{fig:Total}: pipeline and end-to-end \cite{surveye2e}. 

%The contributions of this survey are outlined as follows:

% This survey represents the first comprehensive review of all planning methods in IVs, encompassing both pipeline planning and end-to-end planning approaches.

% This survey provides a thorough analysis and summary of the latest datasets, simulation platforms, and semi-open real-world testing scenarios.

% This survey presents a summary of the current open challenges and future research directions.


% \begin{itemize}
% \item[$\bullet$] 
% %This survey presents the first comprehensive review of all planning methods in IVs including various scenarios, encompassing both pipeline planning and end-to-end planning approaches. \textcolor{black}{Compared with other review articles (such as \cite{surveye2e, Li3}), this work presents a more comprehensive and novel survey.}
% \textcolor{black}{This survey presents the first comprehensive analysis of planning methods in IVs, encompassing both pipeline planning and end-to-end planning methods in various scenarios. }
% %Compared to other review articles, such as \cite{surveye2e, Li3, Review1}, this work proposes a more thorough and innovative survey framework.}


% \item[$\bullet$] This survey provides a thorough analysis and summary of the latest datasets, simulation platforms, and semi-open real-world testing scenarios.

% \item[$\bullet$] This survey gives a summary of the current open challenges and lays out future research directions.
% \end{itemize}

%In Section-\ref{background}, we present a succinct overview of the difference between pipeline and end-to-end methods and concentrate on comparing their respective advantages and disadvantages. 



% \section{Problem Formulation}
\subsection{Pipeline Planning Methods}
The pipeline method also named the modular approach, is widely used by the industry and is nowadays considered the conventional approach. This method is mainly divided into Global Route Planning and Local Behavior/Trajectory Planning. 
\subsubsection{Global Route Planning}
\subsubsection{Local Behavior/Trajectory Planning}

\subsection{End-to-End Planning Methods}
\subsubsection{Imitation Learning}
\subsubsection{Reinforcement Learning}
\subsubsection{Parallel Planning}

% \section{Background and comparison \label{background}}





%\textbf{The pipeline planning method is commonly coupled with perception, location, and control to form the pipeline framework for autonomous driving. The pipeline framework enables developing teams to concentrate on well-defined sub-tasks and independently make improvements across the whole stack, keeping the system operational as long as the intermediate outputs are kept functional.}

\subsection{Background}

The pipeline planning method, also known as the rule-based planning method, is a well-established category of planners. As depicted in \ref{1a_123}, this method serves as a core component of the pipeline framework and must be integrated with other methods, such as perception \cite{TIV___1},  localization, and control, to accomplish autonomous driving tasks. A significant advantage of the pipeline framework is its interpretability, enabling the identification of defective modules when malfunctions or unexpected system behavior occur. In Section-\ref{ppl}, the focus is solely on the planning method within the pipeline framework. The pipeline planning method comprises two primary components: global route planning, which generates a road-level path from the origin to the destination, and local behavior/trajectory planning, which generates a short-term trajectory. Although widely used in the industry, the pipeline planning method requires substantial computational resources and numerous manual heuristic functions \cite{planning_k}. This study specifically addresses the expansion and optimization mechanisms of the pipeline planning method.

The end-to-end planning method, also known as the learning-based approach, is the sole component in the end-to-end framework and has become a trend in autonomous vehicle research. As illustrated in \ref{1b_123}, the entire driving framework is treated as a single machine learning task that converts raw perception data into control commands. The driving model acquires knowledge through imitation learning, develops driving policies through reinforcement learning, and continuously self-optimizes via parallel learning. Despite its appealing concept, determining the reasons for model misbehavior can be challenging. Consequently, this study focuses on the network structure, training techniques, and deployment tasks of the end-to-end model.

\subsection{Comparison}


\textcolor{black}{In this subsection, we provide a concise overview of the distinctions between pipeline and end-to-end methods, particularly highlighting their respective advantages and disadvantages.}

\textcolor{black}{The pipeline framework, widely implemented in the industry, allows engineers to focus on well-defined sub-tasks and independently improve each sub-model within the entire pipeline. Due to its clear intermediate representations and deterministic decision-making rules, this framework facilitates pinpointing the root cause of errors when unexpected behavior occurs. Moreover, it enables reliable reasoning about how the system generates specific control signals. However, the pipeline framework has some drawbacks. Individual sub-models may not be optimal for all driving scenarios, posing a challenge to the generalization of the framework. Additionally, the concatenation of sub-modules and the numerous manual customization constraints in each sub-model can compromise the robustness and real-time capabilities of the method.}

\textcolor{black}{The end-to-end framework optimizes the entire driving task, from raw perception to control signals, as a single deep learning task. By learning optimal intermediate representations for the target task, the framework can attend to any implicit sources of raw data without human-defined information bottlenecks, enhancing its generalization for various scenarios. The end-to-end framework's streamlined architecture, consisting of one or a few networks, also offers superior robustness and real-time capabilities compared to the pipeline framework. However, as research progresses, the end-to-end optimization faces a critical interpretability issue. Without intermediate outputs, tracing the initial cause of an error and explaining why the model arrived at specific control commands or trajectories becomes more challenging.}




\subsection{Paper Structure}


\textcolor{black}{In Section \ref{ppl}, pipeline planning methods are reviewed, including global route planning and local behavior/trajectory planning, with a particular focus on the expansion and optimization mechanisms. In Section \ref{e2e}, end-to-end planning methods are examined, encompassing imitation learning, reinforcement learning, and parallel learning, while exploring network architecture, generalizability, robustness, and validation \& verification methods. Additionally, large datasets, simulation platforms, and physical platforms play auxiliary roles in the development of autonomous driving with higher levels of intelligence and mobility. Therefore, other aspects of autonomous driving are summarized in Section \ref{experiments}, including datasets, simulation platforms, and physical platforms. Finally in Section \ref{future}, current challenges and future directions of autonomous driving are reviewed.}

%In Section-\ref{ppl}, we begin with reviewing pipeline planning methods, including global route planning and local behavior/trajectory planning, where we especially discuss the expansion and optimization mechanisms. The reviewed end-to-end planning methods, including imitation learning, reinforcement learning, and parallel learning are listed in Section-\ref{e2e}. The network architect, generalizability and robustness, and validation \& verification methods are explored in detail. In addition, large datasets, simulation platforms, and physical platforms play auxiliary roles in the development of autonomous driving with higher levels of intelligence and mobility, thus, we summarize other contents of autonomous driving in Section-\ref{experiments}, including the dataset, simulation platforms, and physical platforms. Finally, we review the current challenges and future directions of autonomous driving in Section-\ref{future}.
% \textcolor{black}{Fig. \ref{fig:Mindmap} further clarification of the structural relationships in this survey, highlighting the performance and progressiveness of the reviewed methods}

\subsection{Contributions}
%This paper presents a comprehensive analysis of the general planning method for autonomous driving. Generally speaking, the planning methods for autonomous driving can be classified into two categories, i.e., pipeline and end-to-end. 

\textcolor{black}{This paper presents a comprehensive analysis of the general planning methods for autonomous driving. Broadly speaking, planning methods for autonomous driving can be classified into two categories: pipeline and end-to-end.} 

\textcolor{black}{There have been numerous state-of-the-art works on motion planning for IVs, however, a comprehensive review encompassing both pipeline and end-to-end methods has yet to be conducted. The pipeline is a classical planning method commonly used in the industry, with general categories outlined in previous research \cite{Li3, sur_pipeline2}. In this paper, we propose a new classification of pipeline methods that captures the extensively deployed approaches in a manner more relevant to industry selection, based on the expansion and optimization mechanisms of each method. Our proposed classification includes state grid identification, primitive generation, and other approaches. The end-to-end approach has emerged as a popular research direction in recent years, as demonstrated by previous work \cite{surveye2e, Review1}, which illustrates methods for mapping raw perception inputs to control command outputs. In this survey, we not only review the latest achievements in imitation learning (IL) and reinforcement learning (RL) but also introduce a novel category called parallel planning. This category proposes a virtual-real interaction confusion learning method for a reliable end-to-end planning method. Furthermore, we provide a thorough analysis and summary of the latest datasets, simulation platforms, and semi-open real-world testing scenarios, which serve as essential auxiliary elements for the advancement of IVs. To the best of our knowledge, this survey presents the first comprehensive analysis of motion planning methods in various scenarios and tasks.}



\include*{SubSection/Pipeline}

\section{End-to-End Planning Methods \label{e2e}}


\textcolor{black}{End-to-end stands for the direct mapping from raw sensor data into trajectory points or control signals. Because of its ability to extract task-specific policies, it has achieved great success in a variety of fields \cite{Review1}.} Compared with the pipeline method, there is no external gap between the perception and control modules, and seldom human-customized heuristics are embedded, so the end-to-end method deals with vehicle-environment interactions more efficiently. End-to-end has a higher ceiling, with the potential to achieve expert performance in the autonomous driving field. \textcolor{black}{This section categorizes the end-to-end method into three distinct types from learning methods: imitation learning using supervised learning, reinforcement learning utilizing unsupervised learning, and parallel learning incorporating confusion learning. Fig. \ref{fig:Mindmap} further clarification of the structural relationships fo end-to-end planner, highlighting the performance and progressiveness of the reviewed methods.}

%\textcolor{black}{Table *** shows a comprehensive overview of the advantages and disadvantages of each category.}



\include*{SubSection/ImitationLearning}

\include*{SubSection/ReinforcementLearning}

\include*{SubSection/ParallelPlanning}

\include*{SubSection/ExperimentPlatform}

\section{Challenges and Future Perspectives \label{future}}
Considerable milestones have been achieved in autonomous driving, as evidenced by its successful validation on semi-open roads in various cities. However, its complete commercial deployment is yet to be realized due to numerous obstacles and impending challenges that need to be surmounted.
\subsection{Challenges}

The challenges in IVs are summarised below:
\begin{itemize}
     \item [1)] \textcolor{black}{Perception: autonomous driving frameworks heavily rely on perception data, however, most sensors are vulnerable to environmental effects and suffer from partial perception issues. As a result, potential hazards may be ignored, and these drawbacks present security challenges for autonomous driving.}
     \item [2)] \textcolor{black}{Planning: both pipeline and end-to-end planning have intrinsic limitations, and ensuring the production of high-quality outputs under uncertain and complex scenarios is an indispensable research objective.}
     \item [3)] \textcolor{black}{Safety: hacking for autonomous driving systems is increasing, even minor disruptions have possibly triggered significant deviations. Therefore, the deployment of autonomous driving methods on a massive scale necessitates robust measures to counter adversarial attacks.}
     \item[4)] \textcolor{black}{Dataset: simulators are essential for training and testing autonomous driving models, however, models well-trained in virtual environments often cannot be directly implemented in reality \cite{Proceeding}. Thus, bridging the gap between virtual and real data is imperative for advancing research in this field.}
\end{itemize}
    % \textbf{Limitations of perception.} Autonomous driving frameworks heavily rely on perception data, however, most sensors are vulnerable to environmental effects and suffer from partial perception issues. As a result, potential hazards may be ignored, and these drawbacks present security challenges for autonomous driving.

%     \textbf{Limitations of planning.}  Both pipeline and end-to-end planning have intrinsic limitations, and ensuring the production of high-quality outputs under uncertain and complex scenarios is an indispensable research objective.

%   \textbf{Limitations of safety.} Hacking incidents for autonomous driving systems are increasing, and even minor disruptions have the possibility to trigger significant deviations in decision-making. Therefore, the successful deployment of autonomous driving methods on a massive scale necessitates robust measures to counter adversarial attacks.

% \textbf{Limitations of datasets.} Simulators are essential for training and testing autonomous driving models, however, models well-trained in virtual environments often cannot be directly implemented in reality \cite{Proceeding}. Thus, bridging the gap between virtual and real data is imperative for advancing research in this field.






\subsection{Future Perspectives}
% Currently, IVs are hard to cover all complex scenarios, and the models are also restricted by safety, fairness, and interpretability. Future research trends include:

The mechanism of the end-to-end planner is the closest to the human driver, according to the input state to calculate the output space. However, due to challenges in data, interpretability, generalization, and policies, end-to-end planners are still scarcely implemented in the real world. Herein, we propose some future perspectives in the field of end-to-end planning.


\begin{itemize}

  \item [$\bullet$] 
    %Confronting the constraints of perception. Many researchers try to incorporate cognition into the perception layer. By leveraging human cognitive abilities, it is plausible to overcome autonomous driving challenges.
    \textcolor{black}{Interpretability: Machine learning receives criticism due to its black-box properties. The current intermediate feature representations are insufficient to explain the causality of its inference process. In the case of IV, the consequences of lacking interpretability could be catastrophic. Thus, providing clear and understandable interpretations for the motion planner is crucial in enhancing trust in intelligent vehicles (IVs). Moreover, this approach could assist in predicting and rectifying potential issues that may jeopardize the safety of the passengers.}
  \item [$\bullet$]
    %Tackling the problem of the uninterpretability of end-to-end methods. Many researchers enhance interpretability by generating interpretable intermediate representations in the latent layer. The exploration of employing these representations for end-to-end methods represents a research direction for IVs.
    \textcolor{black}{ Sim2Real: The simulation and the real environment have obvious differences in scenario diversity and environment complexity, making it challenging to align simulation data with real data \cite{SE1, SE2}. Consequently, the well-trained models in simulators may not optimally perform in real settings. Developing a model to bridge the gap between simulated and real environments is critical to address the challenges about data diversity and fairness, which is also a crucial research direction in end-to-end planning.}

  \item [$\bullet$]

    %Managing the problem of hacker attacks on IVs, present defenses are proven inadequate against SOTA attacks, and the development of robust defense techniques against such attacks carries important research implications.
    \textcolor{black}{Reliability: One critical bottleneck that impedes the development and deployment of IVs is the prohibitively high economic and time costs required to validate their reliability. Constructing an artificial-intelligence-based algorithm that can identify the corner cases in a short time is a key direction for the validation of IVs.}
  \item [$\bullet$]
    %Facing challenges for decision-making in complex scenarios. Integrating human cognitive abilities into autonomous driving and a comprehensive understanding of scenario features is an effective way to overcome the present limitations.
    \textcolor{black}{Governance: IV is not only a technical issue, the sound policy is also crucial. Designing a framework that includes safety standards, data privacy regulations, and ethical guidelines is necessary to govern the development and deployment of IVs. This framework will promote accountability and transparency, reduce risks, and ensure that the public interest is defended.}

    % \item [5)]
    % Giving the challenges posed by the robustness and generalizability of planning methods, the well-trained large model in ChatCPT \cite{Chat1} shows surpassing human-level capability in solving complex problems. This holds true in the domain of autonomous driving as well, where a promising future direction concerns rationalizing the application of large models.
    % \item[6)]
    
    % Confronting with the challenge of datasets migration from virtual to real, the description principle \cite{miao2023parallel} of parallel system theory may serve as an efficacious solution. By coupling the two types of data using the description principle, a feedback loop is generated, which enables circular self-optimization.
    

\end{itemize}

%\vfill
\bibliographystyle{IEEEtran}
\bibliography{Mylib.bib}


\begin{IEEEbiography}[{\includegraphics[width=1in,height=1.25in,clip,keepaspectratio]{BIO/SiyuTeng.jpg}}] {Siyu Teng} received M.S. degree from Jilin University in 2021. Now he is a PhD Student at Department of Computer Science, Hong Kong Baptist University. His main interests are parallel planning, end-to-end autonomous driving and interpretable deep learning.
\end{IEEEbiography}

\begin{IEEEbiography}[{\includegraphics[width=1in,height=1.25in,clip,keepaspectratio]{BIO/XueminHU.jpg}}]{Xuemin Hu} is currently an Associate Professor with School of Artificial Intelligence, Hubei University, Wuhan, China. He received the B.S. degree from Huazhong University of Science and Technology and the Ph.D. degree from Wuhan University in 2007 and in 2012, respectively. He was a visiting scholar in the University of Rhode Island, Kingston, RI, US from November 2015 to May 2016. His areas of interest include computer vision, machine learning, motion planning, and autonomous driving.
\end{IEEEbiography}

\begin{IEEEbiography}[{\includegraphics[width=1in,height=1.25in,clip,keepaspectratio]{BIO/PengDeng.png}}]{Peng Deng} received the B.E. degree in vehicle engineering from China Agricultural University, Beijing, China. He is currently pursuing the M.S. degree with the School of Artificial Intelligence, Hubei University, Wuhan, China. His areas of interest include reinforcement learning and autonomous driving.
\end{IEEEbiography}


\begin{IEEEbiography}[{\includegraphics[width=1in,height=1.25in,clip,keepaspectratio]{BIO/baili}}]{Bai Li} (SM’13–M’18) received his B.S. degree in 2013 from the School of Advanced Engineering, Beihang University, China, and his Ph.D. degree in 2018 from the College of Control Science and Engineering, Zhejiang University, China. From Nov. 2016 to June 2017, he visited the Department of Civil and Environmental Engineering, University of Michigan (Ann Arbor), USA, as a joint training Ph.D. student. He is currently an associate professor in Hunan University. Before teaching at Hunan University, he worked in JDX R\&D Center of Automated Driving, JD Inc., China from 2018 to 2020 as an algorithm engineer. Prof. Li has been the first author of more than 70 journal/conference papers and two books related to numerical optimization, motion planning, and robotics. He was a recipient of the International Federation of Automatic Control (IFAC) 2014–2016 Best Journal Paper Prize from Engineering Applications of Artificial Intelligence. He is currently an Associate Editor of IEEE TRANSACTIONS ON INTELLIGENT VEHICLES. He was a recipient of the 2022 TIV Best Associate Editor Award. His research interest is rule-based motion planning methods for IVs.
\end{IEEEbiography}

\begin{IEEEbiography}[{\includegraphics[width=1in,height=1.25in,clip,keepaspectratio]{BIO/yuchenli}}]{Yuchen Li} received the B.E. degree from the University of Science and Technology Beijing in 2016, and the M.E. degrees from Beihang University in 2020. He is pursuing the Ph.D. degree in Hong Kong Baptist University. He is an intern at Waytous. His research interest covers computer vision, 3D object detection, and autonomous driving.
\end{IEEEbiography}



\begin{IEEEbiography}[{\includegraphics[width=1in,height=1.25in,clip,keepaspectratio]{BIO/YunfengAI.png}}]{Yunfeng Ai} received the Ph.D. degree from the University of Chinese Academy of Sciences, Beijing, China in 2006. He is Associate Professor at University of Chinese Academy of Sciences. He was a research fellow at Carnegie Mellon University. His current research interest covers computer vision, machine learning, robots, and autonomous driving.
\end{IEEEbiography}

\begin{IEEEbiography}[{\includegraphics[width=1in,height=1.25in,clip,keepaspectratio]{BIO/dongshengyang.png}}]{Dongsheng Yang} received the Ph.D. degree in information system engineering from the National University of Defense Technology, Changsha, China, in 2004. He is currently a Professor with the School of Public Management/Emergency Management (The Laboratory for Military- Civilian Integration Emergency Command and Control), Jinan University, Guangzhou, China. His research interests include intelligent emergency response of complex systems, multiscale emergency command and control mode and mechanism, and parallel intelligent technology of emergency management.
\end{IEEEbiography}



\begin{IEEEbiography}[{\includegraphics[width=1in,height=1.25in,clip,keepaspectratio]{BIO/LingxiLi.jpg}}]{Lingxi Li}  (S’04-M’08-SM’13) is currently a full professor in the Department of Electrical and Computer Engineering at Purdue School of Engineering and Technology, Indiana University-Purdue University Indianapolis (IUPUI), USA. Dr. Li received his Ph.D. degree in Electrical and Computer Engineering from the University of Illinois at Urbana-Champaign in 2008. Dr. Li’s current research focuses on modeling, analysis, control, and optimization of complex systems, connected and automated vehicles, intelligent transportation systems, digital twins and parallel intelligence, and human-machine interaction. He has authored/co-authored one book and over 130 research articles in refereed journals and conferences. Dr. Li was the recipient of five best paper awards, 2021 IEEE ITSS outstanding application award, 2017 outstanding research contributions award, 2012 T-ITS outstanding editorial service award, and several university research/teaching awards. He is currently serving as an associate editor for five international journals and has served as the General Chair, Program Chair, Program Co-Chair, etc., for 20+ international conferences. 
\end{IEEEbiography}



\begin{IEEEbiography}[{\includegraphics[width=1in,height=1.25in,clip,keepaspectratio]{BIO/ZheXUANYUAN.jpg}}]{Zhe XuanYuan} received the B.S. degree in electronic engineering from Peking University, Beijing, China, in 2005, and the Ph.D. degree in electronic and computer engineering from the Hong Kong University of Science and Technology, Hong Kong, in 2012. He is now an Associate Professor of Data Science with Beijing Normal University-Hong Kong Baptist University United International College, Zhuhai, China. His research interests include robot mapping and navigation, autonomous driving, and vehicular networks.
\end{IEEEbiography}

\begin{IEEEbiography}[{\includegraphics[width=1in,height=1.25in,clip,keepaspectratio]{BIO/fenghuazhu.jpg}}]{Fenghua Zhu} (Senior Member, IEEE) received the Ph.D. degree in control theory and control engineering from the Institute of Automation, Chinese Academy of Sciences, Beijing, China, in 2008. He is currently an Associate Professor with the State Key Laboratory of Multimodal Artificial Intelligence Systems, China. His research interests include artificial transportation systems and parallel transportation management systems.
\end{IEEEbiography}

\begin{IEEEbiography}[{\includegraphics[width=1in,height=1.25in,clip,keepaspectratio]{BIO/LongCHEN.png}}]{Long Chen} (Senior Member, IEEE) received the Ph.D. degree from Wuhan University in 2013, he is currently a Professor with State Key Laboratory of Management and Control for Complex Systems, Institute of Automation, Chinese Academy of Sciences, Beijing, China. His research interests include autonomous driving, robotics, and artificial intelligence, where he has contributed more than 100 publications. He serves as an Associate Editor for the IEEE Transaction on Intelligent Transportation Systems, the IEEE/CAA Journal of Automatica Sinica, the IEEE Transaction on Intelligent Vehicle and the IEEE Technical Committee on Cyber-Physical Systems.
\end{IEEEbiography}

\end{document}


