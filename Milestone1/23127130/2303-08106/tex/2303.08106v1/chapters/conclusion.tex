\section{Conclusion}\label{sec:conclusion}

Studying the impact of distribution shifts on the performance of ML-based algorithms for wireless applications is of paramount importance to our research community to better reflect on the adequacy of adopting the data-driven ML approaches in communication systems engineering. In particular, the investigation of domain generalization will lay the ground for rigorous evaluation protocols of data-driven algorithms for wireless communications systems. In this paper, we presented an overview of state-of-the-art methodologies for domain generalization problems to handle distribution shifts. To justify the need to devise new algorithms with better generalization capabilities, we distinguished the four types of distribution shifts between source and target domains. We also provided an overview of  multiple important fields related to generalization to better put domain generalization in proper perspective across close research areas. Then, we summarized the three existing methodologies to improve the generalization capabilities of deep learning models, namely, data manipulation, data representation, and domain generalization learning paradigms. In doing so, we gave multiple examples and suggestions not covered in the current literature where these methodologies can be applied for wireless communications applications. We then reviewed the recent research contributions to improve the generalization of neural network models when solving wireless communication problems. These problems involve beam prediction, data detection, channel decoding, beamforming, edge networks, etc. We also presented the learned lessons from the existing applications of domain generalization methodologies for wireless communication problems by highlighting the lack of $i)$ algorithms exploiting the domain knowledge from well-established communication models, and $ii)$ open-source benchmarks to accelerate the development of robust algorithms for future wireless networks. Finally, we discussed open questions to enrich and bridge the gap between both domain generalization and wireless communication fields.  