\section{Open Issues}\label{sec:future-direction} 
In this section, we discuss some of the open questions that evolve around the necessity to carefully incorporate the DG concept in ML-oriented communication research. This is because, unlike many other ML-based technologies, most real-world communication applications require real-time operation and seamless adaptation to dynamically changing propagation conditions. This precludes the luxury of repeatedly training ML-oriented models and makes DG-induced robustness a must-have feature in any ML-aided communication system.

\subsection{Beyond End-to-End Learning for Generalization}

Most of the existing studies rely on end-to-end learning to train a holistic over-parametrized DNN architecture by applying gradient-based optimization to the learning system as a whole. This means that all transmit/receive modules of the communication system must be differentiable (in the reverse-mode algorithmic differentiation sense \cite{speelpenning1980compiling}). Few wireless communication libraries have been proposed to study differentiable communication systems \cite{sionna, AIwirelessMatlab}.

\begin{figure*}[!b]
\centering
\begin{minipage}[c]{\textwidth}
\centering
\includegraphics[scale=0.71]{figures/AI-end-to-end-transmitter-receiver.pdf}
\caption{The possible integration steps of ML methods into the conventional transmit/receive communication chain if ML methods will be proven to be robust to domain shifts.}
\label{fig:end-to-end-AI-receiver}
\end{minipage}
\end{figure*}

Before advocating the adequacy of applying ML methods to the building blocks of the wireless physical layer depicted in Fig.~\ref{fig:end-to-end-AI-receiver}, DG has to be meticulously investigated and guaranteed \textit{within} and \textit{across} the blocks. From this perspective, it is not enough to claim the migration from model-based classical signal processing techniques to data-driven ML techniques without analyzing the impact of each migration on the overall system performance in terms of both accuracy and robustness. While such migration is a conceptually profound paradigm shift, its impact continues to be assessed from the accuracy perspective only, and hence must also be carefully analyzed through the lens of generalization/robustness.

The legacy physical layer design strategy relies on the divide-and-conquer approach by decomposing (a.k.a. layering) the entire communication chain into smaller blocks \cite{gallager2008principles}. Designing ML methods to substitute a single block or multiple blocks (see Fig.~\ref{fig:end-to-end-AI-receiver}) raises critical generalization questions justified by the following two facts:
\begin{itemize}
    \item End-to-end learning methods are trained with gradient descent-like optimizers, which exhibit slow convergence on ill-conditioned problems or convergence to possibly poor local optima. In other words, training is performed while hoping that the structural preconditioning is sufficiently strong to steer a method as simple as gradient descent from a random initial state to a highly non-trivial solution \cite{DBLP:conf/acml/Glasmachers17}. This assumption is risky since all ML techniques tailored for wireless applications are exclusively used for non-convex optimization problems.
    \item The valuable wireless communication know-how developed since the 50s is completely neglected during end-to-end training. ``Standing on the shoulders of giants'' (as Sir Isaac Newton once said) is a scientific tradition which promotes building upon the accumulated knowledge and discoveries made by others, and ``end-to-end learning'' must be proven robust to domain shifts to be considered an exception.
\end{itemize}

For these considerations, going beyond conventional end-to-end learning is an important step towards answering critical DG questions in data-driven ML techniques applied to wireless communications. In what follows, we discuss research
directions to cope with some end-to-end learning limitations.

\subsection{Hybrid Data-Driven and Model-Driven Methods}

After more than a century-long research effort in radio communications, state-of-the-art communication modeling and fast estimation algorithms are becoming more essential to high-bandwidth transmissions. From a DG perspective, the power of these classical model-driven tools lies in their guaranteed generalization capabilities because they do not depend on specific domains that are tied to generated/collected datasets. This generalization, however, often comes at the cost of high complexity.

Data-driven methods can come into play as an effective tool to reduce the computational complexity of classical model-based methods at the cost of generalization. As advocated in \cite{pellaco2022machine}, a hybrid framework that combines the benefits of both data-driven and model-based techniques is worth pursuing. Adopting this framework will prevent the generated domains for DG from being fully dependent on $i)$ the convergence of gradient-based optimizers for data-driven methods, or $ii)$ the complexity of model-based methods. For better illustration, we elaborate in what follows on how data-driven methods can be combined with physically consistent model-based methods.

The study of DG for MIMO communication should benefit from the side information provided by the physical laws governing the wave transmission and the circuits of RF components (i.e., amplifiers, and antennas). By employing physically consistent models \cite{DBLP:journals/tcas/IvrlacN10,DBLP:journals/corr/abs-2208-01556,DBLP:journals/twc/PizzoSM22}, it is possible to exploit the inherent symmetries and invariances in communication scenarios owing to Maxwell's equations \cite{jackson1999classical,baum1995symmetry}. From this perspective, physically consistent models for wireless communications offer an opportunity to generate communication datasets which exhibit domain-invariant regularities (e.g., antenna impedances), thereby diminishing the generalization difficulties across domains. As one example, fixing the impedance matrices of transmit and receive linear/planar antenna arrays increases the amount of correlation in the wireless channel, which can be exploited by DNNs for better channel estimation accuracy.

Moreover, this physically consistent direction opens the door for the analysis of DG through the lens of antenna theory. For example, it might be possible to determine which spacing parameter of the antenna array provides the best DNN accuracy for channel estimation. By doing so, realistic wireless communication domains are generated and more faithful representations of the real-world transmissions are simulated, thereby leading to a physically consistent version of digital twins for wireless communications \cite{DBLP:journals/cm/KhanSNHH22}.

\subsection{From Image-Based DG Methods to Signal-Based Methods}

Existing DG methodologies have been predominantly geared towards image-based vision tasks, leaving signal-based tasks almost unexplored despite being versatile in several real-world applications such as healthcare, retail, climate, finance, and communication. This unbalanced exploration impacts the development of specific DG methods for signal-based tasks. For instance, feature alignment approaches for DG are relying heavily on DNNs as feature extractors which are specifically fine-tuned to vision tasks, thereby leaving DG feature extraction for non-image signals severely underexplored. Some work looked at temporal distributional shifts in clinical healthcare \cite{guo2022evaluation,DBLP:conf/iconip/MaLZL19} and climate \cite{DBLP:conf/nips/MalininBGGGCNPP21} applications, but none of the prior work explored it in wireless communication. 

From this perspective, we highlight the importance of taking the first step towards a deeper understanding of temporal distributional shifts in wireless communication due to dynamic changes in the received signal resulting from the varying propagation properties (e.g., coherence time and Doppler shift).

\subsection{Compound Domain Generalization}
As mentioned previously, most of the presented methods for DG assume a homogeneous setting where domain labels are available. However, this assumption may not be realistic in several problems where the domain labels are hard to obtain or define. In this case, several techniques discussed above either become inapplicable (e.g., meta-learning) or their performance degrades drastically \cite{DBLP:journals/corr/abs-2103-02503}. Recently, there has been a surge of interest in studying the compound DG setting in vision problems. Most of the methods for compound DG propose to infer latent domain information from data and then use standard learning techniques to generalize across the latent domains. These solutions are, however, based on different restrictive assumptions such as: $i)$ the latent domains are distinct and separable \cite{hmoe}, $ii)$ the domain heterogeneity originates from stylistic differences \cite{DBLP:conf/cvpr/ChenL0LY22} or $iii)$ the latent domains are balanced \cite{DBLP:journals/corr/abs-1911-07661}. Compound DG is hence still an active research field with a lot of room for improvement, especially in wireless communication problems.

\subsection{Federated Domain Generalization}

Distributed learning algorithms enable devices to cooperatively build a unified learning model across agents with local training. As a result, a wide variety of distributed ML methods have been proposed and extensively analyzed within the federated learning (FL) framework \cite{DBLP:journals/corr/KonecnyMYRSB16}. 

For wireless physical layer applications, FL has been explored to address multiple key communication problems beyond the data security aspect \cite{DBLP:conf/infocom/ZhangYD22} such as channel estimation \cite{DBLP:journals/twc/ElbirC22}, symbol detection \cite{DBLP:conf/ssp/MashhadiSEG21} and beamforming \cite{elbir2020federated}. All of these works do not assume the availability of a central entity (e.g., base station) at which the learning model is trained. However, the question of whether the model learned by each agent generalizes to unseen scenarios is still unanswered and this remains an unexplored research area. In the context of IoT applications, very few efforts started investigating the challenges of DG for IoT devices by aligning each device's domain to a reference distribution in a distributed manner \cite{zhang2023federated}.

Addressing DG in the FL context is known as \textit{federated domain generalization} (FDG)\cite{DBLP:journals/corr/abs-2111-10487}. Distributed agents can collect their local data independently, hence naturally forming a distinct source domain. At the time of writing, no research paper in wireless communication has studied FDG, e.g., in the context of distributed MIMO \cite{DBLP:journals/jsac/WangWYWCH13} consisting of distributed antenna array systems.
