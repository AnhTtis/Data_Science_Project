\documentclass[twocolumn,10pt]{IEEEtran}
%
\usepackage[mathscr]{eucal}
\usepackage{ifpdf}
\usepackage{cite}
\usepackage{graphicx}
\usepackage[cmex10]{amsmath}
\usepackage{amssymb}
\usepackage{amsmath}
\usepackage{placeins}
\DeclareMathOperator*{\argmax}{arg\,max}
\DeclareMathOperator*{\argmin}{arg\,min}
\usepackage{algorithmic}
\usepackage{units}
\usepackage{setspace}
\usepackage{algorithm}
\usepackage{array}
\usepackage{booktabs}
\usepackage{amsthm}
\usepackage{multirow}
\usepackage{enumerate}
\usepackage{color}
\usepackage{mathtools}
\usepackage{caption}
\usepackage[labelformat=simple]{subcaption}
\usepackage{tabularx}
\usepackage{tcolorbox}
\usepackage{supertabular}
\usepackage{tikz}
\usepackage{bm}
\usepackage{xcolor,colortbl}
\usepackage{titlesec}
\usepackage{enumitem}
\usepackage{hyperref}
\usepackage{xfrac}
%\usepackage{dblfloatfix}
%\usepackage{float}
\usepackage{stfloats}
\usetikzlibrary{arrows.meta, positioning,patterns}

\renewcommand\thesubfigure{(\alph{subfigure})}

\usepackage{pifont}% http://ctan.org/pkg/pifont
\newcommand{\cmark}{\ding{51}}%
\newcommand{\xmark}{\ding{55}}%

\newcounter{assume}
\def\theassume{A\arabic{assume}}
\setcounter{assume}{0}
%\newcommand{\inp}[1]{\textcolor{blue}{#1}}
\newcommand{\resl}[1]{}
\newcommand{\tosl}[1]{}
\newcommand{\cosl}[1]{}
\newcommand{\insl}[1]{#1}

\definecolor{mygreen}{RGB}{143, 188, 187}
\definecolor{green1}{RGB}{147, 196, 125}
\definecolor{red1}{RGB}{213,88,88}

\begin{document}
\title{Domain Generalization in Machine
Learning Models for Wireless Communications: Concepts, State-of-the-Art, and Open Issues}
\author{
\IEEEauthorblockN{Mohamed Akrout, Amal Feriani, Faouzi Bellili, \textit{Member, IEEE},\\ Amine Mezghani, \textit{Member, IEEE}, and Ekram Hossain, \textit{Fellow, IEEE}}
}
%
\maketitle
%
%----------------------------------------------Abstract---------------------------------------------
\begin{abstract}
Data-driven machine learning (ML) is promoted as one potential technology to be used in next-generations wireless systems. This led to a large body of research work that applies ML techniques to solve problems in different layers of the wireless transmission link. However, most of these applications rely on supervised learning which assumes that the source (training) and target (test) data are independent and identically distributed (i.i.d). This assumption is often violated in the real world due to domain or distribution shifts between the source and the target data. Thus, it is important to ensure that these algorithms generalize to out-of-distribution (OOD) data. In this context, domain generalization (DG) tackles the OOD-related issues by learning models on different and distinct source domains/datasets with generalization capabilities to unseen new domains without additional finetuning. Motivated by the importance of DG requirements for wireless applications, we present a comprehensive overview of the recent developments in DG and the different sources of domain shift. We also summarize the existing DG methods and review their applications in selected wireless communication problems, and conclude with insights and open questions.

\let\thefootnote\relax\footnotetext{The authors are with the Department of Electrical 
and Computer Engineering, University of Manitoba, Winnipeg, MB, Canada (e-mails: 
\{akroutm,feriania\}@myumanitoba.ca, \{faouzi.bellili, amine.mezghani, ekram.hossain\}@umanitoba.ca). The work was supported by a Discovery Grant from the Natural Sciences and Engineering Research Council of Canada (NSERC).}
\end{abstract}
%
{\em Keywords}: ML-aided wireless networks, Out-of-distribution generalization, Domain generalization

%----------------------------------------------Chapters-----------------------------------------

\section{Introduction}
\label{sec:introduction}
% \begin{itemize}
%     % Diffusion of FL
%     \item {\st{Diffusion of FL}}
%     % Security threats to FL
%     \item {\st{Security threats to FL with particular focus on model poisoning}}
%     % Limitations of existing countermeasures
%     \item {\st{Current countermeasures (e.g., KRUM) and their limitations}}
%     % Proposed method and its advantages
%     \item {\st{Intuitive description of the proposed method and its difference (i.e., advantages) w.r.t. state of the art}}
%     % Main contributions
%     \item {\st{Summary of the main contributions of this work}}
%     % Paper's structure and organization
%     \item {\st{Paper's structure and organization}}
% \end{itemize}

% Diffusion of FL
Recently, {\em federated learning} (FL) has emerged as the leading paradigm for training distributed, large-scale, and privacy-preserving machine learning (ML) systems~\cite{mcmahan2017googleai,mcmahan2017aistats}. 
The core idea of FL is to allow multiple edge clients to collaboratively train a shared, global model without disclosing their local private training data.
%Specifically, an FL system consists of a central server and many edge clients; 
A typical FL round involves the following steps: {\em(i)} the server randomly picks some clients and sends them the current, global model; {\em(ii)} each selected client locally trains its model with its own private data; then, it sends the resulting local model to the server;\footnote{Whenever we refer to global/local model, we mean global/local model {\em parameters}.} {\em(iii)} the server updates the global model by computing an \emph{aggregation function}, usually the average (FedAvg), on the local models received from clients.
% \begin{enumerate}
%     \item[{\em(i)}] the server sends the current, global model to the clients and appoints some of them for training;
%     \item[{\em(ii)}] each selected client locally trains its copy of the global model with its own private data; then, it sends the resulting local model back to the server;\footnote{Whenever we refer to global/local model, we mean global/local model {\em parameters}.}
%     \item[{\em(iii)}] the server updates the global model by computing an \emph{aggregation function} on the local models received from clients (by default, the average, also referred to as FedAvg~\cite{mcmahan2017aistats}).
% \end{enumerate}
This process goes on until the global model converges. %(e.g., after a certain number of rounds or other similar stopping criteria).
%\\
% The advantages of FL over the traditional, centralized learning paradigm are undoubtedly clear in terms of flexibility/scalability (clients can join/disconnect from the FL network dynamically), network communications (only model weights\footnote{We will use \textit{parameters} and \textit{weights} interchangeably.} are exchanged between clients and server), and privacy (each client's private training data is kept local at the client's end and not uploaded to the server).
\\
% Security threats to FL
%However, the growing adoption of FL also raises security concerns~\cite{costa2022covert}, particularly about its confidentiality, integrity, and availability.
Although its advantages over standard ML, FL also raises security concerns~\cite{costa2022covert}. %, particularly about its confidentiality, integrity, and availability~\cite{costa2022covert}.
% OLD, LONG VERSION
% Indeed, some work deals with privacy leakage that may expose the local data of some clients~\cite{melis2019sp}. 
% A large body of work, instead, investigates attacks that usually aim to detriment the predictive accuracy of the learned global model. For instance, \emph{data poisoning} attacks achieve this goal by letting an adversary pollute the training set of some corrupt FL clients with maliciously crafted examples~\cite{jagielski2018sp}.
% Similarly, in \emph{model poisoning} the attacker attempts to tweak the global model weights~\cite{bhagoji2019pmlr} by directly perturbing the local model's weights of some infected FL clients before these are sent to the central server for aggregation, usually via so-called Byzantine attacks. 
% It turns out that Byzantine model poisoning attacks severely impact standard FedAvg; therefore, more robust aggregation functions must be designed to make FL systems secure.
Here, we focus on \emph{untargeted model poisoning} attacks~\cite{bhagoji2019pmlr}, where an adversary attempts to tweak the global model weights %\footnote{We will use the terms \textit{parameters} and \textit{weights} interchangeably.} 
by directly perturbing the local model's parameters of some infected clients before these are sent to the central server for aggregation.
In doing so, the adversary aims to jeopardize the global model \textit{indiscriminately} at inference time.
Such model poisoning attacks severely impact standard FedAvg; therefore, more robust aggregation functions must be designed to secure FL systems.
\\
% In this paper, we focus on designing a novel robust aggregation scheme at the server's end to contrast the effect of Byzantine model poisoning attacks.
%
% Current countermeasures and their limitations
%Several countermeasures have been proposed in the literature to combat model poisoning attacks on FL systems.
% Some methods use simple statistics more robust than plain average to smooth the impact of malicious updates (e.g., Trimmed Mean and FedMedian~\cite{yin2018icml}). 
% Other defenses implement outlier detection techniques to discard malicious updates from the aggregation performed at the server's end. Those are either based on heuristics (e.g., Krum/Multi-Krum~\cite{blanchard2017nips} and Bulyan~\cite{mhamdi2018pmlr}) or data-driven approaches (e.g., K-means clustering~\cite{shen2016acm} or DnC via spectral analysis~\cite{shejwalkar2021ndss}). 
% Finally, some strategies rely on a centralized ``source of trust'' to spot potential malicious updates (e.g., FLTrust~\cite{cao2020fltrust}).
% Several countermeasures have been proposed in the literature to combat model poisoning attacks on FL systems, i.e., to discard possible malicious local updates from the aggregation performed at the server's end. 
% These techniques range from simple statistics more robust than plain average (e.g., Trimmed Mean and FedMedian~\cite{yin2018icml}) to outlier detection heuristics (e.g., Krum/Multi-Krum~\cite{blanchard2017nips} and Bulyan~\cite{mhamdi2018pmlr}) or data-driven approaches (e.g., spectral analysis via K-means clustering~\cite{shen2016acm} or spectral analysis), or methods based on ``source of trust'' (e.g., FLTrust~\cite{cao2020fltrust}).
% OLD, LONG VERSION
%Several countermeasures have been proposed in the literature to combat Byzantine model poisoning attacks on FL systems.
% Descriptive statistics
% For example, Trimmed Mean and FedMedian aggregate local model updates using more robust statistics than standard average~\cite{yin2018icml}.
%
% % Heuristics for outlier detection
% Many existing Byzantine-resilient strategies implement some outlier detection heuristics to discard the model updates sent by potentially malicious clients from the input of the aggregation function.
% One of the most popular heuristics is Krum~\cite{blanchard2017nips}.
% This strategy tries to mitigate the impact of Byzantine attacks by selecting as a global model the local model with the smallest sum of Euclidean distances to {\em all} the other local models.
% Although powerful, Krum requires the server to know (or, at least, estimate) the number of malicious FL clients upfront, which is generally impossible in a realistic attack scenario. %
% Moreover, Krum may become ineffective for complex, high-dimensional model parameter spaces due to the curse of dimensionality.
% Bulyan~\cite{mhamdi2018pmlr} tries to overcome this issue by combining Krum with a variant of Trimmed Mean.
% % Data-driven outlier detection
% Other strategies use data-driven outlier detection techniques -- e.g., via K-means clustering~\cite{shen2016acm} -- to spot potential malicious local model updates. 
% %For instance, Shen et al. propose to cluster local model updates with K-means and thus identify outliers.
%
% % Other techniques
% As far as the server is concerned, any local model received can be from a potential malicious client. 
% FLTrust~\cite{cao2020fltrust} assumes the server acts as a client, i.e., trains a local model on an additional {\em trustworthy} dataset at the server's end and compares it against all the local models from other clients. 
% This way, the server can rely on some ``source of trust'' when discarding potentially malicious clients.
%\\
% Limitations of existing Byzantine-resilient strategies
Unfortunately, existing defense mechanisms either rely on simple heuristics (e.g., Trimmed Mean and FedMedian by~\cite{yin2018icml}) or need strong and unrealistic assumptions to work effectively (e.g., foreknowledge or estimation of the number of malicious clients in the FL system, as for Krum/Multi-Krum~\cite{blanchard2017nips} and Bulyan~\cite{mhamdi2018pmlr}, which, however, cannot exceed a fixed threshold).
Furthermore, outlier detection methods using K-means clustering~\cite{shen2016acm} or spectral analysis like DnC~\cite{shejwalkar2021ndss} do not directly consider the temporal evolution of local model updates received.
Finally, strategies like FLTrust~\cite{cao2020fltrust} require the server to collect its own dataset and act as a proper client, thereby altering the standard FL protocol.
\\
% OLD, LONG VERSION
% Overall, existing Byzantine-resilient strategies are either simple heuristics (e.g., FedMedian) or, if they are more complex, they rely on strong and unrealistic assumptions to work effectively (e.g., knowing the number of malicious clients in the FL system in advance, as for Krum and alike).
% Furthermore, data-driven outlier detection methods do not consider the temporary evolution of local model updates received (e.g., K-means clustering). 
% Finally, strategies like FLTrust requires the server to collect its own dataset and act as a proper client, thereby altering the standard FL protocol.
%
% Description of the proposed method
This work introduces a novel pre-aggregation \textit{filter} robust to untargeted model poisoning attacks. Notably, this filter $(i)$ operates without requiring prior knowledge or constraints on the number of malicious clients and $(ii)$ inherently integrates temporal dependencies. 
The FL server can employ this filter as a preprocessing step before applying \textit{any} aggregation function, be it standard like FedAvg or robust like Krum or Bulyan.
Specifically, we formulate the problem of identifying corrupted updates as a multidimensional (i.e., matrix-valued) time series anomaly detection task. 
The key idea is that legitimate local updates, resulting from well-calibrated iterative procedures like stochastic gradient descent (SGD) with an appropriate learning rate, show \textit{higher predictability} compared to malicious updates. This hypothesis stems from the fact that the sequence of gradients (thus, model parameters) observed during legitimate training exhibit regular patterns, as validated in Section~\ref{subsec:intuition}. %until convergence. 
%This regularity may be more pronounced for smooth convex loss functions, but it can still be captured within an appropriate time window, even for more complex and convoluted loss surfaces. 
%We provide evidence of this claim in Appendix~B, where we show that the average mutual information (i.e., ``predictability''), calculated over pairs of legitimate model updates sent at different FL rounds, is significantly higher than the corresponding computation for a malicious client.
\\
Inspired by the matrix autoregressive (MAR) framework for multidimensional time series forecasting~\cite{chen2021je}, we propose the FLANDERS ({\em \textbf{F}ederated \textbf{L}earning meets \textbf{AN}omaly \textbf{DE}tection for a \textbf{R}obust and \textbf{S}ecure}) filter.
The main advantages of FLANDERS over existing strategies like FLDetector~\cite{zhao2020multivariate} are its resilience to large-scale attacks, where $50\%$ or more FL participants are hostile, and the capability of working under realistic non-iid scenarios.
We attribute such a capability to two key factors: $(i)$ FLANDERS works without knowing a priori the ratio of corrupted clients, and $(ii)$ it embodies temporal dependencies between intra- and inter-client updates, quickly recognizing local model drifts caused by evil players. Below, we summarize our main contributions:

\begin{itemize}
\item[{\em(i)}]
We provide empirical evidence that the sequence of models sent by legitimate clients is more predictable than those of malicious participants performing untargeted model poisoning attacks.
\\
\item[{\em(ii)}] 
We introduce FLANDERS, the first pre-aggregation filter for FL robust to untargeted model poisoning based on multidimensional time series anomaly detection.
\\
\item[{\em(iii)}] 
We integrate FLANDERS into Flower,\footnote{\scriptsize{\url{https://flower.dev/}}} a popular FL simulation framework for reproducibility.
\\
\item[{\em(iv)}] 
We show that FLANDERS improves the robustness of the existing aggregation methods under multiple settings: different datasets, client's data distribution (non-iid), models, and attack scenarios.
\\
\item[{\em(v)}] 
We publicly release all the implementation code of FLANDERS along with our experiments.\footnote{\scriptsize{\url{https://anonymous.4open.science/r/flanders_exp-7EEB}}}
\end{itemize}

% Paper's structure and organization
The remainder of the paper is structured as follows. %some related work and the current state-of-the-art solutions to security issues that FL entails. 
Section~\ref{sec:background} covers background and preliminaries. 
In Section~\ref{sec:related}, we discuss related work.
Section~\ref{sec:problem} and Section~\ref{sec:method} describe the problem formulation and the method proposed. % to tackle it. 
Section~\ref{sec:experiments} gathers experimental results. %, and Section~\ref{sec:limitations} discusses some limitations of this work.
Finally, we conclude in Section~\ref{sec:conclusion}.
 %discusses the limitations of this work and draws future research directions.
%reports conclusions and draws perspectives for future research directions.

%%%%%%% OLD %%%%%%%
%to overcome the resilience of Byzantine failures in distributed Stochastic Gradient Descent computations. 
% The strength of Krum is its time complexity, which is linear in the gradient dimension. 
% However, the robustness of the approach is guaranteed for gradient-based learning applications only when the majority of the clients are not compromised. 
% Besides, the aggregation mechanism of Krum, as well as that of similar methods, is robust from a coarse-grained perspective and does not provide solutions to errors and perturbations that may occur at inference time.
%A related approach to~\cite{blanchard2017nips} is the work of Su et al.~\cite{su2016dc}. Here, the authors propose an iterated approximate agreement to tackle a multi-layer scenario attacked by Byzantine agents. 
%However, the method works efficiently on the sole discrete context and it is inapplicable to continuous state environments.
%\gabri{Maybe, we should just talk about the main limitations of existing countermeasures without digging into their details (or, we can just mention Krum as this is the most popular one). I will move the description of all these methods to the Related Work section.}
\section{Background on Network Calculus}
\label{sec: background}


\begin{figure*}[tbh]
\centering
\begin{subfigure}[b]{0.3\textwidth}
    \centering
    \includegraphics[width=\linewidth]{images/in-out.png}
    \caption{Arrival and departure data and their relation with delay $d(t)$ and backlog $b(t)$. For a FIFO system, the delay is the horizontal distance between $R(t)$ and $R^*(t)$ but some other multiplexing techniques may shift the data to a later priority, causing a longer delay.}
    \label{fig: data in-out}
\end{subfigure}
\hfill
\begin{subfigure}[b]{0.35\textwidth}
    \centering
    \includegraphics[width=\linewidth]{images/arrival-service.png}
    \caption{Characteristics of an arrival curve and a service curve. From any point of observation, the arriving data never exceeds its arrival curve; the departure data is also never less than the service curve with respect to the data arrival.}
    \label{fig: arrival-service curves}
\end{subfigure}
\hfill
\begin{subfigure}[b]{0.33\textwidth}
    \centering
    \includegraphics[width=\linewidth]{images/bound.png}
    \caption{Delay and backlog bounds of a system. Backlog is the maximum vertical distance between $\alpha(t)$ and $\beta(t)$; FIFO delay is their maximum horizontal distance; but for arbitrary multiplexing, the delay guarantee is when the system clears its buffer, thus it's the intersection of $\alpha(t)$ and $\beta(t)$.}
    \label{fig: system bounds}
\end{subfigure}
\caption{Network calculus framework. We let $R(t)$ and $R^*(t)$ be the arrival and departure data flow of a system; $\alpha(t)$ be the piecewise linear concave arrival curve and $\beta(t)$ be the piecewise linear convex service curve of a system.}
% \hossein{Better to show piece-wise linear concave arrival curve and piece-wise linear convex service curve instead of token-bucket and rate-latency.}}
\end{figure*}

We recall some of the network calculus essentials for a better understanding of the framework used in Saihu. In the following context, we use the following notation: $\mbb{R}^+$ is the set of non-negative real numbers; $[x]_+$ denotes $\max(0, x)$

The data flow is by convention modeled as a left-continuous wide-sense increasing function $R(t): \mbb{R}^+ \mapsto \mbb{R}^+$ with respect to time $t$~\cite{ncbook2001leboudec}. 

A system $\mcal{S}$ receives arrival data described as a cumulative function $R(t)$ and delivers departure data as another cumulative function $R^*(t)$. Figure~\ref{fig: data in-out} illustrates such a system $\mcal{S}$. The benefit of representing a system like this is that we can observe system backlog and delay with such a model. 

\begin{definition}[Backlog and Delay~\cite{ncbook2001leboudec}]
    The backlog of a system at time~$t$ is
    \begin{equation}
        b(t) = R(t) - R^*(t)
    \end{equation}
    
    The virtual delay of a FIFO system at time $t$ is
    \begin{equation}
        d_{FIFO}(t) = \inf \lbp \tau \geq 0 : R(t) \leq R^*(t+\tau) \rbp
    \end{equation}
\end{definition}



The backlog of a system can be viewed as the vertical distance between $R$ and $R^*$. The FIFO (\textit{First-in First-out}) delay is the horizontal distance between $R$ and $R^*$. One may obtain other delay values if the multiplexing technique is not FIFO.

% \begin{figure}
%     \centering
%     \includegraphics[width=0.9\linewidth]{images/in-out.png}
%     \caption{In/out data flow; delay and backlog}
%     \label{fig: data in-out}
% \end{figure}

Since we are interested in the system guarantee instead of a single instance of data flow, we would like to have general bounds to the arrival and departure data flows. Therefore, we define \textit{arrival curve} and \textit{service curve} as the bounds of arrival and departure data flows.

\begin{definition}[Arrival Curve~\cite{ncbook2001leboudec}]
    Given a wide-sense increasing function $\alpha: \mbb{R}^+ \mapsto \mbb{R}^+$, we say that a flow $R(t)$ is $\alpha$-constrained if and only if for all $s \leq t$:
    \begin{equation}
        R(t) - R(s) \leq \alpha(t-s)
    \end{equation}
    We say $R(t)$ has $\alpha$ as an arrival curve.
\end{definition}

\begin{definition}[Service Curve~\cite{ncbook2001leboudec}]
    Given a wide-sense increasing function $\beta: \mbb{R}^+ \mapsto \mbb{R}^+$ and $\beta(0) = 0$. A system $\mcal{S}$ having $R(t)$ and $R^*(t)$ as its arrival and departure flows. We say $\mcal{S}$ offers a service curve $\beta$ if and only if
    \begin{equation}
        R^*(t) \geq (R \otimes \beta)(t) =: \inf_{s \leq t} \lbp R(s) + \beta(t-s) \rbp
    \end{equation}
    where $\otimes$ denotes the min-plus convolution
\end{definition}

Figure~\ref{fig: arrival-service curves} illustrates the arrival and service curves. Any segment of arrival flow $R(t)$ is constrained by arrival curve $\alpha$ and the output curve $R^*(t)$ is always no less than the curve $R\otimes\beta$. As a result, an arrival curve upper bounds the incoming traffic, and a service curve lower bounds the outgoing traffic.

% \begin{figure}
%     \centering
%     \includegraphics[width=\linewidth]{images/arrival-service.png}
%     \caption{Arrival/Service curve}
%     \label{fig: arrival-service curves}
% \end{figure}

We consider 2 special types of curves throughout this paper, \textit{token-bucket} (or sometimes called \textit{leaky-bucket}) curve and \textit{rate-Latency} curve.

\begin{definition}[Token-bucket and Rate-latency~\cite{ncbook2001leboudec}]
    A token-bucket curve $\gamma_{r,b}$ with arrival rate $r$ and burst $b$ is defined as
    \begin{equation}
        \gamma_{r,b}(t) = b + rt
    \end{equation}

    A rate-latency curve $\beta_{R,T}$ with service rate $R$ and latency $T$ is defined as
    \begin{equation}
        \beta_{R,T}(t) = R \lb t - T \rb_+
    \end{equation}
\end{definition}

A token-bucket curve is determined by a burst $b$ and an arrival rate~$r$. Burst represents the maximum possible data volume that can arrive simultaneously, and arrival rate represents the maximum long-term data rate~\cite{bouillard2022tradeoff}.
A rate-latency curve is determined by a latency~$T$ and a service rate~$R$. Latency represents the time a server needs before starting to process the incoming data, and service rate represents the minimum rate to process data after the initial latency.

With the help of arrival and service curves, we can derive delay and backlog bounds for a system $\mcal{S}$ illustrated in Figure~\ref{fig: system bounds}. Suppose a system $\mcal{S}$ has arrival curve $\alpha$ and service curve~$\beta$, its worst-case backlog $b^*$ is the maximum vertical distance between~$\alpha$ and~$\beta$. Similarly, depending on the multiplexing technique applied to the system, its worst-case delay bound $d^*$ is the maximum horizontal distance between $\alpha$ and $\beta$ if $\mcal{S}$ is a FIFO system. If we don't have any information about its multiplexing technique, referred to as arbitrary multiplexing, the best we can say is that when $\alpha$ and $\beta$ intersect each other, where all data has been delivered out of the system. Consequently, the worst-case delay bound for arbitrary multiplexing is the time required for $\mcal{S}$ to clear its buffer.

% \begin{figure}
%     \centering
%     \includegraphics[width=\linewidth]{images/bound.png}
%     \caption{System delay/backlog bounds}
%     \label{fig: system bounds}
% \end{figure}

While a service curve captures the slowest possible output speed of a system, a link's transmission capacity limits the speed as well. Hence, we model this phenomenon using a \textit{greedy shaper} with a sub-additive function $\sigma: \mbb{R}^+ \mapsto \mbb{R}^+$ concatenated with a server. We consider a concatenation as shown in Figure \ref{fig: system}. By convention we assume $\sigma(0) = 0$ and $\beta(t) \leq \sigma(t), \forall t \in \mbb{R}^+$, meaning that the buffer is cleared at the beginning and the service never exceed its physical limitation. With the above definition, such greedy shaper conserves the service provided by the system due to theorem \ref{thm: shaping}.

\begin{figure}[thb]
    \centering
    \includegraphics[width=0.7\linewidth]{images/system.png}
    \caption{Shaping of departure data. A flow that has an arrival curve $\alpha$ feeds into a server with an arrival data flow $R(t)$. The server having service curve $\beta$ takes $R(t)$ and gives a departure data flow $R^*(t)$ to a shaper with shaping function $\sigma$. The shaper takes $R^*(t)$ and shape the data flow as another departure $D(t)$.}
    \label{fig: system}
\end{figure}


\begin{theorem}[Shaping conserves service \cite{ncbook2001leboudec}]
\label{thm: shaping}
Following the system shown in Figure \ref{fig: system}, we have
\begin{equation}
     D = R^* \otimes \sigma \geq \lp R \otimes \beta \rp \otimes \sigma = R \otimes \lp \beta \otimes \sigma \rp = R \otimes \beta
\end{equation}
\end{theorem}

In the following context, we model the shaping function $\sigma$ as a token-bucket curve $\gamma_{C,L}$ with transmission capacity $C$ and the packet size $L$ to capture the link capacity and packetization~\cite{bouillard2022tradeoff}.

\section{DG Methods: Data Manipulation}\label{sec:data-manipulation}
In order to generalize to unseen scenarios, this category of methods manipulates the DNN input data. Two types of manipulations are possible either in the raw input space or in the latent input space: $i)$ data augmentation by adding random noise or transformation to the input data, and $ii)$ data generation which generates new training samples using generative models. The main objective of these methods is to increase the quantity and improve the diversity of the training dataset for better generalization capabilities without requiring manual labeling of datasets.

\noindent A data manipulation operation is represented by an arbitrary function $\mathcal{M}(\cdot)$ which transforms the input data $X$ to the manipulated data $X^\prime = \mathcal{M}(X)$. Given a DNN that is represented as an input-output function $g(\cdot)$, the learning objective of data manipulation for DG can be expressed as follows:
\begin{equation}\label{eq:data-manipulation-cost-function}
\min_{g}\,\underbrace{\mathbb{E}_{{\mathsf{X}},{\mathsf{Y}}}\big[\mathcal{L}(g(X),Y)\big]}_{\textrm{task loss}} + \underbrace{\mathbb{E}_{\mathsf{X}^{'},\mathsf{Y}}\big[\mathcal{L}(g(X^\prime),Y)\big]}_{\textrm{data manipulation loss}},
\end{equation}
where $\mathcal{L}(\cdot,\cdot)$ is the DNN cost function. It is worth noting that most data manipulation techniques proposed in the literature are geared towards computer vision applications where all datasets consist of images. In this section, we describe these methods within the context of vision applications and point out their potential use for wireless applications.

\subsection{Data Generation}
Generating new data samples using generative models is a popular technique to augment existing datasets so as to cover richer training scenarios, thereby enhancing the generalization capability of a DNN. The data manipulation function $\mathcal{M}(\cdot)$ in (\ref{eq:data-manipulation-cost-function}) can be represented by deep generative models such as variational auto-encoder (VAE) \cite{DBLP:journals/corr/KingmaW13} and generative adversarial network
(GAN) \cite{DBLP:journals/corr/GoodfellowPMXWOCB14}.

Various distribution distance metrics can be employed to generate high-quality samples including:
\begin{itemize}
    \item \textit{domain discrepancy measures} such as the maximum mean discrepancy (MMD) \cite{DBLP:journals/jmlr/GrettonBRSS12} to minimize the distribution divergence between real and generated data samples.
    \item \textit{the Wasserstein distance} between the prior distribution of the DNN input and a latent target distribution as carried out in Wasserstein auto-encoder (WAE) \cite{DBLP:journals/corr/abs-1711-01558}. This metric is a regularization that encourages the encoded training distribution of a WAE to match the data prior and hence preserves the semantic and domain transfer capabilities.
    \item \textit{semantic consistency loss functions} that maximize the difference between the source and the newly generated distributions, thereby creating new domains that augment the existing source domains \cite{DBLP:conf/eccv/ZhouYHX20}.
\end{itemize}

\noindent It is also possible to generate new domains instead of new data samples using adversarial training \cite{DBLP:conf/cvpr/LiGCHWMYLX21} where one or multiple generative models are trained to progressively generate unseen domains by learning relevant cross-domain invariant representations. Such an alternative involves an entire generative model pipeline composed of multiple DNNs trained in cascade or in parallel, and therefore has a significant computational cost. As one example for channel estimation problems, one can start by generating line-of-sight datasets and then progressively increase the rank of the estimated MIMO channel to multi-path models up to full-rank channels such as rich-scattering MIMO channels.

Furthermore, the data manipulation function $\mathcal{M}(\cdot)$ can also be defined without training generative models. In particular, it is possible to generate new data samples by linearly interpolating any two training samples and their associated labels as done in the low-complexity Mixup method \cite{DBLP:conf/iclr/ZhangCDL18}. More recently, many techniques have built upon Mixup for DG to $i)$ generate new data samples by interpolating either in the raw data space \cite{DBLP:conf/bmvc/WangL0K021,DBLP:conf/icassp/WangLK20,DBLP:conf/cvpr/ShuCW0L21}, or $ii)$ to build robust models with better generalisation capabilities by interpolating in the feature space \cite{DBLP:conf/iclr/ZhouY0X21,DBLP:conf/cvpr/Qiao021,DBLP:conf/cvpr/XuZ0W021}.

\subsection{Data Augmentation}

DNNs are heavily reliant on large datasets to enhance the generalization by avoiding overfitting \cite{DBLP:journals/jbd/ShortenK19}. Data augmentation methods provide a cheap way to augment training datasets. They artificially inflate the dataset size by transforming existing data samples while preserving labels. Data augmentation includes geometric and color transformations for visual tasks, random erasing and/or permutation, adversarial training, and neural style transfer. Every data augmentation operation can be considered as a data manipulation function $\mathcal{M}(\cdot)$ in (\ref{eq:data-manipulation-cost-function}). Here, we classify the data augmentation methods for DG into two categories:
\begin{itemize}
    \item \textit{domain randomization}: this family of methods creates a variety of datasets stemming from data generation processes (e.g., simulated environments) with randomized properties and trains a model that generalizes well across all of them.
    \item \textit{adversarial data augmentation}: this family of methods guides the augmentation by enhancing the diversity of the dataset while ensuring their reliability for better generalization capabilities.
\end{itemize}

\begin{figure*}[ht]
     \centering
         \centering
         \includegraphics[scale=0.65]{figures/domain-randomization.pdf}
         \vspace{0.15cm}
         \caption{Summary of the training and evaluation pipeline of machine learning models under data distribution shifts for communication applications.}
         \label{fig:domain-randomization}
\end{figure*}

\subsubsection{\textbf{Domain randomization}}\label{subsubsec:domain-randomization} The reality gap between the data domains resulting from simulations and real-world data collections often leads to failure due to distribution shifts. This gap is triggered by an inconsistency between the physical parameters of simulations (e.g., channel distribution, noise level) and, more fatally, the incorrect physical modeling (e.g., physical considerations of wireless communication \cite{DBLP:journals/tcas/IvrlacN10,akrout2022achievable}). To perceive how DNNs should be trained and evaluated under data distribution shifts for communication applications, Fig.~\ref{fig:domain-randomization} depicts the training and evaluation pipeline where datasets are generated through communication systems models. There, it is seen that source (i.e., training) and target (i.e., test) domains, $\mathcal{D}^{\textrm{train}}$ and $\mathcal{D}^{\textrm{test}}$, are obtained according to the training and test scenarios, $\mathcal{S}^{\textrm{train}}$ and $\mathcal{S}^{\textrm{test}}$. The latter are determined by defining a set of communication scenarios by varying one or multiple communication parameters of interest. The choice of these parameters dictates the data domains and hence provides a way to control and then analyze the impact of distribution shifts on the performance of DNNs. For instance, research efforts to design broadband ML-aided decoding algorithms should vary the signal frequency and assess the generalization capability of DNNs when trained on carriers in the sub-6 GHz band then evaluated on a different communication band. 

\noindent Domain randomization generates new data samples stemming from simulated dynamics of complex environments. For computer vision applications, the function $\mathcal{M}(\cdot)$ in (\ref{eq:data-manipulation-cost-function}) encloses different manual transformations such as altering object properties (e.g., shape, location, texture), scene editing (e.g., illumination, camera view), or random noise injection \cite{DBLP:conf/iros/TobinFRSZA17}. For real-valued data input vectors, augmentation involves scaling, pattern switching, and random perturbation \cite{pialla2022data}. These augmentation methods are particularly interesting for wireless communication applications because they handle general signal transmission scenarios that are tolerant to variations in the path-loss coefficient, synchronization delays, signal-to-noise ratio, etc.

\subsubsection{\textbf{Adversarial data augmentation}}
\noindent The fact that most domain randomization described in Section \ref{subsubsec:domain-randomization} is performed randomly indicates that there exist potential improvements to remove ineffective randomization that does not help with DNNs' generalization. This optimization is performed by adversarial data augmentation.

Toward this goal, research efforts have been dedicated to designing better strategies for non-random data augmentation. By modeling the dependence between the data sample $X$, its label $Y$, and the domain label $d$ (cf. Definition \ref{def:DG}), it has been shown that the input data can be perturbed along the direction of greatest domain change (i.e., domain gradient) while changing the class label as little as possible \cite{DBLP:conf/iclr/ShankarPCCJS18}. Another line of work devised an adaptive data augmentation procedure where adversarially perturbed samples in the feature space are iteratively added to the training dataset \cite{DBLP:conf/nips/VolpiNSDMS18}. It is also possible to train a dedicated transformation network for data augmentation by $i)$ maximizing the domain classification loss on the transformed data samples to tolerate domain generation differences, and $ii)$ minimizing the label classification loss to ensure that the learned augmentation does not affect the DNN performance \cite{DBLP:conf/aaai/ZhouYHX20}. While adversarial data augmentation can provide richer datasets and fill in data gaps against some adversarial examples, this comes at the cost of a more complex training procedure which is known to be less stable and computationally extensive.

When it comes to wireless communications applications, physics-based models are available to guide data augmentations that are consistent with the law of physics, beyond purely random strategies. For example, the study of the achievable rate of reconfigurable intelligent surface (RIS)-aided communication systems do exhibit the same performance regardless of the carrier frequency due to the scaling invariance property of Maxwell's equations when no source is present (i.e., passive RISs) \cite{jackson1999classical}. Another interesting implication stemming from the symmetry of Maxwell's equations is the frequency independence property of certain wideband antennas that display very similar radiation pattern, gain and impedance above a certain threshold frequency \cite{hohlfeld1999self}. This suggests that the generation of wireless datasets for far-field communication can be made independent of the carrier frequency for specific types of antennas.

From this perspective, data augmentation methods that are aware of the physics of wave propagation do not blindly generate source and target domains for different carrier frequencies. They should instead collapse the data augmentation process to scenarios that do enjoy the scaling invariance property. As a result, not only do data augmentation techniques become efficient but also physically consistent with the electromagnetic properties of RISs. 

\section{DG Methods: Representation Learning}\label{sec:representation-learning}
Generalizing to unseen scenarios is not solely dependent on the DNN prediction approximation function $g(\cdot)$ given in (\ref{eq:data-manipulation-cost-function}). It also depends on the data representations (i.e., features) learned by the DNN \cite{DBLP:journals/pami/BengioCV13}. To better isolate these two distinct tasks, one can view the overall DNN approximation function, $g(h(\cdot))$, as a composition of a prediction/classification function $g(\cdot)$ and a representation learning function $h(\cdot)$. Fig. \ref{fig:DL-rep-learning} depicts this decomposition, $h(X)$, as the output of the representation learning step. In theory, this representation in the feature space comprises two separate representations. The first one denoted by $h_{\textrm{inv}}(X)$ is a domain-invariant representation that is shared across domains (a.k.a., cross-domain representation) and is key to enabling generalization over multiple domains. The second representation $h_{\textrm{spe}}(X)$, however, is domain-specific and represents the variation pertaining to a specific domain. In practice, these two representations can either be non-separable or separable. For instance, several earlier research works \cite{DBLP:conf/icassp/OppenheimLKP79,oppenheim1981importance,hansen2007structural} have shown that in the Fourier spectrum of signals, the phase component predominantly carries low-level statistics whereas the amplitude component mainly contains high-level semantics. Hence, Fourier phase features represent domain-invariant features that cannot be easily affected by domain shifts when used for DG \cite{lu2022domaininvariant}.

\begin{figure}[th!]
     \centering
     %\vspace{-0.2cm}
%\rule{\textwidth}{0.4pt}\vspace{0.2cm}
     \begin{subfigure}[b]{0.49\textwidth}
         \centering
         \includegraphics[scale=0.32]{figures/representation-learning1.pdf}
         \caption{without representation learning\vspace{0.5cm}}
         \label{fig:DL-standard}
     \end{subfigure}
     
     \begin{subfigure}[b]{0.49\textwidth}
         \centering
         \includegraphics[scale=0.32]{figures/representation-learning2.pdf}
         \caption{with representation learning}
         \label{fig:DL-rep-learning}
     \end{subfigure}
    \caption{Illustration of ML-aided classification/prediction (a) without an explicit representation learning step (a.k.a. end-to-end learning), and (b) with a representation learning step.}
    \label{fig:with-without-rep-learning}
    %\vspace{-0.2cm}
\end{figure}

From a mathematical point of view, the optimization problem of representation learning can be written as follows:

\begin{equation}\label{eq:representation-learning-cost-function}
\min_{g,\,h}~\,\underbrace{\mathbb{E}_{{\mathsf{X}},{\mathsf{Y}}}\big[\mathcal{L}(g(h(X)),Y)\big]}_{\textrm{task loss}} + ~\lambda\,\hspace{-0.5cm}\underbrace{r(X)}_{\textrm{regularization loss}},
\end{equation}
where $r(X)$ is a regularization function and $\lambda$ is the associated regularization parameter.

\noindent Depending on the type of the regularization function $r(X)$ or the representation learning function $h(\cdot)$, it is possible to categorize representation learning for DG into two categories:
\begin{itemize}%[leftmargin=*]
    \item \textit{domain-invariant representation learning}: the goal of this family of methods is to learn features that are invariant across different domains. These features are transferable from one domain to another, hence their importance for domain generalization.
    
    \item \textit{feature disentanglement}: these methods decompose a feature representation into one or multiple sub-features, each of which is either domain-specific or domain-invariant.
\end{itemize}

\subsection{Data-Invariant Representation Learning}\label{subsec:data-invariant-representation-learning}

\subsubsection{\textbf{Kernel-based methods}}
Learning representation using kernel methods (e.g., support vector machines \cite{DBLP:journals/ml/CortesV95}, kernel component analysis \cite{scholkopf1997kernel}) is a classical problem in the ML literature. In such a setting, the representation learning function $h(\cdot)$ in (\ref{eq:representation-learning-cost-function}) maps the data samples to the feature space using kernel functions (e.g., radial basis function (RBF), Gaussian, and Laplacian kernels).

For domain generalization, several methods were devised to learn domain-invariant kernels to determine $h(\cdot)$ from the training dataset. Specifically, a positive semi-definite kernel learning approach for DG was proposed in \cite{DBLP:journals/jmlr/BlanchardDDLS21} by considering the conventional supervised learning problem where the original feature space is augmented to include the marginal distribution that generates the features. It is also possible to learn kernel functions by minimizing the distribution discrepancy between all the data samples in the feature space. This method is known as domain-invariant component analysis (DICA) \cite{DBLP:conf/icml/MuandetBS13} and is one of the classical kernel methods for DG.

For classification tasks, in presence of covariate shift only, a randomized kernel algorithm was devised in \cite{DBLP:conf/ijcai/ErfaniBMNLBR16} to extract features that minimize the difference between the marginal distributions across domains. Multi-domain discriminant analysis (MDA) and scatter component analysis (SCA) approaches were proposed in \cite{DBLP:conf/uai/Hu0CC19,DBLP:journals/pami/GhifaryBKZ17} to learn a domain-invariant feature transformation in presence of both covariate and conditional shifts across domains. This is done by jointly minimizing the divergence among domains within each class and maximizing the separability among classes.

\subsubsection{\textbf{Domain adversarial learning}}

Since the presence of spurious features in the data decreases the robustness of DNNs, adversarial learning is a widely used technique to learn invariant features by training generative adversarial networks (GANs). Specifically, the discriminator is trained to distinguish the domains while the generator is trained to fool the discriminator so as to learn domain invariant feature representations for DG \cite{DBLP:conf/cvpr/LiPWK18}. Another line of work in \cite{DBLP:conf/cvpr/GongLCG19} generated a continuous sequence of intermediate domains flowing from one domain to another to gradually reduce the domain discrepancy, and hence improve the DNN generalization ability on unseen target domains. Learning class-wise adversarial networks for DG was also  proposed in \cite{DBLP:conf/eccv/LiTGLLZT18} based on conditional invariant adversarial training when both covariate and conditional shifts coexist.

\subsubsection{Explicit feature alignment}\label{subsubsec:explicit-feature-alignment}

This family of methods learns domain-invariant representations by aligning the features across source domains using one of the following two mechanisms:
\begin{itemize}%[leftmargin=*]
    \item explicit feature distribution alignment through distance minimization or moment matching.
    \item feature normalization addressing data variations to avoid learning nonessential domain-specific features.
\end{itemize}

Feature distribution alignment methods were devised to impose a variety of distribution distances such as the maximum mean discrepancy (MMD) on latent feature distributions \cite{DBLP:conf/cvpr/LiPWK18,DBLP:journals/corr/TzengHZSD14}, and the label similarities for samples of the same classes from different domains using the Wasserstein distance \cite{DBLP:journals/ijon/ZhouJSWC21}. Moment matching for multi-source domain adaptation (M3SDA) was also introduced in \cite{DBLP:conf/iccv/PengBXHSW19} to transfer learned features from multiple labeled source domains to an unlabeled target domain by dynamically aligning moments of their feature distributions.


Feature normalization methods, however, focus on increasing the discrimination capability of DNNs. They do so by normalizing the features to eliminate domain-specific variation while keeping domain-invariant features to enhance generalization. In particular, instance normalization (IN) \cite{DBLP:conf/eccv/PanLST18} and batch instance normalization (BIN) \cite{DBLP:conf/nips/NamK18} have been proposed to enhance the generalization capabilities of convolutional neural networks (CNNs). Instance normalization has been applied in \cite{DBLP:journals/corr/abs-2111-15077} for DG where labels were missing in the training domains to acquire invariant and transferable features. It was also shown that adaptively learning the normalization technique can improve DG without predefining the normalization technique in the DNN architecture a priori \cite{DBLP:conf/cvpr/FanWKYGZ21}.

\subsubsection{\textbf{Invariant risk minimization}}
Another unique perspective on learning domain-invariant representations for DG is to constrain DNNs to have the same output across all domains. The motivation behind this constraint is that an optimal representation for prediction or classification is \textit{the cause} of the DNN output label. This causal relationship from the representation (i.e., the cause) to the label (i.e., the effect) should not be affected by other factors including the domain input. Therefore, the optimal representation is domain invariant and can be learned using invariant risk minimization (IRM) \cite{DBLP:journals/corr/abs-1907-02893}. Given $K$ different domains, the IRM problem can be formulated as follows:
\begin{subequations}\label{eq:IRM}
    \begin{align}
    &\min_{h\in\mathcal{H}}~ \sum\limits_{k=1}^{K}~\mathbb{E}_{\,\mathsf{X}_k,{\mathsf{Y}}_k}\big[\mathcal{L}(g(h(X_k)),Y_k)\big]\label{eq:IRM-1}\\
    &\textrm{subject to~} g \in \bigcap_{k=1}^{K}\,\argmin_{g'\in\,\mathcal{G}}~\mathbb{E}_{\,{\mathsf{X}}_k,{\mathsf{Y}}_k}\big[\mathcal{L}(g'(h(X_k)),Y_k)\big],\label{eq:IRM-2}
    \end{align}
\end{subequations}
where $\mathcal{H}$ and $\mathcal{G}$ are the learnable function classes for representation and task functions, $h(\cdot)$ and $g(\cdot)$, respectively. The optimization in (\ref{eq:IRM}) finds the optimal representation function $h(\cdot)$ that minimizes the sum of all the task losses in (\ref{eq:IRM-1}) given in (\ref{eq:representation-learning-cost-function}). This minimization is carried out under the constraint in (\ref{eq:IRM-2}) which ensures that all domains share
the same optimal representation function $h(\cdot)$.

The idea behind the IRM formulation has drawn significant attention. Specifically, the IRM optimization was extended to text classification \cite{DBLP:journals/corr/abs-2004-05007}, reinforcement learning \cite{DBLP:conf/l4dc/SonarPM21}, self-supervised settings \cite{DBLP:conf/iclr/MitrovicMWBB21}, and to the case of extrapolated task losses among source domains \cite{DBLP:conf/icml/KruegerCJ0BZPC21}. Moreover, it was shown in \cite{DBLP:conf/nips/AhujaCZGBMR21} that constraining the invariance to the task function $g(\cdot)$ only~--- as done in (\ref{eq:IRM})~--- is not enough to guarantee the causal relationship from the representation to the label. A new regularization has thus been proposed to ensure that the representation function $h(\cdot)$ cannot capture fully invariant features that break down the assumed causality as required by the IRM formulation.

\subsection{Feature Disentanglement}

Unlike domain-invariant representation learning, disentangled representation learning relies on DNNs to learn a function that maps a data sample to a feature vector, which factorizes into distinct feature sets as depicted in Fig. \ref{fig:disentangled-representation}. There, it is seen that the entire feature space can be decomposed into a set of feature subspaces. Each feature set is a representation pertaining to a specific feature subspace only. When the feature representation is decomposable into multiple non-overlapping feature subsets, the feature representation is said to be ``disentangled''.
\begin{figure}[h!]
     \centering
         \centering
         \includegraphics[scale=0.48]{figures/disentangled-representation.pdf}
         \caption{Illustration of how a trained neural network transforms a data sample into a disentangled representation vector that factorizes into $N$ small feature vectors.}
         \label{fig:disentangled-representation}
\end{figure}

\noindent The importance of disentanglement-based representation learning for DG stems from the fact that features can be explicitly decomposed into domain-invariant and domain-specific features. As a result, the representation function $h(\cdot)$ defined in (\ref{eq:representation-learning-cost-function}) can be decomposed into two distinct representation functions: $h_{\textrm{inv}}(\cdot)$ for domain-invariant representation and $h_{\textrm{spe}}(\cdot)$ for domain-specific representation. The disentanglement-based optimization can be formulated as follows:
\begin{equation}\label{eq:disentangled-representation-learning-cost-function}
\begin{aligned}
\hspace{-0.3cm}\min_{h_{\textrm{spe}},\,h_{\textrm{inv}},\,g}~&\,\underbrace{\mathbb{E}_{{\mathsf{X}},{\mathsf{Y}}}\big[\mathcal{L}(g(h_{\textrm{inv}}(X)),Y)\big]}_{\textrm{task loss}} +~ \lambda\,\hspace{-0.5cm}\underbrace{r(X)}_{\textrm{regularization loss}} \\
&\hspace{1cm}+ \mu\,\,\underbrace{\mathbb{E}_{{\mathsf{X}}}\big[\mathcal{L}(h_{\textrm{inv}}(X),h_{\textrm{spe}}(X),X)\big]}_{\textrm{reconstruction loss}},
\end{aligned}
\end{equation}
where $\lambda$ and $\mu$ are regularization parameters. In (\ref{eq:representation-learning-cost-function}), the regularization loss encourages the separation between domain-invariant and domain-specific features, while the reconstruction loss ensures that such separation does not lead to significant information loss. In other words, regularization and reconstruction losses are competing penalties that add up to the task loss, and it is the task of the ML designer to find the suitable trade-off that enhances the generalization of DNNs.

\subsubsection{\textbf{Multi-component analysis}}
Multi-component methods dedicate different sets of parameters to learn domain-invariant and domain-specific features. The method ``UndoBias'' proposed in \cite{DBLP:conf/eccv/KhoslaZMET12} learns dedicated SVM models. It represents the dedicated SVM parameters, $\bm{w}_k$, pertaining to the $k$th domain as a perturbation of the domain-invariant parameters $\bm{w}$ with the domain-specific parameters $\Delta\bm{w}_k$, i.e., $\bm{w}_k = \bm{w} + \Delta\bm{w}_k$. This method has been extended for multi-view vision tasks by introducing a regularization to minimize the mismatch between any two view representations \cite{DBLP:conf/iccv/NiuLX15} for better generalization. Neural networks have also been used to capture disentangled representations by learning domain-specific
networks for each domain and one domain-invariant network for all domains \cite{DBLP:journals/tip/DingF18}. Another line of work considered manually comparing specific areas of DNN's attention heatmaps from different domains which proved beneficial to learning disentangled representations and ensuring a more robust generalization \cite{DBLP:conf/cvpr/ZuninoBVS0SMS21}.

\subsubsection{\textbf{Generative modeling}}
Generating data samples whose feature representations are disentangled requires adapting the data generative process of generative models to new constraints. The latter can be incorporated in the loss functions of GANs to encourage feature disentanglement
by separating the domain-specific and domain-invariant features \cite{DBLP:journals/corr/abs-2109-05826}. An autoencoder-based variational approach was devised to disentangle the features by learning three independent latent subspaces, one for the domain, one for the class, and one for any residual variations \cite{DBLP:conf/midl/IlseTLW20}. To generate domains that are different from the source domain, the discrepancy between augmented and sources domains was maximized for out-of-domain augmentation using meta-learning under a semantic consistency constraint \cite{DBLP:conf/cvpr/QiaoZP20}.

For classification tasks, diversifying the inter-class
variation by modeling potential seen or unseen variations across classes was formulated as a disentanglement-constrained optimization problem \cite{DBLP:conf/cvpr/ZhangZLWSX22}. This was made possible by minimizing the discrepancy of the inter-class variation where both intra- and inter-domain variations are regarded as constraints.

% \begin{table*}[t]
% \begin{center}
% \begin{tabular}{ c| c : c: c}
% \multirow{2}{1em}{Input} & Novel View 1 & Novel View 2 & Novel View 3\\
% \cline{2-4}
% & HumanNeRF \hspace{5mm} Ours & HumanNeRF \hspace{5mm} Ours & HumanNeRF \hspace{5mm} Ours   \\
% \hline \\
% \includegraphics[width=0.1\linewidth] {figs/internal/in_1.png} & \includegraphics[width=0.2\linewidth]{figs/internal/internal_1_v1.png} & \includegraphics[width=0.2\linewidth]{figs/internal/internal_1_v2.png} & 
% \includegraphics[width=0.2\linewidth]{figs/internal/internal_1_v3.png}\\  
% \hline
% \includegraphics[width=0.1\linewidth] {figs/internal/in_1.png} & \includegraphics[width=0.2\linewidth]{figs/internal/internal_1_v1.png} & \includegraphics[width=0.2\linewidth]{figs/internal/internal_1_v2.png} & 
% \includegraphics[width=0.2\linewidth]{figs/internal/internal_1_v3.png}\\  
% \hline
% \includegraphics[width=0.1\linewidth] {figs/internal/in_1.png} & \includegraphics[width=0.2\linewidth]{figs/internal/internal_1_v1.png} & \includegraphics[width=0.2\linewidth]{figs/internal/internal_1_v2.png} & 
% \includegraphics[width=0.2\linewidth]{figs/internal/internal_1_v3.png}\\  
% \hline
% \end{tabular}
% \end{center}
%   \caption{Qualitative performance comparison on the self-captured In-The-Wild (ITW) dataset.}
%   \label{tab:all_datasets}
% \end{table*}

% \begin{table*}[t]
% \begin{center}
% \begin{tabular}{ c| c : c: c}
% \multirow{2}{1em}{Input} & Novel View 1 & Novel View 2 & Novel View 3\\
% \cline{2-4}
% & HumanNeRF \hspace{5mm} Ours & HumanNeRF \hspace{5mm} Ours & HumanNeRF \hspace{5mm} Ours   \\
% \hline \\
% \includegraphics[width=0.122\linewidth, trim={0 4cm 0 0.2cm},clip] {figs/internal/in_1.png} & \includegraphics[width=0.25\linewidth, trim={0 4cm 0 0.2cm},clip]{figs/internal/internal_1_v1.png} & \includegraphics[width=0.25\linewidth, trim={0 4cm 0 0.2cm},clip]{figs/internal/internal_1_v2.png} & 
% \includegraphics[width=0.25\linewidth, trim={0 4cm 0 0.2cm},clip]{figs/internal/internal_1_v3.png}\\  
% \hline
% \includegraphics[width=0.122\linewidth, trim={0 2cm 0 0},clip] {figs/internal/in_2.png} & \includegraphics[width=0.25\linewidth, trim={0 2cm 0 0},clip]{figs/internal/internal_2_v1.png} & \includegraphics[width=0.25\linewidth, trim={0 2cm 0 0},clip]{figs/internal/internal_2_v2.png} & 
% \includegraphics[width=0.25\linewidth, trim={0 2cm 0 0},clip]{figs/internal/internal_2_v3.png}\\  
% \hline
% \includegraphics[width=0.1\linewidth, trim={0 0cm 0 0},clip] {figs/pplsnp/in_pplsnp_1.png} & \includegraphics[width=0.25\linewidth, trim={0 2cm 0 0},clip]{figs/pplsnp/pplsnp_1_v1.png} & \includegraphics[width=0.256\linewidth, trim={0 2cm 0 0},clip]{figs/pplsnp/pplsnp_1_v2.png} & 
% \includegraphics[width=0.25\linewidth, trim={0 2cm 0 0},clip]{figs/pplsnp/pplsnp_1_v3.png}\\  
% \hline
% % \includegraphics[width=0.122\linewidth, trim={0 4cm 0 0},clip] {figs/internal/in_1.png} & \includegraphics[width=0.25\linewidth, trim={0 4cm 0 0},clip]{figs/internal/internal_1_v1.png} & \includegraphics[width=0.25\linewidth, trim={0 4cm 0 0},clip]{figs/internal/internal_1_v2.png} & 
% % \includegraphics[width=0.25\linewidth, trim={0 4cm 0 0},clip]{figs/internal/internal_1_v3.png}\\  
% % \hline
% % \includegraphics[width=0.122\linewidth, trim={0 4cm 0 0},clip] {figs/internal/in_1.png} & \includegraphics[width=0.25\linewidth, trim={0 4cm 0 0},clip]{figs/internal/internal_1_v1.png} & \includegraphics[width=0.25\linewidth, trim={0 4cm 0 0},clip]{figs/internal/internal_1_v2.png} & 
% % \includegraphics[width=0.25\linewidth, trim={0 4cm 0 0},clip]{figs/internal/internal_1_v3.png}\\  
% % \hline

% \end{tabular}
% \end{center}
%   \captionof{figure}{Qualitative comparison of novel rendered views on SCF dataset and the People Snapshot dataset re-sampled to sparse views. Notice that our approach significantly improves the results.}
%   \label{fig:itw}
% \end{table*}


\begin{table*}[ht!]
\begin{center}
\begin{tabular}{ c| c : c: c}
\multirow{2}{1em}{Input} & Novel View 1 & Novel View 2 & Novel View 3\\
\cline{2-4}
& \hspace{-4mm} HumanNeRF \hspace{1mm} Ours &  \hspace{-4mm} HumanNeRF \hspace{1mm} Ours & \hspace{-4mm} HumanNeRF \hspace{1mm} Ours   \\
\hline \\
\includegraphics[width=0.07\linewidth, trim={0 0cm 0 0cm},clip] {figs/new_int/in_1.png} & \includegraphics[width=0.2\linewidth, trim={0 2cm 0 0.2cm},clip]{figs/new_int/1_v1.png} & \includegraphics[width=0.2\linewidth, trim={0 2cm 0 0.2cm},clip]{figs/new_int/1_v2.png} & 
\includegraphics[width=0.2\linewidth, trim={0 2cm 0 0.2cm},clip]{figs/new_int/1_v3.png}\\  
\hline
\includegraphics[width=0.07\linewidth, trim={0 0cm 0 0cm},clip] {figs/new_int/in_2.png} & \includegraphics[width=0.2\linewidth, trim={0 2.2cm 0 0},clip]{figs/new_int/2_v1.png} & \includegraphics[width=0.2\linewidth, trim={0 2cm 0 0.2cm},clip]{figs/new_int/2_v2.png} & 
\includegraphics[width=0.2\linewidth, trim={0 2cm 0 0.2cm},clip]{figs/new_int/2_v3.png}\\  
\hline
\includegraphics[width=0.069\linewidth, trim={0 0cm 0 0cm},clip] {figs/new_int/in_4.png} & \includegraphics[width=0.194\linewidth, trim={0 2cm 0 0.2cm},clip]{figs/new_int/4_v1.png} & \includegraphics[width=0.199\linewidth, trim={0 2cm 0 0.2cm},clip]{figs/new_int/4_v2.png} & 
\includegraphics[width=0.197\linewidth, trim={0 2cm 0 0.2cm},clip]{figs/new_int/4_v3.png}\\  
\hline

\end{tabular}
\end{center}
  \captionof{figure}{Qualitative comparison of novel rendered views on SCF dataset (top two rows) and the People Snapshot dataset (bottom row) using sparse views. Our approach significantly improves the results.}
  \label{fig:itw}
  \vspace*{-1ex}
\end{table*}

In this section we describe the loss functions used to learn the FlexNeRF model and discuss details with respect to optimization and ray-sampling.
\vspace{-1mm}
\subsection{Loss Functions}
NeRFs are typically trained with a combination of losses between the rendered and observed frames. In addition, FlexNeRF also uses a combination of segmentation loss, cyclic consistency loss and temporal consistency loss as defined below.

\noindent\textbf{Segmentation Loss:} We apply the BCE-Dice loss between the predicted and ground truth binary segmentation masks
\begin{equation}
\begin{aligned}
    \mathbb{L}_{S} &= \frac{1}{N} \sum \left[ M \text{log} \hat{M} + (1-M)\text{log}(1-\hat{M}) \right] \\
    &+ \frac{2|M \cap \hat{M}|}{|M|+|\hat{M}|},
\end{aligned}
\end{equation}
% \begin{equation}
% \begin{aligned}
%     \mathbb{L}_{S} &= \frac{1}{N} \sum^N_{n=1} \left[ m_n \text{log} \hat{m}_n + (1-m_n)\text{log}(1-\hat{m}_n) \right] \\
%     &+ \frac{2|M \cap \hat{M}|}{|M|+|\hat{M}|},
% \end{aligned}
% \end{equation}
where $N$ is the number of pixels in the segmentation mask. 

\begin{figure}[t]
    \vspace*{-3ex}
  \centering
   \includegraphics[width=0.85\linewidth]{figs/supp/views_comp_flipped.png}
   \caption{LPIPS metric comparison on ZJU-MoCap between HumanNeRF \cite{Weng2022HumanNeRFFR} and our method with decreasing number of views.}
   \label{fig:views_comp}
   \vspace*{-4ex}
\end{figure}

\noindent\textbf{Cyclic Consistency Loss:} We introduce a cyclic consistency constraint on the canonical to observation space transformation, using Mean Squared Error (MSE) between $\hat{x}^o$ and $x^o$ defined by,
\begin{equation}
    \mathbb{L}_{CCL} = \frac{1}{L} \sum_{i=1}^{L} (\hat{x}^o_i- x^o_i)^2,
\end{equation}
where $L$ is the number of positional samples. 
\vspace{1mm}

\begin{table*}[t]
\begin{center}
\begin{tabular}{ |c|c|c|c|c|c| } 
\hline
{Dataset} & {Views} & {Method} & {LPIPS} $\times 10^3$ $\downarrow$ & {PSNR} $\uparrow$ & {SSIM} $\uparrow$\\ 
\hline
\rowcolor{gray!25}
\cellcolor{white} \multirow{6}{8em}{PeopleSnapshot \cite{alldieck2018video}} &  & HumanNeRF \cite{Weng2022HumanNeRFFR} & 39.27 & 27.65 & 0.8816 \\ 
% \cline{3-6}
\hhline{~~|*4{-}|}
\rowcolor{gray!25}
\cellcolor{white} & \multirow{-2}{5em}{Sparse$^\ddagger$} & Ours & \textbf{37.11} & \textbf{28.09} & \textbf{0.9003} \\ 
\cline{2-6}
 & \multirow{3}{5em}{Full} & Neural Body \cite{Peng2021NeuralBI} & 57.67$^*$ & 24.62 & 0.8490 \\ 
\cline{3-6}
% & & H-NeRF \cite{Xu2021HNeRFNR} & 57.31$^*$ & {26.33} & {0.8680} \\ 
% \cline{3-6}
& & HumanNeRF \cite{Weng2022HumanNeRFFR} & 36.79 & 28.05 & 0.8984 \\ 
\cline{3-6}
& & Ours & \textbf{35.63} & \textbf{28.77} & \textbf{0.9043} \\
%-------------------------------------------------------------
\hline
\hline
\rowcolor{gray!25}
\cellcolor{white} \multirow{6}{8em}{ZJU-MoCap \cite{Peng2021NeuralBI, fang2021mirrored}} &  & HumanNeRF \cite{Weng2022HumanNeRFFR} & 36.02 & 29.82 & 0.9597 \\ 
% \cline{3-6}
\arrayrulecolor{gray!25}
\hhline{~|*1{-}|}
\arrayrulecolor{black}
\hhline{~~|*4{-}|}
\rowcolor{gray!25}
\cellcolor{white} & \multirow{-2}{5em}{Sparse$^\ddagger$} & Ours & \textbf{31.68} & \textbf{30.18} & \textbf{0.9685} \\ 
\cline{2-6}
 & \multirow{3}{5em}{Full} & Neural Body \cite{Peng2021NeuralBI} & 52.28 & 29.07 & 0.9615 \\ 
\cline{3-6}
% & & H-NeRF \cite{Xu2021HNeRFNR} & - & - & - \\ 
\cline{3-6}
& & HumanNeRF \cite{Weng2022HumanNeRFFR} & 31.72 & 30.24 & 0.9679 \\ 
\cline{3-6}
& & Ours & \textbf{29.01} & \textbf{31.73} & \textbf{0.9765} \\

\hline
\hline
%-------------------------------------------------------------
\rowcolor{gray!25}
\cellcolor{white} &  & Neural Body \cite{Peng2021NeuralBI} & 48.62 & 25.07 & 0.9131 \\ 
\cline{3-6}
\rowcolor{gray!25}
% \cellcolor{white} & & H-NeRF \cite{Xu2021HNeRFNR} & - & - & - \\ 
% \cline{3-6}
\hhline{~~|*4{-}|}
\rowcolor{gray!25}
\cellcolor{white} & & HumanNeRF \cite{Weng2022HumanNeRFFR} & 39.71 & 26.12 & 0.9366 \\ 
% \cline{3-6}
\hhline{~~|*4{-}|}
\rowcolor{gray!25}
\cellcolor{white} \multirow{-4}{8em}{SCF Dataset$^\dag$} & \multirow{-3}{5em}{\cellcolor{gray!25} Sparse$^\ddagger$} & Ours & \textbf{34.26} & \textbf{29.55} & \textbf{0.9627} \\
\hline
\end{tabular}

\end{center}
  \caption{Comparison of performance across benchmark datasets. $*$ refers to adjusted LPIPS from the values reported in~\cite{Xu2021HNeRFNR} to fit the same scale as our experiments. $\dag$ refers to the Self-Captured Fashion (SCF) dataset. $\ddagger$ indicates the model trained with sparse ($\sim 40$) views.}
  \label{tab:all_datasets}
\end{table*}

% % \begin{table*}[t]
% \begin{center}
% \begin{tabular}{ |c|c|c|c|c|c|c| } 
% \hline
% \multirow{2}{4em}{{Method}} & \multicolumn{3}{|c|}{{Subject 377}} & \multicolumn{3}{|c|}{{Subject 392}} \\
% \cline{2-7}
%  & {LPIPS} $\times \mathbf{10^3}$ $\downarrow$ & {PSNR} $\uparrow$ & {SSIM} $\uparrow$ & {LPIPS} $\times \mathbf{10^3}$ $\downarrow$ & {PSNR} $\uparrow$ & {SSIM} $\uparrow$\\ 
% \hline
% Neural Body \cite{Peng2021NeuralBI} & 40.95 & 29.11 & 0.9674 & 53.27 & 30.10 & 0.9642 \\ 
% \hline
% HumanNeRF \cite{Weng2022HumanNeRFFR} & 24.06 & \textbf{30.41} & 0.9743 & 32.12 & 31.04 & 0.9705 \\ 
% \hline
% Ours & \textbf{23.58} & 30.39 & \textbf{0.9761} & \textbf{30.96} & \textbf{32.17} & \textbf{0.9769}\\ 
% \hline
% \end{tabular}
% \end{center}
% %---------------------------------------------------------------------------
% \begin{center}
% \begin{tabular}{ |c|c|c|c|c|c|c| } 
% \hline
% \multirow{2}{4em}{{Method}} & \multicolumn{3}{|c|}{{Subject 386}} & \multicolumn{3}{|c|}{{Subject 393}} \\
% \cline{2-7}
%  & {LPIPS} $\times \mathbf{10^3}$ $\downarrow$ & {PSNR} $\uparrow$ & {SSIM} $\uparrow$ & {LPIPS} $\times \mathbf{10^3}$ $\downarrow$ & {PSNR} $\uparrow$ & {SSIM} $\uparrow$\\ 
% \hline
% Neural Body \cite{Peng2021NeuralBI} & 46.43 & 30.54 & 0.9678 & 59.05 & 28.61 &	0.959 \\ 
% \hline
% HumanNeRF \cite{Weng2022HumanNeRFFR} & 28.99 & 33.20 & 0.9752 & 36.72 & 28.31 & 0.9603 \\ 
% \hline
% Ours & \textbf{27.29} & \textbf{34.59} & \textbf{0.9803} & \textbf{33.85} & \textbf{29.87} & \textbf{0.9711}\\ 
% \hline
% \end{tabular}
% \end{center}
% % \cellcolor{blue!10}
% %---------------------------------------------------------------------------
% \begin{center}
% \begin{tabular}{ |c|c|c|c|c|c|c| } 
% \hline
% \multirow{2}{4em}{{Method}} & \multicolumn{3}{|c|}{{Subject 387}} & \multicolumn{3}{|c|}{{Subject 394}} \\
% \cline{2-7}
%  & {LPIPS} $\times \mathbf{10^3}$ $\downarrow$ & {PSNR} $\uparrow$ & {SSIM} $\uparrow$ & {LPIPS} $\times \mathbf{10^3}$ $\downarrow$ & {PSNR} $\uparrow$ & {SSIM} $\uparrow$\\ 
% \hline
% Neural Body \cite{Peng2021NeuralBI} & 59.47 & 27.00 & 0.9518 & 54.55 & 29.1 & 0.9593 \\ 
% \hline
% HumanNeRF \cite{Weng2022HumanNeRFFR} & 35.58 & 28.18 & 0.9632 & 32.89 & 30.31 & 0.9642 \\ 
% \hline
% Ours & \textbf{32.49} & \textbf{29.57} & \textbf{0.9786} & \textbf{30.76} & \textbf{31.97} & \textbf{0.9709}\\ 
% \hline
% \end{tabular}

% \end{center}
%   \caption{Comparison of various performance metrics on the ZJU-MoCap dataset.}
%   \label{tab:mocap}
% \end{table*}

% %------------------------------------------------------------------------

% \begin{table*}[t]
% \begin{center}
% \begin{tabular}{ |c|c|c|c|c|c|c| } 
% \hline
% \multirow{2}{4em}{{Method}} & \multicolumn{3}{|c|}{{Subject 377}} & \multicolumn{3}{|c|}{{Subject 392}} \\
% \cline{2-7}
%  & {LPIPS} $\times 10^3$ $\downarrow$ & {PSNR} $\uparrow$ & {SSIM} $\uparrow$ & {LPIPS} $\times 10^3$ $\downarrow$ & {PSNR} $\uparrow$ & {SSIM} $\uparrow$\\ 
% \hline
% Neural Body \cite{Peng2021NeuralBI} & 40.95 & 29.11 & 0.9674 & 53.27 & 30.10 & 0.9642 \\ 
% \hline
% HumanNeRF \cite{Weng2022HumanNeRFFR} & 24.06 & \textbf{30.41} & 0.9743 & 32.12 & 31.04 & 0.9705 \\ 
% \hline
% Ours & \textbf{23.58} & 30.39 & \textbf{0.9761} & \textbf{30.96} & \textbf{32.17} & \textbf{0.9769}\\ 
% \hline
% \end{tabular}
% \end{center}
% %---------------------------------------------------------------------------
% \begin{center}
% \begin{tabular}{ |c|c|c|c|c|c|c| } 
% \hline
% \multirow{2}{4em}{{Method}} & \multicolumn{3}{|c|}{{Subject 386}} & \multicolumn{3}{|c|}{{Subject 393}} \\
% \cline{2-7}
%  & {LPIPS} $\times 10^3$ $\downarrow$ & {PSNR} $\uparrow$ & {SSIM} $\uparrow$ & {LPIPS} $\times 10^3$ $\downarrow$ & {PSNR} $\uparrow$ & {SSIM} $\uparrow$\\ 
% \hline
% Neural Body \cite{Peng2021NeuralBI} & 46.43 & 30.54 & 0.9678 & 59.05 & 28.61 &	0.959 \\ 
% \hline
% HumanNeRF \cite{Weng2022HumanNeRFFR} & 28.99 & 33.20 & 0.9752 & 36.72 & 28.31 & 0.9603 \\ 
% \hline
% Ours & \textbf{27.29} & \textbf{34.59} & \textbf{0.9803} & \textbf{33.85} & \textbf{29.87} & \textbf{0.9711}\\ 
% \hline
% \end{tabular}
% \end{center}
% % \cellcolor{blue!10}
% %---------------------------------------------------------------------------
% \begin{center}
% \begin{tabular}{ |c|c|c|c|c|c|c| } 
% \hline
% \multirow{2}{4em}{{Method}} & \multicolumn{3}{|c|}{{Subject 387}} & \multicolumn{3}{|c|}{{Subject 394}} \\
% \cline{2-7}
%  & {LPIPS} $\times 10^3$ $\downarrow$ & {PSNR} $\uparrow$ & {SSIM} $\uparrow$ & {LPIPS} $\times 10^3$ $\downarrow$ & {PSNR} $\uparrow$ & {SSIM} $\uparrow$\\ 
% \hline
% Neural Body \cite{Peng2021NeuralBI} & 59.47 & 27.00 & 0.9518 & 54.55 & 29.1 & 0.9593 \\ 
% \hline
% HumanNeRF \cite{Weng2022HumanNeRFFR} & 35.58 & 28.18 & 0.9632 & 32.89 & 30.31 & 0.9642 \\ 
% \hline
% Ours & \textbf{32.49} & \textbf{29.57} & \textbf{0.9786} & \textbf{30.76} & \textbf{31.97} & \textbf{0.9709}\\ 
% \hline
% \end{tabular}

% \end{center}
%   \caption{Comparison of various performance metrics on the ZJU-MoCap dataset.}
%   \label{tab:mocap}
% \end{table*}

\noindent\textbf{Temporal Consistency Loss (TCL):} We identify that imposing temporal consistency constraints can be valuable at two instances: a) while rendering consecutive training frames $\{\hat{I}^o\}_{t=-k}^{k}$ and b) while applying temporal deformation from consecutive training frames to the canonical frame $\{{\Delta x}_T\}_{t=-k}^{k}$. To this end, we employ the cycle-back regression consistency loss proposed in \cite{Dwibedi2019TemporalCL}. The cycle-back regression attempts to determine the temporal proximity of rendered frames or deformation vectors, and penalize the model if they are not in close temporal proximity. Given a rendered frame or a deformation vector $u$, and neighbors $\{v_k\}$, we compute the similarity vector $\beta_k$,
\begin{equation}
    \vspace{-1ex}
    \beta_k = \frac{\text{exp}(-\|u-v_k\|^2)}{\sum_j \text{exp}(-\|u-v_j\|^2)},
\end{equation}
where $u, v_k \in \{v_k\}$ and $\beta$ is a discrete distribution of similarities over time. We impose a Gaussian prior on $\beta$ by minimizing the normalized square distance, 
\begin{align}
    \mu = \sum_k k \beta_k \ \ \ \ \sigma^2 = \beta_k (k-\mu)^2
\end{align}
\vspace{-9mm}
\begin{align}
    \mathbb{L}_{TCL} = \frac{|i-\mu|^2}{\sigma^2} + \lambda log(\sigma),
\end{align}

where $\lambda$ is a regularization parameter. 
% Intuitively, it should peak around the index of $u$. 
Finally, the rendering loss
$\mathbb{L}_{rend}$, the canonical loss  $\mathbb{L}_{can}$, and the overall loss $\mathbb{L}$ are defined as 
\vspace{-3mm}
\begin{align}
\mathbb{L}_{rend} &= \mathbb{L}_{LPIPS}(\hat{I}^o,I^o) + \mathbb{L}_{MSE}(\hat{I}^o,I^o) \\
\nonumber &+ \mathbb{L}_{TCL}(\{\hat{I}^o\}_{t=-k}^{k})\\
\mathbb{L}_{can} &= \mathbb{L}_{MSE}(\hat{x}^c,x^c) + \mathbb{L}_{TCL}(\{{\Delta x}_T\}_{t=-k}^{k})\\
\mathbb{L} &= \mathbb{L}_{rend} + \mathbb{L}_{can} + \mathbb{L}_{CCL} + \mathbb{L}_{S}
\end{align}

% the combination of the LPIPS loss $\mathbb{L}_{LPIPS}$, MSE loss $\mathbb{L}_{MSE\_rend}$ between the ground truth frames $\hat{I}^o$ and predicted frames $I^o$, and the TCL $\mathbb{L}_{TCL\_rend}$ amongst the consecutive rendered frames $\{\hat{I}^o\}_{t=-k}^{k}$. Similarly, the canonical loss $\mathbb{L}_{can}$ can be defined as the combination of the MSE loss $\mathbb{L}_{MSE\_can}$ between the predicted canonical point positions $\hat{x}^c$ and $x^c$ and the TCL $\mathbb{L}_{TCL\_can}$ amongst the temporal deformations $\{{\Delta x}_T\}_{t=-k}^{k}$.

\subsection{Optimization Details}
\vspace{-1mm}
\noindent\textbf{Delayed Modular Optimization:} We follow a delayed-optimization approach similar to \cite{Weng2022HumanNeRFFR} to optimize the non-rigid motion, binary segmentation, and the refinement modules of our method. Optimizing these modules from the beginning yields lower performance as they rely on adequate inputs from the rest of the system. Hence, we freeze these modules initially, and unfreeze them gradually during the course of training.

\noindent\textbf{Ray Sampling:} Since LPIPS use a convolution-based approach to extract features, we use patch-based ray sampling following \cite{Weng2022HumanNeRFFR, Schwarz2020GRAFGR} instead of random ray sampling~\cite{Mildenhall2020NeRFRS} from the whole image. 
% We segregate each input image into a fixed number of patches of uniform pre-defined size, and use rays from a given patch for training. 

% \subsection{Optimization Details}
For this chapter, fix a prime $p$. We first discuss deformations of coalgebras from $\F_{p}$
to the $p$-adic integers and further to the $p$-completed sphere $\S_{p}^{\wedge}$ which leads
us to the question of how coalgebras behave with respect to $p$-completion. We introduce the
notion of a $p$-complete coalgebra and show that this is well behaved with respect to the
deformation theory discussed in the previous chapter. We then use this to iterate
Proposition~\ref{witt} and prove our main results, namely the existence of Witt Vectors
and spherical Witt Vectors for formally \'etale coalgebras. Then we specialize to the case
of homology coalgebras, show that for a finite space $X$ the coalgebra $\F_{p}[X]$ is formally
\'etale, and answer our initial question about the relation between $\S[X]^{\wedge}_{p}$
and $\F_{p}[X]$

\subsection{Coalgebras and $p$-completion}

We have seen that the functors that interest us are all \textit{nilcomplete}. For a nilcomplete
functor $X:\rm{CAlg}^{\rm{cn}} \to \cl{S}$ and a connective $\bb{E}_{\infty}$-ring $R$, we can construct
lifts from $X(\pi_{0}R)$ to $X(R)$ inductively along the Postnikov tower
\[ \dots \to \tau_{\leq2}R \to \tau_{\tau\leq 1}R \to \tau_{\leq0} R =\pi_{0}R.\]
This is however not quite enough to obtain our goal of lifting from $\F_{p}$ to the
$p$-completed sphere, we first need to pass to $\Z_{p}= \pi_{0}\S_{p}^{\wedge}$.
Explicitly, this means constructing lifts against the tower
\[\dots \to \Z/p^{3}\to \Z/p^{2}\to \Z/p\to \F_{p}\]
which is clearly presents a different problem. With the machinery developed thus far, we can already
prove the following for a general deformation problem.

\begin{proposition}\label{liftpgen}
  Let $X: \rm{CAlg}^{\rm{cn}} \to \cl{S}$ be a cohesive functor and $A\in X(\F_{p})$
  such that $T_{X_{A}}\simeq 0$. Then there exists a unique lift of $A$ to a point in
  $\flim_{n}X(\Z/p^{n})$.
\end{proposition}
\begin{proof}
  Set $A_{0}= A$, we inductively construct lifts against the tower of square zero extensions
  \[\dots \to \Z/p^{3} \to \Z/p^{2}\to \F_{p}.\]
  Suppose we have already constructed lifts $A_{k}$ for $k\le n$ for some $n$.
  Applying Proposition~\ref{bc} inductively, we get that
  \[T_{X_{A_{n}}}^{\F_{p}} \simeq T^{\F_{p}}_{X_{A_{0}}} \simeq 0.\]
  Thus, since $\Z/p^{n+1}\to \Z/p^{n}$ is a square zero extension with fiber $\F_{p}$,
  Proposition~\ref{deformations} implies that the fiber
  \[X_{A_{n}}^{\Z/p^{n+1}}=\rm{fib}_{A_{n}}(X(\Z/p^{n+1})\to \Z/p^{n})\]
  is contractible and we find an essentially unique lift $A_{n+1}$. This proves the claim.
\end{proof}
 Of course, for an arbitrary functor $X:\rm{CAlg}^{\rm{cn}} \to \cl{S}$ the natural map
$X\to \flim_{n}X(\Z/p^{n})$ might not be an equivalence, meaning that in this generality
we can only construct pro-$p$ objects of $X$ using this inductive method.
In fact, we have that $\rm{cCAlg}_{\Z_{p}}\neq  \flim_{n} \rm{cCAlg}_{\Z/p^{n}}$. To remedy
this problem we show that this limit admits a description via \textit{$p$-complete} coalgebras.
To do this, we first recall some facts about $p$-complete modules.

\begin{definition}
Let $R$ be an $\bb{E}_{\infty}$-ring, then $M \in \rm{Mod}_{R}$ is called
$p$-\textit{complete} if the limit
\[ \lim \left(\dots \rar{\cdot p} M \rar{\cdot p}M \right)\]
vanishes. We denote the full subcategory spanned by the $p$-complete modules by $(\rm{Mod}_{R})_{p}^{\wedge}$.
\end{definition}

\begin{remark}
The inclusion $(\rm{Mod}_{R})_{p}^{\wedge} \rari{} \rm{Mod_{R}}$ admits a left adjoint which takes a module $M$
to its \textit{$p$-completion} given by the limit
\[ \lim \left( \dots \to M/p^{2} \to M/p \right).\]
In fact, $M$ is $p$-complete if and only if the natural map $M \to \lim M/p^{n}$ is an equivalence.
This inherits a natural $R^{\wedge}_{p}$-module structure, thus $p$-completion also gives
an equivalence of categories $(\rm{Mod}_{R})^{\wedge}_{p} \simeq (\rm{Mod}_{R^{\wedge}_{p}})^{\wedge}_{p}$ which
allows us to identify these in what follows.\\
The tensor product of $p$-complete modules is in general not $p$-complete. However, the
category $(\rm{Mod}_{R})_{p}^{\wedge}$ admits a symmetric monoidal structure given by the formula
 \[ M \otimes_{(\rm{Mod}_{R})_{p}^{\wedge}} N := ( M \otimes N )^{\wedge}_{p}.\]
 With this monoidal structure the $p$-completion functor $\rm{Mod}_{R}\to (\rm{Mod}_{R})_{p}^{\wedge}$
 is strong monoidal, while the inclusion is only lax monoidal.
\end{remark}

 \begin{definition}
   Let $R$ be an $\bb{E}_{\infty}$-ring. We define the $\infty$-category of $p$-complete
   $R$-coalgebras is given by.
   \[ {(\rm{cCAlg}_{R})}^{\wedge}_{p}:= \rm{cCAlg}({(\rm{Mod}_{R})}^{\wedge}_{p}).\]
 \end{definition}

 \begin{warning}
   Let $R$ be a $\bb{E}_{\infty}$-ring. Notice that by our definition a $p$-complete $R$-coalgebra
   is the same as a $p$-complete $R^{\wedge}_{p}$-coalgebra and so we do not differentiate between
   the two notions.
   However, this is \textit{not} the same as an $R^{\wedge}_{p}$-coalgebra whose underlying
   spectrum is $p$-complete. The process of $p$-completion does refine to a functor
   $\rm{cCAlg}_{R} \to (\rm{cCAlg}_{R^{\wedge}_{p}})^{\wedge}_{p}$,
   but it does not factor through the category $\rm{cCAlg}_{R^{\wedge}_{p}}$.
 \end{warning}

 We now show check that the assignment $R \mapsto \rm{cCAlg}_{R}^{\rm{cn}}$ is subject to the machinery
 of deformation theory.

 \begin{lemma}\label{conil2}
   The following statements hold:
   \begin{enumerate}
     \item   Suppose we have a pullback diagram of connective $\bb{E}_{\infty}$-rings
   \[\begin{tikzcd}
	R\p & S\p \\
	R & S
	\arrow[from=1-1, to=2-1]
	\arrow[from=2-1, to=2-2]
	\arrow[from=1-2, to=2-2]
	\arrow[from=1-1, to=1-2]
\end{tikzcd}\]
such that the map $\pi_{0}R \to \pi_{0}S$ is surjective. Then the natural map
\[ (\rm{cCAlg}_{R\p}^{\rm{cn}})^{\wedge}_{p} \to (\rm{cCAlg}_{R}^{\rm{cn}})^{\wedge}_{p}\times_{(\rm{cCAlg}_{S}^{\rm{cn}})^{\wedge}_{p}} (\rm{cCAlg}_{S\p}^{\rm{cn}})^{\wedge}_{p}\]
is an equivalence.
     \item For every connective $\bb{E}_{\infty}$-ring $R$, the natural map
           \[ (\rm{cCAlg}_{R}^{\rm{cn}})^{\wedge}_{p} \to\flim_{n} (\rm{cCAlg}_{\tau_{\le n}R}^{\rm{cn}})^{\wedge}_{p}\]
           is an equivalence.
   \end{enumerate}
 \end{lemma}
 \begin{proof}
   Ad 1.: Arguing as in the proof of Proposition~\ref{Mod}, it suffices to show that the
   strong monoidal functor
   \begin{align*}
    (\rm{Mod}_{R\p})^{\wedge}_{p} \to (\rm{Mod}_{R})^{\wedge}_{p}\times_{(\rm{Mod}_{S})^{\wedge}_{p}} (\rm{Mod}_{S\p})^{\wedge}_{p}
   \end{align*}
   is an equivalence. Indeed, given a point $(M,N,h)$ in the pullback, the $R\p$-module $M \times_{M \otimes_{R} S}N$
   is again $p$-complete since $p$-completion commutes with limits. Thus, the inverse functor of
   Proposition~\ref{Mod} also induces a functor on the categories of $p$-complete modules. Moreover,
   we have that
   \[ ((M\times_{M\otimes_{R}S}N)\otimes_{R\p} R)^{\wedge}_{p} \simeq M^{\wedge}_{p} \simeq M\]
   \[ ((M \times_{M\otimes_{R}}N)\otimes_{R\p}S\p)^{\wedge}_{p}\simeq N^{\wedge}_{p} \simeq N,\]
   where the first equivalences hold by Proposition~\ref{Mod}, and the latter since $M$ and $N$ are
   to be $p$-complete. Finally, for $M\in (\rm{Mod}_{R\p})^{\wedge}_{p}$, we compute that
   \[ (M \otimes_{R\p} R)^{\wedge}_{p}\times_{(M \otimes_{R\p} S)^{\wedge}_{p}}(M \otimes_{R\p}S\p)^{\wedge}_{p}
     \simeq \left( M \otimes_{R\p} R \times_{M\otimes_{R\p} S} M \otimes_{R\p} S\p\right)^{\wedge}_{p}
   \simeq M^{\wedge}_{p} \simeq M,\]
 where we have again used the result of Proposition~\ref{Mod} and the fact that $p$-completion commutes
 with limits.\\
 Ad 2: This uses the exact same arguments applied to the equivalence of Corollary~\ref{nilcomplete}.
 \end{proof}

 \begin{corollary}
   For any $n\in \bb{N}$, the functor
   \[ \rm{CAlg}^{\rm{cn}} \to \cl{S} \qquad R \mapsto [(\rm{cCAlg}_{R}^{\rm{cn}})^{\wedge}_{p}]^{\Delta^{n}}\]
   is coherent and nilcomplete.
 \end{corollary}

 We now prove the crucial $p$-completeness result for $\Z_{p}$-modules. As before
 this will enable us to deduce the same result for coalgebras and allow us to tackle the
 actual problem of comparing coalgebras over $\F_{p}$, $\Z_{p}$ and $\S_{p}^{\wedge}$.
\begin{proposition}\label{pcomp}
  Let $\rm{Mod}^{\wedge}_{\Z_p} \subseteq \rm{Mod}_{\Z_{p}}$ denote the full subcategory spanned by the
  $p$-complete $\Z_{p}$-module spectra. Then the natural map
  \[ \rm{Mod}_{\Z_{p}} \to \flim_{n} \rm{Mod}_{\Z/p^{n}} \quad N \mapsto (N\otimes_{\Z_{p}}\Z/p^{n})\]
  restricts to a strong monoidal equivalence
  \[(\rm{Mod}_{\Z_{p}})^{\wedge}_{p} \simeq \flim_{n}\rm{Mod}_{\Z/p^{n}}. \]
\end{proposition}
\begin{proof}
  The functor admits a right adjoint which takes $(M_{n})\in \flim_{n}\rm{Mod}_{\Z/p^{n}}$ to the limit
  $\lim_{n}M_{n}$ taken in the category of $\Z_{p}$-modules. Since $p$-complete modules are closed under
  limits, the essential image of this functor is contained in $\rm{Mod}_{\Z_{p}}^{\wedge}$. Moreover,
  if $M\in \rm{Mod}_{\Z_{p}}^{\wedge}$, then we have that
  \[ \flim_{n}(M \otimes_{\Z_{p}} \Z/p^{n}) \simeq \flim_{n} M/p^{n} \simeq M^{\wedge}_{p}\simeq M.\]
  Hence, the counit of the adjunction is an equivalence on $p$-complete modules.
  Conversely, given $(N_{k})\in \flim_{k}\rm{Mod}_{\Z/p^{k}}$ write $N= \lim_{k}N$. We want
  to show that, for every $n$ the natural map
  \[ N \otimes_{\Z_{p}} \Z/p^{n}\rar{\sim}N_{n}\]
  is an equivalence. Since $N \otimes_{\Z_{p}}Z/p^{n}\simeq N/p^{n}$ and limits are exact, we have an equivalence
  \[N \otimes_{\Z_{p}}\Z/p^{n}\simeq \lim_{k >n}(N_{k}\otimes_{\Z_{p}}\Z/p^{n}).\]
  Thus, the unit of the adjunction may be written as
  \[ \lim_{k>n}(N_{k} \otimes_{\Z_{p}}\Z/p^{n}) \to \lim_{k>n}(N_{k}\otimes_{\Z/p^{k}}\Z/p^{n})\simeq N_{n}\]
  and so has fiber given by
  \[ F_{n}:=\lim_{k>n}\left(N_{k}\otimes_{\Z/p^{k}}\rm{fib}(\Z/p^{k}\otimes_{\Z_{p}}\Z/p^{n}\to \Z/p^{n}) \right).\]
  Now we compute the fiber of $\Z/p^{k}\otimes_{\Z_{p}}\Z/p^{n}\to \Z/p^{n}$ as the module
  \[ \rm{Tor}^{\Z_{p}}(\Z/p^{k}, \Z/p^{n})[1]\simeq \Z/p^{n}[1].\]
  The reduction map $\Z/p^{k}\to \Z/p^{k-1}$ is induced by the map of projective resolutions
\[\begin{tikzcd}
	{\Z_p} & {\Z_p} \\
	{\Z_p} & {\Z_p}
	\arrow["{\cdot p^k}", from=1-1, to=1-2]
	\arrow["\id", from=1-2, to=2-2]
	\arrow["{\cdot p}"', from=1-1, to=2-1]
	\arrow["{\cdot p^{k-1}}"', from=2-1, to=2-2],
\end{tikzcd}\]
hence, on Tor it induces the multiplication by $p$ map
\[ \Z/p^{n}=\rm{Tor}^{\Z_{p}}(\Z/p^{k}, \Z/p^{n})\rar{\cdot p} \rm{Tor}^{\Z_{p}}(\Z/p^{k-1}, \Z/p^{n}) =\Z/p^{n}.\]
Thus, if we have $k\p > k > n$ such that $k\p -k > n$, the transition map
\[ F_{k\p}=N_{k\p} \otimes \rm{Tor}^{\Z_{p}}(\Z/p^{k}, \Z/p^{n})\to N_{k} \otimes \rm{Tor}^{\Z_{p}}(\Z/p^{k-1}, \Z/p^{n})= F_{k}\]
vanishes since the Tor-groups are $p^{n}$-torsion. Choosing a cofinal subset $S\subseteq \bb{N}_{>n}$ such that
$\abs{k\p -k}> n$ for any distinct $k\p,k\in S$, we see that
\[ \lim_{k>n} F_{k}\simeq \lim_{k\in S} F_{k} \simeq 0 \]
vanishes. Thus, since limits are exact, the map $N \otimes_{\Z_{p}} \Z/p^{n}\rar{\sim}N_{n}$ is an equivalence.\\
To see that the functor $\rm{Mod}_{\Z_{p}}^{\wedge} \to \flim_n \rm{Mod}_{\Z/p^{n}}$ is strong monoidal,
we observe that since cofibers and limits are exact, we have for each $n$ equivalences
\begin{align*}
  (M \otimes_{\Z_{p}} N)^{\wedge}_{p} \otimes_{\Z_{p}}\Z/p^{n} &\simeq \lim_{k}(M/p^{k} \otimes_{\Z_{p}}N/p^{k})/p^{n}\\
                                              &\simeq \lim_{k}\left((M/p^{n} \otimes_{\Z_{p}} N/p^{n})\otimes_{Z_{p}}\Z/p^{k}\right) \\
  &\simeq ((N\otimes_{\Z_{p}}\Z/p^{n}) \otimes_{\Z_{p}} (M \otimes_{\Z_{p}}\Z/p^{n}))^{\wedge}_{p}.
\end{align*}
This proves the claim.
\end{proof}

\begin{corollary}\label{pcomp1}
  We have an equivalence of categories
  \[ (\rm{cCAlg}_{\Z_{p}})_{p}^{\wedge} \rar{\sim} \flim_{n} \rm{cCAlg}_{\Z/p^{n}} \quad A \mapsto (A\otimes_{\Z_{p}}\Z/p^{n})\]
  with inverse taking a system of coalgebras $(B_{n})$ to the limit $\lim_{n}B_{n}$ taken in the
  category of ($p$-complete) $\Z_{p}$-modules, equipped with the induced $p$-complete
  $\Z_{p}$-coalgebra structure.
\end{corollary}
\begin{proof}
This follows from Proposition~\ref{pcomp}, arguing as in the proof of Proposition~\ref{Mod}.
\end{proof}

\begin{corollary}\label{obliftzp}
  Let $X(\blank)= (\rm{cCAlg}_{\blank}^{\rm{cn}})^{\Delta^{0}}$ and $A\in X(\F_{p})$ such that $T_{X_{A}}\simeq 0$.
  Then the space of lifts of $A$ to a $p$-complete $\Z_{p}$-coalgebra is contractible
\end{corollary}
 \begin{proof}
 Combine Proposition~\ref{liftpgen} and Corollary~\ref{pcomp1}.
 \end{proof}

\begin{corollary}\label{mapliftzp}
  Let $\varphi: B\to A$ be a map of connective, formally \'etale $\F_{p}$-coalgebras. Then the space of
  lifts of $\varphi$ to a map of $p$-complete $\Z_{p}$-coalgebras $B\p \to A\p$ is contractible.
\end{corollary}
\begin{proof}
    Let $ \cl{X}(\blank)=\rm{cCAlg}_{\blank}^{\rm{cn}}$. By Proposition~\ref{etalchar} the natural map
    \[ T_{\cl{X}^{\Delta^{1}}_{\varphi}} \to T_{\cl{X}^{\Delta^{0}}_{B}}\]
    is an equivalence, but since $B$ is formally \'etale we have $T_{\cl{X}^{\Delta^{0}}_{B}} \simeq 0$.
    Hence, the claim follows by applying Proposition~\ref{liftpgen} to the functor $\cl{X}^{\Delta^{1}}$
    and using Corollary~\ref{pcomp1}.
\end{proof}

Having shown this, we can now construct a functor which is analogous to the classical
Witt-Vectors, which allow us to pass from \'etale $\F_{p}$-algebras to $\Z_{p}$-algebras.

\begin{theorem}
  Let $\cl{C}\subseteq (\rm{cCAlg}_{\Z_{p}}^{\rm{cn}})^{\wedge}_{p}$ denote the full subcategory spanned by those
  coalgebras $A$ for which $A\otimes_{\Z_{p}} \F_{p}$ is formally \'etale. Then the base change functor
  \[ \cl{C} \to \rm{cCAlg}_{\F_{p}}^{\rm{cn}, \rm{f\acute{e}t}}  \qquad A \mapsto A\otimes_{\Z_{p}}\F_{p}\]
  is fully faithful and essentially surjective. In particular, the quasi inverse defines a functor
  \[ W_{p}: \rm{cCAlg}_{\F_{p}}^{\rm{cn,f\acute{e}t}} \to (\rm{cCAlg}_{\Z_{p}}^{\rm{cn}})^{\wedge}_{p}\]
  which is fully faithful and satisfies $W_{p}(A)\otimes_{\Z_{p}}\F_{p} \simeq A$ for every connective, formally
  \'etale $\F_{p}$-coalgebra $A$.
\end{theorem}

\begin{proof}
  Combine Corollary~\ref{obliftzp} and Corollary~\ref{mapliftzp}.
\end{proof}

We now turn our attention to the leap from $\Z_{p}$ to $\S_{p}^{\wedge}$. The following proposition shows that,
for an arbitrary cohesive and nilcomplete functor, a $\Z_{p}$-valued point which has vanishing $\F_{p}$-tangent
complex admits a unique lift to a $\S_{p}^{\wedge}$-valued point. This is surprising, as we do not
actually require any information about the $\Z_{p}$-tangent complex, everything is determined by
what happens modulo $p$.

\begin{proposition}\label{spherelift}
  Let $X: \rm{CAlg}^{\rm{cn}} \to \cl{S}$ be a cohesive and nilcomplete functor and let $A \in X(\Z_{p})$
  such that $T_{X_{A\otimes_{\Z_{p}}\F_{p}}}\simeq 0$. Then $A$ admits an essentially unique lift to $X(\S_{p}^{\wedge})$.
\end{proposition}

\begin{proof}
  We inductively construct lifts against the Postnikov Tower
  \[ \dots \to \tau_{\leq2} \S_{p}^{\wedge}  \to \tau_{\leq 1} \S_{p}^{\wedge} \to \tau_{\leq 0} \S_{p}^{\wedge} \simeq \Z_{p}. \]
  Write $A=A_{0},~S_{n}= \tau_{\leq n}\S_{p}^{\wedge},~ M_{n} = \pi_{n}S_{n}$ and assume we have already constructed
  a unique lift $A_{n}$ to $X(S_{n})$. Consider the square zero extension
  \[ M_{n+1}[n+1] \to S_{n+1}\to S_{n}.\]
  Since $M_{n+1} = \pi_{n+1}S_{n+1}$ is concentrated in a single degree, the $S_{n}$-action factors
  through $S_{0}=\Z_{p}$. Moreover, since $\pi_{n+1}S_{n+1}$ is of finite $p$-torsion, the action
  further factors through $\Z/p^{k}$ for some $k\geq 0$. Thus, Proposition~\ref{bc} implies that
  we have an equivalence
  \[ T_{X_{A_{n}}}^{M_{n+1}[n+1]} \simeq \Sigma^{n}T_{X_{A_{n}}}^{M_{n+1}} \simeq T_{X_{A_{n} \otimes_{S_{n}} \Z/p^{k}}}^{M_{n+1}}.\]
  Arguing as in Proposition~\ref{cofib} with respect to the square zero extension
  \[ \F_{p} \to \Z/p^{k}\to \Z/p^{k-1},\]
  we see that we have a cofiber sequence
  \[  T^{M_{n+1}\otimes_{\Z/p^{k}}\F_{p}}_{X_{A_{n} \otimes_{S_{n}} \Z/p^{k-1}}}
    \to T_{X_{A_{n} \otimes_{S_{n}} \Z/p^{k}}}^{M_{n+1}}
    \to T^{M_{n+1}\otimes_{\Z/p^{k}}\Z/p^{{k-1}}}_{X_{A_{n} \otimes_{S_{n}} \Z/p^{k-1}}}.\]
  For the left hand term, Proposition~\ref{bc} gives the equivalence
  \[ T_{X_{A_{n}\otimes_{S_{n}}\Z/p^{k-1}}}^{M_{n+1}\otimes_{\Z/p^{k}}\F_{p}}
    \simeq T_{X_{A \otimes_{\Z_{p}}\F_{p}}}^{{M_{n+1}\otimes_{\Z/p^{k}}\F_{p}}}
    \simeq T_{X_{A\otimes_{\Z_{p}}\F_{p}}}\otimes_{\F_{p}}( M_{n+1}\otimes_{\Z/p^{k}}\F_{p} ) \simeq 0,\]
  where we have used that, since $M_{n+1}$ is finitely generated, the $\F_{p}$-module
  $M_{n+1}\otimes_{\Z/p^{k}}\F_{p}$ is perfect. For the right hand term we
  replace $M_{n+1}$ with $M_{n+1} \otimes_{\Z/p^{k}}\Z/p^{k-1}$ and repeat the argument,
  inductively yielding equivalences
  \[ T^{M_{n+1}}_{X_{A_{n}\otimes_{S_{n}}\Z/p^{k}}}
    \simeq T^{M_{n+1}\otimes_{\Z/p^{k}}\Z/p^{{k-1}}}_{X_{A_{n-1} \otimes_{S_{n-1}} \Z/p^{k-1}}}
  \simeq \cdots \simeq T^{M_{n+1}\otimes_{\Z/p^{k}} \F_{p}}_{X_{A \otimes_{\Z_{p}}\F_{p}}} \simeq 0.\]
In total, this shows that $T_{X_{A_{n}}}^{M_{n+1}[n+1]} \simeq 0$, and hence $A_{n}$ admits an essentially
unique lift to $X(S_{n+1})$. Thus, the fiber over $A$ of the map
\[ X(\S_{p}^{\wedge})\simeq \flim_{n}X(S_{n})\to X( \Z_{p})\]
is contractible and we are done.
  \end{proof}

  \begin{lemma}\label{pcomparison}
    Write $\cl{X}(\blank)=\rm{cCAlg}^{\rm{cn}}_{\blank}$ and $\cl{Y}(\blank)=
    (\rm{cCAlg}^{\rm{cn}}_{\blank})^{\wedge}_{p}$. Then the $p$-completion map $f:\cl{X}\to \cl{X}\p$
    induces an equivalence
    \[ T^{M}_{(\cl{X}^{\Delta^{n}})_{\xi}} \to  T^{M}_{(\cl{Y}^{\Delta^{n}})_{f(\xi)}}\]
        for every $\F_{p}$-module $M$, $n\in \bb{N}$ and $\xi \in \cl{X}(\F_{p})^{\Delta^{n}}$.
  \end{lemma}
  \begin{proof}
    For any $\F_{p}$-algebra $R$ the $p$-completion map gives an equivalence
    $\rm{Mod}_{R}\rar{\sim} (\rm{Mod}_{R})^{\wedge}_{p}$, since multiplication by some power of $p$
    is nullhomotopic over $\F_{p}$. In particular, this applies to the split square zero
    extension $\F_{p}\oplus M$ for any $M \in \rm{Mod}_{\F_{p}}$ and so the natural map
    $\cl{X}(\F_{p}\oplus M) \to \cl{Y}(\F_{p}\oplus M)$ is an equivalence as well.
    Consequently, we also obtain natural equivalences between the fibers
    \[ (\cl{X}^{\Delta^{n}})_{\xi}^{\F_{p}\oplus M} \to  (\cl{Y}^{\Delta^{n}})_{f(\xi_)}^{\F_{p}\oplus M},\]
    which induces the equivalence of spectra
    \[ T^{M}_{(\cl{X}^{\Delta^{n}})_{\xi}} \to  T^{M}_{(\cl{Y}^{\Delta^{n}})_{f(\xi)}}\]
      as claimed.
  \end{proof}

  \begin{corollary}\label{obliftsp}
    Let $X(\blank)=(\rm{cCAlg}^{\rm{cn}}_{\blank})^{\Delta^{0}}$ and $A \in X(\F_{p})$ such that
    $T_{X_{A}}\simeq 0$, then the space of lifts of $A$ to a $p$-complete $\S_{p}^{\wedge}$-coalgebra
    is contractible.
  \end{corollary}

  \begin{proof}
    Write $Y(\blank)= ((\rm{cCAlg}^{\rm{cn}}_{\blank})^{\wedge}_{p})^{\Delta^{0}}$. Then by Lemma~\ref{pcomparison}
    we have an equivalence $T_{X_{A}}\simeq T_{Y_{A}} \simeq 0$. Hence, we can apply Proposition~\ref{obliftzp} to
    obtain an essentially unique lift $A\p\in Y(Z_{p})$. Further applying Proposition~\ref{spherelift}
    to $A\p$ yields our claim.
  \end{proof}
  Thus, we can pointwise lift $\F_{p}$-coalgebras with vanishing tangent complex to $\S_{p}^{\wedge}$. If
  we moreover consider \textit{formally \'etale coalgebras}, we can make this lifting functorial
  in a coalgebraic analogue of the \textit{Spherical Witt Vectors} construction for
  $\bb{E}_{\infty}$-algebras over $\F_{p}$.

\begin{corollary}\label{mapliftsp}
  Let $\varphi:B\to A$ be a map of $\F_{p}$-coalgebras such that $A$ and $B$ are formally \'etale.
  Then the space of lifts of $\varphi$ to a map $\varphi\p: B\p \to A\p$ of $p$-complete
  $\S_{p}^{\wedge}$-coalgebras is contractible.
\end{corollary}

\begin{proof}
  Let $ \cl{X}(\blank)=\rm{cCAlg}_{\blank}^{\rm{cn}}$ and $\cl{Y}(\blank) =
  (\rm{cCAlg}_{\blank}^{\rm{cn}})^{\wedge}_{p}$. By Proposition~\ref{mapliftzp} the map $\varphi$ admits
  an essentially unique lift to a point $\psi \in \cl{Y}(\Z_{p})^{\Delta^{1}}$. Moreover, Lemma~\ref{pcomparison}
  yields an equivalence $T_{\cl{X}^{\Delta^{1}}_{\varphi}}\simeq T_{\cl{Y}^{\Delta^{1}}_{\varphi}}$. Since both $A$ and $B$ are
  formally \'etale Proposition~\ref{etalchar} gives equivalences
  \[ T_{\cl{X}^{\Delta^{1}}_{\varphi}} \rar{\sim} T_{\cl{X}^{\Delta^{0}}_{B}} \simeq 0\]
  Hence, we can apply Proposition~\ref{spherelift} to the functor $\cl{Y}^{\Delta^{1}}$ and the point
  $\psi \in \cl{Y}^{\Delta^{1}}$, proving the claim.
\end{proof}

\begin{theorem}\label{wittsp}
  Denote by $\cl{C}\subseteq (\rm{cCAlg}_{\S_{p}^{\wedge}}^{\rm{cn}})^{\wedge}_{p} $ the full subcategory spanned by those
  coalgebras $A$ such that $A\otimes_{\S_{p}^{\wedge}}\F_{p}$ is formally \'etale. Then the base change functor
  \[ \cl{C} \to \rm{cCAlg}_{\F_{p}}^{\rm{cn}, \rm{f\acute{e}t}} \qquad A \mapsto A \otimes_{\S_{p}^{\wedge}} \F_{p}\]
  is fully faithful and essentially surjective.
\end{theorem}
\begin{proof}
  Combine Corollary~\ref{obliftsp} and Corollary~\ref{mapliftsp}.
\end{proof}

\begin{remark}
  In the setting of Theorem~\ref{wittsp} the quasi-inverse to $\blank \otimes_{\S^{\wedge}_{p}}\F_{p}$ defines
  a fully faithful functor
  \[ W_{\S_{p}^{\wedge}}: \rm{cCAlg}_{\F_{p}}^{\rm{cn}, \rm{f\acute{e}t}}
    \to (\rm{cCAlg}_{\S_{p}^{\wedge}}^{\rm{cn}})^{\wedge}_{p}\]
  which satisfies $W_{\S_{p}^{\wedge}}(A)\otimes_{\S^{\wedge}_{p}}\F_{p} \simeq A$ for every connective, formally \'etale
  $\F_{p}$-coalgebra $A$. We call $W_{\S_{p}^{\wedge}}(A)$ the \textit{spherical Witt vectors} of $A$.
\end{remark}


\subsection{Homology coalgebras}

As observed in Example~\ref{homology}, for every space $X$ and every $\bb{E}_{\infty}$-ring $R$, the
$R$-homology $R[X]$ carries a natural $R$-coalgebra structure, which is a stronger invariant than its
underlying $R$-module. We now want to apply our results and see what can be said about the deformation
theoretic behavior of homology coalgebras. To do this, we first need to compute the cotangent complex of the
$\F_{p}$-cohomology.

\begin{definition}
  A space $X\in \cl{S}$ is called $p$-finite if the following conditions hold:
  \begin{enumerate}
    \item The space $X$ is truncated.
    \item The set $\pi_{0}X$ is finite.
    \item For each $n\geq 1$ and $x\in X$, we have that $\pi_{n}(X,x)$ is a finite $p$-group.
  \end{enumerate}
  We denote the full subcategory of $\cl{S}$ spanned by the $p$-finite spaces as $\cl{S}_{p}$ and call
 $\cl{S}^{\vee}_{p} =: \rm{Pro}(\cl{S}_{p})$ the category of $p$-\textit{profinite} spaces.
\end{definition}

\begin{remark}
We can regard $\cl{S}_{p}^{\vee}$ as the category of ``formal limits'' of $p$-finite spaces $\varprojlim X_{\alpha}$.
As such there is a functor $\cl{S}^{\vee}_{p}\to \cl{S}$ which takes a formal limit to the actual limit in $\cl{S}$.
This functor admits a left adjoint given by $Y \mapsto \flim_{Y_{\alpha} \to Y} Y_{\alpha}$, where the limit runs over all maps
from a $p$-finite space $Y_{\alpha}$ to $Y$.
\end{remark}

\begin{lemma}
  Let $X$ be a space and $\flim X_{\alpha}$ be its $p$-profinite completion. Then the natural map
  of cohomology rings
  \[ \fcolim \F_{p}^{X_{\alpha}} \to \F_{p}^{X} \]
  is an equivalence.
\end{lemma}
\begin{proof}
  This is immediate since the Eilenberg-MacLane spaces $K(\F_{p},n)$ are $p$-finite.
\end{proof}

\begin{proposition}[Mandell, Lurie]\label{coetal}
  Let $X$ be a space, then the $\F_{p}$-cohomology $\F_{p}^{X}$ is a formally \'etale $\F_{p}$-algebra.
\end{proposition}
\begin{proof}
  Since the functor $R \mapsto L_{R/\F_{p}}$ commutes with colimits, the claim follows from the fact that
  $L_{\F_{p}^{X}/\F_{p}}\simeq 0$ for every $p$-finite space $X$ which is proven
  in~\cite[][Proposition 2.4.12]{dag8}.
\end{proof}

Thus we obtain the following result about the homology coalgebra of a finite space $X$
with coefficients in a connective $\F_{p}$-algebra $R$:

\begin{corollary}\label{goal}
  Let $X$ be a finite space and $R$ be an $\F_{p}$-algebra, then $R[X]$ is a formally
  \'etale $R$-coalgebra.
\end{corollary}
\begin{proof}
  From Proposition~\ref{coetal} we get that
  \[ L_{R^{X}/R}\simeq L_{\F_{p}^{X}/\F_{p}}\otimes_{\F_{p}}R \simeq 0.\]
  Since $X$ is finite, the coalgebra $R[X]$ is dualizable with dual given by $R^{X}$, so the claim
  follows from Proposition~\ref{dualetal}.
\end{proof}

Moreover, for the case $R=\F_{p}$, we can use Theorem~\ref{wittsp} to give a partial answer to our
initial question about lifts of the coalgebra $\F_{p}[X]$.

\begin{corollary}
  Let $X$ be a finite space, then $\F_{p}[X]$ admits a unique lift to a $p$-complete $\S_{p}^{\wedge}$-coalgebra
  given by $W_{\S_{p}^{\wedge}}(\F_{p}[X]) \simeq (\S[X])^{\wedge}_{p}$. Moreover, for any other finite space $Y$
  the natural map
  \[\rm{Map}_{(\rm{cCAlg}_{\S_{p}^{\wedge}})^{\wedge}_{p}}((\S[Y])^{\wedge}_{p}, (\S[X])^{\wedge}_{p})
    \to \rm{Map}_{\rm{cCAlg}_{\F_{p}}}(\F_{p}[Y], \F_{p}[X])\]
  is a homotopy equivalence.
\end{corollary}
\begin{proof}
 Combine Corollary~\ref{goal} and Theorem~\ref{wittsp}.
\end{proof}

\section{Where to go from here}

We finish our discussion by explaining some of the shortcomings of our results and sketch a possible
way to proceed towards a coalgebraic analogue of Mandell's Theorem. The first missing puzzle piece is
the cotangent complex of a coalgebra $A$, which we have been unable to give a solid definition of.
The second and more important one is the relation to the \textit{coalgebra Frobenius}. We conjecture
that the class of \textit{perfect} coalgebras defined via this map give examples of non-dualizable
formally \'etale coalgebras. In particular, this conjecture would imply that the $\F_{p}$-homology
of \textit{any} space $X$ is formally \'etale.

\subsection{The cotangent complex of a coalgebra}
One of the first questions that arose during this project turned out to be one of the most subtle and
tricky ones, namely:

\begin{question}
  What is the cotangent complex of a coalgebra $A$?
\end{question}

Clearly, the existence of a single spectrum controlling the deformation theory of $A$ would be immensely
useful. However, it is not immediately clear what the universal property of such a spectrum should be,
i.e.~which space of derivations it should (co)represent.
Some inspiration can be gleamed from Proposition~\ref{cotangentder}. There we had seen that, for
$\varphi: B \to A$ a map of $R$-coalgebras with $A$ dualizable and $M$ an $R$-module, we have an equivalence
\[ \rm{Der}_{\varphi}(B, C_{A}(M)) \simeq \rm{Map}_{A^{\vee}}(L_{A^{\vee}/R}, \varphi^{\vee}_{\pt}\rm{map}_{R}(B, M)).\]
To get rid of the dependence on the second coalgebra $B$ one is tempted to take $B=R$ such that
$\rm{map}_{R}(B,M)\simeq M$. However, not every coalgebra $A$ admits a map $R\to A$, much less a canonical
one. The only natural choice for a map that is not the initial map would yield the following:

\begin{definition}[Preliminary 1.]
  Let $R$ be an $\bb{E}_{\infty}$-ring and $A\in \rm{cCAlg}_{R}$. The cotangent complex of $A$, if it exists,
  is the $R$-module $L_{A}$ corepresenting the functor
  \[ \rm{Mod}_{R}\to \rm{Mod}_{R} \qquad M \mapsto \rm{der}_{\id}(A, C_{A}(M))\]
\end{definition}

There are however several problems with this. Firstly, it is entirely unclear from the definition
whether $L_{A}$ vanishing would actually imply $A$ being formally \'etale. Moreover, in the dualizable
case it would lead to the rather awkward formula
\[ L_{A} \simeq L_{A^{\vee}/R}\otimes_{A^{\vee}}A.\]
Although somewhat plausible, this again gives us little information about what can actually be
deduced in the case that $L_{A}\simeq 0$.
This leaves us with several options, lest we accept that there is no good notion of one singular
cotangent complex. For one we could work with \textit{coaugmented} coalgebras, namely coalgebras
together with a map $R \to A$. For the purpose of understanding homology coalgebras this would correspond
to considering pointed spaces instead of just spaces, an entirely acceptable compromise, but beyond the
scope of this paper. \\
A different  approach would be to give up on the idea of corepresentability
and instead hope for a colimit preserving functor. For example, the functor
\[ \rm{Mod}_{R}\to \rm{Mod}_{R} \qquad M \mapsto C_{A}(M):=\rm{cofib}( A \rar{\eps} \Omega^{\infty}_{A}M).\]
seems to have no chance of preserving limits, but since colimits of coalgebras are formed underlying,
colimits are not out of the race. This leads us to the following idea:

\begin{definition}[Preliminary 2]\label{dream}
  Let $R$ be an $\bb{E}_{\infty}$-ring and $A\in \rm{cCAlg}_{R}$. We say that $A$ admits a cotangent
  complex $L_{A}:= C_{A}(R)$ if the functor $C_{A}(\blank):\rm{Mod}_{R} \to \rm{Mod}_{R}$ commutes
  with colimits. In this case we have $C_{A}(M)\simeq L_{A}\otimes M$ for every $ M \in \rm{Mod}_{R}$
\end{definition}

This definition is highly speculative, as the only coalgebras we know to admit a cotangent complex
in this sense are the formally \'etale coalgebras, for which the functor $C_{\blank}(A)$ is constant.
Conversely, if $A$ admits a cotangent complex then $L_{A}$ vanishes if and only if $A$ is formally
\'etale. Hence, the spectrum $L_{A}$ is precisely the obstruction to $A$ being formally \'etale,
which is the kind of conceptual clarity we are looking for.
While we lose any direct comparison to the cotangent complex of $A^{\vee}$ this is not entirely surprising,
since the property of being formally \'etale is defined very differently for $A^{\vee}$.
This leaves us with the following:

\begin{question}\label{cotangentdream}
  Let $R$ be an $\bb{E}_{\infty}$-ring. Does every $A \in \rm{cCAlg}_{R}$ admit a cotangent complex in the sense
  of Definition~\ref{dream}?
\end{question}

Regardless of the answer, the takeaway should be that the modules
$C_{A}(M)$ are exactly the obstruction towards $A$ being formally \'etale. Moreover, while the functor
$A\mapsto C_{A}(M)$ is very complicated, the dependence on $M$ should be relatively tame. That is,
for fixed $A$ it should be possible to describe the functor $M \mapsto C_{A}(M)$ in terms of a
formula involving $C_{A}(R)$. However, because $C_{A}(M)$ no longer has a direct relation to any
space of derivations or tangent complex, we cannot leverage results like Proposition~\ref{structure}
to obtain such a formula. We understand this as an indication that for these questions, the formalism may
have reached its limit.

\subsection{The Frobenius}
The most lacking thing about our results is the class of coalgebras that we can currently apply them to.
As of now, we are unable to give examples of formally \'etale coalgebras which are not dualizable. In
particular, we cannot describe the deformation theory of $R[X]$ for spaces $X$ which are not finite.
Attempts to reduce to the dualizable case all seem to fail for the following reason: Even though
we may write $X= \fcolim_{i}X_{i}$ where each $X_{i}$ is finite, giving the formula
$R[X]= \fcolim_{i}R[X_{i}]$, there is no reason why the functor
$\Omega^{\infty}_{\blank}(M): \rm{cCAlg}_{R}\to \rm{cCAlg}_{R}$ should commute with colimits.
Indeed, write $f_{M}:R\to R\oplus M$ for inclusion, then by definition
$\Omega^{\infty}_{\blank}(M) = f_{M,!} f^{\pt}_{M}$. The functor $f^{\pt}_{M}$ commutes with colimits,
and from Proposition~\ref{present} and the converse of the adjoint functor theorem we can deduce
that $f_{M,!}$ commutes with $\kappa$-filtered colimits for some regular cardinal $\kappa$. Thus, the class
of formally \'etale coalgebras is closed under $\kappa$-filtered colimits, but $\kappa$ is, in general, not countable.
% Closely related is the fact the notion of compactness is strangely behaved for coalgebras. For example,
% one can show that $\bb{Q}$ is not a compact object of $\rm{cCAlg}_{\Q}$, see~\cite[][Warning 1.2.15.]{ellII}.
% In particular, this means that
% \[ \rm{cSpec}(\fcolim_{i}\S[X_{i}])(\bb{Q})\neq \fcolim_{i}\rm{cSpec}(\S[X_{i}])(\Q),\]
% so we cannot deduce things about the cospectrum of infinite spaces in this way either. \\
This goes to show that the deformation theory of non-dualizable coalgebras is richer and more
interesting than that of the Ind-completion of dualizable coalgebras and requires additional input.
One contender for this additional input is the \textit{Coalgebra Frobenius} constructed by
Nikolaus:

\begin{theorem}[Nikolaus]
  Let $\cl{C} = (\rm{cCAlg}^{\rm{cn}}_{\S^{\wedge}_{p}})^{\wedge}_{p}$, then there exists a natural transformation
  $\psi_{p}:\id_{\cl{C}}\to \id_{\cl{C}}$ which on an object $A\in \cl{C}$ is given by the composition
  \[ \psi_{p}: A \rar{\Delta_{A}^{\otimes p}} (A^{\otimes p})^{hC_{p}} \rar{\rm{can}} (A^{\otimes p})^{tC_{p}} \rar{\sim} A,\]
  where the final map is the inverse of the \textit{Tate Diagonal}, see~\cite[][Theorem III.1.7]{tch}.
\end{theorem}

Given this map, we are naturally led to define \textit{perfect} coalgebras as follows:

\begin{definition}
  We say that $A \in  (\rm{cCAlg}^{\rm{cn}}_{\S^{\wedge}_{p}})^{\wedge}_{p}$ is \textit{perfect} if the coalgebra
  Frobenius $\psi_{p}: A\to A$ is a homotopy equivalence. We denote the full subcategory spanned by
  the perfect coalgebras by $(\rm{cCAlg}^{\rm{cn}}_{\S^{\wedge}_{p}})^{\wedge ,\rm{perf}}_{p} \subseteq
  (\rm{cCAlg}^{\rm{cn}}_{\S^{\wedge}_{p}})^{\wedge}_{p}$.
\end{definition}

\begin{example}\label{frobchains}
  Let $X$ be any space. Then $(\S[X])^{\wedge}_{p}$ is a perfect coalgebra since we have that
  \[\S[X]^{\wedge}_{p} \simeq (\S_{p}^{\wedge}[\colim_{X}\pt])^{\wedge}_{p} \simeq (\colim_{X} \S_{p}^{\wedge})^{\wedge}_{p}.\]
  On $\S_{p}^{\wedge}$ the map $\psi_{p}$ is necessarily given by the identity, because $\S_{p}^{\wedge}$
  is the terminal $p$-complete $\S_{p}^{\wedge}$-coalgebra. Thus, by naturality $\psi_{p}$ is given
  by the identity on $(\S[X])^{\wedge}_{p}$ as well.
\end{example}

We conjecture that this Frobenius map is related to the deformation theory of coalgebras in a similar
way to the Algebra Frobenius, in that it provides a sufficient condition for a coalgebra to be formally
\'etale.

\begin{conjecture}\label{frobcof}
  Let $A \in (\rm{cCAlg}^{\rm{cn}}_{\S^{\wedge}_{p}})^{\wedge}_{p}$ and write $A\p= A\otimes_{\S^{\wedge}_{p}}\F_{p}$.
  Then for any $M \in \rm{Mod}_{\F_{p}}^{\rm{cn}}$, the coalgebra Frobenius $\psi_p:A\to A$ induces the zero map
  on the $R$-module  $C_{A\p}(M) = \rm{cofib}(A\p \rar{\eta_{A\p}} \Omega^{\infty}_{A}(M))$.
\end{conjecture}

\begin{corollary}
  If Conjecture~\ref{frobcof} holds, then the base change functor
  \[ (\rm{cCAlg}^{\rm{cn}}_{\S^{\wedge}_{p}})^{\wedge ,\rm{perf}}_{p} \to \rm{cCAlg}_{\F_{p}}^{\rm{cn}}
  \qquad A \mapsto A\otimes_{\S_{p}^{\wedge}}\F_{p}\]
is fully faithful and factors through the full subcategory
$\rm{cCAlg}_{\F_{p}}^{\rm{cn}, \rm{f\acute{e}t}}\subseteq \rm{cCAlg}_{\F_{p}}^{\rm{cn}}$.
\end{corollary}
\begin{proof}
  Since $\psi_{p}:A\rar{\sim} A$ is an equivalence it induces an equivalence on $A\otimes_{\S_{p}^{\wedge}}\F_{p}$ and
  thus on $C_{A\otimes_{\S_{p}^{\wedge}}\F_{p}}(M)$ as well. However, since it also induces the zero map on the latter
  we get that $C_{A\otimes_{\S_{p}^{\wedge}}\F_{p}}(M)\simeq 0$. Thus, $A\otimes_{\S_{p}^{\wedge}}\F_{p}$ is formally \'etale and the
  claim follows from Theorem~\ref{wittsp}.
\end{proof}

Combining this with Example~\ref{frobchains} would allow us to fully answer our initial question about
homology coalgebras.

\begin{corollary}\label{dream2}
  If Conjecture~\ref{frobcof} holds, then for any space $X$ the $\F_{p}$-chains $\F_{p}[X]$
  are formally \'etale. In particular $\F_{p}[X]$ admits a unique and functorial lift to a $p$-complete
  $\S_{p}^{\wedge}$-coalgebra given by $\S[X]^{\wedge}_{p}= W_{\S_{p}^{\wedge}}(\F_{p}[X])$.
\end{corollary}

The fact that Conjecture~\ref{frobcof} needs to be checked for every connective $\F_{p}$-module should
be understood as an extension of our failure to find a cotangent complex. Indeed, if $\F_{p}[X]$ admits
a cotangent complex in the sense of Definition~\ref{dream}, then to obtain Corollary~\ref{dream2} it
would suffice to show that $\psi_{p}$ induces the zero map on $C_{A\otimes_{\S_{p}^{\wedge}}\F_{p}}(\F_{p})
= L_{A\otimes_{\S_{p}^{\wedge}}\F_{p}}$. However, even for this specific module the conjecture is difficult
to attack from our present position. The problem is the tricky right adjoint
$\rm{cCAlg}_{\F_{p}\oplus \F_{p}}\to \rm{cCAlg}_{\F_{p}}$ appearing in the definition of
$C_{A\otimes_{\S^{\wedge}_{p}}\F_{p}}(\F_{p})$. Because there is no known formula for this functor, attempts to verify
the conjecture have thus far been unsuccessful in all non-trivial cases. This warrants further investigation
of the coalgebra Frobenius and Conjecture~\ref{dream2}.

\section{Open Issues}\label{sec:future-direction} 
In this section, we discuss some of the open questions that evolve around the necessity to carefully incorporate the DG concept in ML-oriented communication research. This is because, unlike many other ML-based technologies, most real-world communication applications require real-time operation and seamless adaptation to dynamically changing propagation conditions. This precludes the luxury of repeatedly training ML-oriented models and makes DG-induced robustness a must-have feature in any ML-aided communication system.

\subsection{Beyond End-to-End Learning for Generalization}

Most of the existing studies rely on end-to-end learning to train a holistic over-parametrized DNN architecture by applying gradient-based optimization to the learning system as a whole. This means that all transmit/receive modules of the communication system must be differentiable (in the reverse-mode algorithmic differentiation sense \cite{speelpenning1980compiling}). Few wireless communication libraries have been proposed to study differentiable communication systems \cite{sionna, AIwirelessMatlab}.

\begin{figure*}[!b]
\centering
\begin{minipage}[c]{\textwidth}
\centering
\includegraphics[scale=0.71]{figures/AI-end-to-end-transmitter-receiver.pdf}
\caption{The possible integration steps of ML methods into the conventional transmit/receive communication chain if ML methods will be proven to be robust to domain shifts.}
\label{fig:end-to-end-AI-receiver}
\end{minipage}
\end{figure*}

Before advocating the adequacy of applying ML methods to the building blocks of the wireless physical layer depicted in Fig.~\ref{fig:end-to-end-AI-receiver}, DG has to be meticulously investigated and guaranteed \textit{within} and \textit{across} the blocks. From this perspective, it is not enough to claim the migration from model-based classical signal processing techniques to data-driven ML techniques without analyzing the impact of each migration on the overall system performance in terms of both accuracy and robustness. While such migration is a conceptually profound paradigm shift, its impact continues to be assessed from the accuracy perspective only, and hence must also be carefully analyzed through the lens of generalization/robustness.

The legacy physical layer design strategy relies on the divide-and-conquer approach by decomposing (a.k.a. layering) the entire communication chain into smaller blocks \cite{gallager2008principles}. Designing ML methods to substitute a single block or multiple blocks (see Fig.~\ref{fig:end-to-end-AI-receiver}) raises critical generalization questions justified by the following two facts:
\begin{itemize}
    \item End-to-end learning methods are trained with gradient descent-like optimizers, which exhibit slow convergence on ill-conditioned problems or convergence to possibly poor local optima. In other words, training is performed while hoping that the structural preconditioning is sufficiently strong to steer a method as simple as gradient descent from a random initial state to a highly non-trivial solution \cite{DBLP:conf/acml/Glasmachers17}. This assumption is risky since all ML techniques tailored for wireless applications are exclusively used for non-convex optimization problems.
    \item The valuable wireless communication know-how developed since the 50s is completely neglected during end-to-end training. ``Standing on the shoulders of giants'' (as Sir Isaac Newton once said) is a scientific tradition which promotes building upon the accumulated knowledge and discoveries made by others, and ``end-to-end learning'' must be proven robust to domain shifts to be considered an exception.
\end{itemize}

For these considerations, going beyond conventional end-to-end learning is an important step towards answering critical DG questions in data-driven ML techniques applied to wireless communications. In what follows, we discuss research
directions to cope with some end-to-end learning limitations.

\subsection{Hybrid Data-Driven and Model-Driven Methods}

After more than a century-long research effort in radio communications, state-of-the-art communication modeling and fast estimation algorithms are becoming more essential to high-bandwidth transmissions. From a DG perspective, the power of these classical model-driven tools lies in their guaranteed generalization capabilities because they do not depend on specific domains that are tied to generated/collected datasets. This generalization, however, often comes at the cost of high complexity.

Data-driven methods can come into play as an effective tool to reduce the computational complexity of classical model-based methods at the cost of generalization. As advocated in \cite{pellaco2022machine}, a hybrid framework that combines the benefits of both data-driven and model-based techniques is worth pursuing. Adopting this framework will prevent the generated domains for DG from being fully dependent on $i)$ the convergence of gradient-based optimizers for data-driven methods, or $ii)$ the complexity of model-based methods. For better illustration, we elaborate in what follows on how data-driven methods can be combined with physically consistent model-based methods.

The study of DG for MIMO communication should benefit from the side information provided by the physical laws governing the wave transmission and the circuits of RF components (i.e., amplifiers, and antennas). By employing physically consistent models \cite{DBLP:journals/tcas/IvrlacN10,DBLP:journals/corr/abs-2208-01556,DBLP:journals/twc/PizzoSM22}, it is possible to exploit the inherent symmetries and invariances in communication scenarios owing to Maxwell's equations \cite{jackson1999classical,baum1995symmetry}. From this perspective, physically consistent models for wireless communications offer an opportunity to generate communication datasets which exhibit domain-invariant regularities (e.g., antenna impedances), thereby diminishing the generalization difficulties across domains. As one example, fixing the impedance matrices of transmit and receive linear/planar antenna arrays increases the amount of correlation in the wireless channel, which can be exploited by DNNs for better channel estimation accuracy.

Moreover, this physically consistent direction opens the door for the analysis of DG through the lens of antenna theory. For example, it might be possible to determine which spacing parameter of the antenna array provides the best DNN accuracy for channel estimation. By doing so, realistic wireless communication domains are generated and more faithful representations of the real-world transmissions are simulated, thereby leading to a physically consistent version of digital twins for wireless communications \cite{DBLP:journals/cm/KhanSNHH22}.

\subsection{From Image-Based DG Methods to Signal-Based Methods}

Existing DG methodologies have been predominantly geared towards image-based vision tasks, leaving signal-based tasks almost unexplored despite being versatile in several real-world applications such as healthcare, retail, climate, finance, and communication. This unbalanced exploration impacts the development of specific DG methods for signal-based tasks. For instance, feature alignment approaches for DG are relying heavily on DNNs as feature extractors which are specifically fine-tuned to vision tasks, thereby leaving DG feature extraction for non-image signals severely underexplored. Some work looked at temporal distributional shifts in clinical healthcare \cite{guo2022evaluation,DBLP:conf/iconip/MaLZL19} and climate \cite{DBLP:conf/nips/MalininBGGGCNPP21} applications, but none of the prior work explored it in wireless communication. 

From this perspective, we highlight the importance of taking the first step towards a deeper understanding of temporal distributional shifts in wireless communication due to dynamic changes in the received signal resulting from the varying propagation properties (e.g., coherence time and Doppler shift).

\subsection{Compound Domain Generalization}
As mentioned previously, most of the presented methods for DG assume a homogeneous setting where domain labels are available. However, this assumption may not be realistic in several problems where the domain labels are hard to obtain or define. In this case, several techniques discussed above either become inapplicable (e.g., meta-learning) or their performance degrades drastically \cite{DBLP:journals/corr/abs-2103-02503}. Recently, there has been a surge of interest in studying the compound DG setting in vision problems. Most of the methods for compound DG propose to infer latent domain information from data and then use standard learning techniques to generalize across the latent domains. These solutions are, however, based on different restrictive assumptions such as: $i)$ the latent domains are distinct and separable \cite{hmoe}, $ii)$ the domain heterogeneity originates from stylistic differences \cite{DBLP:conf/cvpr/ChenL0LY22} or $iii)$ the latent domains are balanced \cite{DBLP:journals/corr/abs-1911-07661}. Compound DG is hence still an active research field with a lot of room for improvement, especially in wireless communication problems.

\subsection{Federated Domain Generalization}

Distributed learning algorithms enable devices to cooperatively build a unified learning model across agents with local training. As a result, a wide variety of distributed ML methods have been proposed and extensively analyzed within the federated learning (FL) framework \cite{DBLP:journals/corr/KonecnyMYRSB16}. 

For wireless physical layer applications, FL has been explored to address multiple key communication problems beyond the data security aspect \cite{DBLP:conf/infocom/ZhangYD22} such as channel estimation \cite{DBLP:journals/twc/ElbirC22}, symbol detection \cite{DBLP:conf/ssp/MashhadiSEG21} and beamforming \cite{elbir2020federated}. All of these works do not assume the availability of a central entity (e.g., base station) at which the learning model is trained. However, the question of whether the model learned by each agent generalizes to unseen scenarios is still unanswered and this remains an unexplored research area. In the context of IoT applications, very few efforts started investigating the challenges of DG for IoT devices by aligning each device's domain to a reference distribution in a distributed manner \cite{zhang2023federated}.

Addressing DG in the FL context is known as \textit{federated domain generalization} (FDG)\cite{DBLP:journals/corr/abs-2111-10487}. Distributed agents can collect their local data independently, hence naturally forming a distinct source domain. At the time of writing, no research paper in wireless communication has studied FDG, e.g., in the context of distributed MIMO \cite{DBLP:journals/jsac/WangWYWCH13} consisting of distributed antenna array systems.

\section{Conclusion}\label{sec:conclusion}
In this work, we focus on addressing the fundamental challenge of OOD detection tasks, which is how to fully understand the semantic discrepancy between the ID/OOD samples. We reveal that the key to success in the realistic SCOOD task is to allocate as many ID samples in the unlabeled set correctly as possible. To this end, we propose a novel uncertainty-aware optimal transport scheme that introduces class-specific energy scores as guidance for effective label assignment. Experimental results show that our method achieves better performance than previous state-of-the-art methods on SCOOD benchmarks.

\textbf{Limitations.} In addition to temperature scaling, other techniques such as feature clipping applied in ReAct~\cite{sun2021react} also enhance the performance of energy score, so how to obtain an OOD score that best fits the SCOOD task can be further explored. Moreover, a setting highly related to SCOOD has been proposed in \cite{katz2022training} and formulated as a constrained optimization problem. We will also theoretically analyze these practical OOD settings in our feature work.

% \section*{Acknowledgments}
\textbf{Acknowledgments.} 
This work is supported by National Key R\&D Program of China under Grant 2020AAA0105701, National Natural Science Foundation of China (NSFC) under Grants 61872327, Major Special Science and Technology Project of Anhui, National Natural Science Foundation of China (62033012) and Ant Group through Ant Research Intern Program.


%----------------------------------------------Bibliography-------------------------------------------
%\clearpage
\bibliographystyle{ieeetr}
\bibliography{refs}
%---------------------------------------------------------------------------------------------------

\end{document}
