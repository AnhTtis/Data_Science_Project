Traditional approaches for removing identifying features from private images (e.g., faces and medical images) involve direct manipulation of pixels, including masking, blurring, and pixelation~\cite{Ribaric15,Bischoff07,Milchenko13,Schimke11}.
However, these heuristics have been found to provide insufficient privacy protection~\cite{Abramian19,Ravindra21}.
In response, a \emph{learning}-based approach to de-identification has been increasingly studied~\cite{Der21}.
This is enabled by recent advances in GANs, which intuitively provide a more powerful approach to manipulate images according to their intrinsic manifold~\cite{Gui21}.

The existing literature on GAN-based transformation of images for privacy protection largely focuses on face de-identification.
A common approach for formalizing privacy protection in this problem is to combine multiple images to obtain $k$-anonymity through the $k$-Same framework~\cite{Jourabloo15,Meden18,Newton05}.
A key difficulty in these works has been generating high-quality images that capture useful information in the original image.
To this end, recent works have focused on developing techniques to disentangle and preserve non-identity attributes of the image, such as pose and facial expression~\cite{Jeong21,Jourabloo15,Maximov20,Wu19}.
However, these methods are not directly applicable to our setting given the unclear distinction between identity vs. non-identity features in retinal images beyond the blood vessel structure.
GAN-based approaches to generate images with differential privacy~\cite{Dwork14} have also been proposed~\cite{Long21,Xu19}, but current techniques lead to significant degradation of image quality (see supplement for an example application to retinal images).

Several recent works have successfully explored the use of GANs for generating realistic retinal images.
Niu et al.~\cite{Niu19} proposed a method to generate an image consistent with the given pathological descriptors.
Both Zhou et al.~\cite{Zhou20} and Chen et al.~\cite{Chen21} developed GAN models to synthesize retinal images conditioned on a semantic segmentation to improve disease classification performance.
Yu et al.~\cite{Yu19} introduced multi-channel GANs that improve the quality of generated retinal images by separately considering different elements of the image, including the blood vessels and the optic disc. 