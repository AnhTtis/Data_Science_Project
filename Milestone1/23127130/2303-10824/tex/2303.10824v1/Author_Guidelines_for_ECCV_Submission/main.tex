% updated April 2002 by Antje Endemann
% Based on CVPR 07 and LNCS, with modifications by DAF, AZ and elle, 2008 and AA, 2010, and CC, 2011; TT, 2014; AAS, 2016; AAS, 2020; TH, 2022

\documentclass[runningheads]{llncs}
\usepackage{graphicx}
% DO NOT USE \usepackage{times}, it will be removed by typesetters
%\usepackage{times}

\usepackage{tikz}
\usepackage{comment}
\usepackage{amsmath,amssymb} % define this before the line numbering.
\usepackage{color}
\usepackage{xcolor}
\usepackage{algorithm}
\usepackage{algpseudocode}
\usepackage{tabularx, booktabs, makecell, caption}
\usepackage{multirow}
\usepackage{adjustbox}
\usepackage{threeparttable}
\usepackage{graphicx}
% \usepackage{orcidlink}
% \usepackage{siunitx}
% \usepackage{multicol}
% The "axessiblity" package can be found at: https://ctan.org/pkg/axessibility?lang=en
\usepackage[accsupp]{axessibility}  % Improves PDF readability for those with disabilities.
\usepackage{hyperref}
\hypersetup{
    colorlinks=true,
}

\makeatletter
\renewcommand*{\@fnsymbol}[1]{\ensuremath{\ifcase#1\or *\or \dagger\or \ddagger\or
   \mathsection\or \mathparagraph\or \|\or **\or \dagger\dagger
   \or \ddagger\ddagger \else\@ctrerr\fi}}
\makeatother

\begin{document}
\pagestyle{headings}
\mainmatter
\def\ECCVSubNumber{6974}  % Insert your submission number here

\title{$k$-SALSA: $k$-anonymous synthetic averaging of retinal images via local style alignment} % Replace with your title

% INITIAL SUBMISSION 
%\begin{comment}
% \titlerunning{ECCV-22 submission ID \ECCVSubNumber} 
% \authorrunning{ECCV-22 submission ID \ECCVSubNumber} 
% \author{Anonymous ECCV submission}
% \institute{Paper ID \ECCVSubNumber}
%\end{comment}
%******************

% CAMERA READY SUBMISSION
% \begin{comment}
\titlerunning{Synthetic Averaging of Retinal Images via Local Style Alignment}
% If the paper title is too long for the running head, you can set
% an abbreviated paper title here
%
% \author{Minkyu Jeon\inst{1,2}\orcidlink{0000-0003-0572-6065} \and
% Hyeonjin Park\inst{5,}\thanks{This work was performed while the author was at Korea University}\and
% Hyunwoo J. Kim\inst{2,}$^\dagger$\orcidlink{0000-0002-2181-9264} \and
% Michael Morley\inst{3,4,}$^\dagger$\orcidlink{0000-0001-6373-7008} \and
% Hyunghoon Cho\inst{1,}\thanks{Corresponding authors}\orcidlink{0000-0002-2713-0150}}
\author{Minkyu Jeon\inst{1,2} \and
Hyeonjin Park\inst{5,}\thanks{This work was performed while the author was at Korea University}\and
Hyunwoo J. Kim\inst{2,}$^\dagger$ \and
Michael Morley\inst{3,4,}$^\dagger$ \and
Hyunghoon Cho\inst{1,}\thanks{Corresponding authors}}

%
\authorrunning{M. Jeon et al.}
% First names are abbreviated in the running head.
% If there are more than two authors, 'et al.' is used.
%
\institute{Broad Institute of MIT and Harvard, Cambridge, MA, USA\\ 
\email{\{mjeon, hhcho\}@broadinstitute.org} \and
Korea University, Seoul, Republic of Korea\\
\email{hyunwoojkim@korea.ac.kr} \and
Harvard Medical School, Boston, MA, USA \\ \and
Ophthalmic Consultants of Boston, Boston, MA, USA \\
\email{mgmorley@eyeboston.com}\and
NAVER CLOVA, Seoul, Republic of Korea \\ \email{hyeonjin.park.ml@navercorp.com}}
% \end{comment}
%******************
\newcommand{\minkyu}{\textcolor[rgb]{1,0,1}}
\newcommand{\correct}{\textcolor[rgb]{0,0,1}}
\maketitle
\begin{abstract}
\begin{abstract}
The current study investigated possible human-robot kinaesthetic interaction using a variational recurrent neural network model, called PV-RNN, which is based on the free energy principle.
Our prior robotic studies using PV-RNN showed that the nature of interactions between top-down expectation and bottom-up inference is strongly affected by a parameter, called the meta-prior, which regulates the complexity term in free energy.
% The current study examines how the behaviours of robots alter by changing the meta-prior $w$ in human-robot kinaesthetic interaction.
The current study examines how changing the meta-prior $w$ in the interaction phase affects the counter force generated when an experimenter attempts to induce movement pattern transitions familiar to the robot through its prior training.
The study also compares the counter force generated when trained transitions are induced by a human experimenter and when untrained transitions are induced.
Our experimental results indicated that (1) the human experimenter needs more/less force to induce trained transitions when $w$ is set with larger/smaller values, (2) the human experimenter needs more force to act on the robot when he attempts to induce untrained as opposed to trained movement pattern transitions.
Our analysis of time development of essential variables and values in PV-RNN during bodily interaction clarified the mechanism by which gaps in actional intentions between the human experimenter and the robot can be manifested as reaction forces between them.


%% Hiroki writing 2022-11-4
%Current study investigates the dynamics of the latent states during human-robot kinaesthetic interaction using PV-RNN.
%We have achieved to observe and analyse the internal state of an RNN model based on the free energy principle, during real-time human-robot interaction.
%Essential characteristics observed in the previous study of this variational recurrent neural network model, PV-RNN, is that by changing a meta prior $w$, the balance between the top-down intention and the bottom-up perceptual reality changes.
%In the current study, we examined how changing the weighting parameter $w$ between accuracy and complexity in free energy principle affects the humanoid robot's behaviour through human-robot interaction. We have conducted some human-robot kinaesthetic interaction experiments with various $w$ and quantitatively analysed the latent variable and the force applied to the humanoid robot. We have observed that the force required to change the robot's intention has increased, both when the top-down intention was strengthened by changing the $w$ and when corresponding switch of its primitive was against the experience of the RNN during its training. The study confirms through quantitative analysis that by increasing or decreasing the $w$ in PV-RNN, humanoid robot leads or follows the human counterpart during the human-robot kinaesthetic interaction.

\begin{comment}
Comment from Jun #2
・最後にQualitativeな結果(インパクト)が欲しい
・Current study investigates the problem on~と書き出すのが一般的
・最初の一文と最後の一文を対応させる
・最後の一文はもう少しAbstractかつ包括的に
\end{comment}

\begin{comment}
Comment from Jun #1
We investigated how the kinaesthetic human-robot interaction can affect the internal state of a model based on the free energy principle. 
=> how the internal state is affected is not the most important point in this study. This part should be rewritten.

The key function of this variational recurrent neural network model, PV-RNN, is that by changing a meta prior $w$, it takes a balance between the "complexity” term and the ”accuracy” term which corresponds to a top-down intention and a bottom-up perceptual reality in the free energy principle, respectively. 
=> This is not key function of PV-RNN. It is an essential characteristics observed in the previous study. The grammar after $w$ is something strange. Rewrite these.

This research has conducted a human-robot interaction experiment with a robotic agent in a kinaesthetic sense.
=> The sentence is not good. "in a kinaesthetic sense" is grammatically wrong.
MODIFIED => "In the current study human-robot interaction experiments using the kinaesthetic sense were conducted."

We investigated that when human forces the agent to switch primitives from one to another, larger force was required both when the human intention is conflictive against the top-down the intention of the agent and when the agent has a stronger top-down intention by modifying the $w$.
=> You should write the essential results of the experiments rather than what we investigated and also how these results could contribute to the studies on human-robot interaction.
\end{comment}

\end{abstract}
\end{abstract}
\section{Introduction}
\label{sec:introduction}
% \begin{figure}[t]
%     % \begin{subfigure}{1\linewidth}
%     %   \centering
%     % %   \includegraphics[width=1\linewidth]{figs/fig_1_moti_textattn.pdf}  
%     % %   \includegraphics[width=1\linewidth]{figs/fig_1_moti_textattn_v2.pdf}  
%     %   \includegraphics[width=1\linewidth]{figs/fig_1_moti_textattn_v5.pdf}  
%     %   \vspace{-0.5cm}
%     %     \caption{Amount of attention added to each video clip from the source video and query text in the self-attention layers of Moment-DETR encoder.}
%     %     % \caption{Distribution of attention for source and query in Moment-DETR encoder}
%     %     % Visualization of video clip's self-attention score in Moment-DETR encoder.
%     %   \label{fig:fig1_text_attn_ex}
%     % \end{subfigure}%\hfill% or  or \hspace{0.3\textwidth}
%     \vspace{0.2cm}
%     % \begin{subfigure}{1\linewidth}
%       \centering
%     %   \includegraphics[width=1\linewidth]{figs/fig1_moti_negattn.pdf}  
%       \includegraphics[width=1\linewidth]{figs/fig1_moti_negattn_v3.pdf}  
%       \vspace{-0.4cm}
%     %   \caption{Correspondence of saliency scores on the relevance between video clips and the text query.}
%     % \caption{Predicted saliency scores against the video relevant positive query and video irrelevant negative query}
%       \label{fig:fig1_neg_attn_ex}
%     % \end{subfigure}%\hfill% or  or \hspace{0.3\textwidth}
%     \caption{
%     % 원준 원본
%     % (a) Comparison between attention scores of source and query for each video clip~(We sum the attention scores from video and text). 
%     % We observe that the attention scores are dominated by other clips in the source video. 
%     % Text queries do not account for much attention regardless of the relevance to the video clips.
%     % \textbf{(a)} Inspection of the query dependency in Moment-DETR encoder.
%     % % We visualize the attention score of video tokens in the transformer encoder and observe that text query accounts for only a low portion of attention.
%     % % This tendency occurs regardless of the relevance between the text query and video clips. 
%     % We visualize the attention score of video tokens in the transformer encoder and observe 1) text query only accounts for a low portion of attention, and 2) relevance between video-query pair does not affect the attention scores ratio of text.
%     \textbf{(b)} Comparison of highlight-ness when relevant and non-relevant queries are input.
%     As observed in , existing work only uses queries to play an insignificant role, thereby may not be capable of detecting false queries and considering the video-query relevance even when the problem in (a) is resolved. 
%     % \SE{} % 이 부분이 "not capable of" 란 용어가 세다는 피드백이 있는 듯 합니다. 이러한 능력이 없다는 것은 굉장히 강한 어조인거 같기는 하고, 이러한 경우들이 종종 있다거나 좀 약화시킬 필요가 있어보이긴 하네요.
%     On the other hand, our QD-DETR yields a query-dependent representation that the relevance between the source video and query text is updated in the saliency scores.
%     There is a large gap between positive and negative saliency scores, and scores are consistent since the clips are all highly correlated to others.
%     }
%     \label{fig:motivation_ex}
%     % \captionsetup{belowskip=13pt}
%     % \setlength{\belowcaptionskip}{-10pt}
% \end{figure}
\begin{figure}
    \centering
    \includegraphics[width=1\linewidth]{figs/fig1_moti_negattn_1111.pdf}
    % \includegraphics[width=1\linewidth]{figs/fig1_moti_negattn_1109.pdf}
    % \includegraphics[width=1\linewidth]{figs/fig1_moti_negattn_stat.pdf}
    \vspace{-0.6cm}
    \caption{
        % \SE{} % 수정 필요
        Comparison of highlight-ness~(saliency score) when relevant and non-relevant queries are given.
        We found that the existing work only uses queries to play an insignificant role, thereby may not be capable of detecting negative queries and video-query relevance; saliency scores for clips in ground-truth~(GT) moments are low and equivalent for positive and negative queries.
        % This also results in mispredicted moments when ground-truth~(GT) moment is dominated by clips unrelated to GT since their prediction is highly focused on the video.
        % \SE{} % 여기 한번 더 보면 좋을 듯 합니다. GT moment에 unrelated한 clip이 많으면? label이 틀렷을 경우를 말씀하시는건지?
        % As observed in saliency graph, existing work only uses queries to play an insignificant role, thereby may not be capable of detecting false queries and considering the video-query relevance.
        On the other hand, query-dependent representations of QD-DETR result in corresponding saliency scores to the video-query relevance and precisely localized moments.
        % On the other hand, our QD-DETR yields a query-dependent representation that the
        % saliency scores are in accordance with the relevance between the video and query.
        % text is in accordance with the saliency scores.
        % There is a large gap between positive and negative saliency scores, and scores are consistent since the clips are all highly correlated to others.
}
    \label{fig:motivation_ex}
\end{figure}


\section{Introduction}
% 원준 원본
% Along with the advance of digital devices and platforms, video is now one of the most desired data type for consumers. However, although the large information capacity of videos may be beneficial in many aspects, e.g., informative and entertaining, on the contrary perspective, videos are time-consuming, and hard to search for desirable moments. 
% This has led many creators to use extra manpower to crop and edit the video to generate highlight clips to gain the consumer’s attention.
Along with the advance of digital devices and platforms, video is now one of the most desired data types for consumers~\cite{apostolidis2021video,wu2017deep}.
% SE: Video aware deep learning application & survey papers?
Although the large information capacity of videos might be beneficial in many aspects, e.g., informative and entertaining, inspecting the videos is time-consuming, so that it is hard to capture the desired moments~\cite{anne2017localizing,apostolidis2021video}. 
% This has led many creators to use extra manpower to crop and edit the video to generate highlight clips to gain the consumer’s attention.


% On the other side, 
Indeed, the need to retrieve user-requested or highlight moments within videos is greatly raised.
Numerous research efforts were put into the search for the requested moments in the video~\cite{anne2017localizing, gao2017tall, liu2015multi, escorcia2019temporal} and summarizing the video highlights~\cite{zhang2016video, mahasseni2017unsupervised, badamdorj2022contrastive, wei2022learning}.
% Numerous research efforts were put into the search for the requested moments in the video~\cite{anne2017localizing, gao2017tall, liu2015multi, escorcia2019temporal}, summarizing the video to generate highlights was another popular topic~\cite{zhang2016video, mahasseni2017unsupervised, badamdorj2022contrastive, wei2022learning}.
Recently, Moment-DETR~\cite{momentdetr} further spotlighted the topic by proposing a QVHighlights dataset that enables the model to perform both tasks, retrieving the moments with their highlight-ness, simultaneously.

% 원준 원본
% To detect the desired moments, previous works employed transformer encoder-decoder architectural designs to fuse the text query into the video representations. Moment-DETR~\cite{mDETR} modified detection transformer to process capture the moment as a set, and UMT~\cite{umt} implemented transformer decoder as to output clip-wise saliency. 
% Yet to their outstanding breakthroughs in the literature of moment retrieval with the seminal architectures, their limitation is that the role of the given text query is insignificant in representing the query-conditioned video representation; the attention mechanism of moment DETR is not explicitly conditioned on the text query, and the text query is conditioned on multi-modal clips where the differences between the clips are smoothed after encoding process in UMT.



% \begin{figure}[t]
% \centering
%     \begin{subfigure}[l]{0.37\linewidth}
%       \centering
%       \vspace{0.20cm}
%     %   \includegraphics[width=1\linewidth]{figs/fig_1_moti_textattn.pdf}  
%     %   \includegraphics[width=1\linewidth]{figs/fig_1_moti_textattn_v2.pdf}  
%       \includegraphics[width=1\linewidth]{figs/fig1_moti_violin_a.pdf}  
%       \vspace{-0.60cm}
%     %   \caption{text attention}
%         \caption{Importance of queries in video representation}
%       \label{fig:fig1_text_attn}
%     \end{subfigure}%\hfill% or  or \hspace{0.3\textwidth}
%     \vspace{0.2cm}
%     \begin{subfigure}[r]{0.61\linewidth}
%       \centering
%     %   \includegraphics[width=1\linewidth]{figs/fig1_moti_negattn.pdf}  
%       \includegraphics[width=1\linewidth]{figs/fig1_moti_violin_b.pdf}  
%     %   \caption{neg attention}
%         % \caption{Relation between the highlight-ness and the relevance between videos and query texts.}
%         \caption{Highlight-ness~(saliency) histogram of positive and negative video-query pairs\SE{}}
%       \label{fig:fig1_neg_attn}
%     \end{subfigure}%\hfill% or  or \hspace{0.3\textwidth}
%     % \vspace{-0.2cm}
%     \caption{Overall statistics for attention scores in Fig.~\ref{fig:motivation_ex} in QVHighlights dataset. 
%     (a) For the attention scores that measure how much the text query is generally involved in video representation, we use violin plots to show the probability density. We plot the score for each layer in the encoder.
%     % (b) Using the histogram, we compare how the baseline and QD-DETR yield different salient scores given the positive and negative video-text pairs.
%     (b) Saliency histogram shows the distributional gap between positive and negative video-text query pairs of baseline~(Moment-DETR) and proposed QD-DETR.\SE{}
%     }
%     \label{fig:motivation}
%     % \captionsetup{belowskip=13pt}
%     % \setlength{\belowcaptionskip}{-10pt}
% \end{figure}

% \begin{figure}[t]
% \centering

%     \begin{subfigure}[r]{1\linewidth}
%       \centering
%       \hspace{-0.2cm}
%     %   \includegraphics[width=1\linewidth]{figs/fig1_moti_negattn.pdf}  
%       \includegraphics[width=1.1\linewidth]{figs/fig1_moti_violin_a_v2.pdf}  
%     %   \caption{neg attention}
%         % \caption{Relation between the highlight-ness and the relevance between videos and query texts.}
%         \vspace{-0.5cm}
%         % \caption{Saliency histogram of positive and negative video-query pairs}
%         \caption{We plot the histograms and its average value~(dotted line) to compare saliency scores when true and false text queries are given for each method. (left) Since the video representations do not include much textual information, both the true and false queries yield similar saliency scores. (Middle) Even when the video representation is enforced to be updated with the textual information, the issue is not much resolved. (Right) By extracting discriminative features in the text query, distributions are differentiated.
%         % \SE{} % R1@0.5 설명
%         Also, R1@0.5 indicates evaluation metric, Recall at 1 with IoU 0.5 threshold on QVhighlight \textit{val} set.
%         }
%       \label{fig:fig1_neg_attn}
%     \end{subfigure}%\hfill% or  or \hspace{0.3\textwidth}
%     \\
%     \begin{tabular}{cc}
%     \hspace{-0.2cm}
%         \begin{minipage}{.4\linewidth}
%             \begin{subfigure}[l]{1\linewidth}
%               \centering
%             %   \vspace{0.20cm}
%             %   \includegraphics[width=1\linewidth]{figs/fig_1_moti_textattn.pdf}  
%             %   \includegraphics[width=1\linewidth]{figs/fig_1_moti_textattn_v2.pdf}  
%               \includegraphics[width=1\linewidth]{figs/fig1_moti_violin_a.pdf}  
%               \vspace{-0.60cm}
%             %   \caption{text attention}
%                 \caption{Importance of queries in video representation}
%               \label{fig:fig1_text_attn}
%             \end{subfigure}%\hfill% or  or \hspace{0.3\textwidth}
%         \end{minipage}
        
%         \begin{minipage}{.6\linewidth}
%             \vspace{-0.2cm}
%             \caption{Overall statistics of Fig.~\ref{fig:motivation_ex} in QVHighlights dataset. 
%             (a) Saliency histogram shows the distributional gap between positive and negative video-text query pairs.
%             % (a) For the attention scores that measure how much the text query is generally involved in video representation, we use violin plots to show the probability density. We plot the score for each layer in the encoder.
%             % (b) Using the histogram, we compare how the baseline and QD-DETR yield different salient scores given the positive and negative video-text pairs.
%             % (b) Text ratio in self-attention layer to  of Moment-DETR
%             % (b) Ratio of text when representing video tokens in self-attention of Moment-DETR.
%             % (b) Magnitude of attention text query involved.
%             % (b) Attention score of video tokens
%             % (b) Magnitude of text query to refine the video tokens in self-attention layer of Moment-DETR.
%             (b) Probability density depicting the weight of the text query in attention score for video clips. Scores are from the self-attention layers in Moment-DETR encoder.
%             % (b) The text query ratio in attention score of video clips (Self-attention layer in Moment-DETR encoder). We use violin plots to show probability density.
%             % 텍스트 쿼리가, 비디오 피쳐에 얼만큼 attend 하는지
%             }
%         \end{minipage}
    
%     \end{tabular}
%     \vspace{-0.5cm}
%     \label{fig:moti}
%     % \captionsetup{belowskip=13pt}
%     % \setlength{\belowcaptionskip}{-10pt}
% \end{figure}


% \begin{figure}
%     \centering
%     % \includegraphics[width=1\linewidth]{figs/fig1_moti_negattn_1109.pdf}
%     \includegraphics[width=1\linewidth]{figs/fig1_moti_negattn_stat_v2.pdf}
%     \vspace{-0.8cm}
%     \caption{
%         Histogram of saliency when the positive and negative queries are given. We plot the histograms and its average value~(dotted line) to compare saliency scores when relevant~(positive) and irrelevant~(negative) text queries are given for each method. (Left) Since the video representations do not properly reflect textual information, both the positive and negative queries yield similar saliency scores. 
%         % (Middle) Even when the video representation is enforced to be updated with the textual information, the issue is not much resolved. 
%         (Right) By representing video clips in query-dependent manner, distributions are differentiated.
%     }
%     \vspace{-0.6cm}
%     \label{fig:motivation}
% \end{figure}


% One of the demanding task is moment retrieval task, which is detecting the desired moments from the given query, typically the text query.
When describing the moment, one of the most favored types of query is the natural language sentence~(text)\cite{anne2017localizing}. 
While early methods utilized convolution networks~\cite{zhang2020learning, gao2021fast, wang2020temporally}, recent approaches have shown that deploying the attention mechanism of transformer architecture is more effective to fuse the text query into the video representation.
% To handle these modalities, previous works simply employed the attention mechanism of transformer architecture to fuse the text query into the video representation.
For example, Moment-DETR~\cite{momentdetr} introduced the transformer architecture which processes both text and video tokens as input by modifying the detection transformer~(DETR), and UMT~\cite{umt} proposed transformer architectures to take multi-modal sources, e.g., video and audio. 
Also, they utilized the text queries in the transformer decoder.
Although they brought breakthroughs in the field of MR/HD with seminal architectures, they overlooked the role of the text query.
To validate our claim, we investigate the Moment-DETR~\cite{momentdetr} in terms of the impact of text query in MR/HD~(Fig.\ref{fig:motivation_ex}).
Given the video clips with a relevant positive query and an irrelevant negative query, we observe that the baseline often neglects the given text query when estimating the query-relevance scores, i.e., saliency scores, for each video clip.
% the output saliency score, i.e. query-relevance scores.
% Based on the observation, we traced the actual saliency prediction of the model against both the video-relevant query and the irrelevant dummy one where we find that the baseline often neglects the given text query when estimating the query-relevance scores of video clips.
% For example, in Fig.~\ref{fig:motivation_ex}, saliency scores are not affected even when the query is substituted with the dummy.
% % General statistics for Fig.~\ref{fig:motivation_ex} is shown in Fig.~\ref{fig:motivation}. 
% General statistics corresponding to Fig.~\ref{fig:motivation_ex} are also shown in Fig.~\ref{fig:motivation}.



% The limitation of the concrete baseline~\cite{momentdetr} is inspected in two different aspects; 1) Utilization of text-query in the encoding process and 2) the output saliency score, i.e. query-relevance scores.
% Firstly, we visualize the attention score when video clips are given as a query in self-attention. 
% We observe that the text queries have relatively small impacts compared to other video features, as shown in Fig.~\ref{fig:fig1_text_attn_ex}.
% That is, the text does not account for much in representing every video clip, although the goal of MR/HD is to detect query-relevant moments.
% Based on the observation, we traced the actual saliency prediction of the model against both the video-relevant query and the irrelevant dummy one where we find that the baseline often neglects the given text query when estimating the query-relevance scores of video clips.
% For example, in Fig.~\ref{fig:motivation_ex}, saliency scores are not affected even when the query is substituted with the dummy.
% % General statistics for Fig.~\ref{fig:motivation_ex} is shown in Fig.~\ref{fig:motivation}. 
% General statistics are also shown in Fig.~\ref{fig:motivation}.

% Consequently, in Fig.~\ref{fig:fig1_neg_attn_ex}~(b), we found that the baseline often neglects the given text query when estimating the query-relevance scores of video clips; 
% For example, 


% We validate the previous work sometimes neglects the given query when estimating the saliency of video clips.
% For example, there is an example that the saliency scores from positive and negative queries cannot be distinguishable, as shown in Fig.~\ref{fig:fig1_neg_attn_ex}.
% % 우리는 추가로 text attention을 추가도 해봤지만, 효과가 있긴 했으나, still 이슈가 있는 것을 확인하였다?
% % Still, we observe that assuring the high attendance of text queries does not resolve the overlap which motivates us to question the quality of the naive use of task-agnostic text representation~\cite{momentdetr, umt}.
% We found that introducing the text-attention for ensuring the high attendance of text queries relieve the overlap, but there still be a severe overlap.


% To validate their limitations, we inspect the impacts of text queries in the concrete baseline~\cite{momentdetr} with the two different aspects, 1) tendency of attention in self-attention layer and 2) saliency score, i.e. query-relevance scores. \SE{} % attention 이 갑자기 등장하는가?
% Firstly, we visualize the attention score when video clips are given as a query in self-attention. We observe the text queries have relatively low attention scores compared to the video features, as shown in Fig.~\ref{fig:fig1_text_attn_ex}.
% That is, the text does not account for much in representing every video clip, although the goal of MR/HD is to detect query-relevant moments.
% Based on this observation, we trace the actual saliency prediction of the model against both positive and negative text queries.
% We validate the previous work sometimes neglects the given query when estimating the saliency of video clips.
% For example, there is an example that the saliency scores from positive and negative queries cannot be distinguishable, as shown in Fig.~\ref{fig:fig1_neg_attn_ex}.
% % 우리는 추가로 text attention을 추가도 해봤지만, 효과가 있긴 했으나, still 이슈가 있는 것을 확인하였다?
% % Still, we observe that assuring the high attendance of text queries does not resolve the overlap which motivates us to question the quality of the naive use of task-agnostic text representation~\cite{momentdetr, umt}.
% We found that introducing the text-attention for ensuring the high attendance of text queries relieve the overlap, but there still be a severe overlap.



% Thus, we 
% query dependency를 높이기 위해 
% Cross-attention? text-attention? detailed explanation on text-attention should be needed?
% By handling these two issues, we find that more precise retrieval can be achieved.
% 
% 
%
% By projecting video-discriminative text features with high text attendance to source video, we f 
% We also find the need to improve the quality of query features since assuring high text attendance also results in...
% pairs are not finetuned to be discriminative that even the similarity within the pairs does not reflect the relevance between the query and the video clips.
% General statistics for Fig.~\ref{fig:motivation_ex} is shown in Fig.~\ref{fig:motivation}. 
% \SE{} % 이거 ??로 뜨는데, 위처럼 figure 그리면 label이 안되는걸까요
% \SE{}
% 형님 아래 사항 생각 좀 해보는게 좋을 거 같아요.
% fig 1. (a) 그림만 봤을 때 모든 clip에 대해 text attention이 일정이상 존재하긴 하니까, 뭔가 not assured to be conditioned가 와닿지 않는거 같아요.
% + 왜 text가 항상 attend 해야하나?
% not assured to be conditioned --> text shows relatively low affects compared to video 같이 실제 나타난 현상까지 같이 적으면 어떨까 싶어요.
% fig 1. (b) 덜 반영한다?

% \SU{}
% 일단 text가 attend 잘 되어야 한다는 것에 좀 궁금점이 생깁니다. 결국에는 text와 관련있는 frame들을 attend해서 higlight를 찾아야 하는게 아닐까요? 그리고, 현제 저희의 모델 구조상 text query가 Key와 Value로 거의 활용되고 있는데 그렇다면 결국에는 해당 모델은 text에 대한 attention이 전혀 없다고 봐도 무방하지 않을까요? 그런 면에서 text attention을 강조하는게 좀 걸리긴 합니다.

% Specifically, the text query is not assured to be explicitly conditioned on every clip of the video, and as the query texts are evenly treated, discriminative keywords may not be spotlighted.
% attention mechanism of Moment-DETR is not explicitly conditioned on the text query as shown in Fig~\ref{}(d), and in UMT, the text are only used for conditioning the queries while the video representation are refined itself by self-attention.

% \begin{figure}[t]
%     \begin{subfigure}{1\linewidth}
%       \centering
%     %   \includegraphics[width=1\linewidth]{figs/fig_1_moti_textattn.pdf}  
%     %   \includegraphics[width=1\linewidth]{figs/fig_1_moti_textattn_v2.pdf}  
%       \includegraphics[width=1\linewidth]{figs/fig_1_moti_textattn_v4.pdf}  
%       \vspace{-0.5cm}
%     %   \caption{text attention}
%         \caption{Distribution of attention scores in Moment-DETR encoder}
%       \label{fig:fig1_text_attn}
%     \end{subfigure}%\hfill% or  or \hspace{0.3\textwidth}
%     \vspace{0.2cm}
%     \begin{subfigure}{1\linewidth}
%       \centering
%     %   \includegraphics[width=1\linewidth]{figs/fig1_moti_negattn.pdf}  
%       \includegraphics[width=1\linewidth]{figs/fig1_moti_negattn_v2.pdf}  
%       \vspace{-0.5cm}
%     %   \caption{neg attention}
%         \caption{Saliency score against positive and negative text queries}
%       \label{fig:fig1_neg_attn}
%     \end{subfigure}%\hfill% or  or \hspace{0.3\textwidth}
%     \vspace{0.2cm}
%     \begin{subfigure}{1\linewidth}
%       \centering
%     %   \includegraphics[width=1\linewidth]{figs/fig1_moti_violin.pdf}  
%       \includegraphics[width=1\linewidth]{figs/fig1_moti_violin_v2.pdf}  
%       \vspace{-0.5cm}
%       \caption{violin}
%       \label{fig:fig1_violin}
%     \end{subfigure}%\hfill% or  or \hspace{0.3\textwidth}
%     \vspace{-0.2cm}
%     \caption{(a) 1. portion of text attention vs. video attention 2. relation with text query and content (e.g. fg, bg) of clip seems not to affect the attention score
%     (b) 1. high variability even though entire clips are highly correlated with the given text query 2. positive and negative query makes overlaps on saliency score distribution
%     (3) actual distribution on validation dataset.}
%     \label{fig:motivation}
%     % \captionsetup{belowskip=13pt}
%     % \setlength{\belowcaptionskip}{-10pt}
% \end{figure}

To this end, we propose Query-Dependent DETR~(QD-DETR) that produces query-dependent video representation.
% Our key focus is to ensure each clip in predicted moments is explicitly conditioned by the query, particularly on the video-descriptive portion of the text query.
% Our key focus is to ensure that query-relevant clips are predicted by enforcing each clip to be explicitly conditioned by the query.
%Our key focus is to ensure that the model prediction for each clip is highly relevant to the query.
Our key focus is to ensure that the model's prediction for each clip is highly dependent on the query.
% by enforcing each clip to be explicitly conditioned by the query. :)
% hmm...
% \SE {} % "query-relevant clips are predicted" 이 문장이 좀 애매한거 같습니다. relevant 클립을 놓지지 않고 찾는 것을 보장한다? 이런 느낌인지 아니면 높은 saliency 를 주는게 목적이다? model prediction이 query-relevance를 반영하는 것을 보장한다?
% Our key focus is to ensure that the model prediction reflects query-relevance of clips by enforcing each clip to be explicitly conditioned by the query.
First, to fully utilize the contextual information in the query, we revise the transformer encoder to be equipped with cross-attention layers at the very first layers.
% 상익's thought :  single video - query간의 관계만 고려 - 같은 word가 더 많이 쓰이는 것을 보고 
% 교수님's thought : neg pair 를 쓰면 쿼리를 보지 않고서는 video clip간만 고려하는 것이 사라짐. 왜냐면 0으로 내보내야 하기 때문. --> SE: relative difference 만 고려하다가, 
By inserting a video as the query and a text as the key and value of the cross-attention layers, our encoder enforces the engagement of the text query in extracting video representation.
% 원준 교수님 코멘트 반영해서 다시
Then, in order to not only inject a lot of textual information into the video feature but also make it fully exploited, we leverage the negative video-query pairs generated by mixing the original pairs.
Specifically, the model is learned to suppress the saliency scores of such  negative~(irrelevant) pairs.
Our expectation is the increased contribution of the text query in prediction since the videos will be sometimes required to yield high saliency scores and sometimes low ones depending on whether the text query is relevant or not.
% \SE{}
% learns to?
% By suppressing the saliency scores of the irrelevant video-query pairs, the model learns to spotlight only the video-specific discriminative words in the query.
% % \SE{} % ====================== 상익 수정 ========================
% However, this architectural design still lacks the capability of identifying the video-descriptive keywords in the query.
% % However, this architectural design still lacks in identifying proper query relevance.
% This is because the current training scheme only focuses on the interactions of video and clips within a single video while neglecting information shared throughout the entire video.
% % We argue the problem of the current training scheme that only focuses on distinguishing the clips in a single video while neglecting information shared throughout the entire video.
% Therefore, we leverage the negative video-query relationships to enhance the capability of identifying the contextual similarity of query and video clips.
% 
% 원준 원본 
% However, this architectural design heavily relies on the quality of the text query.
% Therefore, we leverage the negative video-query relationships to enable the model to emphasize key corresponding query features.
% By suppressing the saliency scores of the irrelevant video-query pairs, the model learns to spotlight only the video-specific discriminative words in the query.
% =========================================================
Lastly, to apply the dynamic criterion to mark highlights for each instance, we deploy a saliency token to represent the entire video and utilize it as an input-adaptive saliency criterion. 
With all components combined, our QD-DETR produces query-dependent video representation by integrating source and query modalities.
This further allows the use of positional queries~\cite{dabdetr} in the transformer decoder.
% Furthermore, we can exploit the advanced DETR decoder architectures using the positional information, e.g., DAB-DETR, since our encoded tokens consist of identical position representations from a single modality.
% \SE{} % ====================== 상익 수정 ========================
% Furthermore, we can exploit the advanced DETR decoder architectures using the positional information, e.g., DAB-DETR, since our video clip tokens consist of identical position representations from a single modality.
% 원준 원본
% It also enables the use of advanced DETR decoder architectures, e.g., DAB-DETR, for the first time, as these works exploit the position information within a single modality.
% =========================================================
Overall, our superior performances over the existing approaches validate the significance of the role of text query for MR/HD.
% Our extensive experiments on QVHighlights, TVSum, and Charades-STA datasets validate the significance of considering the role and the quality of text query.

% All components combined with dynamic anchor moments for the query of decoder, our FOQUE fosters the query-dependent video representation, thereby making the 
% All components combined, our modified transformer encoding process fosters the query-dependent video representation thereby achieving the state-of-the-art results on various benchmarks of moment-retrieval and highlight detection.
	
% -	Video Platform & Streamer & Consumer의 증가. 
% Video는 다른 데이터 타입보다 정보가 많아 유용하지만, 이는 다른 말로 해석하면 video를 보는 것은 time-consuming 하고, 원하는 것을 찾아보기에는 힘들 수 있음.
% 따라서, 많은 매체에서는 사람들의 더 많은 이목을 끌기 위해 highlight 비디오라는 것을 편집하여 공유도 함.
% 하지만, highlight video를 만들기 위해 사람의 노력이 필요한 현 시점에서, This spotlights the need to retrieve the user-requested / Highlight moments in the video.

% -	이전에도 이러한 문제를 해결하기 위해 (asdfasdf) for moment retrieval, (asdfasdf) for highlight detection 등이 제안 되었지만, 이들은 비디오의 특정 영역을 찾는다는 공통된 목적을 가지고 있으면서도, 데이터 셋의 한계로 인해 따로 연구되었음. 이를 문제 삼으며, 최근에는 두 task를 동시에 학습할 수 있는 dataset이 소개 되었는데, 컴퓨터비전에서 최근 각광을 받고 있는 Transformer 모델 도입과 함께 큰 발전을 거듭하고 있음.

% -	구체적으로, 이 두가지 task를 수행하기 위해서는 transformer를 두가지 방법으로 이용할 수 있는데, moment-DETR 처럼 moment 를 clip의 set 단위로 예측할 수 있고, UMT 처럼 clip-wise prediction을 할 수 있음. 하지만, 이들은 query를 condition이 아닌 video와 동등한 레벨로 취급하거나 [mDETR], 매 클립이 self-attention으로 mixing 된 후에 condition을 걸어주어 clip간의 차이를 확실하지 이용하지 못하였고, 또한, 확실하게 condition으로 주지 못하였고, video와 query 사이의 관계를 한정적으로만 이용하였다.

% -	따라서, we explore three different ways to fully exploit query information. First, we design one-way cross-attention layer to condition every clip with the query features. Then, we utilized the negative video-text pairs to better model the relationships between the video and the text embeddings. Lastly, we define the saliency token to be the video-query dependent saliency estimator.


















% ===================== neg pair 부분 ===========================
% Nevertheless, the current training scheme, only considering the given video-query pair, still disturbs the model from identifying proper query-relevance prediction.
% In detail, the model focus on learning the fine-grained discrepancy between video clips, while neglecting the information they share, which contains significant clues to understand the context of video.
% Therefore, we leverage the negative video-query relationships to enhance the capability of identifying the contextual similarity of query and video clips.
% Therefore, we leverage the negative video-query relationships by suppressing those pairs, so that enhance the capability of identifying the contextual similarity of query and video clips.
% We hypothsize the diversity in query-video pairs are insufficient to learn the general relationship between text query and video.
% Therefore, we leverage the negative video-query relationships by suppressing the saliency scores of the irrelevant video-query pairs.
% However, this architectural design still lacks in identifying proper query relevance.
% We argue that the current training scheme only focuses on learning the fine-grained discrepancy between clips in a single video, while neglecting the information they share, which contains significant clues to understand the context of the video.
% Therefore, we leverage the negative video-query relationships to enhance the capability of identifying the contextual similarity of query and video clips.
% However, this architectural design still lacks in identifying proper query relevance.
% We argue the problem of the current training scheme that only focuses on learning the fine-grained discrepancy between clips in a single video.
% That is, the current design neglects the information shared throughout the video, although it contains significant clues to understand the context of the video.

\section{Related Work}
\section{Related Work} \label{sec:relatedwork}

The section introduces the research related to the paper, which can be divided into three parts: (1) high-utility pattern mining; (2) top-$k$ utility itemset mining; and (3) targeted pattern mining.

\subsection{High-utility pattern mining}

Frequent itemset mining (FIM) \cite{aggarwal2014frequent,agrawal1994fast,han2000mining} has been extensively studied for decades. However, relying only on frequency cannot bring enough benefits to users. Factors such as quantity and profit should also be considered. For this reason, Chen \textit{et al.} \cite{chan2003mining} put forward a new task called high-utility itemset mining (HUIM). Since then, utility mining research has developed rapidly \cite{gan2021survey,lin2016efficient,song2016high,wu2021haop}. For the convenience of discussion, high-utility itemset mining algorithms are grouped into the following three categories:

\textbf{Apriori-based algorithms}: Since Agrawal \textit{et al.} \cite{agrawal1993mining} proposed the Apriori property in 1994, lots of algorithms based on Apriori have been published. For example, Liu \textit{et al.} \cite{liu2005two} introduced the prominent Two-Phase algorithm to handle the difficulty that the utility, unlike frequency, is neither monotone nor anti-monotone. That algorithm uses an overestimation of the utility called \textit{TWU} (Transaction Weighted Utilization) to find candidate itemsets in a first phase. Thereafter, in a second phase, the database is searched again to determine the exact utility value of each candidate itemset. The IIDS algorithm \cite{li2008isolated} is an improved version of Two-Phase that discards isolated items to shrink the search space. However, the common disadvantage of Apriori-like algorithms is that plenty of candidate patterns are generated, resulting in considerable computational costs and memory consumption.

\textbf{Tree-based algorithms}: Tseng \textit{et al.} \cite{tseng2010up} designed the UP-tree structure, a utility-pattern tree, and introduced the UP-Growth algorithm inspired by FP-Growth. Subsequently, other versions of tree-based algorithms \cite{song2014mining,tseng2012efficient} have been presented. In general, utilizing the UP-tree can prevent many meaningless database scans. When working with large-scale databases, however, this structure grows increasingly complex and occupies a massive amount of memory.

\textbf{Other structure-based algorithms}: HUI-Miner \cite{liu2012mining} utilizes a novel data structure known as a utility-list, which avoids the difficulty of generating numerous candidates. Moreover, the FHM algorithm \cite{fournier2014fhm} reduces the cost of join operations by using a tighter upper bound, which results in outperforming HUI-Miner. However, the join operation on lists of these algorithms takes time and memory. Thus, Zida \textit{et al.} \cite{zida2015efim} proposed the EFIM algorithm with high-utility database projection (HDP) and high-utility transaction merging (HTM) techniques to lower the expensive cost of database passes. The utility-list-based CoUPM algorithm for correlated utility-based pattern mining \cite{gan2019correlated}. In summary, these algorithms integrate various strategies to discover HUIs as efficiently as possible.

\subsection{Top-$k$ utility itemset mining}

Although the above algorithms are effective in finding the desired set of itemsets, the efficiency of mining is strongly related to the selection of the minimum utility threshold. However, it is not easy to identify an appropriate threshold. Many top-$k$ pattern mining algorithms were thus designed to directly discover the set of top-$k$ HUIs, rather than asking users to specify a utility threshold. Top-$k$ HUIM algorithms mainly consist of two types: the first is the two-phase algorithms, and the other is the one-phase algorithms.

\textbf{Two-phase algorithms}: The task of discovering the top-$k$ HUIs was proposed by Wu \textit{et al.} \cite{wu2012mining} with the TKU algorithm, which outperformed HUIM algorithms in terms of speed. The TKU algorithm is a two-phase algorithm. In the first phase, a UP-Tree is built, and promising top-$k$ HUIs are generated. Then, in the second phase, the desired top-$k$ HUIs are selected among them. TKU applies several strategies to filter unpromising candidates during the search \cite{tseng2015efficient} and achieve higher efficiency. Subsequently, REPT \cite{ryang2015top} was introduced with optimizations to record and pre-calculate the utility of items to prune the search space effectively and raise the minimum utility threshold. REPT uses a tree structure and pre-evaluation matrixes as tools to store utility information. However, these two-phase algorithms still generate large sets of candidates, which causes unreasonably long runtimes and high memory usage.

\textbf{One-phase algorithms}: For top-$k$ HUIM, the one-phase TKO algorithm \cite{tseng2015efficient} was developed to solve the shortcomings of two-phase algorithms. TKO takes advantage of the utility-list structure of HUI-Miner, and outperforms the TKU and REPT algorithms according to experiments \cite{tseng2015efficient}. Similarly, another one-phase algorithm called KHMC \cite{duong2016efficient} also discovers the top-$k$ HUIs by using the utility-list structure. In KHMC, an estimated utility co-occurrence pruning (EUCP) technique is applied, which is based on precalculating the TWU of 2-itemsets. Moreover, the algorithm also adds another pruning strategy named early abandoning to avoid completely constructing the lists of unpromising itemsets. Three threshold-raising strategies are able to significantly shrink the search space and enhance the algorithm's efficiency. The THUI algorithm \cite{krishnamoorthy2019mining} has better performance thanks to introducing the concept of Leaf Itemset Utility (LIU), a triangular matrix, which can be implemented with only a small amount of memory to store utility information. Besides, the LIU-E and LIU-LB threshold raising strategies also accelerate the mining speed of the algorithm. THUI greatly outperforms TKO and KHMC, especially for dense or large datasets.

In addition, there are various other top-$k$ pattern mining problems and variations, such as mining top-$k$ sequential patterns \cite{zhang2021tkus}, mining top-$k$ HUIs in data streams \cite{cheng2021etkds}, discover top-$k$ high-utility sequential patterns \cite{zhang2021tkus}, and mining top-$k$ HUIs with negative utility values \cite{sun2021mining}.


\subsection{Targeted pattern mining}

Those algorithms listed above are designed to find all itemsets that meet a single predetermined criterion. Target-oriented query algorithms give an alternative solution to this problem by filtering out unnecessary information. Rather than searching for numerous but mostly insignificant items, the user can enter any target and then discover patterns containing the desired items. Several target-oriented query algorithms based on frequency have been developed in earlier studies. These interactive methods are capable of returning results containing a target. Kubat \textit{et al.} \cite{kubat2003itemset} were among the first to address the issue of processing target queries in a transactional database. They implemented target query processing algorithms for association mining by creating itemset trees that can be progressively updated. Fournier-Viger \textit{et al.} \cite{fournier2013meit} developed the Memory Efficient Itemset Tree (MEIT) to further reduce memory requirements. The tree is optimized to perform incremental modifications when new transactions are inserted, and it employs a node-compression method. For multi-objective mining of big data, the guided FP-growth (GFP-growth) algorithm based on FP-Growth was proposed by Shabtay \textit{et al.} \cite{shabtay2018guided}. In particular, many experiments have illustrated the excellent performance of the algorithm on imbalanced data. Target-oriented mining has also been studied and applied to discover sequential patterns. The targeted mining algorithm for sequential patterns proposed by Chueh \textit{et al.} \cite{chueh2010mining} speeds up the search for the target itemsets by using the reversion of the original sequence and comparing the reversed sequence with the related itemsets. Furthermore, clustering analysis is applied to automatically set time partition values for the task of time-interval sequential pattern mining. A novel target-oriented sequential pattern mining approach was presented by Chand \textit{et al.} \cite{chand2012target}, which uses RFM (recency, frequency, and monetary) constraints. As a result, fewer database projections are done, and the space complexity is reduced. To remove some useless or irrelevant patterns in high utility sequential pattern mining \cite{zhang2021shelf,gan2021explainable}, the TUSQ algorithm \cite{zhang2021tusq} first introduced the concept of utility into target sequence queries. The algorithm does not focus on frequency like previous algorithms, but rather on utility. Recently, the TargetUM algorithm \cite{miao2021targeted} has been proposed to fill the gap and perform target-oriented mining in HUIM.

In general, the TargetUM algorithm provides an integrated approach for high-utility mining with a target query, which serves as the foundation for this research. However, there are no studies combining top-$k$ high-utility methods with target pattern queries. This paper introduces the problem of targeted utility mining with the concept of top-$k$ patterns to prevent the generation of large sets of HUIs and to accurately and quickly process target queries.


\section{Method}
\section{Method}
\label{sec: method}
% This section introduces the rendering pipeline of our proposed hierarchical compositional scene. 
% our pipeline consists of three processes, including decomposing the text into editable 3D layout, rendering the compositional views with local (object) NeRFs and global (scene) NeRF and the joint optimization on these hierarchical 3D representations.

% Note that the transformation between the object and the scene frame is defined by ${p}_o$ and ${D}_o$. 
%
% Next, we build a residual connection to add ${\sigma}_o$ and the referenced global color, and the rendering result will be used to calculate the SDS loss based on the global text.  
% Fig.~\ref{fig:framework} illustrates our pipeline, which consists of three main components, including the editable 3D scene layout based on multi-object text (Sec.~\ref{ssec:layout}), the scene rendering pipeline that composites the predictions from all local NeRFs (Sec.~\ref{ssec:render}), and the joint optimization on both local and global representation models (Sec.~\ref{sec:optimization}).
% To elaborate, our editable 3D scene layout represents a global frame of the scene by decomposing it into a set of local frames, where each is parameterized by a local NeRF, a 3D bounding box, and a corresponding local text prompt.
% For instance, the text prompt `A teddy bear and a stuffed monkey sit side by side' is interpreted as a 3D scene layout, as shown in Fig.~\ref{fig:framework}.  
% The whole 3D layout, \ie, scene frame, consists of two 3D bounding boxes, \ie local frames \#1 and \#2, with specific local text prompts, \ie, `a teddy bear' and `a stuffed monkey'. 
% %
% To render the scene view, we first calculate the ray-box intersections between the boxes and rays $({\boldsymbol{r}}_o, \boldsymbol{\phi}_d, {\boldsymbol{\theta}}_d)$, where the ${\boldsymbol{r}}_o$ is the ray origin and the $({\boldsymbol{r}}_o, \boldsymbol{\phi}_d)$ is its direction.
% Then, to infer each object's properties in local NeRFs, we sample the global points $({\boldsymbol{x}}_g, {\boldsymbol{y}}_g, {\boldsymbol{z}}_g)$ in the global frame within the ray-box intersection intervals and project them into the normalized local location $({\boldsymbol{x}}_l, {\boldsymbol{y}}_l, {\boldsymbol{z}}_l)$ in the local frame.
% %
% Given the local sampling points $({\boldsymbol{x}}_l, {\boldsymbol{y}}_l, {\boldsymbol{z}}_l)$, the implicit local NeRF ${\boldsymbol{\theta}}_l$ outputs four pseudo-color channels ${\boldsymbol{C}}_l$ and density $\boldsymbol{\sigma}$, which can be used to render a local view of the local frame to match its local text prompt.
% %
% We further calibrate the predicted pseudo-color $\boldsymbol{C}_l$ from local frames by adding the global embeddings ${\boldsymbol{emb}}_g$ to improve the global view consistency.
% Then, the calibrated predictions after composition are used to reconstruct the scene view by volumetric rendering along the rays.
% %
% Lastly, the rendered views based on local and global frames are guided by score distillation sampling loss $\nabla \mathcal{L}_{\text{SDS}}$~\cite{poole2022dreamfusion} to optimize all the learnable parameters. 
To resolve the issue of guidance collapse, our principal strategy is to \textit{decompose the scene into reusable components and compose/recompose them into a unified and consistent one}.
This enables flexible control over the generated content with direct use of prompts and box layouts, as illustrated in \cref{fig:teaser}.
%
Our proposed CompoNeRF confers several key benefits:
1) \textbf{Semantic Coherence}: It reliably creates 3D objects with detailed textures and global consistency, exemplified by authentic light interactions, such as reflections on the bed surface.
2) \textbf{Modularity and Reusability}: CompoNeRF functions as an ensemble of independently trained NeRF models. These can be efficiently stored and later retrieved from a cached dataset, enabling their reuse in various cases.
3) \textbf{Editability}: Our approach allows for flexible scene modification, such as interchanging the lamp for a vase filled with sunflowers or altering its scale, by simply adjusting the box dimensions for later finetuning. This feature enhances flexibility and creative possibilities. 


% Furthermore, the usage of layout boxes enables more flexible control over the generated content compared with the intricate sketch shape in Latent-NeRF\cite{metzer2022latent}. 
\begin{figure*}[t]
    \centering
    \includegraphics[width=0.9\linewidth]{figures/method.pdf}
    % \vspace{-12pt}
    \caption{\textbf{Framework Overview}.
The CompoNeRF model unfolds in three stages: 1) Editing 3D scene, which initiates the process by structuring the scene with 3D boxes and textual prompts; 2) Scene rendering, which encapsulates the composition/recomposition process, facilitating the transformation of NeRFs to a global frame, ensuring cohesive scene construction. Here, we specify design choices between density-based or color-based(without refining density) composition; 3) Joint Optimization, which leverages textual directives to amplify the rendering quality of both global and local views, while also integrating revised text prompts and NeRFs for refined scene depiction.
  % The model is structured into three components: Composition, Decomposition, and Recomposition. Composition deals with the foundational setup, detailed with choices for density-based and color-based composition. Decomposition utilizes the modularity of the CompoNeRF feature, caching each NeRF module offline for efficient recalibration. Recomposition reuses these cached NeRFs and adjusts the semantic context, providing a revised output with the inclusion of the offline NeRF enhancements.
    % Our model consists of two branches where the upper part is individual NeRFs, and the lower part denotes global calibration with our tailored composition model. The specific designs for density-based and color-based composition modules are highlighted. 
    % CompoNeRF consists of three parts: 1). The editable 3D scene layout configures the scene representations with 3D boxes and text prompts; 2).  The scene rendering includes the global calibration and the compositional process; 3). The joint optimization applies global and local text guidance on global and local render views.
    % The global frame (scene space) contains a set of local frames. Each is  represented by a local NeRF associated with a 3D box and text prompt defined by the editable 3D layout.
    % The scene view is volumetric rendered by sampling the points $({\boldsymbol{x}}_g, \boldsymbol{y}_g, \boldsymbol{z}_g)$ intersected with any local frame along the ray $(\boldsymbol{r}_o, {\boldsymbol{\phi}}_d, \boldsymbol{\theta}_d)$.
    % The sampling points are first inferred through the local NeRF with the local frame locations $({\boldsymbol{x}}_l, \boldsymbol{y}_l, \boldsymbol{z}_l)$ projected from the global location $({\boldsymbol{x}}_g, \boldsymbol{y}_g, \boldsymbol{z}_g)$.
    % And then, all the local predictions are calibrated by a global MLP with conditional input to render the scene view.
    % During the optimization, the text guidance is applied to both local views predicted by local frames only and global views predicted by the composition of all local frame predictions.
    }
    \label{fig:framework}
    % \vspace{-8pt}
\end{figure*}

\subsection{Preliminaries}
Defining individual object bounding boxes as \textit{local frames} and the overall scene coordinate system as the \textit{global frame}, we build the foundation of NeRF and diffusion processes.

\label{sec:background}
\noindent \textbf{3D Representation in Latent Space.}
Our methodology capitalizes on the state-of-the-art text-to-image generative model—Stable Diffusion as described by Rombach et al\cite{rombach2022high}.
We build upon the Latent-NeRF framework~\cite{metzer2022latent}, which computes latent colors for individual objects by considering their sample positions within a localized frame. Specifically, it maps a three-dimensional point in local coordinates \(\boldsymbol{x}_l = (x_l, y_l, z_l)\) to a volumetric density \(\boldsymbol{\sigma}_l\) and an associated color \(\boldsymbol{C}_l\), expressed as \((\boldsymbol{C}_l, \boldsymbol{\sigma}_l) = f_{\boldsymbol{\theta}_l}(x_l, y_l, z_l)\). Here, \(f\) represents a Multi-Layer Perceptron (MLP) characterized by parameters \(\boldsymbol{\theta}_l\).
 This NeRF-generated color is then assessed in the context of the Stable Diffusion model, using text prompts to guide NeRF toward spatially coherent inference with intricate context.
% to infer pseudo-color for each object using local NeRF.
% Specifically, the representation maps a point $\boldsymbol{x}_l = \left({x}_l, {y}_l, {z}_l\right)\in [-1, 1]$ in the local frame to its corresponding volumetric density $\boldsymbol{\sigma}_l$ and emitted color $\boldsymbol{C}_l$, \ie,  $\left(\boldsymbol{C}_l, {\boldsymbol{\sigma}_l}\right)=\boldsymbol{\theta}_{_l}\left({x_l}, {y}_l, {z}_l\right)$.
% The predicted pseudo-color is fed forward into the decoder of the Stable Diffusion model to obtain the final rendering result.

\noindent \textbf{Volume Rendering with Multiple Objects.}
% For each local frame $j$ with NeRF parameterized as $\theta_j$, we follow original NeRF design\cite{nerf} to integrate $(\boldsymbol{C}_l, \boldsymbol{\sigma}_l)$ of   sampled points from any hit ray $r_l=(\boldsymbol{o}_l, \boldsymbol{d}_l)$ by,
% For consistent scene rendering, object transmittance $T_k$ must be recalculated in the global frame based on independent properties inferred from local NeRFs. Hence, we sort predictions according to their distance to $\boldsymbol{o}_g$. 
% Similar to \cref{eq:volrend}, global color $\hat{\boldsymbol{C}}_g$ of ray $\boldsymbol{r}_g=(\boldsymbol{o}_g, \boldsymbol{d}_g)$ is predicted by the volumetric rendering integrating over $m$ objects,
We extend the volume rendering process to accommodate multiple objects by assigning each a local frame, denoted as $j$, with NeRF parameters $\boldsymbol{\theta}_{l, j}$. Drawing from the foundational NeRF approach \cite{nerf}, in each local frame, we integrate the color $\boldsymbol{C}_l$ and density $\boldsymbol{\sigma}_l$ for points $\boldsymbol{x}_l$ sampled along a ray $\boldsymbol{r}_l$, emanates from the camera origin $\boldsymbol{o}_l$ in direction $\boldsymbol{d}_l$. This is formalized in the predicted color integration for $\hat{\boldsymbol{C}}_l$ as:
{\setlength\abovedisplayskip{2pt}
\setlength\belowdisplayskip{2pt}
\begin{equation}
\label{eq:volrend}
{\hat{\boldsymbol{C}}_l}({\boldsymbol{r}_l})=\sum_{k=1}^{N} T_{l, k} \left(1-\exp \left(-\sigma_{l, k} \delta_k\right) \right) {\boldsymbol{C}}_{l,k},
\end{equation}}where $T_{l, k}=\exp \left(-\sum_{j=1}^{k-1} \sigma_{l,j} \delta_j\right)$ represents the transmittance to the $k$-th of total $N$ sample, calculated exponentially over the cumulative density along $\boldsymbol{r}_l$, and $\delta_k$ is the interval between adjacent samples.
%
To synthesize a coherent scene, we transition from processing individual local frames to a collective global frame. Within this global context, we reconcile object attributes inferred from their individual local NeRFs for refined $\boldsymbol{\sigma}_g, \boldsymbol{C}_g$ along with $T_{g, k}$. The samples $\boldsymbol{x}_g$ are ordered based on their spatial distances from the origin $\boldsymbol{o}_g$ following the coordinate transformation. We then express the volumetric rendering of a ray $\boldsymbol{r}_g$ integrating $m$ objects within the global frame as follows:
{
\setlength\abovedisplayskip{2pt}
\setlength\belowdisplayskip{2pt}
\begin{equation}
\label{eq:multi_volrend}
{\hat{\boldsymbol{C}}_g}({\boldsymbol{r}_g})=\sum_{k=1}^{m*N} T_{g, k} \left(1-\exp \left(-\sigma_{g, k} \delta_k\right) \right) {\boldsymbol{C}}_{g,k}. 
\end{equation}}

\noindent \textbf{Score Distillation Sampling.}
% During the SDS process, a noise image $\boldsymbol{X}_t$ is first generated by adding a sampled noise $\epsilon \sim \mathcal{N}(0, I)$ in noise level $t$ into a rendered view $\boldsymbol{X}$ from a NeRF.
To facilitate the conversion from text descriptions to 3D models, DreamFusion~\cite{poole2022dreamfusion} utilizes Score Distillation Sampling (SDS), leveraging the generative capabilities of a diffusion model, denoted as $\phi$, to guide the optimization of NeRF parameters, symbolized as $\boldsymbol{\theta}$.
%
Initially, SDS creates a noisy image $\boldsymbol{X}_t$ by infusing a randomly sampled noise $\epsilon$, which follows a normal distribution $\mathcal{N}(0, I)$, into a NeRF-rendered image $\boldsymbol{X}$ at a given noise level $t$.
The diffusion model $\phi$ then estimates the noise $\epsilon_\phi\left(\boldsymbol{X}_t, t, T\right)$ from this noisy image, conditioned by the noise level $t$ and an optional text prompt $T$. 
The key step in SDS involves calculating the gradient of the loss function, which measures the discrepancy between the estimated noise and the originally added noise:
{\setlength\abovedisplayskip{2pt}
\setlength\belowdisplayskip{2pt}
\begin{equation}
\label{eq:sds_loss}
\nabla_\theta \mathcal{L}_{\text{SDS}}(\boldsymbol{X}_t, T)=  w(t)\left(\epsilon_\phi\left(\boldsymbol{X}_t, t, T\right)-\epsilon\right),
\end{equation}}where $w(t)$ is a weighting function that adjusts the influence of the gradient based on the noise level. 
The gradients across all rendered views direct the update of $\boldsymbol{\theta}$, ensuring that the NeRF-generated images align with the text descriptions. Additionally, we incorporate the 'perturb and average' technique from SJC for more robust $\mathcal{L}_{\text{SDS}}$. For a comprehensive understanding of these methods, the reader is directed to the detailed explanations provided in \cite{poole2022dreamfusion,wang2022score}.

%
%
% \subsection{Editable 3D Scene Layout}
% \label{ssec:layout}
% The 3D scene layout explicitly combines language structures with 3D layouts in an editable way.
% Given the input text prompt $T$, the attribute-object pairs can be easily obtained based on user control.
% Note that the text prompt indicates the multi-object text prompt by default.
% % available for free in many structured representations, such as the constituency tree.
% As shown in Fig.~\ref{fig:framework}, we can extract multiple noun phrases with their binding attributes and map these local text prompts into corresponding regions.
% Specifically, we define the scene structure with $m$ local frames, each employs a local NeRF $\boldsymbol{\theta}_l$ as representation, the local text prompt $T_{l} \subseteq{T}$ and its spatial layout with 3D boxes $\mathbf{b} = \{\mathbf{p}, \mathbf{s}\} \in  \mathbb{R}^6$ of each object entity, where $\mathbf{p}=\{p_x, p_y, p_z\}$ refers to the center point and $\mathbf{s}=\{s_x, s_y, s_z\}$ denotes the box scale. 
% \textit{Our editable 3D layout is easy to be collected and edited with its simplicity, allowing for versatile and interactive user control by modifying the box's or text's properties to define a new scene}.
% Moreover, as depicted in Fig.~\ref{fig:teaser}, each component in a 3D scene layout can be replaced or re-composited with other trained local NeRFs, which is more friendly for flexible user editions compared with using only text prompts.
% We fine-tuned the new layout by global rendering, which enables scalable re-editing.
% Each relationship $r_k \in R$ is a triplet in a <subject-predictive object> format, where a subject node is. After we generate the scene graph from the complex prompts, we can sample the closest relationship with the 2d spatial layout as the initial 3D position. fine-tuned the new layout by global rendering, which enables scalable re-editing
%
% \subsection{Scene Rendering Pipeline}
% \label{ssec:render}
% In CompoNeRF, the scene images are rendered by a ray-casting approach following the design of NeRF.
% % Each ray to be cast is generated based on the camera pose, intrinsic, and transformation.
% The camera is defined by a pinhole camera model, casting a set of rays $(\boldsymbol{r}_o, \boldsymbol{\phi}_d, {\boldsymbol{\theta}}_d)=\boldsymbol{o}+t\boldsymbol{d}$ through each pixel on the frame of size $H \times W$, where the $\boldsymbol{r}_o \in  \mathbb{R}^3$ is the origin and the $(\boldsymbol{\phi}_d, \boldsymbol{\theta}_d)$ is the viewing direction.
% Along this ray, we sample all the points intersected with any layout box of local frames.
% For each hit sampled point, the color and volumetric density are computed through the local NeRF of the hit local frame.
% The ray color perdition is calculated by the differentiable integration applied on all the point-predicted colors and volumetric density along the ray.
%
% \noindent \textbf{Ray-box Intersection with Local Frames.}
% Given a ray $\boldsymbol{r}_i$, each box $\boldsymbol{b}_j$ of the local frame is applied with the AABB ray intersection test algorithm to check the intersections.
% When the ray $r_i$ is hit with a box $\boldsymbol{b}_j$ of the local frame, we use the entrance and exit points as near $\boldsymbol{t}_{in}$ and far $\boldsymbol{t}_{out}$ bounds to sample $N$ equidistant quadrature points, $
% \boldsymbol{t}_{i,j,n}=\frac{n-1}{N-1}\left(\boldsymbol{t}_{out}-\boldsymbol{t}_{in}\right)+\boldsymbol{t}_{in} , n \in \left[1, N\right]$
% % Despite each local frame only having a small number of hit rays compared to the scene, we observe that it is enough to represent each object accurately while maintaining short rendering times.
% Note that the coordinates of sampled points are first projected into normalized coordinates using the box scale of local frames to enable each local NeRF to learn the scale-independent representation.
% The bounding box $\mathbf{b}$ of the local frame in global coordinate can be transformed into a canonical bounding box by ${(\mathbf{b}} - \boldsymbol{p}) / \mathbf{s}$.
% Considering the rendering efficiency, we only calculate the valid points, interacted with the boxes, and set all the empty points with a constant background color.
%
% The appearance of a set object representations depends on its interaction with the scene and illumination which should be decided by the local frame location.
% To ensure the volumetric consistency, we only calibrate the emitted color with scene location, while the gradient still can be propagated.
% Since the overall color depends on both the global  positions $({x}_w, {y}_w, {z}_w)$ and ray directions $({\phi}_d, {\theta}_d)$, the global color embedding is learned based on both the positions and ray directions.
% Since the overall color depends on both the global  positions $({x}_w, {y}_w, {z}_w)$ and ray directions $({\phi}_d, {\theta}_d)$, the global color embedding is learned based on both the positions and ray directions.
% \subsection{The Proposed CompoNeRF}
% \subsubsection{Composition Module}
% CompoNeRF aims to composite multiple NeRFs to reconstruct multi-object scenes with both box and prompt guidance.
% %
% Our framework, as shown in \cref{fig:framework}, applies the AABB ray intersection test algorithm to check for intersections on each box in the global frame. We then samples $\boldsymbol{x}_g$ within the ray box intervals, and project them to $\boldsymbol{x}_l$ to infer  $\left(\boldsymbol{C}_l, {\boldsymbol{\sigma}_l}\right)$ in separate NeRF models. 
% %
% We then utilize volume rendering to obtain rendered views for each local frame respectively. 
% %
% After that, they would be passed on to our tailored composition Module to infer 
% $\left(\boldsymbol{C}_g, {\boldsymbol{\sigma}_g}\right)$
% for global rendering. 
% Next, we match local and global texts with their corresponding image outputs by SDS losses. 
% We also support recomposition by passing samples from cached models into $\boldsymbol{x}_l$ to continue the above process.
\begin{figure}[t!]
    \centering
    \includegraphics[width=\linewidth]{figures/abls.pdf}
    % \vspace{-22pt}
    % \caption{Ablation study on text guidance. (a) without local SDS losses. (b) without global SDS losses. (c) vanilla SDS losses without perturb and average scoring~\cite{wang2022score}. (d) full model.}
    \caption{\textbf{Design Impact Comparison: Density vs. Color-based Methods.} The top row illustrates the density-based approach's detailed rendering and quick convergence in the 'table wine' scene. The bottom row highlights the color-based method's enhancements and its drawbacks, such as geometric and shadow inaccuracies, particularly in close-up views and slow convergence.
    % \textbf{(a)} global text guidance(integrating local frames by \cref{eq:multi_volrend}) and global calibration(integrating local frames, then aligning the rendering result directly with the full text). 
    }
    \label{fig:abls}
    % \vspace{-20pt}
\end{figure}
\subsection{The Proposed CompoNeRF}
\subsubsection{Composition Module}
CompoNeRF is designed to composite multiple NeRFs to reconstruct scenes featuring multiple objects, utilizing guidance from both bounding boxes and textual prompts. Within our framework, depicted in \cref{fig:framework}, the Axis-Aligned Bounding Box (AABB) ray intersection test algorithm is applied to ascertain intersections across each box in the global frame. Subsequently, we sample points \(\boldsymbol{x}_g\) within the intervals of the ray-box and project them to \(\boldsymbol{x}_l\) to deduce the corresponding color \(\boldsymbol{C}_l\) and density \(\boldsymbol{\sigma}_l\) within individual NeRF models.
%
These properties are processed through our composition module to infer the global color \(\boldsymbol{C}_g\) and density \(\boldsymbol{\sigma}_g\), crucial for the global rendering.
%
Volume rendering techniques~\cite{kajiya1984ray} are then employed to procure the rendered views for both local and global frames. We propose dual SDS losses to ensure coherence between the image outputs and their corresponding textual descriptions. Additionally, our approach facilitates recomposition by channeling samples from cached models back into local frames along with the text revision, thereby streamlining the integration.

% As shown in \cref{fig:abls}(a), we verify its necessity by dropping $\nabla \mathcal{L}_{\text{SDS}_g}$. 
% %
% Compared with our full model, its layout does not fit our shared sense of a room, \ie, \emph{nightstand} is usually lower than \emph{bed}; \emph{lamp} needs a base to support it. Additionally,  it lacks global consistency, such as light reflection, to make it more realistic. 
% %
% Therefore, we leverage the full text semantics to ensure consistent global rendering across local frames. 
% %
% Instead of conditioning the global rendering view with the full prompt directly, we note that global calibration is necessary for geometry and color to be learned sufficiently.
% For example, we observe that geometric completeness and texture of \emph{nightstand} are not ideal. Although reflection appears around \emph{nightstand}, \emph{bed} is stripped of the light. 
% %
% Therefore, we opt to leverage the correlation between the rendering output of the combined NeRFs and the overall semantics to perform multi-object scene reconstruction.  
%

\noindent\textbf{Global Composition.}
The independent optimization of each local frame may inadvertently result in a lack of global coherence within the scene. To address this, our scene composition process is designed to integrate these frames, thereby achieving a more consistent result.
%
Before exploring the specifics of the module, it is imperative to discuss two critical design decisions within the composition module, as depicted in \cref{fig:framework}.
%
Upon integrating the properties inferred from \(\boldsymbol{x}_g\) into the composition module, they are fine-tuned through gradients derived from the global SDS loss.  This process leads to a critical consideration: the necessity and implications of refining the global density \(\boldsymbol{\sigma}_g\). This can be divided into two approaches: \textbf{1) Density-based:} The advantage of adjusting \(\boldsymbol{\sigma}_g\) is that it can adjust geometry, thus yielding a scene more congruent with the global text prompt. 
However, this comes at the cost of potentially compromising the optimal color \(\boldsymbol{C}_g\), as calibrating \(\boldsymbol{\sigma}_g\) introduces more uncertainty for subsequent color refinement as it requires prior density features $\boldsymbol{h}$ as shown at \cref{fig:compo}. 
\textbf{2) Color-based:} Conversely, directly employing \(\boldsymbol{\sigma}_l\) mitigates this uncertainty but at the expense of reduced geometric control, presenting a challenging balance to strike in the pursuit of precise scene composition.
% , which may lead to suboptimal outcomes.
%
After thorough experiments, exemplified in \cref{fig:abls}, we have opted for the density-based approach to refine \(\boldsymbol{\sigma}_g\)  prioritizing both \textbf{accuracy and efficiency}. The test revealed that it excels in rendering intricate details, such as enhanced wood grain textures and more naturally contoured 'salad', as accentuated by boxes. This method also demonstrated a swifter convergence rate. Conversely, while the color-based improved reflections and reduced flickering on the 'wine cup', it was plagued by issues such as sparse density, which adversely brings holes at the base of the 'cup' and the corner of the 'table'.
Furthermore, upon close examination, it becomes evident that shadow artifacts of 'wine' on the 'table' are pronounced, suggesting that its disadvantages outweigh its advantages.
%  in this context
% \textbf{Global Composition.}
% Each local frame is optimized independently, causing a lack of global connections for scene composition.
% Before delving into module details, there are two choices (see \cref{fig:framework}) on the composition module design we need to elaborate on first. 
% %
% In \cref{fig:framework}, by taking $\boldsymbol{x}_g$ into the composition module, their inferred properties are calibrated with gradients propagated from the global SDS loss. 
% However, it remains unclear whether $\boldsymbol{\sigma}_g$ should be refined or not. 
% %
% The trade-off on its usage is the density adjustment bringing a more reasonable layout and more geometric details that fit the global text prompt. While its potential downside is that $\boldsymbol{C}_g$ may not be optimal as $\boldsymbol{\sigma}_g$ has more uncertainty compared to $\boldsymbol{\sigma}_l$, bringing sub-optimal rendering results. 

% We choose the density-based method after comparing them with the experiment shown in \cref{fig:abls}. 
% %
% Specifically, we test both designs on the scene \emph{table wine} and discover that the density-based design provides more intrinsic details(as indicated by green boxes), \eg, enriched wood grains, and a more natural shape for \emph{salad} and has much faster convergence speed. In contrast, the color-based method enhances the reflection and smooths flickering on \emph{wine cup}, (as indicated by red boxes), but it suffers from 1) sparse density, resulting in poorly generated geometry at the base of  \emph{cup} and the wood \emph{table} corner. Additionally, shadow artifacts appeared on \emph{table} when viewed up close, outweighing benefits of the color-based method.

\begin{figure}[t!]
    \centering
    \includegraphics[width=\linewidth]{figures/compo_module.pdf}
    % \vspace{-24pt}
    % \caption{Ablation study on text guidance. (a) without local SDS losses. (b) without global SDS losses. (c) vanilla SDS losses without perturb and average scoring~\cite{wang2022score}. (d) full model.}
    \caption{\textbf{Detail of Composition module}: density-based design. 
    }
    \label{fig:compo}
    % \vspace{-18pt}
\end{figure}
\noindent\textbf{Network Design.}
The compositional framework of our network, as delineated in \cref{fig:compo}, is predicated on an architecture that employs a suite of MLPs, represented as \(\{\boldsymbol{\theta}_l\}_{l=1}^{m}\),  each dedicated to a distinct local frame. To harmonize \(\boldsymbol{\sigma}_l\) and \(\boldsymbol{C}_l\), we incorporate global MLPs, including density calibrator $f_{\boldsymbol{\theta}_{g_d}}$ and color calibrator $f_{\boldsymbol{\theta}_{g_c}}$.
%
A transformation module complements this system, tasked with maintaining the spatial coherence between the global and local frames. It governs the transformation of sampling points $\boldsymbol{x}$, ray directions $\boldsymbol{d}$, and adjacent sampling distances $\delta$. This module also orders the points $\{\boldsymbol{x}_{g,j}\}_j$ by their distance to the global camera origin $\boldsymbol{o}_g$, ensuring that each local point $\boldsymbol{x}_l$ is accurately matched with its corresponding global point $\boldsymbol{x}_g$ for subsequent volume rendering. 
%
The network design is:
{
\setlength\abovedisplayskip{4.5pt}
\setlength\belowdisplayskip{4.5pt}
\begin{align}
\label{eq:g_c_d}
{\boldsymbol{\sigma}_g}  &= \alpha_d f_{\boldsymbol{\theta}_{g_d}}({\boldsymbol{x}_g}) + \boldsymbol{\sigma}_l, \\
{\boldsymbol{C}_g}  &= \alpha_c f_{\boldsymbol{\theta}_{g_c}}(\boldsymbol{h}, {\boldsymbol{d}_g}) + \boldsymbol{C}_l. 
\end{align}}In contrast to the local frames, the global frame's color output $\boldsymbol{C}_g$ is inferred based on $\boldsymbol{h}$ and conditional on $\boldsymbol{d}_g$ to enable a view-dependent lighting effect.
% Denote the density features as $\boldsymbol{h}$. 
%
%
Residual learning is leveraged here, where \(\boldsymbol{\sigma}_l, \boldsymbol{C}_l\) serve as foundational elements that support the learning of global density \(\boldsymbol{\sigma}_g\) and color \(\boldsymbol{C}_g\). The parameters \(\alpha_d, \alpha_c\) are adjustable, allowing fine-tuning of the influence that local components exert on the global outputs.
%
It is imperative to acknowledge that in our color-based method, density calibration is intentionally excluded to concentrate solely on the refinement of color dynamics as shown at \cref{fig:framework}. This is achieved by conditioning the process on both spatial and directional global inputs \((\boldsymbol{x}_g, \boldsymbol{d}_g)\), as demonstrated in the following equations:
\begin{align}
\setlength\abovedisplayskip{4.5pt}
\setlength\belowdisplayskip{4.5pt}
\label{eq:g_c_c}
\boldsymbol{\sigma}_g = \boldsymbol{\sigma}_l, \quad
{\boldsymbol{C}_g} = \alpha_c f_{\boldsymbol{\theta}_{g_c}}({\boldsymbol{x}_g}, {\boldsymbol{d}_g}) + \boldsymbol{C}_l.
\end{align}
The integration of extra $\boldsymbol{x}_g$ aims to facilitate a fair comparison under same inputs with the density-based. It enhances the visual appeal of effects like the wine cup's reflection, as demonstrated in \cref{fig:abls}. However, this method is not without its compromises. It tends to produce artifacts and is characterized by a slower convergence rate. Additionally, this approach limits the ability to precisely control density, subsequently impacting the intricate geometric details.


\begin{figure*}[t!]
    \centering
    \includegraphics[width=\linewidth]{figures/sota.pdf}
    % \vspace{-24pt}
    \caption{\textbf{Qualitative comparison with other text-to-3D methods using multi-object text prompts}. Cases 1-3 demonstrate simpler settings characterized by compositions involving two objects. In contrast, Cases 4-8 delve into more intricate scenarios featuring compositions with more than two objects. Smaller images are presented to illustrate the generated local NeRFs(partially shown in Cases 4-8).}
    \label{fig:sota}
    % \vspace{-5pt}
\end{figure*}
%
% \begin{table*}[t!]
% \centering
% \resizebox{\textwidth}{!}
% {
% \begin{tabular}{cccccccc}
% \toprule
% Method            & \rotatebox{60}{table wine}  & \rotatebox{60}{teddy monkey} & \rotatebox{60}{computer mouse} & \rotatebox{60}{bed room}  & \rotatebox{60}{chess} & \rotatebox{60}{pisa tower} & \rotatebox{60}{astronaut} & \rotatebox{60}{tesla}  \\ \midrule
% LatentNeRF  & 21.55 & 27.38 & 17.13 & 21.86 & 31.19 & 24.31 & 27.07 & 25.16 \\
% SJC & 23.33 & 27.37 & 18.00 & 22.54 & 30.53 & \textbf{26.18 }& 27.84 & 23.55 \\
% CompoNeRF & \textbf{32.68} & \textbf{28.57}	 &\textbf{ 22.34} &\textbf{ 28.65} & \textbf{31.45} & \textbf{28.96} & 25.82 & 25.95 & 24.42 & \textbf{32.71} & \textbf{26.13 }& \textbf{26.38} & \textbf{30.98} & \textbf{33.37} \\
% \bottomrule
% \end{tabular}
% }
% \vspace{-10pt}
% \caption{Performance of our CompoNeRF in different 3D scenes. We use CLIP score \cite{parmar2023zero,zhang2023sine,wang2023imagen} as our evaluation metric, which is a common evaluation metric in text-to-image generation tasks to evaluate the similarity of the generated image to the text prompt. }
% \label{perclass}
% \end{table*}
%
\begin{table*}[t!]
% \scalebox{0.8}
\renewcommand{\arraystretch}{1.2}
\fontsize{4pt}{4pt}
\selectfont 
\centering
% \vspace{-8pt}
\resizebox{\textwidth}{!}
{
% \begin{tabular}{lcccccccc}
% \hline
% Method     & table\_wine    & tesla          & pyramid        & chess          & apple and banana      & astronaut      & glass\_balls   & Eiffel\_tower    \\ \hline
% LatentNeRF & 21.55          & 25.16          & 27.43          & 31.19          & 27.69          & 27.07          & 29.51          & 26.32          \\
% SJC        & 23.33          & 23.55          & 25.62          & 30.53          & 28.21          & 27.84          & 28.76          &27.41 \\
% \textbf{CompoNeRF(Ours)}     & \textbf{32.68} & \textbf{26.13} & \textbf{28.96} & \textbf{31.45} & \textbf{33.37} & \textbf{32.71} & \textbf{30.98} & \textbf{28.44}          \\ \hline
% \end{tabular}
\begin{tabular}{lcccccccc}
\hline
Method                   & Case 1         & Case 2         & Case 3         & Case 4         & Case 5         & Case 6         & Case 7         & Case 8         \\ 
\hlineB{1.1}
LatentNeRF               & 25.16          & 27.07          & 27.69          & 31.19          & 21.55          & 26.32          & 27.43          & 29.51          \\
SJC                      & 23.55          & 27.84          & 28.21          & 30.53          & 23.33          & 27.41          & 25.62          & 28.76          \\
\textbf{CompoNeRF (Ours)} & \textbf{26.13} & \textbf{32.71} & \textbf{33.37} & \textbf{31.45} & \textbf{36.06} & \textbf{28.44} & \textbf{28.96} & \textbf{30.98} \\ \hlineB{1.1}
\end{tabular}
}

% \vspace{-6pt}
\caption{\textbf{Performance comparison of our CompoNeRF in different 3D scenes}. For our evaluation metric, we utilize the average of CLIP scores~\cite{parmar2023zero,zhang2023sine,wang2023imagen} across different views, which serve to assess the similarity between the generated images and the global text prompt. }
\label{tb:perclass}
\end{table*}
% \cref{fig:framework} depicts the network architecture of the composition module. Denote $m$ as local MLP $\{\boldsymbol{\theta}_l\}_{l=1}^{m}$ for each local frame. Then, we introduce the global MLPs including density $\boldsymbol{\theta}_{g_d}$ and $\boldsymbol{\theta}_{g_c}$ calibrators to refine $\boldsymbol{\sigma}_l$ and $\boldsymbol{C}_l$. 
% %
% In detail, the network design is, 
% {
% % \setlength\abovedisplayskip{4.5pt}
% % \setlength\belowdisplayskip{4.5pt}
% \begin{align}
% \label{eq:g_c_d}
% {\boldsymbol{\sigma}_g}  &= \alpha_d \boldsymbol{\theta}_{g_d}({\boldsymbol{\sigma}_l}) + \boldsymbol{\sigma}_l, \\  
% {\boldsymbol{C}_g}  &= \alpha_c \boldsymbol{\theta}_{g_c}({\boldsymbol{C}_l},  {\boldsymbol{d}_g}) + \boldsymbol{C}_l, 
% \end{align}}
% %
% where residual $\boldsymbol{\sigma}_l, \boldsymbol{C}_l$ assist in learning $\boldsymbol{\sigma}_g$ and $\boldsymbol{C}_g$, while $\alpha_d, \alpha_c$ balance their contribution as learnable parameters.
% %
% Note that the color-based omits density calibration, and simply uses the shared color refinement.



% The 3D boxes are only used for the spatial configuration of local NeRFs, while the implicit representation of local NeRFs is inferred by the canonical samples inside the local frame without considering the global relationship across different objects.
% To relieve such location-dependent effects, we further calibrate the output color and density from the local NeRF with global coordinates $({\boldsymbol{x}}_g, {\boldsymbol{y}}_g, {\boldsymbol{z}}_g)$ and ray directions $\left({\boldsymbol{\phi}}_{d}, {\boldsymbol{\theta}}_{d}\right)$ as the conditional input.
% % to inject the global visual clues.
% %
% %
% Specifically, we adopt a shared MLP $\boldsymbol{\theta}_{g}$ to calibrate all the predicted object colors, that is,
% {\setlength\abovedisplayskip{4.5pt}
% \setlength\belowdisplayskip{4.5pt}
% \begin{align}
% \label{eq:MLP_dyn_2}
% {\boldsymbol{C}_g} = {\boldsymbol{C}_l} + \boldsymbol{emb}_{g} &= {\boldsymbol{C}_l} + \boldsymbol{\theta}_{g}({\boldsymbol{x}}_g, {\boldsymbol{y}}_g, {\boldsymbol{z}}_g, {\boldsymbol{\phi}}_{d}, {\boldsymbol{\theta}}_{d}),
% \end{align}}
% where ${\boldsymbol{C}_l}$ is the color predicted by the local NeRF.
% Therefore, the scene color can preserve the view-consistent behavior from the original architecture and add consistency across poses for the volumetric density.
% Since the color and density values share the same latent expression in $({\boldsymbol{x}}_l, {\boldsymbol{y}}_l, {\boldsymbol{z}}_l)$, we only calibrate the emitted scene color explicitly with the scene location, as the densities of local NeRFs also are implicitly adjusted during optimization.

% \noindent \textbf{Global and Local Volumetric Rendering.}
% After compositing all the interacted points, each ray $\boldsymbol{r}_i$ collects a set sampling points by $\{\boldsymbol{t}_{i,j,n} \}_{j=1, n=1}^{m_j, N}$, where $m_j$ is the number of the hit object.
% For each sampling point, the inference results with the respective 3D representations are the local color $\boldsymbol{c}_{l}$, global color $\boldsymbol{c}_{g}$, and density $\sigma$.

% In fact, the local view $\hat{C}_{l,j}$ of single object $j$ also can be rendered by the sampled points  belongs to the same local frames as shown at Fig.~\ref{fig:framework}.

\subsubsection{Recomposition}
Our architecture advances scene reconstruction by providing an intuitive interface for layout manipulation.  This capability is crucial for the reconfiguration of scene elements into novel scenes, as depicted in \cref{fig:framework}. Here, the input panel allows for adjustments in the attributes of bounding boxes, such as modifying the position and scale of the 'apple' bounding box prior to composition. The refinement process further involves sampling ray-box intervals from the global frame, leading to transformed coordinates with the corresponding ray samples that are then incorporated into the pipeline, as demonstrated in \cref{fig:compo}.
%
Each bounding box represents an individual NeRF, providing the flexibility to move, scale, or remove elements as needed. CompoNeRF's capabilities also extend to textual edits, exemplified by the transformation of 'wine' into 'juice'.
%
Since NeRFs have been well trained, we only finetune \(\theta_g, \theta_l\) to align text prompts to promote consistency of both local and global views.
%
Moreover, the NeRFs once retrained within the edited scene, are also structured to be decomposable and cacheable in future scene compositions.
% Our CompoNeRF architecture facilitates the seamless reconstruction of scenes leveraging existing models. It enables precise editing of bounding boxes parameterized by \(\{\boldsymbol{\theta}_l\}_{l=1}^{m}\), allowing for their reconfiguration into new layouts. Refer to \cref{fig:framework}, the input panel permits the modification of attributes such as the position and scale of the 'apple' node's bounding box prior to composition. The process is further refined by sampling from the updated ray-box intervals within the global frame, which are then projected onto \(\boldsymbol{x}_l\), ensuring a streamlined reconstruction that integrates the 'apple' effectively. This addition is executed with careful attention to color consistency, positioning the 'apple' adjacent to the 'French bread' to complement the scene's overall palette. Each bounding box represents an individual NeRF, which means they can be manipulated through moving, scaling, and removal operations. CompoNeRF also extends its editing prowess to textual modifications, as evidenced by the 'wine cup' now appearing filled with juice—a change propagated through both subtexts and the global test. 
% %
% Since NeRFs have been well trained, we only finetune $\theta_g, \theta_l$ to align text prompts to promote consistency of both local and global views . 
% %
% Moreover, the NeRFs, once retrained within the reimagined scene, are also structured to be decomposable and cacheable for subsequent scene compositions.

% , as shown in Fig.~\ref{fig:framework}.
% For each scene described by the multi-object text prompt $T$, we
% To enhance the guidance of local representations, we use the local text prompt $T_l \subseteq T$ of a single object to optimize the local NeRFs in local views.
% The scene views $\hat{\boldsymbol{X}}_g=\{\hat{\boldsymbol{C}}_{g,i}\}_{i=1}^{H\times W}$ is obtained from the predicted pixel values of $H \times W$ rays by compositing all the ray-box interaction values.
% Similarly, the rendered view $\hat{\boldsymbol{X}}_{l,j}$ of the local frame $\boldsymbol{\theta}_j$ without compositing other objects can be calculated by $\hat{\boldsymbol{C}}_{l,j}$, as depicted in Sec.~\ref{ssec:render}.
% We use the local color instead of the globally calibrated color to obtain a local view because the local NeRF should learn the object identity unrelated to its placed position, as the position can be different during user edition.
% % Compared to cropping the local region from a global view for training, separate rendering can avoid the undesired information from other objects brought by the occlusion and resolution adjustments.
% Formally, we employ the following loss as the learning objective,
\begin{figure*}[t!]
    \centering
    \includegraphics[width=\linewidth]{figures/editing.pdf}
    % \vspace{-23pt}
    \caption{\textbf{Scene Editing Outcome:} Demonstrated here are the stages of our recomposition, utilizing cached source scenes. Each NeRF is individually identified by colorful labels. These decomposed nodes are then positioned in the initial layout and subsequently calibrated to form the final composition. The detailed description of the ambient environment is underscored, enhancing the scene's realism.}
    \label{fig:app}
    % \vspace{-12pt}
\end{figure*} 
\subsubsection{Optimization}
\label{sec:optimization}
During optimization, our method employs dual text guidance to align rendering results with both global and local textual descriptions. The optimization objective is:
{
\small
\setlength\abovedisplayskip{2pt}
\setlength\belowdisplayskip{2pt}
\begin{equation}
\label{eqn:loss_f}
\mathcal{L}= {\alpha_g}\nabla\mathcal{L}_{\text{SDS}}(\hat{\boldsymbol{X}}_{g}, T) + {\alpha_l}\sum_{j=1}^{m} \nabla\mathcal{L}_{\text{SDS}}(\hat{\boldsymbol{X}}_{l,j}, T_{l,j}) + \beta\mathcal{L}_{\text{sparse}},\nonumber
\end{equation}
}where $T$ signifies the global text prompt, while $T_{l}$ pertains to a specific object within the global context. The hyperparameters $\alpha_{g}, \alpha_{l}$, and $\beta$ modulate the respective loss weights. 
% $\nabla \mathcal{L}_{\text{SDS}}$ is the score distillation sampling loss, as described in Sec.~\ref{sec:background}.
As suggested in~\cite{metzer2022latent}, we use $L_{\text{sparse}}$ included to penalize the binary entropy of local NeRFs' densities, thereby mitigating the issue of extraneous floating radiance.
Additionally, incorporating directional cues such as "front view" or "side view" into the input text, as suggested by \cite{poole2022dreamfusion,metzer2022latent} proves beneficial in specifying camera poses during the training phase, further enhancing the alignment of our generated scenes with the intended perspectives.
% Note that the global calibration in the scene frame can adaptively revise both $({C}_l, {\sigma})$ in local NeRF with $\nabla \mathcal{L}_{SDS}$ along with the back-propagating gradient.


\section{Experiments}
\section{Experimental Setup}
\label{sec:experiments}
\begin{figure}[t]
    \centering 
    \hspace{-.04\columnwidth}
    \includegraphics[width=1.025\columnwidth]{results/VOC/figures/pareto_example.pdf}
    \caption{\textbf{Selecting models for evaluation.} For each configuration, we evaluate every model at every checkpoint and measure its performance across various metrics (\fone, \epg, \iou) on the validation set; \ie every point in the left graph corresponds to one model (for \bcos models optimized via the \epgloss loss at the input layer). Instead of evaluating a single model on the test set, we evaluate \emph{all Pareto-dominant} models, as indicated in the center and right plot.
    % \moritz{Did we not update the results to be consistent with this? I distinctly remember creating the plots for this. (The Pareto front here as a lot more points than those in the result figures...)}
    }
    \label{fig:pareto_example}
\end{figure}

In this section, we describe our experimental setup
and how we select the best models across metrics. {Full training details can be found in the supplement.} We evaluate across the full sweep of combinations of choices for each category, and discuss our results in \cref{sec:results}. 

\myparagraph{Datasets:} We evaluate on \voc \citeMain{everingham2009pascal} and \coco \citeMain{lin2014microsoft} for multi-label image classification. {In \cref{sec:results:waterbirds}, to understand the effectiveness of model guidance in mitigating spurious correlations, we also evaluate on the synthetically constructed Waterbirds-100 dataset \citeMain{sagawa2019distributionally,petryk2022guiding}, where landbirds are perfectly correlated with land backgrounds on the training and validation sets, but are equally likely to occur on land or water in the test set (similar for waterbirds and water). With this dataset, we evaluate model guidance for suppressing undesired features.}

\myparagraph{Attribution Methods and Architectures:} As described in \cref{sec:method:attributions}, we evaluate with \ixg \citeMain{shrikumar2017learning}, \intgrad \citeMain{sundararajan2017axiomatic}, \bcos \citeMain{bohle2022b}, and \gradcam \citeMain{selvaraju2017grad} using models with a \resnet \citeMain{he2016deep} backbone. For \intgrad, we use an \xdnn \resnet \citeMain{hesse2021fast} to reduce the computational cost, and a \bcos \resnet for the \bcos attributions. We optimize the attributions at the input and final layer\footnote{As typically used in \ixg (input) and \gradcam (final) respectively.}; for intermediate layer results, see supplement. Given the similarity of the results between \gradcam and \ixg, and since \bcos attributions performed better than \gradcam for \bcos models, we show \gradcam results in the supplement. 
All models were pretrained on \imagenet \citeMain{imagenet}, and model guidance was performed starting from a baseline model fine-tuned on the target dataset.

\myparagraph{Localization Losses:} As described in \cref{sec:method:losses}, we compare four localization losses in our evaluation: (i) \energyloss, (ii) \loneloss \citeMain{gao2022aligning,gao2022res}, (iii) \ppceloss \citeMain{shen2021human}, and (iv) \rrrloss (cf.~\cref{sec:method:losses}, \citeMain{ross2017right}).

\myparagraph{Evaluation Metrics:} As discussed in \cref{sec:method:metrics}, we evaluate both for classification and localization performance of the models. For classification, we report the F1 scores, similar results with \map scores can be found in the supplement. For localization, we evaluate using the \epg and \iou scores.

\myparagraph{Selecting the best models:} As we evaluate for two distinct objectives (classification and localization), it is non-trivial to decide which models to select during training. \Eg, a model that provides the best classification performance might provide significantly worse localization performance than a model that provides slightly lower classification performance but much better localization. Finding the right balance and deciding which of those models in fact constitutes the `better' model depends on the preference of the end user. 
Hence, instead of selecting models based on a single metric, we select the set of Pareto-dominant models \citeMain{pareto1894massimo,pareto2008maximum,backhaus1980pareto} across three metrics---F1, \epg, and \iou---for each training configuration, as defined by a combination of attribution method, layer, and loss. Specifically, as shown in \cref{fig:pareto_example}, we train for each configuration using three different choices of $\lambda_\text{loc}$, and select the set of Pareto-dominant models among all checkpoints (epochs and $\lambda_\text{loc}$). This provides a more holistic view of the general trends on the effectiveness of model guidance for each configuration.

\section{Discussion and Conclusions}
\section{Discussion and Limitations}

Although we can ablate concepts efficiently for a wide range of object instances, styles, and memorized images, our method is still limited in several ways. First, while our method overwrites a target concept, this does not guarantee that the target concept cannot be generated through a different, distant text prompt. We show an example in \reffig{limitation} (a), where after ablating {\menlo Van Gogh}, the model can still generate {\menlo starry night painting}. However, upon discovery, one can resolve this by explicitly ablating the target concept {\menlo starry night painting}. Secondly, when ablating a target concept, we still sometimes observe slight degradation in its surrounding concepts, as shown in \reffig{limitation} (c). 

\nupur{Our method does not prevent a downstream user with full access to model weights from re-introducing the ablated concept~\cite{ruiz2022dreambooth,kumari2022multi,gal2022image}. Even without access to the model weights, one may be able to iteratively optimize for a text prompt with a particular target concept. Though that may be much more difficult than optimizing the model weights, our work does not guarantee that this is impossible.}

Nevertheless, we believe every creator should have an ``opt-out'' capability. We take a small step towards this goal, creating a computational tool to remove copyrighted images and artworks from large-scale image generative models.
\\

\noindent\textbf{Acknowledgements.} M.J. is supported by the Ministry of Trade, Industry, and Energy in Korea, under Human Resource Development Program for Industrial Innovation (Global) (P0017311) supervised by the Korea Institute for Advancement of Technology. H.C. is supported by NIH DP5 OD029574-01 and by the Schmidt Fellows Program at Broad Institute.

% ---- Bibliography ----
%
% BibTeX users should specify bibliography style 'splncs04'.
% References will then be sorted and formatted in the correct style.
%
\bibliographystyle{splncs04}
% \bibliography{egbib.bbl}
\begin{thebibliography}{10}
\providecommand{\url}[1]{\texttt{#1}}
\providecommand{\urlprefix}{URL }
\providecommand{\doi}[1]{https://doi.org/#1}

\bibitem{Abramian19}
Abramian, D., Eklund, A.: Refacing: Reconstructing anonymized facial features
  using {GANS}. In: 2019 IEEE 16th International Symposium on Biomedical
  Imaging (ISBI 2019) (2019)

\bibitem{Alaluf21}
Alaluf, Y., Patashnik, O., Cohen-Or, D.: Restyle: A residual-based stylegan
  encoder via iterative refinement. In: Proceedings of the IEEE/CVF
  International Conference on Computer Vision (2021)

\bibitem{Bischoff07}
Bischoff-Grethe, A., Ozyurt, I.B., Busa, E., Quinn, B.T., Fennema-Notestine,
  C., Clark, C.P., Morris, S., Bondi, M.W., Jernigan, T.L., Dale, A.M., et~al.:
  A technique for the deidentification of structural brain mr images. Human
  brain mapping  (2007)

\bibitem{Burlina22}
Burlina, P., Paul, W., Liu, T.Y.A., Bressler, N.M.: {Detecting Anomalies in
  Retinal Diseases Using Generative, Discriminative, and Self-supervised Deep
  Learning}. JAMA Ophthalmology  (2022)

\bibitem{Burlina19}
Burlina, P.M., Joshi, N., Pacheco, K.D., Liu, T.Y.A., Bressler, N.M.:
  {Assessment of Deep Generative Models for High-Resolution Synthetic Retinal
  Image Generation of Age-Related Macular Degeneration}. JAMA Ophthalmology
  (2019)

\bibitem{Chen21}
Chen, Y., Long, J., Guo, J.: Rf-gans: A method to synthesize retinal fundus
  images based on generative adversarial network. Computational intelligence
  and neuroscience  (2021)

\bibitem{Coyner20}
Coyner, A.S., Chen, J., Campbell, J.P., Ostmo, S., Singh, P., Kalpathy-Cramer,
  J., Chiang, M.F.: {Diagnosability of Synthetic Retinal Fundus Images for Plus
  Disease Detection in Retinopathy of Prematurity.} AMIA Symposium  (2020)

\bibitem{Dwork14}
Dwork, C., Roth, A., et~al.: The algorithmic foundations of differential
  privacy. Found. Trends Theor. Comput. Sci.  (2014)

\bibitem{Fei09}
Fei-Fei, L., Deng, J., Li, K.: Imagenet: Constructing a large-scale image
  database. Journal of vision  (2009)

\bibitem{Garfinkel15}
Garfinkel, S., et~al.: De-identification of Personal Information:. US
  Department of Commerce, National Institute of Standards and Technology (2015)

\bibitem{Gatys16}
Gatys, L.A., Ecker, A.S., Bethge, M.: Image style transfer using convolutional
  neural networks. In: Proceedings of the IEEE conference on computer vision
  and pattern recognition (2016)

\bibitem{Gkoulalas14}
Gkoulalas-Divanis, A., Loukides, G., Sun, J.: Publishing data from electronic
  health records while preserving privacy: A survey of algorithms. Journal of
  biomedical informatics  (2014)

\bibitem{Goodfellow16}
Goodfellow, I.: Nips 2016 tutorial: Generative adversarial networks. arXiv
  preprint arXiv:1701.00160  (2016)

\bibitem{Der21}
der Goten, V., Alexander, L., Hepp, T., Akata, Z., Smith, K.: Conditional
  de-identification of 3d magnetic resonance images. arXiv preprint
  arXiv:2110.09927  (2021)

\bibitem{Gui21}
Gui, J., Sun, Z., Wen, Y., Tao, D., Ye, J.: A review on generative adversarial
  networks: Algorithms, theory, and applications. IEEE Transactions on
  Knowledge and Data Engineering  (2021)

\bibitem{He20}
He, K., Fan, H., Wu, Y., Xie, S., Girshick, R.: Momentum contrast for
  unsupervised visual representation learning. In: Proceedings of the IEEE/CVF
  conference on computer vision and pattern recognition (2020)

\bibitem{He16}
He, K., Zhang, X., Ren, S., Sun, J.: Deep residual learning for image
  recognition. In: Proceedings of the IEEE conference on computer vision and
  pattern recognition (2016)

\bibitem{he2016deep}
He, K., Zhang, X., Ren, S., Sun, J.: Deep residual learning for image
  recognition. In: Proceedings of the IEEE conference on computer vision and
  pattern recognition (2016)

\bibitem{Heusel17}
Heusel, M., Ramsauer, H., Unterthiner, T., Nessler, B., Hochreiter, S.: Gans
  trained by a two time-scale update rule converge to a local nash equilibrium.
  Advances in neural information processing systems  (2017)

\bibitem{Jakob20}
Jakob, C.E., Kohlmayer, F., Meurers, T., Vehreschild, J.J., Prasser, F.: Design
  and evaluation of a data anonymization pipeline to promote open science on
  covid-19. Scientific data  (2020)

\bibitem{Jeong21}
Jeong, Y., Choi, J., Kim, S., Ro, Y., Oh, T.H., Kim, D., Ha, H., Yoon, S.:
  Ficgan: Facial identity controllable gan for de-identification. arXiv
  preprint arXiv:2110.00740  (2021)

\bibitem{Jourabloo15}
Jourabloo, A., Yin, X., Liu, X.: Attribute preserved face de-identification.
  In: 2015 International conference on biometrics (ICB) (2015)

\bibitem{Karras20b}
Karras, T., Aittala, M., Hellsten, J., Laine, S., Lehtinen, J., Aila, T.:
  Training generative adversarial networks with limited data. Advances in
  Neural Information Processing Systems  (2020)

\bibitem{Karras2020ada}
Karras, T., Aittala, M., Hellsten, J., Laine, S., Lehtinen, J., Aila, T.:
  Training generative adversarial networks with limited data. In: Proc. NeurIPS
  (2020)

\bibitem{Karras19}
Karras, T., Laine, S., Aila, T.: A style-based generator architecture for
  generative adversarial networks. In: Proceedings of the IEEE/CVF conference
  on computer vision and pattern recognition (2019)

\bibitem{Karras20a}
Karras, T., Laine, S., Aittala, M., Hellsten, J., Lehtinen, J., Aila, T.:
  Analyzing and improving the image quality of stylegan. In: Proceedings of the
  IEEE/CVF conference on computer vision and pattern recognition (2020)

\bibitem{Kifer14}
Kifer, D., Machanavajjhala, A.: Pufferfish: A framework for mathematical
  privacy definitions. ACM Transactions on Database Systems (TODS)  (2014)

\bibitem{kingma2014adam}
Kingma, D.P., Ba, J.: Adam: A method for stochastic optimization. arXiv
  preprint arXiv:1412.6980  (2014)

\bibitem{Korot21}
Korot, E., Pontikos, N., Liu, X., Wagner, S.K., Faes, L., Huemer, J., Balaskas,
  K., Denniston, A.K., Khawaja, A., Keane, P.A.: Predicting sex from retinal
  fundus photographs using automated deep learning. Scientific reports  (2021)

\bibitem{Long21}
Long, Y., Wang, B., Yang, Z., Kailkhura, B., Zhang, A., Gunter, C., Li, B.:
  G-pate: Scalable differentially private data generator via private
  aggregation of teacher discriminators. Advances in Neural Information
  Processing Systems  (2021)

\bibitem{long2021gpate}
Long, Y., Wang, B., Yang, Z., Kailkhura, B., Zhang, A., Gunter, C.A., Li, B.:
  G-pate: Scalable differentially private data generator via private
  aggregation of teacher discriminators. NeurIPS 2021  (2021)

\bibitem{Loshchilov16}
Loshchilov, I., Hutter, F.: Sgdr: Stochastic gradient descent with warm
  restarts. arXiv preprint arXiv:1608.03983  (2016)

\bibitem{Marino06}
Mari{\~n}o, C., Penedo, M.G., Penas, M., Carreira, M.J., Gonzalez, F.: Personal
  authentication using digital retinal images. Pattern Analysis and
  Applications  (2006)

\bibitem{Maximov20}
Maximov, M., Elezi, I., Leal-Taix{\'e}, L.: Ciagan: Conditional identity
  anonymization generative adversarial networks. In: Proceedings of the
  IEEE/CVF Conference on Computer Vision and Pattern Recognition (2020)

\bibitem{Meden18}
Meden, B., Emer{\v{s}}i{\v{c}}, {\v{Z}}., {\v{S}}truc, V., Peer, P.:
  k-same-net: k-anonymity with generative deep neural networks for face
  deidentification. Entropy  (2018)

\bibitem{Milchenko13}
Milchenko, M., Marcus, D.: Obscuring surface anatomy in volumetric imaging
  data. Neuroinformatics  (2013)

\bibitem{Mohamed07}
Mohamed, Q., Gillies, M.C., Wong, T.Y.: Management of diabetic retinopathy: a
  systematic review. Jama  (2007)

\bibitem{Nesterov03}
Nesterov, Y.: Introductory lectures on convex optimization: A basic course.
  Springer Science \& Business Media (2003)

\bibitem{Newton05}
Newton, E.M., Sweeney, L., Malin, B.: Preserving privacy by de-identifying face
  images. IEEE transactions on Knowledge and Data Engineering  (2005)

\bibitem{Niu19}
Niu, Y., Gu, L., Lu, F., Lv, F., Wang, Z., Sato, I., Zhang, Z., Xiao, Y., Dai,
  X., Cheng, T.: Pathological evidence exploration in deep retinal image
  diagnosis. In: Proceedings of the AAAI conference on artificial intelligence
  (2019)

\bibitem{paul2021defending}
Paul, W., Cao, Y., Zhang, M., Burlina, P.: Defending medical image diagnostics
  against privacy attacks using generative methods. arXiv preprint
  arXiv:2103.03078  (2021)

\bibitem{Poplin18}
Poplin, R., Varadarajan, A.V., Blumer, K., Liu, Y., McConnell, M.V., Corrado,
  G.S., Peng, L., Webster, D.R.: Prediction of cardiovascular risk factors from
  retinal fundus photographs via deep learning. Nature Biomedical Engineering
  (2018)

\bibitem{radford2015unsupervised}
Radford, A., Metz, L., Chintala, S.: Unsupervised representation learning with
  deep convolutional generative adversarial networks. arXiv preprint
  arXiv:1511.06434  (2015)

\bibitem{Ravindra21}
Ravindra, V., Grama, A.: De-anonymization attacks on neuroimaging datasets. In:
  Proceedings of the 2021 International Conference on Management of Data (2021)

\bibitem{Ribaric15}
Ribaric, S., Pavesic, N.: An overview of face de-identification in still images
  and videos. In: 2015 11th IEEE International Conference and Workshops on
  Automatic Face and Gesture Recognition (FG) (2015)

\bibitem{richardson2021encoding}
Richardson, E., Alaluf, Y., Patashnik, O., Nitzan, Y., Azar, Y., Shapiro, S.,
  Cohen-Or, D.: Encoding in style: a stylegan encoder for image-to-image
  translation. In: IEEE/CVF Conference on Computer Vision and Pattern
  Recognition (CVPR) (June 2021)

\bibitem{Schimke11}
Schimke, N., Kuehler, M., Hale, J.: Preserving privacy in structural
  neuroimages. In: IFIP annual conference on data and applications security and
  privacy (2011)

\bibitem{Simonyan15}
Simonyan, K., Zisserman, A.: Very deep convolutional networks for large-scale
  image recognition. In: International Conference on Learning Representations
  (2015)

\bibitem{Sweeney02}
Sweeney, L.: k-anonymity: A model for protecting privacy. International Journal
  of Uncertainty, Fuzziness and Knowledge-Based Systems  (2002)

\bibitem{Szegedy16}
Szegedy, C., Vanhoucke, V., Ioffe, S., Shlens, J., Wojna, Z.: Rethinking the
  inception architecture for computer vision. In: Proceedings of the IEEE
  conference on computer vision and pattern recognition (2016)

\bibitem{EyeNet21}
Taylor, R.: {AI and the Retina: Finding Patterns of Systemic Disease}. EyeNet
  Magazine  (2021)

\bibitem{Tom20}
Tom, E., Keane, P.A., Blazes, M., Pasquale, L.R., Chiang, M.F., Lee, A.Y., Lee,
  C.S., Force, A.A.I.T.: Protecting data privacy in the age of ai-enabled
  ophthalmology. Translational Vision Science \& Technology  (2020)

\bibitem{Hipaa02}
{U.S. Dept. of Health and Human Services}: {Standards for privacy of
  individually identifiable health information, Final Rule}. Federal Registrar
  (2002)

\bibitem{Wagner20}
Wagner, S.K., Fu, D.J., Faes, L., Liu, X., Huemer, J., Khalid, H., Ferraz, D.,
  Korot, E., Kelly, C., Balaskas, K., et~al.: Insights into systemic disease
  through retinal imaging-based oculomics. Translational vision science \&
  technology  (2020)

\bibitem{Wang21}
Wang, X., Zhang, R., Shen, C., Kong, T., Li, L.: Dense contrastive learning for
  self-supervised visual pre-training. In: Proceedings of the IEEE/CVF
  Conference on Computer Vision and Pattern Recognition (2021)

\bibitem{wang2004image}
Wang, Z., Bovik, A.C., Sheikh, H.R., Simoncelli, E.P.: Image quality
  assessment: from error visibility to structural similarity. IEEE transactions
  on image processing  (2004)

\bibitem{Wisely22}
Wisely, C.E., Wang, D., Henao, R., Grewal, D.S., Thompson, A.C., Robbins, C.B.,
  Yoon, S.P., Soundararajan, S., Polascik, B.W., Burke, J.R., et~al.:
  Convolutional neural network to identify symptomatic alzheimer’s disease
  using multimodal retinal imaging. British Journal of Ophthalmology  (2022)

\bibitem{Wu19}
Wu, Y., Yang, F., Xu, Y., Ling, H.: Privacy-protective-gan for privacy
  preserving face de-identification. Journal of Computer Science and Technology
   (2019)

\bibitem{Xia21}
Xia, W., Zhang, Y., Yang, Y., Xue, J.H., Zhou, B., Yang, M.H.: Gan inversion: A
  survey. arXiv preprint arXiv:2101.05278  (2021)

\bibitem{Xu19}
Xu, C., Ren, J., Zhang, D., Zhang, Y., Qin, Z., Ren, K.: Ganobfuscator:
  Mitigating information leakage under gan via differential privacy. IEEE
  Transactions on Information Forensics and Security  (2019)

\bibitem{Yu19}
Yu, Z., Xiang, Q., Meng, J., Kou, C., Ren, Q., Lu, Y.: Retinal image synthesis
  from multiple-landmarks input with generative adversarial networks.
  Biomedical engineering online  (2019)

\bibitem{Zhou20}
Zhou, Y., Wang, B., He, X., Cui, S., Shao, L.: Dr-gan: Conditional generative
  adversarial network for fine-grained lesion synthesis on diabetic retinopathy
  images. IEEE Journal of Biomedical and Health Informatics  (2020)

\end{thebibliography}

\setcounter{section}{0}
\renewcommand\thesection{\Alph{section}}

\chapter*{Supplementary Materials}

\section{Benchmark datasets}
The APTOS dataset includes 3,662 labeled fundus images. 
We split the data into a training set of 3,000 images and a test set of 662 images.
The EyePACS dataset includes 35,126 labeled fundus images.
We split the data into a training set of 28,100 images and a test set of 7,026 images.
We rescaled the images in both datasets to 512-by-512 RGB pixels.
The training set is used to obtain the GAN and GAN-Inversion models.

\section{Implementation details}

\subsubsection{GAN.}
We trained our GAN models using the official PyTorch implementation\footnote{\url{https://github.com/NVlabs/stylegan2-ada-pytorch}} of StyleGAN2-ADA~\cite{Karras2020ada}. We set the number of mapping networks to two as recommended based on our image resolution and GPU count. Since the desired size of the generated images is $512 \times 512$, we used 16 progressive layers, resulting in the latent space $W$ with dimensions $16 \times 512$. We train the models on 5,000 kimgs in each dataset with batch size 64, using 8 3090-RTX GPUs, Pytorch 1.7.1, CUDA 11.1, and CuDNN 8.1.1.

\subsubsection{GAN Inversion.}
To obtain the GAN inversion encoder we built upon the official PyTorch implementation\footnote{https://github.com/yuval-alaluf/restyle-encoder} of ReStyle~\cite{Alaluf21}. We incorporated the MOCO-based~\cite{He20} similarity loss on pSp~\cite{richardson2021encoding} architecture with ResNetBackboneEncoder~\cite{he2016deep}. We trained each model for 100,000 iterations with a batch size of 8 and 5 refinement iterations per batch. Similar to the GAN setting, the output image size is set to $512\times 512$. We performed the training using 1 3090-RTX GPU with the same environment as that of GAN training.

\subsubsection{$k$-SALSA.}
We split the intermediate-level features into a grid of $4 \times 4$ patches (16 in total) to construct the local style features in all settings.
For the relative ratio of $\mathcal{L}_{\text{content}}$ and $\mathcal{L}_{\text{style}}$, we set the parameter $\lambda$ to 0.1, 0.05 and 0.03 for $k=2,5,10$, respectively, in APTOS. For EyePACS, we set $\lambda$ to 0.01, 0.02 and 0.01, respectively for each $k$.
We optimize our model for synthetic averaging using standard stochastic gradient descent with Adam~\cite{kingma2014adam}, with learning rate 0.1 and $\beta_1 = 0.9$, $\beta_2 = 0.99$. We use the same computational environment as above with a single GPU.

\subsubsection{Downstream classification.}
For all synthetic datasets, we trained a DR classifier using the ResNet50 model~\cite{He16} with batch size 32, 60 epochs, stochastic gradient descent (SGD) with Nesterov momentum 0.9~\cite{Nesterov03}, weight decay 0.0005, and cosine annealing in the learning rate schedule~\cite{Loshchilov16}.

\section{Computational costs}
One-time pre-training of GAN and inversion models took 10~hrs and 3~days, respectively, for APTOS, and  30~hrs and 3~days for EyePACS. %Once these models are trained, they can be re-used.
The main runtime of $k$-SALSA depends on the inference speed of GAN and inversion, only around 0.85~secs/image.
Synthetic averaging takes 19~secs/cluster ($k$=5).
Both steps can be parallelized.
Same-size clustering takes $<$1~min.
Cosine similarity is computed for 16 patches/image (4x4) for 2~ms/image.
Overall, we expect $k$-SALSA to be practical in realistic settings. 
% torch.optim.Adam([w_opt], betas=(0.9, 0.999), lr=initial_learning_rate), initial_learning_rate=0.1
% lr schedule
% We also set $\lambda_1$ as 1e2,  $\lambda_2$ as 1e-6 which is the ratio of $\mathcal{L}_{\text{content}}$ and $\mathcal{L}_{\text{style}}$, respectively.


\section{Choice of similarity metric for local style alignment}
Here we provide additional empirical results supporting our choices of similarity metric in the local style alignment.
Recall that once the local style features are constructed for each batch, we find the optimal matches between the target and source images using cosine similarity (COS).
Given these optimal matches, we then use mean squared error (MSE) to optimize the synthetic average (see the definition of $\mathcal{L}_{\text{style}}$ in Eq.~7), effectively transferring the local styles from the source image to the target average.
We note that these choices are inspired by prior works; dense contrastive learning~\cite{Wang21} uses the COS metric to perform the alignment, whereas style transfer~\cite{Gatys16} is typically done using the MSE---our work combines these two approaches and keeps the respective similarity metrics.
As shown in Supplementary Table~\ref{table:ablation2}, using either of COS and MSE metrics for both components of the model (COS-COS or MSE-MSE) results in worse averaging performance as measured by the downstream classification task evaluated in our work, which intuitively captures how well the clinically relevant features are preserved in the averaged images.
This result supports our hybrid use of both metrics.
\begin{table*}[tb]
\centering
\scalebox{0.95}{
\setlength{\tabcolsep}{0.8mm}
\begin{tabular}{c|c|ccccc|ccc}
\toprule[1pt]
Method & LA & AP(1$\times$) & AR(1$\times$) & AP(3$\times$) & AR(3$\times$) & NMS & forward(ms) & NMS(ms) & FPS \\
 \hline
FCOS~\cite{fcos} & o2m & 38.6 & 57.2 & 41.4 & 59.1 & \Checkmark & 27.7 & 0.7 & 35.2 \\
FCOS~\cite{fcos} & o2m & 17.7 & 52.9 & 19.1 & 54.2 & \XSolidBrush & 27.7 & 0 & 36.1 \\
POTO~\cite{poto} & o2o & 36.5 & 58.9 & 40.2 & 61.1 & \XSolidBrush & 27.7 & 0 & 36.1 \\
POTO+3DMF~\cite{poto} & o2o & 37.0 & 58.8 & 40.5 & 60.9 & \XSolidBrush & 30.3 & 0 & 33.0 \\
POTO+3DMF+Aux~\cite{poto} & o2o+o2m(ATSS) & 37.6 & 58.7 & 41.2 & 61.2 & \XSolidBrush & 30.3 & 0 & 33.0 \\
POTO+3DMF+Aux~\cite{poto} & o2o+o2m(Top-k) & 37.6 & 59.0 & 41.1 & 61.3 & \XSolidBrush & 30.3 & 0 & 33.0 \\
POTO+3DMF+Aux~\cite{poto} & o2o+o2m(FCOS) & 36.5 & 58.0 & 40.3 & 60.6 & \XSolidBrush & 30.3 & 0 & 33.0 \\
Ours-1conv & o2f & 38.4 & 60.5 & 41.6 & 63.3 & \XSolidBrush & 28.1 & 0 & 35.6 \\
Ours-2convs & o2f & 38.7 & 60.6 & 42.0 & 63.5 & \XSolidBrush & 28.2 & 0 & 35.5 \\
Ours-3convs & o2f & 39.0 & 61.2 & 42.2 & 63.5 & \XSolidBrush & 28.2 & 0 & 35.5 \\
%Ours-3convs+DCN & o2f & 39.4 & 61.8 & & & \XSolidBrush & 29.9 & 0 & 33.4 \\
\bottomrule[1pt]
\end{tabular}
}
\caption{Comparisons with state-of-the-art end-to-end dense detectors on COCO \texttt{val} set. All experiments are conducted with ResNet-50 backbone. `LA' means label assignment. `o2o' means one-to-one. `o2m' means one-to-many. `o2f' means one-to-few. `ATSS', `Top-k' and `FCOS' in the `LA' column are the different o2m label assignment strategies used in the auxiliary loss in POTO. The reported runtime (ms) are all evaluated on a Tesla-V100 GPU under the MMDetection toolbox.}
%\vspace{-2mm}
\label{table5}
\end{table*}

\section{Addressing the reduced size of synthetic dataset}
Dataset size is an important issue in medical imaging problems.
The reduced number of images in the synthetic average dataset constructed by $k$-SALSA is a potential concern.
To mitigate the cost of dataset reduction, we investigated an extension of $k$-SALSA based on data augmentation, whereby small random noise is added to $k$-SALSA's average embeddings to generate multiple ``views'' of each cluster.
As shown in Supplementary Table~\ref{table:reduced}, with 5 augmented images per cluster, we observed an improved performance of 0.829 (originally 0.769) for $k$-SALSA, and 0.809 (0.745) for $k$-Centroid in APTOS, $k$=5. 
This demonstrates the potential of our extension in countering the size reduction of the synthetic dataset with data augmentation.
Importantly, the augmented images are independent of private images conditioned on $k$-SALSA's representative embedding of each cluster, thus there is no additional privacy leakage.
\begin{table}[t]
\caption{\textbf{(a)} Ablation on NeFeS architecture. \textbf{(b)} Ablation on the proposed training scheduling}
\label{table:6}
\centering
\resizebox{\linewidth}{!}{

\begin{tabular}{lc}
        \multicolumn{2}{c}{\textbf{(a) NeFeS Architecture Ablation}}\\
        \\
            \toprule
            Method                               & Shop Facade \\
            \midrule
            NeFeS (ours) & 0.14m/0.47\degree  \\
            - Exposure-adaptive ACT & 0.14m/1.20\degree  \\
            - Feature Fusion & 0.37m/1.62\degree\\
            \midrule
            NeRF RGB+CNN & 0.15m/0.87\degree\\
            \bottomrule
        \end{tabular}

\begin{tabular}{lc}
        \multicolumn{2}{c}{\textbf{(b) Training Scheduling Ablation}}\\
        \\
            \toprule
            Method              & Church \\
            \midrule
            Combined & 0.35m/1.13\degree  \\
            Progressive & 0.32m/0.99\degree\\
            \bottomrule
        \end{tabular}

}
\end{table}


\section{Additional comparisons with Original/gan-inverted}
In our main experiments, we subsampled the original and GAN-inverted images to match the number of training images for the classifiers for comparison with different synthetic average datasets.
For completeness, here we include the ``best-case scenario'' classification performance results for these baselines by using the full training dataset in APTOS. 
\begin{table}[tb]
\centering
\scalebox{0.85}{
\setlength{\tabcolsep}{0.8mm}
\begin{tabular}{c|ccccc}
\toprule[1pt]
Method & AP(1$\times$) & AR(1$\times$) & AP(3$\times$) & AR(3$\times$) & NMS \\
 \hline
FCOS~\cite{fcos} & 43.6 & 61.1 & 46.8 &  63.7& \Checkmark \\
FCOS~\cite{fcos} & 20.7 & 57.2 & 23.0 & 59.6 & \XSolidBrush \\
POTO~\cite{poto} & 40.8 & 61.6 & 44.0 & 64.7 & \XSolidBrush \\
POTO+3DMF+Aux~\cite{poto} & 41.8 & 61.7 &  44.8 & 63.9 & \XSolidBrush \\
Ours & 42.9 & 63.7 & 46.1 & 66.7  & \XSolidBrush \\
\bottomrule[1pt]
\end{tabular}
}
\caption{The object detection results with Swin-T backbone.}
\vspace{-4mm}
\label{swin}
\end{table}
The results are shown in Supplementary Table~\ref{table:fairness}.
We obtained a Cohen's $\kappa$ of 0.914 and 0.857 for Original and GAN-Inverted, respectively, without subsampling, compared to 0.888 and 0.828 with subsampling with $k=5$ (i.e., a 20\% sampling rate), respectively.
These results suggest that, while subsampling does reduce the performance, the impact is relatively small and that $k$-SALSA performance is still competitive with the best-case scenario, especially when used with our data augmentation strategy described in the previous section.


\section{Performance dependence on the cluster size $k$}
To further investigate the effect of cluster size $k$ on downstream classification performance, we compared the classifiers trained on $k$-SALSA synthetic datasets for different values of $k$, but subsampled to the same number of clusters.
Note that larger $k$ leads to fewer clusters and thus smaller training data for classification.
At the same time, larger $k$ increases the potential to retain more salient features from source images.
As shown in Supplementary Table~\ref{table:dependence_k}, we observed a performance improvement for larger $k$ in APTOS,
suggesting that summarizing key features across multiple images has a beneficial impact on the classifier training.
\setlength{\tabcolsep}{4pt}
\begin{table}
    \begin{center}
    \caption{Performance dependence on the cluster size $k$}
    \label{table:dependence_k}
    \begin{tabular}{c|c|ccc}
        \toprule
        \multirow{2}{*}{\textbf{Method}} & \multirow{2}{*}{\textbf{{Metric}}} &\multicolumn{3}{c}{\textbf{APTOS}} \\
        & & {$k=2$} & {$k=5$} & {$k=10$}  \\
        \midrule
        {$k$-SALSA} & Cohen's $\kappa$ & 0.688 & 0.740 & 0.761 \\
        
        \bottomrule
    \end{tabular}
    \end{center}
\end{table}
\setlength{\tabcolsep}{1.4pt}

\section{Challenges of differentially private GANs}
To evaluate the performance of differentially private GAN approach to synthetic generation of retinal images, we trained one of the state-of-the-art models, G-PATE~\cite{long2021gpate}, on our fundus image datasets (APTOS) using the official TensorFlow implementation\footnote{https://github.com/AI-secure/G-PATE}.
We follow the setting in the provided code except for the number of teacher networks and batch size, which we changed to 600 and 32 (from 2000 and 64), respectively, to reflect the smaller sizes of our datasets.
In order to use the provided code, we downscaled the retinal images to $64 \times 64$.
Otherwise, we used the following default parameters: number of epochs 1000, $\sigma$ threshold 600, $\sigma$ 100, step size $10^{-4}$, max $\epsilon$ 100 and $z$-dimension 100.
As shown in Supplementary Figure~\ref{fig:g_pate}, the generated images from the differentially private GAN, even with a lenient privacy parameter of $\epsilon=100$, are far from resembling retinal images.
We attribute this failure in training to the relatively small size of our datasets (e.g. 3000) and the high resolution of the images, compared to handwritten digit images considered in the original work.
Note that the GAN architecture used by G-PATE is DC-GAN~\cite{radford2015unsupervised}, which is expected to have difficulties in the limited-data, the high-resolution setting given its low representation power, instability, and vanishing gradients compared with more recent techniques such as StyleGAN2-ADA~\cite{Karras2020ada}.
Consistent with the visual assessment, the Fr\'{e}chet Inception Distance (FID) of the images generated by G-PATE is 441.23, which is vastly higher than that of our approach (20.09), indicating the challenges of differentially private training of GANs in our setting.

\renewcommand{\figurename}{Supplementary Figure\,}
\begin{figure*}
    \centering
    \includegraphics[width=.3\textwidth]{g_pate.png}
    %\vspace{-10pt}
    \caption{64 $\times$ 64 generated images from the G-PATE model~\cite{long2021gpate} trained on retinal images (APTOS).}
    \label{fig:g_pate}
    \vspace{-1.7em}
\end{figure*}

\section{Choice of visual fidelity metric}
To evaluate the fidelity of synthetic images, we use the Fr\'{e}chet inception distance (FID), a standard metric for images generated using GANs. Other common metrics include PSNR and SSIM; however, these metrics quantify the degradation of quality when a source image is transformed, whereas FID measures a distributional similarity to a set of reference images based on high-level activations. FID uniquely assesses whether $k$-SALSA images are realistic compared to real retinal images. Moreover, $k$-SALSA introduces spatial flexibility of image features, which is not captured by pixel-level metrics like PSNR and SSIM.

\section{Additional examples of synthetic averages}
To complement the main results, here we include additional synthetic averages of APTOS images along with their real source images for each $k\in \{2,5,10\}$ (4 examples each): 
Supplementary Figs.~\ref{fig:k10_1}--\ref{fig:k10_4} for $k=10$;
Supplementary Figs.~\ref{fig:k5_1} and \ref{fig:k5_2} for $k=5$; and Supplementary Figs.~\ref{fig:k2_1} and \ref{fig:k2_2} for $k=2$.
The results from the EyePACS dataset are analogous.
Note that the figures for both $k=2$ and $k=5$ include two examples per figure.
Source images (\emph{top}) represent the $k$ real original images in the identified cluster, and the synthetic averages (\emph{bottom}) are generated using $k$-SALSA, $k$-Centroid, $k$-Same-Pixel, $k$-Same-PCA, respectively.
Overall, $k$-SALSA can better detect clinically relevant features in all cases.


\begin{figure*}
    \centering
    \includegraphics[width=1.0\textwidth]{exam1.jpg}
    %\vspace{-10pt}
    \caption{\textbf{Examples of synthetic average of retinal images ($k=10$). }
    One example of $k=10$ real images (\emph{top}) along with synthetic averages generated by different methods (\emph{bottom}). $k$-SALSA better captures a disease-related feature (\emph{A}).}
    \label{fig:k10_1}
    \vspace{-1.7em}
\end{figure*}

\begin{figure*}
    \centering
    \includegraphics[width=1.0\textwidth]{exam2.jpg}
    %\vspace{-10pt}
    \caption{\textbf{Examples of synthetic average of retinal images ($k=10$).}
    One example of $k=10$ real images (\emph{top}) along with synthetic averages generated by different methods (\emph{bottom}). $k$-SALSA better captures a disease-related feature (\emph{A}).}
    \label{fig:k10_2}
    \vspace{-1.7em}
\end{figure*}

\begin{figure*}
    \centering
    \includegraphics[width=1.0\textwidth]{exam3.jpg}
    %\vspace{-10pt}
    \caption{\textbf{Examples of synthetic average of retinal images ($k=10$).}
    One example of $k=10$ real images (\emph{top}) along with synthetic averages generated by different methods (\emph{bottom}). $k$-SALSA better captures a disease-related feature (\emph{A}).}
    \label{fig:k10_3}
    \vspace{-1.7em}
\end{figure*}

\begin{figure*}
    \centering
    \includegraphics[width=1.0\textwidth]{exam4.jpg}
    %\vspace{-10pt}
    \caption{\textbf{Examples of synthetic average of retinal images ($k=10$).}
    One example of $k=10$ real images (\emph{top}) along with synthetic averages generated by different methods (\emph{bottom}). $k$-SALSA better captures  disease-related features (\emph{A}, \emph{B}).}
    \label{fig:k10_4}
    \vspace{-1.7em}
\end{figure*}

\begin{figure*}
    \centering
    \includegraphics[width=1.0\textwidth]{exam5.jpg}
    %\vspace{-10pt}
    \caption{\textbf{Examples of synthetic average of retinal images ($k=5$).}
    Two examples of $k=5$ real images (\emph{top}) along with synthetic averages generated by different methods (\emph{bottom}). $k$-SALSA better captures  disease-related features (\emph{A}, \emph{B}, \emph{C}, \emph{D}).}
    \label{fig:k5_1}
    \vspace{-1.7em}
\end{figure*}

\begin{figure*}
    \centering
    \includegraphics[width=1.0\textwidth]{exam6.jpg}
    %\vspace{-10pt}
    \caption{\textbf{Examples of synthetic average of retinal images ($k=5$).}
    Two examples of $k=5$ real images (\emph{top}) along with synthetic averages generated by different methods (\emph{bottom}). $k$-SALSA better captures disease-related features (\emph{A}, \emph{B}, \emph{C}).}
    \label{fig:k5_2}
    \vspace{-1.7em}
\end{figure*}

\begin{figure*}
    \centering
    \includegraphics[width=1.0\textwidth]{exam7.jpg}
    %\vspace{-10pt}
    \caption{\textbf{Examples of synthetic average of retinal images ($k=2$).}
Two examples of $k=2$ real images (\emph{top}) along with synthetic averages generated by different methods (\emph{bottom}). $k$-SALSA better captures a disease-related feature (\emph{A}).}
    \label{fig:k2_1}
    \vspace{-1.7em}
\end{figure*}

\begin{figure*}
    \centering
    \includegraphics[width=1.0\textwidth]{exam8.jpg}
    %\vspace{-10pt}
    \caption{\textbf{Examples of synthetic average of retinal images ($k=2$).}
Two examples of $k=2$ real images (\emph{top}) along with synthetic averages generated by different methods (\emph{bottom}). $k$-SALSA better captures  disease-related features (\emph{A}, \emph{B}, \emph{C}).}
    \label{fig:k2_2}
    \vspace{-1.7em}
\end{figure*}










% ------------------------------------------------------------------------------------
% \begin{figure*}
%     \centering
%     \includegraphics[width=.9\textwidth]{example1.png}
%     %\vspace{-10pt}
%     \caption{\textbf{Examples of synthetic average of retinal images. ($k=2$)}}
%     \label{fig:g_pate}
%     \vspace{-1.7em}
% \end{figure*}

% \begin{figure*}
%     \centering
%     \includegraphics[width=.9\textwidth]{example2.png}
%     %\vspace{-10pt}
%     \caption{\textbf{Examples of synthetic average of retinal images. ($k=2$)}}
%     \label{fig:g_pate}
%     \vspace{-1.7em}
% \end{figure*}

% \begin{figure*}
%     \centering
%     \includegraphics[width=.9\textwidth]{example3.png}
%     %\vspace{-10pt}
%     \caption{\textbf{Examples of synthetic average of retinal images. ($k=2$)}}
%     \label{fig:g_pate}
%     \vspace{-1.7em}
% \end{figure*}

% \begin{figure*}
%     \centering
%     \includegraphics[width=.9\textwidth]{example4.png}
%     %\vspace{-10pt}
%     \caption{\textbf{Examples of synthetic average of retinal images. ($k=2$)}}
%     \label{fig:g_pate}
%     \vspace{-1.7em}
% \end{figure*}

% \begin{figure*}
%     \centering
%     \includegraphics[width=.9\textwidth]{example5.png}
%     %\vspace{-10pt}
%     \caption{\textbf{Examples of synthetic average of retinal images. ($k=2$)}}
%     \label{fig:g_pate}
%     \vspace{-1.7em}
% \end{figure*}

% \begin{figure*}
%     \centering
%     \includegraphics[width=.9\textwidth]{example6.png}
%     %\vspace{-10pt}
%     \caption{\textbf{Examples of synthetic average of retinal images. ($k=10$)}}
%     \label{fig:g_pate}
%     \vspace{-1.7em}
% \end{figure*}

\end{document}