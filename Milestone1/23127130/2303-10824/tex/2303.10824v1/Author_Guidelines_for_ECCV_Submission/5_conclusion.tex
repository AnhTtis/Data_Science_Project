We presented $k$-SALSA, an end-to-end pipeline for synthesizing a $k$-anonymous retinal image dataset given a private input dataset.
We leverage local style alignment, our new approach for summarizing source images in a cluster while preserving local texture information.
Our results demonstrate that $k$-anonymization of retinal images, preserving both privacy and clinical utility, is feasible.

We would like to address several limitations of the current method in future work. First, $k$-SALSA's performance is dependent on the quality of the underlying GAN and GAN inversion models.
We plan to devise strategies tailored to retinal images (e.g., separately modelling different parts of the image) to further improve GAN models. Next, we plan to explore more rigorous frameworks for privacy such as differential privacy (DP)~\cite{Dwork14}. While it is generally difficult to apply DP to high-dimensional data such as images, certain relaxations of DP~\cite{Kifer14} may lead to a practical solution. Lastly, we plan to explore the application of our methodology to other imaging modalities for the retina, including the OCT.

Our work demonstrates that domain-inspired techniques can be combined with the state-of-the-art GAN techniques to design effective approaches to privatizing sensitive data. 
The methodological insights introduced by our work is of general interest to other domains (e.g. genomics), where privacy-aware aggregation of sensitive data may overcome challenges in data sharing.