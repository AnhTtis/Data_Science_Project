 The application of modern machine learning to retinal image analyses offers valuable insights into a broad range of human health conditions beyond ophthalmic diseases. Additionally, data sharing is key to fully realizing the potential of machine learning models by providing a rich and diverse collection of training data. However, the personally-identifying nature of retinal images, encompassing the unique vascular structure of each individual, often prevents this data from being shared openly. While prior works have explored image de-identification strategies based on synthetic averaging of images in other domains (e.g. facial images), existing techniques face difficulty in preserving both privacy and clinical utility in retinal images, as we demonstrate in our work. We therefore introduce $k$-SALSA, a generative adversarial network (GAN)-based framework for synthesizing retinal fundus images that summarize a given private dataset while satisfying the privacy notion of $k$-anonymity. $k$-SALSA brings together state-of-the-art techniques for training and inverting GANs to achieve practical performance on retinal images. Furthermore, $k$-SALSA leverages a new technique, called local style alignment, to generate a synthetic average that maximizes the retention of fine-grain visual patterns in the source images, thus improving the clinical utility of the generated images. On two benchmark datasets of diabetic retinopathy (EyePACS and APTOS), we demonstrate our improvement upon existing methods with respect to image fidelity, classification performance, and mitigation of membership inference attacks. Our work represents a step toward broader sharing of retinal images for scientific collaboration. Code is available at \url{https://github.com/hcholab/k-salsa}. \footnotemark[1]\footnotetext[1]{This paper has been accepted at ECCV 2022}

\keywords{Medical image privacy, k-anonymity, generative adversarial networks, fundus imaging, synthetic data generation, style transfer}