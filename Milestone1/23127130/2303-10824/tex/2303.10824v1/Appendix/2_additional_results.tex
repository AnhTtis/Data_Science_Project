To complement the main results, here we include additional synthetic averages of APTOS images along with their real source images for each $k\in \{2,5,10\}$ (4 examples each): 
Supplementary Figs.~\ref{fig:k10_1}--\ref{fig:k10_4} for $k=10$;
Supplementary Figs.~\ref{fig:k5_1} and \ref{fig:k5_2} for $k=5$; and Supplementary Figs.~\ref{fig:k2_1} and \ref{fig:k2_2} for $k=2$.
The results from the EyePACS dataset are analogous.
Note that the figures for both $k=2$ and $k=5$ include two examples per figure.
Source images (\emph{top}) represent the $k$ real original images in the identified cluster, and the synthetic averages (\emph{bottom}) are generated using $k$-SALSA, $k$-Centroid, $k$-Same-Pixel, $k$-Same-PCA, respectively.
Overall, $k$-SALSA can better detect clinically relevant features in all cases.


\begin{figure*}
    \centering
    \includegraphics[width=1.0\textwidth]{exam1.jpg}
    %\vspace{-10pt}
    \caption{\textbf{Examples of synthetic average of retinal images ($k=10$). }
    One example of $k=10$ real images (\emph{top}) along with synthetic averages generated by different methods (\emph{bottom}). $k$-SALSA better captures a disease-related feature (\emph{A}).}
    \label{fig:k10_1}
    \vspace{-1.7em}
\end{figure*}

\begin{figure*}
    \centering
    \includegraphics[width=1.0\textwidth]{exam2.jpg}
    %\vspace{-10pt}
    \caption{\textbf{Examples of synthetic average of retinal images ($k=10$).}
    One example of $k=10$ real images (\emph{top}) along with synthetic averages generated by different methods (\emph{bottom}). $k$-SALSA better captures a disease-related feature (\emph{A}).}
    \label{fig:k10_2}
    \vspace{-1.7em}
\end{figure*}

\begin{figure*}
    \centering
    \includegraphics[width=1.0\textwidth]{exam3.jpg}
    %\vspace{-10pt}
    \caption{\textbf{Examples of synthetic average of retinal images ($k=10$).}
    One example of $k=10$ real images (\emph{top}) along with synthetic averages generated by different methods (\emph{bottom}). $k$-SALSA better captures a disease-related feature (\emph{A}).}
    \label{fig:k10_3}
    \vspace{-1.7em}
\end{figure*}

\begin{figure*}
    \centering
    \includegraphics[width=1.0\textwidth]{exam4.jpg}
    %\vspace{-10pt}
    \caption{\textbf{Examples of synthetic average of retinal images ($k=10$).}
    One example of $k=10$ real images (\emph{top}) along with synthetic averages generated by different methods (\emph{bottom}). $k$-SALSA better captures  disease-related features (\emph{A}, \emph{B}).}
    \label{fig:k10_4}
    \vspace{-1.7em}
\end{figure*}

\begin{figure*}
    \centering
    \includegraphics[width=1.0\textwidth]{exam5.jpg}
    %\vspace{-10pt}
    \caption{\textbf{Examples of synthetic average of retinal images ($k=5$).}
    Two examples of $k=5$ real images (\emph{top}) along with synthetic averages generated by different methods (\emph{bottom}). $k$-SALSA better captures  disease-related features (\emph{A}, \emph{B}, \emph{C}, \emph{D}).}
    \label{fig:k5_1}
    \vspace{-1.7em}
\end{figure*}

\begin{figure*}
    \centering
    \includegraphics[width=1.0\textwidth]{exam6.jpg}
    %\vspace{-10pt}
    \caption{\textbf{Examples of synthetic average of retinal images ($k=5$).}
    Two examples of $k=5$ real images (\emph{top}) along with synthetic averages generated by different methods (\emph{bottom}). $k$-SALSA better captures disease-related features (\emph{A}, \emph{B}, \emph{C}).}
    \label{fig:k5_2}
    \vspace{-1.7em}
\end{figure*}

\begin{figure*}
    \centering
    \includegraphics[width=1.0\textwidth]{exam7.jpg}
    %\vspace{-10pt}
    \caption{\textbf{Examples of synthetic average of retinal images ($k=2$).}
Two examples of $k=2$ real images (\emph{top}) along with synthetic averages generated by different methods (\emph{bottom}). $k$-SALSA better captures a disease-related feature (\emph{A}).}
    \label{fig:k2_1}
    \vspace{-1.7em}
\end{figure*}

\begin{figure*}
    \centering
    \includegraphics[width=1.0\textwidth]{exam8.jpg}
    %\vspace{-10pt}
    \caption{\textbf{Examples of synthetic average of retinal images ($k=2$).}
Two examples of $k=2$ real images (\emph{top}) along with synthetic averages generated by different methods (\emph{bottom}). $k$-SALSA better captures  disease-related features (\emph{A}, \emph{B}, \emph{C}).}
    \label{fig:k2_2}
    \vspace{-1.7em}
\end{figure*}










% ------------------------------------------------------------------------------------
% \begin{figure*}
%     \centering
%     \includegraphics[width=.9\textwidth]{example1.png}
%     %\vspace{-10pt}
%     \caption{\textbf{Examples of synthetic average of retinal images. ($k=2$)}}
%     \label{fig:g_pate}
%     \vspace{-1.7em}
% \end{figure*}

% \begin{figure*}
%     \centering
%     \includegraphics[width=.9\textwidth]{example2.png}
%     %\vspace{-10pt}
%     \caption{\textbf{Examples of synthetic average of retinal images. ($k=2$)}}
%     \label{fig:g_pate}
%     \vspace{-1.7em}
% \end{figure*}

% \begin{figure*}
%     \centering
%     \includegraphics[width=.9\textwidth]{example3.png}
%     %\vspace{-10pt}
%     \caption{\textbf{Examples of synthetic average of retinal images. ($k=2$)}}
%     \label{fig:g_pate}
%     \vspace{-1.7em}
% \end{figure*}

% \begin{figure*}
%     \centering
%     \includegraphics[width=.9\textwidth]{example4.png}
%     %\vspace{-10pt}
%     \caption{\textbf{Examples of synthetic average of retinal images. ($k=2$)}}
%     \label{fig:g_pate}
%     \vspace{-1.7em}
% \end{figure*}

% \begin{figure*}
%     \centering
%     \includegraphics[width=.9\textwidth]{example5.png}
%     %\vspace{-10pt}
%     \caption{\textbf{Examples of synthetic average of retinal images. ($k=2$)}}
%     \label{fig:g_pate}
%     \vspace{-1.7em}
% \end{figure*}

% \begin{figure*}
%     \centering
%     \includegraphics[width=.9\textwidth]{example6.png}
%     %\vspace{-10pt}
%     \caption{\textbf{Examples of synthetic average of retinal images. ($k=10$)}}
%     \label{fig:g_pate}
%     \vspace{-1.7em}
% \end{figure*}