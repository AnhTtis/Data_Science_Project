To evaluate the performance of differentially private GAN approach to synthetic generation of retinal images, we trained one of the state-of-the-art models, G-PATE~\cite{long2021gpate}, on our fundus image datasets (APTOS) using the official TensorFlow implementation\footnote{https://github.com/AI-secure/G-PATE}.
We follow the setting in the provided code except for the number of teacher networks and batch size, which we changed to 600 and 32 (from 2000 and 64), respectively, to reflect the smaller sizes of our datasets.
In order to use the provided code, we downscaled the retinal images to $64 \times 64$.
Otherwise, we used the following default parameters: number of epochs 1000, $\sigma$ threshold 600, $\sigma$ 100, step size $10^{-4}$, max $\epsilon$ 100 and $z$-dimension 100.
As shown in Supplementary Figure~\ref{fig:g_pate}, the generated images from the differentially private GAN, even with a lenient privacy parameter of $\epsilon=100$, are far from resembling retinal images.
We attribute this failure in training to the relatively small size of our datasets (e.g. 3000) and the high resolution of the images, compared to handwritten digit images considered in the original work.
Note that the GAN architecture used by G-PATE is DC-GAN~\cite{radford2015unsupervised}, which is expected to have difficulties in the limited-data, the high-resolution setting given its low representation power, instability, and vanishing gradients compared with more recent techniques such as StyleGAN2-ADA~\cite{Karras2020ada}.
Consistent with the visual assessment, the Fr\'{e}chet Inception Distance (FID) of the images generated by G-PATE is 441.23, which is vastly higher than that of our approach (20.09), indicating the challenges of differentially private training of GANs in our setting.

\renewcommand{\figurename}{Supplementary Figure\,}
\begin{figure*}
    \centering
    \includegraphics[width=.3\textwidth]{g_pate.png}
    %\vspace{-10pt}
    \caption{64 $\times$ 64 generated images from the G-PATE model~\cite{long2021gpate} trained on retinal images (APTOS).}
    \label{fig:g_pate}
    \vspace{-1.7em}
\end{figure*}