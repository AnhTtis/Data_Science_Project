\documentclass [12PT]{article}
%\textwidth 6.5in%
%\textheight 8.5in%
%\oddsidemargin -.1in%
%\evensidemargin -1in%
%\topmargin -.2in%
%\renewcommand{\baselinestretch}{1.75}%

%--------------------Initialize---------------------%
%\textheight= 9in  \textwidth = 6.5in \topmargin = -1.4cm
%\oddsidemargin = -0.15in \evensidemargin = 0in
\usepackage{graphicx}%,setspace,lscape,longtable}
%\usepackage{epsfig,graphicx}
%\usepackage{bm,color}

\usepackage[dvipsnames,usenames]{color}

%\usepackage{xcolor}         %for colors
% \usepackage{amsmath}                        %for importing graphics
%\usepackage{graphicx}               %for importing graphics


% ******** colors *************************
\definecolor{DarkGreen}{rgb}{0.5,0.8,0.6}   %define a custom color
\definecolor{RGBblack}{rgb}{0.0,0.0,0.0}    %define a custom color
                                            %  black doesn't work (?) for
                                            %  {h,v}pagecolor so use this

\newcommand{\heilv}[1]{\color{DarkGreen}{#1} \color{black}  }
\newcommand{\huang}[1]{\color{yellow}{#1} \color{black}   }
\newcommand{\hong}[1]{\color{black}{#1} \color{black}  }
\newcommand{\lan}[1]{\color{blue}{#1}\color{black}}
\newcommand{\lv}[1]{\color{green}{#1} \color{black}   }
\newcommand{\shenhong}[1]{\color{magenta}{#1} \color{black}   }
\newcommand{\zong}[1]{\color{brown}{#1} \color{black}   }
\newcommand{\bai}[1]{\color{white}{#1} \color{black} }
\newcommand{\hei}[1]{\color{black}{#1} \color{black} }
\definecolor{grau}{rgb}{0.8,0.8,0.8}
\newcommand{\grau}{\color{grau}}
\newcommand{\chen}[1]{\color{orange}}
\newcommand{\ex}{\color{DarkGreen}}
\newcommand{\refce}[1]{{\braun #1}}
\newcommand{\ma}{\color{magenta}}

\newcommand{\bch}{\color{blue}\it}
\newcommand{\ech}{\color{black}\rm}
%\newcommand{\note}[1]{\color{Blue}
 %  (Note)\footnote{\color{blue}#1}\color{Black}~ }
\newcommand{\note}[1]{\color{blue}  (Note)\footnote{\color{blue}#1}\color{black}~ }
\newcommand{\cut}{\color{red}[cut] \color{black}  }


%%%%%%% ----------- YJ's tracking ------------------%%%%%%
\def\hong{\color{red}}
\def\lv{\color{green}}
\def\huang{\color{yellow}}
\def\hei{\color{black}}
\def\lan{\color{blue}}
\newcommand{\yy}{\hong \it}
\newcommand{\jj}{\hei \rm}
\newcommand{\xx}{\lan \it}


\textwidth 7in%
\textheight 9in%
\oddsidemargin -.3in%
\evensidemargin -.3in%
\topmargin -.4in%
\renewcommand{\baselinestretch}{1.75}%

\usepackage{enumitem}
\usepackage{makecell}
\usepackage{booktabs}
\usepackage{algorithm}
\usepackage{longtable}
\usepackage{algorithmic}
\usepackage{natbib}
\usepackage{multirow}
\usepackage{bm}
\usepackage{amsthm,amsmath,amssymb}
\usepackage{mathrsfs}
\usepackage{url}
\usepackage[title]{appendix}

\newcommand{\obs}{\text{obs}}
\newcommand{\mis}{\text{mis}}
\newcommand*\samethanks[1][\value{footnote}]{\footnotemark[#1]}

\usepackage{color}
% \newcommand{ }{\color{red}}
% \newcommand{  }{\color{black}}

{
	\title{\bf The Backfill i3+3 Design for Dose-Finding Trials in Oncology}

 	\author{
 		Jiaxin Liu\thanks{East China Normal University, Shanghai, CHN} \ \thanks{Cytel Inc., Shanghai, CHN}, Shijie Yuan\samethanks , B. Nebiyou Bekele \thanks{Exelixis, Almeda CA, USA}, and Yuan Ji\thanks{Corresponding email:koaeraser@gmail.com; Department of Public Health Sciences, The University of Chicago, Chicago, USA}
 	}
	\date{\today}
}

\begin{document}
\maketitle


\begin{abstract}
  We consider a formal statistical design that allows   simultaneous enrollment of a main cohort and a backfill cohort of patients in a dose-finding trial. The goal is to accumulate more information at various doses to facilitate dose optimization.    % increase the number of patients tested on safe and potentially efficacious doses in a trial with the primary goal of identifying the maximum tolerated dose and learning the safety profile of a new investigational drug.
  The proposed design, called Bi3+3, combines the simple dose-escalation algorithm in the i3+3 design 
  and a model-based inference under the framework of probability of decisions (POD), both previously published. As a result, Bi3+3 provides a simple algorithm for backfilling patients to lower doses in a dose-finding trial once these doses exhibit safety profile in patients. The POD framework allows dosing decisions to be made when   some backfill patients    are still being followed with incomplete toxicity outcomes, thereby potentially expediting the clinical trial. At the end of the trial, Bi3+3 uses both toxicity and efficacy outcomes to estimate an optimal biological dose (OBD). The proposed inference is based on a  dose-response model that takes into account either a monotone or plateau dose-efficacy relationship, which are frequently encountered in modern oncology drug development. Simulation studies show promising operating characteristics of the Bi3+3 design in comparison to existing designs. 

%With the development of several new treatments, the search for MTD is no longer the only goal of phase I. In order to search for more information of interest and prepare for later phases, there is a trend to enroll additional patients in the dose finding trial at a dose that has already had patients enrolled in the main cohort, called 'backfill'. Some backfill methods have been used in clinical trials. In this article, we combine the i3+3 trial with the backfill strategy and propose a Bckfill i3+3 design to enroll patients. Apart from this, a new 4-parameter model to describe patients' effectiveness data is also proposed to locate a more optimal dose. We run simulations to compare with other methods and the model performed well in selecting the optimal dose and the allocating dose. In addition, because the design allows for pending, the trial can enroll more patients without prolonging the duration of the trial.

\end{abstract}

\textbf{\textit{Keywords}}:  Dose Optimization; Efficacy; Phase I; Probability of Decision; Toxicity.

\section{Introduction}
\label{sec:introduction}

In an oncology phase I dose-finding trial, a binary dose-limiting toxicity (DLT) outcome is used as the primary endpoint to assess the toxicity of ascending doses of a new investigation therapeutics. Toxicity is assumed to be monotonically increasing with dose levels. The primary goal is to establish the toxicity profile of the therapeutics and identify the maximum tolerated dose (MTD). Patients are enrolled in cohorts and sequentially treated at different doses. A statistical design is used to guide the sequential patient enrollment and dosing decisions that specify the dose for the next patient cohort. An appropriate design is expected to achieve two goals simultaneously, first to allocate as many patients to safe and potentially efficacious doses and second to quickly learn the safety profile of the therapeutics and identify the MTD. Over the past three decades, a large number of effective statistical designs have been developed that drastically improved the quality and efficiency of phase I trials. %designs safer and find the MTD more accurately. 
For example, the continual reassessment method (CRM) \citep{o1990continual} applies model-based inference to estimate the MTD whenever new toxicity outcomes become available. Subsequent model-based designs such as mTPI \citep{ji2010modified} and mTPI-2 \citep{guo2017bayesian} attempt to simplify the statistical models and utilize  up-and-down decision rules to generate simple and transparent decision tables for practical trials. More recent design development aims to further simplifies the decision and modeling framework, leading to model-assisted designs like CCD \citep{Ivaccd2007} and BOIN \citep{yuan2015boin}, as well as model-free designs like i3+3 \citep{liu2020i3+}. %\note{put references}

%only allowing patients to be assigned to the current dose or the two adjacent doses. These designs attempt to improve the accuracy of selecting MTD based on the toxicity data in phase I.

The FDA Project Optimus \citep{Optimus} advocates for a more thorough and comprehensive exploration of doses in early-phase oncology drug development. Across many new concepts in the project, an overarching message is to encourage designs that gather more information about doses, especially those less toxic but equally efficacious,   thereby having a better chance identifying an optimal dose in early-phase clinical trials. %, including dose-finding studies. 
Here, we consider a framework that allows patients to be backfilled to lower doses during a dose-finding trial once   these doses exhibit safety profile in patients.    Backfill is useful in settings where efficacy of a drug does not always increase with dose level, as seen in many recent immune and targeted oncology drugs \citep{shah2021drug}. Therefore, backfilling patients at lower doses provide opportunities to further accumulate information (e.g., pharmacology or efficacy data)  at    these doses while the trial continues to explore higher doses. In this way, efficiency in finding an  optimal dose  can be gained since more information will be available at different doses. However, backfilling raises a new logistic and statistical issue, as illustrated in Figure \ref{fig:backfill_problem}. 
% \note{Nice figure. Perhaps remove the time-to-event information for patients 6 \& 7? They are correct, but we do not need them here. The shaded symbols for patients 8 \& 9 are already sufficient. --Yuan} 
To see this, first note that once backfill cohorts are enrolled   at    a lower dose and the trial continues to explore a higher dose, there are two cohorts of patients that are enrolled and treated at different doses, the backfill cohort at the low dose and the main cohort at the high dose. The main cohort is usually enrolled first after which the backfill cohort is enrolled. When   patient outcomes from the main cohort are recorded,    a dosing decision must be made to   determine    the dose for future patients in the next main cohort. Such decision should utilize the information from the patients in the main cohorts    as well as    the backfill cohorts. However, since patients in backfill cohorts are enrolled after the main cohort, it is possible that, at the time of dosing decision, one or more patients (patients 8 and 9 in Figure \ref{fig:backfill_problem}) in the backfill cohort might still be followed without a definitive DLT outcome. In other words, there could be pending patients with incomplete outcomes in the backfill cohorts. These patients might have been followed for a while but no definitive DLT outcomes can be assessed yet. This raises a question of how the information in these patients could be used in statistical modeling and inference for the dosing decision. One solution is to wait and pause trial enrollment until all the pending patients in the backfill cohort complete follow-up and have their DLT outcomes recorded. The proposed Bi3+3 design aims to provide a better solution to this question that does not always require pausing enrollment.
\begin{figure}[!htbp]
    \centering
    \includegraphics[width=0.8\textwidth]{backfill_v1.png}
    \caption{A hypothetical dose-finding trial allowing patient backfill. The main cohort patients are 1-6 and the backfill cohort ones are 7-9. By the time patient 10
arrives, patients 1 - 7 have observed outcomes, and patients 8 and 9 are still being followed without
definitive outcomes.} %Here the observation window is set 21 days.}
    \label{fig:backfill_problem}
\end{figure}



%Despite the appearance of the new designs for phase I, there are also some drawbacks in these designs. As the sample size is always small in phase I, the data can only support to select MTD, while subgroup or rare patient populations are not included in the trial. What's more, as the dose assignment is based on the former patients' assessment results, randomization is reduced in phase I\citep{Ia2021random}. In order to fix these shortcomings, a strategy called 'dose expansion' is proposed to enroll more patients after the dose-finding trial\citep{Ia2016expansion}. It allows allocating subgroup patients to several specified dosages randomly. However, the expansion cohort will definitely prolong the trial duration.

%Apart from dose expansion, a new strategy named 'Backfill' is becoming more and more popular. When the dose-finding trial is at dose $d$, backfill goes 'back', which means enrolling additional patients at doses lower $d$ to gain more information. Patients are backfilled at the same time in dose-finding trial.
There are few formal statistical approaches     for patient backfill in real-world trials. \citet{manasanch2020phase} used a simple backfill algorithm under the 3+3 design to acquire additional information on low doses during the trial.  \citet{dehbi2021controlled} proposed a novel backfill framework using hypothetical testing under a  CRM.   Here, we consider a design that is slightly different. We assume that some patients in the backfill cohorts are still being followed without DLT outcomes  and  apply a time-to-event model to incorporate information from these patients  to allow  %. The proposed Bi3+3 design may shorten the trial duration by allowing 
dosing decisions to be made with pending outcomes in the backfill cohort.  %In addition, we consider both toxicity and efficacy in the selection of the OBD at the end of the trial when efficacy data become available, further improving the chance of finding an appropriate dose for later studies.
%Though backfill can effectively provide more information we concern, it also requires a larger sample size. Since one of the premises of traditional phase I design is the patients are enrolled until the former patients' assessment results are observed, a larger sample size means a longer duration of the trial. 
Designs that allow dosing decisions in the presence of pending outcomes have been developed for dose-finding trials. For example, TITE-CRM \citep{cheung2000sequential} is a pioneering time-to-event (TITE) model under the CRM framework and has been applied in various trials. Motivated by TITE-CRM, a variety of TITE designs have been developed. See \citet{zhou2020review} for a review. Recently, a new framework based on probability of decision (POD) has been proposed, represented by the POD-TPI design \citep{zhou2020pod}, to calculate the probability of different decisions by treating the pending outcomes of patients as random  quantities.  The proposed Bi3+3 design adopts the POD framework for the pending outcomes.

In addition to toxicity data,  efficacy data of phase I trials are collected and should be used for dose selection.  For cytotoxic drugs, the efficacy is assumed to increase with  doses and therefore MTD is  also the optimal biological dose (OBD)   since it gives the highest efficacy among all tolerable doses. However, new targeted and immune therapies are often cytostatic in that the efficacy response rate may reach a plateau and stay flat after certain dose level. When the dose level with highest response is lower than the MTD, the MTD is no longer the OBD, rendering selection of the MTD for future studies suboptimal \citep{shah2021drug}. 
%Due to high toxicity and long-term usage of the cancer drugs, there are several circumstances in which the MTD is testified to be over toxic and the final dose were lowered  \citep{shah2021drug}. Thus 
%In \citep{shah2021drug} safety data, but also efficacy and exposure data need to be evaluated carefully in earlier phases. Therefore, choosing recommended phase 2 dose (RP2D) as well as MTD becomes the goal of some phase I trials.
To this end, %Motivated by the development of new trends, 
  the proposed  backfill i3+3 design (Bi3+3)  selects an OBD based on both efficacy and toxicity data at the end of the trial. %Pending outcomes are modeled for backfill cohorts based on the POD framework. 
  Specifically,  a change-point model is proposed for the dose-efficacy response to capture the potential cytostatic relationship. The remainder of the article is organized as follows. Section \ref{sec:stat_model} introduces the proposed design, and Section \ref{sec:mtd_obd_selection} shows how we select MTD and OBD in the Bi3+3 design. Simulations will be presented in Section \ref{sec:simulation}, and discussion about our work is in the last section, $\S$\ref{sec:discussion}.
% \note{Please complete this part. --Y }   %to use alternative strategies, which is a combination of the i3+3 design \citep{liu2020i3+} and backfill strategy. Also, pendings are allowed at backfill cohort and decisions are made using PoD method \citep{zhou2020pod}, thus making it possible to increase the sample size without prolonging the trial. The assessment results in the backfill cohort and main cohort are both used in the MTD selection. In addition, we also model the time-to-event efficacy data using a 4-parameter regression model after the trial to select RP2D. In this paper, we will introduce the structure of the model and the algorithm of patient allocation in $\S$\ref{sec:stat_model}. Simulations will be presented in $\S$\ref{sec:simulation} and discussions about our work in the last part.

\section{Proposed Bi3+3 Design}
\label{sec:stat_model}

%In this section, we will describe the statistical features of the model. $\S$\ref{sec:notation} will illustrate the notations and statistical assumptions.
%$\S$\ref{sec:pat_allocation} will describe the allocation process of the trial. Following $\S$\ref{sec:mtd_rp2d_selection} shows the proposed model to select RP2D and estimate efficacy.
\subsection{Notations}
\label{sec:notation}
Assume a total of $D$ ascending doses of a new therapeutics is under investigation. %The toxicity outcome of each patient is defined as a binary variable called dose-limiting toxicity (DLT) according to CTCAE \citep{???}\note{Same issue when setting a reference of a website.}.
At a given moment of the trial, suppose $n_d$ patients have been allocated to dose $d$ among whom $y_d$ experience the DLT, $d=1, \ldots, D.$   Let $p_d$ denote the toxicity probability at dose $d$, and we assume  
$$y_d | n_d,p_d \sim Bin(n_d,p_d).$$
Also assume the efficacy outcome of each patient to be binary which  is  usually observed   10-12 weeks    after the treatment for solid tumors in oncology trials. Letting the number of patients experiencing efficacy be $v_d$ and the response rate $q_d$ at dose $d$, we   assume    
$$v_d | n_d,q_d \sim Bin(n_d,q_d) \qquad d=1,\cdots, D.$$

The proposed Bi3+3 design extends the algorithm in the i3+3 design \citep{liu2020i3+} given in the table below. %, which will be illustrated in detail in $\S$\ref{sec:pat_allocation}. We follow the same symbols in the i3+3 design and 
Let $p_T$ denote the target toxicity probability of the MTD, which is the highest toxicity rate that can be tolerated in the trial. Define the equivalence interval (EI) as $[p_T - \epsilon_1, p_T + \epsilon_2]$, which describes a range of toxicity probabilities within which doses are deemed equivalent to the MTD. For example, $p_T = 0.25$ and EI $=[0.2, 0.3].$ Then the dose-finding algorithm of the i3+3 design can be summarised  in Table \ref{tbl:i3+3_algorithm}.  Here, a value  ``below” the EI means that the value is less than $(p_T-\epsilon_1)$, the lower bound of the EI. A value  ``inside” the EI means that the value is larger than or equal to $(p_T-\epsilon_1)$ but less than or equal to $(p_T+\epsilon_2)$, the upper bound of the EI. A value  ``above” the EI mean that the value is larger than $(p_T+\epsilon_2)$. Decisions $E$, $S$, and $D$ represent escalation to the next higher dose, stay at the current dose, and de-escalation to the next lower dose, respectively. 

 \begin{table}
       \begin{center}
        \caption{The  decision rules in the i3+3 design. Notation: $d$ represents the current dose being investigated in the trial; $n_d$ and $y_d$ denote the number of patients enrolled and those with DLT at dose $d$, respectively. }
        \label{tbl:i3+3_algorithm}
	\begin{tabular}{|p{8cm} |{c}| c|}
         % \label{tbl:i3+3_algorithm}
		\hline
	%	\multicolumn{3}{|c|}{  Current dose -- $d$; \# enrolled -- $n_d$; \# DLTs -- $y_d$   } \\
	%	\hline \hline
		{\it Condition} & {\it Decision} & {\it Dose for next cohort}\\
		\hline 
		$\frac{y_d}{n_d}$ below EI	& Escalation ($E$)  & $d+1$ \\ \hline
		$\frac{y_d}{n_d}$ inside EI	& Stay ($S$) & $d$ \\ \hline
		$\frac{y_d}{n_d}$ above EI and $\frac{y_d-1}{n_d}$ below EI	& Stay ($S$) & $d$ \\ \hline
		$\frac{y_d}{n_d}$ above EI and $\frac{y_d-1}{n_d}$ inside EI	& De-escalation ($D$) & $d-1$\\ \hline
		$\frac{y_d}{n_d}$ above EI and $\frac{y_d-1}{n_d}$ above EI	& De-escalation ($D$)  & $d-1$\\
		\hline
	\end{tabular}
 \end{center}
\end{table}




Phase I oncology trials enroll patients in cohorts, say a group of three patients   per cohort.    After a cohort is enrolled and assigned to a dose level for treatment, the patients are followed for   three to four weeks    to evaluate drug safety and record any DLT outcomes. While the patients are being followed, new patients may be eligible for trial enrollment. The proposed Bi3+3 design allocates these patients, as backfill cohorts, to lower doses, which have already been tested and exhibit sufficient safety. %We first introduce the overall Bi3+3 algorithm next. % and explain in details the relevant statistical models. 
% Here we distinguish 'main cohort' from 'backfill cohort' and assume a main cohort consists of three patients. At dose $d$, we allocate patients as follows:

 
 Assume dose $d$ is currently used for treating patients in the trial. We denote ``main cohort" (mc) the cohort of patients for dose escalation, and ``backfill cohort" (bc) the cohort of patients for backfill at lower doses. At a given moment of the trial, assume   an    mc of patients is allocated to dose $d$. %, denoted as the current dose. 
 After the mc is allocated, Bi3+3 allocates bc to one or multiple doses  that are  lower than the current dose $d.$ Denote the set of doses assigned to bc by  ${\cal B}_d = \{k_0, k_0 + 1, \cdots, d-1\}$ where %these doses are lower than dose $d$, and 
 $k_0$ in ${\cal B}_d$ is the lowest dose available for backfill.  The proposed Bi3+3 design continuously and adaptively determine $k_0$   throughout the trial.   
 % and $K$ is the total number of doses to which the bc is allocated. By definition, the highest dose for bc is always $b_K = d-1.$ 
 We will describe a method to determine $k_0$ later in $\S$\ref{sec:update_d0}.     But first,  we describe the main algorithm for the Bi3+3 design next. 
 % dose finding under the proposed design next.

\subsection{Patient allocation for Bi3+3}
\label{sec:pat_allocation}
The following algorithm allows investigators to apply the Bi3+3 design for a practical dose-finding trial. The algorithm requires a design called POD-i3 \citep{zhou2020review}, which is also described in Appendix \ref{appendix:pod_method}. 
\begin{enumerate}
    
    % \item Assume a cohort consists of three patients. No more than two cohorts will be allowed to backfill for dose expansion. When a low dose has six backfilled patients, we call that dose 'full'.
    \item Enroll an mc of patients at dose $d$ in the main cohort. \item Once the enrollment for mc is completed, start enrollment for bc by randomly allocating patients in bc to doses in ${\cal B}_d.$ 
    % Three possible backfilling strategies are considered:
    % \begin{enumerate} [label=(\roman*)]
    %     \item only dose $d$ - 1 
    %     \item the highest dose below $d$ that is not full.
    %     \item random allocate backfill patients at dose below $d$ that are not full.
    % \end{enumerate}
    % The Bi3+3 design accommodates either one of the three strategies. In  $\S$ \ref{sec3}, we use strategy (iii) with the start dose discussed in $\S$ \ref{sec2.1}
    \item After the DLT outcomes of patients at the mc are observed, obtain the dosing decision ${\cal S}_d \in \{E, S, D\}$ of dose $d$ based on the i3+3 design (Table \ref{tbl:i3+3_algorithm}), which is $E$, to escalate to dose $(d+1)$, $S$, to stay at dose $d$, or $D$ to de-escalate to dose $(d-1)$.  Do not execute the decision yet. Simply obtain and record it. 
    \item At the same time, follow the steps below and obtain the dosing decision ${\cal T}_k$ for dose $k$ in ${\cal B}_d$, the set of doses assigned to bc. 
    \begin{enumerate}
        \item If one or more patients at dose $k$ are still being followed for DLT assessment without an outcome, apply the POD-i3 design  \citep{zhou2020review} to assess whether the enrollment must be suspend. If yes for any dose $k \in {\cal B}_d$, suspend the trial enrollment at all doses and continue following the patients with pending DLT outcomes. 
        \item Eventually POD-i3 will not suspend enrollment at any dose in ${\cal B}_d.$ Then obtain dosing decision ${\cal T}_k$ for dose $k$ using the POD-i3 design, which is $E$, to escalate to dose $k+1$, $D$, to de-escalate to dose $k-1$, or $S$, to stay at dose $k$. Follow the steps below:
    \begin{enumerate}
        \item If ${\cal T}_k = D$ for at least one dose $k \in {\cal B}_d$, find $k^* = \min\{k \in {\cal B}_d: {\cal T}_k = D\}$    which corresponds to the lowest dose $k^*$  in the bc doses  with  decision $D$. Enroll the next mc at dose $1\vee (k^* - 1)$.
        \item Otherwise,  enroll the next mc based on ${\cal S}_d$.
    \end{enumerate}
    \end{enumerate}
    \item Throughout the trial, if a new outcome is observed at any dose, apply the safety rule in the i3+3 design at all doses \citep{liu2020i3+},  which is detailed in $\S$\ref{sec:safety_rule}.  %\note{Need to write out the rule.}  That is, whether i3+3 assigns decision DU to any dose. If the lowest dose with a DU decision is dose $d'$, remove dose  $d'$ and all the higher doses from the trial, and enroll patients at $d'$ - 1 in the main cohort.
    \item Repeat steps  2-5  until the number of patients in the main cohort reaches a maximum sample size or no doses are left due to safety rules.
\end{enumerate}

\subsection{Determine the doses for backfill} 
\label{sec:update_d0}
In the Bi3+3 algorithm described above, 
% a main question is how to determine ${\cal B}_d$ 
when the mc is allocated to dose $d$, in principle all doses lower than $d$ are available for backfilling, i.e., ${\cal B}_d = \{k_0 = 1, 2, \cdots, d-1 \}.$ % at the beginning of a trial. As    
% This is equivalent to determining the first dose $b_1$ in ${\cal B}_d$ as the last does $b_K$ is always set to   $(d-1)$.   
Due to the up-and-down nature of the decisions in Table \ref{tbl:i3+3_algorithm}, all doses lower than $d$ must have been tested and their safety has been established by the DLT data observed from the patients treated at these doses. Otherwise, it is impossible for the design to reach dose $d$ for mc if any lower dose exhibits excessive toxicity, in which case the dosing decision would not have been $E$, to escalate. % that would not allow dose escalation. 
Therefore, all the doses in ${\cal B}_d$ are deemed safe based on observed data. 
To this end, to determine ${\cal B}_d$, we consider efficacy data from patients and exclude low doses with insufficient efficacy. These doses are considered having little chance to induce beneficial efficacy response from cancer patients and therefore are not worth further exploration.
%, they should be treated at doses with promising therapeutic effects if safety is not a concern. Otherwise, it would be unethical treat them   at doses with inferior efficacy     since the patients in dose-finding trials have usually failed standard therapies and are in great need of effective treatment. 

We follow the idea  in \citet{dehbi2021controlled}. We slightly abuse the notation $d$ hereinafter to either denote the current dose in the trial or a general index of any doses when needed. Occasionally, we also use $k$  or $i$   to index doses when $d$ is used to denote the current doses. Given context, these notations should be clear to readers. We start with   ${\cal B}_d = \{ k_0 = 1, 2, \cdots, d-1 \}$ assuming the current dose is $d$ for the mc.  %\note{I did not follow the previous sentence.    We need to clarify. --Y } 
When a new efficacy outcome is observed from either mc or bc, we refresh the lowest dose $k_0$ in ${\cal B}_d$ based on the following procedure.  Recall the efficacy is assumed to be a binary outcome, e.g., objective response (OR) as in Response Evaluation Criteria in Solid Tumors (RECIST). %, and then a new patient enrolled for the bc is allocated to a dose in ${\cal B}_d$ randomly.   
% and a new patient is enrolled for bc,

To update $k_0$, we first define for a dose $k \in {\cal B}_d$,
  $$ q_{k+} = \frac{\sum_{i > k}^{D} n_i q_i}{\sum_{i > k}^{D} n_i}, $$ 
where $q_i$ is the true efficacy probability for dose $i$, and $n_i$ denotes the number of patients with available efficacy outcomes at dose $i$. By convention,  $0/0 = 0$. The quantity $q_{k+}$ represents the expected response rate at doses higher than dose $k$. % with numbers of patients at these doses as weights.
Under a Bayesian model that will be presented later, we compute the posterior probability %With a Jeffrey's prior $Beta(0.5, 0.5)$ imposed on the response rate $q_k$ of each dose, $q_k$ and $q_{k+}$ can be compared through the posterior probability
$\xi_k = \Pr\{q_{k+} > q_k \mid \mathcal{D}\}, \; k \in {\cal B}_d, $ where $\mathcal{D} = \{(v_d,n_d), d = 1,\cdots,D\}$, the observed efficacy data. %Note $q_k$ is simply the efficacy probability at dose $k$. % Given these available efficacy outcomes, if
If $\xi_k > \xi_0$, say $\xi_0 = 0.8,$ dose $k$ is deemed to be less efficacious than doses higher than $k$. The first dose $k_0$ in ${\cal B}_d$ is set to dose $(k+1)$, i.e.,  $k_0 = k+1$. 
%   When multiple doses $k$ are believed to be less efficacious than their higher doses (i.e., $\xi_k > \xi_0$), $k_0$ is set to be $(k^*+1)$ where $k^*$ is the largest $k$.  
Throughout the trial, posterior probabilities $\xi_k$ for $k \in {\cal B}_d$ are monitored continuously to determine $k_0$ and update ${\cal B}_d$. 

% We can test the hypothesis $H_0:q_k= q^*_k $ by assuming a uniform prior on the interval $(0,1)$ for the true probabilities $q_k$ and $q^*_k$. \note{what's $q^*_k$?} If the posterior probability $\Pr\{q^*_k > q_k\}$ given these available efficacy outcomes is larger than 80\%, $H_0$ is rejected. The hypothesis would first be tested for dose 1 (i.e., $d = 1$) to distinguish the response rates of dose 1 and higher doses. If the hypothesis is rejected, dose 1 is believed less effective, and thus the starting dose of backfilling will be changed to dose 2, $d_0 = 2$. Then we will test the hypothesis $H_0:q_d= q^*_d $ for $d = 2, \cdots, D-1$, successively. \note{Need some rewording...}
\subsection{Safety Rule} 
\label{sec:safety_rule}
Following the i3+3 design \citep{liu2020i3+},
% the mTPI and mTPI-2 design(\citep{ji2010modified},\citep{guo2017bayesian}), 
two safety rules are added as ethical constraints to avoid excessive toxicity.  Both rules are applied in step 5 of the Bi3+3 design (Section \ref{sec:pat_allocation}).  
\begin{itemize}
    \item  \textbf{Rule 1 (Dose Exclusion)}: If the current dose is considered excessively toxic, i.e., $Pr\{p_d > p_T \mid \mathcal{H}\} > \eta$,   where $\mathcal{H} = \{(y_d,n_d), d = 1,\cdots,D\}$ and the threshold  $\eta$  is close to 1, say 0.95, the current and all higher doses are excluded and never be used again in the remainder of the trial.
    \item \textbf{Rule 2: Early Stop}: If the current dose is the lowest dose (the first dose) and is considered excessively toxic according to Rule 1,  stop the trial for safety. 
\end{itemize}

In safety rules 1 and 2, $ Pr\{p_d > p_T \mid \mathcal{H}\}$ is  under the posterior distribution %a function of the beta distribution,
$Beta(\alpha_0+y_d,\beta_0+n_d-y_d)$ which corresponds to an independent $Beta(\alpha_0, \beta_0)$ for $p_d$,   and $\alpha_0=\beta_0=1$ is used.  As a special case of the safety rule 1, if the decision for  the current dose $d$ is ``E" according to the decision rule in \S\ref{sec:pat_allocation}, and if the next higher dose $(d+1)$ level has been excluded due to Rule 1, the decision for the current dose is changed to  ``S" Stay for the current dose because there is no available dose to escalate to. 


\section{MTD and OBD selection}
\label{sec:mtd_obd_selection}
\subsection{MTD selection}
\label{sec:mtd_selection}
After a prespecified maximum sample size of the trial is reached, all the DLT and efficacy data will be collected for all patients enrolled in the trial. We apply the same MTD selection procedure as in many recent designs based on the isotonic regression \citep{ji2010modified, guo2017bayesian, liu2020i3+}. In particular, an independent Beta(0.005, 0.005) prior is assumed for $p_d, \; d=1, \cdots, D.$
% \note{Please confirm the beta(0.5, 0.5) is used. --Yuan } 
And the pooled adjacent violators algorithm (PAVA) \citep{robertson1988order} is applied to the posterior mean toxicity probabilities for all doses, $\hat{p}_d$, $d=1,\cdots,D$, subject to the order constraints $\hat{p}_k > \hat{p}_d$ for $\forall k > d$. Among all tried doses, select as the estimated MTD $\widetilde{d}$ the dose with the smallest difference $\left| \hat{p}_d - p_T \right|$ and $\hat{p}_d \leq p_T + \epsilon_2$, i.e.,
$$\widetilde{d} = \mathop{\mathrm{argmin}}\limits_{\{d: n_d \neq 0, \hat{p}_d \leq p_T + \epsilon_2\}} \left| \hat{p}_d - p_T \right|.$$

\subsection{OBD selection}
\label{sec:obd_selection}
To select the  OBD,   we modify the change-point model in  \citet{dehbi2021controlled} and construct a four-parameter model for the dose-efficacy response. Specifically, recall that $q_d$ is the probability of efficacy at dose $d$. Then assume 
\begin{eqnarray*}
\text{logit}(q_d)=\beta_{0}+\beta_{1} \displaystyle{\left\{ x_d I(x_d \le h)+(\beta_2 +h)I(x_d >h)\right \}.}
\end{eqnarray*}
Here  $x_d \in \{1, \cdots, D\}$ denotes the $D$ discrete dose levels,  $\beta_0$ is the intercept, $\beta_1$ is the slope,  and  $h \in \{1, \cdots, D\}$ is the change point of the model  after which   efficacy  is assumed to plateau.  Under this model, when the dose level is less than or equal to $h$, by assuming $\beta_1 > 0$, there is a positive monotonic relationship between dose level and efficacy response rate; when the dose level is higher than $h$, efficacy response rate no longer increases with dose level and stays  as constant  $\beta_0 + \beta_1(\beta_2 +h)$ at the logit scale.   Quantity $\beta_1 
 \beta_2$ represents the increment in comparison to the  response rate $\beta_0 + \beta_1 h$ at the change point $h$.  This increment is needed to describe the response rate of the dose immediately after the change point $h$ due to discrete dosing.  

The prior  of $\beta_1$ is set as log-normal distribution with mean 0 and a large variance. Parameter $\beta_2$ describes a jump after the change point in response rate, and its prior distribution follows another log-normal. We use  a normal prior with mean  -2 for $\beta_0$ so that the prior probability of response when there is no treatment is small. As for parameter $h$, we set a discrete prior where $h$ takes the value of each dose level from the starting dose 1 to the highest dose $D$  with a probability. Specifically, we 
set the prior probability to be a small value $e$, say $e=0.05,$ for any dose with no enrolled patients, and divide the remaining prior probability  evenly across doses with enrolled patients.   We use $D' \le D$ to denote the number of doses with enrolled patients.  The  prior distributions are given as follows.
\begin{eqnarray*}
% \begin{split}
    \beta_0  \sim  N(-2,10), 
    \beta_1  \sim  LN(0,10), 
    \beta_2  \sim  LN(0,10), \mbox{ and }\\
    h  \sim  Cat(r_{1}, \cdots, r_{D}) \mbox{ where } r_d = e * I(n_d =0) + \frac{1-(D-D')*e}{D'}I(n_d >0),
% \end{split}
\end{eqnarray*}
  where $Cat(r_{1}, \cdots, r_{D})$ is a discrete distribution taking values in $\{1, \cdots, D\}$ with corresponding probabilities $(r_{1}, \cdots, r_{D})$, respectively.    Note $r_d$ is simply the mathematical expression that gives us the aforementioned prior probability for $h$.% , where $D'$ denotes the highest dose with the patients enrolled.\\

Given available efficacy data $\mathcal{D} = \{(v_d,n_d), d = 1,\cdots,D\}$, the posterior probability $\phi_d = \Pr\{h = d \mid \mathcal{D}\}$, $d = 1,\cdots,D$, can be calculated and  used to select OBD. The change point  $h^*$ is estimated as
$$h^* = D' \wedge ( \mathop{\mathrm{argmax}}\limits_{d} \phi_d ).$$
Considering both toxicity and efficacy effects of these doses, we finally select the OBD $d^*$ based on the selected MTD $\widetilde{d}$ and the estimated change point $h^*$ as
$$d^* = \widetilde{d} \wedge ( h^* + 1 ).$$

% Apparently, the OBD based on efficacy is $h^*$. If $h^*$ is higher than the MTD, due to safety, one must select MTD as the OBD. Algorithm 1 below provides the steps to select the OBD based on $h^*$ and MTD. %However, due to the consideration of safety, we will choose the lower dose of selected MTD and the first-set OBD as the final OBD. Here the selected MTD refers to the final selection using the i3+3 method after the trial ends with the DLT data, non-DLT data and dose assignment information both from main cohort and backfill cohort. With the posterior samples of $h$, a detailed algorithm is presented as follows:
% \begin{algorithm} 
% 	\caption{Select OBD} 
% 	\label{alg3} 
% 	\begin{algorithmic}
% 	    \REQUIRE  Posterior samples of $h$ using Markov chain Monte Carlo for the change-point efficacy model. 
% 		\IF{$h^* = D $} 
% 		\STATE $OBD = D$ 
% 		\ELSE 
% 		\STATE $OBD = h^* + 1 $ 
% 		\ENDIF 
% 		\RETURN  temporary OBD = maximum posterior probability of $OBD_d$
% 		\IF {  temporary OBD $>$ selected MTD}
% 		\STATE OBD = selected MTD 
% 		\ELSE
% 		\STATE OBD= temporary OBD
% 		\ENDIF 
% 	\end{algorithmic} 
% \end{algorithm}


\section{Simulation}
\label{sec:simulation}
  We conduct simulated trials to assess the performance of the Bi3+3 design. We generate patient toxicity and efficacy data based on a set of scenarios. %In the simulation part, we use the algorithm illustrated in $\S$\ref{sec:stat_model} to conduct the trial.
For each scenario, we run 1,000 simulated trials.    To  mimic  real-world situations, we assume the toxicity and efficacy outcomes of any patients are   not immediately observed. Instead, we assume the time to DLT follows a uniform distribution ranging from 0 to the maximum DLT follow-up time if a DLT occurs for the patient;  otherwise we assume DLT is censored.  The efficacy outcomes are observed 90 days after patients' enrollment. Also, the arrival time of patients is assumed to follow an exponential distribution with a mean of 10 days, which means, on average, every 10 days a new patient is eligible for enrollment of the trial, and hence the trial would enroll about three patients per month.% on average.   

In $\S$\ref{sec:comparison_bcrm},we specify  several scenarios with a different target toxicity probability and a different number of doses, and compare Bi3+3 with the Backfill CRM design \citet{dehbi2021controlled}. In $\S$\ref{sec:com_mTPI2} we compare Bi3+3 with the mTPI-2 design for trial duration.  
% \note{Need to fix the reference number here. --Y}


\subsection{Comparison with Backfill CRM}
\label{sec:comparison_bcrm}
We first compare Bi3+3 with the Backfill CRM design in \citet{dehbi2021controlled}.    %For readers' information, below
We briefly summarize the Backfill CRM design below, which provides instructions for allocating patients in the main cohort and backfill cohort. 
\begin{itemize}
    \item Main cohort allocation
    \begin{enumerate}
        \item A one-parameter CRM model \citep{o1990continual} is used for dose assignment and MTD selection, only using data in the main cohort.
    \end{enumerate}
    \item Backfill cohort allocation
    \begin{enumerate}
        \item Starting with the second cohort of patients in the main cohort, a backfill cohort with three patients will be enrolled after the main cohort enrollment is complete.
        \item The patients are randomly assigned to backfill doses  with equal probability. 
        \item Similar to the rules in $\S$\ref{sec:update_d0}, doses with low efficacy response rates are removed from backfill doses.   
    \end{enumerate}
    \item OBD selection
    \begin{enumerate}
        \item After the trial ends, model selection is performed to select a monotonic logistic regression {or} a change-point logistic regression.
        \item The OBD is decided based on the selected model. If the change-point model is selected, doses are divided to two parts, monotone and plateau parts. The dose on the plateau and closest to the change point is selected as the OBD.
    \end{enumerate}
\end{itemize}

We implement the Backfill CRM design and benchmark our implementation against the simulation results in \citep{dehbi2021controlled}. %As CRM design is used in the dose-finding trial, the author use the prespecified skeleton. The reproduced
Our results, shown in Appendix \ref{appendix:reproduction}, are very close to those presented in \citep{dehbi2021controlled}, thus  assuring   our implementation. We then compare the Bi3+3 design with Backfill CRM on a variety of scenarios. See Figure \ref{fig:sc_vs_crm} for the five scenarios used in our simulation. %. using the same skeleton with the author. Their are similar outcomes of Backfill CRM and the reproduced one, which is shown in \ref{tbl:reproduce1} in detail.Since then, we begin to compare Bi3+3 with Backfill CRM using other setups.
In the subsequent simulation,    the target probability of toxicity for MTD  is set to  $p_T =0.3,$   and the equivalence interval for Bi3+3 is  $EI=[0.25,0.35].$  As the backfill strategy is different in Bi3+3 design and Backfill CRM design,   we try to match the overall sample size (main cohorts + backfill cohorts) for both designs for fair comparison.  For the Bi3+3 design,  30 patients are enrolled in the main cohorts which result in an average sample size of 36.7, with the additional 6.7 patients coming from the backfill cohorts.  As for the Backfill CRM design, we set 21 as the main cohort sample size which produces an average of 39.0 patients for the entire trial.  We use the skeleton in \cite{lee2011calibration} for Backfill CRM, a popular choice in practice.    %The Bi3+3 selects the MTD and the OBD at the end of the trial based on the algorithms in \S\ref{sec:mtd_selection} and \ref{sec:obd_selection}.
The operating characteristics of both designs are  presented in Table \ref{tbl:oc_vs_crm}. 
% \note{fix the table number. --Y}   
%Lastly,   

\begin{figure}[!htbp]
    \centering
    \includegraphics[width=1.0\textwidth]{5sce-1.pdf}
    \caption{  The five simulation scenarios with target toxicity probability of 0.3. The (MTD, OBD) are (5,3), (4,3), (5,5), (5,5), and (4,4) for scenarios 1-5, respectively.   Bold fonts indicate the reported values are for the true OBDs.   }
    \label{fig:sc_vs_crm}
\end{figure}

\begin{table}[!htbp]
\scriptsize
\begin{center}
\caption{Comparison of Bi3+3 and Backfill CRM in five scenarios. The target toxicity probability is 0.3 and EI =[0.25,0.35].} %, sample sizes are set at 30 for main cohorts for Bi3+3 and 2 for the entire trial for Backfill CRM. }
\label{tbl:oc_vs_crm}
\begin{tabular}{@{}cccccc|ccccc@{}}
\toprule
 & \multicolumn{5}{c|}{Backfill i3+3}  & \multicolumn{5}{c}{Backfill CRM}\\ \midrule
 Dose level &  1&2  &3  &4  &5 &  1&2  &3  &4  &5 \\ \midrule
\textbf{Scenario 1}&  &  &  &  & &  &  &  &  &  \\
Efficacy probability& 0.1 & 0.3 &0.5  &0.5  &0.5 & 0.1 & 0.3 &0.5  &0.5  &0.5 \\
Toxicity probability& 0.01 &0.05  & 0.1  &0.25  & 0.31 & 0.01 &0.05  & 0.1  &0.25  & 0.31\\
\% of OBD selection& 0\%  & 14.6\% &\textbf{32.9\%}  &25.1\%  &27.4\%  & 0\% &0.8\%  &40.1\% &38.4\%  &20.6\% \\
\% of MTD selection& 0\% & 0.5\% &  20.2\% & 37.6\% & 41.7\% &0\% &0\%  &6\%  &33.8\%  &60.1\% \\
Efficacy estimation& 0.25 &0.33  &0.43  &0.48  &0.52 &0.23  &0.32  & 0.41 & 0.47 &0.51 \\ 
Number of patients enrolled & 4.1 & 5.7 & \textbf{9.0} & 9.7 & 8.1&8.4  &8.6  & 8.7 &7.7  &5.6\\
Number of patients backfill &  1.0 &1.9 &2.4 &1.4 &0& 5.3 & 5.5 & 4.8 & 2.4 &0\\
%Trial Duration& \multicolumn{5}{c}{494}  & \multicolumn{5}{c}{549}\\
\textbf{Scenario 2}&  &  &  &  &  &  &  &  &  &\\
Efficacy probability& 0.05 &0.15  & 0.3 &0.3  &0.3& 0.05 &0.15  & 0.3 &0.3  &0.3 \\
Toxicity probability&0.06 &0.1  & 0.15  &0.3  & 0.38& 0.06 &0.1  & 0.15  &0.3  & 0.38 \\
\% of OBD selection& 0.4\%  & 10.8\% &\textbf{37.4\%}  & 34.6\% &16.8\%  & 0\% &6.6\%  &49.8\%  &35.1\%  &8.2\%\\
\% of MTD selection& 0.4\% & 3.6\% & 33.3\% & 41.5\% & 21.5\%& 0\% & 3.3\% & 24.2\% & 47.3\% & 24.9\%\\
Efficacy estimation& 0.15 &0.19  &0.25  &0.29  &0.34 &0.14&0.18&0.23&0.28&0.31 \\ 
Number of patients enrolled& 5.6 & 7.0 & \textbf{10.0} & 8.9 & 4.8& 10.7 &9.4  &8.9 &6.7  &3.4\\
Number of patients backfill & 1.7  &2.0 &1.7 &0.6 &0& 6.8 &5.7  &3.9 &1.6  &0\\
\textbf{Scenario 3}&  & &  &  & &  &  &  &  & \\
Efficacy probability& 0.07 &0.14  &0.21  &0.28  &0.35& 0.07 &0.14  &0.21  &0.28  &0.35 \\
Toxicity probability& 0.06 &0.12  &0.18  &0.24  &0.3 &0.06 &0.12  &0.18  &0.24  &0.3\\
\% of OBD selection& 0.8\% & 10.6\% & 24.1\% & 31.5\% & \textbf{33.0\%} &0\%&10.6\%  &41.6\%  & 31.1\% & 16.4\%\\
\% of MTD selection& 0.8\% & 6.2\% & 22.7\% & 33.5\% & 36.8\% &0\% &5\%  &22.2\%  &42.0\% & 30.5\%\\
Efficacy estimation& 0.13 & 0.17 &0.21  & 0.26 &0.31 &0.13 &0.16& 0.20 &0.24  &0.28 \\ 
Number of patients enrolled & 6.3 & 8.0 & 9.1 & 7.5 & \textbf{6.1}&11.2  &9.6 &8.5 &5.9  &3.7\\
Number of patients backfill & 2.3  &2.2 &1.6 &0.9 &0& 7.3 & 5.7 &3.6  &1.4  &0\\
%Trial Duration& \multicolumn{5}{c}{492}  & \multicolumn{5}{c}{549}\\
\textbf{Scenario 4}&  &  &  &  &&  &  &  &  &  \\
Efficacy probability& 0.04 &0.08  &0.12  &0.16  &0.2& 0.04 &0.08  &0.12  &0.16  &0.2 \\
Toxicity probability& 0.04 &0.08  &0.15  &0.21  &0.32& 0.04 &0.08  &0.15  &0.21  &0.32\\
\% of OBD selection& 0.1\% & 5.3\% &14.9\%  & 39.2\% & \textbf{40.5\%}& 0\% &4.4\%  &45.5\%  &28.1\%  &21.9\% \\
\% of MTD selection& 0.1\% & 3.2\% & 14.3\% & 40.1\%  & 42.3\%& 0\% & 1.1\% &11.6\%  &43.9\%  &43.3\% \\
Efficacy estimation&  0.09& 0.10 & 0.13  &0.16  &0.19 & 0.08 &0.10  &0.12  &0.14  &0.16 \\ 
Number of patients enrolled & 5.9 & 7.0 & 8.8 & 8.4 & \textbf{7.8}& 10.1 & 9.1 & 8.5& 6.6 &4.7\\
Number of patients backfill & 2.4 &2.4 &2.0 &1.1 &0& 6.7 &5.7  &3.9  &1.7  &0\\
\textbf{Scenario 5}&  &  &  &  & &  &  &  &  & \\
Efficacy probability& 0.1 &0.2  &0.3  &0.4  &0.4& 0.1 &0.2  &0.3  &0.4  &0.4 \\
Toxicity probability& 0.08 &0.16  & 0.24  &0.3  & 0.38& 0.08 &0.16  & 0.24  &0.3  & 0.38 \\
\% of OBD selection& 3.0\% & 26.3\% & 31.8\% & \textbf{26.1\%} & 12.7\% &0\%  &20.7\%  &49.1\% &24.5\%&4.8\%\\
\% of MTD selection& 3.0\% & 19.4\% & 31.6\% & 30.0\% & 15.9\% &0\% & 13.6\% &39.9\%  & 33.9\% &11.7\%\\
Efficacy estimation& 0.17 & 0.22 &0.29  &0.36  &0.40 &0.16  &0.21  & 0.27 & 0.32 &0.36 \\ 
Number of patients enrolled & 7.4 & 9.9 & 9.5 & \textbf{6.0} & 3.3&12.5  &10.4  &  8.7& 5.0 &2.3\\ 
Number of patients backfill &2.5 &2.0 &1.1 &0.4 &0&8.0  &5.8  &3.2  &1.0  &0\\\bottomrule
\end{tabular}
\end{center}
\end{table}


% \begin{table}[!htbp]
% \begin{center}
% \caption{Backfill start doses after the entire trial}
% \label{tbl:back_start}
% \begin{tabular}{@{}cccccccc@{}}
% \toprule
%  & \multicolumn{7}{c}{Dose level} \\ \midrule
%  & 1  &2   & 3 & 4 &5  & 6 &7  \\  \midrule
% Scenario A &  29.7\% & 31.7\%  & 25.8\% & 8.5\% & 3.8\% & 0.5\% &0\%  \\
% Scenario B & 66.8\%  & 18.1\%  & 11.9\% & 2.3\% & 0.9\% &0\%  & 0\% \\
% Scenario C & 60.3\%  & 19.3\%  & 12.0\% & 4.1\% &3.1\%  & 1.2\% &0\%  \\
% Scenario D & 19.5\%  & 13.4\%  & 20.2\% & 16.1\% & 10.4\% & 10.4\% & 0\% \\
% Scenario E & 45.9\%  &23.1\%   & 18.6\% & 9.9\% & 2.0\% &0.5\%  & 0\% \\
% Scenario F &  53.7\% & 20.2\%  & 15.0\% & 7.9\% & 2.1\% & 1.1\% & 0\%  \\
% \bottomrule
% \end{tabular}
% \end{center}
% \end{table}


For scenarios 1 and 2, the true MTD is the fifth dose and fourth dose, respectively. The efficacy response rates of doses reach a plateau at the third dose, which is the true OBD. In scenarios 1 and 2, Bi3+3 selects the true MTD in 41.7\% and 41.5\% of the simulated trials, highest among all the doses, and it selects the true OBD, dose level three,  32.9\% and 37.4\% of the times. The Backfill CRM design has slightly higher percentages in OBD selection in these two scenarios although it also tends to select the next  higher dose with a high frequency. %leading to the lower identification of cutting point h. 33.3\% of the simulations eliminate dose 1 due to the low efficacy in scenario A. Scenario B, however, 63.2\% of the simulations retain dose 1 as the response curve is flatter than scenario A, which makes it harder to recognize the difference between doses. There are 
On average, Bi3+3 assigns 9.0 and 9.7 patients to dose levels three and four in scenario 1, respectively. Compared with Backfill CRM, the total sample size of Bi3+3 is 2.3 patients smaller, although there is not much a difference in the patient  allocation to the OBDs. In scenarios 3 and 4 both toxicity and efficacy increase with dose level and the MTD and OBD are the same, which is dose level five. Bi3+3 selects  dose five as the MTD with 36.8\% and 42.3\% frequencies, and as the OBD with 33.0\% and 40.5\% frequencies in scenarios 3 and 4, respectively. In contrast, the Backfill CRM design selects dose three as OBD with probabilities 41.6\% and 45.5\% in the two scenarios, which is less desirable.    %indicating that the Backfill CRM design has the tendency to select the median dose without the prespecified skeleton.
% Also, the proposed Bi3+3 requires that the selected OBD to be no higher than the selected MTD, making the selection of OBD conservative.
For patient allocation, Bi3+3 allocates on average 6.1 patients to dose five while  Backfill CRM  allocates a slightly lower 3.7 patients to the same dose. In the last scenario 5, efficacy plateaus at a dose same with the true MTD, indicating that the true MTD is the same as the true OBD, which is dose four. Both Bi3+3 and Backfill CRM choose dose   three  as the MTD and OBD. On average, Bi3+3 allocate about one more patient to the true OBD than Backfill CRM.
Backfill CRM is more conservative than Bi3+3 in that it allocates more patients to lower doses. This can be seen across all five scenarios. In addition, it tends to select doses in the middle and is excellent in avoid selecting high doses with high frequencies.  However, in cases where the true OBD is at a high dose, like scenarios 3 and 4, Backfill CRM may have less chance finding them. In summary, both designs exhibit desirable features in these five scenarios.

Apart from these features, we also calculate the mean of efficacy using the regression model in each scenario. The results are reported in Table \ref{tbl:oc_vs_crm} corresponding to the line ``Efficacy estimation". The estimated efficacy response rates in general do not appear to deviate much from the true efficacy probabilities. % efficacy estimation though sometimes higher than the true efficacy, it is still able to distinguish between different scenarios. \note{Either present some results or remove this paragraph? --Y}
%\note{This paragraph is refer to the efficacy estimation in Table \ref{tbl:oc_vs_crm}. --S}

%Apart from this, we also adopt the six scenarios in \citet{dehbi2021controlled} for comparison. The overall sample size in Backfill CRM is 57 patients including those in 10 main cohorts and 9 backfill cohorts. In the proposed Bi3+3 design, because the arrival time of patients is sampled from an exponential distribution, the number of patients in the backfill cohort is random. Therefore, we set the maximum sample size of the main cohort to 42, which leads to the overall sample size of about 55 across the scenarios (see Table \ref{tbl:outcome1} in Appendix \ref{appendix:com_BCRM}). The target probability of toxicity for MTD is set to 0.25, and the equivalence interval for Bi3+3 is [0.2,0.3]. 
% The performance of dose elimination is not satisfying as the high percentage of the backfilling start from dose 1.

% \begin{minipage}{\textwidth}

% \begin{minipage}[t]{0.48\textwidth}
% \makeatletter\def\@captype{table}
% \begin{tabular}{@{}cccccccc@{}}
% \toprule
%  & \multicolumn{7}{c}{Dose level} \\ \midrule
%  & 1  &2   & 3 & 4 &5  & 6 &7  \\  \midrule
% \textbf{Scenario A}&   &   &  &  &  &  &  \\
% Efficacy probability &   &   &  &  &  &  &  \\
% Toxicity probability&   &   &  &  &  &  &  \\
% \% of RP2D selection&   &   &  &  &  &  &  \\
% \% of MTD selection&   &   &  &  &  &  &  \\
% Efficacy estimation&   &   &  &  &  &  &  \\ \bottomrule
% \end{tabular}

% \caption{Sample table title}
% \label{sample-table}
% \end{minipage}
% \begin{minipage}[t]{0.48\textwidth}
% \makeatletter\def\@captype{table}

% \begin{tabular}{@{}cccccccc@{}}
% \toprule
%  & \multicolumn{7}{c}{Dose level} \\ \midrule
%  & 1  &2   & 3 & 4 &5  & 6 &7  \\  \midrule
% \textbf{Scenario A}&   &   &  &  &  &  &  \\
% Efficacy probability &   &   &  &  &  &  &  \\
% Toxicity probability&   &   &  &  &  &  &  \\
% \% of RP2D selection&   &   &  &  &  &  &  \\
% \% of MTD selection&   &   &  &  &  &  &  \\
% Efficacy estimation&   &   &  &  &  &  &  \\ \bottomrule
% \end{tabular}

% \caption{Sample table title}
% \label{sample-table}
% \end{minipage}
% \end{minipage}


\subsection{Comparison with   the    mTPI-2 design given $p_T=0.3$}
\label{sec:com_mTPI2}
%Since the number of dosages reduces, the plateau becomes shorter and the RP2D might become more difficult to identify. Thus we use 5 dose scenarios to check the algorithm's generalization ability. The toxicity probability and response rate are almost the same as the 7 dose scenarios so there are also scenarios with plateau or without plateau. In the following scenarios, 
Next, we apply mTPI-2 \citep{guo2017bayesian} to illustrate some features of Bi3+3 as mTPI-2 is a design that does not allow patient backfill. %The main cohort sample size is set to 30.    when compared to a standard dose-finding design without backfill. %that increase the sample size without prolonging the trial time. 
For fair comparison, we first enroll 30 patients for mTPI-2, select an MTD, and allow six more patients to be enrolled at the selected MTD (to adjust for the fact that Bi3+3 would enroll more patients in the backfill patients). After the six more patients are enrolled and their DLT data observed, we re-apply mTPI-2 and re-select the MTD based on the additional data. We report the frequency of the re-selected MTD in the simulation results for mTPI-2. %we enroll six more patients at the selected MTD for the mTPI-2 method and adjust the final MTD with all the patients' outcomes.\\


%\note{Combine results in Table \ref{tbl:5sce-3} with this table. 
%Summarize results in Table 3 into a few sentences. No need to go through results by each scenarios. I think the main findings are that the Bi3+3 design allows more patients to be tested on various doses without lengthening the trial. Also, I think we probably want to change some scenarios to make doses have larger differences in efficacy and toxicity rates, and make MTD above the OBD.}


%Compared with the 7 dose scenarios, the RP2D selection is similar as scenario 1 chose correctly but scenario 2 didn't. However, the algorithm made the wrong decisions in choosing MTD, as the true MTD is dose 4 but the algorithm inclines to dose 3. Considering there are minor errors in the MTD selection, the RP2D selection provide more information. For example, in scenario 1, the MTD selection of dose 3 and 4 are 32.3\% and 29.4\%, a minor gap. However, the percentage of RP2D selection of these 2 doses are 30.0\% and 25.4\%, indicating the difference in response rate. Scenario 3 and 4 perform well as they choose MTD and RP2D correctly. As for the last scenario, the percentage of selecting dose 3 \& 4 are larger than 60\% in all, indicating the effectiveness of the model.\\
%As the toxicity probability in scenario 2 and 4 are the same as scenario 1 and 3, we compare scenario 1,3 and 5 with the mTPI-2 design. The right part of Table \ref{tbl:5sce-1} shows the MTD selection and patient enrollment of mTPI-2 design. In scenario 1 and 5, the mTPI-2 design didn't select the right MTD either. However, in scenario 3, which mTPI-2 design also failed. For the trial duration average of three scenarios, Backfill i3+3 has a duration of 491.2 days with 34.8 patients enrolled, in the contrast, mTPI-2 enroll 36 patients need 533.3 days on average. That is to say, though Backfill i3+3 design enroll only 1.2 fewer patients, its trial duration is still 42.1 days shorter than mTPI-2, indicating its capability for enrolling more patients in non-extended time.
% For the purpose of comparison, we keep the toxicity probability unchanged and double the response rate as an attempt. The outcome is presented in the right part of \ref{tbl:5sce-1}. As the gap of efficacy between doses become larger, it is easier for scenario 1 and 2 to select the true RP2D with the higher probability. In scenarios 3-5, the outcomes are similar as there are only minor changes. What's more, one need to notice that with the existence of dose elimination with low efficacy, fewer patients are backfilled in the low doses in the double efficacy scenarios.\\

% \begin{table}[!htbp]
% \scriptsize
% \begin{center}
% \caption{Outcomes of 5 doses scenarios}
% \label{tbl:5sce-2}
% \begin{tabular}{@{}cccccc@{}}
% \toprule
%  & \multicolumn{5}{c}{Dose level} \\ \midrule
%  &  1&2  &3  &4  &5  \\ \midrule
% \textbf{Scenario 1}&  &  &  &  &  \\
% Efficacy probability& 0.1 & 0.3 &0.5  &0.5  &0.5  \\
% Toxicity probability& 0.06 &0.13  & 0.19  &0.25  & 0.31 \\
% \% of RP2D selection& 3.2\%  & 31.3\% &35.1\%  &20.6\%  &9.8\%  \\
% \% of MTD selection& 3.2\% & 19.2\% &  31.9\% & 30.5\% & 15.2\% \\
% Efficacy estimation& 0.23 &0.32  &0.43  &0.48  &0.51  \\ 
% Number of patients & 5.9 & 8.7 & 9.4 & 6.4 & 4.3\\
% \textbf{Scenario 2}&  &  &  &  &  \\
% Efficacy probability& 0.1 &0.2  & 0.3 &0.3  &0.3  \\
% Toxicity probability&0.06 &0.13  & 0.19  &0.25  & 0.31 \\
% \% of RP2D selection& 3.1\%  & 31.6\% &29.2\%  & 23.9\% &12.2\%  \\
% \% of MTD selection& 3.1\% & 20.2\% & 31.7\% & 29.3\% & 15.7\% \\
% Efficacy estimation& 0.17 &0.21  &0.27  &0.31  &0.34  \\ 
% Number of patients & 7.2 & 9.3 & 9.4 & 6.3 & 4.4\\
% \textbf{Scenario 3}&  & &  &  &  \\
% Efficacy probability& 0.08 &0.16  &0.24  &0.32  &0.4  \\
% Toxicity probability& 0.04 &0.08  &0.13  &0.17  &0.25  \\
% \% of RP2D selection& 0.6\% & 11.6\% & 17.8\% & 35\% & 35\% \\
% \% of MTD selection& 0.6\% & 5.5\% & 16.9\% & 37.7\% & 39.3\% \\
% Efficacy estimation& 0.16 & 0.20 &0.25  & 0.30 &0.35  \\ 
% Number of patients & 5.8 & 7.2 & 8.2 & 7.8 & 8.5\\
% \textbf{Scenario 4}&  &  &  &  &  \\
% Efficacy probability& 0.14 &0.28  &0.42  &0.56  &0.7  \\
% Toxicity probability& 0.04 &0.08  &0.13  &0.17  &0.25  \\
% \% of RP2D selection& 0.3\% & 12.5\% &20.4\%  & 31.4\% & 35.4\% \\
% \% of MTD selection& 0.3\% & 5.6\% & 17.9\% & 35.7\%  & 40.5\% \\
% Efficacy estimation&  0.24& 0.32 & 0.43  &0.54  &0.60  \\ 
% Number of patients & 4.9 & 6.5 & 7.9 & 7.7 & 8.5\\
% \textbf{Scenario 5}&  &  &  &  &  \\
% Efficacy probability& 0.1 &0.2  &0.3  &0.4  &0.4  \\
% Toxicity probability& 0.04 &0.08  &0.16  &0.25  &0.35  \\
% \% of RP2D selection& 0.3\% & 19.2\% & 36.3\% & 32.7\% & 11.5\% \\
% \% of MTD selection& 0.3\% & 10.8\% & 37.1\% & 37.6\% & 14.2\% \\
% Efficacy estimation& 0.20 & 0.23 &0.30  &0.36  &0.40  \\ 
% Number of patients & 5.7 & 7.7 & 10.1 & 8.0 & 5.0\\\bottomrule
% \end{tabular}
% \end{center}
% \end{table}

%\subsection{Sensitivity analysis}
%\label{sec:sens}
The results comparing Bi3+3 and mTPI-2 are presented in Table \ref{tbl:oc_vs_mtpi2_pt0.3}. Bi3+3 has an advantage in many scenarios as it allows patients to be backfilled. For example, it does not increase trial duration while allowing more patients to be assigned to lower doses, resulting in better OBD selection. See Table \ref{tbl:oc_vs_mtpi2_pt0.3} for a summary.    %For the scenarios that OBD is smaller than MTD or efficacy goes up monotonically, the algorithm can make the true decisions with the highest probability. However, it is hard to select true OBD when the efficacy with plateau and OBD equals MTD, as in scenario 5. 

As a sensitive analysis, we vary the target toxicity probability from $p_T=0.3$ to $p_T=0.25$ and change the EI from [0.25,0.35] to [0.2,0.3] (Table \ref{tbl:oc_vs_mtpi2_pt0.25}).    
We also construct four more scenarios to assess the trial duration and patients enrollment. Bi3+3 shows desirable results as expected, thanks to the ability to backfill patients and time-to-event modeling (Table \ref{tbl:duration_vs_mtpi2}).    %Here we set $p_T$=0.3 and the sample size in the main cohort to be 30. Also, 6 more patients will be enrolled in the mTPI-2 design after the MTD selection.From \ref{tbl:5sce-3} we can see that among all 4 scenarios, the Backfill i3+3 design have a shorter duration with a similar sample size.   % we change $p_T$ to 0.3 in the following scenarios. In addition to target probability, we also used two common toxicity probability here. However, one need to notice that these scenarios have a commonality that the true MTD is at dose 4 or 5. This is due to in the 5 dose scenario, as we require RP2D to be smaller or equal than the MTD, in the scenarios where the true MTD is 1 or 2, it would make little sense to choose RP2D. Also, summarizing from the scenarios before, we set the following scenarios with both high and low efficacy to show the adaptability of the model. The scenario is represented in figure  \ref{fig:5dose-1} and the outcome is shown in Table \ref{tbl:5sce-2}.\\

\begin{table}[!htbp]
\tiny
\begin{center}
\caption{Comparison of Bi3+3 and mTPI-2 in five scenarios. The target toxicity probability is 0.3, EI =[0.25,0.35], sample size is 36.8 for Bi3+3 and 36.0 for mTPI-2.}
\label{tbl:oc_vs_mtpi2_pt0.3}
\begin{tabular}{@{}cccccc|ccccc@{}}
\toprule
 & \multicolumn{5}{c|}{Dose level}  & \multicolumn{5}{c}{m-TPI2}\\ \midrule
 Dose level &  1&2  &3  &4  &5 &  1&2  &3  &4  &5 \\ \midrule
\textbf{Scenario 1}&  &  &  &  & &  &  &  &  &  \\
Efficacy probability& 0.1 & 0.3 &0.5  &0.5  &0.5  &  &  &  &  &\\
Toxicity probability& 0.01 &0.05  & 0.1  &0.25  & 0.31 & 0.01 &0.05  & 0.1  &0.25  & 0.31\\
\% of OBD selection& 0\%  & 14.6\% &\textbf{32.9\%}  &25.1\%  &27.4\%  &  &  &  &  & \\
\% of MTD selection& 0\% & 0.5\% &  20.2\% & 37.6\% & 41.7\% &0\% &0.4\%  &20.2\%  &40.0\%  &39.4\% \\
Efficacy estimation& 0.25 &0.33  &0.43  &0.48  &0.52 &  &  &  &  & \\ 
Number of patients enrolled & 4.1 & 5.7 & \textbf{9.0} & 9.7 & 8.1&3.1  &3.8  & 7.4 &10.8  &10.9\\
Number of patients backfill &  1.0 &1.9 &2.4 &1.4 &0&  &  &  &  &\\
Trial Duration& \multicolumn{5}{c}{494}  & \multicolumn{5}{c}{549}\\
\textbf{Scenario 2}&  &  &  &  &  &  &  &  &  &\\
Efficacy probability& 0.05 &0.15  & 0.3 &0.3  &0.3&  &  &  &  &  \\
Toxicity probability&0.06 &0.1  & 0.15  &0.3  & 0.38& 0.06 &0.1  & 0.15  &0.3  & 0.38 \\
\% of OBD selection& 0.4\%  & 10.8\% &\textbf{37.4\%}  & 34.6\% &16.8\%  &  &  &  &  &\\
\% of MTD selection& 0.4\% & 3.6\% & 33.3\% & 41.5\% & 21.5\%& 0.1\% & 3.6\% & 37.9\% & 39.9\% & 18.5\%\\
Efficacy estimation& 0.15 &0.19  &0.25  &0.29  &0.34 &  &  &  &  & \\ 
Number of patients enrolled& 5.6 & 7.0 & \textbf{10.0} & 8.9 & 4.8& 3.8 &5.1  &9.9 &10.8  &6.3\\
Number of patients backfill & 1.7  &2.0 &1.7 &0.6 &0&  &  &  &  &\\
Trial Duration& \multicolumn{5}{c}{489}  & \multicolumn{5}{c}{545}\\
\textbf{Scenario 3}&  & &  &  & &  &  &  &  & \\
Efficacy probability& 0.07 &0.14  &0.21  &0.28  &0.35&  &  &  &  &  \\
Toxicity probability& 0.06 &0.12  &0.18  &0.24  &0.3 &0.06 &0.12  &0.18  &0.24  &0.3\\
\% of OBD selection& 0.8\% & 10.6\% & 24.1\% & 31.5\% & \textbf{33.0\%} &  &  &  &  &\\
\% of MTD selection& 0.8\% & 6.2\% & 22.7\% & 33.5\% & 36.8\% &0.2\% &6.7\%  &24.8\%  &33.9\% & 34.4\%\\
Efficacy estimation& 0.13 & 0.17 &0.21  & 0.26 &0.31 &  &  &  &  & \\ 
Number of patients enrolled & 6.3 & 8.0 & 9.1 & 7.5 & \textbf{6.1}&3.9  &6.3 &8.8 &8.5  &8.5\\
Number of patients backfill & 2.3  &2.2 &1.6 &0.9 &0&  &  &  &  &\\
Trial Duration& \multicolumn{5}{c}{492}  & \multicolumn{5}{c}{549}\\
\textbf{Scenario 4}&  &  &  &  &&  &  &  &  &  \\
Efficacy probability& 0.04 &0.08  &0.12  &0.16  &0.2&  &  &  &  &  \\
Toxicity probability& 0.04 &0.08  &0.15  &0.21  &0.32& 0.04 &0.08  &0.15  &0.21  &0.32\\
\% of OBD selection& 0.1\% & 5.3\% &14.9\%  & 39.2\% & \textbf{40.5\%}&  &  &  &  & \\
\% of MTD selection& 0.1\% & 3.2\% & 14.3\% & 40.1\%  & 42.3\%& 0\% & 3\% &15.8\%  &43.8\%  &37.4\% \\
Efficacy estimation&  0.09& 0.10 & 0.13  &0.16  &0.19 &  &  &  &  & \\ 
Number of patients enrolled & 5.9 & 7.0 & 8.8 & 8.4 & \textbf{7.8}& 3.5 & 4.7 & 7.6& 9.8 &10.4\\
Number of patients backfill & 2.4 &2.4 &2.0 &1.1 &0&  &  &  &  &\\
Trial Duration& \multicolumn{5}{c}{495}  & \multicolumn{5}{c}{551}\\
\textbf{Scenario 5}&  &  &  &  & &  &  &  &  & \\
Efficacy probability& 0.1 &0.2  &0.3  &0.4  &0.4 &  &  &  &  & \\
Toxicity probability& 0.08 &0.16  & 0.24  &0.3  & 0.38& 0.08 &0.16  & 0.24  &0.3  & 0.38 \\
\% of OBD selection& 3.0\% & 26.3\% & 31.8\% & \textbf{26.1\%} & 12.7\% &  &  &  &  &\\
\% of MTD selection& 3.0\% & 19.4\% & 31.6\% & 30.0\% & 15.9\% &2.4\% & 21.4\% &33.6\%  & 29.2\% &13.3\%\\
Efficacy estimation& 0.17 & 0.22 &0.29  &0.36  &0.40 &  &  &  &  & \\ 
Number of patients enrolled & 7.4 & 9.9 & 9.5 & \textbf{6.0} & 3.3&4.9  &8.9  &  10.4& 7.3 &4.4\\ 
Number of patients backfill &2.5 &2.0 &1.1 &0.4 &0&  &  &  &  &\\
Trial Duration& \multicolumn{5}{c}{488}  & \multicolumn{5}{c}{543}\\\bottomrule
\end{tabular}
\end{center}
\end{table}

%The outcome is similar with the former 5 dose scenarios as we change the probability but keep the cutting point unchanged, as has been discussed above. Scenario 1 and 2 select the OBD successfully with the probability of 32.9\% and 37.4\%, respectively. Also, though true MTD is dose 5 and dose 4 in scenario 1 and 2, more patients are allocated to dose 3 and 4 as we expected. In scenario 5, dose 3 is chose as OBD and MTD, which is lower than the truth.\\
%The right part of Table \ref{tbl:5sce-2} shows the MTD selection and patient enrollment of mTPI-2 design. In scenario 1 and 5, the mTPI-2 design didn't select the right MTD. However, in scenario 4, which mTPI-2 design also failed. For the trial duration average of five scenarios, Bi3+3 has a duration of 491.6 days with 36.8 patients enrolled, in the contrast, mTPI-2 enroll 36.0 patients need 547.4 days on average. That is to say, though Bi3+3 design enroll only 0.8 larger patients, its trial duration is 55.8 days shorter than mTPI-2, indicating its capability for enrolling more patients in non-extended time.

\begin{table}[!htbp]
\scriptsize
\begin{center}
\caption{Comparison of trial duration and patients' enrollment of Backfill i3+3 and mTPI-2}
\label{tbl:duration_vs_mtpi2}
\begin{tabular}{cccccccccc}
\hline
\multirow{2}{*}{} & \multicolumn{5}{c}{\textbf{Dose Toxicity}} & \multirow{2}{*}{\textbf{\makecell{Backfill i3+3\\Trial Duration}}} & \multirow{2}{*}{\textbf{\makecell{mTPI-2\\Trial Duration}}} & \multirow{2}{*}{\textbf{\makecell{Backfill i3+3\\Patients enrolled}}} & \multirow{2}{*}{\textbf{\makecell{mTPI-2\\Patients enrolled}}} \\
                  & 1   &  2  &3    &4   & 5  &                   &                   &                   &                   \\
\textbf{Scenario 1}& 0.15&0.3& 0.45 &0.6& 0.75 &476 & 519&34.2&35.6               \\ 
%\textbf{Scenario 2}& 0.08&0.16&0.24&0.3 &0.38 & 489&543&36.3&36.0                  \\
\textbf{Scenario 2}&0.06&0.12&0.18&0.24& 0.44& 490&544 &36.7&36                   \\
\textbf{Scenario 3}&0.05&0.1&0.15&0.2&0.25&495&556&37.3&36                   \\
\textbf{Scenario 4}&0.27&0.37&0.47&0.57&0.67&434 &483&29.7&32.5                   \\
%\textbf{Scenario 6}&0.06&0.12&0.18&0.24& 0.3& 493&549 &37.1&36                   \\
\textbf{Total Mean}&    &    &    &   &   &473.8&525.3&34.5&35.0   \\ \hline
\end{tabular}
\end{center}
\end{table}

\section{Discussion}
\label{sec:discussion}
The backfill strategy in phase I trials has been routinely applied to allow accumulation of information at low but potentially efficacious doses, despite lacking formal statistical designs. Here, we propose a statistical framework based on model-based inference to allow patient backfilling. The main contributions are to apply POD for patients with pending outcome and to model potentially plateau dose-efficacy response for OBD selection. We show that the propose Bi3+3 design is able to put more patients at lower doses without increase the trial duration. 

%Utilizing the observation window of main cohort patients to backfill patients and PoD method, we are able to enroll more patients without extending the trial duration. Compared to mTPI-2 design with a expansion cohort, backfill i3+3 takes 50+ days fewer to reach the same sample size, and the PCS of MTD is also higher. What's more, different from some others' work, the backfill patients' assessment results are also taken into account, as part of the information used in both MTD and RP2D selection. \\
%Although we assess the efficacy data after the dose-finding trial, we define the efficacy data to be time-to-event during the trial for the correspondence with the real-world data. Only when the efficacy data is available, hypothesis would be tested and doses with lower efficacy might be deleted from the backfill cohort. In situations with larger gaps between efficacy of different doses, patients are more likely to be allocated to dose with higher response rate, which is better for patients and later phase research.\\
For the selection of OBD, we develop a four-parameter regression model that assumes a change-point dose-response relationship. More complex models may be considered, such as model average over possible dose-efficacy models. These models may require larger sample size as they involve more parameters.

We adopt the i3+3 design for the main cohort dose escalation. This is optional. Any sensible design may work well under the proposed framework.   

%However, we only used a single model and the results show that its satisfying performance under different scenarios and dosages. The efficacy estimation is also flexible so the model can be used in both monotone and monotone+plateau curve. The outcome shows the model' ability to choose right RP2D with only 5 doses and a small sample size of around 35. As the simulation settings are similar to the real-world, the model can be used more widely and flexibly.\\
%We worked out a competitive outcome in the 7 dose scenarios. The MTD selection turns out to be correct among 6 scenarios with the probability of near or higher than 40\%. In Scenario A, based on the high probability of dose 2-4, we further choose dose 3 as RP2D successfully. In scenario E and F, the percentage of selecting dose 3 and 4 are high enough, over 60\%. Moreover, the model tends to allocate more patients to the doses of interest rather than a even allocation. Two reasons contribute to the lower elimination rate from backfill cohort in our trials. For one thing, the time-to-event response data delay the evaluation of efficacy. For another, the relatively small sample size used in our trial also makes the elimination rate lower. \\
%However, there is still some room for improvement in our work. In the last 5 scenarios we present, there are minor deviation from MTD selection in scenario 1, 2 and 5. Lower doses are chosen as MTD and the reasons may vary. As we derive the dose allocation from the i3+3 design, which is based on the concept of EI, the dose with toxicity near the lower bound of EI, i.e. a dose with 0.24 toxicity when EI=(0.25,0.35) may be chosen as MTD by error. However, the deviation is bearable because the algorithm tends to choose a lower dose under some circumstances to ensure safety. Though we allow dose elimination of efficacy, the probability of backfill form dose 1 is still high in some situations. In addition, RP2D is selected lower in the scenarios with little gap of efficacy, one may develop more complex models to achieve that goal.\\

\clearpage
\newpage


\bibliographystyle{apalike}
\bibliography{bi3+3}

\newpage

\begin{appendices}
\appendixpage
\setcounter{table}{0}
\renewcommand{\thetable}{A.\arabic{table}}
\setcounter{figure}{0}
\renewcommand{\thefigure}{A.\arabic{figure}}
\section{The POD-i3 design} \label{appendix:pod_method}
After the outcomes of patients in the main cohort (mc) are observed, some pending patients at lower doses are still being followed. Generally, trial designs would make a dose decision immediately after the observation of outcomes at the main cohort and explore more doses as soon as possible. The issue is how the partial information from the pending patients can be incorporated for the dosing decision. We follow the POD framework \citep{zhou2020pod} to analyze and utilize information of pending outcomes. Specifically, we develop a POD-i3 design as suggested by the authors. 

Unlike the TITE-CRM design \citep{cheung2000sequential}, the POD framework considers probability of making a dosing decision ($E$, $S$, or $D$) where the randomness is induced by the uncertainty in the pending DLT outcomes. Let $\mathcal{A}_d$ denote the decision function of a complete-data design for the current dose $d$, such as the i3+3 design, which requires the complete DLT evaluation of all the enrolled patients. Given the target toxicity probability $p_T$ and EI$=[p_T - \epsilon_1, p_T + \epsilon_2]$, the decision of dose $d$, $\mathcal{A}_d$, only depends on $n_d$ and $y_d$, i.e., a deterministic function of $n_d$ and $y_d$, $\mathcal{A}_d = \mathcal{A}_d[y_d,n_d] \in \{-1, 0, 1\}$, where -1, 0, and 1 correspond to the decisions of de-escalating to the previous lower dose (-1), stay at the current dose (0), and escalating to the next higher dose (1), respectively. Then, we use $y_{k,\obs}$ to denote the  observed toxicity outcomes at a backfill dose $k$ ($k < d$) and denote the unobserved DLT outcomes as  $Y_{k,\mis}$. Thus, letting $a$ takes values in $\{-1,0,1\}$, the decision function in the presence of pending outcomes at dose $k$ can be expressed as 
$A_k=\mathcal{A}_k[(y_{k,\obs},Y_{k,\mis}),n_k]$. As $Y_{k,\mis}$ is a random variable, the PoD method calculates the posterior probability $\gamma_{k,a}\equiv Pr(A_k=a \mid \mathcal{H})$ as 
$$\gamma_{k,a} =\displaystyle\sum\limits_{y_{k,\mis}:{A}_k=a} \Pr(Y_{k,\mis}=y_{k,\mis} \mid \mathcal{H}), \quad k =  k_0,\cdots, d-1, \;\; \mbox{ and } a \in \{-1, 0, 1\}.$$
Here, $\mathcal{H}$ denotes the observed data of all enrolled patients, including the dose assigned to each patient, whether or not each patient experiences DLT within the follow-up window, and the follow-up time of each patient. The posterior predictive probability $\gamma_{k,a}$ is the POD for decision $a$ at dose $k$.

Let $A^*_k = \mathop{\mathrm{argmax}}\limits_{a} \gamma_{k,a}$ denote the decision at dose $k$ with the highest POD. If multiple decisions tie for the highest POD, we choose the more conservative one. To ensure the safety of the design, we apply a suspension rule similar to \cite{zhou2020pod}. If $A^*_k = 0$, i.e.,  stay at dose $k$, we recommend to suspend the trial if the posterior probability of de-escalation $\gamma_{k,-1} > \pi_D$ for a pre-specified threshold $\pi_D$. In the original paper by \cite{zhou2020pod}, there is another suspension rule for dose escalation. However, since we are working with backfill doses which have already shown initial safety, we decide not to enforce that rule. See more detail in \cite{zhou2020pod}.


\section{Benchmark Results for the Backfill CRM design}
\label{appendix:reproduction}
The reproduced outcome of the 7 scenarios presented in \citep{dehbi2021controlled} is in \ref{tbl:reproduce1} using the same skeleton with the author.

\begin{table}[!htbp]
\tiny
\begin{center}
\caption{Comparison of Reproduction of Backfill CRM }
\label{tbl:reproduce1}
\begin{tabular}{@{}ccccccccccccccc@{}}
\toprule
 & \multicolumn{7}{c}{Reproduced CRM}& \multicolumn{7}{c}{Backfill CRM} \\ \midrule
Dose level& 1  &2   & 3 & 4 &5  & 6 &7& 1  &2   & 3 & 4 &5  & 6 &7  \\  \midrule
\textbf{Scenario A}&   &   &  &  &  &  & &   &   &  &  &  &  &  \\
Efficacy probability& 0.05 & 0.15  &0.25  & 0.25 &0.25  &0.25  &0.25  &0.05 & 0.15  &0.25  & 0.25 &0.25  &0.25  &0.25  \\
Toxicity probability& 0.01& 0.04  &0.08  & 0.16 &0.25  &0.35  &0.46  & 0.01& 0.04  &0.08  & 0.16 &0.25  &0.35  &0.46 \\
\% of OBD selection& 0\%& 0.3\% &\textbf{29.2\%}  &47.2\%  & 17.4\% & 5.5\% &0.4\%& 0\%& 0\% &\textbf{29\%}  &46\%  & 16\% & 8\% &1\%  \\
\% of MTD selection&  0\%& 0\% & 2.0\%  & 26.9\%  & 52.0\% & 17.6\% &1.5\%    &   &  &  &  &  & &\\
Efficacy estimation&  0.15 & 0.17  &0.20  &0.23  &0.25  &0.28  & 0.29  &   &  &  &  &  & & \\ 
% Number/\% of patients enrolled & 5.3& 6.7 & 9.3 & 12.4 & 11.0 & 5.5 & 1.9& 16\%& 18\% & 20\% & 19\% & 17\% & 9\% & 2\%\\
Number of patients enrolled & 8.9& 10.4 & \textbf{11.1} & 11.9 & 9.2 & 4.3 & 1.3& 9.1& 10.3 & \textbf{11.4} & 10.8 & 9.7 & 5.1 & 1.1\\
Number of patients backfill & 5.8& 7.3& 7.1 & 4.9 &1.7 &0.3 &0  &   &  &  &  &  & &\\
\textbf{Scenario B}&   &   &  &  &  &  &  &   &   &  &  &  &  & \\
Efficacy probability& 0.05 & 0.1  &0.15  & 0.15 &0.15  &0.15  &0.15 & 0.05 & 0.1  &0.15  & 0.15 &0.15  &0.15  &0.15  \\
Toxicity probability& 0.01& 0.04  &0.08  & 0.16 &0.25  &0.35  &0.46& 0.01& 0.04  &0.08  & 0.16 &0.25  &0.35  &0.46  \\
\% of OBD selection& 0\%& 0.4\% & \textbf{28.2\%}  & 49.2\%  & 17.0\% & 4.7\% &0.5\% & 0\%& 0\% & \textbf{32\%}  & 46\%  & 15\% & 6\% &1\%\\
\% of MTD selection&  0\%& 0\% & 1.8\%  & 26.2\%  & 52.9\% & 17.8\% &1.3\%    &   &  &  &  &  & &\\
Efficacy estimation&  0.10 & 0.11  &0.13  &0.14  &0.15  &0.17  & 0.18  &   &  &  &  &  & & \\ 
% Number/\% of patients enrolled & 6.6& 7.5 & 9.2 & 12.0 & 11.0 & 5.5 & 1.9& 18\%& 19\% & 18\% & 18\% & 17\% & 8\% & 2\%\\
Number of patients enrolled & 9.5& 10.6 & \textbf{10.8} & 11.4 & 9.1 & 4.3 & 1.3& 10.3& 10.8 & \textbf{10.3} & 10.3 & 9.7 & 4.6 & 1.1\\
Number of patients backfill & 6.4& 7.5& 6.9 & 4.4 &1.6 &0.3 &0  &   &  &  &  &  & &\\
\textbf{Scenario C}&   &   &  &  &  &  &  &   &   &  &  &  &  & \\
Efficacy probability& 0.04 & 0.08  &0.12  & 0.16 &0.2  &0.24  &0.28  & 0.04 & 0.08  &0.12  & 0.16 &0.2  &0.24  &0.28 \\
Toxicity probability& 0& 0  &0.01  & 0.04 &0.08  &0.16  &0.25 & 0& 0  &0.01  & 0.04 &0.08  &0.16  &0.25  \\
\% of OBD selection& 0\%& 0\% & 0\%  & 25.1\%  & 23.0\% & 4.9\% & \textbf{47.0\%} & 0\%& 0\% &0\%  &30\%  & 10\% & 9\% &\textbf{51\%}  \\
\% of MTD selection&  0\%& 0\% & 0\%  & 0.1\%  & 0.7\% & 6.6\% &92.6\%   &   &  &  &  &  & & \\
Efficacy estimation&  0.10 & 0.12  &0.14  &0.16  &0.19  &0.21  & 0.24  &   &  &  &  &  & & \\ 
% Number/\% of patients enrolled & 4.9 & 5.2 & 5.6 & 6.0 & 7.7 & 10.5 & 13.1& 17\% & 16\% & 14\% & 13\% & 11\% & 10\% & 19\%\\
Number of patients enrolled & 8.3 & 8.0 & 8.1 & 7.7 & 7.4 & 6.8 & \textbf{10.8}& 9.7 & 9.1 & 8.0 & 7.4 & 6.3 & 5.7 & \textbf{10.9}\\
Number of patients backfill & 5.3& 5.0& 5.0 & 4.6 &4.1 &2.9 &0  &   &  &  &  &  & &\\
\textbf{Scenario D}&   &   &  &  &  &  & &   &   &  &  &  &  &  \\
Efficacy probability& 0.07 & 0.14  &0.21  & 0.28 &0.35  &0.42  &0.49& 0.07 & 0.14  &0.21  & 0.28 &0.35  &0.42  &0.49  \\
Toxicity probability& 0& 0  &0.01  & 0.04 &0.08  &0.16  &0.25& 0& 0  &0.01  & 0.04 &0.08  &0.16  &0.25  \\
\% of OBD selection& 0\%& 0\% &0.2\%  &12.2\%  & 19.7\% & 8.7\% & \textbf{59.2\%}& 0\%& 0\% &0\%  &14\%  & 13\% & 13\% &\textbf{60\%}   \\
\% of MTD selection&  0\%& 0\% & 0\%  & 0\%  & 0.8\% & 6.6\% &92.6\%    &   &  &  &  &  & &\\
Efficacy estimation&  0.15 & 0.18  &0.23  &0.28  &0.33  &0.39  & 0.44   &   &  &  &  &  & &\\
% Number/\% of patients enrolled & 3.9 & 4.6 & 5.4 & 6.2 & 8.0 & 10.8 & 13.1& 15\% & 15\% & 14\% & 13\% & 12\% & 11\% & 19\%\\
Number of patients enrolled & 8.0 & 8.0 & 7.9 & 7.8 & 7.6 & 6.9 & \textbf{10.8}& 8.6 & 8.6 & 8.0 & 7.4 & 6.8 & 6.3 & \textbf{10.8}\\
Number of patients backfill & 5.0& 5.0& 4.9 & 4.7 &4.3 &3.1 &0  &   &  &  &  &  & &\\
\textbf{Scenario E}&   &   &  &  &  &  &  &   &   &  &  &  &  & \\
Efficacy probability& 0.05 & 0.1  &0.15  & 0.2 &0.2  &0.2  &0.2 & 0.05 & 0.1  &0.15  & 0.2 &0.2  &0.2  &0.2 \\
Toxicity probability& 0.04& 0.08  &0.16  & 0.25 &0.35  &0.46  &0.56 & 0.04& 0.08  &0.16  & 0.25 &0.35  &0.46  &0.56  \\
\% of OBD selection& 0\%& 3.7\% &47.1\%  & \textbf{38.7\%}  & 9.9\% & 0.6\% &0\% & 0\%& 5\% &45\%  &\textbf{36\%}  & 11\% & 2\% &0\%  \\
\% of MTD selection&  0\%& 1.6\% & 26.1\%  & 49.0\%  & 21.3\% & 2.0\% &0\%   &   &  &  &  &  & & \\
Efficacy estimation&  0.10 & 0.12  &0.14  &0.17  &0.20  &0.22  & 0.25   &   &  &  &  &  & &\\
% Number/\% of patients enrolled & 8.2& 10.4 & 14.0 & 12.3 & 6.1 & 1.5 & 0.3 & 21\% & 22\% & 24\% & 19\% & 10\% & 3\% & 1\%\\
Number of patients enrolled & 12.0& 12.8 & 13.3 & \textbf{11.1} & 5.8 & 1.7 & 0.3 & 12.0 & 12.5 & 13.7 & \textbf{10.8} & 5.7 & 1.7 & 0.6\\
Number of patients backfill & 8.5& 8.4& 6.3 & 3.0 &0.7 &0.1 &0  &   &  &  &  &  & &\\
\textbf{Scenario F}&   &   &  &  &  &  &  &   &   &  &  &  &  & \\
Efficacy probability& 0.04 & 0.08  &0.12  & 0.16 &0.2  &0.24  &0.24& 0.04 & 0.08  &0.12  & 0.16 &0.2  &0.24  &0.24  \\
Toxicity probability& 0.04& 0.08  &0.16  & 0.25 &0.35  &0.46  &0.56 & 0.04& 0.08  &0.16  & 0.25 &0.35  &0.46  &0.56  \\
\% of OBD selection& 0\%& 3.8\% &48.2\%  &\textbf{37.7\%}  & 9.2\% & 1.1\% &0\% & 0\%& 6\% &45\%  &\textbf{34\%}  & 13\% & 3\% &0\%  \\
\% of MTD selection&  0\%& 1.7\% & 25.9\%  & 50.5\%  & 19.5\% & 2.3\% &0.1\%    &   &  &  &  &  & &\\
Efficacy estimation&  0.08 & 0.10  &0.12  &0.14  &0.17  &0.20  & 0.22  &   &  &  &  &  & & \\
% Number/\% of patients enrolled & 8.9& 10.6 & 14.0 & 12.3 & 6.1 & 1.5 & 0.3& 21\%& 22\% & 23\% & 19\% & 11\% & 3\% & 1\%\\
Number of patients enrolled & 11.9& 12.7 & 13.4 & \textbf{11.0} & 6.0 & 1.7 & 0.3& 12.0& 12.5 & 13.1 & \textbf{10.8} & 6.3 & 1.7 & 0.6\\
Number of patients backfill & 8.4& 8.3& 6.3 & 3.0 &0.8 &0.1 &0  &   &  &  &  &  & &\\
\bottomrule
\end{tabular}
\end{center}
\end{table}

%\section{Comparison of Bi3+3 and Backfill CRM}
%\label{appendix:com_BCRM}
% \begin{table}[!htbp]
% \tiny
% \begin{center}
% \caption{Comparison between Backfill i3+3 and Backfill CRM}
% \label{tbl:outcome1}
% \begin{tabular}{@{}ccccccccccccccc@{}}
% \toprule
%  & \multicolumn{7}{c}{Backfill i3+3}& \multicolumn{7}{c}{Backfill CRM} \\ \midrule
% Dose level& 1  &2   & 3 & 4 &5  & 6 &7& 1  &2   & 3 & 4 &5  & 6 &7  \\  \midrule
% \textbf{Scenario A}&   &   &  &  &  &  & &   &   &  &  &  &  &  \\
% Efficacy probability& 0.05 & 0.15  &0.25  & 0.25 &0.25  &0.25  &0.25  &0.05 & 0.15  &0.25  & 0.25 &0.25  &0.25  &0.25  \\
% Toxicity probability& 0.01& 0.04  &0.08  & 0.16 &0.25  &0.35  &0.46  & 0.01& 0.04  &0.08  & 0.16 &0.25  &0.35  &0.46 \\
% \% of OBD selection& 0\%& 19.9\% &\textbf{30.0\%}  &20.4\%  & 21.4\% & 7.8\% &0.5\%& 0\%& 0\% &\textbf{29\%}  &46\%  & 16\% & 8\% &1\%  \\
% \% of MTD selection&  0\%& 0.5\% & 9.4\%  & 34.6\%  & 41.4\% & 13.2\% &0.9\%    &   &  &  &  &  & &\\
% Efficacy estimation&  0.15 & 0.18  &0.21  &0.23  &0.26  &0.29  & 0.31  &   &  &  &  &  & & \\ 
% % Number/\% of patients enrolled & 5.3& 6.7 & 9.3 & 12.4 & 11.0 & 5.5 & 1.9& 16\%& 18\% & 20\% & 19\% & 17\% & 9\% & 2\%\\
% Number of patients enrolled & 4.5& 6.8 & \textbf{10.0} & 13.4 & 11.7 & 5.6 & 1.5& 9.1& 10.3 & \textbf{11.4} & 10.8 & 9.7 & 5.1 & 1.1\\
% Number of patients backfill & 1.4& 3.0& 3.7 & 2.5 &0.7 &0.1 &0  &   &  &  &  &  & &\\
% \textbf{Scenario B}&   &   &  &  &  &  &  &   &   &  &  &  &  & \\
% Efficacy probability& 0.05 & 0.1  &0.15  & 0.15 &0.15  &0.15  &0.15 & 0.05 & 0.1  &0.15  & 0.15 &0.15  &0.15  &0.15  \\
% Toxicity probability& 0.01& 0.04  &0.08  & 0.16 &0.25  &0.35  &0.46& 0.01& 0.04  &0.08  & 0.16 &0.25  &0.35  &0.46  \\
% \% of OBD selection& 0\%& 13.7\% & \textbf{19.8\%}  & 23.5\%  & 29.8\% & 12.4\% &0.8\% & 0\%& 0\% & \textbf{32\%}  & 46\%  & 15\% & 6\% &1\%\\
% \% of MTD selection&  0\%& 0.5\% & 8.7\%  & 34.0\%  & 42.0\% & 13.8\% &1.0\%    &   &  &  &  &  & &\\
% Efficacy estimation&  0.10 & 0.11  &0.13  &0.14  &0.16  &0.19  & 0.21  &   &  &  &  &  & & \\ 
% % Number/\% of patients enrolled & 6.6& 7.5 & 9.2 & 12.0 & 11.0 & 5.5 & 1.9& 18\%& 19\% & 18\% & 18\% & 17\% & 8\% & 2\%\\
% Number of patients enrolled & 5.2& 7.3 & \textbf{10.1} & 13.0 & 11.7 & 5.7 & 1.6& 10.3& 10.8 & \textbf{10.3} & 10.3 & 9.7 & 4.6 & 1.1\\
% Number of patients backfill & 2.1& 3.4& 3.8 & 2.3 &0.7 &0.1 &0  &   &  &  &  &  & &\\
% \textbf{Scenario C}&   &   &  &  &  &  &  &   &   &  &  &  &  & \\
% Efficacy probability& 0.04 & 0.08  &0.12  & 0.16 &0.2  &0.24  &0.28  & 0.04 & 0.08  &0.12  & 0.16 &0.2  &0.24  &0.28 \\
% Toxicity probability& 0& 0  &0.01  & 0.04 &0.08  &0.16  &0.25 & 0& 0  &0.01  & 0.04 &0.08  &0.16  &0.25  \\
% \% of OBD selection& 0\%& 1.8\% & 3.3\%  & 5.3\%  & 11.6\% & 34.9\% & \textbf{43.1\%} & 0\%& 0\% &0\%  &30\%  & 10\% & 9\% &\textbf{51\%}  \\
% \% of MTD selection&  0\%& 0\% & 0\%  & 0.7\%  & 9.7\% & 38.8\% &50.8\%   &   &  &  &  &  & & \\
% Efficacy estimation&  0.11 & 0.13  &0.15  &0.17  &0.20  &0.22  & 0.25  &   &  &  &  &  & & \\ 
% % Number/\% of patients enrolled & 4.9 & 5.2 & 5.6 & 6.0 & 7.7 & 10.5 & 13.1& 17\% & 16\% & 14\% & 13\% & 11\% & 10\% & 19\%\\
% Number of patients enrolled & 3.8 & 4.7 & 5.8 & 6.6 & 8.5 & 11.5 & \textbf{14.0}& 9.7 & 9.1 & 8.0 & 7.4 & 6.3 & 5.7 & \textbf{10.9}\\
% Number of patients backfill & 0.8& 1.7& 2.5 & 2.9 &2.9 &1.8 &0  &   &  &  &  &  & &\\
% \textbf{Scenario D}&   &   &  &  &  &  & &   &   &  &  &  &  &  \\
% Efficacy probability& 0.07 & 0.14  &0.21  & 0.28 &0.35  &0.42  &0.49& 0.07 & 0.14  &0.21  & 0.28 &0.35  &0.42  &0.49  \\
% Toxicity probability& 0& 0  &0.01  & 0.04 &0.08  &0.16  &0.25& 0& 0  &0.01  & 0.04 &0.08  &0.16  &0.25  \\
% \% of OBD selection& 0\%& 2.1\% &5.8\%  &8.1\%  & 14.3\% & 31.8\% & \textbf{37.9\%}& 0\%& 0\% &0\%  &14\%  & 13\% & 13\% &\textbf{60\%}   \\
% \% of MTD selection&  0\%& 0\% & 0\%  & 0.7\%  & 9.3\% & 39.6\% &50.4\%    &   &  &  &  &  & &\\
% Efficacy estimation&  0.16 & 0.20  &0.24  &0.29  &0.35  &0.40  & 0.44   &   &  &  &  &  & &\\
% % Number/\% of patients enrolled & 3.9 & 4.6 & 5.4 & 6.2 & 8.0 & 10.8 & 13.1& 15\% & 15\% & 14\% & 13\% & 12\% & 11\% & 19\%\\
% Number of patients enrolled & 3.6 & 4.8 & 5.4 & 6.8 & 8.5 & 11.7 & \textbf{13.9}& 8.6 & 8.6 & 8.0 & 7.4 & 6.8 & 6.3 & \textbf{10.8}\\
% Number of patients backfill & 0.6& 1.5& 2.3 & 3.1 &2.9 &1.9 &0  &   &  &  &  &  & &\\
% \textbf{Scenario E}&   &   &  &  &  &  &  &   &   &  &  &  &  & \\
% Efficacy probability& 0.05 & 0.1  &0.15  & 0.2 &0.2  &0.2  &0.2 & 0.05 & 0.1  &0.15  & 0.2 &0.2  &0.2  &0.2 \\
% Toxicity probability& 0.04& 0.08  &0.16  & 0.25 &0.35  &0.46  &0.56 & 0.04& 0.08  &0.16  & 0.25 &0.35  &0.46  &0.56  \\
% \% of OBD selection& 0.5\%& 21.0\% &41.2\%  & \textbf{27.5\%}  & 8.2\% & 1.6\% &0\% & 0\%& 5\% &45\%  &\textbf{36\%}  & 11\% & 2\% &0\%  \\
% \% of MTD selection&  0.5\%& 6.8\% & 38.9\%  & 39.7\%  & 12.0\% & 2.1\% &0\%   &   &  &  &  &  & & \\
% Efficacy estimation&  0.10 & 0.12  &0.15  &0.19  &0.22  &0.26  & 0.28   &   &  &  &  &  & &\\
% % Number/\% of patients enrolled & 8.2& 10.4 & 14.0 & 12.3 & 6.1 & 1.5 & 0.3 & 21\% & 22\% & 24\% & 19\% & 10\% & 3\% & 1\%\\
% Number of patients enrolled & 7.0& 10.1 & 14.6 & \textbf{12.7} & 6.1 & 1.7 & 0.3 & 12.0 & 12.5 & 13.7 & \textbf{10.8} & 5.7 & 1.7 & 0.6\\
% Number of patients backfill & 3.2& 3.6& 2.5 & 0.8 &0.2 &0 &0  &   &  &  &  &  & &\\
% \textbf{Scenario F}&   &   &  &  &  &  &  &   &   &  &  &  &  & \\
% Efficacy probability& 0.04 & 0.08  &0.12  & 0.16 &0.2  &0.24  &0.24& 0.04 & 0.08  &0.12  & 0.16 &0.2  &0.24  &0.24  \\
% Toxicity probability& 0.04& 0.08  &0.16  & 0.25 &0.35  &0.46  &0.56 & 0.04& 0.08  &0.16  & 0.25 &0.35  &0.46  &0.56  \\
% \% of OBD selection& 0.5\%& 18.3\% &39.9\%  &\textbf{29.6\%}  & 10.0\% & 1.7\% &0\% & 0\%& 6\% &45\%  &\textbf{34\%}  & 13\% & 3\% &0\%  \\
% \% of MTD selection&  0.5\%& 7.4\% & 38.3\%  & 38.7\%  & 13.3\% & 1.8\% &0\%    &   &  &  &  &  & &\\
% Efficacy estimation&  0.09 & 0.10  &0.12  &0.16  &0.20  &0.23  & 0.26  &   &  &  &  &  & & \\
% % Number/\% of patients enrolled & 8.9& 10.6 & 14.0 & 12.3 & 6.1 & 1.5 & 0.3& 21\%& 22\% & 23\% & 19\% & 11\% & 3\% & 1\%\\
% Number of patients enrolled & 7.3& 10.4 & 14.7 & \textbf{12.7} & 6.0 & 1.7 & 0.3& 12.0& 12.5 & 13.1 & \textbf{10.8} & 6.3 & 1.7 & 0.6\\
% Number of patients backfill & 3.5& 3.8& 2.5 & 0.8 &0.1 &0 &0  &   &  &  &  &  & &\\
% \bottomrule
% \end{tabular}
% \end{center}
% \end{table}

% Table \ref{tbl:back_start} presents the frequency of dose elimination due to insufficient efficacy when selecting doses to backfill patients. These results assess designs' efficiency in detecting doses with low efficacy response rates and stopping enrollment on these doses.
% \begin{table}[!htbp]
% \tiny
% \begin{center}
% \caption{Backfill dose discontinuation percentages in the six scenarios}
% \label{tbl:back_start}
% \begin{tabular}{@{}ccccccccccccc@{}}
% \toprule
%  & \multicolumn{6}{c}{Backfill i3+3}& \multicolumn{6}{c}{Backfill CRM}  \\ \midrule
% Dose level(s) discontinued& None  & 1  & 1+2 & 1+2+3 &1+2+3+4  & 1+2+3+4+5 & None  & 1  & 1+2 & 1+2+3 &1+2+3+4  & 1+2+3+4+5\\  \midrule
% Scenario A &  11.8\% & 27.1\%  & 37.8\% & 14.9\% & 6.9\% & 1.5\% &14\%&37\%&39\%&9\%&1\%&0\%   \\
% Scenario B & 24.1\%  & 29.4\%  & 28.4\% & 13.6\% & 3.5\% &1.0\% &39\%&36\%&20\%&3\%&1\%&0\%  \\
% Scenario C & 16.5\%  & 24.4\%  & 20.7\% & 17.6\% &10.8\%  & 10.0\% &27\%&33\%&25\%&11\%&3\%&0\%  \\
% Scenario D & 11.9\%  & 20.7\%  & 22.0\% & 17.3\% & 13.8\% & 14.3\% &11\%&26\%&30\%&20\%&11\%&2\% \\
% Scenario E & 19.4\%  &28.6\%   & 28.4\% & 19.9\% & 3.3\% &0.4\% &33\%&31\%&21\%&12\%&2\%&0\% \\
% Scenario F &  19.4\% & 29.7\%  & 26.8\% & 17.2\% & 5.9\% & 1.0\% &36\%&31\%&19\%&10\%&3\%&0\%   \\
% \bottomrule
% \end{tabular}
% \end{center}
% \end{table}



\section{Comparison with the mTPI-2 design given $p_T=0.25$  }
\label{appendix:com_mTPI2}
\begin{figure}[!htbp]
    \centering
    \includegraphics[width=1.0\textwidth]{5sce.png}
    \caption{  The five simulation scenarios with target toxicity probability of 0.25. The (MTD, OBD) are (4,3), (4,3), (5,5), (5,5), and (4,4) for scenarios 1-5, respectively.   }
    \label{fig:5dose-1}
\end{figure}

\begin{table}[!htbp]
\tiny
\begin{center}
\caption{Comparison of Bi3+3 and mTPI-2 in five scenarios. The target toxicity probability is 0.25, EI =[0.2,0.3], sample size is 36.5 for Bi3+3 and 36.0 for mTPI-2.}
\label{tbl:oc_vs_mtpi2_pt0.25}
\begin{tabular}{@{}ccccccccccc@{}}
\toprule
 & \multicolumn{5}{c}{Backfill i3+3} & \multicolumn{5}{c}{m-TPI2} \\ \midrule
Dose level&  1&2  &3  &4  &5 &  1&2  &3  &4  &5 \\ \midrule
\textbf{Scenario 1}&  &  &  &  & &  &  &  &  &  \\
% Efficacy probability& 0.05 & 0.15 &0.25  &0.25  &0.25 & 0.1 & 0.3 &0.5  &0.5  &0.5 \\
Efficacy probability& 0.1 & 0.3 &0.5  &0.5  &0.5 &  &  &  &  & \\
Toxicity probability& 0.06 &0.13  & 0.19  &0.25  & 0.31& 0.06 &0.13  & 0.19  &0.25  & 0.31 \\
\% of OBD selection& 3.0\%  & 31.1\% & \textbf{37.0\%}  &19.5\%  &9.4\% &   &  &   &  &  \\
\% of MTD selection& 3.0\%  & 20.0\% & 32.6\%  &28.9\%  &15.5\%& 4.8\% & 26.8\% &  36.7\% & 22.6\% & 9.1\% \\
Efficacy estimation& 0.22 &0.32  &0.43  &0.49  &0.53 &  &  &  &  &  \\ 
Number of patients enrolled & 6.3 &9.0 & \textbf{9.4} & 6.4 &4.3& 6.1& 10.5 & 10.2 & 6.1 & 3.0\\
Number of patients backfill &1.6 &1.6 &1.4 &0.7 &0& & & & &\\
Trial Duration& \multicolumn{5}{c}{490}  & \multicolumn{5}{c}{531}\\
\textbf{Scenario 2}&  &  &  &  & &  &  &  &  &  \\
% Efficacy probability& 0.05 &0.1  & 0.15 &0.15  &0.15 & 0.1 &0.2  & 0.3 &0.3  &0.3  \\
Efficacy probability& 0.1 &0.2  & 0.3 &0.3  &0.3 &  &  &  &  &  \\
Toxicity probability&0.06 &0.13  & 0.19  &0.25  & 0.31& &  &   &  &  \\
\% of OBD selection& 3.0\%  & 27.5\% &\textbf{32.5\%}  & 24.7\% &12.3\% &   &  & &  & \\
\% of MTD selection& 3.0\% & 20.1\% & 32.8\% & 28.4\% & 15.7\% &  &  &  &  &  \\
Efficacy estimation& 0.17 &0.21  &0.27  &0.32  &0.35 &  &  &  &  &  \\ 
Number of patients enrolled & 7.0 & 9.4 & \textbf{9.4} & 6.3 & 4.4&  &  &  &  & \\
Number of patients backfill & 2.4&2.0 &1.3 &0.6 &0& & & & &\\
\textbf{Scenario 3}&  & &  &  &  &  & &  &  &\\
% Efficacy probability& 0.04 &0.08  &0.12  &0.16  &0.2& 0.08 &0.16  &0.24  &0.32  &0.4   \\
Efficacy probability& 0.08 &0.16  &0.24  &0.32  &0.4&  &  &  &  &   \\
Toxicity probability& 0.04 &0.08  &0.13  &0.17  &0.25& 0.04 &0.08  &0.13  &0.17  &0.25  \\
\% of OBD selection& 0.4\% & 9.1\% & 19.4\% & 36.8\% & \textbf{34.3\%} &  &  &  &  & \\
\% of MTD selection& 0.4\% & 5.5\% & 17.3\% & 37.1\% & 39.7\% & 0.6\% & 9.8\% & 26.0\% & 36.4\% & 27.2\%\\
Efficacy estimation& 0.16 & 0.19 &0.25  & 0.31 &0.36&  &  &  &  &  \\ 
Number of patients enrolled& 5.4 & 7.0 & 8.3 & 7.9 & \textbf{8.5}& 4.5 & 7.0 & 8.9 & 8.5 & 7.2\\
Number of patients backfill &1.8 &2.0 &1.9 &1.4 &0 & & & & & \\
Trial Duration& \multicolumn{5}{c}{495}  & \multicolumn{5}{c}{539}\\
\textbf{Scenario 4}&  &  &  &  & &  &  &  &  &  \\
% Efficacy probability& 0.07 &0.14  &0.21  &0.28  &0.35 & 0.14 &0.28  &0.42  &0.56  &0.7  \\
Efficacy probability& 0.07 &0.14  &0.21  &0.28  &0.35 &  &  &  &  &  \\
Toxicity probability& 0.04 &0.08  &0.13  &0.17  &0.25 &  & &  &  &  \\
\% of OBD selection& 0.4\% & 8.3\% &19.5\%  & 35.5\% & \textbf{36.3\%} &  &  &  &  & \\
\% of MTD selection& 0.4\% & 5.4\% & 17.6\% & 36.6\%  & 40.0\% &  &  &  &  & \\
Efficacy estimation&  0.14& 0.17 &0.22  &0.27  &0.31 &  &  &  &  &   \\ 
Number of patients enrolled& 5.5 & 7.0 & 8.4 & 7.8 & \textbf{8.5}&  &  &  &  & \\
Number of patients backfill & 1.9&2.0 &2.0 &1.3 &0& & & & & \\
\textbf{Scenario 5}&  &  &  &  & &  &  &  &  & \\
% Efficacy probability& 0.05 &0.1  &0.15  &0.2  &0.2 & 0.1 &0.2  &0.3  &0.4  &0.4  \\
Efficacy probability& 0.1& 0.2  &0.3  &0.4  &0.4 &  &  &  &  &  \\
Toxicity probability& 0.04 &0.08  &0.16  &0.25  &0.35 & 0.04 &0.08  &0.16  &0.25  &0.35 \\
\% of OBD selection& 0.4\% & 17.4\% & 37.2\% & \textbf{34.2\%} & 10.8\%&  &  &  &  &  \\
\% of MTD selection& 0.4\% & 10.7\% & 36.4\% & 38.7\% & 13.8\% & 0.7\% & 15.0\% & 45.6\% & 31.2\% & 7.5\% \\
Efficacy estimation& 0.18 & 0.23 &0.30  &0.36  &0.41 &  &  &  &  &  \\
Number of patients enrolled & 5.6 & 7.7 & 10.2 & \textbf{8.0} & 4.9& 4.5 & 8.0 & 11.8 & 8.3 & 3.5\\
Number of patients backfill &1.9 &2.1 &1.5 &0.7 & & & & & &\\
Trial Duration& \multicolumn{5}{c}{491}  & \multicolumn{5}{c}{530}\\\bottomrule
\end{tabular}
\end{center}
\end{table}



% \begin{table}[!htbp]
% \scriptsize
% \begin{center}
% \caption{Outcomes of 5 doses scenarios}
% \label{tbl:5sce-1}
% \begin{tabular}{@{}ccccccccccc@{}}
% \toprule
%  & \multicolumn{5}{c}{Backfill i3+3} & \multicolumn{5}{c}{Backfill CRM} \\ \midrule
% Dose level&  1&2  &3  &4  &5 &  1&2  &3  &4  &5 \\ \midrule
% \textbf{Scenario 1}&  &  &  &  & &  &  &  &  &  \\
% % Efficacy probability& 0.05 & 0.15 &0.25  &0.25  &0.25 & 0.1 & 0.3 &0.5  &0.5  &0.5 \\
% Efficacy probability& 0.1 & 0.3 &0.5  &0.5  &0.5 &  &  &  &  & \\
% Toxicity probability& 0.06 &0.13  & 0.19  &0.25  & 0.31& 0.06 &0.13  & 0.19  &0.25  & 0.31 \\
% \% of OBD selection& 3.1\%  & 30.9\% & \textbf{34.7\%}  &21.2\%  &10.1\% &0\%  &12.5\%  &58.1\% &26.1\%  &2.8\%  \\
% \% of MTD selection& 3.1\% & 20.1\% &  31.3\% & 29.5\% & 16.0\%& 0\% & 8.8\% &  41.5\% & 40.5\% & 8.7\% \\
% Efficacy estimation& 0.23 &0.32  &0.43  &0.50  &0.53 &  &  &  &  &  \\ 
% Number of patients enrolled & 5.4 &8.4 & \textbf{9.2} & 6.3 &4.4& 6.1& 10.5 & 10.2 & 6.1 & 3.0\\
% Number of patients backfill &0.8 &1.1 &1.2 &0.7 &0& & & & &\\
% Trial Duration& \multicolumn{5}{c}{490}  & \multicolumn{5}{c}{531}\\
% \textbf{Scenario 2}&  &  &  &  & &  &  &  &  &  \\
% % Efficacy probability& 0.05 &0.1  & 0.15 &0.15  &0.15 & 0.1 &0.2  & 0.3 &0.3  &0.3  \\
% Efficacy probability& 0.1 &0.2  & 0.3 &0.3  &0.3 &  &  &  &  &  \\
% Toxicity probability&0.06 &0.13  & 0.19  &0.25  & 0.31& &  &   &  &  \\
% \% of OBD selection& 3.0\%  & 27.9\% &\textbf{30.4\%}  & 25.6\% &13.1\% &0\% &13.2\% &61.1\% &21.0\% &4.3\% \\
% \% of MTD selection& 3.0\% & 19.7\% & 31.5\% & 29.8\% & 16.0\% &  &  &  &  &  \\
% Efficacy estimation& 0.17 &0.21  &0.27  &0.32  &0.35 &  &  &  &  &  \\ 
% Number of patients enrolled & 6.2 & 8.8 & \textbf{9.2} & 6.4 & 4.4&  &  &  &  & \\
% Number of patients backfill & 1.5&1.5 &1.2 &0.6 &0& & & & &\\
% \textbf{Scenario 3}&  & &  &  &  &  & &  &  &\\
% % Efficacy probability& 0.04 &0.08  &0.12  &0.16  &0.2& 0.08 &0.16  &0.24  &0.32  &0.4   \\
% Efficacy probability& 0.08 &0.16  &0.24  &0.32  &0.4&  &  &  &  &   \\
% Toxicity probability& 0.04 &0.08  &0.13  &0.17  &0.25& 0.04 &0.08  &0.13  &0.17  &0.25  \\
% \% of OBD selection& 0.4\% & 8.5\% & 19.0\% & 36.9\% & \textbf{35.2\%} &0\%  &4.2\% &32.2\%  &40.2\%  & 23.4\%\\
% \% of MTD selection& 0.4\% & 5.7\% & 16.8\% & 36.9\% & 40.2\% & 0.6\% & 9.8\% & 26.0\% & 36.4\% & 27.2\%\\
% Efficacy estimation& 0.16 & 0.20 &0.25  & 0.30 &0.35&  &  &  &  &  \\ 
% Number of patients enrolled& 4.9 & 6.6 & 8.1 & 7.8 & \textbf{8.5}& 4.5 & 7.0 & 8.9 & 8.4 & 7.2\\
% Number of patients backfill &1.2 &1.6 &1.7 &1.3 &0 & & & & & \\
% Trial Duration& \multicolumn{5}{c}{494}  & \multicolumn{5}{c}{539}\\
% \textbf{Scenario 4}&  &  &  &  & &  &  &  &  &  \\
% % Efficacy probability& 0.07 &0.14  &0.21  &0.28  &0.35 & 0.14 &0.28  &0.42  &0.56  &0.7  \\
% Efficacy probability& 0.07 &0.14  &0.21  &0.28  &0.35 &  &  &  &  &  \\
% Toxicity probability& 0.04 &0.08  &0.13  &0.17  &0.25 &  & &  &  &  \\
% \% of OBD selection& 0.4\% & 8.3\% &18.6\%  & 35.7\% & \textbf{37.0\%} &0\%&3.4\%  &36.4\% &39.4\% &20.8\%\\
% \% of MTD selection& 0.4\% & 5.8\% & 17.1\% & 36.3\%  & 40.4\% &  &  &  &  & \\
% Efficacy estimation&  0.14& 0.17 &0.22  &0.27  &0.31 &  &  &  &  &   \\ 
% Number of patients enrolled& 5.0 & 6.6 & 8.1 & 7.8 & \textbf{8.5}&  &  &  &  & \\
% Number of patients backfill & 1.3&1.6 &1.7 &1.3 &0& & & & & \\
% \textbf{Scenario 5}&  &  &  &  & &  &  &  &  & \\
% % Efficacy probability& 0.05 &0.1  &0.15  &0.2  &0.2 & 0.1 &0.2  &0.3  &0.4  &0.4  \\
% Efficacy probability& 0.1& 0.2  &0.3  &0.4  &0.4 &  &  &  &  &  \\
% Toxicity probability& 0.04 &0.08  &0.16  &0.25  &0.35 & 0.04 &0.08  &0.16  &0.25  &0.35 \\
% \% of OBD selection& 0.4\% & 16.6\% & 36.5\% & \textbf{33.9\%} & 12.6\%&0\%&5.2\%&46.6\%&40.7\% & 7.5\% \\
% \% of MTD selection& 0.4\% & 11.0\% & 36.8\% & 37.0\% & 14.8\% & 0.7\% & 15.0\% & 45.6\% & 31.2\% & 7.5\% \\
% Efficacy estimation& 0.18 & 0.23 &0.30  &0.36  &0.41 &  &  &  &  &  \\
% Number of patients enrolled & 4.8 & 7.0 & 10.0 & \textbf{8.0} & 5.0& 4.5 & 8.0 & 11.8 & 8.3 & 3.5\\
% Number of patients backfill &1.2 &1.5 &1.3 &0.7 & & & & & &\\
% Trial Duration& \multicolumn{5}{c}{490}  & \multicolumn{5}{c}{530}\\\bottomrule
% \end{tabular}
% \end{center}
% \end{table}

\end{appendices}



\end{document}