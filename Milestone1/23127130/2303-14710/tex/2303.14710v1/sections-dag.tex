\section{Extension to labelled DAGs}\label{sec:dag}

In this last section we demonstrate the applicability of our method by
establishing a counting formula for the classical model of labelled DAGs,
counted by vertices, edges, and sources.
This corresponds to a sequence obtained by Gessel in~\cite{Gessel1996} using
a generating functions approach.
The recurrence relations we obtain here are different and in particular they
involve no subtraction, which makes them amenable to effective random sampling.
To our knowledge, this is the first such formula for labelled DAGs.

The difference between the formula presented here and the one given in our
previous work~\cite{GPV2021} is that here we take all DAGs into account, not
only those with one sink.
But we are also able to specialise the formula to some sub-classes, like we did
for DOAG in Section~\ref{sec:subclasses}.

\subsection{Recursive decomposition}

In a first attempt to decompose regular vertex-labelled DAGs, one might be
tempted to devise a decomposition similar to DOAGs by removing the
\emph{smallest} source at each step.
However, in this case this makes the recurrence difficult to express.
Instead we count vertex-labelled DAGs with a \emph{distinguished} source (this
operation is called pointing), which makes the decomposition much simpler as we
do not have to maintain an ordering.

Let~$A_{n,m,k}$ denote the number of labelled DAGs with~$n$ vertices,~$k$
sources,~$m$ edges, and any number of sinks.
Just like DOAGs, these objects are thus not necessarily connected.
The number of such DAGs with a distinguished (or pointed) source is given
by~$k \cdot A_{n,m,k}$ since any of the~$k$ sources may be distinguished.
Let~$D$ denote one such DAG and let~$v$ denote its distinguished source.
Removing the distinguished source in~$D$ and decrementing the labels of the
vertices of higher label than~$v$ by one yields a regular vertex-labelled
DAG~$D'$ with~$n-1$ vertices.
Moreover, the three pieces of information that are necessary to reconstruct the
source are the following:
\begin{enumerate}
  \item the label~$\ell$ of the source~$v$ which has been removed;
  \item the set~$S$ of sources of~$D'$ which have been uncovered by
    removing~$v$;
  \item the set~$I$ of internal (non-sources) vertices of~$D'$ that were pointed
    at by~$v$.
\end{enumerate}
The reconstruction is then straightforward:
\begin{itemize}
  \item increment all the labels that are greater or equal to~$\ell$ by one;
  \item create a new vertex labelled~$\ell$ and ``mark'' it: this is the
    distinguished source;
  \item add edges from~$\ell$ to all the vertices from~$S$ and~$I$.
\end{itemize}
This decomposition differs from that of DOAGs in that we have not ordering to
maintain on the set of vertices of the DAG, hence any subset~$S$ of the set of
sources of~$D'$ is licit here.
The triplet~$(\ell, S, I)$ is thus not constrained which leads to a simple
counting formula.
This leads to the following recursive formula where~$p$ denotes the out-degree
of~$v$ (and thus the cardinality of~$S \cup I$ using the notations from above).
\begin{equation}\label{eq:recurrence:A}
  \begin{aligned}
    A_{1, m, k} &= \indicator{m = 0~\wedge~k = 1} \\
    A_{n, m, k} &= 0 & \text{when}~k \leq 0~\text{or}~k > n\\
    A_{n, m, k} &= \frac{n}{k} \sum_{p = 0}^{n-k} \sum_{i = 0}^p
                   A_{n-1, m-p, k-1+p-i} \binom{n-k-p+i}{i} \binom{k-1+p-i}{p-i}
                & \text{otherwise.}
  \end{aligned}
\end{equation}
Here, the first binomial coefficient~$\binom{n-k-p+i}{i}$ counts the number of
possibilities for the set~$I$ of edges to internal vertices.
The second binomial coefficients~$\binom{k-1+p-i}{p-i}$ counts the number of
possibilities for the other edges: that than point at sources of~$D'$.

Again, this formula counts all labelled DAGs, not only those with one sink.
Computing the first terms of the sequence indeed yields the same numbers as the
sequence~\href{https://http://oeis.org/A003024}{\texttt{A003024}} in the OEIS
and first enumerated in~\cite{Robinson1970, Stanley1973, Robinson1973}.
Furthermore, as for DOAGs, we can capture some subclasses of labelled DAGs by
putting restrictions on the out-degrees of the vertices.
For instance, labelled DAGs with only one source and one sink can be counted by
enforcing~$p > 0$ in the summation and computing~$\sum_{m} A_{n,m,1}$.
Computing the first terms of this sequence, we get back that values from
\href{https://oeis.org/A165950}{\texttt{A165950}} in the OEIS related to the
papers~\cite{Gessel1995,Gessel1996}.

\subsection{Random generation}

A recursive random sampling algorithm similar to
Algorithm~\ref{algo:sample:DOAG} from Section~\ref{sec:sampling:rec} can be
obtained from formula~\eqref{eq:recurrence:A}.
The only difference in methodology from  Algorithm~\ref{algo:sample:DOAG} is
that one has to deal with the marking of the sources here and thus the division
by~$k$ at the third line of~\eqref{eq:recurrence:A}.
It can be handled as follows: at every recursive call, first generate a labelled
DAG with a distinguished source (counted by~$k \cdot A_{n, m, k}$) and then
forget which source was distinguished.
Since the recursive formula for~$k \cdot A_{n, m, k}$ has no division, the
uniform sampler of marked DAGs is obtained using the standard recursive method.
Moreover, forgetting which source was marked does not introduce bias in the
distribution since all sources have the same probability to be marked.
A uniform random sampler of labelled DAGs with~$n$ vertices,~$k$ sources,
and~$m$ edges is described in Algorithm~\ref{algo:sample:dag}.

\begin{algorithm}[htb]
  \caption{Uniform random sampler of vertex-labelled DAGs.}%
  \label{algo:sample:dag}
  \begin{algorithmic}
    \Require{Three integers~$(n, m, k)$ such that~$A_{n,m,k} > 0$}
    \Ensure{A uniform random vertex-labelled DAG with~$n$ vertices (including
    $k$ sources and one sink), and~$m$ edges}
    \Function{sample}{$n, m, k$}
      \If{$n \leq 1$}
        generate the (unique) labelled DAG with~$1$ vertex
      \Else{}
        \State{\textbf{pick}~$(p, i)$ with probability
          $\displaystyle {A_{n - 1, m - p, k-1+p-i} \binom{n - k - p + i}{i}
           \binom{k - 1 + p-i}{p-i}} / {A_{n, m, k}}$}
        \assign{$D'$}{\Call{sample}{$n - 1, m - p, k - 1 + p-i$}}
        \assign{$I$}{a uniform subset of size~$i$ of the inner vertices of~$D'$}
        \assign{$S$}{a uniform subset of size~$(p-i)$ the sources of~$D'$}
        \assign{$\ell$}{\Call{Unif}{$\llbracket 1; n \rrbracket$}}
        \State{relabel~$D'$ by adding one to all labels~$\ell' \geq \ell$}
        \return{the DAG obtained by adding a new source to~$D'$ with
        label~$\ell$ and~$I \cup S$ as its out-edges}
      \EndIf{}
    \EndFunction{}
  \end{algorithmic}
\end{algorithm}

Here again, since we can count some sub-classes of labelled DAGs by applying
restrictions on the out-degree of the vertices in their recurrence formula
(equation~\eqref{eq:recurrence:A}), we can get an efficient uniform sampler for
each of these sub-classes.
This is achieved simply by using the new sequence in
Algorithm~\ref{algo:sample:dag}, without any further changes.
In particular, this gives access to a polynomial time algorithm for sampling
labelled DAGs with small bounded degree, which is something that rejection-based
approaches, or the Markov-Chain approach from~\cite{MDB2001,KM2015} would
struggle at.
