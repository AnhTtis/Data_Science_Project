\section{Matrix encoding}\label{sec:matrix}

In this section, we introduce the notion of \emph{labelled transition matrices}
and give a bijection between DOAGs and these matrices, thus offering an
alternative point of view on DOAGs.
In the next two section, we then rely heavily on this encoding to prove an
asymptotic equivalent for the number of DOAGs with~$n$ vertices, and to design a
efficient uniform random sampler for those DOAGs.
We also recall the definition and basic properties of variations, which are an
elementary combinatorial object playing a key role in our analysis.

\subsection{The encoding}

The decomposition scheme described in Section~\ref{sec:def} corresponds to a
traversal of the DOAG\@.
This traversal induces a labelling of the vertices from~$1$ to~$n$, which allows
us to associate the vertices of the graph to these integers in a canonical way.
We then consider its transition matrix using these labels as indices.
Usually, the transition matrix of a directed graph~$D$ is defined as the
matrix~${(a_{i,j})}_{1 \leq i, j \leq n}$ such that~$a_{i,j}$ is~$1$ if there is
an edge from vertex~$i$ to vertex~$j$ in~$D$, and~$0$ otherwise.
This representation encodes the set of the edges of a DAG but not the edge
ordering of DOAGs.
In order to take this ordering into account, we use a slightly different
encoding.

\begin{definition}[Labelled transition matrix of a DOAG]
  Let~$D$ be a DOAG with~$n$ vertices.
  We associate the vertices of~$D$ to the integers from~$1$ to~$n$ corresponding
  to their order in the vertex-by-vertex decomposition.
  The \emph{labelled transition matrix} of~$D$ is the matrix~${(a_{i,j})}_{1
  \leq i, j \leq n}$ with integer coefficients such that~$a_{i,j} = k > 0$ if
  and only if there is an edge from vertex~$i$ to vertex~$j$ and this edge is
  the~$k$-th outgoing edge of~$i$.
  Otherwise~$a_{i,j} = 0$.
\end{definition}

An example of a DOAG and its transition matrix are pictured in
Figure~\ref{fig:matrix}.
The thick lines are not part of the encoding and their meaning will be explained
later.
Let~$\phi$ denote the function mapping a DOAG to its labelled transition matrix.
This function is clearly injective as the edges of the graph can be recovered as
the non-zero entries of the matrix, and the ordering of the outgoing edges of
each vertex is given by the values of the corresponding entries in each row.
Characterising the image of~$\phi$ however requires more work.

\begin{figure}[htb]
  \centering
  \includegraphics[scale=1]{images-matrix}
  \caption{An example DOAG and its labelled transition matrix, the zeros are
  represented by the absence of a number.}%
  \label{fig:matrix}
\end{figure}

We can make some observations.
First, by definition of the traversal of the DOAG, the labelled transition
matrix of a DOAG is strictly upper triangular.
Indeed, since the decomposition algorithm removes one \emph{source} at a time,
the labelling it induces is a topological sorting of the graph.
Moreover, since the non-zero entries of row~$i$ encode the \emph{ordered} set of
outgoing edges of vertex~$i$, these non-zero entries form a permutation
\begin{itemize}
  \item a non-zero value cannot be repeated within a row;
  \item and if a row contains~$d \geq 1$ non-zero entries, then these are the
    integers from~$1$ to~$d$, in any order.
\end{itemize}
Informally, these two properties ensure that a matrix encodes a labelled DOAG (a
DOAG endowed with a labelling of its vertices) and that this labelling is a
topological sorting of the graph.
However, they are not enough to ensure that this topological sorting is
precisely the one that is induced by the decomposition.
The matrices satisfying these two properties will play an important role in the
rest of the paper.
We call them ``variation matrices''.

\begin{definition}[Variation]\label{def:var}
  A variation is a finite sequence of non-negative integers such that
  \begin{enumerate}
    \item each strictly positive number appears at most once;
    \item if~$0 < i < j$ and~$j$ appears in the sequence, then~$i$ appears too.
  \end{enumerate}
  The size of a variation is its length.
\end{definition}
For instance, the sequence~$(6, 2, 3, 0, 0, 1, 4, 0, 5)$ is a variation of
size~$9$ but the sequences~$(1, 0, 3)$ and~$(1, 0, 2, 2)$ are not variations.
Variations can also be defined as interleavings of a permutation with a
sequence of zeros.
One of the earliest references to these objects dates back
to~\citeyear{Izquierdo1659} in Izquierdo's
\citetitle{Izquierdo1659}~\cite[Disputatio 29]{Izquierdo1659}.
They also appear in Stanley's book as the second entry of his \emph{Twelvefold
Way}~\cite[page 79]{Stanley2011}, a collection of twelve basic but fundamental
counting problems.
Variations are relevant to our problem as they naturally appear as rows of the
labelled transition matrices defined in this section.
Some of their combinatorial properties will be exhibited in the next section.

\begin{definition}[Variation matrix]
  Let~$n > 0$ be a positive integer.
  A matrix of integers~${(a_{i,j})}_{1 \leq i, j \leq n}$ is said to be a
  variation matrix if
  \begin{itemize}
    \item it is strictly upper triangular;
    \item for all~$1 \leq i \leq n - 1$, the sub-row~${(a_{i,j})}_{i < j \leq
      n}$ is a variation (of size~$n-i$).
  \end{itemize}
  Equivalently, a variation matrix can be seen as a sequence of variations~$(v_1,
  v_2, \ldots, v_{n-1})$ where for all~$1 \leq i \leq n - 1$, the variation~$v_i$
  has size~$i$.
\end{definition}

We have established that all labelled transition matrices of DOAGs are variation
matrices.
Note that the converse is not true.
For instance, the matrix pictured in Figure~\ref{fig:invalid} is a variation
matrix of size~$3$ that does not correspond to any DOAG\@.
The property of this matrix which explains why it cannot be the image of a DOAG
is pictured in red on the Figure, and will be explained in the rest of this
section, in particular in Theorem~\ref{thm:matrix}.

\begin{figure}[htb]
  \centering
  \includegraphics[scale=1]{images-invalid1}
  \caption{An example of a matrix of variations that cannot be obtained as a
  labelled transition matrix of a DOAG\@.
  The labelled DOAG that it encodes is not labelled according to the
  decomposition order.}%
  \label{fig:invalid}
\end{figure}
We now characterise which of those variation matrices can be obtained as the
labelled transition matrix of a DOAG\@.
Consider such a matrix.

Note that in any column~$j$, the non-zero entry with the \emph{highest}
index~$i$ (that is in the lowest row on the picture with a non-zero element in
column~$j$) has a spacial role: it corresponds to the last edge pointing to
vertex~$j$ when decomposing the DOAG\@.
These elements are underlined by a thick red line in Figure~\ref{fig:matrix},
and when a column has no non-zero entry at all, the top line is pictured in
thick red instead.
When several such cells occur on the same row~$i$ in the matrix, they
correspond to several sources that are discovered at the same time, upon
removing vertex~$i$ in the DOAG\@.
Recall that the decomposition algorithm sorts the labels of these new sources by
following the total order of the outgoing edges of vertex~$i$.
As a consequence, the entries above the thick red line have to be increasingly
sorted (from left to right).
For instance, observe that there are three consecutive underlined cells in the
first row of the matrix in Figure~\ref{fig:matrix}.
Indeed, when removing the first source of the DOAG on the left, we uncover three
new sources which are respectively in first, second and fourth position in the
outgoing edges order of the removed source.

Another important property is that if any underlined cells occur in a row of the
matrix, these are always the first non-zero entries in that row.
This is because, when the decomposition algorithm discovers new sources, it
considers them to be larger than all the previously discovered sources.
As a consequence, the vertices are processed (removed by the algorithm) in the
same order as they become sources during the decomposition.
For a given vertex~$i$, this means that those of its children than become
sources upon removing~$i$ will be processed before the other children.
And thus underlined entries on row~$i$ appear on the left of any other non-zero
entry.
Put differently, the red thick path drawn in Figure~\ref{fig:matrix} is visually
a staircase that only goes down when moving toward the right of the matrix.

The two properties that we just described actually characterise the variation
matrices that can be obtained as the labelled transition matrices of a DOAG\@.

\begin{theorem}\label{thm:matrix}
  All labelled transition matrices of DOAGs are variation matrices.
  Furthermore, let~$A = {(a_{i, j})}_{1 \leq i, j \leq n}$ be a variation
  matrix, and for all~$j \in \llbracket 1; n \rrbracket$, let~$b_j$ denote the
  largest~$i \leq n$ such that~$a_{i,j} > 0$ if such an index exists and~$0$
  otherwise.
  Then,~$A$ is the labelled transition matrix of some DOAG if and only if the
  two following properties hold:
  \begin{itemize}
    \item the sequence~$j \mapsto b_j$ is weakling increasing;
    \item whenever~$b_j = b_{j+1}$, we have that~$a_{b_j,j} < a_{b_j,j+1}$.
  \end{itemize}
\end{theorem}

\begin{proof}
  The fact that the labelled transition matrix of a DOAG is a variation matrix
  is clear from the definition.
  We prove the rest of the theorem in two steps.

  \begin{description}
    \item[Labelled transition matrices satisfy the conditions.]
      Let~$A$ be the labelled transition matrix of a DOAG of size~$n$ and
      let~${(b_j)}_{1 \leq j \leq n}$ be defined as in the statement of the
      theorem.
      We shall prove that it satisfies the two properties of the theorem.

      Observe that for all~$1 \leq j \leq n$,~$b_j$ is the decomposition step at
      which the vertex labelled~$j$ becomes a source.
      This is zero for the sources of the initial DOAG\@.
      As discussed above, since the sources are processed by the decomposition
      algorithm in the same order as they are discovered, the sequence~$j
      \mapsto b_j$ is necessarily weakly increasing.
      The second point is also a consequence of the above discussion: two
      vertices which become sources at the same time get labelled in the same
      order as their position as children of their parent.

    \item[Any matrix satisfying the conditions is a labelled transition matrix.]
      Let~$A$ be a variation matrix of size~$n$ and let~$b$ be as in the
      statement of the theorem and satisfying the two given properties.
      We shall prove that~$A$ is the image by~$\phi$ of some DOAG\@.

      Let~$V = \llbracket 1; n \rrbracket$ and~$E = \left\{(i, j) \in \llbracket
      1; n \rrbracket^2~|~a_{i,j} > 0\right\}$.
      We have that~$(V, E)$ defines an acyclic graph since~$A$ is strictly
      upper-triangular.
      In addition, for each~$v \in V$, define~$\prec_v$ to be the total order on
      the outgoing edges of~$v$ in~$(V, E)$ such that~$u \prec_v u'$ if and only
      if~$a_{v,u} < a_{v,u'}$ in~$A$.
      This is well defined since the outgoing edges of~$v$ are precisely the
      integers~$j$ such that~$a_{v,j} > 0$ and since the non-zero entries of the
      row~$v$ are all different by definition of variation matrices.
      Finally, define~$\prec_\emptyset$ to be the total order on the sources
      of~$(V, E)$ such that~$u \prec_\emptyset v$ if and only if~$u < v$ as
      integers.
      Let~$D$ be the DOAG given by~$(V, E, {(\prec_v)}_{v\,\in V \cup
      \{\emptyset\}})$.

      Remember that DOAGs are considered up to a permutation of their vertices
      that preserves~$E$ and~$\prec$.
      In order to finish this proof, we have to check that the particular
      labelling encoded by~$V$ is indeed the labelling induced by the
      decomposition of~$D$.
      Then it will be clear that~$\phi(D) = A$ and we will thus have exhibited a
      pre-image of~$A$.

      First, since~$A$ is strictly upper-triangular, its first column contains
      only zeros and thus~$1$ is necessarily a source of~$D$.
      In addition, by definition of~$\prec_\emptyset$, it must be the smallest
      source.
      Then, upon removing~$i$, one of two things can happen:
      \begin{itemize}
        \item either~$D$ has more than one source, in which case~$2$ is the
          second source by monotony of the sequence~${(b_j)}_{1 \leq j \leq n}$;
        \item or~$1$ was the unique source of~$D$, in which case the next source
          to be processed is its first child.
          The children of~$1$ are the integers~$j$ such that~$b_j = 1$.
          By monotony of~$b_j$ again (or triangularity of the matrix),~$2$ is
          necessarily a child of~$1$.
          Moreover, by the second property of the sequence~$b$, we have that for
          all~$j < j'$ such that~$b_j = b_j' = 1$,~$a_{1,j} < a_{1,j'}$.
      \end{itemize}
      In both case, we proved that~$2$ is the second vertex to be processed.
      We can then repeat this argument on the DOAG obtained by removing~$1$,
      which corresponds to the matrix~${(a_{i,j})}_{2 \leq i, j \leq n}$ and
      conclude by induction.\qedhere
  \end{description}
\end{proof}

We have now established that the encoding~$\phi$ of DOAGs as labelled transition
matrices is a bijection from DOAGs to the matrices described in
Theorem~\ref{thm:matrix}.
From now on, we will write ``a labelled transition matrix'' to refer to such a
matrix.
We can also state a few simple properties of these matrices.
By definition we have that
\begin{itemize}
  \item the number of vertices of a DOAG is the dimension of its labelled
    transition matrix;
  \item the number of edges of a DOAG is the number of non-zero entries the
    matrix;
  \item the sinks of the DOAG correspond to the zero-filled rows of the matrix;
  \item the sources of the DOAG correspond to the zero-filled column of the
    matrix.
\end{itemize}
Furthermore, the first property of the sequence~${(b_j)}_{1 \leq j \leq n}$
defined in Theorem~\ref{thm:matrix} implies that the zero-filled columns of the
matrix must be contiguous and on the left of the matrix.
The number of sources of the DOAG is thus the maximum~$j$ such that column~$j$
is filled with zeros.

We will see in the next section that working at the level of the labelled
transition matrices, rather than at the level of the graphs, is more handy to
exhibit asymptotic behaviours.
This will also inspire an efficient uniform random sampler of DOAGs with~$n$
vertices in Section~\ref{sec:sampling:rej}.
