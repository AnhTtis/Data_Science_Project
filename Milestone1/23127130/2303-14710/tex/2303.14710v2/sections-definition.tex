\section{Definition and recursive decomposition}\label{sec:def}

We introduce a model of directed acyclic graphs called ``Directed Ordered
Acyclic Graphs'' (or DOAGs) which is similar to the classical model of
unlabelled DAGs but where, in addition, we have a total order on the outgoing
edges of each vertex.

\begin{definition}[Directed Ordered Graph]\label{def:dog}
  A directed ordered graph is a triple~$(V, E, {(\prec_v)}_{v\,\in V \cup
  \{\emptyset\}})$ where:
  \begin{itemize}
    \item $V$ is a finite set of vertices;
    \item $E \subset V \times V$ is a set of edges;
    \item for all~$v \in V$,~$\prec_v$ is a total order over the set of
      \emph{outgoing} edges of~$v$;
    \item and~$\prec_\emptyset$ is a total order over the set of \emph{sources}
      of the graph, that is the vertices without any incoming edge.
  \end{itemize}
  Two such graphs are considered to be \emph{equal} if there exists a bijection
  between their respective sets of vertices that preserves both the edges and
  the order relations~$\prec_v$ and~$\prec_\emptyset$.
\end{definition}

\begin{definition}[Directed Ordered Acyclic Graph]\label{def:doag}
  A directed ordered acyclic graph (or DOAG for short) is a directed ordered
  graph~$(V, E, {(\prec_v)}_{v\,\in V \cup \{\emptyset\}})$ such that~$(V, E)$,
  seen as a directed graph, is acyclic.
\end{definition}

We study this class as a whole, however, some sub-classes are also of special
interest, in particular for the purpose of modelling compacted data structures.
Tree structures representing real data, such as XML documents for instance, are
rooted trees.
When these trees are compacted, the presence of a root translates into a unique
source in the resulting DOAG\@.
Similarly, DOAGs with a single sink will arise naturally when compacting trees
which bear a single type of leaves.
For this reason, we will also discuss how to approach the sub-classes of DOAGs
with a single source and/or a single sink in this document.
As an example, the first line of Figure~\ref{fig:firstdoags} depicts all the
DOAGs with exactly~$3$ edges and~$4$ vertices (among which exactly~$2$ are
sources).
On the second line, we also listed all DOAGs with exactly one sink, one source,
and up to~$4$ edges.

\begin{figure}[htb]
  \centering
  \includegraphics[scale=1]{images-smallcases}

  \bigskip

  \noindent%
  \includegraphics[scale=1]{images-0}\quad%
  \includegraphics[scale=1]{images-1}\quad%
  \includegraphics[scale=1]{images-2}\quad%
  \includegraphics[scale=1]{images-3}\quad%
  \includegraphics[scale=1]{images-4}
  \caption{On the first line: all DOAGs with exactly~$3$ edges and~$4$ vertices,
    among which exactly~$2$ are sources.
    On the second line: all DOAGs with exactly~$1$ source,~$1$ sink, and up
    to~$4$ edges.
    All edges are implicitly oriented from top to bottom.
    The leftmost source is the smallest one in every picture.
    The order of the outgoing edges of each vertex is indicated by the thinner
    blue arrows (always from left to right here).
  }\label{fig:firstdoags}
\end{figure}

\subsection{Recursive decomposition}

We describe a canonical way to recursively decompose a DOAG into smaller
structures.
The idea is to remove vertices one by one in a deterministic order, starting
from the smallest source (with respect to their ordering~$\prec_\emptyset$).
Formally, we define a decomposition step as a bijection between the set of DOAGs
with at least two vertices and the set of DOAGs given with some extra
information.
Let~$D$ be a DOAG with at least~$2$ vertices and consider the new graph~$D'$
obtained from~$D$ by removing its \emph{smallest} source~$v$ and its outgoing
edges.
We also need to specify the ordering of the sources of~$D'$.
We consider the ordering where the \emph{new} sources of~$D'$ (those
that have been uncovered by removing~$v$) are considered to be in the \emph{same
order} (with respect to each other) as they appear as children of~$v$ and all
\emph{larger} than the other sources.
The additional information necessary to reconstruct~$D$ from~$D'$ is the
following:
\begin{enumerate}
  \item the number~$s \geq 0$ of sources of~$D'$ which have been uncovered by
    removing~$v$;
  \item the (possibly empty) set~$I$ of internal (non-sources) vertices of~$D'$
    such that there was an edge in~$D$ from~$v$ to them;
  \item the function $f:I \to \llbracket 1; s + |I| \rrbracket$ identifying the
    positions, in the list of outgoing edges of~$v$, of the edges pointing to
    an element of~$I$.
\end{enumerate}

In fact, this decomposition describes a bijection between DOAGs with at
least~$2$ vertex and quadruples of the form~$(D', s, I, f)$ where~$D'$ is a DOAG
with~$k'$ sources,~$I$ is a subset of its internal vertices,~$0 \leq s \leq k'$
is a non-negative integer, and~$f: I \to \llbracket 1; s + |I| \rrbracket$ is an
injective function.
Indeed, the inverse transformation is as follows.
Create a new source~$v$ with~$s + |I|$ outgoing edges such that the~$i$-th of
these edges is connected to~$f^{-1}(i)$ when~$i \in f(I)$ and is connected to
one of the~$s$ largest sources of~$D'$ otherwise.
The~$s$ largest sources of~$D'$ must be connected to the new source exactly once
and in the same order as they appear in the list of sources of~$D'$.
Note that the order in which the vertices are removed when iterating this
process corresponds to a BFS-based topological sort of the graph.
Fig.~\ref{fig:decomposition} pictures the first~$3$ decomposition steps of an
example DOAG\@.

\begin{figure}[htb]
  \centering
  \includegraphics[scale=1]{images-decomposition}
  \caption{Recursive decomposition of a DOAG by removing sources one by one in a
  breadth first search (BFS) fashion.
  The edges are implicitly oriented from top to bottom and the order of the
  outgoing edges of each vertex is indicated by the thinner blue arrows (always
  from left to right here).
  The integer labels at each stage indicate the ordering of the sources.}%
  \label{fig:decomposition}
\end{figure}

This decomposition can be used to establish a recursive formula for counting
DOAGs, which is given below.
Let~$D_{n,m,k}$ denote the number of DOAGs with~$n$ vertices,~$m$ edges and~$k$
sources, then we have:
\begin{equation}\label{eq:recurrence}
  \begin{aligned}
    D_{1, m, k} &= \indicator{m=0 \, \land \, k=1} \\
    D_{n, m, k} &= 0
                & \text{when}~k \leq 0 \\
    D_{n, m, k} &= \sum_{p=0}^{n - k} \sum_{i=0}^p
                   D_{n-1, m-p, k-1+p-i} \binom{n-k-p+i}{i} \binom{p}{i} i!
                & \text{otherwise,} \\
  \end{aligned}
\end{equation}
where~$p = s + i$ corresponds the out-degree of the smallest source, the
term~$\binom{n-k-p+i}{i} = \binom{n-k-s}{i}$ accounts for the choice of the
set~$I$ and the term~$\binom{p}{i} i{!}$ accounts for the number of injective
functions~$f: I \to \llbracket 1; p \rrbracket$.
The upper bound on~$p$ in the sum is justified by the following combinatorial
arguments.
Since~$p$ is the out-degree of the smallest source, it is upper bounded by the
number of vertices it might have an outgoing edge to, that is the number~$(n-k)$
of non-source vertices of the graph.

\subsubsection{Special sub-classes based on out-degree constraints}%
\label{sec:subclasses}

Since~$p = i + s$ is the out-degree of the removed source in the above
summation, it is easy to adapt this sequence for counting DOAGs with
constraints on the out-degree of the vertices.
For instance, DOAGs with only one sink are obtained by ensuring that every
vertices has at least out-degree one.
In other words, let the summation start at~$p=1$.
Note that restricting DOAGs to have only one single sink or one single source
ensures that they remain connected, however not all connected DOAGs are obtained
this way.
As another example, DOAGs with out-degree bounded by some constant~$d$ are
obtained by letting~$p$ range from~$0$ to~$\min(n-k, d)$.

The general principle is that the DOAGs whose vertices' out-degrees are
constrained to belong to a given set~$\mathcal P$, are enumerated by the
following recursive formula.
\begin{equation}\label{eq:recurrence:general}
  \begin{aligned}
    D^{\mathcal P}_{1, m, k} &= \indicator{m=0 \, \land \, k=1} \\
    D^{\mathcal P}_{n, m, k} &= 0 &  \text{when}~k \leq 0 \\
    D^{\mathcal P}_{n, m, k} &=
      \sum_{p \in \mathcal P} \sum_{i=0}^p D^{\mathcal P}_{n-1, m-p, k-1+p-i}
      \binom{n-k-p+i}{i} \binom{p}{i} i! &  \text{otherwise,}
  \end{aligned}
\end{equation}

The first values of the sequence~$D_{n,m} = \sum_k D_{n,m,k}$ counting DOAGs by
number of vertices and edges only are given in Table~\ref{table:firstterms}.
Table~\ref{table:others} gives the first values of~$D^{\mathcal P}_n =
\sum_{m,k} D^{\mathcal P}_{n,m,k}$ for some choices of~$\mathcal P$.
None of these sequences seem to appear in the online encyclopedia of integer
sequences (OEIS%
\footnote{\url{https://oeis.org/}}%
) yet.

\begin{table}[htb]
  \centering
  \caption{Number of DOAGs with~$n$ vertices and~$m$ edges for small values
  of~$n$ and~$m$.}\label{table:firstterms}
  \begin{tabular}{llp{37em}}
    \toprule
    $n$ & $D_n$ & $D_{n,m} = \sum_k D_{n,m,k}$ for~$m = 0, 1, 2, 3, \ldots$ \\
    \midrule
    $1$ & $1$ & $1$ \\
    $2$ & $2$ & $1, 1$ \\
    $3$ & $8$ & $1, 2, 3, 2$ \\
    $4$ & $95$ & $1, 3, 8, 17, 27, 27, 12$ \\
    $5$ & $4858$ & $1, 4, 15, 48, 139, 349, 718, 1136, 1272, 888, 288$ \\
    $6$ & $1336729$ & $1$, $5$, $24$, $100$, $391$, $1434$, $4868$, $14940$,
                      $40261$, $92493$, $175738$, $266898$, $310096$, $258120$,
                      $136800$, $34560$
    \\
  \end{tabular}
\end{table}

\begin{table}[htb]
  \centering
  \caption{Number of DOAGs with~$n$ vertices and a constrained set of allowed
  degrees.}\label{table:others}

  \begin{tabular}{lp{.75\linewidth}}
    \toprule
    Restrictions & sequence \\
    \midrule
%
    $\mathcal P = \mathbb N$ &
    $1$, $2$, $8$, $95$, $4858$, $1336729$, $2307648716$, $28633470321822$,
    $2891082832793961795$, $2658573971407114263085356$,
    $24663703371794815015576773905384$, $\ldots$ \\
%
    $\mathcal P = \mathbb N$, $k=1$ &
    $1$, $1$, $4$, $57$, $3399$, $1026944$, $1875577035$, $24136664716539$,
    $2499751751065862022$, $2342183655157963146881571$,
    $22043872387559770578846044961204$, $\ldots$ \\
%
    $\mathcal P = \mathbb N^\star$, $k=1$ &
    $1$, $1$, $3$, $37$, $2103$, $627460$, $1142948173$, $14701782996075$,
    $1522511169925136833$, $1426529804350999351686869$,
    $13426022673540053054145359653988$, $\ldots$ \\
%
    $\mathcal P=\{0, 1, 2\}$, $k=1$ &
    $1$, $1$, $4$, $23$, $191$, $2106$, $29294$, $495475$, $9915483$,
    $229898277$, $6074257926$, $180460867600$, $5962588299084$, $\ldots$
  \end{tabular}
\end{table}

\subsection{Computational aspects of the counting problem}\label{sec:count}

In order to implement a random sampler for DOAGs, we will have to pre-compute
the values of~$D_{n,m,k}$ for all~$n,m$, and~$k$ up to a given bound.
This can be achieved easily using equation~\ref{eq:recurrence} and a dynamic
programming approach.
We do not give the algorithm here as it is a straightforward implementation of
the above formula.
But in this section we give some details on its computational aspects.
First, in Lemma~\ref{lem:support} we characterise the indices~$n, m, k$ such
that~$D_{n, m, k} > 0$.
This can be used to avoid unnecessary recursive calls and to choose a
memory-efficient data-structure for storing the results.
Then in Theorem~\ref{thm:counting:cost} we give the complexity of the counting
procedure in terms of bitwise operations.

\begin{lemma}\label{lem:support}
  For~$n > 1$, we have~$D_{n, m, k} \neq 0$ if and only if~$1 \leq k \leq n$
  and~$n - k \leq m \leq \binom{n}{2} - \binom{k}{2}$.
\end{lemma}
\begin{proof}
  There is always at least one source in a DOAG, hence~$1 \leq k \leq n$ is a
  necessary condition for~$D_{n,m,k}$ to be non-zero.
  Now let~$n$ and~$k$ be such that~$1 \leq k \leq n$ and consider~$n$ vertices
  labelled from~$1$ to~$n$.
  The maximum possible number of edges in a DOAG with~$k$ sources is obtained
  by putting an edge from vertex~$i$ to vertex~$j$ if and only if~$i < j$ and~$j
  > k$ as pictured below.
  \begin{center}
    \includegraphics[scale=1]{images-lemma1}
  \end{center}
  This corresponds to~$\binom{n}{2} - \binom{k}{2}$ edges since there
  are~$\binom{n}{2}$ pairs~$(i, j)$ such that~$i < j$ and~$\binom{k}{2}$
  pairs~$(i, j)$ such that~$i < j \leq k$.
  Furthermore, it is possible to remove any number of edges from this maximal
  case while keeping at least one edge to each vertices~$i > k$.
  This accounts for~$(n - k)$ mandatory edges.
\end{proof}

In Theorem~\ref{thm:counting:cost} we get a straightforward upper bound on the
number of arithmetic operations necessary to compute all the~$D_{n, m, k}$ up to
certain bounds.
As it is usual in combinatorial enumeration, there is a hidden cost factor in
the \emph{size} of the numbers at stake: the more they grow the more costly
arithmetic operations become.
To account for this cost we also give an upper bound on the bit-size of all
numbers being multiplied.

\begin{theorem}\label{thm:counting:cost}
  Let~$N, M > 0$ be two integers.
  Computing~$D_{n, m, k}$ for all~$n \leq N$,~$m \leq M$ and all possible~$k$
  can be done with~$O(N^4 M)$ multiplications of integers of size at
  most~$O(\max(M, N) \ln N)$.
\end{theorem}

The first part of Theorem~\ref{thm:counting:cost} is straightforward but we need
a bound on the value of~$D_{n, m, k}$ for the second part, which is the purpose
of Lemma~\ref{lem:D:bound}.

\begin{lemma}\label{lem:D:bound}
  For all~$n, m, k$, we have~$\displaystyle{D_{n, m, k} \leq \binom{\binom{n}{2}
  - \binom{k}{2}}{m} \cdot m!}$
\end{lemma}
\begin{proof}
  This upper bound is based on two combinatorial arguments.
  Consider a sequence of~$n$ vertices obtained by decomposing a DOAG~$D$
  with~$k$ sources and~$m$ edges.
  The first~$k$ vertices of this sequence thus correspond to the~$k$ sources
  of~$D$.

  First, the set of edges of~$D$ is a subset of size~$m$ of the set of all pairs
  of vertices that are not made of two sources.
  Note that not all such subsets form a valid DAG however.
  Hence, the number of ways to choose the~$m$ edges of~$D$ is bounded above
  by~$\binom{\binom{n}{2} - \binom{k}{2}}{m}$.
  Second, the number of ways to order the outgoing edges of all the vertices is
  bounded by~$d_1! d_2! \cdots d_n{!}$ where~$d_j$ denotes the out-degree of
  the~$i$-th vertex.
  Finally, this product is bounded by~$m{!}$, which corresponds to the case
  where all the~$d_j$ but one are equal to~$0$ and the remaining one is equal
  to~$m$.
\end{proof}
This bound on the number of DOAGs is rough but it is precise enough to get an
estimation of the bit-size of these numbers.

\begin{corollary}\label{cor:D:bitsize}
  There exists a constant~$c > 0$ such that for all~$n, m, k$ we
  have~$\log_2(D_{n, m, k}) \leq c \cdot m \cdot \log_2 n$.
\end{corollary}

\begin{proof}
  Let~$L = \binom{n}{2} - \binom{k}{2}$.
  By Lemma~\ref{lem:D:bound}, we have that:
  \begin{equation*}
    D_{n,m,k}
    \leq \frac{L (L - 1) (L - 2) \cdots (L - m + 1)}{m!} \cdot m!
    \leq L^m.
  \end{equation*}
  Hence,~$\log_2(D_{n,m,k}) \leq m \log_2(L)$.
  Moreover~$\log_2(L) \leq \log_2(n^2) = 2 \log_2(n)$, which allows to conclude.
\end{proof}

We now have enough information to prove Theorem~\ref{thm:counting:cost}.
\begin{proof}[Proof of Theorem~\ref{thm:counting:cost}.]
  Since~$D_{n, m, k} = 0$ for~$k > n$, we need to compute~$O(N^2 M)$ numbers.
  Moreover, computing each~$D_{n, m, k}$ requires to compute a sum of at
  most~$n^2$ terms, each of which is the product of a number of bit-length~$O(m
  \ln n)$ with a coefficient of the form~$C(j, p, i) = \binom{j + i}{i}
  \binom{p}{i} i{!}$ (for some~$j, p, i \leq n$) of bit-length~$O(n \ln n)$.
  Overall this accounts for~$O(N^4 M)$ multiplications of
  bit-complexity~$\mulfun{M \ln N}$ since~$m = O(n^2)$.

  There remains to measure the cost of computing the coefficients~$C(j, p, i)$.
  They can be obtained at a small amortized cost using the relation~$C(j, p, i)
  = C(j, p, i - 1) \cdot \frac{(j + i) (p - i + 1)}{i}$ (for all~$1 < i \leq p$)
  to get the value of the coefficient at~$i$ from its value at~$i - 1$ when
  summing the terms of~\ref{eq:recurrence} for increasing values of~$i$.
  Clearly, the cost of multiplying numbers of bit-length~$n \ln n$ and~$\ln n$
  is bounded by~$\mulfun{n \ln n}$ and therefore~$\mulfun{N \ln N}$.

  Combining the these two arguments, we get~$O(\mulfun{\max(N,M) \ln N})$ for
  the cost of each multiplication.
\end{proof}
