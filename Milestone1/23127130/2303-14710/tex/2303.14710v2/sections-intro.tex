\section{Introduction}\label{sec:intro}

Directed Acyclic Graphs (DAGs for short) are directed graphs in which there is
no directed path (sequence of incident edges) from a vertex to itself.
They are an omnipresent data structure in various areas of computer science and
mathematics.
In concurrency theory for instance, scheduling problems usually define a partial
order on a number of tasks, which is naturally encoded as DAG via its Hasse
diagram~\cite{CMPTVW2010,CEH2019}.
DAGs also appear as the result of the compression of some tree-like structures
such as XML documents~\cite{BLMN2015}.
In functional programming in particular, this happens at the memory layout level
of persistent tree-like values, where the term ``hash-consing'' has been coined
to refer to this compression~\cite{Goto1974}.
Computer algebra systems also make use of this idea to store their symbolic
expression~\cite{Ershov1958}.
Finally, complex histories, such as those used in version control systems (see
Git for instance~\cite[p.~17]{git2018}) or genealogy ``trees'' are DAGs as well.

% 1. Background

Two kinds of DAGs have receive a particular interest: \emph{labelled} DAGs and
\emph{unlabelled} DAGs.
The most obvious one the labelled model, in which one has a set~$V$ of
distinguishable vertices (often~$\llbracket 1; n \rrbracket$) connected by a set
of edges~$E \subseteq V \times V$.
The term \emph{labelled} is used because the vertices can be distinguished
here, they can be assigned labels.
On the other hand, unlabelled DAGs are the quotient set obtained by considering
labelled DAGs up to relabelling, that is to say up to a permutation of their
vertices (which is reflected on the edges).
These two types of objects serve a different purpose, the former represents
relations over a given set whereas the latter represents purely structural
objects.
From a combinatorial point of view, a crucial difference between the two models
is that one has to deal with symmetries when enumerating unlabelled DAGs which
makes the counting and sampling problem significantly more involved.

% 1.1 Counting

The problem of counting DAGs has been solved in early 70's by
Robinson~\cite{Robinson1970,Robinson1973,Robinson1977} and
Stanley~\cite{Stanley1973} using different approaches.
Robinson exhibits a recursive decompositions of labelled DAGs leading to a
recurrence satisfied by the numbers~$A_{n,k}$ of DAGs with~$n$ vertices
including~$k$ sources (vertices without any incoming edge).
Stanley on the other hand uses a generating function approach using identities
of the chromatic polynomial.
Robinson also solves the unlabelled case starting from the same ideas but using
Burnside's lemma and cycle index sums to account for the symmetries of these
objects.
He provides a first solution in~\cite{Robinson1970} and makes it more
computationally tractable in~\cite{Robinson1977}.
In the 90's, Gessel uses a novel approach based on so-called \emph{graphical
generating function} in~\cite{Gessel1995,Gessel1996} to take into account more
parameters and count DAGs by vertices and edges, but also sinks and sources.

% 1.2 Random sampling

From the point of view of uniform random generation, the recursive decomposition
exhibited by Robinson in~\cite{Robinson1973} is interesting as it is amenable to
\emph{the recursive method} pioneered by Nijenhuis and Wilf in~\cite{NW1978}.
This yields a polynomial time algorithm for sampling uniform DAGs with~$n$
vertices.
The analysis of this algorithm has been done in~\cite{KM2015} but it had been
acknowledged earlier in~\cite{MDB2001} although the article proposes an
alternative solution.
Both~\cite{KM2015} and~\cite{MDB2001} also offer a Markov chain approach to the
random sampling problem.
Unfortunately, to our knowledge, no efficient uniform random generator of
unlabelled has been found yet.
Moreover, unlike in the labelled case, the method derived by Robinson to exhibit
the number of unlabelled DAGs cannot be easily leveraged into a random sampler
as they make extensive use of Burnside's lemma.
Another interesting question is that of controlling the number of edges in
random those samplers.
Indeed, sampling a uniform DAG with a prescribed number of vertices and edges
cannot be achieved using the Markov chain approach as it constrains the chain
too much, and the formulas of Gessel are not amenable to this either.
In~\cite[\textsection~7]{KM2015}, the authors provide an interesting discussion
on which kind of restrictions can be made on DAGs with the Markov chain
approach.
They discuss in particular the case of bounding the number of edges and
highlight the benefits the having a precise combinatorial enumeration compared
to Markov chains.

% 2. Our contributions

% 2.1 A new model: DOAGs

In the present paper, we propose to study an alternative model of DAGs, which we
call Directed Ordered Acyclic Graphs (DOAGs), and which are enriched with
additional structure on the edges.
More precisely, a DOAG in an unlabelled DAG where (1) set of outgoing edges of
each vertex is totally ordered, (2) the sources are totally ordered as well, and
(3) graphs are constrained to have only one sink.
This \emph{local} ordering of the sources allows to capture more precisely the
structure of existing mathematical objects.
For instance, the compressed formulas and tree-like structures mentioned earlier
(see~\cite{Ershov1958,Goto1974}) indeed present with an ordering as soon as the
underlying tree representation is ordered.
This is the case for most trees used in computer science (\textit{e.g.}\
red-black trees, B-trees, etc.) and for all formulas involving non-commutative
operators.
We present here several results regarding DOAGs, as well as an extension of our
method to classical labelled DAGs.

As a first step of our analysis, we describe a recursive decomposition scheme
that allows us to study DOAGs using tools from enumerative combinatorics.
This allows us to obtain a recurrence formula for counting them, as well as a
polynomial-time uniform random sampler, based on the recursive method
from~\cite{NW1978}, giving full control over their number of vertices and edges.
Our decomposition is based on a ``vertex-by-vertex'' approach, that is we remove
one vertex at a time and we are able to describe exactly what amount of
information is necessary to reconstruct the graph.
This differs from the approach of Robinson to study DAGs, where here removes all
the sources of a DAG at once instead.
Although this is a minor difference, our approach allows us to easily account
for the number of edges of the graph, which is why our random sampler is able to
target DOAGs with a specific number of vertices.
In terms of application, this means that we are able to efficiently sample
DOAGs of low density.
A second by-product of our approach is that it makes straightforward to bound
the out-degree of each vertex, thus allowing to sample DOAGs of low degree as
well.

In order to show the applicability of our method, we devise a similar
decomposition scheme for counting labelled DAGs with only one sink.
This allows us to transfer our results on DOAGs in the context of labelled DAGs.
More precisely, we present a new recurrence formula for counting labelled DAGs
with one sink by number of vertices, edges and sources and which differ from the
formula of Gessel~\cite{Gessel1996} counting the same objects.
Our approach allows us to obtain an efficient uniform random sampler of labelled
DAGs (with one sink) with a prescribed number of vertices and edges.
Here again, in addition to giving control over the number of edges of the
produced objects, our approach can also be adapted to bound the out-degree of
their vertices.
To our knowledge, this is the first such sampler.

Finally, in a second part of our study of DOAGs, we focus on their asymptotic
behaviour and get a first result in this direction.
We consider the number~$D_n$ of DOAGs with~$n$ vertices, one source, and any
number of edges, and we manage to exhibit an asymptotic equivalent of an
uncommon kind:
\begin{equation*}
  D_n \sim c \cdot n^{-1/2} \cdot e^{n-1} \prod_{k=1}^{n-1} k!
  \quad\text{for some constant~$c > 0$.}
\end{equation*}
In the process of proving this equivalent, we state an upper bound on~$D_n$ by
exhibiting a super-set of the set of DOAGs of size~$n$, expressed in terms of
simple combinatorial objects: variations.
This upper-bound is close enough to~$D_n$ so that we can leverage it into an
efficient uniform rejection sampler of DOAGs with~$n$ vertices and any number of
edges.
Combined with an efficient anticipated rejection procedure, allowing to reject
invalid objects as soon as possible, this lead us to an optimal uniform sampler
of DOAGs of size~$n$.

\paragraph{Outline of the paper}

In Section~\ref{sec:def}, we start by introducing the class of Directed Ordered
Acyclic Graphs and their recursive enumeration and describe a recursive
decomposition scheme allowing to count them.
We quickly go over a counting algorithm implementing the recurrence.
In Section~\ref{sec:sampling:rec}, we describe and analyse an effective uniform
random sampler of DOAGs giving full control over the number of edges, vertices,
and sources based on the recursive decomposition.
Then, in Section~\ref{sec:matrix}, we present a bijection between DOAGs and
class of integer matrices.
This bijection proves to be a key element in understanding the structure of
DOAGs in the two following sections.
In Section~\ref{sec:asympt}, we present a first asymptotic result: we give an
asymptotic equivalent of the number of DOAGs of size~$n$ with any number of
sources and edges.
We also state some simple structural properties of those DOAGs.
In light of the matrix encoding and these asymptotic results, we design an
optimal uniform random sampler of DOAGs with a given number of vertices (but no
constraint on the number of edges), that is described in
Section~\ref{sec:sampling:rej}.
Finally in Section~\ref{sec:dag}, we open the way for further research
directions regarding the classical model of labelled DAGs.
We show that our approach, when applied to labelled DAGs, yields a constructive
counting formula for them, that is amenable to efficient uniform random
generation with full control on the number of edges.

An implementation of all the algorithms presented in this paper is available
at~\url{https://github.com/Kerl13/randdag}.
This paper extends~\cite{GPV2021} with a new result on the asymptotics of DOAGs,
with an optimal uniform random sampler for the case when the number of edges is
not prescribed, and covers a larger class of DOAGs and DAGs by drooping a
constraint on the number of sinks.
