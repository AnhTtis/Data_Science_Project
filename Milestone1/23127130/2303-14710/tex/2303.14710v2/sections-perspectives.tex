\section{Conclusion and perspectives}

In this paper, we have studied the new class of directed ordered acyclic graphs,
which are directed acyclic graphs endowed with an ordering of the out-edges of
each of their vertices.
We have provided a recursive decomposition of DOAGs that is amenable to the
effective random sampling of DOAGs with a prescribed number of vertices, edges
and source using the recursive method from Nijenhuis and Wilf.
Using a bijection with a class of integer matrices, we also have provided an
equivalent for the number of DOAGs with~$n$ vertices and designed an optimal
uniform random sampler for them.
We have also showed that our approach allows to approach classical labelled DAGs
and have obtained a new recurrence formula for their enumeration.
The important particularity is that this new formula is amenable to efficient
random sampling when the number of edges is prescribed, which was not the case
for previously known formulas.

\paragraph{Perspectives}

An interesting question that is left open by our work is the case of the
multi-graph variant of this model: what happens if multiple edges are allowed
between two given vertices?
This makes the analysis more challenging since their is now an infinite number
of objects with~$n$ vertices and we must thus take both vertices and edges into
consideration directly.
Estimating the number and behaviour of DOAGs as well as their multi-graph
counterpart, when both parameters~$n$ and~$m$ grow remains an open question and
will certainly yield very different results depending on how~$n$ and~$m$ grow in
relation to each other.

Another interesting question is that of the connectivity.
We do not provide a way to count connected DOAGs directly here.
However we can already prove that in the uniform model from
Section~\ref{sec:asympt}, their are connected with high probability since they
have only one source with high probability.
This implies that sampling a uniform connected DOAG with~$n$ vertices is already
possible, and efficient, by rejection.
The question of the direct enumeration is thus mostly of mathematical interest.

Finally, it is also natural to wonder whether our successful approach at
tackling the asymptotics of DOAG applies to labelled DAGs.
Of course, this asymptotics is known~\cite{BRRW1986}.
But if a proof similar to ours is feasible, then it might be possible as well to
devise an efficient, pre-computation-free, algorithm for sampling them.
We are planning to investigate this question in the near future.
