\section{Extension to labelled DAGs}\label{sec:dag}

In this section we demonstrate how our decomposition scheme can be applied to
the classical model of labelled DAGs to obtain new recurrences on known
sequences.
In the 1990s, Gessel~\cite{Gessel1996} already obtained equations allowing to
count labelled DAGs by vertices, edges, and sources (and also sinks actually)
using a generating functions approach.
These equations involve the inclusion-exclusion principle which has one
drawback: they are usually not amenable to efficient random generation.
The reason for this is that subtractions translate into rejections in the
recursive algorithm, that are here too costly to be usable.
In the present paper, we derive new recurrences with a combinatorial meaning and
that do not involve the inclusion-exclusion principle.
As a consequence, we can obtain an efficient random sampler of DAGs with full
control over the number of vertices, sources, and edges.

As in Section~\ref{sec:def}, we present here a slight generalisation of the
formula given in~\cite[\S 4]{GPV2021} allowing to capture various classes of
labelled DAGs. We omit the proofs here as they follow a straightforward
adaptation of the arguments given in~\cite{GPV2021} and the recursive method.

\subsection{Recursive decomposition}

The key idea to our decomposition is to consider labelled DAGs with a
\emph{distinguished} source (this operation is called pointing) and to decompose
them by removing this source.
This describes a bijection between source-pointed labelled DAGs and labelled
DAGs endowed with some additional structure, in the same fashion as in
Section~\ref{sec:def} of the present paper.

Let~$\mathcal P$ be any subset of~$\mathbb N$ and let~$A_{n,m,k}^{\mathcal P}$
denote the number of labelled DAGs with~$m$ edges and $n$ vertices including~$k$
sources, and in which every vertex except the (first) sink has out-degree
in~$\mathcal P$.
The number of such DAGs with a distinguished (or pointed) source is given by~$k
\cdot A_{n,m,k}^{\mathcal P}$ since any of the~$k$ sources may be distinguished.
Let~$D$ denote one such DAG and let~$v$ denote its distinguished source.
Removing the distinguished source in~$D$ and decrementing the labels of the
vertices of higher label than~$v$ by one yields a regular vertex-labelled
DAG~$D'$ with~$n-1$ vertices.
Moreover, the three pieces of information that are necessary to reconstruct the
source are the following:
\begin{enumerate}
  \item the label~$\ell$ of the source~$v$ which has been removed;
  \item the set~$S$ of sources of~$D'$ which have been uncovered by
    removing~$v$;
  \item the set~$I$ of internal (non-sources) vertices of~$D'$ that were pointed
    at by~$v$.
\end{enumerate}
The reconstruction is then straightforward:
\begin{itemize}
  \item increment all the labels that are greater or equal to~$\ell$ by one;
  \item create a new vertex labelled~$\ell$ and ``mark'' it: this is the
    distinguished source;
  \item add edges from~$\ell$ to all the vertices from~$S$ and~$I$.
\end{itemize}
This decomposition is simpler than that of DOAGs because there is no ordering to
maintain here. Hence, any subset~$S$ of the set of sources of~$D'$ is licit
here.
The triplet~$(\ell, S, I)$ is thus not constrained which leads to the simple
counting formula, given below, where~$p$ denotes the out-degree of~$v$ (and thus
the cardinality of~$S \cup I$).
\begin{equation}\label{eq:recurrence:A}
  \begin{aligned}
    A_{1, m, k}^{\mathcal P} &= \indicator{m = 0~\wedge~k = 1} \\
    k A_{n, m, k}^{\mathcal P} &=
    \begin{cases}
      n
        \sum_{p \in \mathcal P \cap \llbracket 0; n - k\rrbracket} \sum_{i = 0}^p A_{n-1, m-p, k-1+p-i}^{\mathcal P}
        \binom{n-k-p+i}{i} \binom{k-1+p-i}{p-i}
        & \text{if~$1 \leq k$} \\
      0 & \text{otherwise.}
    \end{cases}
  \end{aligned}
\end{equation}
In the last equation:
\begin{itemize}
  \item the factor~$k$ on the left counts the number of ways to choose the
    distinguished source;
  \item the factor~$n$ on the right counts the number of ways to choose the
    label of the new source;
  \item and the two binomial coefficient count the number of ways to select the
    subsets~$I$ and~$S$.
\end{itemize}

When~$\mathcal P = \mathbb N$, we recover the sequence counting all labelled
DAGs, known as~\href{https://http://oeis.org/A003024}{\texttt{A003024}} in the
OEIS and first enumerated in~\cite{Robinson1970, Stanley1973, Robinson1973}.
For~$\mathcal P = \mathbb N^\star$ and with~$k=1$, we find the number of
labelled DAGs with a single source and a single sink, known to Gessel
in~\cite{Gessel1995,Gessel1996} and stored
at~\href{https://oeis.org/A165950}{\texttt{A165950}} in the OEIS.

\subsection{Random generation}

A recursive random sampling algorithm similar to
Algorithm~\ref{algo:sample:DOAG} from Section~\ref{sec:sampling:rec} can be
obtained from formula~\eqref{eq:recurrence:A}.
The only difference in methodology from  Algorithm~\ref{algo:sample:DOAG} is
that one has to deal with the marking of the sources here and thus the division
by~$k$ at the third line of~\eqref{eq:recurrence:A}.
It can be handled as follows: at every recursive call, first generate a labelled
DAG with a distinguished source (counted by~$k \cdot A_{n, m, k}$) and then
forget which source was distinguished.
Since the recursive formula for~$k \cdot A_{n, m, k}$ has no division, the
uniform sampler of marked DAGs is obtained using the standard recursive method.
Moreover, forgetting which source was marked does not introduce bias in the
distribution since all sources have the same probability to be marked.
A uniform random sampler of labelled DAGs with~$n$ vertices,~$k$ sources,
and~$m$ edges is described in Algorithm~\ref{algo:sample:dag}.

\begin{algorithm}[htb]
  \caption{Uniform random sampler of vertex-labelled DAGs.}%
  \label{algo:sample:dag}
  \begin{algorithmic}
    \Require{Three integers~$(n, m, k)$ such that~$A_{n,m,k}^{\mathcal P} > 0$}
    \Ensure{A uniform random labelled DAG with~$n$ vertices (including $k$
    sources and one sink),~$m$ edges, and in which every vertex (except the
    first sink) has out-degree in~$\mathcal P$.}
    \Function{unifDAG${}^{\mathcal P}$}{$n, m, k$}
      \If{$n = 0$ \textbf{or} $n=1$}
        generate the (unique) labelled DAG with~$n$ vertex
      \Else{}
        \State{\textbf{pick}~$(p, i)$ with probability
          $\displaystyle \frac{A_{n - 1, m - p, k-1+p-i}^{\mathcal P} \binom{n - k - p + i}{i}
          \binom{k - 1 + p-i}{p-i}}{A_{n, m, k}^{\mathcal P}}$}
        \assign{$D'$}{\Call{unifDAG${}^{\mathcal P}$}{$n - 1, m - p, k - 1 + p-i$}}
        \assign{$I$}{a uniform subset of size~$i$ of the inner vertices of~$D'$}
        \assign{$S$}{a uniform subset of size~$(p-i)$ the sources of~$D'$}
        \assign{$\ell$}{\Call{Unif}{$\llbracket 1; n \rrbracket$}}
        \State{relabel~$D'$ by adding one to all labels~$\ell' \geq \ell$}
        \return{the DAG obtained by adding a new source to~$D'$ with
        label~$\ell$ and with an outgoing edge to every vertex of~$I \cup S$}
      \EndIf{}
    \EndFunction{}
  \end{algorithmic}
\end{algorithm}
