\section{Introduction}\label{sec:intro}

Directed Acyclic Graphs (DAGs for short) are directed graphs in which there is
no directed path (sequence of incident edges) from a vertex to itself.
They are an omnipresent data structure in various areas of computer science and
mathematics.
In concurrency theory for instance, scheduling problems usually define a partial
order on a number of tasks, which is naturally encoded as DAG via its Hasse
diagram~\cite{CMPTVW2010,CEH2019}: each task corresponds to a vertex in the
graph and task dependencies are materialised by directed edges.
Scheduling then corresponds to finding a good topological order on this graph.
Natural question such as counting or sampling the schedulings of a program are
studied in this context for the purpose of random testing~\cite{GS2005}.
DAGs also appear as the result of the compression of some tree-like structures
such as XML documents~\cite{BLMN2015}.
In functional programming in particular, this happens at the memory layout level
of persistent tree-like values, where the term ``hash-consing'' has been coined
to refer to this compression~\cite{Goto1974}.
Computer algebra systems also make use of this idea to store their symbolic
expression~\cite{Ershov1958}.
These tree use cases leverage the fact that actual memory gains can be obtained
by compacting trees, which has been quantified in~\cite{FSS1990} and has
motivated the study of compacted trees in the recent years~\cite{EFW2021,
GW2024}.
Finally, complex histories, such as those used in version control systems (see
Git for instance~\cite[p.~17]{git2018}) or genealogy ``trees'' are DAGs as well.

Most of the applications presented here actually require to add some more
structure on the space of DAGs in order to faithfully model the objects at play,
which is the main motivation of the present article.
We first give some background on the combinatorics of DAGs and then expand on
our contributions.

\subsection{Background on DAGs}

Two different models of DAGs have received a particular interest:
\emph{labelled} DAGs and \emph{unlabelled} DAGs.
The most obvious one is the labelled model, in which one has a set~$V$ of
vertices (often~$\llbracket 1; n \rrbracket$) connected by a set of edges~$E
\subseteq V \times V$.
The term \emph{labelled} is used because the vertices can be distinguished
here, they can be assigned labels.
On the other hand, unlabelled DAGs are the quotient set obtained by considering
labelled DAGs up to relabelling, that is to say up to a permutation of their
vertices (which is reflected on the edges).
These two types of objects serve a different purpose, the former represents
relations over a given set whereas the latter represents purely structural
objects.
From a combinatorial point of view, a crucial difference between the two models
is that one has to deal with symmetries when enumerating unlabelled DAGs which
makes the counting and sampling problem significantly more involved.

\paragraph{Counting}

The problem of counting DAGs has been solved in early 70's by Robinson and
Stanley using different approaches.
In~\cite{Robinson1970}, Robinson exhibits a recursive decompositions of labelled
DAGs leading to a recurrence satisfied by the numbers~$A_{n,k}$ of DAGs with~$n$
vertices including~$k$ sources (vertices without any incoming edge).
He later reformulates those results in terms of a new kind of generating
functions, now called \emph{graphical generating functions}
in~\cite{Robinson1973}, and also obtains the asymptotic number of size~$n$ DAGs.
Around the same time, Stanley also used a generating function approach
in~\cite{Stanley1973} obtained the same results by deriving identities of the
chromatic polynomial.
Robinson also solves the unlabelled case starting from the same ideas but using
Burnside's lemma and cycle index sums to account for the symmetries of these
objects.
He provides a first solution in~\cite{Robinson1970} and makes it more
computationally tractable in~\cite{Robinson1977}.
In the 90's, Gessel generalised those results, also using the graphical
generating function framework in~\cite{Gessel1995,Gessel1996} to take into
account more parameters and count DAGs by vertices and edges, but also sinks and
sources.

\paragraph{Random sampling}

From the point of view of uniform random generation, the recursive decomposition
exhibited by Robinson in~\cite{Robinson1973} is interesting as it is amenable to
\emph{the recursive method} pioneered by Nijenhuis and Wilf in~\cite{NW1978}.
This yields a polynomial time algorithm for sampling uniform DAGs with~$n$
vertices.
The analysis of this algorithm has been done in~\cite{KM2015} but it had been
acknowledged earlier in~\cite{MDB2001} although the article proposes an
alternative solution.
Both~\cite{KM2015} and~\cite{MDB2001} also offer a Markov chain approach to the
random sampling problem and an interesting discussion on the pros and cons of
both approaches is given in~\cite{KM2015}.
Remote from the field of combinatorics, the random generation of DAGs is also an
active topic in the area of applied statistics and Bayesian inference.
In this context, DAGs encode a relevant structure in a collection of random
variables and the problem of interest is to sample DAGs from a particular
distribution related to those random variables.
To this end, authors resort both to Monte Carlo Markov Chains
approaches~\cite{KSM2022, KM2017} and methods similar to what is referred to as
the \emph{recursive method} in combinatorics~\cite{TVK2020}.
An important point in~\cite{KM2017} is better performance can be achieved by
using a combination of both approaches, in particular by exploiting the
combinatorial properties of DAGs.
Notable is that sampling from the uniform distribution is tackled as a
particular case in~\cite{TVK2020} and solved with the asymptotically
optimal~$O(n^2)$ complexity at the expense of a~$O(n^3)$ pre-processing step.

Unfortunately, to our knowledge, no efficient uniform random generator of
unlabelled has been found yet.
Moreover, unlike in the labelled case, the method derived by Robinson to exhibit
the number of unlabelled DAGs cannot be easily leveraged into a random sampler
as they make extensive use of Burnside's lemma.

Another interesting question is that of controlling the number of edges in
those random samplers.
Indeed, sampling a uniform DAG with a prescribed number of vertices and edges
cannot be achieved using the Markov chain approach as it constrains the chain
too much, and the formulas of Gessel are not amenable to this either.
In~\cite[\textsection~7]{KM2015}, the authors provide an interesting discussion
on which kind of restrictions can be made on DAGs with the Markov chain
approach.
They address in particular the case of bounding the number of edges and
highlight that the Monte Carlo Markov Chain approach fails when the desired
number of edges is too low, thus advocating for having precise combinatorial
enumerations.

\subsection{Contributions}

In the present paper, we propose to study an alternative model of DAGs, which we
call Directed Ordered Acyclic Graphs (DOAGs), and which are enriched with
additional structure on the edges.
More precisely, a DOAG in an unlabelled DAG where (1) set of outgoing edges of
each vertex is totally ordered and (2) the sources are totally ordered as well.
This \emph{local} ordering of the outgoing edges allows to capture more
precisely the structure of existing mathematical objects.
For instance, the compressed formulas and tree-like structures mentioned earlier
(see~\cite{Ershov1958,Goto1974}) indeed present with an ordering as soon as the
underlying tree representation is ordered.
This is the case for most trees used in computer science (\textit{e.g.}\
red-black trees, B-trees, etc.) and for all formulas involving non-commutative
operators.
The model we introduce thus allows for a more faithful modelling of a wide range
of objects.
We present here several results regarding DOAGs, as well as an extension of our
method to classical labelled DAGs.

As a first step of our analysis, we describe a recursive decomposition scheme
that allows us to study DOAGs using tools from enumerative combinatorics.
This allows us to obtain a recurrence formula for counting them, as well as a
polynomial-time uniform random sampler, based on the recursive method
from~\cite{NW1978}, giving full control over their number of vertices and edges.
Our decomposition is based on a ``vertex-by-vertex'' approach, that is we remove
one vertex at a time and we are able to describe exactly what amount of
information is necessary to reconstruct the graph.
This differs from the approach of Robinson to study DAGs, where all the sources
of a DAG are removed at once instead.
Although this is a minor difference, our approach allows us to easily account
for the number of edges of the graph, which is why our random sampler is able to
target DOAGs with a specific number of edges.
In terms of application, this means that we are able to efficiently sample
DOAGs of low density.
A second by-product of our approach is that it makes straightforward to bound
the out-degree of each vertex, thus allowing to sample DOAGs of low degree as
well.

In order to show the applicability of our method, we devise a similar
decomposition scheme for counting labelled DAGs with any number of vertices,
edges, and sources.
This allows us to transfer our results on DOAGs in the context of labelled DAGs.
Our new recurrence differs from the formula of Gessel~\cite{Gessel1996} in that
it does not resort to the inclusion-exclusion principle.
Our approach allows us to obtain an efficient uniform random sampler of labelled
DAGs with a prescribed number of vertices, edges, and sources.
Here again, in addition to giving control over the number of edges of the
produced objects, our approach can also be adapted to bound the out-degree of
their vertices.
To our knowledge, this is the first such sampler.

Finally, in a second part of our study of DOAGs, we focus on their asymptotic
behaviour and get a first result in this direction.
We consider the number~$D_n$ of DOAGs with~$n$ vertices, one source, and any
number of edges, and we manage to exhibit an asymptotic equivalent of an
uncommon kind:
\begin{equation*}
  D_n \sim c \cdot n^{-1/2} \cdot e^{n-1} \prod_{k=1}^{n-1} k!
  \quad\text{for some constant~$c > 0$.}
\end{equation*}
In the process of proving this equivalent, we state an upper bound on~$D_n$ by
exhibiting a super-set of the set of DOAGs of size~$n$, expressed in terms of
simple combinatorial objects: variations.
This upper-bound is close enough to~$D_n$ so that we can leverage it into an
efficient uniform rejection sampler of DOAGs with~$n$ vertices and any number of
edges.
Combined with an efficient anticipated rejection procedure, allowing to reject
invalid objects as soon as possible, this lead us to an asymptotically optimal
uniform sampler of DOAGs of size~$n$.

In terms of applications, our random generation algorithms enable
to experiment with the properties of the objects they model and with the average
complexity of algorithms operating on them.
A similar approach is for instance taken in~\cite{CP2024} where samplers for a
realistic class of Git graphs are developed in order to tackle the average
complexity of a new algorithm introduced in~\cite{Lecoq2024, CDL2022}.
Random testing a also an important application of random sampling, especially as
a building block for property-based testing, a now well-established framework
pioneered Claessen and Hughes in~\cite{CH2000}.

\bigskip

This paper extends an earlier article~\cite{GPV2021} with new results on the
asymptotics of DOAGs, with an optimal uniform random sampler for the case when
the number of edges is not prescribed, and covers a larger class of DOAGs and
DAGs by drooping a constraint on the number of sinks.
For the sake of completeness, the most important results and ideas
from~\cite{GPV2021} will be recalled in the present paper, but the reader will
have to refer the earlier article to get the full proof and algorithmic details.

\subsection{Outline of the paper}

In Section~\ref{sec:def}, we start by introducing the class of Directed Ordered
Acyclic Graphs and their recursive enumeration and describe a recursive
decomposition scheme allowing to count them.
In Section~\ref{sec:sampling:rec}, we quickly go over earlier results regarding
the random generation of DOAGs with a prescribed number of vertices, edges, and
sources.
The presentation given in this paper slightly generalises over the algorithm
given in~\cite{GPV2021} but the ideas and proofs remain unchanged.
Section~\ref{sec:dag} shows that our approach applies to labelled graphs as well
and opens the way for further research regarding this class.
We show that our method, when applied to labelled DAGs, yields a constructive
counting formula for them, that is amenable to efficient uniform random
generation with full control on the number of edges.
Then, in Section~\ref{sec:matrix}, we present a bijection between DOAGs and
class of integer matrices.
This bijection is the key result of this paper as it allows to understand the
structure of DOAGs in detail, and to obtain both asymptotic and algorithm
results in the following sections.
In Section~\ref{sec:asympt}, we present a first asymptotic result: we give an
asymptotic equivalent of the number of DOAGs of size~$n$ with any number of
sources and edges.
We also state some simple structural properties of those DOAGs.
In light of the matrix encoding and these asymptotic results, we design an
optimal uniform random sampler of DOAGs with a given number of vertices (but no
constraint on the number of edges), that is described in
Section~\ref{sec:sampling:rej}.

An implementation of all the algorithms presented in this paper is available
at~\url{https://github.com/Kerl13/randdag}.
