\section{Definition and recursive decomposition}\label{sec:def}

In this section, we recall a model of directed acyclic graphs called ``Directed
Ordered Acyclic Graphs'' (or DOAGs) that we introduced in~\cite{GPV2021}. It is
similar to the classical model of unlabelled DAGs but where, in addition, we
have a total order on the outgoing edges of each vertex.
The presentation we opted for here slightly differs from that of~\cite{GPV2021}
but essentially defines the same objects, the only difference being that we now
allow several sinks for the sake of generality.

\begin{definition}[Directed Ordered Graph]\label{def:dog}
  A directed ordered graph (or DOG for short) is a triple~$(V, E,
  {(\prec_v)}_{v\,\in V \cup \{\top\}})$ where:
  \begin{itemize}
    \item $V$ is a finite set of vertices;
    \item $E \subset V \times V$ is a set of edges;
    \item for all~$v \in V$,~$\prec_v$ is a total order over the set of
      \emph{outgoing} edges of~$v$;
    \item and~$\prec_\top$ is a total order over the set of \emph{sources}
      of the graph, that is the vertices without any incoming edge.
  \end{itemize}
  Two such graphs are considered to be \emph{equal} if there exists a bijection
  between their respective sets of vertices that preserves both the edges and
  the order relations~$\prec_v$ and~$\prec_\top$.
\end{definition}

\begin{definition}[Directed Ordered Acyclic Graph]\label{def:doag}
  A directed ordered acyclic graph (or DOAG for short) is a directed ordered
  graph~$(V, E, {(\prec_v)}_{v\,\in V \cup \{\top\}})$ such that~$(V, E)$,
  seen as a directed graph, is acyclic.
\end{definition}

We study this class as a whole, however, some sub-classes are also of special
interest, in particular for the purpose of modelling compacted data structures.
Tree structures representing real data, such as XML documents for
instance~\cite{BLMN2015}, are rooted trees.
When these trees are compacted, the presence of a root translates into a unique
source in the resulting DOAG\@.
Similarly, DOAGs with a single sink will arise naturally when compacting trees
which bear a single type of leaves.
In particular the model of compacted binary trees, which can also be seen as a
class of cycle-free binary automata, has been shown have unusual combinatorial
properties in~\cite{EFW2020,EFW2021} and corresponds to a restriction of our
model with only binary nodes (and one sink).
For these reasons, we will also discuss how to approach the sub-classes of DOAGs
with a single source and/or a single sink in this document.

In order to illustrate the definition, the first line of
Figure~\ref{fig:firstdoags} depicts all the DOAGs with at most~$3$ vertices and
the second line shows all the DOAGs with exactly~$4$ vertices and~$3$ edges.
There are 17 of them while there are~$95$ DOAGs with~$4$ vertices in total.

\begin{figure}[htb]
  \centering
  \includegraphics[scale=1]{images-smallcases}
  \caption{All DOAGs with respectly $1$ vertex, $2$ vertices, $3$ vertices,
    and simultaneously~$4$ vertices and~$3$ edges.
    All edges are implicitly oriented from top to bottom, the blue labels and
    arrows represent the sources and out-edges orderings (always from left to
    right).
  }\label{fig:firstdoags}
\end{figure}

\subsection{Recursive decomposition}

We describe a canonical way to recursively decompose a DOAG into smaller
structures.
The idea is to remove vertices one by one in a deterministic order, starting
from the smallest source (with respect to their ordering~$\prec_\top$).
Formally, we define a decomposition step as a bijection between the set of DOAGs
with at least two vertices and the set of DOAGs given with some extra
information.

Let~$D$ be a DOAG with at least~$2$ vertices and consider the new graph~$D'$
obtained from~$D$ by removing its \emph{smallest} source~$v$ and its outgoing
edges.
We need to specify the ordering of the sources of~$D'$.
We consider the ordering where the \emph{new} sources of~$D'$ (those
that have been uncovered by removing~$v$) are considered to be in the \emph{same
order} (with respect to each other) as they appear as children of~$v$ and all
\emph{larger} than the other sources.
The additional information necessary to reconstruct~$D$ from~$D'$ is the
following:
\begin{enumerate}
  \item the number~$s \geq 0$ of sources of~$D'$ which have been uncovered by
    removing~$v$;
  \item the (possibly empty) set~$I$ of internal (non-sources) vertices of~$D'$
    such that there was an edge in~$D$ from~$v$ to them;
  \item the function $f:I \to \llbracket 1; s + |I| \rrbracket$ identifying the
    positions, in the list of outgoing edges of~$v$, of the edges pointing to
    an element of~$I$.
\end{enumerate}
More formally, this decomposition describes a bijection between DOAGs with at
least~$2$ vertices and quadruples of the form~$(D', s, I, f)$ where:
\begin{itemize}
  \item $D'$ is a DOAG (obtained by removing~$v$ from~$D$);
  \item $I$ is any subset of the internal vertices of~$D'$ (children of~$v$ in~$D$);
  \item $s$ is any integer between $0$ and the number of sources of~$D'$;
  \item and~$f: I \to \llbracket 1; s + |I| \rrbracket$ is an injective
    function (mapping the vertices of~$I$ to their positions in the list of
    children of~$v$ in~$D$).
\end{itemize}
In order to prove that this is indeed an bijection, we consider the inverse
transformation below.
Start with a quadruple~$(D', s, I, f)$ as described above.
Add a new source~$v$ in~$D'$ with~$s + |I|$ outgoing edges such that the~$i$-th
of these edges is connected to~$f^{-1}(i)$ when~$i \in f(I)$ and is connected to
one of the~$s$ largest sources of~$D'$ otherwise.
The~$s$ largest sources of~$D'$ must be connected to the new source exactly once
and in the same order as they appear in the list of sources of~$D'$.
The resulting graph is a DOAG and it is easy to check that this mapping and the
decomposition are inverses of each other.

Note that the order in which the vertices are removed when iterating this
process corresponds to a variant of the BFS algorithm where only sources are
eligible to be picked next in the search, and their are picked in the order
described above.
Figure~\ref{fig:decomposition} illustrates this decomposition by applying the
first two steps on a large example DOAG\@.


\begin{figure}[htb]
  \centering
  \includegraphics[scale=.9]{images-decomposition}
  \caption{The two first steps of the recursive decomposition of a DOAG by
  removing sources one by one in a breadth first search (BFS) fashion.
  The edges are implicitly oriented from top to bottom and the order of the
  outgoing edges of each vertex is indicated by the thinner blue arrows (always
  from left to right here).
  The integer labels at each stage indicate the ordering of the sources.
  The big disk, square, and triangle are only here to highlight particular
  vertices involved with the functions~$f$ in the decomposition.}%
  \label{fig:decomposition}
\end{figure}

This decomposition can be used to establish a recursive formula for counting
DOAGs, which is given below.
Let~$D_{n,m,k}$ denote the number of DOAGs with~$n$ vertices,~$m$ edges and~$k$
sources, then we have:
\begin{equation}\label{eq:recurrence}
  \begin{aligned}
    D_{1, m, k} &= \indicator{m=0 \, \land \, k=1} \\
    D_{n, m, k} &=
    \begin{cases}
      0 & \text{when}~k \leq 0 \\
      \sum_{p=0}^{n - k} \sum_{i=0}^p D_{n-1, m-p, k-1+p-i} \binom{n-k-p+i}{i}
      \binom{p}{i} i! & \text{otherwise,}
    \end{cases}
  \end{aligned}
\end{equation}
where~$p = s + i$ corresponds the out-degree of the smallest source.
The term~$\binom{n-k-p+i}{i} = \binom{n-k-s}{i}$ accounts for the choice of the
set~$I$ and the term~$\binom{p}{i} i{!}$ accounts for the number of injective
functions~$f: I \to \llbracket 1; p \rrbracket$.
The upper bound on~$p$ in the sum is justified by the fact that the out-degree
of any vertex can be at most the number of non-sources in the graph, that
is~$(n-k)$.

The decomposition scheme presented here differs from the approach described by
Robinson in~\cite{Robinson1970} as it operates on only one source at a time.
It is also reminiscent of the peeling processes used in map enumeration where
maps are decomposed one face at a time, see for instance~\cite{KMSW2019}.
However, the absence of ordering amongst the incoming edges of each vertex in
our setup renders those approaches inapplicable as is.

\subsubsection{Special sub-classes based on out-degree constraints}%
\label{sec:subclasses}

Since~$p = i + s$ is the out-degree of the removed source in the above
summation, it is easy to adapt this sequence for counting DOAGs with
constraints on the out-degree of the vertices.
For instance, DOAGs with only one sink are obtained by ensuring that every
vertex has out-degree at least one.
In other words, let the summation start at~$p=1$.
Note that restricting DOAGs to have only one single sink or one single source
ensures that they remain connected, however not all connected DOAGs are obtained
this way.
As another example, DOAGs with out-degree bounded by some constant~$d$ are
obtained by letting~$p$ range from~$0$ to~$\min(n-k, d)$.

The general principle is that the DOAGs whose vertices' out-degrees are
constrained to belong to a given set~$\mathcal P$, are enumerated by the
following generalised recurrence.
\begin{equation}\label{eq:recurrence:general}
  \begin{aligned}
    D^{\mathcal P}_{1, m, k} &= \indicator{m=0 \, \land \, k=1} \\
    D^{\mathcal P}_{n, m, k} &=
    \begin{cases}
      0 & \text{when}~k \leq 0 \\
      \sum_{p \in \mathcal P} \sum_{i=0}^p D^{\mathcal P}_{n-1, m-p, k-1+p-i}
      \binom{n-k-p+i}{i} \binom{p}{i} i! &  \text{otherwise,}
    \end{cases}
  \end{aligned}
\end{equation}

The first values of the sequence~$D_{n,m} = \sum_k D_{n,m,k}$ counting DOAGs by
number of vertices and edges only are given in Table~\ref{table:firstterms}.
Table~\ref{table:others} gives the first values of~$D^{\mathcal P}_n =
\sum_{m,k} D^{\mathcal P}_{n,m,k}$ for some relevant choices of~$\mathcal P$.
None of these sequences seem to appear in the online encyclopedia of integer
sequences (OEIS%
\footnote{\url{https://oeis.org/}}%
) yet.

\begin{table}[htb]
  \centering
  \caption{Number of DOAGs with~$n$ vertices and~$m$ edges for small values
  of~$n$ and~$m$.}\label{table:firstterms}

  \footnotesize
  \begin{tabular}{llp{37em}}
    \toprule
    $n$ & $D_n$ & $D_{n,m} = \sum_k D_{n,m,k}$ for~$m = 0, 1, 2, 3, \ldots$ \\
    \midrule
    $1$ & $1$ & $1$ \\
    $2$ & $2$ & $1, 1$ \\
    $3$ & $8$ & $1, 2, 3, 2$ \\
    $4$ & $95$ & $1, 3, 8, 17, 27, 27, 12$ \\
    $5$ & $4858$ & $1, 4, 15, 48, 139, 349, 718, 1136, 1272, 888, 288$ \\
    $6$ & $1336729$ & $1$, $5$, $24$, $100$, $391$, $1434$, $4868$, $14940$,
                      $40261$, $92493$, $175738$, $266898$, $310096$, $258120$,
                      $136800$, $34560$
  \end{tabular}
\end{table}

\begin{table}[htb]
  \centering
  \caption{Number of DOAGs with~$n$ vertices and a constrained set of allowed
  degrees.}\label{table:others}

  \footnotesize
  \begin{tabular}{p{.19\linewidth}p{.75\linewidth}}
    \toprule
    Restrictions & sequence \\
    \midrule
%
    $\mathcal P = \mathbb N$ \newline (all DOAGs) &
    $1$, $2$, $8$, $95$, $4858$, $1336729$, $2307648716$, $28633470321822$,
    $2891082832793961795$, $2658573971407114263085356$,
    $24663703371794815015576773905384$, $\ldots$ \\[.5em]
%
    $\mathcal P = \mathbb N$, $k=1$ \newline (1 source) &
    $1$, $1$, $4$, $57$, $3399$, $1026944$, $1875577035$, $24136664716539$,
    $2499751751065862022$, $2342183655157963146881571$,
    $22043872387559770578846044961204$, $\ldots$ \\[.5em]
%
    $\mathcal P = \mathbb N^\star$, $k=1$ \newline (1 source, 1 sink) &
    $1$, $1$, $3$, $37$, $2103$, $627460$, $1142948173$, $14701782996075$,
    $1522511169925136833$, $1426529804350999351686869$,
    $13426022673540053054145359653988$, $\ldots$ \\[.5em]
%
    $\mathcal P=\{0, 1, 2\}$, $k=1$ \newline (unary-binary) &
    $1$, $1$, $4$, $23$, $191$, $2106$, $29294$, $495475$, $9915483$,
    $229898277$, $6074257926$, $180460867600$, $5962588299084$, $\ldots$
  \end{tabular}
\end{table}
