\section{Conclusion and perspectives}

In this paper, we have studied the new class of directed ordered acyclic graphs,
which are directed acyclic graphs endowed with an ordering of the out-edges of
each of their vertices.
We have provided a recursive decomposition of DOAGs that is amenable to the
effective random sampling of DOAGs with a prescribed number of vertices, edges
and source using the recursive method from Nijenhuis and Wilf.
Using a bijection with a class of integer matrices, we also have provided an
equivalent for the number of DOAGs with~$n$ vertices and designed a uniform
random sampler for DOAGs with~$n$ vertices and any number of edges.
This second sampler is asymptotically optimal, both in terms of memory accesses
and random bits consumption.

We have also showed that our approach allows to approach classical labelled DAGs
and have obtained a new recurrence formula for their enumeration.
The important particularity of this new formula is that it is amenable to
effective random sampling when the number of edges is prescribed, which was not
the case for previously known formulas.

\subsection*{Perspectives}

\paragraph{On DOAGs}

So far, we have only approached DOAGs via enumerative tools and
\textit{ad.\ hoc.}\ asymptotic techniques. A common and powerful tool in
combinatorics is the use of generating function to tackle not only asymptotic
estimation, but also the convergence in distribution of parameters, such as the
number of edges.
The book~\cite{FS2009} is a reference on this topic.
Classical approaches using ordinary, exponential, or even graphical
(\cite{PD2019}) generating functions fail in our context due to the super
factorial behaviour of the number of DOAGs.
It remains an open question whether it is possible to design a generating
function approach to our objects, which could help obtaining finer estimates,
not only over~$D_n$, but also over the low of the number of edges.

\paragraph{Multi-graph variant}

An interesting question that is left open by our work is the case of the
multi-graph variant of this model: what happens if multiple edges are allowed
between two given vertices?
This makes the analysis more challenging since there is now an infinite number
of objects with~$n$ vertices.
We thus must change our point of view and take the number~$m$ of edges into
account in addition to, or instead of, the number of vertices.
Estimating the number and behaviour of DOAGs as well as their multi-graph
counterpart, when both parameters~$n$ and~$m$ grow remains an open question and
will certainly yield very different results depending on how~$n$ and~$m$ grow in
relation to each other.
We argue that the model of multi-edge DOAGs is natural, maybe even more so than
that of DOAGs, since they encode (partially) compacted plane trees.
Quantitative aspects of tree compaction, in particular the typical compression
rate, has been studied in the past~\cite{FSS1990} in a general setting.
However, the dual point of view that consists in studying already compacted
structures directly is a more recent topic, see~\cite{EFW2021} and~\cite{GW2024}
for instance.
The class of multi-edge DOAGs generalises the classes studied in those two
papers.
Moreover, being able to sample them efficiently would give a tool to reach every
possible case (including those with double edges) when testing programs
manipulating compacted trees (such as compilers) via random generation.

Another interesting question is that of the connectivity.
We do not provide a way to count connected DOAGs directly here.
However we have already proved that, in the uniform model, they are connected
with high probability since they have only one source with high probability.
Moreover, since~$D_n$ grows extremely fast, we can also foresee that a uniform
DOAG of size~$n$ with two connected components will typically have one tiny
component of size~$1$ and a big component of size~$n-1$, and that the
asymptotic estimations of such graphs is straightforward.
This implies that sampling a uniform connected DOAG with~$n$ vertices is already
possible, and efficient, by rejection and the question of their direct
enumeration is thus mostly of mathematical interest.

\paragraph{Classical labelled DAGs}

Finally, it is also natural to wonder whether our successful approach at
efficiently sampling DOAG applies to labelled DAGs.
Of course, the asymptotics of DAGs is known~\cite{Robinson1973}.
But if a matrix encoding similar to ours is feasible, that is an encoding whose
combinatorial properties are understood well enough to avoid introduction any
bias, then it might be possible to devise an efficient, pre-computation-free,
uniform sampler for DAGs.

A starting point in this direction is the fact that uniform labelled DAGs have,
in average,~$\frac{n^2}{4}$ edges.
In terms of (upper triangular) adjacency matrices, this means that about half
the cells in the upper part of the matrix are non-zero.
This is, in a sense, much less dense than for DOAGs.
However, this is still dense enough in the sense that the DAG analogue of the
red thick path in our figures (described by the sequence~$(b_j)_{1 \leq j \leq
n}$ in Sections~\ref{sec:matrix} and~\ref{sec:sampling:rej}) can be expected to
stay close to the diagonal too.
As a consequence, the approach proposed in Section~\ref{sec:sampling:rej} for
the random generation of DOAGs is still applicable, provided we have an
efficient way to sample those paths.
Indeed, our fast-rejection procedure in Algorithm~\ref{algo:rej:opt} can be seen
as the combination of two algorithms:
\begin{enumerate}
  \item an algorithm to sample the path~$(b_j)_{1 \leq j \leq n}$ under the
    distribution induced by DOAGs;
  \item and the filling of the remaining cells of the matrix by completing the
    random variation in reach row.
\end{enumerate}
In order to design a similar approach for DAGs, we need a way to sample
the~$(b_j)_{1 \leq j \leq n}$ paths (induced by the uniform distribution on
DAGs) and the second step of the algorithm would be to fill the rest of the
matrix with Bernoulli random variables of parameter~$\frac 1 2$.
Our recent ongoing work on this topic suggests that those paths can indeed be
sampled efficiently, which will be investigated further in the near future.

Notable is that the approach presented in~\cite{KM2015} can also be seen as a
way to work with matrices while maintaining uniformity by using a combinatorial
encoding using ordered integer partitions.
A caveat however is that their approach still requires a costly pre-processing,
which we seek to avoid using a rejection-based approach.
