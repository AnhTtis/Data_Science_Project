\documentclass[%
%reprint,
%superscriptaddress,
%groupedaddress,
%unsortedaddress,
%runinaddress,
%frontmatterverbose,
preprint,
%preprintnumbers,
%nofootinbib,
%nobibnotes,
%bibnotes,
amsmath,amssymb,
aps,
%pra,
%prb,
%rmp,
%prstab,
%prstper,
%floatfix,
]{revtex4-2}
\usepackage{graphicx}
\usepackage{bm}
\usepackage[hidelinks]{hyperref}
\usepackage{amsmath}
\usepackage{xcolor}

\begin{document}

\title{Ultrafast control over chiral sum-frequency generation}

\author{Josh Vogwell$^{1}$}
\author{Laura Rego$^{1,2}$}
\author{Olga Smirnova$^{3,4}$}
\author{David Ayuso$^{1,3}$}
\email{d.ayuso@imperial.ac.uk}

\affiliation{$^1$Department of Physics, Imperial College London, SW7 2AZ London, United Kingdom \\
$^2$University of Salamanca, 37008 Salamanca, Spain \\
$^3$Max-Born-Institut, Max-Born-Str. 2A, 12489 Berlin, Germany \\
$^4$Technische Universit\"at Berlin, 10623 Berlin, Germany}

\begin{abstract}
We introduce an ultrafast and all-optical approach for efficient chiral recognition which relies on the interference between two low-order nonlinear processes which are ubiquitous in nonlinear optics: sum-frequency generation (SFG) and third-harmonic generation.
In contrast to traditional SFG, our approach encodes the medium's handedness in the \emph{intensity} of the emitted harmonic signal, rather than in its \emph{phase}, and it enables full control over the enantio-sensitive response.
We show how, by sculpting the sub-optical-cycle oscillations of the driving laser field in time, we can force a pre-selected molecular enantiomer to emit bright harmonic light while its mirror twin remains dark, thus reaching the ultimate efficiency limit of chiral sensitivity.
Our work paves the way for ultrafast and highly efficient imaging and control of the chiral electronic clouds of chiral molecules using lasers with moderate intensities, in all states of matter: from gases to liquids to solids, with molecular specificity and on ultrafast timescales.
\end{abstract}

\maketitle

Chirality is a universal type of asymmetry that naturally arises in molecules, optical fields, viruses, or even galaxies.
Just like a chiral glove would either fit our left or right hand, but not both, the two non-superimposable mirror-reflected versions of a chiral molecule (enantiomers) can behave very differently when they interact with another chiral entity, e.g. another chiral molecule.
It is therefore unsurprising that methods for detecting, quantifying and manipulating molecular chirality are of great importance and interest, particularly in biochemical and pharmaceutical contexts.

Traditional chiroptical methods \cite{Berova2007,Nafie1976,Schellman1975,Polavarapu1998}
rely on weak linear effects which arise beyond the electric-dipole approximation, posing important limitations for ultrafast and high-resolution spectroscopy.
This challenge \cite{Ayuso2022PCCP_persp} can be addressed by developing and applying approaches which rely exclusively on the molecular response to the local polarisation of the electric-field vector of light \cite{Ayuso2022PCCP_persp,Ordonez2018PRA,Ritchie1976PRA,Powis2000JCP,Bowering2001PRL,Garcia2003JCP,Garcia2013NatComm,Janssen2014PCCP,Lux2012Angew,Lehmann2013JCP,Lux2015ChemPhysChem,Kastner2016CPC,Comby2016JPCL,Beaulieu2016FD,Comby2018NatComm,Beaulieu2018NatPhys,Demekhin2018PRL,Goetz2019PRL,Rozen2019PRX,Ordonez2022PCCP,Yachmenev2019PRL,Milner2019PRL,Patterson2013,Eibenberger2017PRL,Shubert2016,Perez2017,Fischer2000PRL,Belkin2000PRL,Belkin2001PRL,Fischer2002CPL,Fischer2003PRL,Fischer2005,Okuno2018JCP,Neufeld2019PRX,Ayuso2019NatPhot,Ayuso2021NatComm,Ayuso2021Optica,Ayuso2022PCCP,Ayuso2022OptExp,Khokhlova2022}.
One strategy is to record the forward-backward asymmetry in the photo-electron angular distributions upon ionisation with circularly polarised light \cite{Ritchie1976PRA,Powis2000JCP,Bowering2001PRL,Garcia2003JCP,Garcia2013NatComm,Janssen2014PCCP,Lux2012Angew,Lehmann2013JCP,Lux2015ChemPhysChem,Kastner2016CPC,Comby2016JPCL,Beaulieu2016FD,Comby2018NatComm,Beaulieu2018NatPhys}.
This method, originally proposed \cite{Ritchie1976PRA} and implemented \cite{Powis2000JCP,Bowering2001PRL,Garcia2003JCP,Garcia2013NatComm,Janssen2014PCCP} in the \emph{linear} regime, has been extended to the \emph{nonlinear} \cite{Janssen2014PCCP,Lux2012Angew,Lehmann2013JCP,Lux2015ChemPhysChem,Kastner2016CPC,Comby2016JPCL,Beaulieu2016FD,Comby2018NatComm,Beaulieu2018NatPhys,Demekhin2018PRL,Goetz2019PRL,Rozen2019PRX,Ordonez2022PCCP} (multi-photon) regime, where the use of tailored light enables unique opportunities for control \cite{Demekhin2018PRL,Goetz2019PRL,Rozen2019PRX,Ordonez2022PCCP}.

The nonlinear regime of purely electric-dipole light-matter interactions also enables efficient all-optical chiral spectroscopy \cite{Fischer2000PRL,Belkin2000PRL,Belkin2001PRL,Fischer2002CPL,Fischer2003PRL,Fischer2005,Okuno2018JCP,Patterson2013,Eibenberger2017PRL,Shubert2016,Perez2017,Neufeld2019PRX,Ayuso2019NatPhot,Ayuso2021NatComm,Ayuso2021Optica,Ayuso2022PCCP,Ayuso2022OptExp,Khokhlova2022}.
Chiral sum-frequency generation \cite{Fischer2000PRL,Belkin2000PRL,Belkin2001PRL,Fischer2002CPL,Fischer2003PRL,Fischer2005,Okuno2018JCP} (SFG) is a well established method for chiral recognition.
It requires two incident beams with frequencies $\omega_1\neq\omega_2$ propagating in different directions, with wave vectors $\hat{\textbf{k}}_1$ and $\hat{\textbf{k}}_2$, and polarisation $\hat{\textbf{e}}_1$ and $\hat{\textbf{e}}_2$. %, see Fig. \ref{fig_SFG}.
The incident fields drive a second-order response in the medium at frequency $\omega_3=\omega_1+\omega_2$ with polarisation $\hat{\textbf{e}}_3=\hat{\textbf{e}}_1\times\hat{\textbf{e}}_2$, which leads to emission of light at this frequency in the direction of $\hat{\textbf{k}}_3=\hat{\textbf{k}}_1+\hat{\textbf{k}}_2$.
SFG is strictly forbidden in the bulk of isotropic media, such as randomly oriented molecules, unless they are chiral.
Since it is driven by purely electric-dipole interactions, the induced signals can be strong if $\omega_3$ is close to resonance \cite{Belkin2001PRL}.
However, the intensity of SFG is not enantio-sensitive, it only depends on whether the molecules are chiral or not.
Their handedness remains hidden in the phase of the signal.

To measure the phase of SFG, and thus the medium's handedness, one can make it interfere with a reference signal using a local oscillator \cite{Okuno2018JCP}.
This achiral reference can also be generated from the chiral sample itself, making the near-field intensity enantio-sensitive.
To this goal, one can take advantage of magnetic interactions \cite{Belkin2000PRL}, or use a constant electric field \cite{Fischer2003PRL}, although these strategies offer limited enantio-sensitivity and opportunities for control.

Here we introduce an all-optical approach for efficient enantio-discrimination that relies on the interference between two low-order nonlinear processes: chiral SFG and achiral third-harmonic generation (THG).
We show how, by sculpting the sub-cycle oscillations of the laser's electric field vector, we can control the ultrafast optical response of the molecules in a highly enantio-sensitive manner: quenching the low-order nonlinear response of one molecular enantiomer while maximising it in its mirror twin.
This work brings giant enantio-sensitivity from the strong-field regime \cite{Neufeld2019PRX,Ayuso2019NatPhot,Ayuso2021NatComm,Ayuso2021Optica,Ayuso2022PCCP,Ayuso2022OptExp,Khokhlova2022} to the perturbative regime.

Chiral SFG can be efficiently driven by any combination of frequencies $\omega_1\neq\omega_2$, as long as the sum-frequency $\omega_3=\omega_1+\omega_2$ is close to resonance \cite{Belkin2001PRL}, as above described.
Let us impose $\omega_2=2\omega_1$, and thus $\omega_3=3\omega$, with $\omega=\omega_1$ being the fundamental frequency, and consider the next-order nonlinear process: third harmonic generation (THG).
The medium can efficiently absorb 3 photons of frequency $\omega$ from the first beam, still at relatively low laser intensities, which leads to achiral polarization at frequency $3\omega$.
Can this achiral response of the molecule interfere with the SFG response to produce an enantio-sensitive interference?
Unfortunately, momentum conservation dictates that, while the SFG and THG signals have the same frequency, they are emitted in different directions.
Indeed, the THG signal co-propagates with the $\omega$ beam, as $\textbf{k}_{\text{THG}}=3\textbf{k}_{\omega}$, whereas the SFG signal is emitted in between the two driving beams, because $\textbf{k}_{\text{SFG}}=\textbf{k}_{\omega}+\textbf{k}_{2\omega}$.
In the following, we show how, a relatively simple modification of the original SFG setup, allows us to overcome this issue.

The proposed optical setup combines a linearly polarised beam with frequency $\omega$ and a second beam that carries cross-polarised $\omega$ and $2\omega$ frequencies.
The $\omega$ components are polarised in the plane of propagation, whereas the $2\omega$ component is polarised orthogonal to this plane, see Fig. \ref{fig_setup}a.
The laser field can be written as:
\begin{align}
\textbf{E} &= \Re\{ \textbf{E}^{(1)} + \textbf{E}^{(2)} \}, \\
\textbf{E}^{(1)} &= E_{\omega}^{(1)} \, e^{-i\omega t + i\textbf{k}_1\cdot\textbf{r}} \, \hat{\textbf{e}}_1, \\
\textbf{E}^{(2)} &= E_{\omega}^{(2)} \, e^{-i\omega t + i\textbf{k}_2\cdot\textbf{r}} \, \hat{\textbf{e}}_2
+ E_{2\omega}^{(2)} \, e^{-2i\omega t -i\phi_{2\omega} + 2i\textbf{k}_2\cdot\textbf{r}} \, \hat{\textbf{e}}_3,
\end{align}
where $E_{\omega}^{(1)}$, $E_{\omega}^{(2)}$ and $E_{2\omega}^{(2)}$ are the amplitudes including the temporal and spatial Gaussian envelopes.
Note that our proposal has two key differences with respect to traditional SFG configurations,
%(Fig. \ref{fig_SFG})
which change dramatically the properties of the field that is created.
First, we set the ratio between the two input frequencies to $\omega_2/\omega_1=2$.
Second, we add the fundamental $\omega$ frequency to the second beam.
In traditional SFG setups, the polarization of the electric-field vector is always confined to a plane, and therefore achiral within the electric-dipole approximation.
In our setup, the combination of the two beams creates a locally chiral field: the polarization of the electric-field vector draws a (three-dimensional) chiral trajectory in time.

\begin{figure}[h]
\centering
\includegraphics[width=\textwidth]{Figures/setup.png}
\caption{\label{fig_setup}
\textbf{Enantio-sensitive SFG.}
\textbf{a,} The proposed optical setup combines a linearly polarised beam with frequency $\omega$ and a second beam that carries cross-polarised $\omega$ and $2\omega$ colours.
The $\omega$ frequency components are polarised in the plane of propagation, the $2\omega$ component is polarised orthogonal to this plane.
\textbf{b,} Multiphoton pathways describing chiral SFG (left) and achiral THG (right). The induced polarisation associated with SFG has the same amplitude and opposite phase in opposite molecular enantiomers, $P_{SFG}^L=-P_{SFG}^R$, whereas the polarisation associated with THG is identical, $P_{THG}^L=P_{THG}^R$.
\textbf{c,} Multiphoton diagrams describing momentum conservation in SFG (left) and THG (right).
Having the $\omega$ frequency component in both beams creates several THG pathways, one of them leading to emission in the same direction as the chiral SFG signal, so the two channels can efficiently interfere.}
\end{figure}

The generated locally chiral field can drive a strongly enantio-sensitive response in a medium of chiral molecules via interference between SFG and THG, see Fig. \ref{fig_setup}b.
The SFG pathway is as in traditional SFG implementations, see Fig. \ref{fig_setup}c (left).
By adding the $\omega$ frequency to the second beam, we open new THG pathways, see Fig. \ref{fig_setup}c (right).
The medium can now absorb the three $\omega$ photons from the same beam, or two photons from one beam and one from the other, giving rise to emission of achiral THG in four different directions.
Importantly, one of these pathways, the one involving absorption of one photon from the first beam and two photons from the second beam, leads to achiral THG emission exactly in the same direction as the chiral SFG signal, see Fig. \ref{fig_setup}c.
The two contributions can now interfere, making the intensity of harmonic emission strongly enantio-sensitive.

To demonstrate our proposal, we have performed state-of-the-art numerical simulations in randomly oriented propylene oxide molecules.
We have considered the following laser parameters:
intensity of the $\omega$ field in the first beam $I_{\omega}^{(1)}=3 \times 10^{12}\,$W/cm$^2$, in the second beam $I_{\omega}^{(2)}=3 \times 10^{12}\,$W/cm$^2$, intensity of the $2\omega$ component $I_{2\omega}^{(2)}=7 \times 10^{11}\,$W/cm$^2$, varying two-colour phase delay $\phi_{\omega,2\omega}$, and opening angle $\alpha=25^\circ$.
Fig. \ref{fig_far}a,b shows the intensity of the two contributions to emission at frequency $3\omega$ in the far field (SFG and THG).
As already anticipated, the SFG contribution is emitted at a divergence angle equal to $\arcsin{(\sin(\alpha)/3)}=8.1^\circ$ (Fig. \ref{fig_setup}c, left), whereas the achiral THG shows 4 peaks, at angles $-25.0^\circ$, $-8.1^\circ$, $8.1^\circ$ and $25.0^\circ$, corresponding to the 4 possible $k$-vectors' combinations (Fig. \ref{fig_setup}c, right).

\begin{figure}[h]
\centering
\includegraphics[width=0.7\textwidth]{Figures/far.png}
\caption{\label{fig_far}
\textbf{Enantio-sensitive SFG in propylene oxide.}
\textbf{a,b,} Intensity at frequency $3\omega$ in the far field associated with chiral SFG (\textbf{a}) and achiral THG (\textbf{b}).
\textbf{c-f} Total intensity emitted from left-handed (\textbf{c,e}) and right-handed (\textbf{d,f}) propylene oxide at frequency $3\omega$, resulting from adding the chiral SFG (\textbf{a}) and achiral THG (\textbf{b}) contributions, for $\phi_{\omega,2\omega}=0.67 \pi $ (\textbf{c,d}) and $\phi_{\omega,2\omega}=1.67 \pi$ (\textbf{e,f}).
}
\end{figure}

The intensity profiles of the SFG and THG contributions are, individually, not enantio-sensitive, see Fig. \ref{fig_far}a,b.
However, as the SFG contribution is out of phase in opposite molecular enantiomers, the total intensity of emission, resulting from adding the two contributions, becomes strongly enantio-sensitive, see Fig. \ref{fig_far}c-f.
Here we have set the amplitude of the $2\omega$ component of the driving field so  the SFG contribution to the induced polarisation and the THG contribution that leads to emission at angle $-8.1^\circ$ have the same amplitude.
Then, by controlling the two-colour delay in the second beam, we can adjust the phase of chiral polarisation associated with SFG, and thus achieve full control over the interference, and thus over the detected intensity at frequency $3\omega$.
For $\phi_{\omega, 2\omega}=0.67\pi$, the SFG and THG contributions interfere constructively in the left-handed molecules (Fig. \ref{fig_far}c) and destructively in the right-handed molecules (Fig. \ref{fig_far}d).
Changing the two-colour delay by $\pi$, i.e. setting $\phi_{\omega,2\omega}=1.67\pi$, also changes the phase of the SFG contribution by $\pi$, and thus it results in the opposite effect: suppression from the left-handed molecules (Fig. \ref{fig_far}e) and strong signal from the right-handed molecules (Fig. \ref{fig_far}f).

Let us emphasise that the possibility of driving such a strong enantio-sensitive response in the far field (Fig. \ref{fig_far}) is enabled by the fact that the driving field is locally chiral, and that the nonlinear response of the molecules is strongly enantio-sensitive already in the near field, at the single-molecule level.
That is, while analysing the SFG and THG contributions to the far-field signal independently (Fig. \ref{fig_far}a,b) provides a clear and intuitive picture, the mechanism giving rise to the enantio-sensitive macroscopic response is not heterodyne detection in the far field, but an enantio-sensitive interference at the single-molecule response level.

Fig. \ref{fig_control} shows how we can fully control the intensity of emission at the sum-frequency $\omega_3=3\omega$ in one spot in the far field, at emission angle $-8.1^\circ$, in a highly enantio-sensitive manner.
Indeed, because the polarisation associated with SFG is out of phase in opposite enantiomers, the values of the two-colour delay that maximise and quench emission from the left- (Fig. \ref{fig_control}a) and right-handed (Fig. \ref{fig_control}b) molecules are shifted by $\pi$.
Note that the specific values of the two-colour delay that maximise or quench the nonlinear response depend on the relative phase between the SFG and THG contributions to light-induced polarisation, which depend on the anisotropy of the chiral molecular potential and thus are molecule-specific quantities.

\begin{figure}[h]
\centering
\includegraphics[width=\textwidth]{Figures/control.png}
\caption{\label{fig_control}
\textbf{Enantio-sensitive control of SFG.}
\textbf{a,b,} Intensity at frequency $3\omega$ emitted from left-handed (\textbf{a}) and right-handed (\textbf{b}) propylene oxide at frequency $3\omega$ as a function of the divergence angle and the two-colour phase delay.
\textbf{c,} Dissymmetry factor $\gamma=2\frac{I_L-I_R}{I_L+I_R}$. The grey color indicates that there is no intensity for any enantiomer.
}
\end{figure}

To quantify the degree of enantio-sensitivity in the nonlinear optical response, we use a standard definition of the dissymmetry factor $\gamma=\frac{I_L-I_R}{(I_L+I_R)/2}$, which reaches the ultimate efficiency limits of $\pm200\%$, see Fig. \ref{fig_control}c.
Indeed, we can maximise emission from the left-handed molecules and fully quench it in the right-handed molecules ($\gamma=200\%$), or vice versa ($\gamma=-200\%$), by adjusting the two-colour delay in the second beam.
Indeed, it is the two-colour delay that defines the local handedness of the locally chiral field, and controls the enantio-sensitive response of the chiral molecules.

Note that, while the intensity of emission at the sum-frequency $\omega_3=3\omega$ at angle $-8.1^\circ$ is strongly enantio-sensitive and can be fully controlled with the two-colour delay $\phi_{\omega,2\omega}$, the intensity peak at $8.1^\circ$ is completely independent on both the molecular handedness and the two-colour delay.
This is due to the different phase matching conditions in SFG and THG, which dictate that the peak at $8.1^\circ$ is exclusively due to THG, as can be understood in terms of conservation of momentum, see Fig. \ref{fig_setup}c.
Thus, the peak at emission angle $8.1^\circ$ constitutes a constant reference which can be used for calibration purposes and for reducing the noise in experimental measures.

Our proposal constitutes a simple, yet highly efficient, all-optical approach for enantio-discrimination, which relies on the interference between well established second- and third-order nonlinear phenomena (SFG and THG) which are ubiquitous in nonlinear optics.
They have both been widely recorded in chiral and achiral media, both from isotropic and anisotropic media, and in the gas, liquid, and solid phases of matter.
Yet, despite the general importance of chiral molecules, to the best of our knowledge these two phenomena had not, until now, been combined to achieve efficient enantio-discrimination.

Our approach takes advantage of the tremendous capabilities of modern optical technology for sculpting the polarization of light with sub-optical-cycle temporal resolution.
Indeed, by controlling the two-colour delay in the proposed optical setup, we can tailor the chirality of the driving field in order to maximise the nonlinear response of a selected molecular enantiomer while suppressing it in its mirror twin.

The possibility of driving strongly enantio-sensitive interactions via low-order nonlinear processes creates tremendous opportunities for imaging and controlling molecular chirality on ultrafast timescales using laser fields with gentle intensities, as well as for developing enantio-sensitive optical traps and tweezers.
In addition, new exciting opportunities may arise from the possibility of creating synthetic chiral light in nano-photonic structures, where the strong light confinement creates strong longitudinal electric-field components.

\section*{Acknowledgments}
We acknowledge enlightening discussions with Misha Ivanov, Andr\'es Ordo\~nez, Rose Picciuto, Mary Matthews and Jon Marangos.
L. R. acknowledges financial support from the European Union-NextGenerationEU and the Spanish Ministry of Universities via her Margarita Salas Fellowship through the University of Salamanca;
J. V., L. R. and D. A. acknowledge funding from the Royal Society URF$\backslash$R1$\backslash$201333 and RF$\backslash$ERE$\backslash$210358.
Funded by the European Union (ERC, ULISSES, 101054696). Views and opinions expressed are however those of the author(s) only and do not necessarily reflect those of the European Union or the European Research Council. Neither the European Union nor the granting authority can be held responsible for them

\bibliography{Bibliography}

\end{document}
