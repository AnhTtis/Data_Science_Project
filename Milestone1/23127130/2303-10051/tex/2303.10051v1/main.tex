 \documentclass[aps,twocolumn,showpacs,amsmath,amssymb,prx,superscriptaddress,floatfix,longbibliography]{revtex4-1}
%\documentclass[aps,preprint,showpacs,preprintnumbers,amsmath,amssymb,superscriptaddress,floatfix]{revtex4-1}
%\documentclass[twocolumn,showpacs,preprintnumbers,amsmath,amssymb]{revtex4}
%\documentclass[preprint,showpacs,preprintnumbers,amsmath,amssymb]{revtex4}
\usepackage{ braket}
\usepackage{amsmath}
\usepackage{amssymb}
\usepackage{mathtools}
\usepackage{graphicx}% Include figure files
\usepackage{dcolumn}% Align table columns on decimal point
\usepackage{bm}% bold math
\usepackage{multirow}
\usepackage{leftidx}
\usepackage{color}
\usepackage{qcircuit}
\usepackage{ifthen}

\usepackage{amsfonts}
%%\usepackage{eufrak}
\usepackage[german,english]{babel}


% Mark S commands
\newcommand \opH {{\hat {\mathcal H}}}
\newcommand \nn{\nonumber}
\newcommand \noi{\noindent}
\newcommand \be{\begin{equation}}
\newcommand \ee{\end{equation}}
\newcommand \bea{\begin{eqnarray}}
\newcommand \eea{\end{eqnarray}}
\newcommand \bse{\begin{subequations}}
\newcommand \ese{\end{subequations}}
\newcommand \mcB{{\mathcal B}}
\newcommand \mcE{{\mathcal E}}
\newcommand \mcF{{\mathcal F}}
\newcommand \mcO{{\mathcal O}}
\newcommand \bfa{{\bf a}}
\newcommand \bfb{{\bf b}}
\newcommand \bfc{{\bf c}}
\newcommand \bfd{{\bf d}}
\newcommand \bfe{{\bf e}}
\newcommand \bfk{{\bf k}}
\newcommand \bfl{{\bf l}}
\newcommand \bfn{{\bf n}}
\newcommand \bfp{{\bf p}}
\newcommand \bfr{{\bf r}}
\newcommand \bfs{{\bf s}}\newcommand \bfu{{\bf u}}
\newcommand \bfv{{\bf v}}
\newcommand \bfA{{\bf A}}
\newcommand \bfB{{\bf B}}
\newcommand \bfE{{\bf E}}
\newcommand \bfJ{{\bf J}}
\newcommand \bfM{{\bf M}}
\newcommand \bfU{{\bf U}}
\newcommand \bfmu{{\boldsymbol \mu}}
\newcommand \bfomega{{\boldsymbol \omega}}
\newcommand \bfsigma{{\boldsymbol \sigma}}
\newcommand {\editms}[1]{\textcolor{blue}{$\spadesuit$~#1}}
\newcommand {\edittg}[1]{\textcolor{red}{$\clubsuit$~#1}}

% color comments 
\newboolean{ShowComments}
\setboolean{ShowComments}{false}  
% change to true to show remaining colorful author comments. 
% change to false to hide all author comments.

\newcommand{\ColorComment}[3]{%
	{\colorbox{#1}{\color{white}   \textsf{\textbf{#2}}} \textcolor{#1}{#3}}}
%  Colorful box, initials, phrase 
\newcommand{\nyacite}[1]{[#1]}
% not yet a cite
	
\definecolor{mscolor}{rgb}{0,0.5,0.5}\newcommand{\ms}[1]{\ColorComment{mscolor}{ms}{#1}}
\definecolor{tgcolor}{rgb}{0.5,0,0.5}\newcommand{\tg}[1]{\ColorComment{tgcolor}{tg}{#1}}

\definecolor{rkccolor}{rgb}{0.5,0.5,0}\newcommand{\rkc}[1]{\ColorComment{rkccolor}{rkc}{#1}}



%%%%%%%%%%%%%%%%%%%%%%%%%%%

\begin{document}
\title{
Mid-circuit measurements on a neutral atom quantum processor} 
\author{T. M. Graham}
\affiliation{Department of Physics, University of Wisconsin-Madison,
Madison, Wisconsin 53706}
\author{L. Phuttitarn}
\affiliation{Department of Physics, University of Wisconsin-Madison,
Madison, Wisconsin 53706}
\author{R. Chinnarasu}
\affiliation{Department of Physics, University of Wisconsin-Madison,
Madison, Wisconsin 53706}
\author{Y. Song}
\affiliation{Department of Physics, University of Wisconsin-Madison,
Madison, Wisconsin 53706}
\affiliation{current address: Korea Research Institute of Standards and Science, Daejeon 34113, Republic of Korea}
\author{C. Poole}
\affiliation{Department of Physics, University of Wisconsin-Madison,
Madison, Wisconsin 53706}
\author{K. Jooya}
\affiliation{Department of Physics, University of Wisconsin-Madison,
Madison, Wisconsin 53706}
\author{J. Scott}
\affiliation{Department of Physics, University of Wisconsin-Madison,
Madison, Wisconsin 53706}
\author{A. Scott}
\affiliation{Department of Physics, University of Wisconsin-Madison,
Madison, Wisconsin 53706}
\affiliation{current address: Department of Computer Science, University of Colorado, Boulder, Colorado, }
\author{P. Eichler}
\affiliation{Department of Physics, University of Wisconsin-Madison,
Madison, Wisconsin 53706}
\author{M. Saffman}
\affiliation{Department of Physics, University of Wisconsin-Madison,
Madison, Wisconsin 53706}
\affiliation{Infleqtion Inc., Madison, Wisconsin, 53703}
% \author{ T. M. Graham}
% \affiliation{Department of Physics, University of Wisconsin-Madison,
%Madison, Wisconsin 53706}
 
% \author{  C. Poole}
% \affiliation{Department of Physics, University of Wisconsin-Madison,
%Madison, Wisconsin 53706}
 
% \author{  L. Phuttitarn and .... }
% \affiliation{Department of Physics, University of Wisconsin-Madison,
%Madison, Wisconsin 53706}
 
 
% \author{  M. Saffman}
%  \affiliation{Department of Physics, University of Wisconsin-Madison,
%Madison, Wisconsin 53706}
% \affiliation{ColdQuanta, Inc., 
%Madison, Wisconsin 53703}

\date{\today}

 
\begin{abstract}
We demonstrate mid-circuit measurements in a  neutral atom array by shelving data qubits in protected hyperfine-Zeeman sub-states while non-destructively measuring an ancilla qubit. Measurement fidelity was enhanced using   microwave repumping of the ancilla during the measurement. The coherence of the shelved data qubits was extended during the ancilla readout with dynamical decoupling pulses, after which the data qubits are returned to $m_f=0$ computational basis states.  We demonstrate that the quantum state of the data qubits is well preserved up to a constant phase shift  with  a state preparation and measurement (SPAM) corrected process fidelity of ${\mathcal F}= 97.0(5)\%$. 
The measurement fidelity on the ancilla qubit after correction for state preparation errors is ${\mathcal F}=94.9(8)\%$ and ${\mathcal F}=95.3(1.1)\%$ for $\ket{0}$ and $\ket{1}$ qubit states, respectively.
We discuss extending this technique to repetitive quantum error correction using quadrupole recooling and microwave-based quantum state resetting. 
\end{abstract}

\maketitle
%\tableofcontents

\section{Introduction}

Scalable quantum computation relies on the coherent, unitary evolution of a large number of qubits. Available qubit technologies have limited coherence times and suffer from errors in gate operations which restricts the usable length of quantum circuits\cite{Preskill2018}. To circumvent these limitations, the leading approach to quantum error correction combines repetitive mid-circuit projective measurements of ancilla qubits for error syndrome extraction, with corrective operations (or frame changes) that are conditioned on the results of measurement outcomes. Repetitive mid-circuit measurements have been demonstrated on quantum processors that use 
trapped ion\cite{Schindler2011,Negnevitsky2018,RyanAnderson2021,Egan2021}, 
superconducting\cite{Kelly2015,Ofek2016,LHu2019,Andersen2020,Bultink2020,ZChen2021,Acharya2023}, 
or spin\cite{Cramer2016} 
qubits. Mid-circuit measurements are also valuable for reducing circuit complexity and increasing the fidelity of a targeted output state, without invoking the full machinery of error correction.  An early example of the utility of mid-circuit measurements was provided by the semi-classical quantum Fourier transform\cite{Griffiths1996,Chiaverini2005}. Such semi-classical techniques use the information provided by mid-circuit measurements to prepare a target state without two-qubit gate operations. More recently, several works have shown how mid-circuit measurements or qubit reset and reuse may enhance the performance of noisy hardware for state preparation\cite{Rattew2021,TCLu2022} or increase the efficacy of noise mitigation techniques\cite{Czarnik2021}.

Fast, reliable mid-circuit measurements in neutral atom quantum computing systems face several challenges. Quantum state measurements of qubits encoded in atomic hyperfine-Zeeman states are generally based on detection of scattered photons from the state that is coupled to imaging light, whereas the other qubit state does not couple, or only couples very weakly, to the light used for imaging\cite{Saffman2005a}. However, during readout, off-resonant Raman transitions cause depumping to states in a far-off-resonant hyperfine manifold that is dark to the readout light, so a repumping laser is needed. The standard protocol for performing state-selective measurements with alkali atom qubits is to use a resonant push-out beam to remove atoms in one of the two computational basis states. Atoms in the remaining computational basis state are then made to fluoresce using near-resonant light. The fluorescence is imaged onto a sensitive camera where it is detected and spatially analyzed to determine the array occupancy. This measurement protocol is incompatible with mid-circuit measurements for two main reasons. First, it is global; a mid-circuit measurement requires only selected sites to be measured without affecting the states of non-selected sites. Second, it is destructive; a qubit measured during a mid-circuit measurement should be reusable in the remainder of the quantum circuit. If the imaging beam is focused to selectively readout an ancilla qubit, then it is possible to achieve a high fidelity, site-selective state measurement, but neighboring data qubits suffer from the fact that they may absorb a scattered photon, leading to a change in the quantum state. There are several strategies to address these issues including electromagnetically induced transparency \cite{Giraldo2022,Saglam2023}, atom transport to isolated imaging regions\cite{Bluvstein2022} or a near-concentric readout cavity\cite{Deist2022}), the use of multiple atomic species\cite{Beterov2015,YZeng2017,Singh2022}, and the use of atoms with a more complex internal structure that allows measurements to be performed using wavelengths that do not interact with atoms in computational basis states \cite{YWu2022}.

 
We present here the first scalable mid-circuit measurements on a single species neutral atom processor compatible with global, parallel ancilla readout. Measurements are implemented in an array of Cs atom qubits and characterized as regards both the measurement fidelity of the ancilla qubit and how the mid-circuit measurement affects the data qubits. 
As described in Sec. \ref{sec.experiment} the approach is  based on  shelving data qubits in hyperfine levels that are only weakly coupled to the light used for state measurement, while the ancilla qubit is measured using previously demonstrated non-destructive state-selective readout techniques \cite{Kwon2017,Martinez-Dorantes2017}. The data qubits are restored to the qubit basis upon completion of the measurement.  The obtained results for ancilla qubit measurements and process fidelity of the quantum information encoded in a  plaquette of data qubits are presented in Sec. \ref{sec.results}.
We conclude in Sec. \ref{sec.outlook} with an outlook for further improvements and applications to quantum state preparation and measurement based quantum computation. 



\section{Experimental approach}
\label{sec.experiment}

The experimental apparatus has been described in detail in \cite{Graham2022}; however, some modifications were made to facilitate mid-circuit measurements. Cs atoms were laser cooled and stochastically loaded into a $3\times 3$ two-dimensional array of red-detuned, 1064-nm wavelength tweezers which had a beam waist of $1.2~\mu$m and a trap depth of approximately $0.4~$mK. The tweezer array was created by diffracting a Gaussian input beam with a two-dimensional acousto-optic deflector (AOD). Each of the deflector's axes was driven with three different frequency tones provided by a Quantum Machines OPX device. The amplitudes of the frequency tones were adjusted for equal trap depths by  balancing the trap light induced Stark shifts \footnote{Ramsey microwave spectroscopy was used to determine the precise clock state resonant frequency, which was then used as a feedback signal to adjust the trap frequency tone amplitudes}.

\begin{figure}[!t]
%\vspace{.1in} 
\includegraphics[width=3.3in]{figures/Figure_1.pdf}
\caption{
\label{fig.1}
a) Experimental layout for mid-circuit measurement. Atoms were trapped and cooled into a $3\times3$ 1064-nm tweezer array.   The microwave horns are each attached to 40 W amplifiers that are driven by 9.2 GHz signals. A 459-nm beam provided site-selective Stark shifts on targeted sites. During detection, near-resonant 852-nm light was used to induce atomic fluorescence. The fluorescence was imaged onto an electron-multiplying CCD camera. b) We shelved the data qubits in the $f=3$ hyperfine manifold in a two step process. In the first step, we applied a microwave pulse which drove transitions from $\ket{4,0}\rightarrow \ket{3,-1}$ and $\ket{3,0}\rightarrow \ket{4,-1}$. A relative phase shift between the two microwave horns provided control of the microwave field polarization such that the former transition undergoes a $2\pi$ rotation while the latter undergoes a $\pi$ rotation. The atoms were then shelved in the $f=4$ manifold. Note that during this pulse, the ancilla qubit was shifted out of resonance using the 459-nm beam to provide a site-selective Stark shift\cite{Xia2015}. c) In the second step of shelving, the atoms were transferred from $f=4$  to the corresponding Zeeman states in the $f=3$ hyperfine manifold using two composite microwave pulses. d) During mid-circuit measurement of the ancilla, excited state hyperfine mixing can cause atoms in the bright $f=4$ state to leak into $f=3$. Three microwave pulses were periodically employed to transfer population back up to $f=4$ to allow fluorescence to resume. e) During the mid-circuit measurement, the data qubits experience dephasing due to magnetic fluctuations. Periodic echoing is needed to mitigate the effects of this dephasing. We used three composite microwave pulses to perform these echoes.    
}
\end{figure}


\begin{table}[!t]
\caption{Readout parameters. $\gamma=2\pi\times 5.2\rm$ MHz is the decay rate of the Cs $6p_{3/2}$ state and $I_{\rm sat}= 1.1~\rm mW/cm^2$ is the saturation intensity of the $6s_{1/2}\ket{4,4}\rightarrow 6p_{3/2}\ket{5,5}$ transition.  }\label{Table.1}
\centering
\begin{tabular}{|c|c|c| }

 \hline
     & Occupation readout  & Mid-circuit readout\\
\hline

Detuning & $-9 \gamma$ & $-2 \gamma$   \\
Intensity & $5 I_{\rm sat}$ & $3 I_{\rm sat}$ \\
Trap depth & $k_{\rm B}\times$  0.85 mK & $k_{\rm B}\times$  1.8 mK \\
Readout time & 30 ms & 4 ms\\

\hline
\end{tabular}
\end{table}
Measurement of the initial atom occupancy was performed using two pairs of counter-propagating $\sigma_+/\sigma_-$ polarized beams. The four beams lay in a single plane perpendicular to the trap $k$-vector. The beams were red-detuned from the $6s_{1/2},f=4 \leftrightarrow 6p_{3/2},f=5$ transition. During the measurement, atoms were occasionally depumped into the $6s_{1/2},f=3$ state which is dark to the readout light. Light tuned to the $6s_{1/2},f=3 \leftrightarrow 6p_{3/2},f=4$ transition  was added to each of the four readout beams to repump atoms back to the $f=4$ state. Typical readout parameters are summarized in Table \ref{Table.1}. To avoid standing waves, we used fiber acousto-optic modulators to apply 1-4 kHz frequency shifts to each readout beam. We also adiabatically ramped the trap depth to 0.85 mK to reduce loss from heating during the readout. The readout light was chopped out-of-phase with the trapping light using a $50\%$ duty cycle and a chopping frequency of $1.42$ MHz. Chopping the trap prevented the atoms from experiencing trap-induced Stark shifts while being illuminated by the readout beams. The atom occupancy was determined from a $30$ ms camera exposure.


After measuring the trap occupancy, the atoms were recooled using a combination of red-detuned polarization gradient cooling on the D2 line and blue-detuned lambda grey molasses using the D1 line \cite{YFHsiao2018} cooling atoms to a temperature of $7~\mu$K. The atoms were then optically pumped into the $\ket{f=4,m_f=4}$ (henceforth denoted $\ket{4,4}$) stretched state using $\sigma_+$ polarized  895-nm light. Photon scattering from optical pumping heated the atoms up to a temperature of $10~\mu$K. We then used a series of CORPSE pulses, composite microwave pulses which are robust against off-resonant errors \cite{Cummins2003}, to transfer the atoms to the $\ket{3,0}$ state ($\ket{0}$ in the qubit computational  basis) (see Appendix \ref{sec.microwave}). After these microwave operations, all atoms were initialized to $\ket{3,0}$.

\subsection{Mid-circuit measurement}

The mid-circuit measurement procedure is composed of 5 steps: 
1) ramp up the trap power on the ancilla atom to be measured, 
2) shelve data qubits (qubits that are not being measured) into the $f=3$ hyperfine manifold, 
3) perform state-selective readout of atoms in the $\ket{1}$ state ($\ket{f=4,m_f=0}$), 
4) unshelve the data qubits to restore the quantum states after the readout, and 
5) ramp the ancilla site trap back to the original level. During this process it is important that all data qubits maintain good phase coherence between $\ket{0}$ and $\ket{1}$ states to prevent the degradation of the qubits. Since the measured qubit is projected into one of the two computational qubit basis states, the coherence between these two states has no importance for the ancilla. However, for the mid-circuit measurement to be compatible with repetitive error correction schemes, the ancilla must be able to be reinitialized back to $\ket{0}$ and recooled back to its initial temperature. This qubit resetting was not performed in this initial mid-circuit measurement demonstration.  Appendix \ref{sec.qubit_reset} presents a path to implementing qubit reset in future experiments. Below, we detail how we implement the mid-circuit measurement technique in the atomic qubit array. 

\subsubsection{Trap Ramping}

To prepare for mid-circuit measurements, we first increased the trap depth of the ancilla site. The state selective measurement technique used for the mid-circuit measurement only cools the atoms in one dimension using counter-propagating $\sigma_+/\sigma_+$ polarized beams for the readout. While this polarization orientation minimizes hyperfine depumping, it  prevents polarization gradient cooling during the readout. Only 1D Doppler cooling is present during the readout and  the ancilla atom is heated in the other two dimensions.  A deeper trap depth allows more photons to be scattered from the ancilla atom before it is heated out of the trap. However, raising the trap depths of the data qubits was not desirable as it heated the data qubits, resulting in faster dephasing and required more total trap power. Instead, we increased the central frequency tone amplitude on each axis of the trap while simultaneously lowering the other two frequency tones. This increased power in the central trap site of the $3\times 3$ array while each of its nearest neighbor sites decreased, further cooling the data qubits \footnote{During the trap ramping, the depth of the four corner traps is lowered, and are thus not considered for experimental demonstration. In future applications, these sites can be used as a reservoir of atoms for atomic rearrangement.}. We use interpolation between the start and end points to smoothly transition between the two settings using a Blackman ramp profile over $200~\mu$s.  

\subsubsection{Data qubit shelving}

To perform a mid-circuit measurement, all data qubits needed to be shelved in the $f=3$ hyperfine manifold to avoid interacting with the readout light; shelved qubits are protected by a $9.2$ GHz detuning. This task is complicated by the fact that the transitions $\ket{3,0} \leftrightarrow \ket{4,-1}$ and $\ket{4, 0} \leftrightarrow \ket{3, -1}$ have nearly the same resonant frequency at the $10.2$ G magnetic bias field that is used in this demonstration. Simply driving the transition to move population from $\ket{4,0} \rightarrow \ket{3,-1}$ in an effort to shelve the $\ket{1}$ qubit state in the $f=3$ manifold, would transfer  population from $\ket{0}=\ket{3,0}$  to $\ket{4,-1}$. To circumvent this problem, we used two microwave horns driven at the same frequency at different orientations. By adjusting the relative driving phases of these horns we controlled the microwave polarization, as described in \cite{Smith2013} (for alternative shelving methods, see Appendix \ref{sec.shelving}). Each horn was driven with a phase controlled 9.2 GHz signal provided by a 40 W amplifier (for microwave system details, see Appendix \ref{sec.microwave}). Given physical space constraints and incomplete microwave polarization control, we found that it was optimal to perform the shelving sequence in two steps. In the first step, we shelve in  $f=4$ by adjusting the microwave polarization such that the $\ket{3,0} \rightarrow \ket{4,-1}$ transition experiences a $\pi$ rotation while $\ket{4,0} \rightarrow \ket{3,-1}$  experiences a $2\pi$ rotation. After this microwave pulse, state $\ket{0}$ was transferred to  $\ket{4,-1}$, and state  $\ket{1}$ remained encoded in $\ket{4,0}$. We then used two composite CORPSE pulses to transition $\ket{4,0} \rightarrow \ket{3, 0}$ and $\ket{4,-1} \rightarrow \ket{3, -1}$.

During the shelving sequence, the ancilla atom must be prevented from being shelved, or it will not be detected. We used  459-nm light that is red-detuned from the $6s_{1/2},f=4 \leftrightarrow 7p_{1/2}$ center of mass by 24 GHz to shift the atom out of resonance with the first half of the two-step shelving process, leaving the ancilla qubit encoded in the clock state basis \cite{Xia2015}.  The CORPSE pulses then swap population between the two clock states. At this point, we detected the population in $\ket{4, 0}$ using readout light. 

\subsubsection{State-selective qubit measurement}

To measure the state of the ancilla qubit, we require population in the $\ket{1}$ state to fluoresce without depumping to the $f=3$ hyperfine manifold. To prevent depumping, these measurements were performed with one pair of counter-propagating beams with $\sigma_+/\sigma_+$ polarization with respect to a $10.2$ G bias magnetic field. When illuminated, an atom occupying any state in the $f=4$ manifold was pumped to the $\ket{4,4}$ stretched state and then cycled on the $6s_{1/2}\ket{4,4} \leftrightarrow  6p_{3/2} \ket{5,5}$ transition \cite{Kwon2017,Martinez-Dorantes2017}.  With this readout beam and magnetic bias geometry, Raman transitions to the $6s_{1/2}$, $f=3$ hyperfine manifold are forbidden by selection rules. However, such methods need to be adapted to be compatible with the mid-circuit measurement protocol. Tensor shifts from the trap cause excited state mixing that result in depumping the ancilla to the $f=3$ manifold. While this problem can be mitigated by chopping the trap and readout light out of phase with each other, as is done in the initial occupancy measurement, the chopping causes either heating of the data atoms at lower chopping frequencies ($\leq 1.3$ MHz) or Zeeman state mixing at higher chopping frequencies ($\geq 1.3$ MHz). These limitations result in the ancilla atom occasionally being depumped into the $f=3$ manifold during readout. However, we found that periodically employing microwave pulses to coherently transfer population from $m_f>0$ Zeeman states in $f=3$ to the corresponding Zeeman state in $f=4$ mitigated this depumping problem (see Fig. \ref{fig.1}d). 

During the mid-circuit measurement, the data qubits are shelved in a magnetically sensitive state, and have a relatively short $T_2^*=3.2$ ms coherence time due to magnetic field and vector Stark shift fluctuations. To accommodate the limited $T_2^*$ time, we applied dynamical decoupling pulses to the data qubits during the readout sequence. We did not have a way of directly driving the $\ket{3,0}\leftrightarrow\ket{3,-1}$ transition, so we periodically paused the readout and performed a three pulse echo sequence (see Fig. \ref{fig.1}e). This sequence is comprised of a $\pi$-pulse to transfer population from $\ket{3,0}$ to $\ket{4,0}$, followed by a $\pi$-pulse to exchange population between $\ket{4,0}$ and $\ket{3,-1}$. A final clock state $\pi$-pulse transfers population from $\ket{4,0}$ back to $\ket{3,0}$. One complication arose in the echoing due to the differences between the trap depths of the ancilla qubit and the data qubits. To accurately determine the state of the ancilla qubit, atoms occupying the state $\ket{3,0}$ at the start of the readout must remain dark; however, the trap Stark shift changed the energy levels of the ancilla qubit such that the echoes were detuned. It was necessary to use CORPSE pulses during the echo to prevent the dark state of the ancilla qubit from leaking into the $f=4$ manifold. In addition, 459-nm light was used to shift the second composite echo pulse closer to resonance. During the 4 ms mid-circuit measurement, we found that 8 Hahn echo pulses were optimal to retain qubit coherence and minimize error due to imperfect microwave rotations.

To minimize dephasing on the data qubits, it was necessary to measure the ancilla atom quickly. This requirement suggested that we should use readout light with small detuning from resonance. However, since the trap is left at a constant intensity during the readout, an ancilla atom in the bright state experiences heating due to dipole force fluctuations (DFF) in addition to heating due to photon recoil (see \cite{Martinez-Dorantes2018} for an extensive treatment of these effects).  To reduce DFF heating, we used a detuning of $-2\gamma$ from the shifted resonance (shifted $+49$ MHz due to the trap Stark shift and Zeeman effect of the bias field) to scatter enough photons to distinguish the bright and dark states of the ancilla atom in 4 ms. This detuning choice balanced the effects of DFF heating and the dephasing of the data qubits. See Table \ref{Table.1} for a summary of readout parameters.

After the mid-circuit measurement, the ancilla atom's dark state was reset to its original state by performing the shelving sequence in reverse order. However, the bright state was significantly heated during the readout and was no longer suitable to use as a qubit. In future work, we will use a quadrupole transition to recool ancilla atoms in the bright state, so they can be restored to the computational basis and reused in the remainder of the calculation (see Appendix  \ref{sec.qubit_reset} for details).

\section{Results}
\label{sec.results}

There were two primary goals for the mid-circuit measurement demonstration. First, the ancilla qubit state must be accurately determined from the measurement; second, the state of the data qubits must not be changed by the measurement. To determine the ancilla qubit state, we imaged the atomic fluorescence onto an electron-multiplying CCD camera (EMCCD) and then applied standard region of interest and threshold techniques to determine if the atom was in a bright or dark state. To characterize the measurement fidelity, we sequentially performed measurements on each of the two computational basis state inputs (see Fig. \ref{fig.Data}). We obtained measurements with a $93.6(5)\%$ ($94.3(5)\%$) probability that a qubit prepared in $\ket{0}$ ($\ket{1}$)  was correctly measured. Some of this error was due to imperfect state preparation; when compensating for state preparation errors, we observed corrected fidelities of $94.9(8)\%$ ($95.3(1.1)\%$) for a qubit prepared in $\ket{0}$ ($\ket{1}$).

This measurement fidelity was limited by constraints imposed by the data qubits. The chief error sources for the dark state were imperfect state transfer during the echoing process, overlap of the bright and dark state histogram peaks, and incomplete shift-out of the microwave pulse which shelves the data qubits into the $f=4$ manifold. Bright state detection was similarly limited by histogram peak overlap and incomplete shift out of the microwave shelving pulse; in addition, it was limited by leakage into the shelved state basis as it was being pumped to the stretched state. Future improvements in fidelity may be achieved by using a higher power and further detuned shift-out laser, using quadrupole recooling during the readout to allow less detuned, faster readout, and using magnetic shielding and faster microwave sources to obtain higher fidelity microwave rotations.

In an ideal mid-circuit measurement, the data qubits would be completely unaffected by the measurement. The process of shelving and echoing the atoms during the measurement induced a phase shift on the data qubits; however, a phase advance can be compensated in quantum circuits in the circuit compilation or through a frame change. We  quantify the impact of the mid-circuit measurement on the data qubits by treating it as a quantum process and evaluating its fidelity relative to a phase gate with a calibrated phase shift. The phase shift was characterized by performing a mid-circuit measurement between two clock-state microwave $\pi/2$ rotations. By adjusting the phase of  the second pulse, we measured a Ramsey curve from which we extracted the phase (see Fig. \ref{fig.Data}).  We then determined the average process fidelity by measuring the fidelity of the inputs ($\ket{x}=\frac{\ket{0}+\ket{1}}{\sqrt{2}}$, $\ket{-x}=\frac{\ket{0}-\ket{1}}{\sqrt{2}}$, $\ket{y}=\frac{\ket{0}+i\ket{1}}{\sqrt{2}}$, $\ket{-y}=\frac{\ket{0}-i\ket{1}}{\sqrt{2}}$, $\ket{z}=\ket{0}$, $\ket{-z}=\ket{1}$) with target outputs calculated from using the extracted phase shift, and averaging the fidelities. This yielded a raw average process fidelity of $93.8(3)\%$. Compensating for state preparation and measurement errors, we obtained a fidelity of $97.0(5)\%$ (see Appendix \ref{sec.state_prep}). The process fidelity was limited by imperfect shelving and echoing and could be improved with faster microwave rotations. Faster readout of the ancilla state would also require fewer data qubit echoes preventing some of the microwave echo error accumulation.  


\begin{figure}[!t]
%\vspace{.1in} 
\includegraphics[width=3.5in]{figures/Figure_2.pdf}
\caption{
\label{fig.Data}
a) To show that the data qubits retain coherence during the mid-circuit measurement, we prepared the atoms in $\frac{1}{\sqrt{2}}\left( \ket{0}+\ket{1} \right)$ using a $\pi/2$-microwave pulse and then performed a mid-circuit measurement on the ancilla.  After the measurement, we applied another $\pi/2$ rotation at a phase $\phi$ with respect to the original $\pi/2$-pulse and then measured the state of the data qubits. Ideally, a full contrast oscillation would be observed; however, imperfections in the microwave rotations limited the average peak-to-peak amplitude of the oscillation to $92.3(6)\%$.   b) Ideally, the mid-circuit measurement  acts as a simple phase rotation of the data qubits, which can be compensated through phase advances. We measured the average process fidelity of the mid-circuit measurement on the data qubits with a unitary phase advance. We accomplished this by characterizing the phase shift from the Ramsey curve shown in a). We prepared 6 mutually unbiased basis states and measured the fidelity with the expected state assuming the mid-circuit measurement acts as a unitary phase shift; the average process fidelity of $93.8(3)\%$ ($97.0(5)\%$ SPAM corrected) was obtained by averaging these 6 inputs.  Error bars represent 1 standard deviation from the mean. c,d) We characterized the fidelity of the ancilla measurement for a bright state input c) and a dark state input d). We observe a detection fidelity of $93.6(5)\%$ ($94.9(8)\%$ when compensating for state preparation errors) for a $\ket{0}$ input and $94.3(5)\%$ ($95.3(1.1)\%$ when compensating for state preparation errors) for a $\ket{1}$ input.   }
\end{figure}

\section{Outlook}
\label{sec.outlook}

This mid-circuit measurement demonstration is the first time single-species mid-circuit measurement has been demonstrated in a 2d array of neutral atom qubits and is a step towards error corrected and fault tolerant quantum computing.   While our demonstration did not include qubit resetting, which is required for repetitive error correction, there is a clear path towards this goal through recooling ancilla atoms in the bright state using a narrow Cs quadrupole transition (see Appendix \ref{sec.qubit_reset}). The narrow line-width of this transition combined with the large hyperfine splitting of the excited state allows cooling without the need for repumping. The coherence on the data qubits can be maintained during this cooling using dynamical decoupling techniques. Once recooled, the ancilla atoms could be reset through a combination of global microwave rotations and previously demonstrated site-selective qubit rotation techniques \cite{Graham2022}. Mid-circuit measurement and qubit resetting would also permit more efficient logical qubit initialization for error correcting codes and quantum state preparation\cite{TCLu2022}. In this application, mid-circuit measurements enable preparation of long-range entangled states faster than the limits imposed by the Lieb-Robinson bounds.   Even without resetting, this demonstration enables many semi-classical quantum computing circuits \cite{Griffiths1996,Chiaverini2005}. These circuits leverage mid-circuit measurement and feed-forward rotations to perform operations, such as a semi-classical version of a quantum Fourier transform, without using two-qubit gates. Similarly, mid-circuit measurements (without requiring qubit resetting), are an important tool for measurement-based (a.k.a. one-way) quantum computing. This computing paradigm uses mid-circuit measurements and feed-forward operations to manipulate a large, highly-entangled resource state to perform quantum computations. Neutral atom platforms are an attractive platform for measurement-based quantum computing since large qubit arrays \cite{Huft2022} and long-range controllable interactions through the Rydberg blockade \cite{Saffman2010} needed for resource states have been demonstrated. Mid-circuit measurements in a neutral atom array provide a missing piece to enable measurement-based quantum computing in a neutral atom array.



This material is based upon work supported by NSF award No. 2210437, NSF Grant No. 2016136 for the Quantum Leap Challenge
Institute center Hybrid Quantum Architectures and Networks,
the U.S. Department of Energy Office of Science National Quantum
Information Science Research Centers as well as support from
the  U.S. Department of Energy, Office of Science, Office of High Energy Physics, under Grant No. DE-SC0019465.
 


\bibliography{atomic,saffman_refs,rydberg,qc_refs,optics}

\appendix 

%\section{Supplemental information}


\section{Experimental system}

In the following sections we provide details of the microwave subsystem, low noise coil drivers, and the setup for laser auto relocking. 


\subsection{Microwave source}
\label{sec.microwave}

We used microwave rotation for all of the coherent ground state transfer in this experimental procedure. We designed and built a system to coherently drive two microwave horns placed in different orientations relative to the atom cloud. To generate the two signals, we mixed each of two output channels of a Quantum Machines OPX box with a 8.992631770 GHz signal provided by a Hewlett Packard HP83623A frequency synthesizer (see Fig. \ref{SI_fig.uwave_layout}). Both the OPX and HP83623A were referenced to a 10 MHz GPS clock source. After the mixer, a 9.2 GHz band-pass filter removed the carrier frequencies. The two channels were boosted with a two-stage amplification process; first pre-amplifiers increased the signals up to the 1 dB compression point of the 40 W primary amplifiers. We were able to change the relative phase between the two channels allowing limited, but precise polarization control of the microwave signal at the atoms. This control was limited by two factors. First, one of the horns could not be positioned as close to the atoms as the other horn resulting in approximately a factor of two different microwave intensities that the two horns could provide to the atoms. Second, reflections off of metal surfaces that surround the cell distorted the polarization seen by the atoms. Despite these limitations, we were able to able to drive all microwave transitions with $>18$ kHz Rabi frequency; see Table \ref{SI_table.1} for a list of all transition Rabi frequencies. This polarization control allowed us to employ the single-pulse qubit shelving procedure described in the main text.



\begin{figure*}[!t]
%\vspace{.1in} 
\includegraphics[width=7in]{figures/SI_Microwave_layout.pdf}
\caption{
\label{SI_fig.uwave_layout}
Microwave signal generation diagram. Two channels from a Quantum Machines OPX at $\sim 200~\rm MHz$ are mixed with a $9~\rm GHz$ from a HP83623A. Both of these machines synchronized to a 10 MHz GPS clock. The resulting signals are filtered and amplified with a two stage amplification process. The final amplifier increases the signal power to 40 W of power before the signals are transferred to the microwave horns.
}
\end{figure*}


\begin{table}[!t]
\caption{Observed Rabi frequencies for various hyperfine transitions. For each transition (excluding the shelving transitions), the phase between the two microwave horns was tuned to optimize the Rabi frequency. For the transitions used during the shelving pulse, $\ket{4,0}\rightarrow\ket{3,-1}$ and $\ket{3,0}\rightarrow\ket{4,-1}$, this phase difference was adjusted such that the Rabi frequency of the former transition was twice that of the latter.}\label{SI_table.1}
\centering
\begin{tabular}{|c|c| }

 \hline
   Transition  &    Rabi frequency\\
\hline
$\ket{4,4}\rightarrow\ket{3,3}$ &  99.9 kHz   \\
$\ket{3,3}\rightarrow\ket{4,3}$ &  37.9 kHz \\
$\ket{4,3}\rightarrow\ket{3,2}$ &  87.7 kHz \\
$\ket{3,2}\rightarrow\ket{4,1}$ &  18.1 kHz   \\
$\ket{4,1}\rightarrow\ket{3,0}$ &  59.9 kHz   \\
$\ket{3,0}\rightarrow\ket{4,0}$ &  62.8 kHz   \\
$\ket{3,2}\rightarrow\ket{4,2}$ &  51.8 kHz   \\
$\ket{3,1}\rightarrow\ket{4,1}$ &  56.2 kHz   \\
$\ket{4,0}\rightarrow\ket{3,-1}$ &  44.8 kHz   \\
$\ket{3,0}\rightarrow\ket{4,-1}$ &  22.4 kHz   \\
$\ket{3,-1}\rightarrow\ket{4,-1}$ &  58.4 kHz   \\

\hline
\end{tabular}
\end{table}

\subsection{Coil drivers}
\label{sec.coil_drivers}
 

To control the current through the coils during the experiment, we developed V-I drivers that supply current  proportionally with an input voltage. These drivers have an additional voltage input that allows for a feed-forward signal that gets added to the main target input (though this compensation was not used in this experiment). The coil drivers function with a feedback loop that controls the amount of current flowing through a pair of MOSFETs, with a buffered sense resistor supplying feedback. Large emphasis was placed into designing the system to be resilient to ground loops caused by parallel channels of the coil driver. The spectral noise density of the current of these coil drivers is less than 1 part per million above 10 Hz, and can drive up to 2 A.


\subsection{Laser auto relock}

An automatic relocking mechanism was implemented and used to maintain the lock of the 459-nm laser that is used for local single qubit shift-out operations to a high finesse stable reference cavity . The 459 nm light is obtained from a 918 nm laser provided by an M-squared Ti:Sapphire laser that is locked to an ultra-low expansion (ULE) glass cavity and is then doubled to 459 nm in a bow-tie resonator. Details of this laser system and its locking system are described in \cite{Graham2022}. The auto-relock system uses information from the spatial profile of the ULE cavity's transmission measured with a Raspberry PiCam, the total transmission power measured with a power meter, and laser frequency data measured with a Moglabs MWM wavemeter. This information is used to control the laser frequency though TCP/IP commands sent to the M Squared SolsTiS. 

An unlock event is detectable if the transmission photodiode drops below a threshold, image analysis on the spatial mode from the camera determines that the cavity is locked to the wrong transverse mode (or not locked), or the wavemeter determines that the axial mode of the cavity has changed. The wavemeter is constantly calibrated to the ULE cavity while the laser is locked. When the laser unlocks, the frequency may drift further away from the correct transverse cavity mode by more than half of a Free spectral range (FSR). In this situation, the laser frequency is ramped in the direction of the correct transverse cavity mode. Once the laser is within half of the FSR to the previous lock point, the wavemeter does not have enough accuracy, so the laser is ramped until the correct spatial mode, determined using the camera, is found. A second higher frequency and smaller amplitude ramp is applied on top of the coarse SolsTiS ramp so that the camera shutter time integrates a wider span of the frequency allowing the spatial mode to be more easily located. Once the correct spatial mode is found, the coarse ramp is disabled and the lock integrator loops are re-enabled (the proportional and derivative loops are always enabled). This scheme has a successful relock rate of about $70\%$. It recursively attempts to lock until it succeeds or too many failed attempts have occurred (indicating that the lock parameters need to be tuned). This system does not tune error offset or loop gain, and simply re-enables loops. Eventually these parameters will saturate and the relocking system will need to be reinitialized. With this system, we  maintain  high finesse locks on the time scale of weeks and unlock events occur on time scales of days. A block diagram of this system is in Fig. \ref{SI_fig.autorelock}.

\begin{figure*}[!t]
%\vspace{.1in} 
\includegraphics[width=6in]{figures/SI_AutoRelock_BlockDiagram.pdf}
\caption{
\label{SI_fig.autorelock}
Block diagram of the auto-relock system for the 459nm laser system described in \cite{Graham2022}. This system uses a photodiode (PD) and Raspberry-Pi Camera (PiCamera) on the transmission side of an ultra-low expansion glass cavity (ULE), as well as a Moglabs MWM wavemeter to perform automatic locking of the laser. The error signal is generated by modulating an electro-optic modulator (EOM) through the Pound-Drever-Hall technique. An acousto-optic modulator (AOM) in a double-pass configuration shifts the laser frequency to the nearest spatial mode of the ULE cavity. The 918-nm light out of the M Squared SolsTiS is converted to 459nm with a second harmonic generation (SHG) crystal/cavity (not shown). 
}
\end{figure*}




\section{Experimental pulse sequence}
\label{sec.pulse_sequence}

The experimental sequence to perform a mid-circuit measurement and characterize it is complex. This sequence can be broken into various subsections (see Fig. \ref{SI_fig.circuit}). First, five composite CORPSE pulses drive the initial, optically pumped state $\ket{4,4}$ to $\ket{3,0}$. Each CORPSE pulse is composed of a $420^{\circ}$ rotation, followed by a $300^{\circ}$ rotation with the opposite phase, followed by a $60^{\circ}$ rotation in phase with the first rotation. This sequence is insensitive to small off-resonant errors caused by magnetic field fluctuations. After the atom is prepared in $\ket{3,0}$, an additional clock state rotation could be applied to control the input state into the mid-circuit measurement. 

After completing the state preparation, we began the mid-circuit measurement process by ramping the trap powers to increase the depth of the ancilla qubit trap. This was followed by a clock-state $\pi$-pulse to both compensate the ancilla qubit for a later clock-state rotation during the shelving process and allow the small amount of data qubit dephasing during the ramp to be echoed during the ramp up at the end of mid-circuit readout process.  The data qubits were then shelved using a the two-step shelving process using microwave polarization control described in the main text. During first step of the shelving process when the date qubits are shelved in $f=4$, a site-selective 459-nm laser shifts the ancilla out of the microwave resonance. During the second step of the shelving process, the data qubits shelved in $f=4$ were transferred to $f=3$ using two microwave CORPSE pulses. During this rotation, the ancilla qubit also experienced a $\pi$-rotation, which restores it the qubit state prior to the mid-circuit readout start.  Once data qubits were shelved, $\sigma_+$ was applied causing any population of the ancilla qubit in $\ket{4,0}$ to fluoresce. Periodically, we applied microwave repumping pulses to prevent depumping of the ancilla bright state into the dark $f=3$ manifold; these three pulses transfer population from $\ket{3,1} \rightarrow \ket{4,1}$, $\ket{4,2} \rightarrow \ket{3,2}$, and $\ket{3,3} \rightarrow \ket{4,3}$. The data qubit coherence was preserved using periodic echoing using the following sequence of three CORPSE pulses: $\ket{3,0} \rightarrow \ket{4,0}$, $\ket{3,-1} \leftrightarrow \ket{4,0}$, and $\ket{4,0} \rightarrow \ket{3,0}$. During the duration of the mid-circuit measurement, 46 microwave repumping cycles and 8 echoes were evenly dispersed in the readout. At the end of the mid-circuit measurement, the data qubits were unshelved using the reverse of the pulse sequence that was used for shelving and the trap powers were ramped back to their initial values. A final microwave $\pi$-pulse on the clock state rotates the data qubits back to their initial state but with a extra phase between the to clock states; this phase may be compensated during the circuit compilation. After completing the mid-circuit measurement, additional clock state rotations to select the measurement basis were applied. Finally, we performed a blowaway-based state-selective measurement on the data qubits.  In all, about 246 microwave pulses were applied to perform this mid-circuit readout characterization circuit (some measurements require 1 or 2 fewer pulses), and 230 of these pulses were applied during the mid-circuit readout.  

\begin{figure*}[!t]
%\vspace{.1in} 
\includegraphics[width=7in]{figures/SI_circuit.pdf}
\caption{
\label{SI_fig.circuit}
The experimental pulse sequence to characterize a mid-circuit measurement. This characterization sequence is composed of a series of pulses of microwaves to rotate qubit frequencies, 459-nm light to provide site selective Stark shifts, and 852 light cause selected atoms to fluoresce. All data and ancilla qubits begin the circuit in the $\ket{4,4}$ stretched state; 5 composite microwave pulse then coherently transition the atoms to the $\ket{3,0}$ state. A microwave pulse resonant with the $\ket{3,0}\leftrightarrow\ket{4,0}$ transition prepares the input state for the mid-circuit measurement. The mid-circuit measurement begins with ramping the trap depth of the ancilla qubit up to allow more photon scatter events before the atom is lost. An extra $\pi$ pulse is applied so that any dephasing arising from the trap ramp down is echoed out when the trap is re-raised. Three pulses shelve the data qubit; during the first of these pulses, focused 459 light shift the ancilla qubit out of resonance. Then, $\sigma_+$ readout light pulses are alternated with microwave pulses that repump the $\ket{3,1}$, $\ket{3,2}$, and $\ket{3,3}$ states to the $f=4$ manifold. After 3 of these cycles, a three pulse microwave echo is applied to the data qubits, then the three readout and microwave repump cycles are repeated. During the middle pulse of this echo, a low-amplitude 459-nm pulse is applied to the ancilla to allow any dark state population to be shifted into resonance. This full block is then repeated eight times in order to allow enough photons to be collected by the EMCCD camera to identify whether or not an atom was detected. The data qubits are the restored to the $\ket{3,0}$, $\ket{4,0}$ basis by running the shelving sequence in reverse order and the ancilla trap is ramped back down. The data qubits are then rotated to the output measurement basis with a microwave pulse and measured with a blowaway-based state selective readout.     
}
\end{figure*}



\subsection{Alternative two-pulse shelving sequence}
\label{sec.shelving}

In addition to the single pulse shelving sequence described in the main text, we designed and implemented a two-pulse shelving sequence which allows qubit shelving without precise polarization control. For this sequence we aligned the horn to have approximately even contributions of $\sigma_+$ and $\sigma_-$ polarization. We then used the difference of Clebsch-Gordan coefficients between the two transitions to design a pulse sequence to shelve the qubit. This sequence is composed of two detuned microwave pulses of the same length but different phases with respect to each other. Both pulses are designed to be detuned $2\pi$ rotations for the $\ket{3,0} \rightarrow \ket{4,-1}$ transition, so an atom in the $\ket{3,0}$ state remains in that state after both pulses (see Fig. \ref{SI_fig.shelve}). However, if an atom starts in $\ket{4,0}$, then the first microwave pulse creates an even super-position of $\ket{4,0}$ and $\ket{3,-1}$ states. By tuning the phase of the second pulse, it is possible to transfer all of the amplitude into the $\ket{3,-1}$ state. Unlike the shelving procedure described in the main text, two-pulse shelving does not require precise microwave polarization control. If the ratio of Rabi frequencies of the two transitions is between $\frac{1}{4} \leq \frac{\Omega_1}{\Omega_2}\leq \frac{3}{4}$, then their exists a combination of detuning, microwave pulse length, and phase that allows the conditions require for this shelving procedure to be met. We used this shelving sequence to obtain a shelving fidelity of $98\%$; however, the single pulse shelving in $\ket{4,0}$ and $\ket{4,-1}$ (followed by CORPSE transfer to $\ket{3,0}$ and $\ket{3,-1}$) technique allowed a higher fidelity of $99\%$ state transfer.  

\begin{figure}[!t]
%\vspace{.1in} 
\includegraphics[width=3.2in]{figures/SI_2pulse_shelving.pdf}
\caption{
\label{SI_fig.shelve}
An alternative two-pulse shelving technique which does not require precise polarization control. In this sequence, differences in polarization and Clebsch-Gordan coefficients result in a difference in Rabi frequency, $\Omega_1$, of the $\ket{4,0}\rightarrow\ket{3,-1}$ transition and the Rabi frequency, $\Omega_2$, of the $\ket{3,0}\rightarrow\ket{4,-1}$ transition.  For Rabi frequency ratios of $\frac{1}{4} \leq \frac{\Omega_1}{\Omega_2}\leq \frac{3}{4}$ this difference can be exploited by tuning the pulse length and detuning to find a condition where one of the transitions (e.g. $\ket{3,0}\rightarrow\ket{4,-1}$) experience a $2\pi$ rotation while the other logical qubit state (e.g. $\ket{4,0}$) is mapped to the equator of the Bloch sphere. A second identical pulse is then applied with a relative phase with respect to the first pulse. The $\ket{3,0}$ state still undergoes a $2\pi$ rotation during the second pulse; however, with a careful selection of $\phi$, the the $\ket{4,0}$ input is mapped to $\ket{3,-1}$. Pictured is an example where the microwave field has equal $\sigma_+$ and $\sigma_-$ components.
}
\end{figure}



\section{State preparation and measurement (SPAM) corrections}

\label{sec.state_prep}

The experiments characterizing the mid-circuit readout performance contained state preparation and measurement errors which need to be accounted for to obtain an accurate assessment of the performance. Below, we examine how such errors act on both the data qubits and ancilla qubit. We used additional experimental measurements to quantify SPAM and find corrected mid-circuit measurement performance estimates.

\subsection{SPAM correction of data qubit measurements}


An ideal mid-circuit measurement would have no effect on the data qubits and only affect the measured ancilla qubits. However, if the measurement instead acts as a known unitary operator, then this rotation can be compensated with either a corrective unitary or by accounting for this rotation in the context of the quantum circuit using a quantum circuit compiler. Errors that cannot be corrected can arise from dephasing in the shelved state or loss outside the qubit basis due to incorrect microwave rotations and/or scattering from the readout light. We wanted to measure the likelihood of this latter error type in order to evaluate how well the mid-circuit readout protocol preserved the data qubits. Our mid-circuit measurements theoretically acted as a unitary Z-phase shift on the data qubits. We measured the process fidelity of the experimental mid-circuit measurement with a Z-phase shift to identify non-unitary errors introduced by the measurement. The fidelity of a single qubit quantum process with a target can be determined by inputting the states ($\ket{0}$,$\ket{1}$,$\frac{\ket{0}+\ket{1}}{\sqrt{2}}$,$\frac{\ket{0}-\ket{1}}{\sqrt{2}}$,$\frac{\ket{0}+i\ket{1}}{\sqrt{2}}$,$\frac{\ket{0}-i\ket{1}}{\sqrt{2}}$)  into the quantum process and then measuring the fidelity of the output with the corresponding target state. The average state fidelity of the six outputs is the average process fidelity. Since all target output states are pure states, we can measure the state fidelity of the experimental output with the target output by rotating the target state to $\ket{0}$ and then performing a blowaway-based state-selective readout.  For these measurements, any atoms in the correct target output state should appear bright in the readout and atoms not in the correct state should be heated out of the trap in the blowaway. However, error in either the state preparation or measurement will affect these fidelity measurements. To compensate for these errors, we performed additional measurements to characterize these them so they could be compensated. To inform our measurements we created an error model below which details how various types of SPAM error affect the state fidelity measurements. For simplicity, we have not corrected for errors due to qubit rotations in the clock state basis during the characterization circuit. We previously characterized such errors in a similar setup to be $<3\times10^{-4}$ \cite{Graham2022}; so they did not contribute significantly to our SPAM error budget.

We first calculated the probability that an atom would be detected in the bright state during the final state-selective readout, $P(D_{B})$, based on the conditional probability of a bright output being detected given the different possible outcomes of the state prep and mid-circuit measurement. this probability is:

\begin{eqnarray}
\label{SI_eqn.D_B}
    P(D_{B}) & = & P(D_{B}|\ket{0}_{\rm out})P(\ket{0}_{\rm out}) \nonumber\\
                      & + & P(D_{B}|\ket{1}_{\rm out})P(\ket{1}_{\rm out}) \nonumber\\
                      & + & P(D_{B}|\ket{3,-1}_{\rm out})P(\ket{3,-1}_{\rm out}) \nonumber\\
                      & + & P(D_{B}|\ket{4,-1}_{\rm out})P(\ket{4,-1}_{\rm out}) \nonumber\\
                      & + & P(D_{B}|\epsilon_{l \rm, MCR})P(\epsilon_{l\rm, MCR}) \nonumber\\
                      & + & P(D_{B}|\epsilon_{\rm prep})P(\epsilon_{\rm prep}) \nonumber\\
                      & + & P(D_{B}|\epsilon_{l \rm,pre})P(\epsilon_{l \rm,pre}),
\end{eqnarray}

where $P(\ket{0}_{\rm out})$ is the probability that a data qubit atom was correctly initialized to $\ket{0}$ before the mid-circuit characterization circuit and was also correctly rotated to $\ket{0}$ after the circuit, $P(\ket{1}_{\rm out})$ the probability that a data qubit atom was correctly initialized to $\ket{0}$ before the mid-circuit characterization circuit but was incorrectly rotated to $\ket{1}$ after the circuit, $P(\ket{3,-1}_{\rm out})$ the probability that a data qubit atom was correctly initialized to $\ket{0}$ before the mid-circuit characterization circuit but was incorrectly rotated to $\ket{3,-1}$ after the circuit due to incorrect unshelving, $P(\ket{4,-1}_{\rm out})$ the probability that a data qubit atom was correctly initialized to $\ket{0}$ before the mid-circuit characterization circuit but was incorrectly rotated to $\ket{4,-1}$ after the circuit due to incorrect unshelving, $P(\epsilon_{l\rm, MCR})$ the probability that a data qubit atom was correctly initialized to $\ket{0}$ before the mid-circuit characterization circuit but was lost during the mid-circuit readout, $P(\epsilon_{\rm prep})$ is the probability that the atom was not properly initialized to the correct state due to either an optical pumping error or an error when rotating $\ket{4,4}$ to $\ket{0}$, and $P(\epsilon_{l \rm,pre})$ is the probability that the atom was lost from the trap before the mid-circuit characterization circuit. The conditional probabilities (e.g. $P(D_{B}|\ket{0}_{\rm out})$) represent the probability that a bright state is detected in the state selective readout given the respective outcomes of the mid-circuit measurement.  These conditional probabilities can be written in terms of how the blowaway-based state-selective readout interprets each of outcome possibilities:

\begin{eqnarray}
    P(D_{B}|\ket{0}_{\rm out}) &=& 1-\epsilon_{l \rm,post} \\
    \nonumber \\
    P(D_{B}|\ket{1}_{\rm out}) &=& \epsilon_{BA}(1-\epsilon_{l \rm,post}) \approx \epsilon_{BA}\\
    \nonumber \\
    P(D_{B}|\ket{3,-1}_{\rm out}) &=&  1-\epsilon_{l \rm,post} \\
    \nonumber \\
    P(D_{B}|\ket{4,-1}_{\rm out}) &=&  \epsilon_{BA}(1-\epsilon_{l \rm,post}) \approx \epsilon_{BA} \\
        \nonumber \\
    P(D_{B}|\epsilon_{l \rm,MCR}) & \approx &  0 \\
    \nonumber \\
    P(D_{B}|\epsilon_{\rm prep}) & = &  \epsilon_{MCR, 4}(1-\epsilon_{l \rm,post}) \nonumber\\
                                 & + &\epsilon_{MCR, 3}(1-\epsilon_{l \rm, post}) \nonumber\\
                                 &\approx&  \epsilon_{MCR, 4}+\epsilon_{MCR, 3}    \\
        \nonumber \\
    P(D_{B}|\epsilon_{l \rm,pre}) & \approx &  0, 
\end{eqnarray}

where $\epsilon_{l \rm,post}$ is the probability that an atom is lost after the mid-circuit measurement so that it is not detected during the state-selective readout, $ \epsilon_{BA}$ is the probability that $\ket{1}$ is erroneously detected during the state selective measurement as a bright state due to an error in the blowaway (likely caused by the atom depumping before it could be heated out of the trap), and $\epsilon_{MCR, 4}$ ($\epsilon_{MCR, 3}$) is the probability that an atom improperly initialized into an m-level in $f=4$ ($f=3$) will not be blown away by the mid-circuit readout process and will occupy a state in $f=3$ going into the state-selective readout.  Note that $\epsilon_{MCR, 4}$ and $\epsilon_{MCR, 3}$ will both be small since the data qubits are in a shallow trap during the mid-circuit readout and the combination of readout light and microwave repumping pulses  during the mid-circuit readout will cause state with $m_f > 0$ to both heat up and collect in $f=4,m=4$, where the blowaway light will then heat them out of the trap (if they survived the mid-circuit readout). It is extremely unlikely for any atoms to be prepared in $m_f<0$ or $\ket{1}$ during state prep since we used stretched-state optical pumping followed by coherent transfer to $\ket{0}$. Almost all of the optical pumping error should reside in states of $\delta m = -1$ and $\delta m = -2$  from the target state. This combined with the coherent transfer from $\ket{4,4}$ to $\ket{0}$ leaves a very small chance that the atom is erroneously prepared in $\ket{1}$ instead of $\ket{0}$. Furthermore, even if an optical pumping error resulted in the atom being erroneously prepared in $\ket{4,3}$ and it was then transferred to $\ket{4,0}$ during the CORPSE transfer segment of the state prep, the population in $\ket{4,0}$ should primarily be mapped back on to $\ket{4,0}$ at the end of the mid-circuit characterization circuit (note that since this error was originally caused by an optical pumping error, it will be incoherent with with the target input state, an therefore should not interfere and skew measurement probabilities). An atom in $\ket{4,0}$ at the end of the measurement will be heated out of the trap by the blowaway beam. The summary of this reasoning is that nearly all optical pumping and state preparation errors result in the atom being ejected from the trap before or during the blowaway of the final state-selective readout.  If the atom is lost either before or during the mid-circuit readout, then the probability that it was be detected as a bright state is negligible, since the dark histogram peak is 6.4 standard deviations from the photon count threshold needed to be classified at bright state detection event.

The output probabilities in eq. (\ref{SI_eqn.D_B}) can be rewritten in terms of conditional probabilities of the outcomes of the state preparation process, i.e properly initialized to $\ket{0}_{\rm in}$ ($P(\ket{0}_{\rm in})$), improperly initialized to the wrong state ($\epsilon_{\rm prep}$), or lost before the mid-circuit measurement characterization circuit ($\epsilon_{l, \rm pre}$).  This probability is:  


\begin{eqnarray}
    \label{SI_eq.probs}
    P(\ket{0}_{\rm out}) &=& P(\ket{0}_{\rm out}|\ket{0}_{\rm in})P(\ket{0}_{\rm in}) \nonumber \\
                         &=& \left[ 1- P(\ket{1}_{\rm in}) - P(\epsilon_{\rm prep})- P(\epsilon_{l, \rm pre}) \right] \nonumber \\
                         &=& P(\ket{0}_{\rm out}|\ket{0}_{\rm in})(1-\epsilon_{l \rm, pre}-\epsilon_{\rm prep}) \\
    \nonumber \\
    \label{SI-eq.p1_out}
    P(\ket{1}_{\rm out}) &=& 1- P(\ket{0}_{\rm out})- P(\ket{3,-1}_{\rm out}) \nonumber \\
                         &-& P(\ket{4,-1}_{\rm out}) - P(\epsilon_{\rm prep})-P(\epsilon_{l, \rm pre}) \nonumber \\
                         \\
    \nonumber \\
    P(\ket{3,-1}_{\rm out}) &=& P(\epsilon_{\rm sh,3}|\ket{0}_{\rm in})P(\ket{0}_{\rm in}) \nonumber \\
                         &\approx& P(\epsilon_{\rm sh,3}|\ket{0}_{\rm in}) \nonumber  \\
                         &\equiv& \epsilon_{\rm sh,3}(\psi)\\
    \nonumber \\
    P(\ket{4,-1}_{\rm out}) &=& P(\epsilon_{\rm sh,4}|\ket{0}_{\rm in})P(\ket{0}_{\rm in}) \nonumber \\
                         &\approx& P(\epsilon_{\rm sh,4}|\ket{0}_{\rm in}) \nonumber  \\
                         &\equiv& \epsilon_{sh,4}(\psi)\\
    \nonumber \\    
    P(\epsilon_{\rm prep}) &\equiv& \epsilon_{\rm prep} \\
    \nonumber \\
    \label{SI_eq.loss_pre}
    P(\epsilon_{l, \rm pre}) &\equiv& \epsilon_{l, \rm pre},
\end{eqnarray}

where $\epsilon_{\rm sh,3}$ ($\epsilon_{\rm sh,4}$) represents the probability that the state was correctly prepared in $\ket{0}$ but an error in unshelving caused the state to transferred to $\ket{3, -1}$ ($\ket{4, -1}$), $P(\ket{1}_{\rm out}|\ket{0}_{\rm in})$ is the probability that the atom was correctly prepared in $\ket{0}$ but incorrectly transferred to $\ket{1}$ due to an error in the mid-circuit calibration circuit. The conditional probability $P(\ket{0}_{\rm out}|\ket{0}_{\rm in})$ is the probability that we measure $\ket{0}$ for the output state given that the qubit was correctly initialized to $\ket{0}$; this probability represents the SPAM corrected state fidelity that we want to calculate.

We can further simplify eq. (\ref{SI-eq.p1_out}) by plugging in the other probabilities from eqs. (\ref{SI_eq.probs})-(\ref{SI_eq.loss_pre}):
\begin{eqnarray}
        P(\ket{1}_{\rm out}) &=& 1- P(\ket{0}_{\rm out}|\ket{0}_{\rm in})(1-\epsilon_{l \rm, pre}-\epsilon_{\rm prep})\nonumber \\
        &-& \epsilon_{\rm sh,3}(\psi) - \epsilon_{sh,4}(\psi) -  \epsilon_{\rm prep}-\epsilon_{l, \rm pre}
\end{eqnarray}



Plugging these probabilities and conditional probabilities into eq. (\ref{SI_eqn.D_B}) yields:
\begin{eqnarray}
    P(D_{B}) & \approx &  (1-\epsilon_{l \rm,post})P(\ket{0}_{\rm out}|\ket{0}_{\rm in}) \nonumber\\
                    & \times &(1-\epsilon_{l \rm, pre}-\epsilon_{\rm prep}) \nonumber\\
                      & + & \epsilon_{BA}P(\ket{1}_{\rm out})  \nonumber\\
                      & + &  (1-\epsilon_{l \rm,post}) \epsilon_{sh,4}(\psi) \nonumber\\
                      & + & \epsilon_{BA}\epsilon_{sh,4}(\psi) \nonumber\\
                      & + &  (\epsilon_{MCR, 4}+\epsilon_{MCR, 3})\epsilon_{\rm prep}.
\end{eqnarray}

All $\epsilon$ terms represent small error probabilities, so approximating to only first order errors, can further simplify to:

\begin{eqnarray}
\label{SI_eqn.DB_simp}
    P(D_{B}) & \approx &  P(\ket{0}_{\rm out}|\ket{0}_{\rm in})\nonumber\\
                      & \times &(1-\epsilon_{l \rm, pre}-\epsilon_{l \rm,post}-\epsilon_{\rm prep}) \nonumber\\
                      & + & \epsilon_{BA}\left[ 1- P(\ket{0}_{\rm out}|\ket{0}_{\rm in}) \right]+\epsilon_{sh,3}(\psi) 
\end{eqnarray}

Then solving for the SPAM corrected error, $ P(\ket{0}_{\rm out}|\ket{0}_{\rm in})$, yields:

\begin{equation}
\label{SI_eqn.SPAM_C}
   P(\ket{0}_{\rm out}|\ket{0}_{\rm in})  \approx  \frac{P(D_{B})-\epsilon_{BA}- \epsilon_{sh,3}(\psi)}{1-\epsilon_{l \rm, pre}-\epsilon_{l \rm,post}-\epsilon_{\rm prep}-\epsilon_{BA}}.
\end{equation}

These values can be determined from three different measurements. Two measurements are needed to determine the state preparation error probability, $\epsilon_{\rm prep}$. For convenience, we break state preparation errors into two parts, the probability that a state is incorrectly prepared in $\ket{f=3, m \neq 0}$, $\epsilon_{3, \rm prep}$, and the probability that a state is incorrectly prepared in $\ket{f=4, m \neq 0}$, $\epsilon_{4, \rm prep}$. Note, that for this analysis, we assume, for reasons listed above, that the probability of erroneously preparing $\ket{4,0}$ is negligible.  To determine, $\epsilon_{3, \rm prep}$, we prepare $\ket{3,0}$ by optically pumping into $\ket{4,4}$, then coherently transfer the atoms to $\ket{3,0}$ using CORPSE pulses. We then perform a state-selective measurement by performing a blowaway and measure the array occupation.  The probability that a data qubits is retained, $ R_{3\rm prep}$, is then:

\begin{eqnarray}
     R_{3\rm prep}&=&(1-\epsilon_{l \rm pre}-\epsilon_{l \rm post}) \nonumber \\
                     &-&\left[1-\epsilon_{BA}(1-\epsilon_{l \rm pre}-\epsilon_{l \rm post}) \right] \epsilon_{4, \rm prep} \nonumber \\
                     &\approx& 1-\epsilon_{l \rm pre}-\epsilon_{l \rm post}-\epsilon_{4, \rm prep}.
\end{eqnarray}

If we perform the same experiment with the addition of a (approximately perfect) microwave rotation from $\ket{3,0}$ to $\ket{4,0}$ then we find a retention probability, $ R_{4\rm prep}$ of: 

\begin{eqnarray}
     R_{4\rm prep}&=&(1-\epsilon_{l \rm pre}-\epsilon_{l \rm post})\epsilon_{3, \rm prep}+ \nonumber \\
                     &-&\epsilon_{BA}(1-\epsilon_{l \rm pre}-\epsilon_{l \rm post})(1-\epsilon_{3, \rm prep}) \nonumber \\
                     &\approx& \epsilon_{3, \rm prep} + \epsilon_{BA}.
\end{eqnarray}
We observe that the denominator in eq. (\ref{SI_eqn.SPAM_C}) can be written as $R_{3\rm prep}-R_{4\rm prep}$. 

One final measurement is needed to determine $\epsilon_{sh,3}(\psi)$. We do not have a straightforward way of directly determining this value since this error since it cannot be easily separated from the other errors arising from the mid-circuit measurement. Furthermore, this error is state dependent, so we would need to disentangle it for every input. What we can do is find an average upper-bound on this error. This will make our SPAM-corrected fidelity a conservative estimate since larger $\epsilon_{sh,3_{\rm ave}}$ results in a lower average fidelity.  We can take this estimate from the minimum value on the Ramsey curve measured shown in figure \ref{fig.Data}a. To estimate how this data relates to $\epsilon_{sh,3}(\psi=x)$ (where $x=(\ket{0}+\ket{1})/\sqrt{2}$), we can use eq. (\ref{SI_eqn.DB_simp}) and set $P(\ket{0}_{\rm out}|\ket{0}_{\rm in})=0$. Then solving for $\epsilon_{sh,3}(\psi=x)$, we find:

\begin{equation}
      \epsilon_{sh,3_{\rm ave}}=\epsilon_{sh,3}(\psi=\ket{x}) \leq P(D_{B,\rm min})-\epsilon_{BA},
\end{equation}

where $P(D_{B,\rm min})$ is the minimum retention probability of the Ramsey curve. Note that this is a conservative over-estimation of the $\epsilon_{sh,3}(\psi=x)$ because various errors in the mid-circuit readout such as dephasing might increase $P(\ket{1}_{\rm out}|\ket{0}_{\rm in})$. An over estimation of $\epsilon_{sh,3}(\psi=x)$ is conservative for the fidelity measurement because this the probability that an error during the mid-circuit measurement might make the fidelity look higher than it actually is, and is subtracted off the true fidelity to find the SPAM corrected fidelity. Note that this value should also be equivalent to the $\epsilon_{sh,3 \rm ave}$, it gives equal weighting to $\ket{0}$ and $\ket{1}$ inputs and in the average process fidelity measurement, the two clock state inputs have equal weighting. Note that this type of error is cause by unshelving errors causing leakage out of the qubit basis and will not depend on a relative phase between clock state inputs.

We can now rewrite eq. (\ref{SI_eqn.SPAM_C}) in terms of experimentally measure quantities:

\begin{equation}
       P(\ket{0}_{\rm out}|\ket{0}_{\rm in}) \geq \frac{P(D_B)-P(D_{B,\rm min})}{R_{3,prep}-R_{4,prep}},
\end{equation}
We have experimentally measured the input values: $P(D_{B,\rm min})=0.01(1)$, $R_{3,prep}=0.970(3)$, and $R_{4,prep}=0.014(2)$. A summary of all data qubit raw and SPAM corrected fidelities are summarized in table \ref{SI_table.2}.  Averaging this quantity for all 6 inputs yields the average process fidelity on the data qubits by the mid-circuit measurement. 

\begin{table}[!t]
\caption{Raw and state preparation and measurement (SPAM) corrected fidelity measurements of the data qubits. We have assumed that $\epsilon_{sh,3}(\psi)$, the probability that the mid-circuit measurement incorrectly moves and input state $\psi$ correctly prepared in the logical qubit basis to a state in $f=3$ where $m_f \neq 0$, is the same for the two inputs $\ket{0}$ and $\ket{1}$.  Note that this assumption is not needed for the average fidelity reported in the final row since on average the inputs have an equal weighting of both terms.}\label{SI_table.2}
\centering
\begin{tabular}{|c|c|c| }

 \hline
   Input state  &   Raw fidelity &   SPAM-corrected fidelity\\
\hline
$(\ket{0}+\ket{1})/\sqrt{2}$ &  93.0(8)\% &  96.2(1.2)\%\\
$(\ket{0}-\ket{1})/\sqrt{2}$ &  94.1(8)\% & 97.4(1.3)\%\\
$(\ket{0}+i\ket{1})/\sqrt{2}$ &  93.4(9)\% & 96.6(1.3)\%\\
$(\ket{0}-i\ket{1})/\sqrt{2}$ &  93.4(1.0)\% & 96.6(1.4)\%\\
$\ket{0}$ &  94.0(4)\% & 97.2(1.3)\%\\
$\ket{1}$ &  94.6(3)\%  & 97.9(1.3)\%\\
\hline
Average & 93.8(3)\% & 97.0(5)\%\\

\hline
\end{tabular}
\end{table}

\subsection{Correction of state preparation errors on  ancilla qubits}

A similar process may be used to compensate mid-circuit measurements on the ancilla qubit for state preparation errors. The raw ancilla qubit measurement characterization was performed in two experiments.  In the first, $\ket{0}$ was prepared, and then a mid-circuit measurement was performed. In the ideal case, $\ket{0}$ would be detected as a dark state by our mid-circuit measurement procedure. We can write the probability of a dark state detection as:
\begin{eqnarray}
 \label{SI_eqn.P1D}
    P_1(D) & = & P(D|\ket{0})P_1(\ket{0}_{\rm in}) \nonumber\\
                   & + & P(D|\epsilon_{l \rm, pre})P(\epsilon_{l \rm, pre}) \nonumber\\
                   & + & P(D|\epsilon_{\rm prep})P(\epsilon_{\rm prep}),
\end{eqnarray}
where $P_1(D)$ is the probability that a dark state is registered by the mid-circuit measurement in this first experiment, $P(D|\ket{0})$ is the probability that a dark state is detected given a $\ket{0}$ input, $P(\ket{0}_{\rm in})$ is the probability that the state $\ket{0}$ is correctly prepared in this first experiment, $P(D|\epsilon_{l \rm, pre})$ is the probability of registering a dark state given loss of the atom before the mid-circuit measurement, $P(\epsilon_{l \rm, pre})$ is the probability that the atom is lost before the mid-circuit measurement, $P(D|\epsilon_{\rm prep})$ is the probability of registering a dark state if the atom in not prepared in the correct state, and $P(\epsilon_{\rm prep})$ is the probability of a state preparation error. Note that for this analysis, for the same reasons as listed in the previous section, we have assumed that a state preparation error resulting in the $\ket{1}$ has negligible probability. 

A second experiment was then performed by preparing the input $\ket{0}$, then performing a clock state rotation to $\ket{1}$, and then performing the mid-circuit measurement. Ideally, a $\ket{1}$ input would be registered as a bright state by the mid-circuit readout procedure. We can write the probability of detecting a bright state in this case as:
\begin{eqnarray}
    \label{SI_eqn.P2B}
    P_2(B) & = & P(B|\ket{1})P_2(\ket{1}_{\rm in}) \nonumber\\
                   & + & P(B|\epsilon_{\rm{loss}})P(\epsilon_{\rm{loss}}) \nonumber\\
                   & + & P(B|\epsilon_{\rm prep})P(\epsilon_{\rm prep}),
\end{eqnarray}
where $P_2(B)$ is the probability that a bright state is registered by the mid-circuit measurement in this second experiment, $P(B|\ket{1})$ is the probability that a bright state is detected given a $\ket{1}$ input, $P(\ket{1}_{\rm in})$ is the probability that the state $\ket{1}$ is correctly prepared in this second experiment, $P(B|\epsilon_{\rm{loss}})$ is the probability that a bright state is registered even though the atoms is lost prior to the mid-circuit measurement, and $P(B|\epsilon_{\rm prep})$ is the probability that a dark count is registered when there is a state preparation error. Note that since we assumed that the probability of erroneously preparing $\ket{1}$ in first experiment was negligible and we have assumed that nearly perfect clock state rotations, it follows that the probability of erroneously preparing $\ket{0}$ in the second experiment is negligible. For the same reasons we may identify that: 
\begin{equation}
P_1(\ket{0}_{\rm in})=P_2(\ket{1}_{\rm in}) = 1-\epsilon_{l,\rm pre}-\epsilon_{\rm prep}.    
\end{equation}
We may also identify that:
\begin{equation}
    P(D|\epsilon_{l \rm, pre}) = 1-P(B|\epsilon_{l \rm, pre})\equiv 1-\epsilon_{l,\rm xtalk},
\end{equation}
where $\epsilon_{l,\rm xtalk}$ represents the probability the a lost atom will be registered as a bright state. Note that $\epsilon_{l,\rm xtalk}$ should be small since an atom that is lost from the trap will not fluoresce and such mislabeling arises only from overlap in the histogram distributions of no atom and atom detected in the bright state.  Furthermore, we observe:
\begin{equation}
    P(B|\epsilon_{\rm prep}) = 1-P(D|\epsilon_{\rm prep}) \equiv 1-\epsilon_{D,\rm prep},
\end{equation}
where $\epsilon_{D,\rm prep}$ represents the probability that an atom prepared in the wrong state will be registered as a dark state. The probability should also be small because atoms prepared in $\ket{4,m>0}$ will be pumped to $\ket{4,4}$ by the readout light just as atoms in $\ket{1}$ are. Similarly, the periodic microwave repumping will result in atoms incorrectly prepared in $\ket{3,m>0}$ undergoing transitions to the $f=4$ manifold, where they too will be pumped to $\ket{4,4}$ by the readout light.  We may now simplify eqs. (\ref{SI_eqn.P1D}): 
\begin{eqnarray}
 \label{SI_eqn.P1D_simp}
    P_1(D) & = & P(D|\ket{0})P_1(\ket{0}_{\rm in}) \nonumber\\
                   & + & (1-\epsilon_{l,\rm xtalk})\epsilon_{l \rm, pre} \nonumber\\
                   & + & \epsilon_{D,\rm prep}\epsilon_{\rm prep} \nonumber\\
                   & \approx & P(D|\ket{0})(1-\epsilon_{l \rm, pre}-\epsilon_{\rm prep}) \nonumber \\
                   & + & \epsilon_{l \rm, pre}
\end{eqnarray}
and \ref{SI_eqn.P2B}
\begin{eqnarray}
    \label{SI_eqn.P2B_simp}
    P_2(B) & = & P(B|\ket{1})P_2(\ket{1}_{\rm in}) \nonumber\\
                   & + & \epsilon_{l,\rm xtalk}\epsilon_{\rm{l \rm, pre}} \nonumber\\
                   & + & (1-\epsilon_{D,\rm prep})\epsilon_{\rm prep} \nonumber\\
                   & \approx & P(B|\ket{1})(1-\epsilon_{l \rm, pre}-\epsilon_{\rm prep}) \nonumber\\
                   & + & \epsilon_{\rm prep}.
\end{eqnarray}
In these equations, we have kept first order terms in $\epsilon$. We can now solve for the state preparation corrected fidelities $P(D|\ket{0})$ and $P(B|\ket{1})$:
\begin{eqnarray}
    \label{SI_eqn.P2B_simp2}
    P(D|\ket{0}) & = &\frac{P_1(D)-\epsilon_{l \rm, pre}}{1-\epsilon_{l \rm, pre}-\epsilon_{\rm prep}} \\
                   \nonumber\\
    P(B|\ket{1}) & = &\frac{P_2(B)-\epsilon_{\rm prep}}{1-\epsilon_{l \rm, pre}-\epsilon_{\rm prep}} \\              
\end{eqnarray}

We wish to express $P(D|\ket{0})$ and $P(B|\ket{1})$, the compensated detection probabilities, in terms of experimentally measured quantities. To do this, we had to perform measurements to estimate $P(\epsilon_{l \rm, pre})$, $P(\epsilon_{\rm prep})$.  These were characterized in four experiments. Two of the experiments, $ R_{3\rm prep}$ and $R_{4\rm prep}$, were the same as those performed in in the last subsection. In addition, we performed an experiment to find the loss present before the mid-circuit measurement by performing a measurement on the atoms followed by a second measurement. The probability that the atom was detected in the second measurement is:
\begin{equation}
     R_{\rm base}=1-\epsilon_{l \rm pre}-\epsilon_{l \rm post}.
\end{equation}
If we assume that $\epsilon_{l \rm pre} \approx \epsilon_{l \rm post}$ then:
\begin{equation}
     \epsilon_{l, \rm pre} \approx \frac{1-R_{\rm base}}{2}
\end{equation}

To disentangle blowaway from $R_{3\rm prep}$ and $R_{4\rm prep}$, we performed a measurement of the blowaway by preparing $\ket{4,4}$ with optical pumping and then performing a blowaway and readout. During our optical pumping we had an excess of repump light. We also measured a depumping to pumping ratio of $>200$. These to facts together imply that there will be a negligible amount of atom population left in the $f=3$ manifold. The resulting atom retention probability for this experiment is:
\begin{equation}
     R_{BA} = \epsilon_{BA}(1-\epsilon_{l, \rm pre}-\epsilon_{l, \rm post})\approx \epsilon_{BA}.
\end{equation}
We used all four experimental results to determine $\epsilon_{\rm prep}$:
\begin{equation}
    \epsilon_{\rm prep} = R_{4 \rm prep}-R_{3 \rm prep}+R_{\rm base}-R_{BA}
\end{equation}
We can now express the state preparation corrected mid-circuit measurement fidelities as:
\begin{eqnarray}
    P(D|\ket{0}) & = & \frac{P_1(D)-\frac{1}{2}(1-R_{\rm base})}{\frac{1}{2}-R_{4,\rm prep}+R_{3,\rm prep}-\frac{R_{\rm base}}{2} + R_{BA}} \nonumber\\
    \\
    P(B|\ket{1}) & = & \frac{P_2(B)-R_{4, \rm prep}+R_{3, \rm prep}-R_{\rm base}+R_{BA}}{\frac{1}{2}-R_{4,\rm prep}+R_{3,\rm prep}-\frac{R_{\rm base}}{2} + R_{BA}}. \nonumber\\
\end{eqnarray}
Plugging in measured values ($P_1(D)=0.936(5)$, $P_2(B)=0.943(5)$, $R_{\rm base}=0.977(5)$, $R_{4,\rm prep}=0.020(2)$, $R_{3,\rm prep}=0.977(5)$, $R_{BA}=0.005(2)$), we find state preparation compensated detection fidelities of $P(D|\ket{0})=0.949(8)$ and $P(B|\ket{1})=0.953(11)$.



% \subsection{Ancilla heating during mid-circuit measurement} \tg{Do we really need this section? These heating mechanisms are discussed in the Meschede paper and in Rb nd readout paper}
%     I. Atom heating\\
%         1. Heating from photon recoil\\
%         2. Heating from dipole force fluctuations\\

\section{Qubit resetting}
\label{sec.qubit_reset}

\subsection{Quadrupole recooling}
In order for the mid-circuit measurement protocol to be compatible with multiple rounds of error correction, it is necessary to recool ancilla qubits after and during the state measurement. In order to realize good two-qubit gate fidelity, the ancilla qubits should be recooled to under $10~\mu\rm K$. This recooling must be done without affecting the quantum state of shelved data qubits, and therefore requires a cooling protocol that does not couple to atoms in the lower $f=3$ hyperfine level. This requirement can be met by using narrow line cooling on the Cs quadrupole transition $6s_{1/2},f=4 \rightarrow 5d_{5/2},f=6$ at 685 nm. Preliminary measurements of laser cooling on this transition were reported in \cite{Carr2014t} and it has also been proposed for implementation of a compact optical atomic clock\cite{Sharma2022}. Here we analyze an alternative version of 685-nm cooling using a single pair of counter-propagating cooling beams based on the Sisyphus cooling techniques proposed in \cite{Taieb1994, Ivanov2011} and demonstrated in Sr neutral atoms \cite{Covey2019a}. 


\begin{figure}[!t]
%\vspace{.1in} 
\includegraphics[width=1.5in]{figures/SI_sisyphus_cooling.pdf}
\caption{
\label{SI-fig_Sisyphys}
Sisyphus cooling of a trapped atom. The atom absorbs a photon of frequency, $\omega$, at position, $x_0$ relative to the center of the trap. While the atom is in the excited state, it moves to a new location, $x$, where it emits a photon of frequency $\omega'$. If the excited state is more tightly confined than the ground state and the atoms moves further from the center of the trap, then $\omega < \omega'$ and the atom will experience cooling.
}
\end{figure}

This technique uses a state-dependent trap depth and a narrow-line transition to enable cooling (see Fig. \ref{SI-fig_Sisyphys}). Specifically, a long-lived excited state is trapped more strongly than the ground state. The cooling lasers are tuned so that they excite atoms near the center of the trap up to the excited state. If the excited state has a long enough lifetime, then the atom has time to travel up the potential well before it decays back to the more weakly trapped ground state. Below we provide a semi-classical analysis of this cooling technique in 1D. In this cycle, the net energy change, $\Delta U$, of the atom is:
\begin{equation}
    \Delta U = -\left[ U_e(x_f) -U_e(x_0)\right]+\left[ U_g(x_f) -U_g(x_0)\right],
\end{equation}
where $U_g(x)$ ($U_e(x)$) is the trap potential of the ground (excited) state as a function of the initial (final) position of the atom position, $x_0$ ($x_f$), of the atom in the cooling cycle. Averaging over the time ($t$) the atom spends in the excited state and assuming harmonic potentials yields:
\begin{eqnarray}
\label{SI_eqn.deltaU}
     \delta U &=& -\frac{\int_{0}^{\infty} dt \Delta U e^{-\gamma t}}{\int_{0}^{\infty} dt e^{-\gamma t}} \nonumber \\
            &=& -\frac{\gamma m (\omega_e^2-\omega_g^2)}{2}\int^{\infty}_0 dt \left[ x^2(t) - x^2_0 \right] e^{- \gamma t},
\end{eqnarray}
where $\omega_g$ ($\omega_e$) is the trap frequency of the ground (excited) state, $m$ is the mass of the atom, and $\gamma$ is the decay rate of the excited state. An atom of temperature $T$ starting at $x_0$ at time $t=0$ will move in the excited state potential well with a trajectory:

\begin{equation}
     x_{\pm}= x_0 cos(\omega_e t) \pm \sqrt{x^2_m - x^2_0} sin(\omega_e t),
\end{equation}
where $x_m=\left( \frac{2 k_B T}{m \omega_e^2} \right)^{1/2}$ with $k_B$ representing Boltzmann's constant. Inserting $ x_{\pm}$ into eq. (\ref{SI_eqn.deltaU}) yields:
\begin{eqnarray}
    \delta U_\pm &=& -\frac{m(1-\omega_g^2/\omega_e^2)}{4(1+(\gamma/(2\omega_e))^2)} [\omega_e^2 (x_m^2-2 x_0^2) \nonumber \\
                &\pm& 2x_0\sqrt{x_m^2-x_0^2}\gamma\omega_e ]
\end{eqnarray}
Averaging the energy change then gives:
\begin{equation}
    \delta U = -\frac{m \omega_e^2 (1-\omega_g^2/\omega_e^2)}{4[1+\gamma^2/(4 \omega_e^2)]}(x_m^2-2x_0^2).
\end{equation}
To obtain a fast cooling rate, we need $\omega_e>\omega_g$ and $\gamma/\omega_e$ to be small. Also, an atom will only experience cooling when $x_0$ is close to the center of the trap.  In this technique, the cooling lasers are tuned so that the center of the trap has the maximum excitation rate; however, other positions in the trap have a finite excitation rate as well. To find the mean energy change, $\overline{\delta U}$, we need to average $\delta U$ over the excitation rate, $r(x)$:
\begin{equation}
    r(x)=\frac{\gamma}{2}\frac{I/I_s}{1+4 \left[ \Delta_0 + \Delta(x) \right]}.
\end{equation}
where $I$ is the intensity of the cooling laser, $I_s$ is the saturation intensity of the transition, $\Delta (x)$ is the position-dependent detuning of the of transition (the variation is due to the trap-induced Stark shift), and $\Delta_0$ is the detuning of the cooling lasers relative to the atomic resonance in the center of the trap. We need to weight the average over the atomic position distribution function, $\rho (x)$:
\begin{equation}
    \rho(x)=\frac{\gamma}{\sqrt{2 \pi \sigma}}e^{-x^2/(2\sigma^2)},
\end{equation}
where
\begin{equation}
    \sigma=\frac{w}{2}\sqrt{k_B T/U_{\rm trap}}.
\end{equation}
Here $T$ is the atom temperature, $w$ is the Gaussian beam waist of the trapping tweezer, and $U_{\rm trap}$ is the depth at the center of the trap.  The mean energy change per cooling cycle is then:
\begin{equation}
    \overline{\delta U}=\int_{-\infty}^{\infty} dx\,  \delta U(x) r(x) \rho (x).
\end{equation}
We have evaluated this integral numerically and found that we could achieve a cooling rate $\geq 40~\mu \rm K/ms$ for realistic trap parameters (see Fig. \ref{SI-fig_Sisyphys_sim}).


\begin{figure}[!t]
%\vspace{.1in} 
\includegraphics[width=3.5in]{figures/SI_sisyphus_cooling_rate.pdf}
\caption{
\label{SI-fig_Sisyphys_sim}
Sisyphus cooling rate with a decay rate $\gamma/2\pi=124~\rm kHz$, an excited state trap frequency $\omega_e=2\omega_g$ (where $\omega_g$ is the ground state trap frequency), a cooling beam  intensity $I/I_s=0.2$ (where $I_s$ is the saturation intensity of the transition), a cooling beam detuning from resonance at the center of the trap of $\Delta_0=0$, an atom temperature of $T=10~\mu\rm K$, and a trap depth of $U_{\rm trap}=500~\mu\rm K$ for a Gaussian beam trap with waist $w=1~\mu\rm m$. a) Shows the scattering rate of the cooling beams versus $\omega_g/\gamma$. b) Shows the cooling rate versus $\omega_g/\gamma$ relative to the transverse heating rate from 1D 852 nm molasses at the indicated $r_{\rm mol}$ scattering rate. 
}
\end{figure}


This cooling technique has an advantage over using a six-beam red molasses in that we can configure the cooling beam pair to have a polarization that allows cycling from the $\ket{4,4}$ state without populating other $m_f$ levels. For the transition to be cyclic, the counter-propagating 685-nm cooling beam pair must be aligned such that their $k$-vector is perpendicular to the bias magnetic field and the beams have a polarization perpendicular to the quantization axis (e.g. if $B$  is along the z-axis, then the cooling beam $k$-vectors are along $x$  and $-x$  and polarized along $y$ ). Quadrupole transition selection rules dictate that the possible $\Delta m_f$ of this transition can be -2, 0, or +2. We can suppress $\Delta m_f=-2$ and $0$ by providing a strong B-field that shifts those transitions out of resonance, so that the $f'=6$ $m_{f'}=6$ transition is strongly preferred. The Land\'e g-factor for $5d_{5/2}$ $f'=6$ is $g_f'(f'=6,m_f'=6)=1/2$, so even the modest field of 10.2 G used in this paper will detune the $\Delta m_f=0$ and $\Delta m_f=-2$ transitions by 14.3 MHz and 28.6 MHz respectively, much greater than the 124 kHz linewidth of the transition. The $5d_{5/2}$ $f'=6$ $m_{f'}=6$ state will decay to $6s_{1/2}\ket{4,4}$  via the $6p_{3/2}$ state. The $\ket{4,4}$ state is the same state to which the mid-circuit readout procedure pumps ancilla atoms in the bright state. Therefore, we may directly recool atoms during the mid-circuit readout without affecting the dark state or the shelved qubits. In fact, since the $5d_{5/2}$ state decays to the ground via the $6p_{3/2}$ state, each cooling cycle provides a photon that can be used to help detect the bright state.\\



\subsection{Quantum state reinitialization}
\label{sec.init}

After recooling, any ancilla qubits that were detected as bright during the mid-circuit readout need to be reset from $\ket{4,4}$ back to initial state, $\ket{3,0}$. This may be accomplished in a 2 step process: 1) Use microwave rotations to transfer from $\ket{4,4} \rightarrow \ket{4,1}$ using CORPSE pulses that are detuned from data qubits encoded in $m_f=0$ states; 2) Perform site selective rotations on ancilla qubits, rotating $\ket{4,1} \rightarrow \ket{3,0}$. Step 1) of this resetting process would use the first four of the five CORPSE pulses which were used in our state preparation described in the main text. Step 2) can be performed using methods described in either \cite{Graham2022} or site-selective Raman rotations \cite{Jones2007,Knoernschild2010}. If an ancilla qubit was detected in the dark state, then it will be rotated to $\ket{1}$ during the mid-circuit measurement process; specifically, the final two-step unshelving process will rotate an ancilla in the dark state from $\ket{0}$ to $\ket{1}$.  To reinitialize, a single site-selective clock state rotation rotates the state to $\ket{0}$   

\end{document}

