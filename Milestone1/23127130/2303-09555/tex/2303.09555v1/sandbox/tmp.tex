% \begin{table}[t] 
%     \centering
%     \caption{Benchmark.}
%     % \begin{adjustbox}{width=0.3\columnwidth}
%     \begin{tabular}{c|ccccccc}
%         \toprule
%         % & \multicolumn{7}{c}{\textbf{Movement Speed}} \\
%         {\textbf{Movement Speed}} & Ground & Ice & Wetland & Clay & Desert & Snow & Aquatic \\
%         \midrule
%         Baby Seal & \\
%         Caterpillar & \\
%         Fish & \\
%         Panda & \\
%         \bottomrule
%     \end{tabular}
%     % \end{adjustbox}
%     \label{tab:benchmark}
% \end{table}

% \begin{table}
%     \begin{minipage}{0.3\linewidth}
%         \caption{Representation of design space.}
%         \label{tab:design_repr}
%         \centering
%         \begin{tabular}{lc}
%             \toprule
%             Representation   & Performance  \\
%             \midrule
%             Particle          &   \\
%             Voxel             &   \\ 
%             \makecell[l]{Implicit\\Function} &   \\
%             SDF Basis         &   \\
%             \makecell[l]{Wasserstein\\Barycenter}     &  \\
%             \bottomrule
%         \end{tabular}
%     \end{minipage}\hfill
%     \begin{minipage}{0.7\linewidth}
%         \centering
%         \includegraphics[width=0.7\linewidth]{example-image}
%         \captionof{figure}{Design space representation}
%     \end{minipage}
% \end{table}


% Contribution:
% \begin{itemize}
%     \item A benchmark of soft robot co-design for locomotion in diverse environments.
%     \item Demonstrate the importance of search space design from the perspective of 3D representation learning.
%     \item Study the role of differentiable physics.
%     \item Interplay between morphology, policy learning, and emergent motions.
% \end{itemize}


% \begin{table}
%     \begin{minipage}{0.3\linewidth}
%         \caption{Representation of design space.}
%         \label{tab:design_repr}
%         \centering
%         \begin{tabular}{lc}
%             \toprule
%             Representation   & Performance  \\
%             \midrule
%             Particle          & 0.027 \\
%             Voxel             & 0.023 \\
%             \midrule
%             Implicit Function & 0.113 \\
%             Diff-CPPN         & 0.091 \\
%             \midrule
%             SDF Basis         & 0.152 \\
%             Wass-Barycenter     & 0.158 \\
%             \bottomrule
%         \end{tabular}
%     \end{minipage}\hfill
%     \begin{minipage}{0.6\linewidth}
%         \centering
%         \includegraphics[width=1\linewidth]{asset/tmp/shape_demo}
%         \captionof{figure}{Design space representation}
%     \end{minipage}
% \end{table}

% \begin{table}
%     \caption{Representation of design space.}
%     \label{tab:design_repr}
%     \centering
%     \begin{tabular}{l|cc}
%         \toprule
%         Representation  & Structural Prior & Performance  \\
%         \midrule
%         Particle          & None & 0.027 \\
%         Voxel             & None & 0.023 \\
%         \midrule
%         Implicit Function & Spatial Smoothness & 0.113 \\
%         Diff-CPPN         & Spatial Smoothness & 0.091 \\
%         \midrule
%         SDF Basis         & Design Basis & 0.152 \\
%         Wass-Barycenter     & Design Basis & 0.158 \\
%         \bottomrule
%     \end{tabular}    
% \end{table}

% \begin{figure}
%     \centering
%     \includegraphics[width=1\linewidth]{asset/tmp/shape_demo}
%     \captionof{figure}{Design space representation}
% \end{figure}

% Inspired by nature.
% IMPORTANT Why virtual environment? driving force for research. (Rapid testbed)

% Many living creatures manifest different behaviors in response to characteristics or challenges of their environments. More interestingly, the strategies developed to survive their habitats are often tightly connected to their morphological structure. Such an intertwined relationship between environment, morphology, and behavior can serve as an inspiring perspective of developing artificial agents with an unified consideration of both body and brain. To this end, soft robotics community has demonstrated, beyond merely policy learning, the potential of leveraging body shape and material properties to achieve a wide variety of tasks. In this work, we introduce a soft robot co-design (learning / optimization in both robot body and policy) platform for locomotion in diverse environments with a differentiable physic engine. We demonstrate a fascinating interplay between environment, robot body, and motion when achieving various locomotion tasks. Furthermore, we identify and investigate several challenges in soft robot co-design from the view of design space representation, ambiguity between design and control, and the role of differentiable physics. We envision the proposed platform will encourage an distinctive way of approaching policy learning and the development of novel representations and learning algorithms for soft robot co-design.

% Story:
% \begin{itemize}
%     \item Nature has incubated a wide variety of creatures that exhibit different levels of intelligence and distinct behaviors in order to survive and thrive.
%     \item How embodied agents interact with the environment depends on not only the brain but also the body, i.e., certain emergent behavior may be unique to certain morphology.
%     \item Further highlight the inter-dependency on environment.
%     \item Overall, sell the story that highlights the interplay between environment, morphology, policy learning, and emergent behaviors.
%     \item In this work, we focus on soft robot co-design for locomotion in diverse environments. Explain why focusing on soft-bodied agents? And why locomotion?
% \end{itemize}

% Contribution:
% \begin{itemize}
%     \item We introduce a platform of soft robot co-design with a differentiable physics engine for locomotion in diverse scenarios, including ground, ice, wetland, clay, desert, snow, and aquatic environments.
%     \item We demonstrate the interplay between environment, morphology, and robot motion.
%     \item We identify and study the challenges in soft robot co-design from various perspectives, including design space representation, ambiguity between design and control, and the role of differentiable physics.
% \end{itemize}



% Benchmark of diverse locomotion:
% \begin{itemize}
%     \item Figures for all environments.
%         \begin{itemize}
%             \item Regular ground.
%             \item Icy.
%             \item Snow.
%             \item Desert (sand).
%             \item Wetland (mud).
%             \item Swamp (plasticine).
%             \item Rubber (jelly).
%             \item Aquatic.
%             \item Subterrain.
%         \end{itemize}
%     \item Introduce tasks and metrics.
%         \begin{itemize}
%             \item Task: Going along a fixed direction. Metric: Average speed along the direction.
%             \item Task: Moving in circles. Metric: Average angular velocity.
%             \item Task: Waypoint following. Metric: Average displacement error.
%             \item Task: Velocity tracking. Metric: Average speed along the varying direction.
%             \item General metric: efficiency.
%         \end{itemize}
%     \item Different animals with manually-annotated muscle and report performance of (i) different environments (ii) different policy learning methods.
%         \begin{itemize}
%             \item Policy architecture: linear combination with bias on periodic function bases with different frequencies and phases.
%             \item Animals: Caterpillar, Earthworm, Walrus/Seal, Panda, Marmot, Fish.
%             \item Policy learning methods: RL, differentiable physics.
%         \end{itemize}
% \end{itemize}

% Interplay between morphology, policy learning, and emergent motions:
% \begin{itemize}
%     \item Qualitative analysis on benchmarking on different animals.
%         \begin{itemize}
%             \item Track 3D motion (trajectory) of different muscle group.
%             \item Measure resistance on robot body in forward direction.
%         \end{itemize}
%     \item Identify different types of motions.
%         \begin{itemize}
%             \item Crawling.
%             \item Swimming.
%             \item Running.
%             \item Hopping.
%         \end{itemize}
%     \item Discuss connection between locomotion behaviors and environments.
% \end{itemize}

% Search space design based on 3D representation learning:
% \begin{itemize}
%     \item In aquatic environments, consider (i) passive dynamics (ii) fixed periodic controller and muscle.
%     \item Consider task of moving forward with metrics including forward speed and efficiency.
%     \item Use differentiable physics for optimization.
%     \item Compare shape search space with different structural priors:
%         \begin{itemize}
%             \item Particle-based representation.
%             \item Voxel-based representation.
%             \item MLP implicit function.
%             \item Diff-CPPN.
%             \item Wasserstein barycenter.
%             \item Base shape SDF interpolation.
%         \end{itemize}
%     \item Include task performance and learning efficiency.
% \end{itemize}

% Challenges in muscle placement and actuation:
% \begin{itemize}
%     \item Highlight the ambiguity between muscle placement and control in optimization.
%     \item Show muscle stiffness is a static version of actuation.
%     \item Show that with fixed shape, we can roughly approximate arbitrary muscle placement by redundant muscle setup with soft actuation mapping.
% \end{itemize}

% Study the role of differentiable physics:
% \begin{itemize}
%     \item Briefly discuss how differentiable physics is studied in the context of policy learning (no need of experiments since many other papers already did a lot of analysis).
%     \item Show loss landscape with respect to a single parameter (of morphology) in worst case,
%         \begin{itemize}
%             \item Shape: A box can fall to the right or to the left due to a single piece of corner being absent or not.
%             \item Softness: With the same contraction behavior, softer body can jump higher to pass a stump and show that there is a cusp between whether you can successfully pass the stump; this is a bit similar to discontinuity mentioned in the ICML diff-sim for control paper (from Russ's group) but here we highlight that controller depends on robot design and this is a similar observation yet from different angle.
%         \end{itemize}
% \end{itemize}

% Comparison of co-design methods:
% \begin{itemize}
%     \item Consider full co-design pipeline, including .
%     \item Focus on two environments: (i) regular land (ii) aquatic. 
%     \item Methods:
%         \begin{itemize}
%             \item CPPN-NEAT.
%             \item RL + NN policy and designer.
%             \item Function basis + differentiable physics.
%         \end{itemize}
%     \item This probably won't yield super meaningful results since it is probably too computationally intractable to actually do straightforward computational co-design in 3D simulation with complex environments and robot body. This is more of making the paper complete and we leave this to future exploration.
% \end{itemize}


\begin{table}[t] 
    \centering
    \caption{Large-scale benchmark of biologically-inspired designs in \namenospace. Each entry shows results from differentiable physics (left) and RL (right). The higher the better.}
    \vspace{-1mm}
    \begin{adjustbox}{width=1\columnwidth}
    \begin{tabular}{c|c|ccccccc}
        \toprule
        \multirow{2}{*}{\textbf{Task}} & \multirow{2}{*}{\textbf{Animal}} & \multicolumn{7}{c}{\textbf{Environment}} \\
        & & \textbf{Ground} & \textbf{Ice} & \textbf{Wetland} & \textbf{Clay} & \textbf{Desert} & \textbf{Snow} & \textbf{Ocean} \\
        \midrule 
        \multirow{4}{*}{\makecell{Movement\\Speed}} & Baby Seal & \textbf{0.122} / 0.241 & \textbf{0.048} / 0.013 & 0.032 / 0.026 & 0.012 / 0.007 & 0.059 / 0.019 & 0.039 / 0.011 & 0.033 / 0.003  \\
        & Caterpillar & 0.080 / 0.068 & 0.023 / 0.002 & \textbf{0.052} / 0.018 & 0.032 / 0.004 & 0.053 / 0.038 & 0.047 / 0.009 & 0.134 / \textbf{0.174} \\
        & Fish & 0.053 / 0.025 & 0.029 / 0.012 & 0.026 / 0.014 & \textbf{0.037} / 0.018 & \textbf{0.115} / 0.015 & \textbf{0.087} / 0.074 & 0.084 / 0.128 \\
        & Panda & 0.046 / 0.026 & 0.038 / 0.000 & 0.016 / 0.008 & 0.019 / 0.005 & 0.023 / 0.008 & 0.031 / 0.006 & 0.024 / 0.007 \\
        \midrule
        \multirow{4}{*}{Turning} & Baby Seal & 0.067 / \textbf{0.083} & \textbf{0.058} / 0.014 & 0.014 / 0.009 & 0.021 / 0.016 & \textbf{0.051} / 0.009 & 0.047 / 0.018 & 0.059 / 0.052 \\
        & Caterpillar & 0.053 / 0.016 & 0.021 / 0.011 & \textbf{0.040} / 0.007 & \textbf{0.027} / 0.005 & 0.034 / 0.005 & \textbf{0.069} / 0.015 & \textbf{0.195} / 0.115 \\
        & Fish & 0.032 / 0.064 & 0.023 / 0.005 & 0.023 / 0.009 & 0.015 / 0.004 & 0.021 / 0.013 & 0.028 / 0.014 & 0.041 / 0.020 \\
        & Panda & 0.064 / 0.037 & 0.023 / 0.004 & 0.009 / 0.004 & 0.008 / 0.001 & 0.014 / 0.003 & 0.013 / 0.003 & 0.035 / 0.034 \\
        \midrule
        \multirow{4}{*}{\makecell{Velocity\\Tracking}} & Baby Seal & 0.343 / 0.501 & 0.257 / 0.136 & 0.249 / 0.235 & 0.231 / 0.246 & 0.290 / 0.261 & 0.215 / 0.223 & 0.379 / 0.252 \\
        & Caterpillar & 0.502 / 0.312 & 0.101 / 0.343 & 0.426 / 0.240 & 0.282 / 0.072 & 0.441 / 0.308 & 0.379 / -0.113 & 0.555 / \textbf{0.739}  \\
        & Fish & 0.216 / 0.382 & 0.416 / 0.204 & 0.221 / 0.271 & 0.234 / \textbf{0.343} & \textbf{0.457} / 0.205 & \textbf{0.638} / 0.330 & 0.657 / 0.455 \\
        & Panda & \textbf{0.575} / 0.415 & \textbf{0.555} / 0.495 & \textbf{0.536} / 0.429 & 0.153 / 0.134 & 0.450 / -0.249 & 0.472 / 0.266 & 0.395 / 0.404 \\
        \midrule
        \multirow{4}{*}{\makecell{Waypoint\\Following}} & Baby Seal & -0.012 / -0.010 & -0.018 / -0.023 & -0.018 / -0.027 & \textbf{-0.020} / -0.026 & -0.012 / -0.027 & -0.014 / -0.025 & -0.010 / -0.026 \\
        & Caterpillar & -0.013 / -0.021 & -0.019 / -0.029 & \textbf{-0.016} / -0.028 & \textbf{-0.020} / -0.027 & -0.013 / -0.026 & -0.018 / -0.028 & \textbf{-0.002} / -0.023 \\
        & Fish & -0.004 / -0.029 & -0.016 / -0.024 & -0.022 / -0.033 & -0.024 / -0.031 & \textbf{-0.007} / -0.032 & \textbf{-0.003} / -0.023 & -0.005 / -0.023 \\
        & Panda & \textbf{-0.003} / -0.014 & \textbf{-0.014} / -0.023 & -0.021 / -0.026 & -0.023 / -0.028 & -0.010 / -0.021 & -0.008 / -0.026 & -0.013 / -0.025 \\
        \bottomrule
    \end{tabular}
    \end{adjustbox}
    \label{tab:benchmark}
    \vspace{-5mm}
\end{table}