\section{Conclusion}
\label{sec:conclusion}

In this work, we introduced \namenospace, a soft robot co-design platform for locomotion in diverse environments. 
%\name supports an extensive, naturally-inspired material set, a variety of locomotion tasks, an unified interface for robot design, and a differentiable physics engine. We benchmark a wide range of representations and learning algorithms, providing a comprehensive understanding of behavioral and morphological intelligence. 
By using \name to investigate the interplay between design representations, environments, tasks, and co-design algorithms, we have found that \textit{1)} emergent motions are driven by environmental and morphological factors in a way that often mirrors nature \textit{2)} injecting design priors in design space representation can improve optimization effectiveness \textit{3)} muscle stiffness and controllers introduce redudnancy in design spaces, and \textit{4)} trapping local minima are common even in very simple morphological design problems.
\noindent\textbf{Impact and Future Direction.} \name provides a well-established platform that allows for systematic training and evaluation of soft robot co-design algorithms; we hope it will accelerate algorithmic co-design development. It also lays a cornerstone for studying morphological and behavioral intelligence under diverse environments. The extensive experiments conducted using \name not only sheds light on the significance of co-design but also identifies concrete challenges from various perspectives. %We envision \name will encourage further research along the line of soft robot co-design. In addition, 
Based on our findings, we suggest several future directions: \textit{1)} 3D representation learning to construct more effective and flexible design space representations \textit{2)} morphology-aware policy learning as an alternative formulation that more elegantly handles the inter-dependency between design and control optimization \textit{3)} principled approaches to combine differentiable physics and gradient-free methods like RL or EAs that marries priors from physics to optimization techniques less susceptible to pathological loss landscapes. Overall, we believe \name paves a road to study morphological and behavioral intelligence, and bridges soft robot co-design with a variety of related research topics.
%, hence making an unique contribution to the broader learning community.

