\begin{abstract}
% Many living creatures manifest different behaviors in response to characteristics or challenges of their environments. More interestingly, the strategies developed to survive their habitats are often tightly connected to their morphological structure. Such an intertwined relationship between environment, morphology, and behavior can serve as an inspiring perspective of developing artificial agents with an unified consideration of both body and brain. To this end, soft robotics community has demonstrated, beyond merely policy learning, the potential of leveraging body shape and material properties to achieve a wide variety of tasks. In this work, we introduce a soft robot co-design (learning / optimization in both robot body and policy) platform for locomotion in diverse environments with a differentiable physic engine. We demonstrate a fascinating interplay between environment, robot body, and motion when achieving various locomotion tasks. Furthermore, we identify and investigate several challenges in soft robot co-design from the view of design space representation, ambiguity between design and control, and the role of differentiable physics. We envision the proposed platform will encourage an distinctive way of approaching policy learning and the development of novel representations and learning algorithms for soft robot co-design.

% \todo{shorten the first sentence. add motivation of virtual environment}

While significant research progress has been made in robot learning for control, unique challenges arise when simultaneously co-optimizing morphology. 
%The fascinatingly intertwined connection between morphology and motion has served as a great inspiration in soft robotics and provides a unique perspective of developing intelligence with a unified consideration of robot body and brain. 
Existing work has typically been tailored for particular environments or representations.  In order to more fully understand inherent design and performance tradeoffs and accelerate the development of new breeds of soft robots, a comprehensive virtual platform --- with well-established tasks, environments, and evaluation metrics --- is needed.  
%Such an environment should be diverse in its ability to simulate various physical phenomena and provide model-based gradients, which are becoming ever-more prevalent in soft robot control and design techniques.  
% Such an environment will allow for rapid iteration on designs as well as the algorithms that generate those designs.
%This urges the need for a comprehensive virtual platform with well-established interface, tasks, and evaluation metric in a wide variety of environments to drive forward development of representations and algorithms.
In this work, we introduce \namenospace, a soft robot co-design platform for locomotion in diverse environments. \name supports an extensive, naturally-inspired material set, including the ability to simulate environments such as flat ground, desert, wetland, clay, ice, snow, shallow water, and ocean.  Further, it provides a variety of tasks relevant for soft robotics, including fast locomotion, agile turning, and path following, as well as 
differentiable design representations for morphology and control.  Combined, these elements form a feature-rich platform for analysis and development of soft robot co-design algorithms.
%and provides a flexible interface of robot shape, stiffness, and muscle placement. The platform is further integrated with a differentiable physic engine to support gradient-based optimization.
%In \namenospace, we observe the interplay between environment, morphology, and motion based on biologically-inspired structures. 
We benchmark prevalent representations and co-design algorithms, and shed light on \textit{1)} the interplay between environment, morphology, and behavior \textit{2)} the importance of design space representations \textit{3)} the ambiguity in muscle formation and controller synthesis and \textit{4)} the value of differentiable physics. We envision that \name will serve as a standard platform and template an approach toward the development of novel representations and algorithms for co-designing soft robots' behavioral and morphological intelligence. Demos are available on our project page\footnote{Project Page: \url{https://sites.google.com/view/softzoo-iclr-2023}}.
\end{abstract}