\section{Experiments}
\label{sec:experiments}




\subsection{Biologically-inspired Morphologies}
\label{ssec:bio_morph}

To study the interplay between environment, morphology, and behavior, we first use a set of animal meshes (selected from {https://www.davidoreilly.com/library}) as robot bodies and  label muscle locations. Through optimizing their controllers, we observe that different designs exhibit better-performing behavior in different environments.

\noindent\textbf{Muscle Annotation.} 
%We select a subset of animal models from an online artist-crafted mesh library\footnote{https://www.davidoreilly.com/library} as robot geometry. 
We 
%then 
implement a semi-automatic muscle annotator that first converts a mesh into a point cloud, secondly performs K-means clustering with user-defined number of body groups, and finally applies principal component analysis on indepdendent clusters to extract muscle direction. 
%The annotator includes an user interface to make manual modification to the annotation process throughout the above steps. 
Users may then fine-tune the resulting muscle placement.
%Overall, we obtain geometry and muscle placement given a mesh.

\noindent\textbf{Animals.} We test four animals: \textit{Baby Seal}, \textit{Caterpillar}, \textit{Fish}, and \textit{Panda}.  These animals are chosen since they exhibit distinct strategies in nature. 
%The muscles are placed according to whether they can generate commonly-seen motion for each animal, e.g., two muscle groups each on the left and right side of a fish to reproduce swimming behavior. 
We refer the readers to \secref{sec:appendix_vis_design} for visualization of each animal-inspired design, including their muscle layouts, and detailed description.

\noindent\textbf{Large-scale Benchmarking.} \tabref{tab:benchmark} shows the performance of the four animals when optimized for each locomotion task, in each environment. We parameterize the controller based on a set of sine function bases with different frequencies, phases, and biases (see \secref{sec:appendix_controller}). We perform control optimization with differentiable physics and RL  \citep{schulman2017proximal}, corresponding to the left/right values in every entry respectively. We would like to emphasize that this experiment is not meant to draw a conclusion on superiority of differentiable physics or RL. Rather, we highlight interesting emergent strategies of different morphologies in response to diverse environments. A We find that the \textit{Baby Seal} morphology is particularly well-suited for \textit{Ground} and \textit{Ice}, the \textit{Caterpillar} morphology is well-suited for  \textit{Wetland} and \textit{Clay}, the \textit{Fish} morphology is well-suited for \textit{Desert} and \textit{Snow}, and both the \textit{Caterpillar} and \textit{Fish} morphologies are well-suited for the \textit{Ocean}. 
%This results demonstrate a promising first step toward studying behavioral and morphological intelligence and their dependency on environments.
% Please check the Appendix for more experimental details.

\noindent\textbf{Environmentally- and Morphologically-driven Motion.} Based on previous quantitative analysis, in \figref{fig:interplay}, we showcase several interesting emergent motion of different animals.  The \textit{Fish} surprisingly demonstrates efficient movement on granular materials (\textit{Desert}), by lifting its torso to reduce friction and pushing itself forward with its tail.
%For \textit{Fish} moving forward in \textit{Desert} (leftmost column), the tailored-for-swimming design surprisingly displays the most efficient movement on granular materials. It lifts its torso to reduce friction and pivots at its tail to generate forward force. 
The \textit{Caterpillar} produces a left-to-right undulation motion to locomote in \textit{Wetland}; the strategy allows it to disperse mud as it moves forward.
%exhibits unique shape and muscle distribution allow left-and-right wiggly movement along heading direction to wipe off mud. 
The \textit{Panda}'s legs act in a spring-like fashion to perform galloping motion on flat terrain.
%\textit{Ground}, its legs act like a spring that stretch asynchronously to achieve galloping-like motion for efficient rotation. 
Finally, the \textit{Baby Seal} uses its tail muscle to create a flapping motion to move on the low-friction \textit{Ice} terrain, where push-off would otherwise be difficult.
%, while the low surface friction prohibits propelling body through tangential force, its distinctive tail muscles realize a flapping motion against the ground to move with a normal force.

\begin{table}
    \begin{minipage}{0.35\linewidth}
        \caption{Quantitative comparison of design space representations.}
        \label{tab:design_repr}
        \centering
        \small
        \vspace{-1mm}
        \begin{tabular}{lc}
            \toprule
            Representation   & Performance  \\
            \midrule
            Particle          & 0.027 \\
            Voxel             & 0.023 \\
            \midrule
            Implicit Function & 0.113 \\
            Diff-CPPN         & 0.091 \\
            \midrule
            SDF-Lerp         & 0.152 \\
            Wass-Barycenter     & 0.158 \\
            \bottomrule
        \end{tabular}
    \end{minipage}
    \hfill
    \begin{minipage}{0.45\linewidth}
        \centering
        \includegraphics[width=1\linewidth]{asset/shape_visualization}
        \vspace{-5mm}
        \captionof{figure}{Visualization of optimized designs.}
        \label{fig:design_repr}
    \end{minipage}
    \vspace{-3mm}
\end{table}

\begin{figure}
     \centering
     \begin{subfigure}[b]{0.35\textwidth}
         \centering
         \includegraphics[width=\textwidth]{./asset/adapt_to_stiffness}
         \vspace{-5mm}
         \caption{Under different stiffness.}
         \label{fig:muscle_ambiguity_a}
     \end{subfigure}
     % \hfill
     \hspace{2mm}
     \begin{subfigure}[b]{0.35\textwidth}
         \centering
         \includegraphics[width=\textwidth]{./asset/adapt_to_placement}
         \vspace{-5mm}
         \caption{Under different muscle placement.}
         \label{fig:muscle_ambiguity_b}
     \end{subfigure}
     \vspace{-1mm}
     \caption{Ambiguity of muscle formation and controller synthesis.}
     \label{fig:muscle_ambiguity}
      \vspace{-6mm}
\end{figure}

\subsection{The Importance Of Design Space Representation}
\label{ssec:design_space_repr}

In this section, we demonstrate the importance of the design space representation for morphology optimization. We apply gradient-based optimization methods across a wide variety of morphological design representations fixing the controller. We focus on achieving fast movement speed in the \textit{Ocean} environment since aquatic creatures tend to manifest a more canonical muscle distribution, typically antagonistic muscle pairs along each side of the body. The low failure rate of this task means that we achieve a wide spread of performances, determined by the design representation, which allows us to compare the effect of each on performance.


%This allows a reasonably-working fixed controller for most possible designs in order to obtain meaningful results (as opposed to randomly initialized controllers that barely generate forward motion even after design optimization). We train a sine-basis model based on a fish inaccessible to all evaluated baselines for the fixed controller.

\noindent\textbf{Baselines.} We implement a variety of design space representations. 
%for geometry, stiffness, and muscle placement. 
We briefly mention the high-level concept of each method as follows. Please refer to \secref{sec:appendix_design_space_repr} for more in-depth description.

\noindent\textit{Particle Representation.} The geometry, stiffness, and muscle placement is directly specified at the particle level, with each particle possessing its own distinct parameterization.

\noindent\textit{Voxel Representation.} This is similar to particle representation but specified at the voxel level.

\noindent\textit{Implicit Function \citep{mescheder2019occupancy}.} We use a shared multi-layer perceptron that takes in particle coordinates and outputs the design specification for the corresponding particle.

\noindent\textit{Diff-CPPN \citep{fernando2016convolution}.} We adapt the differentiable CPPN, which provides a global mapping that converts particle coordinates to a design specification.

\noindent\textit{SDF-Lerp.} Given a set of design primitives (with design specification obtained by using the technique in \secref{ssec:bio_morph}), we compute SDF based on each design primitives for particles in robot design workspace. For each particle, we then linearly interpolate the signed distances and set occupancy for those with negative values to obtain robot geometry.
% we compute SDF of the base particle set with respect to the shape of each design primitive. We then perform linear interpolation using these SDFs and extract subzero level set for robot geometry. 
We directly perform linear interpolation on stiffness and muscle placement of the primitive set. For muscle direction, we use weighted rotation averaging with special orthogonal Procrustes orthonormalization \citep{bregier2021deep}.

\noindent\textit{Wass-Barycenter \citep{ma2021diffaqua}.} We compute Wasserstein barycenter coordinates based on a set of coefficients to obtain robot geometry. We follow \textit{SDF-Lerp} for stiffness and muscle placement. 
%for stiffness and muscle placement as we find the transport map from Wasserstein distance is not sufficiently unstable. We use a fixed muscle direction along the canonical heading direction.

\noindent\textbf{Design Optimization.} In \tabref{tab:design_repr}, we show the results of design optimization. We can roughly categorize these design space representations into (1) no structural prior (\textit{Particle} and \textit{Voxel}) (2) preference for spatial smoothness (\textit{Implicit Function} and \textit{Diff-CPPN}) and (3) highly-structured design basis (\textit{SDF-Lerp} and \textit{Wass-Barycenter}). We observe superior performance as increasing inductive prior is injected into the representation. The use of design primitives casts the problem to composition of existing functional building blocks and greatly reduces the search space, leading to effective optimization. Further, structural priors, lead to designs with fewer thin or sharply changing features; such extreme designs have unwanted artifacts such as numerical fracture during motion. \figref{fig:design_repr} presents optimized designs: meshes indicate geometry and colored point clouds indicate active muscle groups. These results suggest the development of novel representation for robot design that facilitates improved optimization and learning.

\begin{figure}
     \centering
     \begin{subfigure}[b]{0.3\textwidth}
         \centering
         \includegraphics[width=\textwidth]{asset/diffsim_pitfall_actuator.pdf}
         \caption{Multiple local optimum.}
         \label{fig:role_of_diff_physics_a}
     \end{subfigure}
     % \hfill
     \hspace{2mm}
     \begin{subfigure}[b]{0.46\textwidth}
         \centering
         \includegraphics[width=\textwidth]{asset/diffsim_pitfall_shape.pdf}
         \caption{Discontinuity.}
         \label{fig:role_of_diff_physics_b}
     \end{subfigure}
     \vspace{-1mm}
     \caption{Examples of pathological loss landscape for design optimization.}
     \label{fig:role_of_diff_physics}
     \vspace{-6mm}
\end{figure}

\subsection{Ambiguity Between Muscle Stiffness And Actuation Signal Strength}

In this section, we investigate the ambiguity between muscle formation and actuation from the controller. We aim to discover if there exists unidentifiability between design and control optimization. We adopt trajectory optimization for control in this set of experiments since it has the best flexibility.

\noindent\textbf{Stiffness as ``Static'' Actuation.} Here, we explore whether optimizing active actuation can reproduce motion of robots with different static muscle stiffness. With a fixed robot geometry and muscle placement, we randomly sample 100 sets of controllers and muscle stiffness, and record trajectories of all particles for 100 frames. We then try to fit a controller for robots with a different set of stiffness and measure how well they can match the last frame of the pre-collected dataset. We use the Earth Mover's Distance (EMD) loss \citep{achlioptas2018learning} and 
%gradient-based optimization 
differentiable physics for training. Specifically, we use EMD to compare the difference between the motions of each particle set at the end of each trajectory.  In \figref{fig:muscle_ambiguity_a}, we show matching error distribution across different stiffness multipliers (with respect to the original stiffness of the robot). 
%Each trial of simulation randomly samples particles inside a given shape and thus a fixed robot geometry may display the same overall shape yet with discrepancy in particle positions. 
%We compute the EMD loss to the last frame in the dataset as a reference when computing relative error. 
This experiment indicates we can roughly replicate the motion of robot with different stiffness by control optimization. This aligns with the muscle model, where we can interpret stiffness as a static component of actuation.

\noindent\textbf{Approximating Motion of Random Muscle Placement.} Here, we investigate if we can reproduce the motions of robots with different muscle placements by optimizing their controllers. With fixed robot geometry and stiffness, we randomly sample 6 sets of controllers and muscle placements with 3 actuators. We follow the same training procedure mentioned previously to fit controllers for robots with a different set of muscle placement. In \figref{fig:muscle_ambiguity_b}, we show relative matching error at different number of actuators with shaded area as confidence interval from the 6 random seeds. Note that we adopt a fixed muscle direction and soft muscle placement for simplicity. The plot suggests that with redundant ($>3$ in this case) actuators, robot motion can be roughly reproduced with control optimization even under different muscle placement.

Overall, we demonstrate the ambiguity between muscle construction and actuation that may induce challenges in co-design optimization. However, these results also unveil the potential of casting muscle optimization to control optimization in soft robot co-design. 

\subsection{The Good and Bad of Differentiable Physics for Soft Robot Co-design}

\noindent\textbf{The Value of Differentiable Physics.} Gradient-based control optimization methods powered by differentiable simulation, are increasingly popular in computational soft robotics.  Such optimizers can efficiently search for optimal solutions and decrease the number of computationally-intensive simulation episodes needed to achieve optimal results in various computational soft robotics problems compared to model-free approaches such as evolutionary strategies or RL. 
%With proper initialization and search space, differentiable simulation can generate results with a considerably smaller number of iterations as opposed to non-gradient-based method like RL or evolution strategy.
Despite its effectiveness, the local nature of gradient-based methods poses issues for optimization. While recent research has rigorously studied differentiable physics in control \citep{suh2022differentiable}, similar investigation is lacking in the context of design. Here, we take a preliminary step to analyze gradient for design optimization. In \figref{fig:role_of_diff_physics_a}, we show the loss landscape of a single parameter of actuator placement. We consider a voxel grid and smoothly (with soft actuator placement) transition the membership of the actuator in a voxel. We can observe multiple local optima, which may trap gradient-based optimization. In \figref{fig:role_of_diff_physics_b}, we show the discontinuity of loss resulted from shape changes. A box with a missing corner (red) ends up at a very different location from that of a complete box (gray), as it topples when placed in metastable configurations. These results present unique yet understandable challenges for differentiable physics in robot design, which suggests more future research to better leverage gradient information.

\begin{table}
    \begin{minipage}{0.45\linewidth}
        \caption{The efficacy of co-optimization.}
        \label{tab:cooptim_ablation}
        \centering
        \small
        \vspace{-1mm}
        \begin{tabular}{lc}
            \toprule
            \makecell[l]{Optimization Target} & Performance  \\
            \midrule
            Controller & 0.107  \\
            Design & 0.152 \\ 
            Design + Control & 0.332 \\ 
            \bottomrule
        \end{tabular}
    \end{minipage}\hfill
    \begin{minipage}{0.55\linewidth}
        \caption{Co-design with gradient-free/based methods.}
        \label{tab:codesign_baselines}
        \centering
        \small
        \vspace{-1mm}
        \begin{tabular}{lc}
            \toprule
            \makecell[l]{Method} & Performance  \\
            \midrule
            Evolution Strategy & 0.247 \\
            Diff-physics & 0.332 \\
            \bottomrule
        \end{tabular}
    \end{minipage}
    \vspace{-2mm}
\end{table}

\begin{figure*}[t]
    \centering
    \includegraphics[width=1\textwidth]{asset/render_codesign}
    \vspace{-5mm}
    \caption{Swimming motion sequences of co-design baselines.}
    \label{fig:render_codesign}
    \vspace{-6mm}
\end{figure*}

\subsection{Co-Optimizing Soft Robotic Swimmers}
In this section, we demonstrate co-design results for a swimmer in the \textit{Ocean} environment. First, in \tabref{tab:cooptim_ablation}, we compare between (1) optimizing control only (2) optimizing design only (3) co-optimizing design and control. We use \textit{SDF-Lerp} as design space representation and differentiable physics for optimization. To generate meaningful results, for the control-only case, we use a fish-like design; and for the design-only case, we use a controller trained for a fish not included in the design primitive of \textit{SDF-Lerp}. The outcome verifies the effectiveness of generating superior soft robotic swimmer with co-design. In \tabref{tab:codesign_baselines}, we compare co-design results from evolution strategy (ES) and differentiable physics. We follow the above-mentioned setup for differentiable physics. For evolution strategy, we adopt a common baseline: CPPN representation \citep{stanley2007compositional} with HyperNEAT \citep{stanley2009hypercube}; to make control optimization consistent with evolution strategy, we use CMA-ES \citep{hansen2003reducing}. Furthermore, we show the optimized design and the corresponding motion sequence of both methods in \figref{fig:render_codesign}. Interestingly, the better-performing co-design result much resembles the results of performing design optimization only, shown in \figref{fig:design_repr}.

% \citep{hansen2003reducing}

