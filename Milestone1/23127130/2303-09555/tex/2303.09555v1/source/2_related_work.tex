\section{Related Work}
\label{sec:related_work}

%We buld upon prior work in \emph{a)} the co-design of soft bodied machines, \emph{b)} differentiable simulation for soft robot optimization, and \emph{c)} environmentally-driven computational agent design.

%Our work builds upon a long line of work in computationally designing soft robots.  Here, we describe our work in relationship to prior work in soft-bodied co-design, differentiable soft robot simulation for computational design, and environmentally-driven computational design.

\textbf{Co-Design of Soft Bodied Machines.}
Evolutionary algorithms (EAs) have been used to design virtual agents since the pioneering work of \cite{sims1994evolving}; as these algorithms improved the problems they could be applied to grew in complexity.  \cite{cheney2014unshackling} demonstrated EAs for co-designing soft robots over shape, materials, and actuation for open-loop cyclic controllers; follow-on work explored soft robots with circuitry \citep{cheney2014evolved} and those grown from virtual cells \citep{joachimczak2016artificial}.
Evolutionarily designed soft robots were demonstrated to be physically manufacturable using soft foams \citep{hiller2011automatic}, and biological cells \citep{kriegman2020scalable}.  Virtually all such results have been powered by the neuroevolution of augmenting topologies (NEAT) algorithm  \citep{stanley2002evolving}, and, typically, compositional pattern producing networks (CPPN) \citep{stanley2007compositional} for morphological design representation.  Recent work has explored other effectual heuristic search algorithms, such as simulated annealing, for optimizing locomoting soft robots represented by grammars \citep{van2019spatial, van2022co}.  Recently, \cite{bhatia2021evolution} provided an expansive suite of benchmark tasks for evaluating algorithms for co-designing soft robots with closed-loop controllers.  That work provides a baseline co-design method that employs CPPN-NEAT for morphological search and reinforcement learning for control optimization \citep{schulman2017proximal}.
Similar to \citep{bhatia2021evolution}, our work benchmarks EAs alongside competing methods.  We provide a suite of environmentally-diverse tasks in a differentiable simulation environment, which allows the use of efficient gradient-based search algorithms.  Further, we focus on a systematic decoupling of design space representation, control, task and environment, and search algorithms in order to distill the influence of each.  


\iffalse
Long line of work in soft robot co-design that is based in evolutionary-based algorithms.  This work inludes:
-(First, maybe ground this in evolutionary algorithms, like Sims et al)
-Lipson work (Unshackling Evolution + a few follow ons)
-Bongard/Kriegman work
-Evolving Modular Robots Using Direct and Generative Encodings
-Harnessing evolutionary creativity: evolving soft-bodied animals in simulated physical environments
-Note cool successes in real world
Other approaches:
Shea + Van den pien

Note key differences between our work and their work:
-Differentiability - much faster (and more in the next section)
-Many tasks - most of these focus on a single task
-Focused more on a systematic approach to understanding the quality of a ``design''

\fi

\textbf{Differentiable Simulation for Soft Robot Optimization.}
A differentiable simulator is one in which the gradient of any measurement of the system can be analytically computed with respect to any variable of the system, which can include behavioral (control) and physical (morphological and environmental) parameters.  
%While differentiability has long been employed in physical simulators \citep{drake, giftthaler2017automatic}, focus on efficient backpropagation schemes and incorporation of deep neural networks has unlocked efficient algorithms for robot control and inference for both rigid \citep{degrave2019differentiable, de2018end} and soft robotics.  
Differentiability provides particular value for soft robotics; gradient-based optimization algorithms can reduce the number of (typically expensive) simulations needed to solve computational control and design problems.  Differentiable material point method simulation (MPM) has been used in co-optimizing soft robots over closed-loop controllers and spatially varying material parameters \citep{hu2019chainqueen, spielberg2021advanced}, as well as proprioceptive models \cite{spielberg2019learning}.  Co-optimization procedures rooted in differentiable simulation have also been applied to more traditional finite element representations of soft robots.  Notably, \citep{ma2021diffaqua} demonstrated efficient co-optimization of swimming soft robots' geometry, actuators, and control, incorporating a learned (differentiable) Wasserstein basis for tractable search over high-dimensional morphological design spaces.  Follow-on work further showed that differentiable simulators can be combined with learned dynamics for co-design \citep{nava2022fast}.  Such finite-element-based representations, however, have struggled with smoothly differentiating through rich contact.

%As a drawback, gradient-based co-design methods using finite element simulation have historically struggled with optimization in the presence of rich contact.

%Most gradient-based computational design using finite element solvers avoid problems with contact, as differentiating through rich soft-bodied contact can be expensive.  

Our work presents a rich comparison of differentiable and non-differentiable approaches for soft-bodied co-design, as well as a set of baseline methods.  We present a differentiable soft-bodied simulation environment capable of modeling multiphysical materials.  Our environment is based in MPM, because of its ability to naturally and differentiably handle contact and multiphysical coupling.  Similar to other recent work, such as \cite{suh2022differentiable} and \cite{huang2021plasticinelab}, we perform a thorough analysis (in our case, empirical) of the value and limitations of differentiable simulation environments, with an emphasis on cyberphysical co-design.

%P1: Explain what differentiability is and how it is different from the previous section's approach.  Note a few preceding simulators for rigid bodies (DeGrave, de Avila) and other domains (JAX M.D., etc.), note that soft bodied simulation is particularly interesting becuase of how expensive simulation is, making model-free design expensive.

%Note a few differentiable-based co-design work:
%MPM: ChainQueen/Advanced ChainQueen/LITL
%FEM: DiffAqua/follow-on work with Katzschmann/Recent work from Daniele Panozzo
%(There's some work on TopoOpt but not peer reviewed yet)
\iffalse
P2:
Here we present baselines, focus on showing limits and opportunities.  Explain how previous work on understanding differentiable simulation has been a bit limited to simple cases and rigid systems (Tedrake's, Kragic's, Bohg's groups all have some results here, I think).  Here, we have the most expansive comparisons to date.

\fi

\textbf{Environmentally-Driven Computational Agent Design.}
Similar to their biological counterparts, virtual creatures have different optimized forms depending on their environment.  \cite{auerbach2014environmental} and \cite{corucci2017evolutionary} analyzed real-world biological morphological features from the perspective of EAs; \cite{cheney2015evolving} studied the interplay between of virtual terrain and robot geometry for soft-bodied locomotion locomotion.  In rigid rob\rebuttal{o}tics where simulation is less computationally expensive,
%and more abundant data can be quickly collected, 
online data-driven neural generative design methods are emerging.  These include methods that reason over parameterized geometries \cite{ha2019reinforcement}, topological structure \cite{zhao2020robogrammar, xu2021multi, hu2022modular}, and shapeshifting behavior \cite{pathak2019learning}.  These methods currently require too much simulation-based data generation for soft robotics applications, but are useful as inspiration for future research directions.

\iffalse

%Analyze how different environmental factors and terrain affect robot optimization.

Evolutionary work \citep{auerbach2014environmental}, and some other of the work like Corucci 2017

RL work (Pathak, David Ha)

Other learning-based methods (RoboGrammar1+2, Howie Choset's group's work on design)

\fi


\iffalse
\begin{itemize}
    \item Environmental influence on morphological complexity \citep{auerbach2014environmental}
    \item Intelligence and emergent behavior through evolution \citep{corucci2017evolutionary}
    \item Learning-in-the-loop codesign \citep{spielberg2019learning}
    \item Codesign in aquatic environment \citep{ma2021diffaqua}
\end{itemize}


\aes{Probably want at least the following sections: Soft Robot Co-Design; Environmentally-Driven Design; Differentiable Simulation for Design + Control}
\fi