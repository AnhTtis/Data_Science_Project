\subsection{Proof of Proposition~\ref{prop:monotone-prob}}\label{app:proof-monotone-prob}

In order to prove the proposition, we first note the well-known fact [\cite{whitt1979note,milgrom1981good,shaked2007stochastic}] that monotone likelihood ratio (MLR) implies first-order stochastic dominance (FOSD) of the distribution of the signal conditioned on a higher parameter (as well as for the posterior distribution of the parameter conditioned on a higher signal).

\begin{lemma} \label{lem:FOSD}
Assume that the family of review signal distributions has MLR.
Then, whenever $q' > q$, the signal distribution for $q'$ first-order stochastically dominates the distribution for $q$; that is, the distributions satisfy that $\Prob[\RevSig{} \sim {\RevSigDist[q']}]{\REVSIG \geq x} > \Prob[\REVSIG \sim {\RevSigDist[q]}]{\REVSIG \geq x}$ for all $x \in (\inf \SigSet, \sup \SigSet)$. 
\end{lemma}

We mentioned above that a higher signal also implies FOSD of the posterior quality distributions. The following lemma captures the stronger property that under the MLR property, this holds even for vectors of signals.

\begin{lemma} \label{lem:monotone_expected_quality}
Suppose $\RevSigV'$ and $\RevSigV$ are two vectors of signals that have MLR and satisfy $\RevSigV'\ge \RevSigV$ component-wise, and the inequality is strict for at least one of the components. 
Then, $U(\RevSigV')> U(\RevSigV)$ holds for any prior $\QualDist$.
\end{lemma}

For continuous distributions, this lemma is proved by \citet{torres2005multivariate}.
We give a self-contained proof for the categorical case, which is largely analogous, in \cref{app:FOSD-proof}.
We are now ready to prove Proposition~\ref{prop:monotone-prob}.

\proof{Proof of Proposition~\ref{prop:monotone-prob}.}
We give the proof in the categorical model; it can be straightforwardly generalized to the continuous model.

Let $q' > q$. 
For each reviewer $i$, couple the draws of $\RevSig{i}$ from $\RevSigDist[q]$ and $\RevSigP{i}$ from $\RevSigDist[q']$ by drawing a (common) uniformly random quantile $x$ in $[0,1]$ and letting $\RevSig{i}, \RevSigP{i}$ be the respective signals at quantile $x$ of the corresponding CDFs. 
Because, by \cref{lem:FOSD}, $\RevSigDist[q']$ (strictly) first-order stochastically dominates $\RevSigDist[q]$, this coupling ensures that $\RevSigP{i} \ge \RevSig{i}$;
furthermore, the inequality is strict with positive probability unless $\RevSig{i}=\max_s\in\SigSet$.
By applying this coupling to each individual review (which, recall, is drawn independently of other reviews), we obtain a coupling of vectors of reviews such that $\RevSigVP \geq \RevSigV$ always holds component-wise, and the inequality is strict for at least one of the components with positive probability. By Lemma~\ref{lem:monotone_expected_quality}, this coupling has the property that $U(\RevSigVP) \geq U(\RevSigV)$ always holds, and, conditional on \RevSigV, the inequality is strict with positive probability unless every component of $\RevSigV$ is the maximum signal (if the maximum exists).
Therefore, if \ACCMAP is a monotone acceptance policy, $\AccP{\ACCMAP}{q}$ can never decrease in $q$.

It remains to show that $\AccP{\ACCMAP}{q}$ is \emph{strictly} increasing in $q$ for non-trivial threshold policies $\ACCMAP[\tau,r]$.
First, we may assume w.l.o.g.~that $0 < r < 1$.
For if $r=0$, the policy is equivalent to the policy $\ACCMAP[\tau',\half]$ with any $\tau' \in (\max \Set{U(\RevSigV)}{U(\RevSigV) < \tau}, \tau)$, 
and if $r=1$, it is equivalent to the policy $\ACCMAP[\tau',\half]$ with any $\tau' \in (\tau, \min \Set{U(\RevSigV)}{U(\RevSigV) > \tau})$. Here, the minimum and maximum will be finite because the policy is assumed to be non-trivial.

By Lemma~\ref{lem:monotone_expected_quality}, there must exist \RevSigV, \RevSigVH with $U(\RevSigVH) \geq \tau \geq U(\RevSigV)$ such that at least one of the two inequalities is strict. Let \RevSigV be a vector of reviews maximizing $U(\RevSigV)$ subject to $U(\RevSigV) \leq \tau$.
Because $U(\RevSigVH) > U(\RevSigV)$, the vector \RevSigV cannot be maximal in all components.
Therefore, by the preceding coupling argument, when $\RevSigV$ is drawn with quality $q$, the corresponding vector $\RevSigVP$ drawn with quality $q'$ satisfies $U(\RevSigVP) > U(\RevSigV)$. 
By definition of \RevSigV, the review vector \RevSigVP gives rise to strictly higher acceptance probability than \RevSigV. For if $U(\RevSigV) < \tau$, then \RevSigV always leads to rejection, whereas (by maximality of $U(\RevSigV)$) \RevSigVP leads to acceptance with probability at least $r > 0$.
And if $U(\RevSigV) = \tau$, then \RevSigV leads to acceptance with probability $r < 1$, whereas \RevSigVP leads to acceptance with probability 1.

% \gs{I am having trouble with this proof.  It seems to require that "$\RevSigV$ is drawn with quality $q$ and the corresponding vector $\RevSigVP$ drawn with quality $q'$" with some positive probability, but I don't see where we show that.}
% \dkcomment{I thought that this is because we couple by quantile. You draw $s$. Each component has a quantile. You draw $s'$ in that component as having the same quantile according to the distribution parametrized with $q'$. I don't see how that leads to problems.}\fangcomment{I agree with David. We can first sample a quantile that determines $s$ then $s'$ that ensures $U(s')\ge U(s)$. 
%  However, it is unclear to me why $U(s')>U(s)$ instead of $U(s')\ge U(s)$ and strict with positive probability (but this should not be an issue).  This may be stated in the second paragraph.}

Therefore, for non-trivial threshold policies, the coupling ensures that a paper of quality $q'$ is accepted at least whenever a paper of quality $q$ is accepted, and with strictly positive probability, only the paper with quality $q'$ is accepted.
This completes the proof. \Halmos
\endproof

\subsection{Proof of Lemma~\ref{lem:monotone_expected_quality}}\label{app:FOSD-proof}

Here, we prove Lemma~\ref{lem:monotone_expected_quality}. 

\proof{Proof of \Cref{lem:monotone_expected_quality}.}

% \dkreplace{The proof follows by induction on the size of quality set $\QualSet$. Note that the proof is based on the categorical model, but can be straightforwardly generalized to the continuous model. For a more detailed proof for the continuous model, one can refer Theorem 1 of \cite{torres2005multivariate}.}{
Here, we provide a proof for the categorical model. It can be easily modified for the continuous model, and the result also is shown as Theorem~1 by \citet{torres2005multivariate}.
We show the result by induction on the size of the quality set $\QualSet$.

\emph{Base case:} We show that the inequality holds for any binary quality set $\QualSet=\SET{q_1, q_2}$ with $q_2>q_1$.  For $\RevSigV \in \SigSet^{\NumReviews}$, let $\gamma(\RevSigV)=\frac{\RevSigProb[q_2]{\RevSigV}}{\RevSigProb[q_1]{\RevSigV}}$ be the likelihood ratio. We first rewrite the ex-post expected quality:
\begin{align*}
    U(\RevSigV) & = \ExpectC{Q}{\RevSigV}\\
    &= \sum_q q\cdot \ProbC{Q=q}{\RevSigV}\\
    &= \sum_q q \cdot \frac{\Prob{q}\cdot \ProbC{\RevSigV}{Q=q}}{\sum_{q'}\Prob{q'} \cdot \ProbC{\RevSigV}{Q=q'}}\\
    &= q_1 \cdot \frac{\QualProb{q_1}\RevSigProb[q_1]{\RevSigV}}{\QualProb{q_1}\RevSigProb[q_1]{\RevSigV}+\QualProb{q_2}\RevSigProb[q_2]{\RevSigV}} + q_2\cdot \frac{\QualProb{q_2}\RevSigProb[q_2]{\RevSigV}}{\QualProb{q_1}\RevSigProb[q_1]{\RevSigV}+\QualProb{q_2}\RevSigProb[q_2]{\RevSigV}}\\
    &=  \frac{q_1 \cdot \QualProb{q_1}\RevSigProb[q_1]{\RevSigV} + q_2\cdot\QualProb{q_2}\RevSigProb[q_2]{\RevSigV}}{\QualProb{q_1}\RevSigProb[q_1]{\RevSigV}+\QualProb{q_2}\RevSigProb[q_2]{\RevSigV}}\\
    &= \frac{q_1\cdot\QualProb{q_1} + q_2\cdot\QualProb{q_2}\gamma(\RevSigV)}{\QualProb{q_1}+\QualProb{q_2}\gamma(\RevSigV)}.
\end{align*}
    
Now, we consider the difference $U(\RevSigVP) - U(\RevSigV)$, write it using a common denominator, and cancel common terms. Then, 
% \dkdeletecomment{we'd also need to specify the one strict inequality, no?}{because $\RevSigVP \ge \RevSigV$ component-wise,} 
by \cref{def:informative}, we obtain that $\gamma(\RevSigVP)>\gamma(\RevSigV)$, which we substitute:
\begin{align*}
    U(\RevSigVP) - U(\RevSigV)
    &=  \frac{\QualProb{q_1}\QualProb{q_2} \cdot \left(q_1\gamma(\RevSigV) + q_2\gamma(\RevSigVP) - q_1\gamma(\RevSigVP) - q_2\gamma(\RevSigV)\right)}{\left(\QualProb{q_1}+\QualProb{q_2}\gamma(\RevSigVP)\right) \cdot \left(\QualProb{q_1}+\QualProb{q_2}\gamma(\RevSigV)\right)}\\
    &= \frac{\QualProb{q_1}\QualProb{q_2} \cdot (\gamma(\RevSigVP) - \gamma(\RevSigV)) \cdot (q_2-q_1)}{\left(\QualProb{q_1}+\QualProb{q_2}\gamma(\RevSigVP)\right) \cdot \left(\QualProb{q_1}+\QualProb{q_2}\gamma(\RevSigV)\right)}\\
    &>0.
\end{align*}

\emph{Induction step: } We show that if the inequality holds for all categorical models with support size $|\QualSet|=n-1$, it also holds for categorical models with support size $|\QualSet|=n$. 
Let $\QualSet=\SET{q_1, q_2, \ldots, q_n}$ with $q_n>\cdots>q_1$. Using the same derivation as in the base case, 
\begin{align*}
    U(\RevSigVP) - U(\RevSigV)
    &= \frac{\left(\sum_{i=1}^n q_i\QualProb{q_i}\RevSigProb[q_i]{\RevSigVP}\right) \cdot \left(\sum_{i=1}^n \QualProb{q_i}\RevSigProb[q_i]{\RevSigV}\right) - \left(\sum_{i=1}^n q_i\QualProb{q_i}\RevSigProb[q_i]{\RevSigV}\right) \cdot \left(\sum_{i=1}^n \QualProb{q_i}\RevSigProb[q_i]{\RevSigVP}\right)}{\left(\sum_{i=1}^n \QualProb{q_i}\RevSigProb[q_i]{\RevSigVP}\right) \cdot \left(\sum_{i=1}^n \QualProb{q_i}\RevSigProb[q_i]{\RevSigV}\right)}.
\end{align*}

% \dkreplace{
% Let $P(n)$ be the numerator of the above equation.
% Assuming $P(k)$ is positive, we want to show that $P(k+1)$ is also positive.

% \begin{align*}
%     P(k+1) =& \left(\sum_{i=1}^{k+1} q_i\QualProb{q_i}\RevSigProb[q_i]{\RevSigVP}\right)\left(\sum_{i=1}^{k+1} \QualProb{q_i}\RevSigProb[q_i]{\RevSigV}\right) - \left(\sum_{i=1}^{k+1} q_i\QualProb{q_i}\RevSigProb[q_i]{\RevSigV}\right)\left(\sum_{i=1}^{k+1} \QualProb{q_i}\RevSigProb[q_i]{\RevSigVP}\right)\\
%     =& P(k) + q_{k+1}\QualProb{q_{k+1}}\RevSigProb[q_{k+1}]{\RevSigVP}\left(\sum_{i=1}^{k} \QualProb{q_i}\RevSigProb[q_i]{\RevSigV}\right) +\QualProb{q_{k+1}}\RevSigProb[q_{k+1}]{\RevSigV}\left(\sum_{i=1}^{k} q_i\QualProb{q_i}\RevSigProb[q_i]{\RevSigVP}\right)\\
%     &- q_{k+1}\QualProb{q_{k+1}}\RevSigProb[q_{k+1}]{\RevSigV}\left(\sum_{i=1}^{k} \QualProb{q_i}\RevSigProb[q_i]{\RevSigVP}\right) - \QualProb{q_{k+1}}\RevSigProb[q_{k+1}]{\RevSigVP}\left(\sum_{i=1}^{k} q_i\QualProb{q_i}\RevSigProb[q_i]{\RevSigV}\right)\\
%     % =& P(k) + \sum_{i=1}^{k}\QualProb{q_i}\QualProb{q_{k+1}}\left(q_{k+1}\left(\RevSigProb[q_{k+1}]{\RevSigVP}\RevSigProb[q_i]{\RevSigV} - \RevSigProb[q_{k+1}]{\RevSigV}\RevSigProb[q_i]{\RevSigVP}\right) + q_i\left(\RevSigProb[q_{k+1}]{\RevSigV}\RevSigProb[q_i]{\RevSigVP} - \RevSigProb[q_{k+1}]{\RevSigVP}\RevSigProb[q_i]{\RevSigV}\right)\right)\\
%     =& P(k) + \sum_{i=1}^{k}\QualProb{q_i}\QualProb{q_{k+1}}\left(\RevSigProb[q_{k+1}]{\RevSigVP}\RevSigProb[q_i]{\RevSigV} - \RevSigProb[q_{k+1}]{\RevSigV}\RevSigProb[q_i]{\RevSigVP}\right)\left(q_{k+1} - q_i\right)\\
%     >& \sum_{i=1}^{k}\QualProb{q_i}\QualProb{q_{k+1}}\left(\RevSigProb[q_{k+1}]{\RevSigVP}\RevSigProb[q_i]{\RevSigV} - \RevSigProb[q_{k+1}]{\RevSigV}\RevSigProb[q_i]{\RevSigVP}\right)\left(q_{k+1} - q_i\right).
% \end{align*}
% }{
Because the denominator is strictly positive, we now focus on showing that the numerator is as well.
We show that the difference between the actual numerator, and the version where the upper bound of each summation is $n-1$, is positive. Then, because the latter is positive by induction hypothesis, we will have shown the claim.
\begin{align*}
& \left( \left(\sum_{i=1}^{n} q_i\QualProb{q_i}\RevSigProb[q_i]{\RevSigVP}\right) \cdot \left(\sum_{i=1}^{n} \QualProb{q_i}\RevSigProb[q_i]{\RevSigV}\right) - \left(\sum_{i=1}^{n} q_i\QualProb{q_i}\RevSigProb[q_i]{\RevSigV}\right)\left(\sum_{i=1}^{n} \QualProb{q_i}\RevSigProb[q_i]{\RevSigVP}\right) \right)
\\ - & 
 \left( \left(\sum_{i=1}^{n-1} q_i\QualProb{q_i}\RevSigProb[q_i]{\RevSigVP}\right) \cdot \left(\sum_{i=1}^{n-1} \QualProb{q_i}\RevSigProb[q_i]{\RevSigV}\right) - \left(\sum_{i=1}^{n-1} q_i\QualProb{q_i}\RevSigProb[q_i]{\RevSigV}\right)\left(\sum_{i=1}^{n-1} \QualProb{q_i}\RevSigProb[q_i]{\RevSigVP}\right) \right)
\\ = & 
 q_n \QualProb{q_n} \RevSigProb[q_{n}]{\RevSigVP} \cdot \left( \sum_{i=1}^{n-1} \QualProb{q_i}\RevSigProb[q_i]{\RevSigV} \right) 
 + \QualProb{q_n} \RevSigProb[q_n]{\RevSigV} \cdot \left( \sum_{i=1}^{n-1} q_i \QualProb{q_i}\RevSigProb[q_i]{\RevSigVP} \right)
\\ - & q_n \QualProb{q_n} \RevSigProb[q_n]{\RevSigV} \cdot \left( \sum_{i=1}^{n-1} \QualProb{q_i} \RevSigProb[q_i]{\RevSigVP} \right) 
- \QualProb{q_n} \RevSigProb[q_n]{\RevSigVP} \cdot \left(\sum_{i=1}^{n-1} q_i \QualProb{q_i}\RevSigProb[q_i]{\RevSigV} \right)
\\ = & 
\sum_{i=1}^{n-1} \QualProb{q_i} \QualProb{q_n} \cdot \left( \RevSigProb[q_n]{\RevSigVP}\RevSigProb[q_i]{\RevSigV} - \RevSigProb[q_n]{\RevSigV} \RevSigProb[q_i]{\RevSigVP} \right) \cdot \left( q_n - q_i \right).
\end{align*}


By \cref{def:informative}, 
% \dkreplace{for any $i<k+1$ and $\RevSigVP\ge \RevSigV$ component-wise with strict inequality for at least one of the components, we know that 

% \begin{align*}
%     &\frac{\RevSigProb[q_{k+1}]{\RevSigVP}}{\RevSigProb[q_i]{\RevSigVP}}>\frac{\RevSigProb[q_{k+1}]{\RevSigV}}{\RevSigProb[q_i]{\RevSigV}}\\
%     \Leftrightarrow\qquad & \RevSigProb[q_{k+1}]{\RevSigVP}\RevSigProb[q_{i}]{\RevSigV} > \RevSigProb[q_{k+1}]{\RevSigV}\RevSigProb[q_i]{\RevSigVP}.
%     \end{align*}

% This implies that $P(k+1)>0$, establishing the induction step.
% }{
$\RevSigProb[q_n]{\RevSigVP} \RevSigProb[q_i]{\RevSigV} > \RevSigProb[q_n]{\RevSigV}\RevSigProb[q_i]{\RevSigVP}$, proving that the difference is strictly positive. Thus, by induction hypothesis, the utility difference is strictly positive, completing the proof. \Halmos
\endproof

\subsection{Proof of \cref{lem:author_response}}\label{app:proof-author_resp}

Let $\ACCMAP$ be the acceptance policy. 
The probability of acceptance for a paper of quality $\Qual=q$ is $\AccP{\ACCMAP}{q}$. By Proposition~\ref{prop:monotone-prob}, this probability is non-decreasing in $q$ for monotone acceptance policies, and strictly increasing for non-trivial threshold policies.
The author submits the paper if the expected utility of submitting is greater than 1 (the utility of the outside option), does not submit the paper if the expected utility is less than 1, and is indifferent between submitting or not if the expected utility is equal to 1. 
Because a noiseless author does not learn any new information from rejection in previous rounds, she will make the same decision in future rounds, implying that she will submit until acceptance.
% Since the author will face the same tradeoff in future rounds\footnote{Crucially, a noiseless author does not learn any new information from rejection in previous rounds.}, she will make the same decision, so she will submit until acceptance. 
Let $\ConfValue$ be fixed. The expected utility can be obtained as the time-discounted sum of the utility from acceptance:
\begin{align}
\AUTHUTIL(\ACCMAP,q)
& = 
\sum_{t\ge 1} \ConfValue \cdot \TD^{t-1} \AccP{\ACCMAP}{q} \cdot (1-\AccP{\ACCMAP}{q})^{t-1} 
\; = \; \frac{\ConfValue \cdot \AccP{\ACCMAP}{q}}{1-\TD \cdot (1-\AccP{\ACCMAP}{q})}.
\label{eqn:submission-inequality}
\end{align}
Solving the inequalities $\AUTHUTIL(\ACCMAP,q) > 1$, $\AUTHUTIL(\ACCMAP,q) < 1$, and $\AUTHUTIL(\ACCMAP,q) = 1$ for $q$, the author submits the paper if $\AccP{\ACCMAP}{q} > 1/\rho$, does not submit if $\AccP{\ACCMAP}{q} < 1/\rho$, and is indifferent between submitting or not if $\AccP{\ACCMAP}{q} = 1/\rho$, respectively. 
This completes the proof of the lemma.

\subsection{Proof of \cref{prop:de_facto}}\label{app:proof-de_facto}
By \cref{lem:author_response}, an author with a paper of quality $Q=q$ decides whether to submit based on whether the acceptance probability $\AccP{\ACCMAP}{q}$ is greater than the inverse of the attractiveness factor, $1/\rho$.
Recall that $\AccP{\ACCMAP}{q}$ is weakly increasing in $q$ by \cref{prop:monotone-prob}, while $1/\rho(q,r)$ is strictly decreasing in $q$.
Since the conference's policy is responsive, there must exist some $\bar{q}$ such that $\AccP{\ACCMAP}{\bar{q}} > 1/\rho(\bar{q},1)$.
If $\AccP{\ACCMAP}{q} \ge 1/\rho(q,1)$ for all $q\in \QualSet$, then every author prefers to submit and $\theta = -\infty$ is a de facto threshold.
Therefore, we only have to consider the case where there exists some $\ubar{q}$ such that $\AccP{\ACCMAP}{\ubar{q}} < 1/\rho(\ubar{q},1)$.
% Moreover, under our model's assumption 
% \dkcomment{I did not see that assumption anywhere in either the model or the statement of the lemma. Where is it?}\yzcomment{It was commented out for some reason. I put it back in section 2.2} 
% that the conference's value is less than 1 if all papers are accepted, there exists some $\ubar{q}$ such that $\AccP{\ACCMAP}{\ubar{q}} < 1 < 1/\rho(\ubar{q},1)$.

Combining these observations, we conclude that either there exists some $\theta'\in \QualSet$ such that $\AccP{\ACCMAP}{\theta'} = 1/\rho(\theta',1)$, or there exist two qualities $\ubar{\theta}, \bar{\theta}\in \QualSet$, with no intermediate qualities between them, such that $\AccP{\ACCMAP}{\ubar{\theta}} < 1/\rho(\ubar{\theta},1)$ and $\AccP{\ACCMAP}{\bar{\theta}} > 1/\rho(\bar{\theta},1)$.
We show that a threshold best response exists for either case.

First, if there exists a $\theta'\in \QualSet$ such that $\AccP{\ACCMAP}{\theta'} = 1/\rho(\theta',1)$, then it is a threshold best response:  every author submits and keeps resubmitting if her paper has quality $Q = q\ge \theta'$ and takes the outside option if $q < \theta'$.
To see this, note that authors with a paper of quality $q\ge \theta'$ all have an acceptance probability $\AccP{\ACCMAP}{q} \ge 1/\rho(\theta', 1)$, meaning that they are weakly happier to submit; on the other hand, the acceptance probability $\AccP{\ACCMAP}{q} \le 1/\rho(\theta',1)$ for authors with $q< \theta'$, meaning that they are weakly happier to take the outside option.

If there exist two adjacent qualities $\ubar{\theta}, \bar{\theta}\in \QualSet$ such that $\AccP{\ACCMAP}{\ubar{\theta}} < 1/\rho(\ubar{\theta},1)$ and $\AccP{\ACCMAP}{\bar{\theta}} > 1/\rho(\bar{\theta},1)$, we further distinguish two cases.
In the first case,  $\AccP{\ACCMAP}{\ubar{\theta}}\le 1/\rho(\bar{\theta},1)$. This means that $\AccP{\ACCMAP}{\ubar{\theta}}\le 1/\rho(\ubar{\theta},r)$ for any $r\in [0,1]$, because $\ubar{\theta}<\bar{\theta}$.\fangcomment{Minor point: we use $r$ for three different meaning 1) superscript for reviewer $F^{(r)}$, 2) threshold acceptance policy $\phi_{\tau, r}$, and 3) author's threshold strategy.} \dkcomment{Good point! At least the review noise should perhaps be $R$? Unfortunately, we haven't been consistent in using macros, so this will now be hard to fix without missing one. But we should probably do it after we submit, before the next round of reviews/acceptance.}
In other words, if authors with $Q \ge \bar{\theta}>\ubar{\theta}$ all decide to submit and authors with $Q < \ubar{\theta}$ all decide to take the outside option, then authors with $Q = \ubar{\theta}$ all (weakly) prefer to take the outside option no matter with what probability the other authors with $Q= \ubar{\theta}$ submit.
Therefore, the following is an equilibrium for authors: every author with $Q \ge \bar{\theta}$ submits, and every author with $Q < \bar{\theta}$ takes the outside option.

In the second case, $\AccP{\ACCMAP}{\ubar{\theta}}> 1/\rho(\bar{\theta},1)$. 
This corresponds to the following case: if authors with paper quality $Q \ge \bar{\theta}$ submit and those with $Q < \ubar{\theta}$ opt for the outside option, then authors with paper quality $Q = \ubar{\theta}$ prefer submitting if no one at that quality submits but prefer the outside option when everyone at that quality submits.
% Therefore, there exists some probability $r$ such that it is a best response for authors with $Q = \ubar{\theta}$ to submit with probability $r$, while authors with $Q > \ubar{\theta}$ submit with probability 1, and those with $Q < \ubar{\theta}$ choose the outside option.

At equilibrium, the probability $r$ with which such authors submit must make authors with paper quality $Q = \ubar{\theta}$ indifferent between submitting or not submitting. Therefore, $r$ must be is the solution to $\AccP{\ACCMAP}{\ubar{\theta}} = 1/\rho(\ubar{\theta},r)$, i.e.,
\begin{equation*} 
        \frac{r\cdot \ubar{\theta} \cdot \QualProb{\ubar{\theta}} + \sum_{q \in \QualSet, q \geq \bar{\theta}} q \cdot \QualProb{q}}{%
        r\cdot\QualProb{\ubar{\theta}} + \sum_{q \in \QualSet, q \geq \bar{\theta}}\QualProb{q}} 
        = \frac{1-\TD \cdot (1-\AccP{\ACCMAP}{\ubar{\theta}})}{\AccP{\ACCMAP}{\ubar{\theta}}}.
\end{equation*}
Note that the solution must exist because by the relationship between $P_{\text{acc}}$ and $1/\rho$ in this case, when $r = 0$, the left-hand side of the above equation is larger than the right-hand side, while if $r = 1$, the left-hand side is smaller than the right-hand side.
By the continuity of the left-hand-side as a function of $r$, a solution must exist.
This completes the proof of the first part.

As noted above, by \cref{prop:monotone-prob}, under a monotone acceptance policy, the acceptance probability is weakly increasing in $q$.
As a result, multiple adjacent quality levels may share the same acceptance probability, which can lead to non-threshold best responses.
For example, suppose that there are three adjacent quality levels $q_1<q_2<q_3$ such that $\AccP{\ACCMAP}{q_1} = \AccP{\ACCMAP}{q_2} = \AccP{\ACCMAP}{q_3} = 1/\rho(q_2,1)$.
Based on the earlier argument, there exists a threshold best response at $\theta = q_2$, where the authors submit if and only if $Q\ge q_2$ and the corresponding conference value is $\ConfValue(q_2,1)$.
However, it is also possible for a non-threshold best response to exist: for example, there may be a submission strategy where authors with qualities $q_1, q_2$, and $q_3$ submit with probabilities $r_1, r_2$, and $r_3$, respectively, and $r_1, r_3 \in (0,1)$. As long as the induced conference value remains at $\ConfValue(q_2)$, such a mixed strategy can also constitute a non-threshold best response.

However, when the conference applies a responsive threshold acceptance policy, $\AccP{\ACCMAP}{q}$ is \emph{strictly} increasing in $q$.
This implies that no matter what the conference value is, there exists at most one quality level at which authors are indifferent between submitting and not submitting. 
This immediately guarantees that every best response must be a threshold strategy.
If such a quality level $\theta'$ where authors are indifferent between submitting or not submitting exists, then $\theta = \theta'$ is a de facto threshold.
If not, there again exist two adjacent qualities $\ubar{\theta}, \bar{\theta}\in \QualSet$ such that $\AccP{\ACCMAP}{\ubar{\theta}} < 1/\rho(\ubar{\theta},1)$ and $\AccP{\ACCMAP}{\bar{\theta}} > 1/\rho(\bar{\theta},1)$.
Then, based on the previous arguments, there are again two cases: if $\AccP{\ACCMAP}{\ubar{\theta}}\le 1/\rho(\bar{\theta},1)$, any $\theta \in [\ubar{\theta}, \bar{\theta}]$ is a de facto threshold; if $\AccP{\ACCMAP}{\ubar{\theta}}> 1/\rho(\bar{\theta},1)$, then $\theta = \ubar{\theta}$ is a de facto threshold.
This completes the second part.

% We next prove the third part, so we assume that the conference's policy is a non-trivial threshold acceptance policy.
% By \cref{prop:monotone-prob}, $\AccP{\ACCMAP}{q}$ is strictly monotone in $q$. 
% Therefore, in the categorical model, there exists at most one $\hat{q} \in \QualSet$ such that $\AccP{\ACCMAP}{\hat{q}}=1/\rho$.
% If such a $\hat{q}$ exists, it is a de facto threshold. If not, then because the threshold acceptance policy is non-trivial, there exist qualities $\ubar{q} < \bar{q}$ such that $\AccP{\ACCMAP}{\ubar{q}} < 1/\rho$, $\AccP{\ACCMAP}{\bar{q}} > 1/\rho$, and there are no qualities in $\QualSet$ between $\ubar{q}$ and $\bar{q}$.  
% Then, any $\theta \in [\ubar{q}, \bar{q}]$ is a de facto threshold. To complete the third part of the proof, notice that if $\theta$ is a de facto threshold, then every best response by the author is a $\theta$-threshold strategy.
% {the author is strictly better off to submit any paper with quality $q\ge \bar{q}$ and not submit any paper with quality $q< \bar{q}$, which results in a threshold best response. In this case, any $\theta \in [\ubar{q}, \bar{q}]$ is a de facto threshold.}

Finally, we prove the third part. 
In the continuous model, the distribution of the review noise is assumed to be a bijection, which implies that $\AccP{\ACCMAP}{q}$ is continuously increasing in $q$. 
Moreover, as the density of $\QualDist$ has full support, $1/\rho(q)$ is continuously decreasing in $q$.
Because the acceptance policy is responsive, there must exist a unique intersection such that $\AccP{\ACCMAP}{\theta} = 1/\rho(\theta)$, meaning that $\theta$ is a unique de facto threshold. \Halmos

\subsection{Proof of \cref{prop:threshold-policy}}
\label{app:proof-threshold-policy}

% \dkcomment{We are restating some of the lemmas/propositions from the main text when we prove them in the appendix, but not others. Is there a concrete reason why? If not, should we be consistent? E.g., restate all of them?}

To prove the proposition, we want to relate the acceptance threshold to how ``strict'' the policy is. We begin by defining a comparison between the strictness of two policies:

\begin{definition} \label{def:stricter}
An acceptance policy \ACCMAP['] is \emph{(weakly) stricter} than another policy $\ACCMAP$ if it accepts every paper with a (weakly) smaller probability, i.e., $\AccP{\ACCMAP[']}{q} < \AccP{\ACCMAP}{q}$ for all $q$ (resp., $\AccP{\ACCMAP[']}{q} \leq \AccP{\ACCMAP}{q}$ for all $q$ for the weak version).
\end{definition}

Being stricter appears to be a very demanding requirement, in that it requires an inequality for all paper qualities. 
We next show that for threshold policies, it in fact follows from a strictly smaller acceptance probability for just one paper. 


\begin{lemma}\label{lem:stricter_policy}
Let $\ACCMAP$ and $\ACCMAP[']$ be two threshold acceptance policies. If there exists a $q\in \QualSet$ such that $\AccP{\ACCMAP[']}{q} < \AccP{\ACCMAP}{q}$, then $\ACCMAP[']$ is stricter than $\ACCMAP$.
\end{lemma}

\proof{Proof of \Cref{lem:stricter_policy}.}

  We want to show that if a threshold policy accepts one type of paper with strictly smaller probability, it accepts \emph{every} paper with strictly smaller probability. 

  First, recall that we assumed that the conditional signal distribution has full support on the signal space. Because multiple reviews are i.i.d., conditioned on any paper quality $q$, the review signal distribution has full support over all vectors of review signals.

  Because \ACCMAP, \ACCMAP['] are \emph{threshold} policies, any review vector that leads to acceptance under \ACCMAP['] must lead to acceptance under \ACCMAP with at least the same probability. And because a paper of quality $q$ is accepted with strictly higher probability by \ACCMAP, there must exist a set $S$ of review vectors \RevSigV which are all accepted with strictly higher probability under \ACCMAP than under \ACCMAP['], such that $S$ has strictly positive probability mass under the combination of the distributions $\QualDist$ and $\RevSigDist[q]$.
  
  Because $S$ occurs with positive probability for every paper quality $q'$ (by the full-support assumption on the $\RevSigDist[q]$), every paper is accepted with strictly higher probability by \ACCMAP than by \ACCMAP['], completing the proof. \Halmos
\endproof

The following lemma relates the strictness of a threshold acceptance policy with its acceptance threshold.


\begin{lemma} \label{prop:monotone-prob-threshold}
In the categorical model, let $\ACCMAP[\tau,r]$ and $\ACCMAP[\tau',r']$ be two threshold acceptance policies with $r, r'\in (0,1]$.
Then, if either $\tau'>\tau$ or $\tau'=\tau$ and $r'< r$, $\ACCMAP[\tau',r']$ is weakly stricter than $\ACCMAP[\tau,r]$. 

 In the continuous model, let $\tau$ and $\tau'$ be the thresholds for two non-trivial threshold policies. Then, $\ACCMAP[\tau']$ is stricter than $\ACCMAP[\tau]$ if and only if $\tau' > \tau$.
\end{lemma}

\proof{Proof of \Cref{prop:monotone-prob-threshold}.}

We begin by proving the first part of the lemma, regarding the categorical model. By the definition of $\ACCMAP[\tau,r]$ and $\ACCMAP[\tau',r']$, whenever a paper with some review vector $\RevSigV$ is accepted by $\ACCMAP[\tau',r']$ with positive probability, it will be accepted by $\ACCMAP[\tau,r]$ with at least the same probability. This means that $\ACCMAP[\tau,r]$ accepts every paper with at least the same probability as  $\ACCMAP[\tau',r']$.

% \dkcomment{Switching order of ``if'' vs.~``only if'', because ``if'' takes more work.}
Next, we prove the second statement, regarding the continuous model. For the ``only if'' direction, we know that a stricter threshold policy accepts every paper with strictly smaller probability. In the continuous model, $\ACCMAP[\tau']$ must reject some review vectors that are accepted under $\ACCMAP[\tau]$. This implies $\tau'>\tau$. 

For the ``if'' direction, we begin with some basic observations. First, by \cref{lem:monotone_expected_quality}, $U(\RevSigV)$ is monotone increasing in \RevSigV.
Second, by standard Bayesian updating formulas and using that the review noise is additive and independent of $q$, the expected quality conditioned on the signal \RevSigV can be written as 
\begin{align*}
    U(\RevSigV)
    & = \int q \cdot \frac{\QualDens{q} \cdot \prod_{i=1}^{\NumReviews} \RevSigProb[q]{\RevSig{i}}}{\int \QualDens{q'} \cdot \prod_{i=1}^{\NumReviews} \RevSigProb[q']{\RevSig{i}}\, d q'} \, d q
    = \int q \cdot \frac{\QualDens{q} \cdot \prod_{i=1}^{\NumReviews} f^{(r)}(\RevSig{i}-q)}{\int \QualDens{q'} \cdot \prod_{i=1}^{\NumReviews} f^{(r)}(\RevSig{i}-q') \, d q'} \, d q.
\end{align*}

In this form, because we assumed $f^{(r)}$ to be a continuous function, and $\QualDens{p}$ to be strictly positive, it is easy to see that $U(\RevSigV)$ is also a continuous function of \RevSigV.

Because neither threshold acceptance policy is trivial by assumption, there exist review vectors $\RevSigV, \RevSigVP$ with $U(\RevSigV) < \tau < \tau' < U(\RevSigVP)$, and in particular, because $U$ is monotone, writing $\mathbf{1}$ for the all-1 vector, $\lim_{y \to - \infty} U(y \cdot \mathbf{1}) < \tau$, and $\lim_{y \to \infty} U(y \cdot \mathbf{1}) > \tau'$.
By continuity and monotonicity of $U$, the function $U(y \cdot \mathbf{1})$ is continuous and monotone as a function of $y$, so there exists a $y$ with $U(y \cdot \mathbf{1}) = \frac{\tau+\tau'}{2}$.

Again by continuity of $U$, there is a sufficiently small $\epsilon > 0$ such that for every $\RevSigV$ with $|| \RevSigV -  y \cdot \mathbf{1} ||_2 \leq \epsilon$, we have that 
$U(\RevSigV) \in (\tau, \tau')$.
Let $S = \Set{\RevSigV}{|| \RevSigV - y \cdot \mathbf{1} ||_2 \leq \epsilon}$.
$S$ has positive volume, and because we assumed that $f^{(r)}(x) > 0$ for all $x$, for every quality $q$, the event of obtaining a review vector $\RevSigV \in S$ has positive probability.
Furthermore, whenever the review vector $\RevSigV \in S$, the paper is accepted by $\ACCMAP[\tau]$, but rejected by $\ACCMAP[\tau']$.
Thus, every paper quality $q$ is strictly more likely to be accepted under $\ACCMAP[\tau']$ than under $\ACCMAP[\tau]$.
\Halmos
\endproof

\proof{Proof of \cref{prop:threshold-policy}.}

We prove the proposition based on two cases, depending on whether $\theta\in \QualSet$.
If $\theta \in \QualSet$, let $r\in [0,1]$ be any value such that $\ConfValue(\theta,r)> 1$; the existence of such an $r$ is guaranteed because $\theta$ is a candidate threshold.
In this case, let $\hat{\theta} = \theta$ and $\rho = \frac{\ConfValue(\theta, r)-\TD}{1-\TD}$.
If $\theta \notin \QualSet$, let $\hat{\theta} = \inf\Set{q \in \QualSet}{q > \theta}$ and $\rho = \frac{\ConfValue(\hat{\theta}, 1)-\TD}{1-\TD}$.
Note that because $\theta$ is a candidate threshold, $V(\hat{\theta},1) > 1$. 
This, in both cases (discrete and continuous), we obtain that $1/\rho  < 1$. 

To prove the proposition, it is sufficient to show that, in either case, there exists a threshold acceptance policy such that $\AccP{\ACCMAP}{\hat{\theta}} = 1/\rho$.
To see this, when $\theta \in \QualSet$, it is a best response for authors with $Q > \theta$ to submit, while authors with $Q < \theta$ take the outside option, and authors with $Q = \theta$ submit with probability $r$.
When $\theta \notin \QualSet$, it is a best response for authors with $Q > \theta$ to submit, while authors with $Q < \theta$ take the outside option.
This means that as long as $\AccP{\ACCMAP}{\hat{\theta}} = 1/\rho$, $\theta$ is a de facto threshold.

% Let $\hat{\theta}$ be a value in $\QualSet$ closest to $\theta$, i.e.~$\hat{\theta} \in \arg\min_{q\in \QualSet} (q-\theta)^2$ (breaking ties arbitrarily).
We now show that there always exists a threshold acceptance policy such that $\AccP{\ACCMAP}{\hat{\theta}} = 1/\rho$.
Let $f(\tau) := \AccP{\ACCMAP[\tau,0]}{\hat{\theta}}$ be the probability that a paper of quality $\hat{\theta}$ is accepted under the policy $\ACCMAP[\tau, 0]$, which accepts the paper if and only if its expected posterior quality is greater than $\tau$.
Then, $\lim_{\tau \to -\infty} f(\tau) = 1 > 1/\rho > 0 = \lim_{\tau \to +\infty} f(\tau)$. Furthermore, by Lemma \ref{prop:monotone-prob-threshold}, $f(\tau)$ is a non-increasing function of $\tau$. 
Therefore, there must exist a $\hat{\tau}$ such that $\lim_{\tau \to \hat{\tau} \uparrow} f(\tau) \geq 1/\rho \geq \lim_{\tau \to \hat{\tau} \downarrow} f(\tau)$.
  
If $f$ is continuous at $\hat{\tau}$, then the threshold policy $\ACCMAP[\hat{\tau},0]$ has $\AccP{\ACCMAP[\hat{\tau},0]}{\hat{\theta}} = 1/\rho$ by definition.
Otherwise, let $z = \lim_{\tau \to \hat{\tau} \uparrow} f(\tau) - \lim_{\tau \to \hat{\tau} \downarrow} f(\tau) > 0$. We can infer that there must be a discrete probability of $z$ for the event that \fangcomment{do we use $U(\RevSigV)$?}$\ExpectC{\Qual}{\RevSigV} = \hat{\tau}$, i.e., that $\Prob[{\RevSigV \sim \RevSigDist[{\hat{\theta}}]}]{\ExpectC{\Qual}{\RevSigV} = \hat{\tau}} = z$.
We then consider the threshold policy $\ACCMAP[\hat{\tau},\hat{r}]$ with threshold $\hat{\tau}$ which conditioned on $\ExpectC{\Qual}{\RevSigV} = \hat{\tau}$ accepts a paper with probability $\hat{r} := \frac{1/\rho-\lim_{\tau \to \hat{\tau} \downarrow} f(\tau)}{z}$. The overall acceptance probability of $\ACCMAP[\hat{\tau},\hat{r}]$ for a paper with quality $\hat{\theta}$ is therefore
\begin{align*} 
 & \Prob[{\RevSigV \sim \RevSigDist[{\hat{\theta}}]}]{\ExpectC{\Qual}{\RevSigV} > \hat{\tau}} + \Prob[{\RevSigV \sim \RevSigDist[{\hat{\theta}}]}]{\ExpectC{\Qual}{\RevSigV} = \hat{\tau}} \cdot \frac{1/\rho-\lim_{\tau \to \hat{\tau} \downarrow} f(\tau)}{z}
\\ & = \lim_{\tau \to \hat{\tau} \downarrow} f(\tau)
+ z \cdot \frac{1/\rho-\lim_{\tau \to \hat{\tau} \downarrow} f(\tau)}{z}
\; = \; 1/\rho.
\end{align*}

Thus, under the threshold acceptance policy $\ACCMAP[\hat{\tau},\hat{r}]$, authors of papers with quality $\hat{\theta}$ are indifferent between submitting and not submitting.\fangcomment{do we need a proof for the moreover part?}
% Note that if $\theta\in\QualSet$, then $\hat{\theta}=\theta$. The result for the special case in the proposition statement straightforwardly follows.
% For the general case, assume w.l.o.g.~(the other case is symmetric) that $\hat{\theta} \geq \theta$.
% Because the author is indifferent between submitting and not submitting papers of quality $\Qual = \hat{\theta}$, we may assume that such papers are submitted, and hence all papers of quality at least $\hat{\theta}$. 
% By definition of $\hat{\theta}$, no papers have quality $\Qual \in [\theta, \hat{\theta})$, so all papers of quality at least $\theta$ are submitted. Thus, $\theta$ is also a de facto threshold for the author.

Finally, uniqueness of the conference's threshold policy in the continuous model when the author's submission decision is non-trivial follows directly because by Lemma~\ref{prop:monotone-prob-threshold}, the solution for $\hat{\tau}$ of $\AccP{\ACCMAP[\hat{\tau},0]}{\hat{\theta}} = 1/\rho$ is unique in the continuous model when agents are neither submitting all papers nor submitting no papers.
\Halmos
\endproof

\subsection{Proof of \cref{prop:gap-QB-dominance}}\label{app:proof-gap-QB-dominance}

In the continuous model, fixing the quality prior, the conference quality only depends on the de facto threshold.
Therefore, to prove the proposition, it is sufficient to show that for every candidate threshold $\theta$, the review burden of the first setting is larger than that of the second setting.

Recall that the review burden is given by $R(\theta) = m\int_\theta^\infty \QualDens{q}/\AccP{\ACCMAP[\tau(\theta)]}{q} dq$, suppress parameter dependence and denote the corresponding acceptance thresholds by $\tau(\theta)$ and $\tau'(\theta)$ for the two settings, respectively.
It is thus sufficient to show that $\AccP{\ACCMAP[\tau(\theta)]}{q} < \AccP{\ACCMAP[\tau'(\theta)]}{q}$ for every candidate threshold $\theta$.
This inequality follows directly from the fact that the resubmission gap in the first setting is larger, i.e., $\tau(\theta) > \tau'(\theta)$ for any $\theta$. 
By \cref{prop:monotone-prob-threshold}, the claim follows, completing the proof.

\subsection{Proof of \cref{prop:blackwell}}\label{app:proof-blackwell}

We will show that any memoryless acceptance policy $\ACCMAP[']: \SigSet' \to [0,1]$ in the second setting can be simulated in the first setting with the same expected conference quality and review burden. It follows that both the QB-tradeoff and the QA-tradeoff in the first setting weakly dominate those in the second setting.

Because $\RevSigDist$ Blackwell dominates $\RevSigDistP$, there exists a garbling $\gamma$ from $\SigSet$ to $\SigSet'$.  
We define a memoryless acceptance policy $\ACCMAP$ in the first setting: for any review signal $\REVSIG$, we set
\begin{align*}
\AccMap{\REVSIG} & = \sum_{\REVSIGP \in \SigSet'} 
\AccMap[']{\REVSIGP} \cdot \gamma (\REVSIG, \REVSIGP). 
\end{align*}

First, because $(\gamma(\REVSIG, \REVSIGP))_{\REVSIGP \in \SigSet'}$ is a distribution on $\SigSet'$ and $\AccMap[']{\REVSIGP} \in [0,1]$ for all $\REVSIGP$, the output of $\ACCMAP$ is in $[0,1]$. 
Moreover, for any paper quality $\qual \in \QualSet$,
\begin{align*}
\AccP{\ACCMAP}{\qual} 
& = \sum_{\dkreplace{\RevSigV}{\REVSIG} \in \SigSet} 
\RevSigProb[\qual]{\REVSIG}
\cdot \AccMap{\REVSIG} \\
& = \sum_{\dkreplace{\RevSigV}{\REVSIG} \in \SigSet} 
 \RevSigProb[\qual]{\REVSIG}
\cdot \sum_{\REVSIGP \in \SigSet'} \AccMap[']{\REVSIGP} \cdot
\gamma(\REVSIG, \REVSIGP) \tag{Definition of $\ACCMAP$}\\
& = \sum_{\REVSIGP \in \SigSet'} \left( \sum_{\REVSIG \in \SigSet}
\RevSigProb[\qual]{\REVSIG} \cdot \gamma(\REVSIG, \REVSIGP) \right)
\cdot \AccMap[']{\REVSIGP} \tag{Changing order of summation} \\
& = \sum_{\REVSIGP \in \SigSet'}  \RevSigProbP[\qual]{\REVSIGP} \cdot \AccMap[']{\REVSIGP} \tag{$\gamma$ is a garbling} \\
& = \AccP{\ACCMAP[']}{\qual}.
\end{align*}

Thus, the acceptance policies $\ACCMAP$ and $\ACCMAP[']$ have identical acceptance probabilities for each paper quality; this makes them indistinguishable to authors. In particular, both acceptance policies have the same expected conference quality and review burden. \Halmos


\subsection{Proof of \cref{prop:blackwell-threshold}}\label{app:proof-blackwell-threshold}

We first prove \cref{claim:blackwell-RB-better}, which is the key to the proof of \cref{prop:blackwell-threshold}.
We note that our proof is written for the categorical model. However, the proof for the continuous model is analogous, simply by replacing the summations with integrals.
\yzcomment{Note for future: better to check if this is precisely true.}
% \fangdelete{
% \restatedlemma{claim:blackwell-RB-better}{
%     Consider two threshold acceptance policies $\ACCMAP$ and $\ACCMAP[']$ which accept papers of quality $\bar{q}$ with \yichiedit{equal} probability in the first and the second setting, respectively.  
%     Then, under the author's $\bar{q}$-threshold strategy, 
%     % the review burden of $\ACCMAP$ in the first setting is no larger than the review burden of $\ACCMAP'$ in the second setting.  
%     \yichiedit{the acceptance probability of a paper of quality $q$ in the first setting is no less than that in the second setting, for any $q>\bar{q}$.}
%     }}


\proof{Proof of \cref{claim:blackwell-RB-better}.}
    Let $(\tau, r)$ and $(\tau', r')$ be the parameters corresponding to the threshold policies $\ACCMAP$ and $\ACCMAP'$, respectively.
    Given that the acceptance probability of $\bar{q}$ is the same under $\ACCMAP[\tau,r]$ and $\ACCMAP[\tau', r']$, we want to show that for every $q > \bar{q}$, the acceptance probability is weakly higher under $\ACCMAP[\tau,r]$ than under $\ACCMAP[\tau', r']$.
    By decomposing the acceptance probability into the individual signals, we need to show that
    \begin{equation} \label{eq:blackwell_ineq}
        r \cdot \RevSigProb[q]{\tau} + \sum_{s>\tau} \RevSigProb[q]{\REVSIG} 
        - \left(r'\cdot \RevSigProbP[q]{\tau'} + \sum_{\REVSIGP>\tau'} \RevSigProbP[q]{\REVSIGP} \right) 
        \ge 0,
    \end{equation}
    for any $q > \bar{q}$.

    Let $\gamma$ be the garbling from $\RevSigDist$ to $\RevSigDistP$ (see \cref{def:blackwell}), so that $\RevSigProbP[q']{\REVSIGP} = \sum_{\REVSIG} \RevSigProb[q']{\REVSIG} \cdot \gamma(\REVSIG, \REVSIGP)$ for any $q'$ and $\REVSIGP$. 
    Substituting the garbling-based characterization and changing the order of summation, the left-hand side of \cref{eq:blackwell_ineq} can be rewritten as
    \begin{align*}
        \lefteqn{r \cdot \RevSigProb[q]{\tau} + \sum_{s>\tau} \RevSigProb[q]{\REVSIG}
        - \left( 
        r' \cdot \sum_{\REVSIG} \RevSigProb[q]{\REVSIG} \cdot \gamma(\REVSIG, \tau')
        + \sum_{\REVSIGP>\tau'} \sum_{\REVSIG} \RevSigProb[q]{\REVSIG} \cdot \gamma(\REVSIG, \REVSIGP)  \right)}
        \\ & =
        r \cdot \RevSigProb[q]{\tau} + \sum_{s>\tau} \RevSigProb[q]{\REVSIG} 
        - \sum_{\REVSIG} \left( r'\cdot\gamma(\REVSIG, \tau') + \sum_{\REVSIGP>\tau'} \gamma(\REVSIG, \REVSIGP) \right) \cdot \RevSigProb[q]{\REVSIG}.
    \end{align*}
    
    For notational simplicity, let $h(\REVSIG)=  r'\cdot\gamma(\REVSIG, \tau') + \sum_{\REVSIGP>\tau'} \gamma(\REVSIG, \REVSIGP)$. 
    Because $h(s)$ is a probability (the probability of observing an instantiation of the Blackwell dominated signal that is accepted by $\ACCMAP[\tau', r']$ when the instance of the Blackwell dominating signal is $s$), $0\le h(\REVSIG)\le 1$. 
    We now break the summation over $s$ in the above equation into three summations, namely, the summation over $s<\tau$, $s=\tau$, and $s>\tau$. Combining the summations over the same subset of signals allows us to rewrite the preceding equation as
    \begin{align}
        \lefteqn{r \cdot \RevSigProb[q]{\tau} + \sum_{s>\tau}\RevSigProb[q]{\REVSIG} 
        - \sum_{\REVSIG} h(\REVSIG) \cdot \RevSigProb[q]{\REVSIG}} \nonumber 
       \\ & =
          (r-h(\tau)) \cdot \RevSigProb[q]{\tau} 
        + \sum_{s>\tau} (1-h(\REVSIG)) \cdot \RevSigProb[q]{\REVSIG} 
        - \sum_{s<\tau} h(\REVSIG) \cdot \RevSigProb[q]{\REVSIG}.\label{eq:blackwell_acc_prob_diff}
    \end{align}

    Next, we use the monotone likelihood ratio property.
    Let $\eta_s := \frac{\RevSigProb[q]{s}}{\RevSigProb[\bar{q}]{s}}$ denote the likelihood ratio for signal $s$. 
    Because signals have MLR, we have $\eta_{s'} > \eta_s$ for any $s'>s$.
    Using this monotonicity, we can bound \eqref{eq:blackwell_acc_prob_diff} as
    \begin{align*}
        \eqref{eq:blackwell_acc_prob_diff} 
        & = (r-h(\tau)) \cdot \eta_\tau \RevSigProb[\bar{q}]{\tau} 
        + \sum_{s>\tau} (1-h(\REVSIG)) \cdot \eta_s \RevSigProb[\bar{q}]{\REVSIG} 
        - \sum_{s<\tau} h(\REVSIG) \cdot \eta_s \RevSigProb[\bar{q}]{\REVSIG}
        \\ & \ge
        (r-h(\tau)) \cdot \eta_\tau \RevSigProb[\bar{q}]{\tau} 
        + \sum_{s>\tau} (1-h(\REVSIG)) \cdot \eta_\tau \RevSigProb[\bar{q}]{\REVSIG} 
        - \sum_{s<\tau} h(\REVSIG) \cdot \eta_\tau \RevSigProb[\bar{q}]{\REVSIG} 
        \\ & =
        \eta_\tau \cdot \left(
        r\cdot \RevSigProb[\bar{q}]{\tau} 
        + \sum_{s>\tau} \RevSigProb[\bar{q}]{\REVSIG} 
        - \sum_{\REVSIG}h(\REVSIG)\RevSigProb[\bar{q}]{\REVSIG}
        \right)
        \\ & =
        \eta_\tau \cdot 
        \left( \AccP{\ACCMAP[\tau,r]}{\bar{q}} - \AccP{\ACCMAP[\tau',r']}{\bar{q}} \right)
        \\ & = 0.
    \end{align*}
    Here, the inequality uses the monotone likelihood property separately for $s > \tau$ and $s < \tau$ (observing the signs of the multipliers of $\eta_s$), and the final step follows from the assumption that the acceptance probabilities of papers of quality $\bar{q}$ in both settings are equal. \Halmos
\endproof

With this lemma, we are able to prove \cref{prop:blackwell-threshold}.
\proof{Proof of \cref{prop:blackwell-threshold}.}  
For any responsive threshold acceptance policy $\ACCMAP[']$ in the second setting (the one that has weaker review signals), let $\vartheta = (\theta, r)$ be any of the author's equilibrium strategies to $\ACCMAP[']$.
% (As discussed in \cref{prop:de_facto}, there might be more than one best response.)
We will show the existence of a threshold acceptance policy $\ACCMAP$ in the first setting such that: 1) $\vartheta$ is also an equilibrium strategy to $\ACCMAP$; 2) the conference quality is the same in both settings; 3) the review burden induced by the policy-response pair $(\ACCMAP, \vartheta)$  in the first setting is at most the review burden induced by the policy-response pair  $(\ACCMAP', \vartheta)$ in the second setting; and 4) the author welfare induced by the policy-response pair $(\ACCMAP, \vartheta)$ in the first setting is at least the author welfare induced by the policy-response pair  $(\ACCMAP', \vartheta)$ in the second setting. If the above is true, then every point on the QB-tradeoff curve of the second setting is weakly dominated by a point on the QB-tradeoff curve of the first setting, which completes the proof. The same arguments apply to QA-tradeoff curves.
Note that it is sufficient to consider only responsive threshold policies because any policy that is not responsive either attracts no submissions (where the claim trivially holds) or accepts all submissions with a probability of 1 (which leads to a conference value smaller than 1 and thus is not a candidate threshold).

% First note that if $\ACCMAP[']$ always accepts or always rejects, then the proposition trivially holds.  
% Thus, we can assume that $\ACCMAP[']$ is non-trivial. 
By \cref{prop:de_facto}, the author's equilibrium strategy under a responsive threshold acceptance policy is a threshold strategy with some threshold $\theta$ (and probability $r$): the author will always submit a paper with quality $\Qual > \theta$, not submit if $\Qual < \theta$, and submit with probability $r$ if $\Qual = \theta$.
% Additionally, if $\vartheta$ is the strategy under which the author always submits everything, then letting $\ACCMAP$ be the policy that accepts everything minimizes the review burden without changing the conference quality. Therefore, without loss of generality, we can assume that not every paper is submitted under $\vartheta$.

Now, we consider two cases based on whether there are authors who are indifferent between submitting and not submitting.

\begin{enumerate}
\item The first case is that there exists a quality $\bar{q}$ such that an author with threshold strategy $\vartheta$ submits papers of quality $\bar{q}$ with probability $r$ in equilibrium. Because $\vartheta$ is a threshold strategy, there can be at most one such quality. Let $\rho = \rho(\theta, r)$ be the conference attractiveness under this setting. By \cref{prop:de_facto}, the acceptance probability at $\bar{q}$ must be exactly $1/\rho$.\fangcomment{Minor issue: this cannot be directly derived from \cref{prop:de_facto} }

Now, consider a threshold acceptance policy $\ACCMAP$ in the first setting that also induces $\vartheta$ as the author's best response. By construction, $\bar{q}$ is a candidate threshold. Therefore, the existence of $\ACCMAP$ is guaranteed by \cref{prop:threshold-policy}.
% Note that $\theta$ is a best response to $\phi'$ because by \gs{fill in ref here.} the acceptance rate is strictly monotone in the paper quality. 
Note that the conference quality and the attractiveness $\rho$ remain unchanged: papers with qualities strictly greater than $\bar{q}$ are all accepted, no paper with quality strictly less than $\bar{q}$ is accepted; and papers with quality $\bar{q}$ are submitted with probability $r$, and these papers are accepted i.i.d.~with probability $1/\rho$ in each setting. 
Therefore, by \cref{claim:blackwell-RB-better}, all submitted papers have a weakly larger acceptance probability under $\ACCMAP$ in the first setting, resulting in a weakly smaller review burden and a weakly larger author welfare.

\item In the second case, for any possible paper quality $q$, an author with threshold strategy $\vartheta$ either always or never submits papers of quality $q$.  
% Such a $\vartheta$ exists because we are assuming the categorical model, and we assumed that not every paper is submitted. 
Let $\bar{q}$ be the highest paper quality which is never submitted and let $\rho = \rho(\bar{q}, 0)$ be the conference attractiveness.

Let $\hat{\phi}'$ denote the threshold policy which accepts papers of quality $\bar{q}$ with probability $1/\rho$ in the second setting. 
Note that both $\ACCMAP'$ and $\hat{\phi}'$ induce $\vartheta$ as the author's best response, because we can assume that authors with papers of quality $\bar{q}$ never submit under $\hat{\phi}'$, given that $\AccP{\hat{\phi}'}{\bar{q}} = 1/\rho$, i.e., the authors are indifferent between submitting papers of quality $\bar{q}$ or not. 
This implies that $\hat{\phi}'$ induces the same conference quality as $\ACCMAP'$. Fixing the author's strategy $\vartheta$, the review burden of $\hat{\phi}'$ is no larger than that of $\ACCMAP'$ and the author welfare of $\hat{\phi}'$ is no less than that of $\ACCMAP'$. This is because by definition, $\ACCMAP'$ accepts papers of quality $\bar{q}$ with probability at most $1/\rho$, which is the acceptance probability of papers of quality $\bar{q}$ under $\hat{\phi}'$. Then, by \cref{lem:stricter_policy}, the acceptance probability of papers of any quality is weakly larger under $\hat{\phi}'$ than $\ACCMAP'$, resulting in a weakly smaller review burden and a weakly larger author welfare under $\hat{\phi}'$.

% First, we observe that the review burden of $\hat{\phi}$ is at most that of $\phi$ in setting one with author strategy $\theta$.  This is because the threshold of $\hat{\phi}$ is at most that of $\phi$.  Notice that $\phi$ accepts papers with quality $\bar{q}$ with probability at most $\rho$, because otherwise, $\theta$ would not be a best response.  If $\phi$ accepts papers with quality $\bar{q}$ with probability exactly $\rho$, then by definition, the threshold of $\hat{\phi}$ is at most that of $\phi$.   If $\phi$ accepts papers with quality $\bar{q}$ with probability strictly less than  $\rho$, then, the threshold of $\hat{\phi}$ must be strictly less than that of $\phi$ because acceptance probability is monotone in threshold.  \gs{should we cite something here?}

Finally, in the first setting, let $\ACCMAP$ be a threshold policy that induces a threshold best response with $(\bar{q}, 0)$ for the author such that authors with paper quality $\bar{q}$ prefer not to submit and $\AccP{\ACCMAP}{\bar{q}} = 1/\rho(\bar{q}, 0)$. This implies that the conference attractiveness is $\rho = \rho(\bar{q}, 0)$, the same as in the second setting. Because $\ACCMAP$ is a responsive threshold policy, by \cref{prop:de_facto}, $\bar{q}$ is a de facto threshold. Therefore, the existence of $\ACCMAP$ is guaranteed by \cref{prop:threshold-policy}. By this construction, first note that $\vartheta$ is also a best response to $\ACCMAP$.
Second, the conference quality under $\ACCMAP$ is identical to the conference quality under $\ACCMAP'$ because the same set of papers is submitted, and all are eventually accepted. Finally, by \cref{claim:blackwell-RB-better}, because papers of quality $\bar{q}$ are accepted with probability $1/\rho$ under $\ACCMAP$ in the first setting and under $\hat{\phi}'$ in the second setting, all papers with quality larger than $\bar{q}$ have a larger acceptance probability in the first setting. Therefore, the review burden of $\ACCMAP$ in the first setting is no larger than the review burden of $\hat{\phi}'$ which is no larger than the review burden of $\ACCMAP'$ in the second setting. The same argument holds analogously for author welfare. This completes the proof. \Halmos
\end{enumerate}
\endproof

\subsection{Proof of \cref{lemma:query burden value}}\label{app:proof-QB-tradeoff}

% To verify the definition of the dominance of QB-tradeoff curves, we will show that for any point on $\mathcal{C'}$ that neither corresponds to accepting all papers nor to rejecting all papers, there is a point on $\mathcal{C}$ that has the same conference quality but strictly smaller review burden.

We first consider the continuous model. Fixing a de facto threshold $\theta$, whose existence as a candidate threshold is ensured by \cref{prop:de_facto}), the conference value $\ConfValue(\theta)$ is fixed. Therefore, the conference attractiveness factor in the setting with a larger $\TD$, which is $\rho = \frac{\ConfValue(\theta) - \TD}{1-\TD}$, is larger than that in the setting with a smaller $\rho'$.

Next, we show that the setting with a larger attractiveness factor $\rho$ has a QB-tradeoff curve dominated by the setting with a smaller factor $\rho' < \rho$.

By \cref{prop:threshold-policy}, both settings have respective unique corresponding non-trivial threshold acceptance policies with thresholds $\tau$ and $\tau'$. By \cref{prop:de_facto}, a paper with quality $\theta$ is accepted with probabilities $1/\rho$ and $1/\rho'$, respectively, in the two settings.  
This means that papers with quality $\theta$ are accepted with strictly smaller probability by $\ACCMAP[\tau]$ than $\ACCMAP[\tau']$. 
By \cref{lem:stricter_policy}, $\ACCMAP[\tau]$ is stricter than $\ACCMAP[\tau']$.
Thus, for papers of every quality $q$, the expected number of rounds that a paper of quality $q$ has to be resubmitted is strictly larger in the setting with attractiveness factor $\rho$, which implies a strictly larger review burden.
Therefore, the QB-tradeoff curve in the setting with a larger attractiveness factor, is dominated by the QB-tradeoff curve in the other setting.

The argument for categorical models is essentially the same. We fix the authors' threshold strategy $(\bar{q}, r)$ such that authors with papers of quality $\Qual>\bar{q}$ submit, authors with $\Qual<\bar{q}$ take the outside option, and authors with $\Qual = \bar{q}$ submit with probability $r$. This fixes the conference quality and the conference value $\ConfValue(\bar{q}, r)$. 

We want to show that the acceptance policy which induces such an author's best response has a smaller review burden in the setting with $\rho'$ than in the setting with $\rho>\rho'$. Without loss of generality, suppose $r<1$. If $r\in (0,1)$, by \cref{lem:author_response}, the acceptance probability $\AccP{\ACCMAP}{\bar{q}} = 1/\rho < \AccP{\ACCMAP'}{\bar{q}} = 1/\rho'$. This means that the acceptance policy $\ACCMAP$ is stricter than $\ACCMAP'$. Based on the same argument as in the continuous model, this means that $\ACCMAP$ leads to a larger review burden. 
If $r = 0$, there might be multiple acceptance policies that induce the same threshold equilibrium. Let $\ACCMAP$ be any one of them. Because papers of quality $\bar{q}$ are not submitted, we must have $\AccP{\ACCMAP}{\bar{q}} \leq 1/\rho$. Now, let $\ACCMAP[']$ be a threshold acceptance policy in the setting with attractiveness $\rho'$ which accepts papers of quality $\bar{q}$ with probability $1/\rho'$. Such a policy $\ACCMAP[']$ exists by Proposition~\ref{prop:threshold-policy}, i.e., $\ACCMAP[']$ induces $\bar{q}$ as a de facto threshold. Because $1/\rho' > 1/\rho$, by \cref{lem:stricter_policy}, $\ACCMAP$ is stricter than $\ACCMAP[']$. Again, based on the same argument, this completes the proof.
%
% The argument for categorical models is essentially the same, but slight care needs to be taken to account for the non-uniqueness of threshold policies. Note that fixing a de facto threshold does not fix the conference quality in the categorical model when $\theta\in\QualSet$. However, we can always additionally fix the author's strategy (for both settings considered) if there exists a paper quality for which the authors are indifferent between submitting or not; this fixes the conference quality. Therefore, fixing a de facto threshold $\theta$ is still sufficient to complete the proof.
% Let $q < \theta$ be the largest quality in the quality set $\QualSet$ that is strictly smaller than $\theta$. (We define $q$ this way even if $\theta$ itself is in the quality set.)
% Such a $q$ must exist because $\theta$ was a non-trivial threshold.
% Let $\ACCMAP$ be any threshold acceptance policy which induces $\theta$ as the de facto threshold in the setting with conference attractiveness $\rho$.
% Because papers of quality $q$ are not submitted, we must have $\AccP{\ACCMAP}{q} \leq 1/\rho$.
% Now, let $\ACCMAP[']$ be a threshold acceptance policy in the setting with attractiveness $\rho'$ which accepts papers of quality $q$ with probability $1/\rho'$. Such a policy $\ACCMAP[']$ exists by Proposition~\ref{prop:threshold-policy}. Because $1/\rho' > 1/\rho$, by Lemma~\ref{lem:stricter_policy}, $\ACCMAP$ is stricter than $\ACCMAP[']$.
% Therefore, all papers are accepted with at least the same probability under $\ACCMAP[']$ as under $\ACCMAP$. 
% Therefore, the review burden under $\ACCMAP[']$ is strictly smaller than under $\ACCMAP$, and since this holds for all non-trivial $\theta$, we have shown domination of the QB-tradeoff.
\Halmos

\subsection{Proof of \cref{lem:QA_eta}}\label{app:proof-QA-eta}

% \dkcomment{Also, why are we writing $\partial$? Why not just $d$? Isn't everything just scalar?}\yzcomment{chanegd}

  Define $g(\TD) = \frac{1-\TD}{\AccP{\tau(\TD)}{q}} + \TD$. 
  Because $u^{(a)}(q,\TD) = \frac{\ConfValue}{g(\TD)}$, we know that $u^{(a)}$ is increasing in $\TD$ if and only if $g$ is decreasing in $\TD$.
  Note that $\AccP{\tau(\TD)}{q} = 1 - \REVNOISEDIST(\tau(\TD) - q)$ so that $\frac{d \AccP{\tau(\TD)}{q}}{d \TD} = - f^{(r)}(\tau(\TD) - q) \cdot \frac{d \tau(\TD)}{d \TD}$.
    
    Recall that $\tau(\TD)$ is chosen so that authors with a paper of quality $\theta$ are indifferent between submitting and taking the outside option, i.e., $\AccP{\tau(\TD)}{\theta} = \frac{1-\TD}{\ConfValue-\TD}$. Taking the derivative of both sides of the equation w.r.t.~$\TD$, and combining with the previous derivative calculation, we have $\frac{d \tau(\TD)}{d \TD} = \frac{\ConfValue-1}{(\ConfValue-\TD)^2 \cdot f^{(r)}(\tau(\TD) - \theta)}$.

    Then, taking the derivative of $g$ w.r.t.~$\TD$ yields
    \begin{align*}
        g'(\TD) &= 1 + \frac{-(1-\REVNOISEDIST(\tau(\TD) - q)) + (1-\TD) \cdot \frac{(\ConfValue-1)f^{(r)}(\tau(\TD) - q)}{(\ConfValue-\TD)^2 \cdot f^{(r)}(\tau(\TD) - \theta)}}{(1-\REVNOISEDIST(\tau(\TD) - q))^2}.
    \end{align*}
    $g'(\TD)$ is positive if and only if 
    \begin{align*}
        (1-\REVNOISEDIST(\tau(\TD) - q))^2 -(1-\REVNOISEDIST(\tau(\TD) - q)) + (1-\TD) \cdot \frac{(\ConfValue-1) \cdot f^{(r)}(\tau(\TD) - q)}{(\ConfValue-\TD)^2 \cdot f^{(r)}(\tau(\TD) - \theta)} 
        & > 0,
    \end{align*}
    which can be rearranged to
    \begin{align*}
    \frac{(1-\TD) \cdot (\ConfValue-1) \cdot f^{(r)}(\tau(\TD) - q)}{(\ConfValue-\TD)^2 \cdot f^{(r)}(\tau(\TD) - \theta)} 
    & > \REVNOISEDIST(\tau(\TD) - q) \cdot (1-\REVNOISEDIST(\tau(\TD) - q)).
    \end{align*}
    Because $\frac{1-\TD}{\ConfValue-\TD} = 1 - \REVNOISEDIST(\tau(\TD) - \theta)$ and $\frac{\ConfValue-1}{\ConfValue-\TD} = \REVNOISEDIST(\tau(\TD) - \theta)$, the preceding inequality can be further rearranged to
    \begin{align*}
       \frac{\REVNOISEDIST(\tau(\TD) - \theta) \cdot (1-\REVNOISEDIST(\tau(\TD) - \theta))}{f^{(r)}(\tau(\TD) - \theta)} 
       & > \frac{\REVNOISEDIST(\tau(\TD) - q) \cdot (1-\REVNOISEDIST(\tau(\TD) - q))}{f^{(r)}(\tau(\TD) - q)},
    \end{align*}
    which holds if and only if $h(q) > h(\theta)$.

Therefore, the author's utility is increasing in $\TD$ if and only if $h(q) < h(\theta)$. By reversing all inequalities in the preceding calculations (i.e., comparison of the derivative with 0), we can show that the author's utility is decreasing in $\TD$ if and only if $h(q) > h(\theta)$, and remains unchanged in $\TD$ if $h(q) = h(\theta)$.\Halmos