% \gscomment{Reorg.  Define threhsold acc policy and threshold response.  Give intuitive overview of current 3.1  (say that threshold policies are general, and authors typically have de facto threshold.  Then give powerful story of resubmission gap.  Then 3.1 becomes "threshold policies and the resubmission paradox".  3.2 "an illustratrative example with Gaussians"  do an example of Gaussian (what is currently 3.3).  Then 3.3 "Authors best response policies".  3.4 "Generality of threshold strategies" becomes the current technical work from 3.1. "Generality of threshold strategies" (last two might be combined?)   }

We first focus on the case of noiseless authors. Recall that in this setting, because the author will not update her belief about the paper's quality, she will either submit to the conference until the paper is accepted (or, in the case of limited submissions, until the paper may not be resubmitted anymore), or immediately take the side option. 
In turn, this allows us to analytically characterize the relationship between the author's submission equilibrium and the conference's acceptance policy.

% \dkdeletecomment{I really don't like ``Table of Contents'' paragraphs, and even less for just a single section. I don't think a reader gets anything out of this paragraph.}{
% \yzreplace{
% In this section, we focus on the \emph{resubmission gap}: the difference between the threshold the conference sets for acceptance and the actual threshold of accepted papers, taking into account the resubmission of previously rejected papers. An analysis of the resubmission gap is of interest in its own right (since it may crucially inform conference acceptance policies), and also serves as the foundation of our further investigation of tradeoffs in \cref{sec:continuous-tradeoffs}.
% }{
% This section is organized as follows. In \cref{subsec:paradox}, we \gsreplace{present}{state} the high-level technical results and highlight their key practical implications\textemdash an explanation of the resubmission paradox. In \cref{sec:additive-noise}, we then introduce a simple \gsreplace{yet}{and} intuitive additive noise model to illustrate how model parameters shape the author’s equilibrium behavior. Finally, we provide general theoretical justifications for our claims across a range of models and acceptance policies in \cref{subsec:threshold_broader}.}}

\subsection{Threshold Policies and The Resubmission Paradox}
\label{subsec:paradox}

% \begin{quote}
% \emph{At many prestigious conferences, a nontrivial fraction of papers rejected in one year are resubmitted to the same or closely related venues in subsequent years and are eventually accepted --- sometimes with only minor changes.}
% \end{quote}

% \dkcomment{Are we actually quoting someone here? If not, quote environment seems wrong. Instead, I would suggest something like the following:}
Our analysis begins from the following often puzzling real-world phenomenon: 
At many prestigious conferences, a nontrivial fraction of rejected papers are subsequently resubmitted (often multiple times) to the same or closely related venues, only to be eventually accepted --- sometimes with only minor changes.

We refer to this counterintuitive phenomenon as the \emph{resubmission paradox}.
Its prevalence has led to suggestions of raising acceptance rates so that more of these papers are accepted in their initial round of submission, thereby reducing repeated reviews and lowering the review burden on the community.
\yzcomment{It'd be nice to have a citation.}

In this subsection, we explain the resubmission paradox by uncovering the relationship between a conference’s acceptance policy and authors’ equilibrium behavior. The paradox turns out not to be an accident, but a byproduct of two forces: the conference’s desire to maintain its prestige and authors’ self-selection in deciding where to submit. Simply lowering the acceptance threshold sounds like it should help, but it risks drawing in more lower-quality submissions and thus fails to reduce --- and may even increase --- the overall review burden.

\paragraph{Threshold equilibrium.}
In order to understand the effects of a conference's acceptance policy, it is necessary to characterize the authors' equilibrium under the policy. Fortunately, under mild conditions on the acceptance policy, authors follow a threshold strategy at equilibrium, submitting papers if and only if the quality exceeds some threshold. This observation motivates a central concept in our analysis, which offers a simple and intuitive explanation for the resubmission paradox.


\begin{definition}[De Facto Threshold]
Consider a conference with a memoryless acceptance policy~$\ACCMAP$ and noiseless authors.  
We say that a value~$\theta$ is the \emph{de facto threshold under $\ACCMAP$} if, in any non-atomic symmetric equilibrium\footnote{``Non-atomic'' here means that any unilateral deviation by a single author does not affect the conference value $\ConfValue$, and ``symmetric'' means that all authors follow the same strategy.}, authors submit --- and continue resubmitting --- any paper with quality~$\Qual = q$ whenever $q > \theta$, and take the outside option if $q < \theta$.
%\dkedit{$\theta$ is a \emph{de facto threshold} if it is a de facto threshold under some policy $\ACCMAP$.}
\end{definition}

We call $\theta$ the de facto threshold under $\ACCMAP$ because every paper whose quality is above $\theta$ will be submitted until accepted, so de facto, the conference will accept all papers above $\theta$. 
Based on the definition, the de facto threshold exists for an acceptance policy \ACCMAP if and only if authors submit papers according to the above threshold strategy in every non-atomic symmetric equilibrium. In particular, if in a strategy profile, there are qualities $q \neq q'$ such that an author is indifferent between submitting or not submitting papers of qualities $q$ and $q'$, then a de facto threshold does not exist under \ACCMAP.


% \gscomment{I think that the order of the bullet points below should be reversed.  Then it goes from most easy to understand to harder to understand.  I think that will help the reader.}
% \yzcomment{Sounds good. Do we also want to change the proposition?}\gscomment{I think so, unless it makes the proof more difficult, in which case just leave it.}
We defer a detailed discussion of the existence of the de facto threshold to \cref{sec:submission-threshold}, and provide only a high-level summary (of \cref{prop:de_facto}) here:  
\begin{itemize}[leftmargin=2em]
    \item In the continuous model, under a threshold acceptance policy~$\ACCMAP[\tau]$, every equilibrium corresponds to a unique de facto threshold~$\theta$, and $\theta$ and $\tau$ are in one-to-one correspondence.
    \item In the categorical model, under a threshold acceptance policy~$\ACCMAP[\tau,r]$, every equilibrium corresponds to some de facto threshold~$\theta$, but the mapping from $\tau$ to $\theta$ may not be unique.
    \item In both the categorical and continuous models, a de facto threshold~$\theta$ exists under any monotone \ACCMAP. However, other non-threshold equilibria may also exist.
\end{itemize}

Threshold acceptance policies comprise a very natural class of policies for a conference to apply. We further justify this by showing the following in \cref{prop:threshold-policy}: If $\theta$ is such that including all papers of quality strictly exceeding $\theta$, and possibly some papers of quality exactly $\theta$, would result in a conference of value $\ConfValue > 1$, then there exists a threshold acceptance policy inducing $\theta$ as its de facto threshold.

The concept of de facto thresholds, along with the characterization from the aforementioned \cref{prop:threshold-policy}, allows for a clean characterization of the policy maximizing conference quality:

% \dkcomment{Moved up the proposition, proof, and discussion from 3.3, per Grant's suggestion.}
\begin{proposition}\label{prop:max-general}
The conference quality-maximizing policy induces a de facto threshold of $\theta = 0$, and there is a threshold acceptance policy inducing this de facto threshold.
 \end{proposition}

\proof{Proof of \cref{prop:max-general}}
Given a de facto threshold $\theta$, the conference's quality $\CONFUTIL$ is $\int_{\theta}^\infty \qual \,d\QualDens{\qual}$. (For discrete quality distributions, the corresponding conference quality is obtained by replacing the integral with the sum and the density function with the probability function.)

This integral/sum is maximized at $\theta=0$.
Furthermore, because we assumed that papers of positive quality occur with positive probability, the integral is strictly positive for $\theta = 0$, so that $\ConfValue > 1$.
By the discussion preceding this proposition (and proved in \cref{prop:threshold-policy}), $\theta=0$ is induced as a de facto threshold by some threshold acceptance policy.
% \gscomment{it is not clear to me that there is a general way to find a threshold policy that maps to some defacto threshold.  Could we modify prop 5 so that this is more clear?}
\Halmos
\endproof


\paragraph{Explaining the resubmission paradox.}
The acceptance threshold and the de facto threshold tend to be different.
The threshold $\tau$ can be interpreted as the ``declared'' quality goal of the conference, attained in isolation. $\theta$ determines the actual quality of the conference, taking into consideration resubmissions and reviewing noise. The difference between $\tau$ and $\theta$ is a key concept we study, which we term the ``resubmission gap''.

\begin{definition}[Resubmission Gap]
\label{def:resubmission-gap}
Consider a memoryless conference with a non-trivial threshold acceptance policy $\ACCMAP[\tau,r]$, and a noiseless author.
Let $\theta$ be the smallest de facto threshold under $\ACCMAP[\tau,r]$. 
The difference between the acceptance threshold $\tau$ and $\theta$ is called the \emph{resubmission gap}.
\end{definition}
In general, the de facto threshold is not unique in the categorical model; we define the resubmission gap based on the \emph{smallest} de facto threshold. This issue disappears in the continuous model, where the resubmission gap is unique for every non-trivial threshold acceptance policy. 

The name ``resubmission gap'' reflects the discrepancy between the ``stated'' and ``de facto'' quality of accepted papers, caused by the fact that authors are free to resubmit papers and that the reviews are noisy.
The fact that the actual threshold of papers at the conference differs from the declared acceptance threshold for each year in isolation (namely, by the resubmission gap) may appear somewhat unexpected at first, and is frequently missing from conversations about policy changes. 

% \yzreplace{
% The resubmission gap leads to the following oddity, which we call the \emph{resubmission paradox}: all submitted papers will eventually be accepted, but in any one round, many (and often most) are rejected. This creates a substantial and seemingly unnecessary burden on the review process, as papers are reviewed multiple times before final acceptance. While accepting all papers initially seems like a solution, in our model, it is not.  The only way to initially accept all papers is to lower the acceptance threshold, which in turn lowers the de facto threshold and incentivizes additional submissions.  Moreover, these additional papers will often be rejected multiple times, but will be resubmitted until they are accepted.  Thus, this apparent solution actually reproduces the problem, just with a lower de facto threshold.
% }
% {
The resubmission gap is precisely what gives rise to the resubmission paradox highlighted at the start of this subsection. As mentioned there, a common proposal is to lower the acceptance threshold in order to reduce the review burden by cutting down on resubmissions. However, this approach is unlikely to help. While a smaller threshold does allow more papers to be accepted earlier, it simultaneously lowers the de facto threshold and encourages additional submissions. Moreover, many of these new submissions are sub-standard and will still be rejected multiple times until eventual acceptance. Thus, this apparent solution actually reproduces the problem, just with a lower de facto threshold.


% \gsreplace{Combining \cref{prop:max-general} with \cref{lem:author_response} \gscomment{Lemma 1 is in section 3.3--the future.  Ideally we would restate summary Prop 5 --- and perhaps prop 5 so that we don't need prop 1.  Another possibility would be to intutively state prop 1 above as well.  I really like this tighter story, but currently there are a few loose ends. } yields some insights into the right acceptance threshold for a conference aiming for maximum quality.
% Let $\hat{q} = \dkreplace{\min}{\inf}\Set{q \in \QualSet}{q\ge 0}$ be the smallest non-negative \gsedit{paper} quality.
% Under the conference quality-maximizing threshold acceptance policy, $\AccP{\ACCMAP[\hat{\tau},r]}{\hat{q}} = 1/\rho(\hat{q}, 1)$ so that all non-negative-quality papers are submitted and accepted. \gscomment{This section is currently really hard to read.  Much of this stuff has been defined.  It seems like this was copy/pasted but not modified to fit here.}  \gscomment{I feel like the next part would be better moved after the discusison on resubmission gap.} This equation suggests that the more attractive the conference is (i.e., the larger $\rho$), \gscomment{what is rho?  We never say.  Conference value is V later.} the more often borderline papers must be rejected. Because borderline papers are resubmitted until accepted, this means that for the same optimal quality, \yzreplace{an attractive conference (or patient authors)}{patient authors} \gscomment{IT is not at all clear how this relates to patient authors.}
% % \yzcomment{We still have a meaningful definition for $\ConfValue$}\fangcomment{The current wording, larger $V$ leads to big $R$, is unclear as $V$ is endogenous.} 
% leads to a higher reviewing load. We will investigate this phenomenon in significantly more depth in \cref{sec:dominating-value-discount}.}{

% NEW STUFF HERE
\paragraph{The importance of the resubmission gap.}
% \yzreplace{
% First, we want to highlight the importance of the size of the resubmission gap for many of our key results.  Next, we briefly discuss what determines its size.

% If we fix a de facto threshold, this will fix the conference quality (in the categortical model, ).  In particular, we saw above, that there was a particular de facto threshold which maximizes the conference quality.  It also fixes the value of the conference.   For the sake of argument, let's fix some de facto threshold.  Then the larger the rebmission gap, the higher the acceptance threshold.  Moreover, the higher the acceptance threshold, the more often each paper will need to be submitted before it is accepted.  Thus, a large resubmission gap will intutively lead to a higher review burden (as each submitted paper may need to be reviewed more), and lower author utility, as, for a fixed discount factor, the authors would like their papers accepted sooner rather than later.  

% We now describe, at a high level, what will impact the resubmission gap.  Let $\hat{\theta}$ \gscomment{is this notation okay?} be some de facto threshold which we seek to achieve with a threshold policy $\ACCMAP[\hat{\tau}]$.  We will see in \cref{lem:author_response} that to make this happen requires that the acceptance probability of a paper at the de facto threshold $\hat{\theta}$,  $\AccP{\ACCMAP[\hat{\tau},r]}{\hat{\theta}} = 1/\rho$ where $\rho$ is the attractiveness of the conference, which we will describe momentarily. Already, we can intuit that the more attractive the conference $\rho$ is, the smaller $1/\rho$ and the smaller the acceptance rate \AccP{\ACCMAP[\hat{\tau},r]}{\hat{\theta}} should be, and thus the higher the acceptance threshold.  It turns out that $\rho$ increases with the value of the conference (assuming the fixed de facto threshold) and the discount factor.  We  will be able identify three factors that reliably increase the resubmission gap: 1) the noise of the reviews, which, if large, will require a larger acceptance threshold to keep out borderline papers, 2) the discount factor which increases $\rho$, and 3) the distribution of paper, which, if shifted right, will increase the conference value at the prespecified de facto threshold and thus increase $\rho$.
% }
% {
We highlight the central role of the resubmission gap in many of our key results.

Fixing a de facto threshold fixes the conference quality $\CONFUTIL$ and the conference value $\ConfValue$. (In the categorical model, we also need to fix the submission probability in the edge case $\Qual = \theta$.) 
As we have seen, a particular de facto threshold $\theta = 0$ maximizes the conference quality (\cref{prop:max-general}).
Now, for a fixed de facto threshold, a larger resubmission gap implies a higher acceptance threshold. A higher threshold, in turn, means that each paper is expected to be submitted more times before eventual acceptance. Consequently, a larger resubmission gap naturally increases the review burden --- since each paper generates more reviews\textemdash and decreases author utility, as authors prefer earlier acceptance given a fixed discount factor.  

% Under the conference quality-maximizing threshold acceptance policy, $\AccP{\ACCMAP[\hat{\tau},r]}{\hat{q}} = 1/\rho(\hat{q}, 1)$ so that all non-negative-quality papers are submitted and accepted.

% First, leAbove, we asserted that, given a de facto threshold, there is an acceptance threshold that implements that de facto threshold.  The size of this gap, conceptually drives several of our key results.  



% Combining \cref{prop:max-general} with \cref{lem:author_response} \gscomment{Lemma 1 is in section 3.3--the future.  Ideally we would restate summary Prop 5 --- and perhaps prop 5 so that we don't need prop 1.  Another possibility would be to intutively state prop 1 above as well.  I really like this tighter story, but currently there are a few loose ends. } yields some insights into the right acceptance threshold for a conference aiming for maximum quality.
% Let $\hat{q} = \dkreplace{\min}{\inf}\Set{q \in \QualSet}{q\ge 0}$ be the smallest non-negative \gsedit{paper} quality.
% Under the conference quality-maximizing threshold acceptance policy, $\AccP{\ACCMAP[\hat{\tau},r]}{\hat{q}} = 1/\rho(\hat{q}, 1)$ so that all non-negative-quality papers are submitted and accepted. \gscomment{This section is currently really hard to read.  Much of this stuff has been defined.  It seems like this was copy/pasted but not modified to fit here.}  \gscomment{I feel like the next part would be better moved after the discusison on resubmission gap.} This equation suggests that the more attractive the conference is (i.e., the larger $\rho$), \gscomment{what is rho?  We never say.  Conference value is V later.} the more often borderline papers must be rejected. Because borderline papers are resubmitted until accepted, this means that for the same optimal quality, \yzreplace{an attractive conference (or patient authors)}{patient authors} \gscomment{IT is not at all clear how this relates to patient authors.}
% % \yzcomment{We still have a meaningful definition for $\ConfValue$}\fangcomment{The current wording, larger $V$ leads to big $R$, is unclear as $V$ is endogenous.} 

% \yzreplace{We will investigate this phenomenon in significantly more depth in \cref{sec:continuous-tradeoffs}.}

We will investigate the factors that affect the resubmission gap in \cref{sec:additive-noise}, and examine how the resubmission gap relates to another key concept in our paper\textemdash the QB- or QA-tradeoff\textemdash in \cref{sec:continuous-tradeoffs}.
% }



\subsection{An Illustrative Example in the Continuous Model With a Single Review}
\label{sec:additive-noise}
In order to illustrate the somewhat abstract definitions more concretely, we now characterize the relationship between the acceptance threshold $\tau$ and the de facto threshold $\theta$ specifically for the continuous model with $\NumReviews = 1$ review per paper. 
Importantly, recall that the review noise is additive and the noise distribution has zero mean and is independent of the true underlying quality.
That is, for any quality $\Qual=q$, the reviewer observes a signal of $q+X$ where the distribution of the review noise, $\RevNoiseDist{X}$, is independent of $q$. 

We first note that the conference value depends on the author's submission strategy. As we primarily consider symmetric threshold equilibria, we write the conference value $\ConfValue(\theta)$ as a function of a threshold quality $\theta$ such that authors with papers of quality $\Qual \ge \theta$ submit and resubmit until acceptance, and authors with $\Qual < \theta$ opt for the outside option. It follows that the conference value is strictly increasing in~$q$:
\[\ConfValue(\theta) = 1 + \frac{\int_\theta^\infty q \QualDens{q} dq}{\int_\theta^\infty \QualDens{q} dq}.\]

Under the single-review model, it is convenient to define a \emph{signal-based threshold acceptance policy}, which accepts a paper if and only if its review signal exceeds a threshold~$\tau_s$. Since review noise is additive to the true paper quality, and the noise distribution has an invertible cdf, there is a monotone one-to-one mapping between a signal threshold~$\tau_s$ and a threshold~$\tau$ applied to the paper’s expected quality (given a fixed prior on quality). 
The following proposition formalizes this correspondence.
% Under this model, the following proposition shows that
% % the resubmission gap of a threshold acceptance policy does not depend on its threshold $\tau$.
% there is a monotonic one-to-one correspondence between the acceptance threshold $\tau$ and the de facto threshold $\theta$.

\begin{proposition} \label{prop:gap-invariant}
  Given $\theta$ such that $\ConfValue(\theta)>1$, consider the continuous model with a single review and additive noise drawn from $\REVNOISEDIST$. 
  The acceptance threshold~$\tau_s$ of a signal-based threshold policy that induces $\theta$ as the de facto threshold is
  \[\tau_s = 
  % \theta + \left(\REVNOISEDIST\right)^{-1}{\left(1-\frac{1}{\rho(\theta)}\right)} = 
  \theta + \left(\REVNOISEDIST\right)^{-1}{\left(\frac{\ConfValue(\theta) - 1}{\ConfValue(\theta) - \TD}\right)},\]
  which is increasing in $\theta$.
  % In particular, the resubmission gap $\tau - \theta$ is independent of $\tau$ and the prior distribution $\QualDist$ over paper qualities.
\end{proposition}
% \fangcomment{this part is fishy.  The proposition provides the *threshold on the signal*, but $\tau$ should be the threshold on the conditional expectation $U(s)$.  I am worry, because the threshold acceptance policy $\tau$ should also depend on the prior distribution $\QualDist$. If there is no simple fix, we may need to further restrict to the two step Gaussian.}
% \fangcomment{This seems like a corollary of proposition 2.}
\proof{Proof of \cref{prop:gap-invariant}}
  % Consider an acceptance threshold $\tau$, and corresponding policy $\ACCMAP[\tau]$. Recall that under a continuous model, the probability that the conditional expected quality of a paper is exactly $\tau$ is 0, so $\tau$ uniquely defines the threshold acceptance policy.
  % By Proposition~\ref{prop:de_facto}, the de facto threshold $\theta$ satisfies $\AccP{\ACCMAP[\tau]}{\theta} = 1/\rho$.

We first show that the threshold acceptance policy can induce $\theta$ as a de facto threshold as long as $\ConfValue(\theta)>1$.
An author submits the paper if the expected utility of submitting is greater than 1 (the utility of the outside option), does not submit the paper if the expected utility is less than 1, and is indifferent between submitting and not submitting if the expected utility is equal to 1. 
Because a noiseless author does not learn any new information from rejection in previous rounds, she will make the same decision in future rounds, implying that she will submit until acceptance.
Let $\AccP{\ACCMAP[\tau_s]}{q}$ be the acceptance probability of a paper with quality $Q=q$ under a signal-based threshold acceptance policy. By Proposition~\ref{prop:monotone-prob}, this probability is strictly increasing in $q$.
The expected utility can be obtained as the time-discounted sum of the utility from acceptance:
% \dkcomment{Should this be $u^{(a)}$ or $U^{(a)}$? I am not quite clear on which we use for what.}\yzcomment{Should be small u. Small u is individual utility, U is welfare.}
\begin{align}
u^{(a)}(\ACCMAP[\tau_s],q)
& = 
\sum_{t\ge 1} \ConfValue \cdot \TD^{t-1} \AccP{\ACCMAP[\tau_s]}{q} \cdot (1-\AccP{\ACCMAP[\tau_s]}{q})^{t-1} 
\; = \; \frac{\ConfValue \cdot \AccP{\ACCMAP[\tau_s]}{q}}{1-\TD \cdot (1-\AccP{\ACCMAP[\tau_s]}{q})}.
\end{align}
Solving the inequalities $u^{(a)}(\ACCMAP[\tau_s],q) \ge 1$ and $u^{(a)}(\ACCMAP[\tau_s],q) < 1$, and $u^{(a)}(\ACCMAP[\tau_s],q) = 1$ for $q$, the author submits the paper if $\AccP{\ACCMAP[\tau_s]}{q} > (1-\TD)/(\ConfValue(\theta)-\TD)$, does not submit if $\AccP{\ACCMAP[\tau_s]}{q} < (1-\TD)/(\ConfValue(\theta)-\TD)$, and is indifferent between submitting and not submitting if $\AccP{\ACCMAP[\tau_s]}{q} = (1-\TD)/(\ConfValue(\theta)-\TD)$, respectively. 


  % In the continuous model, the de facto threshold \dkreplace{together with the}{and} quality distribution \dkedit{together} determine the conference value, i.e., $\ConfValue(\theta) = 1+ \frac{\int_\theta^\infty q\QualDens{q} dq}{\int_\theta^\infty \QualDens{q} dq}$.
  % By definition of the de facto threshold and \cref{lem:author_response}, $\AccP{\ACCMAP[\tau]}{\theta} = 1/\rho(\theta)$, where $\rho(\theta) = \frac{\ConfValue(\theta)-\TD}{1-\TD}$.

% \yzreplace{
% In the continuous model with a single review, the conference will accept a paper with quality $q$ if and only if the review $s$ satisfies $s=q+x>\tau$. This happens with probability $\AccP{\ACCMAP[\tau]}{q} = 1-\RevNoiseDist{\tau-q}$.
% Thus, $\tau$ solves $1/\rho(\theta) = 1-\RevNoiseDist{\tau-\theta}$. }

Under the signal-based threshold acceptance policy, a paper with quality $q$ is accepted if and only if the review $s$ satisfies $s=q+x>\tau_s$.
This happens with probability $\AccP{\ACCMAP[\tau_s]}{q} = 1-\RevNoiseDist{\tau_s-q}$.
Thus, $\tau_s$ solves $(1-\TD)/(\ConfValue(\theta)-\TD) = 1-\RevNoiseDist{\tau_s-\theta}$. 


A solution for $\tau$ exists because the left-hand side of the above equation lies between 0 and 1 (because $\TD \in (0, 1)$ and $\ConfValue(\theta) > 1$) and $\RevNoiseDist{\cdot}$ has a continuous cdf. 
The uniqueness of $\theta$ follows because we assumed $\RevNoiseDist{\cdot}$ to be \emph{strictly} increasing.
Rearranging the equation leads to the relationship as shown in the statement of the proposition.
\Halmos
\endproof

% \yzdelete{
% Substituting $\theta = 0$ from Proposition~\ref{prop:max-general} into Proposition~\ref{prop:gap-invariant}, 
% %we immediately obtain the following corollary.
% \yichiedit{we immediately observe that the conference quality-maximizing acceptance threshold is $\left(\REVNOISEDIST\right)^{-1}{\left(\frac{\ConfValue(0) - 1}{\ConfValue(0) - \TD}\right)}$.
% Under this acceptance policy, the author submits and resubmits the paper if and only if the quality is non-negative. The resulting maximum quality for a conference is $\CONFUTIL = \int_{0}^\infty \qual \,\QualDens{\qual}\,dq$.}
% }

Because there is a monotone one-to-one mapping between $\tau_s$ and $\tau$, the resubmission gap $\tau-\theta$ is monotone increasing in the term $\left(\REVNOISEDIST\right)^{-1}{\left(\frac{\ConfValue(\theta) - 1}{\ConfValue(\theta) - \TD}\right)}$.
\cref{prop:gap-invariant} thus implies that there are four factors that affect the resubmission gap: the author's discount factor $\TD$, the paper quality distribution $\QualDist$, the review noise distribution $\REVNOISEDIST$, and the de facto threshold $\theta$ itself. A large discount factor $\TD$ (very patient authors), a high-quality-skewed quality distribution $\QualDist$ (which leads to a larger $\ConfValue$ given the same $\theta$), a large review noise, and a higher de facto threshold (which leads to a larger $\ConfValue$) all contribute to a large resubmission gap.
This means that the conference has to set a significantly higher bar to sufficiently discourage such resubmissions, and will reject many good papers repeatedly before they are finally accepted. 
% \yzdelete{In contrast, the combination of a lower-tier conference with a low de facto threshold (and thus a low conference value) and impatient authors may result in a negative resubmission gap\fangcomment{Why can the resubmission gap be negative?  Isn't $V\ge 1$?}\textemdash meaning that the conference needs to significantly lower its acceptance threshold to provide strong enough assurance to good papers that they will be immediately accepted. Note that this can only occur for $\ConfValue(\theta) < 2$; otherwise, the resubmission gap is non-negative for authors with any level of patience.}
% \yzcomment{Removed the discussion on negative gap, which was wrong as we did not consider $\tau$ on the expected quality space. Whether a resubmission gap can be negative depends on the mapping from $\tau$ to $\tau_s$, which depends on the prior. This may need some numerical tests. No time for this revision.}

The above intuition carries over to the more general setting where a larger resubmission gap has an important practical implication: a setting with a smaller resubmission gap can achieve the same conference quality at a lower review burden.  In \cref{subsec:gap-tradeoffs}, we will see this formally and then study what factors reliably increase this gap.

% \begin{proposition} \label{prop:max}
%   For the continuous model with a single review and additive noise drawn from $\REVNOISEDIST$ (independently of the true paper quality), the threshold that maximizes the conference quality is
%   $\tau^* = \left(\REVNOISEDIST\right)^{-1}\left(\frac{\ConfValue-1}{\ConfValue-\TD}\right)$.

%   Then, the author submits (and resubmits until accepted) the paper if and only if
%   % \footnote{In this case, the event $\Qual = 0$ has probability 0, so there is a unique optimal strategy for the author.} 
%   the quality is non-negative: $\Qual \geq 0$. \yichicomment{Removed a footnote here.}
%   The resulting (maximum) quality for the conference is 
%   $\CONFUTIL = \int_{0}^\infty \qual \,d\QualDens{\qual}$.
% \end{proposition}

% First, note that \cref{prop:max} also implies that the quality-maximizing acceptance threshold does not depend on the distribution $\QualDist$ of the papers' qualities.   
% Second, notice that a large conference value $\ConfValue$ (very high prestige) or a large discount factor $\TD$ (very patient authors) encourages authors to consistently resubmit bad papers. This leads to a large resubmission gap: the conference has to set a significantly higher bar to sufficiently discourage such resubmissions, and will reject many good papers repeatedly before they are finally accepted. 
% In contrast, when the conference is not attractive enough or authors are not patient enough, the conference has to lower the acceptance threshold even below the de facto threshold, to provide strong enough assurance to good papers that they will be immediately accepted. 
% However, this can only occur for $\ConfValue < 2$; otherwise, the resubmission gap is non-negative for authors with any level of patience.



\subsection{Justifying Threshold Equilibria and Policies in Broader Settings}
\label{subsec:threshold_broader}

We now establish that de facto thresholds exist broadly across the models considered in this paper, and that any candidate threshold can be induced by a threshold acceptance policy as a de facto threshold. These results demonstrate the robustness of our central concepts --- such as the resubmission gap --- and thereby strengthen the practical insights derived from our model. 

\subsubsection{Characterizing De Facto Thresholds}
\label{sec:submission-threshold}

When the conference's acceptance policy $\ACCMAP$ is fixed, from the author's perspective, it induces an acceptance probability $\AccP{\ACCMAP}{q}$ for each paper quality $q$.
By Proposition~\ref{prop:monotone-prob}, this probability is weakly increasing in $q$ if the acceptance policy is monotone, and strictly increasing under a non-trivial threshold policy. 
We will show that as a result, if the acceptance policy is monotone, the author's equilibrium strategy can be characterized by a submission threshold $\theta$.
Note, however, that there may be other best responses that do not fit this pattern.
% the threshold best response clearly illustrates the impact of the conference's acceptance policy, aligning well with our goal.
% \dkcomment{I agree with the first half of the preceding sentence, but am unsure about what the second one is supposed to say. Are we justifying a focus only on author threshold policies? We didn't actually say that we focus on such policies.} \yichicomment{Commented out the second half.}

To state this result, we first define suitable nomenclature and notation.

\begin{definition}
In a \emph{$\theta$-threshold strategy} for the author, an author submits (and resubmits until they are accepted)  all papers of quality $\Qual = q > \theta$, and no paper of quality $q < \theta$. The author may handle papers of quality $q = \theta$ arbitrarily.  
An author strategy is called a \emph{threshold strategy} if it is a $\theta$-threshold strategy for some $\theta$.
\end{definition}

When $q = \theta$, the author is typically indifferent between submitting and not submitting. In this case, we assume that we can prescribe the author’s tie-breaking behavior for ease of analysis. 

% \yzdelete{
% We now formally define a central concept of our paper:\gscomment{this was already defined in 3.1.  I think we should remove it here. and the commentary below.}

% \begin{definition}[De Facto Threshold]
% Consider a conference with memoryless acceptance policy $\ACCMAP$ and noiseless authors.
% A value $\theta$ such that under every best response the author submits a paper of quality $\Qual = q$ if $q > \theta$ and does not submit any paper with $q < \theta$ is called a \emph{de facto threshold}.
% \end{definition}

% We call $\theta$ the de facto threshold because every paper whose quality is above $\theta$ will be submitted until accepted, so de facto, the conference will accept all papers above $\theta$.
% Based on the definition, the de facto threshold exists for an acceptance policy if and only if every best response of the author is a threshold strategy. In particular, if there are qualities $q \neq q'$ such that the author is indifferent between submitting or not submitting papers of qualities $q$ and $q'$, then the de facto threshold does not exist.
% }

% We begin our analysis by characterizing the best responses of an author to monotone acceptance policies.
% The following proposition characterizes the best responses of an author to monotone and threshold acceptance policies.
We capture the appeal of submitting to a conference by an \emph{attractiveness factor} $\rho$ (where larger values correspond to conferences more attractive to submit to), based on its value $\ConfValue$ and the discount factor $\TD$:
\begin{align}
\rho & := \frac{\ConfValue - \TD}{1-\TD}.
\label{eqn:rho-definition}
\end{align}

\begin{lemma}\label{lem:author_response}
    Consider a memoryless conference with a monotone acceptance policy $\ACCMAP$. 
    Suppose that the conference value $\ConfValue$ is fixed, and accordingly $\rho$ is fixed.
    Then, the author's best response is to submit the paper (in each round) if $\AccP{\ACCMAP}{\Qual} > 1/\rho$, take the side option when $\AccP{\ACCMAP}{\Qual} < 1/\rho$, and the author is indifferent between submitting or not submitting if $\AccP{\ACCMAP}{\Qual} = 1/\rho$.
\end{lemma}
We defer the proof to \cref{app:proof-author_resp}; it closely parallels the proof of \cref{prop:gap-invariant}. Note that we use the term ``best response'' rather than ``symmetric equilibrium'' in the statement of this lemma, since when $\ConfValue$ is fixed, each author’s utility is independent of the strategies of others. 
% \yzdelete{Intuitively, the acceptance probability of a paper, and thus the expected utility of the author, is increasing in its quality under monotone acceptance policies. This results in a threshold best response for the authors. Then, \cref{lem:author_response} follows from reasoning about the critical quality that makes the author indifferent between submitting and taking the side option in different settings.}

% \yzreplace{
% We further show that under a monotone acceptance policy, threshold strategies are always one of the author's best responses, and are the unique best response when the acceptance policy is a threshold one.
% To appreciate the result, first note that both the conference value $\ConfValue$ and the attractiveness factor $\rho$ depend on the author's response.}
We now examine how the dependence of the conference value~$\ConfValue$ and the attractiveness factor~$\rho$ on the author’s strategy influences the equilibrium.
Let $\ConfValue(\theta, r)$ denote the conference value when all authors with paper quality $Q > \theta$ choose to submit and resubmit until acceptance, those with $Q < \theta$ opt for the outside option, and authors with $Q = \theta$ submit with probability $r$. 
We then have the following formula for the conference value in the categorical model: 
% \gscomment{we just said above that we are ignoring r, but then it still apears below.}
% \dkreplace{\begin{equation*}
%     \ConfValue(q, r) = 1 + \frac{r\cdot q \cdot \QualProb{q} + \sum_{q' \in \QualSet, q' > q} q' \cdot \QualProb{q'}}{r\cdot \QualProb{q} + \sum_{q' \in \QualSet, q' > q} \QualProb{q'}}.
% \end{equation*}}{
\begin{equation*}
    \ConfValue(\theta, r) = 1 + \frac{r\cdot \theta \cdot \QualProb{\theta} + \sum_{q \in \QualSet, q > \theta} q \cdot \QualProb{q}}{r\cdot \QualProb{\theta} + \sum_{q \in \QualSet, q > q} \QualProb{q}}.
\end{equation*}
% \dkcomment{replaced all $q$ with $\theta$ and $q'$ with $q$.}
Because $\QualDist$ has full support on $\QualSet$, $\ConfValue(\theta,r)$ is strictly increasing in $\theta$ and strictly decreasing in $r$. This implies that the corresponding attractiveness factor $\rho(\theta,r)=\frac{\ConfValue(\theta,r) - \TD}{1-\TD}\in (0,1)$ is strictly increasing in $\theta$ and strictly decreasing in $r$.  

% \yzdelete{
% In the continuous setting, since papers with quality exactly $q$ have measure zero, the parameter $r$ can be omitted, and we write the conference value simply as $\ConfValue(q)$.
% The formula for the conference value is analogous to the categorical model, while we replace the summations with integrals.
% }

% Note that conference value has to be at least 1, implying that not all de facto thresholds are feasible. A de facto threshold $\theta\in \QualSet$ is feasible if there exists an $r\in [0,1]$ such that $\ConfValue(\theta, r)>1$, or if $\theta\notin \QualSet$, it is feasible if $\ConfValue(\bar{\theta}, 1)>1$ where $\bar{\theta} = \min\{q\in\QualSet\mid q>\theta\}$. In the continuous model, $\theta$ is feasible if $\ConfValue(\theta)>1$. 


For the existence of a meaningful conference value, the acceptance policy should at least encourage \emph{some} papers to submit.
The following definition captures this desideratum.
\begin{definition}
    We say that an acceptance policy $\ACCMAP$ is \emph{responsive} if it is non-trivial and there exists a quality $\bar{q}\in \QualSet$ such that $\AccP{\ACCMAP}{\bar{q}} > 1/\rho(\bar{q},1)$.
\end{definition}
\yichicomment{better name than responsive?}
\dkcomment{inviting? non-discouraging?}
\yzcomment{I don't feel strongly that these are better than the current one.}

\yzcomment{There is a suggestion of reversing the order of the following 3 points. I feel the wording and the proof are already well-adjusted for the ordering, and it doesn't seem like an essential change for this submission. I'll mark the idea and leave it for now.}
\begin{proposition} \label{prop:de_facto}
Assume that authors are noiseless, and the conference uses a monotone acceptance policy $\ACCMAP$ that is memoryless and responsive.
\begin{enumerate}
   \item There exists a $\theta$ such that it is a non-atomic symmetric equilibrium for every author to play a $\theta$-threshold strategy.%
   \footnote{There may also exist non-threshold equilibria, and thus $\theta$ may not be a de facto threshold.}
  \item If the conference policy $\ACCMAP$ is additionally a threshold one, then there exists a de facto threshold, implying that there exists only one non-atomic symmetric equilibrium; in this equilibrium, every author plays a $\theta$-threshold strategy.
  \item If the conference applies a threshold acceptance policy with threshold $\tau$ and the model is continuous, then the de facto threshold $\theta$ is unique: no best response is a $\theta'$-threshold strategy for $\theta' \neq \theta$.  
  Moreover, $\AccP{\ACCMAP[\tau]}{\theta} = 1/\rho(\theta)$. 
\end{enumerate}
\end{proposition}

% \begin{proposition} \label{prop:de_facto}
% Assume that authors are noiseless, and the conference uses a monotone acceptance policy $\ACCMAP$ that is memoryless and responsive.
% \begin{enumerate}
%   \item If the conference applies a threshold acceptance policy with threshold $\tau$ and the model is continuous, then there exists a unique de facto threshold $\theta$ such that no best response is a $\theta'$-threshold strategy for $\theta' \neq \theta$.  
%   Moreover, $\AccP{\ACCMAP[\tau]}{\theta} = 1/\rho(\theta)$. 
%   \item  If the conference applies a threshold acceptance policy with threshold $\tau$ and the model is categorical, then there exists a de facto threshold only one non-atomic symmetric equilibrium; in this equilibrium, every author plays a $\theta$-threshold strategy.
%    \item There exists a $\theta$ such that it is a non-atomic symmetric equilibrium for every author to play a $\theta$-threshold strategy.%
%    \footnote{There may also exist non-threshold equilibria, and thus $\theta$ may not be a de facto threshold}.
% \end{enumerate}
% \end{proposition}
% Note that our results (here and later) depend on $\TD$ and $\ConfValue$ only through $\rho$. Therefore, in a sense, they are ``interchangeable,'' albeit not linearly. That is, an increase in author patience ($\TD$) is tantamount to a (different) increase in conference prestige, as far as author behavior is concerned.

\Cref{prop:de_facto} follows from \cref{lem:author_response}, together with the monotonicity of $\AccP{\ACCMAP}{q}$ and $\rho(q)$.
This result implies that any monotone acceptance policy induces a form of self-selection among authors, whereby clearly substandard papers are not submitted in equilibrium.
In the following sections, we explore what factors affect the de facto threshold and how the conference can leverage this self-selection behavior to balance the utilities of different stakeholders in the system.


% \yichicomment{To do: what's below is unlikely to hold anymore.}
% \cref{prop:de_facto} assumes that the agents are allowed to resubmit their papers arbitrarily many times. This assumption is not essential: an essentially identical calculation (in \cref{sec:time_limited_policy}) shows that the de facto threshold does not change when the conference restricts the number of times a paper can be resubmitted.

\subsubsection{Sufficiency of Threshold Policies}
\label{sec:threshold-resubmission}  

% \dkreplace{
% \yzreplace{
% Threshold acceptance policies (see \cref{def:threshold-policy}) comprise a very natural class of policies for a conference to apply. 
% Recall that they accept all papers whose posterior (based on the reviews) expected quality strictly exceeds some threshold, and reject all papers whose posterior expected quality falls short of the threshold. }
% {Now, we justify the focus on threshold acceptance policies by showing} that every feasible de facto threshold can be induced by a threshold acceptance policy.}
Having shown that threshold best responses are ``typically'' best for \emph{authors}, we next show that threshold acceptance policies are sufficient for the \emph{conference}, in that they let it induce every candidate threshold as a de facto threshold.

\begin{definition}\label{def:candidate_threshold}
    We say that a value $\theta \in \R$ is a \emph{candidate threshold} if the resulting conference value is larger than the utility of the outside option. Formally, $\theta$ is \a candidate threshold if:
% \gsreplace{\[
% \begin{cases}
% \ConfValue(\dkreplace{\bar{\theta}}{\min\{q \in \QualSet \mid q > \theta\}}, 1) > 1 & \text{if } \theta \notin \QualSet%, \text{ where } \bar{\theta} = \min\{q \in \QualSet \mid q > \theta\}
% , \\
% \dkreplace{\exists\,}{\text{there exists an }} r \in [0,1] \text{ such that } \ConfValue(\theta, r) > 1 & \text{if } \theta \in \QualSet.
% \end{cases}
% \]}
\[
\begin{cases}
\ConfValue(\theta, r) > 1 \text{ for some } r \in [0,1]   & \text{if } \theta \in \QualSet
, \\
\ConfValue(\inf \Set{q \in \QualSet}{q > \theta}, 1) > 1 & \text{if } \theta \notin \QualSet%, \text{ where } \bar{\theta} = \min\{q \in \QualSet \mid q > \theta\}
.
\end{cases}
\]
Here, we use the standard convention that $\inf \emptyset = \infty$, so $\theta$ can only be a candidate threshold if there exists at least one $q \in \QualSet$ with $q \geq \theta$.
% \dkcomment{Replaced min by inf, as for continuous domains, the minimum of that set doesn't exist, and for discrete, there's no difference. Explicitly addressed the empty set.}

Let $\mathcal{C}$ denote the set of all candidate thresholds.
\end{definition}
% Intuitively, a feasible de facto threshold cannot be too low, as this would make the highest achievable conference value worse than the outside option. In such cases, even an acceptance probability of 1 would not make authors with borderline papers ($Q = \bar{\theta}$) indifferent between submitting and opting out.
Because $\ConfValue(q,r)$ is increasing in $q$, if $\theta$ is a candidate threshold and $\theta' > \theta$, then $\theta'$ is also a candidate threshold, so long as at least one $q \in \QualSet$ weakly exceeds $\theta'$. 

% \dkreplace{Therefore, there exists an infimum feasible de facto threshold $\theta_{\inf}$ such that any $\theta > \theta_{\inf}$ is feasible and any $\theta < \theta_{\inf}$ is infeasible. 
% Whether $\theta = \theta_{\inf}$ is feasible depends on whether the model is continuous or categorical.
% In the continuous model, there exists a $\theta_{\inf}$ such that $\ConfValue(\theta_{\inf}) = 1$, implying that $\theta = \theta_{\inf}$ is not feasible. In the categorical model, there exists $(\theta_{\inf}, r_{\inf})$ such that $\ConfValue(\theta_{\inf}, r_{\inf}) = 1$ for some probability $r_{\inf}\in (0,1]$. Therefore, $\ConfValue(\theta, r) > 1$ for any $\theta > \theta_{\inf}$ and $r\in (0,1]$ while $\ConfValue(\theta, r) \le 1$ for any $\theta < \theta_{\inf}$ and $r\in (0,1]$, meaning that $\theta = \theta_{\inf}$ is feasible but any $\theta < \theta_{\inf}$ is infeasible.}

Let $\theta_{\inf} = \inf \mathcal{C}$ be the infimum of all candidate thresholds.
By the preceding paragraph, we obtain that either $\mathcal{C} = [\theta_{\inf}, \max \QualSet]$ or $\mathcal{C} = (\theta_{\inf}, \max \QualSet]$.
Both cases can occur, depending on whether the model is continuous or categorical.
In the continuous model, $\ConfValue(\theta_{\inf}) = 1$, implying that $\theta_{\inf}$ is not a candidate threshold. 
In the categorical model, there exists an $r_{\inf} \in (0,1]$ such that $\ConfValue(\theta_{\inf}, r_{\inf}) = 1$. 
Therefore, $\ConfValue(\theta_{\inf}, r_{\inf}/2) > 1$, and $\theta_{\inf}$ is indeed a candidate threshold.

% \yichicomment{ the current statement sounds circular. Consider replacing “Let $\theta\in R$ be any de facto threshold”
% with “For every $\theta\in R$” (since we don’t know whether $\theta$ is a de facto threshold
% until a conference acceptance rule has been specified).}
% \dkcomment{I think I resolved this by calling it a candidate threshold, and removing the ``de facto threshold'' part of the definition, which I think was not needed --- this was only a property of conference values conditioned on included papers. I hope that I did not terribly mess this up.}

\begin{proposition} 
\label{prop:threshold-policy}
Let $\theta \in \mathcal{C}$ be a candidate threshold. 
Then, there exists a threshold acceptance policy with threshold $\hat{\tau}$ and probability $\hat{r}$ under which the following is a symmetric equilibrium for noiseless authors: submit to the prestigious conference which uses $\ACCMAP[\hat{\tau},\hat{r}]$ if $\Qual > \theta$ and take the side option if $\Qual < \theta$ (and the author is indifferent between submitting or not submitting if $\Qual=\theta\in \QualSet$). \dkcomment{addressed?}
Moreover, if the model is continuous and the authors are neither submitting all papers nor submitting no papers, the threshold $\hat{\tau}$ is unique.
\end{proposition}

We defer the proof to \cref{app:proof-threshold-policy}, while providing a high-level sketch here. 
% We say an acceptance policy is stricter than another if it accepts every paper with a (weakly) smaller probability. While restricting to threshold acceptance policies, we show that a threshold policy is stricter than another if and only if its threshold is higher. 
% First, note that a paper of quality $\theta$ has an acceptance probability of $1/\rho$ (\cref{prop:de_facto}). Then, we show that for threshold acceptance policies, the acceptance probability of any paper is monotone decreasing in the acceptance threshold. Thus, the proposition follows because the acceptance probability of a threshold policy with $\tau = -\infty$ is $1$, and with $\tau = \infty$, it is $0$. Therefore, because $1/\rho \in (0,1)$, there must exist (at least) one acceptance threshold such that the acceptance probability of a paper of quality $\theta$ equals $1/\rho$.
For any $\theta \in \mathcal{C}$, there exists a borderline quality $\hat{q} = \inf\Set{q \in \QualSet}{q \ge \theta}$ (which may differ from $\theta$ in the categorical model), such that if $\AccP{\ACCMAP}{\hat{q}} = 1/\rho(\hat{q}, r)$ for some $r$, then $\theta$ is a de facto threshold.
The key insight is that, under threshold acceptance policies, the acceptance probability decreases monotonically with the threshold $\tau$, ranging from 1 as $\tau = -\infty$ to 0 as $\tau = \infty$.
This implies that for any candidate threshold $\theta$ where the corresponding $\hat{q}$ satisfies that $1/\rho(\hat{q}, r)\in (0,1)$, there must exist an acceptance threshold such that the acceptance probability of a paper of quality $\hat{q}$ equals $1/\rho(\hat{q}, r)$. 

Proposition~\ref{prop:threshold-policy} in part justifies our focus on threshold policies.  For any monotone policy, Proposition~\ref{prop:de_facto} implies the existence of a symmetric equilibrium where authors play a threshold strategy, and thus, Proposition~\ref{prop:threshold-policy} implies the existence of a threshold policy for which the author best-responds in the same way.
\fangcomment{We need to say the threshold in \cref{prop:de_facto} are in $\mathcal{C}$ \cref{def:candidate_threshold}.} \yzcomment{Do we have a proof relying on this?}\fangcomment{This paragraph suggests we may use threshold acceptance policy to replace any monotone policy.  However, under \cref{prop:de_facto} there may be some monotone policy whose threshold equilibrium is not in $\mathcal{C}$.}
Thus, assuming that authors break ties in favor of using threshold strategies, any conference quality that can be achieved with a monotone policy can be achieved with a threshold acceptance policy.  
Interestingly, it does not follow that the QB-tradeoff for threshold policies weakly dominates that of all monotone acceptance policies, as we will show in \cref{sec:testing_tradeoff}.

While Proposition~\ref{prop:threshold-policy} implies the existence of an acceptance threshold $\tau$ inducing the desired submission threshold $\theta$, these two thresholds will typically be different. 
% \dkreplace{The threshold $\tau$ can be interpreted as the ``declared'' quality goal of the conference, attained in isolation. $\theta$ is the actual quality of the conference, taking into consideration resubmissions and reviewing noise. The difference is a key concept we study, and term the ``resubmission gap''.

% \begin{definition}[Resubmission Gap] 
% Consider a memoryless conference with a non-trivial threshold acceptance policy $\ACCMAP[\tau,r]$, and a noiseless author.
% Let $\theta$ be the smallest de facto threshold in response to $\ACCMAP[\tau,r]$. 
% The difference between the acceptance threshold $\tau$ and $\theta$ is called the \emph{resubmission gap}.
% \end{definition}
% % \fangcomment{Is it possible to have this at the beginning of section 3 or section 3.2}
% In general, the de facto threshold is not unique in the categorical model; we define the resubmission gap based on the \emph{smallest} de facto threshold. This issue disappears in the continuous model, where the resubmission gap is unique for every non-trivial threshold acceptance policy. 

% The name resubmission gap reflects the discrepancy between the ``stated'' and ``de facto'' quality of accepted papers, caused by the fact that authors are free to resubmit papers.
% The fact that the actual threshold of papers at the conference differs from the declared acceptance threshold for each year in isolation (namely, by the resubmission gap) may appear somewhat unexpected at first, and is frequently missing from conversations about policy changes. 

% The resubmission gap leads to the following oddity, which we call the \emph{resubmission paradox}:  all submitted papers will eventually be accepted, but in any one round, many (and often most) are rejected. This creates a substantial and seemingly unnecessary burden on the review process, as papers are reviewed multiple times before final acceptance. While accepting all papers initially seems like a solution, in our model, it is not.  The only way to initially accept all papers is to lower the acceptance threshold, which in turn lowers the de facto threshold and incentivizes additional submissions.  Moreover, these additional papers will often be rejected multiple times, but will be resubmitted until they are accepted.  Thus, this apparent solution actually reproduces the problem, just with a lower de facto threshold.
% }
{Indeed, the difference is exactly the resubmission gap we defined in \cref{def:resubmission-gap}, and which is one of our central quantities of interest.}

% \dkcomment{There was a lot of duplication of actual text, and definitely content, defining resubmission gap and resubmission paradox again. I assume it was just a remnant of editing, not intentional.}

% \gscomment{I feel like the below could actually be moved into section 3.1.  3.3 seems to be for more technical analysis justifying particular assumptions and how general we can make things.  This is notes and bolts kinda stuff that we will use repeatedly, and it only require de facto threshold.}
% \dkcomment{Implemented this suggestion.}
