Our analysis of QB-tradeoffs in \cref{subsec:gap-tradeoffs} is largely based on threshold acceptance policies. To contextualize this focus, we first investigate when using such policies is optimal in terms of achieving Pareto frontiers of the QB-tradeoff.
%In \cref{prop:de_facto}, we showed that every monotone acceptance policy induces a de facto threshold above which all papers are submitted (and eventually accepted), and below which no papers are submitted. Then, in \cref{prop:threshold-policy}, we showed that every de facto threshold can be implemented with a threshold acceptance policy. In particular, this applies to the optimal de facto threshold of $\theta=0$\textemdash see \cref{prop:max-general}.
%While this implies that every desired conference quality \yichidelete{(including the optimal conference quality)} that is achievable by a monotone policy can be achieved by a threshold policy, it raises the question whether insisting on a threshold acceptance policy might come at a cost in terms of the \emph{review burden}. 
%Here, we answer this question. We show that when only $\NumReviews = 1$ review is obtained, there always is an optimal threshold acceptance policy. 
%However, even for $\NumReviews = 2$, there are in general instances for which monotone non-threshold policies strictly outperform threshold policies in terms of the review burden, while achieving the same (optimal) conference quality.
While analyzing monotone acceptance policies, we showed that they always induce a threshold strategy for authors as one of the best responses (see \cref{prop:threshold-policy}). Furthermore, under threshold acceptance policies, non-threshold best responses do not exist; and even for monotone policies, they exist only in knife-edge cases, where authors are indifferent between submitting or not submitting for multiple paper qualities. 
We therefore restrict our attention to threshold best responses for the authors in the following discussions. 

In \cref{prop:threshold-policy}, we showed that every candidate threshold can be implemented with a threshold acceptance policy. In particular, this applies to the optimal de facto threshold of $\theta=0$; see \cref{prop:max-general}. Therefore, every conference quality that is achievable by a monotone policy can be achieved by a threshold policy. 
It remains to understand when a threshold acceptance policy minimizes the corresponding review burden (among all monotone policies) for every conference quality. Our results suggest that when only $\NumReviews = 1$ review is obtained, there always is an optimal threshold acceptance policy. 
However, even for $\NumReviews = 2$, there are in general instances for which monotone non-threshold policies strictly outperform threshold policies in terms of the review burden, while achieving the same (optimal) conference quality.


\begin{proposition}\label{prop.threshold_opt}
When $\NumReviews = 1$ and authors always respond with threshold strategies if possible, the QB-tradeoff for threshold policies weakly dominates that of all monotone acceptance policies.
\end{proposition}
Our proof will use the following theorem about uniformly most powerful tests in statistical testing.

\begin{theorem}[Karlin–Rubin Theorem (Theorem~12.9 of \cite{keener2010theoretical})] \label{thm.ump}  Consider two sets $\QualSet$ and $\SigSet\subseteq \R$ and a family of pdfs or pmfs $\{g(\REVSIG |\qual):\qual\in \QualSet\}$ on $\SigSet$ possessing the monotone likelihood property (\cref{def:informative}).  
For any $\theta\in \QualSet$, $\tau\in \R$, $r\in [0,1]$, and two (randomized) tests $h, h^*: \SigSet\to [0,1]$, if $h^*$ is a threshold function, i.e., of the form
$$h^*(\REVSIG) = \begin{cases} 1\text{ if }\REVSIG>\tau\\
r \text{ if }\REVSIG = \tau\\
0 \text{ if }\REVSIG<\tau\end{cases},$$ and 
$\sup_{\qual\in \Theta_0}\E_{\REVSIG\sim g(\cdot|\qual)}[h^*(\REVSIG)]\ge \sup_{\qual\in \Theta_0}\E_{\REVSIG\sim g(\cdot|\qual)}[h(\REVSIG)]$ with $\Theta_0:=\{\qual\in \QualSet: \qual\le \theta\}$, then 
$$\E_{\REVSIG\sim g(\cdot|\qual)}[h^*(\REVSIG)]\ge \E_{\REVSIG\sim g(\cdot|\qual)}[h(\REVSIG)]\text{, for all } \qual\in \QualSet\setminus \Theta_0.$$

We call $h^*$ \emph{uniformly most powerful} among all tests $h$ with 
$$\sup_{\qual\in \Theta_0}\E_{\REVSIG\sim g(\cdot|\qual)}[h^*(\REVSIG)]\ge \sup_{\qual\in \Theta_0}\E_{\REVSIG\sim g(\cdot|\qual)}[h(\REVSIG)].$$
\end{theorem}

\proof{Proof of Proposition~\ref{prop.threshold_opt}.}

As discussed after \cref{prop:threshold-policy}, by assuming that authors break ties in favor of using threshold strategies, any conference quality that can be achieved with a monotone policy is 
% $\int^\infty_\theta q\,dp(q)$ for some de facto threshold $\theta\in \R$, 
determined by some de facto threshold $\theta$ (or a threshold-probability pair $(\theta, r)$ in the categorical model), and the quality can be achieved with a threshold acceptance policy. 
Thus, it is sufficient to show that for any de facto threshold $\theta$ (or $(\theta, r)$ in the categorical model) there exists a threshold policy that minimizes the review burden over all monotone policies.  

Given an attractiveness $\rho$ induced by the threshold best response, by \cref{prop:de_facto}, a monotone policy with de facto threshold $\theta$ can be considered as
a statistic test on $\{q \in \QualSet: q \le \theta\}$ vs.~$\{q \in \QualSet: q > \theta\}$. 
% \dkcomment{I think we're interested in the actual values here, not the random variable.}
% \dkreplace{so that 1) the size is less than $1/\rho$ and 2) the power is greater than $1/\rho$ for all papers with quality greater than $\theta$.  When the review signal is informative (monotone likelihood ratio property), by Karlin–Rubin theorem~(e.g., Theorem 8.3.17 of \cite{keener2010theoretical}),
% there exists a uniformly most powerful \fangreplace{test $\ACCMAP^*$ at size $1/\rho$}{ size $1/\rho$ test $\ACCMAP^*$} so that for any acceptance policy $\ACCMAP$ with size less than or equal to $1/\rho$ and any paper quality $q>\theta$, $\AccP{\ACCMAP^*}{q}\ge \AccP{\ACCMAP}{q}$.  Moreover, $\ACCMAP^*$ has a critical value $c$ and accepts a paper when the signal is greater than $c$.  Therefore, $\ACCMAP^*$ has the same de facto threshold $\theta$ and minimizes review burden.  Finally, by the definition of monotone likelihood ratio property with $m = 1$, the $\ACCMAP^*$ is a threshold policy.}
To ensure that authors do not submit papers of quality less than $\theta$, the false positive rate (accepting any paper of quality $q < \theta$) must be at most $1/\rho$. Subject to those constraints, the false negative rate (rejecting any paper of quality $q > \theta$) should be minimized.  By \cref{prop:threshold-policy}, there exists a threshold policy $\ACCMAP^*$ with de facto threshold $\theta$.
The Karlin–Rubin Theorem (as given in \cref{thm.ump}) states that when the observed signal (in our case: the review) has the monotone likelihood ratio property, 
% \fangreplace{there exists a uniformly most powerful test $\ACCMAP^*$ which is a threshold test subject to the constraint on false positive, and it satisfies}
the above threshold test (policy) $\ACCMAP^*$ is uniformly most powerful so that $\AccP{\ACCMAP^*}{q}\ge \AccP{\ACCMAP}{q}$ for all $q > \theta$.  
Thus, $\ACCMAP^*$ has the desired de facto threshold $\theta$ and minimizes the probability of false negatives (rejections) for all paper qualities that should be eventually accepted.
Thus, it minimizes the review burden. \Halmos
\endproof


The optimality of threshold policies ceases to hold when $\NumReviews > 1$, as we illustrate next with a counter-example.

\begin{table}[ht]
\caption{Likelihood function in \cref{ex:threshold-counterexample}. \label{tab:threshold-counterexample}}
    \centering
     \renewcommand{\arraystretch}{1.2}
    \begin{tabular}{|l|c|c|c|c|c|c|}
    \hline
 $\RevSigV$ & $(L,L)$ &     
 \begin{tabular}{@{}c@{}}$(L,M)$ or\\ $(M,L)$ \end{tabular}
 & \begin{tabular}{@{}c@{}}$(L,H)$ or\\ $(H,L)$ \end{tabular} 
 & $(M,M)$ 
 &  \begin{tabular}{@{}c@{}}$(M,H)$ or\\ $(H,M)$ \end{tabular} 
 & $(H,H)$ \\
         \hline
        $\ProbC{\RevSigV}{q = -2}$ & $\frac{4}{9}$ & $\frac{2}{9}$ & $\frac{2}{9}$ & $\frac{1}{36}$ & $\frac{1}{18}$ & $\frac{1}{36}$\\
        $\ProbC{\RevSigV}{q = 1}$ & $\frac{1}{9}$ & $\frac{1}{9}$ & $\frac{1}{3}$ & $\frac{1}{36}$ & $\frac{1}{6}$ & $\frac{1}{4}$\\
        $\ProbC{\RevSigV}{q = 5}$ & $\frac{1}{36}$ & $\frac{1}{18}$ & $\frac{2}{9}$ & $\frac{1}{36}$ & $\frac{2}{9}$ & $\frac{4}{9}$\\
        \hline
    \end{tabular}
\end{table}

% \begin{table}
%      \TABLE
%      {Likelihood function in \cref{ex:threshold-counterexample}.\label{tab:threshold-counterexample}}
%      \begin{tabular}{|c|c|c|c|c|c|c|}
%           \hline
%          $\RevSigV$ & $(L,L)$ &     
%          \begin{tabular}{@{}c@{}}$(L,M)$ or\\ $(M,L)$ \end{tabular}
%          & \begin{tabular}{@{}c@{}}$(L,H)$ or\\ $(H,L)$ \end{tabular} 
%          & $(M,M)$ 
%          &  \begin{tabular}{@{}c@{}}$(M,H)$ or\\ $(H,M)$ \end{tabular} 
%          & $(H,H)$ \\
%          \hline
%         $\ProbC{\RevSigV}{q = -2}$ & $\frac{4}{9}$ & $\frac{2}{9}$ & $\frac{2}{9}$ & $\frac{1}{36}$ & $\frac{1}{18}$ & $\frac{1}{36}$\\
%         $\ProbC{\RevSigV}{q = 1}$ & $\frac{1}{9}$ & $\frac{1}{9}$ & $\frac{1}{3}$ & $\frac{1}{36}$ & $\frac{1}{6}$ & $\frac{1}{4}$\\
%         $\ProbC{\RevSigV}{q = 5}$ & $\frac{1}{36}$ & $\frac{1}{18}$ & $\frac{2}{9}$ & $\frac{1}{36}$ & $\frac{2}{9}$ & $\frac{4}{9}$\\
%         \hline
%     \end{tabular}
%     \vspace{2mm}
%      {Columns are ordered based on expected conference quality (from low to high) conditioned on the review vector.}
% \end{table}

\begin{example} \label{ex:threshold-counterexample}
The set of paper qualities is $\{-2,1,5\}$, with uniform prior $\QualDist = (1/3,1/3,1/3)$.  The review signal set is $\SigSet = \{L,M,H\}$, and the number of reviews is $\NumReviews = 2$.  The conditional distributions of review signals are $\RevSigDist[-2] = (2/3,1/6,1/6), \RevSigDist[1] = (1/3,1/6,1/2)$, and $\RevSigDist[5] = (1/6,1/6,2/3)$; it can be verified that this information structure satisfies the monotone likelihood ratio (MLR) property. 
\Cref{tab:threshold-counterexample} shows the resulting likelihood of each vector of review signals, with $\NumReviews = 2$. The columns of the table are ordered based on expected conference quality (from low to high) conditioned on the review vector.

When only positive-quality papers ($q = 1, 5$) are submitted, the conference value is $V=3$. Under an author discount factor of $\eta = 9/19$, the conference's attractiveness is then $\rho = 24/5$.


Consider the following policy $\ACCMAP[']$. 
(1) When the review vectors are $(H,H), (M,H)$ or $(H,M)$,
accept the paper with probability 1;
(2) when the review vectors are $(M,M), (L,H)$ or $(H,L)$,
accept the paper with probability $1/2$; and
(3) otherwise, reject the paper with probability 1.
Notice that this policy is monotone (because the acceptance probability is non-decreasing in the conditional expected quality), but it is not a threshold policy. 
This is because the conditional expected paper qualities of the signal vectors $(M,M)$ and $(L,H)$ (or $(H,L)$) are
$U(M,M) = \frac{4}{3} > \frac{9}{7} = U(H,L) = U(L,H)$,
yet $(H,L)$ and $(L,H)$ lead to acceptance with positive probability, while $(M,M)$ does not lead to acceptance with probability 1.
Next, we compute the acceptance probabilities of papers with different qualities:
\begin{align*}
\AccP{\ACCMAP[']}{5} & = 19/24 > 1/\rho
& \AccP{\ACCMAP[']}{1} & = 43/72 > 1/\rho
& \AccP{\ACCMAP[']}{-2} & = 5/24 = 1/\rho.
\end{align*}

Thus, all papers of positive quality are submitted, while the papers of negative quality are not submitted. As a result, $\ACCMAP[']$ is a monotone policy with de facto threshold 0, maximizing the conference quality.  

Next, consider any threshold policy $\ACCMAP$ implementing the de facto threshold of $\theta=0$. It cannot accept papers with review vectors $(L,H), (H,L)$ with probability exceeding $7/16$. 
Otherwise, by virtue of being a threshold policy, $\ACCMAP$ would have to accept all papers with higher review vectors with probability 1; as a result, the acceptance probability of a paper with quality $-2$ would exceed $7/16 \cdot 2/9 + 1 \cdot (1/36 + 1/18 + 1/36) = 5/24$.

Again by virtue of being a threshold policy, $\ACCMAP$ must reject all papers with review vectors $(L,L), (L,M), (M,L)$. 
Thus, a paper of quality $5$ is accepted with probability at most $1 \cdot (4/9 + 2/9 + 1/36) + 7/16 \cdot 2/9 = 19/24$, while a paper of quality $1$ is accepted with probability at most $1 \cdot (1/4+1/6+1/36) + 7/16 \cdot 1/3 = 85/144$.
Thus, papers of quality $5$ are accepted with the same probability as under the policy $\ACCMAP[']$, while papers of quality $1$ are accepted with strictly smaller probability. As a result, a strictly higher review load is required.
\end{example}

We further note that \cref{ex:threshold-counterexample} also shows that the combination of independent review signals with MLR do not necessarily have MLR. To see this, in \cref{tab:threshold-counterexample}, $(M, M)$ is a better review signal than $(L, H)$ or $(H, L)$ in the sense that the expected paper quality is higher conditioned on the former. However, it is not hard to observe that these two signals violate the definition of MLR. In particular, $\frac{\RevSigDist[1]((M,M))}{\RevSigDist[-2]((M,M))}= 1 < 1.5 = \frac{\RevSigDist[1]((L,H)\text{ or }(H,L))}{\RevSigDist[-2]((L,H)\text{ or }(H,L))}$.