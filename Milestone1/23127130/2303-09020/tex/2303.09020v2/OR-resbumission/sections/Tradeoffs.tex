

% \gscomment{ Define QB trade off and QA tradeoff.  some pictures. 4.0.1 summary of results (little chart about noise and discount). 4.2 Dominanting Threshold.  4.3 Point-wise worse resubmission gap implies weakly dominant QB-tradeoff curve.  4.4 reviewer noise (general)  4.5 Author discount (continuous model only for QA).
% 4.6 Acceptance Rate.

% Look at two cases:  

% Current: 4.2 what thresholds work best (high versus low) 4.3 QB (insert how this is related to resubmission gap)  4.4 QA. 4.4 Review noise (more noise worse for  QB and QA), 4.5 discount good for QB mixed for QA.  Acceptance Rate (add comment about how acceptance rate depends on the distribution as well as the thresold change mu and see what happens.).     }


In this section, we build on the fundamental concepts of threshold policies, de facto thresholds, and resubmission gap, to undertake a more in-depth investigation of the tradeoffs a conference may face. 
% In particular, we consider the tradeoff between conference quality and review burden on the community, and between acceptance thresholds and acceptance rate. 
In particular, we focus on the utility of the three stakeholders in our model, as reflected by conference quality, review burden, and authors' welfare.
We also examine another central feature of a conference --- its acceptance rate --- and analyze how it depends on the acceptance threshold.
% As we will show, review burden and author utility are often aligned\textemdash fixing the number of reviews per paper, a lower review burden typically implies fewer rounds of resubmission, which in turn leads to a higher author welfare.
% Therefore, we first focus on the tradeoff between conference quality and review burden, and examine when review burden aligns with author welfare as we change the review noise and the author's discount factor.
% Finally, we analyze how acceptance thresholds relate to the overall acceptance rate.

% \yzdeletecomment{Old starting paragraphs}{
% We \yichireplace{primarily}{first} study the tradeoff between the conference's quality and the review burden \fangedit{, formalized as QB-tradeoffs in~\cref{subsec:QB-tradeoff}, }\dkcomment{As in Section 2, we are making a very quick pivot without explanation away from the authors. It might be enough to replace ``primarily'' by ``first''.} based on threshold acceptance policies. In \cref{subsec:QB-tradeoff}, we use numerical examples to visualize the QB-tradeoffs and the Pareto frontiers in the continuous model. 
% Then, the rest of this section investigates factors that affect QB-tradeoffs. Finally, we examine the relationship between the acceptance threshold and the acceptance rate.
% In \cref{sec:testing_tradeoff}, we further provide a discussion of when using \dkreplace{the}{a} threshold acceptance policy is optimal in terms of achieving Pareto frontiers of the QB-tradeoff. This result partially justifies why we primarily focus on studying threshold acceptance policies in this paper.

% \yichiedit{We also analyze the relationship between review burden and author welfare \fangedit{ in \cref{subsec:QB-tradeoff}}. Our results show that these two objectives are perfectly aligned when varying the level of review noise; however, this alignment does not always hold when the author discount factor changes. 
% Finally, we examine how the connection between a conference’s acceptance rate and its acceptance threshold is influenced by the distribution of paper quality\fangedit{ in \cref{sec:acc_rate}}.}\fangcomment{this seems to be an incomplete outline.  I hope to add pointers for each subsections}
% }

\subsection{Tradeoffs in the Continuous Model}
\label{subsec:QB-tradeoff}
We begin by characterizing the utilities of the three stakeholders in the continuous model, where there is a clean one-to-one mapping between the acceptance threshold and the de facto threshold, and visualize the resulting tradeoff curves using numerical examples. As noted in \cref{sec:model}, we are primarily interested in two tradeoffs: conference quality versus review burden (the QB-tradeoff) and conference quality versus author welfare (the QA-tradeoff). 

We have discussed the conference quality in the previous section, and observed that the conference will eventually accept every paper whose quality is above the de facto threshold $\theta$. Thus, the conference quality is $\CONFUTIL = \int_\theta^\infty q \QualDens{q}dq$. This expression shows that, once the quality prior is fixed, the de facto threshold~$\theta$ alone determines the conference quality.

As for the review burden, we consider the average number of reviews per \emph{paper} (including papers that were never submitted, and thus incurred $0$ reviews) as the relevant measure; note that because the total number of papers is constant, this is equivalent to considering the total number of reviews provided by the community.
This accounts for all reviews a paper receives over multiple rounds of submission --- first submission, second submission, and so on --- until it is eventually accepted.
Let $\theta$ be a candidate threshold, and let the corresponding acceptance policy be $\ACCMAP$. The total review burden under the continuous model is given by:
\begin{align}\label{eq:review_burden}
    R(\theta) &= m \int_{\theta}^\infty \QualDens{q} \cdot \left(\sum_{t=0}^{\infty} (1-\AccP{\ACCMAP}{q})^t \right) \; dq =  m\int_{\theta}^\infty \QualDens{q}/\AccP{\ACCMAP}{q} \; dq.
\end{align}

We next consider author welfare. In the continuous model, a de facto threshold $\theta$ determines a conference value $\ConfValue$ and induces an acceptance policy $\ACCMAP$, as shown in \cref{prop:de_facto}.
The expected utility of an author with a paper of quality $q\ge \theta$ who decides to submit and keep resubmitting until acceptance is
\begin{equation*}
    u^{(a)}(q, \ACCMAP, \ConfValue) 
    = \sum_{t=0}^{\infty} \ConfValue \cdot \AccP{\ACCMAP}{q} \cdot (\TD \cdot (1-\AccP{\ACCMAP}{q}))^t
    = \frac{\AccP{\ACCMAP}{q} \cdot V}{1-\eta \cdot (1-\AccP{\ACCMAP}{q})}.
\end{equation*}
The author welfare of all submitted papers is then $U^{(a)}(\theta) = \int_{\theta}^\infty \QualDens{q} \cdot u^{(a)}(q, \ACCMAP, \ConfValue) \, dq$.

Given a quality prior, varying the acceptance threshold (and thus the de facto threshold) varies the utilities of the three stakeholders in the above discussed ways. This allows us to visualize the QB tradeoffs and the QA tradeoffs.
In this section (and the following sections that contain discussions on the continuous model), we frequently use the following special cases of our general continuous model as examples for our analysis and plots.

The \emph{$(\sigma, \QualDist, \NumReviews, \TD)$-Gaussian model} is a continuous model with noiseless authors; the noise for each review is drawn from a Gaussian distribution $\REVNOISEDIST = \Gaussian{0}{\sigma}$. The parameters $\QualDist, \NumReviews$, and $\TD$ are, as before, the prior, the number of solicited reviews, and the discount factor, respectively. 
The \emph{$(\sigma, \mu_{\QualDist}, \sigma_{\QualDist}, \NumReviews, \TD)$-Double Gaussian model} is a Gaussian model with the prior $\QualDist = \Gaussian{\mu_{\QualDist}}{\sigma_{\QualDist}}$.


\begin{figure}[htb]
     \FIGURE
     {\begin{subfigure}[b]{0.48\textwidth}
         \centering
         \includegraphics[width=\textwidth]{Plots/Pareto_optimal_vs_noise_continuous_example_1.pdf}
         \captionsetup{size=}
         \caption{}
     \end{subfigure}
     \hfill
     \begin{subfigure}[b]{0.47\textwidth}
         \centering
         \includegraphics[width=\textwidth]{Plots/Pareto_optimal_vs_noise_continuous_example_2.pdf}
         \captionsetup{size=}
         \caption{}
     \end{subfigure}
     \hfill
     }
     {QB-tradeoff Curves in the Continuous Model. \label{fig:Pareto_frontier_continuous}}
     {(a) shows QB-tradeoff curves of a \emph{$(\sigma, \mu_{\QualDist} = -1, \sigma_{\QualDist} = 1, \NumReviews = 1, \TD = .7)$-Double Gaussian model}.  (b) shows QB-tradeoff curves of a \emph{$(\sigma, \mu_{\QualDist} = .5, \sigma_{\QualDist} = 2, \NumReviews = 1, \TD = .7)$-Double Gaussian model}.  In each case, the review quality $\sigma$ is varied over five discrete options.  For each $\sigma$, the curve shows the possible QB-tradeoffs as the acceptance threshold is varied continuously. The Pareto frontier is shown with solid lines, while dominated points are shown with dashed lines. 
     % \yzedit{Note that the \fangedit{conference }quality-maximizing point on each curve corresponds to the case $\theta = 0$\fangedit{; these points are indicated by dots}.}\fangcomment{I guess the text: larger $\tau/\theta$ is not ratio.  may be we use $\tau$ or $\theta$.  Is it possible to present some $\tau$ or $\theta$ values in the figures?}
     }
     \end{figure}
     

\Cref{fig:Pareto_frontier_continuous} maps the QB-tradeoff for various settings. 
In each setting, there is a point at (0, 0) that corresponds to rejecting all submissions. As the threshold is decreased, high-quality papers start being submitted, increasing both the conference quality and review burden.  When the de facto threshold is 0, conference quality is maximized. Subsequently, a further decrease in the threshold leads to more low-quality papers being accepted, lowering the conference quality.  We see here that the effect on the review burden is mixed: sometimes it increases (e.g., the curves in \cref{fig:Pareto_frontier_continuous} (b) with $\sigma < 1$) while other times it decreases (e.g., the curves in \cref{fig:Pareto_frontier_continuous} (b) with $\sigma \ge 1$).  
As the threshold decreases further, 
% \dkreplace{we see a point where all submitted papers are accepted with one round of review.
% In this case, either all papers are submitted and accepted immediately, resulting in a review burden of 1 (e.g., panel (b)), or papers with quality above $\theta_{\inf}$ are submitted, resulting in a conference value $\ConfValue = 1$ and a review burden smaller than 1 (e.g., panel (a)).}{
two types of behaviors may emerge. If the expected paper quality under \QualDist is non-negative, it is strictly positive conditioned on being above any de facto threshold $\theta > -\infty$, so as the de facto threshold goes to $-\infty$, all papers are submitted and accepted in a single round of review. This behavior can be observed in \cref{fig:Pareto_frontier_continuous} (b), where the curves converge to a point with $R=1$. On the other hand, if the expected paper quality is negative, a sufficiently low de facto threshold $\theta$ leads to a conference value $\ConfValue < 1$, so no authors submit any more, and the QB-tradeoff becomes $(0,0)$. Thus, the lowest meaningful de facto threshold is $\theta = \theta_{\inf}$; this can be observed in \cref{fig:Pareto_frontier_continuous} (a).

% \dkcomment{Rewrote the preceding on 08/17.}

\begin{figure}[htb]
     \FIGURE
     {\includegraphics[width=0.6\textwidth]{Plots/QA_tradeoff_continuous_example_1.pdf}}
     {QA-tradeoff Curves in the Continuous Model. \label{fig:QA_Pareto_frontier_continuous}}
     {The figure shows QA-tradeoff curves of a \emph{$(\sigma, \mu_{\QualDist} = 1, \sigma_{\QualDist} = 3, \NumReviews = 1, \TD = .7)$-Double Gaussian model}. Similar to \cref{fig:Pareto_frontier_continuous}, the Pareto frontier is shown with solid lines, while dominated points are shown with dashed lines.}
\end{figure}

\Cref{fig:QA_Pareto_frontier_continuous} presents an example of QA-tradeoff curves under various review noise levels, where the paper quality prior has a mean of $1$ and a standard deviation of $3$.
The curve starts at the point where all papers are accepted after a single review round: since $\mu_{\QualDist} = 1$, the conference quality is 1, and author welfare equals the conference value, which is 2.
In this example, as the acceptance threshold increases, conference quality improves because authors with negative-quality papers opt not to submit. However, the effect on author welfare depends on the review noise. When the reviews are noisy (high $\sigma$), increasing the threshold primarily reduces the acceptance probability, leading to a monotonic decline in author welfare. When the reviews are accurate (low $\sigma$), the increased acceptance threshold may significantly increase the conference value, because it is much harder for low-quality papers to get accepted. The increased utility upon acceptance can outweigh the drop in acceptance rate and raise author welfare.
Yet, as the acceptance threshold continues to increase (in the left-hand side of \cref{fig:QA_Pareto_frontier_continuous}) so that the corresponding de facto threshold $\theta > 0$, both conference quality and author welfare eventually decline to 0.

\subsubsection{Dominating Acceptance Thresholds}

Using the preceding numerical examples, we examine which acceptance thresholds or de facto thresholds are QB- or QA-dominating, and how the model parameters influence the dominating thresholds.

% \gscomment{we might warn folks earlier that our discussion will center around whether, if you are willing to give up on perfect conference quality, for either review burden or author utility, will you will be better off letting in more borderline papers, or rejecting more borderline papers.  Both will reduce conference quality, but how will review burden (or author utility) react? }

\paragraph{QB-tradeoff curves.}
Notice that deviations from the quality-maximizing de facto threshold ($\theta = 0$) in \emph{either} direction could be QB-dominating. First, the conference can decrease the threshold to accept some negative-quality papers, in order to accept the positive-quality papers in fewer rounds; alternatively, the conference can increase the threshold to give up on some borderline papers with positive quality which might otherwise take a large number of rounds until acceptance.

Clearly, which intervals of strategies are Pareto optimal depends on the distribution of paper quality and the review noise. For example, if there is a substantially larger number of borderline papers with negative quality than positive quality, marginally lowering the threshold will both degrade conference quality and increase review burden, but marginally increasing the threshold will decrease the review burden, though still degrade conference quality.  The latter will be Pareto optimal while the former will not. This can be seen from panel (a) of \cref{fig:Pareto_frontier_continuous}: increasing $\tau$ or $\theta$ from the quality-maximizing point ($\theta = 0$) yields acceptance policies that are QB-dominating.

Furthermore, as observed from panel (b) of \cref{fig:Pareto_frontier_continuous}, Pareto-optimal tradeoffs are typically achieved by decreasing $\tau$ when $\theta = 0$ under high review noise, and by increasing $\tau$ when review noise is low. This stems from the fact that with high noise, increasing the de facto threshold $\theta$ by a fixed amount (or equivalently, changing the conference quality by a fixed amount), requires a larger increase in the acceptance threshold $\tau$ (\cref{prop:gap-invariant}). This leads to a significant drop in the acceptance probability for all papers, resulting in more rounds of resubmission and thus a larger review burden. In contrast, when review noise is low, a small change in $\tau$ suffices to induce the same increase in $\theta$ (and conference quality). This means that the marginal benefit of increasing $\tau$ outweighs the marginal cost: the gain from reducing review burden by forgoing borderline papers is greater than the loss from slightly lowering the acceptance probability for all papers.
% the effect of giving up borderline papers to save review burden more impactful than the effect of decreasing the acceptance probability of every paper. 


The preceding paragraph's insight can be expressed intuitively as follows: if the reviewing quality is low, then striving for high quality will not dissuade authors of low quality from submitting sufficiently, and the primary effect of imposing ``high standards'' will be to impose multiple rounds of review on most papers (good or bad), before ultimately they are all accepted anyway. Thus, the conference might as well admit that it cannot distinguish paper qualities well, and be lenient in accepting. If the review quality is high, however, things change: bad papers are sufficiently likely to be rejected that the deterrence effect may save the reviewers significant work.
% \dkcomment{Added this on 08/17. Is it sufficiently accurate and ``useful''? Does it belong here, or in the conclusions/discussion? Or should it just go?}
\gscomment{I like this observation.  Not sure this is possible, but could we move it to the front and then explain why.   Currently we are explaining why and then summarizing what is happening.}


% \gscomment{I am not really sure what the below paragraph is adding.  Could we consider removing it?  In general, the discussion in this "paragraph" of the subsubsection makes lots of observations and is not organized in a way that the takeaway is easy to grasp.  The exception is that with less noise, it is easier to impose quality constraints, and with more noise, it is easier to just let stuff slide.}
% Panels (a) and (b) of \cref{fig:Pareto_frontier_continuous} show two cases where the expected quality under $\QualDist$ is smaller\fangcomment{equal?} and larger than 0, respectively.
% We observe that the Pareto optimal de facto thresholds are always greater than 0 in panel (a), while in panel (b) with $\sigma\ge 1$, negative de facto thresholds can be Pareto optimal.\yzcomment{$\theta = 0$ is the quality-maximizing point.}
% This means that for all curves in panel (a) and $\sigma\le 0.5$ in panel (b), the Pareto optimal QB-tradeoffs either optimize the conference quality, or increase the threshold to trade off conference quality for a reduced review burden. 
% However, for $\sigma\ge 2$ in panel (b), the Pareto optimal QB-tradeoffs correspond to a set of de facto thresholds that is not convex. In particular, there exists a $\bar{\theta} > 0$, such that the Pareto optimal QB-tradeoffs either (1) only accept papers with quality at least $\bar{\theta}$ or (2) keep out the bad papers (with quality less than some threshold $\theta \leq 0$). Here, having a de facto threshold of $0 < \theta < \bar{\theta}$ is Pareto dominated by some threshold of $\theta' < 0$ which yields a higher conference quality, and with a lower review burden.
% Furthermore, when $\sigma = 1$ in panel (b), there exists a $\ubar{\theta} < 0$, such that every point corresponding to $\theta < \ubar{\theta}$ is dominated by some threshold $\dkreplace{\theta \in}{\theta'} > \bar{\theta}$.

\paragraph{QA-tradeoff curves.}
Different from the QB-tradeoff curves, where a sufficiently high acceptance threshold is never dominated, in QA-tradeoff curves, 
% the dominating regime \dkcomment{I don't understand the role of ``the dominating regime'' in this sentence.}, 
the dominating tradeoffs are usually achieved when $\theta\in [\ubar{\theta},0]$ for some negative value $\ubar{\theta}$ (see \cref{fig:QA_Pareto_frontier_continuous}). This means that at the point $\theta = 0$, lowering the acceptance threshold tends to be QA-dominating, primarily because authors prefer lenient acceptance policies.

Recall the suggestion discussed in \cref{subsec:paradox}: the conference can lower the acceptance threshold to accept more papers quickly. 
We have seen, both in the previous section and in \cref{fig:Pareto_frontier_continuous}, that this strategy does not necessarily reduce the review burden due to the resubmission gap. Here, we can see that sometimes it can also harm author welfare.
For example, in \cref{fig:QA_Pareto_frontier_continuous}, we can observe that the policies near the point (2,1) are typically dominated for curves with small $\sigma$. This is because the conference value is endogenous: the policy that trivially accepts all papers tends to have a low conference value, which harms the authors' utilities.


% \subsection{Continuous Models Studied}\label{sec:conti_model}
% In this section (and the following sections that contain discussions on the continuous model), we frequently use the following special cases of our general continuous model as examples for our analysis and plots.

% The \emph{$(\sigma, \QualDist, \NumReviews, \TD)$-Gaussian model} is a continuous model with noiseless authors; the noise for each review is drawn from a Gaussian distribution $\REVNOISEDIST = \Gaussian{0}{\sigma}$. The parameters $\QualDist, \NumReviews$, and $\TD$ are, as before, the prior, the number of solicited reviews, and the discount factor, respectively. 
% The \emph{$(\sigma, \mu_{\QualDist}, \sigma_{\QualDist}, \NumReviews, \TD)$-Double Gaussian model} is a Gaussian model with the prior $\QualDist = \Gaussian{\mu_{\QualDist}}{\sigma_{\QualDist}}$.


%%%%%%%%%%%%%%%% Old QB-tradeoff section
% \subsection{Tradeoff between Conference Quality and Review Burden: QB-tradeoff} 
% \label{subsec:QB-tradeoff}

% We first consider the tradeoff between the conference quality and the review burden. We focus on the average number of reviews per \emph{paper} (including papers that were never submitted, and thus incurred $0$ reviews) as the relevant measure of the review burden.
% \yichiedit{
% This accounts for all reviews a paper receives over multiple rounds of submission\textemdash first submission, second submission, and so on\textemdash until it is eventually accepted.
% Let the de facto threshold be $\theta$ and let the corresponding acceptance policy be $\ACCMAP$. The total review burden under the continuous model is given by:
% \begin{align}\label{eq:review_burden}
%     R(\theta) &= m \int_{\theta}^\infty \QualDens{q} (1 + (1-\AccP{\ACCMAP}{q}) + (1-\AccP{\ACCMAP}{q})^2 + \dots) dq =  m\int_{\theta}^\infty \QualDens{q}/\AccP{\ACCMAP}{q} dq.
% \end{align}
% }
% Intuitively, if there are many borderline papers, whose quality is near the acceptance threshold $\tau$, they may go through many rounds of resubmission and increase the review burden.  Raising or lowering the threshold slightly might lead to a different borderline regime with a smaller fraction of papers on the border.  We already saw that the conference quality is maximized by a de facto threshold of 0.  At this point, changing the threshold will come at a cost to the conference quality, either by losing out on some good papers or by accepting some bad papers.  In the extremes, rejecting everything will lead to a review burden of 0; accepting everything\yichiedit{, which is sometimes infeasible,} will lead to a review burden of 1.  The \emph{QB-Tradeoff} traces how the review burden and conference quality jointly vary across all possible acceptance policy thresholds \yichiedit{that yield feasible de facto thresholds.}  

% We would like to understand the Pareto frontier of the QB-tradeoff over threshold acceptance policies\textemdash that is, holding all other parameters fixed, we aim to identify which acceptance thresholds are Pareto optimal.

% Notice that deviations from the optimal de facto threshold of 0 in \emph{either} direction could be Pareto optimal. First, the conference can decrease the threshold to accept some negative-quality papers, in order to accept the positive-quality papers in fewer rounds; alternatively, the conference can increase the threshold to give up on some borderline papers with positive quality which might otherwise take a large number of rounds until acceptance.

% Clearly, which intervals of strategies are Pareto optimal depends on the distribution of paper quality \yichiedit{and the review noise}. For example, if there is a substantially larger number of borderline papers with negative quality than positive quality, marginally lowering the threshold will both degrade conference quality and increase review burden, but marginally increasing the threshold will decrease the review burden, though still degrade conference quality.  The latter will be Pareto optimal while the former will not.

% \yichiedit{Furthermore, Pareto-optimal tradeoffs are typically achieved by decreasing $\tau$ when $\theta = 0$ under high review noise, and by increasing $\tau$ when review noise is low.\fangcomment{do we open the section saying we focus on threshold acceptance policy?} This stems from the fact that with high noise, increasing the de facto threshold $\theta$ by a fixed amount (or equivalently, changing the conference quality by a fixed amount), requires a larger increase in the acceptance threshold $\tau$ (\cref{prop:gap-invariant}). This leads to a significant drop in the acceptance probability for all papers, resulting in more rounds of resubmission and thus a larger review burden. In contrast, when review noise is low, a small change in $\tau$ suffices to induce the same increase in $\theta$ (and conference quality), making the effect of giving up borderline papers to save review burden more impactful than the effect of decreasing the acceptance probability of every paper.
% }\fangcomment{we need to update this after fixing proposition 5.  I find it is difficult to map Figure 1 to $\tau$ and $\theta$.  It may be good to mention how to get the value of $\tau$ and $\theta$ from the plots.  }
% \begin{figure}[htb]
%      \centering
%      \begin{subfigure}[b]{0.4\textwidth}
%          \centering
%          \includegraphics[width=\textwidth]{Plots/Pareto_optimal_vs_noise_continuous.png}
%          \captionsetup{size=}
%          \caption{}
%      \end{subfigure}
%      \hfill
%      \begin{subfigure}[b]{0.4\textwidth}
%          \centering
%          \includegraphics[width=\textwidth]{Plots/Pareto_optimal_vs_noise_continuous_case2.png}
%          \captionsetup{size=}
%          \caption{}
%      \end{subfigure}
%      \hfill
%      \caption{(a) shows the QB-tradeoff of a \emph{$\boldsymbol{(\sigma, \mu_{\QualDist} = 0, \sigma_{\QualDist} = 1, \NumReviews = 1, \ConfValue = 5, \TD = .7)}$-Double Gaussian model}.  (b) shows the QB-tradeoff of a \emph{$\boldsymbol{(\sigma, \mu_{\QualDist} = .8, \sigma_{\QualDist} = 1, \NumReviews = 1, \ConfValue = 5, \TD = .7)}$-Double Gaussian model}.  In each case, the review quality $\sigma$ is varied over five discrete options.  For each $\sigma$, the curve shows the possible QB-tradeoffs as the acceptance threshold is varied continuously. The Pareto frontier is shown with solid lines, while dominated points are shown with dashed lines.     \label{fig:Pareto_frontier_continuous}}
% \end{figure}

% \cref{fig:Pareto_frontier_continuous} maps the QB-tradeoff for various settings. 
% In each setting, there is a point at (0, 0) that corresponds to rejecting all submissions. As the threshold is decreased, high-quality papers start being submitted, increasing both the conference quality and review burden.  When the de facto threshold is 0, conference quality is maximized. Subsequently, a further decrease in the threshold leads to more low-quality papers being accepted, lowering the conference quality.  We see here that the effect on the review burden is mixed: sometimes it increases \yichiedit{(e.g., the curves in \cref{fig:Pareto_frontier_continuous} (b) with $\sigma < 1$)} while other times it decreases \yichiedit{(e.g., the curves in \cref{fig:Pareto_frontier_continuous} (b) with $\sigma \ge 1$)}.  
% As the threshold decreases further, we see a point \yichiedit{where all submitted papers are accepted with one round of review.
% In this case, either all papers are submitted and accepted immediately, resulting in a review burden of 1 (e.g., panel (b)), or papers with quality above $\theta_{\inf}$ are submitted, resulting in a conference value $\ConfValue = 1$ and a review burden smaller than 1 (e.g., panel (a)).
% }

% At that point, the conference quality is 0 in panel (a), but .8 in panel (b) due to the different priors on paper quality.     

% In Panel (a), the Pareto optimal de facto thresholds are always greater than 0.  This is because, for any $\theta > 0$, the de facto thresholds of $\theta$ and $-\theta$ have the same conference quality, but $\theta$ will have a smaller review burden because the number of submitted papers overall is smaller, and the number of submitted papers just above the boundary will also be smaller.   
% \yichiedit{
% Panels (a) and (b) show two cases where the expected quality under $\QualDist$ is smaller and larger than 0, respectively.
% As expected, the Pareto optimal de facto thresholds are always greater than 0 in panel (a), while in panel (b) with $\sigma\ge 1$, negative de facto thresholds can be Pareto optimal.\fangcomment{I am pretty lost here.  how can reader get the de factor thresholds from the figure 1?   }
% This means that for all curves in panel (a) and $\sigma\le 0.5$ in panel (b), the Pareto optimal QB-tradeoffs either optimize the conference quality, or increase the threshold to trade off conference quality for a reduced review burden. 
% However, for $\sigma\ge 2$ in panel (b), the Pareto optimal QB-tradeoffs correspond to a set of de facto thresholds that is not convex. In particular, there exists a $\bar{\theta} > 0$, such that the Pareto optimal QB-tradeoffs either (1) only accept papers with quality at least $\bar{\theta}$ or (2) keep out the bad papers (with quality less than some threshold $\theta \leq 0$). Here, having a de facto threshold of $0 < \theta < \bar{\theta}$ is Pareto dominated by some threshold of $\theta' < 0$ which yields a \dkreplace{larger}{higher} conference quality, \dkreplace{but at}{and with} a lower review burden.
% Furthermore, when $\sigma = 1$ in panel (b), there exists a $\ubar{\theta} < 0$, such that every point \dkreplace{corresponds}{corresponding} to $\theta < \ubar{\theta}$ is dominated by some threshold $\dkreplace{\theta \in}{\theta} > \bar{\theta}$
% }

% Panel (b) contains some interesting observations. First, for $\sigma=0.2, 2$ and $5$, marginally increasing the de facto threshold from 0 remains Pareto optimal, while marginally decreasing it is Pareto dominated.  However, the opposite is true for $\sigma=0.5$ and 1. This is because, in all cases, when the threshold decreases, the number of submitted papers increases and the number of papers just above the borderline (that require more rounds of submission in expectation) decreases.  However, the rate at which the latter decreases depends non-monotonically on the review quality.  Second, notice that in the four largest settings of $\sigma$, some Pareto optimal thresholds lie both above and below the de facto threshold of 0.  

% Thus, for all curves in panel (a) and $\sigma=0.2$ in panel (b), we have that the Pareto optimal QB-tradeoffs either optimize the conference quality, or increase the threshold to trade off conference quality for a reduced review burden.  

% However, for $\sigma=0.5$ and 1 in panel (b), the Pareto optimal QB-tradeoffs correspond to a set of de-facto thresholds that is not convex. In particular, there exists a $\theta_0 > 0$, such that the Pareto optimal QB-tradeoffs either (1) only accept papers with quality at least $\theta_0$ or (2) keep out the bad papers (with quality less than some threshold $\theta \leq 0$). Here, having a de facto threshold of $0 < \theta < \theta_0$ is Pareto dominated by some threshold of $\theta' < 0$ which yields the same conference quality, but at a lower review burden.


% Finally, for $\sigma=2$ and 5 in panel (b), there are three ranges of Pareto optimal QB-tradeoffs! That is, there exists $\theta_2 < 0 < \theta_1 < \theta_0$ such that the Pareto optimal policies either (1) only accept papers above some threshold $\theta \geq \theta_0$;  (2) reject the really bad papers by setting the threshold to $\theta \leq \theta_2$; or (3) maximize conference quality despite a relatively high review burden with  $0 \leq \theta \leq \theta_1$.

% \subsection{Tradeoff between Conference Quality and Author Welfare: QA-tradeoff} 
% \label{subsec:QA-tradeoff}

% Recall that we define the QA-tradeoff analogously to the QB-tradeoff, by replacing the review burden with the author welfare in \cref{subsec:number-reviews}.
% In the continuous model, a de facto threshold $\theta$ determines a conference value $\ConfValue$ and induces an acceptance policy $\ACCMAP$\fangedit{, as shown in \cref{prop:de_facto}}.
% The expected utility of an author with a paper of quality $q\ge \theta$ who decides to submit and keep resubmitting until acceptance is
% \begin{equation*}
%     u^{(a)}(q, \ACCMAP, \ConfValue) = \AccP{\ACCMAP}{q} \cdot V + \eta \cdot (1-\AccP{\ACCMAP}{q}) \cdot \AccP{\ACCMAP}{q} \cdot V + \cdots = \frac{\AccP{\ACCMAP}{q} \cdot V}{1-\eta \cdot (1-\AccP{\ACCMAP}{q})}.
% \end{equation*}
% \fangreplace{In the continuous model,}{The author welfare of all submitted papers is thus } $U^{(a)}(\theta) = \int_{\theta}^\infty \QualDens{q} u^{(a)}(q, \ACCMAP, \ConfValue) dq$.\fangcomment{I move the continuous before categorical}

% In the categorical model, the \gsreplace{social}{author} welfare \fangreplace{of all submitted papers is thus }{ under a threshold acceptance policy $\ACCMAP[\tau,r]$ is } $U^{(a)}(\theta, r) = r\cdot \QualProb{\theta} u^{(a)}(\theta, \ACCMAP, \ConfValue) + \sum_{q>\theta} \QualProb{q} u^{(a)}(q, \ACCMAP, \ConfValue)$.


% In \cref{fig:QA_Pareto_frontier_continuous}, we present an example of QA-tradeoff curves under various review noise levels, where the paper quality prior has a mean of $1$ and a standard deviation of $3$.
% The curve starts at the point where all papers are accepted after a single review round: the conference quality is 1,\fangcomment{do we want exogenous conference quality?} and author welfare equals the conference value, which is 2

% \fangcomment{I guess this statement only holds for certain range of threshold}\fangreplace{As}{Initially, in the upper right corner of \cref{fig:QA_Pareto_frontier_continuous}, when} the acceptance threshold increases, conference quality improves because authors with low-quality papers opt not to submit.\fangcomment{endogenous ?} However, the effect on author welfare depends on the review noise. When the reviews are noisy (high $\sigma$), increasing the threshold primarily reduces the acceptance probability, leading to a monotonic decline in author welfare. When the reviews are accurate (low $\sigma$), the increased acceptance threshold may significantly increase the conference value, because it is much harder for low-quality papers to get accepted. The increased utility upon acceptance can outweigh the drop in acceptance rate and raise author welfare.
% Yet, as the acceptance threshold continues \fangedit{(in the left hand side of \cref{fig:QA_Pareto_frontier_continuous}) }to increase so that the corresponding de facto threshold $\theta > 0$, both conference quality and author welfare eventually decline to 0.

% Different from the QB-tradeoff curves, where a significantly high acceptance threshold is never dominated, in QA-tradeoff curves, 
% % the dominating regime \dkcomment{I don't understand the role of ``the dominating regime'' in this sentence.}, 
% the dominating tradeoffs are usually achieved when $\theta\in [\ubar{\theta},0]$. This means that at the point $\theta = 0$\fangcomment{what is $\ubar{\theta}$? we may say negative value around 0?}, keep lowering the acceptance threshold \dkcomment{``to'' instead of ``at''?} \yichicomment{the sentence is modified} tends to be Pareto optimal, primarily because authors prefer lenient acceptance policies.


% \gscomment{should we also compare review burden and author welfare?}\yzcomment{This seems less interesting: accepting everything simply maximizes both.}\gscomment{good point, maybe include a note on this.}

\subsection{Resubmission Gap and Dominating QB-tradeoffs}
\label{subsec:gap-tradeoffs}

Here, we present an intuitive connection between the key concept in \cref{sec:thresholds-gaps}, the resubmission gap, and the key concept in this section, the QB-tradeoff, using the continuous model.
Recall that fixing the number of reviews per paper, the resubmission gap depends on the quality prior, the review noise, the author's discount factor, and the de facto threshold. 
Let $\tau(\theta\mid \QualDist, \RevSigDist, \eta, m)$ be the acceptance threshold that induces a candidate threshold $\theta$ as a de facto threshold, which is a function of $\theta$ conditioned on other model parameters. We present the following result.

\begin{proposition}\label{prop:gap-QB-dominance}
    Consider the continuous model with a fixed number of reviews per paper $m$ and a fixed prior distribution of paper quality $\QualDist$.
    We compare two settings: one with review signal distribution and author-discount factor $(\RevSigDist, \eta)$ and another with $(\RevSigDist', \eta')$. If for every candidate threshold $\theta$, the corresponding resubmission gap is larger in the first setting than in the second, i.e., $\tau(\theta\mid \QualDist, \RevSigDist, \eta, m) >  \tau'(\theta\mid \QualDist, \RevSigDist', \eta', m)$, then the QB-tradeoff of the second setting weakly dominates that of the first.
\end{proposition}

We defer the proof to \cref{app:proof-gap-QB-dominance}.
Intuitively, a setting with a larger resubmission gap requires a larger acceptance threshold to induce the same de facto threshold, resulting in a lower acceptance probability for every submitted paper.
This thus leads to more rounds of resubmissions and increases the review burden.

In the next two subsections, we examine how a larger review noise and a larger author discount factor each enlarge the resubmission gap, thereby leading to a dominated QB-tradeoff. We also analyze how these parameters affect the QA-tradeoff. 
% \dkcomment{Should it be singular or plural?}


\subsection{Dominating Tradeoffs: Review Noise}
\label{sec:QB-trade-noise}

In comparing the different QB-tradeoff curves of \cref{fig:Pareto_frontier_continuous} and the QA-tradeoff curves of \cref{fig:QA_Pareto_frontier_continuous}, we observe that any curves corresponding to higher-quality (i.e., lower-variance) reviews dominate similar curves corresponding to lower-quality reviews.
We show that this is not a coincidence and holds not just for Gaussian noise in the reviews, but for Blackwell dominating review quality (defined in \cref{def:blackwell}).
Note that the former is a special case of the latter.
In other words, we show that a better review quality can simultaneously benefit all three stakeholders of the review system.
In this subsection, unless otherwise specified, tradeoffs without qualification refer to both QA- and QB-tradeoffs.
% Later in \cref{subsec:qa_tradeoff}, we further show that a better review quality also leads to dominating QA-tradeoffs\textemdash review policies with Blackwell dominating review quality can achieve the same conference quality at a higher author welfare.

The full story is a bit more subtle. Whether Blackwell-dominating reviews imply better tradeoffs depends on what space of acceptance policies the conference can optimize over. We show that if the conference has all memoryless acceptance policies available, then better reviews can always be used to simulate worse reviews, and the conference can thus obtain at least the same tradeoff. Therefore, better reviews weakly dominate worse reviews even when the signals do not satisfy the monotone likelihood ratio (MLR) property. 
However, if the conference is restricted to threshold policies and the reviews do not necessarily have MLR (\cref{def:informative}), carefully chosen ``worse'' reviews may actually permit the use of a better threshold policy, achieving a better tradeoff. 
However, such behavior is indeed the result of signals violating the MLR property: if the review signals have MLR, Blackwell dominance again implies a weakly better QB-tradeoff under threshold policies.

\begin{definition}[Blackwell Dominance \citep{bohnenblust1949reconnaissance,blackwell1953equivalent}]
\label{def:blackwell}
Let $\RevSigDist: \QualSet \times \SigSet \to [0,1]$ and $\RevSigDistP: \QualSet \times \SigSet' \to [0,1]$ be two review signal distributions.
$\RevSigDist$ \emph{Blackwell dominates} $\RevSigDistP$ if there exists a garbling $\gamma: \SigSet \times \SigSet' \to [0,1]$ from $\SigSet$ to $\SigSet'$, where for all $\REVSIG \in \SigSet$,
% \dkcomment{Replacing signal \emph{vector} with signals everywhere here. Please keep consistent with future edits.} 
$(\gamma(\REVSIG, \REVSIGP))_{\REVSIGP \in \SigSet'}$ is a distribution on $\SigSet'$, such that for all $\REVSIGP \in \SigSet'$ and all $q \in \QualSet$:
\begin{align*}
\RevSigProbP[q]{\REVSIGP} 
& = \sum_{\REVSIG \in \SigSet} \RevSigProb[q]{\REVSIG} \cdot \gamma(\REVSIG, \REVSIGP).
\end{align*}
\end{definition}

\subsubsection{General memoryless acceptance policies.}

We state the following proposition in the categorical model with $\NumReviews = 1$ review. We discuss the (straightforward) extension to the continuous model and multiple reviews below.

\begin{proposition}
\label{prop:blackwell}
Consider two settings with $\NumReviews = 1$ in the categorical model that are identical except for the review signal distributions (which need not have MLR): 
the distribution $\RevSigDist$ of the first setting Blackwell-dominates the distribution $\RevSigDistP$ of the second setting.
Then, over memoryless acceptance policies, the tradeoff in the first setting weakly dominates the tradeoff in the second. 
\end{proposition}

The proof is given in \cref{app:proof-blackwell}. At a high level, the proposition holds because if $\RevSigDist$ Blackwell-dominates $\RevSigDistP$, then in the setting with $\RevSigDist$, a policy $\ACCMAP$ can perform the garbling $\gamma$ from \cref{def:blackwell} itself and then apply $\ACCMAP[']$; if $\ACCMAP[']$ is monotone, one can show that so is the resulting $\ACCMAP$.
Thus, the tradeoff in the first setting must weakly dominate the second setting.

% \dkdeletecomment{Since we are not giving the proof here, this is meaningless?}{For the continuous setting, the proof can be modified by replacing summation with integration.}
When the two settings have $\NumReviews > 1$ reviews drawn independently from $\RevSigDist$ and $\RevSigDistP$, respectively, where $\RevSigDist$ Blackwell-dominates $\RevSigDistP$, we can use the fact that applying the same garbling independently in each dimension gives a garbling on the $\NumReviews$-dimensional signal vectors. Therefore, viewing the entire vector as just one signal, the distribution in the first setting Blackwell-dominates that in the second setting, and Proposition~\ref{prop:blackwell} applies directly. 
Performing this reduction relies on the fact that Proposition~\ref{prop:blackwell} did not require signals to have MLR. After all, the MLR property is defined only for scalar-valued signals.
Similarly to the case of \emph{better} reviews, when \emph{more} reviews are obtained in the first setting, and the reviews in both settings are drawn from the same distribution, the signal of the combined reviews in the first setting Blackwell-dominates the signal in the second setting: this is because discarding the additional signals is easily seen to be a garbling.

\subsubsection{Threshold acceptance policies.}
\label{subsec:threshold-BW-domi}

Proposition~\ref{prop:blackwell} shows that a better review quality (in the Blackwell sense) implies a better tradeoff if we allow the conference to apply any memoryless acceptance policy. This result even holds for review signals that do not necessarily have MLR. In the following proposition, we further show that even if the conference is restricted to applying threshold policies, the same result holds if the review signals  \emph{do} satisfy the monotone likelihood ratio property.

\begin{proposition} \label{prop:blackwell-threshold}
  Consider two settings with $\NumReviews = 1$ \fangcomment{continuous?}\yzcomment{this should hold for both model, even though the proof is written for the categorical model.} review that both satisfy the MLR property and are identical except for the review signal distributions: 
  the distribution $\RevSigDist$ of the first setting Blackwell-dominates the distribution $\RevSigDistP$ of the second setting.
  Then, over threshold acceptance policies, the QA- and QB-tradeoff curves in the first setting weakly dominate those in the second.
\end{proposition}

The following lemma is central to the proof of \cref{prop:blackwell-threshold}.

\begin{lemma}\label{claim:blackwell-RB-better}
    Consider two threshold acceptance policies $\ACCMAP$ and $\ACCMAP[']$ which accept papers of quality $\bar{q}$ with equal probability in the first and the second setting in \cref{prop:blackwell-threshold}, respectively.  
    Then, 
    % {Given two settings in \cref{prop:blackwell-threshold} with two threshold acceptance policies $\ACCMAP$ and $\ACCMAP[']$ respectively, if there exists $\bar{q}$ so that the probability of accepting papers of quality $\bar{q}$ are identical under both settings,} 
    % under the author's $\bar{q}$-threshold strategy, the review burden of $\ACCMAP$ in the first setting is no larger than the review burden of $\ACCMAP'$ in the second setting.  
    the acceptance probability of a paper of quality $q$ in the first setting is no less than that in the second setting, for any $q>\bar{q}$.
\end{lemma}

The intuition behind \cref{claim:blackwell-RB-better} is that a larger review noise enlarges the resubmission gap, requiring a larger acceptance threshold to induce the same de facto threshold. Consequently, in the setting with better review quality, every submitted paper has a higher acceptance probability.

% This lemma says that if the policies $\ACCMAP$ and $\ACCMAP[']$ induce indifference at the same quality threshold in the authors, then the first setting has a lower review burden than the second.
% This lemma says that if the policies $\ACCMAP$ and $\ACCMAP[']$ induce the same acceptance probability for a paper with quality $\bar{q}$ under two settings with different review quality, then every paper with a higher quality than $\bar{q}$ will be accepted with a larger probability under the setting with better review quality.
The proof of \cref{prop:blackwell-threshold}, which is deferred to \cref{app:proof-blackwell-threshold}, then follows by creating an appropriate $\ACCMAP$ from $\ACCMAP[']$ such that there exists a paper quality with the same acceptance probability in both settings and \cref{claim:blackwell-RB-better} can be applied.
The lemma then suggests that the first setting can accept every submitted paper with fewer rounds of resubmissions than the second setting, leading to a smaller review burden and a larger author welfare.
%to compare. 
% \yzdelete{Given a $\ACCMAP[']$ for the second setting, if there is a paper quality in the second setting that agents submit with a probability $r\in (0,1)$, this is pretty straightforward.  If there is no such paper quality, we make $\ACCMAP[']$ slightly less strict so that such a quality exists. 
% In this case, the modified $\ACCMAP[']$ has a smaller review burden\yzedit{ and a larger author welfare} than the old $\ACCMAP[']$ while maintaining the same conference quality. 
% Then, by \cref{claim:blackwell-RB-better}, we can show that there is a $\ACCMAP$ inducing the same threshold in the first setting which has a smaller review burden\yzedit{ and a larger author welfare} than the modified $\ACCMAP[']$.}
% \yzcomment{Removed the detailed explanation for the proof. I feel it's not clear at all, and the current intuition is good.}


While the result again generalizes from the categorical model to continuous signals, it does not generalize to $\NumReviews > 1$ signals. The reason is that it relies on signals having MLR, a property that is not preserved when combining signals (shown in \cref{ex:threshold-counterexample} in \cref{sec:testing_tradeoff}). 
In particular, we have numerically found counterexamples which suggest that even though the review signal distribution of one setting Blackwell-dominates the distribution of another setting (and both review signals have MLR), it is possible that after combining two independent signals in each setting, the tradeoff achieved by threshold policies in the first setting does not (weakly) dominate the tradeoff in the second setting. Our counterexamples use three types of paper qualities, and the signal set contains three signals. Unfortunately, these counterexamples are both complicated and unintuitive, so we omit them from this paper.
% The counter-examples were found by exhaustive computational search, and the code for this search is available at \url{https://github.com/yichiz97/Conference-Peer-Review}.
 
These examples also suggest that if the reviews do not have MLR, Blackwell dominance does not imply better tradeoffs under threshold acceptance policies.

% \subsubsection{QA-tradeoffs: Review Noise}
% \label{subsec:qa_tradeoff}

% We have shown that \gsedit{when} fixing an author's threshold best response, which fixes the conference quality, an improved review quality always leads to a smaller review burden. In this subsection, we investigate the effect of review quality on author \gsdelete{social} welfare. 
% We show that, conditioned on the same de facto threshold, the acceptance policy under the setting with a better review quality always leads to larger author welfare compared to the acceptance policy under the setting with a weaker review quality. This pattern is \fangreplace{confirmed}{supported} in \cref{fig:QA_Pareto_frontier_continuous}.

% Fixing an author's threshold best response $(\theta, r)$, suppose the corresponding conference value is $\ConfValue$ and the acceptance policy is $\ACCMAP$ (both are functions of $(\theta, r)$).
% The expected utility of an author with a paper of quality $q\ge \theta$ who decides to submit and keep resubmitting until acceptance has an utility 
% \begin{equation*}
%     u^{(a)}(q, \ACCMAP, \ConfValue) = \AccP{\ACCMAP}{q} V + \eta (1-\AccP{\ACCMAP}{q}) \AccP{\ACCMAP}{q} V + \cdots = \frac{\AccP{\ACCMAP}{q} V}{1-\eta(1-\AccP{\ACCMAP}{q})}.
% \end{equation*}
% In the categorical model, the social welfare of all submitted papers is thus $U^{(a)}(\theta, r) = r\cdot \QualProb{\theta} u^{(a)}(\theta, \ACCMAP, \ConfValue) + \sum_{q>\theta} \QualProb{q} u^{(a)}(q, \ACCMAP, \ConfValue)$.
% In the continuous model, $U^{(a)}(\theta) = \int_{\theta}^\infty \QualDens{q} u^{(a)}(q, \ACCMAP, \ConfValue) dq$.

% We define QA-tradeoff analogously to QB-tradeoff by replacing review burden with the author social welfare.
% \gscomment{I think we need to again use better notation here.  Howe about ${U}^{(c)}{U}^{(a)}$ or $CA$-weakly dominates}
% We say \dkedit{that} a policy with\fangcomment{not sure how careful we want to be.  The values should depends on both acceptance policy and author's strategy} \fangreplace{$U^{(c)}$}{$\CONFUTIL$} and \fangreplace{$U^{(a)}$}{$\AUTHUTIL$} weakly dominates another policy with $\hat{U}^{(c)}$ and $\hat{U}^{(a)}$ if it has both a higher (or equal) conference quality and a higher (or equal) author welfare.
% A policy strictly dominates another if weak dominance holds and at least one of the inequalities is strict.
% We show that review quality has the same effect on QA-tradeoff curves and QB-tradeoff curves.
% \begin{corollary}\label{coro:QA-tradeoff}
%    Consider two settings with $\NumReviews = 1$ review that are identical except for the review signal distributions: 
%   the distribution $\RevSigDist$ of the first setting Blackwell-dominates the distribution $\RevSigDistP$ of the second setting.
%   Then, over monotone acceptance policies, the QA-tradeoff curve in the first setting weakly dominates the QA-tradeoff curve in the second setting.
%   Moreover, if the review signals \yzreplace{are informative}{satisfy MLR}, the statement holds for threshold acceptance policies.
% \end{corollary}

% We omit the proof of the above corollary, which follows directly from \cref{prop:blackwell} and \cref{prop:blackwell-threshold}.\fangcomment{why \cref{prop:blackwell}?  I think the new part is to show the author welfare is better.}
% The key observation is that while changing the review quality, review burden and author welfare are well aligned. This is because the expected number of reviews for a paper of quality $q$ is $\frac{1}{\AccP{\ACCMAP}{q}}$, which is decreasing in $\AccP{\ACCMAP}{q}$, and the expected utility of the author with this paper is $\frac{\AccP{\ACCMAP}{q} \ConfValue}{1-\eta(1-\AccP{\ACCMAP}{q})}$, which is increasing in $\AccP{\ACCMAP}{q}$. By \cref{claim:blackwell-RB-better}, the acceptance policy in the setting with better review quality accepts every submitted paper with a higher probability. Therefore, both the QB-tradeoffs and the QA-tradeoff improve with review quality.

% This result suggests that an improved review quality can benefit all three stakeholders of the system.

\subsection{Dominating Tradeoffs: Discount Factor}
\label{sec:dominating-value-discount}

We now explore the effect of the author discount factor on QA- and QB-tradeoffs.
We show that more patient authors unequivocally lead to a worse QB-tradeoff. 
However, the effect on QA-tradeoff is mixed --- whether author welfare increases with $\eta$ depends on the quality prior, the review noise distribution, and the de facto threshold.
% Next, we elaborate on an observation made in \cref{sec:additive-noise}, namely, that more patient authors lead to a worse QB-tradeoff, because authors will be more persistent in resubmitting borderline papers.

\paragraph{QB-tradeoffs.} As we have seen in \cref{sec:additive-noise}, a larger $\eta$ \dkreplace{enlarges}{increases} the resubmission gap, \dkreplace{meaning}{implying} that authors are more persistent in resubmitting their rejected papers under the same condition. Therefore, the conference has to raise its acceptance threshold so as to maintain its quality. This then results in more resubmissions and thus leads to a dominated QB-tradeoff. We formalize this insight in the following proposition.
\begin{proposition} \label{lemma:query burden value}
% Consider two settings that are identical except that they have attractiveness factors $\rho$ and $\rho'<\rho$, respectively.
% Consider their corresponding QB-tradeoff curves as the acceptance threshold is varied from $-\infty$ to $\infty$.
% The QB-tradeoff curve of the setting with attractiveness factor $\rho'$ dominates the QB-tradeoff curve of the setting with $\rho$.
Consider two settings that are identical except that they have different author discount factors, $\eta > \eta'$. Then, the QB-tradeoff curve in the setting with discount factor $\eta'$ dominates the QB-tradeoff curve in the setting with $\eta$.
\end{proposition}

% \gscomment{Seems like this should mention that increasing the discount factor increases the resubmission gap, and that is what is driving this result.}

We defer the proof to \cref{app:proof-QB-tradeoff}. 
\cref{lemma:query burden value} suggests that having more patient authors will harm the QB-tradeoff, in the sense that it will be dominated by the original setting. 
% This result holds across all the models that we consider; it is also confirmed by our analysis based on real data in \cref{sec:noiseless-ICLR}.
It also cautions against certain peer review experiments that may increase the author discount factor --- for example, shortening the review cycle or eliminating the rebuttal phase.

\paragraph{QA-tradeoffs.}
% \gscomment{I am not sure if it is more complicated than in my mind, but can't we just say that there are two forces at work when the discount factor decreases: 1) the the resubmission gap increases; 2) that the authors utility for events in the future decrease.  The first of these means that the authors utility will increase because for the same defacto threshold, the acceptance threshold will be lower and thus every paper will be accepted sooner.  However, the second point is bad for every author that does not get his paper accepted in the first round.}
We now investigate how author welfare changes with the discount factor $\eta$. 
Intuitively, while increasing $\eta$ raises utility upon acceptance for a fixed number of resubmissions, it also widens the resubmission gap, forcing the conference to raise its threshold. This, in turn, increases the expected number of resubmissions, potentially reducing authors’ utility.

We formalize the above intuition in the continuous model.
Suppose that the de facto threshold is fixed at $\theta$, which implies fixed conference value $\ConfValue$ and conference quality $\CONFUTIL$.
The utility of an author with a paper of quality $q$ is then a function of the acceptance threshold $\tau$ (which depends on $\eta$), i.e., $u^{(a)}(q, \tau(\eta)) = \frac{\AccP{\tau(\eta)}{q} \cdot V}{1-\eta \cdot (1-\AccP{\tau(\eta)}{q})}$.
The acceptance threshold should make the authors with borderline papers indifferent between submitting and taking the outside option, i.e., $\tau(\eta)$ is the solution in $\tau$ to $\AccP{\tau}{\theta} = \frac{1-\eta}{V-\eta}$ with $V$ being a constant.
Writing the author's utility as a function of $q$ and $\eta$, we are primarily interested in the conditions under which $u^{(a)}(q, \eta) = \frac{\AccP{\tau(\eta)}{q} \cdot V}{1-\eta \cdot (1-\AccP{\tau(\eta)}{q})}$ is increasing in $\eta$.
% \fangcomment{instead of the following local characterization, can we have a figure showing $u^{(a)}$ under a fixed de facto threshold?}
% \yzcomment{I tried. The figures don't look pretty. the curves for most of $u_a$ are flat, with one curve having a very large gradient. I slightly prefer the current form.}

\begin{proposition}\label{lem:QA_eta}
Consider the continuous model with a fixed candidate threshold $\theta$ and a discount factor $\eta$. Let $\tau$ be the acceptance threshold that induces $\theta$ as the de facto threshold, as defined in \cref{prop:gap-invariant}. Then, for any paper of quality $q \ge \theta$, the author's marginal utility with respect to $\eta$ is positive
% \fangcomment{increasing for all $\eta\in (0,1)$?  Based on the proof, it is only increasing locally: there exists a small enough $\epsilon$ so that $u^{(a)}(q,\eta)>u^{(a)}(u,\eta')$ if $\eta<\eta'<\eta+\epsilon$} 
if $h(q) < h(\theta)$, negative if $h(q) > h(\theta)$, and zero if $h(q) = h(\theta)$, where
\[
h(q) = \frac{f^{(r)}(\tau - q)}{F^{(r)}(\tau - q) \cdot \left(1 - F^{(r)}(\tau - q)\right)},
\]
and $f^{(r)}$ and $F^{(r)}$ are the pdf and cdf of the review noise distribution, respectively.
\end{proposition}

% \begin{lemma}\label{lem:QA_eta}
% In the continuous model, fix a feasible de facto threshold $\theta$ and a discount factor $\eta$. Let $\tau$ be the acceptance threshold corresponding to $\theta$ as given by \cref{prop:gap-invariant}. Then, the utility of an author with a paper of quality $q$ increases, decrease, or doesn't change with $\eta$ if 
% $h(q) = \frac{f^{(r)}(\tau-q)}{\REVNOISEDIST(\tau-q)\left(1-\REVNOISEDIST(\tau-q)\right)} <, =, \text{or } > h(\theta)$. 
% \end{lemma}
We defer the proof of \cref{lem:QA_eta} to \cref{app:proof-QA-eta}.
The proposition highlights that whether an author’s utility increases with the discount factor $\eta$ depends on the behavior of the function $h(q)$ for $q > \theta$, which in turn depends on the shape of the review noise distribution. 
Notably, even for standard noise models such as the normal distribution, there is no general guarantee that $h(q) > h(\theta)$ or $h(q) < h(\theta)$, meaning that an author's utility is in general non-monotonic in $\eta$.

For example, when $f^{(r)}$ follows a zero-mean normal distribution, $h(q)$ is symmetric at $\tau$ and convex, with $h(q) \to \infty$ as $q \to \pm\infty$, and minimized at $q = \tau$. 
This implies that when $\tau > \theta$\textemdash as is typically the case\textemdash there exists a threshold $\bar{q} > \theta$ such that author utility increases with $\eta$ for $\theta < q < \bar{q}$, but decreases for $q > \bar{q}$. 
Intuitively, borderline papers near the de facto threshold benefit more from a higher discount factor as they usually experience more rounds of resubmissions.
In contrast, authors with high-quality papers usually experience fewer rounds of rejections and thus can benefit less from an increase in $\eta$. 
Their utilities decrease in $\eta$ because the conference has to raise $\tau$ so as to preserve the conference quality in response to the increase in $\eta$.

% This pattern is typically unique for single-peak thin-tailed noise distributions. 
However, for heavy-tailed review noise distributions such as the Cauchy distribution, the function $h(q)$ behaves differently: it increases from $0$ to its maximum value as $q$ increases toward $\tau$, and then decreases back to $0$ as $q \to \infty$. As a result, the pattern is reversed: authors’ marginal utility is positive in $\eta$ for high-quality papers, but is negative in $\eta$ for papers with relatively lower quality.
Intuitively, under heavy-tailed review noise, even high-quality papers have a non-trivial probability of being rejected in a single round of submission.

Because author welfare is the integral of individual author utilities across the quality distribution, it can both increase or decrease with $\eta$, depending on the review noise distribution and the paper quality prior. Therefore, attempts that aim to reduce the cost of resubmissions not only exacerbate the review burden but can also, in some cases, reduce overall author welfare.


\subsection{Acceptance Rate}\label{sec:acc_rate}

One may suspect that the higher the threshold, the more selective the conference, so the lower the acceptance rate.
But this is not always true. The reason is self-selection by authors of weaker papers, who may not submit in the first place.  As a result, those papers will not be rejected. 

We first develop some mathematical tools to help us reason about the interaction between the selectivity of the conference (the de facto threshold) and the acceptance rate.

Consider the continuous model.
Let $\tau$ be a non-trivial acceptance threshold, and $\theta$ the corresponding de facto threshold. As before, let $\AccP{\tau}{q}$ be the probability that a paper of quality $\Qual=q$ is accepted at the conference.  
In round $t$, the total resubmission ``density''
of papers with quality equal to $q\ge \theta$ is equal to 
$\QualDens{q}\cdot \sum_{j=0}^{t} (1-\AccP{\tau}{q})^{t-j}.$
As $t$ gets larger, 
% \footnote{Taking $t$ large enables us to take into account the full history of previously rejected papers that are resubmitted.} 
% \yichicomment{Removed a footnote here.}
this converges to $\QualDens{q}/\AccP{\tau}{q}$. 
Of these papers, a $\AccP{\tau}{q}$ fraction will be accepted in each round.
Hence, the acceptance rate converges to
\begin{equation}\label{eq:accept-rate}
    \AccRate = \frac{\text{Number of papers accepted this round}}{\text{Number of papers submitted this round}}  = \frac{\int_{\theta}^{\infty}\QualDens{q} dq}{%
        \int_{\theta}^{\infty} \QualDens{q}/\AccP{\tau}{q} \; dq}. 
\end{equation}

We now use \cref{eq:accept-rate} to intuitively reason about how the prior distribution $\QualDist$ affects the acceptance rate $\AccRate$.  Notice that it is the papers with low acceptance probabilities that disproportionally decrease the acceptance rate.  This is because they add only their mass to the numerator, but add their mass scaled by $1/\AccP{\tau}{q}$ to the denominator.  Thus, intuitively, for papers with quality at least $\theta$, if a $z$ fraction are borderline with quality  ``near'' $\theta$, and the other $1-z$ fraction has a very high acceptance probability,  the acceptance rate can be approximated by $\frac{1}{z/\AccP{\tau}{\theta} + (1 - z)}$.  Notice that this quantity is decreasing with $z$: the larger $z$, the smaller the acceptance rate.   

The measurement of $z$ intuitively resembles the \emph{hazard rate} of a distribution\fangcomment{Can I say the hazard rate of $\QualDist$?}, which is defined as $\frac{f(x)}{1-F(x)}$, where $f$ is the probability density function, and $F$ is the cumulative distribution function.  Similar to $z$, the hazard rate measures the probability of a paper on a boundary at $x$, $f(x)$, relative to the mass of papers larger than $x$, $1 - F(x)$.   The hazard rate is known to be monotone for thin-tailed distributions, like the Gaussian distribution. 
% Conversely,  the hazard rate is known to be non-monotone for heavy-tailed distributions, like the Cauchy distribution.  Finally, the hazard rate is known to be (eventually) constant for the Laplace distribution.  

The acceptance rate also depends on the acceptance probability of the borderline papers, $\AccP{\tau}{\theta}$. Based on \cref{lem:author_response}, this probability must be equal to $1/\rho(\theta)$, where the attractiveness factor $\rho$ is a function of $\theta$. As we argued before, $\rho$ is increasing in $\theta$, which means that $\AccP{\tau}{\theta}$ is decreasing in $\theta$. Therefore, the effect of $\AccP{\tau}{\theta}$ on the acceptance rate is monotone: fixing $z$, a larger acceptance probability of borderline papers always leads to a larger acceptance rate of the conference.

Using the above intuition, we might expect the acceptance rates to decrease as $\theta$ increases for the Gaussian prior.  This is because Gaussian distributions have a monotone (increasing) hazard rate; thus, both $z$ and $\AccP{\tau}{\theta}$ change in the direction to drive down the acceptance rate as $\theta$ increases.  
However, we might expect the acceptance rates to increase for quality distributions with non-monotone hazard rate. This is because for a quality prior with a non-monotone hazard rate, $z$ may decrease at some point and drive up the acceptance rate, since raising the acceptance threshold can substantially reduce the proportion of borderline papers among all submissions.

\cref{fig:Acc_rate_continuous} gives evidence to support this intuition: we observe that the acceptance rate is non-monotone in $\theta$ when the quality prior is a two-peaked Gaussian distribution (panel (c)). 
However, a non-monotone hazard rate in the quality prior does not necessarily imply a non-monotone acceptance rate. As illustrated in panel (b), the Cauchy prior, despite having a non-monotone hazard rate, yields a monotonically decreasing acceptance rate.

We also observe that compared with settings with smaller $\sigma$, settings with noisier review quality tend to have a monotone decreasing acceptance rate. This is because by \cref{prop:gap-invariant}, a larger review noise leads to a larger resubmission gap. Therefore, with the same increase in $\theta$, a larger rise in $\tau$ is required, and thus borderline papers are rejected with a larger probability.  
\gscomment{I don't think we have the energy/time to do this now, but I think this is a very nice insight if it formallizes.  In particular, this seems to be driving the results of section 4.1.   Overall, connecting observations to a few technical drivers helps to organize the results.} 
In other words, the derivative of $\AccP{\tau}{\theta}$ in $\theta$ is larger when the review is noisier. This enhances the original intuition that a more selective conference has a lower acceptance rate.
% Additionally, we might expect the acceptance rates to increase for the Cauchy distribution. This is because Cauchy distributions have a non-monotone hazard rate; thus, $z$ is likely to decrease at some point and drive up the acceptance rate.  Finally, we might expect that the acceptance rate is relatively constant after some point for the Laplace prior.   
% \cref{fig:Acc_rate_continuous} gives evidence to support this intuition.  
% We find it remarkable that despite the heuristic reasoning ``by analogy'' to the hazard rate, we exactly observe these outcomes for the paradigmatic distributions in their respective classes. 

% \begin{figure}[htb]
%      \centering
%      \begin{subfigure}[b]{0.32\textwidth}
%          \centering
%          \includegraphics[width=\textwidth]{Plots/Acc_rate_vs_threshold_Gaussian.png}
%          \captionsetup{size=}
%          \caption{Gaussian prior.}
%      \end{subfigure}
%      \hfill
%      \begin{subfigure}[b]{0.32\textwidth}
%          \centering
%          \includegraphics[width=\textwidth]{Plots/Acc_rate_vs_threshold_Laplace.png}
%          \captionsetup{size=}
%          \caption{Laplace prior.}
%      \end{subfigure}
%      \hfill
%      \begin{subfigure}[b]{0.32\textwidth}
%          \centering
%          \includegraphics[width=\textwidth]{Plots/Acc_rate_vs_threshold_Cauchy.png}
%          \captionsetup{size=}
%          \caption{Cauchy prior.}
%      \end{subfigure}
%      \hfill
%      \caption{
%      The acceptance rate vs.~acceptance threshold under different prior distributions. Here, the noise distribution $\boldsymbol{\REVNOISEDIST}$ is fixed as a normal distribution with zero mean and standard deviation of $\boldsymbol{1}$. Three types of prior distributions of the paper quality, all with zero mean and standard deviation (scaling factor) $\boldsymbol{\sigma}$, are considered: (a) the Normal distribution; (b) the Laplace distribution; and (c) the Cauchy distribution. \label{fig:Acc_rate_continuous}}
% \end{figure}

\begin{figure}[htb]
     \FIGURE
     {\begin{subfigure}[b]{0.32\textwidth}
         \centering
         \includegraphics[width=\textwidth]{Plots/acc_rate_normal_sigma.pdf}
         \captionsetup{size=}
         \caption{Gaussian prior.}
     \end{subfigure}
     \hfill
     \begin{subfigure}[b]{0.32\textwidth}
         \centering
         \includegraphics[width=\textwidth]{Plots/acc_rate_cauchy_sigma.pdf}
         \captionsetup{size=}
         \caption{Cauchy prior.}
     \end{subfigure}
     \hfill
     \begin{subfigure}[b]{0.32\textwidth}
         \centering
         \includegraphics[width=\textwidth]{Plots/acc_rate_mixture_sigma.pdf}
         \captionsetup{size=}
         \caption{Mixture Gaussian prior.}
     \end{subfigure}
     \hfill
     }
     {The Acceptance Rate vs.~De Facto Threshold under Different Prior Distributions. \label{fig:Acc_rate_continuous}}
     {The noise distribution $\REVNOISEDIST$ is fixed as a normal distribution with zero mean and standard deviation of $1$. Three types of prior distributions of the paper quality are considered: (a) the Normal distribution with $\mu_q = 0$ and $\sigma_q = 1$; 
     (b) the Cauchy distribution with $\mu_q = 0$ and $\sigma_q = 1$;
     (c) the mixture of two Normal distributions with $\lambda = 0.5$, $\mu_{q} = 0$, $\mu_q' = 4$, and $\sigma_q = \sigma_q' = 1$.}
\end{figure}

Another interesting observation is that the quality distribution of submitted papers is not a good reflection of the prior quality distribution of papers, even conditional on being above the de facto threshold.  The reason is that papers nearer the de facto threshold need to be submitted more times (on average) before being accepted than higher-quality papers.  Therefore, they are over-represented among the submitted papers. This aligns well with many reviewers' observation in the real world that many of their assigned papers seem to be borderline. However, by \cref{eq:accept-rate}, the quality distribution of accepted papers is an accurate reflection of the prior quality distribution of papers conditional on being above the de facto threshold. 
