%% This is file `elsarticle-template-1-num.tex',
%%
%% Copyright 2009 Elsevier Ltd
%%
%% This file is part of the 'Elsarticle Bundle'.
%% ---------------------------------------------
%%
%% It may be distributed under the conditions of the LaTeX Project Public
%% License, either version 1.2 of this license or (at your option) any
%% later version.  The latest version of this license is in
%%    http://www.latex-project.org/lppl.txt
%% and version 1.2 or later is part of all distributions of LaTeX
%% version 1999/12/01 or later.
%%
%% Template article for Elsevier's document class `elsarticle'
%% with numbered style bibliographic references
%%
%% $Id: elsarticle-template-1-num.tex 149 2009-10-08 05:01:15Z rishi $
%% $URL: http://lenova.river-valley.com/svn/elsbst/trunk/elsarticle-template-1-num.tex $
%%
\documentclass[authoryear,11pt,preprint]{elsarticle}
% \documentclass[authoryear, review]{elsarticle}
%% Use the option review to obtain double line spacing
% \documentclass[preprint,review,12pt]{elsarticle}

%% Use the options 1p,twocolumn; 3p; 3p,twocolumn; 5p; or 5p,twocolumn
%% for a journal layout:
%% \documentclass[final,1p,times]{elsarticle}
%% \documentclass[final,1p,times,twocolumn]{elsarticle}
%% \documentclass[final,3p,times]{elsarticle}
%% \documentclass[final,3p,times,twocolumn]{elsarticle}
%% \documentclass[final,5p,times]{elsarticle}
%% \documentclass[final,5p,times,twocolumn]{elsarticle}
\usepackage{color}
\usepackage{lineno}
\usepackage{pdflscape}
\usepackage{amsmath}
\usepackage{booktabs}
\usepackage[hyphens]{url}
\usepackage{hyperref}
\usepackage{multirow}
\usepackage{addlines}
\usepackage{adjustbox}
\usepackage{natbib} %When changing reference style
\usepackage[english]{babel}
\usepackage[dvipsnames]{xcolor}
\usepackage{soul}
\usepackage{adjustbox}
\usepackage{verbatim}
\usepackage[margin=2.5cm]{geometry}% by courtesy of Mico
\usepackage{mathtools}
\usepackage{tikz}
\usepackage{pgfplots}
\usepackage{enumitem}
\usepackage{calc}

%\modulolinenumbers[30]
\definecolor{darkblue}{rgb}{0,0.2,0.4}
%% The graphicx package provides the includegraphics command.
\usepackage{graphicx}
\usepackage{subcaption}
%% The amssymb package provides various useful mathematical symbols
\usepackage{amssymb}
%% The amsthm package provides extended theorem environments
\usepackage{amsthm}
\usepackage{etoolbox}
\usepackage{algorithm}
\usepackage{algpseudocode}

\newtheorem{prop}{Proposition}

% start with some helper code
%%%%% algorithm
% This is the vertical rule that is inserted
\makeatletter
\newcommand*{\algrule}[1][\algorithmicindent]{%
  \hspace*{.2em}% <------------- This is where the rule starts from
  \vrule %height .75\baselineskip depth .25\baselineskip
  \hspace*{\dimexpr#1-.2em-.4pt}%
}

\newcommand{\StatePar}[1]{%
  \State\parbox[t]{\dimexpr\linewidth-\ALG@thistlm}{\strut #1\strut}%
}
\renewcommand{\ALG@beginalgorithmic}{\offinterlineskip}% Remove all interline skips

\newcount\ALG@printindent@tempcnta
\def\ALG@printindent{%
  \ifnum \theALG@nested > 0% is there anything to print
    \ifx\ALG@text\ALG@x@notext% is this an end group without any text?
      % do nothing
    \else
      \unskip
      % draw a rule for each indent level
      \ALG@printindent@tempcnta=1
      \loop
        \algrule[\csname ALG@ind@\the\ALG@printindent@tempcnta\endcsname]%
        \advance \ALG@printindent@tempcnta 1
        \ifnum \ALG@printindent@tempcnta<\numexpr\theALG@nested+1\relax
      \repeat
        \fi
    \fi
}
% the following line injects our new indent handling code in place of the default spacing
\patchcmd{\ALG@doentity}{\noindent\hskip\ALG@tlm}{\ALG@printindent}{}{\errmessage{failed to patch}}
% end vertical rule patch for algorithmicx
\makeatother

\algrenewcommand\algorithmicend{\strut\textbf{end}}
\algrenewcommand\algorithmicdo{\strut\textbf{do}}
\algrenewcommand\algorithmicwhile{\strut\textbf{while}}
\algrenewcommand\algorithmicfor{\strut\textbf{for}}
\algrenewcommand\algorithmicforall{\strut\textbf{for all}}
\algrenewcommand\algorithmicloop{\strut\textbf{loop}}
\algrenewcommand\algorithmicrepeat{\strut\textbf{repeat}}
\algrenewcommand\algorithmicuntil{\strut\textbf{until}}
\algrenewcommand\algorithmicprocedure{\strut\textbf{procedure}}
\algrenewcommand\algorithmicfunction{\strut\textbf{function}}
\algrenewcommand\algorithmicif{\strut\textbf{if}}
\algrenewcommand\algorithmicthen{\strut\textbf{then}}
\algrenewcommand\algorithmicelse{\strut\textbf{else}}
\algrenewcommand\algorithmicrequire{\strut\textbf{Input:}}
\algrenewcommand\algorithmicensure{\strut\textbf{Output:}}
\let\oldState\State
\renewcommand{\State}{\oldState\strut}
\algnewcommand{\IIf}[1]{\State\algorithmicif\ #1\ \algorithmicthen}
\algnewcommand{\EndIIf}{\unskip\ \algorithmicend\ \algorithmicif}

%%%%% nomenclature 
\usepackage{nomencl}
\makenomenclature
\setlength{\nomitemsep}{-\parsep}
\usepackage{framed} % Framing content
\usepackage{multicol} % Multiple columns environment

%%%%%%%%% euro symbol
\usepackage[utf8]{inputenc}
\usepackage{newunicodechar}
\newcommand\cedilla{}
\let\cedilla\c
\DeclareUnicodeCharacter{0229}{\c{e}}
\DeclareUnicodeCharacter{2009}{ °}
\usepackage[T1]{fontenc}
\usepackage{textcomp}
\usepackage{lmodern}
\usepackage{ltxcmds}
\usepackage{gensymb}

%%%%% for table format
\usepackage{siunitx}
\sisetup{output-decimal-marker={.}}
\DeclareSIUnit{\euro}{\texteuro}
%%%%% for coloring text
\makeatother
\definecolor{royalpurple}{rgb}{0.58, 0.44, 0.86}
\newcommand{\commentsb}[1]{{\color{royalpurple}SB: #1 }}
\newcommand{\comments}[1]{{\color{royalpurple}#1 }}
\newcommand{\red}[1]{{\color{red}#1}}
\newcommand{\blue}[1]{{\color{blue} #1}}
\newcommand{\orange}[1]{{\color{orange} #1}}
\allowdisplaybreaks

\DeclareMathOperator*{\argmax}{argmax}

%%%%% norm symbol
\newcommand\norm[1]{\left\lVert#1\right\rVert}

\usepackage{setspace}
\onehalfspacing
\hypersetup{colorlinks = true,linkcolor = blue,anchorcolor =red,citecolor = blue,filecolor = red,urlcolor = red, pdfauthor=author}

%%%%%%%%%%%%%%%%
% \renewcommand*\nompreamble{\begin{multicols}{2}}
% \renewcommand*\nompostamble{\end{multicols}}

%% The lineno packages adds line numbers. Start line numbering with
%% \begin{linenumbers}, end it with \end{linenumbers}. Or switch it on
%% for the whole article with \linenumbers after \end{frontmatter}.

%\usepackage{lineno}


%% natbib.sty is loaded by default. However, natbib options can be
%% provided with \biboptions{...} command. Following options are
%% valid:

%%   round  -  round parentheses are used (default)
%%   square -  square brackets are used   [option]
%%   curly  -  curly braces are used      {option}
%%   angle  -  angle brackets are used    <option>
%%   semicolon  -  multiple citations separated by semi-colon
%%   colon  - same as semicolon, an earlier confusion
%%   comma  -  separated by comma
%%   numbers-  selects numerical citations
%%   super  -  numerical citations as superscripts
%%   sort   -  sorts multiple citations according to order in ref. list
%%   sort&compress   -  like sort, but also compresses numerical citations
%%   compress - compresses without sorting
%%
%% \biboptions{comma,round}

% \biboptions{}
% \biboptions{}

\journal{}

\pgfplotsset{compat=1.17} % for plotting
\begin{document}

\begin{frontmatter}
%% Title, authors and addresses

\title{Integrated investment, retrofit and abandonment planning of energy systems with short-term and long-term uncertainty using enhanced Benders decomposition}

%% use the tnoteref command within \title for footnotes;
%% use the tnotetext command for the associated footnote;
%% use the fnref command within \author or \address for footnotes;
%% use the fntext command for the associated footnote;
%% use the corref command within \author for corresponding author footnotes;
%% use the cortext command for the associated footnote;
%% use the ead command for the email address,
%% and the form \ead[url] for the home page:
%%
%% \title{Title\tnoteref{label1}}
%% \tnotetext[label1]{}
%% \author{Name\corref{cor1}\fnref{label2}}
%% \ead{email address}
%% \ead[url]{home page}
%% \fntext[label2]{}
%% \cortext[cor1]{}
%% \address{Address\fnref{label3}}
%% \fntext[label3]{}


%% use optional labels to link authors explicitly to addresses:
%% \author[label1,label2]{<author name>}
%% \address[label1]{<address>}
%% \address[label2]{<address>}
\author[a]{Hongyu Zhang\corref{cor1}}
\ead{hongyu.zhang@ntnu.no}
\author[b]{Ignacio E. Grossmann}
\ead{grossmann@cmu.edu}
\author[c]{Brage Rugstad Knudsen}
\ead{brage.knudsen@sintef.no}
\author[d]{Ken McKinnon}
\ead{K.McKinnon@ed.ac.uk}
\author[d]{Rodrigo Garcia Nava}
\ead{rodrigo.garciana@gmail.com}
\author[a]{Asgeir Tomasgard}
\ead{asgeir.tomasgard@ntnu.no}

\cortext[cor1]{Corresponding author}

\address[a]{Department of Industrial Economics and Technology Management, Norwegian University of Science and Technology, Høgskoleringen 1, 7491, Trondheim, Norway}
\address[b]{Department of Chemical Engineering, Carnegie Mellon University, 5000 Forbes Avenue, Pittsburgh, PA 15213, USA}
\address[c]{SINTEF Energy Research, Kolbjørn Hejes vei 1B, 7491, Trondheim, Norway}
\address[c]{School of Mathematics, University of Edinburgh, James Clerk Maxwell Building, Peter Guthrie Tait Road, Edinburgh, EH9 3FD, United Kingdom}

\begin{abstract}
We propose the REORIENT (REnewable resOuRce Investment for the ENergy Transition) model for energy systems planning with the following novelties: (1) integrating capacity expansion, retrofit and abandonment planning, and (2) using multi-horizon stochastic mixed-integer linear programming with short-term and long-term uncertainty. We apply the model to the European energy system considering: (a) investment in new hydrogen infrastructures, (b) capacity expansion of the European power system, (c) retrofitting oil and gas infrastructures in the North Sea region for hydrogen production and distribution, and abandoning existing infrastructures, and (d) long-term uncertainty in oil and gas prices and short-term uncertainty in time series parameters. We utilise the special structure of multi-horizon stochastic programming and propose an enhanced Benders decomposition to solve the model efficiently. We first conduct a sensitivity analysis on retrofitting costs of oil and gas infrastructures. We then compare the REORIENT model with a conventional investment planning model regarding costs and investment decisions. Finally, the computational performance of the algorithm is presented. The results show that: (1) when the retrofitting cost is below 20\% of the cost of building new ones, retrofitting is economical for most of the existing pipelines, (2) platform clusters keep producing oil due to the massive profit, and the clusters are abandoned in the last investment stage, (3) compared with a traditional investment planning model, the REORIENT model yields 24\% lower investment cost in the North Sea region, and (4) the enhanced Benders algorithm is up to 6.8 times faster than the reference algorithm.
\end{abstract}

\begin{keyword}
OR in energy \sep Stochastic programming \sep Multi-horizon stochastic programming \sep Mixed-integer linear programming \sep Retrofit of energy systems
% keywords here, in the form: keyword \sep keyword

% MSC codes here, in the form: \MSC code \sep code
% or \MSC[2008] code \sep code (2000 is the default)

\end{keyword}
%Highlights
% Presemt a new vision on creating vir
\end{frontmatter}
%%
%% Start line numbering here if you want
%%
%% main text

% \linenumbers

\section{Introduction}
\label{sec:introduction}
Accelerating energy transition in all sectors is vital to achieve a carbon-neutral economy by 2050 \citep{REPowerEU, green_deal}. The committed emissions from existing  energy infrastructure jeopardise the 1.5 °C target \citep{Tong2019CommittedTarget}. It may be more beneficial to retrofit existing energy infrastructure than to abandon it. Abandoning existing energy infrastructure, such as oil and gas infrastructure, may cost substantially \citep{Bakker2019AnCampaigns}. Also, the oil and gas industry involves multi-billion-dollar investments and profits. Therefore, there is motivation to retrofit existing oil and gas infrastructure for clean energy production and transportation to (1) help the oil and gas industry transition to a clean energy producer, and (2) accelerate the energy transition by financing it using the gain from the avoided abandonment cost. Retrofitting existing oil and gas infrastructure for hydrogen production and transportation is drawing more attention due to the increasing demand for hydrogen. Most offshore pipelines can be used for hydrogen transport in Europe \citep{ValentinGentile2021Re-StreamEurope}. The European hydrogen infrastructure can grow to a pan-European network by 2040, which is largely based on repurposed existing natural gas infrastructure \citep{RikvanRossum2022EuropeanBackbone}. Retrofitting an existing offshore platform to a green hydrogen production platform is under exploration \citep{NeptuneEnergy2023PosHYdonSea}. We note that retrofitting may become an important pillar for accelerating the energy transition.  Therefore, in this paper, we analyse cost drivers that have the possibility to trigger widespread retrofit of offshore oil and gas infrastructure to clean energy generation and decarbonisation purposes. We use a high-fidelity, detailed spatial-temporal stochastic programming model to analyse these drivers for a large region with a substantial role in the energy supply to the surrounding countries.

Energy system infrastructure planning is crucial during the energy transition towards zero emission by $2050$. Optimisation models are widely used for the investment \citep{zhang2022_OEH,Cho2022RecentOptimization, Munoz2015} and operational \citep{Schulze2016,Philpott2000Hydro-electricDemand} planning of energy systems. Traditionally, capacity expansion, retrofitting \citep{store2018} and abandonment are planned separately using different models. However, as sector coupling and Power-to-X become more important, as well as the possibility to retrofit existing infrastructure for renewable energy production and distribution, the ability to analyse investments, retrofit and abandonment planning in a single integrated model becomes important to include all degrees of freedom in the planning process and by such minimise the risk of suboptimality. However, such integrated models have been less explored than their counterparts that treat investment, retrofit and abandonment independently. 

Managing uncertainty is crucial in a long-term planning model. Long-term energy system planning problems are normally prone to uncertainty on strategic and operational time horizons. Strategic uncertainty includes oil and gas prices, CO$_2$ budget, and CO$_2$ tax. Short-term uncertainty normally includes the availability of non-dispatchable renewable technologies and the failure of conventional generators. Multi-Horizon Stochastic Programming (MHSP) includes uncertainty in both time horizons more efficiently than traditional multi-stage stochastic programming \citep{Kaut2014}. Most previous studies on energy system planning consider only short-term uncertainty \citep{Backe2022EMPIRE:Analyses}. In this paper, the proposed model utilises MHSP and includes uncertainty from both time horizons.

The computational difficulty needs to be addressed. The special structure of MHSP allows decomposing of a problem with Benders-type algorithms. A stabilised adaptive Benders decomposition algorithm was proposed in \citep{zhang2022_benders} and demonstrated on a power system investment planning problem with up to 1 billion variables and 4.5 billion constraints. The algorithm showed a significant reduction in computational time. However, \cite{zhang2022_benders} dealt with a large-scale linear programming problem. In this paper, we consider a problem with binary variables in the investment planning part in order to capture the economic scale and model retrofit and abandonment decisions. Because the binary variables exist in the investment planning part, which is the reduced master problem in the Benders-type algorithm, we can still apply stabilised adaptive Benders directly. However, the algorithm may slow down due to the combinatorial part of the problem.

To fill the research gaps mentioned above, we first propose the REORIENT (REnewable resOuRce Investment for the ENergy Transition) model, which is a multi-horizon stochastic Mixed-Integer Linear Programming (MILP) model with short-term and long-term uncertainty for the investment, retrofit and abandonment planning for energy systems.  We consider retrofitting existing platforms for offshore green hydrogen production and pipelines for green and blue hydrogen distribution. We then demonstrate the REORIENT model on a European energy system planning problem. An enhanced Benders decomposition is proposed to solve the problem efficiently. we improve the method in \citep{zhang2022_benders} to allow it to solve problems with binary variables faster.

The contributions of the paper are the following: (1) we integrate investment planning, retrofitting, and abandonment in a single stochastic optimisation model, (2) we formulate the problem using multi-horizon stochastic MILP, (3) we propose an enhanced Benders decomposition method to solve such large scale model, and (4) we demonstrate the model on a large-scale planning problem for the European energy system to analyse the planning decisions and costs and provide global insights.

The outline of the paper is as follows: Section \ref{sec:literature_review} introduces the background knowledge regarding capacity expansion planning, retrofit planning and abandonment planning, stochastic programming, MHSP, and Benders decomposition. Section \ref{sec:problem_description_modelling_assumptions} provides the problem description, modelling strategies and assumptions. Section \ref{sec:Benders decomposition in MHSP} presents the proposed enhanced Benders decomposition. Section \ref{sec:model} presents the REORIENT model. Section \ref{sec:results} reports the computational results and numerical analysis. Section \ref{sec:discussion} discusses the implications of the method and results and summaries the limitations of the research. Section \ref{sec:conclusions} concludes the paper and suggests further research.

\section{Literature review}
\label{sec:literature_review}
In the following, we present a  brief overview of relevant literature on capacity expansion planning, abandonment planning, retrofit planning, MHSP, and Benders decomposition.

\subsection{Capacity expansion planning}
\label{sec:capacity expansion planning}
Capacity expansion planning problems normally consider an existing system with historical capacity or a new system and make investment planning to fulfil the demand under, among others, environmental restrictions. Capacity expansion problems are formulated either in deterministic models \citep{Lara2018DeterministicAlgorithm} or stochastic programming models \citep{Backe2022EMPIRE:Analyses, Conejo2016}. \cite{Backe2022EMPIRE:Analyses} utilised multi-horizon formulation but did not include long-term uncertainty. To capture the economic scale of investment decisions, sometimes MILP is used \citep{lara2020}. To gain enough environmental and economic insights from such models, sometimes a large-scale problem needs to be modelled, such as \citep{Li2022,zhang2022_OEH}. \cite{Munoz2016} proposed a new bounding scheme and combine it with Benders decomposition to solve a large investment planning problem that is formulated as MILP. In addition to planning for power, natural gas and heat systems separately, planning for integrated multi-carrier systems is also studied. Energy hubs that convert, condition and store multiple energy carriers in an investment planning problem were studied in \citep{zhang2022_benders}. Stochastic programming is also exploited in natural gas systems \citep{Fodstad2016}, offshore oil and gas infrastructure planning \citep{Gupta2014}, and hydrogen network \citep{Galan2019}.

\subsection{Retrofit planning}
\label{sec:retrofit planning}
In grassroots design, the process decisions are made first to be followed by equipment decisions, but the retrofit design also requires models that can rate existing equipment for proper analysis. A comparison of grassroots and retrofit design has been presented in \citep{grossmann1987}. The combinatorial nature of the retrofit planning problems makes such models much more complex. There are several reasons for retrofit, including (1) processing a new feedstock, (2) improving the economics by the use of less energy per unit of production, (3) making a new product, and (4) increasing the conversion of the feedstocks, among many other reasons. In this paper, we consider retrofitting to make a new product. A debottlenecking strategy was proposed for retrofit problems \citep{Harsh1989AProblem}. A systematic procedure for the retrofit of heat exchanger networks was presented in \citep{yee1991}. Retrofit of heat exchangers has been extensively studied in the past decades \citep{Pan2013ADetails,Wang2012ApplicationNetwork}. A high-level optimisation model for the retrofit planning of process networks was presented in \citep{Jackson2002High-levelNetworks}, in which the retrofit over several time periods was addressed. The proposed strategy consists of a high-level analysis of the entire network and a low-level analysis of a specific process flowsheet. The problem is formulated using a multi-period generalised disjunctive programming model, which is reformulated using a mixed-integer linear program using the convex hull formulation. In this paper, considering the scope and size of the problem, we only consider the high-level modelling of the retrofit. 

\subsection{Abandonment planning}
\label{sec:plug and abandonment planning}
Abandonment planning includes abandon of power plants that exceed their lifetimes, and mature oil and gas fields, among others. Existing literature focuses on the plug and abandonment campaign in the oil and gas industry. It is because many wells are planned to be plugged and abandoned, and such activity will cost substantially \citep{Bakker2019AnCampaigns}. The plug and abandonment cost is estimated at £5-15 million per well for the next decade, and thousands are to be abandoned just in offshore regions. Plug and abandonment planning is usually formulated as a profit maximisation problem \citep{Bakker2021MatureProgramming} or cost minimisation problem \citep{Bakker2019AnCampaigns, Bakker2021VehiclePlanning}. Real options theory is also used for oil and gas field development \citep{Stein-ErikFleten2011RealTie-ins,store2018,Bakker2019AnCampaigns}. In addition to abandonment, investment and operational scheduling optimisation in the oil and gas sector can be found in \citep{Oliveira2013AConsiderations,Goel2004AReserves,Gupta2012AnInfrastructure,iyer1998}.

From the literature above, we find that the planning of investment, retrofit, and plug and abandonment are often treated separately. To overcome the limitations of the separated approach for energy systems planning where such decisions are deemed tightly coupled, we propose a modelling framework that integrates investment, plug and abandonment and retrofit. An illustrative example is presented in Figure \ref{fig:retrofit}. The parts marked with red represent the new integrated planning compared with traditional investment planning in the literature. 
\begin{figure}[!htb]
    \centering
    \includegraphics[scale=0.8]{retrofit_v3.pdf}
    \caption{Integrated investment, retrofit and abandonment planning. The grey dotted box includes some technologies that can be invested in. The offshore oil and gas platform can be retrofitted or abandoned. Otherwise, it can keep producing. The Offshore Energy Hubs (OEHs) \citep{zhang2022_OEH} can generate electricity, and produce and store hydrogen.}
    \label{fig:retrofit}
\end{figure}

\subsection{MHSP}
\label{sec:multi_horizon_programming}
\begin{figure}[t]
    \centering
    \includegraphics[scale=0.9]{node_form_time.pdf}
    \caption{Illustration of MHSP with short-term and long-term uncertainty (blue circles: strategic nodes, red squares: operational periods, $i:$ index of the strategic nodes).}
    \label{fig:multi-horizon_tree}
\end{figure}
Investment planning of an energy system often faces uncertainty from two time horizons \citep{Kaut2014,lara2020}: (a) the uncertainty from the operational time horizon, such as the availability of renewable energies. The operational uncertainty becomes even more crucial for a system with higher penetration of intermittent renewable energies, and (b) the uncertainty from the strategic time horizon, e.g., oil and gas price and demand. In traditional multi-stage stochastic programming, uncertainty from operational and strategic time horizons can lead to a large scenario tree, thus, an intractable planning model. The multi-horizon approach was proposed as an alternative formulation that reduces the model size significantly \citep{Kaut2014}. An illustration of MHSP with short-term and long-term uncertainty is shown in Figure \ref{fig:multi-horizon_tree}. One can have a much smaller model by disconnecting operational nodes between successive planning stages and embedding them into their respective strategic nodes. We call an operational problem embedded in a strategic node an operational node. The resulting model is called MHSP. MHSP can be understood as an approximation to multi-stage stochastic programming unless two requirements are met \citep{Kaut2014}: (a) strategic and operational uncertainties are independent, and the strategic decisions must not depend on any particular operational decisions, and (b) the operational decisions in the last operational period in a stage do not affect the system operation in the first operational period in the next stage. MHSP has been applied in several energy system planning problems \citep{Skar2016,Backe2022EMPIRE:Analyses,Zhang2022OffshoreShelf,Durakovic2023PoweringPrices} Furthermore, the bounds in MHSP have been studied in \citep{Maggioni2020BoundsPrograms}.

\subsection{Benders decomposition}
Benders decomposition was proposed to solve problems with complicating variables \citep{Geoffrion1972}. Then generalised Benders for nonconvex problems were proposed \citep{Steeger2017DynamicProblem,Li2019AVariables}. Extensive research was conducted on accelerating Benders decomposition, such as choosing and adding strong cuts \citep{Magnanti1981AcceleratingCriteria, Oliveira2014AcceleratingChain}, developing a multi-cut version of Benders decomposition \citep{You2011MulticutUncertainty}, regularised Benders decomposition \citep{Zverovich2012,zhang2022_benders}, and using inexact oracles \citep{Mazzi2020,zhang2022_benders,vanAckooij2016InexactSupport}. In this paper, we adopt the approach from \citep{zhang2022_benders} and improve the method for solving MILP.

\section{Problem description, modelling strategies, modelling assumptions}
\label{sec:problem_description_modelling_assumptions}
In this section, we present the problem description and modelling strategies, including price modelling, scenario generation, temporal and geographical representations of the problem, and the modelling assumptions. 

The problem under consideration aims to choose (a) the optimal strategy for investment, abandonment and retrofit planning, and (b) operating scheduling for an energy system to achieve emission targets at minimum overall costs under short-term uncertainty, including renewable energy availability, hydropower production profile and load profile, and long-term uncertainty, including oil and gas prices. 

For the investment planning, we consider: (a) thermal generators (Coal-fired plant, OCGT, CCGT, Diesel, nuclear plants, co-firing biomass with 10\% lignite, lignite); (b) generators with Carbon Capture and Storage (CCS) (Coal-fired plant with CCS and advanced CCS, gas-fired plant with CCS and advanced CCS, co-firing biomass with 10\% lignite with CCS, lignite with CCS, lignite with advanced CCS); (c) renewable generators (onshore and offshore wind and solar, wave, biomass, run-of-the-river hydropower, geothermal and regulated hydropower); (d) electric storage (hydro pump storage and lithium); (e) onshore and offshore transmission lines; (f) onshore and offshore clean energy hubs (electrolyser, fuel cell, hydrogen storage); (g) onshore steam reforming plant with CCS (SMRCCS) and (h) offshore and onshore hydrogen pipelines. The capital expenditures and fixed operational costs coefficients are assumed to be known. 

For the retrofit planning, we consider: (a) retrofitting existing natural gas pipelines for hydrogen transport, and (b) retrofitting existing offshore platforms for clean OEHs. Finally, we consider the abandonment of mature fields.  The problem is to determine: (a) the capacities of technologies and retrofit and abandonment decisions, and (b) operational strategies that include scheduling of generators, storage and approximate power flow among regions to meet the energy demand with minimum overall investment and operational and environmental costs.

\subsection{Modelling strategies and assumptions}
In this section, we present the modelling strategies and assumptions we use in the long-term integrated planning problem.

\subsubsection{Price process}
\label{sec:price_process}
We use a two-factor Short-Term Long-Term (STLT) model to capture a realistic price behaviour of oil and gas \citep{Schwartz2000Short-termPrices}. The STLT model has a stochastic equilibrium price and stochastic short-term deviations \citep{Bakker2021MatureProgramming}. In the STLT model, the logarithm of the spot price at time $t$ is, 
\begin{equation}
    \log p_t=\chi_t+\xi_t,
\end{equation}
where $\chi_t$ is the short-term factor in prices and $\xi_t$ is the long-term factor. \citep{Bakker2021MatureProgramming} presents a risk-neutral STLT model in discrete time, which is used in this paper for price modelling. The price is represented by, 
\begin{alignat}{1}
    \tilde{p}_t&=e^{\tilde{\chi}_t+\tilde{\xi}_t},\\
    \tilde{\chi}_t&=\tilde{\chi}_{t-1}-\left(1-e^{-\kappa\Delta t}\right)+\sigma_{\chi}\epsilon_{\chi}\sqrt{\frac{1-e^-2\kappa \Delta t}{2\kappa}},\\
    \tilde{\xi}&=\tilde{\xi}_{t-1}+\mu^{*}_{\xi}\Delta t+\sigma_{\xi}\epsilon_{\xi}\sqrt{\Delta t},
\end{alignat}
where $\tilde{p}_t$, $\tilde{\chi}_{t}$ and $\tilde{\xi}_{t}$ are risk-neutral equivalents to the spot price, short-term factor and long-term factor. The volatilities are represented by $\sigma_{\chi}$ and $\sigma_{\xi}$, while $\epsilon_{\chi}$ and $\epsilon_{\xi}$ are correlated standard normal random variables with correlation coefficient $\rho_{\chi \xi}$. The parameter $\kappa$ is the mean reverse coefficient, $\lambda_\chi$ is a risk premium that specifies a reduction in the drift got the short-term process, and $\mu^{*}_{\xi}$ is the risk-neutral drift of the equilibrium level, $\tilde{\xi}_{t}$. The length of the time period $t$ (in years) is represented by $\Delta_t$.

Based on the historical record, we assume the gas price is correlated with the oil price, although recently, they have been less correlated due to the energy crisis.

\subsubsection{Long-term production profile modelling}
There are three phases during the lifetime of oil fields \citep{store2018}. We adopt the production modelling from \citep{wallace1987}.  The long-term production profile of oil and gas is presented by,
\begin{equation}
    q_t=\begin{cases}
        q^P & t\leq t^P,\\
        q^Pe^{-m(t-t^p)} &  t> t^P,
        \end{cases}
\end{equation}
where $q^P$ is the production rate during plateau period, $t^P$ is the length of the plateau period, $m$ is the decline constant. We also calibrate the model with the average decline rate of the giant oil fields in the world \citep{hook2009}.

\subsubsection{Scenario generation}
For short-term uncertainty, we randomly select time intervals with an hourly resolution of four seasons of a year, and scale them up to represent an operational scenario. 

For long-term uncertainty, we consider oil and gas price uncertainty. We first generate a large number of projections using the price process described in Section \ref{sec:price_process}. Then we obtain the mean values of the prices for each stage and construct the mean scenario. Next, we use a Julia package \texttt{ScenTrees.jl} \citep{Kirui2020ScenTrees.jl:Programming} that utilises the methodology proposed in \citep{Pflug2015DynamicTrees}  to generate a multi-stage scenario tree. Finally, we reduce the scenario tree by removing the decision nodes whose probability is zero.

\subsubsection{Geographical representation of the problem}
The problem potentially consists of many regions and results in a large model. Therefore, we use a k-means cluster approach to group platform locations into representative ones. We use dedicated locations to represent the regions that require higher resolution. The generators and storage units in one aggregated region with the same characteristics are aggregated into clusters. In such a way, for the aggregated regions, the model does not make the investment in a certain unit but in that type of technology.

\subsubsection{Modelling assumptions}
We assume that: (a) the Kirchhoff voltage law is omitted and we use a linear direct current power flow model for the power system part, (b) the switch from natural gas to hydrogen has little impact on the capacity of a pipeline to transport energy \citep{siemens_2021}, (c) linepack in hydrogen pipelines is omitted, (d) investment in new wells is not considered, (e) we simplify the modelling of pressure and temperature of the production processes on platforms and use typical values in the North Sea region, and (f) no more oil and gas profit once a platform is retrofitted, and the associated gas profit is lost once a pipeline is retrofitted. 

\section{Benders decomposition in MHSP}
\label{sec:Benders decomposition in MHSP}
MHSP has a unique structure that allows the application of Benders-type decomposition for solving large-scale stochastic programming problems. In the following, we first present how a MHSP problem can be decomposed by Benders decomposition. We then propose an enhanced Benders decomposition for solving the proposed model efficiently. 
\subsection{Benders decomposition in MHSP}
Here, we give a general formulation for MHSP and show that it has a special structure to be decomposed by Benders decomposition. 

A MHSP model for strategic planning problems can be formulated as follows
\begin{subequations}
\label{eq:MHSP}
\begin{alignat}{3}
    & \min_{x, y}  &&\sum_{i \in \mathcal{I}} \pi_{i}\left(c_{i}^{\top}x_{i} + \sum_{s \in \mathcal{S}_i}\omega_{is} q_{is}^{\top}Q_{is}y_{is}\right)\\
    & \text{s.t.} && T^{I}_{j}x_{j} + W^{I}_i x_i \leq h^{I}_i, & i \in \mathcal{I} \setminus \{1\}, j=I_i,\\
    & \quad && T^{0} x_i \leq h^{0}, & i =1,\\
    & \quad && T^O_{is} x_i + W^O_{is} y_{is} \leq h^{O}_{is}, & i \in \mathcal{I}, s \in \mathcal{S}_i,
\end{alignat}
\end{subequations}
where $x$ are the strategic decision variables, $x=\{x_i| i \in \mathcal{I}\}$, and $x_i=\{x_{ij} \in \{0,1\}, j=1,...,p; x_{ij} \in \mathbb{R}, j=p+1, ..., n_i\}$ and $y$ are the operational variables, $y=\{y_{is}|i \in \mathcal{I}, s \in \mathcal{S}_i\}$,  $y_{is} \in \mathbb{R}^{n_i}$. The $\pi_i$ is the probability of strategic node $i$, sum of $\pi_i$ in each strategic stage is equal to 1, $c_i \in \mathbb{R}^{n_i}$, $h^I_i \in \mathbb{R}^{m_i}$, $W^I_i \in \mathbb{R}^{m_i \times n_i}$, are vectors and matrices at strategic node $i \in \mathcal{I}$, and $T^I_j \in \mathbb{R}^{m_i \times n_j}$ is the matrix for its ancestor node $j=I_i$. The probability of operational scenario $s$ that is embedded in strategic node $i$ is denoted by $\omega_{is}$, and $\sum_{s \in \mathcal{S}_{i}} \omega_{is}=1$. Operational vectors and matrices at operational node $i$, in operational scenario $s$, are given by $T^O_{is} \in \mathbb{R}^{m_{i} \times n_{i}}$,  $W^O_{is} \in \mathbb{R}^{m_{i} \times n_{i}}$, $q_{is} \in \mathbb{R}^{n_{i}}$, $h^{O}_{is} \in \mathbb{R}^{n_{i}}$.

In MHSP, the complicating variables are the strategic decisions, $x_i$, that link all the decision nodes. By fixing the complicating variables $x_i$, we can decompose the full size problem using Benders-type decomposition. For a given node $i$, the subproblem is formulated as
\begin{subequations}
\begin{alignat}{3}
    & \min_{y}  && \pi_{i}\sum_{s \in \mathcal{S}_i}\omega_{is}q_{is}^{\top}Q_{is}y_{is}\\
    & \text{s.t.} &&  T^O_{is} x_i + W^O_{is} y_{is} \leq h^{O}_{is}, \qquad i \in \mathcal{I}, s \in \mathcal{S}_i,
\end{alignat}
\end{subequations}
the subproblems can be solved in parallel. The reduced master problem is 
\begin{subequations}
\begin{alignat}{3}
    & \min_{x}  &&\sum_{i \in \mathcal{I}} \pi_{i}(c_{i}^{\top}x_{i} + \beta_i)\\
    & \text{s.t.} && T^{I}_{j}x_{j} + W^{I}_i x_i \leq h^{I}_i, & \qquad i \in \mathcal{I} \setminus \{1\}, j=I_i,\\
    & \quad && T^{0} x_i \leq h^{0}, & i =1,\\
    & \quad &&\beta_{i} \geq \theta + \lambda^{\top}(x_{i}-x),& \qquad (x,\theta,\lambda) \in \mathcal{F}_{i(j-1)}, i \in \mathcal{I}, \label{eq:benders_cut_general}
\end{alignat}
\end{subequations}
where $\beta_i$ is a variable for the approximated cost of the operational subproblem that is embedded in strategic node $i$. Constraint \eqref{eq:benders_cut_general} are the Benders cuts built up to iteration $j-1$. To simplify the notation, we write the operational subproblem as
\begin{equation}
    g(x_{i},c_{i}):=\min_{y_{i} \in \mathcal{Y}}\{c_{i}^{\top}Cy_{i}| Ay_{i}\leq Bx_{i}\},
    \label{eq:SP}
\end{equation}
where $\mathcal{Y}$ is a convex polyhedron. Also, the Reduced Master Problem (RMP) is,
\begin{subequations}
\label{eq:MP}
\begin{alignat}{3}
    &\min_{\mathbf{x} \in \mathcal{X}, \beta}&& f(\mathbf{x}) + \sum_{i \in \mathcal{I}}\pi_{i}\beta_{i}\\
    &\text{s.t. }&&\beta_{i} \geq \theta + \lambda^{\top}(x_{i}-x),\qquad  (x,\theta,\lambda) \in \mathcal{F}_{i(j-1)}, i \in \mathcal{I},
\end{alignat}
\end{subequations}
where $f(\mathbf{x})=\sum_{i \in \mathcal{I}}\pi_{i}c_{i}x_{i}$, $\mathcal{X}$ is the feasible region of variable $\mathbf{x}$, $\mathcal{F}_{i(j-1)}$ is the set of cuts associated with subproblem $i$ built up to iteration $j-1$. 

\subsection{Enhanced Benders decomposition}
\label{sec:enhanced benders}
In this section, we present the solution method for solving the proposed problem. The algorithm is adapted from \citep{zhang2022_benders} and improved for solving MILP. \cite{zhang2022_benders} utilised adaptive oracles \citep{Mazzi2020} and level method regularisation and achieved a significant reduction in solution time. However, the level method problem is a quadratic program if $L_2$ norm is used \citep{Zverovich2012} and becomes a mixed-integer quadratic program after adding integer variables. In this paper, we use a centre point approach to avoid solving the mixed-integer quadratic programming but still obtain a stabilised solution. Finding a well centred point is a built-in function in solvers like Gurobi \citep{gurobi} with certain solver parameter settings. The solvers utilise methods like analytic centre \citep{Gondzio1996, Goffin1990} to get a well centred points. By doing so, the time spent on regularisation is reduced significantly. 
\begin{figure*}[!htb]
    \centering
    \begin{subfigure}[]{0.475\textwidth}
        \centering
        \includegraphics[scale=1]{step_1}
        \caption{\small standard Benders}
    \end{subfigure}
    \hfill
    \begin{subfigure}[]{0.475\textwidth}
        \centering
        \includegraphics[scale=1]{step_2}
        \caption{\small Benders with centre point}
    \end{subfigure}
    \caption{\small An illustrative example of standard Benders and Benders with centre point from iteration $j$ to iteration $j+1$. (Red dots: upper bounds, blue solid lines: cuts built up to iteration $j-1$, green lines: cuts built at iteration $j$.)}
    \label{fig:algorithm_illustration}
\end{figure*}

The Centre Point problem (CP) is the relaxed RMP problem that excludes the objective function and subjects to, in addition to RMP constraints, one extra level set constraint,
\begin{equation}
    f(\mathbf{x})+\sum_{i \in \mathcal{I}}\pi_{i}\beta_{i}\leq L^{*}_{j}+\gamma(U^{*}_{j-1}-L^{*}_{j}), 
\end{equation}
where $U^{*}_{j-1}$ and $L^{*}_{j}$ are the best upper bound at iteration $j-1$ and lower bound at iteration $j$, respectively. The CP is a feasibility problem because there is no objective function. An illustration of the centre point approach is presented in Figure \ref{fig:algorithm_illustration}.

The method is presented in Algorithm \ref{alg:Level set method stabilised Benders decomposition with adaptive oracles}.  We omit to present the level set management step, and we refer the readers to \citep{zhang2022_benders} for details. 

\subsection{Convergence of Algorithm \ref{alg:Level set method stabilised Benders decomposition with adaptive oracles}}
To prove the convergence of Algorithm \ref{alg:Level set method stabilised Benders decomposition with adaptive oracles}, we first give the following proposition.
\begin{prop}
\label{prpo:valid cut}
The Benders cuts generated by calling oracles at $\mathbf{x}^{CP}$  underestimate the objective function value of the original MILP problem. 
\end{prop}
\begin{proof}
The CP is a linear relaxation of the RMP. The $\mathbf{x}^{CP}$ is a solution of CP. It is trivial that the optimal value of the linear relaxation underestimates the optimal value of the original problem. Therefore, following the proof by \citep{Birge1985DecompositionPrograms} that Benders cuts are valid for linear programming problems, the Benders cuts are also valid for the original MILP problem.
\end{proof}

Following the convergence proof of Algorithm 2 in \citep{Mazzi2020} and \textbf{Proposition} \ref{prpo:valid cut}, Algorithm \ref{alg:Level set method stabilised Benders decomposition with adaptive oracles} terminates in a finite number of iterations with an $\epsilon$-optimal solution to problem \eqref{eq:MHSP}.

\begin{algorithm}[!htb]
    \caption{Regularised Benders decomposition with adaptive oracles}\label{alg:Level set method stabilised Benders decomposition with adaptive oracles}
    \begin{algorithmic}[1]
        \State choose $\epsilon$ (convergence tolerance), $\gamma \in (0,1)$ (stabilisation factor), $\underline{\beta}$ (initial lower bound $\beta_{i}$), $U^{*}_0:=M$ (initial upper bound), and level set management steps related parameters;
        \StatePar{set  $j:=0$, $\mathcal{F}_{i0}:=\{(\beta_{i0},0,0)\}$ for each $i \in \mathcal{I}$;}
        \State solve subproblem at the special point $(\underline{x},\underline{c})$ and obtain $\theta$, $\lambda$ and $\phi$; set $\mathcal{S}:=\{(\underline{x},\underline{c},\theta,\lambda,\phi)\}$;
        \Repeat
            \State set $j:=j+1$;
            \State solve RMP and obtain $\beta_{ij}$ and $\mathbf{x}^{RMP}_{j}$; set $L^{*}_{j}:=f(\mathbf{x}^{RMP}_{j})+\sum_{i \in \mathcal{I}}\pi_{i}\beta_{ij}$;
            \StatePar{set CP target: $L^{*}_{j}+\gamma(U^{*}_{j-1}-L^{*}_{j})$;}
            \StatePar{solve CP and obtain $\mathbf{x}^{CP}_{j}$;}
            \For{$i \in \mathcal{I}$}
                \StatePar{call adaptive oracles at $(x^{CP}_{ij},c_{i})$ and  obtain $\underline{\theta}_{ij}$, $\overline{\theta}_{ij}$, $\overline{\phi}_{ij}$ and $\underline{\lambda}_{ij}$;}
            \EndFor
            \State set $n:=0$;
            \Repeat
                \State $n:=n+1;$\;
                \State set $i:= \argmax_{i \in \mathcal{I}} (\overline{\theta}_{ij} - \underline{\theta}_{ij})$; 
                \State solve subproblem $i$ at $(x^{CP}_{ij},c_{i})$ exactly and obtain $\theta_{ij}$, $\lambda_{ij}$, $\phi_{ij}$;
                \State set $\mathcal{S}:=\mathcal{S}\cup\{(x^{CP}_{ij},c_{i},\theta_{ij},\lambda_{ij},\phi_{ij})\}$;
                \For{$i \in \mathcal{I}$}
                    \StatePar{call adaptive oracles at $(x^{CP}_{ij},c_{i})$ and obtain $\underline{\theta}_{ij}$, $\overline{\theta}_{ij}$, $\overline{\phi}_{ij}$ and $\underline{\lambda}_{ij}$;}
                \EndFor
                \StatePar{set $L^{LBO}_{j}:= f(\mathbf{x}^{CP}_{j}) + \sum_{i \in \mathcal{I}}\pi_{i}\underline{\theta}_{ij}$;}
                \StatePar{set $U^{UBO}_{j}:= f(\mathbf{x}^{CP}_{j}) + \sum_{i \in \mathcal{I}}\pi_{i}\overline{\theta}_{ij}$;}
            \Until{$U^{UBO}_{j}-L^{LBO}_{j} \leq U^{*}_{j-1}-L^{*}_{j-1} $ \normalfont{\textbf{or}} $n>|\mathcal{I}|$ \normalfont{\textbf{or}} $L^{LBO}_{j}\geq U^{*}_{j-1};$}
            \For{$i \in \mathcal{I}$}
                    \StatePar{set $\mathcal{F}_{ij}:=\mathcal{F}_{i(j-1)}\cup \{(x^{CP}_{ij},\underline{\theta}_{ij},\underline{\lambda}_{ij})\}$;}
            \EndFor
            \For{$i \in \mathcal{I}$}
            \StatePar{call adaptive oracles at $(x^{RMP}_{ij}, c_i)$ and obtain $\underline{\theta}_{ij}$, $\overline{\theta}_{ij}$, $\overline{\phi}_{ij}$ and $\underline{\lambda}_{ij}$;}
            \EndFor
            \StatePar{set $U_{j}:=f(\mathbf{x}^{RMP}_{j}) + \sum_{i \in \mathcal{I}}\pi_{i}\overline{\theta}_{ij}$;}
            \StatePar{set $U^{*}_{j}:=\min(U^{*}_{j-1},U_{j})$;}
            \StatePar{level set management steps;}
        \Until{$U^{*}_{j}-L^{*}_{j} \leq \epsilon.$}
    %   \algstore{testCont}
    \end{algorithmic}
\end{algorithm}

\section{The REORIENT model}
\label{sec:model}
This section presents the energy system integrated planning and operational optimisation model. The full model is decomposed by having an investment planning master problem and an operational subproblem. We use the conventions that calligraphic capitalised Roman letters denote sets, upper case Roman and lower case Greek letters denote parameters, and lower case Roman letters denote variables. The indices are subscripts, and name extensions are superscripts. The same lead symbol represents the same type of thing. The names of variables, parameters, sets and indices are single symbols. We give a brief definition of some of the main sets and variables, and their corresponding domains as we explain the equations. For a complete overview of all sets and indices, parameters and variables used in the RIORIENT model, we refer to \ref{sec:nonmenclature}.

\subsection{Investment planning model (RMP in Benders decomposition)}
The investment master problem Equations \eqref{eq:investment_cost}-\eqref{eq:mp_domain} follows the general formulation given by Equations \eqref{eq:MP}. The total cost for investment planning, Equation \eqref{eq:investment_cost}, consists of actual investment costs $c^{INV}$ as well as the expected operational cost of the system over the time horizon $\kappa\sum_{i \in \mathcal{I}^{Ope}}\pi_{i}c^{OPE}_i$ which is total approximated subproblem costs in Benders decomposition. Here, $\kappa$ is a scaling factor that depends on the time step between two successive investment nodes. Equation \eqref{eq:cinv} calculates the investment cost which comprises capacity-dependent and capacity-independent costs, retrofitting costs, abandonment costs, fixed operating and maintenance costs, and profit of existing technology (e.g., oil and gas platform).
\begin{equation}
    \min c^{INV}+\kappa\sum_{i \in \mathcal{I}^{Ope}}\pi_{i}c^{OPE}_i ,\label{eq:investment_cost}
\end{equation}
where
\begin{equation}
\begin{split}
    c^{INV}=\sum_{i\in \mathcal{I}^{Inv}}\pi^{Inv}_{i}\left(\sum_{p \in \mathcal{P}}\left(C^{InvV}_{pi}x^{Inv}_{pi}+C^{InvF}_{pi}y^{Inv}_{pi}\right)+\sum_{p \in \mathcal{P}^{RT}}\left(C^{ReTV}_{pi}x^{ReT}_{pi}+C^{ReTF}_{pi}y^{ReT}_{pi}\right)\right)+\\
    \phantom{ass}\kappa \sum_{i \in \mathcal{I}^{Ope}}\pi^{Ope}_{i}\left(\sum_{p \in \mathcal{P}}C^{Fix}_{pi}x^{Acc}_{pi} + \sum_{p \in \mathcal{P}^{RT}}C^{ReTFixO}_{pi}x^{AccReT}_{pi}+\sum_{p \in \mathcal{P}^{R}}C^{ReFFixO}_{pi}x^{AccReF}_{pi} \right). \label{eq:cinv}
\end{split}
\end{equation}

Constraint \eqref{eq:cap_tech} states that the accumulated capacity of a technology $x^{Acc}_{pi}$ in an operational node equals the sum of the historical capacity $X^{Hist}_{p}$ and newly invested capacities $x^{Inv}_{pi}$ in its ancestor investment nodes $\mathcal{I}^{Inv}_{i}$ that are not retired.
\begin{equation}
    x^{Acc}_{pi}=X^{Hist}_{pi}+\sum_{j \in\mathcal{I}^{Inv}_{i}|\kappa(S^{Ope}_{i}-S^{Inv}_{j})\leq H_{p}}x_{pj}^{Inv}, \qquad p \in \mathcal{P},  i \in \mathcal{I}^{Ope}.\label{eq:cap_tech}
\end{equation}

Constraint \eqref{eq:max_built} ensures the maximum $X^{MaxInv}_{pi}$ capacity that is built in an investment node. The binary variable $y^{Inv}_{pi}$ equals 1 if a technology $p\in \mathcal{P}$ in investment node $i \in \mathcal{I}^{Inv}$ is invested. Parameter $X^{MaxAcc}_{p}$ gives the maximum capacity that can be installed for different technologies.
\begin{equation}
    x^{Inv}_{pi} \leq X^{MaxInv}_{pi}y^{Inv}_{pi}, \qquad p \in \mathcal{P}, i \in \mathcal{I}^{Inv}. \label{eq:max_built}
\end{equation}

Constraints \eqref{eq:max_acc}-\eqref{eq:retrofit_cap} establish that the invested capacity and accumulated capacity of newly invested technologies and retrofitted technologies should be within the capacity limits. 
\begin{equation}
    x^{Acc}_{pi} \leq X^{MaxAcc}_{p},  \qquad  p \in \mathcal{P},  i \in \mathcal{I}^{Ope}, \label{eq:max_acc}
\end{equation}
\begin{equation}
    x^{AccReT}_{pi} \leq X^{MaxAccReT}_{p}, \qquad  p \in \mathcal{P}^{RT},  i \in \mathcal{I}^{Ope}, \label{eq:max_acc_retrofit}
\end{equation}
\begin{equation}
    x^{ReT}_{pj} \leq X^{MaxReT}_{p}y^{ReT}_{pi}, \qquad p \in \mathcal{P}^{RT}, i \in \mathcal{I}^{Inv},j \in \mathcal{I}^{Inv}_{i}. \label{eq:retrofit_cap}
\end{equation}

Constraint \eqref{eq:retrofit_from} dictates that the existing capacity is zero if a technology is retrofitted to a new technology. 
\begin{equation}
    x^{AccReF}_{pj} = X^{HistReF}_{pj}(1-y^{ReF}_{pi}), \qquad p \in \mathcal{P}^{R}, i \in \mathcal{I}^{Inv}, j \in \mathcal{I}^{Ope}_{i}. \label{eq:retrofit_from}
\end{equation}

Constraint \eqref{eq:retrofit_to_limits} states that only one technology can be retrofitted to. 
\begin{equation}
    \sum_{p \in \mathcal{P}^{R}_{p}}y^{ReT}_{pi} \leq y^{ReF}_{pi}, \qquad p \in \mathcal{P}^{R}, i \in \mathcal{I}^{Inv}. \label{eq:retrofit_to_limits}
\end{equation}

Constraint \eqref{eq:one_time_retrofit} ensures that retrofit can only happen once for a technology during the planning horizon. 
\begin{equation}
    \sum_{i \in \mathcal{I}^{Inv}}y^{ReF}_{pi} \leq 1, \qquad p \in \mathcal{P}^{R}. \label{eq:one_time_retrofit}
\end{equation}

Constraint \eqref{eq:retrofit_acc} states that the accumulated capacity of a technology $x^{AccReT}_{pi}$ in an operational node equals the newly invested capacities $x^{ReT}_{pi}$ in its ancestor investment nodes $\mathcal{I}^{Inv}_{i}$ that are not retired.  Parameter $X^{MaxAccReT}_{p}$ is the maximum accumulated capacity of a technology that is retrofitted from another. 
\begin{equation}
    x^{AccReT}_{pi} = \sum_{j \in\mathcal{I}^{Inv}_{i}|\kappa(S^{Ope}_{i}-S^{Inv}_{j})\leq H_{p}}x^{ReT}_{pj}, \qquad  p \in \mathcal{P}^{RT}, i \in \mathcal{I}^{Ope}. \label{eq:retrofit_acc}
\end{equation}

The Benders cuts built up to iteration $k-1$ are given by Equation \eqref{eq:benders_cut}.
\begin{equation}
    c^{OPE}_i \geq \theta + \lambda^{\top}(x_{i}-x), \qquad (x,\theta,\lambda) \in \mathcal{F}_{i(k-1)}, i \in \mathcal{I}. \label{eq:benders_cut}
\end{equation}

The domains of variables are given as follows
\begin{equation}
   x^{Inv}_{pi}, x^{Acc}_{pi}, x^{ReT}_{pi}, x^{AccReT}_{pi}, x^{AccReF}_{pi}, c^{INV} \in \mathbb{R}^{+}_{0}, \qquad y^{Inv}_{pi}, y^{ReF}_{pi}, y^{ReT}_{pi} \in \{0,1\}. \label{eq:mp_domain}
\end{equation}

The vector $x_i=\left( \{x^{Acc}_{pi}, p \in \mathcal{P}\}, \{x^{AccReT}_{pi}, p \in \mathcal{P}^{RT}\}, \{x^{AccReF}_{pi}, p \in \mathcal{P}^{R}\}\right)$ collects all right-hand side coefficients that will be fixed in operational subproblem, Equations \eqref{eq:SP_objective}-\eqref{eq:sp_domain}. The vector $c_i=\left(C^{CO_2}_{i}\right)$ collects all the cost coefficients. The vectors $x_i$ and $c_i$ will be fixed as parameters in the operational problem. 

\subsection{Operational problem (subproblem in Benders decomposition)}
We now present the operational problem and note that we omit index $i$ in the operational model for ease of notation. 

The right hand side parameters $P^{Acc}_p$, $V^{Acc}_c$, $P^{AccG}_g$, $P^{AccHRor}_g$, $P^{AccSE}$, $Q^{AccSE}_s$, $P^{AccL}_l$, $V^{AccLHy}_l$, $\mu^{DP}$, $\mu^{DH}$, and $\mu^E$ are fixed by the solution $x_i$ from solving the master problem Equations \eqref{eq:investment_cost}-\eqref{eq:mp_domain}. The CO$_2$ cost of generators that is included in parameter $C^G_g$ is fixed by $c_i$ from the master problem.

The operational cost $c^{OPE}(x_i,c_i)$ at one operational node $i$ is computed by solving subproblem Equations \eqref{eq:SP_objective}-\eqref{eq:sp_domain} given the decisions $x_i$ and $c_i$ made in Equations \eqref{eq:investment_cost}-\eqref{eq:benders_cut}. The operational subproblem Equations \eqref{eq:SP_objective}-\eqref{eq:sp_domain} correspond to the general formulation Equations \eqref{eq:SP}. The objective function, the operational cost, includes total operating costs of generators $C^{G}_{g}p^{G}_{gt}$, energy load shedding costs for heat, power, and hydrogen $C^{Shed}p_{zt}^{Shed}$ and fuel cost of steam reforming plants $C^{R}v^{R}_{rt}$. $C^{G}_{g}$ includes the variable operational cost, fuel cost and the CO$_2$ tax, $C^{CO_2}$, charged on the emissions of generator $g$. 
\begin{equation}
    \min \sum_{t \in \mathcal{T}}\pi_{t}H_t\left(\sum_{g \in \mathcal{G}}C_g^{G}p_{gt}^{G}+\sum_{r \in \mathcal{R}}C^{R}v_{rt}^{R}+\sum_{z \in \mathcal{Z}}\left(\sum_{l \in \{H,P\}}C^{Shed,l}p_{zt}^{Shed,l}+C^{ShedHy}v_{zt}^{ShedHy}\right)\right). \label{eq:SP_objective}
\end{equation}

Constraints \eqref{eq:all_cap} ensure that the technologies operate within their capacity limits.
\begin{subequations}
\label{eq:all_cap}
    \begin{alignat}{3}
    &\quad&&p_{pt}\leq P_{p}^{Acc}, & p \in \mathcal{P}^{*}, t \in \mathcal{T}, \label{eq:multiple_cap}\\
    &\quad&& v_{vt} \leq V_{v}^{Acc},&   v \in \mathcal{V}^{*}, t \in \mathcal{T}, \label{eq:multiple_gas_cap}\\
    &\quad&& p_{gt}^{G}+p_{gt}^{ResG}\leq P_{g}^{AccG}, & g \in \mathcal{G}, t \in \mathcal{T}, \label{eq:gen_cap}\\
    &\quad&& p_{st}^{SE-} + p_{st}^{ResSE} \leq P_{s}^{AccSE}, & s \in \mathcal{S}^{E}, t \in \mathcal{T}, \label{eq:estore_discharge_cap}\\
    &\quad&& q_{st}^{SE} \leq Q_{s}^{AccSE}, & s \in \mathcal{S}^{E}, t \in \mathcal{T}, \label{eq:estore_energy_cap}\\
    &\quad&&-P_{l}^{AccL} \leq p_{lt}^{L} \leq P_{l}^{AccL}, & l \in \mathcal{L}, t \in \mathcal{T}, \label{line_cap}\\
    &\quad&&-V_{l}^{AccLHy} \leq v_{lt}^{LHy} \leq V_{l}^{AccLHy}, & \qquad l \in \mathcal{L}^{Hy}, t \in \mathcal{T}. \label{eq:pipeline_cap}
    \end{alignat}
\end{subequations}

Constraint \eqref{eq:ramp} captures how fast generators can ramp up or ramp down their power output, respectively. 
\begin{equation}
    -\alpha^{G}_{g}P_{g}^{AccG} \leq p_{gt}^{G}+p_{gt}^{ResG}-p_{g(t-1)}^{G}-p_{g(t-1)}^{ResG} \leq \alpha^{G}_{g}P_{g}^{AccG}, \qquad  g \in \mathcal{G}, n \in \mathcal{N}, t \in \mathcal{T}_{n}. \label{eq:ramp}
\end{equation}

Constraint \eqref{eq:reserve} dictates that the spinning reserve of generator $p^{ResG}_{gt}$, plus the reserve of the electricity storage $p^{ResES}_{st}$ must exceed the minimum reserve requirement, where $\sigma^{Res}$ is a percentage of the power load.
\begin{equation}
    \sum_{g \in \mathcal{G}_{z}} p^{ResG}_{gt}+ \sum_{s \in \mathcal{S}^{E}_{z}}p^{ResSE}_{st} \geq \sigma_{z}^{Res}P^{DP}_{zt}, \qquad z \in \mathcal{Z}, t \in \mathcal{T}. \label{eq:reserve}
\end{equation}

Constraints \eqref{eq:hydro_seasonal} and \eqref{eq:hydro_ror} ensure that run-of-the-river hydropower and regulated hydropower production are within their limits and according to the generation profiles, separately.
\begin{equation}
    \sum_{t \in \mathcal{T}_{n}} p_{gt}^{H} \leq \sum_{t \in \mathcal{T}_{n}} P_{gt}^{HSea}, \qquad g \in \mathcal{G}^{HSea}, n \in \mathcal{N}, \label{eq:hydro_seasonal}
\end{equation}
\begin{equation}
    p_{gt}^{H} \leq P_{gt}^{HRor}P^{AccHRor}_g, \qquad g \in \mathcal{G}^{HRor}, t \in \mathcal{T}. \label{eq:hydro_ror}\\ 
\end{equation}

Constraint \eqref{eq:kcl} ensures that, in one operational period $t$, the sum of total power generation of generators $p^{G}_{gt}$, power discharged from all the electricity storage $p^{SE-}_{st}$, renewable generation $R^{R}_{zt}p^{AccR}_{rt}$, hydro power generation $p^{H}_{gt}$, fuel cell generation $p^{F}_{ft}$, power transmitted to this region, and load shed $p^{ShedP}_{zt}$ equals the sum of  power demand $\mu^{DP}P^{DP}_{zt}$, power consumption of electric boilers $p^{BE}_{bt}$, power consumption of all electrolysers $p^{E}_{et}$, power transmitted to other regions, and power generation shed $p^{GShedP}_{zt}$. The parameter $R^{GR}_{rt}$ is the capacity factor of the renewable unit that is a fraction of the nameplate capacity $P^{AccR}$. The subset of a technology in the region $z$ is represented by $R_{z}:=\{r \in \mathcal{R}: r \text{ is available in region } z\}$, where $\mathcal{R}$ can be replaced by other sets of technologies. \begin{equation}
\begin{split}
\sum_{g \in \mathcal{G}_{z}}p_{gt}^{G}+\sum_{l \in \mathcal{L}^{In}_{z}}\eta^{L}p_{lt}^{L}+\sum_{s \in \mathcal{S}^{E}_{z}}p_{st}^{SE-}+\sum_{r \in \mathcal{G}^{R}_{z}}R^{GR}_{rt}P_{r}^{AccGR}+\sum_{g \in \mathcal{G}^{H}_{z}}p_{gt}^{H}+
\sum_{f \in \mathcal{F}_{z}}p^{F}_{ft}+p_{zt}^{ShedP}= \quad\\
 \mu^{DP} P^{DP}_{zt}+\sum_{b \in \mathcal{B}^{E}_{z}}p^{BE}_{bt}+\sum_{e \in \mathcal{E}_{z}}p^{E}_{et}+
\sum_{l \in \mathcal{L}^{Out}_{z}}\eta^{L}_{l}p_{lt}^{L}+\sum_{s \in \mathcal{S}^{E}_{z}}p_{st}^{SE+}+p_{zt}^{GShedP}, \qquad z \in \mathcal{Z}, t \in \mathcal{T}. \label{eq:kcl}
\end{split}
\end{equation}

The hydrogen mass balance Constraint \eqref{hydrogen_balance} dictates that hydrogen produced by electrolyser $H_t\rho^{E}p^{E}_{et}$ and steam reforming plant $v^{R}_{rt}$, hydrogen transmitted to this region, withdraw from a hydrogen storage $v^{SHy-}_{st}$ and hydrogen production shed $v^{GShedHy}_{zt}$ equals the hydrogen demand $V^{DHy}_{zt}$, fuel supply to fuel cell $H_t\rho^{F}p^{F}_{ft}$, hydrogen injected into the storage $v^{SHy+}$, hydrogen transmitted from this region plus the hydrogen load shed $v^{ShedHy}_{zt}$.
\begin{equation}
    \begin{split}
    \sum_{s \in \mathcal{S}^{Hy}_{z}}v^{SHy+}_{st}+\sum_{l \in \mathcal{L}^{HyOut}_{z}}v^{LHy}_{lt}+\sum_{f \in \mathcal{F}_{z}}H_{t}\rho^{F}p^{F}_{ft}+v^{ShedHy}_{zt}+\mu^{DHy}V^{DHy}_{zt}= \phantom{abcdedmsssss}\\
    \sum_{l \in \mathcal{L}^{HyIn}_{z}}v^{LHy}_{lt}+\sum_{e \in \mathcal{E}_{z}}H_t\rho^{E}p^{E}_{et}+\sum_{r \in \mathcal{R}_{z}}v^{R}_{rt}+\sum_{s \in \mathcal{S}^{Hy}_{z}}v^{SHy-}+v^{GShedHy}_{zt}, \qquad z \in \mathcal{Z}, t \in \mathcal{T}.\label{hydrogen_balance}
    \end{split}
\end{equation}

The heat energy balance Constraint \eqref{heat_balance} states that the heat recovery of gas turbines $\eta^{HrG}_{g}p^{G}_{gt}$, plus electric boiler heat generation $\eta^{BE}_{b}p^{BE}_{bt}$, plus heat load shed $p^{ShedH}_{zt}$ equals the heat demand $\mu^{DH}P^{DH}_{zt}$ plus the heat generation shed $p^{GShedH}_{zt}$.
\begin{equation}
    \begin{split}
    \sum_{g \in \mathcal{G}}\eta^{HrG}_gp^{G}_{gt} +\sum_{b\in \mathcal{B}^{E}_{z}}\eta^{BE}_{b}p^{BE}_{bt}+p^{ShedH}_{zt}=\mu^{DH}P^{DH}_{zt}+p^{GShedH}_{zt}, \qquad z \in \mathcal{Z}^{P}, t \in \mathcal{T}. \label{heat_balance}
    \end{split}
\end{equation}

Constraint \eqref{eq:storage_balance} states that the state of charge $q^{SE}_{st}$ in period $t+1$ depends on the previous state of charge $q^{SE}_{st}$, the charged power $\mu^{SE}_sp^{SE+}_{st}$ and discharged power $p^{SE-}_{st}$. The parameter $\eta^{SE}_{s}$ represent the charging efficiency. 
\begin{equation}
    q_{s(t+1)}^{SE}=q_{st}^{SE}+H_{t}(\eta_{s}^{SE}p_{st}^{SE+}-p_{st}^{SE-}), \qquad  s \in \mathcal{S}^{E}, n \in \mathcal{N}, t \in \mathcal{T}_{n}. \label{eq:storage_balance}
\end{equation}

The hydrogen storage balance Constraint \eqref{eq:hydrogen_storage_balance} shows that the hydrogen storage level $v^{SHy}_{st}$ at period $t+1$ equals to storage level at the previous period, plus the hydrogen injected $v^{SHy+}_{st}$, minus the hydrogen withdrawn $v^{SHy-}_{st}$.
\begin{equation}
    v^{SHy}_{s(t+1)}=v^{SHy}_{st}+v^{SHy+}_{st}-v^{SHy-}_{st}, \qquad  s \in \mathcal{S}^{Hy}, n \in \mathcal{N}, t \in \mathcal{T}_{n}. \label{eq:hydrogen_storage_balance}\\
\end{equation}

Constraint \eqref{eq:co2budget} restricts the total emission. The parameter $\mu^{E}$ is the CO$_2$ budget.
\begin{equation}
    \sum_{t \in \mathcal{T}}\pi_{t}\left(\sum_{g \in \mathcal{G}} E^{G}_{g}p^{G}_{gt}+\sum_{r \in \mathcal{R}}E^{R}v^{R}_{rt}\right)\leq \mu^{E}. \label{eq:co2budget}
\end{equation}

The domains of variables are given as follows
\begin{equation}
\begin{split}
    p^{L}_{lt}, v^{LHy}_{lt} \in \mathbb{R},\quad p^{G}_{gt}, p^{ShedP}, p^{ShedH}, v^{ShedHy}_{zt}, p^{Acc}_{p}, p^{BE}_{bt}, p^{ResG}, p^{ResSE}\in \mathbb{R}^{+}_{0}, \\
    p^{AccG}_{g}, p^{GShedP}_{zt}, p^{GShedH}_{zt}, v^{GShedHy}_{zt}, v^{SHy+}_{st}, v^{SHy-}_{st}, v^{SHy}_{st}, p^{E}_{et}, p^{H}_{gt}, p^{AccGR}_{r} \in \mathbb{R}^{+}_{0},\\
    p^{SE+}_{st}, v^{AccLHy}_{l}, p^{SE-}_{st}, q^{AccSE}_{s}, q^{SE}_{st}, p^{AccL}_{l}, p^{AccR}_{r}, p^{F}_{ft}, v^{R}_{rt} \in \mathbb{R}^{+}_{0}.\label{eq:sp_domain}
\end{split}
\end{equation}

\section{Results}
\label{sec:results}
In this section, we first present the case study. Then we report the computational performance of the enhanced Benders decomposition, followed by the sensitivity analysis of the retrofitting cost of natural gas pipelines and offshore platforms. Finally, we compare the solutions and costs between the REORIENT and investment-only models. The investment-only model is the REORIENT model without the retrofit and abandonment planning functions. 

\subsection{Case study}
We demonstrate the REORIENT model on the integrated strategic planning of the European energy system. The network topology is shown in Figure \ref{fig:ns_grid}. 
\begin{figure}[!thb]
    \centering
    \includegraphics[scale=0.9]{initial_grid.pdf}
    \caption{Illustration of the considered European energy system. The considered system includes 27 regions (each region can deploy 36 technologies), 87 transmission lines, 7 existing natural gas pipelines that can be retrofitted for hydrogen transport (some are overlapped), and 87 candidate new hydrogen pipelines.}
    \label{fig:ns_grid}
\end{figure}
We make investment planning towards 2050 with a 5-year planning step. We implemented the algorithm and model in Julia 1.8.2 using JuMP \citep{jump} and solved with Gurobi 10.0 \citep{gurobi}. The problem instances contain up to 13 million continuous variables, 1860 binary variables, and 30 million constraints. We run the code on nodes of a computer cluster with a 2x 3.6GHz 8 core Intel Xeon Gold 6244 CPU and 384 GB of RAM, running on CentOS Linux 7.9.2009. 
% The Julia code and data for the case study have been made publicly available at \citep{reorient2022}. 
The parameters for the price process for oil and gas prices are presented in Table \ref{table:price_process_parameters}. 

\begin{table}[!htb]
\centering
\caption{ Estimated parameters for the price process, taken from \citep{Bakker2021MatureProgramming}.}
\label{table:price_process_parameters}
    \resizebox{8cm}{!}{%
    \begin{tabular}{ccccccc}
        \toprule
        & $\kappa$ & $\sigma_{\chi}$ & $\lambda_{\chi}$ & $\sigma_{\xi}$ &$\mu^{*}_{\xi}$ &$\rho_{\chi \xi}$\\ \hline
        Estimate    &0.407  &0.273  &-0.147 &0.149  &-0.007 &0.306\\
    \bottomrule
    \end{tabular}
    }
\end{table}

\begin{table}[!htb]
\centering
    \caption{Existing natural gas pipelines considered in the case study and their potential hydrogen transport capacity.}
    \label{table:Existing natural gas pipelines considered the case study.}
    \resizebox{10cm}{!}{%
        \begin{tabular}
            {llllS[table-format=2.0]}
            \toprule
            Model name  &   Name    & From      & To    & {Capacity (Ktonne/hour)}\\\hline
            Pipeline 1  &   Vesterled	&	NOO3	&	UK1	&	0.46\\
            Pipeline 2  &   Langeled	&	NOO4	&	UK2	&	0.98\\
            Pipeline 3  &   Zeepipe 1	&	NOO4	&	WE	&	0.58\\
            Pipeline 4  &   Franpipe	&	NOO4	&	WE	&	0.75\\
            Pipeline 5  &   Norpipe	    &	NOO5	&	EE	&	0.61\\
            Pipeline 6  &   Europipe 1	&	NOO4	&	EE	&	0.69\\
            Pipeline 7  &   Europipe 2	&	NOO4	&	EE	&	0.92\\
            \bottomrule
        \end{tabular}
    }
\end{table}

We use Gurobi as the base solver. We use the dual simplex algorithm to solve the RMP due to its relatively small size. The parameter \texttt{DegenMoves} has been turned on because we notice degeneracy makes the solver slow. We use the Barrier algorithm to solve the centre point problem to obtain a centre point. If \texttt{Presolve} is off and \texttt{Crossover} is off, then Gurobi will give a centre point. However, we turn on \texttt{Presolve} to reduce the problem size further. In addition, considering the scale of the problem, we choose to solve all the following instances to 1\% convergence tolerance.

\subsubsection{Computational results}
This section presents an overview of the problem instances and a performance analysis of the proposed algorithm. An overview of the test instances is presented in Table \ref{table:Overview of the cases used in the computational study.}. The computational time is given in Tables \ref{table:Computational time of level set stabilised Benders} and \ref{table:Computational time of centre point stabilised Benders}, and note that Gurobi cannot solve Cases 2 and 3.

\begin{table}[!htb]
    \caption{Overview of the cases used in the computational study.}
    \label{table:Overview of the cases used in the computational study.}
    \resizebox{\columnwidth}{!}{%
        \begin{tabular}
            {cS[table-format=4.0]S[table-format=1.0]S[table-format=2.0]|S[table-format=3.0]S[table-format=2.0]|S[table-format=1.1e1]S[table-format=4.0]S[table-format=1.1e1]S[table-format=1.1e1]}
            \toprule
            & {Operational periods} & \multicolumn{1}{c}{Short-term} & \multicolumn{1}{c}{Long-term} & \multicolumn{2}{c}{Number of decision nodes} & \multicolumn{3}{c}{Problem size (undecomposed)}\\
            & {per short-term scenario} & \multicolumn{1}{c}{scenarios} & \multicolumn{1}{c}{scenarios} & \multicolumn{1}{c}{Operational nodes} & \multicolumn{1}{c}{Investment nodes} & \multicolumn{1}{c}{Continuous variables} & \multicolumn{1}{c}{Binary variables} & \multicolumn{1}{c}{Constraints} \\ \hline
            Case	1	&	96	&	4	&	1	&	6	&	6	&	7.0e5	&	180	&	1.6e6	\\
            Case	2	&	672	&	4	&	1	&	6	&	6	&	4.8e6	&	180	&	1.1e7	\\
            Case	3	&	96	&	4	&	53	&	114	&	62	&	1.3e7	&	1860	&	3.0e7	\\
            \bottomrule
        \end{tabular}
    }
\end{table}

\begin{table}[!htb]
    \caption{Computational time of level set stabilised Benders. (Iters: iterations, Evals: subproblem evaluations)}
    \label{table:Computational time of level set stabilised Benders}
    \centering
    \resizebox{\columnwidth}{!}{%
    \begin{tabular}{lrS[table-format=2.2]S[table-format=2.2]S[table-format=2.2]S[table-format=2.2]}
        \toprule
        & {Iters/Evals} &{Total time spent (h)} & {Master problem (\%)} & {Stabilisation problem (\%)} & {Subproblems and adaptive oracles (\%)}\\ \hline
        Case 1	&	714/3615	&	5.53	&	16.34	&	43.06	&	40.58	\\
        Case 2	&	894/4249	&	31.04	&	4.02	&	12.76	&	83.21	\\
        \bottomrule
        \multicolumn{6}{r}{This method did not solve Case 3 after 10 days of running, and it reached 34.6\% convergence tolerance before termination.}
    \end{tabular}
    }
\end{table}

\begin{table}[!htb]
    \caption{Computational time of centre point stabilised Benders. (Iters: iterations, Evals: subproblem evaluations)}
    \label{table:Computational time of centre point stabilised Benders}
    \centering
    \resizebox{\columnwidth}{!}{%
    \begin{tabular}{lrS[table-format=2.2e1]S[table-format=2.2e1]S[table-format=2.2e1]S[table-format=2.2e1]}
        \toprule
        & {Iters/Evals} & {Total time spent (h)} & {Master problem (\%)} & {Stabilisation problem (\%)} & {Subproblems and adaptive oracles (\%)}\\ \hline
        Case 1	&	152/641	&	0.37	&	8.42	&	3.35	&	88.08	\\
        Case 2	&	119/516	&	4.54	&	0.44	&	0.20	&	99.35	\\
        Case 3	&	161/6502	&	49.61	&	9.26	&	8.88	&	81.85	\\
        \bottomrule
    \end{tabular}
    }
\end{table}
By comparing Tables \ref{table:Computational time of level set stabilised Benders} and \ref{table:Computational time of centre point stabilised Benders}, we can see that by utilising the centre point, we reduce the computational time significantly. By comparing the percentage of the time spent on solving the stabilisation problem, we can see that solving CP takes much less of the total time than solving a quadratic programming stabilisation problem. We can also see that as we increase the number of strategic nodes, the percentage of time spent on the RMP and CP increases in both algorithms. This is because we add one cut per node per iteration, and as we have more nodes, the RMP and CP grow faster every iteration. Also, by comparing Cases 2 and 3, we observe no significant difference in the number of iterations, but the increase in time is more significant. This is because Cases 2 and 3 have the same amount of strategic decisions, but the subproblem in Case 3 is larger. 

\subsubsection{Sensitivity analysis}
\label{sec:sensitivity_analysis}
In this section, we use Case 3 to conduct a sensitivity analysis on the fixed retrofitting cost of pipelines and platforms. In addition, we also present the results of the investment decisions for a future energy system with a large amount of green and blue hydrogen production and transportation. 

We first conduct a sensitivity analysis on the retrofitting cost of natural gas pipelines. To this end, we consider two cases: Case A,  oil and gas production has stopped, and there is no natural gas transportation value in the pipelines, and Case B, oil and gas production is ongoing, and the natural gas pipelines are used for natural gas transport. Using Case A, our motivation is to understand under what cost range it would be more beneficial to retrofit natural gas pipelines that are not in operation compared with building new hydrogen pipelines. Case B is a more realistic case because most of the pipelines in the North Sea are in operation and have an export role. Using Case B, we want to analyse if retrofit to hydrogen will occur if that means the loss of oil and gas export profit. According to \cite{siemens_2021}, the cost of retrofitting the pipelines can be estimated at around 10-15\% of the new construction. Here, we conduct sensitivity on the cost from 5\%-30\% with a 5\% step. 

From Table \ref{table:Results of the expected retrofitting decisions in Case A.}, for pipelines 2-7, if the retrofitting cost is below 15\% of newly built cost, they will be retrofitted in all scenarios. However, for pipeline 1, when the retrofitting cost is less than 15\% of building a new one, it is retrofitted in 28 scenarios, and the retrofitting takes place in the third strategic stage. Pipeline 5 is only retrofitted at the end of the planning horizon when the cost is 15\% of building a new one. When the retrofitting cost is higher than 20\%, some of the pipelines are not retrofitted. Instead, the model decides to build new pipelines and a different network topology to achieve the minimum cost. We can see that different oil and gas price scenarios affect the retrofitting decisions. 
\begin{table}[!htb]
\centering
    \caption{Results of the expected retrofitting decisions in Case A.}
    \label{table:Results of the expected retrofitting decisions in Case A.}
    \resizebox{16cm}{!}{%
        \begin{tabular}
            {lrrrrrrr}
            \toprule
            Cost (\% of new one)	&	{Pipeline 1}	&	{Pipeline 2}	&	{Pipeline 3}	&	{Pipeline 4}	&	{Pipeline 5}	&	{Pipeline 6}	&	{Pipeline 7}	\\ \hline
            5\%	&	(0, 0, 2, 4, 8, 16, 28)	&	$*$	&	$*$	&	$*$	&	$*$	&	$*$	&	$*$	\\
            10\%	&	(0, 0, 2, 4, 8, 16, 28)	&	$*$	&	$*$	&	$*$	&	$*$	&	$*$	&	$*$	\\
            15\%	&	(0, 0, 2, 4, 8, 16, 28)	&	$*$	&	$*$	&	$*$	&	(0, 0, 0, 0, 0, 2, 3)	&	$*$	&	$*$	\\
            20\%	&	$-$	&	$*$	&	$*$	&	$-$	&	$-$	&	$*$	&	$*$	\\
            25\%	&	$-$	&	$-$	&	$*$	&	$*$	&	(0, 0, 0, 2, 4, 7, 11)	&	$*$	&	$*$	\\
            30\%	&	$-$	&	$-$	&	$*$	&	$-$	&	(0, 0, 0, 0, 2, 4, 7)	&	$*$	&	$*$	\\
            \bottomrule
            \multicolumn{8}{r}{$*$: the pipeline is retrofitted in all strategic nodes, $-$: the pipeline is not retrofitted.}\\
            \multicolumn{8}{r}{$(\{x_i, i=1,..,7\}):$ the number of decision nodes, $x_i$, that retrofitting of the pipeline takes place in stage $i$.}
        \end{tabular}
    }
\end{table} 

\begin{table}[!htb]
\centering
    \caption{Results of the expected retrofitting decisions in Case B.}
    \label{table:Results of the expected retrofitting decisions in Case B.}
    \resizebox{\columnwidth}{!}{%
        \begin{tabular}
            {lrrrrrrr}
            \toprule
            Cost (\% of new one)	&	{Pipeline 1}	&	{Pipeline 2}	&	{Pipeline 3}	&	{Pipeline 4}	&	{Pipeline 5}	&	{Pipeline 6}	&	{Pipeline 7}	\\ \hline
            5\%	&	$-$	&	(0, 0, 0, 0, 2, 4, 6)	&	(0, 0, 0, 2, 4, 8, 14)	&	(0, 0, 0, 0, 2, 4, 7)	&	(0, 0, 0, 0, 2, 4, 8)	&	(0, 0, 0, 0, 2, 4, 7)	&	(0, 0, 0, 0, 2, 4, 7)	\\
            10\%	&	$-$	&	$-$	&	(0, 0, 0, 2, 4, 8, 14)	&	(0, 0, 2, 4, 8, 16, 28)	&	(0, 0, 0, 0, 0, 2, 4) 	&	(0, 0, 0, 2, 4, 8, 14)	&	(0, 0, 0, 2, 4, 8, 14)	\\
            15\%	&	$-$	&	$-$	&	(0, 0, 0, 2, 4, 8, 14)	&	(0, 0, 0, 2, 4, 8, 14)	&	(0, 0, 0, 0, 2, 4, 6)	&	(0, 0, 0, 2, 4, 8, 14)	&	(0, 0, 0, 0, 2, 4, 7)	\\
            20\%	&	$-$	&	$-$	&	(0 ,0, 0, 0, 2, 4, 8)	&	(0, 0, 0, 0, 2, 4, 6)	&	(0, 0, 0, 0, 2, 4, 6)	&	(0, 0, 0, 0, 2, 4, 8)	&	(0, 0, 0, 0, 2, 4, 8)	\\
            25\%	&	$-$	&	$-$	&	(0, 0, 0, 2, 4, 8, 14)	&	(0, 0, 0, 0, 2, 4, 6)	&	(0, 0, 0, 0, 2, 4, 6)	&	(0, 0, 0, 2, 4, 8, 14) 	&	(0, 0, 0, 2, 4, 8, 14)	\\
            30\%	&	$-$	&	$-$	&	$-$	&	$-$	&	$-$	&	(0, 0, 0, 0, 2, 4, 8)	&	(0, 0, 0, 0, 2, 4, 6)	\\
            \bottomrule
\multicolumn{8}{r}{$-$: the pipeline is not retrofitted.}\\
\multicolumn{8}{r}{$(\{x_i, i=1,..,7\}):$ the number of decision nodes, $x_i$, that retrofitting of the pipeline takes place in stage $i$.}
        \end{tabular}
    }
\end{table} 

From Table \ref{table:Results of the expected retrofitting decisions in Case B.}, it can be observed, compared with Case A, that the economic viability of pipeline retrofit is harder if the pipelines are already used for natural gas transport. However, most pipelines are still retrofitted for hydrogen transportation in later investment stages. From Tables \ref{table:Results of the expected retrofitting decisions in Case A.} and \ref{table:Results of the expected retrofitting decisions in Case B.}, we can see that retrofit decisions are sensitive to the retrofitting cost, and oil and gas prices. Also, retrofit sometimes only take place in specific price scenarios.  

Secondly, we conduct a sensitivity analysis on the retrofitting cost of oil and gas platforms. By doing so, we aim to analyse: (1) if retrofitting can help delay or even avoid the costly abandonment campaign and (2) understand the relation between retrofitting existing platforms for OEHs and building new OEHs. We assume that the fixed part of the retrofitting cost is half of the removal cost, and conduct sensitivity around this cost. For each platform cluster, we consider a fixed part of the retrofitting cost ranging from €10 million to €2 billion. However, the results suggest that it is not economical to retrofit platforms for hydrogen production under this price range due to the massive loss of oil and gas export profit. The model decides to conduct an abandonment campaign for all platform clusters by the end of the planning horizon. This means that based on the cost models that are used, retrofitting platforms for hydrogen production is more costly than abandonment. Also, due to the oil and gas export profit, the platforms will produce as long as possible until they must be abandoned. In this case study, the platforms must be retrofitted or abandoned by 2050. This suggests that repurposing platforms for other use may need stronger incentives in addition to economic factors. 
\begin{figure}[!thb]
    \centering
    \includegraphics[scale=0.8]{grid_io.pdf}
    \caption{Expected solution of the grid design towards 2050 (investment only model).}
    \label{fig:grid_investment_only}
\end{figure}
\subsubsection{Comparison between the REORIENT model and an investment planning only model}
In this section, we use Case 3 and analyse the difference between an investment-planning-only model and the proposed integrated model regarding investment decisions and costs. We fix the retrofitting cost of pipelines to 15\% of the cost of its newly built counterpart. In the following, we report the results of expected strategic decisions regarding the grid design and capacities of the technologies in each decision stage. 

\begin{figure}[!thb]
    \centering
    \includegraphics[scale=0.8]{grid_reorient.pdf}
    \caption{Expected solution of the grid design towards 2050 (REORIENT model).}
    \label{fig:grid_REORIENT}
\end{figure}

\begin{figure}[!thb]
    \centering
    \includegraphics[scale=0.9]{tech_investment_only.pdf}
    \caption{Expected investment decisions towards 2050 (investment only model).}
    \label{fig:tech_investment_only}
\end{figure}
\begin{figure}[!thb]
    \centering
    \includegraphics[scale=0.9]{investment_reorient.pdf}
    \caption{Expected investment decisions towards 2050 (REORIENT model).}
    \label{fig:tech_REORIENT}
\end{figure}

From Figures \ref{fig:grid_investment_only} and \ref{fig:grid_REORIENT}, we can see that the network topology is noticeably different. By 2050, there will be 32 pipelines built compared with 28 pipelines in the investment-only model. The line connecting NE and NO1 has less capacity in the REORIENT model compared with the investment-only model. In both cases, the UK onshore power system transmission is reinforced, however, a 3 GW difference in the line connecting UK3 and UK5 is observed, followed by a 2 GW difference in UK4-UK5. By 2050, the line NEO-EEO will have 44.90 GW capacity in REORIENT model compared with 36.81 GW in its counterpart. NOO3-NEO presents a significant difference as well with 4.4 GW capacity in REORIENT model and 12.26 GW in the investment-only model by 2050. NOO2 and NOO3 are not connected in the REORIENT model but are connected in the other model. 

By comparing Figures \ref{fig:tech_investment_only} and \ref{fig:tech_REORIENT}, we notice that in both models,  NOO3 is an important offshore region which receives significant investment in offshore wind and electrolysers due to its location and high wind availability. A major difference is found in offshore wind capacity in NEO and UKO4. The REORIENT model has a higher investment in offshore wind in NEO in all investment steps. 
\begin{table}[!htb]
    \centering
    \caption{Results of accumulated capacity in Europe (investment-only model).}
    \label{table:Results of accumulated capacity in Europe (investment-only model).}
    \resizebox{\columnwidth}{!}{%
    \begin{tabular}{lS[table-format=3.2]S[table-format=3.2]S[table-format=3.2]S[table-format=3.2]S[table-format=3.2]S[table-format=4.2]S[table-format=2.2]}
    \toprule
     {Year} & {Offshore wind}	&	{Onshore wind}	&	{Onshore solar} &		{Electrolyser offshore} &		{Electrolyser onshore}	&	{Transmission line}	&	{Hydrogen pipeline}\\
    & {(GW)}	&	{(GW)}	&	{(GW)} &		{(GW)} &		{(GW)}	&	{(GW)}	&	{(ktonne/hour)}\\ \hline
     2025	&	35.72	&	279.24	&	119.83	&	0.00	&	0.00	&	148.66	&	0.00	\\
    2030	&	36.26	&	354.17	&	119.83	&	0.00	&	0.00	&	190.46	&	0.00	\\
    2035	&	212.08	&	364.02	&	146.29	&	12.46	&	8.66	&	586.61	&	0.73	\\
    2040	&	397.01	&	373.98	&	146.29	&	161.64	&	14.74	&	909.83	&	6.98	\\
    2045	&	465.95	&	355.11	&	150.86	&	174.57	&	20.51	&	985.28	&	7.80	\\
    2050	&	608.14	&	269.46	&	171.42	&	248.01	&	27.19	&	1084.87	&	15.70	\\
    \bottomrule
    \end{tabular}
    }
\end{table}

\begin{table}[!htb]
    \centering
    \caption{Results of accumulated capacity in Europe (REORIENT model).}
    \label{table:Results of accumulated capacity in Europe (REORIENT model).}
    \resizebox{\columnwidth}{!}{%
    \begin{tabular}{lS[table-format=3.2]S[table-format=3.2]S[table-format=3.2]S[table-format=3.2]S[table-format=3.2]S[table-format=4.2]S[table-format=2.2]}
    \toprule
     {Year} & {Offshore wind}	&	{Onshore wind}	&	{Onshore solar} &		{Electrolyser offshore} &		{Electrolyser onshore}	&	{Transmission line}	&	{Hydrogen pipeline}\\
    & {(GW)}	&	{(GW)}	&	{(GW)} &		{(GW)} &		{(GW)}	&	{(GW)}	&	{(ktonne/hour)}\\ \hline
    2025	&	35.92	&	277.39	&	119.83	&	0.00	&	0.00	&	148.84	&	2.94	\\
    2030	&	38.33	&	348.67	&	119.83	&	0.00	&	0.00	&	190.73	&	3.19	\\
    2035	&	215.35	&	366.35	&	146.29	&	20.25	&	9.35	&	536.56	&	3.59	\\
    2040	&	394.34	&	371.60	&	146.29	&	156.90	&	20.38	&	904.63	&	6.19	\\
    2045	&	461.80	&	357.82	&	151.91	&	172.95	&	21.44	&	998.00	&	6.93	\\
    2050	&	593.89	&	293.15	&	171.42	&	253.37	&	24.85	&	1061.02	&	13.63	\\
    \bottomrule
    \end{tabular}
    }
\end{table}

Tables \ref{table:Results of accumulated capacity in Europe (investment-only model).} and \ref{table:Results of accumulated capacity in Europe (REORIENT model).} present the accumulated capacity of each technology in each region offshore wind will surpass onshore wind and become the most important renewable power supply by 2050. In the investment-only model, more than 23 GW more transmission line capacity is needed by 2050 compared with the results using the REORIENT model. In both models, hydrogen is produced mainly from SMRCCS at the initial stages but gradually replaced by electrolysers. Also, both models decide to produce green hydrogen mainly offshore. The hydrogen pipeline capacity is lower in the REORIENT model compared with the counterpart by the end of the planning stage. 

In addition, the total cost over the given planning horizon is €1694.47 billion in the investment-only model and €1691.50 billion in the REORIENT model. Furthermore, the REORIENT yields 24\% lower investment cost in the North Sea region compared with the traditional investment-only model. This shows the potential value of doing integrated planning. The value of REORIENT model can be further revealed once more retrofitting options are included, e.g., by including more existing natural gas pipelines. 

\section{Discussion}
\label{sec:discussion}
In this paper, we integrated investment, retrofit and abandonment planning in a multi-horizon stochastic MILP model. The model is generally applicable to studying a specific planning problem for a production plant or a large-scale energy system planning problem for a region.

We used the model to study the investment planning of a European energy system. We considered regional retrofit at a high level and conduct a techno-economical analysis. Unlike traditional retrofit models for process systems, we have omitted detailed modelling of the processes, which is a compromise due to the large scale of the study. The sensitivity analysis presented in this paper can be used as a benchmark for future studies. 

In the case study, we find that although reducing retrofitting costs can trigger the retrofitting of some oil and gas infrastructures, it may not be a sufficiently strong incentive for platform retrofitting compared with pipeline retrofitting. This is because the loss of oil and gas profit is much larger than the reduction in retrofitting cost. Additionally, for platforms, a lot of investment needs to be made for producing green hydrogen upon removing the existing structure. Other driving factors, such as policies, are therefore needed to encourage oil and gas operators to retrofit their infrastructure or reduce production for the energy transition. 

From an algorithm perspective, the proposed algorithm solved the problem instances efficiently. The problem instances have many regions and technologies and, therefore, are highly degenerate. The CP helps the proposed enhanced Benders algorithm to converge faster. In addition, the proposed enhanced Benders decomposition can be applied to a class of problems that can be formulated as Equations \eqref{eq:MP} and \eqref{eq:SP}. Other strategies to accelerate Benders decomposition, including adding combinatorial cuts, trust region, local branching methods, and partial surrogate are tested. However, the improvement in performance is not significant. 

There are some limitations of the case study: (1) in the case study, the offshore fields are aggregated into representative fields, which loses the modelling of the retrofit and abandonment for specific fields; (2) there are other parameters that may affect the investment decisions such as CO$_2$ budget can be an uncertain parameter and change the results. We believe that oil and gas prices have a more direct relation and economic trade-off with retrofit and abandonment decisions; therefore, we choose to consider oil and gas prices as the uncertain parameter, and (3) we only consider green and blue hydrogen. However, hydrogen produced by other means may also be relevant and affect the results. 

\section{Conclusions and future work}
\label{sec:conclusions}
This paper has presented the REORIENT model, a multi-horizon stochastic MILP for integrated investment, retrofit and abandonment energy-system planning. The major novelties and contributions are: (1) we developed an MHSP model for integrated investment, retrofit and abandonment planning of energy systems, (2) we included uncertainty from both strategic and operational time horizons in such a model, (3) an enhanced Benders algorithm was developed to solve large-scale MILP faster and (4) the triggering parameters for retrofitting is investigated by conducting sensitivity analysis and a comparison between the REORIENT model and investment planning only model is made. Results from our case study indicate that: (1) for pipelines that are not in use, when the retrofitting cost is below 20\% of the cost of building new ones, it is more economical to retrofit most of the pipelines than building new ones. For pipelines that transport natural gas, it is economical to be retrofitted in some natural gas price scenarios, (2) platform clusters keep producing oil and gas rather than being retrofitted for hydrogen use, and the clusters abandonment takes place at the last investment stage, (3) compared with an investment planning model, the REORIENT model yields €3 billion lower total cost, and 24\% lower investment cost in the North Sea region, and (4) the proposed Benders algorithm can solve the model efficiently and is 6.8 times faster than the reference algorithm which cannot solve the largest instance.

In the future, the REORIENT model can be used for more energy systems analysis, such as investigating the integrated planning for other regions, such as the continental shelf of the United States, or focusing on some specific platforms in a smaller region. In addition, other solution methods, such as Lagrangean type algorithms and progressive hedging algorithms, can be further developed for solving large-scale MHSP more efficiently. 

\section*{CRediT author statement}
\textbf{Hongyu Zhang:} Conceptualisation, Methodology, Software, Validation, Formal analysis, Investigation, Visualisation, Data curation, Writing - original draft, Writing - review \& editing. \textbf{Ignacio E. Grossmann:} Conceptualisation, Methodology, Supervision, Writing - review \& editing. \textbf{Brage Rugstad Knudsen:} Conceptualisation, Methodology, Supervision, Writing - review \& editing, Funding acquisition. \textbf{Ken McKinnon:} Conceptualisation, Methodology, Writing - review \& editing. \textbf{Rodrigo Garcia Nava:} Conceptualisation, Methodology, Software. \textbf{Asgeir Tomasgard:} Conceptualisation, Methodology, Supervision, Writing - review \& editing, Funding acquisition.

\section*{Declaration of competing interest}
The authors declare that they have no known competing financial interests or personal relationships that could have appeared to influence the work reported in this paper.

\section*{Acknowledgements}
This work was supported by the Research Council of Norway through PETROSENTER LowEmission [project code 296207].

%% Authors are advised to submit their bibtex database files. They are
%% requested to list a bibtex style file in the manuscript if they do
%% not want to use model1-num-names.bst.

%% References without bibTeX database:

% \begin{thebibliography}{00}

%% \bibitem must have the following form:
%%   \bibitem{key}...
%%
 % \footnote{For further information, please refer to \url{https://docs.julialang.org/en/v1/manual/asynchronous-programming/}}

% \section*{References}
%\bibliographystyle{elsarticle-harv}
\clearpage

\setlength{\bibsep}{0pt plus 0.3ex}
\footnotesize{
% \bibliographystyle{ieeetr}%When changing reference style
% \bibliographystyle{elsarticle-harv}
% \bibliographystyle{apalike}
\bibliographystyle{model5-names}
\bibliography{retrofit_offshore_system_final}}
\clearpage

\appendix

\section{Nomenclature}
\label{sec:nonmenclature}
\begin{description}[itemsep=-5pt, leftmargin=!,labelwidth=\widthof{\bfseries $X^{MaxReT/MinReT}_{pi}$}]
\item [\textbf{Investment planning model indices and sets}]
\item [$p \in \mathcal{P}$] set of technologies
\item [$p \in \mathcal{P}^{R}$] set of candidate retrofit technologies
\item [$p \in \mathcal{P}^{R}_{p}$] set of candidate technologies that an existing technology $p$ can be retrofitted to $(p \in \mathcal{P}^{R})$ which including abandonment and prolong
\item [$p \in \mathcal{P}^{RT}$] set of candidate technologies be retrofitted to
\item [$i \in \mathcal{I}^{Ope}$] set of operational nodes
\item [$i \in \mathcal{I}^{Inv}$] set of investment nodes
\item [$j \in \mathcal{I}^{Inv}_{i}$] set of investment nodes $j$ $(j \in \mathcal{I}^{Inv})$ succeed to investment node $i$ $(i \in \mathcal{I}^{Inv})$
\item [$j \in \mathcal{I}^{Ope}_{i}$] set of operational nodes $j$ $(j \in \mathcal{I}^{Ope})$ succeed to investment node $i$ $(i \in \mathcal{I}^{Inv})$
\item [$(x, \theta, \lambda) \in \mathcal{F}_{i(k-1)}$] set of the Benders cut built up to iteration $k-1$, where $x$ is the vector of sampled points, $theta$ and $\lambda$ are the actually cost of subproblem at the sampled points, and the vector of subgradients at the sampled points, respectively. 

\item [\textbf{Investment planning model parameters}]
\item [$C^{InvV}_{pi}$] unitary investment cost of technology $p$ in investment node $i$ ($p \in \mathcal{P}, i \in \mathcal{I}^{Inv}$) [€/MW, €/MWh, €/kg]
\item [$C^{InvF}_{pi}$] fixed capacity independent investment cost of technology $p$ in investment node $i$ ($p \in \mathcal{P}, i \in \mathcal{I}^{Inv}$) [€]
\item [$C^{Fix}_{pi}$] unitary fix operational and maintenance cost of technology $p$ in operational node $i$ ($p \in \mathcal{P},i \in \mathcal{I}^{Ope}$) [€/MW, €/MWh, €/kg]
\item [$C^{ReTV}_{pi}$] unitary investment cost of retrofitted technology $p$ in investment node $i$ ($p \in \mathcal{P}^{RT}, i \in \mathcal{I}^{Inv}$) [€/MW, €/MWh, €/kg]
\item [$C^{ReTF}_{pi}$] fixed capacity independent investment cost of retrofitted to technology $p$ in investment node $i$ ($p \in \mathcal{P}^{RT}, i \in \mathcal{I}^{Inv}$) [€]
\item [$C^{ReTFixO}$] fix operational cost of the technology that is retrofitted to $p$ in investment node $i$ ($p \in \mathcal{P}^{RT}, i \in \mathcal{I}^{Inv}$) [€]
\item [$C^{ReFFixO}_{pi}$] fix operational cost of retrofitted technology $p$ in investment node $i$ ($p \in \mathcal{P}^{RT}, i \in \mathcal{I}^{Inv}$) [€]
\item [$X^{MaxInv/MinInv}_{pi}$] maximum/minimum built capacity of technology $p$ in investment node $i$ ($p \in \mathcal{P}, i \in \mathcal{I}^{Inv}$) [MW, MWh, kg]
\item [$X^{MaxAcc}_{p}$] maximum installed capacity of technology over the planning horizon $p$ ($p \in \mathcal{P}$) [MW, MWh, kg]
\item [$\kappa$] scaling effect depending on time step between successive investment nodes
\item [$H_p$] lifetime of technology $p$ ($p \in \mathcal{P}$)
\item [$X^{HistReF}_{pi}$] historical capacity of existing technology that can be retrofitted [MW, MWh, kg]
\item [$X^{Hist}_{pi}$] historical capacity of technology $p$ in operational node $i$ ($p \in \mathcal{P}, i \in \mathcal{I}^{Ope}$) [MW, MWh, kg]
\item [$X^{MaxReT/MinReT}_{pi}$] maximum/minimum built capacity of technology $p$ in investment node $i$ ($p \in \mathcal{P}^{RT}, i \in \mathcal{I}^{Inv}$) [MW, MWh, kg]
\item [$X^{MaxAccReT}_{pi}$] maximum installed capacity of technology $p$ ($p \in \mathcal{P}^{RT}$) [MW, MWh, kg]
\item [$x_{i}$] right hand side of the operational problem
\item [$c_{i}$] cost coefficients of the operational problem
\item [$\pi^{Inv/Ope}_{i}$] probability of investment/operational node $i$, ($i \in \mathcal{I}^{Inv}$/$i \in \mathcal{I}^{Ope}$)
\item [$\mu^{E}_{i}$] CO$_2$ budget at operational node $i$ ($i \in \mathcal{I}^{Ope}$)
\item [$\mu^{DP}_{i}$] scaling factor on power demand at operational node $i$ ($i \in \mathcal{I}^{Ope}$)
\item [$\mu^{P}_{i}$] scaling factor on oil and gas production at operational node $i$ ($i \in \mathcal{I}^{Ope}$)
\item [$\mu^{DHy}_{i}$] scaling factor on hydrogen demand at operational node $i$ ($i \in \mathcal{I}^{Ope}$)
\item [$C^{CO2}_{i}$] CO$_2$ emission price at operational node $i$ ($i \in \mathcal{I}^{Ope}$)
\item [$\mathcal{S}^{Ope}_{i}$] strategic stage of operational node $i$ $(i \in \mathcal{I}^{Ope})$
\item [$\mathcal{S}^{Inv}_{i}$] strategic stage of investment node $i$ $(i \in \mathcal{I}^{Inv})$

\item [\textbf{Investment planning model variables}]
\item [$x_{pi}^{Acc}$] accumulated capacity of device $p$ in operational node $i$ ($p \in \mathcal{P}, i \in \mathcal{I}^{Ope}$) [MW, MWh, kg]
\item [$x_{pi}^{Inv}$] newly invested capacity of device $p$ in investment node $i_{0}$ ($p \in \mathcal{P}, i \in \mathcal{I}^{Inv}$) [MW, MWh, kg]
\item [$y^{Inv}_{pi}$] 1 if technology $p$ is newly invested in investment node $i$, 0 otherwise ($p \in \mathcal{P}, i \in \mathcal{I}^{Inv}$)
\item [$y^{ReT}_{pi}$] 1 if technology $p$ is retrofitted to in investment node $i$, 0 otherwise ($p \in \mathcal{P}^{RT}, i \in \mathcal{I}^{Inv}$)
\item [$x^{AccReT}_{pi}$] accumulated capacity of technology $p$ that is retrofitted to in operational node $i$ ($p \in \mathcal{P}^{RT}, i \in \mathcal{I}^{Ope}$)
\item [$y^{ReF}_{pi}$] 1 if technology $p$ is retrofitted from in investment node $i$, 0 otherwise ($p \in \mathcal{P}^{R}, i \in \mathcal{I}^{Inv}$)
\item [$x^{AccReF}_{pi}$] accumulated capacity of retrofitted from technology in operational node $i$ ($p \in \mathcal{P}^{R}, i \in \mathcal{I}^{Ope}$)
\item [$x^{ReT}_{pi}$]  in operational node $i$ ($p \in \mathcal{P}^{R}, i \in \mathcal{I}^{Ope}$)
\item [$c^{INV}$] total investment and fixed operating and maintenance costs [€]
\item [$c^{OPE}_i$] approximated operational cost in operational node $i$ in Benders decomposition ($i \in \mathcal{I}^{Ope}$) [€]

\item [\textbf{Operational model indices and sets}]
\item [$n \in \mathcal{N}$] set of time slices
\item [$t \in \mathcal{T}$] set of hours in all time slices
\item [$t \in \mathcal{T}_{n}$] set of hours in time slice $n$ $(n \in \mathcal{N})$
\item [$l \in \mathcal{L}$] set of transmission lines
\item [$l \in \mathcal{L}^{Hy}$] set of hydrogen pipelines
\item [$l \in \mathcal{L}^{Out/In}_z$] set of transmission lines go out of/into region $z$
\item [$l \in \mathcal{L}^{HyOut/HyIn}_z$] set of hydrogen pipelines go out of/into region $z$
\item [$g \in \mathcal{G}$] set of thermal generation
\item [$r \in \mathcal{G}^{R}$] set of renewable generation
\item [$g \in \mathcal{G}^{H}$] set of hydropower generation including run of the river $\mathcal{G}^{HRor}$ and seasonal $\mathcal{G}^{HSea}$
\item [$s \in \mathcal{S}^{E}$] set of electricity storage
\item [$s \in \mathcal{S}^{Hy}$] set of hydrogen storage
\item [$b \in \mathcal{B}^{E}$] set of electric boilers
\item [$r \in \mathcal{R}$] set of SMRCCS
\item [$e \in \mathcal{E}$] set of electrolysers
\item [$f \in \mathcal{F}$] set of fuel cells
\item [$z \in \mathcal{Z}^{P}$] set of all platform clusters
\item [$z \in \mathcal{Z}$] set of all locations
\item [$p \in \mathcal{P}^{*}$] set of all thermal generators, electric boilers, electrolysers, electricity storage, fuel cells and seasonal hydropower generation $(\mathcal{P}^{*}=\mathcal{G} \cup \mathcal{B}^{E} \cup \mathcal{E} \cup \mathcal{S}^{E} \cup \mathcal{F} \cup \mathcal{G}^{HSea})$
\item [$v \in \mathcal{V}^{*}$] set of hydrogen storage and SMRCCS plants $(\mathcal{V}^{*}=\mathcal{S}^{Hy} \cup \mathcal{R})$ 

\item [\textbf{Operational model parameters}]
\item [$\mu^{E}$] CO$_2$ emission limit (tonne)
\item [$\mu^{DP/DH/DHy}$] scaling effect on power demand/heat demand/hydrogen demand
\item [$H_t$] number of hour(s) in one operational period $t$
\item [$\pi_t$] weighted length of one operational period $t$
\item [$R_{rt}^{GR}$] capacity factor of renewable unit $r$ in period $t$ ($r \in \mathcal{R}, t \in \mathcal{T}$)
\item [$\eta^*$] efficiency of electric boilers, fuel cells, thermal generators, electric storage and transmission lines $*=\{\text{BE, SE, L, HrG\}}$ indexed by related sets
\item [$E_{g}^{G}$]  CO$_2$ emission factor of thermal generation $g$ ($g \in \mathcal{G}$) [t/MWh]
\item [$C_g^G$] total operational cost of generating 1 MW power from thermal generation $g$ ($g \in \mathcal{G}$) [€/MW]
\item [$C^{Shed,l}$] load shed penalty cost of power $(l=P)$, heat $(l=H)$ and hydrogen $l=Hy$ [€/MW, €/kg]
\item [$\sigma^{Res}_{z}$] spinning reserve factor in region $z$ ($z \in \mathcal{Z}$)
\item [$\alpha^{G}_{g}$] maximum ramp rate of generators ($g \in \mathcal{G}$) [MW/MW]
\item [$\rho^{E}$] conversion factor of electrolyser to hydrogen [MWh/kg]
\item [$P^{DP/DH}_{zt}$] power demand/heat demand in location $z$ period $t$ $(z \in \mathcal{Z}, t \in \mathcal{T})$ [MW]
\item [$\rho^{F}$] hydrogen consumption factor of fuel cell [kg/MW]
\item [$P_{g}^{AccG}$] accumulated capacity of thermal generator $g$ $(g \in \mathcal{G})$ [MW]
\item [$P_{g}^{AccHRor}$] accumulated capacity of run of the river hydropower generation $g$ $(g \in \mathcal{G}^{HRor})$ [MW]
\item [$P_{g}^{Acc}$] accumulated capacity of technology $p$ $(p \in \mathcal{P}^{*})$ [MW]
\item [$Q_{s}^{AccSE}$] accumulated storage capacity of electricity store $s$ ($s \in \mathcal{S}^{E}$) [MWh]
\item [$P_{l}^{AccL}$] accumulated capacity of line $l$ ($l \in \mathcal{L}$) [MW]
\item [$C^{R}$] operational cost of producing 1 kg hydrogen from SMRCCS [€/kg]
\item [$P^{HSea/HRor}_{gt}$] production profile of seasonal hydropower/run of the river hydropower in location $z$ period $t$ $(z \in \mathcal{Z}, t \in \mathcal{T})$ [MW]
\item [$V^{DHy}_{zt}$] hydrogen demand in region $z$ period $t$ $(z \in \mathcal{Z}, t \in \mathcal{T})$ [MW]
\item [$E^{R}$] emission factor of SMRCCS
\item [$V^{Acc}$] storage level, injection and withdrawal capacities of hydrogen storage and capacity of SMRCCS [kg]

\item [\textbf{Operational model variables}]
\item [$p_{gt}^{G}$] power generation of thermal generator $g$ in period $t$ ($g \in \mathcal{G}, t \in \mathcal{T}$) [MW]
\item [$p_{gt}^{ResG}$] power reserved of thermal generator $g$ for spinning reserve requirement in period $t$ ($g \in \mathcal{G}, t \in \mathcal{T}$) [MW]
\item [$p_{st}^{SE+}/p_{st}^{SE-}$] charge/discharge power of electricity store $s$ in period $t$ ($s \in \mathcal{S}^{E}, t \in \mathcal{T}$) [MW]
\item [$p_{st}^{ResSE}$] power reserved in electricity store $s$ for spinning reserve requirement in period $t$  ($s \in \mathcal{S}^{E}, t \in \mathcal{T}$) [MW]
\item [$q_{st}^{SE}$] energy level of electricity store $s$ at the start of period $t$ ($s \in \mathcal{S}^{E}, t \in \mathcal{T}$) [MWh]
\item [$p_{zt}^{GShed,l}$] generation shed for power $(l=P)$ and heat $(l=H)$ in location $z$ in period $t$ ($z \in \mathcal{Z}, t \in \mathcal{T}$) [MW]
\item [$p_{zt}^{Shed,l}$] load shed for power $(l=P)$ and heat $(l=H)$ in location $z$ in period $t$ ($z \in \mathcal{Z}, t \in \mathcal{T}$) [MW]
\item [$v_{zt}^{GShedHy}$] hydrogen production shed in location $z$ in period $t$ ($z \in \mathcal{Z}, t \in \mathcal{T}$) [kg]
\item [$v_{zt}^{ShedHy}$] hydrogen load shed in location $z$ in period $t$ ($z \in \mathcal{Z}, t \in \mathcal{T}$) [kg]
\item [$p_{lt}^{L}$] power flow in line $l$ in period $t$ ($l \in \mathcal{L}, t \in \mathcal{T}$) [MW]
\item [$p_{bt}^{BE}$] power consumption of electric boiler $b$ in period $t$ ($b \in \mathcal{B}^{E}, t \in \mathcal{T}$) [MW]
\item [$p^{F}_{ft}$] power generation of fuel cell $f$ in period $t$ ($f \in \mathcal{F}, t \in \mathcal{T}$) [MW]
\item [$p^{E}_{et}$] power consumption of electrolyser $e$ in period $t$ ($e \in \mathcal{E}, t \in \mathcal{T}$) [MW]
\item [$v^{SHy+}_{st}/v^{SHy-}_{st}$] injection/withdraw of hydrogen to (from) hydrogen storage $s$ in period $t$ ($s \in \mathcal{S}^{Hy}, t \in \mathcal{T}$) [kg]
\item [$v^{SHy}_{st}$] storage level of hydrogen storage $s$ in period $t$ $(s \in \mathcal{S}^{Hy}, t \in \mathcal{T})$ [kg]
\item [$v^{R}_{rt}$] hydrogen production of SMRCCS $r$ in period $t$ $(r \in \mathcal{R}, t \in \mathcal{T})$ [kg]
\item [$v^{LHy}_{lt}$] hydrogen flow in pipeline $l$ period $t$ $(l \in \mathcal{L}^{Hy}, t \in \mathcal{T})$
\item [$v_{vt}$] hydrogen injection, withdraw, storage level of hydrogen storage, and hydrogen production of SMRCCS in period $t$ $(v \in \mathcal{S}^{Hy} \cup \mathcal{R})$ [kg]
\end{description}

% 
\end{document}

%% End of file `elsarticle-template-1-num.tex'.