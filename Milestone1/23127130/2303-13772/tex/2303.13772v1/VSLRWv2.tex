%%%%%%%%%%%%%%%%%%%%%%%%%%%%%%%%%%%%%%%%%%%%%%%%%%%%%%%%%%%%%%%%%%%%%
%%                                                                 %%
%% Please do not use \input{...} to include other tex files.       %%
%% Submit your LaTeX manuscript as one .tex document.              %%
%%                                                                 %%
%% All additional figures and files should be attached             %%
%% separately and not embedded in the \TeX\ document itself.       %%
%%                                                                 %%
%%%%%%%%%%%%%%%%%%%%%%%%%%%%%%%%%%%%%%%%%%%%%%%%%%%%%%%%%%%%%%%%%%%%%

%%\documentclass[referee,sn-basic]{sn-jnl}% referee option is meant for double line spacing

%%=======================================================%%
%% to print line numbers in the margin use lineno option %%
%%=======================================================%%

%%\documentclass[lineno,sn-basic]{sn-jnl}% Basic Springer Nature Reference Style/Chemistry Reference Style

%%======================================================%%
%% to compile with pdflatex/xelatex use pdflatex option %%
%%======================================================%%

%%\documentclass[pdflatex,sn-basic]{sn-jnl}% Basic Springer Nature Reference Style/Chemistry Reference Style

%%\documentclass[sn-basic]{sn-jnl}% Basic Springer Nature Reference Style/Chemistry Reference Style
\documentclass[sn-mathphys]{sn-jnl}% Math and Physical Sciences Reference Style
%%\documentclass[sn-aps]{sn-jnl}% American Physical Society (APS) Reference Style
%%\documentclass[sn-vancouver]{sn-jnl}% Vancouver Reference Style
%%\documentclass[sn-apa]{sn-jnl}% APA Reference Style
%%\documentclass[sn-chicago]{sn-jnl}% Chicago-based Humanities Reference Style
%%\documentclass[sn-standardnature]{sn-jnl}% Standard Nature Portfolio Reference Style
%%\documentclass[default]{sn-jnl}% Default
%%\documentclass[default,iicol]{sn-jnl}% Default with double column layout

%\usepackage[comma,authoryear]{natbib}

%%%% Standard Packages
%%<additional latex packages if required can be included here>
%%%%

%%%%%=============================================================================%%%%
%%%%  Remarks: This template is provided to aid authors with the preparation
%%%%  of original research articles intended for submission to journals published 
%%%%  by Springer Nature. The guidance has been prepared in partnership with 
%%%%  production teams to conform to Springer Nature technical requirements. 
%%%%  Editorial and presentation requirements differ among journal portfolios and 
%%%%  research disciplines. You may find sections in this template are irrelevant 
%%%%  to your work and are empowered to omit any such section if allowed by the 
%%%%  journal you intend to submit to. The submission guidelines and policies 
%%%%  of the journal take precedence. A detailed User Manual is available in the 
%%%%  template package for technical guidance.
%%%%%=============================================================================%%%%

\jyear{2021}%

\usepackage{adjustbox}  %resize table size

%% as per the requirement new theorem styles can be included as shown below
\theoremstyle{thmstyleone}%
\newtheorem{theorem}{Theorem}%  meant for continuous numbers
%%\newtheorem{theorem}{Theorem}[section]% meant for sectionwise numbers
%% optional argument [theorem] produces theorem numbering sequence instead of independent numbers for Proposition
\newtheorem{proposition}[theorem]{Proposition}% 
%%\newtheorem{proposition}{Proposition}% to get separate numbers for theorem and proposition etc.

\theoremstyle{thmstyletwo}%
\newtheorem{example}{Example}%
\newtheorem{remark}{Remark}%

\theoremstyle{thmstylethree}%
\newtheorem{definition}{Definition}%

\raggedbottom
%%\unnumbered% uncomment this for unnumbered level heads

\usepackage{tensor}

\newcommand{\TB}{\text{B}}
\newcommand{\tT}{\tilde{T}}
\newcommand{\tm}{\tilde{m}}
\newcommand{\tF}{\tilde{F}}
\newcommand{\tR}{\tilde{R}}
\newcommand{\tr}{\tilde{r}}
\newcommand{\tg}{\tilde{g}}
\newcommand{\tmg}{\tilde{\mathcal{G}}}
\newcommand{\tc}{\tilde{c}}
\newcommand{\ttc}{\tilde{\tilde{c}}}
\newcommand{\tG}{\tilde{G}}
\newcommand{\tsigma}{\tilde{\sigma}}
\newcommand{\ta}{\tilde{a}}
\newcommand{\te}{\tilde{e}}
\newcommand{\GR}{\text{GR}}
\newcommand{\tgam}{\tilde{\gamma}}
\newcommand{\tlambda}{\tilde{\lambda}}
\newcommand{\tbeta}{\tilde{\beta}}
\newcommand{\tGam}{\tilde{\Gamma}}
\newcommand{\tkapp}{\tilde{\kappa}}
\newcommand{\tk}{\tilde{k}}
\newcommand{\trho}{\tilde{\rho}}
\newcommand{\tP}{\tilde{P}}
\newcommand{\tp}{\tilde{p}}
\newcommand{\tnu}{\tilde{\nu}}
\newcommand{\tmu}{\tilde{\mu}}
\newcommand{\ttmu}{\tilde{\tilde{\mu}}}
\newcommand{\tepsilon}{\tilde{\epsilon}}
\newcommand{\ttepsilon}{\tilde{\tilde{\epsilon}}}
\newcommand{\Ch}{\text{Ch}}
\newcommand{\Tpc}{\text{pc}}
\newcommand{\TMpc}{\text{Mpc}}
\newcommand{\Temis}{\text{emis}}
\newcommand{\Tchirp}{\text{chirp}}
\newcommand{\Tgw}{\text{gw}}
\newcommand{\Ts}{\text{s}}
\newcommand{\Tret}{\text{ret}}
\newcommand{\Teff}{\text{eff}}
\newcommand{\Ttot}{\text{tot}}
\newcommand{\Tobs}{\text{obs}}
\newcommand{\Tbias}{\text{bias}}
\newcommand{\Treq}{\text{req}}
\newcommand{\Tls}{\text{ls}}
\newcommand{\Tso}{\text{so}}
\newcommand{\YPBBN}{Y_{\text{P}}^{(\text{BBN})}}
\newcommand{\thbar}{\tilde{\hbar}}
\newcommand{\tilh}{\tilde{h}}
\newcommand{\hTT}{\tilde{h}} 


\begin{document}

\title[Article Title]{A viable varying speed of light model in the RW metric}

%%=============================================================%%
%% Prefix	-> \pfx{Dr}
%% GivenName	-> \fnm{Joergen W.}
%% Particle	-> \spfx{van der} -> surname prefix
%% FamilyName	-> \sur{Ploeg}
%% Suffix	-> \sfx{IV}
%% NatureName	-> \tanm{Poet Laureate} -> Title after name
%% Degrees	-> \dgr{MSc, PhD}
%% \author*[1,2]{\pfx{Dr} \fnm{Joergen W.} \spfx{van der} \sur{Ploeg} \sfx{IV} \tanm{Poet Laureate} 
%%                 \dgr{MSc, PhD}}\email{iauthor@gmail.com}
%%=============================================================%%

\author*[1]{\fnm{Seokcheon} \sur{Lee}}\email{skylee@skku.edu}

%\author[2,3]{\fnm{Second} \sur{Author}}\email{iiauthor@gmail.com}
%\equalcont{These authors contributed equally to this work.}

%\author[1,2]{\fnm{Third} \sur{Author}}\email{iiiauthor@gmail.com}
%\equalcont{These authors contributed equally to this work.}

\affil*[1]{\orgdiv{Department of Physics}, \orgname{Institute of Basic Science, Sungkyunkwan University,}, \orgaddress{ \city{Suwon}, \postcode{16419},  \country{Korea}}}

%\affil[2]{\orgdiv{Department}, \orgname{Organization}, \orgaddress{\street{Street}, \city{City}, \postcode{10587}, \state{State}, \country{Country}}}

%\affil[3]{\orgdiv{Department}, \orgname{Organization}, \orgaddress{\street{Street}, \city{City}, \postcode{610101}, \state{State}, \country{Country}}}

%%==================================%%
%% sample for unstructured abstract %%
%%==================================%%

\abstract{The Robertson-Walker (RW) metric allows us to apply general relativity to model the behavior of the Universe as a whole (\textit{i.e.}, cosmology). We can properly interpret various cosmological observations, like the cosmological redshift, the Hubble parameter, geometrical distances, and so on, if we identify fundamental observers with individual galaxies. That is to say that the interpretation of observations of modern cosmology relies on the RW metric. The RW model satisfies the cosmological principle in which the 3-space always remains isotropic and homogeneous. One can derive the cosmological redshift relation from this condition. We show that it is still possible for us to obtain consistent results in a specific time-varying speed-of-light model without spoiling the success of the standard model.  The validity of this model needs to be determined by observations. }

%%================================%%
%% Sample for structured abstract %%
%%================================%%

% \abstract{\textbf{Purpose:} The abstract serves both as a general introduction to the topic and as a brief, non-technical summary of the main results and their implications. The abstract must not include subheadings (unless expressly permitted in the journal's Instructions to Authors), equations or citations. As a guide the abstract should not exceed 200 words. Most journals do not set a hard limit however authors are advised to check the author instructions for the journal they are submitting to.
% 
% \textbf{Methods:} The abstract serves both as a general introduction to the topic and as a brief, non-technical summary of the main results and their implications. The abstract must not include subheadings (unless expressly permitted in the journal's Instructions to Authors), equations or citations. As a guide the abstract should not exceed 200 words. Most journals do not set a hard limit however authors are advised to check the author instructions for the journal they are submitting to.
% 
% \textbf{Results:} The abstract serves both as a general introduction to the topic and as a brief, non-technical summary of the main results and their implications. The abstract must not include subheadings (unless expressly permitted in the journal's Instructions to Authors), equations or citations. As a guide the abstract should not exceed 200 words. Most journals do not set a hard limit however authors are advised to check the author instructions for the journal they are submitting to.
% 
% \textbf{Conclusion:} The abstract serves both as a general introduction to the topic and as a brief, non-technical summary of the main results and their implications. The abstract must not include subheadings (unless expressly permitted in the journal's Instructions to Authors), equations or citations. As a guide the abstract should not exceed 200 words. Most journals do not set a hard limit however authors are advised to check the author instructions for the journal they are submitting to.}

\keywords{Varying speed of light, Robertson-Walker metric, General covariance}

%%\pacs[JEL Classification]{D8, H51}

%%\pacs[MSC Classification]{35A01, 65L10, 65L12, 65L20, 65L70}

\maketitle

\section{Introduction}\label{sec1}

Cosmology has been one of the most successful and relevant applications of Einstein's General Theory of Relativity (GR) to model the behavior of the Universe as a whole.  In other words,  cosmology as science can exist.  This fact becomes possible by adopting assumptions consistent with our observing Universe.  From the constancy of the temperature of cosmic microwave background (CMB) in a different direction in the sky,  we have good evidence that the Universe is  {\it isotropic} on the very largest scales \cite{Hinshaw:2013,Planck:2018nkj}.  If the Universe has no preferred center,  then the isotropy also implies the {\it homogeneity} consistent with the observation that the matter distribution looks uniform on scales of more than 100 million light-years \cite{Guzzo:2018xbe,DES:2020sjz}.  From these observations,  one reaches the {\it Cosmological Principle} (CP) that states that the Universe looks the same from all positions in space {\it at a particular time} and that all directions at any point are equivalent.  Thus,  it is possible to adopt the standard form of the Robertson-Walker (RW) metric in cosmology for the cosmic time $t$ \cite{Robertson:1929,Robertson:1933,Walker:1935,Walker:1937}.

The Lorentz transformation (LT) between two Galilean frames (GFs) derives consequences of special relativity (SR). In the general theory of relativity (GR), the inertial frame (IF) means a freely falling one. One can establish a Lorentz-invariant spacetime interval from the coordinate differences between two events. This spacetime interval is light-like ({\it i.e.}, null) if it equals zero. One can interpret this as signals moving at the speed of light connecting events in Minkowski spacetime separated by the null interval \cite{Morin07}. However, in GR, it is impossible to define a global time owing to the absence of a universal IF. Nevertheless, one can define a global time for the Universe when a set of requirements is satisfied and a metric embodying the cosmological principle meets these requirements. Then, one can define a global time by a foliation of spacetime as a sequence of non-intersecting spacelike 3D surfaces \cite{Weyl:1923,Islam02,Narlikar02,Hobson06,Gron07,Ryder09,CB15,Roos15,Guidry19,Ferrari21}. 

SR contains only one parameter, $c$, the speed of light in a vacuum.  We have shown that the universal Lorentz covariance, or, equivalently, the single postulate of Minkowski spacetime is good enough to satisfy the SR \cite{Das93,Schutz97,Lee:2020zts}. Thus, it is possible to make the Lorentz invariant (LI) varying speed of light (VSL) model as long as $c$ is locally constant and changes at cosmological scales. To avoid trivial rescaling of units, one must test the simultaneous variation of $c$ and Newton's gravitational constant $G$ because $c$ and $G$ enter as the combination $G/c^4$ in the Einstein action \cite{Barrow:1998eh}.  %with $c = c_0 a^{n}$ 

All galaxies are assumed to lie on a hypersurface so that the surface of simultaneity for the local Lorentz frame (LF) of each galaxy coincides locally with the hypersurface. One can conclude that a global cosmic time for all galaxies on the hypersurface provided that space is homogeneous and isotropic (implying that the spatial curvature is constant). Cosmic time is the one measured by a comoving observer who observes the universe expanding uniformly around her. Light travels through the expanding space. The observation that sufficiently distant light sources show cosmological redshift corresponding to their distance from us follows the so-called Hubble’s law. We can induce the cosmological redshift relation from the cosmological principle (CP).

As a possible way to explain problematic observational results based on GR, the possibility of various VSLs has sometimes been invoked.  Einstein claimed that a shorter wavelength $\lambda$ leads to a lower speed of light using $c = \nu \lambda$ with the constant frequency $\nu$. He assumed that a gravitational field makes the clock run slower by $\nu_1 = \nu_2 (1 + GM/rc^2)$ \cite{Einstein:1911}. Dicke proposed that the wavelength and the frequency vary by defining a refractive index $n \equiv c/c_0 = 1 + 2GM/rc^2$ \cite{Dicke:1957}. He also considered a cosmology with an alternative description to the cosmological redshift by using a decreasing $c$ in time.  The assumption of these pioneer works is that the time dilation is due to the local gravity.  There have been cosmology-based VSL models to explain the horizon problem of the Big Bang model and provide an alternative method to cosmic inflation \cite{Barrow:1998eh,Petit:1988,Petit:1988-2,Petit:1989,Midy:1989,Moffat:1992ud,Petit:1995ass,Albrecht:1998ir,Barrow:1998he,Clayton:1998hv,Barrow:1999jq,Clayton:1999zs,Brandenberger:1999bi,Bassett:2000wj,Gopakumar:2000kp,Magueijo:2000zt,Magueijo:2000au,Magueijo:2003gj,Magueijo:2007gf,Petit:2008eb,Roshan:2009yb,Sanejouand:2009,Nassif:2012dr,Moffat:2014poa,Ravanpak:2017kdg,Costa:2017abc,Nassif:2018pdu}. A VSL model which proposed the change of the speed of light only without allowing the variations of other physical constants is called minimal VSL (mVSL).  Petit argued that if $c$ varies as a function of cosmic time, then one should include the joint variations of all related physical constants. These variations should be based on the consistency of all physical equations, and measurements of these constants remain consistent with physics laws during the evolution of the Universe. From this consideration, one might be able to obtain a universal gauge relationship and the temporal variation of the parameters that are regarded as constants \cite{Petit:1995ass,Petit:2008eb}
\begin{align}
G &= G_0 a^{-1} \,, \quad m = m_{0} a  \,, \quad c = c_0 a^{\frac{1}{2}}  \,, \quad h = h_0 a^{\frac{3}{2}} \,, \quad e = e_{0} a^{\frac{1}{2}}  \,, \quad \mu = \mu_0 a \label{Petitconst} \,.
\end{align}
We should emphasize those cosmology-based VSL models are different from those based on local gravity such as \cite{Einstein:1911,Dicke:1957}.  In these models (based on the RW metric), the cosmic time dilation is due to both the expansion of the Universe and the difference in the values of the local speed of light \cite{Barrow:1998he,Lee:2023rqv}. 

In Sec. ~\ref{sec:FLRW}, we review how to derive the cosmological redshift in the RW metric.  We extend this idea into the specific time-varying speed of light model (\textit{i.e.},  meVSL \cite{Lee:2020zts}) in section~\ref{sec:VSLz}. In section~\ref{sec:reason}, we explain how we can obtain consistent results in this model. We summarize the consequences of meVSL in the FLRW universe to keep the cosmological principle in section~\ref{sec:Cons}. We conclude in Sec. ~\ref{sec:Conc}. 

\section{Review of Cosmological redshift}
\label{sec:FLRW}

The RW metric is 
\begin{align}
ds^2 = -(d X^0)^2 +  a^2(t) \left( \frac{dr^2}{1-kr^2} + r^2 \left( d \theta^2 + \sin^2 \theta d \phi^2 \right) \right) \equiv -(d X^0)^2 +  a^2(t) dl_{3\textrm{D}}^2 \label{RW} \,,
\end{align}
where $X^0 = ct$.  In this metric, the light signal propagates along the null geodesic $ds^2 = 0$, and from metric \eqref{RW}, we obtain outgoing light signals
\begin{align}
dl_{3\textrm{D}}(r\,, \theta\,, \phi) = \frac{dX^0}{a(t)} \label{dl3D} \,.
\end{align}
The spatial infinitesimal line element $dl_{3\textrm{D}}$ is a function of the comoving coordinates ($\sigma\,,\theta\,,\phi$) only and thus should be the same value at any given time.  From this fact, one traditionally (\textit{i.e.}, by assuming a constant speed of light) obtains the cosmological redshift relation 
\begin{align}
\frac{dX^0(t_1)}{a(t_1)} = \frac{dX^0(t_2)}{a(t_2)}\quad \Rightarrow \quad \frac{dt_1}{a_1} = \frac{dt_2}{a_2} \quad \Rightarrow \quad \lambda_1 = c dt_1 \equiv c \nu_1^{-1} = \frac{a_1}{a_2} \lambda_2 \label{z} \,,
\end{align}
where $d t_i$ is the time interval of successive crests of light at $t_i$ (\textit{i.e.},  the inverse of the frequency $\nu_i$ at $t_i$) \cite{Weinberg:2008}.


\section{Cosmological redshift including VSL}
\label{sec:VSLz}

The standard cosmology has been very successful, and any viable new model should maintain results from the standard one. Now, we rederive the cosmological redshift relation in the meVSL model where we include the possibility of the varying speed of light at cosmological scales. Traditionally, we obtain the cosmological redshift under the assumption that the speed of light is constant at these scales.  However, the Lorentz invariance (LI) is a local symmetry that is only meaningful at each spacetime point (event), but GR is valid at cosmological scales. Therefore, the quibble about whether SR is generally adaptable at cosmological distances and time scales should be determined by observations \cite{Roos15}.  In this case, we can rewrite Eq.~\eqref{z} as
\begin{align}
\frac{dX^0(t_1)}{a(t_1)}  = \frac{dX^0(t_2)}{a(t_2)} \quad \Rightarrow \quad \frac{\tc_1 dt_1}{a_1} = \frac{\tc_2 dt_2}{a_2} \quad \Rightarrow \quad \lambda_1 = \tc_1 dt_1 = \frac{a_1}{a_2} \lambda_2 \label{zVSL} \,,
\end{align}
where $\tc_i \equiv \tc (t_i)$, $a_i \equiv a(t_i)$, and 
\begin{align}
dX^{0} = d \left(c t \right) = \left(\frac{d \ln c}{d \ln t} + 1 \right) c dt \equiv \tc dt \quad \textrm{and} \quad \delta c \equiv \frac{\tc}{c} = \left(\frac{d \ln c}{d \ln t} + 1 \right) \label{dX0} \,,
\end{align}
Thus, the cosmological redshift relation still holds even if we allow the speed of light can vary as a function of cosmic time.  These results have been derived before in a so-called meVSL model to satisfy Einstein field equations \cite{Lee:2020zts}.

\section{Reason for this possibility}
\label{sec:reason}

One measures the spectrum of light coming from a distant source.  One can determine the redshift by searching for features in the spectra like absorption lines, emission lines, or other variations in light intensity.  Cosmological redshift due to the expansion of the Universe may be characterized by the relative difference between the observed and emitted wavelengths of an object \cite{Weinberg:2008}.  The wavelength of the light at a given cosmic epoch $t_i$ is $\lambda_i$. Traditionally,  we make two assumptions. One is the clocks run at a different rate at different epochs $d t_i \neq d t_j$ when $i \neq j$, and the other is that the speed of light is constant during the entire history of the Universe.  But it is not a necessary condition to keep the LI because it is only true for the local inertial observer. There have been several projects to measure cosmological time dilation. Direct observation of the time dilation measures the decay time of distance supernova (SN) light curves and spectra \cite{Leibundgut:1996qm,SupernovaSearchTeam:1997gem,Foley:2005qu,Blondin:2007ua,Blondin:2008mz}. Another method is measuring time dilation by searching the stretching of peak-to-peak timescales of gamma-ray bursters (GRBs) \cite{Norris:1993hda,Wijers:1994qf,Band:1994ee,Meszaros:1995gj,Lee:1996zu,Chang:2001fy,Crawford:2009be,Zhang:2013yna,Singh:2021jgr}. There has been a search for the time dilation effect in the light curves of quasars (QSOs) located at cosmological distances \cite{Hawkins:2001be,Dai:2012wp}.  So far, it seems fair to say that no convincing detection has been made for cosmic time dilation with the conflict between different measurements. Also, there is no mechanism to determine $dt_i$ from the RW model.  Therefore, it might still be meaningful to investigate the possibility of varying speed of light in this observation as long as its results are consistent with the standard model ones.

\section{Consequences}
\label{sec:Cons}

If $c$ is not a constant in cosmic time, then one needs to rederive the Einstein field equations again by including this VSL effect.  Previously, we have shown one viable model \cite{Lee:2020zts}.  Metric and four-position are given by 
\begin{align}
g_{\mu\nu} = \textrm{diag} \left( -1\,,  \frac{a^2}{1-kr^2}\,, a^2 r^2\,, a^2 r^2 \sin^2 \theta \right) \quad , \quad x^{\mu} = \left( c[a]t \,, x \,, y \,, z  \right) \label{gmunuxmu} \,,
\end{align}
where we will denote the 3D spatial metric as $\gamma_{ij}$. Christoffel symbols are obtained from their definition
\begin{align}
&\Gamma^{\mu}_{\nu\lambda} \equiv \frac{1}{2} g^{\mu\alpha} \left( g_{\alpha\nu,\lambda} + g_{\alpha\lambda,\nu} - g_{\nu\lambda,\alpha} \right) \label{GammaApp} \, \\
&\Gamma^{0}_{ij} \equiv \frac{1}{2} g^{00} \left( g_{0j,i} + g_{i0,j} - g_{ij,0} \right) = \frac{1}{2} (-1) \left(- \frac{d a^2}{d x^0}  \gamma_{ij}\right) = \frac{a \dot{a}}{\tc} \gamma_{ij} \label{Gamma0ij} \\ &\Gamma^{i}_{0j} = \frac{1}{\tc}  \frac{\dot{a}}{a} \delta^i_j \quad , \quad \Gamma^{i}_{jk} = ^{s}\Gamma^{i}_{jk}  \label{GammacompApp} \,,
\end{align}
where $^{s}\Gamma^{i}_{jk}$ denote the Christoffel symbols for the spatial metric $\gamma_{ij}$ and we use
\begin{align}
dx^0 = d \left(c[a] t] \right) = \left( \frac{d \ln c}{d \ln a} \frac{d \ln a}{dt} t + 1 \right) c dt =  \left( \frac{d \ln c}{d \ln a}  H t + 1 \right) c dt 
\equiv \tc dt \label{dx0} \,. \end{align}
One obtains Riemann curvature tensors, Ricci curvature tensors, and a Ricci scalar.
\begin{align}
&\tensor{R}{^\alpha_\beta_\mu_\nu} = \Gamma^{\alpha}_{\beta\nu,\mu} - \Gamma^{\alpha}_{\beta\mu,\nu} + \Gamma^{\alpha}_{\lambda\mu} \Gamma^{\lambda}_{\beta\nu} - \Gamma^{\alpha}_{\lambda\nu} \Gamma^{\lambda}_{\beta\mu} \label{RabmuApp} \,, \\
&\tensor{R}{^0_i_0_j} = \frac{\gamma_{ij}}{\tc^2} \left( a \ddot{a} - \dot{a}^2 \frac{d \ln \tc}{d \ln a} \right) \quad , \quad \tensor{R}{^i_0_0_j} = \frac{\delta^{i}_{j}}{\tc^2} \left( \frac{\ddot{a}}{a} - \frac{\dot{a}^2}{a^2} \frac{d \ln \tc}{ d \ln a}  \right) \,, \label{R0i0jApp} \\
&\tensor{R}{^i_j_k_m} = \frac{\dot{a}^2}{\tc^2} \left( \delta^{i}_{k} \gamma_{jm} - \delta^i_m \gamma_{jk} \right) + \tensor[^s]{R}{^i_j_k_m} \quad , \quad \tensor[^s]{R}{^i_j_k_m} = k \left( \delta^i_k \gamma_{jm} - \delta^i_m \gamma_{jk} \right) \,. \label{RijkmApp} \\
&R_{\mu\nu} = \Gamma^{\lambda}_{\mu\nu,\lambda} - \Gamma^{\lambda}_{\mu\lambda,\nu} + \Gamma^{\lambda}_{\mu\nu} \Gamma^{\sigma}_{\lambda\sigma} - \Gamma^{\sigma}_{\mu\lambda} \Gamma^{\lambda}_{\nu \sigma} \label{RmnApp} \,, \\
%R_{00} &= \Gamma^{\lambda}_{00,\lambda} - \Gamma^{\lambda}_{0\lambda,0} + \Gamma^{\lambda}_{00} \Gamma^{\sigma}_{\lambda\sigma} - \Gamma^{\sigma}_{0\lambda} \Gamma^{\lambda}_{0\sigma} = - \frac{3}{\tc^2} \left( \frac{\ddot{a}}{a} - \frac{\dot{\tc}}{\tc} H \right)\label{R00mpApp} \,, \\
&R_{00} = -\frac{3}{\tc^2} \left( \frac{\ddot{a}}{a} - \frac{\dot{a}^2}{a^2} \frac{d \ln \tc}{ d \ln a}  \right) \quad , \quad
R_{ij} = \frac{\gamma_{ij}}{\tc^2} a^2 \left( 2 \frac{\dot{a}^2}{a^2} + \frac{\ddot{a}}{a} + 2 k \frac{\tc^2}{a^2} - \frac{\dot{a}^2}{a^2} \frac{d \ln \tc}{ d \ln a}  \right) \label{RijApp} \,, \\
&R = \frac{6}{\tc^2} \left( \frac{\ddot{a}}{a} + \frac{\dot{a}^2}{a^2} + k \frac{\tc^2}{a^2} - \frac{\dot{a}^2}{a^2} \frac{d \ln \tc}{ d \ln a}  \right) \label{RmpApp} \,.
\end{align} 
It causes the modification of the usual Friedmann equations in the VSL model shown in the previous work \cite{Lee:2020zts}. All of this argument relies on an RW metric and requires the homogeneity and isotropy of the 3D space. The adiabatic expansion of the Universe is a prerequisite to keep the CP. As $c$ evolves as a function of cosmic time, other physical constants and quantities also need to do as a function of $t$ to satisfy the CP \cite{Lee:2212}. In the meVSL model, one obtains the same result as a constant $c$ calculation for the comoving distance $l_{3\textrm{D}}$ by integrating $dl_{3\textrm{D}}$ for the cosmological redshift $z$ 

\begin{align}
\Delta s^2 = 0 \Rightarrow d l_{3\textrm{D}} = \frac{\tilde{c} dt}{a} = \frac{\tilde{c} da}{a^2 H(a)} = -\frac{c_0 dz}{H^{(\text{GR})}(z)} 
\end{align}
where $H(a) = H^{(\text{GR})} a^{b/4}$ when $\tilde{c} = c_0 a^{b/4}$.  $H^{(\text{GR})}$ denotes the Hubble parameter when $c$ is a constant. Thus,  the Hubble radius $c/H$ is the same as that of GR for the meVSL model and does not solve the horizon problem of the standard big bang model. However, the Hubble parameter is modified from that of GR by the extra factor $a^{b/4}$. It might solve the $H$-tension problem. In the meVSL model, we consider local thermodynamics, energy conservation, and other local physics. These considerations induce the time evolutions of other physical constants and quantities shown in table~\ref{tab:table-1}.

\begin{table}[htbp]
	%\centering
\caption{Summary for cosmological evolutions of physical constants and quantities of the meVSL model. These relations satisfy all known local physics laws, including special relativity, thermodynamics, and electromagnetic force \cite{Lee:2020zts,Lee:2212}. One should remind that the wavelength, frequency, and temperature of the photon (dimensionful physical quantities) evolve as cosmic time even in the standard model.}
\label{tab:table-1}
%\begin{minipage}{140pt}%{<preferred-table-width>}
\begin{adjustbox}{width=\columnwidth,center}
\begin{tabular}{|c||c|c|c|}
	\hline
	local physics laws & Special Relativity & Electromagnetism & Thermodynamics \\
	\hline \hline
	quantities & $\tm = \tm_0 a^{-b/2}$ & $\te = \te_0 a^{-b/4}\,, \tlambda = \tlambda_0 a \,, \tnu = \tnu_0 a^{-1+b/4}$ & $\tT = \tT_0 a^{-1}$ \\
	\hline
	constants & $\tc = \tc_0 a^{b/4} \,, \tG = \tG_0 a^{b}$ & $\tepsilon = \tepsilon_0 a^{-b/4} \,, \tmu = \tmu_0 a^{-b/4} \,, \tc = \tc_0 a^{b/4}$ & $\tk_{\TB 0} \,, \thbar = \thbar_0 a^{-b/4}$ \\
	\hline
	energies & $\tm \tc^2 = \tm_0 \tc_0^2$ & $\tilde{h} \tnu = \tilde{h}_0 \tnu_0 a^{-1}$ & $\tk_{\TB} \tT = \tk_{\TB 0} \tT_0 a^{-1}$ \\
	\hline
\end{tabular}
\end{adjustbox}
%\end{minipage}
\end{table}


\section{Conclusion}\label{sec:Conc}

We interpret most current cosmological observations based on the $\Lambda$CDM cosmological model. It uses the RW metric, and it is valuable for us to clarify any possibility of the extension of this model. We show that the meVSL model provides consistent results for various measurements even if we allow the speed of light to change in the expanding Universe. However, other physical constants, including the Planck constant, should also evolve as a function of cosmic time to be a viable model. The consequences of modifications of cosmological observations compared to the standard model would determine the validity of the meVSL model.

\bmhead{Acknowledgments}

SL is supported by Basic Science Research Program through the National Research Foundation of Korea (NRF) funded by the Ministry of Science, ICT, and Future Planning (Grant No. NRF-2017R1A2B4011168 and No. NRF-2019R1A6A1A10073079). SL appreciates professor C. Ahn, K. Ahn, S. Appleby, G. F. R. Ellis, B.S.~Kyae, K. W.Ng, K. A. Olive, W. I.  Park, and S.Y. Tsai  for their useful discussions and comments. 

\section*{Availability of data and materials}

Data sharing is not applicable to this article as no new data were created or analyzed in this study.

%%=============================================%%
%% For submissions to Nature Portfolio Journals %%
%% please use the heading ``Extended Data''.   %%
%%=============================================%%

%%=============================================================%%
%% Sample for another appendix section			       %%
%%=============================================================%%

%% \section{Example of another appendix section}\label{secA2}%
%% Appendices may be used for helpful, supporting or essential material that would otherwise 
%% clutter, break up or be distracting to the text. Appendices can consist of sections, figures, 
%% tables and equations etc.


%%===========================================================================================%%
%% If you are submitting to one of the Nature Portfolio journals, using the eJP submission   %%
%% system, please include the references within the manuscript file itself. You may do this  %%
%% by copying the reference list from your .bbl file, paste it into the main manuscript .tex %%
%% file, and delete the associated \verb+\bibliography+ commands.                            %%
%%===========================================================================================%%

%\bibliographystyle{plain}

%\bibliography{VSLRW-bib}% common bib file

\begin{thebibliography}{9}

%CMB

\bibitem{Hinshaw:2013}
G.~Hinshaw \textit{et al.},  ``Nine-year Wilkinson microwave anisotropy probe (WMAP) observations: Cosmological parameter results.,'' Astrophys.  J.  Suppl.  Ser.  \textbf{208},  20 (2013) doi:10.1088/0067-0049/208/2/19 [arXiv:1212.5226 [astro-ph.CO]].  

%\cite{Planck:2018nkj}
\bibitem{Planck:2018nkj}
N.~Aghanim \textit{et al.} [Planck],
``Planck 2018 results. I. Overview and the cosmological legacy of Planck,''
Astron. Astrophys. \textbf{641}, A1 (2020)
doi:10.1051/0004-6361/201833880
[arXiv:1807.06205 [astro-ph.CO]].
%930 citations counted in INSPIRE as of 24 Jun 2022

LSS

%\cite{Guzzo:2018xbe,DES:2020sjz}
\bibitem{Guzzo:2018xbe}
L.~Guzzo, J.~Bel, D.~Bianchi, C.~Carbone, B.~R.~Granett, A.~J.~Hawken, F.~G.~Mohammad, A.~Pezzotta, S.~Rota and M.~Zennaro,
``Measuring the Universe with galaxy redshift surveys,''
doi:10.1007/978-3-030-01629-6\_1
[arXiv:1803.10814 [astro-ph.CO]].
%1 citations counted in INSPIRE as of 24 Jun 2022

%\cite{DES:2020sjz}
\bibitem{DES:2020sjz}
R.~Cawthon \textit{et al.} [DES],
``Dark Energy Survey Year 3 Results: Calibration of Lens Sample Redshift Distributions using Clustering Redshifts with BOSS/eBOSS,''
Mon. Not. Roy. Astron. Soc. \textbf{513}, 5517 (2022)
doi:10.1093/mnras/stac1160
[arXiv:2012.12826 [astro-ph.CO]].
%14 citations counted in INSPIRE as of 24 Jun 2022

%RW 

\bibitem{Robertson:1929}
Robertson, H. P. , ``On the Foundations of Relativistic Cosmology," Proceedings of the National Academy of Sciences. {\bf 15}, (11): 822–829 (1929) doi:10.1073/pnas.15.11.822. PMC 522564. PMID 16577245.
% 

\bibitem{Robertson:1933}
H.~P.~Robertson,  ``Relativistic Cosmology,'' Rev. Mod. Phys. {\bf 5}, 62 (1933) https://doi.org/10.1103/RevModPhys.5.62.
%

\bibitem{Walker:1935}
%
A.~G. ~Walker,  E. ~A. ~Milne,  ``On the Formal Comparison of Milne's Kinematical System with the Systems of General Relativity,'' Mon. Not. R. Astron. Soc. {\bf 95}, 263 (1935) https://doi.org/10.1093/mnras/95.3.263.

\bibitem{Walker:1937}
Walker, A. G. ,  ``On Milne's theory of world-structure,'' Proceedings of the London Mathematical Society, Series {\bf 42}, (1): 90–127,  (1937) doi:10.1112/plms/s2-42.1.90.
%"

%LT

\bibitem{Morin07}
D.~Morin \textit{Introduction to Classical Mechanics} (Cambridge University Press, 2007). 

%% cosmic time \cite{Weyl:1923,Islam02,Narlikar02,Hobson06,Gron07,Ryder09,CB15,Roos15,Guidry19,Ferrari21}

\bibitem{Weyl:1923}
H.~Weyl, Phys. Z. {\bf 24}, 230 (1923) (English translation: H.~Weyl, ``Republication of: On the general relativity theory'' Gen. Rel. Grav. \textbf{41}, 1661 (2009)).

\bibitem{Islam02}
J.~N.~Islam, \textit{An Introduction to Mathematical Cosmology} (Cambridge University Press, 2001).

\bibitem{Narlikar02}
J. ~V. ~Narlikar,\textit{An Introduction to Cosmology} (Cambridge University Press,  3rd Ed 2002).

\bibitem{Hobson06}
M. ~P. ~Hobson,  G.~P. ~Efstathiou,  and A. ~N. ~Lasenby, \textit{General Relativity: An Introduction for Physicists}  (Cambridge University Press,  2006).

\bibitem{Gron07} \O.~Gr\o n and S.~Hervik, \textit{Einstein's General Theory of Relativity} (Springer, 2007). 

\bibitem{Ryder09} L.~Ryder, \textit{Introduction to General Relativity} (Cambridge University Press,  2009).

\bibitem{CB15}   %\cite{Narlikar02,CB15}
Y.~ Choquet-Bruhat,  \textit{Introduction to General Relativity, Black Holes and Cosmology} (Oxford University Press,  2015). 

\bibitem{Roos15}
M.~Roos, \textit{Introduction to Cosmology} (John Wiley and Sons, 2015).

\bibitem{Guidry19}
M.~Guidry, \textit{Modern General Relativity: Black Holes, Gravitational Waves, and Cosmology} (Cambridge University Press, 2019).

\bibitem{Ferrari21}
V.~Ferrari, L.~Gualtieri, and P.~Pani, \textit{General Relativity and its Applications: Black Holes, Compact Stars and Gravitational Waves} (CRC Press, 2021).

\bibitem{Das93} A.~Das, {\it The Special Theory of Relativity, A Mathematical Exposition} (Springer, 1993). 

\bibitem{Schutz97} J.~Schutz, {\it Independent Axioms for Minkowski Spacetime} (Addison Wesley Longman Limited, 1997).  %ISBN 0-582-31760-6.

%\cite{Lee:2020zts}
\bibitem{Lee:2020zts}
S.~Lee,
``The minimally extended Varying Speed of Light (meVSL),''
JCAP \textbf{08}, 054 (2021)
doi:10.1088/1475-7516/2021/08/054
[arXiv:2011.09274 [astro-ph.CO]].
%5 citations counted in INSPIRE as of 16 Dec 2021

\bibitem{Barrow:1998eh}
J.~D.~Barrow,  ``Cosmologies with varying light speed,'' Phy.~Rev.~D. \textbf{59}, 043515 (1988) doi:10.1103/PhysRevD.59.043515
[arXiv:astro-ph/9811022 [astro-ph]]. 
%36 citations counted in INSPIRE as of 03 Sep 2020

\bibitem{Einstein:1911} A.~Einstein,  ``Über den Einfluß der Schwerkraft auf die Ausbreitung des Lichtes,'' Annalen.~der.~Physik. \textbf{35}, 898–906 (1911) %"" (PDF).. Bibcode:1911AnP...340..898E. 
 doi:10.1002/andp.19113401005.

\bibitem{Dicke:1957} R.~Dicke,  ``Gravitation without a Principle of Equivalence,'' Rev.~Mod.~Phys. \textbf{29}, 363–376 (1957) %"".  %Bibcode:1957RvMP...29..363D. 
doi:10.1103/RevModPhys.29.363.

%**************  VSL as cosmic inflation

\bibitem{Petit:1988} J.~P.~Petit,  ``An interpretation of cosmological model with variable light velocity,'' Mod.~Phys.~Lett.~A. \textbf{3}, 1527–1532 (1988) %"" (PDF). Bibcode:1988MPLA....3.1527P. CiteSeerX 10.1.1.692.9603. 
doi:10.1142/S0217732388001823.

\bibitem{Petit:1988-2} J.~P.~Petit,  ``Cosmological model with variable light velocity: the interpretation of red shifts,'' Mod.~Phys.~Lett.~A. \textbf{3}, 1733–1744 (1988) %"" (PDF). Bibcode:1988MPLA....3.1733P. CiteSeerX 10.1.1.692.9067. 
doi:10.1142/S0217732388002099.
 
\bibitem{Petit:1989} J.~P.~Petit and M.~Viton,  ``Gauge cosmological model with variable light velocity. Comparizon with QSO observational data,'' Mod.~Phys.~Lett.~A. \textbf{4}, 2201–2210 (1989) %"" (PDF).  Bibcode:1989MPLA....4.2201P. 
doi:10.1142/S0217732389002471.
 
\bibitem{Midy:1989} P.~Midy and J~P.~Petit,  ``Scale invariant cosmology,'' Int.~J.~Mod.~Phys.~D \textbf{8}, 271–280 (1989) %"" (PDF). . Archived from the original (PDF) on 2014-07-17. Retrieved 2014-12-24.

%\cite{Moffat:1992ud}
\bibitem{Moffat:1992ud}
J.~W.~Moffat,
``Superluminary universe: A Possible solution to the initial value problem in cosmology,''
Int. J. Mod. Phys. D \textbf{2}, 351-366 (1993)
doi:10.1142/S0218271893000246
[arXiv:gr-qc/9211020 [gr-qc]].
%279 citations counted in INSPIRE as of 03 Sep 2020

\bibitem{Petit:1995ass}
J.~P.~Petit, ``Twin Universe Cosmology,'' Astrophys. \ Sp.\ Science. {\bf 226}, 273 (1995) Bibcode:1995Ap\&SS.226..273P. CiteSeerX 10.1.1.692.7762. doi:10.1007/bf00627375. %"". . 

%\cite{Albrecht:1998ir}
\bibitem{Albrecht:1998ir}
A.~Albrecht and J.~Magueijo,
``A Time varying speed of light as a solution to cosmological puzzles,''
Phys. Rev. D \textbf{59}, 043516 (1999)
doi:10.1103/PhysRevD.59.043516
[arXiv:astro-ph/9811018 [astro-ph]].
%385 citations counted in INSPIRE as of 03 Sep 2020

%\cite{Barrow:1998he}
\bibitem{Barrow:1998he}
J.~D.~Barrow and J.~Magueijo,
``Solutions to the quasi-flatness and quasi lambda problems,''
Phys. Lett. B \textbf{447}, 246 (1999)
doi:10.1016/S0370-2693(99)00008-8
[arXiv:astro-ph/9811073 [astro-ph]].
%101 citations counted in INSPIRE as of 17 Sep 2020

%\cite{Clayton:1998hv}
\bibitem{Clayton:1998hv}
M.~A.~Clayton and J.~W.~Moffat,
``Dynamical mechanism for varying light velocity as a solution to cosmological problems,''
Phys. Lett. B \textbf{460}, 263-270 (1999)
doi:10.1016/S0370-2693(99)00774-1
[arXiv:astro-ph/9812481 [astro-ph]].
%127 citations counted in INSPIRE as of 04 Sep 2020

%\cite{Barrow:1999jq}
\bibitem{Barrow:1999jq}
J.~D.~Barrow and J.~Magueijo,
``Solving the flatness and quasiflatness problems in Brans-Dicke cosmologies with a varying light speed,''
Class. Quant. Grav. \textbf{16}, 1435-1454 (1999)
doi:10.1088/0264-9381/16/4/030
[arXiv:astro-ph/9901049 [astro-ph]].
%83 citations counted in INSPIRE as of 09 Sep 2020

%\cite{Clayton:1999zs}
\bibitem{Clayton:1999zs}
M.~A.~Clayton and J.~W.~Moffat,
``Scalar tensor gravity theory for dynamical light velocity,''
Phys. Lett. B \textbf{477}, 269-275 (2000)
doi:10.1016/S0370-2693(00)00192-1
[arXiv:gr-qc/9910112 [gr-qc]].
%86 citations counted in INSPIRE as of 17 Sep 2020

%\cite{Brandenberger:1999bi}
\bibitem{Brandenberger:1999bi}
R.~H.~Brandenberger and J.~Magueijo,
``Imaginative cosmology,''
[arXiv:hep-ph/9912247 [hep-ph]].
%17 citations counted in INSPIRE as of 08 Oct 2020

%\cite{Bassett:2000wj}
\bibitem{Bassett:2000wj}
B.~A.~Bassett, S.~Liberati, C.~Molina-Paris and M.~Visser,
``Geometrodynamics of variable speed of light cosmologies,''
Phys. Rev. D \textbf{62}, 103518 (2000)
doi:10.1103/PhysRevD.62.103518
[arXiv:astro-ph/0001441 [astro-ph]].
%94 citations counted in INSPIRE as of 17 Sep 2020

%\cite{Gopakumar:2000kp}
\bibitem{Gopakumar:2000kp}
P.~Gopakumar and G.~V.~Vijayagovindan,
``Solutions to cosmological problems with energy conservation and varying c, G and Lambda,''
Mod. Phys. Lett. A \textbf{16}, 957-962 (2001)
doi:10.1142/S0217732301004042
[arXiv:gr-qc/0003098 [gr-qc]].
%15 citations counted in INSPIRE as of 09 Sep 2020


%\cite{Magueijo:2000zt}
\bibitem{Magueijo:2000zt}
J.~Magueijo,
``Covariant and locally Lorentz invariant varying speed of light theories,''
Phys. Rev. D \textbf{62}, 103521 (2000)
doi:10.1103/PhysRevD.62.103521
[arXiv:gr-qc/0007036 [gr-qc]].
%117 citations counted in INSPIRE as of 03 Sep 2020

%\cite{Magueijo:2000au}
\bibitem{Magueijo:2000au}
J.~Magueijo,
``Stars and black holes in varying speed of light theories,''
Phys. Rev. D \textbf{63}, 043502 (2001)
doi:10.1103/PhysRevD.63.043502
[arXiv:astro-ph/0010591 [astro-ph]].
%59 citations counted in INSPIRE as of 03 Sep 2020

%\cite{Magueijo:2003gj}
\bibitem{Magueijo:2003gj}
J.~Magueijo,
``New varying speed of light theories,''
Rept. Prog. Phys. \textbf{66}, 2025 (2003)
doi:10.1088/0034-4885/66/11/R04
[arXiv:astro-ph/0305457 [astro-ph]].
%208 citations counted in INSPIRE as of 03 Sep 2020


%\cite{Magueijo:2007gf}
\bibitem{Magueijo:2007gf}
J.~Magueijo and J.~W.~Moffat,
``Comments on 'Note on varying speed of light theories',''
Gen. Rel. Grav. \textbf{40}, 1797-1806 (2008)
doi:10.1007/s10714-007-0568-2
[arXiv:0705.4507 [gr-qc]].
%26 citations counted in INSPIRE as of 23 Sep 2020

%\cite{Petit:2008eb}
\bibitem{Petit:2008eb}
J.~P.~Petit and G.~d'Agostini,
``Bigravity: A Bimetric model of the Universe with variable constants, inluding VSL (variable speed of light),''
[arXiv:0803.1362 [math-ph]].
%1 citations counted in INSPIRE as of 03 Sep 2020

%\cite{Roshan:2009yb}
\bibitem{Roshan:2009yb}
M.~Roshan, M.~Nouri and F.~Shojai,
``Cosmological solutions of time varying speed of light theories,''
Phys. Lett. B \textbf{672}, 197-202 (2009)
doi:10.1016/j.physletb.2009.01.042
[arXiv:0901.3191 [gr-qc]].
%4 citations counted in INSPIRE as of 23 Sep 2020

\bibitem{Sanejouand:2009} Y.~H.~Sanejouand,  ``About some possible empirical evidences in favor of a cosmological time variation of the speed of light.,'' Europhys.\ Lett.\ \textbf{88}, 59002 (2009) %  Europhys. Lett. 88(5), 59002. https://arxiv.
[arXiv:0908.0249].

%\cite{Nassif:2012dr}
\bibitem{Nassif:2012dr}
C.~Nassif and A.~C.~Amaro de Faria,
``Variation of the speed of light with temperature of the expanding universe,''
Phys. Rev. D \textbf{86} (2012), 027703
doi:10.1103/PhysRevD.86.027703
[arXiv:1205.2298 [gr-qc]].
%2 citations counted in INSPIRE as of 08 Oct 2020

%\cite{Moffat:2014poa}
\bibitem{Moffat:2014poa}
J.~W.~Moffat,
``Variable Speed of Light Cosmology, Primordial Fluctuations and Gravitational Waves,''
Eur. Phys. J. C \textbf{76}, no.3, 130 (2016)
doi:10.1140/epjc/s10052-016-3971-6
[arXiv:1404.5567 [astro-ph.CO]].
%22 citations counted in INSPIRE as of 09 Sep 2020

%\cite{Ravanpak:2017kdg}
\bibitem{Ravanpak:2017kdg}
A.~Ravanpak, H.~Farajollahi and G.~F.~Fadakar,
``Normal DGP in varying speed of light cosmology,''
Res. Astron. Astrophys. \textbf{17}, no.3, 26 (2017)
doi:10.1088/1674-4527/17/3/26
[arXiv:1703.09811 [gr-qc]].
%0 citations counted in INSPIRE as of 17 Sep 2020

%\cite{Costa:2017abc}
\bibitem{Costa:2017abc}
R.~Costa, R.~R.~Cuzinatto, E.~M.~G.~Ferreira and G.~Franzmann,
``Covariant c-flation: a variational approach,''
Int. J. Mod. Phys. D \textbf{28}, no.09, 1950119 (2019)
doi:10.1142/S0218271819501190
[arXiv:1705.03461 [gr-qc]].
%5 citations counted in INSPIRE as of 17 Sep 2020

\bibitem{Nassif:2018pdu}
C.~Nassif and F.~A.~Silva,  ``Variation of the speed of light and a minimum speed in the scenario of an inflationary universe with accelerated expansion,'' Phys.\ Dark\ Universe, {\bf 22}, 127 (2018) [arXiv:2009.05397 [physics.gen-ph]]. %

%\cite{Lee:2023rqv}
\bibitem{Lee:2023rqv}
S.~Lee,
``Constraining minimally extended varying speed of light by cosmological chronometers,''
[arXiv:2301.06947 [astro-ph.CO]].
%1 citations counted in INSPIRE as of 13 Feb 2023

\bibitem{Weinberg:2008}
S.~Weinberg, \textit{Cosmology} (Oxford University Press, 2008). %p. 11. ISBN 978-0-19-852682-7.

 %——————— time dilation from SNe (positive) ——————————

%\cite{Leibundgut:1996qm,SupernovaSearchTeam:1997gem,Foley:2005qu,Blondin:2007ua,Blondin:2008mz}
\bibitem{Leibundgut:1996qm}
B.~Leibundgut,
``Time dilation in the light curve of the distant type ia supernovae sn 1995k,''
Astrophys. J. Lett. \textbf{466}, L21 (1996)
doi:10.1086/310164
[arXiv:astro-ph/9605134 [astro-ph]].
%64 citations counted in INSPIRE as of 09 Dec 2022

%\cite{SupernovaSearchTeam:1997gem}
\bibitem{SupernovaSearchTeam:1997gem}
A.~G.~Riess \textit{et al.} [Supernova Search Team],
``Time dilation from spectral feature age measurements of type ia supernovae,''
Astron. J. \textbf{114}, 722 (1997)
doi:10.1086/118506
[arXiv:astro-ph/9707260 [astro-ph]].
%60 citations counted in INSPIRE as of 09 Dec 2022


%\cite{Foley:2005qu}
\bibitem{Foley:2005qu}
R.~J.~Foley, A.~V.~Filippenko, D.~C.~Leonard, A.~G.~Riess, P.~Nugent and S.~Perlmutter,
``A Definitive measurement of time dilation in the spectral evolution of the moderate-redshift Type Ia supernova 1997ex,''
Astrophys. J. Lett. \textbf{626}, L11-L14 (2005)
doi:10.1086/431241
[arXiv:astro-ph/0504481 [astro-ph]].
%27 citations counted in INSPIRE as of 09 Dec 2022

%\cite{Blondin:2007ua}
\bibitem{Blondin:2007ua}
S.~Blondin and J.~L.~Tonry,
``Determining the Type, Redshift, and Age of a Supernova Spectrum,''
Astrophys. J. \textbf{666}, 1024-1047 (2007)
doi:10.1086/520494
[arXiv:0709.4488 [astro-ph]].
%321 citations counted in INSPIRE as of 09 Dec 2022

%\cite{Blondin:2008mz}
\bibitem{Blondin:2008mz}
S.~Blondin, T.~M.~Davis, K.~Krisciunas, B.~P.~Schmidt, J.~Sollerman, W.~M.~Wood-Vasey, A.~C.~Becker, P.~Challis, A.~Clocchiatti and G.~Damke, \textit{et al.},``Time Dilation in Type Ia Supernova Spectra at High Redshift,''
Astrophys. J. \textbf{682}, 724-736 (2008)
doi:10.1086/589568
[arXiv:0804.3595 [astro-ph]].
%37 citations counted in INSPIRE as of 09 Dec 2022




%————————— time dilation from GRBs ————————


%\cite{Norris:1993hda,Wijers:1994qf,Band:1994ee,Meszaros:1995gj,Lee:1996zu,Chang:2001fy,Crawford:2009be,Zhang:2013yna,Singh:2021jgr}
\bibitem{Norris:1993hda}
J.~P.~Norris, R.~J.~Nemiroff, J.~D.~Scargle, C.~Kouveliotou, G.~J.~Fishman, C.~A.~Meegan, W.~S.~Paciesas and J.~T.~Bonnell,
``Detection of signature consistent with cosmological time dilation in gamma-ray bursts,''
Astrophys. J. \textbf{424}, 540 (1994)
doi:10.1086/173912
[arXiv:astro-ph/9312049 [astro-ph]].
%99 citations counted in INSPIRE as of 09 Dec 2022


%\cite{Wijers:1994qf}
\bibitem{Wijers:1994qf}
R.~A.~M.~J.~Wijers and B.~Paczynski,
``On the nature of gamma-ray burst time dilations,''
Astrophys. J. Lett. \textbf{437}, L107 (1994)
doi:10.1086/187694
[arXiv:astro-ph/9406007 [astro-ph]].
%16 citations counted in INSPIRE as of 09 Dec 2022


%\cite{Band:1994ee}
\bibitem{Band:1994ee}
D.~Band,
``Cosmological time dilation in gamma-ray bursts?,''
Astrophys. J. Lett. \textbf{432}, L23 (1994)
doi:10.1086/187502
[arXiv:astro-ph/9407007 [astro-ph]].
%10 citations counted in INSPIRE as of 09 Dec 2022


%\cite{Meszaros:1995gj}
\bibitem{Meszaros:1995gj}
A.~Meszaros and P.~Meszaros,
``Cosmological evolution and luminosity function effects on number counts, redshift and time dilation of bursting sources,''
Astrophys. J. \textbf{466}, 29 (1996)
doi:10.1086/177491
[arXiv:astro-ph/9512164 [astro-ph]].
%29 citations counted in INSPIRE as of 09 Dec 2022


%\cite{Lee:1996zu}
\bibitem{Lee:1996zu}
T.~T.~Lee and V.~Petrosian,
``Time dilation of batse gamma-ray bursts,''
Astrophys. J. \textbf{474}, 37 (1997)
doi:10.1086/303458
[arXiv:astro-ph/9607127 [astro-ph]].
%17 citations counted in INSPIRE as of 09 Dec 2022


%\cite{Chang:2001fy}
\bibitem{Chang:2001fy}
H.~Y.~Chang,
``Fourier analysis of gamma-ray burst light curves: searching for direct signature of cosmological time dilation,''
Astrophys. J. Lett. \textbf{557}, L85 (2001)
doi:10.1086/323331
[arXiv:astro-ph/0106220 [astro-ph]].
%13 citations counted in INSPIRE as of 09 Dec 2022

%Claiming time dilation in light curves of GRBs with the anticorrelation of a timescale measure and a brightness measure has several difficulties. One difficulty is that this effect is correct only for standard candle sources with a standard duration, which we have evidence that it is not necessarily true (Kim et al. 2001; Chang & Yi 2001). A broad luminosity function and/or an intrinsic spread in the durations could smear out the signature. Another possible difficulty with this anticorrelation is that it could be mimicked by intrinsic properties of the sources (Brainerd 1994, 1997; Yi & Mao 1994; Wijers & Paczy´nski 1994). An additional complication is that an intrinsic redshift of the time profiles from higher energy bands to lower energy bands may be present (Fenimore & Bloom 1995), which would bleach the cosmological signature.

%\cite{Crawford:2009be}
\bibitem{Crawford:2009be}
D.~F.~Crawford,
``No Evidence of Time Dilation in Gamma-Ray Burst Data,''
[arXiv:0901.4169 [astro-ph.CO]].
%8 citations counted in INSPIRE as of 09 Dec 2022

%\cite{Zhang:2013yna}
\bibitem{Zhang:2013yna}
F.~W.~Zhang, Y.~Z.~Fan, L.~Shao and D.~M.~Wei,
``Cosmological Time Dilation in Durations of Swift Long Gamma-Ray Bursts,''
Astrophys. J. Lett. \textbf{778}, L11 (2013)
doi:10.1088/2041-8205/778/1/L11
[arXiv:1309.5612 [astro-ph.HE]].
%17 citations counted in INSPIRE as of 09 Dec 2022


%\cite{Singh:2021jgr}
\bibitem{Singh:2021jgr}
A.~Singh and S.~Desai,
``Search for cosmological time dilation from gamma-ray bursts \textemdash{} a 2021 status update,''
JCAP \textbf{02}, no.02, 010 (2022)
doi:10.1088/1475-7516/2022/02/010
[arXiv:2108.00395 [astro-ph.HE]].
%4 citations counted in INSPIRE as of 09 Dec 2022


%———————————— cosmological time dilation QSO (negative)————

%\cite{Hawkins:2001be,Dai:2012wp}
\bibitem{Hawkins:2001be}
M.~R.~S.~Hawkins,
``Time dilation and quasar variability,''
Astrophys. J. Lett. \textbf{553}, L97 (2001)
doi:10.1086/320683
[arXiv:astro-ph/0105073 [astro-ph]].
%17 citations counted in INSPIRE as of 09 Dec 2022

%\cite{Dai:2012wp}
\bibitem{Dai:2012wp}
D.~C.~Dai, G.~D.~Starkman, B.~Stojkovic, D.~Stojkovic and A.~Weltman,
``Using quasars as standard clocks for measuring cosmological redshift,''
Phys. Rev. Lett. \textbf{108}, 231302 (2012)
doi:10.1103/PhysRevLett.108.231302
[arXiv:1204.5191 [astro-ph.CO]].
%7 citations counted in INSPIRE as of 09 Dec 2022


\bibitem{Lee:2212}
S.~Lee,
``Adiabatic expansion for varying speed of light model,''
[arXiv:2212.03728 [astro-ph.CO]].
%5 citations counted in INSPIRE as of 16 Dec 2021


\end{thebibliography}



%% if required, the content of .bbl file can be included here once bbl is generated
%%\input sn-article.bbl

%% Default %%
%%\input sn-sample-bib.tex%

\end{document}
