% CVPR 2023 Paper Template
% based on the CVPR template provided by Ming-Ming Cheng (https://github.com/MCG-NKU/CVPR_Template)
% modified and extended by Stefan Roth (stefan.roth@NOSPAMtu-darmstadt.de)

\documentclass[10pt,twocolumn,letterpaper]{article}

%%%%%%%%% PAPER TYPE  - PLEASE UPDATE FOR FINAL VERSION
%\usepackage[review]{cvpr}      % To produce the REVIEW version
%\usepackage{cvpr}              % To produce the CAMERA-READY version
\usepackage[pagenumbers]{cvpr} % To force page numbers, e.g. for an arXiv version

% Include other packages here, before hyperref.
\usepackage{graphicx}
\usepackage{amsmath}
\usepackage{amssymb}
\usepackage{booktabs}
\usepackage{tabulary,multirow,overpic,xcolor}
\usepackage{algorithm}
\usepackage{algpseudocode}
\usepackage{listings}
\usepackage{color}
\usepackage{xcolor,colortbl}
%\usepackage{minted}
\usepackage{setspace}

\DeclareMathOperator{\E}{\mathbb{E}}


% It is strongly recommended to use hyperref, especially for the review version.
% hyperref with option pagebackref eases the reviewers' job.
% Please disable hyperref *only* if you encounter grave issues, e.g. with the
% file validation for the camera-ready version.
%
% If you comment hyperref and then uncomment it, you should delete
% ReviewTempalte.aux before re-running LaTeX.
% (Or just hit 'q' on the first LaTeX run, let it finish, and you
%  should be clear).
\usepackage[pagebackref,breaklinks,colorlinks]{hyperref}


% Support for easy cross-referencing
\usepackage[capitalize]{cleveref}
\crefname{section}{Sec.}{Secs.}
\crefname{section}{Section}{Sections}
\crefname{table}{Table}{Tables}
\crefname{table}{Tab.}{Tabs.}


%%%%%%%%% PAPER ID  - PLEASE UPDATE
\def\cvprPaperID{10172} % *** Enter the CVPR Paper ID here
\def\confName{CVPR}
\def\confYear{2023}
\renewcommand{\baselinestretch}{0.961}

\begin{document}

%%%%%%%%% TITLE - PLEASE UPDATE
\title{MAGVLT: Masked Generative Vision-and-Language Transformer}

% \author{Sungwoong Kim$^{*}^{\dagger}$\\
% Korea University\\
% Seoul, South Korea\\
% {\tt\small swkim01@korea.ac.kr}
% \and
% Daejin Jo$^*$\\
% Kakao Brain\\
% Seongnam, South Korea\\
% {\tt\small daejin.jo@kakaobrain.com}
% \and
% Donghoon Lee$^*$\\
% Kakao Brain\\
% Seongnam, South Korea\\
% {\tt\small dhlee@kakaobrain.com}
% \and
% Jongmin Kim$^*$\\
% Kakao Brain\\
% Seongnam, South Korea\\
% {\tt\small jmkim@kakaobrain.com}
% }
% \maketitle
% \def\thefootnote{*}\footnotetext{These authors contributed equally to this work.}\def\thefootnote{\arabic{footnote}}
% text text text\footnote{normal footnote}

\author{Sungwoong Kim$^{1,*,\dagger}$,\qquad Daejin Jo$^{2,*}$,\qquad Donghoon Lee$^{2,*}$,\qquad Jongmin Kim$^{2,*}$\\
$^{1}$Department of Artificial Intelligence, Korea University, Seoul, South Korea\\  $^{2}$Kakao Brain, Seongnam, South Korea\\
{\tt\small swkim01@korea.ac.kr, \{daejin.jo, dhlee, jmkim\}@kakaobrain.com}
}
\maketitle
\def\thefootnote{$*$}\footnotetext{Contributed equally.}
\def\thefootnote{$\dagger$}\footnotetext{Work done at Kakao Brain. Corresponding author.}

%%%%%%%%% ABSTRACT
\begin{abstract}
% While generative modeling on multimodal image-text data has been actively developed with large-scale paired datasets, there have been limited attempts to generate both image and text data by a single model rather than a generation of one fixed modality conditioned on the other modality. In this paper, we explore a unified generative vision-and-language (VL) model that can produce both images and text sequences. Especially, we propose a generative VL transformer based on the non-autoregressive mask prediction, named {\bf MAGVLT}, and compare it with an autoregressive generative VL transformer (ARGVLT). In comparison to ARGVLT, the proposed MAGVLT enables bidirectional context encoding, fast decoding by parallel token predictions in an iterative refinement, and extended editing capabilities such as image and text infilling. For rigorous training of our MAGVLT with image-text pairs from scratch, we combine the image-to-text, text-to-image, and joint image-and-text mask prediction tasks. Moreover, we devise two additional tasks based on the step-unrolled mask prediction and the selective prediction on the mixture of two image-text pairs. Here, the former training task simulates the intermediate refinement steps at the inference time while the latter encourages the mask prediction to be performed by a more appropriate cross-modal attention. Experimental results on various downstream generation tasks of VL benchmarks show that our MAGVLT outperforms ARGVLT by a large margin even with significant inference speedup. Particularly, MAGVLT achieves competitive results on both zero-shot image-to-text and text-to-image generation tasks from MS-COCO by one moderate-sized model (fewer than 500M parameters) even without the use of monomodal data and networks.
While generative modeling on multimodal image-text data has been actively developed with large-scale paired datasets, there have been limited attempts to generate both image and text data by a single model rather than a generation of one fixed modality conditioned on the other modality. In this paper, we explore a unified generative vision-and-language (VL) model that can produce both images and text sequences. Especially, we propose a generative VL transformer based on the non-autoregressive mask prediction, named {\bf MAGVLT}, and compare it with an autoregressive generative VL transformer (ARGVLT). In comparison to ARGVLT, the proposed MAGVLT enables bidirectional context encoding, fast decoding by parallel token predictions in an iterative refinement, and extended editing capabilities such as image and text infilling. For rigorous training of our MAGVLT with image-text pairs from scratch, we combine the image-to-text, text-to-image, and joint image-and-text mask prediction tasks. Moreover, we devise two additional tasks based on the step-unrolled mask prediction and the selective prediction on the mixture of two image-text pairs. Experimental results on various downstream generation tasks of VL benchmarks show that our MAGVLT outperforms ARGVLT by a large margin even with significant inference speedup. Particularly, MAGVLT achieves competitive results on both zero-shot image-to-text and text-to-image generation tasks from MS-COCO by one moderate-sized model (fewer than 500M parameters) even without the use of monomodal data and networks.
\end{abstract}
\vspace{-0.5cm}

%%%%%%%%% BODY TEXT
\section{Introduction}
\label{sec:intro}

Generalizable multimodal modeling has recently made a lot of progress, especially in the field of vision-and-language (VL) modeling \cite{clip, align, dalle, dalle2, imagen, parti, beit3, ding2022cogview2, alayrac2022flamingo, coca, ofa}. In particular, a massive amount of image-text data \cite{wit, cc3m, cc12m, LAION-400M, LAION-5B, coyo-700m} allows robust pretraining of large-scale multimodal VL models that can be easily transferred to various downstream tasks including image captioning \cite{coco, nocaps}, text-guided image generation \cite{nichol2021glide, dalle, dalle2, ding2021cogview, ding2022cogview2, imagen, parti, ERNIE-ViLG2}, visual question answering \cite{vqa2017}, and image-text retrieval \cite{coco, clip, align, flickr30k}. In this respect, many multimodal VL pretraining algorithms have been proposed in the literature, and these can be broadly categorized into either discriminative or generative learning algorithms. Discriminative pretraining such as contrastive learning \cite{clip, align} aims to obtain semantic representations effective for discriminative tasks while generative pretraining learns them to reconstruct an input \cite{wang2021simvlm, vl-beit, beit3, git, anonymous2023connecting, kim2022verse, hao2022language, maskvlm, vl-t5}. The recent growth of model capacity and data size has led to more interest in generative pretraining since it can provide more diverse and improved generalization ability both for VL understanding and VL generation tasks.

While generative VL pretraining has been widely exploited, most existing algorithms focus on representation learning for VL understanding tasks \cite{vl-beit, beit3, maskvlm, vilt, vlmo, uniter, vl-t5} or conditional generation tasks where a generation is performed on one fixed modality conditioned on the other modality \cite{nichol2021glide, dalle, dalle2, ding2021cogview, ding2022cogview2, imagen, parti, ERNIE-ViLG2, wang2021simvlm, git, hao2022language, vl-t5, dai2022enabling, jin2021good, alayrac2022flamingo, coca}. A few algorithms have tried to produce data in both modalities from a single VL model \cite{anonymous2023connecting, kim2022verse}. If one universal model can generate both modalities, it would be beneficial in concentrating training efforts on a single model as well as resource-saving under a resource-constrained deployment. Moreover, we can expect task extension as well as synergetic performance improvement between the modalities from this multimodal generation. Therefore, in this work, we develop a unified generative VL model.

There are two prevalent paradigms of the generative modeling for image and text processing: autoregressive (AR) generative modeling \cite{dalle, ding2021cogview, kim2022verse, parti, git} and non-AR generative modeling \cite{dalle2, imagen, d3pm, anonymous2023connecting}. Many previous algorithms adopt AR modeling due to its excellent generation results and high training scalability, especially with transformer networks. However, AR modeling has restrictions in unidirectional conditioning in that an image needs to be flattened into a 1D sequence by an unnatural ordering. In addition, AR sampling is performed by one-by-one predictions of elements, which incurs very slow generation for a long sequence. Recently, in order to overcome these limitations of AR modeling, non-AR generative modeling based on the mask prediction has been proposed for language \cite{ghazvininejad-etal-2019-mask}, image \cite{chang2022maskgit, Draft-and-Revise}, and video processing \cite{maskvit, phenaki}. Masked modeling is usually employed for representation learning to solve understanding tasks in language, vision, and VL domains. However, with an iterative refinement-based generation and a variable mask ratio during training, it has been shown to be used as a promising generative modeling. In this regard, for our generative VL modeling, we propose {\bf Ma}sked {\bf G}enerative {\bf VL T}ransformer (MAGVLT). In contrast to AR-based generative VL transformer (ARGVLT), the proposed MAGVLT is able to exploit bidirectional conditioning and fast generation through a small number of refinement steps and parallel token predictions.

In specific, MAGVLT can generate any or both of an image and a text sequence conditioned also on any or both of them. Namely, it can perform any kind of task in a form of image-and-text-to-image-and-text (IT2IT), including image-to-text (I2T) and text-to-image (T2I) tasks. Following the previous masked generative modeling \cite{ghazvininejad-etal-2019-mask, chang2022maskgit}, we conduct sampling by iterative denoising based on the masked token prediction and train MAGVLT by the masked token prediction objective with a randomly sampled mask ratio to take into account various denoising steps. Here, to perform robust training of MAGVLT especially with only image-text pairs from scratch, MAGVLT is learned basically by the composition of image-to-text, text-to-image, and joint image-and-text mask prediction objectives. We observe that our cross-modal masking (joint image-and-text mask prediction) during training helps in improving both performances of I2T and T2I tasks over single-modal masking (image-to-text + text-to-image mask predictions). Note that only masked generative modeling used in MAGVLT enables this cross-modal mask prediction during training.

In addition, we propose to use two additional tasks based on the step-unrolled mask prediction and the selective prediction on the mixture of two image-text pairs. The former one is motivated by SUNDAE \cite{savinov2022stepunrolled} and is modified to perform the mask prediction on the unrolled prediction, which simulates the masked input samples encountered at the intermediate refinement steps. On the other hand, the latter one learns to reconstruct the masked tokens in accordance with a selected context between two VL contexts that are mixed as a noisy input. This selective prediction improves cross-modal attention for an accurate generation. 

Through experiments on various downstream VL generation tasks, we empirically demonstrate that our MAGVLT significantly outperforms ARGVLT even with greatly reduced inference time. Especially, to the best of our knowledge, MAGVLT is the first model that obtains strong performances on both zero-shot I2T and zero-shot T2I generation tasks of MS-COCO benchmark \cite{coco} by a single moderate-sized model (fewer than 500M parameters) without relying on monomodal data and networks. Previously, as unified generative VL models, L-Verse \cite{kim2022verse} and UPGen \cite{anonymous2023connecting} have not showed zero-shot I2T results while OFA \cite{ofa} has used monomodal data and also has not showed zero-shot I2T and T2I results. Extensive ablations also validate the contribution of each component for MAGVLT.

To summarize, our main contributions are: (1) a masked generative VL transformer as a unified generative VL model that can produce both images and texts; (2) a robust training on image-text pairs by multiple training tasks that include the cross-modal mask prediction in tandem with the step-unrolled mask prediction and the selective prediction on the mixed context; and (3) an empirical validation of MAGVLT that outperforms the autoregressive model and moreover shows competitive performances on both of zero-shot I2T and T2I generation tasks for the first time without employing extra monomodal data and networks.

% To summarize, our main contributions are:
% \begin{itemize}
% \item Masked generative VL transformer (MAGVLT) is proposed as a unified generative VL model that can produce both images and text sequences.
% \item MAGVLT is robustly learned on image-text pairs by multiple training tasks that include image-to-text, text-to-image, and joint image-and-text mask predictions in tandem with the step-unrolled mask prediction and the selective prediction on the mixed context.
% \item MAGVLT empirically outperforms the autoregressive model and moreover shows competitive performances on both of zero-shot image-to-text and text-to-image generation tasks for the first time without employing extra monomodal data and networks.
% \end{itemize}

\section{Related Work}
\label{sec:background}

\subsection{Multimodal Vision-and-Language Modeling}

There has been a lot of research on multimodal VL training. Especially, in recent years, large-scale image-text datasets have made great progress in VL pretraining with various training objectives. For example, image-text matching \cite{lxmert, vilbert, e2e-vlp} and contrastive learning on image-text data \cite{coca, align, clip, ufo, Florence} has been widely used for discriminative representation learning. 

Since BERT \cite{bert} has shown impressive performances on many natural language processing tasks, masked language modeling has been widely adopted for VL pretraining. In particular, \cite{vl-t5} formulated multiple VL tasks as a text generation task and applied masked token prediction objective. \cite{vl-beit, beit3, m3ae, MultiMAE, maskvlm} have proposed to use a unified masked data modeling with a shared multimodal transformer while several algorithms have combined a number of objectives including image-text matching, contrastive VL loss, and masked language modeling \cite{dai2022enabling, vilt, vlmo, VinVL}. However, most of these masked VL pretraining algorithms have been developed for VL understanding tasks.

Meanwhile, AR generative modeling has also recently received lots of interest for VL pretraining due to its powerful generalization ability. While many algorithms have proposed to utilize the generative language modeling paradigm for VL understanding tasks \cite{git, hao2022language, wang2021simvlm, jin2021good, alayrac2022flamingo}, recent large-scale AR transformers trained on a large amount of image-text pairs have shown powerful performances for text-guided image generation \cite{dalle, parti, ding2021cogview, ding2022cogview2}. A few generative models based on AR decoding have been shown to produce both images and text sequences \cite{kim2022verse, ofa}, however, they have not shown competitive performances on both modalities.

\subsection{Non-Autoregressive Generative Modeling}
\begin{figure}[t]
\begin{minipage}{.47 \textwidth}\centering
    \includegraphics[width=1\textwidth]{figures/framework.jpg}
    \caption{\label{fig:framework} Masked generative VL training via three multimodal masked token prediction tasks (I2T, T2I, IT2IT). Here, we represent VQ-GAN and BPE as image encoder and text encoder. MAGVLT predicts only masked tokens according to each task.}
\end{minipage}
\vspace{-0.4cm}
\end{figure}

Non-AR generative models have been increasingly used to lift certain limitations of AR models such as unidirectional attention and slow decoding. Among them, diffusion-based models have recently shown remarkable performances on the task of text-guided image generation \cite{nichol2021glide, dalle2, imagen, ERNIE-ViLG2, ldm}. However, for text generation, diffusion models \cite{Hoogeboom2021, d3pm} are still limited in achieving competitive performances compared to AR models.

Similar to diffusion-based generative models, masked generative models also perform iterative refinements for data generation and simulate various denoising steps during training. On top of that, masked generative models often conduct fewer refinement steps leading to faster generations. Therefore, masked generative models have recently been employed for language \cite{ghazvininejad-etal-2019-mask}, image \cite{chang2022maskgit, Draft-and-Revise}, and video processing \cite{maskvit, phenaki}. However, there have been almost no masked generative models that can generate both text and image data. Very recently, a concurrent work \cite{anonymous2023connecting} has tried to combine representation and generative learning for VL tasks into a single model that is based on masked token prediction. However, their generation performances are very poor on both modalities in contrast to the strong performances of MAGVLT on both modalities, and furthermore, our MAGVLT differs in that we use multiple cross-modal tasks for robust generative training on image-text pairs.

\section{Masked Generative Vision-and-Language Transformer} \label{sec:method}
% In this section, we propose \textit{MAGVLT} which is able to generate visual and text tokens by learning three different mask prediction based on multi-modal tasks.
% Moreover, two additional objectives are presented for further improvement.
% \subsection{Notation}
% Let $X$ denotes an instance of a paired image $\text{I}$ and text $\text{T}$ in dataset $\mathcal{D}$, $S_{\text{task}} \in \{{\scriptstyle T2I, I2T, IT2IT}\}$ denotes a set of pretraining tasks.
% where \textit{T2I}, \textit{I2T}, and \textit{IT2IT} represent \textit{text-to-image}, \textit{image-to-text}, and \textit{image-and-text-to-image-and-text}, respectively.
% Each task will be described in \cref{sec:multitask} details.
% \subsection{Masked Generative Vision-and-Language Transformer} \label{sec:magvlt}
% In a recent progress of development generative vision-language models, non-autoregressive based methods has shown promising direction\cite{} especially in multi-modal generation due to its efficient iterative decoding with a rich bi-directional encoding.
\subsection{Masked Image-Text Modeling}

MAGVLT is based on the previous masked generative modeling for image and language processing \cite{chang2022maskgit, ghazvininejad-etal-2019-mask}. Given an image-text pair $(I,T)$, the image input $I$ is mapped to latent tokens $X=[x_i]_{i=1}^{N_I}$ by VQ-GAN \cite{Esser2021TamingTF}, where $N_I$ is the number of image tokens (\eg., $16 \times 16$), and the text sequence $T$ is also converted to the tokenized sequence $Y=[y_j]_{j=1}^{N_T}$ by byte pair encoding (BPE) \cite{bpe}, where $N_T$ is the number of text tokens (\eg, $64$). Then, $(X,Y)$ is fed into a bidirectional transformer with full attention. In contrast to the causal attention in the AR transformer, this full attention allows to fully utilize the whole context information for decoding and therefore leads to better output predictions. 
% In addition, MAGVLT is a decoder-only model that is similar to an encoder-decoder model with shared parameters and full cross attention.
Here, to indicate which modality is processed by the shared transformer, we prepend learnable special tokens representing the modality such as {\fontfamily{qcr}\selectfont <BOI>} (\textit{begin-of-image}) and {\fontfamily{qcr}\selectfont <BOT>} (\textit{begin-of-text}) to the input tokens of each modality.

% We develop MAGVLT based on previous non-AR image generative model \cite{chang2022maskgit} and language generative model \cite{ghazvininejad-etal-2019-mask}. 
% For cross-modal generation, we extend MaskGIT \cite{chang2022maskgit} to allow text as its input condition and output as well.
% We follow the two-stage recipe that uses VQ-encoder \cite{Esser2021TamingTF} in order to transform an input image to fixed $N_{img}$ length tokens, denoted $X.img=VQ(\text{I})= [x_i^{img}]_{i=1}^{N_{img}}$.
% Likewise, we use an off-the-shelf natural language tokenizer \cite{wolf-etal-2020-transformers} in order to transform an input raw text to variable length $N_{txt}$ tokens, denoted $X.txt=\text{Tokenizer}(\text{T})=[x_i^{txt}]_{i=1}^{N_{txt}}$.
% \\~\\
% \noindent\textbf{Training}. 

\begin{figure}[t]
\begin{minipage}{.47 \textwidth}\centering
    \includegraphics[width=1\textwidth]{figures/sampling.jpg}
    \caption{\label{fig:sampling} Iterative decoding process of MAGVLT for T2I task.}
\end{minipage}
\vspace{-0.3cm}
\end{figure}
For training MAGVLT on image-text pairs, we first sample a mask ratio $\gamma(r) \in(0,1]$ where $r \in[0,1)$ indicates the simulated refinement step ratio and is also uniformly sampled considering various steps during generation. Then, we uniformly sample $\lceil \gamma \cdot N \rceil$ tokens where $N=N_{I}+N_{T}$ and replace them with a special token {\fontfamily{qcr}\selectfont <MASK>}. Here, we separately apply this masking for each modality such that $\lceil \gamma_{I} \cdot N_{I} \rceil$ image tokens and $\lceil \gamma_{T} \cdot N_{T} \rceil$ text tokens are masked. Let $M_I=[m^I_i]_{i=1}^{N_I}$ and $M_T=[m^T_j]_{j=1}^{N_T}$ be the resulting binary image mask and binary text mask, respectively, such that $x_i$ is replaced with {\fontfamily{qcr}\selectfont <MASK>} if $m^I_i = 1$ while $y_j$ is replaced with {\fontfamily{qcr}\selectfont <MASK>} if $m^T_j = 1$. Note that we set $\gamma_{I}(\cdot)$ and $\gamma_{T}(\cdot)$ as the cosine function and the linear function, respectively, following \cite{chang2022maskgit} and \cite{ghazvininejad-etal-2019-mask}.

As shown in \cref{fig:framework}, MAGVLT is trained basically by the composition of three mask prediction losses: ${\mathcal L_{\text{I2T}}}$ for the I2T task, ${\mathcal L_{\text{T2I}}}$ for the T2I task, and ${\mathcal L_{\text{IT2IT}}}$ for the IT2IT task. And these losses are defined by the negative log-likelihood of the masked tokens:
\begin{eqnarray}
\label{eq:basic_loss}
{\mathcal L_{\text{I2T}}}\!\!&\!\!=\!\!&\!\!\! - \!\!\!\!\mathop{\mathbb{E}}_{(X,Y)\in {\mathcal D}} \bigg [ \sum_{\forall j \in [1, N_T], m^T_j=1} \!\!\!\!
\log p(y_j | Y_{{\bar M_T}}, X)
\bigg ], \\
{\mathcal L_{\text{T2I}}} \!\!&\!\!=\!\!&\!\!\! - \!\!\!\!\mathop{\mathbb{E}}_{(X,Y)\in {\mathcal D}} \bigg [ 
\sum_{\forall i \in [1, N_I], m^I_i=1} \!
\log p(x_i | X_{{\bar M_I}}, Y)
\bigg ], \\
{\mathcal L_{\text{IT2IT}}} \!\!&\!\!=\!\!&\!\!\! - \!\!\!\!\mathop{\mathbb{E}}_{(X,Y)\in {\mathcal D}} \bigg [
\sum_{\forall j \in [1, N_T], m^T_j=1} \!
\log p(y_j | Y_{{\bar M_T}}, X_{{\bar M_I}})\nonumber\\
&&~~~~~~~+\!\!\sum_{\forall i \in [1, N_I], m^I_i=1}
\log p(x_i | X_{{\bar M_I}}, Y_{{\bar M_T}})
\bigg ],
\end{eqnarray}
where ${\mathcal D}$ is the training dataset, and $X_{{\bar M_I}}$ and $Y_{{\bar M_T}}$ denote the masked image and the masked text, respectively.
% During training, a number of masked tokens $N_{mask}^{\ast}$ is sampled by a mask scheduling function $\gamma_{\ast}(r) \in(0,1]$ and executing  $N_{mask}^{\ast}=\gamma_{\ast}(r)\cdot N_{\ast}$ where $\ast$ is a variable of modality (i.e. $img$ or $txt$). 
% Here, we follow the cosine scheduling function $\gamma_{img}(r)$ as in \cite{chang2022maskgit} and the linear scheduling function $\gamma_{txt}(r)$ as in \cite{ghazvininejad-etal-2019-mask}.
% $N_{mask}^{\ast}$ tokens are replaced randomly with a special token {\fontfamily{qcr}\selectfont <MASK>} in $X.\ast$ and the $N_{\ast}-N_{mask}^{\ast}$ other tokens will be left intact.
% Denote $X_{\bar{M}(X, \tau)}$ the masked result which is defined by a mask function $\bar{M}(X, \tau)$ where $\tau$ is a sampled task from $S_{\text{task}}$.
% % for example, $X.txt$ will have no mask while $X.img$ will be masked in T2I task. 
% MAGVLT is trained to minimize the negative log-likelihood of the masked tokens:
% \begin{eqnarray} \label{eq:mask_loss}
% {\mathcal L_{mask}} = - \displaystyle \mathop{\mathbb{E}}_{\substack{X \sim \mathcal{D},\\ \tau \sim S_{\text{task}}}} \bigg[ log p \bigg ( \hat{Y}|X_{\bar{M}(X, \tau)} \bigg) \bigg ],
% \end{eqnarray}
% where $\hat{Y}$ is the target to be predicted.
% \\~\\
% \noindent\textbf{Inference}. 
\begin{figure}[t]
\begin{minipage}{.47 \textwidth}\centering
    \includegraphics[width=1 \textwidth]{figures/unroll.jpg}
    \caption{\label{fig:unroll} Step-unrolled mask prediction for I2T task. The one-step unrolled sequence is re-masked and then forwarded to the model. Throughout this process, the model faces inputs it would encounter during iterative inference.}
\end{minipage}
\vspace{-0.4cm}
\end{figure}

\subsection{Iterative Inference}

During inference, the target sequence is predicted by iterative decoding. The mask ratio is defined as a function of the decoding steps as $\gamma(\frac{k}{K})$ where $k \in \{0, 1, ..., K-1\}$ and $K$ is the total number of iterations.
% The mask ratio $r$ is defined as $\frac{N_{iter}^{\ast}-t}{N_{iter}^{\ast}}$ where $N_{iter}^{\ast}$ is the total number of iteration and $t$ is the step of iteration (we fix $N_{iter}^{img} = 10$ and $N_{iter}^{txt} = 12$ for all our
% experiments). 
% % The modality symbol $\ast$ will be omitted left in this paper for simplicity.
For the first iteration ($k=0$), all the tokens are masked, and the model predicts all the tokens in parallel. 
For the next $k$th iteration, the most $\lceil \gamma(\frac{k}{K})N \rceil$ unconfident tokens are masked out and predicted again. 
This process is described in \cref{fig:sampling}. 
Here, it is noted that following the masking strategy in \cite{chang2022maskgit}, unmasked image tokens in previous steps are excluded in computing confidences and accordingly will never be masked again. On the other hand, unmasked text tokens can be selected as masked tokens again in the following iterations. Since the number of refinement steps is generally small (\eg, ~10), this iterative decoding with parallelizable predictions is significantly faster than the autoregressive decoding, especially when the number of tokens is very large.
% At each later iteration, the model predicts all the tokens simultaneously but only unmasks the most confident $N_{mask}^{\ast}$ tokens. 
% The confidence of the unmasked tokens are stored with their index by a storing scheme and then use it in the next iteration.
% Here, we use different storing scheme according to target generation.
% In image generation, following \cite{chang2022maskgit}, the maximum confidence (i.e. 1) is stored after a token is unmasked. Thus, it will never be masked again.
% On the other hand, in text generation, the confidence of the unmasked token is stored as in \cite{ghazvininejad-etal-2019-mask}.
% Thus, the unmasked token can be masked again during iteration.
\\[3pt]
\noindent\textbf{Target Length Prediction}. 
Since the length of text sequence is varied in contrast to the fixed number of image tokens, non-AR models like MAGVLT need to perform target length prediction for text generation. We follow the length predictor proposed in \cite{ghazvininejad-etal-2019-mask} where the output of {\fontfamily{qcr}\selectfont <BOT>} that is located between $X$ and $Y$ is the predicted text length. At test time, the text sequence is generated after the length is predicted. The loss of the auxiliary target length predictor, ${\mathcal L}_{\text{TL}}(N_T, {\hat N}_T)$, is defined by the cross entropy loss between the ground-truth length $N_T$ and the predicted length ${\hat N}_T$ given the maximum possible number of text tokens.
% For text generation, non-AR models like MAGVLT do not explicitly learn to predict an end of sequence.
% A common way is to use an auxiliary target length predictor which is parameterized on encoder's outputs of a special token such as {\fontfamily{qcr}\selectfont <EOS>} (end of sentence). 
% We follow the length predictor of  \cite{ghazvininejad-etal-2019-mask}. 
% We use an intermediate special token {\fontfamily{qcr}\selectfont <BOT>} which represents \textit{begin-of-text}, and is located between $X.img$ and $X.txt$.
% The loss of the auxiliary target length predictor is denoted as ${\mathcal L}_{TL}$.
\begin{figure}[t]
\begin{minipage}{.47 \textwidth}\centering
    \includegraphics[width=.7\textwidth]{figures/bias_stat.png}
    \caption{\label{fig:bias_stat}  An example of a paired data with masked caption \textbf{(Left)}, and biased statistics of word compositions  \textbf{(Right)}.}
\end{minipage}
\vspace{-0.4cm}
\end{figure}
\begin{figure*}[t]
\begin{minipage}{1 \textwidth}\centering
    \includegraphics[width=.95\textwidth]{figures/mixsel.jpg}
    \caption{\label{fig:MixSel} \textit{MixSel} learning tasks corresponding to three multimodal tasks.}
\end{minipage}
\vspace{-0.4cm}
\end{figure*}
\subsection{Step-Unrolled Mask Prediction} \label{sec:unroll}
Although a variable mask ratio during training reflects various intermediate refinement steps, there still exists a gap between a corruption on the target tokens at training time and a corruption on the partially predicted tokens at test time.
SUNDAE \cite{savinov2022stepunrolled} tries to resolve this issue by optimizing the model conditioned on a corrupted target sequence which is sampled through one step generative unrolling during training and achieves significant performance improvements of the non-AR autoencoder for text generation.
% More specifically, it first corrupts the original text tokens with sampled tokens from the vocabulary and then re-generates another noisy tokens by sampling from the denoising autoencoder. The resulting noisy tokens are used as a training input for updating the denoising autoencoder.

Here, we adopt and modify this step-unrolled denoising as an additional training task for MAGVLT.
Since MAGVLT is based on the masked token prediction, we re-mask the one-step predicted sequence where the mask ratio is reduced from the previous mask ratio and then predict the re-masked tokens by MAGVLT. We call this task as step-unrolled mask prediction, dubbed UnrollMask. We apply UnrollMask only to the I2T and T2I training tasks to maintain the uncorrupted cross-modal context. \cref{fig:unroll} visually depicts this UnrollMask especially for the I2T task. We denote the UnrollMask loss as ${\mathcal L_{\text{UM}}}$, and for example ${\mathcal L_{\text{UM}}}$ on the I2T task can be defined as
\begin{eqnarray}
\label{eq:unroll_loss}
\!\!\!\!\!{\mathcal L_{\text{UM,I2T}}} \!\!&\!\!=\!\!&\!\!\! - \!\!\!\!\!\!\!\mathop{\mathbb{E}}_{(X,Y)\in {\mathcal D}} \!\! \bigg [
\sum_{\forall j \in [1, N_T ], {m^{T}_j}^{(+1)}=1} \!\!\!\!\!\!\!\!\!\!\!\!\!\!\!
\log p(y_j | {\hat Y}^{(+1)}_{{\bar M_T}^{(+1)}}, X)
\bigg ],
\end{eqnarray}
where ${\hat Y}^{(+1)}_{{\bar M_T}^{(+1)}}$ indicates the re-masked one-step unrolled prediction of $Y_{\bar M_T}$.
% Note that we use different unmasking scheme according to given task during generation as described in \cref{sec:magvlt}, therefore, we apply the different unroll masking for the task.
% In specific, the masked tokens according to $r^{+1}$ are sampled within the unmasked tokens of $X^{+1}$ in T2I.
% On the other hand, the masked tokens according to $r^{+1}$ are sampled within all the tokens of $X^{+1}$ in I2T.
% The step-unrolled mask prediction loss is defined:
% \begin{eqnarray} \label{eq:unroll_loss}
% {\mathcal L}_{UM} = - \displaystyle \mathop{\mathbb{E}}_{\substack{X \sim \mathcal{D},\\ \tau \sim S_{\text{task}}}} \bigg[ log p \bigg ( \hat{Y}|X_{\bar{M}(X^{+1}, \tau)} \bigg) \bigg ].
% \end{eqnarray}

\subsection{Selective Prediction on Mixed Context} \label{sec:mixsel}
% Since many tokens to be predicted are highly correlated to the given masked input sequence $X_{\bar{M}}$, the model is often exposed a situation where no penalty is incurred even if the model ignores the context during in training, and the mask prediction of learned model can easy to be biased by the masked sequence regardless of the context.
% It can be seen as an short-cut and sometimes useful property, but such bias issue can suffer the generalization ability of the model.
% For example, as shown in \cref{fig:bias_stat}, although the model should predict the masked word token \textit{cat} by considering the given image, the model is easy to output \textit{dog} rather than \textit{cat} since most of training data likely have more samples about \textit{dog jumps} rather than \textit{cat jumps} on grass field. 
In this multimodal generative modeling, the model often ignores the cross-modal context and produces an output that is biased to the within-modal statistics. For example, in the I2T task of \cref{fig:bias_stat}, the model should predict the masked word token as `\textit{cat}' by the given image, however, the model often rather outputs `\textit{dog}' since `\textit{dog jumps}' is more likely occurred than `\textit{cat jumps}' before the text of `\textit{on grass field}' in the set of training text sequences.

Thus, in order to reduce such bias, we propose a simple yet effective additional learning task, named selective prediction on the mixed context (MixSel), which is described in \cref{fig:MixSel}.
As shown in the figure, two different input contexts are mixed in a half-and-half concatenated manner, and one of them is randomly selected to be the target context in generation. Here, a special token is appended to inform the selected context, for instance {\fontfamily{qcr}\selectfont <LEFT>} or {\fontfamily{qcr}\selectfont <RIGHT>} is used for the horizontally combined image or the concatenated text sequence while {\fontfamily{qcr}\selectfont <TOP>} or {\fontfamily{qcr}\selectfont <BOTTOM>} is used for the vertically combined image. Also, when two different text sequences are concatenated, another special token, {\fontfamily{qcr}\selectfont <SEP>}, is inserted between them.
% In I2T, as shown in the top of the figure, the two different images is combined in half-and-half manner vertically or horizontally and the masked input sequence is generated according to the selected context. 
% For example in the top-left of the figure, the bottom half image is selected and the caption according to the bottom image was masked.
% In T2I, as shown in the middle of the figure, the two different texts is combined with a special token denoted {\fontfamily{qcr}\selectfont <SEP>}. In this task, the model should predict the masked image tokens according to the selected caption.
% In IT2IT, as shown in bottom of the figure, the model should predict the masked selected image tokens and text tokens simultaneously.
The MixSel objective is denoted as ${\mathcal L_{\text{MS}}}$, and for instance ${\mathcal L_{\text{MS}}}$ on the I2T task can then be defined as
\begin{align}
\label{eq:hnh_loss}
&{\mathcal L_{\text{MS, I2T}}} =\nonumber\\~~ 
&- \!\!\!\!\mathop{\mathbb{E}}_{(X,Y)\in {\mathcal D}} \bigg [
\sum_{\forall j \in [1, N_T], m^T_j=1} \!
\log p(y^{\ell}_j | {\hat Y}^{\ell}_{{\bar M_T}}, \phi(X^1, X^2))
\bigg ],
\end{align}
where $\phi$ is the mixture function on the two images $X^1$ and $X^2$, and $\ell \in \{1,2\}$ represents the selected context. 

% From this MixSel training task, the model is able to attend more carefully to the appropriate span of the cross-modal context since the input context is perturbed and we randomly select a target between the two instances. Namely, MixSel allows to improve the accuracy of the cross-modal attention by mixing the original cross-modal content with randomly unrelated one.

From this MixSel training task, the model is able to attend more carefully to the appropriate span of the cross-modal context and improve the accuracy of the cross-modal attention by mixing the original cross-modal content with randomly unrelated one. This could make the model to utilize the cross-modal context more trustfully
and hence more often in generation as the training progresses. Therefore, MixSel indeed helps in reducing the overlooking of the cross-modal context and circumventing the within-modal bias problem in test-time generation. Note that it is different from the previous mix-based data augmentation techniques \cite{mixup, cutmix, augmix} in that we retain the information of the original contexts entirely and randomly select the target one for generation. Moreover, although MixSel training is relevant to classifier-free guidance (CFG) \cite{ho_cfg} in that both try to strengthen the effect of the condition, MixSel does not have to perform the forward processing twice at
test time. Also, we can adapt CFG along with MixSel training.

% can not be done in previous data augmentation techniques \cite{} since they are limited to perturb the input statistics without removing semantic information of data.
% Moreover, when the model starts to depend the given masked target sequence $X_{\bar{M}}$ only the model would be suffered rapidly by the MixSel objectives, and therefore ${\mathcal L_{MS}}$ can regularize the model learning simply by combining it to the loss. 

% \begin{figure}[t]
% \begin{minipage}{.5\textwidth}\centering
%     \includegraphics[width=.98\textwidth]{figures/mix.pdf}
%     \caption{\label{fig:mix_code} Pytorch-like pseudocode of the mixture function of \textit{MixSel}}
% \end{minipage}
% \end{figure}

\subsection{Multitask Pretraining}\label{sec:multitask}
As we mentioned above, MAGVLT is basically trained via three types of multimodal tasks: I2T, T2I, and IT2IT. 
During training, a task $\tau \in \{\text{I2T, T2I, IT2IT}\}$ is sampled from the categorical distribution with the predefined sampling probability $p_{\tau}$ for each iteration (batch-wise), and then apply the associated mask prediction loss ${\mathcal L_{\text{mask}, \tau}} \in \{{\mathcal L_{\text{I2T}}}, {\mathcal L_{\text{T2I}}}, {\mathcal L_{\text{IT2IT}}}\}$. Along with this mask prediction loss, we also add the target length prediction loss ${\mathcal L_{\text{TL}, \tau}}$, the UnrollMask loss ${\mathcal L_{\text{UM}, \tau}}$, and the MixSel loss ${\mathcal L_{\text{MS}, \tau}}$ according to the selected task $\tau$. Overall, the final objective of MAGVLT with respect to $\tau$ is
\begin{eqnarray} 
\label{eq:final_loss}
 {\mathcal L}_{\tau} &=& {\mathcal L}_{\text{mask}, \tau} + \lambda_{\text{TL}}{\mathcal L}_{\text{TL}, \tau} \cdot {\mathbb I} [\tau \neq \text{T2I}] \nonumber \\ &&+ \lambda_{\text{UM}}{\mathcal L}_{\text{UM}, \tau} \cdot {\mathbb I}[\tau \neq \text{IT2IT}] + \lambda_{\text{MS}} {\mathcal L}_{\text{MS}, \tau},
\end{eqnarray}
where ${\mathbb I}[\cdot]$ is the indicator function, and $\lambda_{\text{TL}}$, $\lambda_{\text{UM}}$, and $\lambda_{\text{MS}}$ are relative loss weights. Here, we fix $\lambda_{\text{TL}}=0.01$, $\lambda_{\text{UM}}=1.0$, and $\lambda_{\text{MS}}=0.5$ for all our experiments.

\section{Experiment}
\label{sec:exp}
In this section, we elaborate the experiments on the VL generation tasks with extensive ablation studies to identify the contribution of each factor in the proposed algorithm. Official codes will be available\footnote{\url{https://github.com/kakaobrain/magvlt}}.
\subsection{Experimental Setup}
\noindent\textbf{Model}.
VLTs (\ie ARGVLT and MAGVLT) have 447M parameters (24 layers, 1024 hidden dimension, and 8 attention heads) including VQ-GAN in total.
We also perform experiments about scaling of VLTs, and the results are presented in Appendix.
As an image encoder, VQ-GAN \cite{Esser2021TamingTF} converts a 256$\times$256 image into 16$\times$16 tokens with 16,384 codebook size. 
For text sequence, we adopt the BPE tokenizer \cite{bpe} used in CLIP \cite{clip} with 49,408 vocabulary size. We fix the text sequence length to 64.
% since it is long enough to cover the length of most ground-truth captions exist in the dataset we used.
% 모델spec, image token 길이, text token 길이
\\[3pt]
\noindent\textbf{Dataset}. 
We pretrain ARGVLT and MAGVLT from scratch using paired image-text datasets. 
Our pretraining data consists of Conceptual Captions 3M (CC3M)\cite{cc3m}, Conceptual Captions 12M (CC12M)\cite{cc12m}, SBU Caption\cite{NIPS2011_SBU}, and Visual Genome\cite{krishna_VG} datasets.
Together, there are about 17M image-text pairs.
\\[3pt]
\noindent\textbf{Pretraining}. 
There are many options to train VLTs. 
Note that T2I and I2T are available for both ARGVLT and MAGVLT while IT2IT is only available for MAGVLT.
We experiment various subsets of the multimodal tasks in order to investigate the effectiveness of each task.
In specific, we pretrain ARGVLT on three different subsets which consist of  \textit{T2I only}, \textit{I2T only}, and \textit{T2I} \& \textit{I2T}. 
Likewise, we pretrain MAGVLT on three different subsets which consist of  \textit{T2I only}, \textit{I2T only}, and \textit{T2I} \& \textit{I2T} \& \textit{IT2IT}.
More details of pretraining will be found in Appendix.
\\[3pt]
\noindent\textbf{Evaluation}. 
We compare generative VL models based on cross-modality generation tasks especially in \textit{zero-shot} settings in order to evaluate the generalization ability of the proposed method.
We also provide more results including finetuning VLTs on downstream tasks in Appendix.
% zeroshot setup, fintuning, 다양한 downstream task에 대한것  은 appendix
\\[3pt]
\noindent\textbf{Sampling}. 
For text-to-image generation, following\cite{kim2022verse},  we obtain 32 samples from each trained VLT (\ie ARGVLT and MAGVLT) and calculate similarity scores between the sampled images and the conditioning text by CLIP \cite{clip} to select a top ranked image (\textit{clip reranking}). 
Likewise, for image-to-text generation (image captioning), we produce 64 samples from each trained VLT and select a top ranked text by CLIP scores. The number of refinement steps is set to $K=10$ and $K=12$ for image generation tasks and text generation tasks, respectively.

\begin{figure}[t]
\begin{minipage}{.45 \textwidth}\centering
    \includegraphics[width=.95\textwidth]{figures/t2i.png}
    \caption{\label{fig:t2i} Text-to-Image samples on MS-COCO captions. The images in the second column are sampled from MAGVLT trained without \textit{UnrollMask} and \textit{MixSel}.
    MAGVLT generated more appropriate images on the corresponding caption.
    More samples will be found in Appendix.
    }
\end{minipage}
\vspace{-0.4cm}
\end{figure}

% \begin{figure}[t]
% \begin{minipage}{.45 \textwidth}\centering
%     \includegraphics[width=.95\textwidth]{figures/inpainting.png}
%     \caption{\label{fig:inpainting} Image inpainting samples on MS-COCO dataset. MAGVLT generated the masked parts to be more blended with the surrounding context, and more proper to the captions.
%     More samples will be found in Appendix.
%     }
% \end{minipage}
% \vspace{-0.15cm}
% \end{figure}



\bgroup
\def\arraystretch{1.05}%
\begin{table}[t]
\centering
\small
\begin{tabular}{lccc}
\hline
Model & FID ($\downarrow$) & IS ($\uparrow$) & Speed\\
\hline
\rowcolor[gray]{0.85}\multicolumn{4}{l}{\textit{\textbf{AR based}}} \\
CM3-Medium (2.7B) \cite{aghajanyan2022cm3} & 36.78  & -  & - \\
DALL-E (12B) \cite{dalle} & 27.5  & 17.9 & - \\
CogView (4B) \cite{ding2021cogview} & 27.1  & 18.2  & -\\
CogView2 (6B) \cite{ding2022cogview2} & 24.0  & 22.4  & - \\
Parti-350M (350M) \cite{parti} & 14.10  & -  & - \\
Make-A-Scene (4B) \cite{gafni2022make} & \bf{11.84}  & -  & - \\
ARGVLT (\textit{T2I only}) (447M)  & 21.80 & 19.27 & $1.00\times$ \\
\hline
\rowcolor[gray]{0.85}\multicolumn{4}{l}{\textit{\textbf{Non-AR based}}} \\ 
GLIDE (3.5B) \cite{nichol2021glide} & 12.24  & - & -  \\
DALL-E-2 (6.5B) \cite{dalle2} & 10.39  & - & - \\
Imagen (4.9B) \cite{imagen} & 7.27  & - & - \\
ERNIE-ViLG 2.0 (24B) \cite{ERNIE-ViLG2} & {\bf 6.75}  & - & - \\
{\bf MAGVLT (\textit{T2I only}) (447M)}  & 10.74 & 23.94 & $8.12\times$ \\
\hline
\rowcolor[gray]{0.85}\multicolumn{4}{l}{\textit{\textbf{Available for both T2I \& I2T}}} \\ 
UPGen (307M) \cite{anonymous2023connecting} & 65.25  & -  & - \\
L-Verse (500M) \cite{kim2022verse} & 37.2  & -  & - \\
ARGVLT (447M) & 16.93 & 22.50 & $1.00\times$ \\
% ARVLT-L (IT2IT) & -& - \\
{\bf MAGVLT (447M)}  & \bf{12.08} & \bf{22.75} & $8.12\times$ \\
\hline
\end{tabular}
\caption{\textit{Zero-shot} T2I results on MS-COCO validation set. Here, we compute FID and IS on a subset of 30,000 captions sampled from COCO validation. \label{tab:t2i}}
\end{table}
\egroup



\begin{center}
    \begin{tabular}{cccc}
        & PSNR$\uparrow$ & SSIM$\uparrow$ &LPIPS$\downarrow$ \\\hline
        PnP-HVAE  & $\mathbf{29.54}$ & $\mathbf{0.93}$ & $\mathbf{0.06}$\\
        GS-PNP & $28.52$ & $\mathbf{0.93}$ & $0.09$\\
        EPLL & $\underline{29.16}$ & $\mathbf{0.93}$ & $\mathbf{0.06}$\\
    \end{tabular}
    \caption{\label{tab:inpainting_bsd}Quantitative evaluation for inpainting on BSD.}
    \end{center}

\subsection{Image Generation}
\label{sec:image gen}
\noindent\textbf{Text-to-Image}. 
We evaluate the zero-shot generalization capability of MAGVLT under text-to-image generation.
We measure quantitative metrics of quality of generated images by Fréchet Inception Distance (FID) \cite{FID} and Inception Score (IS) \cite{salimans2016improved} on MS-COCO \cite{coco} validation.
In addition, we compare the relative decoding speed of MAGVLT against ARGVLT. The results are shown in \cref{tab:t2i}, where the models are grouped according to: (1) whether they are AR-based or not, and (2) the modality they can generate; the models in the first two groups are able to generate image only while the models in the last group can generate both image and text.
MAGVLTs significantly outperform AR-based methods including ARGVLTs as well as obtain comparable scores to the state-of-the-art diffusion-based methods. 
Note that all the other models in the non-AR group have more than 1B parameters while MAGVLTs have less than 500M parameters. And the performance gap between the task-specific MAGVLT (T2I only) and the universal MAGVLT is small.
Moreover, MAGVLTs generate an image more than eight times faster than ARGVLTs.
We provide qualitative samples in \cref{fig:t2i}.
% We can find that the model of ARGVLT on I2T and T2I significantly improve the model of ARGVLT on T2I only, and it means that the multi-task learning can make a synergy in this task. 
% AR에서는 multi task learning이 t2i에효과적이고 MASK에서는 multi task learning이 i2t에 효과적인 이유는?
\\[4pt]
\noindent\textbf{Image Inpainting}. 
One of the key advantages of MAGVLT over ARGVLT is that it enables bidirectional encoding of conditional information. MaskGIT \cite{chang2022maskgit} already demonstrated this advantage. To reconfirm it on MAGVLT, we conduct similar image inpainting experiments. In detail, the central 8$\times$8 image tokens corresponding to the central 50\% of the whole 16$\times$16 image tokens are masked out, and then replaced with newly-generated tokens conditioning on the unmasked image tokens and the ground-truth text tokens. The output images are blended with the input images along the mask boundary following \cite{chang2022maskgit}.
% ARGVLT can encode upper-left-sided context only for each token generation, whereas MAGVLT can refer to all the context image tokens, and therefore is expected to show more accurate inpainting results.
Quantitatively, as shown in \cref{tab:inpainting}, MAGVLT outperforms ARGVLT. which is also observed in qualitative samples of \cref{fig:t2i_inpaint_more} in Appendix.

\bgroup
\def\arraystretch{1.05}%
\begin{table}[t]
\centering
\small
\setlength{\tabcolsep}{3pt}
\scalebox{1}{
    \begin{tabular}{lccccc}
    \hline
    \multicolumn{1}{l}{Model} & B-4  & M    & C  & S & Speed
    \\ \hline
    % \rowcolor[gray]{0.85}\multicolumn{6}{l}{\textit{\textbf{with External LLMs}}} \\ 
    % \hline
    \rowcolor[gray]{0.85}\multicolumn{6}{l}{\textit{\textbf{with external language model}}} \\ 
    ZeroCap (345M) \cite{tewel2022zerocap} &2.6 & 11.5 & 14.6 & 5.5 & - \\
    MAGIC (1.5B) % GPT-2
    \cite{su2022magic} & 12.9& 17.4 &  49.3& 11.3 & -\\
    VLKD$_{\text{ViT-B/16}} $ (406M) 
    \cite{dai2022enabling} & {\bf 16.7} & {\bf 19.7} & 58.3 & {\bf 13.4} & -\\
    Flamingo-3B (3B) \cite{alayrac2022flamingo} &- & -& {\bf 73.0} & - & -\\
    % MetaLM \cite{hao2022language} & 24.5 & 22.5 & 82.2 & 15.7 & - \\
    %SimVLM$_{large}$ \cite{wang2021simvlm}& 10.5 & 12.0 & - & 24.9 & 8.3 \\
    \hline
    \rowcolor[gray]{0.85}\multicolumn{6}{l}{\textit{\textbf{without external language model}}} \\ 
    SimVLM$_{\text{huge}}$ (632M) \cite{wang2021simvlm} & 11.2& 14.7 & 32.2 & 8.5 & -\\
    %SimVLM$_{base}$ \cite{wang2021simvlm} & 9.5 & 11.5 & - & 24.0 & 7.5 \\
    ARGVLT (\textit{I2T only}) (447M) & 11.4 & 15.1 & 47.4 & 11.4 & $1.00\times$ \\
    ARGVLT (447M) & 10.9 & 14.9 & 45.5 & 11.2 & $1.00\times$ \\
    {\bf MAGVLT (\textit{I2T only})} (447M)& 12.9& 17.1& 53.5& 12.9 & $1.56\times$\\
    {\bf MAGVLT} (447M)& {\bf 14.6} & {\bf 19.0} & {\bf 60.4} & {\bf 14.3} & $1.56\times$
    \\ \hline
    \end{tabular}
}
\caption{\textit{Zero-shot} I2T results on MS-COCO Karpathy test. \label{tab:i2t_coco}}
\vspace{-0.1cm}
\end{table}
\egroup

\begin{figure}[t]
\begin{minipage}{.47 \textwidth}\centering
    \includegraphics[width=.95\textwidth]{figures/i2t_sample_ver3.png}
    \caption{\label{fig:i2t} Image-to-text samples on MS-COCO images. MAGVLT generated more proper captions on the corresponding images. More samples will be found in Appendix.}
\end{minipage}
\vspace{-0.2cm}
\end{figure}

% \begin{figure}[t]
% \begin{minipage}{.45 \textwidth}\centering
%     \includegraphics[width=.95\textwidth]{figures/i2t_infill_ver2.png}
%     \caption{\label{fig:infilling} Text infilling samples on MS-COCO dataset. The locations to be infilled are shaded with orange color, and the red and blue-colored words denote infilled words and their corresponding right-sided context words, respectively. The words infilled by MAGVLT are better aligned with the surrounding context words, and more appropriate on the corresponding images. More samples will be found in Appendix.}
% \end{minipage}
% \vspace{-0.5cm}
% \end{figure}


% Attention map or MixSel result
\subsection{Text Generation}
\label{sec:text gen}
\noindent\textbf{Image Captioning}. 
We evaluate the zero-shot image caption generation on MS-COCO Caption\cite{coco}, and NoCaps\cite{nocaps}. 
We measure quantitative metrics of the quality of the generated caption compared to the ground truth by BLEU-4 (B-4), METEOR (M), CIDEr (C), and SPICE (S).
% In addition, we compare the relative decoding speed of MAGVLT against ARGVLT. 
% Here, the decoding speed is measured from the average inference time taken to generate 64 length tokens per image.
The evaluation results on MS-COCO are shown in \cref{tab:i2t_coco}. 
Likewise as in text-to-image generation, we can verify that MAGVLTs improve performances over ARGVLTs significantly.
Moreover, MAGVLT outperforms some baselines that leverage external language models in CIDEr and SPICE which are specifically designed for the captioning task.
Note that MAGVLT has about six times smaller parameters than Flamingo-3B \cite{alayrac2022flamingo}, and even can generate images by a single model. In addition, the universal MAGVLT performs better than the task-specific MAGVLT (I2T only) maybe due to the synergetic improvement between the modalities. Regarding the marginal speedup ($1.56\times$) by MAGVLT over ARGVLT for I2T, compared to the fixed number of image tokens (256), the numbers of text tokens are quite small and generally less than 32, thus, the relative speedup by parallel predictions for
12 steps is reduced in text generation. Here, MAGVLT generates a text sequence given the pre-predicted target length. We provide qualitative samples in \cref{fig:i2t}.

Here, since L-Verse \cite{kim2022verse}, which can generate both modalities, provides I2T results only by scratch training on MS-COCO, for more comparison to L-Verse, we also perform the training from scratch on MS-COCO alone and obtain a much better FID score for T2I but a slightly lower CIDEr score for I2T compared to L-Verse (18.49 vs. 45.8 in FID, 85.3 vs. 102.2 in CIDEr).

% Although we found the zero-shot captioning result of COCO in \cite{hao2022language}, it is not compared here due to the unfair setting of pretraining. 
\bgroup
\def\arraystretch{1.05}%
\begin{table}[t]
\centering
\small
\begin{tabular}{lcc}
\hline
Model & CIDEr & SPICE
\\ \hline
\rowcolor[gray]{0.85}\multicolumn{3}{l}{\textit{\textbf{with MS-COCO in training}}} \\ 
%FewVLM$_{base}$ \cite{jin2021good} & - & - & - & - & - & - & 42.2& 8.5 \\
FewVLM$_{\text{large}}$ (740M)\cite{jin2021good} & 47.7& {\bf 9.1} \\
MetaLM (545M) % from section 4.2. VL encoder (192M) + LM (GPT2 w/ 24-layer, 353M) 
\cite{hao2022language} 
& {\bf 58.7}& 8.6 \\
% SimVLM$_{base}$  \cite{wang2021simvlm} & 83.2 & - & 84.1& - & 82.5& - &  83.5& - \\
% SimVLM$_{large}$ \cite{wang2021simvlm}  & 97.6& - & 96.5& - & 96.3& - &  96.6& - \\
% SimVLM$_{huge}$ \cite{wang2021simvlm}  & 101.2& - & 100.4& - & 102.3& - &  101.4& - \\
\hline
\rowcolor[gray]{0.85}\multicolumn{3}{l}{\textit{\textbf{without MS-COCO in training}}} \\ 
VLKD$_{\text{RN50x16}} $ (406M)\cite{dai2022enabling} & 54.0 & {\bf 9.6} \\
SimVLM$_{\text{huge}}$ (632M)\cite{wang2021simvlm} & {\bf 101.4} & -\\
ARGVLT (\textit{I2T only}) (447M) & 34.8& 6.5 \\
ARGVLT (447M) & 33.4& 6.4  \\
{\bf MAGVLT (\textit{I2T only})} (447M)& 37.7 & 7.2  \\
{\bf MAGVLT} (447M)& 46.3 & 8.7 
\\ \hline
\end{tabular}
\caption{\textit{Zero-shot} I2T results on NoCaps validation. \label{tab:i2t_nocaps}}
\end{table}
\vspace{-0.2cm}
\egroup
The evaluation results on NoCaps are shown in \cref{tab:i2t_nocaps}.
Basically, including MS-COCO in training is beneficial in performing on NoCaps since the interface of the caption collection for NoCaps closely resembles that used for the collection of the MS-COCO captions. 
Yet, MAGVLT shows comparable performances compared to FewVLM \cite{jin2021good}.
Also in this task, MAGVLT significantly outperforms ARGVLT.
MAGVLTs underperform in comparisons to VLKD \cite{dai2022enabling} and SimVLM \cite{wang2021simvlm}.
This might be due to that VLKD (RN50x16) has larger parameters ($\simeq$700M) than MAGVLT and also leverages an external language model (BART\cite{lewis-etal-2020-bart}) unlike MAGVLT.
SimVLM uses ALIGN dataset \cite{align} which contains 1.8B image-text pairs as well as C4 \cite{c4data} dataset which contains 360M text-only instances while MAGVLT uses only 17M image-text pairs for pretraining.
\\[5pt]
\noindent\textbf{Text Infilling}. 
Similar to image inpainting, we conduct the text infilling experiment where the central 50\% part of the text is erased by a mask and then replaced with generated tokens from the trained model. The qualitative samples are shown in \cref{fig:i2t_infill_more} in Appendix, where we can see that the infilled words generated by MAGVLT are better aligned with surrounding context words, compared to ARGVLT. Also, as shown in \cref{tab:infilling}, MAGVLT quantitatively outperforms ARGVLT. 

\bgroup
\def\arraystretch{1.05}%
    \begin{table}[t]
    \centering
    \small
    \begin{tabular}{lcccc}
    \hline
    \multicolumn{1}{l}{Model} & B-4  & M & C  & S\\
    \hline
    %ARGVLT (\textit{I2T only}) & 39.8  &  32.3 & 120.6 & 24.0\\
    ARGVLT & 39.5  & 32.2 & 119.8 & 23.8\\
    %{\bf MAGVLT} (\textit{I2T only}) & todo& todo  & todo & todo\\
    {\bf MAGVLT} & {\bf 42.7}  & {\bf 35.2} & {\bf 135.3} & {\bf 26.3}\\
    \hline
    \end{tabular}
    \caption{\textit{Zero-shot} text infilling results on MS-COCO Karpathy test. \label{tab:infilling}}
    \vspace{-0.3cm}
    \end{table}
\egroup

\subsection{Ablation Studies} \label{sec:ablation}
The objective function of the densest subgraph problem, i.e., the degree density,
has been generalized to various forms for extracting a more sophisticated structure in a graph.
Section~\ref{subsec:numerator} covers variants that generalize the numerator $e[S]$ of the density,
while Section~\ref{subsec:denominator} discusses variants that generalize the denominator $|S|$ of the density.
Section~\ref{subsec:generalization_others} reviews the other generalizations that do not fall in the above categorization.

\subsection{Generalizing the numerator}\label{subsec:numerator}

Tsourakakis~\cite{Tsourakakis15} generalized the notion of density to the $k$-clique density.
For $G=(V,E)$ and $S\subseteq V$, the $k$-clique density for some fixed positive integer $k$ is defined as
\begin{align*}
h_k(S)=c_k(S)/|S|,
\end{align*}
where $c_k(S)$ is the number of $k$-cliques contained in $G[S]$.
Obviously, when $k=2$, it reduces to the original density.
In the $k$-clique densest subgraph problem ($k$-clique DSP),
given an undirected graph $G=(V,E)$, we are asked to find $S\subseteq V$ that maximizes $h_k(S)$.
In particular, when $k=3$, the problem is referred to as the triangle densest subgraph problem (triangle DSP).
Tsourakakis~\cite{Tsourakakis15} proved that unlike many optimization problems for detecting a large near-clique,
the $k$-clique DSP is polynomial-time solvable when $k$ is constant.
Indeed, the author designed a maximum-flow-based exact algorithm and a supermodular-function-maximization-based exact algorithm
for the problem with constant $k$.
The author also demonstrated that a generalization of the greedy peeling algorithm,
which in each iteration removes a vertex participating in the minimum number of $k$-cliques, attains $k$-approximation.
Furthermore, the author presented a MapReduce implementation of the above greedy peeling algorithm to address large-scale graphs.
The results of computational experiments show that even the triangle densest subgraph is much closer to large near-cliques compared with the densest subgraph, as desired.

Later Mitzenmacher et al.~\cite{mitzenmacher2015scalable} conducted a follow-up work.
Their work is motivated by the fact that it is prohibitive to compute an exact or even well-approximate solution
to the $k$-clique DSP for reasonably large $k$ (e.g., $k>3$) on large graphs, due to the expensive cost of counting $k$-cliques.
To overcome this issue, they presented a sampling scheme called the densest subgraph sparsifier,
yielding a randomized algorithm that produces a well-approximate solution to the $k$-clique DSP
while providing significantly reduced time and space complexities.
Specifically, the sampling scheme samples each $k$-clique independently with an appropriate probability,
which can be incorporated as a preprocessing in any exact algorithm for the problem.
In addition to the sampling scheme, they also devised two simpler exact algorithms for the $k$-clique DSP.
Finally, the authors extended the $k$-clique DSP to the bipartite graph setting.
For an undirected bipartite graph $G=(L\cup R, E)$, positive integers $p,q$, and $S\subseteq L\cup R$, they defined the $(p,q)$-biclique density as $b_{p,q}(S)=c_{p,q}(S)/|S|$,
where $c_{p,q}(S)$ is the number of $(p,q)$-cliques contained in $G[S]$.
In the $(p,q)$-biclique densest subgraph problem ($(p,q)$-biclique DSP), given an undirected bipartite graph $G=(L\cup R, E)$,
we seek $S\subseteq L\cup R$ that maximizes $b_{p,q}(S)$.
They showed that all the above results for the $k$-clique densest subgraph problem can be extended to the $(p,q)$-biclique DSP.
Computational experiments demonstrate that the proposed sampling-based algorithms output near-optimal solutions to the problems,
and such solutions tend to be close to near-cliques and near-bicliques as the parameters $k$ and $p,q$, respectively, become large.

Fang et al.~\cite{Fang2019Efficient} devised more efficient exact and approximation algorithms for the $k$-clique DSP.
To this end, they introduced a generalization of the $k$-core called the $(k,\Psi)$-core.
%Note that the value $k$ of the $k$-clique DSP and the value $(k,\Psi)$-core are not
For a positive integer $k$ and an $h$-clique $\Psi$, a $(k,\Psi)$-core is a maximal subgraph
in which every vertex participates in at least $k$ $h$-cliques.
Therefore, if we take a 2-clique (i.e., an edge) as $\Psi$, the concept reduces to the ordinary $k$-core.
Note that the concept of $(k,\Psi)$-core is a special case of $k$-$(r,s)$ nucleus
introduced by Sar\i{}y\"{u}ce et al.~\cite{Sariyuce2015nucleus,Sariyuce2017nucleus}.
Using the concept, their exact algorithm for the $k$-clique DSP runs as follows:
It computes lower and upper bounds on the $k$-clique density value for each $(\ell,\Psi)$-core computed,
and based on those bounds, it derives lower and upper bounds on the optimal value of the problem.
Then the algorithm specifies some $(\ell,\Psi)$-cores that may contain an optimal solution to the problem,
and solve the problem on them.
A useful fact here is that such $(\ell,\Psi)$-cores tend to be much smaller than the original graph,
enabling us to compute an optimal solution in much shorter time in practice.
On the other hand, their approximation algorithm is based on the fact
that the $(\ell,\Psi)$-core with the maximum value of $\ell$ is a good approximation to an optimal solution.
The algorithm computes the structure without conducting core decomposition from scratch.

Recently, Gao et al.~\cite{gao2022colorful} designed a graph reduction technique to accelerate approximation algorithms for the $k$-clique DSP.
To this end, they introduced the novel concept called the colorful $h$-star.
Assume that the vertices of a graph are colored so that any pair of vertices having an edge receives different colors.
Then, for a positive integer $h$, a colorful $h$-star is a star contained in the graph as a (not necessarily induced) subgraph
in which all vertices have different colors.
Note that the colorful $h$-star is a relaxed concept of $h$-clique; indeed, every $h$-clique is a colorful $h$-star.
They showed that unlike $k$-cliques, the colorful $h$-stars can be counted efficiently using a newly devised dynamic programming method, and designed an efficient colorful $h$-star core decomposition algorithm.
Based on this, they designed a graph reduction technique to accelerate any approximation algorithm for the $k$-clique DSP.
Moreover, they showed that the colorful $h$-star core itself can be a good heuristic solution for the $k$-clique DSP.

Konar and Sidiropoulos~\cite{konar2022triangle} studied the triangle densest $k$-subgraph problem (TD$k$S).
The problem is a variant of D$k$S, where given an undirected graph $G=(V,E)$ and a positive integer $k$,
we are asked to find $S\subseteq V$ that maximizes the $3$-clique density $h_3(S)=c_3(S)/|S|$ (or simply $c_3(S)$) subject to $|S|=k$.
They showed that TD$k$S is \NP-hard, and as a counterpart of the authors' algorithm for D$k$S (reviewed in Section~\ref{subsubsec:DkS}),
they presented a heuristic algorithm based on a mirror descent algorithm for a convex relaxation derived by the Lov\'asz extension,
followed by a simple rounding procedure.
The proposed algorithm is shown to be empirically effective in terms of TD$k$S,
and moreover, it sometimes obtains a better solution even in terms of D$k$S than state-of-the-art algorithms for D$k$S.

Bonchi et al. \cite{BonchiKS19} generalized the notion of density by considering the $h$-degree of a node, i.e., the number of other nodes at distance no more than $h$ from the node. Based on this they defined the following problem:

\begin{problem}[Distance-$h$ densest subgraph]\label{prob:densestsubgraph}
Given a graph $G=(V,E)$ and a distance threshold $h \in \mathbb{N}^+$,  find a subset $S^* \subseteq V$ with the maximum average $h$-degree.
$$
 S^*=\underset{S\subseteq V}{\argmax}\frac{\sum_{v\in S} deg^h_{G[S]}(v)}{|S|}
$$
\end{problem}

It is easy to see that for $h = 1$, Problem~\ref{prob:densestsubgraph} corresponds to the traditional DSP. As the focus of their work was mostly on
distance generalized core decomposition, Bonchi et al. \cite{BonchiKS19} showed that, analogously to the $h = 1$ case, the inner-most core of the core decomposition, i.e., the $(k,h)$-core such that there is no non-empty $(j,h)$-core with $j>k$, provides an approximation to the distance-$h$ densest subgraph.


Miyauchi and Kakimura~\cite{Miyauchi-Kakimura18} aims to find a community,
i.e., a dense subgraph that is only sparsely connected to the rest of the graph, based on DSP.
They generalized the density as follows:
\begin{align*}
d_\alpha(S)=\frac{e[S]-\alpha\cdot e[S,\overline{S}]}{|S|}\ \text{($\alpha \in [0,\infty)$)},
\end{align*}
where $\alpha$ is a nonnegative parameter and $e[S,\overline{S}]$ is the cut size of $S$, i.e., the number of edges between $S$ and $V\setminus S$.
That is, this quality function penalizes the connection between $S$ and $V\setminus S$, resulting in a preferential treatment for community structure.
The authors studied the problem of maximizing this quality function,
and designed an LP-based exact algorithm and a maximum-flow-based exact algorithm.
Moreover, they presented a linear-time algorithm with some quality guarantee.
Computational experiments demonstrate that the proposed algorithms are highly effective in finding community structure in a graph.
For example, for the well-known Web graph called Web-Google, their linear-time approximation algorithm finds a vertex subset with more than 99.1\% density and less than 3.1\% cut size,
compared with an approximate densest subgraph obtained by the greedy peeling algorithm.

Recently, Chekuri, Quanrud, and Torres~\cite{Chekuri2022supermod} have introduced the densest supermodular subset problem (DSS),
where given a finite set $V$ and a nonnegative supermodular function $f:2^V\rightarrow \mathbb{R}_+$, we are asked to find $S\subseteq V$ that maximizes $f(S)/|S|$.
As $e[S]$ is a supermodular function over $V$ given $G=(V,E)$, this problem is a generalization of DSP.
For the above generalized problem, they presented a natural generalization of the iterative greedy peeling algorithm for DSP (reviewed in Section~\ref{subsubsec:iterative_peeling}).
Their significant contribution is the proof of the fact that the generalized algorithm outputs a $(1+\epsilon)$-approximate solution for the generalized problem,
after $T=O\left(\frac{\Delta_f \log n}{\lambda^* \epsilon^2}\right)$ iterations,
where $\Delta_f=\max_{v\in V}(f(V)-f(V\setminus \{v\}))$ and $\lambda^*$ is the optimal value of the problem.
This result affirmatively answers the conjecture of Boob et al.~\cite{boob2020flowless} that the iterative peeling algorithm converges to an optimal solution to DSP.
The proof is based on a consideration of an LP that is derived via the Lov\'asz extension of a supermodular function.


\subsection{Generalizing the denominator}\label{subsec:denominator}

Kawase and Miyauchi~\cite{Kawase-Miyauchi18} addressed the size issue of DSP.
The size issue means that when we solve DSP,
it may happen that the obtained subset is too large or too small in comparison with the size desired in the application at hand.
As mentioned in Section~\ref{sec:cons}, there are size-constrained variants of the densest subgraph problem, e.g., D$k$S, Dal$k$S, and Dam$k$S,
which explicitly specify the size range.
Unlike these variants, Kawase and Miyauchi~\cite{Kawase-Miyauchi18} generalized the density without putting any constraint.
Specifically, they introduced the $f$-density of $S\subseteq V$, which defined as $e[S]/f(|S|)$,
where $f\colon \mathbb{Z}_+\rightarrow \mathbb{R}_+$ is a monotonically non-decreasing function.
Note that earlier than this, Yanagisawa and Hara~\cite{yanagisawa2018discounted} introduced an intermediate generalization
called the discounted average degree, i.e., $e[S]/|S|^\alpha$ for $\alpha \in [1,2]$.
In the $f$-densest subgraph problem ($f$-DS), we are asked to find $S\subseteq V$ that maximizes the $f$-density $e[S]/f(|S|)$.
Although the $f$-DS does not explicitly specify the size of vertex subsets, the above size issue can be handled using convex or concave function $f$ appropriately.
Indeed, the authors showed that any optimal solution to $f$-DS with convex/concave function $f$ has a size smaller/larger than or equal to that of a densest subgraph.
Here a function $f:\mathbb{Z}_+\rightarrow \mathbb{R}_+$ is said to be convex (resp. concave) if $f(x)-2f(x+1)+f(x+2)\geq\, (\text{resp.} \leq)\, 0$ holds for any $x\in \mathbb{Z}_+$.
For the $f$-DS with convex $f$, they proved the \NP-hardness with some concrete $f$,
and designed a polynomial-time
$\min\left\{\frac{f(2)/2}{f(S^*)/|S^*|^2},\, \frac{2f(n)/n}{f(|S|^*)-f(|S^*|-1)}\right\}$-approximation algorithm,
where $S^*\subseteq V$ is an optimal solution to the $f$-DS.
The approximation ratio looks a bit complicated but it reduces to a simpler form by considering some concrete $f$, e.g., $f(x)=x^\alpha$ ($\alpha \in [1,2]$) and $f(x)=\lambda x + (1-\lambda)x^2$ ($\lambda \in [0,1]$).
For the $f$-DS with concave $f$, they designed an LP-based exact algorithm and a maximum-flow-based exact algorithm.
In particular, the LP-based exact algorithm computes not only an optimal solution but also vertex subsets corresponding to dense frontier points, as explained in Section~\ref{sec:cons}.
Finally, they presented linear-time $3$-approximation algorithm based on the greedy peeling algorithm.

\subsection{Other variants}\label{subsec:generalization_others}
Tsourakakis et al.~\cite{tsourakakis2013denser} defined the Optimal Quasi-Clique (OQC) problem of finding the subgraph $S$ that maximizes
$$
e[S] - \alpha \binom{|S|}{2}
$$
thus trying to find a subgraph which is denser in terms of the edge density $e[S]/{|S|\choose 2}$, instead of the degree density adopted by DSP.

Feng et al. \cite{feng2021specgreedy} proposed a generalized framework for addressing DSP and related problems (e.g., \cite{hooi2016fraudar, Miyauchi-Kakimura18, tsourakakis2019novel}) by introducing SpecGreedy, an algorithm that uses graph spectral properties and a greedy peeling strategy to solve the generalized problem. Their framework is a generalization of DSP, which accounts for node weights and a second input graph defined on the same set of nodes $V$. Specifically, the objective is to maximize a function that prioritizes edge-density in the first input graph and node weights while discarding nodes that are dense even in the second input graph.

Recently, Veldt, Benson, and Kleinberg~\cite{veldt2021meandensest} generalized the density to the single-parameter family of quality functions.
Specifically, they introduced the $p$-density for $S\subseteq V$, based on the concept of generalized mean (also called power mean or $p$-mean) of real values, as follows:
\begin{align*}
M_p(S)=\left(\frac{1}{|S|}\sum_{v\in S}\deg_S(v)^p\right)^{1/p}\quad \text{($p\in [-\infty,\infty]$)},
\end{align*}
where for $p\in \{-\infty, 0, \infty\}$, $M_p(S)$ is defined as its limit,
i.e., $M_{-\infty}(S)=\lim_{p\rightarrow -\infty}M_p(S)=\min_{v\in S}\deg_S(v)$, $M_{0}(S)=\lim_{p\rightarrow 0}M_p(S)=\prod_{v\in S}\deg_S(v)$,
and $M_{\infty}(S)=\lim_{p\rightarrow \infty}M_p(S)=\max_{v\in S}\deg_S(v)$.
When $p=1$, the $p$-density reduces to the original density.
The generalized mean densest subgraph problem asks for finding $S\subseteq V$ that maximizes $M_p(S)$.
It is worth mentioning that the generalized mean densest subgraph problem deals with the densest subgraph problem and the problem of finding $k$-core with maximum $k$ in a unified manner ($p=1$ and $p=0$, respectively).
They first proved that when $p\geq 1$, the generalized mean densest subgraph problem can be solved exactly in polynomial time by repeatedly solving submodular function minimization.
They then designed a faster $(p+1)^{1/p}$-approximation algorithm based on the greedy peeling algorithm.
They specified a class of graphs for which this approximation ratio is tight, and showed that as $p\rightarrow \infty$, the approximation ratio converges to $1$.
They also proved that for any $p >1$, the greedy peeling algorithm for DSP outputs an arbitrarily bad solution to the problem, on some graph classes.
Computational experiments demonstrate that the proposed algorithm obtains an extremely good approximate solution (even for the original densest subgraph problem),
scales to large graphs, and highlights a range of different meaningful notions of density.

Balalau et al. \cite{balalau2015topkoverlapping} defined the $(k,\alpha)$-Dense Subgraph with Limited Overlap problem ($(k,\alpha)$-DSLO): given an integer $k >
0$ as well as a real number $\alpha \in [0, 1]$, find at most $k$
subgraphs that maximize the total aggregate density,
i.e., the sum of the average degree of each subgraph,
under the constraint that the maximum pairwise
Jaccard coefficient between the set of nodes in the subgraphs be at most $\alpha$. They proved that the problem is \NP-hard even when $\alpha = 0$ (disjoint subgraphs) and showed that the simple heuristic that greedily finds one densest subgraph in the current graph, remove all its vertices and edges, and iterate until $k$
subgraphs are found or the current graph is empty, can produce arbitrarily bad solutions. Balalau et al. thus presented an efficient algorithm for $(k,\alpha)$-DSLO which comes with provable guarantees in some cases of
interest.

\noindent\textbf{Variants of MAGVLT.}
Here, we investigate the effects of sampling weights for three multimodal tasks corresponding to $p_{\tau}$ in learning MAGVLT, and the results are shown in \cref{tab:variants}.
We denote the three weights in a form of {\fontfamily{qcr}\selectfont T2I:I2T:IT2IT} in the table.
% For example, {\fontfamily{qcr}\selectfont 1:0:0} means that the model is trained on T2I task only.
% T2I only가 image generation에서 젤 좋다. 그런데 반대로 I2T only의 경우 captioning 에서 젤 나쁘다. 우리는 I2T only model의 training loss가 다른 어떤 모델들보다 낮은것을 관측했다. 이것은 non-AR 모델에서 captioning task 만을 고려했을 때 우리가 section 3.4에서 언급한 bias 이슈가 더 커지고 일반화 성능이 악화될 수 있음을 의미한다.
The trained model by T2I only produces the best performance for text-to-image generation. 
However, in contrast to this, the trained model by I2T only shows the worst performance for captioning. 
We observe that the training loss of the mask prediction of this I2T only model is the lowest compared to the other models. 
This may suggest that the bias issue we mentioned in \cref{sec:mixsel} can be more serious, especially in non-AR methods.
% and therefore, the generalization ability becomes worse especially in non-AR methods.
As shown in the result, learning T2I along with I2T resolves the issue to some extent. 
% We believe that learning image generation can make the model further understand also when image 
The ratio of {\fontfamily{qcr}\selectfont 8:2:0} shows better I2T performance in CIDEr but inferior
T2I performance in FID than that of {\fontfamily{qcr}\selectfont 0:0:1}. We observe that the inclusion of the uncorrupted cross-modal context is necessary. The model trained with {\fontfamily{qcr}\selectfont 2:1:1} weights shows the best captioning performance but the worst T2I performance at a time.
As the portion of T2I training is getting larger, the model performs better in T2I but worsens in I2T. 
It means that there is a trade-off between T2I and I2T according to the sampling ratio.
We use the most balanced one ({\fontfamily{qcr}\selectfont 8:1:1}) super-scripted by * as the default setting for MAGVLT.

It is noted that regarding the performance drop by IT2IT modeling rather than T2I-only modeling for the T2I task, compared to I2T, in T2I the benefit of focusing on the cross-modal context is less significant, especially in the later refinement steps. Therefore, improved cross-modal attention by IT2IT training could be less effective for the T2I task. Having said that, the performance drop by MAGVLT for T2I is small even though it can also perform I2T. And we observe that in terms of CLIP scores, our IT2IT training is slightly better than T2I-only training (0.3176 vs. 0.3145) due to the enhanced cross-modal alignment. Moreover, the capacity of our moderate-sized model would be still limited in generating both modalities. Under this limited capacity, there exist some performance trade-offs, and here it would be more leaned to I2T. \cref{tab:scaling_t2i} in Appendix shows that when we increase the model capacity about two times, the large model with IT2IT performs better than the medium model with T2I-only training on the T2I task.
\bgroup
\def\arraystretch{1.05}%
    \begin{table}[t]
    \centering
    \small
    \begin{tabular}{lcc}
    \hline
    \multicolumn{1}{l}{Model} & CIDEr ($\uparrow$)  & FID ($\downarrow$) \\
    \hline
    % ARGVLT & 45.5 & 16.93 \\
    % ARGVLT + MixSel & 30.9 & 22.29 \\
    MAGVLT (\textit{T2I only}) & - & \bf{10.74}\\
    ~~w/o MixSel & - & 10.97 \\
    ~~w/o UnrollMask and MixSel & - & 11.72 \\
    MAGVLT (\textit{I2T only}) & \bf{53.5} & - \\
    ~~w/o MixSel & 51.3 & - \\
    ~~w/o UnrollMask and MixSel & 48.0 & - \\
    MAGVLT & \bf{60.4} & \bf{12.08} \\
    ~~w/o MixSel & 58.9 & \bf{12.07} \\
    ~~w/o UnrollMask & 56.5 & 13.26 \\
    ~~w/o UnrollMask and MixSel & 53.8 & 14.12 \\
    \hline
    \end{tabular}
    \caption{Effectiveness of additional training tasks. \label{tab:addtask}}
    \vspace{-0.3cm}
    \end{table}
\egroup
\\[5pt]
\noindent\textbf{Effectiveness of Additional Tasks}. 
Here, we investigate the effects of the additional tasks (\ie \textit{UnrollMask} and \textit{MixSel}), and the results are 
 shown in \cref{tab:addtask}.
MAGVLTs without the additional tasks are clearly underperformed in total.
Applying UnrollMask in pretraining significantly improves the performances of the base models on both T2I and I2T tasks.
Including MixSel along with UnrollMask also improves the performances of the T2I-only model and especially the models on I2T. Similar to IT2IT training, this improved cross-modal
attention by MixSel could be less effective for T2I task compared to I2T task. We also experimentally found that the performance gain by MixSel is somewhat marginal for ARGVLT.
We hypothesize that the causal attention of AR models would be hard to encode the randomly changing span of context.
% However, it does not improve the model trained by multitask on T2I.
% todo - why?
% In contrast to the result of MAGVLT, MixSel hinders learning the model of ARGVLT. 




\section{Conclusion} \label{sec:concls}

In this work, we propose MAGVLT as a unified generative VL model that can produce both image and text data. MAGVLT is robustly trained on image-text pairs by multiple cross-modal tasks and significantly outperforms ARGVLT achieving strong performances on both of zero-shot I2T and T2I tasks. In future work, we first need to scale-up MAGVLT in terms of both model and data for better generalizability. Also, we aim to leverage a pretrained language model to enhance natural language processing and eventually to enable in-context VL learning. We will also try to exploit an encoder-decoder architecture to robustly perform on both understanding and generation tasks.

{\small
\subsection*{Acknowledgements}
% We would like to thank Brain Cloud Team at Kakao Brain. This work was also supported by Korea University Grant (K2304351).
We would like to thank Brain Cloud Team at Kakao Brain for their support. This work was also supported by Korea University Grant (K2304351) and Institute of Information \& communications Technology Planning \& Evaluation (IITP) grant funded by the Korea government (MSIT) (No. 2022-0-00612, Geometric and Physical Commonsense Reasoning based Behavior Intelligence for Embodied AI).
}

%%%%%%%%% REFERENCES
{\small
\bibliographystyle{ieee_fullname}
\bibliography{egbib}
}

\newpage
\appendix

\section{Training Details} 
We apply BPE dropout with a rate of 0.1. We also apply residual and attention dropouts with a rate of 0.1, and label smoothing for both image and text loss computation with a rate of 0.1. We train both ARGVLT and MAGVLT models using AdamW optimizer with $\beta_1 = 0.9$, $\beta_2 = 0.96$, $\epsilon = 10^{-8}$, weight decay coefficient of $4.5\times 10^{-2}$, and the learning rate of $4.5\times 10^{-4}$ with a cosine annealing. The gradients are clipped by a norm using a threshold of 4, prior to applying the Adam update. When training ARGVLT, we observe that calculating the predictive losses on the context tokens along with the generation tokens improves the overall performance. Hence, we compute the losses on the whole concatenated token sequence with the loss 
coefficients to 0.9 and 0.1 for generation modality and conditional modality, respectively. The data augmentations used in \cite{dalle} are applied to the images before encoding them using VQ-GAN. 
For positional embedding, we adopt a learnable absolute position encoding, for both image and text modalities. The encoded image tokens are flattened by the raster scan order before being fed into the transformer. MAGVLT was trained on 128 V100 GPUs for 40K updates with a batch size of 4,096, which takes about 3 days.

\cref{tab:arch_hparam} describes the detailed architecture hyperparameters for the transformers we used including the large models. 
 
% TODO: ARGVLT loss weight 
% TODO: epochs=10, batchsize=4096, pretraining for VLTs
% TODO: epochs=15, batchsize=4096, pretraining for VLT large models
\bgroup
\def\arraystretch{1.05}%
    \begin{table}[h]
    \centering
    \small
    \begin{tabular}{ccc}
    \hline
    \bf{Parameter} & \multicolumn{2}{c}{\textbf{Model}} \\
    \cmidrule(r){2-3}
    & ARG/MAGVLT & ARG/MAGVLT$_{\text{Large}}$ \\
    % \midrule
    \hline
    %ARGVLT (\textit{I2T only}) & 39.8  &  32.3 & 120.6 & 24.0\\
    Params & 371M & 840M \\  
    Layers & 24 & 36  \\ 
    Embed Dim & 1024 & 1280 \\
    Heads & 8 & 10 \\ 
    \hline
    \end{tabular}
    \caption{Detailed architecture hyperparmeters. The left model column represents the default model described in the main paper, while the right column indicates the large model that will be presented in the next section.
    % The result of the large models will be reported in the Supplementary Materials.
    % We follow the standard convention that the hidden/embedding dimension of transformers is equal to the head dimension multiplied by the number of heads. 
    \label{tab:arch_hparam}}
    \end{table}
\egroup

\section{Model Scaling}
It is well known that scaling up the pretrained generative model generally improves the generalization ability, and recently VL models often have more than 1B parameters. Therefore, we also scale up our VLTs and evaluate those for the tasks of zero-shot I2T and T2I on MS-COCO. As shown in \cref{tab:arch_hparam}, the large model (MAGVLT$_\text{Large}$) contains 840M parameters for the transformer and 916M parameters including VQ-GAN in total. 
MAGVLT$_\text{Large}$ was trained on 128 V100 GPUs for 80K updates with a batch size of 4096, which takes about 12 days.
\bgroup
\def\arraystretch{1.05}%
    \begin{table}[h]
    \centering
    \small
    \begin{tabular}{lccc}
    \hline
    Model & FID ($\downarrow$) & IS ($\uparrow$) & Speed\\
    \hline
    ARGVLT  & 16.93 & 22.50 & 1.00$\times$ \\
    % ARGVLT$_\text{Large}$  & 12.90 & 24.27 & 0.51$\times$ \\ % @60K
    ARGVLT$_\text{Large}$  & 13.01 & 23.75 & 0.51$\times$ \\ % @80K
    MAGVLT  & 12.08 & 22.75 & \bf{8.12}$\times$ \\
    % MAGVLT$_\text{Large}$  & \bf{10.69} & \bf{25.04} & 6.97$\times$ \\ % 60K
    MAGVLT$_\text{Large}$  & \bf{10.14} & \bf{25.15} & 6.97$\times$ \\
    % 80K
    \hline
    \end{tabular}
    \caption{\textit{Zero-shot} T2I results on MS-COCO validation. \label{tab:scaling_t2i}}
    \end{table}
\egroup

The zero-shot T2I results on MS-COCO are shown in \cref{tab:scaling_t2i}. Notably, the large-scale models of both VLTs significantly improve FID and IS scores with large margin, compared to their respective default models. In addition, the degree of sampling speed reduction by model scaling is relatively smaller in MAGVLT than that in ARGVLT. Note that  MAGVLT$_\text{Large}$ is slightly slower than the default MAGVLT (6.97$\times$ vs 8.12$\times$), however it is still much faster than the default ARGVLT which has much fewer parameters.  
\bgroup
\def\arraystretch{1.05}%
\begin{table}[h]
\centering
\small
\scalebox{1}{
    \begin{tabular}{lcc}
    \hline
    \multicolumn{1}{l}{Model} & CIDEr  & SPICE
    \\ \hline
    % \rowcolor[gray]{0.85}\multicolumn{6}{l}{\textit{\textbf{with External LLMs}}} \\ 
    % \hline
    \rowcolor[gray]{0.85}\multicolumn{3}{l}{\textit{\textbf{MS-COCO}}} \\ 
    ARGVLT & 45.5 & 11.2 \\
    %ARGVLT$_\text{Large}$ & 34.8 & 9.4 \\ % top_p = 0.0
    %ARGVLT$_\text{Large}$ & 45.0 & 11.5 \\ % @60k
    ARGVLT$_\text{Large}$ & 43.6 & 11.2 \\ % @80k
    MAGVLT & 60.4 & 14.3 \\
    %MAGVLT$_\text{Large}$ & \bf{63.8} & \bf{15.1} \\ % @60k
    MAGVLT$_\text{Large}$ & \bf{68.1} & \bf{15.5} \\ % @80k
    \hline
    \rowcolor[gray]{0.85}\multicolumn{3}{l}{\textit{\textbf{NoCaps}}} \\ 
    ARGVLT & 33.4 & 6.4  \\
    %ARGVLT$_\text{Large}$ & 27.2 & 5.4 \\ % top_p = 0.0
    %ARGVLT$_\text{Large}$ & 34.6  &  6.7\\ % @60k
    ARGVLT$_\text{Large}$ & 34.1  &  6.1 \\ % @80k
    MAGVLT & 46.3 & 8.7  \\
    %MAGVLT$_\text{Large}$ & \bf{51.8} & \bf{9.2} \\ % @60k
    MAGVLT$_\text{Large}$ & \bf{55.8} & \bf{9.8} \\ % @80k
    \hline
    \end{tabular}
}
\caption{\textit{Zero-shot} I2T results on MS-COCO Karpathy test ({\bf Top}) and NoCaps validation ({\bf Bottom}). \label{tab:scaling_i2t}}
\end{table}
\egroup

The zero-shot I2T results on MS-COCO and NoCaps datasets are presented in \cref{tab:scaling_i2t}. Similar to the T2I results, the large-scale models of both VLTs show better I2T scores compared to their respective default models. Note that in case of ARGVLT, the performance gap between the default and large models is marginal on MS-COCO dataset, while MAGVLT improves the performance significantly on both datasets, as the model size is increased. These results imply that our MAGVLT is more effective in model scaling.  \\

\section{Finetuning on Downstream Tasks}
In order to verify the transferability of MAGVLT by task-specific finetuning, we perform finetuning on two downstream tasks, one for generation and the other for understanding.
In this finetuning setting, ARGVLT and MAGVLT are initialized from their 40K pretrained checkpoint, and MAGVLT$_\text{Large}$ is initialized from 60K pretrained checkpoint.
\\[5pt]
\noindent{\bf{Image Captioning.}}
We finetune ARGVLT and MAGVLT on the image caption generation task of MS-COCO 2014 dataset. In specific, we finetune the VLTs with the cross entropy loss for 100 epochs with a batch size of 512.
The learning rate is set to $10^{-5}$ for ARGVLT and MAGVLT, and $2 \times 10^{-5}$ for MAGVLT$_\text{Large}$.
Note that we do not use the additional tasks, UnrollMask and MixSel, in finetuning.
The captioning performances are presented in \cref{tab:ds_caption}.
Similar to zero-shot I2T results, MAGVLT shows better results compared to ARGVLT. Moreover, the large-scale model of MAGVLT improves the performances compared to its respective default model.
\bgroup
\def\arraystretch{1.05}%
\begin{table}[h]
\centering
\small
\scalebox{1}{
    \begin{tabular}{lcccc}
    \hline
    \multicolumn{1}{l}{Model} & B-4  & M & C  & S
    \\ \hline
    % % \rowcolor[gray]{0.85}\multicolumn{6}{l}{\textit{\textbf{with external language model}}} \\ 
    % % ZeroCap \cite{tewel2022zerocap} &2.6 & 11.5 & 14.6 & 5.5 & - \\
    % % MAGIC\cite{su2022magic} & 12.9& 17.4 &  49.3& 11.3 & -\\
    % % VLKD$_{\text{ViT-B/16}}$ \cite{dai2022enabling} & {\bf 16.7} & {\bf 19.7} & 58.3 & {\bf 13.4} & -\\
    % % Flamingo-3B \cite{alayrac2022flamingo} &- & -& {\bf 73.0} & - & -\\
    % % \hline
    % \rowcolor[gray]{0.85}\multicolumn{6}{l}{\textit{\textbf{without external language model}}} \\ 
    ARGVLT  & 28.6 & 25.2 & 94.7 & 18.1 \\
    MAGVLT  & 29.3 & 27.1 & 103.3 & 20.5 \\
    MAGVLT$_\text{Large}$  & \bf{32.3} & \bf{27.9} & \bf{110.7} & \bf{21.0} \\ % @60K 
    % MAGVLT$_\text{Large}$  & \bf{29.2} & \bf{27.3} & \bf{105.6} & \bf{21.3} \\ % @80K
    \hline 
    \end{tabular}
}
\caption{Comparisons of finetuned models on MS-COCO Karpathy splits. \label{tab:ds_caption}}
\end{table}
\egroup
\\[1pt]
\noindent{\bf{Visual Question Answering.}}
Masked pretraining is well known as a good representation learning approach for VL \emph{understanding} tasks. Therefore, even though we use a variable mask ratio rather than a low fixed ratio during training for obtaining generation capability of MAGVLT, we can also evaluate the transferability of MAGVLT on a discriminative task. For this, we perform experiments on visual question answering (VQA) task, which is a VL understanding task that requires a model to answer a question given an image, on the commonly used VQAv2 dataset \cite{vqa2017}.
Following \cite{wang2021simvlm}, we treat this task as a classification task where an auxiliary classifier predicts an answer from 3,129 candidates.
The tokens of the question mark `{\fontfamily{qcr}\selectfont ?}' and {\fontfamily{qcr}\selectfont <MASK>} token are sequentially added to the tail of the input sequence $[X; Y]$ where $[\cdot]$ is the concatenation operator.
The top layer output of {\fontfamily{qcr}\selectfont <MASK>} is used as an input for the classifier.
We finetune the classifier and the corresponding model with the cross entropy loss for 20 epochs with a batch size of 2,048 and a learning rate of $5 \times 10^{-5}$, and the dropout rate of the top layer output is set to 0.6.

The results are shown in \cref{tab:ds_vqa}. Compared to the latest algorithms \cite{hao2022language, dai2022enabling}, MAGVLT performs slightly worse, however it can be confirmed that the discriminative representation for understanding has been learned by MAGVLT to some extent. While VLKD \cite{dai2022enabling} and MetaLM \cite{hao2022language} use large-scale language-only data and leverage a language model, we pretrain our model from scratch using only paired image-text datasets. And, our model is basically trained for generation, and moreover, it can even generate images by a single model. 
% TODO: describe the results
% 최신  알고리즘들과 비교했을때 수치가 조금 떨어지지만 understanding을 위한 representation도 어느정도 잘 학습해 놓고 있음을 확인할수 있다. VLKD나 MetaLM 은 large-scale language only data를 사용하고, language model을 leverage 했는데 반해 우리는 그런거 전혀 없이 순수 image-text pair data set만을 사용해서 scratch 부터 pretraining 하였다. 그리고, 더나아가 우리는 generation 용 모델이고 더더군다나 하나의 모델이 image generation까지 커버한다.
\bgroup
\def\arraystretch{1.05}%
\begin{table}[h]
\centering
\small
\scalebox{1}{
    \begin{tabular}{lcc}
    \hline
    \multicolumn{1}{l}{Model} & test-dev & test-std
    \\ \hline
    % % \rowcolor[gray]{0.85}\multicolumn{6}{l}{\textit{\textbf{with external language model}}} \\ 
    % % ZeroCap \cite{tewel2022zerocap} &2.6 & 11.5 & 14.6 & 5.5 & - \\
    % % MAGIC\cite{su2022magic} & 12.9& 17.4 &  49.3& 11.3 & -\\
    % % VLKD$_{\text{ViT-B/16}}$ \cite{dai2022enabling} & {\bf 16.7} & {\bf 19.7} & 58.3 & {\bf 13.4} & -\\
    % % Flamingo-3B \cite{alayrac2022flamingo} &- & -& {\bf 73.0} & - & -\\
    % % \hline
    % \rowcolor[gray]{0.85}\multicolumn{6}{l}{\textit{\textbf{without external language model}}} \\ 
    VLKD$_{\text{ViT-B/16}}$ \cite{dai2022enabling} & 69.8 & - \\
    MetaLM \cite{hao2022language} & \bf{74.4} & \bf{74.5} \\
    \hline
    MAGVLT  & 63.0 & 63.4 \\
    MAGVLT$_\text{Large}$ & 65.7 & 66.2 \\ % @60k
    % MAGVLT$_\text{Large}$ & 65.9 & 66.19  \\ % @80k
    \hline
    \end{tabular}
}
\caption{Experimental results on VQAv2. \label{tab:ds_vqa}}
\end{table}
\egroup
\section{Unconditional Image+Text Generation Result}

Since we train MAGVLT with the three multi-modal tasks including IT2IT, the model is able to produce both image and text at a time. 
Namely, all of the tokens of $X$ and $Y$ are masked at first, and then refined through the iterative decoding. For the target length prediction, the target length is randomly initialized in a range from 8 to 16 and then iteratively predicted as the refinement step proceeds. 
Here, we provide unconditional image+text generation results which are presented in \cref{fig:unconditioanl_ti}. Note that the generated images are very diverse and generally have high quality, and the generated texts also describe the images properly. 
\section{MixSel Analysis}

Here, we demonstrate the effectiveness of the proposed \textit{MixSel} task.
As described in \cref{sec:mixsel}, MixSel mixes two different contexts and selects one of them to be used for generation. 
% We hypothesize that if the model is not allowed to the information about \textit{what the cross-modal context corresponding to the masked target sequence is}, then the model can ignore the cross-modal context more easily since the mixed noisy context interferes the model attending. 
% Therefore, the model would more depend on the within-modal context, and it makes the model biased.
We hypothesize that our MixSel training task allows the model to attend more carefully to the proper cross-modal context and accordingly to reduce the overlooking of the cross-modal context.
In order to verify this, we first consider \textit{MixRandom} setting which is the same as MixSel, but different in that the target is randomly selected without the additional special token to inform which one is selected, \ie {\fontfamily{qcr}\selectfont <LEFT>} and {\fontfamily{qcr}\selectfont <RIGHT>} or {\fontfamily{qcr}\selectfont <TOP>} and {\fontfamily{qcr}\selectfont <BOTTOM>}. This MixRandom can be seen as the perturbation of the input context alone for regularization like data augmentations.
In \cref{tab:aug}, MAGVLT$_\text{MixRandom}$, which indicates the trained MAGVLT along with UnrollMask and MixRandom, deteriorates the performances of both the zero-shot I2T and the zero-shot T2I, in comparison to MAGVLT with the use of MixSel training.
\bgroup
\def\arraystretch{1.05}%
    \begin{table}[h]
    \centering
    \small
    \begin{tabular}{lcc}
    \hline
    Model & CIDEr ($\uparrow$) & FID ($\downarrow$) \\
    \hline
    MAGVLT$_\text{MixSel}$  & \bf{60.4} & \bf{12.08} \\
    % MAGVLT$_\text{w/o MixSel}$ & 58.9 & \bf{12.07}  \\
    MAGVLT$_\text{MixRandom}$ & 57.9 & 13.43  \\
    \hline
    \end{tabular}
    \caption{Comparison of MixSel and MixRandom on \textit{Zero-shot} I2T and T2I. \label{tab:aug}}
    \end{table}
\egroup

Furthermore, in \cref{fig:mixsel_attn}, we qualitatively show by visualization of cross-modal attention maps that MixSel pretraining task makes the model to attend more to the cross-modal context appropriately compared to the model trained without MixSel training.
% TODO: attention map samples and description

\section{Additional Samples}
Here, we present more qualitative results of image and text generation tasks described in \cref{sec:image gen} and \cref{sec:text gen}. The image generation and inpainting results are presented in \cref{fig:t2i_more}, \cref{fig:t2i_inpaint_more}, respectively. The image captioning and text infilling results are shown in \cref{fig:i2t_more} and \cref{fig:i2t_infill_more}, respectively. For text generation tasks, we resize and center-crop the validation images. Overall, our proposed MAGVLT shows better results than ARGVLT.

% \section{Code Submission}
% Our code for MAGVLT is involved in Appendix. Due to the file size limit, we include only the training code that runs on a small subset of the original training data. We will release our entire code that runs on the full training data, the sampling code, and the trained model file upon acceptance.

\begin{figure*}[t]
\centering
\includegraphics[width=.65\textwidth]{figures/unconditional_ti_ver2.png}
\caption{\label{fig:unconditioanl_ti} Unconditional image+text generation results obtained by MAGVLT. Note that the generated images cover diverse categories, such as natural scenery (1st row), indoor scenes \& foods (2nd row), animals (3rd row), objects (4th row), and illustrations (5th row). Also, the generated texts are well aligned with generated images.}
\vspace{-0.5cm}
\end{figure*}

\begin{figure*}[t]
\centering
\includegraphics[width=.9\textwidth]{figures/mixsel_attn.jpg}
\caption{\label{fig:mixsel_attn} Visualization of cross-modal attention maps and generated images at different refinement steps. Given the text "a picture of bus in the street.", images are generated using MAGVLTs trained without the use of UnrollMask and MixSel (\textbf{Top}) and with the use of UnrollMask and MixSel (\textbf{Bottom}). To visualize each attention map, cross-attention scores between all 256 image tokens (queries) and a specific text token (a key, corresponds to each row) are computed and then reshaped to 16x16. Image tokens more attend to object text tokens ({\fontfamily{qcr}\selectfont <bus>} and {\fontfamily{qcr}\selectfont <street>}) when the model trained with the use of UnrollMask and MixSel.}
\vspace{-0.5cm}
\end{figure*}

\begin{figure*}[t]
\centering
\includegraphics[width=.6\textwidth]{figures/t2i_more_v2.png}
\caption{\label{fig:t2i_more} More samples of text to image on MS-COCO dataset. }
\vspace{-0.5cm}
\end{figure*}

% \begin{figure*}[t]
% \centering
% \includegraphics[width=.6\textwidth]{figures/t2i_inpaint_more_1.png}
% \caption{\label{fig:t2i_inpaint_more} More samples of image inpainting  on MS-COCO dataset. }
% \vspace{-0.5cm}
% \end{figure*}

\begin{figure*}[t]
\centering
\includegraphics[width=.5\textwidth]{figures/t2i_inpaint_more_3.png}
\caption{\label{fig:t2i_inpaint_more} Image inpainting samples on MS-COCO dataset. MAGVLT generated the masked parts to be more blended with the surrounding context, and more proper to the captions.}
\vspace{-0.5cm}
\end{figure*}

\begin{figure*}[t]
\centering
\includegraphics[width=.85\textwidth]{figures/i2t_more.png}
\caption{\label{fig:i2t_more} More samples of image captioning on MS-COCO dataset. }
\vspace{-0.5cm}
\end{figure*}

\begin{figure*}[t]
\centering
\includegraphics[width=.7\textwidth]{figures/i2t_infill_more_2.png}
\caption{\label{fig:i2t_infill_more} Text infilling samples on MS-COCO dataset. The locations to be infilled are shaded with orange color. The words infilled by MAGVLT are better aligned with the surrounding context words, and more appropriate on the corresponding images.}
\vspace{-0.5cm}
\end{figure*}



\end{document}

