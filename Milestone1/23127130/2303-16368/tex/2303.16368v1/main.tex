%\documentclass[prl,10pt,twocolumn,groupedaddress,floatfix,showpacs]{revtex4-2}

\documentclass[aps,reprint,amsmath,amssymb,superscriptaddress,nobibnotes,showpacs,showkeys]{revtex4-2}


%#####################################
\usepackage[utf8]{inputenc}  
\usepackage[T1]{fontenc}     %Output what you want e.g., é, ł, a, ü
\usepackage[british]{babel}  %Do hyphenation according to british english
%\usepackage[sc,osf]{mathpazo}\linespread{1.05}  %Palatino font
\usepackage[scaled=0.86]{berasans}  % URL font that go well wtih palatino
\usepackage[colorlinks=true, allcolors=blue, urlcolor=blue]{hyperref}  %Hyperlinks (pink, green, blue)
\usepackage{graphicx} % Package to insert exteral figures
\usepackage[babel]{microtype}  %Improves text justification
\usepackage{amsmath,amssymb,amsthm,bm,amsfonts,mathrsfs,bbm} %Usefull math packages
\renewcommand{\qedsymbol}{\rule{0.7em}{0.7em}}
\usepackage{xspace}  %Useful to add space in macros
\usepackage{pgfplots}
\usepackage{xcolor,colortbl}
\usepackage{array}
\usepackage{bigstrut}

%%%%%%%%%%%%%%%%%%

%\usepackage{IEEEtrantools}
\usepackage{multirow}

% ------------------------------------------------------------------------------
\newcommand{\ashu}{\color{blue}}
\newcommand{\bae}{\color{purple}}
\newcommand{\ji}{\color{gray}}


\newcommand{\R}{\mathbb{R}}
\newcommand{\N}{\mathbb{N}}
\newcommand{\C}{\mathbb{C}}
\newcommand{\Z}{\mathbb{Z}}
\newcommand{\Q}{\mathbb{Q}}
\newcommand{\I}{\mathbbm{1}}

\newcommand{\tr}{\text{tr}}
\newcommand{\ket}[1]{| #1 \rangle}
\newcommand{\bra}[1]{\langle #1|}
\newcommand{\ip}[2]{\langle #1|#2 \rangle}
\newcommand{\bracket}[3]{\langle #1|#2|#3 \rangle}
\newcommand{\sm}[1]{\left( \begin{smallmatrix} #1 \end{smallmatrix} \right)}

\newcommand{\be}{\begin{equation}}
\newcommand{\ee}{\end{equation}}
\newcommand{\bea}{\begin{eqnarray}}
\newcommand{\eea}{\end{eqnarray}}
\newcommand{\bes}{\begin{equation*}}
\newcommand{\ees}{\end{equation*}}
\newcommand{\beas}{\begin{eqnarray*}}
\newcommand{\eeas}{\end{eqnarray*}}

% ------------------------------------------------------------------------------

\newcommand{\x}{\mathrm{x}}
%\newcommand{\ket}[1]{|#1\rangle}
\newcommand{\ketbra}[1]{\ket{#1}\!\bra{#1}}
%\newcommand{\bra}[1]{\langle#1|}
\newcommand{\proj}[1]{\ket{#1}\!\bra{#1}}

\newcommand{\ten}{\otimes}
\newcommand{\Id}{\mathds{1}}
\newcommand{\zero}{\mathbf{0}}
\newcommand{\KBDS}{C^{s}}
\newcommand{\KBDSs}{C}
%\newcommand{\KBDS}{C^{\mbox{\tiny U}}}
\newcommand{\KTwo}{C^{2\times2}}
\renewcommand{\H}{\mathcal{H}}
\def\A{\mathcal{A}}
\def\B{\mathcal{B}}

\def\x{\mathrm{x}}
\def\y{\mathrm{y}}

\def\W{\mathcal{W}}


\def\sig{\widetilde{\sigma}}
\def\F{\mathcal{F}}
\def\N{\mathcal{N}}
\def\P{\mathbbm{P}}
\def\tr{\mathrm{tr}}

\def\L{\mathcal{L}}
\def\Q{\mathcal{Q}}
\def\NL{\mathcal{NS}}

\newtheorem{thm}{Theorem}[section]
\newtheorem*{thm*}{Theorem}
\newtheorem{cor}[thm]{Corollary}
\newtheorem{con}[thm]{Conjecture}
\newtheorem{lem}[thm]{Lemma}
\newtheorem*{lem*}{Lemma}
\newtheorem{prop}[thm]{Proposition}
\newtheorem{defn}[thm]{Definition}
\newtheorem{rem}[thm]{Remark}
\newtheorem{eg}[thm]{Example}

\newtheorem*{lipschitzLem*}{Lemma \ref{lipschitz}}
\newtheorem*{lipschitzCubeLem*}{Lemma \ref{lipschitzCube}}
\newtheorem*{pgmNearlyOptimalThm*}{Theorem \ref{pgmNearlyOptimal}}

\theoremstyle{definition}
\newtheorem{remark}{Remark}

\DeclareMathOperator*{\Oplus}{\bigoplus}
\DeclareMathOperator*{\Otimes}{\bigotimes}
% \DeclareMathOperator*{\Oplus}{\scalerel*{\bigoplus}{\textstyle\sum}}
% ------------------------------------------------------------------------------








%#####################################

\begin{document}

\title{Detecting Entanglement by State Preparation and a Fixed Measurement}
 
\author{ Jaemin Kim  }
\affiliation{School of Electrical Engineering, Korea Advanced Institute of Science and Technology (KAIST), 291 Daehak-ro, Yuseong-gu, Daejeon 34141, Republic of Korea }

\author{ Anindita Bera}
\affiliation{Institute of Physics, Faculty of Physics, Astronomy, and Informatics, Nicolaus Copernicus University, Grudziadzka 5, 87-100 Torun, Poland}

\author{Joonwoo Bae}
%\email{bae.joonwoo@gmail.com}
\affiliation{School of Electrical Engineering, Korea Advanced Institute of Science and Technology (KAIST), 291 Daehak-ro, Yuseong-gu, Daejeon 34141, Republic of Korea }


\author{Dariusz Chru\'sci\'nski}
\affiliation{Institute of Physics, Faculty of Physics, Astronomy, and Informatics, Nicolaus Copernicus University, Grudziadzka 5, 87-100 Torun, Poland}

%########################################

\begin{abstract}

It is shown that a fixed measurement setting, e.g., a measurement in the computational basis, can detect {\it all} entangled states by preparing multipartite quantum states, called {\it network states}. We present network states for both cases to construct decomposable entanglement witnesses (EWs) equivalent to the partial transpose criteria and also non-decomposable EWs that detect undistillable entangled states beyond the partial transpose criteria. Entanglement detection by state preparation can be extended to multipartite states such as graph states, a resource for measurement-based quantum computing. Our results readily apply to a realistic scenario, for instance, an array of superconducting qubits. neutral atoms, or photons, in which the preparation of a multipartite state and a fixed measurement are experimentally feasible. %From a fundamental point of view, our results show how to realize dynamics of dynamics of positive but not completely positive maps that distinguish entangled states from separable ones. 
% Our results provide a detour to implement nonphysical (nonpositive) operators using physically realizable quantum states.
\end{abstract}

%\pacs{03.65.Ud, 02.50.Le, 03.67.Ac}

\maketitle

\section{Introduction}

A set of observables, called entanglement witnesses (EWs), can distinguish entangled states from separable ones both theoretically and experimentally \cite{TERHAL2000319, PhysRevA.62.052310}. EWs are a versatile tool to characterize entangled states in general, i.e., multipartite quantum systems in arbitrary dimensions \cite{HORODECKI19961, GUHNE20091, RevModPhys.81.865, Chru_ci_ski_2014, Friis:2019aa}. They have also been developed for the verification of entanglement in a practical scenario where assumptions, e.g., measurement devices or dimensions of quantum systems, cannot be justified \cite{PhysRevLett.110.060405, PhysRevLett.110.060405, PhysRevLett.107.170403}. Remarkably, all entangled states can be verified in a fully device-independent manner \cite{Bowles:2018aa}. Experimentally certified entangled states enable one to achieve quantum advantages, such as efficient computation \cite{PhysRevLett.79.325}, higher channel capacities \cite{PhysRevA.95.052329, PhysRevLett.125.150502}, and a higher level of security in cryptographic protocols \cite{PhysRevLett.98.230501, Pironio_2009}.
 
In a realistic experimental scenario for detecting entangled states, particularly in the era of noisy-intermediate-scale-quantum technologies, limitations exist in manipulating quantum systems, where imperfections introducing quantum errors are naturally present \cite{Preskill2018quantumcomputingin}. For instance, one may attempt to circumvent varying measurement settings in most of the physical systems, superconducting qubits, e.g., \cite{doi:10.1063/1.5089550}, and neutral atoms, e.g.,\cite{Henriet2020quantumcomputing, Graham:2022aa}, where a fixed measurement setting in the computational basis $\{ |0\rangle_z, |1\rangle_z\}$ applies. Therefore, on the one hand, while noise is present in the current technologies, the certification of quantum properties such as entanglement is vital to achieving quantum advantages. However, on the other hand, for detecting entangled states, all of the schemes mentioned above relying on EWs ask experimenters to be able to handle experimental settings. 

 
In addition, general measurements, i.e., non-projective positive-operator-valued-measures, are often essential to construct EWs. They can be realized after interactions between systems and auxiliary systems followed by projective measurements on the auxiliary ones \cite{doran1994c}, see also \cite{PhysRevA.87.012106, PhysRevA.100.062317}. However, such interactions and measurements are also noisy within the currently available quantum technologies. Noisy EWs lead to loopholes in the detection of entanglement. On top of that, there are also quantum systems that hardly interact with each other such as photons, for which thus measurement strategies are limited. 

 
In this work, we establish a framework for detecting entangled states with a fixed measurement, say the $z$-direction, by preparing multipartite states that we call {\it network states}. Similarly to measurement-based quantum computation that realizes arbitrary unitary transformations by state preparation, we show that entanglement witnesses (EWs) can be estimated by preparing multipartite states. We present the construction of network states for decomposable EWs, which are equivalent to the partial transpose criteria, and also for non-decomposable EWs that detect bound entangled states beyond the partial transpose criteria, such as the Bell-diagonal EWs \cite{chruscinski2014class, PhysRevA.105.052401} from the Choi map \cite{Choi:1975aa} and its various generalizations \cite{PhysRevA.84.024302}, and the Breuer-Hall EW \cite{PhysRevLett.97.080501, Hall_2006}. Our results apply to multipartite systems: graph states \cite{PhysRevA.69.062311}, a resource for measurement-based quantum computing \cite{PhysRevLett.86.5188}, can be detected by state preparation and a fixed measurement. 
 
 
\section{Entangled states}

Let us begin by summarizing EWs and collecting related results. Let $W$ denote an observable for bipartite systems on a Hilbert space $\H\otimes \H$ where $\dim\H=d$. An observable $W$ is an EW if we have, for some entangled state $\rho_{\mathrm{ent}}$
\bea
\tr[W\rho_{\mathrm{ent}}] <0,~~\mathrm{whereas}~~\tr[W\sigma_{\mathrm{sep}}] \geq 0,~\forall \sigma_{\mathrm{sep}} \in \mathrm{SEP} \nonumber\label{eq:ew}
\eea
where $\mathrm{SEP}$ denotes the set of separable states. EWs can be extended to multipartite states and characterize their various properties, such as the $k$-separability that characterizes $n$-partite states, which are separable in $k$-partite splittings. EWs can also be used to certify the fidelity in the state preparation \cite{PhysRevA.76.030305}. We also emphasize that an EW corresponds to an observable: its experimental estimation concludes entangled states without state identification by quantum tomography. 

One of the intriguing properties of entangled states is the irreversibility in manipulations of entanglement. Entangled states from which no entanglement can be extracted, though entanglement is needed for their preparation, are identified as {\it undistillable} or {\it bound} entangled states \cite{PhysRevA.61.062313, PhysRevLett.86.2681}. Multipartite quantum states that remain positive after partial transpose (PPT) turn out to be undistillable. Remarkably, PPT entangled states can be used to activate other entangled states \cite{horodecki1999bound}. 

Then, non-PPT entangled states can be characterized by decomposable EWs that have a general form as follows,
\bea
W = P+Q^{\Gamma},~~\mathrm{for}~~P,~Q\geq 0\nonumber.
\eea 
Here $\Gamma$ denotes the partial transpose. Then, non-decomposable EWs, which cannot be in the form above, can detect PPT entangled states. In general, it is highly non-trivial to construct non-decomposable EWs, which are the main object in the mathematically challenging problem of classifying positive linear maps of Operator Algebras \cite{Stormer:1963aa}, see also Refs. \cite{bera2022generalizing, bera2022class}. 


\begin{figure}[t]
\centering
\includegraphics[width=9.0cm ]{fig2}
\caption{ A bipartite network state $N_{23}$ is prepared on $A_2A_3B_2B_3$ to detect an entangled state on $A_1B_1$. Bi-interactions and measurements in the computational basis can construct EWs, see also Eq. (\ref{eq:expew}) in the text. Gray boxes denote a measurement in the basis $|\phi^{+}\rangle$, which is equivalent to a measurement in the computational basis after applying of a controlled-NOT gate followed by a Hadamard gate. }
\label{fig:fig1}
\end{figure}

\section{Measurement-Based EWs }

Let us now illustrate entanglement detection by state preparation with two-qubit EWs, in which all EWs are decomposable. We in particular consider an EW $W = |\phi^+\rangle\langle \phi^+|^{\Gamma} $ that detects a state $|\psi^{-}\rangle$, where four Bell states are written by $|\phi^{\pm}\rangle = (|00\rangle \pm |11\rangle)/\sqrt{2}$ and $|\psi^{\pm}\rangle = (|01\rangle \pm |10\rangle)/\sqrt{2}$.

\subsection{Two-qubit network states}

To realize entanglement detection by state preparation, we introduce a multipartite state, called {\it a network state}, to construct an EW $W = |\phi^+\rangle\langle \phi^+|^{\Gamma}$ as follows,
\bea
N_{23} &=& \frac{1}{4} |\psi^-\rangle_{A_2B_2} \langle \psi^-| \otimes |\phi^+\rangle_{A_3B_3} \langle \phi^+| + \nonumber \\
&&\frac{1}{12} (\mathbbm{1} - |\psi^-\rangle_{A_2B_2} \langle \psi^-|) \otimes (\mathbbm{1} - |\phi^+\rangle_{A_3B_3} \langle \phi^+|),~~~~ ~\label{eq:wr}
\eea
which is located at sites $A_2A_3B_2B_3$, see Fig. \ref{fig:fig1}. We then place a state of interest $\rho$ in at $A_1B_1$, denoted by $\rho_{1}:=\rho^{( A_1B_1)}$.
It holds that
\bea
\tr[\rho W ] =  16~ \tr  [\rho_1\otimes N_{23} (\frac{1}{2}\mathbbm{1} - | \phi^+\rangle_{A_3B_3} \langle \phi^+|) \otimes P^{(12)}].~~ 
\label{eq:ewr}
\eea
where $P^{(12 )} = |\phi^{+}\rangle_{A_1A_2}\langle \phi^+| \otimes |\phi^{+}\rangle_{B_1B_2}\langle \phi^+|$. One can find the expectation value by estimating a singlet fraction,
\bea
_{A_3 B_3}  \langle \phi^{+}| \tr_{12}[ \rho_1 \otimes N_{23} ~P^{(12)} ]  |\phi^{+}\rangle_{A_3 B_3}   = \frac{1}{8} - \frac{1}{4}\tr[\rho W],~~
\label{eq:expew}
\eea
where the left-hand-side can be obtained by preparing a network state followed by a fixed measurement. Once a Bell measurement reports an outcome $P^{(12)} $, a singlet fraction is estimated by finding the probability of having outcome $|\phi^{+}\rangle$ on $A_3B_3$. In fact, an entangled state $\rho$ is detected if the probability, i.e., the left-hand-side in Eq. (\ref{eq:expew}), is greater than $1/8$, since $\tr[\sigma_{\mathrm{sep}}W ]\geq 0 $ for all separable states $\sigma_{\mathrm{sep}}$. 


Experimental resources to obtain the left-hand side in Eq. (\ref{eq:expew}) are summarized as follows. One prepares a four-partite network state $N_{23}$. Note that for two-qubit cases, a network state in Eq. (\ref{eq:wr}) is a variation of a Smolin state, a four-partite bound entangled state \cite{PhysRevA.63.032306}. Note that a Smolin state has been realized with photonic qubits \cite{ Amselem:2009aa, PhysRevLett.105.130501, PhysRevLett.109.040501}. The detection scheme also needs a measurement in the basis $|\phi^+\rangle$, which is equivalent to the capability of realizing a controlled-NOT gate, a Hadamard gate, and a fixed-measurement in the $z$-direction; see also Fig. \ref{fig:fig1}. All these are compatible with the resources to realize measurement-based quantum computing.



\subsection{ EWs via entanglement activation}

To show a general construction of network states for arbitrary EWs for high-dimensional quantum systems, let us first recall the result in Ref. \cite{PhysRevLett.96.150501} that all EWs can be expressed in the following form,
\bea
W_{A_2 B_2 } = \tr_{A_3B_3} \big[N (\eta \mathbbm{I} - |\phi_{d}^+ \rangle_{A_3B_3} \langle \phi_{d}^+ |)\big], \label{eq:ewm}
\eea
for some multipartite network state $N := N^{(A_2 B_2 A_3 B_3)}$ and a parameter $\eta \in [1/d,1 )$, where $|\phi_{d}^+\rangle = \sum_{j=0}^{d-1}|jj\rangle /\sqrt{d}$. Note that the parameter $\eta$ satisfies the condition, $\eta\geq E_d [N]$ where $E_d$ is called a maximal singlet fraction,
\bea
E_d[N] &=& \sup_{\F}~ \langle \phi_{d}^+| \F[ N ] |\phi_{d}^+\rangle ~\mathrm{for~a~local ~filtering~}\F   \nonumber \\
&& \F[N] = \frac{ K_{A}\otimes K_B N K_{A}^{\dagger}\otimes K_{B}^{\dagger} }{ \tr[  K_{A}^{\dagger}K_{A} \otimes K_{B}^{\dagger}K_{B} N ]},\label{eq:fil}
\eea
where $K_A: \H_{A_1A_2}\rightarrow \mathbbm{C}^d $ and $K_B: \H_{B_1B_2}\rightarrow \mathbbm{C}^d $. It is worth mentioning that EWs in Eq. (\ref{eq:ewm}) identify entangled states that activate a network state in the sense that 
\bea
E_d[ \sigma\otimes N]>\eta, ~\mathrm{whereas}~ {E_d [N]\leq \eta}.\label{eq:ac}
\eea 
In fact, all entangled states can be used to activate some other state: in Eq. (\ref{eq:ac}) a state $\sigma$ is entangled if and only if it can activate some other entangled state $N$.  

 
\subsection{ General construction of network states }

We now present a construction of a network state for a given EW, see Eq. (\ref{eq:ewm}). For convenience, let us consider an EW $W$ on $d\otimes d$, and its decomposition may be found as follows,
\bea
W = \sum_{j} a_j W(j)^T ~\mathrm{with}~ W(j) \geq 0 ~\mathrm{and}~ a_j \in \mathbbm{R}, \nonumber
\eea
so that one can choose normalized non-negative operators $\{ \Pi(i) \geq 0\}$ and constants $\{ c_j\}$ such that,
\bea
N_{23} = \sum_{j} c_j  W(j)_{A_2B_2} \otimes  \Pi(j)_{A_3B_3}  \label{eq:ns}
\eea
and % {\color{blue} there exists a constant $k > 0$ such that
\bea
W_2^T = ~ k \, \tr_{3} [N_{23}  ( \eta \mathbbm{1} - |\phi_{d}^+ \rangle_{A_3B_3} \langle \phi_{d}^+ |  )  ]. \label{eq:ewn}
\eea
for some $\eta\geq 1/d$ and $k>0$. One can find $\{a_j\}$ and $\{c_j \}$ are related as follow,
$$
a_j = k \, c_j (\eta - \bracket{\phi_{d}^+}{\Pi(j)}{\phi_{d}^+}).\nonumber
$$
For a state $\rho_1 = \rho^{(A_1B_1)}$ it holds that
\bea
\tr[\rho W] %&=& d^2 ~ \tr[\rho_1 \otimes W_{2}^T P^{(12)}  ]  \nonumber\\ 
&\propto& \tr[\rho_1 \otimes N_{23} ~ P^{(12)} \otimes ( \eta \mathbbm{1} - |\phi_{d}^+ \rangle_{A_3B_3} \langle \phi_d^+ |  ) ]. ~~~ \label{eq:m}
\eea
For an entangled state $\rho$ detected by $W$, i.e., $\tr[\rho W]<0$, Eq. (\ref{eq:m}) shows that
\bea
\eta &<&  _{A_3B_3}\langle \phi_{d}^+| \Lambda^{(1\rightarrow 3)} [\rho_1]|\phi_{d}^+\rangle_{A_3B_3} \nonumber \\
&&\mathrm{where}~ \Lambda^{( 1\rightarrow 3)} [\rho_1]= \frac{ \tr_{12}[\rho_{1}\otimes N_{23} P^{(12)}] }{ \tr[\rho_1\otimes N_{23} P^{(12)}]  }. ~~\label{eq:v}
\eea
The above may be rephrased by a teleportation protocol: once a measurement $P^{(12)}$ is successful, a state $\rho$ prepared at $A_1B_1$ is sent to $A_3B_3$ via a network state $N_{23}$. Then, a singlet fraction is estimated and compared with $\eta$, which is pre-determined by a network state to realize an EW. As mentioned, experimental resources for the realization contain preparing a network state and Bell measurements that require bi-interactions and a fixed local measurement setting; see also Fig. \ref{fig:fig1}. In Appendix \ref{app1}, we reproduce a network state in Eq. (\ref{eq:wr}) by applying the general construction above. 


\section{Examples}


Let us then apply the general construction of network states and present network states for decomposable and non-decomposable EWs. We recall that identifying all EWs, equivalent to characterizing the set of separable states, is a challenging mathematical problem \cite{Stormer:1963aa}. Its computational complexity also belongs to NP-Hard \cite{10.1145/780542.780545}. In what follows, we consider EWs known so far and show network states to construct them. 

To this end, let $P_{st}$ denote a projection onto a Bell state, for $s,t = 0,\ldots, d-1$, in a dimension $d$,
\bea
P_{st} = |\phi_{st}\rangle \langle \phi_{st}| ~\mathrm{where}~|\phi_{st} \rangle = \frac{1}{\sqrt{d}} \sum_{j = 0}^{d-1} \omega^{t j } \ket{j } \ket{j + s}, ~~ \label{eq:1} 
\eea
where $\omega = e^{2\pi i/d}$.
Projectors onto symmetric and anti-symmetric subspaces are denoted by $S_d$ and $A_d$, respectively,
\bea
S_d = \frac{\mathbbm{1+F} }{2}~~\mathrm{and} ~~A_d = \frac{\mathbbm{1 - F} }{2},
\eea
where $\mathbbm{F} = d P_{00}^{\Gamma}$ is a flip operator \cite{PhysRevA.61.062313}. Note that $\tr[A_d]=d(d-1)/2$ and $\tr[S_d] = d(d+1)/2$. Interestingly, high-dimensional Bell states and projections onto symmetric and anti-symmetric subspaces suffice to construct network states for known non-decomposable maps. 



\subsection{ Decomposable EWs: the partial transposition }

Firstly, we consider a decomposable EW $W = Q^\Gamma$ for $Q\geq 0$ and $\tr[Q]=1$, for which a network state can be constructed as follows. We write by $\lambda:=\max_{i}|\lambda_i|$ where $\{\lambda_i\} $ are eigenvalues of an EW $W$ and a network state is obtained as,
\bea
N_{23}&=& c_1\left( \frac{\lambda\mathbbm{1} - Q^\Gamma }{\lambda d^2 - 1 }\right)^{(2)} \otimes P_{00}^{(3)} \nonumber\\
&&+ c_2 \left( \frac{ \lambda \mathbbm{1} +  Q^{\Gamma} }{\lambda d^2 + 1}\right)^{(2)} \otimes \left( \frac{\mathbbm{1} - P_{00}}{d^2 - 1}\right)^{(3)}, \label{eq:nsd}
\eea
where superscript $(j)$ stands for systems $A_jB_j$ and 
\bea
c_1 = \frac{d^2 \lambda - 1}{d^3 \lambda  + d -2} ~~\mathrm{and}~ ~ c_2 =\frac{(d-1)(d^2 \lambda +1)}{d^3 \lambda  +d-2}. \nonumber
\eea
In the other way around, from a network state $N_{23}$ in Eq. (\ref{eq:nsd}) one can reproduce a decomposable EW, see Eq. (\ref{eq:ewm})
\bea
\tr_3[ N_{23 } (\frac{1}{d} \mathbbm{1} - |\phi_{00}\rangle_{A_3B_3}\langle \phi_{00}|)] = \frac{2(d-1)}{d(d^3 \lambda + d -2)} Q^\Gamma \propto Q^\Gamma.\nonumber
% \\ \tr_3[ N_{23}^{(\mathrm{BH})}~ (\frac{1}{d} \mathbbm{1} - |\phi_{00}\rangle_{A_3B_3}\langle \phi_{00}|)] = \frac{2(d-1)}{d(d^3 \lambda + d -2)} Q^\Gamma \propto Q^\Gamma.\nonumber
\eea
Hence, the partial transpose criteria \cite{PhysRevLett.77.1413} can be generally realized by preparing a network state with a fixed measurement. 

As an instance, a network state for the decomposable and optimal EW $W = P_{00}^{\Gamma}$ can be found as 
\bea
\frac{1}{d+2} \left( \frac{A_d}{\tr A_d} \right)^{(2)} \otimes P_{00}^{(3)}  +  \frac{d+1}{d+2} \left( \frac{S_d}{\tr S_d} \right)^{(2)} \otimes \left( \frac{\mathbbm{1}-P_{00}}{d^2-1} \right)^{(3)}. \nonumber
\eea
The network state above is known as a symmetric state being $UUVV^*$-invariant, and has been used to activate entanglement distillation with an infinitesimal amount of bound entanglement \cite{vollbrecht2002activating}.

\subsection{ Non-decomposable EWs }

Secondly, to construct network states for non-decomposable EWs, we introduce paired Bell-diagonal (PBD) states as follows,
\bea
N_{23}^{(\mathrm{PBD})}(\Vec{\lambda}) = \sum_{s=0}^{d-1} \lambda_s ~\frac{1}{d} \sum_{t=0}^{d-1}  P_{st}^{(2)} \otimes P_{st}^{(3)}, \label{eq:PBD}
\eea
where $\Vec{\lambda}=(\lambda_0, \ldots, \lambda_{d-1})$ and $\sum_{s=0}^{d-1} \lambda_s = 1$. 


\subsubsection{ Bell-diagonal EWs}

A network state in Eq. (\ref{eq:PBD}) can be used to estimate expectation values of Bell-diagonal EWs \cite{chruscinski2014class},
 \bea
W[\Vec{\lambda}] = \sum_{s=0}^{d-1} \lambda_s \Pi_s - P_{00},~~\mathrm{where}~~ \Pi_s = \sum_{t=0}^{d-1} P_{st}.
\label{eq:Wa}
\eea
Note that the Choi map and its generalizations are well-known instances. Then, PBD network states construct Bell-diagonal EWs as follows, 
\bea
\frac{\lambda_0}{d} W_2^T [\Vec{\lambda}] &=& \tr_3[ N_{23}^{(\mathrm{PBD})}(\Vec{\lambda}) (\lambda_0 \mathbbm{1} - |\phi_{00}\rangle_{A_3B_3}\langle \phi_{00}|)]. \nonumber 
\eea
Hence, it is shown that all entangled states characterized by Bell-diagonal EWs can be detected by a fixed measurement and state preparation. 

\subsubsection{Choi EWs}

Instances of Bell-diagonal EWs for $d=3$ contain the Choi map \cite{Choi:1975aa} and its generalizations \cite{PhysRevA.84.024302, doi:10.1142/S1230161213500066}, that detect PPT entangled states. As it is shown in Eq. (\ref{eq:v}), once a filtering operation with a PBD network state in Eq. (\ref{eq:PBD}) is successful, entangled states are concluded by finding a singlet fraction. For the case the Choi map, entangled states are detected if the singlet fraction is larger than $2/3$. The proof is provided in Appendix \ref{app2}. 

\subsubsection{Multipartite bound entangled states as a network state}

We also observe that a PBD state for $d=2$ with $\vec{\lambda} = (1/2,1/2)$ corresponds to a Smolin state \cite{PhysRevA.63.032306}, 
\bea
\rho_S = \frac{1}{4}\sum_{s,t=0,1 } P_{st}^{(A_2B_2)} \otimes P_{st}^{(A_3B_3)}. \nonumber
\eea
The state is invariant under permutations of $A_2A_3B_2B_3$ and remains PPT in any bipartite splitting: it is called a four-partite unlockable and undistillable entangled state. A Smolin state can be used to activate distillation of entanglement. 

A Smolin state can be generalized to higher dimensions, with $\vec{\lambda} = (1/d,\ldots,1/d)$, 
\bea
N_{23} (\Vec{\lambda}) = \frac{1}{d^2} \sum_{s=0}^{d-1} \sum_{t=0}^{d-1}  P_{st}^{(A_2 B_2)} \otimes P_{st}^{(A_3B_3)}. \nonumber
\eea
However, a Smolin state in a higher dimension $d>2$ no longer remains PPT in the bipartite splitting $A_2A_3:B_2B_3$. The network state then realizes an EW,
\bea
W = \frac{1}{d}\sum_{s=0}^{d-1}  \Pi_s - P_{00} = \frac{1}{d} \mathbbm{1} - P_{00}  \nonumber
\eea
which is decomposable. It is also an EW that is derived from a reduction map \cite{PhysRevA.59.4206}. Note that a Smolin state corresponds to a network state that realizes a reduction EW for $d=2$.

\subsubsection{ EWs from the Breuer-Hall map}


The Breuer-Hall (BH) map shown in Refs. \cite{PhysRevLett.97.080501, Hall_2006} derives highly non-trivial non-decomposable EWs,
\bea
\Lambda_{\textrm{BH}}(\rho) = \frac{1}{d-2} ( \tr(\rho) \mathbbm{1} - \rho - U \rho^T U^\dagger) \label{eq:BH map}
\eea
where $U$ is an skew-symmetric unitary operator satisfying $UU^\dagger = \mathbbm{1}$ and $U^T = -U$.
Then the BH EW is obtained as follows,
\bea
W_{\mathrm{BH}} = \frac{1}{d-2} ( \frac{1}{d}\mathbbm{1} - P_{00} - \frac{1}{d}\mathbbm{F}'), \label{eq:BH EW}
\eea
where $\mathbbm{F}' \equiv (\mathbbm{1} \otimes U) \mathbbm{F} (\mathbbm{1} \otimes U^\dagger)$. Note that the BH EW is optimal.  

A network state for the BH EW is obtained as follows,
\bea
N_{23}^{(\textrm{BH})}  %& = & c_0 N^{(23)}_{\textrm{PBD}} + (1-c_0) N^{(23)}_{\textrm{skew}}, \label{eq:bhn} \\
& = & c_0 \frac{1}{d^2} \sum_{s=0}^{d-1} \sum_{t=0}^{d-1} P_{st}^{(2)}  \otimes P_{st}^{(3)} \nonumber \\
&& + c_1 \left( \frac{\mathbbm{1} + \mathbbm{F}' }{d^2+d} \right)^{(2)} \otimes P_{00}^{(3)}   \nonumber \\
&& + c_2 \left( \frac{\mathbbm{1} - \mathbbm{F}' }{d^2-d} \right)^{(2)} \otimes \left( \frac{\mathbbm{1} - P_{00}}{d^2-1} \right)^{(3)}, \label{eq:nbhh} 
\eea
where
\bea
c_0 = \frac{2d^2-2d}{3d^2 -3d +2},~\mathrm{and}~ c_1= \frac{d+1}{3d^2-3d+2}, \nonumber
\eea
and $c_2 = 1-c_0-c_1$. One can find that, from Eq. (\ref{eq:ewn})
\bea
W_{\mathrm{BH}}^T ~~\propto~~ \tr_3[ N_{23}^{(\mathrm{BH})}~ (\frac{1}{d} \mathbbm{1} - |\phi_{00}\rangle_{A_3B_3}\langle \phi_{00}|)]. \label{eq:bh} 
\eea
Once a filtering operation in Eq. (\ref{eq:v}) is successful, entangled states are detected if a singlet fraction of a resulting state on $A_3B_3$ is larger than $1/d$.


\begin{figure}[t]
\centering
\includegraphics[width=9.2cm ]{chain}
\caption{ A tripartite graph state can be detected by preparing a network state, Bell measurements, and a fixed measurement. Entangled states of arrayed qubits can be detected by state preparation and a fixed measurement. }
\label{fig:chain}
\end{figure}

 
\subsection{ EWs for multipartite systems}

Thirdly, entanglement detection by state preparation can be extended to multipartite quantum states. We here, in particular, consider graph states, a class of states as a resource for measurement-based quantum computing \cite{PhysRevLett.86.5188}. Let us again present an instance for a three-qubit graph state, a Greenberger–Horne–Zeilinger (GHZ) state $|\psi\rangle = (|000\rangle +|111\rangle)/\sqrt{2}$ \cite{PhysRevA.62.062314}, see Fig. \ref{fig:chain}. An EW to detect a GHZ state may be given as,
\bea
W = \frac{1}{2} \mathbbm{1} - |\psi \rangle \langle \psi|.
\eea
A network state for an EW above can be constructed as,
\bea
N_{23} = \frac{1}{8} \sum_{a,b,c=0}^{1} \psi_{abc}^{(A_2 B_2 C_2)} \otimes \psi_{abc}^{(A_3 B_3 C_3)}.
\eea
where $\psi_{abc} = |\psi_{abc}\rangle \langle \psi_{abc}|$,
\bea
|\psi_{abc}\rangle = Z^a \otimes X^b \otimes X^c |\psi\rangle, ~~a,b,c \in \{0,1\}
\eea
with Pauli matrices $X$ and $Z$. It holds that 
%\bea
%\tr[\rho W] =  8 \tr[\rho_1\otimes N_{23} P_{12} \otimes (\frac{\mathbbm{1}}{2} - |\psi \rangle\langle \psi |)_{3}], \nonumber
%\eea
\bea
{}_{A_3B_3C_3} \langle \psi |\rho_1 \otimes N_{23} ~P^{(12)} |\psi \rangle_{A_3B_3C_3} = \frac{1}{16} - \frac{1}{8}\tr[\rho W], \nonumber
\eea
which shows detection of a genuinely multipartite entangled state $\rho$ by finding that the left-hand-side is greater than $1/16$. Further generalization for detecting entangled $n$-qubit graph states is provided in Appendix \ref{app3}. \\


\subsection{ To construct non-decomposable EWs } 


Finally, let us investigate two entangled states defined by an EW. One denotes an entangled state $\rho_1$ detected by an EW, and the other $N_{23}$ realizing an EW by its preparation; see also Eq. (\ref{eq:ewm}). The result in Ref. \cite{PhysRevLett.96.150501} shows that an EW detects a set of entangled states that can activate its network state. Since a pair of PPT states cannot activate each other, either the states $\rho_1$ or $N_{23}$ must be non-PPT. Hence, a network state $N_{23}$ to detect a PPT entangled state $\rho_1$ should be non-PPT. We thus conclude that multipartite non-PPT entangled states can construct non-decomposable EWs, which are then highly non-trivial. \\


\section*{Conclusion} 
In conclusion, we have established the framework of detecting entangled states in terms of state preparation and a fixed measurement. We have presented the construction of network states that allow one to estimate EWs. Network states for EWs known so far are explicitly provided, both decomposable and non-decomposable cases. Our results shed new light on detecting entangled states: a measurement setting for estimating EWs is replaced by a state preparation and then simplified to a fixed one. 


%From the fundamental point of view, our results show an entanglement-based scheme to realize positive but not completely positive (PnCP) maps, that distinguish entangled states from separable ones. Positive maps may lead to more advantages in information processing, such as channel discrimination tasks \cite{Regula2021operational}. 


\section*{Acknowledgement}

This work is supported by National Research Foundation of Korea (NRF-2021R1A2C2006309, NRF-2022M1A3C2069728) and the Institute for Information \& Communication Technology Promotion (IITP) (the ITRC Program/IITP-2023-2018-0-01402). AB and DC were supported by the Polish National Science Center project No. 2018/30/A/ST2/00837.

%\bibliography{bibedsp}

%apsrev4-2.bst 2019-01-14 (MD) hand-edited version of apsrev4-1.bst
%Control: key (0)
%Control: author (8) initials jnrlst
%Control: editor formatted (1) identically to author
%Control: production of article title (0) allowed
%Control: page (0) single
%Control: year (1) truncated
%Control: production of eprint (0) enabled
\begin{thebibliography}{49}%
\makeatletter
\providecommand \@ifxundefined [1]{%
 \@ifx{#1\undefined}
}%
\providecommand \@ifnum [1]{%
 \ifnum #1\expandafter \@firstoftwo
 \else \expandafter \@secondoftwo
 \fi
}%
\providecommand \@ifx [1]{%
 \ifx #1\expandafter \@firstoftwo
 \else \expandafter \@secondoftwo
 \fi
}%
\providecommand \natexlab [1]{#1}%
\providecommand \enquote  [1]{``#1''}%
\providecommand \bibnamefont  [1]{#1}%
\providecommand \bibfnamefont [1]{#1}%
\providecommand \citenamefont [1]{#1}%
\providecommand \href@noop [0]{\@secondoftwo}%
\providecommand \href [0]{\begingroup \@sanitize@url \@href}%
\providecommand \@href[1]{\@@startlink{#1}\@@href}%
\providecommand \@@href[1]{\endgroup#1\@@endlink}%
\providecommand \@sanitize@url [0]{\catcode `\\12\catcode `\$12\catcode
  `\&12\catcode `\#12\catcode `\^12\catcode `\_12\catcode `\%12\relax}%
\providecommand \@@startlink[1]{}%
\providecommand \@@endlink[0]{}%
\providecommand \url  [0]{\begingroup\@sanitize@url \@url }%
\providecommand \@url [1]{\endgroup\@href {#1}{\urlprefix }}%
\providecommand \urlprefix  [0]{URL }%
\providecommand \Eprint [0]{\href }%
\providecommand \doibase [0]{https://doi.org/}%
\providecommand \selectlanguage [0]{\@gobble}%
\providecommand \bibinfo  [0]{\@secondoftwo}%
\providecommand \bibfield  [0]{\@secondoftwo}%
\providecommand \translation [1]{[#1]}%
\providecommand \BibitemOpen [0]{}%
\providecommand \bibitemStop [0]{}%
\providecommand \bibitemNoStop [0]{.\EOS\space}%
\providecommand \EOS [0]{\spacefactor3000\relax}%
\providecommand \BibitemShut  [1]{\csname bibitem#1\endcsname}%
\let\auto@bib@innerbib\@empty
%</preamble>
\bibitem [{\citenamefont {Terhal}(2000)}]{TERHAL2000319}%
  \BibitemOpen
  \bibfield  {author} {\bibinfo {author} {\bibfnamefont {B.~M.}\ \bibnamefont
  {Terhal}},\ }\bibfield  {title} {\bibinfo {title} {Bell inequalities and the
  separability criterion},\ }\href
  {https://doi.org/https://doi.org/10.1016/S0375-9601(00)00401-1} {\bibfield
  {journal} {\bibinfo  {journal} {Physics Letters A}\ }\textbf {\bibinfo
  {volume} {271}},\ \bibinfo {pages} {319} (\bibinfo {year}
  {2000})}\BibitemShut {NoStop}%
\bibitem [{\citenamefont {Lewenstein}\ \emph {et~al.}(2000)\citenamefont
  {Lewenstein}, \citenamefont {Kraus}, \citenamefont {Cirac},\ and\
  \citenamefont {Horodecki}}]{PhysRevA.62.052310}%
  \BibitemOpen
  \bibfield  {author} {\bibinfo {author} {\bibfnamefont {M.}~\bibnamefont
  {Lewenstein}}, \bibinfo {author} {\bibfnamefont {B.}~\bibnamefont {Kraus}},
  \bibinfo {author} {\bibfnamefont {J.~I.}\ \bibnamefont {Cirac}},\ and\
  \bibinfo {author} {\bibfnamefont {P.}~\bibnamefont {Horodecki}},\ }\bibfield
  {title} {\bibinfo {title} {Optimization of entanglement witnesses},\ }\href
  {https://doi.org/10.1103/PhysRevA.62.052310} {\bibfield  {journal} {\bibinfo
  {journal} {Phys. Rev. A}\ }\textbf {\bibinfo {volume} {62}},\ \bibinfo
  {pages} {052310} (\bibinfo {year} {2000})}\BibitemShut {NoStop}%
\bibitem [{\citenamefont {Horodecki}\ \emph {et~al.}(1996)\citenamefont
  {Horodecki}, \citenamefont {Horodecki},\ and\ \citenamefont
  {Horodecki}}]{HORODECKI19961}%
  \BibitemOpen
  \bibfield  {author} {\bibinfo {author} {\bibfnamefont {M.}~\bibnamefont
  {Horodecki}}, \bibinfo {author} {\bibfnamefont {P.}~\bibnamefont
  {Horodecki}},\ and\ \bibinfo {author} {\bibfnamefont {R.}~\bibnamefont
  {Horodecki}},\ }\bibfield  {title} {\bibinfo {title} {Separability of mixed
  states: necessary and sufficient conditions},\ }\href
  {https://doi.org/https://doi.org/10.1016/S0375-9601(96)00706-2} {\bibfield
  {journal} {\bibinfo  {journal} {Physics Letters A}\ }\textbf {\bibinfo
  {volume} {223}},\ \bibinfo {pages} {1} (\bibinfo {year} {1996})}\BibitemShut
  {NoStop}%
\bibitem [{\citenamefont {G{\"u}hne}\ and\ \citenamefont
  {T{\'o}th}(2009)}]{GUHNE20091}%
  \BibitemOpen
  \bibfield  {author} {\bibinfo {author} {\bibfnamefont {O.}~\bibnamefont
  {G{\"u}hne}}\ and\ \bibinfo {author} {\bibfnamefont {G.}~\bibnamefont
  {T{\'o}th}},\ }\bibfield  {title} {\bibinfo {title} {Entanglement
  detection},\ }\href
  {https://doi.org/https://doi.org/10.1016/j.physrep.2009.02.004} {\bibfield
  {journal} {\bibinfo  {journal} {Physics Reports}\ }\textbf {\bibinfo {volume}
  {474}},\ \bibinfo {pages} {1} (\bibinfo {year} {2009})}\BibitemShut {NoStop}%
\bibitem [{\citenamefont {Horodecki}\ \emph {et~al.}(2009)\citenamefont
  {Horodecki}, \citenamefont {Horodecki}, \citenamefont {Horodecki},\ and\
  \citenamefont {Horodecki}}]{RevModPhys.81.865}%
  \BibitemOpen
  \bibfield  {author} {\bibinfo {author} {\bibfnamefont {R.}~\bibnamefont
  {Horodecki}}, \bibinfo {author} {\bibfnamefont {P.}~\bibnamefont
  {Horodecki}}, \bibinfo {author} {\bibfnamefont {M.}~\bibnamefont
  {Horodecki}},\ and\ \bibinfo {author} {\bibfnamefont {K.}~\bibnamefont
  {Horodecki}},\ }\bibfield  {title} {\bibinfo {title} {Quantum entanglement},\
  }\href {https://doi.org/10.1103/RevModPhys.81.865} {\bibfield  {journal}
  {\bibinfo  {journal} {Rev. Mod. Phys.}\ }\textbf {\bibinfo {volume} {81}},\
  \bibinfo {pages} {865} (\bibinfo {year} {2009})}\BibitemShut {NoStop}%
\bibitem [{\citenamefont {Chru{\'{s}}ci{\'{n}}ski}\ and\ \citenamefont
  {Sarbicki}(2014)}]{Chru_ci_ski_2014}%
  \BibitemOpen
  \bibfield  {author} {\bibinfo {author} {\bibfnamefont {D.}~\bibnamefont
  {Chru{\'{s}}ci{\'{n}}ski}}\ and\ \bibinfo {author} {\bibfnamefont
  {G.}~\bibnamefont {Sarbicki}},\ }\bibfield  {title} {\bibinfo {title}
  {Entanglement witnesses: construction, analysis and classification},\ }\href
  {https://doi.org/10.1088/1751-8113/47/48/483001} {\bibfield  {journal}
  {\bibinfo  {journal} {Journal of Physics A: Mathematical and Theoretical}\
  }\textbf {\bibinfo {volume} {47}},\ \bibinfo {pages} {483001} (\bibinfo
  {year} {2014})}\BibitemShut {NoStop}%
\bibitem [{\citenamefont {Friis}\ \emph {et~al.}(2019)\citenamefont {Friis},
  \citenamefont {Vitagliano}, \citenamefont {Malik},\ and\ \citenamefont
  {Huber}}]{Friis:2019aa}%
  \BibitemOpen
  \bibfield  {author} {\bibinfo {author} {\bibfnamefont {N.}~\bibnamefont
  {Friis}}, \bibinfo {author} {\bibfnamefont {G.}~\bibnamefont {Vitagliano}},
  \bibinfo {author} {\bibfnamefont {M.}~\bibnamefont {Malik}},\ and\ \bibinfo
  {author} {\bibfnamefont {M.}~\bibnamefont {Huber}},\ }\bibfield  {title}
  {\bibinfo {title} {Entanglement certification from theory to experiment},\
  }\href {https://doi.org/10.1038/s42254-018-0003-5} {\bibfield  {journal}
  {\bibinfo  {journal} {Nature Reviews Physics}\ }\textbf {\bibinfo {volume}
  {1}},\ \bibinfo {pages} {72} (\bibinfo {year} {2019})}\BibitemShut {NoStop}%
\bibitem [{\citenamefont {Branciard}\ \emph {et~al.}(2013)\citenamefont
  {Branciard}, \citenamefont {Rosset}, \citenamefont {Liang},\ and\
  \citenamefont {Gisin}}]{PhysRevLett.110.060405}%
  \BibitemOpen
  \bibfield  {author} {\bibinfo {author} {\bibfnamefont {C.}~\bibnamefont
  {Branciard}}, \bibinfo {author} {\bibfnamefont {D.}~\bibnamefont {Rosset}},
  \bibinfo {author} {\bibfnamefont {Y.-C.}\ \bibnamefont {Liang}},\ and\
  \bibinfo {author} {\bibfnamefont {N.}~\bibnamefont {Gisin}},\ }\bibfield
  {title} {\bibinfo {title} {Measurement-device-independent entanglement
  witnesses for all entangled quantum states},\ }\href
  {https://doi.org/10.1103/PhysRevLett.110.060405} {\bibfield  {journal}
  {\bibinfo  {journal} {Phys. Rev. Lett.}\ }\textbf {\bibinfo {volume} {110}},\
  \bibinfo {pages} {060405} (\bibinfo {year} {2013})}\BibitemShut {NoStop}%
\bibitem [{\citenamefont {Bae}\ \emph {et~al.}(2011)\citenamefont {Bae},
  \citenamefont {Hwang},\ and\ \citenamefont {Han}}]{PhysRevLett.107.170403}%
  \BibitemOpen
  \bibfield  {author} {\bibinfo {author} {\bibfnamefont {J.}~\bibnamefont
  {Bae}}, \bibinfo {author} {\bibfnamefont {W.-Y.}\ \bibnamefont {Hwang}},\
  and\ \bibinfo {author} {\bibfnamefont {Y.-D.}\ \bibnamefont {Han}},\
  }\bibfield  {title} {\bibinfo {title} {No-signaling principle can determine
  optimal quantum state discrimination},\ }\href
  {https://doi.org/10.1103/PhysRevLett.107.170403} {\bibfield  {journal}
  {\bibinfo  {journal} {Phys. Rev. Lett.}\ }\textbf {\bibinfo {volume} {107}},\
  \bibinfo {pages} {170403} (\bibinfo {year} {2011})}\BibitemShut {NoStop}%
\bibitem [{\citenamefont {Bowles}\ \emph {et~al.}(2018)\citenamefont {Bowles},
  \citenamefont {{\v S}upi{\'c}}, \citenamefont {Cavalcanti},\ and\
  \citenamefont {Ac{\'\i}n}}]{Bowles:2018aa}%
  \BibitemOpen
  \bibfield  {author} {\bibinfo {author} {\bibfnamefont {J.}~\bibnamefont
  {Bowles}}, \bibinfo {author} {\bibfnamefont {I.}~\bibnamefont {{\v
  S}upi{\'c}}}, \bibinfo {author} {\bibfnamefont {D.}~\bibnamefont
  {Cavalcanti}},\ and\ \bibinfo {author} {\bibfnamefont {A.}~\bibnamefont
  {Ac{\'\i}n}},\ }\bibfield  {title} {\bibinfo {title} {Device-independent
  entanglement certification of all entangled states},\ }\href
  {https://doi.org/10.1103/PhysRevLett.121.180503} {\bibfield  {journal}
  {\bibinfo  {journal} {Physical Review Letters}\ }\textbf {\bibinfo {volume}
  {121}},\ \bibinfo {pages} {180503} (\bibinfo {year} {2018})}\BibitemShut
  {NoStop}%
\bibitem [{\citenamefont {Grover}(1997)}]{PhysRevLett.79.325}%
  \BibitemOpen
  \bibfield  {author} {\bibinfo {author} {\bibfnamefont {L.~K.}\ \bibnamefont
  {Grover}},\ }\bibfield  {title} {\bibinfo {title} {Quantum mechanics helps in
  searching for a needle in a haystack},\ }\href
  {https://doi.org/10.1103/PhysRevLett.79.325} {\bibfield  {journal} {\bibinfo
  {journal} {Phys. Rev. Lett.}\ }\textbf {\bibinfo {volume} {79}},\ \bibinfo
  {pages} {325} (\bibinfo {year} {1997})}\BibitemShut {NoStop}%
\bibitem [{\citenamefont {Quek}\ and\ \citenamefont
  {Shor}(2017)}]{PhysRevA.95.052329}%
  \BibitemOpen
  \bibfield  {author} {\bibinfo {author} {\bibfnamefont {Y.}~\bibnamefont
  {Quek}}\ and\ \bibinfo {author} {\bibfnamefont {P.~W.}\ \bibnamefont
  {Shor}},\ }\bibfield  {title} {\bibinfo {title} {Quantum and superquantum
  enhancements to two-sender, two-receiver channels},\ }\href
  {https://doi.org/10.1103/PhysRevA.95.052329} {\bibfield  {journal} {\bibinfo
  {journal} {Phys. Rev. A}\ }\textbf {\bibinfo {volume} {95}},\ \bibinfo
  {pages} {052329} (\bibinfo {year} {2017})}\BibitemShut {NoStop}%
\bibitem [{\citenamefont {Yun}\ \emph {et~al.}(2020)\citenamefont {Yun},
  \citenamefont {Rai},\ and\ \citenamefont {Bae}}]{PhysRevLett.125.150502}%
  \BibitemOpen
  \bibfield  {author} {\bibinfo {author} {\bibfnamefont {J.}~\bibnamefont
  {Yun}}, \bibinfo {author} {\bibfnamefont {A.}~\bibnamefont {Rai}},\ and\
  \bibinfo {author} {\bibfnamefont {J.}~\bibnamefont {Bae}},\ }\bibfield
  {title} {\bibinfo {title} {Nonlocal network coding in interference
  channels},\ }\href {https://doi.org/10.1103/PhysRevLett.125.150502}
  {\bibfield  {journal} {\bibinfo  {journal} {Phys. Rev. Lett.}\ }\textbf
  {\bibinfo {volume} {125}},\ \bibinfo {pages} {150502} (\bibinfo {year}
  {2020})}\BibitemShut {NoStop}%
\bibitem [{\citenamefont {Ac\'{\i}n}\ \emph {et~al.}(2007)\citenamefont
  {Ac\'{\i}n}, \citenamefont {Brunner}, \citenamefont {Gisin}, \citenamefont
  {Massar}, \citenamefont {Pironio},\ and\ \citenamefont
  {Scarani}}]{PhysRevLett.98.230501}%
  \BibitemOpen
  \bibfield  {author} {\bibinfo {author} {\bibfnamefont {A.}~\bibnamefont
  {Ac\'{\i}n}}, \bibinfo {author} {\bibfnamefont {N.}~\bibnamefont {Brunner}},
  \bibinfo {author} {\bibfnamefont {N.}~\bibnamefont {Gisin}}, \bibinfo
  {author} {\bibfnamefont {S.}~\bibnamefont {Massar}}, \bibinfo {author}
  {\bibfnamefont {S.}~\bibnamefont {Pironio}},\ and\ \bibinfo {author}
  {\bibfnamefont {V.}~\bibnamefont {Scarani}},\ }\bibfield  {title} {\bibinfo
  {title} {Device-independent security of quantum cryptography against
  collective attacks},\ }\href {https://doi.org/10.1103/PhysRevLett.98.230501}
  {\bibfield  {journal} {\bibinfo  {journal} {Phys. Rev. Lett.}\ }\textbf
  {\bibinfo {volume} {98}},\ \bibinfo {pages} {230501} (\bibinfo {year}
  {2007})}\BibitemShut {NoStop}%
\bibitem [{\citenamefont {Pironio}\ \emph {et~al.}(2009)\citenamefont
  {Pironio}, \citenamefont {Ac{\'\i}n}, \citenamefont {Brunner}, \citenamefont
  {Gisin}, \citenamefont {Massar},\ and\ \citenamefont
  {Scarani}}]{Pironio_2009}%
  \BibitemOpen
  \bibfield  {author} {\bibinfo {author} {\bibfnamefont {S.}~\bibnamefont
  {Pironio}}, \bibinfo {author} {\bibfnamefont {A.}~\bibnamefont {Ac{\'\i}n}},
  \bibinfo {author} {\bibfnamefont {N.}~\bibnamefont {Brunner}}, \bibinfo
  {author} {\bibfnamefont {N.}~\bibnamefont {Gisin}}, \bibinfo {author}
  {\bibfnamefont {S.}~\bibnamefont {Massar}},\ and\ \bibinfo {author}
  {\bibfnamefont {V.}~\bibnamefont {Scarani}},\ }\bibfield  {title} {\bibinfo
  {title} {Device-independent quantum key distribution secure against
  collective attacks},\ }\href {https://doi.org/10.1088/1367-2630/11/4/045021}
  {\bibfield  {journal} {\bibinfo  {journal} {New Journal of Physics}\ }\textbf
  {\bibinfo {volume} {11}},\ \bibinfo {pages} {045021} (\bibinfo {year}
  {2009})}\BibitemShut {NoStop}%
\bibitem [{\citenamefont {Preskill}(2018)}]{Preskill2018quantumcomputingin}%
  \BibitemOpen
  \bibfield  {author} {\bibinfo {author} {\bibfnamefont {J.}~\bibnamefont
  {Preskill}},\ }\bibfield  {title} {\bibinfo {title} {Quantum {C}omputing in
  the {NISQ} era and beyond},\ }\href@noop {} {\bibfield  {journal} {\bibinfo
  {journal} {{Quantum}}\ }\textbf {\bibinfo {volume} {{2}}},\ \bibinfo {pages}
  {{79}} (\bibinfo {year} {{2018}})}\BibitemShut {NoStop}%
\bibitem [{\citenamefont {Krantz}\ \emph {et~al.}(2019)\citenamefont {Krantz},
  \citenamefont {Kjaergaard}, \citenamefont {Yan}, \citenamefont {Orlando},
  \citenamefont {Gustavsson},\ and\ \citenamefont
  {Oliver}}]{doi:10.1063/1.5089550}%
  \BibitemOpen
  \bibfield  {author} {\bibinfo {author} {\bibfnamefont {P.}~\bibnamefont
  {Krantz}}, \bibinfo {author} {\bibfnamefont {M.}~\bibnamefont {Kjaergaard}},
  \bibinfo {author} {\bibfnamefont {F.}~\bibnamefont {Yan}}, \bibinfo {author}
  {\bibfnamefont {T.~P.}\ \bibnamefont {Orlando}}, \bibinfo {author}
  {\bibfnamefont {S.}~\bibnamefont {Gustavsson}},\ and\ \bibinfo {author}
  {\bibfnamefont {W.~D.}\ \bibnamefont {Oliver}},\ }\bibfield  {title}
  {\bibinfo {title} {A quantum engineer's guide to superconducting qubits},\
  }\href {https://doi.org/10.1063/1.5089550} {\bibfield  {journal} {\bibinfo
  {journal} {Applied Physics Reviews}\ }\textbf {\bibinfo {volume} {6}},\
  \bibinfo {pages} {021318} (\bibinfo {year} {2019})},\ \Eprint
  {https://arxiv.org/abs/https://doi.org/10.1063/1.5089550}
  {https://doi.org/10.1063/1.5089550} \BibitemShut {NoStop}%
\bibitem [{\citenamefont {Henriet}\ \emph {et~al.}(2020)\citenamefont
  {Henriet}, \citenamefont {Beguin}, \citenamefont {Signoles}, \citenamefont
  {Lahaye}, \citenamefont {Browaeys}, \citenamefont {Reymond},\ and\
  \citenamefont {Jurczak}}]{Henriet2020quantumcomputing}%
  \BibitemOpen
  \bibfield  {author} {\bibinfo {author} {\bibfnamefont {L.}~\bibnamefont
  {Henriet}}, \bibinfo {author} {\bibfnamefont {L.}~\bibnamefont {Beguin}},
  \bibinfo {author} {\bibfnamefont {A.}~\bibnamefont {Signoles}}, \bibinfo
  {author} {\bibfnamefont {T.}~\bibnamefont {Lahaye}}, \bibinfo {author}
  {\bibfnamefont {A.}~\bibnamefont {Browaeys}}, \bibinfo {author}
  {\bibfnamefont {G.-O.}\ \bibnamefont {Reymond}},\ and\ \bibinfo {author}
  {\bibfnamefont {C.}~\bibnamefont {Jurczak}},\ }\bibfield  {title} {\bibinfo
  {title} {Quantum computing with neutral atoms},\ }\href@noop {} {\bibfield
  {journal} {\bibinfo  {journal} {{Quantum}}\ }\textbf {\bibinfo {volume}
  {{4}}},\ \bibinfo {pages} {{327}} (\bibinfo {year} {2020})}\BibitemShut
  {NoStop}%
\bibitem [{\citenamefont {Graham}\ \emph {et~al.}(2022)\citenamefont {Graham},
  \citenamefont {Song}, \citenamefont {Scott}, \citenamefont {Poole},
  \citenamefont {Phuttitarn}, \citenamefont {Jooya}, \citenamefont {Eichler},
  \citenamefont {Jiang}, \citenamefont {Marra}, \citenamefont {Grinkemeyer},
  \citenamefont {Kwon}, \citenamefont {Ebert}, \citenamefont {Cherek},
  \citenamefont {Lichtman}, \citenamefont {Gillette}, \citenamefont {Gilbert},
  \citenamefont {Bowman}, \citenamefont {Ballance}, \citenamefont {Campbell},
  \citenamefont {Dahl}, \citenamefont {Crawford}, \citenamefont {Blunt},
  \citenamefont {Rogers}, \citenamefont {Noel},\ and\ \citenamefont
  {Saffman}}]{Graham:2022aa}%
  \BibitemOpen
  \bibfield  {author} {\bibinfo {author} {\bibfnamefont {T.~M.}\ \bibnamefont
  {Graham}}, \bibinfo {author} {\bibfnamefont {Y.}~\bibnamefont {Song}},
  \bibinfo {author} {\bibfnamefont {J.}~\bibnamefont {Scott}}, \bibinfo
  {author} {\bibfnamefont {C.}~\bibnamefont {Poole}}, \bibinfo {author}
  {\bibfnamefont {L.}~\bibnamefont {Phuttitarn}}, \bibinfo {author}
  {\bibfnamefont {K.}~\bibnamefont {Jooya}}, \bibinfo {author} {\bibfnamefont
  {P.}~\bibnamefont {Eichler}}, \bibinfo {author} {\bibfnamefont
  {X.}~\bibnamefont {Jiang}}, \bibinfo {author} {\bibfnamefont
  {A.}~\bibnamefont {Marra}}, \bibinfo {author} {\bibfnamefont
  {B.}~\bibnamefont {Grinkemeyer}}, \bibinfo {author} {\bibfnamefont
  {M.}~\bibnamefont {Kwon}}, \bibinfo {author} {\bibfnamefont {M.}~\bibnamefont
  {Ebert}}, \bibinfo {author} {\bibfnamefont {J.}~\bibnamefont {Cherek}},
  \bibinfo {author} {\bibfnamefont {M.~T.}\ \bibnamefont {Lichtman}}, \bibinfo
  {author} {\bibfnamefont {M.}~\bibnamefont {Gillette}}, \bibinfo {author}
  {\bibfnamefont {J.}~\bibnamefont {Gilbert}}, \bibinfo {author} {\bibfnamefont
  {D.}~\bibnamefont {Bowman}}, \bibinfo {author} {\bibfnamefont
  {T.}~\bibnamefont {Ballance}}, \bibinfo {author} {\bibfnamefont
  {C.}~\bibnamefont {Campbell}}, \bibinfo {author} {\bibfnamefont {E.~D.}\
  \bibnamefont {Dahl}}, \bibinfo {author} {\bibfnamefont {O.}~\bibnamefont
  {Crawford}}, \bibinfo {author} {\bibfnamefont {N.~S.}\ \bibnamefont {Blunt}},
  \bibinfo {author} {\bibfnamefont {B.}~\bibnamefont {Rogers}}, \bibinfo
  {author} {\bibfnamefont {T.}~\bibnamefont {Noel}},\ and\ \bibinfo {author}
  {\bibfnamefont {M.}~\bibnamefont {Saffman}},\ }\bibfield  {title} {\bibinfo
  {title} {Multi-qubit entanglement and algorithms on a neutral-atom quantum
  computer},\ }\href {https://doi.org/10.1038/s41586-022-04603-6} {\bibfield
  {journal} {\bibinfo  {journal} {Nature}\ }\textbf {\bibinfo {volume} {604}},\
  \bibinfo {pages} {457} (\bibinfo {year} {2022})}\BibitemShut {NoStop}%
\bibitem [{\citenamefont {Doran}\ and\ \citenamefont
  {Society}(1994)}]{doran1994c}%
  \BibitemOpen
  \bibfield  {author} {\bibinfo {author} {\bibfnamefont {R.}~\bibnamefont
  {Doran}}\ and\ \bibinfo {author} {\bibfnamefont {A.}~\bibnamefont
  {Society}},\ }\href {https://books.google.co.kr/books?id=DYCUp0JYU6sC} {\emph
  {\bibinfo {title} {C*-algebras: 1943-1993 : a Fifty Year Celebration : AMS
  Special Session Commemorating the First Fifty Years of C*-algebra Theory,
  January 13-14, 1993, San Antonio, Texas}}},\ Contemporary mathematics\
  (\bibinfo  {publisher} {American Mathematical Society},\ \bibinfo {year}
  {1994})\BibitemShut {NoStop}%
\bibitem [{\citenamefont {Sparaciari}\ and\ \citenamefont
  {Paris}(2013)}]{PhysRevA.87.012106}%
  \BibitemOpen
  \bibfield  {author} {\bibinfo {author} {\bibfnamefont {C.}~\bibnamefont
  {Sparaciari}}\ and\ \bibinfo {author} {\bibfnamefont {M.~G.~A.}\ \bibnamefont
  {Paris}},\ }\bibfield  {title} {\bibinfo {title} {Canonical naimark extension
  for generalized measurements involving sets of pauli quantum observables
  chosen at random},\ }\href {https://doi.org/10.1103/PhysRevA.87.012106}
  {\bibfield  {journal} {\bibinfo  {journal} {Phys. Rev. A}\ }\textbf {\bibinfo
  {volume} {87}},\ \bibinfo {pages} {012106} (\bibinfo {year}
  {2013})}\BibitemShut {NoStop}%
\bibitem [{\citenamefont {Yordanov}\ and\ \citenamefont
  {Barnes}(2019)}]{PhysRevA.100.062317}%
  \BibitemOpen
  \bibfield  {author} {\bibinfo {author} {\bibfnamefont {Y.~S.}\ \bibnamefont
  {Yordanov}}\ and\ \bibinfo {author} {\bibfnamefont {C.~H.~W.}\ \bibnamefont
  {Barnes}},\ }\bibfield  {title} {\bibinfo {title} {Implementation of a
  general single-qubit positive operator-valued measure on a circuit-based
  quantum computer},\ }\href {https://doi.org/10.1103/PhysRevA.100.062317}
  {\bibfield  {journal} {\bibinfo  {journal} {Phys. Rev. A}\ }\textbf {\bibinfo
  {volume} {100}},\ \bibinfo {pages} {062317} (\bibinfo {year}
  {2019})}\BibitemShut {NoStop}%
\bibitem [{\citenamefont
  {Chru{\'{s}}ci{\'{n}}ski}(2014)}]{chruscinski2014class}%
  \BibitemOpen
  \bibfield  {author} {\bibinfo {author} {\bibfnamefont {D.}~\bibnamefont
  {Chru{\'{s}}ci{\'{n}}ski}},\ }\bibfield  {title} {\bibinfo {title} {A class
  of symmetric bell diagonal entanglement witnesses{\textemdash}a geometric
  perspective},\ }\href {https://doi.org/10.1088/1751-8113/47/42/424033}
  {\bibfield  {journal} {\bibinfo  {journal} {Journal of Physics A:
  Mathematical and Theoretical}\ }\textbf {\bibinfo {volume} {47}},\ \bibinfo
  {pages} {424033} (\bibinfo {year} {2014})}\BibitemShut {NoStop}%
\bibitem [{\citenamefont {Bera}\ \emph
  {et~al.}(2022{\natexlab{a}})\citenamefont {Bera}, \citenamefont {Wudarski},
  \citenamefont {Sarbicki},\ and\ \citenamefont {Chru\ifmmode \acute{s}\else
  \'{s}\fi{}ci\ifmmode~\acute{n}\else \'{n}\fi{}ski}}]{PhysRevA.105.052401}%
  \BibitemOpen
  \bibfield  {author} {\bibinfo {author} {\bibfnamefont {A.}~\bibnamefont
  {Bera}}, \bibinfo {author} {\bibfnamefont {F.~A.}\ \bibnamefont {Wudarski}},
  \bibinfo {author} {\bibfnamefont {G.}~\bibnamefont {Sarbicki}},\ and\
  \bibinfo {author} {\bibfnamefont {D.}~\bibnamefont {Chru\ifmmode
  \acute{s}\else \'{s}\fi{}ci\ifmmode~\acute{n}\else \'{n}\fi{}ski}},\
  }\bibfield  {title} {\bibinfo {title} {Class of bell-diagonal entanglement
  witnesses in $\mathbbm{C}^4\otimes \mathbbm{C}^4$: Optimization and the
  spanning property},\ }\href {https://doi.org/10.1103/PhysRevA.105.052401}
  {\bibfield  {journal} {\bibinfo  {journal} {Phys. Rev. A}\ }\textbf {\bibinfo
  {volume} {105}},\ \bibinfo {pages} {052401} (\bibinfo {year}
  {2022}{\natexlab{a}})}\BibitemShut {NoStop}%
\bibitem [{\citenamefont {Choi}(1975)}]{Choi:1975aa}%
  \BibitemOpen
  \bibfield  {author} {\bibinfo {author} {\bibfnamefont {M.-D.}\ \bibnamefont
  {Choi}},\ }\bibfield  {title} {\bibinfo {title} {Completely positive linear
  maps on complex matrices},\ }\href
  {https://doi.org/http://dx.doi.org/10.1016/0024-3795(75)90075-0} {\bibfield
  {journal} {\bibinfo  {journal} {Linear Algebra and its Applications}\
  }\textbf {\bibinfo {volume} {10}},\ \bibinfo {pages} {285} (\bibinfo {year}
  {1975})}\BibitemShut {NoStop}%
\bibitem [{\citenamefont {Ha}\ and\ \citenamefont
  {Kye}(2011)}]{PhysRevA.84.024302}%
  \BibitemOpen
  \bibfield  {author} {\bibinfo {author} {\bibfnamefont {K.-C.}\ \bibnamefont
  {Ha}}\ and\ \bibinfo {author} {\bibfnamefont {S.-H.}\ \bibnamefont {Kye}},\
  }\bibfield  {title} {\bibinfo {title} {One-parameter family of indecomposable
  optimal entanglement witnesses arising from generalized choi maps},\ }\href
  {https://doi.org/10.1103/PhysRevA.84.024302} {\bibfield  {journal} {\bibinfo
  {journal} {Phys. Rev. A}\ }\textbf {\bibinfo {volume} {84}},\ \bibinfo
  {pages} {024302} (\bibinfo {year} {2011})}\BibitemShut {NoStop}%
\bibitem [{\citenamefont {Breuer}(2006)}]{PhysRevLett.97.080501}%
  \BibitemOpen
  \bibfield  {author} {\bibinfo {author} {\bibfnamefont {H.-P.}\ \bibnamefont
  {Breuer}},\ }\bibfield  {title} {\bibinfo {title} {Optimal entanglement
  criterion for mixed quantum states},\ }\href
  {https://doi.org/10.1103/PhysRevLett.97.080501} {\bibfield  {journal}
  {\bibinfo  {journal} {Phys. Rev. Lett.}\ }\textbf {\bibinfo {volume} {97}},\
  \bibinfo {pages} {080501} (\bibinfo {year} {2006})}\BibitemShut {NoStop}%
\bibitem [{\citenamefont {Hall}(2006)}]{Hall_2006}%
  \BibitemOpen
  \bibfield  {author} {\bibinfo {author} {\bibfnamefont {W.}~\bibnamefont
  {Hall}},\ }\bibfield  {title} {\bibinfo {title} {A new criterion for
  indecomposability of positive maps},\ }\href
  {https://doi.org/10.1088/0305-4470/39/45/020} {\bibfield  {journal} {\bibinfo
   {journal} {Journal of Physics A: Mathematical and General}\ }\textbf
  {\bibinfo {volume} {39}},\ \bibinfo {pages} {14119} (\bibinfo {year}
  {2006})}\BibitemShut {NoStop}%
\bibitem [{\citenamefont {Hein}\ \emph {et~al.}(2004)\citenamefont {Hein},
  \citenamefont {Eisert},\ and\ \citenamefont {Briegel}}]{PhysRevA.69.062311}%
  \BibitemOpen
  \bibfield  {author} {\bibinfo {author} {\bibfnamefont {M.}~\bibnamefont
  {Hein}}, \bibinfo {author} {\bibfnamefont {J.}~\bibnamefont {Eisert}},\ and\
  \bibinfo {author} {\bibfnamefont {H.~J.}\ \bibnamefont {Briegel}},\
  }\bibfield  {title} {\bibinfo {title} {Multiparty entanglement in graph
  states},\ }\href {https://doi.org/10.1103/PhysRevA.69.062311} {\bibfield
  {journal} {\bibinfo  {journal} {Phys. Rev. A}\ }\textbf {\bibinfo {volume}
  {69}},\ \bibinfo {pages} {062311} (\bibinfo {year} {2004})}\BibitemShut
  {NoStop}%
\bibitem [{\citenamefont {Raussendorf}\ and\ \citenamefont
  {Briegel}(2001)}]{PhysRevLett.86.5188}%
  \BibitemOpen
  \bibfield  {author} {\bibinfo {author} {\bibfnamefont {R.}~\bibnamefont
  {Raussendorf}}\ and\ \bibinfo {author} {\bibfnamefont {H.~J.}\ \bibnamefont
  {Briegel}},\ }\bibfield  {title} {\bibinfo {title} {A one-way quantum
  computer},\ }\href {https://doi.org/10.1103/PhysRevLett.86.5188} {\bibfield
  {journal} {\bibinfo  {journal} {Phys. Rev. Lett.}\ }\textbf {\bibinfo
  {volume} {86}},\ \bibinfo {pages} {5188} (\bibinfo {year}
  {2001})}\BibitemShut {NoStop}%
\bibitem [{\citenamefont {G\"uhne}\ \emph {et~al.}(2007)\citenamefont
  {G\"uhne}, \citenamefont {Lu}, \citenamefont {Gao},\ and\ \citenamefont
  {Pan}}]{PhysRevA.76.030305}%
  \BibitemOpen
  \bibfield  {author} {\bibinfo {author} {\bibfnamefont {O.}~\bibnamefont
  {G\"uhne}}, \bibinfo {author} {\bibfnamefont {C.-Y.}\ \bibnamefont {Lu}},
  \bibinfo {author} {\bibfnamefont {W.-B.}\ \bibnamefont {Gao}},\ and\ \bibinfo
  {author} {\bibfnamefont {J.-W.}\ \bibnamefont {Pan}},\ }\bibfield  {title}
  {\bibinfo {title} {Toolbox for entanglement detection and fidelity
  estimation},\ }\href {https://doi.org/10.1103/PhysRevA.76.030305} {\bibfield
  {journal} {\bibinfo  {journal} {Phys. Rev. A}\ }\textbf {\bibinfo {volume}
  {76}},\ \bibinfo {pages} {030305} (\bibinfo {year} {2007})}\BibitemShut
  {NoStop}%
\bibitem [{\citenamefont {D\"ur}\ \emph
  {et~al.}(2000{\natexlab{a}})\citenamefont {D\"ur}, \citenamefont {Cirac},
  \citenamefont {Lewenstein},\ and\ \citenamefont
  {Bru\ss{}}}]{PhysRevA.61.062313}%
  \BibitemOpen
  \bibfield  {author} {\bibinfo {author} {\bibfnamefont {W.}~\bibnamefont
  {D\"ur}}, \bibinfo {author} {\bibfnamefont {J.~I.}\ \bibnamefont {Cirac}},
  \bibinfo {author} {\bibfnamefont {M.}~\bibnamefont {Lewenstein}},\ and\
  \bibinfo {author} {\bibfnamefont {D.}~\bibnamefont {Bru\ss{}}},\ }\bibfield
  {title} {\bibinfo {title} {Distillability and partial transposition in
  bipartite systems},\ }\href {https://doi.org/10.1103/PhysRevA.61.062313}
  {\bibfield  {journal} {\bibinfo  {journal} {Phys. Rev. A}\ }\textbf {\bibinfo
  {volume} {61}},\ \bibinfo {pages} {062313} (\bibinfo {year}
  {2000}{\natexlab{a}})}\BibitemShut {NoStop}%
\bibitem [{\citenamefont {Shor}\ \emph {et~al.}(2001)\citenamefont {Shor},
  \citenamefont {Smolin},\ and\ \citenamefont {Terhal}}]{PhysRevLett.86.2681}%
  \BibitemOpen
  \bibfield  {author} {\bibinfo {author} {\bibfnamefont {P.~W.}\ \bibnamefont
  {Shor}}, \bibinfo {author} {\bibfnamefont {J.~A.}\ \bibnamefont {Smolin}},\
  and\ \bibinfo {author} {\bibfnamefont {B.~M.}\ \bibnamefont {Terhal}},\
  }\bibfield  {title} {\bibinfo {title} {Nonadditivity of bipartite distillable
  entanglement follows from a conjecture on bound entangled werner states},\
  }\href {https://doi.org/10.1103/PhysRevLett.86.2681} {\bibfield  {journal}
  {\bibinfo  {journal} {Phys. Rev. Lett.}\ }\textbf {\bibinfo {volume} {86}},\
  \bibinfo {pages} {2681} (\bibinfo {year} {2001})}\BibitemShut {NoStop}%
\bibitem [{\citenamefont {Horodecki}\ \emph {et~al.}(1999)\citenamefont
  {Horodecki}, \citenamefont {Horodecki},\ and\ \citenamefont
  {Horodecki}}]{horodecki1999bound}%
  \BibitemOpen
  \bibfield  {author} {\bibinfo {author} {\bibfnamefont {P.}~\bibnamefont
  {Horodecki}}, \bibinfo {author} {\bibfnamefont {M.}~\bibnamefont
  {Horodecki}},\ and\ \bibinfo {author} {\bibfnamefont {R.}~\bibnamefont
  {Horodecki}},\ }\bibfield  {title} {\bibinfo {title} {Bound entanglement can
  be activated},\ }\href@noop {} {\bibfield  {journal} {\bibinfo  {journal}
  {Physical review letters}\ }\textbf {\bibinfo {volume} {82}},\ \bibinfo
  {pages} {1056} (\bibinfo {year} {1999})}\BibitemShut {NoStop}%
\bibitem [{\citenamefont {St{\o}rmer}(1963)}]{Stormer:1963aa}%
  \BibitemOpen
  \bibfield  {author} {\bibinfo {author} {\bibfnamefont {E.}~\bibnamefont
  {St{\o}rmer}},\ }\bibfield  {title} {\bibinfo {title} {Positive linear maps
  of operator algebras},\ }\href {https://doi.org/10.1007/BF02391860}
  {\bibfield  {journal} {\bibinfo  {journal} {Acta Mathematica}\ }\textbf
  {\bibinfo {volume} {110}},\ \bibinfo {pages} {233} (\bibinfo {year}
  {1963})}\BibitemShut {NoStop}%
\bibitem [{\citenamefont {Bera}\ \emph
  {et~al.}(2022{\natexlab{b}})\citenamefont {Bera}, \citenamefont {Scala},
  \citenamefont {Sarbicki},\ and\ \citenamefont
  {Chru{\'s}ci{\'n}ski}}]{bera2022generalizing}%
  \BibitemOpen
  \bibfield  {author} {\bibinfo {author} {\bibfnamefont {A.}~\bibnamefont
  {Bera}}, \bibinfo {author} {\bibfnamefont {G.}~\bibnamefont {Scala}},
  \bibinfo {author} {\bibfnamefont {G.}~\bibnamefont {Sarbicki}},\ and\
  \bibinfo {author} {\bibfnamefont {D.}~\bibnamefont {Chru{\'s}ci{\'n}ski}},\
  }\href@noop {} {\bibinfo {title} {Generalizing choi map in $m_3$ beyond
  circulant scenario}},\ \bibinfo {howpublished} {arXiv:2212.03807} (\bibinfo
  {year} {2022}{\natexlab{b}})\BibitemShut {NoStop}%
\bibitem [{\citenamefont {Bera}\ \emph
  {et~al.}(2022{\natexlab{c}})\citenamefont {Bera}, \citenamefont {Sarbicki},\
  and\ \citenamefont {Chru{\'s}ci{\'n}ski}}]{bera2022class}%
  \BibitemOpen
  \bibfield  {author} {\bibinfo {author} {\bibfnamefont {A.}~\bibnamefont
  {Bera}}, \bibinfo {author} {\bibfnamefont {G.}~\bibnamefont {Sarbicki}},\
  and\ \bibinfo {author} {\bibfnamefont {D.}~\bibnamefont
  {Chru{\'s}ci{\'n}ski}},\ }\href@noop {} {\bibinfo {title} {A class of optimal
  positive maps in $m_n$}},\ \bibinfo {howpublished} {arXiv:2207.03821}
  (\bibinfo {year} {2022}{\natexlab{c}})\BibitemShut {NoStop}%
\bibitem [{\citenamefont {Smolin}(2001)}]{PhysRevA.63.032306}%
  \BibitemOpen
  \bibfield  {author} {\bibinfo {author} {\bibfnamefont {J.~A.}\ \bibnamefont
  {Smolin}},\ }\bibfield  {title} {\bibinfo {title} {Four-party unlockable
  bound entangled state},\ }\href {https://doi.org/10.1103/PhysRevA.63.032306}
  {\bibfield  {journal} {\bibinfo  {journal} {Phys. Rev. A}\ }\textbf {\bibinfo
  {volume} {63}},\ \bibinfo {pages} {032306} (\bibinfo {year}
  {2001})}\BibitemShut {NoStop}%
\bibitem [{\citenamefont {Amselem}\ and\ \citenamefont
  {Bourennane}(2009)}]{Amselem:2009aa}%
  \BibitemOpen
  \bibfield  {author} {\bibinfo {author} {\bibfnamefont {E.}~\bibnamefont
  {Amselem}}\ and\ \bibinfo {author} {\bibfnamefont {M.}~\bibnamefont
  {Bourennane}},\ }\bibfield  {title} {\bibinfo {title} {Experimental
  four-qubit bound entanglement},\ }\href {https://doi.org/10.1038/nphys1372}
  {\bibfield  {journal} {\bibinfo  {journal} {Nature Physics}\ }\textbf
  {\bibinfo {volume} {5}},\ \bibinfo {pages} {748} (\bibinfo {year}
  {2009})}\BibitemShut {NoStop}%
\bibitem [{\citenamefont {Lavoie}\ \emph {et~al.}(2010)\citenamefont {Lavoie},
  \citenamefont {Kaltenbaek}, \citenamefont {Piani},\ and\ \citenamefont
  {Resch}}]{PhysRevLett.105.130501}%
  \BibitemOpen
  \bibfield  {author} {\bibinfo {author} {\bibfnamefont {J.}~\bibnamefont
  {Lavoie}}, \bibinfo {author} {\bibfnamefont {R.}~\bibnamefont {Kaltenbaek}},
  \bibinfo {author} {\bibfnamefont {M.}~\bibnamefont {Piani}},\ and\ \bibinfo
  {author} {\bibfnamefont {K.~J.}\ \bibnamefont {Resch}},\ }\bibfield  {title}
  {\bibinfo {title} {Experimental bound entanglement in a four-photon state},\
  }\href {https://doi.org/10.1103/PhysRevLett.105.130501} {\bibfield  {journal}
  {\bibinfo  {journal} {Phys. Rev. Lett.}\ }\textbf {\bibinfo {volume} {105}},\
  \bibinfo {pages} {130501} (\bibinfo {year} {2010})}\BibitemShut {NoStop}%
\bibitem [{\citenamefont {Kaneda}\ \emph {et~al.}(2012)\citenamefont {Kaneda},
  \citenamefont {Shimizu}, \citenamefont {Ishizaka}, \citenamefont {Mitsumori},
  \citenamefont {Kosaka},\ and\ \citenamefont
  {Edamatsu}}]{PhysRevLett.109.040501}%
  \BibitemOpen
  \bibfield  {author} {\bibinfo {author} {\bibfnamefont {F.}~\bibnamefont
  {Kaneda}}, \bibinfo {author} {\bibfnamefont {R.}~\bibnamefont {Shimizu}},
  \bibinfo {author} {\bibfnamefont {S.}~\bibnamefont {Ishizaka}}, \bibinfo
  {author} {\bibfnamefont {Y.}~\bibnamefont {Mitsumori}}, \bibinfo {author}
  {\bibfnamefont {H.}~\bibnamefont {Kosaka}},\ and\ \bibinfo {author}
  {\bibfnamefont {K.}~\bibnamefont {Edamatsu}},\ }\bibfield  {title} {\bibinfo
  {title} {Experimental activation of bound entanglement},\ }\href
  {https://doi.org/10.1103/PhysRevLett.109.040501} {\bibfield  {journal}
  {\bibinfo  {journal} {Phys. Rev. Lett.}\ }\textbf {\bibinfo {volume} {109}},\
  \bibinfo {pages} {040501} (\bibinfo {year} {2012})}\BibitemShut {NoStop}%
\bibitem [{\citenamefont {Masanes}(2006)}]{PhysRevLett.96.150501}%
  \BibitemOpen
  \bibfield  {author} {\bibinfo {author} {\bibfnamefont {L.}~\bibnamefont
  {Masanes}},\ }\bibfield  {title} {\bibinfo {title} {All bipartite entangled
  states are useful for information processing},\ }\href
  {https://doi.org/10.1103/PhysRevLett.96.150501} {\bibfield  {journal}
  {\bibinfo  {journal} {Phys. Rev. Lett.}\ }\textbf {\bibinfo {volume} {96}},\
  \bibinfo {pages} {150501} (\bibinfo {year} {2006})}\BibitemShut {NoStop}%
\bibitem [{\citenamefont {Gurvits}(2003)}]{10.1145/780542.780545}%
  \BibitemOpen
  \bibfield  {author} {\bibinfo {author} {\bibfnamefont {L.}~\bibnamefont
  {Gurvits}},\ }\bibfield  {title} {\bibinfo {title} {Classical deterministic
  complexity of edmonds' problem and quantum entanglement},\ }in\ \href
  {https://doi.org/10.1145/780542.780545} {\emph {\bibinfo {booktitle}
  {Proceedings of the Thirty-Fifth Annual ACM Symposium on Theory of
  Computing}}},\ \bibinfo {series and number} {STOC '03}\ (\bibinfo
  {publisher} {Association for Computing Machinery},\ \bibinfo {address} {New
  York, NY, USA},\ \bibinfo {year} {2003})\ pp.\ \bibinfo {pages}
  {10--19}\BibitemShut {NoStop}%
\bibitem [{\citenamefont {Peres}(1996)}]{PhysRevLett.77.1413}%
  \BibitemOpen
  \bibfield  {author} {\bibinfo {author} {\bibfnamefont {A.}~\bibnamefont
  {Peres}},\ }\bibfield  {title} {\bibinfo {title} {Separability criterion for
  density matrices},\ }\href {https://doi.org/10.1103/PhysRevLett.77.1413}
  {\bibfield  {journal} {\bibinfo  {journal} {Phys. Rev. Lett.}\ }\textbf
  {\bibinfo {volume} {77}},\ \bibinfo {pages} {1413} (\bibinfo {year}
  {1996})}\BibitemShut {NoStop}%
\bibitem [{\citenamefont {Vollbrecht}\ and\ \citenamefont
  {Wolf}(2002)}]{vollbrecht2002activating}%
  \BibitemOpen
  \bibfield  {author} {\bibinfo {author} {\bibfnamefont {K.~G.~H.}\
  \bibnamefont {Vollbrecht}}\ and\ \bibinfo {author} {\bibfnamefont {M.~M.}\
  \bibnamefont {Wolf}},\ }\bibfield  {title} {\bibinfo {title} {Activating
  distillation with an infinitesimal amount of bound entanglement},\
  }\href@noop {} {\bibfield  {journal} {\bibinfo  {journal} {Physical review
  letters}\ }\textbf {\bibinfo {volume} {88}},\ \bibinfo {pages} {247901}
  (\bibinfo {year} {2002})}\BibitemShut {NoStop}%
\bibitem [{\citenamefont {Chru{\'s}ci\'{n}ski}\ and\ \citenamefont
  {Sarbicki}(2013)}]{doi:10.1142/S1230161213500066}%
  \BibitemOpen
  \bibfield  {author} {\bibinfo {author} {\bibfnamefont {D.}~\bibnamefont
  {Chru{\'s}ci\'{n}ski}}\ and\ \bibinfo {author} {\bibfnamefont
  {G.}~\bibnamefont {Sarbicki}},\ }\bibfield  {title} {\bibinfo {title}
  {Optimal entanglement witnesses for two qutrits},\ }\href
  {https://doi.org/10.1142/S1230161213500066} {\bibfield  {journal} {\bibinfo
  {journal} {Open Systems \& Information Dynamics}\ }\textbf {\bibinfo {volume}
  {20}},\ \bibinfo {pages} {1350006} (\bibinfo {year} {2013})},\ \Eprint
  {https://arxiv.org/abs/https://doi.org/10.1142/S1230161213500066}
  {https://doi.org/10.1142/S1230161213500066} \BibitemShut {NoStop}%
\bibitem [{\citenamefont {Horodecki}\ and\ \citenamefont
  {Horodecki}(1999)}]{PhysRevA.59.4206}%
  \BibitemOpen
  \bibfield  {author} {\bibinfo {author} {\bibfnamefont {M.}~\bibnamefont
  {Horodecki}}\ and\ \bibinfo {author} {\bibfnamefont {P.}~\bibnamefont
  {Horodecki}},\ }\bibfield  {title} {\bibinfo {title} {Reduction criterion of
  separability and limits for a class of distillation protocols},\ }\href
  {https://doi.org/10.1103/PhysRevA.59.4206} {\bibfield  {journal} {\bibinfo
  {journal} {Phys. Rev. A}\ }\textbf {\bibinfo {volume} {59}},\ \bibinfo
  {pages} {4206} (\bibinfo {year} {1999})}\BibitemShut {NoStop}%
\bibitem [{\citenamefont {D\"ur}\ \emph
  {et~al.}(2000{\natexlab{b}})\citenamefont {D\"ur}, \citenamefont {Vidal},\
  and\ \citenamefont {Cirac}}]{PhysRevA.62.062314}%
  \BibitemOpen
  \bibfield  {author} {\bibinfo {author} {\bibfnamefont {W.}~\bibnamefont
  {D\"ur}}, \bibinfo {author} {\bibfnamefont {G.}~\bibnamefont {Vidal}},\ and\
  \bibinfo {author} {\bibfnamefont {J.~I.}\ \bibnamefont {Cirac}},\ }\bibfield
  {title} {\bibinfo {title} {Three qubits can be entangled in two inequivalent
  ways},\ }\href {https://doi.org/10.1103/PhysRevA.62.062314} {\bibfield
  {journal} {\bibinfo  {journal} {Phys. Rev. A}\ }\textbf {\bibinfo {volume}
  {62}},\ \bibinfo {pages} {062314} (\bibinfo {year}a
  {2000}{\natexlab{b}})}\BibitemShut {NoStop}%
\bibitem [{\citenamefont {Jungnitsch}\ \emph {et~al.}(2011)\citenamefont
  {Jungnitsch}, \citenamefont {Moroder},\ and\ \citenamefont
  {G\"uhne}}]{PhysRevA.84.032310}%
  \BibitemOpen
  \bibfield  {author} {\bibinfo {author} {\bibfnamefont {B.}~\bibnamefont
  {Jungnitsch}}, \bibinfo {author} {\bibfnamefont {T.}~\bibnamefont
  {Moroder}},\ and\ \bibinfo {author} {\bibfnamefont {O.}~\bibnamefont
  {G\"uhne}},\ }\bibfield  {title} {\bibinfo {title} {Entanglement witnesses
  for graph states: General theory and examples},\ }\href
  {https://doi.org/10.1103/PhysRevA.84.032310} {\bibfield  {journal} {\bibinfo
  {journal} {Phys. Rev. A}\ }\textbf {\bibinfo {volume} {84}},\ \bibinfo
  {pages} {032310} (\bibinfo {year} {2011})}\BibitemShut {NoStop}%
\end{thebibliography}%

\appendix

\section{Entanglement detection by state preparation for two-qubit states}
\label{app1}

We here reproduce a network state for two-qubit EWs. Let $|\phi^{\pm}\rangle = (|00\rangle\pm |11\rangle )/\sqrt{2}$ and $|\psi^{\pm}\rangle = (|01\rangle\pm |10\rangle )/\sqrt{2}$ denote four Bell states. We show how to construct a network state for an EW 
\bea
W = |\phi^+\rangle \langle \phi^+|^{\Gamma} = \frac{1}{2}\mathbbm{I} - |\psi^-\rangle \langle \psi^-|. \nonumber
\eea
One may find a decomposition of the above EW in the following way
\bea
W = \frac{1}{2} (\mathbbm{I} -  |\psi^-\rangle \langle \psi^-|) -\frac{1}{2} |\psi^-\rangle \langle \psi^-|, \nonumber
\eea
where two non-negative operators are obtained as $ |\psi^-\rangle \langle \psi^-|$ and $(\mathbbm{I} -  |\psi^-\rangle \langle \psi^-|)/2$. A network state may be written as
\bea
N_{23} &=& c_1 |\psi^-\rangle_{A_2 B_2 } \langle \psi^-| \otimes \Pi_{A_3B_3}(1) + \nonumber\\
&& \frac{c_2}{3} (\mathbbm{I} - |\psi^-\rangle_{A_2 B_2 } \langle \psi^-| ) \otimes \Pi_{A_3B_3}(2), \nonumber
\eea
for some positive constants $c_1,~c_2=1-c_1>0$ and non-negative normalized operators $\Pi(1),~ \Pi(2)\geq 0$. To realize entanglement detection of a state of interest $\rho$ using an entanglement witness $W$, one may seek $N_{23}$ that satisfy 
\bea
\tr[\rho_1 W_1] =  16~ \tr  [\rho_1\otimes N_{23} (\eta \mathbbm{1} - | \phi^+\rangle_{A_3B_3} \langle \phi^+|) \otimes P^{(12)}  ], \nonumber 
\eea
with $\eta = 1/2$, since all two-qubit entangled states are distillable. The goal is now to find the parameters $c_1,~c_2,~ \Pi(1)$ and $\Pi(2)$ that satisfy the relation in the above. The left-hand-side (lhs) is given by
\bea
\mathrm{lhs}= \frac{1}{2} - \langle \psi^-| \rho|\psi^-\rangle, \nonumber
\eea
and the right-hand-side (rhs) by
\bea 
\mathrm{rhs}&=& 2 \tr [\rho_{2}^T N_{23}] - 4 \tr [\rho_{2}^T \otimes |\psi^-\rangle \langle \psi^-|  N_{23} ]    \nonumber \\
&=& c_2( \frac{2}{3}  - \frac{4}{3} \langle \phi^+| \Pi (2) |\phi^+\rangle )  + L \langle \psi^-| \rho |\psi^-\rangle, \nonumber
\eea
where 
\bea
L &=& c_1 (2-4 \langle \phi^+| \Pi(1) | \phi^+\rangle ) +  \nonumber \\
 && c_2 ( - \frac{2}{3}  + \frac{4}{3} \langle \phi^+| \Pi(2) | \phi^+\rangle ). \nonumber
\eea
From the lhs and the rhs, one can find that
\bea
c_2( \frac{2}{3}  - \frac{4}{3} \langle \phi^+| \Pi (1) |\phi^+\rangle )= \frac{1}{2}~\mathrm{and} ~L=-1, \nonumber
\eea
from which 
\bea
&& c_1 (2-4 \langle \phi^+| \Pi(1) | \phi^+\rangle ) = - \frac{1}{2} \nonumber\\
& \iff &c_1 = \frac{1}{8 \langle \phi^+ | \Pi(1) |\phi^+\rangle -4 } >0. \nonumber
\eea
For convenience, we choose $\Pi(1) = |\phi^+\rangle \langle \phi^+|$ although it is not a unique choice. It follows that $c_1=1/4$ and $c_2=3/4$. The consequence is that $\langle \phi^+| \Pi(2) |\phi^+\rangle = 0$. Thus, we have 
\bea
\Pi(2) = \frac{1}{3} ( \mathbbm{I} -|\phi^+\rangle \langle \phi^+| ).\nonumber
\eea
All these conclude a network state
\bea
N_{23} &=& \frac{1}{4} |\psi^-\rangle_{A_2B_2} \langle \psi^-| \otimes |\phi^+\rangle_{A_3B_3} \langle \phi^+| + \nonumber \\
&&\frac{1}{12} (\mathbbm{1} - |\psi^-\rangle_{A_2B_2} \langle \psi^-|) \otimes (\mathbbm{1} - |\phi^+\rangle_{A_3B_3} \langle \phi^+|). \nonumber
\eea
Note that a network state for an EW is not unique. 



\section{Network states for high-dimensional EWs}
\label{app2}

\subsection{The framework}

Recall that for a given EW $W$, we are looking for a network state $N_{23} = N^{(A_2 B_2 A_3 B_3)}$, which are separable in $A_2 B_2: A_3 B_3$, satisfying the following condition:
\bea
W_2^T \propto \tr_3[N_{23} (\eta \mathbbm{1} - P_{00})_3], 
\eea
for some $\eta \in [\frac{1}{d}, 1)$.
It is easy to see that the following relation holds:
\bea
\tr[\rho_1 W_1] &=& d^2 \tr[\rho_1 \otimes W_2^T P^{(12)}]
\\ &\propto& \tr[\rho_1 \otimes N_{23} ~ P^{(12)} \otimes (\eta \mathbbm{1} - P_{00})_3], \label{prop}
\eea
where $P^{(12)} = P_{00}^{(A_1 A_2)} \otimes P_{00}^{(B_1 B_2)}$ denotes the Bell measurements on both sides.
The scheme can be understood as follows. First, a filtering operation $\Lambda^{(1 \to 3)}$ teleports the state $\rho_1$ to $A_3 B_3$, leaving a result state $\Lambda(\rho)$:
\bea
\Lambda^{(1 \to 3)} (\rho) = \frac{\tr_{12}[\rho_1 \otimes N_{23} P^{(12)}] }{\tr[\rho_1 \otimes N_{23} P^{(12)}] }. \label{filtering}
\eea
Then the singlet fraction, or the overlap with the Bell state $P_{00}$, of the resulting state $\Lambda(\rho)$ is checked whether it is higher than $\eta$ or not. 

The singlet fraction can also be estimated with a fixed measurement on individual quantum systems. The main idea is to place unitary interactions before a measurement. A $d$-dimensional Hadamard gate and a $d$-dimensional CNOT gate may be obtained as,
\begin{align*}
H &= \sum_{j,k=0}^{d-1} e^{2 \pi ijk/d} \ket{j}\bra{k},~ \mathrm{and}
\\ U_{CNOT} &= \sum_{j=0}^{d-1} \ket{j}\bra{j} \otimes \sum_{k=0}^{d-1} \ket{k+j}\bra{k}.    
\end{align*}
Note that a maximally entangled state can be generated,  $\ket{\phi_{00}} = U_{CNOT}(H \otimes \I) \ket{00}$. Then, instead of a joint measurement, one can first apply $(H^\dagger \otimes \I) U_{CNOT}^\dagger $ to a resulting state $\Lambda^{(1 \to 3)} (\rho)$ and then perform a measurement in the computational basis. The probability of having outcomes $00$ gives the singlet fraction $\langle \phi_{00} | \Lambda(\rho) | \phi_{00} \rangle$.
It holds that $\langle \phi_{00} | \Lambda(\rho) | \phi_{00} \rangle > \eta$ if and only if $\tr[\rho W] < 0$, which certifies that a state $\rho$ given in the beginning is entangled. \\

\subsection{ Decomposable EW}

Consider a decomposable EW $W = Q^\Gamma$ for $Q\geq 0$ and $\tr[Q]=1$. Let $\lambda:=\max_{i}|\lambda_i|$ where $\{\lambda_i\} $ are eigenvalues of $W$. Then, a network state for the EW is obtained as
\bea
N_{23}^{\text{(dec)}} &=& c_1\left( \frac{\lambda\mathbbm{1} - Q^\Gamma }{\lambda d^2 - 1 }\right)^{(2)} \otimes P_{00}^{(3)} \nonumber\\
&&+ c_2 \left( \frac{ \lambda \mathbbm{1} +  Q^{\Gamma} }{\lambda d^2 + 1}\right)^{(2)} \otimes \left( \frac{\mathbbm{1} - P_{00}}{d^2 - 1}\right)^{(3)},~~ \label{decom NS}
\eea
with the threshold value $\eta = \frac{1}{d}$, 
\bea
c_1 = \frac{d^2 \lambda - 1}{d^3 \lambda  + d -2} ~~\mathrm{and}~ ~ c_2= \frac{(d-1)(d^2 \lambda +1)}{d^3 \lambda  +d-2}. \nonumber
\eea
The superscript $(j)$ stands for the composite space $A_j B_j$ for $j \in \{1,2,3\}$.
From the equation
\bea
\tr_3[ N_{23 } (\frac{1}{d}\mathbbm{1}-P_{00})_3] = \frac{2(d-1)}{d(d^3 \lambda + d -2)} Q^\Gamma, \nonumber
\eea
it holds that
\bea
& \tr[\rho W] = k ~ \tr[\rho_1 \otimes N_{23} ~ P^{(12)} \otimes (\frac{1}{d}\mathbbm{1}-P_{00})_3], & \nonumber
\\ & \bracket{\phi_{00}}{\Lambda(\rho)}{\phi_{00}} > \frac{1}{d }\iff \tr[\rho W] <0, &
\eea
where $k=\frac{d(d^3 \lambda + d -2)}{2(d-1)}$.

Hence, the partial transpose criteria can be realized by preparing a network state in Eq. (\ref{decom NS}). In particular, a network state for the decomposable EW 
\bea
W = P_{00}^{\Gamma} = \frac{\mathbbm{F}}{d},
\eea
which is proportional to the flip operator $\mathbbm{F}$ that detects entangled Werner states, can be found as 
\bea
N_{23}^{\text{(flip)}} &=& \frac{1}{d+2} \left( \frac{\mathbbm{1}-\mathbbm{F}}{d^2-d} \right)^{(2)} \otimes P_{00}^{(3)}  \nonumber
\\ && + \frac{d+1}{d+2} \left( \frac{\mathbbm{1}+\mathbbm{F}}{d^2+d} \right)^{(2)} \otimes \left( \frac{\mathbbm{1}-P_{00}}{d^2-1} \right)^{(3)}.~~ \label{flip NS}
\eea
Note that this network state is invariant under $U_{A_2} \otimes U_{B_2} \otimes V_{A_3} \otimes V^*_{B_3}$ for any unitary operation $U, V$. This network state is positive under the partial transpose $A_2 A_3: B_2 B_3$, so it is undistillable. This state (\ref{flip NS}) has been used in the activation of non-PPT entangled states and proved to be PPT in \cite{vollbrecht2002activating}.

\subsection{ Bell-diagonal EW}

The next examples are Bell-diagonal witnesses $W(\Vec{\lambda})$, which are decomposable or non-decomposable depending on the parameter $\Vec{\lambda} = (\lambda_0, \ldots, \lambda_{d-1})$:
\bea
W(\Vec{\lambda}) = \sum_{s=0}^{d-1} \lambda_s \Pi_s - P_{00}, \label{eq:Wa2}
\eea
where $\lambda_s \ge 0 ~\forall s$ and $\sum_{s=0}^{d-1} \lambda_s = 1$. 
Note also that $W[\Vec{\lambda}]$ in Eq. (\ref{eq:Wa2}) is an EW if a vector $\Vec{\lambda}$ satisfies the cyclic inequalities in the following, 
\bea
\sum_{j=0}^{d-1} \frac{t_{j}^2}{\sum_{s=0}^{d-1} \lambda_s t_{j+s}^2} \le d, \nonumber
\eea
for all $t_0, \ldots, t_d \ge 0$.
The value $\lambda_0$ is critical in implementing this witness with state preparation, as shown below. To construct a network state for a Bell-diagonal EW, we use paired Bell-diagonal (PBD) states:
\bea
N_{23}^{\textrm{(PBD)}} (\Vec{\lambda}) = \sum_{s=0}^{d-1} \lambda_s \frac{1}{d} \sum_{t=0}^{d-1} P_{st}^{(2)} \otimes P_{st}^{(3)},
\eea
with the threshold value $\eta = \lambda_0$. 
Then the following holds:
%\bea
%& \tr_3[ N_{23}^{\textrm{(PBD)}} (\lambda_0 \mathbbm{1} - P_{00})_3 ] = \frac{\lambda_0}{d} W_2^T (\Vec{\lambda}), & \nonumber
%\eea
\bea
& \tr[\rho W(\Vec{\lambda})] = k ~\tr[\rho_1 \otimes N_{23}^{\textrm{(PBD)}}(\Vec{\lambda})~ P^{(12)} \otimes (\lambda_0 \mathbbm{1} - P_{00})_3 ], & \nonumber
\\ & \bracket{\phi_{00}}{\Lambda(\rho)}{\phi_{00}} > \lambda_0 \iff \tr[\rho W(\Vec{\lambda})] <0. \nonumber &
\eea
where $k = d^3 / \lambda_0$. It is possible to achieve $\bracket{\phi_{00}}{\Lambda(\sigma)}{\phi_{00}} = \lambda_0$ with a separable state $\sigma$:
\bea
\Lambda^{(1 \to 3)} (\sigma) &=& \frac{\tr_{12}[\sigma_1 \otimes N_{23} P^{(12)}] }{\tr[\sigma_1 \otimes N_{23} P^{(12)}] }, \nonumber
\\ \textrm{where } \sigma &=& \frac{1}{d} P_{00} + \frac{1}{d} \sum_{s=1}^{d-1} \frac{\Pi_s}{d}. \nonumber
\eea
In particular, EWs from a reduction map and  the Choi map can be written in the form of Bell-diagonal witness and can be implemented by preparing the corresponding PBD state.
A decomposable EW from the reduction map corresponds to a Bell-diagonal EW with $\Vec{\lambda} = (\frac{1}{d}, \ldots, \frac{1}{d})$:
\bea
W_{\textrm{red}} &=& \sum_{s=0}^{d-1} \frac{1}{d} \Pi_s - P_{00} = \frac{1}{d} \mathbbm{1} - P_{00} \label{red},
\\ N_{23}^{\textrm{(red)}} &=& \frac{1}{d^2} \sum_{s=0}^{d-1} \sum_{t=0}^{d-1} P_{st}^{(2)} \otimes P_{st}^{(3)},
\eea
with the threshold value $\eta = \frac{1}{d}$.

The PBD state $N_{23}^{\textrm{(red)}}$ can be seen as a direct generalization of Smolin state \cite{PhysRevA.63.032306} into $d$-dimension. In $d=2$, Smolin state is PPT in $A_2 A_3:B_2 B_3$. However, the state $N_{23}^{\textrm{(red)}}$ is not PPT in higher dimensions $d \ge 3$ in general. Smolin state in two-dimension is undistillable, but the distillability of $N_{23}^{\textrm{(red)}}$ in higher dimensions is unknown.

Although it can be generalized into higher dimensions \cite{PhysRevA.84.024302, doi:10.1142/S1230161213500066}, the nondecomposable witness from Choi map \cite{Choi:1975aa} is defined in $d=3$ and corresponds to a Bell-diagonal witness with $\Vec{\lambda} = (\frac{2}{3}, \frac{1}{3}, 0)$:
\bea
W_{\textrm{Choi}} &=& \frac{2}{3} \Pi_0 + \frac{1}{3} \Pi_1 - P_{00},
\\ N_{23}^{\textrm{(Choi)}} &=& \frac{2}{9} \sum_{t=0}^{2} P_{0,t}^{(2)} \otimes P_{0,t}^{(3)} + \frac{1}{9} \sum_{t=0}^{2} P_{1,t}^{(2)} \otimes P_{1,t}^{(3)},~~~~~~
\eea
with the threshold value $\eta = \frac{2}{3}$.


\subsection{  Breuer-Hall EW}

The Breuer-Hall map \cite{PhysRevLett.97.080501, Hall_2006} finds an EW,
\bea
W_{\textrm{BH}} = \frac{1}{d} (\mathbbm{1}- \mathbbm{F}') - P_{00} ,
\eea
where $\mathbbm{F}' = (\mathbbm{1} \otimes U) \mathbbm{F} (\mathbbm{1} \otimes U^\dagger)$ for any skew-symmetric unitary operator $U$ such that $UU^\dagger = \mathbbm{1}$ and $U^T = -U$. In even dimensions $d=2n$, one can set $U$ as
\bea
U = \Oplus_{i=1}^{n} 
\begin{bmatrix}
0 & 1 \\
-1 & 0 
\end{bmatrix}, \nonumber
\eea
then it acts as $U \ket{i} = (-1)^{j} \ket{j}$ where $j = i+1$ for even $i$ and $j = i-1$ for odd $i$.

One can think of $W_{\textrm{BH}}$ as a combination of reduction EW (\ref{red}) and the flip operator from reduction map with additional $U$.
A network state for the BH EW is as follows,
\bea
N_{23}^{(\textrm{BH})}  %& = & c_0 N^{(23)}_{\textrm{PBD}} + (1-c_0) N^{(23)}_{\textrm{skew}}, \label{eq:bhn} \\
& = & c_0 \frac{1}{d^2} \sum_{s=0}^{d-1} \sum_{t=0}^{d-1} P_{st}^{(2)}  \otimes P_{st}^{(3)} \nonumber \\
&& + c_1 \left( \frac{\mathbbm{1} + \mathbbm{F}' }{d^2+d} \right)^{(2)} \otimes P_{00}^{(3)}   \nonumber \\
&& + c_2 \left( \frac{\mathbbm{1} - \mathbbm{F}' }{d^2-d} \right)^{(2)} \otimes \left( \frac{\mathbbm{1} - P_{00}}{d^2-1} \right)^{(3)},~~~~ \label{eq:nbh} 
\eea
with the threshold value $\eta = \frac{1}{d}$, where $c_0 = \frac{2d^2-2d}{3d^2 -3d +2}$, $c_1= \frac{d+1}{3d^2-3d+2}$, and $c_2 = \frac{d^2-2d+1}{3d^2-3d+2}$. One can show that
\bea
 \tr_3[ N_{23}^{(\mathrm{BH})}~ (\eta \mathbbm{1} - P_{00})_3] = \frac{c_0}{d^2} W_{\mathrm{BH}}^T,
\eea
which leads to
\bea
& \tr[\rho W_{\textrm{BH}}] = k ~\tr[\rho_1 \otimes N_{23}^{\textrm{(BH)}}~ P^{(12)} \otimes (\eta \mathbbm{1} - P_{00})_3 ], & \nonumber
\\ & \bracket{\phi_{00}}{\Lambda(\rho)}{\phi_{00}} > \frac{1}{d} \iff \tr[\rho W_{\textrm{BH}}] <0, \nonumber &
\eea
where $k =  d^4 / c_0$.

%There is a final remark on the relation between the network state $N_{23}$ and the corresponding EW $W$. If the state $N_{23}$ is PPT under $A_2A_3:B_2B_3$, then the EW $W$ must be decomposable, and the threshold value is $\eta=\frac{1}{d}$. This can be understood as a view of activating distillability. Suppose that both states $\rho_1$ and $N_{23}$ are PPT, so undistillable. They remain PPT after the local projections $P_{00}^{(A_1 A_2)} \otimes P_{00}^{(B_1 B_2)}$, so the upper bound of the singlet fraction of $\Lambda(\rho)$ is $\frac{1}{d}$. For some state $\tau$, $\Lambda(\tau)$ having a singlet fraction higher than $\frac{1}{d}$ implies that $\tau_1 \otimes N_{23}$ is distillable \cite{PhysRevA.59.4206}, which implies that $\tau$ is non-PPT. The EW constructed from a PPT network state cannot detect PPT entangled states, i.e., it is decomposable.
%The contraposition states that to make a nondecomposable EW, the network state must be non-PPT. However, the converse does not hold: an EW constructed from a non-PPT network state is not always nondecomposable. A counterexample is already given in the paper: the reduction witness is decomposable in any dimension, while high-dimensional Smolin states are non-PPT. The relations between other properties of EW, e.g., atomicity and optimality, and the properties of network states remain open questions.



\begin{figure}[t]
\centering
\includegraphics[width=8.8cm ]{figm}
\caption{ A graph state can be detected by preparing a network state, Bell measurements, and a fixed measurement. }
\label{fig:figm}
\end{figure}




\section{Entanglement witnesses for graph states}
\label{app3}
A graph $G = (V ,E)$ is defined by a set of vertices $V$ and a set of edges $E$:
\bea
V &=& \{1,\ldots,n\}, ~ \text{where $n$ is the number of vertices,} \nonumber
\\ E &=& \{(i,j)| i,j \in V, i<j, \text{Vertex $i$ and $j$ are connected.}\} \nonumber
\eea
Also define the neighborhood of vertex $i$: $E_i = \{j \in V| (i,j) \in E \text{ or } (j,i) \in E\}$, which is a set of vertices connected to vertex $i$.
% A graph $G = (V ,E)$ is defined by a set $V$ of vertices corresponding to qubits and a set $E = (E_1, \ldots, E_n)$ of edges connecting some of these vertices, where $E_i$ is the set of vertices connected to the $i$-th vertex.

The generators $g_i$ and the projectors $\gamma_i$ of a graph state determined by the graph $G$ are given by
\bea
g_i &=& X_i \prod_{j \in E_i} Z_j, \nonumber
\\ \gamma_i^{(x_i)} &=& \frac{\mathbbm{1}+(-1)^{x_i} g_i}{2}.
\eea
Note that the eigenvalue of $g_i$ is either 1 or -1, and the eigenvalue of $\gamma_i$ is either 1 or 0. Now we define an orthonormal graph state basis consisting of $2^n$ states:
\bea
\ket{\Vec{x}}\bra{\Vec{x}}_G = \prod_{i=1}^{n} \gamma_i^{(x_i)}, ~~ \text{where } \Vec{x} = (x_1, \ldots, x_n) \in \{0,1\}^n. \nonumber
\eea
The states $\ket{\Vec{x}}_G$ are the eigenstates of the generators and projectors: 
\bea
g_i \ket{\Vec{x}}_G &=& (-1)^{x_i} \ket{\Vec{x}}_G, \\
\gamma_i^{(k)} \ket{\Vec{x}}_G &=&
\begin{cases}
\ket{\Vec{x}}_G & \mbox{if}\; x_i = k,\\
0 & \mbox{if}\; x_i \ne k.
\end{cases}
\eea
A graph state, denoted by $\ket{\Vec{0}}\bra{\Vec{0}}_G$, corresponds to an eigenstate with eigenvalue $+1$ for all generators $g_i$ ($i \in \{1, \ldots, n\}$). A state can be obtained by preparing $\ket{+} = (\ket{0}+\ket{1})/\sqrt{2}$ placed at vertices in $V$, and applying the controlled-Z ($CZ$) gate to all edges in $E$, see Fig. \ref{fig:graph}.

\begin{figure}[t]
\centering
\includegraphics[width=7cm]{cluster.pdf}
\caption{A graph consisting of a set of vertices and a set of edges uniquely defines the graph state. The graph state can be obtained by preparing $\ket{+} = (\ket{0}+\ket{1})/\sqrt{2}$ at all vertices in $V$, and applying the controlled-Z ($CZ$) gats to all edges in $E$.}
\label{fig:graph}
\end{figure}

\begin{figure}[t]
\centering
\includegraphics[width=8cm]{Cl4.pdf}
\caption{The four-qubit linear cluster state on the graph $G_{Cl4}$. One can obatin the graph state $\ketbra{0000}_G$ by preparing four $\ket{+}$ states and applying the controlled-Z gate to three edges.}
\label{fig:Cl4}
\end{figure}

For instance, the four-qubit linear cluster state can be defined by a graph $G_{Cl4} = (V_4,E_{Cl4})$, where $V_4 = \{1,2,3,4\}$ and $E_{Cl4} = \{(1,2),(2,3),(3,4)\}$, see Fig. \ref{fig:Cl4}. 
% The neighborhood of vertices are given by $E_1=\{2\}$, $E_2=\{1,3\}$, $E_3=\{2,4\}$, $E_4=\{3\}$. 
The generators define $16$ graph states $\ket{0000}\bra{0000}_G, \ldots,$ and $\ket{1111}\bra{1111}_G$. The graph state $\ket{0000}\bra{0000}_G$ is detected by an EW in the following,
\bea
W_{Cl4} = \frac{1}{2} \sum_{\Vec{x} \in S} \ket{\Vec{x}}\bra{\Vec{x}}_G - \ketbra{0000}_G,
\eea
where $S = \{$0000, 0001, 0010, 0011, 0100, 0101, 0110, 0111, 1000, 1010, 1100, 1110$\}$. Then the network state for this graph state can be constructed by 
\bea
N_{23}^{(Cl4)} = \sum_{\Vec{x} \in S} \frac{1}{12} \ketbra{\Vec{x}}_G^{(2)} \otimes \ketbra{\Vec{x}}_G^{(3)}.
\eea
It holds that
\bea
&& \tr[\rho_1 \otimes N_{23}^{(Cl4)} P^{(12)} \otimes ( \frac{1}{2}\mathbbm{1} - \ketbra{0000}_G ) ] \nonumber
\\ && \propto \tr[\rho W_{Cl4}],
\eea
where
\bea
P^{(12 )} &=& |\phi^{+}\rangle_{A_1A_2}\langle \phi^+| \otimes |\phi^{+}\rangle_{B_1B_2}\langle \phi^+| \nonumber
\\ && \otimes |\phi^{+}\rangle_{C_1C_2}\langle \phi^+| \otimes |\phi^{+}\rangle_{D_1D_2}\langle \phi^+|. \nonumber
\eea
Then a four-qubit entangled state $\rho$ is detected by $W_{Cl4}$ when
\bea
{}_{G} \langle 0000| \Lambda [\rho]| 0000\rangle_G > \frac{1}{2},
\eea
where the map $\Lambda^{(1 \to 3)}$ is defined as,
\bea
\Lambda^{( 1\rightarrow 3)} [\rho]= \frac{ \tr_{12}[\rho_{1}\otimes N_{23} P^{(12)}] }{ \tr[\rho_1\otimes N_{23} P^{(12)}]  }. \nonumber
\eea
\\

A typical decomposable EW can be written in the following form \cite{PhysRevA.84.032310}:
\bea
W_G = \frac{1}{2} \sum_{\Vec{x} \in S} \ketbra{\Vec{x}}_G - \ketbra{\Vec{0}}_G,
\eea
where the set $S \subseteq \{0,1\}^n$ depends on $W_G$.
Then the network states $N_{23}^{G}$ for this entanglement witnesses are given by
\bea
N_{23}^{(G)} = \sum_{\Vec{x} \in S} \frac{1}{|S|} \ketbra{\Vec{x}}_G^{(2)} \otimes \ketbra{\Vec{x}}_G^{(3)}.
\eea
An $n$-qubit entangled state $\rho$ is detected by $W_G$ when
\bea
{}_{G} \langle \Vec{0}| \Lambda [\rho]| \Vec{0}\rangle_G &>& \frac{1}{2},
\eea
for
\bea
\Lambda^{( 1\rightarrow 3)} [\rho] &=& \frac{ \tr_{12}[\rho_{1}\otimes N_{23} P^{(12)}] }{ \tr[\rho_1\otimes N_{23} P^{(12)}]  }, \nonumber\\ 
P^{(12)} &=& \Otimes_{v \in V} |\phi^+\rangle_{v_1 v_2}\langle\phi^+|, \nonumber
\eea
where $V$ denotes the set of vertices of the graph $G$. 

% Then the overlap with the graph state $\ket{\Vec{0}}_G$ of the result state $\Lambda[\rho]$ is checked whether it is higher than $\frac{1}{2}$ or not. 
The overlap ${}_{G} \langle \Vec{0}| \Lambda [\rho]| \Vec{0}\rangle_G$ can be estimated with a fixed measurement on individual qubits. Note that a graph state is generated as follows, $\ket{\Vec{0}}_G$ is obtained as $\ket{\Vec{0}}_G = (\Otimes_{e \in E} U_{CZ}) (\Otimes_{v \in V} H) \ket{0}^{\otimes n}$. Then, the estimation of a singlet fraction can be achieved by applying an interaction $(\Otimes_{v \in V} H) (\Otimes_{e \in E} U_{CZ})$ to a resulting state $\Lambda^{(1 \to 3)} [\rho]$ and then performing a measurement in the computational basis.  The probability of obtaining an outcome $00$ gives the overlap ${}_{G} \langle \Vec{0}| \Lambda [\rho]| \Vec{0}\rangle_G$.
It holds that ${}_{G} \langle \Vec{0}| \Lambda [\rho]| \Vec{0}\rangle_G > \frac{1}{2}$ if and only if $\tr[\rho W] < 0$, which certifies that $\rho$ is entangled.

\end{document}
 

