
\section{Introduction}

The past decade has witnessed dramatic progress in Artificial Intelligence (AI), which has made a profound impact in almost every domain, such as natural language processing~\cite{chowdhary2020natural}, computer vision~\cite{voulodimos2018deep}, recommender system~\cite{zhang2019deep}, healthcare~\cite{miotto2018deep}, biology~\cite{webb2018deep}, finance~\cite{ozbayoglu2020deep}, and so forth. A vital enabler of these great successes is the availability of abundant and high-quality data. Many major AI breakthroughs occur only after we have the access to the right training data. For example, AlexNet~\cite{krizhevsky2017imagenet}, one of the first successful convolutional neural networks, was designed based on the ImageNet dataset~\cite{deng2009imagenet}. AlphaFold~\cite{jumper2021highly}, a breakthrough of AI in scientific discovery, will not be possible without annotated protein sequences~\cite{mirdita2017uniclust}. The recent advances in large language models rely on large text data for training~\cite{kenton2019bert,radford2018improving,radford2019language,brown2020language} (left of Figure~\ref{fig:motivation}). Besides training data, well-designed inference data has facilitated the initial recognition of numerous critical issues in AI and unlocked new model capabilities. A famous example is adversarial samples~\cite{kurakin2018adversarial} that confuse neural networks through specialized modifications of input data, which causes a surge of interest in studying AI security.
%Another notable example is ProPublica's COMPAS dataset~\cite{barenstein2019propublica}, which shows that the prediction of recidivism was racially biased and raises serious concerns about AI fairness. 
Another example is prompt engineering~\cite{liu2023pre}, which accomplishes various tasks by solely tuning the input data to probe knowledge from the model while keeping the model fixed (right of Figure~\ref{fig:motivation}). In parallel, the value of data has been well-recognized in industries. Many big tech companies have built infrastructures to organize, understand, and debug data for building AI systems~\cite{thusoo2010data,barclay2000microsoft,armbrust2021lakehouse,varia2014overview}. All these efforts in constructing training data, inference data, and the infrastructure to maintain data have paved the path for the achievements in AI today.



\begin{figure}[t]
  \centering
  \begin{subfigure}[b]{0.545\textwidth}
    \centering
    \includegraphics[width=1.0\textwidth]{figures/motivation_1.pdf}
  \end{subfigure}%
  \begin{subfigure}[b]{0.455\textwidth}
    \centering
    \includegraphics[width=1.0\textwidth]{figures/motivation_2.pdf}
  \end{subfigure}%
  \vspace{-7pt}
  \caption{Motivating examples that highlight the central role of data in AI. On the left, large and high-quality training data are the driving force of recent successes of GPT models, while model architectures remain similar, except for more model weights. The detailed data collection strategies of GPT models are provided in~\cite{zhu2015aligning,radford2018improving,radford2019language,brown2020language,ouyang2022training,gpt4}. On the right, when the model becomes sufficiently powerful, we only need to engineer prompts (inference data) to accomplish our objectives, with the model being fixed.}
  \vspace{-10pt}
  \label{fig:motivation}
\end{figure}
% advocates for a more data-centric rather than model-centric strategy in AI projects. 
Recently, the role of data in AI has been significantly magnified, giving rise to the emerging concept of \emph{data-centric AI}~\cite{polyzotis2021can,jarrahi2022principles,jakubik2022data,zha2023data,whang2023data}. In the conventional model-centric AI lifecycle, researchers and developers primarily focus on identifying more effective models to improve AI performance while keeping the data largely unchanged. However, this model-centric paradigm overlooks the potential quality issues and undesirable flaws of data, such as missing values, incorrect labels, and anomalies. Complementing the existing efforts in model advancement, data-centric AI emphasizes the systematic engineering of data to build AI systems, shifting our focus from model to data. It is important to note that ``data-centric'' differs fundamentally from ``data-driven'', as the latter only emphasizes the use of data to guide AI development, which typically still centers on developing models rather than engineering data.

Several initiatives have already been dedicated to the data-centric AI movement. A notable one is a competition launched by Ng et al.~\cite{ng2021data}, which asks the participants to iterate on the dataset only to improve the performance. Snorkel~\cite{ratner2017snorkel} builds a system that enables automatic data annotation with heuristic functions without hand labeling. A few rising AI companies have placed data in the central role because of many benefits, such as improved accuracy, faster deployment, and standardized workflow~\cite{landingai,snorkelai,scaleai}. These collective initiatives across academia and industry demonstrate the necessity of building AI systems using data-centric approaches.

With the growing need for data-centric AI, various methods have been proposed. Some relevant research subjects are not new. For instance, data augmentation~\cite{feng2021survey} has been extensively investigated to improve data diversity. Feature selection~\cite{li2017feature} has been studied since decades ago for preparing more concise data. Meanwhile, some new research directions have emerged recently, such as data programming for labeling data quickly~\cite{ratner2016data}, algorithmic
recourse for understanding model decisions~\cite{karimi2021algorithmic}, and prompt engineering that modifies the input of large language models to obtain the desirable predictions~\cite{liu2023pre}. From another dimension, some works are dedicated to making data processing more automated, such as automated data augmentation~\cite{cubuk2019autoaugment}, and automated pipeline discovery~\cite{drori2021alphad3m,lai2021tods}. Some other methods emphasize human-machine collaboration in creating data so that the model can align with human intentions. For example, the remarkable success of ChatGPT and GPT-4~\cite{gpt4} is largely attributed to the reinforcement learning from human feedback procedure~\cite{christiano2017deep}, which asks humans to provide appropriate responses to prompts and rank the outputs to serve as the rewards~\cite{ouyang2022training}. Although the above methods are independently developed for different purposes, their common objective is to ensure data quality, quantity, and reliability so that the models behave as intended.

Motivated by the need for data-centric AI and the numerous proposed methods, this survey provides a holistic view of the technological advances in data-centric AI and summarizes the existing research directions. In particular, this survey centers on the following research questions:


% With the great successes of data-centric approaches in many fields, data-centric AI has become one of the hottest concepts in the AI and machine learning community. 

%  that focus on engineering the data used to build AI systems

\begin{itemize}
    \item RQ1: What are the necessary tasks to make AI data-centric?
    \item RQ2: Why is automation significant for developing and maintaining data?
    \item RQ3: In which cases and why is human participation essential in data-centric AI?
    \item RQ4: What is the current progress of data-centric AI?
\end{itemize}



By answering these questions, we make three contributions. \emph{Firstly}, we provide a comprehensive overview to help readers efficiently grasp a broad picture of data-centric AI from different perspectives, including definitions, tasks, algorithms, challenges, and benchmarks. \emph{Secondly}, we organize the existing literature under a goal-driven taxonomy. We further identify whether human involvement is needed in each method and label the method with a level of automation or a degree of human participation. \emph{Lastly}, we analyze the existing research and discuss potential future opportunities.


This survey is structured as follows. Section~\ref{sec:2} presents an overview of the concepts and tasks related to data-centric AI. Then, we elaborate on the needs, representative methods, and challenges of three general data-centric AI goals, including training data development (Section~\ref{sec:3}), inference data development (Section~\ref{sec:4}), and data maintenance (Section~\ref{sec:5}). Section~\ref{sec:6} summarizes benchmarks for various tasks. Section~\ref{sec:7} discusses data-centric AI from a global view and highlights the potential future directions. Finally, we conclude this survey in Section~\ref{sec:8}.


