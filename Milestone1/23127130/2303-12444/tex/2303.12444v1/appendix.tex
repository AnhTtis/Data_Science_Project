\begin{appendix}
%\label{sec: Appendix}


\section{APX-hardness for computing MMS and APS for submodular valuations}
\label{sec:APX}

The current section is presented in a somewhat sketchy way, and without detailed proofs, as it is not the main focus of the current paper.

In general, computing the MMS and the APS are both NP-hard tasks. For example, for the case of two agents with equal entitlements and additive valuations, weak NP-hardness is a straightforward consequence of the NP-hardness of the PARTITION problem (given a set of integers, is there a subset whose sum of values is exactly half of the total sum?). Computing the APS is in some sense an easier task than computing the MMS. In particular, for additive valuations there are pseudo-polynomial time algorithms for computing the APS~\cite{BEF21}, whereas computing the MMS is strongly NP-hard. Also, as the value of the APS is a solution to a linear program with exponentially many constraints (Definition~\ref{def:APS}), the APS can be computed in polynomial time (using the ellipsoid algorithm) if there is a separation oracle for the linear program. This separation oracle corresponds to a computational problem that has a natural economic interpretation: given prices to the items and a budget for the agent, which is the highest value bundle that the agent can afford? For a given valuation function, if one can answer such queries in polynomial time, then the APS can be computed in polynomial time. 

In the case of equal entitlement, the APS is at least as large as the MMS, and sometimes strictly larger. For submodular valuations, a ratio of $\frac{5}{6}$ between the MMS and the APS is demonstrated in~\cite{BEF21}, for an allocation instance with six items and two agents with equal entitlements.


For submodular valuations, both the MMS and the APS are APX-hard to compute. We are not aware of a reference in which such a statement is proved explicitly. However, $(1 - \frac{1}{e})$ approximation hardness can be proved using known techniques. (For simplicity, we omit here low order additive terms when stating approximation ratios.) Let us briefly explain how. A certain reduction template (starting from the APX-hard problem Max 3SAT) described in~\cite{Feige98} established hardness of approximation results for Min Set Cover  (within a ratio of $\ln n$) and Max $k$-Coverage (within a ratio of $1 - \frac{1}{e}$). The same reduction template, but starting from a different APX-hard problem (Max 3-Coloring), was used in~\cite{Feige_domatic02} to prove hardness of approximation of the Domatic Number (within a ratio of $\ln n$), and  in~\cite{DBLP:journals/algorithmica/KhotLMM08} to prove that the maximum welfare problem with submodular valuations is hard to approximate within a ratio of $1 - \frac{1}{e}$. The reduction used to prove this last result implies that for submodular valuations it is NP-hard to approximate the MMS within a ratio better than $1 - \frac{1}{e}$. The reason for this is that in the maximum welfare instance constructed by the reduction, all agents have the same submodular valuation function (call it $v$). Moreover, on {\em yes} instances, the maximum welfare allocation gives all agents the same value (call it $t$, with the total welfare being $n\cdot t$). Hence on {\em yes} instances, the MMS of $v$ is $t$ (as the maximum welfare allocation serves as an MMS partition). On {\em no} instances, the fact that the welfare is at most $(1-e)n\cdot t$ implies that in every partition to $n$ bundles, at least one of the bundles has value at most $(1-e)t$, showing that the MMS is at most $(1-\frac{1}{e})t$. As it is NP-hard to distinguish between {\em yes} and {\em no} instances, we get a hardness of approximation for the MMS.

To derive $(1 - \frac{1}{e})$ approximation hardness for the APS, one further observes the following property of {\em no} instances that result from using the reduction template: every set that contains a $\frac{1}{n}$ fraction of the items has value at most $(1 - \frac{1}{e})t$. This property is inherited from the fact that the same reduction template is used in the proof in~\cite{Feige98} that Max $k$-Coverage is hard to approximate within a ratio better than $1 - \frac{1}{e}$. When the entitlement is $\frac{1}{n}$, the APS fractional partition (of Definition~\ref{def:dual}) must contain at least one bundle with at most  a $\frac{1}{n}$ fraction of the items, and hence the APS for {\em no} instances is at most $(1 - \frac{1}{e})t$.





%\ufc{At some place say that for submodular there are examples (random set cover instances) in which $MMS \simeq (1 - \frac{1}{e}) \cdot APS$.}
%\gbc{Talk with Uri about this point in the meeting.}





\section{A simple technical claim}


%\ufc{The proof should be presented as an easy corollary of Claim~\ref{cl:item_reduce}, and not as an independent proof. This was the point of having Claim~\ref{cl:item_reduce}, so that we do not need to reprove it every time that we use it. }
%\gbc{\cref{claim:APS not decrease in I_s} and the below presentation of $\hat{I_s}$ is used in the lemma of $\rho=\frac{1}{3}$ in the arbitrary entitlements case. 
%I wonder whether to keep it here in the preliminaries or to shift it inside the proof of the arbitrary entitlements case. 
%Also, This is the second place where I am using the notation of residual instance $I_s$, and defining the normalized instance $I_s$. (It is also presented in the end of \Cref{thm:1/3_APS_guarantee})}\\

{We recall notation used when describing the $proportional(\rho)$ bidding strategy for the bidding game in Section~\ref{sec:APS}. Let $I_{s}$ denote the residual instance after round $s$, which is the first point in time after which no items satisfy $v_{p}(e)>2\rho APS_{p}$. The total budget $b^s$ remaining for all agents is denoted by $\gamma$. We consider a new instance
$\hat{I_{s}}$ to be the following:
\begin{itemize}
\item The set of agents is those who are active in $I_{s}$.
\item The entitlement of an agent in $\hat{I_{s}}$ is her budget in
$I_{s}$, scaled by $\frac{1}{\gamma}$. Consequently, the entitlements
are non-negative and sum up to 1.
\item The items of $\hat{I_{s}}$ are those of $I_{s}$ (denoted by $\items^s$), and $v_p$ remains unchanged (over subsets of $\items^s$).
\end{itemize}}

{
\begin{claim}
\label{claim:APS not decrease in I_s}
Given that agent $p$ did not win an item yet (in the first $s$ rounds),
then $APS(\mathcal{M}^{s},v_{p},\frac{1}{\gamma}b_{p})\geq APS(\mathcal{M},v_{p},b_{p})$.
In other words, the $APS$ of agent $p$ in $\hat{I_{s}}$ is at least
as her APS in $I_{0}$, the original instance. (Referring \Cref{def:$v^t_p$}, in the special case of $v_p=v_p^t$ with $t=APS_p$, then the claim holds with equality)
\end{claim}
\begin{proof}
Agent $p$ did not win an item in the first $s$ rounds, while there
is an item with a value greater than $2\rho$ of her APS. Therefore
by the proportional bidding strategy, she bids her entire budget,
$b_{p}$, in each of these rounds. Thus, after these $s$ rounds,
$\gamma=b^{s}\geq1-s\cdot b_{p}$. Let $P_{0}=\{\lambda_{S}S\}$ be
a fractional partition associated with the APS of agent $p$. then
we define a new fractional partition $P_{s}=\{\lambda'_{S}S\}$ to
be the following
\[
\lambda'_{S}=\begin{cases}
\frac{1}{\gamma}\lambda_{S} & \text{ if }S\subseteq\mathcal{M}^{s}\\
0 & \text{otherwise}
\end{cases}
\]

We claim that $P_{s}$ witness that indeed $APS(\mathcal{M}^{s},v_{p},\frac{1}{\gamma}b_{p})\geq APS(\mathcal{M},v_{p},b_{p})$.
\begin{itemize}
\item 
\begin{align*}
\sum_{S\subseteq\mathcal{M}^{s}}\lambda'_{S} & =\frac{1}{\gamma}\sum_{S\subseteq\mathcal{M}^{s}}\lambda{}_{S}=\frac{1}{\gamma}(\sum_{S\subseteq\mathcal{M}}\lambda_{S}-\sum_{e\in\mathcal{M\setminus\mathcal{M}}^{s}}\sum_{S\mid e\in S}\lambda_{S})\\
 & \underset{*}{\geq}\frac{1}{\gamma}(1-\sum_{e\in\mathcal{M\setminus\mathcal{M}}^{s}}b_{p})\\
 & =\frac{1}{\gamma}(1-s\cdot b_{p})\\
 & \geq1
\end{align*}
By definition of $P_o$, for each $e\in\items$, $\sum_{S\mid e\in S}\lambda_S\leq b_p$, which justify inequality *.
\item Each $e\in\mathcal{M}^{s}$ satisfies:
\[
\sum_{S\mid e\in S}\lambda'_{S}=\frac{1}{\gamma}\sum_{S\mid e\in S}\lambda_{S}\le\frac{1}{\gamma}b_{p}
\]
 Where $\frac{1}{\gamma}b_{p}$ is the entitlement of $p$ in $\hat{I_{s}}$.
\item Each $S\subseteq\mathcal{M}^{s}$ with strictly positive weight in $P_s$ is of value $v_{p}(S)\geq APS_{p}$ (since a bundle with positive weight
also has positive weight in $P_0$). In the special case of $v_p=v_p^t$ with $t=APS_p$, (no bundle equals more than $APS_p$) then clearly also $v_p(S)\leq APS_p$, and therefore $v_p(S)=APS_p$
\end{itemize}
\end{proof}
}

\section{A negative example}
\label{sec:example}

We restate and prove \Cref{no rho larger than 1/3 for original bidding game}.

\exampleThird*

\begin{proof}
We present a series of instances in which agent $p$ with a submodular valuation function executes the proportional bidding strategy.

The instances are parameterized by $k\in\mathbb{N}$. The $k$th instance will be as follows:
Define
\begin{align*}
q_{1} & =2\\
q_{k} & =1+\prod_{i=1}^{k-1}q_{i}
\end{align*}

(This sequence is known as the Sylvester sequence)

The number of agents will be: $n_{k}=q_{k+1}-1$ (for example, for
$k=2$, $n_{k}=2\cdot3\cdot7=43$)

The set of item is $\mathcal{M}=\{e_{i,j}\}$ for $1\leq i\leq k+1$,
$1\leq j\leq n$ ($n\cdot(k+1)$ items)

If we think of $e_{i,j}$ as arranged in a matrix, then all the items
in a row are copies of the same item and are substitutes. The value
of items from different rows is additive.

For any $1\le i\leq k$ and for any $j$, $v_{p}(e_{i,j})=\frac{2}{q_{i}}$.
For $i=k+1$ and any $j$, $v_{p}(e_{k+1,j})=1$. For example, if $k=3$,
there are $43$ agents and columns, and in each column $j$, $v_{p}(e_{1,j})=1$, $v_{p}(e_{2,j})=\frac{2}{3}$, $v_{p}(e_{3,j})=\frac{2}{7}$, $v_{p}(e_{4,j})=1$.

\begin{itemize}

\item $v_{p}$ is submodular. The marginal value of each item is weakly decreasing (the marginal value of item $e_{i,j}$ to a set $S$ is either $v_{p}(e_{i,j})$ or $0$, depending on whether the set $S$ already contains an item from the $i$'th row).

\item The columns $C_j$ of the matrix $\{e_{i,j}\}$ form an MMS partition. The value of every bundle is at most $v_{p}(\mathcal{M})$, and in this partition, the value of each bundle (column) is exactly $v_{p}(\mathcal{M})$.

\item $APS_p = MMS_{p} =  v_{p}(\mathcal{M})=v_{p}(C_{j})=v_{p}(e_{k+1,j})+\sum_{i=1}^{k}v_{p}(e_{i,j})=1+\sum_{i=1}^{k}\frac{2}{q_{i}}=1+2\sum_{i=1}^{k}\frac{1}{q_{i}}=1+2\cdot(1-\frac{1}{q_{k+1}-1})\underset{*}{=}3-\frac{2}{q_{k+1}-1}$,
where equality {*} is a known property of the partial sums of Sylvester's inverse series (this can be proved by induction, Wikipedia value of Sylvester sequence).

\item $q_{i}$ divides $n$, for every $i \le k$.

\end{itemize}

For convenience, we assume w.l.o.g that the budget of each agent is $2$.

For every $k$, we first show a run of the bidding game with adversarial bidding of the other agents, in which agent $p$ executes the proportional bidding strategy with $\rho_{k}=\frac{1}{APS_{p}}$, and she receives a value of precisely $1$ (she gets the bundle that consists only of items from row $k+1$) which is a $\frac{1}{APS_{p}}$ of her $APS$.
For that instance, $\rho_{k}=\frac{1}{3-\frac{2}{q_{k+1}-1}}$. The series of $\rho_{k}$ is monotonically decreasing to a limit of $\frac{1}{3}$. (Sylvester's sequence grows at a doubly exponential rate. Hence, the sequence of $\rho_{k}$ converges very fast.)

% \ufc{remove: We now turn to present the adversarial bidding of the other agents, yielding the adversarial run, which induces a run of the bidding game with agent $p$ obtaining a value of exactly $\rho_{k}$ of her APS.}

Consider the $I_{k}$ instance parameterized by $k$. For convenience, assume the budget of each agent is $2$ (which is $2\rho_{k}APS_{p})$. Then, by $proportional(\rho_{k})$, in each round, agent $p$ bids the highest marginal value of the remained items.
We now present the adversarial run.

In round $1$, $p$ bids $1$, and is allowed to win. She selects an item of value $1$ from row $k+1$. 

In each of the next $n$ rounds, at least one of the first $\frac{n}{2}$ other agents bids $1$, and upon winning (note that $p$ bids $1$ in each of these rounds, and the algorithm is assumed to brake the ties adversarially), takes an item from the first row (i.e., $e_{1,j}$). 
%Since the highest value of an item is $1,$ , so $p$ does not win any item from the first row, and 
All items of the first row are taken by $\frac{n}{2}$ of the other agents. These $\frac{n}{2}$ agents exhaust their budget and become inactive. 

%From now on, since items are substitutes along rows, an item with the highest marginal value for $p$ is taken from the lower row available.

In each of the next $n$ rounds, at least one of the next $\frac{n}{3}$ other agents bids $\frac{2}{3}$, and upon winning (note that $p$ bids $\frac{2}{3}$ in each of these rounds), takes an item from the second row (i.e., $e_{1,j})$. Each such agent becomes inactive after taking three items.

%$p$ bids $\frac{2}{3}$, again the algorithm brake ties adversarially so $\frac{n}{3}$ of the other agents win the items in the second rows exhausting their budget. 
The run proceeds in the same way, where for every $i$, $\frac{n}{q_{i}}$ of the other agents bid $\frac{2}{q_{i}}$, win all the items in the $i$'th row, and become inactive. Note that we use the property of $q_{i}\mid n$ for every $i\leq k$.

Thus, the number of other agents that take all items from rows $1$ to $k$ is:
\[
\sum_{i=1}^{k}\frac{n}{q_{i}}=n\cdot\sum_{i=1}^{k}\frac{1}{q_{i}}=n\cdot(1-\frac{1}{q_{k+1}-1})=n\cdot(1-\frac{1}{n})=n-1
\]

Thus, there are sufficiently many other agents to take all items from rows $1$ to $k$, and agent $p$ gets items only from row $k+1$. As they are substitutes, the total value received by $p$ is $1$.

Notice that if $p$ executes $proportional(\rho'$) with $\rho'>\rho_{k}$, then the bids of $p$ in each round are strictly smaller than those described above. Hence same run of the algorithm holds, and $p$ does not get a bundle of value $\rho' APS_p$, but rather only $\rho_k APS_p$. Hence, $I_{k}$ serves as an example showing for every $\rho'>\rho_{k}$ that $proportional(\rho')$ does not guarantee $p$ a value of $\rho' APS_p$.

Since $\rho_{k}$ is close to $\frac{1}{3}$ as we wish, for any $\rho>\frac{1}{3}$, there exist a witness $I_{k}$ on which $proportional(\rho)$ does not guarantee $p$ a $(\rho)$-fraction of her $APS$.
\end{proof}


\end{appendix}