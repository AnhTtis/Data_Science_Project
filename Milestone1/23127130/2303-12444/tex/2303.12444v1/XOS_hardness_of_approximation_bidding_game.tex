%Now we show that agents with XOS valuation functions do not have safe strategies that guarantee them a constant fraction of their MMS in the bidding game.
%\gbc{As you changed \Cref{no rho larger than 1/3 for original bidding game} from lemma to proposition, do you want to change \Cref{prop:XOS_hardness} from lemma to proposition? Recall that we state \Cref{prop:XOS_hardness} in the introduction}

We now restate and prove \Cref{prop:XOS_hardness}, showing that our proof techniques do not extend to XOS valuations.
\XosHardness*

\begin{proof}
For parameters $n,k$, define the instance $I(n,k)$ as follows. There
are $n$ agents with equal entitlements. The set $\mathcal{M}$ of items  consists of $nk$ items $e_{ij}$ for $1\leq i\leq k$ and $1\leq j\leq n$. We think of $\mathcal{M}$ as arranged in an $k\times n$ matrix, with $e_{ij}$ in the $ij$ entry. For every column $j$, let $c_{j}$ be the additive valuation function defined by giving value $1$ to items in column $j$, and $0$ to all other items. Let $v$ be the pointwise maximum of the functions $c_{j}$, that is, $v(S)=\max_{j}c_{j}(S)$ for every $S\subseteq\mathcal{M}$.
Then the valuation function $v$ is an XOS function by its definition.
We focus on a specific agent $p$ whose valuation function is $v_{p}=v$.
(The other agents may have arbitrary valuations.)

\begin{claim}
\label{claim:XOS_hardness}
If $n\geq4k^{2}$, no bidding strategy can guarantee $p$ more than a $1/k$-fraction
of $MMS_{p}$.
\end{claim}

\begin{proof}
For convenience, assume all agents are given a budget of $k$. We give
the other agents adversarial bidding strategies, as follows. There
are two types of agents.
\begin{itemize}
\item Type 1 agents consist of $n/2$ of the agents that always bid $1/2$,
and take an arbitrary available item upon winning.
\item Type 2 agents consist of the rest $n/2-1$ agents, which operate
as follows. Once agent $p$ wins an item $e_{ij}$, an agent of this
type bids all of her budget in the next $k-1$ rounds, and upon winning,
chooses an available item from column $j$ (and becomes inactive).
\end{itemize}
An agent of type 1 becomes inactive after winning exactly $2k$ items.
As the number of items is $nk$, it follows that there exists an active
agent of type 1 in every round. Thus, agent $p$ must pay $1/2$ for
every won item, so she can win at most $2k$ items overall. Once $p$
wins her first item from some column $j$, if there exist at least
$k-1$ active agents of type 2, all other items in column $j$ will
be taken by them in the next $k-1$ rounds. So, if we start with at
least $(2k)\cdot(k-1)$ agents of type 2, $p$ will not win more than
one item from every column. As $(2k)\cdot(k-1)\leq2k^{2}-1\leq n/2-1$,
this indeed holds, so agent $p$ cannot win a bundle of value more
than $1$. Observe that $MMS_{p}=k$, so the claim follows.
\end{proof}

Proposition~\ref{prop:XOS_hardness} is an immediate consequence of Claim~\ref{claim:XOS_hardness}.
\end{proof}
