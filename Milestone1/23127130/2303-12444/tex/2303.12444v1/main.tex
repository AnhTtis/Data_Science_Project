\documentclass[11pt]{article}
\usepackage[utf8]{inputenc}

% \setlength{\textheight}{8.8in}
% \setlength{\textwidth}{6.5in}
% \setlength{\evensidemargin}{-0.18in}
% \setlength{\oddsidemargin}{-0.18in}
% \setlength{\headheight}{10pt}
% \setlength{\headsep}{10pt}
% \setlength{\topsep}{0in}
% \setlength{\topmargin}{0.0in}
% \setlength{\itemsep}{0in}
% \renewcommand{\baselinestretch}{1.2}

\usepackage{fullpage}
\renewcommand{\baselinestretch}{1.1}

\usepackage{hyperref}
\usepackage{amsmath}
\usepackage{amsthm}
\usepackage{cleveref}
\usepackage{amssymb}
\usepackage{mathtools}
\usepackage[linesnumbered]{algorithm2e}
\usepackage{thmtools}
\usepackage{thm-restate}



\def\proofname{proof}
\newtheorem{theorem}{Theorem}
\newtheorem{corollary}[theorem]{Corollary}
\newtheorem{conjecture}[theorem]{Conjecture}
\newtheorem{lemma}[theorem]{Lemma}
\newtheorem{claim}[theorem]{Claim}
\newtheorem{definition}[theorem]{Definition}
\newtheorem{proposition}[theorem]{Proposition}
\newtheorem{observation}[theorem]{Observation}
\newtheorem{remark}[theorem]{Remark}
\newtheorem{example}[theorem]{Example}
\newtheorem{question}[theorem]{Question}
% \numberwithin{equation}{section}
\renewenvironment{proof}{\noindent\bf{Proof.}\rm}{\hfill$\blacksquare$\bigskip}

\declaretheorem[name=Theorem,sibling=theorem]{rethm}
\declaretheorem[name=Corollary,sibling=theorem]{recor}
\declaretheorem[name=Proposition,sibling=theorem]{reprop}


\newcommand{\lexmax}{{\emph{lex-max-WS}}}
\newcommand{\items}{\mathcal{M}} 
\newcommand{\agents}{\mathcal{N}} 

\usepackage[]{color-edits}% suppress
\addauthor{UF}{red}
\newcommand{\ufc}[1]{{\UFcomment{#1}}}
\newcommand{\ufe}[1]{{\UFedit{#1}}}

\addauthor{GB}{blue}
\newcommand{\gbc}[1]{{\GBcomment{#1}}}
\newcommand{\gbe}[1]{{\GBedit{#1}}}


\title{On fair allocation of indivisible goods to submodular agents}

\author{Gilad Ben Uziahu\thanks{Weizmann Institute ---  E-mail: \texttt{gilad.ben-uziahu@weizmann.ac.il}.}\;\, and Uriel Feige\thanks{Weizmann Institute ---  E-mail: \texttt{uriel.feige@weizmann.ac.il}.}}

\begin{document}

\maketitle

\section{Introduction}
\label{sec:introduction}
% \begin{itemize}
%     % Diffusion of FL
%     \item {\st{Diffusion of FL}}
%     % Security threats to FL
%     \item {\st{Security threats to FL with particular focus on model poisoning}}
%     % Limitations of existing countermeasures
%     \item {\st{Current countermeasures (e.g., KRUM) and their limitations}}
%     % Proposed method and its advantages
%     \item {\st{Intuitive description of the proposed method and its difference (i.e., advantages) w.r.t. state of the art}}
%     % Main contributions
%     \item {\st{Summary of the main contributions of this work}}
%     % Paper's structure and organization
%     \item {\st{Paper's structure and organization}}
% \end{itemize}

% Diffusion of FL
Recently, {\em federated learning} (FL) has emerged as the leading paradigm for training distributed, large-scale, and privacy-preserving machine learning (ML) systems~\cite{mcmahan2017googleai,mcmahan2017aistats}. 
The core idea of FL is to allow multiple edge clients to collaboratively train a shared, global model without disclosing their local private training data.
%Specifically, an FL system consists of a central server and many edge clients; 
A typical FL round involves the following steps: {\em(i)} the server randomly picks some clients and sends them the current, global model; {\em(ii)} each selected client locally trains its model with its own private data; then, it sends the resulting local model to the server;\footnote{Whenever we refer to global/local model, we mean global/local model {\em parameters}.} {\em(iii)} the server updates the global model by computing an \emph{aggregation function}, usually the average (FedAvg), on the local models received from clients.
% \begin{enumerate}
%     \item[{\em(i)}] the server sends the current, global model to the clients and appoints some of them for training;
%     \item[{\em(ii)}] each selected client locally trains its copy of the global model with its own private data; then, it sends the resulting local model back to the server;\footnote{Whenever we refer to global/local model, we mean global/local model {\em parameters}.}
%     \item[{\em(iii)}] the server updates the global model by computing an \emph{aggregation function} on the local models received from clients (by default, the average, also referred to as FedAvg~\cite{mcmahan2017aistats}).
% \end{enumerate}
This process goes on until the global model converges. %(e.g., after a certain number of rounds or other similar stopping criteria).
%\\
% The advantages of FL over the traditional, centralized learning paradigm are undoubtedly clear in terms of flexibility/scalability (clients can join/disconnect from the FL network dynamically), network communications (only model weights\footnote{We will use \textit{parameters} and \textit{weights} interchangeably.} are exchanged between clients and server), and privacy (each client's private training data is kept local at the client's end and not uploaded to the server).
\\
% Security threats to FL
%However, the growing adoption of FL also raises security concerns~\cite{costa2022covert}, particularly about its confidentiality, integrity, and availability.
Although its advantages over standard ML, FL also raises security concerns~\cite{costa2022covert}. %, particularly about its confidentiality, integrity, and availability~\cite{costa2022covert}.
% OLD, LONG VERSION
% Indeed, some work deals with privacy leakage that may expose the local data of some clients~\cite{melis2019sp}. 
% A large body of work, instead, investigates attacks that usually aim to detriment the predictive accuracy of the learned global model. For instance, \emph{data poisoning} attacks achieve this goal by letting an adversary pollute the training set of some corrupt FL clients with maliciously crafted examples~\cite{jagielski2018sp}.
% Similarly, in \emph{model poisoning} the attacker attempts to tweak the global model weights~\cite{bhagoji2019pmlr} by directly perturbing the local model's weights of some infected FL clients before these are sent to the central server for aggregation, usually via so-called Byzantine attacks. 
% It turns out that Byzantine model poisoning attacks severely impact standard FedAvg; therefore, more robust aggregation functions must be designed to make FL systems secure.
Here, we focus on \emph{untargeted model poisoning} attacks~\cite{bhagoji2019pmlr}, where an adversary attempts to tweak the global model weights %\footnote{We will use the terms \textit{parameters} and \textit{weights} interchangeably.} 
by directly perturbing the local model's parameters of some infected clients before these are sent to the central server for aggregation.
In doing so, the adversary aims to jeopardize the global model \textit{indiscriminately} at inference time.
Such model poisoning attacks severely impact standard FedAvg; therefore, more robust aggregation functions must be designed to secure FL systems.
\\
% In this paper, we focus on designing a novel robust aggregation scheme at the server's end to contrast the effect of Byzantine model poisoning attacks.
%
% Current countermeasures and their limitations
%Several countermeasures have been proposed in the literature to combat model poisoning attacks on FL systems.
% Some methods use simple statistics more robust than plain average to smooth the impact of malicious updates (e.g., Trimmed Mean and FedMedian~\cite{yin2018icml}). 
% Other defenses implement outlier detection techniques to discard malicious updates from the aggregation performed at the server's end. Those are either based on heuristics (e.g., Krum/Multi-Krum~\cite{blanchard2017nips} and Bulyan~\cite{mhamdi2018pmlr}) or data-driven approaches (e.g., K-means clustering~\cite{shen2016acm} or DnC via spectral analysis~\cite{shejwalkar2021ndss}). 
% Finally, some strategies rely on a centralized ``source of trust'' to spot potential malicious updates (e.g., FLTrust~\cite{cao2020fltrust}).
% Several countermeasures have been proposed in the literature to combat model poisoning attacks on FL systems, i.e., to discard possible malicious local updates from the aggregation performed at the server's end. 
% These techniques range from simple statistics more robust than plain average (e.g., Trimmed Mean and FedMedian~\cite{yin2018icml}) to outlier detection heuristics (e.g., Krum/Multi-Krum~\cite{blanchard2017nips} and Bulyan~\cite{mhamdi2018pmlr}) or data-driven approaches (e.g., spectral analysis via K-means clustering~\cite{shen2016acm} or spectral analysis), or methods based on ``source of trust'' (e.g., FLTrust~\cite{cao2020fltrust}).
% OLD, LONG VERSION
%Several countermeasures have been proposed in the literature to combat Byzantine model poisoning attacks on FL systems.
% Descriptive statistics
% For example, Trimmed Mean and FedMedian aggregate local model updates using more robust statistics than standard average~\cite{yin2018icml}.
%
% % Heuristics for outlier detection
% Many existing Byzantine-resilient strategies implement some outlier detection heuristics to discard the model updates sent by potentially malicious clients from the input of the aggregation function.
% One of the most popular heuristics is Krum~\cite{blanchard2017nips}.
% This strategy tries to mitigate the impact of Byzantine attacks by selecting as a global model the local model with the smallest sum of Euclidean distances to {\em all} the other local models.
% Although powerful, Krum requires the server to know (or, at least, estimate) the number of malicious FL clients upfront, which is generally impossible in a realistic attack scenario. %
% Moreover, Krum may become ineffective for complex, high-dimensional model parameter spaces due to the curse of dimensionality.
% Bulyan~\cite{mhamdi2018pmlr} tries to overcome this issue by combining Krum with a variant of Trimmed Mean.
% % Data-driven outlier detection
% Other strategies use data-driven outlier detection techniques -- e.g., via K-means clustering~\cite{shen2016acm} -- to spot potential malicious local model updates. 
% %For instance, Shen et al. propose to cluster local model updates with K-means and thus identify outliers.
%
% % Other techniques
% As far as the server is concerned, any local model received can be from a potential malicious client. 
% FLTrust~\cite{cao2020fltrust} assumes the server acts as a client, i.e., trains a local model on an additional {\em trustworthy} dataset at the server's end and compares it against all the local models from other clients. 
% This way, the server can rely on some ``source of trust'' when discarding potentially malicious clients.
%\\
% Limitations of existing Byzantine-resilient strategies
Unfortunately, existing defense mechanisms either rely on simple heuristics (e.g., Trimmed Mean and FedMedian by~\cite{yin2018icml}) or need strong and unrealistic assumptions to work effectively (e.g., foreknowledge or estimation of the number of malicious clients in the FL system, as for Krum/Multi-Krum~\cite{blanchard2017nips} and Bulyan~\cite{mhamdi2018pmlr}, which, however, cannot exceed a fixed threshold).
Furthermore, outlier detection methods using K-means clustering~\cite{shen2016acm} or spectral analysis like DnC~\cite{shejwalkar2021ndss} do not directly consider the temporal evolution of local model updates received.
Finally, strategies like FLTrust~\cite{cao2020fltrust} require the server to collect its own dataset and act as a proper client, thereby altering the standard FL protocol.
\\
% OLD, LONG VERSION
% Overall, existing Byzantine-resilient strategies are either simple heuristics (e.g., FedMedian) or, if they are more complex, they rely on strong and unrealistic assumptions to work effectively (e.g., knowing the number of malicious clients in the FL system in advance, as for Krum and alike).
% Furthermore, data-driven outlier detection methods do not consider the temporary evolution of local model updates received (e.g., K-means clustering). 
% Finally, strategies like FLTrust requires the server to collect its own dataset and act as a proper client, thereby altering the standard FL protocol.
%
% Description of the proposed method
This work introduces a novel pre-aggregation \textit{filter} robust to untargeted model poisoning attacks. Notably, this filter $(i)$ operates without requiring prior knowledge or constraints on the number of malicious clients and $(ii)$ inherently integrates temporal dependencies. 
The FL server can employ this filter as a preprocessing step before applying \textit{any} aggregation function, be it standard like FedAvg or robust like Krum or Bulyan.
Specifically, we formulate the problem of identifying corrupted updates as a multidimensional (i.e., matrix-valued) time series anomaly detection task. 
The key idea is that legitimate local updates, resulting from well-calibrated iterative procedures like stochastic gradient descent (SGD) with an appropriate learning rate, show \textit{higher predictability} compared to malicious updates. This hypothesis stems from the fact that the sequence of gradients (thus, model parameters) observed during legitimate training exhibit regular patterns, as validated in Section~\ref{subsec:intuition}. %until convergence. 
%This regularity may be more pronounced for smooth convex loss functions, but it can still be captured within an appropriate time window, even for more complex and convoluted loss surfaces. 
%We provide evidence of this claim in Appendix~B, where we show that the average mutual information (i.e., ``predictability''), calculated over pairs of legitimate model updates sent at different FL rounds, is significantly higher than the corresponding computation for a malicious client.
\\
Inspired by the matrix autoregressive (MAR) framework for multidimensional time series forecasting~\cite{chen2021je}, we propose the FLANDERS ({\em \textbf{F}ederated \textbf{L}earning meets \textbf{AN}omaly \textbf{DE}tection for a \textbf{R}obust and \textbf{S}ecure}) filter.
The main advantages of FLANDERS over existing strategies like FLDetector~\cite{zhao2020multivariate} are its resilience to large-scale attacks, where $50\%$ or more FL participants are hostile, and the capability of working under realistic non-iid scenarios.
We attribute such a capability to two key factors: $(i)$ FLANDERS works without knowing a priori the ratio of corrupted clients, and $(ii)$ it embodies temporal dependencies between intra- and inter-client updates, quickly recognizing local model drifts caused by evil players. Below, we summarize our main contributions:

\begin{itemize}
\item[{\em(i)}]
We provide empirical evidence that the sequence of models sent by legitimate clients is more predictable than those of malicious participants performing untargeted model poisoning attacks.
\\
\item[{\em(ii)}] 
We introduce FLANDERS, the first pre-aggregation filter for FL robust to untargeted model poisoning based on multidimensional time series anomaly detection.
\\
\item[{\em(iii)}] 
We integrate FLANDERS into Flower,\footnote{\scriptsize{\url{https://flower.dev/}}} a popular FL simulation framework for reproducibility.
\\
\item[{\em(iv)}] 
We show that FLANDERS improves the robustness of the existing aggregation methods under multiple settings: different datasets, client's data distribution (non-iid), models, and attack scenarios.
\\
\item[{\em(v)}] 
We publicly release all the implementation code of FLANDERS along with our experiments.\footnote{\scriptsize{\url{https://anonymous.4open.science/r/flanders_exp-7EEB}}}
\end{itemize}

% Paper's structure and organization
The remainder of the paper is structured as follows. %some related work and the current state-of-the-art solutions to security issues that FL entails. 
Section~\ref{sec:background} covers background and preliminaries. 
In Section~\ref{sec:related}, we discuss related work.
Section~\ref{sec:problem} and Section~\ref{sec:method} describe the problem formulation and the method proposed. % to tackle it. 
Section~\ref{sec:experiments} gathers experimental results. %, and Section~\ref{sec:limitations} discusses some limitations of this work.
Finally, we conclude in Section~\ref{sec:conclusion}.
 %discusses the limitations of this work and draws future research directions.
%reports conclusions and draws perspectives for future research directions.

%%%%%%% OLD %%%%%%%
%to overcome the resilience of Byzantine failures in distributed Stochastic Gradient Descent computations. 
% The strength of Krum is its time complexity, which is linear in the gradient dimension. 
% However, the robustness of the approach is guaranteed for gradient-based learning applications only when the majority of the clients are not compromised. 
% Besides, the aggregation mechanism of Krum, as well as that of similar methods, is robust from a coarse-grained perspective and does not provide solutions to errors and perturbations that may occur at inference time.
%A related approach to~\cite{blanchard2017nips} is the work of Su et al.~\cite{su2016dc}. Here, the authors propose an iterated approximate agreement to tackle a multi-layer scenario attacked by Byzantine agents. 
%However, the method works efficiently on the sole discrete context and it is inapplicable to continuous state environments.
%\gabri{Maybe, we should just talk about the main limitations of existing countermeasures without digging into their details (or, we can just mention Krum as this is the most popular one). I will move the description of all these methods to the Related Work section.}

\section{Proofs of our results}

%\ufc{Short introduction to this section (optional).}

\section{Notation and Preliminaries}\label{sec_prel}
Let $\mathbb{Z}_{>0}$ denote the set of positive integers and let $\mathbb{Z}_{[a,b]}$ denote the set of integers in the interval $[a,b]$. The $m\times m$ identity matrix is denoted by $I_m$ and its columns by $e_i$ for $i\in\mathbb{Z}_{[1,m]}$. We use $\mathbf{0}$ to denote a vector or a matrix of zeros of appropriate dimensions. For a sequence $\{z_k\}_{k=0}^{N-1}$ with $z_k\in\mathbb{R}^\eta$, we denote its stacked vector as $z = \begin{bmatrix}z_0^\top &z_1^\top & \dots & z_{N-1}^\top\end{bmatrix}^\top$ and a stacked window of it as $z_{[l,j]} = \begin{bmatrix}z_l^\top &z_{l+1}^\top & \dots & z_{j}^\top\end{bmatrix}^\top$ with $0\leq l<j$.\par
Persistence of excitation of a sequence and its extension to multiple sequences \cite{vanWaarde20} are defined as follows.
\begin{definition} The sequence \(\{z_k\}_{k=0}^{N-1}\), $z_k\in\mathbb{R}^{\eta}$, is said to be persistently exciting of order \(L\) if \(\textup{rank}(\mathscr{H}_{L}(z))=\eta L\), where $\mathscr{H}_L(z) = \begin{bmatrix}
		z_{[0,L-1]} & z_{[1,L]} & \cdots & z_{[N-L,N-1]}
	\end{bmatrix}$.
	\label{def_PE}
\end{definition}
\begin{definition}[\cite{vanWaarde20}]\label{def_cPE}
	The sequences $\{z_k^{(j)}\}_{k=0}^{N_j-1}$, with $z_k^{(j)}\in\mathbb{R}^\eta$ and $j\in\mathbb{Z}_{[1,r]}$, are said to be \textit{collectively persistently exciting} of order $L$ if rank$(\mathcal{H}_L(\mathscr{Z}))=\eta L$, where $\mathscr{Z} = \begin{bmatrix}
		(z^{(1)})^\top & \cdots & (z^{(r)})^\top
	\end{bmatrix}^\top,$ and
	\begin{equation*}
		\mathcal{H}_L(\mathscr{Z}) = \begin{bmatrix}
			\mathscr{H}_L(z^{(1)}) & \cdots & \mathscr{H}_L(z^{(r)})
		\end{bmatrix}.
	\end{equation*}
\end{definition}




\subsection{{Approximate} APS-fair allocations for submodular agents}
\label{sec:APS}


%\gbc{Removed:
%\begin{theorem}
%\label{thm:APSsubmodular}
%There is an allocation algorithm, for indivisible goods to agents with arbitrary entitlements and submodular valuations that gives every agent at least a $\rho$ fraction of her APS. The algorithm runs in polynomial time if 
%the valuation functions are accessed using value queries, and furthermore, the APS value of every agent is given. 
%\end{theorem}}

%\ufc{As secondary goal, we want to show that safe strategies in natural games give such a result}

%\gbc{Indeed the assumption of APS equals entitlement is not nedded}
%\gbc{Remove? We assume \ufc{the assumption is not needed at this stage.} that valuation functions are scaled so that for every agent $i$, her APS is equal to her entitlement $b_i$, and truncated so that no single item has value more than $b_i$.  \ufc{For submodular, truncated so that no bundle has value larger than $APS_i$.} (Note that we do not assume that the instance is ordered, as the transformation of Bouvert and Lamaitre assumes additive valuations.)}

We now describe an allocation game (introduced in~\cite{BEF21}), that we refer to as the bidding game.
Initially, every agent $i$ is {\em active}, is given a budget of $b^0_i = b_i$ (in particular, in the equal entitlement case $b^0_i = \frac{1}{n}$), and has an empty bundle $S^0_i$ of items. The set of initially unallocated items is denoted by $M^0$. 

The game proceeds in rounds, and in every round, one item is allocated. In round $r \ge 1$, to decide which item is allocated, we do the following.

\begin{enumerate}

\item If there are no active agents, end the allocation algorithm. (The remaining items, if there are any, can be allocated arbitrarily.)

%\gbc{I think we should remove this second bullet, since it is not describing the algorithm\textbackslash bidding-game, but the proportional-bidding-strategy, which presented when proving \Cref{thm:1/3_APS_guarantee}} \ufc{The second bullet should just describe what a legal bid is, and not how the agent chooses the bid.}
    
    \item {Every active agent $i$ submits a nonnegative bid $p_i^r$ of her choice, not exceeding her budget. Namely, $0 \le p_i^r \le b_i^{r-1}$.}
    % let $e_i^r$ be the item in $M^{r-1}$ of highest marginal value given $S_i^{r-1}$, and let $v_i^r$ be its marginal value. The agent bids $p_i^r = \min[v_i^r, b_i^{r-1}]$. 

    \item  The agent $i$ with the highest bid (breaking ties arbitrarily) wins, {and selects an arbitrary item of her choice. Denote this selected item as $e^r$.}  We update $M^r = M^{r-1} \setminus \{e^r\}$ and $S_i^r = S_i^{r-1} \cup \{e^r\}$. Her budget is updated to $b^r_i = b^{r-1}_i - p_i^r$. If $b^r_i =0$, then agent $i$ stops being active. In any case, for agents $j \not= i$, we have $S_j^r = S_j^{r-1}$ and $b^r_j = b^{r-1}_j$.
    %\gbc{Notice that in the above third bullet, I updated the original\textbackslash first version of the bidding-game to be where each agent can exploit her entire budget. Only later in the article we present the altruistic version of the bidding game, in which the algorithm turns the agents to inactive when surpassing a threshold of their budget.}

\end{enumerate}


%\ufc{For COMSOC, present a one page sketch of the proof for $\frac{1}{3}$-MMS, explain that full proof needs to handle APS and arbitrary entitlement, and move next five and a half pages to the appendix}


To help illustrate the key methods used in the proof of \Cref{thm:1/3_APS_guarantee}, we first present a sketch of proof for a weaker version. This will pave the way for the subsequent proof of \Cref{thm:1/3_APS_guarantee}.

\begin{proposition}
\label{pro:third}
    Consider the bidding game described above and an agent $p$ with a submodular valuation function in an equal entitlements setting. Employing a bidding-strategy referred to as $proportional(\frac{1}{3})$ guarantees agent $p$ a minimum value of $\frac{1}{3}\cdot MMS_p$.
\end{proposition}

\begin{proof}
In the bidding game, each agent is initially assigned a budget equal to her entitlement. In the equal entitlement case, each agent receives a budget of $\frac{1}{n}$. The $proportional(\frac{1}{3})$ bidding strategy involves the following steps. Initially, agent $p$ calculates $MMS_p$ (her $MMS$ value).  At the beginning of round $r$, let $\items^r$ denote the set of items that are still available, let $C^r$ denote the set of items that agent $p$ won prior to round $r$, and let $b_p^r$ denote the budget that the agent still holds. In round $r$, agent $p$ bids $\frac{3}{2}\cdot\frac{1}{n}\cdot\frac{1}{MMS_p}\cdot\max_{e\in\items^r}[v_p(e\mid C^r)]$. In other words, the bid of the agent is equal to $\frac{3}{2}$ times the {\em scaled value} of the marginal value of the item of highest marginal value that still remains, where the scaling factor $\frac{1}{n}\cdot\frac{1}{MMS_p}$ is such that after this scaling, the MMS value of $p$ equals the original budget of $p$.
If this bid value exceeds $b_p^r$ (the remaining budget of $p$), then $p$ bids $b_p^r$. In any case, if $p$ wins the bid, she selects the item of highest marginal value in $\items^r$, and pays her bid. We note here that the factor $\frac{3}{2}$ was chosen so as to equal $\frac{1}{2\rho}$, for our choice of $\rho = \frac{1}{3}$. The same type of expression, $\frac{1}{2\rho}$, will appear also in the proof of Theorem~\ref{thm:1/3_APS_guarantee}.

{We now provide a sketch of proof that the above bidding strategy guarantees agent $p$ a bundle of value at least $\frac{1}{3} MMS_p$. In this sketch, $\{B_i\}_{i=1}^n$ denotes an MMS partition for agent $p$. Namely, $v_p(B_i) \ge MMS_p$ for every $i$.}

{Call an item $e$ {\em large} if $v_p(e)>\frac{2}{3}MMS_p$. We claim that, without loss of generality, we can assume that there are no large items. As long as a large item exists, $p$ bids her entire budget. If agent $p$ wins the round, {we are done}. If a different agent $q$ wins the round, that agent spends her entire budget and leaves the bidding game after winning only a single item. 
%Thus, we have disposed of an item $e$ and an agent $q$. 
%Removing a single item and an agent does not decrease the MMS value of agent $p$, since $n-1$ bundles of the MMS partition of $p$ are still available. Therefore, 
Intuitively, $q$ did not ``hurt" $p$, since $n-1$ bundles of the MMS partition of $p$ still contain all their items, whereas only $n-1$ agents remain to compete on them. Further details of this argument are omitted.}



According to the bidding strategy, if agent $p$ manages to spend at least half of her budget during the bidding game, she will receive a $\frac{1}{3}$-fraction of her MMS. Therefore, our goal is to show that $p$ manages to spend at least half of her budget.
% Therefore we would like to show that $p$ manages to spend at least half of her budget.

The main idea is that until $p$ spent half of her budget, other agents cannot do much damage to $p$. The bidding strategy of $p$ has two key properties that are easy to verify. First, the bidding sequence is non-increasing (this is a consequence of submodularity of $v_p$). Second, {in the absence of large items (an assumption that we can make without loss of generality)}, as long as $p$ has spent at most half of her budget, her remaining budget does not constrain her from providing a full bid according to the bidding strategy. Due to this latter property, if another agent $q$ wins an item, {$q$ pays for the item that she takes {at least $\frac{3}{2}$ times} the scaled marginal value that $p$ has for the item.}

Next, we analyze how much harm the other agents {can cause $p$ up to the point when she spends} half of her budget. We denote by $C$ the bundle that $p$ holds {at the last point in time in which she has not spent half her budget.} 
A sufficient condition for $p$ to exceed $\frac{1}{3}MMS_p$ is if there exists a bundle $B_i$ from $p$'s MMS partition that has sufficiently high marginal value (relative to $C$, and after excluding from $B_i$ those items won by other agents) so that together with $C$, the value exceeds $\frac{1}{3}MMS_p$. 
For every item $e$ that another agent wins, that agent pays at least $\frac{3}{2}\frac{1}{n}\frac{1}{MMS_p}v_p(e\mid C)$. {(The fact that we can compare to marginal value relative to $C$ is a consequence of submodularity of $v_p$.)} Hence, the ratio between the value taken from bundles of the MMS partition, and the payment done by the other agents is $\alpha=\frac{3}{2}\frac{1}{n}\frac{1}{MMS_p}$.
Therefore, using the fact that the total budget of all the agents together is 1, we obtain an upper bound on the total value taken by the other agents, which is $\frac{2}{3}\cdot n\cdot MMS_p$. Hence, there must exist a bundle $B_i$ in which the other agents took items of marginal value relative to $C$ of at most $\frac{2}{3}MMS_p$. Hence, together with $C$, there is enough value left in $B_i$ for $p$ to surpass $\frac{1}{3}MMS_p$. (This last argument again uses submodularity of $v_p$.)
\end{proof}

Before presenting the proof of \Cref{thm:1/3_APS_guarantee}, we discuss some of the challenges we will face when extending the (sketch of) proof of Proposition~\ref{pro:third} to the more general setting of \Cref{thm:1/3_APS_guarantee}.

\begin{itemize}
    \item \Cref{pro:third} considers the MMS, whereas \Cref{thm:1/3_APS_guarantee} considers the APS, which is always at least as large as the MMS, and sometimes larger. The analysis needs to be extended to hold relative to this stronger notion, and in particular, can no longer assume the existence of an integral MMS partition.
    \item The setting considered in \Cref{thm:1/3_APS_guarantee} allows for arbitrary entitlements. The proof of \Cref{pro:third} uses the assumption that the setting is that of equal entitlement (for example, in its treatment of large items). 
    \item \Cref{thm:1/3_APS_guarantee} provides a guarantee that is somewhat better than $\frac{1}{3}$-fraction of the APS (that becomes significant if the entitlement is large).
\end{itemize}

We now proceed to present the proof of \Cref{thm:1/3_APS_guarantee}.

%In order to address the third challenge, we will utilize a bidding strategy we refer to as $proportional(\rho)$, in the proof of \Cref{thm:1/3_APS_guarantee}. This strategy takes a parameter $\rho$ as an input, which determines the proportion between the bids $p$ will raise with respect to the values of the items.
% The bidding strategy is simple: in round $r$, agent $p$ should bid $\max_{e\in\items^r}\frac{1}{2\rho}\frac{b_p}{APS_p}v_p(e|C^r)$.If this bid value exceeds her remaining budget, she bids her remaining budget.
%However, we present the strategy in a more complicated way to aid in the analysis of the proof of \Cref{thm:1/3_APS_guarantee}.


%\gbc{Todo - change the strategy to proportional-1 and proportional-2. Proportional-1 is conducted when there exists large items (satisfying $v_p(e)\geq2\rho APS$ in this case the agent bids her entire budget. If she wins we are good - if not, the agent conducts the proportional strategy on the reduced instance $I_s$ (or $\hat{I_s}$). When for the first time there are no more large items - then the agent bids $\frac{1}{2\rho}\frac{b_p}{APS}v_p(e)$ (in this case, this value does not exceed   her remained budget, unless she already won a value of $\rho APS_p$)} 

%\ufc{Our strategy should perhaps be called $2\rho$-proportional and not $\rho$-proportional, so that 1-proportional is proportional. Maybe keep $proportional(\rho)$, as something different from $\rho$-proportional.}
{We describe a bidding strategy for the bidding game in the arbitrary entitlement case. It has a parameter $\rho > 0$,
%satisfying $0 < \rho \le \frac{1}{2}$ \gbc{Consider removing this restriction of $\rho$, and change it to $\rho\leq\frac{1}{3-2b_p}$, As explained in my comment before \Cref{lem:1/3 guarantee no large items}}, 
and we refer to it as the proportional bidding strategy $proportional(\rho)$.}  
As we shall later prove, if $\rho$ is chosen to satisfy $\rho \le \frac{1}{3-2b_p}$, then the $proportional(\rho)$ bidding strategy will guarantee that agent $p$ receives a bundle of value at least $\rho APS_p$.
A player $p$ (with valuation function $v_p$ and entitlement $b_p$) 
%where $b_p = \frac{1}{n}$ in the equal entitlement case)  
that uses $proportional(\rho)$ first computes her $APS(\items, v_p, b_p)$  value, which we refer to as $APS_p$. 
Up to some minor technical details (that manifest themselves only if other agents spend their budgets at a rate that is higher than that dictated by the bids of $p$ -- these details do not affect the guarantees offered by the bidding strategy), $proportional(\rho)$ is equivalent to the following simple strategy. Scale the valuation function of $p$ such that her $APS$ equals her entitlement (and budget) $b_p$. In each round, bid $\frac{1}{2\rho}$ times the marginal value (with respect to the items that $p$ already holds) of the item of highest marginal value that is not yet allocated, if $p$ has sufficient budget to do so, and bid the total remaining budget otherwise. If $p$ wins the bid, she selects the item of highest marginal value.
%\gbc{For COMSOC from this point shift the rest of the section to the Appendix}
%\gbe{The more detailed presentation of the $proportional(\rho)$, and the proof of \Cref{thm:1/3_APS_guarantee}, is deferred to the appendix.}\gbc{Give an explicit reference to the subsection in the appendix?}

{To simplify the proofs that follow later, we now present the $proportional(\rho)$ bidding strategy in more detail, and introduce terminology that will be used in the proofs.}
%The partition of $proportional(\rho)$ into $proportional_1(\rho)$ and $proportional_2(\rho)$ is made only for convenience in the analysis.
%The strategy $proportional(\rho)$ is the combination of $proportional_1(\rho)$ and $proportional_2(\rho)$. 
%Before presenting our strategy in its full generality, 
We first present $proportional(\rho)$ in the special case in which $v_p(e)\leq 2\rho APS_p$ holds for all items. We refer to the strategy in this special case as $proportional_1(\rho)$. {At the beginning of round $r$, let $\items^r$ denote the set of items not yet allocated, let $C^r$ denote the set of items already allocated to $p$, and let $b_p^{r-1}$ denote the budget remaining for agent $p$. Then the agent bids $\frac{1}{2\rho}\cdot\frac{b_p}{APS_p}\cdot\max_{e\in\items^r}[v_p(e\mid C^r)]$ (the highest marginal value that a yet unallocated item has, scaled by $\frac{1}{2\rho}\frac{b_p}{APS_p}$) if this bid is not larger than $b_p^{r-1}$, and
bids her remaining budget $b_p^{r-1}$ otherwise. If the agent
wins her bid, she selects the item with the highest marginal value with respect to $C^r$.}

We now present the strategy for the remaining case, that in which there are large items that satisfy $v_p(e)>2\rho APS_p$. We refer to the strategy in this case as $proportional_2(\rho)$.

Prior to the beginning of the bidding game, agent $p$ truncates her valuation function, so that the value of each bundle $S$ is $\min[v_p(S), APS_p]$. In the notation of \Cref{def:$v^t_p$}, this new valuation function is denoted as $v_p^t$, with $t=APS_p$. This does not affect $APS_p$, {and preserves submodularity}. To keep notation simple, we still use the notation $v_p$ for this new valuation function.

In the bidding game itself, as long as there exists a large item that satisfies $v_p(e) > 2\rho APS_p$, $p$ bids her entire budget. If the agent
wins her bid, she selects the item with the highest value (which is at least $2\rho APS_p$) and leaves the game (as her budget is exhausted). If the agent does not win any of the large items, let $s$ denote the last round in which there are large items, i.e., from round $s+1$ onward, all unallocated items satisfy $v_p(e) \leq 2\rho APS_p$. At this point, agent $p$ basically switches to using $proportional_1(\rho)$. Below we introduce terminology that describes how this switch is done. This terminology will later be used in our proofs.
%\ufc{The natural thing now is to simply switch to $proportional_1(\rho)$. The paragraph and a half that follows here is part of the analysis, not part of the bidding strategy.}
%\gbc{In a few sentences it is written that $p$ simulates $proportional_1$ on $\hat{I_s}$. This statement makes sense only if $p$ did not win a large item, i.e., she is an active agent in $\hat{I_s}$} \ufc{We already said that in the previous sentence.} \ufc{Remove, and note that you keep on making the mistake of confusing $s$ with $s+1$. If $p$ did not win an item in the first $s+1$ rounds, then she continue as follows.} 

Let $I_s$ denote the {\em residual instance}  that remains after round $s$. It includes the set of items that were not yet allocated (which we denote by $\mathcal{\hat{M}}$), 
% \ufc{removed because $p$ took no items: the valuation function (over the remaining items) of {agent $p$ is} her marginal valuation function with respect to the items that she already has,}
and each agent has whatever remains from her budget after the $s$ rounds.
View $I_{s}$ as a new allocation instance, that we refer to as $\hat{I_{s}}$. The agents of $\hat{I_{s}}$ are the remaining active agents of $I_{s}$. The set of items of $\hat{I_{s}}$ is $\mathcal{\hat{M}}$ (those items remaining
in $I_{s}$). Setting $\gamma=b^s$ to be the total remaining budget of agents, we update the entitlement of each agent to be $\hat{b_i}=\frac{1}{\gamma} b_i$. Agent $p$ simulates $proportional_1(\rho)$ on $\hat{I_s}$. To do so, for every agent $i \not=p$, a bid $x_{i}^{r}$ in $I_{s}$ is viewed as a bid of $\frac{1}{\gamma}\hat{x}_{i}^{r-s}$ in 
    $\hat{I_{s}}$, and any intended bid $x^r$ of agent $p$ in $\hat{I_{s}}$ is translated to a bid $\gamma x^{r+s}$ in $I_s$. 
% \ufc{remove: Namely other agent bids $p_{i}^{r}$ in $I_{s}$ then $p$ simulate her bid in $\hat{I_{s}}$ as $\hat{p}_{i}^{r-s}=\frac{1}{\gamma}p_{i}^{r}$.
% Then, agent $p$ knows how to bid by considering scaling her bids in $\hat{I_{s}}$ by $\gamma$.}
(Since the ratio between a budget of an agent in $I_{s}$ and in $\hat{I_{s}}$ is $\frac{1}{\gamma}$, by taking a bid of another agent $i$ in $I_{s}$ and scaling it by $\frac{1}{\gamma}$ we get a legal bid of the agent in $\hat{I_{s}}$. Similarly, scaling by $\gamma$ a bid of agent $p$ in $\hat{I_{s}}$ induces a legal bid in $I_{s}$.)

Observe that by \Cref{claim:APS not decrease in I_s}, the APS of $p$ in $\hat{I_s}$ remains the same as the original value of $APS_p$.

   
    
    %The strategy $proportional(\rho)$ is the combination of $proportional_1(\rho)$ and $proportional_2(\rho)$. Up to some minor technical details (that manifest themselves only if other agents spend their budgets at a rate that is higher than that dictated by the bids of $p$ -- these details do not affect the guarantees offered by the proportional bidding strategy), it is equivalent to the following simple strategy. Scale the valuation function of $p$ such that her $APS$ equals her entitlement (and budget) $b_p$. In each round, bid $\frac{1}{2\rho}$ times the marginal value (with respect to the items that $p$ already holds) of the item of highest marginal value that is not yet allocated, if $p$ has sufficient budget to do so, and bid the total remaining budget otherwise. If $p$ wins the bid, she selects the item of highest marginal value. The partition of $proportional(\rho)$ into $proportional_1(\rho)$ and $proportional_2(\rho)$ is made only for convenience in the analysis.
    
    %\item In the case agent $p$ did not win an item by round $s$, she bids her entire budget in each of these rounds. Thus, any other agent wins an item and spends at least $b_p$ of her budget, and by round $s$, the total budget spent by other agents is at least $s\cdot b_p$. If we consider the worst case \gbc{I don't want to get into a formal explanation why this is the worst case} the budget spent by other agents is exactly $s\cdot b_p$, then the bids of $p$ in the remaining $I_s$ in round $r$ will be exactly $\frac{1}{2\rho}\frac{b_p}{APS_p}\max_{e\in \items^r} [v_p(e\mid C^r)]$ or the $b_p^r$ if the bid exceeds the remained budget. Hence one can verify that by simply bidding in every round $\frac{1}{2\rho}\frac{b_p}{APS_p}\max_{e\in \items^r} [v_p(e\mid C^r)]$ (or $b_p^r$ if the bid exceeds the remained budget), the same guarantees will hold as of the proportional bidding strategy presented above. \gbc{What I am trying to say by that is that the algorithm is basically simple to apply - the proportional bidding strategy is basically simple - in round $r$ bid $\frac{1}{2\rho}\frac{b_p}{APS_p}\max_{e\in \items^r} [v_p(e\mid C^r)]$ or $b_p^r$ if this exceeds the remaining budget }




%\gbc{Removed: The first stage of the bidding strategy is used as long as there are items with $v_p$ value greater than $2\rho \cdot APS_p$. {In the first stage, in every round $r$} agent $p$ bids her entire budget, and picks the item with the highest value if she wins. {If agent $p$ is still active} after no more items with a value greater than $2\rho \cdot APS_p$ remain, {then she moves to the second stage of the bidding strategy}. \ufc{We over complicated things. If we originally truncate the valuation function at APS, then at this stage $APS' = APS$, because the APS cannot increase. This simplifies the presentation.} She computes her new {$APS(\items', v_p, b'_p)$, that were refer to as $APS'$. Here}  $\items'$ is the {set of remaining} items, and {$b'_p=\frac{1}{n-s}$, where $s$ is the number of inactive agents.}  \ufc{Explain how to computed $b'_p$ in the arbitrary entitlement case.  $b'_p = \frac{1}{b'}b_p$, where $b'$ is the sum of entitlements that still remain, for active agents.}
%\ufc{removed: $b'=\frac{n}{n-s}\cdot b_p$, where $s$ is the number of inactive agents (in the equal entitlement case $b_p=\frac{1}{n}$ where $n$ is the number of agents).}\gbc{I think we should keep $b_p'=\frac{n}{n-s}\cdot b_p$ as I originally suggested, since it takes into consider both cases of equal and unequal entitlements} Then the agent scales her valuation function {$v_p$ to $v'_p$ so that $APS(\items', v'_p, b'_p) = b'_p$.}
%\ufc{removed: by $\frac{b_p}{APS}$.} \gbc{I still think that the agent need to scale her valuation function by $\frac{b_p}{APS(\items ', b_p'})$ so after scaling her new APS is equal $b_p=\frac{1}{n}$. This implies that all items satisfy $v_p'(e)\leq2\rho b_p$ and hence and the agent bids $\frac{1}{2\rho}v_p'(e\mid C^r)$} 
%At the beginning of a round $r$, let $\items^r$ denote the set of items not yet allocated, let $C^r$ denote the set of items already allocated to $p$, and let $b_p^r$ \ufc{you previously used $b_p^{r-1}$. stick to one version} denote the budget remaining for agent $p$. Then the agent bids $\frac{1}{2\rho}\cdot\frac{b_p}{b'_p}\cdot\max_{e \in \items^r}[v_p(e \mid C^r)]$ (the highest marginal value that a yet unallocated item has, scaled by $\frac{1}{2\rho}$) if this bid is not larger than $b_p^r$, and bids her remaining budget $b_p^r$ otherwise. If the agent wins her bid, she selects the item with the highest marginal value with respect to $C^r$.}

% The reader should keep in mind that the parameter $\rho$ will represent the fraction that the agent will be guaranteed to have from her $APS$ (or $MMS$). 
%\gbc{Removed:
% \ufc{State proposition $v'_p(e) \le 2\rho\cdot APS(\items', v'_p, b'_p)$, and in the proof say the contents of items 1 and 2 below.}

% \ufc{replace the observations below by the above proposition}

% \gbe{
% We state here convenient observations:
% \begin{enumerate}
% \ufe{\item $APS' \ge APS$.} \ufc{Refer to a proof that appears later, or alternatively, prove here.} \ufe{Consequently, at the end of the first stage, every remaining item has $v_p$ value at most $2\rho APS'$.}
     
%    \item After the agent scales her valuation function, her new $APS_p=b_p$ (respectively $MMS_p=b_p$).
     
%     \item\ufe{By observations~1 and~2 above}, at the second phase of the bidding strategy, i.e., after the agent scales her valuation function, \ufe{every remaining item $e$ satisfies} \ufc{remove: the items respect} \ufe{$v'_p(e)\leq 2\rho b'_p$.} 
%     \ufc{The following is not needed if you refer the reader to a proof of Observation 1.}
%     \gbc{Wondering whether a more detailed explanation is needed, such as the following. Denote $APS^0$ the $APS$ of the $p$ at the beginning. And $APS^s$ the APS of agent $p$ after $s$ rounds when there are no more items with value $2\rho APS^0$. Then $APS^s\geq APS^0$. Then by the proportional bidding, in the second phase of the strategy, the agent scales her valuation function by $\frac{b_p}{APS^s}$, so her new APS equals her budget $b_p$. Therefore, if we know that after $s$ rounds, no more items with value $2\rho APS^0$ exists, then, this guarantee translates to the guarantee that all the items have a value less than $2\rho b_p$ after scaling the valuation function by $\frac{b_p}{APS^s}$.} Since $\frac{1}{2\rho}\cdot 2\rho b_p\leq b_p$, up until the agent wins her first item; all her bids are $\frac{1}{2\rho}$ times the value of the item with the highest value (and the agent does not encounter the situation where this value is greater than her budget).
% \end{enumerate}
%  }

%}

%At every round, to compute the bids of the agents, value queries suffice. \ufc{This confuses two issues. One is the complexity of computing bids according the proportional strategy, given that the APS value is known, Here we need to compute marginal values, and this can be done by value queries. The other is that we will later claim theorems about wat happens when all agents use the proportional strategy.}

%\ufc{Needs to be rephrased so as to present the point of view of $p$, as not all agents use the proportional strategy.} Every agent that stops being active gets value at least $\rho$ times her APS. Hence it remains to choose a value for $\rho$ that guarantees that the set of remaining items does not become empty while there still are active agents. 


%\ufc{Either change to Observation (and when referring to it, refer to it as an observation), or provide a proof, or explain why not proof is needed.}
%\gbc{If you are ok with it, I changed it to observation - There is no proof for it since it's immediate.}

\begin{observation}
\label{obsrv: bidding sequence decreasing}
The sequence of bids of an agent that uses the proportional bidding strategy is weakly decreasing.
\end{observation}

%\gbc{The following changes are made in order to focus on agent $p$ instead of any agent $i$}
Consider an APS fractional partition %\gbc{Removed:$\{\lambda_j B_i^j\}_{j \ge 1}$ for agent $i$} 
{$\{\lambda_S\}_{S\subseteq\mathcal{M}}$ for agent $p$} , where {$\sum_S \lambda_S = 1$, $\sum_{S \; | \; e\in S} \lambda_S \le b_p$} %\gbc{Removed:$\sum_j \lambda_j = 1$, $\sum_{j \; | \; e\in B_i^j} \lambda_j \le b_i$} 
for every item $e$, and {$v_p(S) \ge APS_p$ for every $S$ with $\lambda_S$ in the support}.
%\gbc{Removed:$v_i(B_i^j) \ge APS_i$ for every $j$} 
We trace three parameters throughout the execution of the algorithm. One is a {\em lower bound} on the total marginal value of the set of all remaining items to agent {$p$}, %\gbc{Removed:$i$}
given her set of items at the time. Initially it is {$L^0 = \sum_S \lambda_S v_p(S) \ge {APS_p}$}. %\gbc{Removed:$L_i^0 = \sum_j \lambda_j v_i(B_i^j) \ge {APS_i}$}. 
Another is the budget of agent $i$, which initially is $b_i^0 = b_i$.  Another is the total remaining budget of all agents. Initially it is $b^0 = 1$. 


%Recall .
%\gbc{Removed: \APSbidding* }
% \begin{lemma}
% \label{thm:1/3_APS_guarantee}
% For an agent $p$ with a submodular valuation function, setting $\rho={\frac{1}{3-2b_p}> \frac{1}{3}}$, the $proportional(\rho)$ bidding strategy guarantees agent $p$ a value of at least $\rho \cdot APS_p$. (In the case of equal entitlements, this gives $\rho=\frac{n}{3n-2}$.)
% \end{lemma}

%\gbc{Should we remove this presentation of $p$? - it appears in the presentation of $proportional(\rho)$. }
Consider an agent $p$ with valuation function $v_{p}$. 
We begin by proving \Cref{thm:1/3_APS_guarantee} under the simplifying assumption {that there are no large items i.e., every item satisfies $v_{p}(e) \le 2\rho APS_p$}. 
%\ufc{In the following lemma, does the agent underbid if $b_p > \frac{1}{2}$?)}
%\gbc{In the presentation of $proportional(\rho)$ we defined it for $0<\rho\leq\frac{1}{2}$. The next lemma states that we can plug in $\rho=\frac{1}{3-2b_p}$ and get at $\rho$ fraction guarantee. In this case $\rho>\frac{1}{2}$ iff $b_p>\frac{1}{2}$ which leads to underbidding as you suggest. Therefore I think we should change the restriction of $\rho\leq\frac{1}{2}$ in the presentation of the proportional strategy. Why is underbidding problematic? I agree it does not seem natural. I think this restriction came originally in order to claim that if there are no large items, i.e., $v_p(e)\leq 2\rho APS_p$, then the agent is able to raise a full bid. But, by truncating $v_p$ with $APS_p$ we obtain that even if we restrict $\rho\leq\frac{1}{3-2b_p}$ it suffices to show that $p$ is able to raise a full bid when no large items exist}

\begin{lemma}
\label{lem:1/3 guarantee no large items}
For an agent $p$ with a submodular valuation function, if $v_{p}(e) \le 2\rho {APS_p}$ for every item $e$, then by setting $\rho={\frac{1}{3-2b_p}> \frac{1}{3}}$, the $proportional(\rho)$ bidding strategy guarantees agent $p$ a value of at least $\rho \cdot APS_p$.
\end{lemma}

We shall present a sequence of claims that proves \Cref{lem:1/3 guarantee no large items}.

\begin{claim}
\label{claim:2alternatives}
In every round $t$ of the algorithm, at least one of
the following two conditions hold:
\begin{enumerate}
\item Agent $p$'s bid is equal to {$\frac{1}{2\rho}{\cdot\frac{b_p}{APS_p}}\cdot\max_{e \in \items^r}[v_p(e  \mid C^r)]$ (the highest marginal value that a yet unallocated item has, scaled by $\frac{1}{2\rho}{\frac{b_p}{APS_p}}$)} 
\item Agent $p$ already won a bundle with value at least $\rho{APS_p}$.
\end{enumerate}
\end{claim}

\begin{proof}
Suppose that the second condition does not hold. If $p$ did not
win an item yet, then she still has her entire budget, and using the
assumption that no item has a value greater than $2\rho {APS_p}$, condition~1 holds. Otherwise, agent $p$ won items with a total value less than $\rho{APS_p}$ by time $t$. Submodularity of $v_{p}$ implies that the sequence of remaining maximal marginal values  $\max_{e \in \items^r}[v_p(e \; \mid C^r)]$ is
non-increasing in $r$. %\gbc{or should I refer to \Cref{obsrv: bidding sequence decreasing}?}. 
Hence {$\max_{e \in \items^t}[v_p(e \; \mid C^t)] \le \rho APS_p$. Therefore } $\frac{1}{2\rho}{\frac{b_p}{APS_p}}\max_{e \in \items^t}[v_p(e \; \mid C^t)] \le{\frac{1}{2\rho}\rho b_p\leq\frac{1}{2}b_p}\le {b_p^t}$, and condition~1 holds.
\end{proof}

%\gbc{Removed: Denote the entitlement of agent $p$ by $b=\frac{1}{n}$. As mentioned above, we assume her APS is also equal to $b.$} 
Let $\{\lambda_{S}\}_{S\subseteq\mathcal{M}}$
be the set of weights associated with the APS {for $v_p$} (i.e., for every $S\subseteq\mathcal{M},$
$\lambda_{S}>0\implies v_p(S)\geq {APS_p}$, also $\sum_{S}\lambda_{S}=1$,
and $\forall e\in\mathcal{M},$ $\sum_{S|e\in S}\lambda_{S}\leq b_p$). 

Let $L^{0}\coloneqq\sum_{S}\lambda_{S}v(S)$. Observe that by the
definition of $APS$, $L^{0}\geq APS_{p}$. At the beginning of the
algorithm (when no item has been allocated yet), $L^{0}$ is a lower
bound on the marginal value {that agent $p$ has for} the set of all items. 

Let $f$ denote 
%\ufc{removed: the beginning of} \gbc{I mentioned explicitly here that when referring to time or round $f$, I refer to \textbf{the beginning of the round}, i.e., before the bidding, in order to prevent ambiguity} 
the earliest round after which either all other agents become inactive, or all items have been allocated. Let
$C\subseteq\mathcal{M}$ denote the set of items agent $p$ has by the
end of round $f$, and let $O$ denote the set of items that the other
agents have by the end of round $f$. Define $L^{f}=\sum_{S}\lambda_{S}\cdot v_{p}(S\setminus O\dot{\cup}C\mid C)$.
Namely, $L^{f}$ is the expected marginal value to agent $p$ (who
already holds the set $C$ of items) of a bundle $S$ selected at
random according to the probability distribution over bundles implied
by the coefficients $\lambda_{S}$, after one removes from $S$ those
items that were allocated by the end of round $f$.

\begin{claim}
\label{Claim_first_Bound_Lf}
Let $\tilde{\mathcal{M}}$ be the set of items that remain unallocated after round $f$, i.e., $\tilde{\mathcal{M}}=\mathcal{M}\setminus(O\dot{\cup}C)$. Then $v_{p}(C\cup\tilde{\mathcal{M}})=v_{p}(\mathcal{M}\setminus O)\geq v_{p}(C)+L^{f}$.
In other words, $v_{p}(C)+L^{f}$ is a lower bound on the total
value that agent $p$ will have, if she receives all the remaining items
($\tilde{\mathcal{M}}$).
\end{claim}

\begin{proof}
%\ufc{remove: If there are other active agents at the end of round $f$, {then it is the end of the algorithm, and} all the items have been allocated to agents (i.e., $C\dot{\cup}O=\mathcal{M})$. Thus,} 
{If no items remain after round $f$ (i.e., $C\dot{\cup}O=\mathcal{M}$), then} for
each $S\subseteq\mathcal{M}$, $S\setminus O\dot{\cup}C=\emptyset$
and $L^{f}=0$. Hence, $v_{p}(C\cup\tilde{\mathcal{M}}) =  v_p(C) = v_{p}(C)+L^{f}$, proving the claim.

%\ufc{Removed: "Otherwise, $p$ is the only active agent." This is not relevant here, and also, nothing in the claim or in the definition of $f$ implies that $p$ is active.} 
{If items do remain after round $f$, then} every term $v_{p}(S\setminus O\dot{\cup}C\mid C)$
in the sum of $L^{f}=\sum_{S}\lambda_{S}\cdot v_{p}(S\setminus O\dot{\cup}C\mid C)$,
is a marginal value of a partial set of the not-yet-allocated items
(i.e., $(S\setminus O\dot{\cup}C)\subseteq\tilde{\mathcal{M}).}$
Hence $v_{p}(S\setminus O\dot{\cup}C\mid C)\leq v_{p}(\tilde{\mathcal{M}}\mid C)$.
Since the scalars $\{\lambda_{S}\}$ in the sum $L^{f}=\sum_{S}\lambda_{S}\cdot v_{p}(S\setminus O\dot{\cup}C\mid C)$
are non-negative and add up to $1$, we obtain:
\begin{align*}
L^{f}= & \sum_{S}\lambda_{S}\cdot v_{p}(S\setminus O\dot{\cup}C\mid C)\\
\leq & \sum_{S}\lambda_{S}\cdot v_{p}(\tilde{\mathcal{M}}\mid C)\\
= & v_{p}(\tilde{\mathcal{M}}\mid C)
\end{align*}
Hence
\begin{align*}
v_{p}(C)+L^{f}\leq & v_{p}(C)+v_{p}(\tilde{\mathcal{M}}\mid C)=v_{p}(C\cup\tilde{\mathcal{M}})
\end{align*}
\end{proof}

\begin{claim}
\label{Claim_second_bound_Lf}
$\min\{L^{\text{f}}+v_{p}(C),2\rho{APS_p}\}$ is a lower bound on the final
total value of agent $p$.
\end{claim}
\begin{proof}
First, notice that if $p$ is not active in time $f$, then $p$ spent her entire budget, ${b_p}$. The bidding strategy of $p$ (and submodularity of $v_{p})$ implies that in that case,
$p$ has a value of at least $2\rho {APS_p}$ in time $f$, i.e., $v_{p}(C)\geq 2\rho {APS_p}$
and the claim follows in this case.

Otherwise, $p$ is an active agent {after round} $f$. We consider two cases.
If some items remain after round $f$, then agent $p$ is the only remaining active agent. Hence $p$ is the only agent to win items {from $\tilde{\mathcal{M}}$.}
Then, the agent will keep winning items until she becomes inactive or until she wins all remaining items. By \Cref{Claim_first_Bound_Lf}, $v_{p}(\mathcal{M}\setminus O)\geq v_{p}(C)+L^{f}$. The fact that the agent bids at most {$\frac{1}{2\rho}{\cdot\frac{b_p}{APS_p}}$ times} the marginal value of the item she wins in each round guarantees that agent $p$ gets at least $\min\{L^{f}+v_{p}(C), 2\rho {APS_p}\}$. It remains to handle the case of agent $p$ being active at time $f$ while no items remain. In this case, $L^{f}=0$, so the bound is trivial.
\end{proof}

\begin{claim}
\label{claim_L0_Lf}
The following holds:
\[
L^{0}\leq L^{f}+v_{p}(C)+b_p\cdot\sum_{e\in O}v_{p}(e\mid C)
\]
\end{claim}


\begin{proof}
$\!$
\[
L^{f}=\sum_{S}\lambda_{S}\cdot v_{p}(S\setminus O\dot{\cup}C\mid C)=\sum_{S}\lambda_{S}\cdot v_{p}(S\setminus O\mid C)
\]
For every $S\subseteq\mathcal{M}$ we claim:
\[
v_{p}(S)\underset{1.}{\leq}v_{p}(S\mid C)+v_{p}(C)\underset{2.}{\leq}v_{p}(S\setminus O\mid C)+v_{p}(C)+\sum_{e\in S\cap O}v_{p}(e\mid C)
\]
\emph{proof of inequality 1:}
\[
v_{p}(S\mid C)+v_{p}(C)=v_{p}(S\cup C)-v_{p}(C)+v_{p}(C)=v_{p}(S\cup C)\ge v_{p}(S)
\]
\emph{proof of inequality 2:}

Consider an arbitrary order of the set $S\cap O=\{e_{1},\dots,e_{k}\}$.
Then:

\begin{align*}
v_{p}(S\mid C) 
& =v_{p}\left(S\setminus O\mid C\right)+\sum_{i=1}^{k}v_{p}\left(e_{i}\mid\left(S\setminus O\right)\cup C\cup\left(\bigcup_{j=1}^{i-1}e_{j}\right)\right)\\
& \leq v_{p}\left(S\setminus O\mid C\right)+\sum_{i=1}^{k}v_{p}\left(e_{i}\mid C\right)
\end{align*}

Thus, using the last inequality, we obtain the following:

\begin{align*}
L^{0}=\sum_{S\subseteq\mathcal{M}}\lambda_{S}v_{p}(S)
& \leq\sum_{S\subseteq\mathcal{M}}\lambda_{S}\left(
v_p(S\mid C)+v_p(C)
\right) \\
& \leq\sum_{S\subseteq\mathcal{M}}\lambda_{S}\left(v_{p}(S\setminus O\mid C)+v_{p}(C)+\sum_{e\in S\cap O}v_{p}(e\mid C)\right) \\
& \underset{*}{\leq}v_{p}(C)+b_p\cdot\sum_{e\in O}v_{p}(e\mid C)+\sum_{S\subseteq\mathcal{M}}\lambda_{S}v_{p}(S\setminus O\mid C)\\
& =v_{p}(C)+b_p\cdot\sum_{e\in O}v_{p}(e\mid C)+L^{f}
\end{align*}
where inequality * is since each item $e\in\mathcal{M}$ has a total
weight of at most $b_p$ (by the APS definition). Overall, as we wanted
to show, we obtained the following:
\[
L^{0}\leq L^{f}+v_{p}(C)+b_p\cdot\sum_{e\in O}v_{p}(e\mid C)
\]
\end{proof}



\begin{claim}\label{claim_other_agent_items_reducing_Lf}
Either $v_p(C) \ge \rho {APS_p}$, or
\[
\sum_{e\in O}v_{p}(e\mid C)\leq {2\rho(b^0-b_p){\cdot \frac{APS_p}{b_p}}}
\]
\end{claim}

\begin{proof}
Suppose that $v_p(C) < \rho {APS_p}$. Let $C^{e}\subseteq C$ be the set of items agent $p$ already won
when another agent wins item $e$.
\Cref{claim:2alternatives} implies that
agent $p$ bids {$\frac{1}{2\rho}{\cdot\frac{b_p}{APS_p}}$ times }the highest marginal value of an item w.r.t $C^{e}$.
Hence, when the other agent $i$ wins item $e$, agent $p$ bid is
at least ${\frac{1}{2\rho}{\frac{b_p}{APS_p}}}v_p(e\mid C^{e})\geq {\frac{1}{2\rho}{\frac{b_p}{APS_p}}}v_p(e\mid C)$, and winning item $e$ reduces the budget of agent $i$ by at least ${\frac{1}{2\rho}{\frac{b_p}{APS_p}}}v_{p}(e\mid C)$. Since the
budget of all other agents at the beginning is $b^{0}-{b_p}$, we
obtain 
\[
\sum_{e\in O}\frac{b_p}{2\rho APS_p}v_{p}(e\mid C)\leq\sum_{e\in O}\frac{b_p}{2\rho APS_p}v_{p}(e\mid C^{e})\leq {(b^0-b_p)}
\]
The claim follows by rearranging (scaling both sides by $\frac{2\rho APS_p}{b_p}$).
%\gbc{I think the above inequality is clearer than the one below - so I suggest removing it}
%\[
%\sum_{e\in O}v_{p}(e\mid C)\leq\sum_{e\in O}v_{p}(e\mid C^{e})\leq %{2\rho(b^0-b_p){\frac{APS_p}{b_p}}}
%\]
\end{proof}



We are now ready to prove Lemma~\ref{lem:1/3 guarantee no large items}.

\begin{proof}
Considering \Cref{claim_L0_Lf} and \Cref{claim_other_agent_items_reducing_Lf}, we have that either $v_p(C) \ge \rho{b_p}$, or the following holds:
\[
APS_p\leq L^0\leq v_{p}(C)+L^{f}+b_p\cdot\sum_{e\in O}v_{p}(e\mid C) \leq v_{p}(C)+L^{f}+{b_p\cdot2\rho{\frac{APS_p}{b_p}}(b^0-b_p)}
\]
%\gbc{This is the first place where we use $b_p=\frac{1}{n}$, and the proof splits for the equal and arbitrary entitlements. Therefore, I suggest an alternation, so the assumption of $b_p=\frac{1}{n}$ will arrive at the very last line of the proof.}
%\gbc{Removed:
%Recalling that \gbc{Removed:$APS_p=$}$b_p=\frac{1}{n}$ and $b^0 = 1$, we obtain:
%\[
%v_{p}(C)+L^{f} \ge APS_p\cdot {(1-2\rho b^0+2\rho b_p)} = APS_p\cdot(1 - 2\rho\frac{n-1}{n})
%\]
%By setting $\rho=\frac{n}{3n-2}$}

By rearranging the above, and plugging $b^0=1$ we obtain:
\[
 APS_p\cdot {(1-2\rho b^0+2\rho b_p)} = APS_p\cdot {(1-2\rho +2\rho b_p)} 
 \le v_{p}(C)+L^{f}
\]

By setting $\rho=\frac{1}{3-2b_p}$ 
the above gives $L^{f}+v_{p}(C) \ge \rho{APS_p}$. {So far we obtained that either $v_p(C)\geq\rho APS_p$, or } %\gbc{Removed:This, together with the fact that} 
agent $p$ is guaranteed to have a total final value of $\min\{L^{f}+v_{p}(C),2\rho {APS_p}\}$ (\Cref{Claim_second_bound_Lf}). %\gbc{Remove:, implies that agent} 
{ Hence, in both cases, when setting $\rho=\frac{1}{3-2b_p}$, agent }$p$ is guaranteed to have at least $\rho$ fraction of her $APS$. In the special case of equal entitlements (where $b_p=\frac{1}{n}$) it implies $\rho=\frac{n}{3n-2}$. This completes the proof of ~\Cref{lem:1/3 guarantee no large items}.


\end{proof}

We now restate and prove \Cref{thm:1/3_APS_guarantee}.
\APSbidding*
%\gbc{Maybe restate \Cref{thm:1/3_APS_guarantee} here? as we removed its restating from above.\\}
\begin{proof}
~\Cref{lem:1/3 guarantee no large items} handles the case that $v_p(e) \le 2\rho {APS_p}$ for every item $e$.
It remains to handle the case that there are items $e$ of value $v_p(e) > 2\rho {APS_p}$.

Consider an input instance $I_0$. 
Run the bidding game with $p$ using the proportional bidding strategy. As described in $proportional_2$, {$s$ denotes the {last round} in which there was an unallocated item $e$ with $v_p(e) > 2\rho APS_p$.} 
%\ufc{remove: $s$ denotes the {first round} in which all yet unallocated items satisfy $v_p(e) \le 2\rho APS_p$.} \ufc{Check definition of $s$. Seems that it is used inconsistently. Perhaps the intention is not "in which" but "after which"?} \gbc{It is consistent now - $s$ is the first time/round in which the assumption of no large items holds.} 
If agent $p$ won some item by the end of round $s$, then she has a value of at least $2\rho APS_p$, and we are done. Hence, we may assume that agent $p$ did not win any item in the first $s$ rounds. {Recall the definitions of residual instance $I_s$, $\hat{I_s}$ and $\gamma$ (which is the scaling factor between bids in $I_s$ and $\hat{I_s}$) presented in $proportional_2$.}
%\gbc{Removed: $I_s$ denote the
%{\em residual instance}  that remains after round $s$. 
%\gbc{In the residual instance, we do not really care about the valuation function of other agents, as our promise is against any adversarial bidding of other agents. Furthermore, as we describe next, in each of the first $s$ rounds, another agent wins an item, and becomes inactive (spent her entire budget). So the agents who survive to the residual instance have the same valuation function (since they did not win any item by round $s$). Therefore the following sentence should be removed} \ufc{You are jumping ahead to $I_{\hat{s}}$. First there is a description of the residual instance  $I_s$. In the next paragraph it is explained which parts are redundant, giving rise to $I_{\hat{s}}$.}
%It includes the set of items that were not yet allocated (which we denote by $\mathcal{\hat{M}}$), the valuation function (over the remaining items) of each agent is her marginal valuation function with respect to the items that she already has (for $p$, this is her original valuation function), and each agent has whatever remains from her budget after the $s$ rounds (for $p$, this is her entire original budget). }


By the definition of $s$, in each of the first $s$ rounds,
there is an available item with $v_{p}(e)>2\rho APS_p$. Thus agent {$p$ bids her entire budget $b_p$ in each such round}, and the (other) agent who wins the round spends at least $b_p$. (In the special case of equal entitlement, this means that in each of the first $s$ rounds, some agent wins a single item and becomes inactive. Hence, in the residual instance $I_s$ there are $n-s$ active agents (including $p$), each active agent has her entire original budget, and no active agent has any items.)

%\gbc{Removed: View $I_{s}$ as a new allocation instance, that we refer to as $I_{\hat{s}}$. The agents of $I_{\hat{s}}$ are the $n-s$ active agents of $I_{s}$. The set of items of $I_{\hat{s}}$ is $\mathcal{\hat{M}}$ (those items remaining in $I_{s}$)}. 
Setting $\gamma=b^s\leq 1 - s\cdot b_p$, the entitlement of each agent {in $\hat{I_s}$} is $\hat{b_i^s}=\frac{1}{\gamma} b_i^s$. 
%\gbc{Removed: By \Cref{claim:APS not decrease in I_s} \ufc{need to check that this claim appears and is written properly} the $APS$ of $p$ (with valuation function $v_{p}$) in $I_{\hat{s}}$ is at least as large as the $APS$ of $p$ in $I_{0}$. Scaling the valuation of $p$ to be $\hat{v}_{p}=\gbe{\frac{\gamma b_p}{APS(\hat{\items},\hat{b})}v_p}$, the APS of agent $p$ in $I_{\hat{s}}$ is  $\hat{b}$.}
%\ufc{Need to make notation here consistent with that of page 2. Best way of doing so might include changes in both pages.}

%\gbc{Removed:Observe that $I_{\hat{s}}$ is an allocation instance in which {the APS of $p$ is $\hat{b}$, and} $\hat{v}_{p}(e) \le 2\rho\hat{b}$ for every item $e \in \mathcal{\hat{M}}$. }

{
Recall that agent $p$ simulates the bidding strategy ($proportional_1$) on $\hat{I_{s}}$. A bid $p_{i}^{r}$ in $I_{s}$ is interpreted as a bid of $\frac{1}{\gamma}\hat{p}_{i}^{r-s}$ in $\hat{I_{s}}$. 
% Remove: Namely other agent bids $p_{i}^{r}$ in $I_{s}$ then $p$ simulate her bid in $\hat{I_{s}}$ as $\hat{p}_{i}^{r-s}=\frac{1}{\gamma}p_{i}^{r}$.
% Then, agent $p$ knows how to bid by considering scaling her bids in $\hat{I_{s}}$ by $\gamma$. Since the ratio between a budget of an agent in $I_{s}$ and in $\hat{I_{s}}$ is $\frac{1}{\gamma}$, by taking a bid of another agent $i$ in $I_{s}$ and scaling it by $\frac{1}{\gamma}$ we get a legal bid of the agent in $\hat{I_{s}}$. Similarly, scaling by $\gamma$ a bid of agent $p$ in $\hat{I_{s}}$ induces a legal bid in $I_{s}$.
By \Cref{claim:APS not decrease in I_s}, the APS of agent $p$ at $\hat{I_{s}}$ 
%\ufc{it would be better if the APS is the same, as otherwise the statement is not true} %\gbc{I corrected \cref{claim:APS not decrease in I_s}} \gbc{Removed: is at least as the}
{stays the same as the} APS of $p$ in the original instance. Hence, in $I_s$, agent $p$ will get the same bundle as she gets in the run on $\hat{I_{s}}$. By \Cref{lem:1/3 guarantee no large items}, this bundle is of value at least $\rho APS_p$, as desired.}
%\gbc{Removed:
%By Lemma~\ref{lem:1/3 guarantee no large items}, the proportional bidding strategy \gbe{($proportional_1$)} guarantees $p$ a value (according to $\hat{v}_p$) of at least $\rho\gbe{APS_p}\gbc{Removed:\hat{b}}$. We claim that this implies that the proportional bidding strategy of $p$ in $I_{0}$ guarantees a value of at least $\rho \gbe{APS_p}\gbc{Removed:b_p}$ \gbc{Removed:(according to $v_p$)}. After $s$ rounds in $I_0$, the proportional bidding strategy of $p$ in $I_s$ is a scaling of the proportional bidding strategy in $I_{\hat{s}}$, with a scaling factor of \gbe{$\gamma$}\gbc{$\frac{1}{\gamma}$}. \ufc{Since we changed the definition of the proportional bidding strategy, there is no scaling here.} Namely, a bid of $\alpha$ in $I_{\hat{s}}$ is replaced by a bid of \gbe{$\gamma\alpha$}\gbc{Removed:$\frac{1}{\gamma}\alpha$} in $I_{s}.$ 
%We now show that this strategy guarantees a $\rho b$ value to $p$ in $I_{s}$. 
%Assume for the sake of contradiction that for some adversarial bidding of the other agents, $p$ does not receive at least $\rho \gbe{APS_p}\gbc{Removed:b_p}$ in $I_{s}$. Then, by scaling the adversarial bids of the other agents by \gbe{$\frac{1}{\gamma}$}\gbc{Removed:$\gamma$}, we have a run of the algorithm on $I_{\hat{s}}$ in which the proportional strategy of $p$ does not yield a $\rho\gbe{APS_p}\gbc{Removed:\hat{b}}$ value, contradicting Lemma~\ref{lem:1/3 guarantee no large items}. 
%Hence, we have a strategy for $p$ to ensure $\rho b$ value in $I_{s}$.}
\end{proof}


% \gbe{
% \begin{lemma*}
% \textbf{Lemma 3.} In the equal entitlement case, where $b_{j}=\frac{1}{n}$
% for all $n$, we can set $\rho=\frac{n}{3n-2}$.
% \end{lemma*}

% In the equal entitlement case we assume that no single item $e$ has
% value $\rho\cdot b_{j}$ for any agent $j$, as otherwise we give
% $e$ to the agent and remove the agent. For the APS fractional partition
% of other agents, we remove the bundles containing $e$, and the APS
% does not decrease. \ufc{If the point of you rewriting the proof of Lemma 3 is to fill in missing details, then here you also need to explain that after removing the agent, the given allocation algorithm gives each other individual agent exactly the same items that she would get before removing the agent.}

% Suppose agent $p$ did not receive a $\rho$ fraction of her APS when
% the algorithm finished. First, some notation.
% Denote the valuation function of agent $p$ as $v_p$ and her entitlement $b=\frac{1}{n}$. As above,
% we assume her APS is also equal to $b.$ Denote $\{\lambda_{J}\}_{J\subseteq\mathcal{M}}$
% to be a set of weights associated with her APS {fractional partition} (i.e {for every $J\subseteq\mathcal{M}$,} 
% $\lambda_{J}>0\implies v(J)\geq b,$ also $\sum_{J}\lambda_{J}=1$,
% and $\forall e\in\mathcal{M},$ $\sum_{j|e\in j}\lambda_{j}\leq b$).

% Let $L^{0} = \sum_{J}\lambda_{J}v(j)$.}
% At the beginning of the algorithm (when no item has been
% allocated yet), $L^{0}$ is a lower bound on the marginal value
% of the set of all items to agent $p$

% Let $f$ denote the earliest round after which either all other agents become
% inactive, or all items have been allocated.
% Let $C\subseteq\mathcal{M}$ denote the set of items that agent $p$
% has by the end of round $f$, and let $O$ denote the set of items that the other agents
% have by the end of round $f$. Define $L^{f}=\sum_{J}\lambda_{J}\cdot v(J\setminus O\dot{\cup}C\mid C)$. namely, $L^f$ is the expected marginal value to agent $p$ (who already holds the set $C$ of items), of a bundle $J$ selected at random according to the probability distribution over bundles implied by the coefficients $\lambda_J$, after one removes from $J$ those items that were allocated by the end of round $f$.

% \ufc{The next claim appears to be formulated incorrectly.}

% \begin{claim}
% $L^{f}$ is a lower bound of the agent $p$'s value \ufc{not clear, value for what?} at the end of
% the algorithm. \ufc{Perhaps you mean to say (and then prove) that $\min[\rho b, v_p(C) + L^f]$ is a lower bound on the final total value of agent $p$.}
% \end{claim}
% \begin{proof}
% If there are other active agents at the end of the algorithm, then
% time $f$ is the end of the algorithm, and all the items have been
% allocated to agents (i.e., $C\dot{\cup}O=\mathcal{M})$. Thus, for
% each $J\subseteq\mathcal{M}$, $J\setminus O\dot{\cup}C=\emptyset$
% and $L^{f}=0$, so the bound is trivial.

% Otherwise, $p$ is the only active agent. Since $p$ is the only remained
% active agent, the rest of the not-yet allocated items will assign
% to her. \ufc{Not true -- that's not what the algorithm says.} So every term in the sum is a marginal value of a partial
% set of the not-yet-allocated items. Since the sum is convex, we obtain
% that$L^{f}$ is a lower bound on the value agent $p$ will has at
% the end of the algorithm.
% \end{proof}

% \ufc{The formulation of the next claim is problematic, as the notion of $e$ reducing $L^0$ is not defined. If you introduce notation for individual rounds, things become clearer. If $e$ is allocated in round $r$, then you compare $L^r - L^{r-1}$ with $B^r - B^{r-1}$, where $B^t$ denotes the total budget after round $t$.}

% \begin{claim}
% Let $\alpha\ge0.$ Consider the order of the allocation of items throughout
% the algorithm. For each item $e\in\mathcal{M}$ which is allocated
% to other agents, if it reduces $L^{0}$ by $b\cdot\alpha\geq o$,
% then the budget deduced from the agent winning the item is at least
% $\alpha$
% \end{claim}
% \begin{proof}
% Let $C'\subseteq C$ be the set of items agent $p$ already won when
% another agent wins item $e$. When the other agent wins item $e$,
% agent $p$ bids $v(e\mid C')\geq v(e\mid C)$. Thus, after deducting
% all the items of $C$ from the APS bundles, removing item $e$ decreases
% $L$ by at most $b\cdot v(e\mid C)$ (since the bundles of the APS
% partition in which item $e$ participates have a total weight of at
% most $b$), while the budget of the other agent decreased by at least
% $v(e\mid C')\geq v(e\mid C)$ (i.e., the budget of the other agent
% is reduced at least as the amount it reduces $L$)
% \end{proof}
% \begin{claim}
% Let $C\subseteq\mathcal{M}$ be the set of items agent $p$ won by
% time $f$. Then $C$ is reducing $L^{0}$ to $L^{f}$ by at most $v(C)$.
% \end{claim}
% \begin{proof}
% Consider one term in the sum of $L^{f}$ associated with the set $J\subseteq\mathcal{M}$.
% Then,
% \[
% v(J\setminus O\dot{\cup}C\mid C)=v((J\setminus O)\cup C)-v(C)\geq v(J\setminus O)-v(C)
% \]

% Since the above inequality holds for each term in the convex sum,
% the claim follows.
% \end{proof}

% By the above claims, other agent winning items reduce $L^{0}$ to
% $L^{f}$ by at most $b\cdot(b^{0}-2\rho b)=2\rho(b-b^{2})$ and hence $L^{f}$
% is at least
% \[
% L^{f}\ge APS-2\rho b+2\rho b^{2}-v(C)=b-2\rho b+2\rho b^{2}-v(C)=b\cdot(1-2\rho+2\rho b)-v(C)
% \]

% Recall that $L^{f}$ is a lower bound on the total \textbf{marginal
% value} of all remaining items (not yet allocated) in time $f$, and
% thus, agent $p$ is guaranteed to have a value of at least $v(C)+L^{f}\ge b\cdot(1-2\rho+2\rho b)=b\cdot(1-2\rho\cdot\frac{n-1}{n})$.

% Overall, by setting $\rho=\frac{n}{3n-2}$, agent $p$ is guaranteed
% to have at $\rho$ of her $APS$ as we wanted to show.

%\ufc{Review up to here}


%\begin{remark}
%There is slackness in the proof for equal entitlement, and hence the value of $\rho$ can probably be improved. Consider an agent $j \not= i$. If $j$ spent all her budget, she  won only two items. In general, if she won $k$ items she spent at most $\frac{k}{2k-2}$ fraction of her budget. If all agents but $i$ spent all their budget, then a weight of $\frac{2}{n}$ bundles still have value $1 - \rho$ left in them, and we can set $\rho = \frac{1}{2}$. Hence some agents do not spend all their budget, which should allow for a value of $\rho$ strictly larger than $\frac{1}{3}$. Try to provide rigorous proofs with as good bounds as possible, and further improve them when $n$ is small by having an additive $\Omega(\frac{1}{n})$ term.
%\end{remark}

%\gbc{The following 2 lemmas regarding $\rho=\frac{1}{3-2b_p}$ in the arbitrary entitlement case, since both cases of equal and arbitrary entitlements appears in \Cref{thm:1/3_APS_guarantee} and \Cref{lem:1/3 guarantee no large items}}
%\begin{lemma}
%\label{lemma:1-3 APS Arbitrary - using assumption no large items} In the case of the arbitrary entitlements, if $\rho = \frac{1}{3-2 b_p} (\le \frac{1}{3})$  and  there are no large items, i.e., $v_{p}(e) \le 2\rho APS_p$ for every item $e$, then the proportional bidding strategy guarantees agent $p$ a value of at least $\rho \cdot APS_p$.
%\end{lemma}




%\begin{lemma}
%\label{lem:1/3 arbitrary ent}
%When agents have arbitrary entitlements and agent $p$ has a submodular valuation function, setting $\rho$ to  $\frac{1}{3 - 2b_p}$, the proportional bidding strategy guarantees agent $p$ a value of at least $\rho \cdot APS_p$.
%\end{lemma}
%\begin{proof}
%In the proportional bidding strategy, in any round in which items with a value larger than $2\rho APS_p$ exist, agent $p$ bids her entire budget. If she wins one of these rounds, she guarantees herself $2\rho APS_{p}$. In case she did not win any of the first $s$ rounds, the agent simulates the bidding strategy on $\hat{I_{s}}$. A bid $p_{i}^{r}$ in $I_{s}$ is correlated with a bid of $\frac{1}{\gamma}\hat{p}_{i}^{r-s}$ in $\hat{I_{s}}$. Namely other agent bids $p_{i}^{r}$ in $I_{s}$ then $p$ simulate her bid in $\hat{I_{s}}$ as $\hat{p}_{i}^{r-s}=\frac{1}{\gamma}p_{i}^{r}$. Then, agent $p$ knows how to bid by considering scaling her bids in $\hat{I_{s}}$ by $\gamma$. Since the ratio between a budget of an agent in $I_{s}$ and in $\hat{I_{s}}$ is $\frac{1}{\gamma}$, by taking a bid of another agent $i$ in $I_{s}$ and scaling it by $\frac{1}{\gamma}$ we get a legal bid of the agent in $\hat{I_{s}}$. Similarly, scaling by $\gamma$ a bid of agent $p$ in $\hat{I_{s}}$ induces a legal bid in $I_{s}$. By \Cref{claim:APS not decrease in I_s}, the APS of agent $p$ at $\hat{I_{s}}$ is at least as the APS of $p$ in the original instance. Hence, by the end of the algorithm, agent $p$ will get the same bundle as she gets in the run on $\hat{I_{s}}$. Thus, by \Cref{lem:1/3 guarantee no large items}, this bundle is of value at least $\rho APS_p$ as desired.
%\end{proof}


%\begin{remark}
%For $n=42$, here is an example for which the proportional bidding strategy does not give $\rho > \frac{3}{8}$. There are $n-1$ additive agents and one submodular one. For every agent, the MMS partition is the same, and item values in every bundle are $(6,6,3,1)$. For the submodular agent, the $n$ items that are first in their bundle are all substitutes of each other. In the bidding, the submodular agent may first win a first item of value~6. Then 21 additive agents each win two items of value~6 (the second items in two different bundles), 14 additive agents each win three items of value~3 (the submodular agent does not bid~6 because the items of value~6 that remain are substitutes for the item that the agent already has), and 6 additive agents each win seven items of value~1. The items that remain have no value for the submodular agent.
%The conjectured worst case (for large number of agents) might be when some agents win two items, some three, and some seven. Then every bundle lost $(1 + \frac{1}{2} + \frac{1}{6})\rho = \frac{5\rho}{3}$, and using $1 - \frac{5\rho}{3} = \rho$ we get $\rho = \frac{3}{8}$. 
%\end{remark}



%\ufc{define the altruistic version of the bidding game and the $\rho$-proportional strategy, perhaps under different names.}


%\ufc{Remove:    So far, in \Cref{thm:1/3_APS_guarantee} we presented an algorithm based on a bidding game in which every agent has an initial budget, and in each round, the highest bidder chooses to pick an item. Each agent plays until she finishes her budget, and the game continues until there are no more items or the agents finish their budgets. Now we define another version of the bidding game in which agents might leave the game before finishing their budget.}


\subsection{Approximate MMS-fair allocations for submodular agents}
\label{sec:Approximated MMS-fair existance}

In the standard version of the bidding game, every agent has an initial budget, and in each round, the highest bidder picks an item. Each agent plays until she exhausts her budget, and the game ends when either there are no more items left, or all agents exhaust their budgets. Now we define another version of the bidding game, in which agents might leave the game before exhausting their budget.

%\ufc{Note to myself: should be $\rho$-altruistic game and 1-proportional bidding}
    
\begin{definition}
    The  $\rho$-altruistic version of the bidding game is the same bidding game with the change that every agent becomes inactive after spending a $\rho$-fraction of her budget. 
    % \gbc{Related to the next comment denoted with * - we need to choose one of the two. the budget of the agents is $b_p$ and she becomes inactive after spending $\frac{1}{2}$ of her budget
    % 2. the agent's budget is $2\rho MMS_p$ and she becomes inactive after spending $\rho$ of her budget
    % I think we should pick the second option. In the proofs we want to think of other agnets payments as the value of the items for agent $p$ (or at least an upper bound). I think considering each agent gets a budget of $2\rho b_p$ makes the arguments clearer}
\end{definition}

%\ufc{edit below, explaining the proportional bidding strategy when MMS=2.7 and so is the budget. Assume also that valuation is truncated at 2.7.}

We shall consider a bidding strategy for the $\rho$-altruistic bidding game, that we shall refer to as the {\em proportional bidding strategy}. We make two assumptions that simplify our presentation. These assumptions have no effect on the correctness of Theorem~\ref{thm:equal-10/27} that follows. These assumptions concern the submodular valuation function $v_p$ of agent $p$ that uses the proportional bidding strategy.

\begin{enumerate}
    \item The valuation function is scaled so that $MMS_p = b_p$ (the MMS of agent $p$ equals her entitlement).
    \item The valuation function is truncated at $MMS_p$ (as in Definition~\ref{def:$v^t_p$}). By Claim~\ref{claim:truncation}, this truncation does not affect the MMS value. 
    %\gbc{In the presentation of the proportional bidding strategy for the original version of the bidding game, I put the requirement of $p$ truncating her valuation function into the proportional bidding strategy. i.e, an agent who executes proportional strategy first truncates her valuation function}
\end{enumerate}

We now present the proportional bidding strategy, as used by agent $p$. At the beginning of round $r$, let $\items^r$ denote the set of items not yet allocated, let $C^r$ denote the set of items already allocated to $p$, and let $b_p^r$ denote the budget remaining for agent $p$. Then the agent bids $\max_{e\in\items^r}[v_p(e\mid C^r)]$  (the highest marginal value that a yet unallocated item has) %\gbc{Removed:, scaled by $\frac{b_p}{APS_p}$} 
if this bid is not larger than $b_p^{r}$, and bids her remaining budget $b_p^{r}$ otherwise. 
% \gbc{ * There is a problem - in the case of agent $MMS_p = b_p$, the bids of $p$ should be $\bold{\frac{1}{2\rho}}\cdot\max_{e\in\items^r}[v_p(e\mid C^r)]$, unless we assume the initial budget of the agent is $2\rho b_p$ and not $b_p$ (as done in the initial proof)} If the agent
% wins her bid, she selects the item with the highest marginal value with respect to $C^r$.




Recall \Cref{thm:equal-10/27}:
\altruistic*


% \begin{theorem}
% \label{thm:equal-10/27}
% In the equal entitlement case with submodular valuations, for $\rho = \frac{10}{27}$, every agent that uses the proportional strategy in the altruistic version of the bidding game is guaranteed to get a bundle of value at least a $\rho = \frac{10}{27}$ fraction of her MMS. 
% \end{theorem}


Before presenting the proof of Theorem~\ref{thm:equal-10/27}, we sketch the proof for a weaker version of the theorem, setting $\rho = \frac{4}{11}$ instead of $\rho = \frac{10}{27} > \frac{4}{11}$. Already this weaker version improves over the ratio of $\rho = \frac{n}{3n-2}$ of Theorem~\ref{thm:1/3_APS_guarantee} (when $n > 8$). Moreover, the proof of this weaker version of Theorem~\ref{thm:equal-10/27} conveys some intuition that may be helpful for following the proof of Theorem~\ref{thm:equal-10/27}. 

%\ufc{does this (and main theorem) work for APS with equal entitlement?}\gbc{No, at least not immediately. In our proof, we analyze inequalities implied by an MMS partition. for example when we analyze the distribution of the primary items in the MMS bundles. We also use claims such as no two primary items of type $X_1$ in the same MMS bundle.}

\begin{proposition}
\label{pro:4/11}
In the equal entitlement case with submodular valuations, for $\rho = \frac{4}{11}$, every agent that uses the proportional strategy in the altruistic version of the bidding game is guaranteed to get a bundle of value at least a $\rho = \frac{4}{11}$ fraction of her MMS. 
\end{proposition}

\begin{proof}
We only sketch the proof, as we shall later present a full proof for Theorem~\ref{thm:equal-10/27}.

If agent $p$ that uses the proportional bidding strategy manages to spend $\rho \cdot b_p$, then she also wins items of total value at least $\rho \cdot MMS_p$, and we are done. Hence it remains to exclude the case that agent $p$ failed to spend $\rho b_p$. In this case,  partition the agents other than $p$ into three classes, $X_0, X_1, Y$. 

Class $X_0$ contains those agents that by the end of the bidding game take only one item. %\gbc{Why add another demand on $X_0$ agent. Why not just agents that take only 1 item?} and at the time that the item was taken, the bid of agent $p$ was strictly larger than $\rho APS$. 
Intuitively, an agent $i$ of class $X_0$ who took one item does not ``hurt" $p$, because there are $n-1$ bundles in the MMS partition of $p$ from which agent $i$ does not take any item, and only $n-1$ agents (including $p$) compete for items in these bundles. Hence we may assume that no agent is in class $X_0$. See further details in \Cref{claim: no X_0 agents}. 
%\ufe{(This argument appears in a more rigorous form in the proof of Theorem~\ref{thm:1/3_APS_guarantee}, and hence further details are omitted here.)} \ufc{Instead: see more details in claim?} \gbc{I prefer: see more details in \Cref{claim: no X_0 agents}}

Class $X_1$ contains those agents that by the end of the bidding game take two items. Being in the $\rho$-altruistic bidding game implies that for the first item that such an agent $i$ took she paid at most $\rho b_p$ (here we use the fact that agents have equal entitlements), implying that the bid of $p$ at the time was at most $\rho b_p = \rho MMS_p$. By the proportional strategy, this bid was equal to the highest marginal value for $p$ for any of the remaining items at the time, and as the sequence of highest marginal values cannot increase as rounds progress, this further implies that the marginal $v_p$ values of items taken by agent $i$ is at most $2 \rho MMS_p$. A key observation is that if there are more than $\frac{n}{2}$ agents in class $X_1$, then there must be a bundle $B$ in the MMS partition of $p$ from which they took at least two items. As it does not matter for $p$ which of the agents in $X_1$ takes which of the items that the agents in $X_1$ collectively take (as long as each agent in $X_1$ takes two items, and each such item has marginal value at most $\rho b_p$), we may pretend that there is an agent $i$ in $X_1$ for which the two items that she takes are from this bundle $B$. This agent $i$ does not ``hurt" $p$, because there are $n-1$ bundles in the MMS partition of $p$ from which agent $i$ does not take any item, and only $n-1$ agents (including $p$) compete for items in these bundles. Hence, we may assume that at most $n/2$ agents are in class $X_1$. See further details in \Cref{claim: at most n/2 X_1  agents}.

Class $Y$ contains all remaining agents. Such an agent $i$ either takes no items (and then of course she does not hurt $p$), or takes only one item with marginal value (to $p$) of at most $\rho APS_p$, or takes $k \ge 3$ items. In the latter case, being in the $\rho$-altruistic bidding game implies that for her first $k-1$ items agent $i$ spent at most $\rho b_p$, and hence the bid of $p$ for the last item taken by $i$ was at most $\frac{1}{k-1} \rho b_p$. This implies that the total marginal $v_p$ values (with respect to items held by $p$) of items taken by $i$ is at most $\frac{k}{k-1} \rho MMS_p$. For $k \ge 3$, this is maximized when $k=3$, giving $\frac{3}{2} \rho MMS_p$.

Summing up, agents other than $p$ take a total marginal $v_p$ value of at most $\frac{n}{2}\cdot 2\rho MMS_p + (\frac{n}{2}-1)\cdot \frac{3}{2} \rho MMS_p = (\frac{7n}{4} - \frac{3}{2}) \rho MMS_p$. Hence from at least one of the bundles $B$ of the MMS partition of $p$, the total marginal value taken by other agents is at most $(\frac{7}{4} - \frac{3}{2n}) \rho MMS_p$. As the bidding game ended with $p$ spending strictly less than $\rho b_p = \rho MMS_p$, no items left in $B$ have any marginal value for $p$. Denoting the bundle that $p$ receives by $C$, this implies that $v_p(B\mid C)\leq (\frac{7}{4} - \frac{3}{2n})\rho MMS_p$. Hence. we have that $MMS_p \le v_p(B) \le v_p(B\mid C) + v_p(C) \le (\frac{7}{4} - \frac{3}{2n})\rho MMS_p + v_p(C)$. For $\rho = \frac{4}{11}$, the assumption that $v_p(C) < \rho MMS_p$ leads to a contradiction in the above inequality, thus proving that $v_p(C) \ge \rho MMS_p$, as desired.
\end{proof}

Observe that due to the slackness factor of $\frac{3}{2n}$ in the last paragraph of the proof of Proposition~\ref{pro:4/11}, we can adapt the proof to get a slightly better bound for $\rho$, of the order of $\frac{4}{11} + \Theta(\frac{1}{n})$. Though this slackness term does not substantially change the approximation ratio, it does prove useful for designing a polynomial time algorithm that outputs an allocation in which each agent receives at least a $\frac{4}{11}$ fraction of her APS. See more details in Section~\ref{sec:polynomial}.




Having seen the proof of Proposition~\ref{pro:4/11}, let us explain the source of improvement that leads to the proof of Theorem~\ref{thm:equal-10/27}. The $\frac{4}{11}$ approximation ratio (rather than a better one) comes from the possibility that agents other than $p$ take $2 \cdot \frac{n}{2}$ items of value $\rho MMS_p$, and $3 \cdot (\frac{n}{2}-1)$ items of value $\frac{1}{2}\rho MMS_p$, for a total value of nearly $\frac{7}{4} \rho MMS_p$. However, in this case the other agents take fewer than $3n$ items, implying that in at least one of the bundles of the MMS partition of $p$, they take at most two items, of values $\rho MMS_p$ and ${\frac{1}{2}\rho} MMS_p$ (recall that we may assume that no two items of value $\rho MMS_p$ are in the same bundle of $p$'s MMS partition). Hence one of these bundles still has value of $MMS_p - \frac{3}{2}{\rho}MMS_p$. The assumption that $p$ gets a value of at most $\rho MMS_p$ then implies {that} $MMS_p \le \frac{5}{2}\rho MMS_p$, implying that $\rho \ge \frac{2}{5}$.



The other agents may do damage to $p$ in a different way. $\frac{n}{2}$ agents might each take two items of value $\rho MMS_p$, $\frac{n}{3}$ agents might each take three items of value $\frac{1}{2}\rho MMS_p$, and $\frac{n}{6} - 1$ agents might each take six items of value $\frac{1}{5}\rho MMS_p$. Ignoring the missing one agent (that is important, but is ignored only for the sake of the argument), this allows the other agents to take three items from each MMS bundle, of values $\rho MMS_p$, $\frac{1}{2}\rho MMS_p$ and $\frac{1}{5}\rho MMS_p$. This leads to the inequality $MMS_p \le \frac{27}{10}\rho MMS_p$, implying that $\rho \ge \frac{10}{27}$. The proof of Theorem~\ref{thm:equal-10/27} shows that this is the most damage that the other agents can do. 
\bigskip
%\ufc{For COMSOC, rest of proof can be moved to appendix.}

We now present rigorous proofs for two claims that were used in the proof of \Cref{pro:4/11}, and will also be used in the proof of \Cref{thm:equal-10/27}. These claims imply that we can assume without loss of generality that there are no agents of type $X_0$, and at most $\frac{n}{2}$ agents of type $X_1$. For both claims, we present their proofs under a framework in which we prove by induction on $n$ (the number of players) the statement that the proportional bidding strategy guarantees a $\rho$ fraction of $MMS_{p}$ to agent $p$. The statement is clearly true for the base case of $n = 1$. Hence for a given value of $n > 1$, we just prove the inductive step (assuming that the statement has already been proved for all $n' < n$). To simplify the presentation of the proofs of the claims, we make two assumptions. We stress that these assumptions do not affect the correctness of the claims. (Alternatively, these assumptions can be turned into facts, by simply incorporating them as part of the description of the proportional bidding strategy.)


\begin{itemize}

\item Agent $p$ truncates her valuation function at $t=MMS_{p}$, (i.e.,
$v_{p}\leftarrow v_{p}^{t}$).

\item Agent $p$ fixes some order over $\mathcal{M}$ in order to break
ties consistently. If she wins a round and there is more than one item with the highest marginal value, she will break the tie by picking the item appearing earlier in this order.
%\ufc{So as to be able to compare bundles received in different runs}

\end{itemize}


\begin{claim}
\label{claim: no X_0 agents}
In the inductive framework presented above, suppose that there is an allocation instance $I$ with $n$ agents, and that agent $p$ has a submodular valuation function $v_p$ and uses the proportional bidding strategy. If in the run of the bidding game there is at least one agent of type $X_{0}$, then agent $p$ is guaranteed to receive a bundle of value at least $\rho MMS_p$.
\end{claim}

\begin{proof}
Recall that by our assumption, the original valuation function $v_p$ is truncated so that no bundle has value larger than $MMS_p$. Consider a run $R$ of the bidding game, in which agent $p$ uses the proportional bidding strategy with an order 
%\gbc{the notation of $O$ will be related soon to the set of items taken by \textbf{other} agents, therefore consider the notation $\Pi$ for permutation/order}. \gbc{Removed:$O$}
$\Pi$ over the items $\items$.
Suppose that in this run agent $i$ is of type $X_0$. Let $e$ denote the single item that agent $i$ takes, and let $r$ denote the round number in which $i$ took item $e$. Suppose for the sake of contradiction that in this run $R$ agent $p$ receives a bundle of value strictly smaller than $\rho MMS_p$. Then we show a new allocation instance $I'$ with only $n-1$ agents, and a run $R'$ in which an agent $p'$ that uses the proportional bidding strategy gets value smaller than $\rho {MMS_{p'}}$. This contradicts our induction hypothesis. 

The instance $I'$ contains $n-1$ agents and the set $\items' = \items \setminus \{e\}$ of items. The valuation $v_{p'}$ is identical to $v_p$, though defined only over $\items'$ (for every $S \subseteq \items'$ we have that $v_{p'}(S) = v_p(S)$). Note that even though there are $n-1$ agents, $MMS_{p'} = MMS_p$. (The MMS partition for $v_p$, after dropping the bundle containing $e$, can serve as an MMS partition for $v_{p'}$, showing that $MMS_{p'} \ge MMS_p$. Being truncated at $MMS_p$, we have that $MMS_{p'} \le MMS_p$.) The order {$\Pi'$} used by $p'$ over $\items'$ is identical to {$\Pi$} (but with item $e$ removed). The run $R'$ is identical to $R$, except for the following four changes:
\begin{itemize}
    \item Agent $i$ and her bids are removed.
    \item Agent $p'$ plays in $R'$ the role that agent $p$ played in $R$.
    \item Round $r$ (the one in which item $e$ was taken) is removed.
    \item All bids are scaled by $\frac{n-1}{n}$ (as the budgets of agents are $\frac{1}{n-1}$ rather than $\frac{1}{n}$)
\end{itemize} 
Consequently, the sequence of bids and item choices that $p'$ makes in $R'$, being derived from the bids of $p$ in $R$, is consistent with $p'$ using the proportional strategy. The bundle received by $p'$ in $R'$ is exactly the same bundle that $p$ received in $R$, and hence of value below $\rho {MMS_{p'}}$. This contradicts our induction hypothesis. 
%Now, consider the adversarial bids of other agents where agent $i$ gives a bid of her entire budget and takes item $e$, and the rest of the other agents raise the same bids as in the run in the original instance. Then each agent will have the same bundle at the end of the bidding game in both instances. In the latter instance, by removing agent $i$ and item $e$, we obtain a new instance with $n-1$ agents, and the $MMS$ value of $p$ stays the same (take the rest of $n-1$ MMS bundles do not contain $e$ in the original $MMS$ partition, with the rest of the items from $e$'s bundle spread arbitrarily, this partition for $n-1$ bundles serves as a witness for the $MMS$ value of $p$ stays the same). By induction, the proportional strategy guarantees $p$ a value of $\rho MMS_{p}$, and we handled this case.
\end{proof}







\begin{claim}
\label{claim: at most n/2 X_1  agents}
In the inductive framework presented above, suppose that there is an allocation instance $I$ with $n$ agents, and that agent $p$ has a submodular valuation function $v_p$ and uses the proportional bidding strategy. If in the run of the bidding game, more than $\frac{n}{2}$ agents are of type $X_{1}$, then agent $p$ is guaranteed to receive a bundle of value at least $\rho MMS_p$.
\end{claim}



\begin{proof}
Consider a run $R$ of the bidding game, in which agent $p$ uses the proportional bidding strategy with an order $\Pi$ over the items $\mathcal{M}$. Suppose that in this run, more than $\frac{n}{2}$ agents are of type $X_1$. Suppose for the sake of contradiction that in this run $R$ agent $p$ receives a bundle of value strictly smaller than $\rho MMS_p$. Then we show a new instance $I'$ with only $n-1$ agents, and a run $R'$ in which agent $p'$ that uses the proportional bidding strategy gets a value strictly smaller than $\rho MMS_{p'}$. This contradicts our induction hypothesis.


Consider an MMS partition $\{B_i\}_{i=1}^n$ with respect to $v_p$. Since there are more than $\frac{n}{2}$ agents of type $X_1$, there are at least $n+1$ items taken by these agents. Hence, there exists a bundle $B_i$ in the MMS partition that contains at least two of these items. Denote these items by $e_1$ and $e_2$, the agents who win them by $a_1$ and $a_2$, and the rounds in $R$ in which these items were taken by $r_1$ and $r_2$. 

The instance $I'$ contains $n-1$ agents and the set $\mathcal{M}'=\mathcal{M}\setminus\{e_1,e_2\}$ of items. The valuation $v_p'$ is identical to $v_p$, though defined only over $\mathcal{M}'$ (for every $S\subseteq\mathcal{M'}$ we have that $v_p'(S)=v_p(S)$). Note that even though there are $n-1$ agents, $MMS_{p'}=MMS_p$. ($\{B_i\}_{i=1}^n$, the MMS partition for $p$, can serve as an MMS partition for $p'$, after dropping the bundle containing $e_1$ and $e_2$, showing that $MMS_{p'}\geq MMS_p$. Being truncated at $MMS_p$, we have that $MMS_{p'}\leq MMS_p$). The order $\Pi'$ used by $p'$ over $\mathcal{M}'$ is identical to $\Pi$ (but with items $e_1,e_2$ removed). The run $R'$ is identical to $R$, except for the following changes:

\begin{itemize}

    \item Rounds $r_1$ and $r_2$ (the rounds in which items $e_1, e_2$ were taken) are removed.
    
    \item Agent $p'$ plays in $R'$ the role that agent $p$ played in $R$.

    \item Agent $a_1$ and her bids are removed. 
    
    \item If $a_1\neq a_2$ (i.e $e_1$ and $e_2$ were taken by different agents) then let $e_3$ denote the other item taken by $a_1$ in $R$, and let $r_3$ denote the round in which it was taken. In $R'$ 
    agent $a_2$ takes item $e_3$ in round $r_3$ (instead of $e_2$ that $a_2$ took in the run $R$ - recall that $e_2 \not\in \items'$).
    
    \item All bids are scaled by $\frac{n-1}{n}$ (as the budgets of agents are $\frac{1}{n-1}$ rather than $\frac{1}{n}$)

\end{itemize}

Consequently, the sequence of bids and item choices that $p'$ makes in $R'$, being derived from the bids of $p$ in $R$, is consistent with $p'$ using the proportional strategy. The bundle received by $p'$ in $R'$ is exactly the same bundle that $p$ received in $R$, and hence of value below $\rho MMS_{p'}$. This contradicts our induction hypothesis.
% Consider an agent $i$ who takes two items. We can assume each item is of value exactly $\rho MMS_{p}$. If it is not the case, consider the same instance but with the value of these two items raised to exactly $\rho MMS_{p}$ (more precisely for $S\mathcal{\subseteq M}$ includes this item, the new value of $S$ will be $\min\{MMS_{p},v_{p}(S)\}$). In this new instance, $p$ has the same $MMS$ (it can not go up due to the truncation step of $v_{p}$), and since there are no $X_{0}$ agents (\Cref{claim: no X_0 agents}), if there is an item worth more than $\rho MMS_{p}$, agent $p$ wins it and satisfied. Otherwise, these two items are with the highest value for $p$. Therefore, if agent $i$ gives two bids of $\rho$ in the very first two rounds, by adversarially tie-breaking of the bidding game, she can get these two items with value $\rho MMS_{p}$ each, and the same run of the original instance is a legal run of the bidding game in the new instance. At the end of the run, each agent will have the exact same items as in the original instance. Since we did not increase the MMS of $p$ in the new instance, she gets the same fraction of her $MMS_{p}$ at the end of the game in both instances.
% So far, we have shown that we can assume w.l.o.g, each agent of type $X_{1}$ wins items of value $\rho MMS_{p}$ each. Together with \Cref{claim: no X_0 agents} states that there are no agents of type $X_{0}$, we get that all the $X_{1}$ agents win their items at the beginning of the game (in consecutive steps). Thus, it is clear that the internal allocation of items of $X_{1}$ to the agents of $X_{1}$ does not affect the final bundle $p$ will have at the end of the algorithm. Hence we can assume w.l.o.g that if two items of $X_{1}$ agents have been taken from a bundle of the MMS partition of $p$, then they have been taken by the same agent and at the very first two rounds. Now by induction (similar to \Cref{claim: no X_0 agents}), in the instance without this agent and the two items, agent $p$ gets a bundle of value at least $\rho MMS_{p}$ as we wanted to show. 
\end{proof}

%\ufc{Also here you need to consider separately the case that there are items of value larger than $2\rho \cdot MMS$.}

%\gbc{
%The exact same proof as \cref{claim:APS not decrease in I_s} holds here, except the last argument - in \cref{claim:APS not decrease in I_s} we use \Cref{cl:item_reduce} to state that the $APS$ of agent $p$ in the instance $I_{\hat{s}}$ did not reduced. In the proof of this \Cref{thm:equal-10/27} our guarantee is based on the $MMS$. Hence, the argument should be that the $MMS$ of agent $p$ did not reduce in $I_{\hat{s}}$. This is because if we reduce an item and an agent from the instance, the original $MMS$ partition induces a partition in $I_{\hat{s}}$ for the remained agents, and the value of the minimal bundle did not reduce. (i.e., removing an agent and an item does not reduce the $MMS$ of the remained agents)}

%\ufc{Proof of what}
{We are now ready to prove \Cref{thm:equal-10/27}.}

\begin{proof}
Consider an arbitrary allocation instance with equal entitlement, and an arbitrary agent $p$ with a submodular valuation function $v_p$. We wish to show that by using the proportional strategy in the $\rho$-altruistic version of the bidding game (with a choice of $\rho = \frac{10}{27}$), $p$ receives a bundle of value at least $\frac{10}{27} \cdot MMS_p$. Equivalently (by scaling $v_p$ so that $MMS_p = \frac{27}{10}$), we wish to show that if $MMS_p = 2.7$, then $p$ receives a bundle of value at least~1. 

We begin by presenting a claim that will be used later.
Let $\left\{ B_{i}\right\} _{i=1}^{n}$ be an MMS partition of agent $p$, i.e., for each $i$ $v_{p}(B_{i})\geq MMS_{p}$. Let $f$ be the earliest round after which no other agents are active. Let $C$ denote the bundle $p$ has by the end of round $f$, and let $O\subseteq\mathcal{M}$ be the set of items taken by other agents.
The following claim is similar in nature to \Cref{Claim_second_bound_Lf}

\begin{claim}
\label{claim: No large value at each MMS bundle}
In the altruistic version of the bidding game, for each $i\in[n]$, the final value that $p$ has is at least 
\[\min\left\{ \rho MMS_{p},v_{p}(C)+v_{p}(B_{i}\setminus O\mid C)\right\} = \min\left\{ \rho MMS_{p},v_{p}(C
\cup\left\{  B_{i}\setminus O\right\})\right\}\]
\end{claim}

\begin{proof}
If $p$ is not active at the end of round $f$, then $p$ has a value of at least $\rho MMS_{p}$ at that time, i.e., $v_{p}(C)\geq\rho APS_{p}$, as desired. 

If $p$ is active at the end of round $f$, we consider two cases. If some items remain after round $f$, then agent $p$ is the only remaining active agent. Hence $p$ is the only agent to win items from $\mathcal{M}$. Then, the agent will keep winning items until either she becomes inactive (and has value at least $\rho MMS_{p}$) or until she wins all remaining items from $B_{i}\setminus O$. It remains to handle the case of agent $p$ being active after time $f$ while no items remain. In this case, $v_{p}(B_{i}\setminus O)=v_{p}(\emptyset)=0$, so the bound is trivial. 
\end{proof}

%\gbc{Removed: We set up a system of linear inequalities in which the variable $b$ expresses the $v_p$ value of the bundle that $p$ receives, and variable $z$ expresses the value of $MMS_p$ (hence we shall have the constraint $z = 2.7$). The linear inequalities encode various constraints on the $v_p$ value of items that agents other than $p$ receive, given that $p$ is using the proportional strategy and ends up with a value of $b$. We show that the system of inequalities is feasible only if $b \ge 1$, proving the lemma. }

%such that if any of the constraints is violated, the bundle received by $p$ under the $\rho$-proportional strategy necessarily has value strictly larger than~1.  Hence for the optimal value of $z$, we have that  $\alpha \ge \frac{1}{z}$. We show that by choosing $\rho = \frac{10}{27}$ in the $\rho$-proportional strategy, the constraints of the LP imply that $z \le \frac{27}{10}$, establishing that for this strategy $\alpha \ge \frac{10}{27}$, and proving the lemma.




 
Every negative example (i.e., an instance of the problem in which agent $p$ does not get at least $\rho MMS_p$) implies (by scaling the valuation function of the agent) the existence of another instance of the problem, in which $MMS_p = \frac{1}{\rho}$ and $p$ gets a bundle of value less than 1. Denote the $MMS$ of agent $p$ by $z$.
%and her MMS (which denoted $z$) is $\frac{1}{\rho}$.
%By scaling her valuation function, we can assume that $z=\frac{1}{\rho}$.
We set up a system of linear inequalities. The linear inequalities encode various constraints on the $v_p$ value of items that agents other than $p$ receive, given that $p$ is using the proportional strategy and ends up with a value of at most $1$. We show that the system of inequalities is feasible only if $z \le 2.7$. {This implies that %the fact that if $z \geq 2.7$ $p$ gets at least $1$ and having a $\rho$ guarantee, and 
$\rho \geq \frac{10}{27}$}. 

%\gbc{Removed: We claim that if $p$ uses the $\rho$-proportional strategy (with $\rho = \frac{10}{27}$), she gets a bundle of value at least~1 (and hence, at least her $\rho$-MMS).} 
% \gbe{Note that throughout the proof, we assume that there is no item $e$ which satisfies $v_p(e)\geq 2\cdot MMS_p$. Notice that the assumption does not harm loss of generality, since an existance of a negative instance on which  }

% The Linear Program's solution, which maximizes $z$, has a value of $2.7$. In order to show that this proves the
% lemma, assume by contradiction there exists a negative example, for $\rho'<\frac{10}{27}$. Then by the above argument, we will have another negative example in which the unsatisfied agent gets a value of at most $1$ and her MMS is $z=\frac{1}{\rho'}>2.7$. Now, this instance induces a feasible solution to the Linear Program with an objective value of $\frac{1}{\rho'}>2.7$, contradicting that the maximum is
% $2.7$. }



To simplify the presentation of the system of inequalities, we shall have a slight abuse of terminology. The term {\em payment} of an agent (say, for an item in round $r$) will correspond to the bid of agent $p$ (in round $r$), even though the actual payment might be larger (if the bid of the winning agent was strictly higher than the bid of agent $p$). With this abuse of notation, the sequence of payments that an agent makes is nonincreasing. This implies that the total payment of an agent that wins $t > 1$ items is at most $\frac{t}{t-1}$, because otherwise the agent spent more than~1 on the first $t-1$ items that she picked, and would become inactive before winning its $t$th item. Observe that the total payment of $p$ can be assumed to be not more than~1, as otherwise, the bundle that $p$ receives according to the proportional strategy has $v_p$ value at least~1, and we are done.


Consider a negative example for some $\rho$, i.e., an instance in which an agent does not get a $\rho$ fraction of her MMS, and she gets a value of at most $1$. Keeping this instance in mind, we will present our set of inequalities, and explain why the instance respects each of them.


The system of inequalities has nonnegative variables, $x_{1}, x_{2}, x_{3}, x_{4}, y, q, z$. For each $1 \le i \le 4$, $x_{i}$ represents the fraction of agents who satisfy the following two conditions:
\begin{enumerate}
\item The agent takes $i+1$ items.
\item The total payments that the agent made is at least $\frac{6}{5}$.
\end{enumerate}

The reason why we do not introduce a variable $x_0$ (for agents who take one item and pay at least $\frac{6}{5}$) is because Claim~\ref{claim: no X_0 agents} implies that we can assume that there are no such agents. This also implies that a payment for an item is never larger than~1.
Variable $y$ represents the fraction of the rest of the agents, those that either take at least six items or paid at most $\frac{6}{5}$. Observe that in either case, any such agent paid a total of at most $\frac{6}{5}$. 
Variable $z$ represents the MMS of agent $p$ (and recall that we scale the valuation $v_p$ so that $MMS_p = \frac{1}{\rho}$).
%\ufc{remove $a$ and use $v_p(C)$?} The variable $a$ represents the value that agent $p$ holds after round $f$. \ufe{Observe that if $a \ge 1$, then $p$ gets a bundle of value at least $\rho MMS_p$.}
%\gbc{Remove?: Notice that if at this time agent $p$ itself is also not active, then $p$ already holds a value of at least~1.} 
The variable $q$ needs a more detailed explanation. For every $0\le s\le\frac{1}{2}$, let $\alpha_{s}$ denote the fraction of agents that satisfy the following conditions:

\begin{enumerate}
\item The agent takes at least three items.
\item The total payments made by the agent is larger than $\frac{6}{5}$.
\item Her payment for the first item that she takes is $\frac{1}{2}+s$. 
\end{enumerate}

Note that this implies that the number of items that such an agent takes is either three, in which case her total payments are at most $\frac{3}{2}-s$, or four, in which case her total payments are at most $\frac{5}{4} - \frac{s}{2}$, which is smaller than $\frac{4}{3}-s$ for $s < \frac{1}{6}$ (note that if $s \ge \frac{1}{6}$ then her total payments when taking four items are at most $\frac{6}{5}$, and hence this case is excluded). The variable $q$ represents $\int_{0}^{\frac{1}{2}}s\cdot\alpha_{s}ds$.

%\gbe{Before we continue, we state an essential observation here. 
%Consider the MMS partition of agent $p$. In each step of the algorithm, assuming that agent $p$ is still active (this assumption is made without loss of generality, as otherwise she already got a value of at least $\rho \cdot MMS_p$), her bid is equal to the maximal marginal value among the remaining items. Therefore, if an agent other than $p$ wins the bid, whatever item $e$ she picks, she pays at least the marginal value that $e$ has to $p$.
%\gbc{Should we remove the two previous sentences? It is explained in the beginning of the proof when presenting the notion of \emph{payments} and why it is a slight abuse of notation} \ufc{Please write and explain this properly: Thus, if we look at a bundle $B_{i}$ from the MMS partition of agent $p$, and we denote by $B\subseteq B_{i}$ as the items other agents win throughout the algorithm, the total payments which other agents paid for items $B$ are at least as the marginal value of $B$ to agent $p$, w.r.t the items she won until all agents become inactive.} \gbc{Uri, I explained this properly in \Cref{claim: No large value at each MMS bundle}} \gbc{remove the next text in blue? We present the notation of time $f$, bundle $C\mathcal{M}$ at the beginning of the proof, before \Cref{claim: No large value at each MMS bundle}}\gbe{More formally, as in \Cref{thm:1/3_APS_guarantee} proof, we denote $f$ to be the earliest round, after which either all other agents become inactive or all items have been allocated. In addition, we denote $C\subseteq\mathcal{M}$ to be the set of items agent $p$ has by the end of round $f$. In a similar way to  \Cref{claim_other_agent_items_reducing_Lf} proof, we obtain that any item $e$ won by another agent $i$ she pays at least $v_p(e\mid C)$. Consider $\{B_i\}_i$ to be an $MMS_p$ partition of agent $p$. Then, the payment of other agents who won a bundle $B\subseteq B_i$ is at least $\sum_{e\in B}v_p(e\mid C)\geq v_p(B\mid C)$}}

{We turn to present the linear inequalities, and explain why each of these inequalities must hold if $p$ executes the proportional bidding strategy}

\begin{enumerate}

\item $x_{1}+x_{2}+x_{3}+x_{4}+y \le 1 - \frac{1}{n}$. The variables $x_1, x_2, x_3, x_4, y$ represent fractions of the total number of agents, and agent $p$ is not included in these fractions. As the total number of agents is $n$, the sum of the fractions is at most $1 - \frac{1}{n}$.


{\item $z-2x_{1}-(3/2)x_{2}-(4/3)x_{3}-(5/4)x_{4}-(6/5)y+q \leq 1$. Fix
$\left\{ B_{i}\right\} _{i=1}^{n}$, an MMS partition for agent $p$.
Since we assume $p$ is active at the end of the algorithm, \Cref{claim: No large value at each MMS bundle} implies that the final value of $p$ is at least $v_{p}(C) + v_{p}\left\{ B_{i}\setminus O\mid C\right\} )$. %\ufc{remove next sentence?}\gbc{This is a justification for the first inequality in the coming chain of inequalities.} 
Since the final value of $p$ is at most $1$, we obtain $1\geq v_{p}(C) + v_{p}\left\{ B_{i}\setminus O\mid C\right\} )$

Recall that $O\mathcal{\subseteq M}$ is the set of items taken by other agents. Consider the following partition of $O$ , $\left\{ O_{i}=B_{i}\cap O\right\} _{i=1}^{n}$. Then, for each $i$, 
%\ufc{More simply: $v_p(C) + v_p(O_i|C) = v_p(B_i) \ge z$ implies that $v_p(C) \ge z - v_p(O_i|C)$.} \gbc{I believe your statement is not true, since $v_p(C) + v_p(O_i|C) \leq v_p(B_i)$ But the inequality might be strict. Recall that $C$ is the bundle $p$ has by round $f$ which is not necessarily the end of the bidding game. I.e., $(C\cap B_i)\cup O_i\subseteq B_i$ but the inclusion might be strict.}
\begin{align*}
1 & \geq v_{p}(C)+v_{p}\left\{ B_{i}\setminus O\mid C\right\} )\\
 & =v_{p}(C)+v_p(B_{i}\setminus O_{i}\mid C)\\
 & \geq v_{p}(C)+v_{p}(B_{i}\mid C)-v_{p}(O_{i}\mid C)\\
 & =v_{p}(C\cup B_{i})-v_{p}(O_{i}\mid C)\\
 & \geq v_{p}(B_{i})-v_{p}(O_{i}\mid C)\\
 & =z-v_{p}(O_{i}\mid C)
\end{align*}


Denote the total payments of agents other than $p$ as $P_O$. Notice that $\sum_{i}v_{p}(O_{i}\mid C)\leq\sum_{i}\sum_{e\in O_i} v_{p}(e\mid C)\leq P_O$.
 Hence, there exists $i\in[n]$ for which $v_{p}(O_{i}\mid C)\leq\frac{1}{n}\cdot P_O$.
 
We now upper bound $\frac{1}{n}\cdot P_O$. Recall that the total payment of an agent that takes $t > 1$ items is at most $\frac{t}{t-1}$. Moreover, if agents of type $x_{2}$ and $x_{3}$  have a payment strictly larger than $\frac{1}{2}$ for their first item, then they do not reach the maximum payment they can achieve ($\frac{3}{2}$ and $\frac{4}{3}$). By the definition of $q$, we obtain that their total payment is reduced by at least $q$. (Recall the discussion that follows the definition of $q$. It shows that if an agent of type $x_{2}$ pays $\frac{1}{2}+s$ for her first item, then the maximum payment she might reach after taking her three items is at most $\frac{3}{2}-s$. Likewise, it shows that if an agent of type $x_{3}$ pays $\frac{1}{2}+s$ for her first item, then the maximum payment she might reach after taking her four items is at
most $\frac{4}{3}-s$.) Therefore we have
\[
\frac{1}{n}\cdot P_O {\leq}2x_{1}+(3/2)x_{2}+(4/3)x_{3}+(5/4)x_{4}+(6/5)y-q
\]
The constraint follows by using this last inequality and the two previous inequalities $1 \geq z-v_{p}(O_{i}\mid C)$ and $v_{p}(O_{i}\mid C) \le \frac{1}{n}P_0$.
%\gbc{Removed:, and the requirement that $v_p(C) \le 1$.}
% \gbc{Consider the above constraint instead of this and Remove:\\ $-2x_{1}-(3/2)x_{2}-(4/3)x_{3}-(5/4)x_{4}-(6/5)y+(2/3)q-a+z\le1-a$.
% $\quad$ Consider the time when all other agents are inactive. Then,
% every bundle $B_{i}$ of the MMS partition of the first agent will
% have less than $1-a$ marginal-value w.r.t the items the first agent
% already won throughout the algorithm (with a total value of $a$).
% Otherwise, the agent will take one of the remaining bundles, ensuring
% herself a value of 1. So z is upper bounded by the sum of payments
% paid by agents, plus 1. In order to upper bound the payments
% of agents, recall that the bidding sequence of each agent is decreasing.
% Combining this with the fact that an agent becomes inactive when surpassing
% a payment of 1, we have that agent of type $x_{1}$ pays at most
% 2 , agent of type $x_{2}$ pays at most $\frac{3}{2}$, etc. Now,
% if agents of type $x_{2}$ and $x_{3}$ pay for their first item,
% a payment strictly more than $\frac{1}{2}$, then they do not reach
% the maximum payment they can achieve (2, and $\frac{3}{2}$). By the
% definition of $q$ we obtain that the amount of payment reduced is
% at least $\frac{2}{3}q$ (if an agent of type $x_{2}$ pays $\frac{1}{2}+s$
% in her first item, then the maximum payment she might have after taking
% her three items is at most $\frac{3}{2}-s$. Likewise, an agent of
% type $x_{3}$ who pays for an item a payment of $\frac{1}{2}+s$ in
% the first item will have reached a total payment of at most $\frac{4}{3}-\frac{2}{3}s$).
% Therefore we have $z\le2x_{1}+(3/2)x_{2}+(4/3)x_{3}+(5/4)x_{4}+(6/5)y-\frac{2}{3}q+a+1-a$.
% By rearranging, we obtain the above constraint.}
\item $2x_{1} \le 1$. 
% \gbc{Removed:We can assume that the instance is reduced and is a
% minimal negative example in the number of agents. We claim that in the reduced instance, at most half the agents pick two items. Otherwise, there is a bundle in which two items are picked. Remove it and one of the agents, and we get a smaller negative example.}
We can assume that at most half of the agents are of type $x_1$, by \Cref{claim: at most n/2 X_1  agents}.

{
\item $\frac{6}{5}y+q \geq (z-1-\frac{3}{2})\cdot\left(3-2x_{1}-3x_{2}-4x_{3}-5x_{4}\right)$. {(This is not a linear inequality, but it will be linearized before it will be used.)}
% $-(2/5)x_{1}-(3/5)x_{2}-(4/5)x_{3}-x_{4}-(6/5)y-q\le-(3/5)$. 
We refer to items taken by agents represented by $x_{1},x_{2},x_{3},x_{4}$ as \emph{primary}, and to items taken by agents represented by $y$ as \emph{secondary}. How much payment do secondary items need to have so that payment of at least $z-1$ is paid for items in each bundle $B_i$ of the MMS partition (which is a necessary condition for $p$ having a total value of at most $1$ \Cref{claim: No large value at each MMS bundle})? The total number of primary items is $(2x_{1}+3x_{2}+4x_{3}+5x_{4})n$. We split the argument into two cases.

In the first case, there are at most $3n$ \emph{primary} items. The number of primary items missing to complete this number to $3n$ is $(3-2x_{1}-3x_{2}-4x_{3}-5x_{4})n$. {We refer to each such missing item as a \emph{deficiency unit}.} As noted above, we assume no two primary items of agents of type $x_{1}$ are taken from a single bundle. 
{We analyze the distribution of \emph{deficiency units} over the MMS bundles when a bundle with $\ell\leq3$ \emph{primary} items has $3-\ell$ \emph{deficiencies}. Then, we will find properties of the distribution of \emph{deficiencies} that minimizes the amount of payment to be paid by $y$ agents to reach $z-1$ payment in each MMS-bundle.

\begin{itemize}

    %\item A bundle $B_j$ in the MMS-partition with 0 deficiencies has three \emph{primary} items. In these three \emph{primary} items, there is at most one primary item of type $X_1$ agent, with a payment of $~1$. The remaining two primary items are valued at most $\frac{1}{2}+s$ each. Summing these three payments together, we obtain a total payment larger than $z-1$ for $z\leq3$. Thus, there is no guaranteed payment that needs to be paid by $y$ agents in such bundle $B_j$ with 0 \emph{deficiency units}
    %\gbc{Uri,  I think we should consider keeping the above bullet, analyzing a bundle of 0 unit-deficiencies. In every other bullet, we take the amount of payment needed to complete to $z-1$ and divide it by the amount deficiency units. In a sense, this bullet shows that in our analysis, the deficiency units indeed responsible for most of the payment should be paid by $y$ agents, so a bundle with no deficiencies might not have payments of $y$ agents.}
    %\ufc{I recommend removing this bullet. Otherwise, it needs serious rewriting. Each how can it have two primary items of payment $\frac{1}{2} + s$?}
    %\gbc{If you recommend removing this bullet, I will remove it. However, a bundle $B_j$ might have two primary items with payments $\frac{1}{2}+s_1$ and $\frac{1}{2}+s_2$ if they have been taken by different agents of type (Of type $x_i$ for $i>1$)}

    \item A bundle $B_j$ in the MMS partition with one \emph{deficiency unit}, has a payment for \emph{primary} items of at most $1+\frac{1}{2}+s$.  At least $z-1-\frac{3}{2}-s$ needs to be paid  by $y$ agents in order to surpass a $z-1$ payment in $B_j$. I.e., a $z-1-\frac{3}{2}-s$ for the one unit of deficiency.

    \item A bundle $B_j$ in the MMS partition with two \emph{deficiency units} has a payment for \emph{primary} items of at most $1$. At least $z-1-1$ needs to be paid by $y$ agents to surpass a $z-1$ payment in $B_j$. I.e., a $\frac{z-2}{2}$ on average per \emph{deficiency unit}.

    \item A bundle with three \emph{deficiency units}, needs a payment of at least $z-1$ by $y$ agents. I.e., a $\frac{z-1}{3}$ on average for per deficiency unit.
    \end{itemize}
    
    For $z\leq 3$, the minimum payment per deficiency occurs when every deficiency unit is in a different bundle (i.e., these bundles have one deficiency unit, the case of bullet two).}
%\ufc{impossible to understand next sentence} \gbc{Removed: The worst case, when there is the maximum payment to be paid by $y$ agents to reach $z-1$ payment in each bundle, is when each bundle is missing a primary item in a different bundle.} 
Then, in each bundle with two primary items, one item can have payment arbitrarily close to~1, and the other $\frac{1}{2}+s$, with $s$ as above. Hence, to reach $\sim(z-1)$, the bundle needs $\sim(z-1-\frac{3}{2})-s$. As integrating over \emph{all} $s$ we get $q$, the above considerations give the constraint $(6/5)y\ge(z-1-\frac{3}{2})\cdot(3-2x_{1}-3x_{2}-4x_{3}-5x_{4})-q$ as desired. 

In the second case, the number of {primary} items is greater than $3n$. {If $z > 2.5$, then the right-hand side of constraint~4 is a product of a positive scalar and a negative scalar, resulting in a negative scalar. By non-negativity of variables $q$ and $y$, the left-hand side of the constraint is a non-negative scalar. Hence constraint~4 is valid for the range of values of $z$ which will be considered in the proof, which only includes values larger than $2.5$.}

%\ufc{remove: Here we observe that we may assume \ufc{not true for small $n$} that $z > 2.5$. This follows from \Cref{hard instance altruistic version} (setting $k=2$ it its proof), which implies that for $\rho \geq 0.4$, the $proportional(\rho)$ does not guarantee a $\rho$ fraction of the $MMS$. As $z$ is supposed to equal $\frac{1}{\rho}$, we get that the range of interest for $z$ is only $z > 2.5$. Thus, in case there are more than $3n$ primary items, the right-hand side of constraint~4 is a product of a positive scalar and a negative scalar, resulting in a negative scalar. By non-negativity of variables $q$ and $y$, we obtain that the left-hand side of the constraint is a non-negative scalar. Overall, in case there are more than $3n$ \emph{primary}, the constraint states that a non-negative scalar is greater than a non-positive scalar, which is always true.}
% Observe that if the number of primary items is at least $3n$, the constraint holds immediately since the right-hand side of the inequality is negative while the left-hand side is positive.
}
}
\end{enumerate}

{The fourth constraint is not a linear inequality. Nevertheless we may make use of the system of four constraints as if it is a system of linear inequalities. We do so by substituting candidate values for $z$ (these values are larger than $2.5$, so that the fourth constraint remains valid), simplifying the second and fourth constraints after making this substitution, and checking whether the resulting system of inequalities is feasible. If it is not feasible, this certifies that the substituted value for $z$ was too high, and hence we obtain an upper bound on $z$. Substituting $z = 2.7$, the constraints become:

\begin{enumerate}
\item $x_{1}+x_{2}+x_{3}+x_{4}+y \leq 1 - \frac{1}{n}$
\item $- 2x_{1}-\frac{3}{2}x_{2}-\frac{4}{3}x_{3}-\frac{5}{4}x_{4}-\frac{6}{5}y+q \leq -1.7$
\item $2x_{1} \leq 1$
\item $-\frac{2}{5}x_{1}-\frac{3}{5}x_{2}-\frac{4}{5}x_{3}-x_{4}-\frac{6}{5}y-q \leq -\frac{3}{5}$. 
\end{enumerate}

Summing up the four constraints multiplied by coefficients $(1.8,1,0.2,0.5)$ 
respectively, we obtain:

$$(\frac{2}{5} - \frac{1}{3})x_3 + \frac{x_4}{20} + \frac{q}{2}  \le -\frac{1.8}{n}$$

As $x_3$, $x_4$ and $q$ are non-negative, this is a contradiction. Hence $z < 2.7$. In fact, the term $-\frac{1.8}{n}$ on the right hand side implies that $z \le 2.7 - \Omega(\frac{1}{n})$, which in turn implies that $\rho \ge \frac{10}{27} + \Omega(\frac{1}{n})$.}
\end{proof} 
%\gbc{Discuss with Uri about this proof.
% I believe the mistake in the previous version was assuming $n>1$. In case $n=1$ constraint 4 is no longer true. When $n=1$, there are no other agents, thus $x_i, y, q$ must be zero, and the forth constraint is reduced to $0\geq(z-1-\frac{3}{2})\cdot(3-0)$}

%\ufc{Note to Gilad. There are edits that we are doing in the arxiv version that you need not transfer to the thesis. However, the edits done in this last proof will have to be transferred into the thesis, as without them the proof is not readable.}


%\gbe{Observe that due to the slackness factor of $\frac{1}{n}$ in the first constraint (namely, the other agent are $(1-\frac{1}{n})$ of all agents), we can adapt the proof to get a slightly better bound for $\rho$, of the order of $\frac{10}{27}+\Theta\left(\frac{1}{n}\right)$. Though this slackness term does not substantially change the approximation ratio, it does prove useful for designing a polynomial time algorithm that outputs an allocation in which each agent receives at least a $\frac{10}{27}$ fraction of her APS. See more details in Section \Cref{sec:polynomial}}\\

%\ufc{remove next proof after approving previous one}

%\begin{proof}
%Consider the LP over the nonnegative variables, $x_1, x_2, x_3. x_4, y, q, z$. The LP assumes that item values are scaled by $\frac{1}{\rho}$, so that an item that had value $\rho$ now has value~1. $x_i$ represents the fraction of agents who take $i+1$ items, and moreover, the total value of items that they take is greater than $\frac{6}{5}$. $y$ represents the fraction of agents who take items of total value at most $\frac{6}{5}$, which in particular includes all agents that take at least six items. $z$ represents $\frac{1}{\rho}$, so maximizing $z$ minimizes $\rho$. We shall prove that $z \le 2.7$, implying that $\rho \ge \frac{10}{27}$. The variable $q$ needs a more detailed explanation. For every $0 \le s \le \frac{1}{2}$, let $\alpha_s$ denote the fraction of agents that satisfy three conditions: the first item that she takes has value $\frac{1}{2} + s$, the total values of items that she takes is larger than $\frac{6}{5}$, and the number of items that she takes is at least three. Note that this implies that the number of items that such an agent takes is either three (and then the total value that she takes is at most $\frac{3}{2} - s$) or four (and then the total value that she takes is smaller than $\frac{4}{3} - s)$. The variable $q$ represents $\int_0^\frac{1}{2} s\cdot \alpha_s ds$.

%The LP maximizes $z$ under the following constraints (writing them to be in canonical form):

%\begin{enumerate}
    
%    \item $x1 + x2 + x3 + x4 + y \le 1$. (The fraction of agents that pick items is at most~1.)
    
%    \item $-2*x1 - (3/2)*x2 - (4/3)*x3 - (5/4)*x4 - (6/5)*y + q + z \le 1$. ($z$ is upper bounded by the sum of item values.)
    
 %   \item  $2*x1 \le 1$. (At most half the agents pick two items. Otherwise, there is a bundle in which two items are picked. Remove it and one of the agents, and we get a smaller negative example.)
    
 %   \item $-(2/5)*x1 - (3/5)*x2 - (4/5)*x3 - x4 - (6/5)*y5 - q \le -(3/5)$. (We refer to items taken by agents represented by $x_1, x_2, x_3, x_4$ as {\em primary}, and to items taken by agents represented by $y$ as {\em secondary}. How much value do secondary items need to have so that a value of at least~1.7 is taken from each bundle (which is a necessary condition in order to have $z \ge 2.7$)? The total number of primary items is $(2x_1 + 3x_2 + 4x_3 + 5x_4)n$. The number of primary items missing to complete this number to $3n$ is $(3 - 2x_1 - 3x_2 - 4x_3 - 5x_4)n$. The worst case is when each missing primary item is missing in a different bundle. Then in each bundle with two items, one item can have value~1, and the other value $\frac{1}{2}+s$, with $s$ as above. Hence to reach~1.7 the bundle needs~$0.2-s$. As integrating over {\em all} $s$ we get $q$, the above considerations give the constraint  $(6/5)*y5 \ge 0.2(3 - 2x_1 - 3x_2 - 4x_3 - 5x_4) - q$.) 
%\end{enumerate}

%Summing up the four constraints multiplied by coefficients \ufe{$(1.8, 1, 0.2, 0.5)$} respectively, and using nonnegativity of the variables, we obtain the desired inequality $z \le 2.7$.
%\end{proof}


%In the arbitrary entitlement case, it is no longer true that if we remove a single agent and a single item, the APS of other agents does not decrease. Due to this reason, we add the following preprocessing stage to the algorithm.


%Fix $\rho' = 2\rho^2$. The preprocessing stage guarantees each agent that becomes inactive at least a $\rho'$ fraction of her APS. Taking $\rho = \frac{1}{3} + \Omega(\frac{1}{n})$ we get $\rho' = \frac{2}{9} + \Omega(\frac{1}{n})$.


%Initially all agents are active. Sort the agents from highest entitlement to lowest entitlement. Scan the agents in this order. For each agent $i$, if there is no item of value $\rho' b_i$, move to the next agent. If there is an item $e$ with $v_i(e) \ge \rho' b_i$, then do the following. Let $j$ be the active agent among those that have higher entitlement than $i$ (that is, $j < i$) for which $e$ has highest value. 

%\begin{enumerate}

%\item If $v_j(e) \le 2\rho b_i$, then give item $e$ to agent $i$ and make agent $i$ inactive. For agents with lower entitlement than $i$, the APS does not decrease \gbe{(by Claim \ref{cl:item_reduce})}. For agents with higher entitlement than $i$, this does not harm the analysis in the proof of Lemma~\ref{lem:equal}.

%\item If $v_j(e) > 2\rho b_i$, then give $e$ to agent $j$, and decrease the budget of $j$ by $b_i$ \gbc{It might be that agent $j$ has a total budget less than $b_i$}. For agent $i$ and agents with lower entitlement than $i$, the APS does not decrease \gbe{(by Claim \ref{cl:item_reduce})}. For agents with higher entitlement than $i$, this does not harm the analysis in the proof of Lemma~\ref{lem:equal}. If agent $j$ spent already $\rho b_j$, then make agent $j$ inactive. Observe that in this case agent $j$ has value at least $\rho \cdot 2\rho b_j = \rho' b_j$.

%\end{enumerate}

%The parameters of the algorithm above can be improved to $\rho' = \frac{7 - \sqrt{33}}{4}$ and $\rho = \frac{\sqrt{33} - 3}{6}$. For simplicity of the computations, let us present the analysis for suboptimal parameters $\rho' = \frac{3}{10}$ and $\rho = \frac{3}{7}$. Observe that if there are no items of value $\rho' b_j$, then in the main part of the algorithm every agent $j\not= i$ that becomes inactive spends budget of at most $\max[2\rho', \frac{3}{2}\rho]b_j < (1 - \rho')b_j$ \gbe{(if the agent wins 2 items, she spends at most $2\rho'b_j$. If she wins 3 items or more, she spends at most $\frac{3}{2}\rho b_j$)} . Hence the last active agent $i$ can take items of value at least $\rho' b_i$, and become inactive as well. In the preprocessing stage change the test to $v_j(e) \le (1 - \rho') b_i$. Note that if agent $j$ spends $\rho b_j$, she gets a value of at least $(1 - \rho')\rho b_j = \rho' b_j$.


%Using the above analysis we can plug $\rho = 0.3$ (in fact, a slightly higher value of $\frac{7 - \sqrt{33}}{4}$) in Theorem~\ref{thm:APSsubmodular} (note that our $\rho'$ is the approximation ratio $\rho$). 

% \begin{remark}
% Future work can try to improve the approximation ratio for submodular valuations further, both for equal entitlement and for arbitrary entitlement. One may also consider XOS valuations, subaddtive valuations, and subclasses of submodular such as GS, budget additive or coverage.
% \end{remark}


\subsection{Polynomial time algorithms}
\label{sec:polynomial}

%\gbe{We begin by briefly reviewing our results concerning polynomial time implementations of our result.}

Theorems~\ref{thm:1/3_APS_guarantee} and~\ref{thm:equal-10/27} imply (among other things) the existence of allocations that give each agent with a submodular valuation a certain fraction of her APS (or MMS). However, they do not provide polynomial time algorithms to find such allocations, because they assume that the value of the APS (or MMS) is known (or can be computed by the agent), whereas computing this value is NP-hard. Nevertheless, by using a technique presented in \cite{DBLP:conf/sigecom/GhodsiHSSY18}, we obtain a polynomial time implementation, proving \Cref{cor:polyTimeSubmodular}. The basic idea is as follows. One runs the bidding game with all agents using our proposed proportional strategy, but each agent starts with an estimate for her MMS (or APS) that is higher than the true value. If all agents get the desired fraction of their estimated MMS, we are done. If not, then for those agents that get a fraction that is too small, we lower their estimate for their MMS by a factor of $(1 - \epsilon)$, and repeat the whole process. No agent will ever need to lower her estimate to below a $(1 - \epsilon)$ fraction of her true MMS. We now provide more details.





%\gbc{Removed: Theorems~\ref{thm:1/3_APS_guarantee} and~\ref{thm:equal-10/27} imply (among other things) the existence of allocations that give each agent with a submodular valuation a certain fraction of her APS (or MMS). However, they do not provide polynomal time algorithms to find such allocations, because they assume that the value of the APS (or MMS) is known, whereas computing this value is NP-hard. In this section we show that this assumption can be removed, and consequently that there are polynomial time allocation algorithms that output allocations that approximate the APS and MMS within ratios almost as good as those shown in Theorems~\ref{thm:1/3_APS_guarantee} and~\ref{thm:equal-10/27}. Our proof is based on a technique that was introduced in~\cite{DBLP:conf/sigecom/GhodsiHSSY18}. }
%with some minor modifications.
%\ufc{remove: After showing existential results of approximate $MMS/APS$ allocations, we turn to show a polynomial algorithm that finds those allocations.} 
%\gbe{We continue by stating explicitly what parameters we consider when addressing a polynomial algorithm, and proving some useful theorems towards \Cref{cor:polyTimeSubmodular} and ~\ref{cor:submodular, additive, and unit demand algorithm} proofs.}


Consider an allocation instance $I$ with $n$ agents, $m$ items, integer valued %submodular
valuation functions, with values of bundles in the range $[0,K]$ (that is, $K=\max_{i\in\mathcal{N}}v_i(\items)$). (Alternatively, for a valuation function $v_i$ that is not integer valued, $K$ denotes an upper bound on  $\frac{v_i(\items)}{v_i(S)}$, over all sets $S \subset \items$ for which $v_i(S) > 0$. Namely, $K$ is an upper bound on the ratio between the values of the most valuable bundle and least valuable bundle of positive value.) We assume that all allocation algorithms have value query access to the valuation functions of the agents. We say that an allocation algorithm runs in polynomial time if both its running time and the number of value queries that it makes are polynomial in $n$, $m$ and $\log K$. In the following presentation, we consider the APS as our share notion, and allow for agents of arbitrary entitlement. We remark that the same principles apply (with straightforward modifications) to settings with equal entitlement, and the MMS as a fairness notion. 
%\ufc{remove: If we denote as $n$ the number of agents, $m$ to be the number of items, and $K=\max_{i\in\mathcal{N}}v_i(\items)$, then we would like to show an algorithm} \ufc{remove: \ufe{We design an allocation algorithm} that runs in time polynomial in $n$, $m$ and $\log K$.}

\begin{remark}
    Definition~\ref{def:conditional algorithm} and Theorem~\ref{thm:polytime} (that will be presented shortly) involve an approximation ratio $\rho$. All results extend without any change in the proofs to settings in which $\rho$ is not a fixed constant, but instead a function of the entitlement (such approximation ratios appear in Theorem~\ref{thm:1/3_APS_guarantee}, for example). That is, for agent $i$ with entitlement $b_i$, the approximation ratio is $\rho(b_i)$. 
\end{remark}


\begin{definition}
\label{def:conditional algorithm}
For $\rho > 0$, we say that an allocation algorithm is a $\rho$-APS algorithm for a class $C$ of valuations, if given any allocation instance with valuations from the class $C$, the algorithm outputs an allocation $(A_1, \ldots, A_n)$ in which every agent $i$ gets a bundle $A_i$ of value $v_i(A_i) \ge \rho \cdot APS_i$. 
We say that an allocation algorithm is a {\em conditional} $\rho$-APS algorithm for a class $C$ of valuations, if given any allocation instance with valuations from the class $C$, and given any vector $(t_1, \ldots, t_n)$, the algorithm outputs an allocation $(A_1, \ldots, A_n)$ that satisfies the following property: for every agent $i$, if it happens that $t_i \le APS_i$, then $v_i(A_i) \ge \rho \cdot t_i$. %$t_i = APS_i$, then $v_i(A_i) \ge \rho \cdot APS_i$. 
\end{definition}

%\begin{remark}
%    A variation on the notion of conditional $\rho$-APS algorithm is to have the following property: for every agent $i$, if it happens that $t_i \le APS_i$, then $v_i(A_i) \ge \rho \cdot t_i$. This variation is not needed in the current work in which $C$ is the class of submodular valuations, but has implicitly been used in~\cite{BEF21} for the class of additive valuations. Moreover, with this variation, the proof of Theorem~\ref{thm:polytime} becomes slightly simpler, and extends to all classes of valuation functions.
%\end{remark}

The proof of the following theorem is similar to a proof of a related theorem proved in~\cite{DBLP:conf/sigecom/GhodsiHSSY18}. We present its proof for completeness.
%\gbc{Talk with Uri about this point - I think it is too much credit, I think they proved it only for their algorithms, and not in such a general setting as presented in the next Theorem}

\begin{theorem}
\label{thm:polytime}
    Fix arbitrary $\rho > 0$ and arbitrary $\varepsilon > 0$. Then every polynomial time conditional $\rho$-APS algorithm for a class $C$ of valuations can be transformed into a polynomial time (unconditional) $(1 - \varepsilon)\rho$-APS  algorithm for the class $C$ of valuations. Here, the dependence on $\varepsilon$ of the running time of the unconditional algorithm is a multiplicative factor of $O(\frac{1}{\varepsilon})$. 
\end{theorem}



\begin{proof}
%\ufc{Need to refer the reader to the algorithm.}
{In Algorithm~\ref{alg:conditional to unconditional algorithm} we give an unconditional $(1 - \varepsilon)\rho$-APS algorithm, using a conditional $\rho$-APS algorithm as a blackbox.}


\RestyleAlgo{ruled}
\SetKwComment{Comment}{/* }{ */}
\begin{algorithm}[hbt!]
\caption{An \emph{unconditional} $(\rho - \varepsilon)$-APS algorithm using a \emph{conditional} $rho$-APS algorithm as a blackbox }\label{alg:conditional to unconditional algorithm}
\KwData{$\mathcal{M},\mathcal{N},\langle v_1,\dots,v_n\rangle,  \varepsilon{,K}$}
For every $i\in\mathcal{N},\quad t_i\gets v_i(\items)$\;

\While{true}{
    Run \emph{conditional}-$\rho$-APS algorithm with guesses $\langle t_1,\dots,t_n\rangle$, resulting $\mathcal{A}=\langle A_1,\dots,A_n\rangle$\; 
  \eIf{$\exists i,$ such that $v_i(A_i)<\rho_i\cdot t_i$ {{\bf and} $t_i\geq v_i(\items)\cdot\frac{1}{K}$}}{
    $i = i$ which satisfies the condition\;
    $t_i\gets (1 -\varepsilon)t_i$\;
  }{ Return $\mathcal{A}$ and exit\;
  }
}
\end{algorithm}

{
\begin{remark}
    In Algorithm~\ref{alg:conditional to unconditional algorithm}, we require having $K$ (the maximum ratio between largest and smallest value bundles) as an input. However, in the case of agents with submodular valuations, $K$ is not needed as input. Instead,  $K$ can be computed efficiently as it equals $\max_{i\in\mathcal{N}}\{\max_{\{e\in\mathcal{M}\mid v_i(e)>0\}}\{\frac{v_i(\items)}{v_i(e)}\}\}$. 
\end{remark}
}
% \end{figure}


\begin{claim}
\label{lower bound of guesses of APS}
During the whole run of Algorithm~\ref{alg:conditional to unconditional algorithm}, for every agent $i$, $t_i\geq (1-\varepsilon)APS_i$.

\end{claim}

\begin{proof}
    Fix an agent $i$. At the beginning of the algorithm, $t_i=v_i(\items)\geq APS_i$. during the algorithm we only reduce the value of $t_i$ each time by a factor of $(1-\varepsilon)$. Consider the first time when $t_i<APS_i$. Then, $t_i\geq (1-\varepsilon)APS_i$ (since in its previous value, the variable $t_i$ was greater than $APS_i$).
    Since $t_i<APS_i$, by \Cref{def:conditional algorithm}, every time we run command 3, the bundle of agent $i$ in the resulted allocation $\mathcal{A}$ is of value $\geq\rho t_i$, so we will not reduce the value of $t_i$.
\end{proof}

We prove the correctness of Algorithm~\ref{alg:conditional to unconditional algorithm}. If the algorithm terminates, then it returns an allocation $\mathcal{A}=\langle A_1,\dots,A_n\rangle$  with the property that for every agent $i$ { with $t_i\geq v_i(\items)\cdot\frac{1}{K}$}, $v_i(A_i)\geq \rho_i \cdot t_i$. By \Cref{lower bound of guesses of APS}, we have that each variable $t_i$ holds during the algorithm a value greater than $(1-\varepsilon) APS_i$. Thus, given that the algorithm terminates, {each such} agent $i$ gets a bundle $A_i$ of value at least $(1-\varepsilon) \rho APS_i$. 
%So the correctness boils down to the termination of the algorithm.
It remains to consider those agents $i$ with $t_i<v_i(\items)\cdot\frac{1}{K}$, and for which furthermore, $APS_i > 0$. The assumption that $APS_i>0$ implies that $APS_i \ge v_i(S)$, where $S$ is the bundle of minimum value. Also, $t_i<v_i(\items)\cdot\frac{1}{K} \le v_i(S)$. %$v_i(\items)\cdot\frac{1}{K}\geq v_i(S)$ , which is no greater than $APS_i$. 
Thus, by \Cref{lower bound of guesses of APS}, $v_i(A_i)\geq \rho t_i\geq v_i(S) \ge t_i$ (where the middle inequality holds because every bundle of positive value has value at least $v_i(S)$). As $APS_i < \frac{t_i}{1-\varepsilon}$ (otherwise, the condition of step~4 of the algorithm would not allow the value $t_i$ to be reached), we have that $v_i(A_i)\geq (1 - \varepsilon)APS_i$.


We turn to analyze algorithm's time complexity. 
%showing it is finite and polynomial. \ufc{not true} From \Cref{def:conditional algorithm}, we obtain that command 3 of Algorithm ~\ref{alg:conditional to unconditional algorithm} runs in polynomial time, given value queries. Denote its running time as 
Let $T(n,m,\log K)$ denote the running time of the \emph{conditional}-$\rho$-APS algorithm (called in command 3 of Algorithm ~\ref{alg:conditional to unconditional algorithm}). The number of times that command~3 is run is at most $n\cdot\log_{(1-\varepsilon)}K=n\cdot\frac{1}{\varepsilon}\cdot\log K$ (because in each iteration, at least one $t_i$ drops by a factor of $1-\epsilon$, and the total drop in value of $t_i$ is at most a factor of $K$).
%We need to analyze how many times we run command 3 (i.e., how many iterations for the ``while'' loop). Denote as $K$ to be $\max_{i\in\mathcal{N}}(v_i(\items))$. We begin with every agent $i$ having $t_i=v_i(\items)\leq K$, and we reduce one $t_i$ by an $\varepsilon$ factor in every iteration. Since there are $n$ agents, the ``while'' loop runs for at most $n\cdot\log_{(1-\varepsilon)}K=n\cdot\frac{1}{\varepsilon}\cdot\log K$ many iterations. 
Thus the overall runtime of Algorithm ~\ref{alg:conditional to unconditional algorithm} is $O\left( n\cdot\log(K)\cdot \frac{1}{\varepsilon}\cdot T(n,m,\log K)\right)$, yielding a polynomial time algorithm (under the assumption that $T(n,m,\log K)$ is polynomial).

\end{proof}


%\begin{remark}
%    The proof of Theorem~\ref{thm:polytime} uses the assumption that valuation functions are integer valued, with values in the range $[0,K]$. The statement of the theorem can be modified to a setting in which valuation functions are not integer valued, and instead values of bundles are expressed as rational numbers with numerators and denominators of values at most $K$. The modification is that for every desired $\epsilon > 0$, one can produce an (unconditional) $(\rho - \epsilon)$-APS algorithm, and the running time of the algorithm is polynomial in $n$, $m$, $\log K$, and $\log \frac{1}{\epsilon}$.  The proof of the modified theorem is very similar to that of Theorem~\ref{thm:polytime}, and is omitted.
%\end{remark}

The following theorem, Theorem~\ref{thm:conditionalSubmodular}, is a relatively straightforward consequence of Theorems~\ref{thm:1/3_APS_guarantee} and~\ref{thm:equal-10/27}.

\begin{theorem}
\label{thm:conditionalSubmodular}
    There is a polynomial time conditional $\rho$-APS algorithm for submodular valuations, with $\rho = \frac{1}{3 - 2b_i}$. For the equal entitlement case, there is a polynomial time conditional $\rho$-MMS algorithm for submodular valuations, with $\rho = \frac{10}{27} + {\Omega(\frac{1}{n})}$.
\end{theorem}

\begin{proof}
%    \ufc{Insert a proof. Given the vector of $t_i$, run the bidding game with the respective strategies. Ties can be broken arbitrarily. Explain that there are two ways to address $t_i < APS_i$. One is to observe that the analysis still works. The other, a more generic way, it to truncate at $t_i$, which is allowed because we are dealing with submodular functions, and truncation preserves submodularity.}
As described in the \emph{proportional bidding strategy}, both in the original and \emph{altruistic} version of the bidding game, an agent $i$ executes the \emph{proportional bidding strategy} is required to know/compute their APS (MMS). Based on her $APS_i$ ($MMS_i$), the agent knows how to bid. Consider a modification of the bidding strategy, in which agent $i$ receives as an auxiliary input a value $t_i$ (instead of computing her true APS (MMS) value), and truncates her valuation function at $t_i$, namely $v_i\gets v_i^{t_i}$ (\Cref{def:$v^t_p$}). Then, the agent infers her bidding using the value of $t_i$ instead of $APS_i$ ($MMS_i$). 

The conditional $\rho$-APS algorithm (or $\rho$-MMS algorithm) is simply to simulate the bidding game (or altruistic bidding game) with the $t_i$ values as auxiliary inputs to the agents, and having each agent follow her respective modified proportional bidding strategy.  Now, giving value queries, the bidding game is simulated in polynomial time, where we break ties arbitrarily. It remains to show the correctness of the algorithm. For this, notice that given a value $t_i\leq APS_i$ ($MMS_i$), then by \Cref{claim:truncation} the $APS$ ($MMS$) value of agent $i$ is reduced to $t_i$, and the truncation preserves submodularity. Now the conditions of \Cref{thm:1/3_APS_guarantee} (\Cref{thm:equal-10/27}) are met, and agent $i$ gets a bundle of value at least $\rho$-APS (-MMS), as desired.
\end{proof}

Combining theorems~\ref{thm:polytime} and~\ref{thm:conditionalSubmodular} we prove \Cref{cor:polyTimeSubmodular}, which is restated here for convenience.

\PolySubmodular*

\begin{proof}
\Cref{thm:conditionalSubmodular} states that there is a conditional $\rho$-APS algorithm for submodular valuations with $\rho=\frac{1}{3-2b_i}$. We can assume $b_i\geq\frac{1}{m}$ (as otherwise the APS of agent $i$ is~0,
%\gbc{Should I explain further that if  $b_i<\frac{1}{m}$, each item in a fractional partition associated with $APS_i$ have a total weight strictly less than $\frac{1}{m}$ implying a total weight $< m\cdot\frac{1}{m}$ leading to contradiction?}, 
implying that $\rho = \frac{1}{3} + \Omega(\frac{1}{m})$. Thus, setting $\varepsilon$ to be $O(\frac{1}{m})$ in Algorithm~\ref{alg:conditional to unconditional algorithm} yields the existence of an unconditional $\frac{1}{3}$-APS polynomial time algorithm, as desired. In a similar way for the $MMS$, by setting $\varepsilon$ to be $O(\frac{1}{n})$ we obtain an unconditional $\frac{10}{27}$-MMS polynomial time algorithm.
\end{proof}

%{Corollary~\ref{cor:submodular, additive, and unit demand algorithm} states that if valuation functions of agents come from different classes, then there is an allocation that simultaneously gives each agent a bundle of value that is a certain fraction of her APS, where this fraction depends on the class ($1$ for unit demand, $\frac{3}{5}$ for additive, $\frac{1}{3}$ for submudular). This is a consequence of the fact that all these ratios can be achieved by bidding strategies for a certain bidding game. The bidding game that is used is a variation on the bidding games considered in the current paper, in which agents who win a bid may pick more than a single item (as long as they can afford to pay for the items that they pick). This variation is used in~\cite{BEF21} in their proof of the $\frac{3}{5}$ ratio for additive valuations. Our $\frac{1}{3}$ ratio for submudular valuations extends also to this version of the bidding game (and unit demand bidders have trivial bidding strategies), establishing Corollary~\ref{cor:submodular, additive, and unit demand algorithm}.} \ufe{We now provide more details.}

%\ufe{Recall Definition~\ref{def:conditional algorithm} concerning conditional algorithms.}

Finally, we restate and prove Corollary~\ref{cor:submodular, additive, and unit demand algorithm}.
\PolyEnsamble*




\begin{proof}
For the sake of this proof, we  consider a modified version of the original bidding game. When an agent wins a round (i.e., she is the highest bidder), instead of picking one item, she can pick $k\geq1$ of the remaining items and pay $k$ times her bid.
We consider this version of bidding game because this is the version for which it was previously shown (in~\cite{BEF21}) that additive agents have a bidding strategy that guarantees themselves $\frac{3}{5}$ of their APS. That strategy is implementable in polynomial time. 

We show that in this version of the bidding game, any submodular, additive, or unit-demand agent has a strategy that guarantees herself the relevant guarantee of her APS.
\begin{itemize}
    \item Submodular agents have a strategy that guarantees $\frac{1}{3}+\Omega(\frac{1}{n})$ fraction of their APS. The key property that enables the proportional bidding strategy presented in \Cref{thm:1/3_APS_guarantee} to maintain its guarantee also in the current version of he bidding game is the fact the bidding sequence of a submodular agent executing the proportional bidding strategy is non-increasing. (\Cref{obsrv: bidding sequence decreasing}).
     Therefore, if another agent, $o$, wins a round and decides to pick $k>1$ items, for the submodular agent perspective, this is equivalent to $k$ rounds in which $o$ raises the same bid, and all other agents raise a bid of zero. Since the bids of the submodular agent are non-increasing, the submodular agent will not raise a bid strictly greater than $o$'s bid. Thus, with adversarial tie-breaking, we can guarantee that in both cases of the bidding game, the bundle of the submodular agent will remain the same, and the $\frac{1}{3}+\Omega(\frac{1}{n})$ guarantee from \Cref{thm:1/3_APS_guarantee} holds.
    
    \item Additive agents have a strategy that guarantees a $\frac{3}{5}$ fraction of their APS. The proof is presented in \cite{BEF21}. In addition, \cite{BEF21} presented a polynomial time implementation of this strategy, that does not require knowing the APS value of an agent.
    
    \item Unit-demand agents have a strategy that guarantees a $1$-APS. Consider a unit-demand agent $p$ with entitlement $b_p$. It is easy to verify that the APS of $p$ is the $\lceil \frac{1}{b_p} \rceil$th most valuable item, and $0$ if there are fewer than $\lceil \frac{1}{b_p} \rceil$ items.
    We claim that the simple strategy of $p$ bidding her entire budget in each round and, upon winning, taking the most valuable remaining item guarantees $p$ her APS.
    Since $p$ bids her entire budget until she wins an item, in every round she does not win, at least $b_p$ of the total budget of agents is spent. Hence, $p$ must win one of the first $\lfloor\frac{1}{b_p}\rfloor$ rounds. By that, she guarantees herself one of the $\lfloor\frac{1}{b_p}\rfloor$ most valuable items, which is at least her APS. Note that this strategy does not require agent $p$ to know her APS value, and given access to value queries, her strategy is polynomial-time.
\end{itemize}

It remains to show a transformation from the existence of approximate fair allocation (induced by the above strategies) to a polynomial time algorithm. The proof of this is similar to the proof of \Cref{cor:polyTimeSubmodular}, and is omitted. 
%However, we do stress that the main property that allows the polynomial implementation is that the guarantees of each bidding strategy, as the above, is the independency of the other agents' classes of valuation or their bidding strategies.
\end{proof}

% \gbe{
%     With slight modification, the above algorithm suggests a more general algorithm, in which different agents might have different bidding strategies for their bidding, and hence different fairness guarantees.
%     \cite{BEF21} suggests that an additive agent has a bidding strategy for the original version of the bidding game, which guarantees herself a $\frac{3}{5}$ fraction of her $APS$. Moreover, for unit-demand agents, \cite{BEF21} presents bidding guarantees each such agent an entire APS value. In terms of \Cref{def:conditional algorithm}, \cite{BEF21} presents a conditional $\frac{3}{5}$-APS for additive agents and a conditional $1$-APS for unit demand agents. However, \cite{BEF21} also presents a transformation from a conditional algorithm to an unconditional, in polynomial time, without losing an $\varepsilon$ factor from the approximation ratio. Combining \cite{BEF21} results with \Cref{cor:polyTimeSubmodular} we obtain Corollary ~\ref{cor:submodular, additive, and unit demand algorithm}
%     \gbc{Need to make sure why /Cref does not work here}
    % Therefore, if we have an instance consisting of submodular and additive agents, combining both strategies to agents guarantees the submodular agents a $\frac{1}{3}$-fraction of their APS, and $\frac{3}{5}$-fraction of the APS for the additive agents.}






\subsection{Negative examples}

%\ufc{State the values of  $\rho_k$ for small $k$, and an approximate value for the limit.}
{
\begin{proposition}
\label{hard instance altruistic version}
For every constant $\rho > \lim_{k\to\infty}\rho_k \simeq 0.3716$ (where for each $k\in\mathbb{N}$ we will define $\rho_k$ in the proof), there is an allocation instance with equal entitlements and an adversarial run of the \emph{altruistic} version of bidding game, in which an agent $p$ that has a submodular valuation function and uses the proportional bidding strategy gets a bundle of value smaller than $\rho MMS_p$.
\end{proposition}


\begin{proof}
We present a series of instances in which agent
$p$ with a submodular valuation function executes the proportional bidding strategy.

The instances are parameterized by $k\in\mathbb{N}$. The $k$th instance will be as follows:
Define
\begin{align*}
q_{1} & =2\\
q_{k} & =1+\prod_{i=1}^{k-1}q_{i}
\end{align*}

(This sequence is known as the Sylvester sequence)

The number of agents will be: $n_{k}=q_{k+1}-1$ (for example, for
$k=2$, $n_{k}=2\cdot3\cdot7=43$).

The set of item is $\mathcal{M}=\{e_{i,j}\}$ for $1\leq i\leq k+1$,
$1\leq j\leq n$ ($n\cdot(k+1)$ items).

If we think of $e_{i,j}$ as arranged in a matrix, then all the items
in a row are copies of the same item and are substitutes. The value
of items from different rows is additive.

For any $1\le i\leq k$ and for any $j$, $v_{p}(e_{i,j})=\frac{1}{q_i-1}$.
For $i=k+1$ and any $j$, $v_{p}(e_{k+1,j})=1$. For example, if $k=3$,
there are $43$ agents and columns, and in each column $j$, $v_p(e_{1,j})=1$, $v_p(e_{2,j})=\frac{1}{2}$, $v_p(e_{3,j})=\frac{1}{6}$, $v_p(e_{4,j})=1$.

\begin{itemize}

\item $v_{p}$ is submodular. The marginal value of each item is weakly decreasing (the marginal value of item $e_{i,j}$ to a set $S$ is either $v_{p}(e_{i,j})$ or $0$, depending on whether the set $S$ already contains an item from the $i$'th row).

\item The columns $C_j$ of the matrix $\{e_{i,j}\}$ form an MMS partition. The value of every bundle is at most $v_{p}(\mathcal{M})$, and in this partition, the value of each bundle (column) is exactly $v_{p}(\mathcal{M})$.

\item $APS_p = MMS_p =  v_p(\mathcal{M})=v_{p}(C_j)=v_p(e_{k+1,j})+\sum_{i=1}^{k}v_{p}(e_{i,j})=1+\sum_{i=1}^{k}\frac{1}{q_i-1}$
% $=1+2\sum_{i=1}^{k}\frac{1}{q_{i}}=1+2\cdot(1-\frac{1}{q_{k+1}-1})\underset{*}{=}3-\frac{2}{q_{k+1}-1}$,
% where equality {*} is a known property of the partial sums of Sylvester's inverse series (this can be proved by induction, Wikipedia value of Sylvester sequence).

\item $q_{i}$ divides $n$, for every $i \le k$.

\end{itemize}

For convenience, we assume w.l.o.g that the budget of each agent equal her $MMS$.

For every $k$, we first show a run of the bidding game with adversarial bidding of the other agents, in which agent $p$ executes the proportional bidding strategy with $\rho_{k}=\frac{1}{MMS_p}=\frac{1}{1+\sum_{i=1}^{k}\frac{1}{q_i-1}}$, and she receives a value of precisely $1$ (she gets the bundle that consists only of items from row $k+1$), which is a $\frac{1}{MMS_{p}}$ of her $MMS$.
% Moreover, each agent who spends more than $1$ value from her budget becomes inactive.
% For that instance, $\rho_{k}=\frac{1}{3-\frac{2}{q_{k+1}-1}}$.
The series of $\rho_{k}$ is monotonically decreasing and bounded by 0, so $\lim_{k\to\infty}\rho_k$ exists. (Sylvester's sequence grows at a doubly exponential rate. Hence, the sequence of $\rho_{k}$ converges very fast.)

% \ufc{remove: We now turn to present the adversarial bidding of the other agents, yielding the adversarial run, which induces a run of the bidding game with agent $p$ obtaining a value of exactly $\rho_{k}$ of her APS.}

% Consider the $I_{k}$ instance parameterized by $k$. 
% For convenience, assume the budget of each agent is  (which is $2\rho_{k}APS_{p})$.
Then, by $proportional(\rho_{k})$, in each round, agent $p$ bids the highest marginal value of the remaining items. Moreover, each agent who spends more than $1$ value from her budget becomes inactive.
We now present the adversarial run.

In round $1$, $p$ bids $1$, and is allowed to win. She selects an item of value $1$ from row $k+1$. 

In each of the next $n$ rounds, at least one of the first $\frac{n}{2}$ other agents bids $1$, and upon winning (note that $p$ bids $1$ in each of these rounds, and the algorithm is assumed to brake the ties adversarially), takes an item from the first row (i.e., $e_{1,j}$). 
%Since the highest value of an item is $1,$ , so $p$ does not win any item from the first row, and 
All items of the first row are taken by $\frac{n}{2}$ of the other agents. These $\frac{n}{2}$ agents surpass a payment of $1$ and become inactive. 

%From now on, since items are substitutes along rows, an item with the highest marginal value for $p$ is taken from the lower row available.

In each of the next $n$ rounds, at least one of the next $\frac{n}{3}$ other agents bids $\frac{1}{2}$, and upon winning (note that $p$ bids $\frac{1}{2}$ in each of these rounds), takes an item from the second row (i.e., $e_{1,j})$. Each such agent surpasses a payment of 1 exactly when winning her 3rd item, and becomes inactive.
% after taking three items.

%$p$ bids $\frac{2}{3}$, again the algorithm brake ties adversarially so $\frac{n}{3}$ of the other agents win the items in the second rows exhausting their budget. 
The run proceeds in the same way, where for every $i$, $\frac{n}{q_{i}}$ of the other agents bid $\frac{1}{q_{i}-1}$, win all the items in the $i$'th row, and become inactive. Note that  each such agent surpasses a payment of $1$ exactly when winning her $q_i$th item, and becomes inactive. Note that we use the property of $q_{i}\mid n$ for every $i\leq k$.

Thus, the number of other agents that take all items from rows $1$ to $k$ is:
\[
\sum_{i=1}^{k}\frac{n}{q_{i}}=n\cdot\sum_{i=1}^{k}\frac{1}{q_{i}}=n\cdot(1-\frac{1}{q_{k+1}-1})=n\cdot(1-\frac{1}{n})=n-1
\]

Thus, there are sufficiently many other agents to take all items from rows $1$ to $k$, and agent $p$ gets items only from row $k+1$. As they are substitutes, the total value received by $p$ is $1$.

Consider for $\rho'>\rho_{k}$ the altruistic version of the bidding game in which an agent becomes inactive after spending a $\rho'$ fraction of her budget, and suppose that $p$ executes the $proportional(\rho')$ bidding strategy. 
%then each agent becomes inactive when surpassing a payment of some $\alpha>1$. 
% the bids of $p$ in each round are strictly smaller than those described above.
On the instance $I_k$ described above, the same run of the bidding game holds, and $p$ does not get a bundle of value $\rho' APS_p$, but rather only $\rho_k APS_p$. Hence, $I_{k}$ serves as an example showing that our proof of Theorem~\ref{thm:1/3_APS_guarantee} does not extend to values of $\rho$ larger than $\rho_{k}$. The same holds for every $\rho>\lim_{k\to\infty}\rho_k$ (by enlarging $k$, we can make $\rho_k$ as close as we wish to $\lim_{k\to\infty}\rho_k$).
%\gbc{Uri, notice that this paragraph below suggest how to use this claim in \Cref{thm:equal-10/27}}
%\ufc{remove: Since obviously $\rho_{k}$ is close to} \ufe{By enlarging $k$, we can make $\rho_k$ as close as we wish to $\lim_{k\to\infty}\rho_k$. Hence,  for any $\rho>\lim_{k\to\infty}\rho_k$, there is an \ufc{remove: witness} \ufe{instance} $I_{k}$ on which $proportional(\rho)$ does not guarantee $p$ a $(\rho)$-fraction of her $APS$.
\end{proof}
}




\begin{remark}
The negative example in Proposition~\ref{hard instance altruistic version} can be modified by replacing (for every $j$) the single item $e_{k+1,j}$ of value~1 by $q_k - 1$ items, each of value $\frac{1}{q_k - 1}$ (the same value as that of item $e_{k,j}$). In this modified version, the adversarial run is changed so that other agents win all items $e_{ij}$ for $i \le k$ (their budgets exactly suffice for this), and agent $p$ can take the remaining items from one of the bundles of the MMS partition (the remaining items in different MMS bundles are substitutes to each other and do not provide additional marginal value), getting a value of~1. This modified example is useful in illustrating that for certain variations of the bidding game (considered by the authors but omitted here), bidding strategies similar to the ones considered in the proof of Theorem~\ref{thm:equal-10/27}  do not lead to approximation ratios that are significantly better than those proved in Theorem~\ref{thm:equal-10/27}).  
\end{remark}

%\ufc{remove:
%\begin{remark}
%One may try to improve $\rho$ by allowing some form of bidding on bundles (where the winner is the agent whose bid per item is highest). In the above example, the submodular agent will start with a double bid of~12, and after $n/2$ additive agents win, switch to a double bid of~9 (instead of a single bid of~6). Even if other agents win with a double bid of~9 (or single bid of $\frac{9}{2}$) and take items of value~6, the submodular agent is happy, as these items are substitutes.

%However, the, example can be modified, replacing $(6,6,3,1)$ by $(6,3,1, 1, 1, 1, 1, 1, 1)$, and full budget is 12. Now even if the agent starts bidding 12 on pairs, $n/2$ agents might take all the items of value~6. If she then bids by item values, $n/3$ agents might take all the items of value~3. (We assume that once an agent paid more than 6, she is out of the game. Hence other agents can take only three items of value~3, and not~4. Also, if our agent overbids, she may be allowed to win, but herself will be out of the game once she reaches a value of~6.) Then $n/7$ agents might take $n$ items, from each bundle one of the items of value~1. Then every bundle has only content $(1, 1, 1, 1, 1, 1)$, and if the items are substitutes across bundles, the total value that remains is~6. 
%\end{remark}}


The following proposition shows that in \Cref{thm:1/3_APS_guarantee}, the value of $\rho$ cannot be improved to a constant (independent of $b_p$) larger than $\frac{1}{3}$. Its proof is similar to the proof of Proposition~\ref{hard instance altruistic version}, with some relatively straightforward modifications. For completeness, its full proof is presented in the appendix (Section~\ref{sec:example}).

\begin{restatable}{reprop}{exampleThird}
%\begin{proposition}
\label{no rho larger than 1/3 for original bidding game}
% \ufc{remove: $\rho=\frac{1}{3}$ is the largest fraction, of which an agent $p$
% executes $proportional(\rho)$, is guaranteed to have $\rho$ fraction
% of her $MMS$ (and hence APS). In other words, for every constant
% $\varepsilon>0$, there exists an instance and an adversarial run
% of the bidding game, on which an agent that executes $proportional(\frac{1}{3}+\varepsilon)$ does not obtain a $(\frac{1}{3}+\varepsilon)$ fraction of her MMS, even in the case of equal entitlements.}
For every constant $\rho > \frac{1}{3}$, there is an allocation instance with equal entitlements and an adversarial run of the bidding game, in which an agent $p$ that has a submodular valuation function and uses the $proportional(\rho)$ bidding strategy gets a bundle of value smaller than $\rho MMS_p$.
\end{restatable}
%\end{proposition}


% ##################################################

%\gbc{Once you approve the above shore version of the proof, we can remove the below full proof. don't worry it is saved in the History file.}



%\ufc{Perhaps comment out this remark, as it is too much of a digression.}




%Now we show that agents with XOS valuation functions do not have safe strategies that guarantee them a constant fraction of their MMS in the bidding game.
%\gbc{As you changed \Cref{no rho larger than 1/3 for original bidding game} from lemma to proposition, do you want to change \Cref{prop:XOS_hardness} from lemma to proposition? Recall that we state \Cref{prop:XOS_hardness} in the introduction}

We now restate and prove \Cref{prop:XOS_hardness}, showing that our proof techniques do not extend to XOS valuations.
\XosHardness*

\begin{proof}
For parameters $n,k$, define the instance $I(n,k)$ as follows. There
are $n$ agents with equal entitlements. The set $\mathcal{M}$ of items  consists of $nk$ items $e_{ij}$ for $1\leq i\leq k$ and $1\leq j\leq n$. We think of $\mathcal{M}$ as arranged in an $k\times n$ matrix, with $e_{ij}$ in the $ij$ entry. For every column $j$, let $c_{j}$ be the additive valuation function defined by giving value $1$ to items in column $j$, and $0$ to all other items. Let $v$ be the pointwise maximum of the functions $c_{j}$, that is, $v(S)=\max_{j}c_{j}(S)$ for every $S\subseteq\mathcal{M}$.
Then the valuation function $v$ is an XOS function by its definition.
We focus on a specific agent $p$ whose valuation function is $v_{p}=v$.
(The other agents may have arbitrary valuations.)

\begin{claim}
\label{claim:XOS_hardness}
If $n\geq4k^{2}$, no bidding strategy can guarantee $p$ more than a $1/k$-fraction
of $MMS_{p}$.
\end{claim}

\begin{proof}
For convenience, assume all agents are given a budget of $k$. We give
the other agents adversarial bidding strategies, as follows. There
are two types of agents.
\begin{itemize}
\item Type 1 agents consist of $n/2$ of the agents that always bid $1/2$,
and take an arbitrary available item upon winning.
\item Type 2 agents consist of the rest $n/2-1$ agents, which operate
as follows. Once agent $p$ wins an item $e_{ij}$, an agent of this
type bids all of her budget in the next $k-1$ rounds, and upon winning,
chooses an available item from column $j$ (and becomes inactive).
\end{itemize}
An agent of type 1 becomes inactive after winning exactly $2k$ items.
As the number of items is $nk$, it follows that there exists an active
agent of type 1 in every round. Thus, agent $p$ must pay $1/2$ for
every won item, so she can win at most $2k$ items overall. Once $p$
wins her first item from some column $j$, if there exist at least
$k-1$ active agents of type 2, all other items in column $j$ will
be taken by them in the next $k-1$ rounds. So, if we start with at
least $(2k)\cdot(k-1)$ agents of type 2, $p$ will not win more than
one item from every column. As $(2k)\cdot(k-1)\leq2k^{2}-1\leq n/2-1$,
this indeed holds, so agent $p$ cannot win a bundle of value more
than $1$. Observe that $MMS_{p}=k$, so the claim follows.
\end{proof}

Proposition~\ref{prop:XOS_hardness} is an immediate consequence of Claim~\ref{claim:XOS_hardness}.
\end{proof}




\bibliographystyle{alpha}
\bibliography{Refrences_DB}

\section{Appendix for Proofs}

\paragraph{Proof of Theorem \ref{thm:main}.}

\begin{proof}
\label{proof:main}
Our proof has two steps. In Step 1, we will show that SimCLR is equivalent to minimizing the cross entropy loss defined in Eqn.~(\ref{eqn:cross-entropy}). 
In Step 2, we will show  that minimizing the cross-entropy loss 
is equivalent to spectral clustering on $\bfpi$. 
Combining the two steps together, we have proved our theorem. 

\textbf{Step 1: } SimCLR is equivalent to minimizing the cross entropy loss.

The cross-entropy loss takes expectation over 
$\bfW_\bfX\sim \mathbb{P}(\cdot ; \bfpi)$, 
which means $\bfW_\bfX$ has exactly one non-zero entry in each row $i$. By Lemma~\ref{lem:multinomial}, we know every row $i$ of $\bfW_\bfX$ is independent of other rows. Moreover, 
$\bfW_{\bfX,i}\sim \mathcal{M}(1, \bfpi_i/\sum_j \bfpi_{i,j})=\mathcal{M}(1, \bfpi_i)$, because $\bfpi_i$ itself is a probability distribution.
Similarly, we know $\bfW_\bfZ$ also has the row-independent property by sampling over $\mathbb{P}(\cdot;\bfK_\bfZ)$.
Therefore, by Lemma~\ref{lem:cross_split}, we know Eqn.~(\ref{eqn:cross-entropy}) is equivalent to:
\[
 -\sum_{i=1}^n \mathbb{E}_{\bfW_{\bfX,i}}[\log \mathbb{P}(\bfW_{\bfZ,i}=\bfW_{\bfX,i};\bfK_\bfZ)],
\]

This expression takes expectation over $\bfW_{\bfX,i}$ for the given row $i$. Notice that 
$\bfW_{\bfX,i}$ has exactly one non-zero entry, which equals $1$ (same for $\bfW_{\bfZ,i}$). 
As a result
we expand the above expression to be:
\begin{equation}
 -\sum_{i=1}^n \sum_{j\neq i} \Pr(\bfW_{\bfX,i,j}=1)\log \Pr(\bfW_{\bfZ,i,j}=1).
\label{eqn:detailed-expansion}    
\end{equation}


By Lemma~\ref{lem:multinomial}, $\Pr(\bfW_{\bfZ,i,j}=1)=\bfK_{\bfZ,i,j}/\|\bfK_{\bfZ,i}\|_1$ for $j\neq i$. Recall that $\bfK_\bfZ=(k(\bfZ_i-\bfZ_j))_{(i,j)\in[n]^2}$, which means 
$\bfK_{\bfZ,i,j}/\|\bfK_{\bfZ,i}\|_1=\frac{\exp(-\|\bfZ_i-\bfZ_j\|^2/{2\tau})}{\sum_{k\neq i}
\exp(-\|\bfZ_i-\bfZ_k\|^2/{2\tau})
}$ for $j\neq i$, when $k$ is the Gaussian kernel with variance $\tau$. 

Notice that $\bfZ_i=f(\bfX_i)$, so we know
\begin{equation}
-\log \Pr(\bfW_{\bfZ,i,j}=1)=
-\log \frac{\exp(-\|f(\bfX_i)-f(\bfX_j)\|^2/{2\tau})}{\sum_{k\neq i}
\exp(-\|f(\bfX_i)-f(\bfX_k)\|^2/{2\tau}),
}
\label{eqn:infonce-equivalence}    
\end{equation}


The right hand side is exactly the InfoNCE loss defined in Eqn.~(\ref{eqn:infonce}).
Inserting Eqn.~(\ref{eqn:infonce-equivalence}) into Eqn.~(\ref{eqn:detailed-expansion}), we get the SimCLR algorithm, which first samples augmentation pairs $(i,j)$ with $\Pr(\bfW_{\bfX,i,j}=1)$ for each row $i$, and then optimize the InfoNCE loss. 

\textbf{Step 2: } minimizing the cross entropy loss 
is equivalent to spectral clustering on $\bfpi$.


By Lemma~\ref{lem:convert_to_spectral}, we may further convert the loss to 
\begin{equation}
\label{eqn:main-theorem-repul-attr}
\min_{\bfZ}
-\sum_{(i,j)\in [n]^2} \mathbf{P}_{i,j}
\log k (\bfZ_i-\bfZ_j)+\log \mathbf{R}(\bfZ).
\end{equation}
Since $k$ is the Gaussian kernel, this reduces to \[
\min_\bfZ \mathrm{tr}(\bfZ^\top \mathbf{L}(\bfpi) \bfZ)
+\log \mathbf{R}(\bfZ),
\]

where we use the fact that $\mathbb{E}_{\bfW_\bfX\sim \mathbb{P}(\cdot; \bfpi)}[\mathbf{L}(\bfW_\bfX)]
=\mathbf{L}(\bfpi)
$, because the Laplacian operator is linear and $
\mathbb{E}_{\bfW_\bfX\sim \mathbb{P}(\cdot; \bfpi)}(\bfW_\bfX)=\bfpi
$.
\end{proof}

\paragraph{Proof of Theorem \ref{thm:clip}.}
\begin{proof}
Since $\bfW_\bfX\sim \mathbb{P}(\cdot;\bfpi_{\mathbf{A}, \mathbf{B}})$, we know 
$\bfW_\bfX$ has exactly one non-zero entry in each row, denoting the pair that got sampled. 
A notable difference compared to the previous proof is we now have $n_\mathcal{A}+n_\mathcal{B}$ objects in our graph. CLIP deals with this by taking a mini-batch of size $2N$, 
such that $n_\mathcal{A}=n_\mathcal{B}=N$, and adding the $2N$ InfoNCE losses together. We label the objects in $\mathcal{A}$ as $[n_\mathcal{A}]$, and the objects in $\mathcal{B}$ as $\{n_\mathcal{A}+1, \cdots, n_\mathcal{A}+n_\mathcal{B}\}$. 

Notice that $\bfpi_{\mathbf{A}, \mathbf{B}}$ is a bipartite graph, so the edges of objects in $\mathcal{A}$ will only connect to object in $\mathcal{B}$ and vice versa. We can define the similarity matrix in $\cZ$ as $\bfK_\bfZ$, 
where $\bfK_\bfZ(i, j+n_\mathcal{A})=\bfK_\bfZ(j+n_\mathcal{A},i)= k(\bfZ_i-\bfZ_j)$ for $i\in [n_\mathcal{A}], j\in [n_\mathcal{B}]$, and otherwise we set $\bfK_\bfZ(i,j)=0$. 
The rest is same as the previous proof. 
\end{proof}

\paragraph{Proof of Theorem \ref{thm:exponential}.}

\begin{proof}
\label{proof:exponential}
Since the objective function consists of a linear term combined with an entropy regularization, which is a strongly concave function, the maximization problem is a convex optimization problem. Owing to the implicit constraints provided by the entropy function, the problem is equivalent to having only the equality constraint. We then introduce the Lagrangian multiplier $\lambda$ and obtain the following relaxed problem:

$$
\widetilde{E}(\boldsymbol{\alpha})=\psi_{1}-\sum_{i=1}^n \alpha_{i} \psi_{i}+\tau \sum_{i=1}^n \alpha_{i}\log \alpha_{i}+\lambda\left(\boldsymbol{\alpha}^{\top} \mathbf{1}_n-1\right).
$$

As the relaxed problem is unconstrained, taking the derivative with respect to $\alpha_{i}$ yields

$$
\frac{\partial \widetilde{E}(\boldsymbol{\alpha})}{\partial \alpha_{i}}=-\psi_{i}+\tau\left(\log \alpha_{i}+\alpha_{i} \frac{1}{\alpha_{i}}\right)+\lambda=0.
$$

Solving the above equation implies that $\alpha_{i}$ takes the form
$
\alpha_{i}=\exp \left(\frac{1}{\tau} \psi_{i}\right) \exp \left(\frac{-\lambda}{\tau}-1\right).
$ Since $\alpha_{i}$ lies on the probability simplex, the optimal $\alpha_{i}$ is explicitly given by
$
\alpha^{*}_{i}=\frac{\exp \left(\frac{1}{\tau} \psi_{i}\right)}{\sum_{i^{\prime}=1}^n \exp \left(\frac{1}{\tau} \psi_{i^{\prime}}\right)} .
$ Substituting the optimal point into the objective function, we obtain
$$
\begin{aligned}
E\left(\boldsymbol{\alpha}^*\right)  &=\psi_1-\sum_{i=1}^n \frac{\exp \left(\frac{1}{\tau} \psi_{i}\right)}{\sum_{i^{\prime}=1}^n \exp \left(\frac{1}{\tau} \psi_{i^{\prime}}\right)} \psi_{i}+\tau \sum_{i=1}^n \frac{\exp \left(\frac{1}{\tau} \psi_{i}\right)}{\sum_{i^{\prime}=1}^n \exp \left(\frac{1}{\tau} \psi_{i^{\prime}}\right)}\log \frac{\exp \left(\frac{1}{\tau} \psi_{i}\right)}{\sum_{i^{\prime}=1}^n \exp \left(\frac{1}{\tau} \psi_{i^{\prime}}\right)} \\
& =\psi_1 - \tau \log \left(\sum_{i=1}^n \exp \left(\frac{1}{\tau} \psi_{i}\right)\right).
\end{aligned}
$$
Thus, the Lagrangian dual function is given by
\begin{equation*}
-E\left(\boldsymbol{\alpha}^*\right)= -\tau \log \frac{\exp \left(\frac{1}{\tau} \psi_{1}\right)}{\sum_{i=1}^n \exp \left(\frac{1}{\tau} \psi_{i}\right)}.\qedhere
\end{equation*}
\end{proof}



\section{More on Experiments} \label{section: experiment_details}

\paragraph{CIFAR-10 and CIFAR-100} CIFAR-10 ~\citep{krizhevsky2009learning} and CIFAR-100 ~\citep{krizhevsky2009learning} are well-known classic image classification datasets. Both CIFAR-10 and CIFAR-100 contain a total of 60k $32 \times 32$ labeled images of different classes, with 50k for training and 10k for testing. CIFAR-10 is similar to CIFAR-100, except there are 10 different classes in CIFAR-10 and 100 classes in CIFAR-100.

\paragraph{TinyImageNet} TinyImageNet ~\citep{le2015tiny} is a subset of ImageNet ~\citep{deng2009imagenet}. There are 200 different object classes in TinyImageNet, with 500 training images, 50 validation images, and 50 test images for each class. All the images in TinyImageNet are colored and labeled with a size of $64 \times 64$.

\textbf{Pseudo-code.} Algorithm \ref{alg:Training Procedure} presents the pseudo-code for our empirical training procedure.

\begin{algorithm}[!htbp]
\caption{Training Procedure}
\label{alg:Training Procedure}
\begin{algorithmic}[1]
\REQUIRE trainable encoder network $f$, batch size $N$, augmentation strategy \textit{aug}, loss function $L$ with hyperparameters \textit{args}
\FOR {sampled minibatch ${x_i}_{i=1}^N$}
\FORALL{$i \in { 1, ..., N }$}
\STATE draw two augmentations $t_i = \textit{aug}\left(x_i\right) $, $t_i' = \textit{aug}\left(x_i\right) $
\STATE $z_i = f\left(t_i\right)$, $z_i' = f\left(t_i'\right)$
\ENDFOR
\STATE compute loss $\mathcal{L} = L(N, z, z', \textit{args})$
\STATE update encoder network $f$ to minimize $\mathcal{L}$
\ENDFOR
\STATE \textbf{Return} encoder network $f$
\end{algorithmic}
\end{algorithm}

We also provide the pseudo-code for our core loss function used in the training procedure in Algorithm \ref{alg:Core loss}. The pseudo-code is almost identical to SimCLR's loss function, with the exception of an extra parameter $\gamma$.

\begin{algorithm}[!htbp]
\caption{Core loss function $\mathcal{C}$}
\label{alg:Core loss}
\begin{algorithmic}[1]
\REQUIRE batch size $N$, two encoded minibatches $z_1, z_2$, $\gamma$, temperature $\tau$
\STATE $z = \textit{concat}\left(z_1, z_2\right)$
\FOR {$i \in {1, ..., 2N }, j \in {1, ..., 2N}$ }
\STATE $s_{i,j} = \Vert z_i - z_j \Vert_2^{\gamma}$
\ENDFOR
\STATE \textbf{define} $l(i, j)$ \textbf{as} $l(i, j) = - \log \frac{exp\left(s_{i,j}/\tau \right)}{\sum_{k=1}^{2N} \mathbf{1}{[k \ne i]} exp\left(s{i, j} / \tau \right)} $
\STATE \textbf{Return} $\frac{1}{2N} \sum_{k=1}^N\left[l(i, i+N) + l(i+N, i)\right]$
\end{algorithmic}
\end{algorithm}

Utilizing the core loss function $\mathcal{C}$, we can define all kernel loss functions used in our experiments in Table \ref{table: loss definition}. For all $z_i \in z$ with even dimensions $n$, we define $z_{L_i} = z_i\left[0:n/2\right]$ and $z_{R_i} = z_i\left[n/2:n\right]$.

\begin{table}[ht]
\centering
\begin{tabular}{{@{}l|l@{}}}
Kernel  &  Loss function \\ \midrule
Laplacian & $\mathcal{C}\left(N, z, z', \gamma=1, \tau\right)$\\ \midrule
Sum       & $\lambda * \mathcal{C}\left(N, z, z', \gamma=1, \tau_1\right) + (1-\lambda) * \mathcal{C}\left(N, z, z', \gamma=2, \tau_2\right)$  \\ \midrule
Concatenation Sum&$\lambda * \mathcal{C}\left(N, z_L, z'_L, \gamma=1, \tau_1\right) + (1-\lambda) * \mathcal{C}\left(N, z_R, z'_R, \gamma=2, \tau_2\right)$\\ \midrule
$\gamma = 0.5$ & $\mathcal{C}\left(N, z, z', \gamma=0.5, \tau\right)$          \\ 

\end{tabular}

\caption{Definition of kernel loss functions in our experiments}
\label {table: loss definition}
\end{table}

\textbf{Baselines.} We reproduce the SimCLR algorithm using PyTorch Lightning~\citep{PytorchLightning}.

\textbf{Encoder details.}
The encoder $f$ consists of a backbone network and a projection network. We employ ResNet50~\citep{ResNet} as the backbone and a 2-layer MLP (connected by a batch normalization~\citep{ioffe2015batch} layer and a ReLU \cite{nair2010rectified} layer) with hidden dimensions 2048 and output dimensions 128 (or 256 in the concatenation kernel case).

\textbf{Encoder hyperparameter tuning.}
For each encoder training case, we randomly sample 500 hyperparameter groups (sample details are shown in Table \ref{table: Hyperparameter sample}) and train these samples simultaneously using Ray Tune ~\citep{RayTune}, with the ASHA scheduler~\citep{li2018massively}. Ultimately, the hyperparameter group that maximizes the online validation accuracy (integrated in PyTorch Lightning) within 5000 validation steps is chosen for the given encoder training case.

\begin{table}[ht]
\centering

\begin{tabular}{@{}l|l|l@{}}
\midrule
Hyperparameter  & Sample Range & Sample Strategy \\ \midrule
start learning rate & $\left[10^{-2}, 10\right]$ & log uniform \\ \midrule
$\lambda$       & $\left[0, 1\right]$ & uniform \\ \midrule
$\tau$, $\tau_1$, $\tau_2$ & $\left[0, 1\right]$ & log uniform \\ \midrule
\end{tabular}

\caption{Hyperparameters sample strategy}
\label {table: Hyperparameter sample}
\end{table}

\textbf{Encoder training.} 
We train each encoder using the LARS optimizer~\citep{LARSOptimizer}, LambdaLR Scheduler in PyTorch, momentum 0.9, weight decay $10^{-6}$, batch size 256, and the aforementioned hyperparameters for 400 epochs on a single A-100 GPU.

\textbf{Image transformation.} The image transformation strategy, including augmentation, is identical to the default transformation strategy provided by PyTorch Lightning.

\textbf{Linear evaluation.}
The linear head is trained using the SGD optimizer with a cosine learning rate scheduler, batch size 64, and weight decay $10^{-6}$ for 100 epochs. The learning rate starts at $0.3$ and ends at $0$.

\textbf{Moco Experiments.} We also tested our method based on MoCo~\citep{he2019moco}. The results are summarized in Table \ref{tab:results-moco}. Here we choose ResNet18~\citep{ResNet} as the backbone and set a temperature of $0.1$ as default. For our simple sum kernel, we set $\lambda=0.8$. The results show that our method outperforms the original MoCo method.

\begin{table}[thb]
\centering
\caption{MoCo Experiment Results on CIFAR-10 and CIFAR-100.}
\label{tab:results-moco}
\resizebox{\textwidth}{!}{%
\begin{tabular}{@{}c|ccc|ccc@{}}
\toprule
\multirow{3}{*}{Method} & \multicolumn{3}{c|}{CIFAR-10} & \multicolumn{3}{c}{CIFAR-100} \\ \cmidrule(lr){2-4} \cmidrule(lr){5-7} 
                        & 200 epochs & 400 epochs    & 1000 epochs   & 200 epochs & 400 epochs & 1000 epochs         \\ \midrule
MoCo (repro.)         & $76.41 \pm 0.12$    & $80.01 \pm 0.15$          & $84.45 \pm 0.08$    & $\mathbf{47.02 \pm 0.11}$ & $52.50 \pm 0.07$ & $57.62 \pm 0.15$            \\
\midrule
Laplacian Kernel        & ${78.09 \pm 0.10}$    & $\mathbf{83.85 \pm 0.09}$          & $\mathbf{88.34 \pm 0.16}$    & $46.12 \pm 0.22$   & $53.44 \pm 0.17$ & $59.10 \pm 0.14$        \\
Simple Sum Kernel & $\mathbf{78.12 \pm 0.15}$   & $83.23 \pm 0.18$ & $87.50 \pm 0.20$ & $46.65 \pm 0.06$ & $\mathbf{53.62 \pm 0.19}$ & $\mathbf{59.83 \pm 0.12}$\\
\bottomrule
\end{tabular}
}
\end{table}



\section{More Experiments on Synthetic Data}


Consider a scenario with $n$ clusters, each containing $k$ vertices. Let the probability of vertices $u$ and $v$ from the same cluster belonging to $\bfpi$ be $p$. Conversely, for vertices $u$ and $v$ from different clusters, let the probability of belonging to $\pi$ be $q$. We generate the graph $\bfpi$ randomly, based on $p$ and $q$. We experiment with values of $k=100$ and $n=6$ for ease of visualization, embedding all points in a two-dimensional space. Each vertex's initial position originates from a normal distribution. In each iteration, we sample a subgraph of $\bfpi$ uniformly, ensuring each vertex has an out-degree of $1$. We then optimize the corresponding vectors using InfoNCE loss with an SGD optimizer and iterate until convergence. Our experimental setup consists of an SGD learning rate of $1$, an InfoNCE loss temperature of $0.5$, and a batch size of $50$. We evaluate two scenarios with different $p$ and $q$ values: $p=1$, $q=0$, and $p=0.75$, $q=0.2$. The results of these experiments are visualized in Figure \ref{fig:vis-spectral-cluster}. The obtained embeddings exhibit the hallmark pattern of spectral clustering of graph $\bfpi$.

\begin{figure}[!tb]
\centering
\subfigure{
\includegraphics[width=1\textwidth]{Figures/cluster_pi.png}
\label{fig:vis-cluster}
}
\subfigure{
\includegraphics[width=1\textwidth]{Figures/noised_cluster_pi.png}
\label{fig:vis-noised-cluster}
}
\caption{Visualizations of the optimization process using InfoNCE Loss on the vectors corresponding to $\bfpi$. Points of identical color belong to the same cluster within $\bfpi$. To showcase the internal structure of $\bfpi$, we randomly select 10 vertices from each cluster to display the edge distribution of $\bfpi$.}
\label{fig:vis-spectral-cluster}
\end{figure}



\end{document}

There is slackness in our allocation algorithm, and hence it is likely that more sophisticated algorithms will give each agent an even higher fraction of her APS.



The proposed algorithm is a bidding strategy for the bidding game of [BEF21] (the version in which the winner of the bid can buy as many items as her budget allows).

The proposed strategy is as follows (for valuation functions normalized as in the theorem). (Note that we do not assume that the instance is ordered, as the transformation of Bouvert and Lamaitre assumes additive valuations.) Each agent bids the value of the item of highest marginal value to her. Note that we may assume that no item has value larger than the initial budget of the player, as truncating values would not lower the APS. The highest bidder wins. If the winner bid more than half her remaining budget, she takes a single item. If she bid at most half her remaining budget she make take more than one item. The optimal rule for how many items to take might be complicated. The simple rule of always taking one item should suffice in order to prove the theorem. The following extension may suffice in order to improve $\rho$: if this is the first time the agent wins a bid, and taking two items suffices in order to reach $\rho$ fraction of the APS, then take two items. In any other case, take only one item.

When an agent spends $\rho$ fraction of her budget, she leaves the game.

After each round, the winning agent updates her valuation function for the remaining items, to be the marginal value with respect to the items that she already holds. Consequently, the sum of marginal values of remaining items goes down.  A better value of $\rho$ may possibly be obtained by a careful choice of scaling the marginal valuation function, but might not be necessary if one only wants to prove the theorem. (If scaling is used, then an agent leaves the game  once she accumulated a $\rho$ fraction of her APS. This might requiring spending more than a $\rho$ fraction of her budget.)

The analysis will need to bound how much the marginal values decrease when an agent wins an item. They cannot decrease by too much due to the existence of an APS fractional partition, as it implies that for an item of low value, it cannot be that all other items are substitutes for it. 

Let us first provide analysis for the case of equal entitlement (MMS partition). For an agent with entitlement $1/n$, a lower bound on the initial sum of marginal values is~1, as there are $n$ bundles, each of value at least $\frac{1}{n}$. Whenever another agent wins an item, the lower bound decreases by at most the bid of the other agent. Whenever the agent wins an item of value $\delta$, the lower bound decreases by at most $n\delta$ (as each of the $n$ bundles in the MMS partition loses at most $\delta$ in its total value).

For MMS, we can take $\rho=\frac{1}{3}$. If some agent wins only a single item, she does not harm the MMS of other agents. If an agent wins at least two items, she spends at most $\frac{2}{3n}$ altogether. Hence other agents can cause the sum of marginal values to decrease by at most $\frac{2(n-1)}{3n} < \frac{2}{3}$. The agent herself can win items of total value $\frac{1}{3n}$ until no value is left. 

If true, the above recovers the previously known result of an allocation giving $\frac{1}{3}$ fraction of the MMS in the equal entitlement case, and suggests also an improvement by an $\Omega({1}{n})$ term. This new proof for MMS is more amenable to extensions to APS, as it does not make the explicit assumption that no item is worth more than a third of the APS.  


 