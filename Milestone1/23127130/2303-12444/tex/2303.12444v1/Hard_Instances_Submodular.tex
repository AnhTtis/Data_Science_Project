
\subsection{Negative examples}

%\ufc{State the values of  $\rho_k$ for small $k$, and an approximate value for the limit.}
{
\begin{proposition}
\label{hard instance altruistic version}
For every constant $\rho > \lim_{k\to\infty}\rho_k \simeq 0.3716$ (where for each $k\in\mathbb{N}$ we will define $\rho_k$ in the proof), there is an allocation instance with equal entitlements and an adversarial run of the \emph{altruistic} version of bidding game, in which an agent $p$ that has a submodular valuation function and uses the proportional bidding strategy gets a bundle of value smaller than $\rho MMS_p$.
\end{proposition}


\begin{proof}
We present a series of instances in which agent
$p$ with a submodular valuation function executes the proportional bidding strategy.

The instances are parameterized by $k\in\mathbb{N}$. The $k$th instance will be as follows:
Define
\begin{align*}
q_{1} & =2\\
q_{k} & =1+\prod_{i=1}^{k-1}q_{i}
\end{align*}

(This sequence is known as the Sylvester sequence)

The number of agents will be: $n_{k}=q_{k+1}-1$ (for example, for
$k=2$, $n_{k}=2\cdot3\cdot7=43$).

The set of item is $\mathcal{M}=\{e_{i,j}\}$ for $1\leq i\leq k+1$,
$1\leq j\leq n$ ($n\cdot(k+1)$ items).

If we think of $e_{i,j}$ as arranged in a matrix, then all the items
in a row are copies of the same item and are substitutes. The value
of items from different rows is additive.

For any $1\le i\leq k$ and for any $j$, $v_{p}(e_{i,j})=\frac{1}{q_i-1}$.
For $i=k+1$ and any $j$, $v_{p}(e_{k+1,j})=1$. For example, if $k=3$,
there are $43$ agents and columns, and in each column $j$, $v_p(e_{1,j})=1$, $v_p(e_{2,j})=\frac{1}{2}$, $v_p(e_{3,j})=\frac{1}{6}$, $v_p(e_{4,j})=1$.

\begin{itemize}

\item $v_{p}$ is submodular. The marginal value of each item is weakly decreasing (the marginal value of item $e_{i,j}$ to a set $S$ is either $v_{p}(e_{i,j})$ or $0$, depending on whether the set $S$ already contains an item from the $i$'th row).

\item The columns $C_j$ of the matrix $\{e_{i,j}\}$ form an MMS partition. The value of every bundle is at most $v_{p}(\mathcal{M})$, and in this partition, the value of each bundle (column) is exactly $v_{p}(\mathcal{M})$.

\item $APS_p = MMS_p =  v_p(\mathcal{M})=v_{p}(C_j)=v_p(e_{k+1,j})+\sum_{i=1}^{k}v_{p}(e_{i,j})=1+\sum_{i=1}^{k}\frac{1}{q_i-1}$
% $=1+2\sum_{i=1}^{k}\frac{1}{q_{i}}=1+2\cdot(1-\frac{1}{q_{k+1}-1})\underset{*}{=}3-\frac{2}{q_{k+1}-1}$,
% where equality {*} is a known property of the partial sums of Sylvester's inverse series (this can be proved by induction, Wikipedia value of Sylvester sequence).

\item $q_{i}$ divides $n$, for every $i \le k$.

\end{itemize}

For convenience, we assume w.l.o.g that the budget of each agent equal her $MMS$.

For every $k$, we first show a run of the bidding game with adversarial bidding of the other agents, in which agent $p$ executes the proportional bidding strategy with $\rho_{k}=\frac{1}{MMS_p}=\frac{1}{1+\sum_{i=1}^{k}\frac{1}{q_i-1}}$, and she receives a value of precisely $1$ (she gets the bundle that consists only of items from row $k+1$), which is a $\frac{1}{MMS_{p}}$ of her $MMS$.
% Moreover, each agent who spends more than $1$ value from her budget becomes inactive.
% For that instance, $\rho_{k}=\frac{1}{3-\frac{2}{q_{k+1}-1}}$.
The series of $\rho_{k}$ is monotonically decreasing and bounded by 0, so $\lim_{k\to\infty}\rho_k$ exists. (Sylvester's sequence grows at a doubly exponential rate. Hence, the sequence of $\rho_{k}$ converges very fast.)

% \ufc{remove: We now turn to present the adversarial bidding of the other agents, yielding the adversarial run, which induces a run of the bidding game with agent $p$ obtaining a value of exactly $\rho_{k}$ of her APS.}

% Consider the $I_{k}$ instance parameterized by $k$. 
% For convenience, assume the budget of each agent is  (which is $2\rho_{k}APS_{p})$.
Then, by $proportional(\rho_{k})$, in each round, agent $p$ bids the highest marginal value of the remaining items. Moreover, each agent who spends more than $1$ value from her budget becomes inactive.
We now present the adversarial run.

In round $1$, $p$ bids $1$, and is allowed to win. She selects an item of value $1$ from row $k+1$. 

In each of the next $n$ rounds, at least one of the first $\frac{n}{2}$ other agents bids $1$, and upon winning (note that $p$ bids $1$ in each of these rounds, and the algorithm is assumed to brake the ties adversarially), takes an item from the first row (i.e., $e_{1,j}$). 
%Since the highest value of an item is $1,$ , so $p$ does not win any item from the first row, and 
All items of the first row are taken by $\frac{n}{2}$ of the other agents. These $\frac{n}{2}$ agents surpass a payment of $1$ and become inactive. 

%From now on, since items are substitutes along rows, an item with the highest marginal value for $p$ is taken from the lower row available.

In each of the next $n$ rounds, at least one of the next $\frac{n}{3}$ other agents bids $\frac{1}{2}$, and upon winning (note that $p$ bids $\frac{1}{2}$ in each of these rounds), takes an item from the second row (i.e., $e_{1,j})$. Each such agent surpasses a payment of 1 exactly when winning her 3rd item, and becomes inactive.
% after taking three items.

%$p$ bids $\frac{2}{3}$, again the algorithm brake ties adversarially so $\frac{n}{3}$ of the other agents win the items in the second rows exhausting their budget. 
The run proceeds in the same way, where for every $i$, $\frac{n}{q_{i}}$ of the other agents bid $\frac{1}{q_{i}-1}$, win all the items in the $i$'th row, and become inactive. Note that  each such agent surpasses a payment of $1$ exactly when winning her $q_i$th item, and becomes inactive. Note that we use the property of $q_{i}\mid n$ for every $i\leq k$.

Thus, the number of other agents that take all items from rows $1$ to $k$ is:
\[
\sum_{i=1}^{k}\frac{n}{q_{i}}=n\cdot\sum_{i=1}^{k}\frac{1}{q_{i}}=n\cdot(1-\frac{1}{q_{k+1}-1})=n\cdot(1-\frac{1}{n})=n-1
\]

Thus, there are sufficiently many other agents to take all items from rows $1$ to $k$, and agent $p$ gets items only from row $k+1$. As they are substitutes, the total value received by $p$ is $1$.

Consider for $\rho'>\rho_{k}$ the altruistic version of the bidding game in which an agent becomes inactive after spending a $\rho'$ fraction of her budget, and suppose that $p$ executes the $proportional(\rho')$ bidding strategy. 
%then each agent becomes inactive when surpassing a payment of some $\alpha>1$. 
% the bids of $p$ in each round are strictly smaller than those described above.
On the instance $I_k$ described above, the same run of the bidding game holds, and $p$ does not get a bundle of value $\rho' APS_p$, but rather only $\rho_k APS_p$. Hence, $I_{k}$ serves as an example showing that our proof of Theorem~\ref{thm:1/3_APS_guarantee} does not extend to values of $\rho$ larger than $\rho_{k}$. The same holds for every $\rho>\lim_{k\to\infty}\rho_k$ (by enlarging $k$, we can make $\rho_k$ as close as we wish to $\lim_{k\to\infty}\rho_k$).
%\gbc{Uri, notice that this paragraph below suggest how to use this claim in \Cref{thm:equal-10/27}}
%\ufc{remove: Since obviously $\rho_{k}$ is close to} \ufe{By enlarging $k$, we can make $\rho_k$ as close as we wish to $\lim_{k\to\infty}\rho_k$. Hence,  for any $\rho>\lim_{k\to\infty}\rho_k$, there is an \ufc{remove: witness} \ufe{instance} $I_{k}$ on which $proportional(\rho)$ does not guarantee $p$ a $(\rho)$-fraction of her $APS$.
\end{proof}
}




\begin{remark}
The negative example in Proposition~\ref{hard instance altruistic version} can be modified by replacing (for every $j$) the single item $e_{k+1,j}$ of value~1 by $q_k - 1$ items, each of value $\frac{1}{q_k - 1}$ (the same value as that of item $e_{k,j}$). In this modified version, the adversarial run is changed so that other agents win all items $e_{ij}$ for $i \le k$ (their budgets exactly suffice for this), and agent $p$ can take the remaining items from one of the bundles of the MMS partition (the remaining items in different MMS bundles are substitutes to each other and do not provide additional marginal value), getting a value of~1. This modified example is useful in illustrating that for certain variations of the bidding game (considered by the authors but omitted here), bidding strategies similar to the ones considered in the proof of Theorem~\ref{thm:equal-10/27}  do not lead to approximation ratios that are significantly better than those proved in Theorem~\ref{thm:equal-10/27}).  
\end{remark}

%\ufc{remove:
%\begin{remark}
%One may try to improve $\rho$ by allowing some form of bidding on bundles (where the winner is the agent whose bid per item is highest). In the above example, the submodular agent will start with a double bid of~12, and after $n/2$ additive agents win, switch to a double bid of~9 (instead of a single bid of~6). Even if other agents win with a double bid of~9 (or single bid of $\frac{9}{2}$) and take items of value~6, the submodular agent is happy, as these items are substitutes.

%However, the, example can be modified, replacing $(6,6,3,1)$ by $(6,3,1, 1, 1, 1, 1, 1, 1)$, and full budget is 12. Now even if the agent starts bidding 12 on pairs, $n/2$ agents might take all the items of value~6. If she then bids by item values, $n/3$ agents might take all the items of value~3. (We assume that once an agent paid more than 6, she is out of the game. Hence other agents can take only three items of value~3, and not~4. Also, if our agent overbids, she may be allowed to win, but herself will be out of the game once she reaches a value of~6.) Then $n/7$ agents might take $n$ items, from each bundle one of the items of value~1. Then every bundle has only content $(1, 1, 1, 1, 1, 1)$, and if the items are substitutes across bundles, the total value that remains is~6. 
%\end{remark}}


The following proposition shows that in \Cref{thm:1/3_APS_guarantee}, the value of $\rho$ cannot be improved to a constant (independent of $b_p$) larger than $\frac{1}{3}$. Its proof is similar to the proof of Proposition~\ref{hard instance altruistic version}, with some relatively straightforward modifications. For completeness, its full proof is presented in the appendix (Section~\ref{sec:example}).

\begin{restatable}{reprop}{exampleThird}
%\begin{proposition}
\label{no rho larger than 1/3 for original bidding game}
% \ufc{remove: $\rho=\frac{1}{3}$ is the largest fraction, of which an agent $p$
% executes $proportional(\rho)$, is guaranteed to have $\rho$ fraction
% of her $MMS$ (and hence APS). In other words, for every constant
% $\varepsilon>0$, there exists an instance and an adversarial run
% of the bidding game, on which an agent that executes $proportional(\frac{1}{3}+\varepsilon)$ does not obtain a $(\frac{1}{3}+\varepsilon)$ fraction of her MMS, even in the case of equal entitlements.}
For every constant $\rho > \frac{1}{3}$, there is an allocation instance with equal entitlements and an adversarial run of the bidding game, in which an agent $p$ that has a submodular valuation function and uses the $proportional(\rho)$ bidding strategy gets a bundle of value smaller than $\rho MMS_p$.
\end{restatable}
%\end{proposition}


% ##################################################

%\gbc{Once you approve the above shore version of the proof, we can remove the below full proof. don't worry it is saved in the History file.}



%\ufc{Perhaps comment out this remark, as it is too much of a digression.}


