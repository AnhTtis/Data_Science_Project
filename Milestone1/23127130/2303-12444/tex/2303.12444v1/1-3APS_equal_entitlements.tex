
%Recall .
%\gbc{Removed: \APSbidding* }
% \begin{lemma}
% \label{thm:1/3_APS_guarantee}
% For an agent $p$ with a submodular valuation function, setting $\rho={\frac{1}{3-2b_p}> \frac{1}{3}}$, the $proportional(\rho)$ bidding strategy guarantees agent $p$ a value of at least $\rho \cdot APS_p$. (In the case of equal entitlements, this gives $\rho=\frac{n}{3n-2}$.)
% \end{lemma}

%\gbc{Should we remove this presentation of $p$? - it appears in the presentation of $proportional(\rho)$. }
Consider an agent $p$ with valuation function $v_{p}$. 
We begin by proving \Cref{thm:1/3_APS_guarantee} under the simplifying assumption {that there are no large items i.e., every item satisfies $v_{p}(e) \le 2\rho APS_p$}. 
%\ufc{In the following lemma, does the agent underbid if $b_p > \frac{1}{2}$?)}
%\gbc{In the presentation of $proportional(\rho)$ we defined it for $0<\rho\leq\frac{1}{2}$. The next lemma states that we can plug in $\rho=\frac{1}{3-2b_p}$ and get at $\rho$ fraction guarantee. In this case $\rho>\frac{1}{2}$ iff $b_p>\frac{1}{2}$ which leads to underbidding as you suggest. Therefore I think we should change the restriction of $\rho\leq\frac{1}{2}$ in the presentation of the proportional strategy. Why is underbidding problematic? I agree it does not seem natural. I think this restriction came originally in order to claim that if there are no large items, i.e., $v_p(e)\leq 2\rho APS_p$, then the agent is able to raise a full bid. But, by truncating $v_p$ with $APS_p$ we obtain that even if we restrict $\rho\leq\frac{1}{3-2b_p}$ it suffices to show that $p$ is able to raise a full bid when no large items exist}

\begin{lemma}
\label{lem:1/3 guarantee no large items}
For an agent $p$ with a submodular valuation function, if $v_{p}(e) \le 2\rho {APS_p}$ for every item $e$, then by setting $\rho={\frac{1}{3-2b_p}> \frac{1}{3}}$, the $proportional(\rho)$ bidding strategy guarantees agent $p$ a value of at least $\rho \cdot APS_p$.
\end{lemma}

We shall present a sequence of claims that proves \Cref{lem:1/3 guarantee no large items}.

\begin{claim}
\label{claim:2alternatives}
In every round $t$ of the algorithm, at least one of
the following two conditions hold:
\begin{enumerate}
\item Agent $p$'s bid is equal to {$\frac{1}{2\rho}{\cdot\frac{b_p}{APS_p}}\cdot\max_{e \in \items^r}[v_p(e  \mid C^r)]$ (the highest marginal value that a yet unallocated item has, scaled by $\frac{1}{2\rho}{\frac{b_p}{APS_p}}$)} 
\item Agent $p$ already won a bundle with value at least $\rho{APS_p}$.
\end{enumerate}
\end{claim}

\begin{proof}
Suppose that the second condition does not hold. If $p$ did not
win an item yet, then she still has her entire budget, and using the
assumption that no item has a value greater than $2\rho {APS_p}$, condition~1 holds. Otherwise, agent $p$ won items with a total value less than $\rho{APS_p}$ by time $t$. Submodularity of $v_{p}$ implies that the sequence of remaining maximal marginal values  $\max_{e \in \items^r}[v_p(e \; \mid C^r)]$ is
non-increasing in $r$. %\gbc{or should I refer to \Cref{obsrv: bidding sequence decreasing}?}. 
Hence {$\max_{e \in \items^t}[v_p(e \; \mid C^t)] \le \rho APS_p$. Therefore } $\frac{1}{2\rho}{\frac{b_p}{APS_p}}\max_{e \in \items^t}[v_p(e \; \mid C^t)] \le{\frac{1}{2\rho}\rho b_p\leq\frac{1}{2}b_p}\le {b_p^t}$, and condition~1 holds.
\end{proof}

%\gbc{Removed: Denote the entitlement of agent $p$ by $b=\frac{1}{n}$. As mentioned above, we assume her APS is also equal to $b.$} 
Let $\{\lambda_{S}\}_{S\subseteq\mathcal{M}}$
be the set of weights associated with the APS {for $v_p$} (i.e., for every $S\subseteq\mathcal{M},$
$\lambda_{S}>0\implies v_p(S)\geq {APS_p}$, also $\sum_{S}\lambda_{S}=1$,
and $\forall e\in\mathcal{M},$ $\sum_{S|e\in S}\lambda_{S}\leq b_p$). 

Let $L^{0}\coloneqq\sum_{S}\lambda_{S}v(S)$. Observe that by the
definition of $APS$, $L^{0}\geq APS_{p}$. At the beginning of the
algorithm (when no item has been allocated yet), $L^{0}$ is a lower
bound on the marginal value {that agent $p$ has for} the set of all items. 

Let $f$ denote 
%\ufc{removed: the beginning of} \gbc{I mentioned explicitly here that when referring to time or round $f$, I refer to \textbf{the beginning of the round}, i.e., before the bidding, in order to prevent ambiguity} 
the earliest round after which either all other agents become inactive, or all items have been allocated. Let
$C\subseteq\mathcal{M}$ denote the set of items agent $p$ has by the
end of round $f$, and let $O$ denote the set of items that the other
agents have by the end of round $f$. Define $L^{f}=\sum_{S}\lambda_{S}\cdot v_{p}(S\setminus O\dot{\cup}C\mid C)$.
Namely, $L^{f}$ is the expected marginal value to agent $p$ (who
already holds the set $C$ of items) of a bundle $S$ selected at
random according to the probability distribution over bundles implied
by the coefficients $\lambda_{S}$, after one removes from $S$ those
items that were allocated by the end of round $f$.

\begin{claim}
\label{Claim_first_Bound_Lf}
Let $\tilde{\mathcal{M}}$ be the set of items that remain unallocated after round $f$, i.e., $\tilde{\mathcal{M}}=\mathcal{M}\setminus(O\dot{\cup}C)$. Then $v_{p}(C\cup\tilde{\mathcal{M}})=v_{p}(\mathcal{M}\setminus O)\geq v_{p}(C)+L^{f}$.
In other words, $v_{p}(C)+L^{f}$ is a lower bound on the total
value that agent $p$ will have, if she receives all the remaining items
($\tilde{\mathcal{M}}$).
\end{claim}

\begin{proof}
%\ufc{remove: If there are other active agents at the end of round $f$, {then it is the end of the algorithm, and} all the items have been allocated to agents (i.e., $C\dot{\cup}O=\mathcal{M})$. Thus,} 
{If no items remain after round $f$ (i.e., $C\dot{\cup}O=\mathcal{M}$), then} for
each $S\subseteq\mathcal{M}$, $S\setminus O\dot{\cup}C=\emptyset$
and $L^{f}=0$. Hence, $v_{p}(C\cup\tilde{\mathcal{M}}) =  v_p(C) = v_{p}(C)+L^{f}$, proving the claim.

%\ufc{Removed: "Otherwise, $p$ is the only active agent." This is not relevant here, and also, nothing in the claim or in the definition of $f$ implies that $p$ is active.} 
{If items do remain after round $f$, then} every term $v_{p}(S\setminus O\dot{\cup}C\mid C)$
in the sum of $L^{f}=\sum_{S}\lambda_{S}\cdot v_{p}(S\setminus O\dot{\cup}C\mid C)$,
is a marginal value of a partial set of the not-yet-allocated items
(i.e., $(S\setminus O\dot{\cup}C)\subseteq\tilde{\mathcal{M}).}$
Hence $v_{p}(S\setminus O\dot{\cup}C\mid C)\leq v_{p}(\tilde{\mathcal{M}}\mid C)$.
Since the scalars $\{\lambda_{S}\}$ in the sum $L^{f}=\sum_{S}\lambda_{S}\cdot v_{p}(S\setminus O\dot{\cup}C\mid C)$
are non-negative and add up to $1$, we obtain:
\begin{align*}
L^{f}= & \sum_{S}\lambda_{S}\cdot v_{p}(S\setminus O\dot{\cup}C\mid C)\\
\leq & \sum_{S}\lambda_{S}\cdot v_{p}(\tilde{\mathcal{M}}\mid C)\\
= & v_{p}(\tilde{\mathcal{M}}\mid C)
\end{align*}
Hence
\begin{align*}
v_{p}(C)+L^{f}\leq & v_{p}(C)+v_{p}(\tilde{\mathcal{M}}\mid C)=v_{p}(C\cup\tilde{\mathcal{M}})
\end{align*}
\end{proof}

\begin{claim}
\label{Claim_second_bound_Lf}
$\min\{L^{\text{f}}+v_{p}(C),2\rho{APS_p}\}$ is a lower bound on the final
total value of agent $p$.
\end{claim}
\begin{proof}
First, notice that if $p$ is not active in time $f$, then $p$ spent her entire budget, ${b_p}$. The bidding strategy of $p$ (and submodularity of $v_{p})$ implies that in that case,
$p$ has a value of at least $2\rho {APS_p}$ in time $f$, i.e., $v_{p}(C)\geq 2\rho {APS_p}$
and the claim follows in this case.

Otherwise, $p$ is an active agent {after round} $f$. We consider two cases.
If some items remain after round $f$, then agent $p$ is the only remaining active agent. Hence $p$ is the only agent to win items {from $\tilde{\mathcal{M}}$.}
Then, the agent will keep winning items until she becomes inactive or until she wins all remaining items. By \Cref{Claim_first_Bound_Lf}, $v_{p}(\mathcal{M}\setminus O)\geq v_{p}(C)+L^{f}$. The fact that the agent bids at most {$\frac{1}{2\rho}{\cdot\frac{b_p}{APS_p}}$ times} the marginal value of the item she wins in each round guarantees that agent $p$ gets at least $\min\{L^{f}+v_{p}(C), 2\rho {APS_p}\}$. It remains to handle the case of agent $p$ being active at time $f$ while no items remain. In this case, $L^{f}=0$, so the bound is trivial.
\end{proof}

\begin{claim}
\label{claim_L0_Lf}
The following holds:
\[
L^{0}\leq L^{f}+v_{p}(C)+b_p\cdot\sum_{e\in O}v_{p}(e\mid C)
\]
\end{claim}


\begin{proof}
$\!$
\[
L^{f}=\sum_{S}\lambda_{S}\cdot v_{p}(S\setminus O\dot{\cup}C\mid C)=\sum_{S}\lambda_{S}\cdot v_{p}(S\setminus O\mid C)
\]
For every $S\subseteq\mathcal{M}$ we claim:
\[
v_{p}(S)\underset{1.}{\leq}v_{p}(S\mid C)+v_{p}(C)\underset{2.}{\leq}v_{p}(S\setminus O\mid C)+v_{p}(C)+\sum_{e\in S\cap O}v_{p}(e\mid C)
\]
\emph{proof of inequality 1:}
\[
v_{p}(S\mid C)+v_{p}(C)=v_{p}(S\cup C)-v_{p}(C)+v_{p}(C)=v_{p}(S\cup C)\ge v_{p}(S)
\]
\emph{proof of inequality 2:}

Consider an arbitrary order of the set $S\cap O=\{e_{1},\dots,e_{k}\}$.
Then:

\begin{align*}
v_{p}(S\mid C) 
& =v_{p}\left(S\setminus O\mid C\right)+\sum_{i=1}^{k}v_{p}\left(e_{i}\mid\left(S\setminus O\right)\cup C\cup\left(\bigcup_{j=1}^{i-1}e_{j}\right)\right)\\
& \leq v_{p}\left(S\setminus O\mid C\right)+\sum_{i=1}^{k}v_{p}\left(e_{i}\mid C\right)
\end{align*}

Thus, using the last inequality, we obtain the following:

\begin{align*}
L^{0}=\sum_{S\subseteq\mathcal{M}}\lambda_{S}v_{p}(S)
& \leq\sum_{S\subseteq\mathcal{M}}\lambda_{S}\left(
v_p(S\mid C)+v_p(C)
\right) \\
& \leq\sum_{S\subseteq\mathcal{M}}\lambda_{S}\left(v_{p}(S\setminus O\mid C)+v_{p}(C)+\sum_{e\in S\cap O}v_{p}(e\mid C)\right) \\
& \underset{*}{\leq}v_{p}(C)+b_p\cdot\sum_{e\in O}v_{p}(e\mid C)+\sum_{S\subseteq\mathcal{M}}\lambda_{S}v_{p}(S\setminus O\mid C)\\
& =v_{p}(C)+b_p\cdot\sum_{e\in O}v_{p}(e\mid C)+L^{f}
\end{align*}
where inequality * is since each item $e\in\mathcal{M}$ has a total
weight of at most $b_p$ (by the APS definition). Overall, as we wanted
to show, we obtained the following:
\[
L^{0}\leq L^{f}+v_{p}(C)+b_p\cdot\sum_{e\in O}v_{p}(e\mid C)
\]
\end{proof}



\begin{claim}\label{claim_other_agent_items_reducing_Lf}
Either $v_p(C) \ge \rho {APS_p}$, or
\[
\sum_{e\in O}v_{p}(e\mid C)\leq {2\rho(b^0-b_p){\cdot \frac{APS_p}{b_p}}}
\]
\end{claim}

\begin{proof}
Suppose that $v_p(C) < \rho {APS_p}$. Let $C^{e}\subseteq C$ be the set of items agent $p$ already won
when another agent wins item $e$.
\Cref{claim:2alternatives} implies that
agent $p$ bids {$\frac{1}{2\rho}{\cdot\frac{b_p}{APS_p}}$ times }the highest marginal value of an item w.r.t $C^{e}$.
Hence, when the other agent $i$ wins item $e$, agent $p$ bid is
at least ${\frac{1}{2\rho}{\frac{b_p}{APS_p}}}v_p(e\mid C^{e})\geq {\frac{1}{2\rho}{\frac{b_p}{APS_p}}}v_p(e\mid C)$, and winning item $e$ reduces the budget of agent $i$ by at least ${\frac{1}{2\rho}{\frac{b_p}{APS_p}}}v_{p}(e\mid C)$. Since the
budget of all other agents at the beginning is $b^{0}-{b_p}$, we
obtain 
\[
\sum_{e\in O}\frac{b_p}{2\rho APS_p}v_{p}(e\mid C)\leq\sum_{e\in O}\frac{b_p}{2\rho APS_p}v_{p}(e\mid C^{e})\leq {(b^0-b_p)}
\]
The claim follows by rearranging (scaling both sides by $\frac{2\rho APS_p}{b_p}$).
%\gbc{I think the above inequality is clearer than the one below - so I suggest removing it}
%\[
%\sum_{e\in O}v_{p}(e\mid C)\leq\sum_{e\in O}v_{p}(e\mid C^{e})\leq %{2\rho(b^0-b_p){\frac{APS_p}{b_p}}}
%\]
\end{proof}



We are now ready to prove Lemma~\ref{lem:1/3 guarantee no large items}.

\begin{proof}
Considering \Cref{claim_L0_Lf} and \Cref{claim_other_agent_items_reducing_Lf}, we have that either $v_p(C) \ge \rho{b_p}$, or the following holds:
\[
APS_p\leq L^0\leq v_{p}(C)+L^{f}+b_p\cdot\sum_{e\in O}v_{p}(e\mid C) \leq v_{p}(C)+L^{f}+{b_p\cdot2\rho{\frac{APS_p}{b_p}}(b^0-b_p)}
\]
%\gbc{This is the first place where we use $b_p=\frac{1}{n}$, and the proof splits for the equal and arbitrary entitlements. Therefore, I suggest an alternation, so the assumption of $b_p=\frac{1}{n}$ will arrive at the very last line of the proof.}
%\gbc{Removed:
%Recalling that \gbc{Removed:$APS_p=$}$b_p=\frac{1}{n}$ and $b^0 = 1$, we obtain:
%\[
%v_{p}(C)+L^{f} \ge APS_p\cdot {(1-2\rho b^0+2\rho b_p)} = APS_p\cdot(1 - 2\rho\frac{n-1}{n})
%\]
%By setting $\rho=\frac{n}{3n-2}$}

By rearranging the above, and plugging $b^0=1$ we obtain:
\[
 APS_p\cdot {(1-2\rho b^0+2\rho b_p)} = APS_p\cdot {(1-2\rho +2\rho b_p)} 
 \le v_{p}(C)+L^{f}
\]

By setting $\rho=\frac{1}{3-2b_p}$ 
the above gives $L^{f}+v_{p}(C) \ge \rho{APS_p}$. {So far we obtained that either $v_p(C)\geq\rho APS_p$, or } %\gbc{Removed:This, together with the fact that} 
agent $p$ is guaranteed to have a total final value of $\min\{L^{f}+v_{p}(C),2\rho {APS_p}\}$ (\Cref{Claim_second_bound_Lf}). %\gbc{Remove:, implies that agent} 
{ Hence, in both cases, when setting $\rho=\frac{1}{3-2b_p}$, agent }$p$ is guaranteed to have at least $\rho$ fraction of her $APS$. In the special case of equal entitlements (where $b_p=\frac{1}{n}$) it implies $\rho=\frac{n}{3n-2}$. This completes the proof of ~\Cref{lem:1/3 guarantee no large items}.


\end{proof}

We now restate and prove \Cref{thm:1/3_APS_guarantee}.
\APSbidding*
%\gbc{Maybe restate \Cref{thm:1/3_APS_guarantee} here? as we removed its restating from above.\\}
\begin{proof}
~\Cref{lem:1/3 guarantee no large items} handles the case that $v_p(e) \le 2\rho {APS_p}$ for every item $e$.
It remains to handle the case that there are items $e$ of value $v_p(e) > 2\rho {APS_p}$.

Consider an input instance $I_0$. 
Run the bidding game with $p$ using the proportional bidding strategy. As described in $proportional_2$, {$s$ denotes the {last round} in which there was an unallocated item $e$ with $v_p(e) > 2\rho APS_p$.} 
%\ufc{remove: $s$ denotes the {first round} in which all yet unallocated items satisfy $v_p(e) \le 2\rho APS_p$.} \ufc{Check definition of $s$. Seems that it is used inconsistently. Perhaps the intention is not "in which" but "after which"?} \gbc{It is consistent now - $s$ is the first time/round in which the assumption of no large items holds.} 
If agent $p$ won some item by the end of round $s$, then she has a value of at least $2\rho APS_p$, and we are done. Hence, we may assume that agent $p$ did not win any item in the first $s$ rounds. {Recall the definitions of residual instance $I_s$, $\hat{I_s}$ and $\gamma$ (which is the scaling factor between bids in $I_s$ and $\hat{I_s}$) presented in $proportional_2$.}
%\gbc{Removed: $I_s$ denote the
%{\em residual instance}  that remains after round $s$. 
%\gbc{In the residual instance, we do not really care about the valuation function of other agents, as our promise is against any adversarial bidding of other agents. Furthermore, as we describe next, in each of the first $s$ rounds, another agent wins an item, and becomes inactive (spent her entire budget). So the agents who survive to the residual instance have the same valuation function (since they did not win any item by round $s$). Therefore the following sentence should be removed} \ufc{You are jumping ahead to $I_{\hat{s}}$. First there is a description of the residual instance  $I_s$. In the next paragraph it is explained which parts are redundant, giving rise to $I_{\hat{s}}$.}
%It includes the set of items that were not yet allocated (which we denote by $\mathcal{\hat{M}}$), the valuation function (over the remaining items) of each agent is her marginal valuation function with respect to the items that she already has (for $p$, this is her original valuation function), and each agent has whatever remains from her budget after the $s$ rounds (for $p$, this is her entire original budget). }


By the definition of $s$, in each of the first $s$ rounds,
there is an available item with $v_{p}(e)>2\rho APS_p$. Thus agent {$p$ bids her entire budget $b_p$ in each such round}, and the (other) agent who wins the round spends at least $b_p$. (In the special case of equal entitlement, this means that in each of the first $s$ rounds, some agent wins a single item and becomes inactive. Hence, in the residual instance $I_s$ there are $n-s$ active agents (including $p$), each active agent has her entire original budget, and no active agent has any items.)

%\gbc{Removed: View $I_{s}$ as a new allocation instance, that we refer to as $I_{\hat{s}}$. The agents of $I_{\hat{s}}$ are the $n-s$ active agents of $I_{s}$. The set of items of $I_{\hat{s}}$ is $\mathcal{\hat{M}}$ (those items remaining in $I_{s}$)}. 
Setting $\gamma=b^s\leq 1 - s\cdot b_p$, the entitlement of each agent {in $\hat{I_s}$} is $\hat{b_i^s}=\frac{1}{\gamma} b_i^s$. 
%\gbc{Removed: By \Cref{claim:APS not decrease in I_s} \ufc{need to check that this claim appears and is written properly} the $APS$ of $p$ (with valuation function $v_{p}$) in $I_{\hat{s}}$ is at least as large as the $APS$ of $p$ in $I_{0}$. Scaling the valuation of $p$ to be $\hat{v}_{p}=\gbe{\frac{\gamma b_p}{APS(\hat{\items},\hat{b})}v_p}$, the APS of agent $p$ in $I_{\hat{s}}$ is  $\hat{b}$.}
%\ufc{Need to make notation here consistent with that of page 2. Best way of doing so might include changes in both pages.}

%\gbc{Removed:Observe that $I_{\hat{s}}$ is an allocation instance in which {the APS of $p$ is $\hat{b}$, and} $\hat{v}_{p}(e) \le 2\rho\hat{b}$ for every item $e \in \mathcal{\hat{M}}$. }

{
Recall that agent $p$ simulates the bidding strategy ($proportional_1$) on $\hat{I_{s}}$. A bid $p_{i}^{r}$ in $I_{s}$ is interpreted as a bid of $\frac{1}{\gamma}\hat{p}_{i}^{r-s}$ in $\hat{I_{s}}$. 
% Remove: Namely other agent bids $p_{i}^{r}$ in $I_{s}$ then $p$ simulate her bid in $\hat{I_{s}}$ as $\hat{p}_{i}^{r-s}=\frac{1}{\gamma}p_{i}^{r}$.
% Then, agent $p$ knows how to bid by considering scaling her bids in $\hat{I_{s}}$ by $\gamma$. Since the ratio between a budget of an agent in $I_{s}$ and in $\hat{I_{s}}$ is $\frac{1}{\gamma}$, by taking a bid of another agent $i$ in $I_{s}$ and scaling it by $\frac{1}{\gamma}$ we get a legal bid of the agent in $\hat{I_{s}}$. Similarly, scaling by $\gamma$ a bid of agent $p$ in $\hat{I_{s}}$ induces a legal bid in $I_{s}$.
By \Cref{claim:APS not decrease in I_s}, the APS of agent $p$ at $\hat{I_{s}}$ 
%\ufc{it would be better if the APS is the same, as otherwise the statement is not true} %\gbc{I corrected \cref{claim:APS not decrease in I_s}} \gbc{Removed: is at least as the}
{stays the same as the} APS of $p$ in the original instance. Hence, in $I_s$, agent $p$ will get the same bundle as she gets in the run on $\hat{I_{s}}$. By \Cref{lem:1/3 guarantee no large items}, this bundle is of value at least $\rho APS_p$, as desired.}
%\gbc{Removed:
%By Lemma~\ref{lem:1/3 guarantee no large items}, the proportional bidding strategy \gbe{($proportional_1$)} guarantees $p$ a value (according to $\hat{v}_p$) of at least $\rho\gbe{APS_p}\gbc{Removed:\hat{b}}$. We claim that this implies that the proportional bidding strategy of $p$ in $I_{0}$ guarantees a value of at least $\rho \gbe{APS_p}\gbc{Removed:b_p}$ \gbc{Removed:(according to $v_p$)}. After $s$ rounds in $I_0$, the proportional bidding strategy of $p$ in $I_s$ is a scaling of the proportional bidding strategy in $I_{\hat{s}}$, with a scaling factor of \gbe{$\gamma$}\gbc{$\frac{1}{\gamma}$}. \ufc{Since we changed the definition of the proportional bidding strategy, there is no scaling here.} Namely, a bid of $\alpha$ in $I_{\hat{s}}$ is replaced by a bid of \gbe{$\gamma\alpha$}\gbc{Removed:$\frac{1}{\gamma}\alpha$} in $I_{s}.$ 
%We now show that this strategy guarantees a $\rho b$ value to $p$ in $I_{s}$. 
%Assume for the sake of contradiction that for some adversarial bidding of the other agents, $p$ does not receive at least $\rho \gbe{APS_p}\gbc{Removed:b_p}$ in $I_{s}$. Then, by scaling the adversarial bids of the other agents by \gbe{$\frac{1}{\gamma}$}\gbc{Removed:$\gamma$}, we have a run of the algorithm on $I_{\hat{s}}$ in which the proportional strategy of $p$ does not yield a $\rho\gbe{APS_p}\gbc{Removed:\hat{b}}$ value, contradicting Lemma~\ref{lem:1/3 guarantee no large items}. 
%Hence, we have a strategy for $p$ to ensure $\rho b$ value in $I_{s}$.}
\end{proof}
