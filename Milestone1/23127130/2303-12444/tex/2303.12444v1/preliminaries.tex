\subsection{Preliminaries}
\label{sec:preliminaries}

\begin{claim}\label{cl:item_reduce}
Let $I$ be an allocation instance with a set $\items$ of items, let $i$ be an agent with entitlement $0 < b_i < 1$, let $j$ be an agent with entitlement $b_j \le b_i$, and let $e \in \items$ be an arbitrary item. Consider an allocation {instance} $I'$ that differs from $I$ only in that agent $i$ is removed, the entitlements of all other agents are scaled by $\frac{1}{1 - b_i}$ (so that they sum up to~1), and item $e$ is removed. Then the APS of $j$ in $I'$ is at least as large as the APS of $j$ in $I$. Moreover, if $I$ was an instance with equal entitlement, then the same applies to the MMS of $j$
\end{claim}


\begin{proof}
The claim above states that for each agent $j$ with $b_{j}\leq b_{i}$,
if we define $b_{j}^{*}:=\frac{1}{1-b_{i}}\cdot b_{j}$, then:
\[
APS_{j}(\mathcal{M}\setminus\{e\},b_{j}^{*})\geq APS_{j}(\mathcal{M},b_{j})
\]

Let $\{\lambda_{S}S\}$ be a fractional APS  partition for $j$ in the original instance $I$. Let $w=\sum_{S|e\in S}\lambda_{S}$ denote the total weight of bundles in which $e$ participates in the fractional partition. Note that by definition of the APS, $w \le b_j$, and since $b_j \le b_i$, we also have $w \le b_i$. Define the following new weights:
\[
\lambda_{S}^{*}=\begin{cases}
\frac{1}{(1-w)}\cdot\lambda_{S} & \text{ if }e\not\in S\\
0 & \text{ otherwise}
\end{cases}
\]

We show that the new weights induce a fractional APS partition for the new instance $I'$ (with agent $i$ and item $e$ removed). Indeed, every item is in bundles of total weight at most $b^*_j$:
\begin{align*}
\forall e' & \in\mathcal{M}\setminus\{e\},\quad\sum_{S|e'\in S}\lambda_{S}^*\leq\frac{1}{(1-w)}\sum_{S|e'\in S}\lambda_{S}\leq\frac{1}{(1-w)}b_{j}\leq\frac{1}{(1-b_{i})}b_{j}=b_{j}^{*}
\end{align*}

The sum of weights of all bundles is 1:
\[
\sum_{S}\lambda_{S}^{*}=\sum_{S|e\notin S}\lambda_{S}^{*}=\sum_{S|e\notin S}\frac{1}{(1-w)}\lambda_{S}=\frac{1}{(1-w)}\cdot(1-w)=1
\]

Trivially all bundles in the support of $\{\lambda_{S}^{*}S\}$
are of value at least $APS_{j}(\mathcal{M},b_{j})$, proving the claim.
\end{proof}



\begin{definition}
\label{def:$v^t_p$}
    Let $v_p$ be the valuation function of agent $p$ and let $t\geq 0$ be a scalar, then we define $v_p^t$, the valuation $v_p$ truncated at $t$, as follows:
\[
v_p^t(B)\coloneqq\min\{v_p(B),t\}
\]
\end{definition}

Observe that if $v_p$ is submodular, then also $v_p^t$ is submodular.


\begin{claim}
\label{claim:truncation}
     Let $v_p$ be the valuation function of an agent $p$, and set {$t\leq MMS_p$} (respectively {$t\leq APS_p$}). Then the {new }$MMS$ ($APS$) of agent $p$ {is} $t$ if we consider her valuation function to be $v_p^t$ (\Cref{def:$v^t_p$}). {In the special case of $t=MMS_p$ ($t=APS_p$), this implies that the MMS (APS) of the agent remains unchanged.}
\end{claim}

\begin{proof}
    The partition (fractional partition) that certifies that the MMS (APS) with respect to $v_p$ is at least $t$, certifies the same with respect to $v_p^t$ (because every bundle that has value at least $t$ with respect to $v_p$ has value $t$ with respect to $v_p^t$). The MMS (APS) with respect to $v_p^t$ cannot be larger than $t$, as no bundle has $v_p^t$ value larger than $t$.
\end{proof}

%%%%% Beginning of claims for 1/3 Arbitrary-entitlement APS %%%%%%%
