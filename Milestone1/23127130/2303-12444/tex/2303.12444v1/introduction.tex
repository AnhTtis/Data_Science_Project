\begin{abstract}
We consider the problem of fair allocation of indivisible goods to agents with submodular valuation functions, where agents may have either equal entitlements or arbitrary (possibly unequal) entitlements. We focus on share-based fairness notions, specifically, the maximin share (MMS) for equal entitlements and the anyprice share (APS) for arbitrary entitlements, and design allocation algorithms that give each agent a bundle of value at least some constant fraction of her share value. For the equal entitlement case (and submodular valuations), Ghodsi, Hajiaghayi, Seddighin, Seddighin, and Yami [EC 2018] designed a polynomial-time algorithm for $\frac{1}{3}$-maximin-fair allocation. We improve this result in two different ways. We consider the general case of arbitrary entitlements, and present a polynomial time algorithm that guarantees submodular agents $\frac{1}{3}$ of their APS. For the equal entitlement case, we improve the approximation ratio and obtain $\frac{10}{27}$-maximin-fair allocations. Our algorithms are based on designing strategies for a certain bidding game that was previously introduced by Babaioff, Ezra and Feige [EC 2021].

% \ufc{remove the rest} In both algorithms, items are allocated via a bidding game. Each agent gets a budget (correlated to her entitlement), and in every round, the highest bidder gets to choose an item and pay her bid. We show that in the case of arbitrary entitlement, a submodular agent has a bidding strategy that guarantees herself a $\frac{1}{3}$ fraction of her APS, independently of other agents' bids or valuation function. Moreover, as shown in \cite{BEF21}, in this bidding game, additive agents have a bidding strategy ensuring themselves a $\frac{3}{5}$-fraction of their APS.
% \gbc{Should we also stress some of the tightness results by negative examples? Or mentioning the fact that this bidding game is not ideal for agents with a valuation function drawn from a family larger than submodular, such as an XOS, which we show that no bidding strategy guarantees a constant fraction}
\end{abstract}

\section{Introduction}


    
%\item Explain the fairness notions that we desire. Assume that agents have valuation functions. Care only about their own bundle. Want MMS (reference Budish) for equal entitlement and APS (reference BEF) for arbitrary entitlement. Alert the reader that for equal entitlement, APS might be larger than MMS. Explain that we need to settle for approximations (cite KPW that even for additive valuations).


%\item Related work. There are many different definitions of fairness, and this is somewhat unavoidable (depends on social norms, etc.) Alternative notions of fairness (envy based, share based for unequal entitlement). Known MMS and APS approximations for various classes of valuations. Also there is work on chores (just give references). 

%\item Definitions and preliminaries. Oracle access to valuations. Formal definition of MMS and APS (including dual definition). Explanation that they are NP-hard to compute (even for additive), and for submodular, even to approximate (reference) . Note that nevertheless, this will not prevent us from achieving polynomial time allocation algorithms (assuming value queries). Bidding game. Truncated valuations (preserves MMS, APS). Other stuff as needed. (Some standard proofs can be moved to appendix (for completeness).)
%\gbc{I discussed NP hardness of computing APS and MMS values in the 'notions of approximation' subsection}

%\gbc{As we discussed, there is no result stating hardness of approximation of the APS for submodular agent. After referring me to the paper 'Inapproximability Results for Combinatorial Auctions with Submodular Utility Functions' of Khot, Lipton, Markakis, and Mehta -  showing an inaproximability result of the \emph{maximum welfare} problem with submodular agent, I suspect that the same reduction to an allocation problem they built has the local problem that the yes instance translated to a trivial APS of value $1$ and for the NO instance, any bundle of size $\leq$ average size of bundl ($\frac{m}{n}$) has the property that it is of value less than $\alpha+\varepsilon$ where $\alpha$ is the constant gap of approximation. As we discussed, this property implies that the APS is less equal $\alpha+\varepsilon$ (a sufficient but not necessary condition for the APS being $\leq\alpha+\varepsilon$}




We study the problem of fair allocation of a set $\mathcal{M}$ of $m$ indivisible items to a set $\mathcal{N}$ of $n$ agents, where each agent $i$ has a monotone non-negative submodular valuation function $v_{i}:2^{\mathcal{M}}\to\mathbb{R}_{\geq0}$ (the definition of submodularity appears in \Cref{sec: Classes of valuation functions}). In fair allocation settings, agents do not pay for the items. Instead, agents have arbitrary, possibly unequal entitlements to the items. Specifically, each agent $i$ has an individual entitlement $0<b_i\leq1$, and the entitlements sum up to~1 ($\sum_{i=1}^n b_i = 1$). We focus on share-based fairness notions, specifically, the maximin share (MMS) for equal entitlements~\cite{Budish11} and the anyprice share (APS) for arbitrary entitlements~\cite{BEF21} (definitions of these notions appear in Section~\ref{sec:basic definitions}). We design allocation algorithms that give each agent a bundle of value at least some constant fraction of her share value.

% \ufc{remove this paragraph}
% \gbe{The agents have submodular valuations (further details in \Cref{sec: Classes of valuation functions}) and no money or payments involved. (Though one should distinguish that there is no restriction for the items to encapsulate a monetary value, and the algorithms finding such \emph{fair} allocations may use techniques involving payments such as auctions, where each agent has an initial virtual budget.)}
%\gbc{Remove:
%\ufc{Move to the related work section?} There has been significant previous work on the special case of agents having equal entitlements to the items \gbe{\cite{AMNS17}, \cite{BarmanKumar17}, \cite{GargTaki21}} \ufc{cite examples}, and on the general case of agents have arbitrary entitlement to the items (for example \cite{DBLP:journals/mor/BabaioffNT21}, \cite{FGT19}, \cite{CSZ19}, \cite{AzizMS20}).}
{The problem of fairly allocating indivisible items has been extensively studied, with various settings of the problem considered (items might be goods or chores, agents may have equal or unequal entitlements), and different fairness criteria explored, such as envy-based notions and share-based principles. See for example \cite{Aziz_Survey22, Amanatidis_survey22} and references therein.}

The problem of allocating indivisible items to agents arises naturally in the real-world. We present one such example that illustrates aspects addressed in our work (unequal entitlement, no payments, non-additive valuation functions). The NBA draft is an annual event of the National Basketball Association (NBA) in which new eligible basketball players (typically graduate college players) are allocated to NBA teams. 
%via a picking sequence\gbe{. To keep things simple, we bring here only the essence of the NBA draft and allow ourselves a small amount of inaccuracy in our description. 
The allocation mechanism is a picking sequence composed of two rounds. In each round, each team in its turn picks a player among the eligible players. In this example, teams correspond to the agents, and basketball players correspond to the items.
%This example is interesting since it emphasizes some aspects of the problem we focus on here in our paper. First, 
The teams do not have equal entitlement to the players. Teams of poorer performance in the previous season have higher entitlement that those of better performance (a policy that tries to maintain the competitiveness of the teams). This unequality of entitlement is reflected in the allocation mechanism, by having teams with higher entitlement pick earlier than teams of lower entitlement, in each of the rounds of the picking sequence. (In practice, the allocation mechanism is somewhat more complicated than described above, but these additional complications are not relevant to our presentation, and hence omitted.) 
%the order of selection of the teams is determined basically with an opposite correlation to the team performance in the previous season - 
Teams do not pay for the right to pick a player. (They will of course later pay the salary of the player, but these monetary aspects are only an aspect that determines how desirable the player is for the team, and are not part of the allocation processes.)
The interests of teams do not seem to fit a model of an additive valuation function. For example, 
it is likely that the combined value for a team of two players that play in the ``center" position is smaller than the sum of values of the individual players.

 

{
\subsection{Fairness notions}

In our paper, we focus on notions of \emph{fairness} known as \emph{share-based} notions. In share-based fairness, each agent cares only about her own bundle in the allocation, and expects its value to reach at least a certain target value.
% Thus, other notions of \emph{fairness} for the indivisible setting have been proposed. 
One such fairness notion is the \emph{maximin share}, which was introduced by Budish \cite{Budish11}. The \emph{maximin share} (abbreviated as \emph{MMS}) of an agent is defined to be the maximum value she can ensure for herself if she were to partition the goods into $n$ bundles and then receive a minimum valued bundle. A \emph{maximin fair} allocation is an allocation in which each agent gets a bundle that she values at least as her maximin share.
}
%\gbc{Removed: \ufc{Inconsistent: you motivate arbitrary entitlement but do not motivate equal entitlement.} There is a large body of work about fair allocations that considers the case of equal entitlement agents. \ufc{give references} But real-life occasions sometimes suggest that this might not be the case. \ufc{Replace the previous ambiguous sentence by saying what you wish to consider.} For example consider company shareholders, where each investor owns a different amount of shares. Then, if we think of the share-holders as the agents, we would like to have an allocation that takes into account the distribution of the shares among investors. Consequently, our interest here will be focused also on the more general case, where agents have arbitrary (possibly unequal) entitlements to the items. More formally the problem is now equipped with a non-negative vector \textbf{$\boldsymbol{b}$} of entitlements that satisfies $\sum_{i=1}^{n}b_{i}=1$, and the entitlement of agent $i$ is $b_{i}$.} 

The maximin share notion is applicable when agents have equal entitlement. A notion of fairness for the case of arbitrary entitlements was presented by Babaioff, Ezra, and Feige \cite{BEF21}, and is referred to as the \emph{AnyPrice share} (abbreviated as APS). See Definition~\ref{def:APS}.
In the special case of equal entitlements, the AnyPrice share of an agent is at least as large as her \emph{Maximin share}, and sometimes strictly larger.


%\ufc{remove: where the inequality is sometimes strict.}
%\gbc{In contrast to the MMS, the APS does not have an easy intuitive literal explanation. The most straightforward way to understand the APS is by looking at the formal definition. I am worried that the reader might suspect we are not giving a formal definition. Should we add a sentence states 'formal definition of the \emph{APS} will be shown soon'}

\subsection{Notions of approximation}

In the context of the notions of the MMS and APS, there are two different tasks that involve approximations.

\begin{itemize}

\item Approximating the value of the MMS (or APS) of an agent. Both the MMS and APS of an agent are NP-hard to compute even if the valuation function is additive, in which case computing the exact value of the MMS is strongly NP-hard~\cite{Woeginger97}, whereas computing the exact value of the APS is weakly NP-hard~\cite{BEF21}. For submodular valuation functions (a class that is considered in this paper), computing the MMS and the APS is APX-hard. See Section~\ref{sec:APX} for more details.


\item Approximating a fair allocation (maximin-fair allocation, AnyPrice-fair allocation, etc.). For $\alpha\in(0,1)$, we say that an allocation is $\alpha$-\emph{maximin-fair }(resp. $\alpha$-\emph{AnyPrice-fair)} if it gives every agent at least an $\alpha$ fraction of her MMS (resp. APS). Kurokawa, Procaccia, and Wang \cite{KPW18} showed that for every $n\geq3$, there exists an instance with $n$ additive agents, such that no \emph{maximin-fair} allocation exist, i.e., for each allocation, there exist an agent which gets a bundle she values strictly less than her MMS. As the APS of an agent is at least as large as her MMS, there are instances with additive valuations and no APS-allocation.
\end{itemize}


In our paper we will focus on the latter task, approximating MMS-fair (APS-fair) allocations.
%\ufc{A bit out of place. Perhaps remove.} Even though the MMS and APS are NP-hard to compute using value queries, it will not prevent us from achieving polynomial time allocation algorithms (assuming value queries).}


\subsection{Classes of valuation functions}
\label{sec: Classes of valuation functions}

Throughout this paper we assume that valuation functions are {\em normalized} (the value of the empty set is~0) and {\em monotone} ($v(S) \le v(T)$ for $S \subset T$).

% \ufc{remove: To tackle the fair allocation problem, there are some assumptions we can make. A natural assumption is to limit the valuation functions of the agents by assuming that they belong to a specific family of valuation functions. For example one may assume that valuation functions are additive, namely, every item has a value, and the value of a bundle is the sum of the values of the items.} 
Lehman, Lehman, and Nissan \cite{DBLP:journals/geb/LehmannLN06}introduce a hierarchy of families of valuation functions, and two prominent members of this hierarchy are \emph{Submodular} and \emph{XOS} valuations, as defined below.

\begin{definition}
\textbf{(Submodular valuation) }a valuation function $v\colon2^{\mathcal{M}}\to\mathbb{R}_{\geq0}$ is submodular if the following (equivalent) conditions hold: 
\begin{itemize}
\item $\forall S,T\subseteq\mathcal{M}$ we have $v(S)+v(T)\geq v(S\cup T)+v(S\cap T)$
\item $\forall S,T\subseteq\mathcal{M}$ with $S\subseteq T$, and for any
$j\in\mathcal{M\setminus}T$ we have $v(S\cup\{j\})-v(S)\geq v(T\cup\{j\})-v(T)$
\end{itemize}
\end{definition}
\begin{definition}
\textbf{(XOS valuation)} a valuation function $v\colon2^{\mathcal{M}}\to\mathbb{R}_{\geq0}$
is XOS (also referred to as {\em fractionally subadditive}) if there exist a finite set of \textbf{additive} valuations $\{v_{1},v_{2},\dots,v_{k}\}$ such that
\[
\forall T\in\mathcal{M},\quad v(T)=\max_{j\in[k]}v_{j}(T)
\]

\end{definition}

%\gbc{Removed:$\quad$
%\begin{definition}
%\textbf{(Subadditive valuation) }a valuation function $v\colon2^{\mathcal{M}}\to\mathbb{R}_{\geq0}$
%is subadditive if the following holds:
%\[
%\forall S,T\in\mathcal{M},\quad v(S)+v(T)\geq v(S\cup T)
%\]
%As shown in \cite{DBLP:journals/geb/LehmannLN06}, the hierarchy
%of these classes is as follows:
%\[
%Additive\subsetneq Submodular\subsetneq XOS\subsetneq Subadditve
%\]
%\end{definition}
%}\\


As shown in \cite{DBLP:journals/geb/LehmannLN06}, the hierarchy of these classes is as follows:
\[
Additive\subsetneq Submodular\subsetneq XOS
\]

{
Let us briefly discuss the representation of valuation functions.
%Recall that our allocation problem is defined by the parameters $|\mathcal{M}|=m$, $|\mathcal{N}|=n$, and the valuations functions of the players. 
The explicit representation of a valuation function requires exponential space in $m$ (its domain size is $2^{m}$). 
%However, one may strive to measure the complexity of an algorithm as a function of $m$ and $n$. Thus, 
Consequently, as described in \cite{DBLP:journals/geb/LehmannLN06}, one typically assumes query access to valuation functions, rather than having an explicit representation for them.
%when designing an algorithm, one may consider various types of access to the valuation functions of the agents.
Our paper will focus on the value queries model (i.e., the function is implicitly given through a value oracle). In this model, a query is a set of items, and the answer is the value of the function on this set of items. We assume that each such query takes unit time. Consequently, polynomial allocation algorithms may make only polynomially many value queries to the underlying valuation functions. %assuming value query access executes a polynomial amount of queries.
}




\subsection{Our main results}

Our main results concern the existence (and polynomial time computability) of approximate MMS-fair (APS-fair) allocations in the equal entitlements (arbitrary entitlements) case for submodular agents. Previously, for the equal entitlement case, Ghodsi, Hajiaghayi, Seddighin, Seddighin and Yami \cite{DBLP:conf/sigecom/GhodsiHSSY18} designed a polynomial-time algorithm for $\frac{1}{3}$-\emph{maximin-fair} allocations, and designed instances in which in every allocation, at least one agent gets a bundle of value not larger than a $\frac{3}{4}$ fraction of her MMS. 

Our results are based on a bidding game mechanism presented by \cite{BEF21}, where each agent gets an initial budget, and in every round, the highest bidder gets to choose an item. We show here that in this bidding game, a submodular agent has a bidding strategy that guarantees at least a $\frac{1}{3}$ fraction of her APS in the arbitrary entitlements case. 
%\gbc{Should I be more accurate and state that the constant fraction of the guarantee is $\frac{1}{3-2b_p}$?}. 
In the case of equal entitlements, we show that with a slight modification of the bidding game (named as the \emph{altruistic version} of the bidding game, {see \Cref{sec:Approximated MMS-fair existance}}), there is a bidding strategy for a submodular agent that guarantees at least a $\frac{10}{27}$ fraction of her MMS.

\begin{restatable}{rethm}{APSbidding}\label{thm:1/3_APS_guarantee}
 Consider the bidding game described above, and an agent $p$ with a submodular valuation function and entitlement $b_p$. Setting $\rho={\frac{1}{3-2b_p} > \frac{1}{3}}$, a bidding strategy referred to as $proportional(\rho)$ guarantees agent $p$ a value of at least $\rho \cdot APS_p$. (In the case of equal entitlements, this gives $\rho=\frac{n}{3n-2}$.)
\end{restatable}

\begin{restatable}{rethm}{altruistic}
\label{thm:equal-10/27}
    Consider the altruistic version of the bidding game in the equal entitlement case. Every agent with a submodular valuation that uses a bidding strategy referred to as the proportional bidding strategy is guaranteed to get a bundle of value at least a $\rho = \frac{10}{27} + \Omega(\frac{1}{n}) > 0.37037$ fraction of her MMS. 
\end{restatable}

The bidding strategies used in our two main theorems require computing the APS (or MMS) of the agents, tasks which are APX-hard. Nevertheless, known techniques~\cite{DBLP:conf/sigecom/GhodsiHSSY18} allow us to deduce the following corollary:

\begin{restatable}{recor}{PolySubmodular}
\label{cor:polyTimeSubmodular}
There is a polynomial time $\frac{1}{3}$-APS algorithm for submodular valuations with arbitrary entitlements. For the equal entitlement case, there is a polynomial time $\frac{10}{27}$-MMS algorithm for submodular valuations.
\end{restatable}

Our results concerning submodular valuations can be combined with previous results of \cite{BEF21} that concern bidding strategies for agents with subclasses of submodular valuations, namely additive valuations and unit demand valuations. This results in allocation algorithms for setting with submodular valuations, in which agents that have valuations coming from simple sub-classes of submodular valuations get improved guarantees.
%\gbc{A general note - from the above sentence one might thinks that the resulting algorithm uses the fact that the other agent belong to a subclass of submodular valuation. but this property does not necessary - if one comes up with a strategy for a class $c$ not contained in submodular, this may lead to a similar algorithm. (Although class $C$ can not contain XOS, as we have the negative result).}

% \ufc{remove}\gbe{\cite{BEF21} showed that additive an agent has a strategy for guaranteeing  $\frac{3}{5}$ of her APS, and an agent with unit-demand has a strategy for guaranteeing herself her entire APS value. Combining these results with our result, we have that in an instance with a combination of submodular, additive, and unit-demand agents, these bidding strategies guarantee each agent their approximated share simultaneously.}

%Corollary ~\ref{cor:submodular, additive, and unit demand algorithm}

\begin{restatable}{recor}{PolyEnsamble}
\label{cor:submodular, additive, and unit demand algorithm}
There is a polynomial time allocation algorithm, which simultaneously guarantees for submodular agents $\frac{1}{3}$-APS, for additive agents $\frac{3}{5}$-APS, and for Unit-demand agents $1$-APS.
\end{restatable}

A class of valuations that is more general than submodular valuations is XOS valuations. 
A natural question concerning the bidding game is whether agents with XOS valuations have strategies that guarantee a constant fraction of their APS.  The answer for this question is negative.



\begin{restatable}{reprop}{XosHardness}
\label{prop:XOS_hardness}
There is no bidding strategy that guarantees a constant fraction of the MMS to an agent with an XOS valuation function (not even in the case of equal entitlements).
\end{restatable}


\subsection{Basic definitions}
\label{sec:basic definitions}

\begin{definition}
\textbf{(Maximin share (MMS))} Consider an allocation instance with a set $\mathcal{M}=\{e_{1},\dots,e_{m}\}$ of $m$ items and a set $\mathcal{N}=\{1,\dots,n\}$ of $n$ agents, where each agent $i$ has an individual non-negative valuation function $v_{i}\colon2^{\mathcal{M}}\to\mathbb{R}_{\geq0}$. Then the maximin share of agent $i$, denoted by $MMS_{i}$, is the maximum over all $n$-partitions of $\mathcal{M}$, of the minimum value under $v_{i}$ of a bundle in the $n$-partition
\[
MMS_{i}=\max_{A_{1},A_{2},\dots,A_{n}\in P_{n}(\mathcal{M})}\min_{j}v_{i}(A_{j})
\]

(where $P_{n}(\mathcal{M})$ is the set of all partitions of $\mathcal{M}$ to $n$ pairwise disjoint sets)
\end{definition}
% $\quad$
\begin{definition}
\label{def:APS}
\textbf{(AnyPrice share)} Consider a setting in which agent $i$ with valuation $v_{i}$ has entitlement $b_{i}$ to a set of indivisible items $\mathcal{M}$. The AnyPrice share (APS) of agent $i$, denoted by $AnyPrice(b_{i},v_{i},\mathcal{M})$, is the value she can guarantee herself whenever the items in $\mathcal{M}$ are adversarially priced with non-negative prices that sum up to 1, and she picks her favorite affordable bundle. More formally, if $P=\left\{ (p_{1},\dots,p_{m})|\sum p_{j}=1,\text{ and }\forall j,p_{j}\geq0\right\}$ is the set of all possible pricing of $\mathcal{M}$, then the definition of the APS is:
\[
AnyPrice(b_{i},v_{i},\mathcal{M})=\min_{(p_{1},p_{2},\dots,p_{m})\in P}\max_{S\subseteq\mathcal{M}}\left\{ v_{i}(S) \mid \sum_{j\in S}p_{j}\leq b_{i}\right\}
\]

When $\mathcal{M}$ and $v_{i}$ are clear from context we denote the APS share of an agent $i$ with entitlement $b_{i}$ by $AnyPrice(b_{i})$, instead of $AnyPrice(b_{i},v_{i},\mathcal{M})$.
\end{definition}
$\quad$

As shown in \cite{BEF21}, the AnyPrice share has the following equivalent definition.
\begin{definition}
\label{def:dual}
\textbf{(AnyPrice share dual definition)} Consider a setting in which agent $i$ with valuation $v_{i}$ has entitlement $b_{i}$ to a set of indivisible items $\mathcal{M}$. The AnyPrice share of $i$, denoted by $AnyPrice(b_{i},v_{i},\mathcal{M})$, is the maximum value $z$ she can get by coming up with nonnegative weights $\left\{ \lambda_{T}\right\} _{T\in\mathcal{M}}$ that total to 1 (a distribution over sets), such that any set $T$ of value below $z$ has a weight of zero, and any item appears in
sets of a total weight at most $b_{i}$:
\[
AnyPrice(b_{i},v_{i},\mathcal{M})=\max z
\]
 subject to the following set of constraints being feasible for $z$:
\begin{itemize}
\item $\sum_{T\subseteq\mathcal{M}}\lambda_{T}=1$
\item $\lambda_{T}\geq0,\forall T\subseteq\mathcal{M}$
\item $\lambda_{T}=0,\forall T\subseteq\mathcal{M}\text{ s.t }v_{i}(T)<z$
\item $\sum_{T:j\in T}\lambda_{T}\leq b_{i},\forall j\in\mathcal{M}$
\end{itemize}
\end{definition}

% #########furthue discussion on the APS properties
%\ufc{remove until the end of subsection, as we are not the ones introducing the APS}
%An equivalent set of constraints that will sometimes be convenient to use is the following. Let $\mathcal{G}_{i}(z)$ be the family of sets $T\subseteq\mathcal{M}$ such that $v_{i}(T)\geq z$. The set of constraints now becomes $\sum_{T\in\mathcal{G}_{i}(z)}\lambda_{T}=1$, and $\sum_{T\in\mathcal{G}_{i}(z):j\in T}\lambda_{T}\leq b_{i},\forall j\in\mathcal{M}$.

%In order to show the basic result that $APS\geq MMS$ in the equal entitlements case, consider $\langle T_{1},T_{2},\dots T_{n}\rangle$ to be a partition of $\mathcal{M}$ to $n$ bundles, such that $v_{i}(T_{k})\geq MMS_{i}$ for each $k\in[n]$ (such a partition exist from MMS definition).
%Then, since the equal entitlements case can be represented by entitlements $b_{i}=\frac{1}{n},\forall i\in[n]$, by setting
%\[
%\lambda_{T}=\begin{cases}
%\frac{1}{n} & \text{if }T\in\langle T_{1},T_{2},\dots T_{n}\rangle\\
%0 & \text{else}
%\end{cases}
%\]

%we get that the APS is at least as large as the MMS. Also, the dual definition of the APS can be viewed as a fractional relaxation of the definition of MMS. Specifically, as shown, one can get the definition of the MMS by restricting every weight $\lambda_{T}$ to be either $b_{i}$ or 0 for any non-empty set. As the total weight is 1, there must be at most $n$ bundles with non-zero weight. The constraint that every item belongs to bundles with a total weight of at most $b_{i}$ implies that all non-empty positive-weight bundles are disjoint. The APS on the other hand relaxes the constraint $\lambda_{T}\in\{0,b_{i}\}$ for non-empty sets to the fractional constraint $\lambda_{T}\in[0,b_{i}]$, and maintains the constraint that every item belongs to bundles with a total weight of at most $b_{i}$.\\

%\gbc{ As described in the guiding bullets you wrote in the beginning of the paper, you want to remove only the title of the next subsection, and let it be part of the 'basic definitions bullet. However I wonder maybe keeping this title of the subsection 'computational aspects of fair-allocation approximation'}

%\subsection{Computational aspects of fair-allocation approximation}

%\ufc{I think that for this paper it suffices to define value queries. This should probably be done already before stating corollary~\ref{cor:polyTimeSubmodular}, so that the corollary makes sense}

%\gbc{Removed, since already stated above:From now on we will focus on the problem of approximating a fair allocation (Maximin-fair or AnyPrice-fair).}
%Recall that the problem \ufc{what problem?} is defined by the parameters, $|\mathcal{M}|=m$, $|\mathcal{N}|=n$ and the valuations functions of the players. The explicit representation of each valuation function requires exponential space in $m$ (its domain size is $2^{m}$)\gbc{Removed:, having the valuations functions in their explicit representation as an input}. However, one may strive to measure the complexity of an algorithm as a function of $m$ and $n$. Thus, as described in \cite{DBLP:journals/geb/LehmannLN06}, when designing an algorithm, one may consider one of the following types of access to the valuation functions of the agents.
%\begin{itemize}
%\item Value Queries/Oracles - in this model, a query is a set of items, and the answer is the value that the function gives to this set of items.
%\item Demand Oracles - in this model, a query is a pricing function $p:2^{\mathcal{M}}\to\mathbb{R}$ (usually the pricing function is additive so it can be described only by the prices of the items) and the answer will be a bundle that maximizes the utility, i.e a set $S\subseteq\mathcal{M}$ which maximizes $\max_{s\subseteq\mathcal{M}}\left\{ p(S)-v_{i}(S)\right\} $ (notice that there might be more than one maximizer). Demand oracles are more powerful than value oracles, in the sense that given a demand oracle, one may answer value queries using the demand oracles, with polynomial (in $m$ and $n$) number of calls. An example for an application of the demand oracles is solving a configuration-LP in polynomial time as shown in \cite{DBLP:journals/toc/FeigeV10} and \cite{10.5555/1109557.1109675}, which in turn leads to a polynomial approximation algorithm with ratio $(1-\frac{1}{e}+\varepsilon)$ (with a tiny constant epsilon) for the Submodular Welfare Maximization problem, while there is an upper bound on the approximation ratio of $(1-\frac{1}{e})$ assuming value queries \cite{DBLP:journals/algorithmica/KhotLMM08}. (In Submodular Welfare Maximization we need to find an allocation $A=(A_{1},A_{2},\dots,A_{n})\in P_{n}(\mathcal{M})$ which maximizes the objective $\sum_{i\in n}v_{i}(A_{i})$ assuming submodular valuations.) 
% Consider rephrasing this explanation, and giving a reference or a
% formal definition to the configuration-LP

%\end{itemize}

\subsection{Related work}

%\ufc{Check references before posting}

There are several different approaches trying to define fairness criteria for allocation of items. One approach concerns elimination (or minimization) of envy among agents. An allocation is {\em envy-free}  if no agent strictly prefers a bundle of another agent over her own bundle {\cite{Foley67}}. Envy-free allocations exist in setting with divisible items, but need not exist in settings with indivisible items (e.g., when there are fewer agents than items). Consequently, various relaxations of the envy-free property have been introduced, among them EF1 {(\cite{LiptonEF1},\cite{Budish11})} and EFX {\cite{MaximumNashWelfare}}. In this work we do not consider envy-based fairness notions.

Perhaps the first share-based fairness notion to have been introduced is the proportional share, $b_i \cdot v_{i}(\mathcal{M})$. This notion and various relaxations of it (Prop1) may be appropriate when valuations functions are additive, but is hard to justify for other classes of valuation functions. In this work we are concerned with submodular valuations. For the case of equal entitlements, we consider the maximin share (MMS) \cite{Budish11}, which is the share notion that is most commonly used for allocation of indivisible items to agents with equal entitlements. For the case of arbitrary entitlements, we use the anyprice share (APS) \cite{BEF21}. We remark that there are other notions of shares that have been proposed for settings with unequal entitlements and are not considered in our work. These include the \emph{weighted maximin share} (WMMS) of~\cite{FGT19}, and the \emph{l-out-of-d} share~\cite{DBLP:journals/mor/BabaioffNT21}.

% \ufc{remove}
% \gbe{
% Multiple notions have been proposed to capture an allocation's \emph{fairness} property, such as \emph{envy-freeness} and \emph{proportionality}. In \emph{envy-freeness}, we require an allocation where no agent prefers a bundle of another agent, strictly more than her own bundle. In the \emph{proportionality} notion, we require that each agent gets a bundle which she values at least as much as her proportional share, i.e., $\frac{1}{n}v_{i}(\mathcal{M})$.
% Unlike the divisible setting of the problem (in which an item can be split), these \emph{fairness} concepts are sometimes not feasible, i.e., no allocation guarantees the respective property. For example,  \emph{Envy-freeness} does not exist if the number of items is strictly less than the number of agents. In that case, there will always be an agent who gets nothing and will envy.
% As described above, our paper will focus on \emph{MMS-fairness} and \emph{APS-fairness}.
% }

% \ufc{remove}
% \gbe{Moreover, there have been a few attempts to generalize the \emph{maximin share} to the case of arbitrary entitlements. One was in \cite{FGT19}, by introducing the \emph{weighted maximin share }(abbreviated \emph{as WMMS}). Babaioff, Nisan, and Talgam-Cohen suggested another generalization of the MMS in \cite{DBLP:journals/mor/BabaioffNT21} when they presented the \emph{l-out-of-d} notion. \gbc{I feel like we should add a sentence stating that the notion we focus on in our paper, the AnyPrice share, seems to be more relevant to non-additive agents, and maybe state that \cite{BEF21} shows more advantages of the APS over other notions, such as the AnyPrice-share value of an agent does not depend on other agents' valuation}
% % All of them seem to be irrelevant beyond the special case (but yet important) of agents with additive valuations
% }

We present here some known approximation results for \emph{MMS-fair} and \emph{APS-fair} allocations:

\begin{itemize}

\item Additive valuations with equal entitlements (approximate MMS-allocations).

\begin{itemize}

\item Impossibility results. As mentioned above, Kurokawa, Procaccia, and Wang \cite{KPW18} were the first to show that for every $n\geq3$, there exists an instance with $n$ additive agents, such that no \emph{maximin-fair }allocation exists. Later, Feige, Sapir, and Tauber \cite{DBLP:journals/corr/abs-2104-04977} showed an example of an instance with $n=3$ agents with additive valuations, where for any allocation, at least one of the agents gets a bundle she values at most $\frac{39}{40}$ of her MMS). 

\item Existence results. \cite{KPW18} showed existence of $\approx\frac{2}{3}$-\emph{maximin-fair }allocation (up to $O(\frac{1}{n})$). Ghodsi et al \cite{DBLP:conf/sigecom/GhodsiHSSY18} showed existence of $\frac{3}{4}$-\emph{maximin-fair} allocations. 

\item Algorithmic results. Garg and Taki \cite{GargTaki21}, presented a polynomial time algorithm that finds a $\frac{3}{4}$-\emph{maximin-fair} allocation, assuming value queries. 

\end{itemize}

\item 
%Before our result, \cite{BarmanKumar17} were the first to initiate the study of approximate maximin fair division under \emph{submodular valuations}. 
%\gbc{Uri, notice that \cite{BarmanKumar17} first published in 2017 in a conference, and in 2020 the journal version published. I suspect there are some differences in these versions - for example, when \cite{DBLP:conf/sigecom/GhodsiHSSY18} cite this work, they cite a result of $\frac{1}{10}$-MMS instead of $0.21$-MMS}
Submodular valuations with equal entitlements. \cite{BarmanKumar17} showed existence and polynomial time computability of $\approx0.21$-MMS fair allocations. Their algorithm is based on a simple round-robin algorithm. Ghodsi et al \cite{DBLP:conf/sigecom/GhodsiHSSY18} showed a polynomial-time algorithm for $\frac{1}{3}$-\emph{maximin-fair} allocations, and examples in which $\rho$-\emph{maximin-fair} allocations do not exist, for any $\rho > \frac{3}{4}$ and any $n \ge 2$ (number of agents).

\item For XOS valuations with equal entitlements, Ghodsi et al \cite{DBLP:conf/sigecom/GhodsiHSSY18} show the existence of $\frac{1}{5}$-MMS allocations, and presented examples in which $\rho$-\emph{maximin-fair} allocations do not exist, for any $\rho > \frac{1}{2}$ and any $n \ge 2$. 

%\gbc{Removed:
%\begin{itemize}
%\item Impossibility result - and upper bound of $\frac{1}{2}$-fraction of a \emph{maximin-fair }allocation 
%\item Existence result - an existence of $\frac{1}{5}$-fraction of a \emph{maximin-fair} allocation
%\item Algorithmic result - a polynomial-time algorithm for $\frac{1}{8}$-\emph{maximin-fair}allocation.
%\end{itemize}
%\item Subadditive valuations with equal entitlements \cite{DBLP:conf/sigecom/GhodsiHSSY18}:
%\begin{itemize}
%\item Impossibility result - an upper bound of $\frac{1}{2}$-fraction of a \emph{maximin-fair }allocation
%\item Existence result - there exist a $\frac{1}{10\log(m)}$\emph{ maximin-fair} allocation (not known to be algorithmic).
%\end{itemize}
%}
%\gbc{Keep the below known results?}

\item Additive valuations with arbitrary entitlements (approximate \emph{AnyPrice-fair} allocations).
\begin{itemize}
\item Impossibility result. The negative results stated for MMS allocations (such as upper bound of $\frac{39}{40}$ for $n = 3$ agents) extend to APS allocations, as the APS is at least as large as the MMS. 
\item There exists a polynomial-time algorithm for computing a $\frac{3}{5}$-APS
allocation, i.e., an algorithm which returns an allocation where each agent gets a bundle she values at least $\frac{3}{5}$ of her AnyPrice share \cite{BEF21}.
\end{itemize}
\end{itemize}

We are not aware of previous work concerning approximate APS-allocations for valuation functions that are submodular.

%\gbc{Removed:
%\subsection{Useful techniques}
%Before the presentation of the Maximin-fair notion, there has been extensive research focused on the Welfare maximization problem. The welfare of an allocation $A=(A_{1},A_{2},\dots,A_{n})\in P_{n}(\mathcal{M})$ is defined to be $\sum_{i\in n}v_{i}(A_{i})$. In the additive case, finding the allocation which maximizes the welfare is easy, simply by assigning each item to the agent which values it the most. But in the more general valuation families such as Submodular, XOS, and Subadditive, the problem is not trivial and hard to approximate. Some of the techniques applied in the context of \emph{welfare maximization} may also be useful in the context of fair allocations. For example as mentioned above, \cite{DBLP:journals/toc/FeigeV10} and \cite{10.5555/1109557.1109675} presented a polynomial algorithm for constant-fraction approximation to the problem assuming the demand oracle model, by using the configuration LP of the problem and solving its dual.
%$}
%\\
