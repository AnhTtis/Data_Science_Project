\documentclass[12pt,reqno]{amsart}

\usepackage[active]{srcltx}

%%%% DELETE THESE LINES FOR FINAL VERSION
%\newcommand{\ver}{{\it 9}}%<---PUT VERSION NUMBER HERE
\newcommand{\club}{\clubsuit} % THIS IS TO MARK COMMENTS, CHANGES
\newcommand{\cs}{$\club$}
%\usepackage{showkeys}  % THIS SHOWS LABELS
%%%%

\headheight=6.15pt \textheight=8.75in \textwidth=6.5in
\oddsidemargin=0in \evensidemargin=0in \topmargin=0in

\usepackage{latexsym}
\usepackage{graphicx}

\usepackage{graphicx,amsfonts,amssymb,amsmath,amsthm,xpatch}
\usepackage{fullpage}
\usepackage[all,arc]{xy}
\usepackage[dvipsnames]{xcolor}
\usepackage{hyperref}
\usepackage{mathrsfs,mathtools}
\usepackage{caption}
\usepackage{enumitem}
\usepackage{tikz-cd,float}
\usepackage{pdflscape,bm}

\usepackage{color}\newcommand{\blue}{\color{blue}}
\newcommand{\red}{\color{red}}


\newcommand{\edit}[1]{{\color{red}{$\clubsuit$#1$\clubsuit$}}}

\newtheorem{theo}{{\sc Theorem}}[section]
\newtheorem{mainlem}{{\sc Lemma}}
\newtheorem{maintheo}{{\sc Theorem}}
\newtheorem{maindefn}{{\sc Definition}}
\newtheorem{mainprop}{{\sc Proposition}}
\newtheorem{cor}[theo]{{\sc Corollary}}
\newtheorem{conj}[theo]{{\sc Conjecture}}
\newtheorem{maincor}[maintheo]{{\sc Corollary}}
\newtheorem{lem}[theo]{{\sc Lemma}}
\newtheorem{prop}[theo]{{\sc Proposition}}
\newtheorem{proposition}[theo]{{\sc Proposition}}
\newenvironment{example}{\medskip\noindent{\it Example:\/} }{\medskip}
\newenvironment{rem}{\medskip\noindent{\it Remark:\/} }{\medskip}


\newtheorem{defn}[theo]{{\sc Definition}}
\newenvironment{claim}{\medskip\noindent{\it Claim:\/} }{\medskip}
\makeatletter
\def\blfootnote{\gdef\@thefnmark{}\@footnotetext}
\makeatother

\makeatletter
\newcounter{proofpart}
\xpretocmd{\proof}{\setcounter{proofpart}{0}}{}{}
\newcommand{\proofpart}[2]{%
  \par
  \addvspace{\medskipamount}%
  \stepcounter{proofpart}%
  \noindent\emph{#1 \theproofpart: #2}\par\nobreak\smallskip
  \@afterheading
}
\makeatother



\title{Semiclassical Measures of Eigenfunctions of the Hydrogen Atom}




\author{Nicholas Lohr}
\address{Northwestern University, Evanston, IL, 60208}
\email{nlohr@math.northwestern.edu}
\date{\today}


\begin{document}
\blfootnote{The author was partially supported by NSF RTG grant DMS-2136217.}
\begin{abstract}
The main result of this article characterizes the set of semiclassical measures corresponding to sequences of eigenfunctions of the hydrogen atom. In particular, any Radon probability measure on the fixed negative energy surface $\Sigma_E$ that is invariant under the Hamiltonian flow is a semiclassical measure of a sequence of eigenfunctions of hydrogen. We first prove that there is a sequence of eigenfunctions, called hydrogen coherent states, that converge to a delta measure concentrating on any given geodesic $\gamma$ on $\Sigma_E$, and we finish using a density argument in the weak-* topology. 
\end{abstract}
\maketitle%\tableofcontents
\section{Introduction}
In this article, we characterize the semiclassical
measures corresponding to eigenfunctions of the hydrogen
atom on $L^2(\mathbb{R}^3)$:
\begin{equation}\label{eq:operator}
\widehat{H}_{\hbar}\coloneqq -\frac{\hbar^2}{2}\Delta-\frac{1}{|x|}.
\end{equation}
For $\hbar_j \to 0$ and $L^2$-normalized $\Psi_{j}$, we say
that $\Psi_{j}$ converges to a nonnegative Radon measure $\mu$ in the sense of semiclassical
measures if for any $a \in C_c^{\infty}(T^*(\mathbb{R}^3-\{0\}))$, we have
$$\langle \operatorname{Op}_{\hbar_j}(a)\Psi_{j},\Psi_{j}
\rangle \xrightarrow{\hbar_j \to 0}\int_{T^*(\mathbb{R}^3-\{0\})}a(x,\xi)
d\mu(x,\xi), $$
where $\operatorname{Op}_{\hbar}$ denotes semiclassical Weyl quantization.
\par In general, it is hard to characterize the set of all semiclassical measures $\mu$ for a given operator. The high-frequency quantum limit of eigenfunctions of the Laplace-Beltrami operator on the flat torus and sphere was studied in \cite{J97,JZ99}, respectively. This research continued to Schr\"odinger operators on the torus and sphere in \cite{AM14,MR19}, respectively. In addition, this problem on general Zoll manifolds was studied in \cite{MR16}. Characterizing the semiclassical measures for the isotropic harmonic oscillator on $\mathbb{R}^d$ was first done in \cite{A20} and with different methods in \cite{S19}. The semiclassical measures of the anistropic oscillator on $\mathbb{R}^d$ were studied in \cite{AM22}, and this note adds hydrogen to this research.
\par To this end, we study specific eigenfunctions, called hydrogen coherent states, of $\widehat{H}_{\hbar}$. Hydrogen coherent states were first studied in \cite{GDB89,N89} and later in \cite{K96,TV-B97}. It is well-known that $\widehat{H}_{\hbar}$ has a discrete
negative spectrum with energies (i.e. eigenvalues)
\begin{equation}\label{eq:energy}
E_N(\hbar) \coloneqq -\frac{1}{2\hbar^2(N+1)^2} \quad N=0,1,2,\ldots.
\end{equation}
With fixed energy $E<0$, the hydrogen coherent states
$\Psi_{\alpha,N}$ (defined in \eqref{eq:hcs}) are eigenfunctions
of $\widehat{H}_{\hbar}$:
\begin{equation*}
\widehat{H}_{\hbar}\Psi_{\alpha,N}=E_N(\hbar)\Psi_{\alpha,N}.
\end{equation*}
These eigenfunctions are specifically designed to concentrate on
classical Kepler orbits
in the semiclassical limit $\hbar\to 0$ and $N \to
\infty$. This was proved in configuration and momentum space separately in \cite{TV-B97} provided that the orbit has nonzero angular momentum (i.e. not a collision orbit). The point of this article is to prove this result in phase space, including the collision orbits with zero angular momentum, and use this to characterize all of the semiclassical measures of eigenfunctions of $\widehat{H}_{\hbar}$.
\subsection{Statement of Results}
Define the set
\begin{equation*}
\mathcal{A} \coloneqq \{ \alpha \in \mathbb{C}^4\ ; \ |\Re \alpha|=|\Im \alpha|=1, \Re (\alpha) \cdot \Im (\alpha) =0\}.
\end{equation*}
Note that $\mathcal{A}$ is a parametrization $T_1S^3$, and hence
$T_1^*S^3$. Recall the highest weight spherical harmonics (also called spherical coherent states) $\Phi_{\alpha,N}\in L^2(S^3)$ are defined by
$$\Phi_{\alpha,N}(u)\coloneqq c_N(\alpha \cdot u)^N,$$
for any $\alpha \in \mathcal{A}$ where $c_N \coloneqq \frac{1}{\pi \sqrt{2}}\sqrt{N+1}$ is a normalization constant so that $\lVert \Phi_{\alpha,N}\rVert_{L^2(S^3)}=1$. It is well-known that as $N \to \infty$, $\Phi_{\alpha,N}$
concentrates on the great circle $\gamma(\theta) \coloneqq
\Re(\alpha) \cos \theta +\Im(\alpha)\sin \theta$ in $S^3$ (see \cite{TV-B97,JZ99}). Now
we define the hydrogen coherent states.
\begin{defn}[Hydrogen Coherent States]
For $E<0$ fixed and any $\alpha \in \mathcal{A}$, we define $\Psi_{\alpha,N} \in
L^2(\mathbb{R}^3)$ by
\begin{equation}\label{eq:hcs}
\Psi_{\alpha,N}\coloneqq \mathcal{V}^{-1}(\Phi_{\alpha,N}),
\end{equation}
where $\mathcal{V}^{-1}$ is the inverse of the Fock map, defined in Definition \ref{def:fock} (and again in \eqref{eq:inverse}).
\end{defn}
We begin with a theorem on the concentration of these states in phase space.
\begin{theo}\label{theo:1}
Let $E<0$ and $a \in C_c^{\infty}(T^*(\mathbb{R}^3-\{0\}))$. If $\gamma$ is the Kepler orbit in $\Sigma_E$, then let $\alpha \in \mathcal{A}$ generate the great circle that $\gamma$ corresponds to in Theorem \ref{theo:moser}. If $\gamma$ has nonzero momentum, then in the limit as $N \to \infty$ with $E=-\frac{1}{2\hbar^2(N+1)^2}$, we have
$$\langle
\operatorname{Op}_{\hbar}(a)\Psi_{\alpha,N},\Psi_{\alpha,N} \rangle
\xrightarrow{N \to \infty}\frac{p_0^3}{2\pi}
\int_{0}^{2\pi/p_0^3}a(\gamma(t))dt $$
where $p_0 \coloneqq \sqrt{-2E}$ and $\operatorname{Op}_{\hbar}$
denotes semi-classical Weyl quantization. For $\gamma$ with zero angular momentum, we have
$$\langle
\operatorname{Op}_{\hbar}(a)\Psi_{\alpha,N},\Psi_{\alpha,N} \rangle
\xrightarrow{N \to \infty}\frac{p_0^3}{2\pi}
\int_{t_{\gamma}-2\pi/p_0^3}^{t_{\gamma}}a(\gamma(t))dt $$
where $t_{\gamma}$ is the collision time of $\gamma$ defined in \eqref{eq:collisiontime}.
\end{theo}
Using Theorem \ref{theo:1}, we prove the main result of the
article, Theorem \ref{theo:2}. Before we state the theorem, we
introduce more notation. Let $H:T^*(\mathbb{R}^3-\{0\}) \to \mathbb{R}$ be defined by
\begin{equation}\label{eq:clasham}
H(x,\xi)\coloneqq \frac{|\xi|^2}{2}-\frac{1}{|x|},
\end{equation}
the classical Hamiltonian of $\widehat{H}_{\hbar}$. Denote $$\Sigma_E \coloneqq \{(x,\xi) \in T^*(\mathbb{R}^3-\{0\}) : H(x,\xi)=E\}$$
the energy surface at energy level $E$.
\begin{theo}\label{theo:2}
Let $E<0$ and let $\mu$ be a Radon probability
measure on $\Sigma_E$ invariant under the Hamiltonian flow. Then
$\mu$ is a semiclassical measure of a sequence $\Psi_{N_j}$
of eigenfunctions of $\widehat{H}_h$. That is, there exists $N_j \to \infty$ and $L^2(\mathbb{R}^3)$-normalized
$\Psi_{N_j}$ such that $\widehat{H}_{\hbar_j}\Psi_{N_j}=E_{N_j}(\hbar_{j})\Psi_{N_j}$ (where $E=-\frac{1}{2\hbar_j^2(N_j+1)^2}$) and 
$$\langle \operatorname{Op}_{\hbar_j}(a)\Psi_{N_j},\Psi_{N_j} \rangle
\xrightarrow{N_j \to \infty}\int_{\Sigma_E}a(x,\xi) d\mu(x,\xi)  $$
for any $a \in C_c^{\infty}(T^*(\mathbb{R}^3-\{0\}))$.
\end{theo}
\subsection{Future Work}
In future work, we plan to study the finer pointwise asymptotics of the Wigner distributions of the hydrogen coherent states in a similar fashion as in \cite{L23}. We also plan on applying these results to the perturbed hydrogen equation and the Klein-Gordon equation.
\subsection{Acknowledgments}
This article is part of the Ph.D. thesis of the author at Northwestern University under the guidance of Steve Zelditch. The author thanks Jared Wunsch for continued conversations and support after the passing of Steve Zelditch. The author also thanks Erik Hupp, Ruoyu P. T. Wang, and Jeff Xia for helpful conversations.
\subsection{Background}
In this section, we introduce the relevant classical and quantum mechanical maps that are involved with this problem. For completeness, we reproduce proofs of basic facts about these maps, and further properties and generalizations to $\mathbb{R}^d$ can be found in \cite{M70,HdL12} for the Moser map and \cite{F35,BI66,RC21} for the Fock map.
 \subsubsection{The Classical Mechanical Moser Map}
 In this section, we define the classical Moser map, first defined by Moser in \cite{M70} (see \cite{HdL12} for an overview). This map is a symplectomorphism that maps the Hamiltonian orbits on $\Sigma_E$ to the geodesics of $T_1^*S^3$.
 We use the notation
 $$\widehat{S}^3 \coloneqq S^3-\{(0,0,0,1)\}$$
 to denote the punctured sphere. In this section, we define the Moser map from $T^*\mathbb{R}^3$ to $T^*\widehat{S}^3$.
Let $\omega: \mathbb{R}^3 \to \widehat{S}^3$ be inverse stereographic projection to the sphere minus the north pole. That is, 
$$\omega(x) \coloneqq \frac{1}{|x|^2+1}\begin{cases}
2x_k & \text{if }k<4\\

|x|^2-1 &\text{if }k=4
\end{cases}.$$
It can be easily computed that the pullback $\omega^*:T^*\mathbb{R}^3 \to T^*\widehat{S}^3$ is
\begin{align}
\omega^*(x,\xi)=(\omega(x),\eta) \quad \text{with} \quad \eta_j =\begin{cases}
           \xi_j \frac{|x|^2+1}{2}-(x\cdot \xi) x_j &\text{ if }
                                               j<4\\
           x \cdot \xi &\text{ if }
                                               j=4
                                                                  \end{cases}\label{eq:mos}
\end{align}
where we use the identification of the cotangent bundle with the tangent bundle via the Riemannian metric.
          \begin{defn} Let $E<0$ and define $p_0 \coloneqq
          \sqrt{-2E}$. Define the Moser map
          $\mathcal{M}_E:T^*\mathbb{R}^3 \to T^*\widehat{S}^3$ to
          be $\mathcal{M}_E \coloneqq   \omega^* \circ R_{-\pi/2}\circ S 
          \circ D_{p_0}$ where $D_{p_0}(x,\xi) \coloneqq (p_0 x
          ,\frac{1}{p_0}\xi)$ is a dilation by $p_0$,
          $R_{-\pi/2}(x,\xi) \coloneqq (\xi,-x)$ is rotation by $-\pi/2$, and $S(x,\xi) \coloneqq (p_0x,\xi)$ is a shearing. Using
          \eqref{eq:mos}, we can write $\mathcal{M}_E$ explicitly
          as 
          \begin{align}
\mathcal{M}_E(x,\xi)=\big(\omega(\tfrac{1}{p_0}\xi),\eta\big) \quad \text{where} \quad \eta_j =\begin{cases}
           -x_j \frac{|\xi|^2+p_0^2}{2}+(x\cdot \xi) \xi_j &\text{ if }
                                               j<4\\
           -p_0(x \cdot \xi) &\text{ if }
                                               j=4
                                                                                               \end{cases}.\label{eq:mos2}
          \end{align}
          The inverse is given by
          \begin{align}
\mathcal{M}_E^{-1}(u,\eta)=(x,p_0\omega^{-1}(u) ) \quad \text{where} \quad x_k=\tfrac{1}{p_0^2}\big(\eta_k(u_4-1)-\eta_4u_k\big) \quad \text{for } k=1,2,3.
\end{align}
\end{defn}
\begin{rem}
It is easy to see that $\mathcal{M}_E$ is a symplectomorphism that pulls back the canonical symplectic form on $T^* \widehat{S}^3$ to $p_0\ dx \wedge d\xi$. By a calculation, one easily sees that $\mathcal{M}_E|_{\Sigma_E}=T_1^*S^3$. It is worth noting that the Moser map depends on the energy level $E$. In the following, we will drop the subscript $E$ in the notation and write $\mathcal{M}=\mathcal{M}_E$, but it is worth remembering this dependence.
\end{rem}
\begin{theo}[\cite{M70}, Theorem 1]\label{theo:moser}
Up to a reparametrization of time, the Moser map transforms the Kepler flow on the energy surface $H(x,\xi)=E$ onto the cogeodesic flow on $T_1^*\widehat{S}^3$ parametrized by arc length. More specifically, 
if $\gamma(t)=(x(t),\xi(t)) \in T^*\mathbb{R}^3$ is a Kepler orbit on the energy surface $H(x,\xi)=E \eqqcolon \frac{-p_0^2}{2}$, then $\varphi(s)=(u(s),\eta(s)) \coloneqq \mathcal{M}(\gamma(t(s)) \in T_1^* \widehat{S}^3$ is a 
geodesic on $\widehat{S}^3$ parametrized by arc length $s$ where
$t(s)$ satisfies
\begin{equation}\label{eq:timechange}
\frac{dt}{ds}=\frac{|x(t(s))|}{p_0}=\frac{1-u_4(s)}{p_0^3},\quad t(0)=0.
\end{equation}
\end{theo}
\begin{proof}
Let $\mathcal{M}(x,\xi)=(u,\eta)$. From \eqref{eq:mos2}, one can
compute
\begin{equation}\label{eq:relate}
\frac{1}{2}|\eta|^2=\frac{|x|^2(|\xi|^2+p_0^2)^2}{8}.
\end{equation}
On $T^*\widehat{S}^3$, define $K(u,\eta) \coloneqq \frac{1}{2}|\eta|^2$. Note that the
Hamiltonian flow of $K$ on the level hypersurface
$K=\frac{1}{2}$ is the cogeodesic flow on $T_1^* \widehat{S}^3$ parametrized by arc
length time $s$. By \eqref{eq:relate}, the
Hamiltonian orbits of $$F(x,\xi) \coloneqq
\frac{|x|^2(|\xi|^2+p_0^2)^2}{8}$$ on the level
hypersurface $F=\frac{1}{2}$ parametrized in time parameter $t'$ are images under
$\mathcal{M}^{-1}$ of the Hamiltonian orbits of $K$ on the level
hypersurface $K=\frac{1}{2}$ parametrized by arc length $s$ where
$$\frac{dt'}{ds}=\frac{1}{p_0}. $$
Define
$$G(x,\xi)=\sqrt{2
  F(x,\xi)}-1=\frac{|x|(|\xi|^2+p_0^2)}{2}-1.$$
It is easy to see that the Hamiltonian flow of $F$ on the level
hypersurface $F=\frac{1}{2}$
equivalent to the Hamiltonian flow of $G$ on the level
hypersurface $G=0$. Finally, note that
$$H(x,\xi)=\frac{1}{|x|}G(x,\xi)-\frac{p_0^2}{2}. $$
Again, it is easy to see that the Hamiltonian flow of $G$ on the
level hypersurface $G=0$ in the time parameter $t'$ is
equivalent to the Hamiltonian flow of $H$ on
$H=-\frac{p_0^2}{2}=E$ in the time parameter $t$ where
$\frac{dt}{dt'}=|x(t(t'))|$. Altogether, we have
$$\frac{dt}{ds}=\frac{dt}{dt'}\frac{dt'}{ds}=\frac{|x(t(s))|}{p_0}=\frac{1}{p_0}\frac{2}{|\xi(t(s))|^2+p_0^2}=\frac{1}{p_0}\frac{2}{|p_0\omega^{-1}(u(s))|^2+p_0^2}=\frac{1-u_4(s)}{p_0^3}, $$
and we are done.
\end{proof}
\begin{rem}
Note that the Hamiltonian orbits with zero angular momentum
(i.e. the collision orbits) correspond under the Moser map to
great circle arcs on $S^3$ that end at the north pole. Let
$\varphi(s)=(u(s),\eta(s)) \in T_1^*S^3$ be a great circle on
$S^3$ parametrized by arc length passing through the north
pole. Let $s_{\varphi} \in (0,2\pi)$ be such that
$\varphi(s_{\varphi})$ is at the north pole. Then we think of
$\varphi$ as a curve in $T_1^*\widehat{S}^3$ by restricting the
parameter $s$ to the interval
$(s_{\varphi}-2\pi,s_{\varphi})$. By Theorem \ref{theo:moser}, $\varphi$
corresponds to a Hamiltonian orbit $\gamma$ with time restricted
to $(t_{\gamma}-2\pi/p_0^3,t_{\gamma})$ where
\begin{equation}\label{eq:collisiontime}
t_{\gamma} \coloneqq t(s_{\varphi})
\end{equation}
where $t(s)$ is defined by \eqref{eq:timechange}. We call $t_{\gamma}$ the collision time. The quantity $t_{\gamma}-2\pi/p_0^3$ is the collision time provided we reversed time. We frequently extend the Hamiltonian orbits with zero momentum to be $2\pi/p_0^3$ periodic where we leave the times $t_{\gamma}+2\pi/p_0^3 \mathbb{Z}$ undefined.
\end{rem}
\subsubsection{The Quantum Mechanical Fock Map} In this section, we define the Fock map, first defined by Fock in \cite{F35} (see \cite{BI66,RC21} for overviews). The Fock map is the `quantization' of the Moser map. Eigenfunctions $\psi$ of the hydrogen atom $\widehat{H}_{\hbar}$ with energy $E<0$ satisfy
  \begin{align}\label{eq:eq}
\widehat{H}_{\hbar}\psi \coloneqq -\frac{\hbar^2}{2}\Delta \psi -\frac{1}{|x|}\psi=E \psi
\end{align}
which implies
\begin{align}\label{eq:closeha}
\Big(\frac{|\xi|^2}{2}-E \Big) \mathcal{F}_{\hbar}(\psi)(\xi)=\frac{1}{2\pi^2\hbar}\int_{\mathbb{R}^3}\frac{\mathcal{F}_{\hbar}(\psi)(p)}{|p-\xi|^2}dp
\end{align}
where $\mathcal{F}_{\hbar}(\psi)(\xi)\coloneqq (2\pi
\hbar)^{-3/2}\int_{\mathbb{R}^3}\psi(v)e^{-i \frac{v\cdot
    \xi}{\hbar}}dv$ is the semiclassical Fourier
transform. This is because
$\mathcal{F}_{\hbar}(|\bullet|^{-1})=\frac{1}{\pi}\cdot \frac{\sqrt{2\pi \hbar}}{|\bullet|^2}$ and $\mathcal{F}_{\hbar}(f \cdot
  g)=(2\pi \hbar
  )^{-3/2}\mathcal{F}_{\hbar}(f)*\mathcal{F}_{\hbar}(g)$. Now define 
$p_0 \coloneqq \sqrt{-2E}$ and the dilation operator
$$D_{p_0}(f) \coloneqq p_0^{3/2}f(p_0 \bullet).$$ We apply $D_{p_0}$ on both sides
of \eqref{eq:closeha} and see that
\begin{align}
  \frac{|\xi|^2+1}{2}(D_{p_0}\circ
  \mathcal{F}_{\hbar})(\psi)(\xi)&=\frac{1}{2\pi^2 \hbar
                                   p_0}\int_{\mathbb{R}^3}\frac{(D_{p_0}
\circ
                                   \mathcal{F}_{\hbar})(\psi)(p)}{|p-\xi|^2}dp \nonumber\\
  &=\frac{1}{2\pi^2 \hbar p_0}\int_{\mathbb{R}^3}\frac{(D_{p_0}
\circ
                                   \mathcal{F}_{\hbar})(\psi)(p)}{|p-\xi|^2}dp \label{eq:lilwayne}
\end{align}
Define stereographic projection $\omega:\mathbb{R}^3 \to  \widehat{S}^3$ by
$$\omega(p)_i=\frac{1}{|p|^2+1}\begin{cases}2 p_i
&\text{if } i=1,2,3\\
|p|^2-1 &\text{if }i=4\end{cases}, \qquad
\omega^{-1}(u)_i=\frac{u_i}{1-u_4}.$$ Recall that the pullback of the
Euclidean sphere measure $d\Omega$ under $\omega$ is
\begin{equation}\label{eq:measure}
\omega^*d\Omega=\Big(\frac{2}{|p|^2+1} \Big)^3dp.
\end{equation}
Also recall that stereographic projection distorts distances by
the formula
\begin{equation}\label{eq:distort}
|p-\xi|^2=\frac{(|p|^2+1)(|\xi|^2+1)}{4}|\omega(p)-\omega(\xi)|^2.
\end{equation}
We now perform the change of variables of $\xi=\omega^{-1}(u)$ and
$p=\omega^{-1}(y)$ to \eqref{eq:lilwayne}. By \eqref{eq:measure}, we have
\begin{align}
  &\frac{|\omega^{-1}(u)|^2+1}{2}(D_{p_0}\circ
  \mathcal{F}_{\hbar})(\psi)(\omega^{-1}(u))\nonumber \\
  &\qquad \qquad =\frac{1}{2\pi^2\hbar p_0 }\int_{S^3}\frac{( D_{p_0}\circ \mathcal{F}_{\hbar})(\psi)(\omega^{-1}(y))}{|\omega^{-1}(u)-\omega^{-1}(y)|^2}\Big(\frac{|\omega^{-1}(y)|^2+1}{2}\Big)^3d\Omega(y), \nonumber
\end{align}
which, by \eqref{eq:distort}, implies 
\begin{align}
  &\Big(\frac{|\omega^{-1}(u)|^2+1}{2}\Big)^2(D_{p_0} \circ
    \mathcal{F}_{\hbar})(\psi)(\omega^{-1}(u)) \nonumber \\
  &\qquad \qquad =\frac{1}{2\pi^2\hbar p_0}\int_{S^3}\frac{(D_{p_0}\circ \mathcal{F}_{\hbar})(\psi)(\omega^{-1}(y))}{|u-y|^2}\Big(\frac{|\omega^{-1}(y)|^2+1}{2}\Big)^2d\Omega(y), \label{eq:so}
\end{align}
Define $\Phi(u) \coloneqq (\frac{|\omega^{-1}(y)|^2+1}{2})^2(D_{p_0} \circ
    \mathcal{F}_{\hbar})(\psi)(\omega^{-1}(u))$. Then
    \eqref{eq:so} reads
    \begin{align}\label{eq:so4in}
  \Phi(u)=\frac{1}{2\pi^2\hbar p_0}\int_{S^3}\frac{\Phi(y)}{|u-y|^2}d\Omega(y).
    \end{align}
    Note that \eqref{eq:so4in} reflects $\operatorname{SO}(4)$ symmetry: if
    $\Phi$ satisfies \eqref{eq:so4in}, then so does $y \mapsto
    \Phi(A^{-1}y)$ for any $A \in \operatorname{SO}(4)$. Also note that
    \begin{align}
\lVert \Phi \rVert_{L^2(S^3)}^2
      &\overset{\mathclap{\eqref{eq:measure}}}{=}\Big\lVert\big(\tfrac{|\bullet|^2+1}{2}\big)^{\frac{1}{2}}
        (D_{p_0}\circ
        \mathcal{F}_{\hbar})(\psi)\Big\rVert_{L^2(\mathbb{R}^3)}^2\nonumber\\
      &=\Big\lVert
        D_{p_0}\Big[\big(\tfrac{|\bullet|^2+p_0^2}{2p_0^2}\big)^{\frac{1}{2}}
        \mathcal{F}_{\hbar}(\psi)\big]\Big\rVert_{L^2(\mathbb{R}^3)}^2\nonumber\\
      &=\Big\lVert
        \big(\tfrac{|\bullet|^2+p_0^2}{2p_0^2}\big)^{\frac{1}{2}}
        \mathcal{F}_{\hbar}(\psi)\Big\rVert_{L^2(\mathbb{R}^3)}^2\nonumber\\
      &=\tfrac{1}{p_0^2}\big\langle
        (\tfrac{|\bullet|^2}{2}-E)\mathcal{F}_{\hbar}(\psi),\mathcal{F}_{\hbar}(\psi)
        \big\rangle_{L^2(\mathbb{R}^3)}\nonumber\\
        &=\tfrac{1}{p_0^2}\big\langle
          (-\tfrac{\hbar^2}{2}\Delta-E)\psi,\psi
          \big\rangle_{L^2(\mathbb{R}^3)}\nonumber\\
      &\overset{\mathclap{\eqref{eq:eq}}}{=}\tfrac{1}{p_0^2}\big\langle(-\hbar^2\Delta-\tfrac{1}{|\bullet|}-2E)\psi,\psi
        \big\rangle_{L^2(\mathbb{R}^3)}\nonumber\\
      &=\lVert \psi \rVert_{L^2(\mathbb{R}^3)}^2+\tfrac{1}{p_0^2}\big\langle(-\hbar^2\Delta-\tfrac{1}{|\bullet|})\psi,\psi
        \big\rangle_{L^2(\mathbb{R}^3)}\nonumber\\
       &=\lVert \psi \rVert_{L^2(\mathbb{R}^3)}^2+\tfrac{1}{p_0^2}\big\langle[\widehat{H}_{\hbar}-E,r \partial_r]\psi,\psi
         \big\rangle_{L^2(\mathbb{R}^3)}\nonumber \\
      &=\lVert \psi \rVert_{L^2(\mathbb{R}^3)}^2 \label{eq:unit}
\end{align}
where the last line we use the self-adjointness of
$\widehat{H}_{\hbar}-E$ (i.e. the quantum virial theorem).
\begin{defn}\label{def:fock} Fix $E<0$ and let $\mathcal{E}_N$ be the eigenspace of
$\widehat{H}_{\hbar}$ with energy $E=-\frac{1}{2\hbar^2(N+1)^2}$, and define the pure point spectrum $\mathcal{E}_{pp} \coloneqq \bigoplus_{N=0}^{\infty} \mathcal{E}_N$. The Fock map
$\mathcal{V}_{E}:\mathcal{E}_{pp} \to L^2(S^3)$ is the linear operator defined by
\begin{equation*}
\mathcal{V}_{E}(\psi)(u) \coloneqq \Big(\frac{|\omega^{-1}(y)|^2+1}{2}\Big)^2(D_{p_0} \circ
\mathcal{F}_{\hbar})(\psi)(\omega^{-1}(u))
\end{equation*}
where $\omega:\mathbb{R}^3 \to \widehat{S}^3$ is stereographic projection, $\mathcal{F}_{\hbar}$ is the semiclassical Fourier transform, and $D_{p_0}$ is dilation by $p_0$. That is $D_{p_0}(f) \coloneqq p_0^{3/2}f(p_0 \bullet)$ where $p_0 \coloneqq \sqrt{-2E}$.
\end{defn}
From \eqref{eq:unit}, we see that $\mathcal{V}_E$ is $L^2$ norm-preserving. The next theorem shows that it is in fact unitary.
\begin{theo}[\cite{F35}] The Fock map $\mathcal{V}_{E}:\mathcal{E}_{pp} \to
L^2(S^3)$ is a unitary map that conjugates $\widehat{H}_{\hbar}$ to $\frac{1}{2\hbar^2}(\Delta_{S^3}-1)^{-1}$ where $\Delta_{S^3}$ is the usual negative Laplacian on $S^3$. That is,
\begin{equation}\label{eq:conj}
\mathcal{V}_E\widehat{H}_{\hbar}\mathcal{V}_E^{-1}=\frac{1}{2\hbar^2}(\Delta_{S^3}-1)^{-1}.
\end{equation}
\end{theo}
\begin{proof}
Let $\mathcal{H}_4^{(N)}$ be the space of harmonic
polynomials of degree $N$ on $\mathbb{R}^4$ restricted to
$S^3$. We will show that $\mathcal{V}_E|_{\mathcal{E}_N}:\mathcal{E}_N \to \mathcal{H}_4^{(N)}$ is unitary, and the unitarity of $\mathcal{V}_E$ will follow by linearity. Define the Riesz potential-type operator $T:L^2(S^3) \to L^2(S^3)$ by
$$T(\Phi)(u) \coloneqq \int_{S^3}\frac{\Phi(y)}{|y-u|^2}d\Omega(y). $$
It is easy to see that $T$ commutes with the standard $\operatorname{SO}(4)$
action on $L^2(S^3)$. It well known that $\operatorname{SO}(4)$ rotations give
an irreducible representation on $\mathcal{H}_4^{(N)}$. By
Schur's lemma, $T|_{\mathcal{H}_4^{(N)}}=\lambda_N
\operatorname{Id}_{\mathcal{H}_4^{(N)}}$ for some $\lambda_N \in
\mathbb{C}$. By a direct computation with
$\Phi(y)=(y_1+iy_2)^N$, one sees that
$\lambda_N=\frac{N+1}{2\pi^2}$. Since $p_0 \coloneqq
\sqrt{-2E}=(\hbar(N+1))^{-1}$, \eqref{eq:so4in} and the fact
that $L^2(S^3)= \bigoplus_{N=0}^{\infty}\mathcal{H}_4^{(N)}$
imply that $\mathcal{V}_E|_{\mathcal{E}_N}$ maps into
$\mathcal{H}_4^{(N)}$. The map $\mathcal{V}_E|_{\mathcal{E}_N}$
preserves the $L^2$ norm by \eqref{eq:unit}, and since
$\operatorname{dim}\mathcal{H}_{4}^{(N)}=\operatorname{dim}\mathcal{E}_N=(N+1)^2$,
we see that $\mathcal{V}_E|_{\mathcal{E}_N}:\mathcal{E}_N \to
\mathcal{H}_{4}^{(N)}$ is indeed unitary, which implies
$\mathcal{V}_E$ is unitary by linearity. The conjugation
statement follows from linearity and the fact that
\begin{equation*}
\frac{1}{2\hbar^2}(\Delta_{S^3}-1)^{-1}\Big|_{\mathcal{H}_4^{(N)}}=-\frac{1}{2\hbar^2(N+1)^2}\operatorname{Id}_{\mathcal{H}_4^{(N)}}=\mathcal{V}_E\widehat{H}_{\hbar}\mathcal{V}_E^{-1}|_{\mathcal{H}_4^{(N)}}. \qedhere
\end{equation*}
\end{proof}
\begin{rem} In the following, we will drop the dependence on $E$
from the notation of the Fock map and write
$\mathcal{V}=\mathcal{V}_E$, but, again, it is good to remember
that this map depends on the energy level. For
$\mathcal{V}^{-1}$, we write it as a composition of operators
\begin{equation}\label{eq:inverse}
\mathcal{V}^{-1} \coloneqq D_{p_0} \circ \mathcal{F}_{\hbar}^{-1} \circ J^{1/2} \circ K 
\end{equation}
where the unitary map $K:L^2(S^3) \to L^2(\mathbb{R}^3)$ and the
multiplication map $J:L^2(\mathbb{R}^3) \to L^2(\mathbb{R}^3)$
are defined by
\begin{equation}\label{eq:JK}
K(f) \coloneqq \Big(\frac{2}{| \bullet |^2+1}\Big)^{3/2}f \circ \omega \quad \text{and} \quad J(f) \coloneqq \frac{2}{| \bullet |^2+1}f.
\end{equation}
It is easy to see that $K$ is unitary by \eqref{eq:measure}. It
is also easy to see that if $R \in \operatorname{SO}(3)$ with
$T_R:L^2(\mathbb{R}^3) \to L^2(\mathbb{R}^3)$ defined by
rotation $T_R(f) \coloneqq f(R^{-1}\bullet)$, then we have
\begin{equation}\label{eq:kk}
\mathcal{V}^{-1}T_{R'}=T_{R}\mathcal{V}^{-1}
\end{equation}
where $R' \coloneqq \begin{psmallmatrix}R & 0\\ 0& 1 \end{psmallmatrix} \in \operatorname{SO}(4)$.

\end{rem}

\section{Proof of Theorem \ref{theo:1}}
\proofpart{Step}{Reduction to $\alpha=e_1+i(\cos
  (\theta_0)e_2+\sin(\theta_0)e_4)$}
We claim that if the result is true for $\alpha_0 \coloneqq e_1+i(\cos
  (\theta_0)e_2+\sin(\theta_0)e_4)$, then it is also true for any $\alpha \in \mathcal{A}$. Indeed, let $\gamma$ be the orbit generated by an arbitrary $\alpha \in \mathcal{A}$.
Let $\varphi(s)=\Re \alpha \cos s + \Im \alpha \sin s \in S^3$ be the great circle that $\gamma$ corresponds to by Theorem \ref{theo:moser}. There exists an $s_0$ such that the fourth coordinate of $\varphi(s_0)$ is nonzero. By reparametrizing $\varphi$ to begin at $s_0$, we can assume the fourth coordinate of $\Re \alpha$ is nonzero. So there exists an $R \in
\operatorname{SO}(3)$ such that
\begin{equation*}\label{eq:wlog}
R'\varphi(s)= e_1 \cos s +(a_2e_2+a_3e_3+a_4e_4)\sin s.
\end{equation*}
where $R'=\begin{psmallmatrix}R & 0\\ 0& 1 \end{psmallmatrix} \in \operatorname{SO}(4)$, $e_j$ is the $j$th standard basis vector in $\mathbb{R}^4$, and $a_j \in \mathbb{R}$ are such that $a_2^2+a_3^2+a_4^2=1$. We can apply a further rotation in the $e_2e_3$-plane so as to make $a_3=0$, so altogether there exists $R \in \operatorname{SO}(3)$ such that
\begin{equation}\label{eq:wlog}
R'\varphi(s)= e_1 \cos s +(\cos (\theta_0)e_2+\sin(\theta_0)e_4)\sin s.
\end{equation}
for some $\theta_0 \in [0,2\pi)$. Then 
\begin{align}
  \langle
  \operatorname{Op}_{\hbar}(a)\Psi_{\alpha_0,N},\Psi_{\alpha_0,N}
  \rangle&=\langle
  \operatorname{Op}_{\hbar}(a)\Psi_{R'\alpha,N},\Psi_{R'\alpha,N}
  \rangle\nonumber \\
&\overset{\mathclap{\eqref{eq:kk}}}{=}\langle
           \operatorname{Op}_{\hbar}(a)T_{R}\Psi_{\alpha,N},T_{R}\Psi_{\alpha,N}
           \rangle\nonumber \\
  &=\langle
    T_{R}\operatorname{Op}_{\hbar}(a(Rx,R\xi))\Psi_{\alpha,N},T_{R}\Psi_{\alpha,N}
    \rangle\nonumber \\
    &=\langle \operatorname{Op}_{\hbar}(a(Rx,R\xi))\Psi_{\alpha,N},\Psi_{\alpha,N} \rangle, \label{eq:eqnn}
\end{align}
where $T_R(f) \coloneqq f(R^{-1}\bullet)$ is rotation of
functions. This shows
\begin{align*}
\lim_{N \to \infty}\langle \operatorname{Op}_{\hbar}(a(Rx,R\xi))\Psi_{\alpha,N},\Psi_{\alpha,N} \rangle\overset{\mathclap{\eqref{eq:eqnn}}}{=}\lim_{N \to \infty}  \langle
  \operatorname{Op}_{\hbar}(a)\Psi_{\alpha_0,N},\Psi_{\alpha_0,N}
  \rangle \overset{\mathclap{\eqref{eq:kk}}}{=}\int_{T_R\gamma}a=\int_{\gamma}a(R \bullet,R\bullet),
\end{align*}
as desired.
\proofpart{Step}{$\gamma$ is not a collision orbit (i.e. $\theta_0 \neq \pi/2,3\pi/2$)}
We prove the theorem for $\gamma$ not being a collision orbit, which will be important to the statement of Lemma \ref{lem:crit}.
If $a \in C^{\infty}(T^*(\mathbb{R}^3-\{0\}))$, then we have
$$\langle \operatorname{Op}_{\hbar}(a)\Psi_{\alpha,N},\Psi_{\alpha,N}\rangle=\int_{T^*\mathbb{R}^3}a(x,\xi)W_{\Psi_{\alpha,N},\Psi_{\alpha,N}}(x,\xi)dxd\xi,$$
where
$$W_{\Psi_{\alpha,N},\Psi_{\alpha,N}}(x,\xi) \coloneqq
\frac{1}{(2\pi
  \hbar)^3}\int_{\mathbb{R}^3}\Psi_{\alpha,N}(x+\tfrac{v}{2})\overline{\Psi_{\alpha,N}(x-\tfrac{v}{2})}e^{-\frac{i}{\hbar}\langle
  v,\xi \rangle}dv. $$
Using basic facts about Wigner distributions (see \cite{F89}), we see
\begin{align*}
W_{\Psi_{\alpha,N},\Psi_{\alpha,N}}(x,\xi) &=
                                             W_{\mathcal{V}^{-1}(\Phi_{\alpha,N}),\mathcal{V}^{-1}(\Phi_{\alpha,N})}(x,\xi)\\
  &= W_{(D_{p_0}\circ \mathcal{F}_{\hbar}^{-1} \circ J^{1/2}
    \circ K)(\Phi_{\alpha,N}),(D_{p_0}\circ
    \mathcal{F}_{\hbar}^{-1} \circ J^{1/2} \circ
    K)(\Phi_{\alpha,N})}(x,\xi)\\
  &= W_{(\mathcal{F}_{\hbar}^{-1} \circ J^{1/2} \circ
    K)(\Phi_{\alpha,N}),( \mathcal{F}_{\hbar}^{-1} \circ J^{1/2}
    \circ K)(\Phi_{\alpha,N})}(p_0x,\tfrac{1}{p_0}\xi)\\
  &= W_{( J^{1/2} \circ K)(\Phi_{\alpha,N}),( J^{1/2} \circ K)(\Phi_{\alpha,N})}(\tfrac{1}{p_0}\xi,-p_0x).
\end{align*}
So we have
\begin{align}
&\langle \operatorname{Op}_{\hbar}(a)\Psi_{\alpha,N},\Psi_{\alpha,N}\rangle =
                                           \int_{T^*\mathbb{R}^3}a(x,\xi)W_{\Psi_{\alpha,N},\Psi_{\alpha,N}}(x,\xi)dxd\xi\nonumber \\
                                         &=\int_{T^*\mathbb{R}^3}a(\tfrac{1}{p_0}x,p_0\xi)W_{\Psi_{\alpha,N},\Psi_{\alpha,N}}(\tfrac{1}{p_0}x,p_0\xi)dxd\xi \nonumber\\
  &=\int_{T^*\mathbb{R}^3}a(\tfrac{1}{p_0}x,p_0\xi)W_{( J^{1/2}
    \circ K)(\Phi_{\alpha,N}),( J^{1/2} \circ
    K)(\Phi_{\alpha,N})}(\xi,-x)dxd\xi\nonumber\\
  &=\frac{c_N^2}{(2\pi \hbar)^3}\int_{\mathbb{R}^3}\int_{T^*\mathbb{R}^3} \tfrac{16 a(\frac{1}{p_0}x,p_0\xi)}{ (|\xi+\frac{v}{2}|^2 +
               1)^2( |\xi-\frac{v}{2}|^2
               + 1)^2}\big(\alpha
\cdot \omega(\xi+\tfrac{v}{2})\big)^N\big(\overline{\alpha}
\cdot \omega(\xi-\tfrac{v}{2})\big)^Ne^{\frac{i}{\hbar}\langle
    v,x \rangle}dxd\xi dv\nonumber \\
    &=\frac{(N+1)^4}{16\pi^5}\int_{\mathbb{R}^3}\int_{T^*\mathbb{R}^3} f(x,\xi,v)e^{iN P(x,\xi,v)}dxd\xi dv, \label{eq:newish2}
\end{align}
where the last line we use $p_0^{-1}=\hbar(N+1)$ and the substitution $x \mapsto p_0^{-1}x$ and define
\begin{align*}
f(x,\xi,v) &\coloneqq \frac{ 16a(\frac{1}{p_0^2}x,p_0\xi)e^{i\langle v,x \rangle}}{ (|\xi+\frac{v}{2}|^2 +
               1)^2( |\xi-\frac{v}{2}|^2
               + 1)^2} \\
P(x,\xi,v) &\coloneqq -i\log\big(\alpha
\cdot \omega(\xi+\tfrac{v}{2})\big)-i\log\big(\overline{\alpha}
\cdot \omega(\xi-\tfrac{v}{2})\big)+\langle v,x \rangle.
\end{align*}
First note that $\Im P(x,\xi,v) \geq 0$. This is because $|\alpha
\cdot \omega(\xi\pm \tfrac{v}{2})| \leq 1$ since $|\alpha
\cdot \omega(\xi\pm \tfrac{v}{2})|$ is the norm of
projection of $\omega(\xi\pm \tfrac{v}{2})$ on the
$\operatorname{span}_{\mathbb{R}}(\Re\alpha,\Im\alpha)$. In particular, we have equality if and only if $\omega(\xi\pm \tfrac{v}{2})\in \operatorname{span}_{\mathbb{R}}(\Re\alpha,\Im\alpha)$.
We would like to apply stationary phase methods to formula
\eqref{eq:newish2}. We have the following lemma.
\begin{lem}\label{lem:crit}
For the complex phase $P$ as above, let $\mathcal{C} \coloneqq \{\nabla_x P =\nabla_{\xi} P = \nabla_v P=0,\Im P(x,\xi,v)=0\}$ be the critical manifold. Then
$$\mathcal{C}=
\bigg\{(x,\xi,v)=\bigg(\begin{psmallmatrix}\sin \beta -\sin \theta_0 \\ -\cos \theta_0 \cos \beta\\ 0 \end{psmallmatrix},\tfrac{1}{1-\sin
                         \theta_0 \sin \beta}\begin{psmallmatrix}\cos(\beta)\\ \sin(\beta)\cos(\theta_0)\\
                 0 \end{psmallmatrix},\begin{psmallmatrix} 0 \\ 0\\ 0 \end{psmallmatrix}\bigg) \mid \beta \in [0,2\pi)\bigg\}.$$
That is, $\mathcal{C}$ is precisely the Kepler orbit obtained by the image of the great circle generated by $\alpha$ under the Moser map with $E=-1/2$.               
\end{lem}
\begin{proof}
The condition $\nabla_x P =0$ implies $v=0$. As noted above, the second condition is equivalent to the
condition $\omega(\xi\pm \tfrac{v}{2})\in
\operatorname{span}_{\mathbb{R}}(\Re\alpha,\Im\alpha)$. Let $\beta$ be such that $\alpha \cdot \omega(\xi)=e^{i\beta}$. Since $\alpha = e_1+i(\cos(\theta_0)e_2+\sin(\theta_0)e_4)$, we have
$$\omega(\xi)=\cos(\beta)e_1+\sin(\beta)(e_2\cos
\theta_0+ e_4\sin \theta_0).$$
 Taking $\omega^{-1}$ on both sides, we have
\begin{align*}
\xi_1 =\frac{\cos(\beta)}{1-\sin
                         \theta_0 \sin \beta},\quad \xi_2=\frac{\sin(\beta)\cos(\theta_0)}{1-\sin\theta_0 \sin \beta},\quad \xi_3=0.
\end{align*}
Finally, the $\partial_{v_j}P(x,\xi,0)=0$
reads
\begin{align*}
-i\frac{\alpha_j+[\alpha_4-\alpha \cdot \omega(\xi)]\xi_j}{(|\xi|^2+1)(\alpha \cdot \omega(\xi))}+i\frac{\overline{\alpha}_j+[\overline{\alpha}_4-\overline{\alpha} \cdot \omega(\xi)]\xi_j}{(|\xi|^2+1)(\overline{\alpha} \cdot \omega(\xi))}+x_j=0,
\end{align*}
which implies
\begin{align*}
x_j&=  \Re \bigg( i\frac{2\alpha_j}{(|\xi|^2+1)(\alpha \cdot
     \omega(\xi))}+i\omega(\xi)_j\frac{\alpha_4-\alpha \cdot
     \omega(\xi)}{(\alpha \cdot \omega(\xi))}\bigg)\\
  &=(1-\sin \theta_0 \sin \beta)\Re \big(i\alpha_je^{-i\beta}\big)-\omega(\xi)_j\cos \beta \sin \theta_0.
\end{align*}
We see that
\begin{align*}
  x_1 = \sin \beta-\sin\theta_0,\quad x_2 = -\cos \beta\cos \theta_0,\quad x_3=0,
\end{align*}
as desired.
\end{proof}
Let $\pi_x \mathcal{C}$ denote the projection of $\mathcal{C}$
to configuration space, and let $\chi(x)\in
C_c^{\infty}(\mathbb{R}^3)$ be a non-negative smooth bump
function that is $1$ on $\pi_x \mathcal{C}$ and $0$ off of a
small tubular neighborhood of $\pi_x \mathcal{C}$ with $\chi \in
[0,1]$. Then the integral in \eqref{eq:newish2} becomes
\begin{align}
\int_{\mathbb{R}^3}\int_{T^*\mathbb{R}^3}
  f(x,\xi,v)e^{iN P(x,\xi,v)}dxd\xi
  dv&=\int_{\mathbb{R}^3}\int_{T^*\mathbb{R}^3}\chi(x)
      f(x,\xi,v)e^{iN P(x,\xi,v)}dxd\xi dv \nonumber \\
  &\qquad +\int_{\mathbb{R}^3}\int_{T^*\mathbb{R}^3}(1-\chi(x)) f(x,\xi,v)e^{iN P(x,\xi,v)}dxd\xi dv. \label{eq:newish3}
\end{align}
The second integral is $O(N^{-\infty})$ since the integral is
outside of $\mathcal{C}$. For the first integral, we do the
change of variables $x \mapsto (\beta,t,s)$ where
$$x=x(\beta)+tn_{\beta}+se_3 \quad \text{where} \quad x(\beta)
\coloneqq \begin{psmallmatrix}\sin \beta -\sin \theta_0 \\ -\cos
            \theta_0 \cos \beta\\ 0 \end{psmallmatrix},
            n_{\beta} \coloneqq \tfrac{1}{\sqrt{1-\sin^2\beta
                \sin^2\theta_0}}\begin{psmallmatrix}
                        -\sin\beta \cos \theta_0\\ \cos \beta\\
                                  0 \end{psmallmatrix},$$
where $\beta \in [0,2\pi)$ and $t^2+s^2<\delta$ for some $\delta>0$. Note that $\lVert n_{\beta} \rVert =1$ and $x(\beta) \cdot n_{\beta}=0$, so the change of variables is parametrizing a tubular neighborhood of $\pi_x \mathcal{C}$. With this change of variables, it can be easily computed that $dx=|\sqrt{1-\sin^2\beta
                \sin^2\theta_0}+t \cos \theta_0|dtdsd\beta$
              Altogether, by \eqref{eq:newish3}, we have
\begin{align}
\int_{\mathbb{R}^3}\int_{T^*\mathbb{R}^3}
  \chi(x)f(x,\xi,v)e^{iN P(x,\xi,v)}dxd\xi=\int_0^{2\pi}\int_{t^2+s^2<\delta}\int_{\mathbb{R}^6}\widetilde{f}_{\beta}(t,s,\xi,v)e^{iN \widetilde{P}_{\beta}(t,s,\xi,v)}d\xi dv dtds d\beta,\label{eq:barely}
\end{align}
where
\begin{align*}
\widetilde{f}_{\beta}(t,s,\xi,v)\coloneqq&\ f(x(\beta)+tn_{\beta}+se_3,\xi,v)\chi(x(\beta)+tn_{\beta}+se_3)\Big|\sqrt{1-\sin^2\beta
                                         \sin^2\theta_0}+t \cos \theta_0\Big|\\
\widetilde{P}_{\beta}( t,s,\xi,v)  \coloneqq& \  P(x(\beta)+tn_{\beta}+se_3,\xi,v).
\end{align*}
For fixed $\beta$, we apply the method of stationary phase in the variables $(t,s,\xi,v)$. By Lemma \ref{lem:crit}, the only critical point of $\widetilde{P}$ is at $(0,0,\xi(\beta),0)$ where $\xi(\beta) \coloneqq \tfrac{1}{1-\sin
                         \theta_0 \sin
                         \beta}\begin{psmallmatrix}\cos(\beta)\\
                                 \sin(\beta)\cos(\theta_0)\\
  0 \end{psmallmatrix}$. The
                                 Hessian of $\widetilde{P}$
                                 evaluated at this critical
                                 point is
                                 \begin{equation}\label{eq:HESS}
                                 \operatorname{Hess}(\widetilde{P})_{crit}=\bordermatrix{     
            & t     & s   & \xi & v      \cr
    t     & 0     & 0    & 0 & n_{\beta}^T      \cr
    s    & 0    & 0     & 0 & e_3^T      \cr
    \xi  & 0  & 0  & -2i \Re H_{\beta} & \Im H_{\beta}   \cr
    v     & n_{\beta}     & e_3     & \Im H_{\beta} & -\tfrac{i}{2}\Re H_{\beta}      \cr
  }, \end{equation}
  where $H_{\beta}$ is the Hessian of the function $\xi \mapsto
  \log(\alpha \cdot \omega(\xi))$ evaluated at $\xi(\beta)$. The
  calculation of the determinant of the matrix $H_{\beta}$ was
  studied in the Appendix A.3 of \cite{RC21}. Using a lower triangular block matrix
  identity, we have
\begin{equation}\label{eq:determinant}
\det \operatorname{Hess}(\widetilde{P})_{\beta}=-\det \begin{pmatrix}-2i \Re H_{\beta} & \Im H_{\beta}   \cr
   \Im H_{\beta} & -\tfrac{i}{2}\Re H_{\beta} \end{pmatrix} \det \begin{pmatrix}0 & n_{\beta}^T\\ 0 & e_3^T \end{pmatrix}\begin{pmatrix}-2i \Re H_{\beta} & \Im H_{\beta}   \cr
   \Im H_{\beta} & -\tfrac{i}{2}\Re H_{\beta} \end{pmatrix}^{-1}\begin{pmatrix}0 & 0\\ n_{\beta} & e_3 \end{pmatrix}.
\end{equation}
  Using the block matrix
  identity
  \begin{equation}\label{eq:blockid}
\begin{pmatrix}\frac{1}{2}I & -I \\ 0 &
                                          I \end{pmatrix}\begin{pmatrix}2A & B \\ B & \frac{1}{2}A \end{pmatrix}\begin{pmatrix}I &  \frac{1}{2}I\\ 0 & I \end{pmatrix}=\begin{pmatrix}A-B &  0\\ B &\frac{1}{2}(A+B) \end{pmatrix},
                                          \end{equation}
                                          we see
\begin{equation}\label{eq:hellomatrix}
\Big|\det \begin{pmatrix}-2i \Re H_{\beta} & \Im H_{\beta}   \cr
   \Im H_{\beta} & -\tfrac{i}{2}\Re H_{\beta} \end{pmatrix}\Big|=|\det H_{\beta}|^2.
   \end{equation}
Inverting the identity \eqref{eq:blockid} and using the formula
for the inverse of a triangular block matrix, we see
$$\begin{pmatrix}-2i \Re H_{\beta} & \Im H_{\beta}   \cr
   \Im H_{\beta} & -\tfrac{i}{2}\Re
                   H_{\beta} \end{pmatrix}^{-1}=\begin{pmatrix}
                  * & * \\ * & 2
                                                               \Re(H_{\beta}^{-1})\end{pmatrix}. $$
But note $n_{\beta}$ and $e_3$ are eigenvectors of $
\Re(H_{\beta}^{-1})$ with eigenvalues $\lambda_2,\lambda_3$
defined at (9.39) of \cite{RC21}. Thus
\begin{equation}\label{eq:hessfinal}
\sqrt{|\det \operatorname{Hess}(\widetilde{P})_{\beta}|}\overset{\eqref{eq:determinant},\eqref{eq:hellomatrix}}{=}2|\det H_{\beta}| \sqrt{\lambda_2\lambda_3}=2(1-\sin \beta \sin \theta_0)^3\sqrt{1-\sin^2 \beta \sin^2 \theta_0},
\end{equation}
where the last equality follows from (9.33) and (9.39) in
Chapter 9 of \cite{RC21} ($n_{\beta}$ is the normalized
$v_{\beta}$ in \cite{RC21}). Now we apply stationary phase to
\eqref{eq:barely}, and with \eqref{eq:newish3}, we see
\begin{align*}
&\frac{(N+1)^4}{16\pi^5}\int_{\mathbb{R}^3}\int_{T^*\mathbb{R}^3}
  f(x,\xi,v)e^{iN P(x,\xi,v)}dxd\xi dv\\
  &=\frac{(N+1)^4}{16\pi^5}\Big(\frac{2\pi}{N}\Big)^4\int_{0}^{2\pi}\frac{16a(\tfrac{1}{p_0^2}x(\beta),p_0
    \xi(\beta))}{(|\xi(\beta)|^2+1)^4}\frac{\sqrt{1-\sin^2\beta
    \sin^2\theta_0}}{2(1-\sin \beta \sin
    \theta_0)^3\sqrt{1-\sin^2 \beta \sin^2
    \theta_0}}d\beta+O(\tfrac{1}{N}) \\
  &=\frac{1}{2\pi} \int_{0}^{2\pi}a(\tfrac{1}{p_0^2}x(\beta),p_0
    \xi(\beta))(1-\sin \beta \sin
    \theta_0)d\beta+O(\tfrac{1}{N})\\
  &=\frac{p_0^3}{2\pi} \int_{0}^{2\pi/p_0^3}a\big(\gamma(t)\big)dt+O(\tfrac{1}{N}),
\end{align*}
where the last line we change variables $\beta \to t$ where $t$
is as in Theorem \ref{theo:moser}.
\stepcounter{proofpart}\stepcounter{proofpart}
\proofpart{Step}{$\gamma$ is a collsion orbit
  (i.e. $\theta_0=\pi/2,3\pi/2)$}
By reversing time, we can assume without loss of generality that
$\theta_0=\pi/2$. The setup is the same as in Step 2. We still
consider the integral \eqref{eq:newish2}, but the critical
manifold is now
\begin{equation*}
\mathcal{C} =\bigg\{\bigg(\begin{psmallmatrix}\sin \beta-1 \\ 0\\ 0 \end{psmallmatrix},\tfrac{1}{1-\sin \beta}\begin{psmallmatrix}\cos \beta\\ 0\\
                 0 \end{psmallmatrix},\begin{psmallmatrix} 0 \\ 0\\ 0 \end{psmallmatrix}\bigg) \mid \beta \in (-3\pi/2,\pi/2)\bigg\}.
                 \end{equation*}
We cannot apply the same change of variables in only the $x$
variables as before since the manifold degenerates into a line segment
when projected to configuration space. We instead consider a tubular neighborhood of $\mathcal{C} \cap \operatorname{supp} a(\frac{1}{p_0^2}\bullet, p_0 \bullet)$ in phase space. Let $\chi(x,\xi) \in C_{c}^{\infty}(T^*\mathbb{R}^3)$ be a non-negative smooth bump function that is $1$ on $\mathcal{C} \cap \operatorname{supp} a(\frac{1}{p_0^2}\bullet, p_0 \bullet)$ and $0$ off of a
small tubular neighborhood of $\mathcal{C} \cap \operatorname{supp} a(\frac{1}{p_0^2}\bullet, p_0 \bullet)$ with $\chi \in
[0,1]$. Then we have
\begin{align}
\int_{\mathbb{R}^3}\int_{T^*\mathbb{R}^3}
  f(x,\xi,v)e^{iN P(x,\xi,v)}dxd\xi
  dv&=\int_{\mathbb{R}^3}\int_{T^*\mathbb{R}^3}\chi(x,\xi)
      f(x,\xi,v)e^{iN P(x,\xi,v)}dxd\xi dv \nonumber \\
  &\qquad +\int_{\mathbb{R}^3}\int_{T^*\mathbb{R}^3}(1-\chi(x,\xi)) f(x,\xi,v)e^{iN P(x,\xi,v)}dxd\xi dv. \label{eq:newish'3}
\end{align}
The second integral is $O(N^{-\infty})$ since the integral is
outside of $\mathcal{C}$. For the first integral, we do a
change of variables. We define the following vectors:
$$x(\beta) \coloneqq \begin{psmallmatrix}\sin \beta-1 \\ 0\\ 0 \end{psmallmatrix}, \xi(\beta) \coloneqq \tfrac{1}{1-\sin \beta}\begin{psmallmatrix}\cos \beta\\ 0\\
                 0 \end{psmallmatrix},m_{\beta} \coloneqq
                 c_{\beta}\begin{psmallmatrix}\frac{1}{\sin
                            \beta-1} \\ 0\\
                            0 \end{psmallmatrix},m_{\beta}'
                     \coloneqq
                            c_{\beta}\begin{psmallmatrix}\cos
                                       \beta \\ 0\\
                                       0 \end{psmallmatrix} $$
                                       where $c_{\beta}
                                       \coloneqq (\cos^2
                                       \beta+\frac{1}{(1-\sin
                                        \beta)^2})^{-1/2}$ is a
                                       normalization factor. Now
                                       we do the change of
                                       variables
                                       $(x,\xi,v) \to
       (\beta,t_1,t_2,s_1,s_2,s_3,s_1',s_2',s_3')$
                                       where
                                       $$x=x(\beta)+t_1e_2+t_2e_3+s_1m_{\beta},
                                       \quad
                                       \xi=\xi(\beta)+s_1m_{\beta}'+s_2e_2+s_3e_3,
                                       \quad
                                       v=s_1'm_{\beta}'+s_2'e_2+s_3'e_3.$$
On $\mathcal{C}$, one can easily compute that $dxd\xi dv=|\cos \beta| dt ds ds'd\beta$. We proceed the same as before: we apply the method of stationary phase in the variables $(t,s,s')$ at the only critical point $(0,0,0)$. The Hessian is very similar to \eqref{eq:HESS} (in fact, this case is easier as the block matrices are diagonal), and one can compute that \begin{equation*}
\sqrt{|\det \operatorname{Hess}(\widetilde{P})_{\beta}|}=2(1-\sin \beta)^{3}|\cos \beta|.
\end{equation*}
We then have
           \begin{align*}
&\frac{(N+1)^4}{16\pi^5}\int_{\mathbb{R}^3}\int_{T^*\mathbb{R}^3}
  f(x,\xi,v)e^{iN P(x,\xi,v)}dxd\xi dv\\
  &=\frac{(N+1)^4}{16\pi^5}\Big(\frac{2\pi}{N}\Big)^4\int_{-3\pi/2}^{\pi/2}\frac{16a(\tfrac{1}{p_0^2}x(\beta),p_0
    \xi(\beta))}{(|\xi(\beta)|^2+1)^4}\frac{|\cos \beta|}{2(1-\sin \beta)^3|\cos \beta|}d\beta+O(\tfrac{1}{N}) \\
  &=\frac{1}{2\pi} \int_{-3\pi/2}^{\pi/2}a(\tfrac{1}{p_0^2}x(\beta),p_0
    \xi(\beta))(1-\sin \beta)d\beta+O(\tfrac{1}{N})\\
  &=\frac{p_0^3}{2\pi} \int_{t_{\gamma}-2\pi/p_0^3}^{t_{\gamma}}a\big(\gamma(t)\big)dt+O(\tfrac{1}{N}),
           \end{align*}
where the last line we change variables $\beta \to t$ where $t$
is as in Theorem \ref{theo:moser} and $t_{\gamma}$ is the collision time (defined at \eqref{eq:collisiontime}).         
\section{Proof of Theorem \ref{theo:2}}
We start by reducing the limit of the theorem to an integral on the space of oriented geodesics on $\Sigma_E$. This space is
$G(\Sigma_E) \coloneqq \Sigma_E/\sim$ where we quotient out by points on the same
geodesic. By Theorem \ref{theo:moser}, we have
$$G(\Sigma_E) = T_1^*\widehat{S}^3/S^1
=T_1^*S^3/S^1=\operatorname{SO}(4)/(\operatorname{SO}(2) \times \operatorname{SO}(2))=\widetilde{\textbf{Gr}}(2,4), $$
where $\widetilde{\textbf{Gr}}(2,4)$ is the oriented Grassmanian
manifold (i.e. the double cover of $\textbf{Gr}(2,4)$). That is,
the space of geodesics on $\Sigma_E$ is the same as the space of
geodesics on $S^3$. In particular, the space of geodesics is a
compact manifold. If we denote $\pi:\Sigma_E \to G(\Sigma_E)$
the projection, then the disintegration theorem says
\begin{equation}\label{eq:disint}
\int_{\Sigma_E}a(x,\xi)d\mu(x,\xi)= \int_{G(\Sigma_E)}\int_{\pi^{-1}(\gamma)}a(x,\xi)d\nu_{\gamma}(x,\xi) d \overline{\mu}(\gamma),
\end{equation}
where $\overline{\mu}\coloneqq \pi_*\mu$ and $\nu_{\gamma}$ are
probability measures on $\Sigma_E$ such that
$\operatorname{supp}\nu_{\gamma} \subseteq \pi^{-1}(\gamma)$ for
$\overline{\mu}$-almost all $\gamma \in G(\Sigma_E)$. Note that
\eqref{eq:disint} is true for merely $\mu$-integrable $a$, so
since $\mu$ is invariant under the Hamiltonian flow, we see that
(almost all) $\nu_{\gamma}$ is invariant under the Hamiltonian
flow. Then we have
$$\int_{\pi^{-1}(\gamma)}a(x,\xi)d\nu_{\gamma}(x,\xi) =\overline{a}(\gamma)\coloneqq \frac{p_0^3}{2\pi}\int_{0}^{2\pi/p_0^3}a(\gamma(t))dt,$$
where
$\overline{a}\in C(G(\Sigma_E))$ is
the Randon transform (we have extended the Hamiltonian flow of
collision orbits to be $2\pi/p_0^3$ periodic). With
\eqref{eq:disint}, this implies
\begin{equation}\label{eq:disint2}
\int_{\Sigma_E}a(x,\xi)d\mu(x,\xi)= \int_{G(\Sigma_E)} \overline{a}(\gamma) d \overline{\mu}(\gamma).
\end{equation}
For any $\gamma_0 \in G(\Sigma_E)$, Theorem \ref{theo:1}
says that for $\alpha$ generating $\gamma_0$, we have
\begin{equation}\label{eq:steppingstone}
\langle \operatorname{Op}_{\hbar}(a)\Psi_{\alpha,N},\Psi_{\alpha,N} \rangle \xrightarrow{N \to \infty}\delta_{\gamma_0}(\overline{a}) \coloneqq \int_{G(\Sigma_E)}\overline{a}(\gamma) d\delta_{\gamma_0}(\gamma).
\end{equation}
Now we would like to show the analogous statement to \eqref{eq:steppingstone} for convex
combinations of delta masses. Let $c_1,\ldots,c_n \in [0,1]$ be
such that $c_j>0$ and $\sum c_j=1$. Let
$\gamma_1,\gamma_2,\ldots,\gamma_n \in G(\Sigma_E)$ be distinct
geodesics with generators $\alpha_1,\ldots, \alpha_n \in
\mathcal{A}$, respectively. Then consider $\Psi_N \coloneqq
\sqrt{c_1}\Psi_{\alpha_1,N}+\cdots+\sqrt{c_n}
\Psi_{\alpha_n,N}$. We claim
\begin{equation}\label{eq:convex}
\langle \operatorname{Op}_{\hbar}(a)\Psi_{N},\Psi_{N} \rangle \xrightarrow{N \to \infty}\sum_jc_j\delta_{\gamma_j}(\overline{a}).
\end{equation}
Indeed, this follows immediately from \eqref{eq:steppingstone}
and the fact that $\langle
\operatorname{Op}_{\hbar}(a)\Psi_{\alpha_j,N},\Psi_{\alpha_k,N}
\rangle= O(N^{-\infty})$ for $j \neq k$, which we prove after this argument in Lemma \ref{lem:lem} (the hydrogen analog of Lemma 2.1 in \cite{TV-B97}). It is well-known (by the Krein-Milman theorem) that convex combinations of delta measures are weak-* dense in the compact convex set of probability measures on $G(\Sigma_E)$ (equipped with the weak-* topology). We can find eigenfunctions whose semiclassical limit coincides with any given convex combination of delta measures applied to $\overline{a}$ by \eqref{eq:convex}, so we are done by \eqref{eq:disint2}.
\begin{lem}\label{lem:lem} For $a \in C_c^{\infty}(T^*(\mathbb{R}^3-\{0\}))$ and $\alpha, \beta \in \mathcal{A}$ not generating
the same geodesic, we
have $\langle
\operatorname{Op}_{\hbar}(a)\Psi_{\alpha,N},\Psi_{\beta,N}
\rangle= O(N^{-\infty})$
\end{lem}
\begin{proof}
By the same calculations in Step 2 of the proof of Theorem
\ref{theo:1} we have
\begin{equation}\label{eq:laststationaryphase}
\langle \operatorname{Op}_{\hbar}(a)\Psi_{\alpha,N},\Psi_{\beta,N}
\rangle=\frac{(N+1)^4}{16\pi^5}\int_{\mathbb{R}^3}\int_{T^*\mathbb{R}^3} f(x,\xi,v)e^{iN P(x,\xi,v)}dxd\xi dv
\end{equation}
where
\begin{align*}
f(x,\xi,v) &\coloneqq \frac{ 16a(\frac{1}{p_0^2}x,p_0\xi)e^{i\langle v,x \rangle}}{ (|\xi+\frac{v}{2}|^2 +
               1)^2( |\xi-\frac{v}{2}|^2
               + 1)^2} \\
P(x,\xi,v) &\coloneqq -i\log\big(\alpha
\cdot \omega(\xi+\tfrac{v}{2})\big)-i\log\big(\overline{\beta}
\cdot \omega(\xi-\tfrac{v}{2})\big)+\langle v,x \rangle.
\end{align*}
First note that $\Im P(x,\xi,v) \geq 0$. This is because $|\alpha
\cdot \omega(\xi+ \tfrac{v}{2})| \leq 1$ since $|\alpha
\cdot \omega(\xi+ \tfrac{v}{2})|$ is the norm of
projection of $\omega(\xi\pm \tfrac{v}{2})$ on the
$\operatorname{span}_{\mathbb{R}}(\Re\alpha,\Im\alpha)$ (and analogously for the $\beta$ term). In particular, we have equality if and only if $\omega(\xi + \tfrac{v}{2})\in \operatorname{span}_{\mathbb{R}}(\Re\alpha,\Im\alpha)$ and $\omega(\xi - \tfrac{v}{2})\in \operatorname{span}_{\mathbb{R}}(\Re\beta,\Im\beta)$.
We would like to apply stationary phase methods to formula
\eqref{eq:laststationaryphase}. Note that
\begin{equation}\label{eq:xv}
\begin{cases}
  \nabla_x P =0\\
  \Im P=0
\end{cases}=\begin{cases}v=0\\
            |\alpha \cdot \omega(\xi)|=|\beta \cdot \omega(\xi)|=1\end{cases}
\end{equation}
Note that if $
\operatorname{span}_{\mathbb{R}}(\Re\alpha,\Im\alpha) \cap
\operatorname{span}_{\mathbb{R}}(\Re\beta,\Im\beta)=\{0\}$, then
the above cannot be satisfied since this would imply
$\omega(\xi)=0$. In this case we have no critical points,
and then \eqref{eq:laststationaryphase} is
$O(N^{-\infty})$. We split this up into two further cases:
\proofpart{Case}{$\operatorname{span}_{\mathbb{R}}(\Re\alpha,\Im\alpha) \cap
  \operatorname{span}_{\mathbb{R}}(\Re\beta,\Im\beta)$ is a
  line.}
If $\alpha$ undergoes a transformation $\alpha \to e^{i\theta}\alpha$, then $\Psi_{\alpha,N}$ undergoes just a phase change by a factor $e^{iN\theta}$ (and similarly for $\beta$). By multiplying both sides of the statement of the Lemma by phase changes, we may assume that $\Re \alpha=\Re \beta \in \operatorname{span}_{\mathbb{R}}(\Re\alpha,\Im\alpha) \cap
  \operatorname{span}_{\mathbb{R}}(\Re\beta,\Im\beta)$. By applying a rotation in $\operatorname{SO}(3)$, we may assume $$\alpha=e_1+i(\cos
  (\theta_0)e_2+\sin(\theta_0)e_4) \quad \text{and} \quad
  \beta=e_1+i (b_2e_2+b_3e_3+b_4e_4)$$ for some $\theta_0 \in
  [0,2\pi)$ and $b_2^2+b_3^2+b_4^2=1$ (see Step 1 of the proof
  of Theorem \ref{theo:1}). Since $\omega(\xi) \in
  \operatorname{span}(e_1)$ and $|\omega(\xi)|=1$, we must have
  $\omega(\xi)=\pm e_1 \in \mathbb{R}^4$, which implies $\xi
  =\pm e_1 \in \mathbb{R}^3$. On the other hand,
  $\partial_{\xi_j}P(x,\xi,0)=0$ is equivalent to 
\begin{equation*}
\frac{1}{\alpha \cdot \omega(\xi)}\big(\alpha_j+(\alpha_4-\alpha \cdot \omega(\xi))\xi_j \big)=-\frac{1}{\overline{\beta} \cdot \omega(\xi)}\big(\overline{\beta}_j+(\overline{\beta}_4-\overline{\beta} \cdot \omega(\xi))\xi_j \big)
\end{equation*}
Rearranging, we obtain
\begin{equation}\label{eq:lastequation}
\big[2(\alpha \cdot \omega(\xi))(\overline{\beta}\cdot
\omega(\xi))-\alpha_4(\overline{\beta}\cdot
\omega(\xi))-\overline{\beta}_4(\alpha \cdot
\omega(\xi))\big]\xi_j=\alpha_j(\overline{\beta}\cdot
\omega(\xi))+\overline{\beta}_j(\alpha \cdot \omega(\xi)).
\end{equation}
Since $\alpha \cdot \omega(\xi)=\overline{\beta }\cdot
\omega(\xi)=\pm 1$, we have
$$(2\mp \alpha_4\mp \overline{\beta}_4)\xi_j=\pm \alpha_j \pm
\overline{\beta}_j. $$
For $j=1$, this implies $\Im \alpha_4=\Im \beta_4$, and for $j=2,3$, we conclude $\Im \alpha=\Im \beta$. This contradicts $\operatorname{span}_{\mathbb{R}}(\Re\alpha,\Im\alpha) \cap
  \operatorname{span}_{\mathbb{R}}(\Re\beta,\Im\beta)$ being a line, so we have no critical points,
and so \eqref{eq:laststationaryphase} is
$O(N^{-\infty})$. 
\proofpart{Case}{$\operatorname{span}_{\mathbb{R}}(\Re\alpha,\Im\alpha)=
  \operatorname{span}_{\mathbb{R}}(\Re\beta,\Im\beta)$.}
By similar reductions that we applied to Case 1, we may assume $\alpha=e_1+i(\cos
  (\theta_0)e_2+\sin(\theta_0)e_4)$ and $\beta=
  \overline{\alpha}$. Since \eqref{eq:xv} implies $\alpha \cdot
  \omega(\xi)=e^{i \varphi}$ for some $\varphi$, then by
  \eqref{eq:lastequation} we have $$(e^{i \varphi}-i \sin
  \theta_0)\xi_j=\alpha_j. $$
  Since $\xi$ must be real, we are forced to have $\sin \varphi=\sin \theta_0$ (from $j=1$), but then $\xi_2$ would be complex unless $\theta_0=\varphi \in \pi/2+ \pi \mathbb{Z}$. But this contradicts $\alpha_1 =1$. We then have no critical points, and so \eqref{eq:laststationaryphase} is
$O(N^{-\infty})$.

\end{proof}

\bibliographystyle{alpha-reverse}
\bibliography{refs}

\end{document}