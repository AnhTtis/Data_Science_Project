Vehicle electrification is becoming mainstream globally to reduce carbon emissions and achieve sustainable transportation. As reported in \cite{GlobalEVSales}, a total of $10.5$ million new electric vehicles were delivered in 2022, with an increase of $55\%$ compared to 2021, accounting for $13\%$ of the market share and is expected to continue to grow in the coming years. 

As part of vehicle electrification, road freight electrification is crucial for reducing greenhouse gas emissions caused by diesel-powered trucks in freight operations, which are responsible for around $25\%$ of the vehicle-related carbon emissions in Europe \cite{siskos2019assessing}. However, the process of freight truck electrification today is lagging far behind that of electric passenger vehicles \cite{gloableev}. A major concern with electrifying trucks, among others, is their limited driving ranges, known as \emph{range anxiety}. Currently, the average travel range of a commercial electric truck on a full battery is varied between $200$ and $600$ kilometers, depending on diverse truckload and battery capacity~\cite{wassiliadis2021review}. This is typically insufficient to sustain trucks to complete their delivery missions without stopping and refilling batteries, especially for long-haul journeys. To diminish range anxiety, increase electric truck adoptions, and accelerate road freight electrification, reliable and efficient charging strategies are needed. In addition to charging batteries, truck drivers also need to stop and take rests during trips to avoid driving fatigue. The so-called hours-of-service (HoS) regulations \cite{poliak2018social} address exactly this issue and put restrictions on how long one can drive consecutively without rest, as well as during one day. As a result, charging strategies for electric trucks should be designed not only for mission completion but also to align with the HoS regulations.

To date, there have been extensive works developing viable charging strategies. A majority of these approaches integrate charging stops into conventional routing problems and minimize the travel time or energy required on the route, as in \cite{storandt2012quick,schneider2014electric,huber2020optimization}. The authors in \cite{cussigh2019optimal} propose an optimal driving and charging strategy for electric vehicles. It includes the driving speed as an additional control variable when minimizing the total travel time. However, none of these works incorporates the HoS regulations in optimal charging problems. To the best of our knowledge, \cite{zahringer2022time} is the first to incorporate today's HoS regulations in their charging strategy, which is obtained via a genetic algorithm. Our work differs from the approach in \cite{zahringer2022time} in two ways. Firstly, we model the route and optimal charging problem in a more general framework, allowing for multiple rests within the maximum daily driving time before delivery deadlines. Secondly, a rollout-based online solution scheme of high efficiency is developed, which provides great potential for real-time optimization to deal with travel time uncertainties or model mismatches as opposed to the genetic algorithm, which is offline and time-consuming.   

In this letter, we study the optimal charging strategy for electric trucks with realistic HoS regulations. In particular, we consider an electrified transportation system where every electric truck has a pre-planned route for a delivery task. With the knowledge of a collection of charging and rest stations available along the route, the truck driver could design an optimal charging strategy to determine where and how long to recharge the truck and take rests so that the extra operational costs due to the charging and rest decisions are minimized. The main contributions are: \emph{i)} we model the optimal charging problems of electric trucks as mixed integer programs with \emph{bilinear constraints} that incorporate the delivery deadlines and HoS regulations; \emph{ii)} a rollout-based approximate solution method, as well as its variants, is developed for addressing the problem, through which the computational demands required by exact solution approaches are significantly decreased while offering solid performance guarantees. Simulation studies performed using realistic road and truck data illustrate the effectiveness of the proposed method.