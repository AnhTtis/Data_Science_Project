\documentclass{ieeeaccess}
\usepackage{cite}
\usepackage{amsmath,amssymb,amsfonts}
\usepackage{algorithmic}
\usepackage{graphicx}
\usepackage{textcomp}

% Authors' packages
\usepackage{color, colortbl}
\definecolor{shadecolor}{gray}{0.9}
\usepackage{booktabs}
\usepackage{tabularx}
\usepackage{threeparttable}
\usepackage{comment}
\usepackage{multirow}
% Acronyms 
\usepackage[acronym]{glossaries}

\glsdisablehyper
\newacronym{ai}{AI}{Artificial Intelligence}
\newacronym{if}{IF}{Integrate and Fire}
\newacronym{iot}{IoT}{Internet of Things}
\newacronym{ann}{ANN}{Artificial Neural Network}
\newacronym{snn}{SNN}{Spiking Neural Network}
\newacronym{rsnn}{RSNN}{Recurrent Spiking Neural Network}
\newacronym{dnn}{DNN}{Deep Neural Network}
\newacronym{cnn}{CNN}{Convolutional Neural Network}
\newacronym{lif}{LIF}{Leaky Integrate and Fire}
\newacronym{asic}{ASIC}{Application-Specific Integrated Circuit}
\newacronym{fpga}{FPGA}{Field Programmable Gate Array}
\newacronym{stdp}{STDP}{Spike-Timing-Dependent Plasticity}
\newacronym{wta}{WTA}{Winner Takes All}
\newacronym{rtl}{RTL}{Register Transfer Level}
\newacronym{hdl}{HDL}{Hardware Description Language}
\newacronym{vhdl}{VHDL}{VHSIC Hardware Description Language}
\newacronym{sota}{SOTA}{State Of The Art}
\newacronym{shd}{SHD}{Spiking Heildelberg Digits}
\newacronym{sbs}{SbS}{Spike-by-Spike}
\newacronym{ml}{ML}{Machine Learning}
\newacronym{dl}{DL}{Deep Learning}
\newacronym{scnn}{SCNN}{Spiking Convolutional Neural Networks}
\newacronym{fc}{FC}{Fully-Connected}
\newacronym{fffc}{FF-FC}{Feed Forward Fully-Connected}
\newacronym{fsm}{FSM}{Finite State Machine}
\newacronym{MAC}{MAC}{Multiply and Accumulate}
\newacronym{BRAM}{BRAM}{Block RAM}
\newacronym{SRAM}{SRAM}{Static Random Access Memory}
\newacronym{DRAM}{DRAM}{Dynamic Random Access Memory}
\newacronym{LUT}{LUT}{Look Up Table}
\newacronym{FF}{FF}{Flip Flop}
\newacronym{DVS}{DVS}{Dynamic Vision Sensor}
\newacronym{SHD}{SHD}{Spiking Heidelberg Dataset}
\newacronym{BPTT}{BPTT}{Back-Propagation Through Time}
\newacronym{CU}{CU}{Control Unit}
\newacronym{ROM}{ROM}{Read Only Memory}
\newacronym{RAM}{RAM}{Random Access Memory}
\newacronym{DP}{DP}{Data Path}
\newacronym{CPS}{CPS}{Cyber-Physical System}
\newacronym{MLP}{MLP}{Multilayer Perceptron}
\newacronym{PE}{PE}{Portable Executable}
\newacronym{OS}{OS}{Operating System}
\newacronym{SBO}{SBO}{Stack Buffer Overflow}
\newacronym{ROP}{ROP}{Return-Oriented Programming}
\newacronym{JOP}{JOP}{Jump-Oriented Programming}
\newacronym{TP}{TP}{True Positive}
\newacronym{TN}{TN}{True Negative}
\newacronym{FP}{FP}{False Positive}
\newacronym{FN}{FN}{False Negative}
\newacronym{TPR}{TPR}{True Positive Rate}
\newacronym{FPR}{FPR}{False Positive Rate}
\newacronym{TNR}{TNR}{True Negative Rate}
\newacronym{ROC}{ROC}{Receiver Operating Characteristic}
\newacronym{AUC}{AUC}{Area Under the Curve}
\newacronym{HMD}{HMD}{Hardware-Supported Malware Detection}
\newacronym{HPC}{HPC}{Hardware Performance Counter}
\newacronym{PMU}{PMU}{Performance Monitoring Unit}
\newacronym{FE}{FE}{Feature Extraction}
\newacronym{FS}{FS}{Feature Selection}
\newacronym{PCA}{PCA}{Principal Component Analysis}
\newacronym{KNN}{KNN}{K-Nearest Neighbors}
\newacronym{SVM}{SVM}{Support Vector Machines}
\newacronym{HLS}{HLS}{High-Level Synthesis}
\newacronym{RADICS}{RADICS}{Rapid Attack Detection, Isolation and Characterization Systems}
\newacronym{TDT}{TDT}{Threat Detection Technology}
\newacronym{FCNN}{FCNN}{Fully Convolutional Neural Network}
\newacronym{FFT}{FFT}{Fast Fourier Transform}


\def\BibTeX{{\rm B\kern-.05em{\sc i\kern-.025em b}\kern-.08em
    T\kern-.1667em\lower.7ex\hbox{E}\kern-.125emX}}
\begin{document}
\history{\textcolor{white}{Date of publication xxxx 00, 0000, date of current version xxxx 00, 0000.}}
\doi{\textcolor{white}{XX.XXX/XXXX.XX.XX}}

\title{A survey on hardware-based malware detection approaches}

\author{
\uppercase{Cristiano Pegoraro Chenet}\authorrefmark{1}, \IEEEmembership{Student member, IEEE},
\uppercase{Alessandro Savino}\authorrefmark{1}, \IEEEmembership{Senior member, IEEE}, and
\uppercase {Stefano di Carlo}\authorrefmark{1}, \IEEEmembership{Senior Member, IEEE}.}
\address[1]{Department of Control and Computer Engineering, Politecnico di Torino, Corso Duca Degli Abruzzi, 24, 10129, Torino (TO), Italy (e-mails: {cristiano.chenet, alessandro.savino, stefano.dicarlo}@polito.it)}

\tfootnote{This work was partially supported by project SERICS (PE00000014) under the MUR National Recovery and Resilience Plan funded by the European Union - NextGenerationEU and by the Vitamin-V project (Project number: 101093062) funded by the European Union. Views and opinions expressed are, however, those of the author(s) only and do not necessarily reflect those of the European Union or the HaDEA. Neither the European Union nor the granting authority can be held responsible for them.}

\markboth
{Chenet et. al \headeretal: A survey on hardware-based malware detection approach}
{Chenet et. al \headeretal: A survey on hardware-based malware detection approach}

\corresp{Corresponding author: Stefano Di Carlo (e-mail: stefano.dicarlo@polito.it).}

\begin{abstract} 
This paper delves into the dynamic landscape of computer security, where malware poses a paramount threat. Our focus is a riveting exploration of the recent and promising hardware-based malware detection approaches. Leveraging hardware performance counters and machine learning prowess, hardware-based malware detection approaches bring forth compelling advantages such as real-time detection, resilience to code variations, minimal performance overhead, protection disablement fortitude, and cost-effectiveness.
Navigating through a generic hardware-based detection framework, we meticulously analyze the approach, unraveling the most common methods, algorithms, tools, and datasets that shape its contours. This survey is not only a resource for seasoned experts but also an inviting starting point for those venturing into the field of malware detection.
However, challenges emerge in detecting malware based on hardware events. We struggle with the imperative of accuracy improvements and strategies to address the remaining classification errors. The discussion extends to crafting mixed hardware and software approaches for collaborative efficacy, essential enhancements in hardware monitoring units, and a better understanding of the correlation between hardware events and malware applications.
\end{abstract}

\begin{keywords}
Cybersecurity, malware, hardware-based detection, hardware-based framework
\end{keywords}

\titlepgskip=-15pt

\maketitle

\section{Introduction}
\label{sec:introduction}
% \begin{itemize}
%     % Diffusion of FL
%     \item {\st{Diffusion of FL}}
%     % Security threats to FL
%     \item {\st{Security threats to FL with particular focus on model poisoning}}
%     % Limitations of existing countermeasures
%     \item {\st{Current countermeasures (e.g., KRUM) and their limitations}}
%     % Proposed method and its advantages
%     \item {\st{Intuitive description of the proposed method and its difference (i.e., advantages) w.r.t. state of the art}}
%     % Main contributions
%     \item {\st{Summary of the main contributions of this work}}
%     % Paper's structure and organization
%     \item {\st{Paper's structure and organization}}
% \end{itemize}

% Diffusion of FL
Recently, {\em federated learning} (FL) has emerged as the leading paradigm for training distributed, large-scale, and privacy-preserving machine learning (ML) systems~\cite{mcmahan2017googleai,mcmahan2017aistats}. 
The core idea of FL is to allow multiple edge clients to collaboratively train a shared, global model without disclosing their local private training data.
%Specifically, an FL system consists of a central server and many edge clients; 
A typical FL round involves the following steps: {\em(i)} the server randomly picks some clients and sends them the current, global model; {\em(ii)} each selected client locally trains its model with its own private data; then, it sends the resulting local model to the server;\footnote{Whenever we refer to global/local model, we mean global/local model {\em parameters}.} {\em(iii)} the server updates the global model by computing an \emph{aggregation function}, usually the average (FedAvg), on the local models received from clients.
% \begin{enumerate}
%     \item[{\em(i)}] the server sends the current, global model to the clients and appoints some of them for training;
%     \item[{\em(ii)}] each selected client locally trains its copy of the global model with its own private data; then, it sends the resulting local model back to the server;\footnote{Whenever we refer to global/local model, we mean global/local model {\em parameters}.}
%     \item[{\em(iii)}] the server updates the global model by computing an \emph{aggregation function} on the local models received from clients (by default, the average, also referred to as FedAvg~\cite{mcmahan2017aistats}).
% \end{enumerate}
This process goes on until the global model converges. %(e.g., after a certain number of rounds or other similar stopping criteria).
%\\
% The advantages of FL over the traditional, centralized learning paradigm are undoubtedly clear in terms of flexibility/scalability (clients can join/disconnect from the FL network dynamically), network communications (only model weights\footnote{We will use \textit{parameters} and \textit{weights} interchangeably.} are exchanged between clients and server), and privacy (each client's private training data is kept local at the client's end and not uploaded to the server).
\\
% Security threats to FL
%However, the growing adoption of FL also raises security concerns~\cite{costa2022covert}, particularly about its confidentiality, integrity, and availability.
Although its advantages over standard ML, FL also raises security concerns~\cite{costa2022covert}. %, particularly about its confidentiality, integrity, and availability~\cite{costa2022covert}.
% OLD, LONG VERSION
% Indeed, some work deals with privacy leakage that may expose the local data of some clients~\cite{melis2019sp}. 
% A large body of work, instead, investigates attacks that usually aim to detriment the predictive accuracy of the learned global model. For instance, \emph{data poisoning} attacks achieve this goal by letting an adversary pollute the training set of some corrupt FL clients with maliciously crafted examples~\cite{jagielski2018sp}.
% Similarly, in \emph{model poisoning} the attacker attempts to tweak the global model weights~\cite{bhagoji2019pmlr} by directly perturbing the local model's weights of some infected FL clients before these are sent to the central server for aggregation, usually via so-called Byzantine attacks. 
% It turns out that Byzantine model poisoning attacks severely impact standard FedAvg; therefore, more robust aggregation functions must be designed to make FL systems secure.
Here, we focus on \emph{untargeted model poisoning} attacks~\cite{bhagoji2019pmlr}, where an adversary attempts to tweak the global model weights %\footnote{We will use the terms \textit{parameters} and \textit{weights} interchangeably.} 
by directly perturbing the local model's parameters of some infected clients before these are sent to the central server for aggregation.
In doing so, the adversary aims to jeopardize the global model \textit{indiscriminately} at inference time.
Such model poisoning attacks severely impact standard FedAvg; therefore, more robust aggregation functions must be designed to secure FL systems.
\\
% In this paper, we focus on designing a novel robust aggregation scheme at the server's end to contrast the effect of Byzantine model poisoning attacks.
%
% Current countermeasures and their limitations
%Several countermeasures have been proposed in the literature to combat model poisoning attacks on FL systems.
% Some methods use simple statistics more robust than plain average to smooth the impact of malicious updates (e.g., Trimmed Mean and FedMedian~\cite{yin2018icml}). 
% Other defenses implement outlier detection techniques to discard malicious updates from the aggregation performed at the server's end. Those are either based on heuristics (e.g., Krum/Multi-Krum~\cite{blanchard2017nips} and Bulyan~\cite{mhamdi2018pmlr}) or data-driven approaches (e.g., K-means clustering~\cite{shen2016acm} or DnC via spectral analysis~\cite{shejwalkar2021ndss}). 
% Finally, some strategies rely on a centralized ``source of trust'' to spot potential malicious updates (e.g., FLTrust~\cite{cao2020fltrust}).
% Several countermeasures have been proposed in the literature to combat model poisoning attacks on FL systems, i.e., to discard possible malicious local updates from the aggregation performed at the server's end. 
% These techniques range from simple statistics more robust than plain average (e.g., Trimmed Mean and FedMedian~\cite{yin2018icml}) to outlier detection heuristics (e.g., Krum/Multi-Krum~\cite{blanchard2017nips} and Bulyan~\cite{mhamdi2018pmlr}) or data-driven approaches (e.g., spectral analysis via K-means clustering~\cite{shen2016acm} or spectral analysis), or methods based on ``source of trust'' (e.g., FLTrust~\cite{cao2020fltrust}).
% OLD, LONG VERSION
%Several countermeasures have been proposed in the literature to combat Byzantine model poisoning attacks on FL systems.
% Descriptive statistics
% For example, Trimmed Mean and FedMedian aggregate local model updates using more robust statistics than standard average~\cite{yin2018icml}.
%
% % Heuristics for outlier detection
% Many existing Byzantine-resilient strategies implement some outlier detection heuristics to discard the model updates sent by potentially malicious clients from the input of the aggregation function.
% One of the most popular heuristics is Krum~\cite{blanchard2017nips}.
% This strategy tries to mitigate the impact of Byzantine attacks by selecting as a global model the local model with the smallest sum of Euclidean distances to {\em all} the other local models.
% Although powerful, Krum requires the server to know (or, at least, estimate) the number of malicious FL clients upfront, which is generally impossible in a realistic attack scenario. %
% Moreover, Krum may become ineffective for complex, high-dimensional model parameter spaces due to the curse of dimensionality.
% Bulyan~\cite{mhamdi2018pmlr} tries to overcome this issue by combining Krum with a variant of Trimmed Mean.
% % Data-driven outlier detection
% Other strategies use data-driven outlier detection techniques -- e.g., via K-means clustering~\cite{shen2016acm} -- to spot potential malicious local model updates. 
% %For instance, Shen et al. propose to cluster local model updates with K-means and thus identify outliers.
%
% % Other techniques
% As far as the server is concerned, any local model received can be from a potential malicious client. 
% FLTrust~\cite{cao2020fltrust} assumes the server acts as a client, i.e., trains a local model on an additional {\em trustworthy} dataset at the server's end and compares it against all the local models from other clients. 
% This way, the server can rely on some ``source of trust'' when discarding potentially malicious clients.
%\\
% Limitations of existing Byzantine-resilient strategies
Unfortunately, existing defense mechanisms either rely on simple heuristics (e.g., Trimmed Mean and FedMedian by~\cite{yin2018icml}) or need strong and unrealistic assumptions to work effectively (e.g., foreknowledge or estimation of the number of malicious clients in the FL system, as for Krum/Multi-Krum~\cite{blanchard2017nips} and Bulyan~\cite{mhamdi2018pmlr}, which, however, cannot exceed a fixed threshold).
Furthermore, outlier detection methods using K-means clustering~\cite{shen2016acm} or spectral analysis like DnC~\cite{shejwalkar2021ndss} do not directly consider the temporal evolution of local model updates received.
Finally, strategies like FLTrust~\cite{cao2020fltrust} require the server to collect its own dataset and act as a proper client, thereby altering the standard FL protocol.
\\
% OLD, LONG VERSION
% Overall, existing Byzantine-resilient strategies are either simple heuristics (e.g., FedMedian) or, if they are more complex, they rely on strong and unrealistic assumptions to work effectively (e.g., knowing the number of malicious clients in the FL system in advance, as for Krum and alike).
% Furthermore, data-driven outlier detection methods do not consider the temporary evolution of local model updates received (e.g., K-means clustering). 
% Finally, strategies like FLTrust requires the server to collect its own dataset and act as a proper client, thereby altering the standard FL protocol.
%
% Description of the proposed method
This work introduces a novel pre-aggregation \textit{filter} robust to untargeted model poisoning attacks. Notably, this filter $(i)$ operates without requiring prior knowledge or constraints on the number of malicious clients and $(ii)$ inherently integrates temporal dependencies. 
The FL server can employ this filter as a preprocessing step before applying \textit{any} aggregation function, be it standard like FedAvg or robust like Krum or Bulyan.
Specifically, we formulate the problem of identifying corrupted updates as a multidimensional (i.e., matrix-valued) time series anomaly detection task. 
The key idea is that legitimate local updates, resulting from well-calibrated iterative procedures like stochastic gradient descent (SGD) with an appropriate learning rate, show \textit{higher predictability} compared to malicious updates. This hypothesis stems from the fact that the sequence of gradients (thus, model parameters) observed during legitimate training exhibit regular patterns, as validated in Section~\ref{subsec:intuition}. %until convergence. 
%This regularity may be more pronounced for smooth convex loss functions, but it can still be captured within an appropriate time window, even for more complex and convoluted loss surfaces. 
%We provide evidence of this claim in Appendix~B, where we show that the average mutual information (i.e., ``predictability''), calculated over pairs of legitimate model updates sent at different FL rounds, is significantly higher than the corresponding computation for a malicious client.
\\
Inspired by the matrix autoregressive (MAR) framework for multidimensional time series forecasting~\cite{chen2021je}, we propose the FLANDERS ({\em \textbf{F}ederated \textbf{L}earning meets \textbf{AN}omaly \textbf{DE}tection for a \textbf{R}obust and \textbf{S}ecure}) filter.
The main advantages of FLANDERS over existing strategies like FLDetector~\cite{zhao2020multivariate} are its resilience to large-scale attacks, where $50\%$ or more FL participants are hostile, and the capability of working under realistic non-iid scenarios.
We attribute such a capability to two key factors: $(i)$ FLANDERS works without knowing a priori the ratio of corrupted clients, and $(ii)$ it embodies temporal dependencies between intra- and inter-client updates, quickly recognizing local model drifts caused by evil players. Below, we summarize our main contributions:

\begin{itemize}
\item[{\em(i)}]
We provide empirical evidence that the sequence of models sent by legitimate clients is more predictable than those of malicious participants performing untargeted model poisoning attacks.
\\
\item[{\em(ii)}] 
We introduce FLANDERS, the first pre-aggregation filter for FL robust to untargeted model poisoning based on multidimensional time series anomaly detection.
\\
\item[{\em(iii)}] 
We integrate FLANDERS into Flower,\footnote{\scriptsize{\url{https://flower.dev/}}} a popular FL simulation framework for reproducibility.
\\
\item[{\em(iv)}] 
We show that FLANDERS improves the robustness of the existing aggregation methods under multiple settings: different datasets, client's data distribution (non-iid), models, and attack scenarios.
\\
\item[{\em(v)}] 
We publicly release all the implementation code of FLANDERS along with our experiments.\footnote{\scriptsize{\url{https://anonymous.4open.science/r/flanders_exp-7EEB}}}
\end{itemize}

% Paper's structure and organization
The remainder of the paper is structured as follows. %some related work and the current state-of-the-art solutions to security issues that FL entails. 
Section~\ref{sec:background} covers background and preliminaries. 
In Section~\ref{sec:related}, we discuss related work.
Section~\ref{sec:problem} and Section~\ref{sec:method} describe the problem formulation and the method proposed. % to tackle it. 
Section~\ref{sec:experiments} gathers experimental results. %, and Section~\ref{sec:limitations} discusses some limitations of this work.
Finally, we conclude in Section~\ref{sec:conclusion}.
 %discusses the limitations of this work and draws future research directions.
%reports conclusions and draws perspectives for future research directions.

%%%%%%% OLD %%%%%%%
%to overcome the resilience of Byzantine failures in distributed Stochastic Gradient Descent computations. 
% The strength of Krum is its time complexity, which is linear in the gradient dimension. 
% However, the robustness of the approach is guaranteed for gradient-based learning applications only when the majority of the clients are not compromised. 
% Besides, the aggregation mechanism of Krum, as well as that of similar methods, is robust from a coarse-grained perspective and does not provide solutions to errors and perturbations that may occur at inference time.
%A related approach to~\cite{blanchard2017nips} is the work of Su et al.~\cite{su2016dc}. Here, the authors propose an iterated approximate agreement to tackle a multi-layer scenario attacked by Byzantine agents. 
%However, the method works efficiently on the sole discrete context and it is inapplicable to continuous state environments.
%\gabri{Maybe, we should just talk about the main limitations of existing countermeasures without digging into their details (or, we can just mention Krum as this is the most popular one). I will move the description of all these methods to the Related Work section.}
\section{Malware Fundamentals}
\label{sec:malware_basics}

Categorizing malware is difficult because of its growing complexity and diverse properties. Yet, creating a malware taxonomy provides valuable insights into understanding it better. Before exploring the fundamentals of malware operation, let us define a set of keywords commonly used to describe different malware categories \cite{McGraw_Morrisett_2000,Christodorescu_2007}:

\begin{itemize}
	\item \textbf{Virus}: malicious code with the capability of inserting itself into other programs;
 	\item \textbf{Worm}: malicious code that propagates similarly to viruses but does not require a target software to replicate, often exploiting connectivity such as emails;
	\item \textbf{Trojan horse}: malicious code that masquerades as a useful program;
	\item \textbf{Spyware}: malicious code secretly installed into an information system to transmit private user data to an external entity;
	\item \textbf{Adware}: malicious code that displays computer advertisements, primarily aiming for financial benefits;
	\item \textbf{Ransomware}: malicious code that denies access to a user’s data, usually by encrypting it until a ransom is paid;
	\item \textbf{Backdoor}: malicious code that opens systems to external entities by subverting local security policies to allow remote access and control over a network;
	\item \textbf{Keylogger}: malicious code designed to record keystrokes, used to obtain passwords or encryption keys to bypass security measures;
	\item \textbf{Botnet}: a network of infected computers controlled by a remote criminal;
	\item \textbf{Rootkit}: malicious application attackers use to conceal their activities and maintain control over a host.
\end{itemize}

Organizations like NIST \cite{NIST_Glossary_2023} and ENISA \cite{ENISA_Botnets_2023} recognize these malware types. In literature, three common properties describe malware: (i) propagation method, categorizing based on spread and purpose; (ii) concealment strategy, focusing on hiding tactics against users and detection; and (iii) data structure manipulation, dealing with software vulnerability exploitation. Table \ref{tab:classifiers_analysis} organizes malware based on these categories.

\begin{table*}[hbt]
 \centering
\caption{Malware categories based on propagation method, concealment strategy, and data structure manipulation.}\label{tab:malware_classification}
\begin{tabular}{lccc}
\toprule
\textbf{Malware Type} & \textbf{Propagation Method} & \textbf{Concealment Strategy} & \textbf{Data Structure Manipulation} \\
\midrule\\
\rowcolor{shadecolor}
Worms & Network-based transmission & Polymorphisms or metamorphism & Exploitation of memory corruption vulnerabilities \\
\midrule
Viruses & File-based transmission & Polymorphism & Manipulation of data structures \\
\midrule
\rowcolor{shadecolor}
Trojans & Social engineering & No concealment & - \\
\midrule
Spyware & Internet downloads & Encryption & - \\
\midrule
\rowcolor{shadecolor}
Ransomware & Email attachments & Encryption & File system manipulation \\
\midrule
Adware & Software bundling & No concealment & - \\
\midrule
\rowcolor{shadecolor}
Rootkits & Kernel-level exploits & Obfuscation & Manipulation of system structures \\
\midrule
Backdoors & Remote access & Encryption & - \\
\midrule
\rowcolor{shadecolor}
Keyloggers & Phishing, infected software & Encryption & - \\
\midrule
Botnets & Exploitation, social engineering & Encryption and polymorphism & - \\
\bottomrule
\end{tabular}
\end{table*}

Regarding concealment strategy, malware can be categorized into two main groups: (i) no concealment and (ii) stealthy malware \cite{Aycock_2010,You_Yim_2010,Rad_2012}. No concealed malicious code lacks techniques to hide itself, making it easy to detect. However, as shown in Table \ref{tab:malware_classification}, only a small subset of malware does not employ concealment. File infectors like traditional viruses or worms may not heavily focus on concealment, spreading by attaching to executable files. Adware may not invest heavily in hiding and may rely on user interactions. Similarly, if achieved without sophisticated evasion, simple trojans may prioritize their primary goal over concealment.

Conversely, stealthy malware is a general term for all kinds of malicious code capable of hiding from users and detection mechanisms \cite{Stolfo_2007, Rudd_2017}. Its primary purpose is to remain undetected for an extended period in the computing system, allowing compromising computers and stealing information before a suitable detection mechanism can be deployed to protect against it. In general, the concealment actions aim to hide the malware's trails or code. Stealthy malware may employ several techniques:

\begin{itemize}
    \item \textbf{Encryption/obfuscation}: the oldest and simplest technique consists of a decryptor and an encrypted main body. When the infected file runs, the decryptor recovers the main body. The malware may use a different key for each infection to hide its signature, making the encrypted part unique. The decryptor small size compared to the main body reduces detection probability. Encryption complexity ranges from basic operations to strong encryption methods \cite{Aycock_2010,You_Yim_2010, Nadim:2021aa};
    
    \item \textbf{Oligomorphism and polymorphism}: the encryption technique limitation lies in the constant decryptor across exploitations, enabling detection based on code patterns. Oligomorphism employs a small set of decryptors, using a different one for each infection. Polymorphism, similar but with theoretically infinite decryptor variations, relies on obfuscation methods like dead-code insertion and register reassignment for distinct decryptor creation \cite{Aycock_2010,You_Yim_2010,Wong_Stamp_2006,Konstantinou_2008};
    
    \item \textbf{Metamorphism}: the binary sequence is altered by making a new malware version for each new infection through a mutation engine. The mutation engine uses code transforming and obfuscation to change the malicious code \cite{Aycock_2010,You_Yim_2010,Brezinski_2023}.
 \end{itemize}

Several classes of software vulnerabilities can be explored to perform security attacks. This paper focuses on the prevalent memory errors enabling memory corruption for security attacks \cite{Van_der_Veen_2012}, which lead to two main exploit categories: control-flow attacks and data-only attacks.

Control-flow attacks are common, easy to construct, and demand minimal application-specific knowledge. They exploit vulnerabilities like buffer overflows or injection attacks to redirect the program's execution flow, enabling arbitrary code execution \cite{Demme_2013, Khasawneh_2015, Ozsoy_2015, Tang_2014, Wang_Karri_2013}. Techniques such as code injection \cite{Ray_Ligatti_2012}, \gls{ROP} \cite{Prandini_Ramilli_2012}, or \gls{JOP} \cite{Bletsch_2011} divert execution to specific memory locations housing malicious code, bypassing standard security measures.

In contrast, data-only attacks are rarer, subtler, and require advanced knowledge of program semantics. They manipulate critical data while maintaining a valid control flow, compromising target programs without injecting additional code. These attacks alter essential data elements, such as identification or configuration data, influencing target application behaviors during runtime \cite{Chen_2005}.
\section{Overview of Malware detection}
\label{sec:malware_detection}

Malware detection involves determining whether a given program exhibits malicious intent. Figure \ref{fig:overview_malware_detection_approaches} offers an overview of contemporary solutions for malware detection, categorized into two main groups: software-based and hardware-based approaches. This division is rooted in differing observation points within the system stack and different detection methodologies. Recent advancements, as underscored by \cite{Deldar:2023aa} and \cite{He:2021aa}, increasingly rely on \gls{ml} or \gls{ai} techniques to facilitate detection.

%Historically, software-based detection has been the traditional solution. It relies on the analysis of the software and its behavior to determine if there is a malicious code behavior. A common example is the popular antivirus software, which deals against several malware classes of Section \ref{sec:malware_basics}, despite the name. Nevertheless, 

\Figure[htb]()[width=0.98\columnwidth]{Figures/Overview_malware_detection_approaches}
   {Overview of the contemporary solutions for malware detection. Elaborated by the authors based on \cite{Aslan_Samet_2020}.\label{fig:overview_malware_detection_approaches}}

This section presents an overview of software-based and hardware-based malware detection (sub-sections \ref{subsec:software_malware_detection} and \ref{subsec:hardware_malware_detection}), starting by reviewing the metrics used for evaluating the performance and efficiency of the detectors (sub-section \ref{subsec:evaluation_metrics}). %Eventually, Sub-section \ref{subsec:hardware_strengths_weaknesses} discusses the strengths and weaknesses of the hardware-based approach.

\subsection{Evaluation metrics}
\label{subsec:evaluation_metrics}

Before delving into specific malware detection techniques, readers need to consider the evaluation metrics used to assess their effectiveness. These metrics serve as quality indicators, pivotal in determining the adoption of a technique on a commercial scale. Since malware detection is a classification problem, the quality evaluation of the detectors is based on the standard classification metrics. They can be grouped as performance metrics and efficiency metrics. Performance is the degree to which a system or component accomplishes its designated functions within given constraints, i.e., correctly detects the malware. Efficiency is the degree to which a system or component performs its specified functions with minimum consumption of resources \cite{ISO_IEC_IEEE_Vocabulary_2017}. 

The primary evaluation tool for performance is the confusion matrix. This matrix is fundamental in \gls{ml} and classification tasks, summarizing results in a tabular form. It comprises four elements (see Table \ref{tab:confusion_matrix}): \glspl{TP} represent instances where the model correctly predicts malware presence, \glspl{TN} indicate correct predictions of malware absence. In contrast, \glspl{FP} and \glspl{FN} denote incorrect predictions of malware presence or absence, respectively.

\begin{table}[htb]
\centering
\caption{Confusion matrix for malware detection.}
\label{tab:confusion_matrix}
\begin{tabular}{ccc}
\toprule
 & \textbf{Predicted Negative} & \textbf{Predicted Positive} \\
\midrule\\
\rowcolor{shadecolor}
\textbf{Actual Negative} & \glspl{TN} & \glspl{FP} \\
\midrule
\textbf{Actual Positive} & \glspl{FN} & \glspl{TP} \\
\bottomrule
\end{tabular}
\end{table}

Such a matrix allows for the definition of more descriptive metrics, and Table \ref{tab:usefull_matrix} summarizes the most common ones \cite{Ye:2017aa}. The \textit{accuracy} summarizes the overall correctness of the classification model by expressing the number of correct predictions, making it one of the most widely used metrics. In scenarios where it is crucial to avoid incorrect malware predictions, \textit{precision} provides an accurate measure of the \glspl{TP} among all positive predictions. Shifting the evaluation focus to ensure no malware passes unnoticed, the \textit{\gls{TPR}} (also known as \textit{Recall} or \textit{Sensitivity}) weighs \glspl{TP} against all positive samples. It has two counterparts: (i) the \textit{\gls{FPR}}, representing the probability of a \gls{TP} being missed, and (ii) the \textit{specificity}, also known as \textit{\gls{TNR}}, indicating the probability of an actual negative (\gls{TN}) being correctly classified. Balancing Precision and Recall is often essential, and the evaluation can be accomplished using the \textit{F1-score}, which represents their harmonic mean.

Eventually, the \textit{\gls{ROC} curve} offers a visual perspective to performance evaluation. It plots the \gls{TPR} against the \gls{FPR} on a 2D graph, enabling a visual comparison of different models and capturing multiple classification aspects by inspecting the \textit{\gls{AUC}}. In simple terms, the larger the \gls{AUC}, the better the model. \gls{AUC} is closely related to the robustness of the classifier, indicating how effectively the classifier distinguishes between malware and benign applications.

\begin{table}[htb]
\centering
\caption{Most common metrics for performance evaluation of classification.}
\label{tab:usefull_matrix}
\begin{tabular}{cc}
\toprule
 \textbf{Matrix} & \textbf{Expression} \\
\midrule\\
\rowcolor{shadecolor}
\textbf{Accuracy (A)} & $A=\frac{TP+TN}{TP+TN+FP+FN}$ \\
\midrule
\textbf{Precision (P)} & $P=\frac{TP}{TP+FP}$ \\
\midrule
\rowcolor{shadecolor}
\textbf{\gls{TPR}} & $TPR=\frac{TP}{TP+FN}$ \\
\midrule
\textbf{\gls{FPR}} & $FPR=\frac{FN}{FN+TP}$ \\
\midrule
\rowcolor{shadecolor}
\textbf{Specificity (S)} & $S = \frac{TN}{TN+FP}$ \\
\midrule
\textbf{F1-Score (F1)} & $F1 = \frac{2 \times (P \times R)}{(P + R)}$ \\
\midrule
\rowcolor{shadecolor}
\textbf{ROC} & $ROC = 1 - S = \frac{FP}{FP+TN}$\\
\midrule
\textbf{AUC} & $AUC = \int_{0}^{1} R (FPR) \,dFPR$\\
\bottomrule
\end{tabular}
\end{table}

According to \cite{ISO_IEC_IEEE_Vocabulary_2017}, efficiency is related to the resources used for malware detection. Many metrics can be used to evaluate the efficiency \cite{Sze_2020}, but in the malware detection field, latency, power consumption, and hardware cost are the main interest:

    \begin{itemize}
    
    \item \textbf{Latency} is the time between collecting all features analyzed by the malware detector and concluding its detection. A low latency is vital for run-time detection of malware that acts in a short interval of time;
   
    \item \textbf{Power consumption} indicates the energy the detector consumes per unit of time. Two factors primarily impact the power consumption of the detector: the hardware that implements or where the classifier runs and the detection algorithm (those with higher computing processing tend to consume more);

    \item \textbf{Hardware cost} indicates the monetary cost of building the detection system. This is important from both an industry and a research perspective to dictate whether a system is financially viable. The main parameter to evaluate the hardware cost is the chip area (usually reported in square millimeters) in conjunction with the process technology (for example, 45 nm). Sometimes, the amount of memory is also used to evaluate the hardware cost. 
    \end{itemize}

%Such costs are often referred to as \textit{overhead}, a deviation from the average operational costs.

%In the hardware-based detection approach experiments, efficiency is often addressed when there is control over the resources. This is the case when classifiers are directly implemented in software (without machine learning tools, like Scikit-learn, Weka, and Matlab) or when the classifiers are directly implemented in hardware (like an FPGA).

\subsection{Software-based malware detection}
\label{subsec:software_malware_detection}

Software-based protection relies on specific software running in the system and analyzing the potential malware presence using different approaches. Authors in \cite{Aslan_Samet_2020} and \cite{Deldar:2023aa} proposed a very comprehensive selection of them:

\begin{itemize}

    \item \textbf{Signature-based}: the signature is a unique malware feature extracted from structural properties (e.g., code sequences) or run-time properties \cite{Idika_Mathur_2007}. The detection works as follows: features extracted from the executable generate a signature stored in a signature database. When the system is required to classify a potential threat, the detector extracts the related features and computes the signature, comparing it with signatures on the database. The potential threat is marked as malware if a hit occurs during the comparison. This approach is widely used within commercial antivirus and does not allow zero-day detection~\cite{Deldar:2023aa};

    \item \textbf{Software behavior analysis}: this approach is based on dynamic characteristics from run-time executions of programs \cite{Aslan_Samet_2020}. Dynamic characteristics might include processor and memory information, kernel usage (system calls), file system activities, and network communications. They are extracted with monitoring tools, a dataset is created, and a \gls{ml} detector distinguishes malicious and harmless applications. Software behavior analysis can detect malware variants often missed by the signature-based approach;

    %\item \textbf{Software Behavior Analysis}: This approach assesses program behaviors using monitoring tools to distinguish between malicious and harmless software. The process involves automatic analysis through sandboxing, monitoring system calls, tracking file changes, comparing registry snapshots, observing network activities, and monitoring processes. A dataset is created using these procedures to detect threats, specific features are extracted, and machine learning algorithms are employed for classification.

 
    \item \textbf{Heuristic-based detection}: this method relies on experiences and techniques, including rules and \gls{ml}. The process involves two phases: first, the detector system is trained with normal and abnormal data to identify relevant characteristics. In the second phase, known as monitoring or detection, the trained detector intelligently assesses new samples to make decisions \cite{Alzarooni_2012};
    
    \item \textbf{Deep Learning}: this falls under the umbrella of \gls{ml} algorithms, enabling computational models with multiple layers to extract more advanced features from raw input~\cite{Deldar:2023aa}. The \gls{FE} aspect combines elements from previous approaches, making it a novel method. Additionally, it proves highly effective for zero-day detection, as the \gls{FE}, employing multiple techniques, facilitates context adaptation and model updates, as highlighted in \cite{Deldar:2023aa}.
    \end{itemize}

Regarding software-based detection, it is also crucial to distinguish among the types of analysis carried out to extract the required information. According to \cite{Idika_Mathur_2007}, three ways are possible: (i) via \textit{static} analysis, using syntax or structural properties of the program/process (e.g., code sequences), (ii) via \textit{dynamic} analysis, extracting the necessary data during or after program execution, leveraging run-time information, and (iii) via \textit{hybrid} analysis, combining the two previous. Selecting one of those also affects the expected latency of the detection. While a static analysis aims to detect the threat even before executing the malicious program, the other two might require an entire execution before detection.

\subsection{Hardware-based malware detection}
\label{subsec:hardware_malware_detection}

Hardware-based detection, or \gls{HMD}, addresses the performance and computational overhead challenges of traditional malware detection techniques by utilizing low-level micro-architectural features of running applications on the target system \cite{He:2021aa}. The concept that malware can be identified through micro-architecture hardware events stems from the observation that programs exhibit phase behaviors \cite{Sherwood_2003, Isci_2006}. Program phases, which vary significantly between programs, manifest as patterns in architectural and micro-architectural events. This variation enables the discrimination of programs based on their time-behavioral hardware event patterns, facilitating the differentiation between malicious and benign applications. In 2011, Malone et al. \cite{Malone_2011} demonstrated the feasibility of detecting program code modifications based on the deviation of hardware events. In 2013, Demme et al. \cite{Demme_2013} showed the feasibility of detecting Android malware and Linux rootkits using hardware events values analyzed by a \gls{ml} classifier. 

The idea of \gls{HMD} is to perform dynamic analysis leveraging micro-architecture hardware events monitored by most modern microprocessors using \glspl{HPC} \cite{Alonso:2023aa}. Various \gls{ml} techniques can be applied to the \glspl{HPC} collected data \cite{He:2021aa}. 
One of the primary advantages of \gls{HMD} is that the analysis relies on real-time hardware collected data, enabling fast \gls{ml} classification; a few milliseconds suffice to identify threats. This translates to low latency, enabling runtime detection \cite{Sayadi_2018, Sayadi_2019, Patel_2017}. Unlike static technique analysis employed by most software-based antivirus solutions, which can be easily subverted by stealthy malware using concealment techniques, dynamic analysis via hardware-based approaches facilitates the detection of code variants and unknown malware \cite{Demme_2013}. Moreover, while software-based detection tools are software-based and susceptible to bugs or oversights in the underlying system software, hardware-based detection with secure hardware significantly reduces the possibility of malware subverting protection mechanisms \cite{Demme_2013, Tang_2014}.

On the performance front, the dynamic analysis conducted by software-based detection necessitates sophisticated computation, often at the expense of significant performance overhead. The increasing software size further complicates dynamic software analysis \cite{Demme_2013}. Conversely, in the hardware-based approach, understanding software behavior provided by micro-architectural events simplifies the analysis, reducing computational processing efforts and the cost of hardware-based detection \cite{Demme_2013, Sayadi_2022}.

%The problem of the computational overhead imposed by the software-based detection approach was probably already observed by many computer users with antivirus software installed on the operating system. When the antivirus scans, the computational overhead frequently causes the user to experience a slowdown in the applications, sometimes even making usability unfeasible.

deHowever, while the \glspl{HPC} demonstrate their ability to track behavioral deviations~\cite{Dutto:2021aa, Torres_Liu_2022, Kasap:2023aa}, their effectiveness remains open to discussion. On the positive side, \cite{Demme_2013, Tang_2014} demonstrated detector performance using this approach, reporting accuracy consistently exceeding 80\%, deeming it effective. Conversely, \cite{Zhou_2018} and \cite{Zhou_2021} conducted experiments challenging the effectiveness of hardware-based detection. They argued that reported detection capabilities often stem from tiny sample sizes and experimental setups favoring the detection mechanism unrealistically. Even if accurate, an 80\% accuracy is insufficient in scenarios with thousands of executables, risking many benign applications being misclassified as malware. They also questioned the causal link between low-level micro-architectural events and high-level software behavior. Lastly, they illustrated the hardware-based detector inability to distinguish ransomware embedded in a benign application like Notepad++. In a recent contribution, \cite{Botacin_Gregio_2022} acknowledged the absence of a perfect malware detector and argued that hardware-based detection is only effective for specific malware types. In particular, \cite{Botacin_Gregio_2022} proposes its effectiveness in identifying attacks exploiting architectural side-effects, citing examples such as RowHammer \cite{Kim_2014, Mutlu_2020} (detectable through excessive cache flushes \cite{Li_Gaudiot_2019}), \gls{ROP} attacks \cite{Prandini_Ramilli_2012} (identified by an abundance of instruction misses \cite{Wang_Backer_2016}), and DirtyCoW \cite{NIST_CVE-2016-5195} (detectable through heightened paging activity). The authors also emphasized the necessity for a maliciousness theory to enhance the understanding of malware threats and assess proposed defenses.

While \glspl{HPC} have been used in the past for safety and security, performance analysis, and optimization \cite{Weaver_McKee_2008,Carelli:2018aa,Carelli:2019aa}, it is well-known that they may suffer from inconsistency in implementation, leading to non-determinism and overcounting \cite{Weaver_Terpstra_Moore_2013}. Das et al. highlighted some of these \gls{HPC} challenges in security \cite{Das_2019}. Recent studies address \gls{HPC} discrepancies, propose methodologies, analyze resilience, and compare HPCs in various machines \cite{Barrera_2020,Kadiyala_2020,Ritter_2022,Sasongko_2023}. Given that \glspl{HPC} are hardware-based protections, detectors may be designed for specific devices with characteristics defined by the architecture and manufacturer. For instance, processors may track different numbers of events simultaneously, and discrepancies in instruction counting methods are possible \cite{Weaver_McKee_2008}. These factors underscore the need for malware detection applications to abstract software from the hardware level.

Among the inconsistencies and limitations of \glspl{HPC}, some countermeasures can be deployed to stabilize the generated data~\cite{Weaver_McKee_2008, Das_2019}. They include per-process filtering of events (applied by saving and restoring the counter values at context switches), proper interrupt handling, and minimizing the impact of non-deterministic events. In general, all works acknowledge that the evolution and improvement of the processors hardware monitoring units also tend to reduce this issue. 
Eventually, the classification task built on top of the \gls{HPC} data is commonly a \gls{ml} one. This frequently leads to techniques that increase the complexity of such algorithms, like ensemble learning and time series or even \glspl{dnn}~\cite{He:2021aa}.
%The problem of HPC consistency and accuracy of HPCs is relevant and deserves a little more discussion here. Based on systematization proposed by Das et al. \cite{Das_2019}, the sources for the inconsistency and inaccuracy are:
%
%    \begin{itemize}
%    
%    \item \textbf{Non-determinism}: refers to identical runs returning different values. Hardware interrupts and page faults were observed as the causes of non-determinism in X86 processors. For example, hardware interrupts are non-deterministic, and when they occur, they cause an extra time increment in the event, making it also non-deterministic. Another example: the first time a page of memory is accessed, it causes a page fault that triggers an interruption and consequently affects the event determinism \cite{Weaver_Terpstra_Moore_2013};
%  
%    \item \textbf{Overcounting}: means some instructions counting multiple times. There are various cases of overcounting in X86 processors: if the X87 top-of-stack pointer overflows, when the floating point unit is used for the first time due to missing terms in the instruction classifying hardware on the monitoring unit, and when an event measures microcoded events rather than retired events \cite{Weaver_Terpstra_Moore_2013};
%    
%    \item \textbf{External sources}: refers to variations in the run-time environment. For example, operating system activity, scheduling of programs in multitasking environments, memory layout, pressure, and multi-processor interactions may change between different runs \cite{Das_2019};
%    
%    \item \textbf{Variations in tool implementations}: the several tools to help obtain performance counter measurements often yield different results for the same application, even in a strictly controlled environment. The variation of sizes may result from the techniques involved in acquiring them (the point at which they start the counters), the reading technique (polling or sampling), the measurement level (thread, process, core, multiple cores), and the noise-filtering approach used \cite{Das_2019}.
%    
%    \end{itemize}



\section{Hardware-based Malware Detection Basics}
\label{sec:hw_based_detection_app}

This section focuses on \gls{HMD} techniques, outlining their key components. 

\subsection{Hardware events and performance counters}
\label{subsec:hw_events_perf_count}

Modern processors have units to monitor hardware events. In 2002, Sprunt \cite{Sprunt_2002} published a seminal paper on the basics of \glspl{PMU}. These units were developed to collect data about the performance of applications, operating systems, and processors and to help programmers tune algorithms and codes. Software dynamically adjusted to resource utilization would also benefit from the information collected. The proven advantages of utilizing the \glspl{PMU}, the continuous improvements of these units, and their constant spreading among different devices have led to their leverage for safety and security purposes~\cite{Dutto:2021aa,Kasap:2023aa,Carelli:2018aa,Carelli:2019aa}.

Nowadays, \glspl{PMU} can monitor several hardware events (see Figure \ref{fig:HW_events_counters}). Complex devices like high-end processors have hundreds of events to monitor. These events include retired instructions (branches, load, store, etc.), branch predictions, cache hits and misses, floating-point operations, hardware interrupts, elapsed core clock ticks, core frequency, and temperature. However, to minimize hardware complexity, only a few \glspl{HPC} (e.g., 2 to 8 in high-end processors) are generally available, thus limiting the number of parallel events that can be monitored. Each \gls{HPC} has an event detector and an associated counter \cite{Doyle_2017}.

\Figure[htb]()[width=0.98\columnwidth]{Figures/HW_events_counters}
   {Hardware events and performance counters in a processor. Elaborated by the author. \label{fig:HW_events_counters}}

\subsection{Hardware-based detection framework}
\label{subsec:hardware-based_detection_framework}

A generic framework can be a guiding structure to facilitate the implementation of \gls{HMD}, as illustrated in Figure \ref{fig:generic_hardware-based_framework}. The framework leverages the existing \gls{PMU} within the processor and consists of two primary components: (i) data collection and preprocessing and (ii) malware detection. This section provides a detailed overview of the implementation process.

\Figure[htb]{Figures/generic_hardware-based_framework}
   {A generic hardware-based detection framework. Elaborated by the author. \label{fig:generic_hardware-based_framework}}
   
Data collection involves \gls{FE} and \gls{FS} \cite{Abdulwahab_2022, Chandrashekar_Sahin_2014}. \gls{FE} captures and stores \glspl{HPC} in a vector space, enabling the \gls{FS} to select a subset that efficiently describes the input data while minimizing noise and irrelevant variables, ensuring optimal prediction results. \gls{FE} can occur in the time or event domain \cite{Sprunt_2002}. In the time-based domain, the application execution is periodically interrupted to record \gls{HPC} values. Conversely, the event domain triggers interruptions based on specific events or a set number of executed instructions rather than regular intervals.

In terms of strategies to perform \gls{FE}, we envision four alternatives: (i) instrument the source code with the employment of a library, like \texttt{PAPI} \cite{PAPI_1999}; (ii) develop of a proprietary kernel module or driver, as performed in \cite{Tang_2014}; (iii) use of an available utility that performs tasks mainly in the \gls{OS} kernel, like \texttt{PERF} \cite{PERF_2009}; and (iv) use of a micro-architectural simulator to model the processor as it executes the application, like \texttt{gem5} \cite{Binkert_2011} and \texttt{GVSoC} \cite{Bruschi_2021}.

During \gls{FE}, the sampling strategy is crucial. In the time-based domain, parameters such as period, frequency, or number of cycles determine when \glspl{HPC} are sampled. In the event-based domain, sampling depends on the number of event or instruction occurrences. The chosen \gls{FE} strategy influences these definitions. A proprietary kernel module or driver allows programmers to choose between time-based or event-based domains, set parameters for sampling triggering, and specify values. However, configurations are limited when libraries like \texttt{PAPI} and \texttt{PERF} are used.
Regarding sampling values, in time-based sampling, there is no fixed ideal period or frequency, varying based on the experiment and goal. Hardware-based detection experiments typically use periods in the order of milliseconds or seconds. Striking a balance between low and high sampling frequencies is essential, considering the trade-off between computational processing, data quantity, and system effects.

\gls{FS} offers multiple advantages, including addressing the Curse of Dimensionality in \gls{ml} \cite{Goodfellow_2016}, enhancing data understanding, reducing computation requirements, and improving predictor performance. Filter-based algorithms dominate the \gls{FS} in the \gls{HMD} field, ranking features based on a scoring criterion, using a threshold for variable selection. They are valued for simplicity and practical application success, focusing on the relevancy of features. Prominent methods include \gls{PCA} (used by \cite{Zhou_2018, Sayadi_2019, Gao_2021, Sayadi_2021}), Fisher Score \cite{Duda_2000} (used by \cite{Tang_2014, Torres_Liu_2022}), Pearson Correlation Coefficient \cite{Pearson_1895} (used by \cite{Patel_2017, Sayadi_2018, Sayadi_2019, Gao_2021, Sayadi_2021}) and Information Gain (Mutual Information) \cite{Peng_2005} (used by \cite{Singh_2017, Kwan_2020}). The Scikit-learn \cite {Scikit-learn} library for the Python and Weka \cite{Frank:2005aa} are tools frequently used in the \gls{HMD} field for \gls{FS}.

Since the number of events that can be potentially monitored exceeds the available \glspl{HPC}, some studies (for example, \cite{Patel_2017, Malone_2011, Khasawneh_2015}) also perform a preliminary manual \gls{FS} before data collection, thus reducing the number of software executions required to collect data. The selection is based on architectural and micro-architectural knowledge and other studies.

Eventually, in \gls{HMD}, \gls{ml} algorithms play a crucial role. Supervised and unsupervised learning techniques are employed in hardware-based malware detection. While for supervised detection, both benign and malignant samples, adequately annotated, are necessary, in unsupervised malware detection, the classifier is trained only with benign applications to perform anomaly detection \cite{Chandola_2009}. Unsupervised detection has two exciting advantages: (i) it does not require a malware dataset for training, and (ii) the classifier can detect zero-day malware~\cite{He:2021aa}. On the other side, unsupervised algorithms are complex, requiring more sophisticated analysis and resulting in complex hardware implementations.

%For the first time, Tang et al. \cite{Tang_2014} showed the feasibility of unsupervised techniques for hardware-based malware detection. In the following year, Garcia-Serrano \cite{Garcia-serrano_2015} also investigated the application of unsupervised technique, trying to simplify the approach of Tang et al.

Several traditional classification algorithm families are employed in \gls{HMD}: linear regression (LinearRegression and SimpleLinearRegression), logistic regression (Logistic and SimpleLogistic), Bayesian network (BayesNet and NaiveBayes), decision trees (J48 and REPTree), rule-based (JRIP, OneR and PART), \gls{ann} (MultiLayerPerceptron), \gls{KNN} (IBk), ensemble learning (AdaBoostM1, Bagging and RandomForest) and \gls{SVM} (SMO) \cite{Goodfellow_2016}. The algorithms in parentheses refer to specific Weka implementations, which are commonly used in the context of \gls{HMD}. Further details on these families and their implementations in Weka can be found in \cite{Frank:2005aa}.

Eventually, a crucial consideration is the trade-off between monitoring more events for better application characterization and detector performance and the impact on runtime applicability. Some studies used many events, exceeding available \glspl{HPC}, necessitating multiple application runs \cite{Demme_2013, Singh_2017, Sayadi_2017}. This trade-off is further addressed in \gls{ml} solutions discussed in Section \ref{subsec:machine_learning_techniques}.
\section{Hardware-based detection Assessment}
\label{sec:hw-based_performance_efficiency}

The following sections analyze the performance and efficiency of the state-of-the-art in \gls{HMD} and explore \gls{ml} techniques to enhance detector performance.

%Sub-section \ref{subsec:evaluation_metrics} reviewed the metrics used for evaluating the performance and efficiency of malware detectors. Sub-section \ref{subsec:hardware-based_detection_framework} discussed its implementation in the light of a generic framework. With these basements, this sub-section presents some considerations about the performance and efficiency of the hardware-based detection approach.

\subsection{Performance}
\label{subsubsec:performance}

Tables \ref{tab:performance_analysis} and \ref{tab:main_studies} provide a comprehensive overview of the literature contributions in the field, aiming to facilitate fair comparisons by presenting the best-case results in Table \ref{tab:performance_analysis}. Metrics were directly sourced from the paper's text whenever feasible, with manual extraction from reported \gls{ROC} curves employed only when necessary. The "Classification" column denotes the classification algorithm associated with the best result, with the Weka implementation serving as a reference. Conversely, Table \ref{tab:main_studies} outlines, for each contribution in Table \ref{tab:performance_analysis}, the range of considered scenarios in terms of malware, classifiers, and system characteristics. The values in Table \ref{tab:performance_analysis} underscore the efficacy of \gls{HMD} in supporting malware detection and highlight the overall high quality of the findings.
 
\begin{table*}
\begin{threeparttable}[b]
\caption{Summary of best-case performance from main studies in the hardware-based malware detection approach. \# \glspl{HPC} column refers to the number of hardware events the classifiers consider. Classification algorithm labels are based on Weka implementations used in the referenced studies. Evaluation metrics as defined in Section \ref{subsec:evaluation_metrics}: A is Accuracy, P is Precision, S is Specificity, and F1 is the F1-Score.}
\label{tab:performance_analysis}
\centering
%\resizebox{\textwidth}{!}{%
\begin{tabularx}{\linewidth}
{
>{\hsize=.1\hsize\linewidth=\hsize}c
>{\hsize=.1\hsize\linewidth=\hsize}c
>{\hsize=.4\hsize\linewidth=\hsize}X
>{\hsize=.1\hsize\linewidth=\hsize}c
>{\hsize=.2\hsize\linewidth=\hsize}X
>{\hsize=.1\hsize\linewidth=\hsize}X
>{\hsize=.1\hsize\linewidth=\hsize}X
>{\hsize=.1\hsize\linewidth=\hsize}c
>{\hsize=.1\hsize\linewidth=\hsize}c
>{\hsize=.1\hsize\linewidth=\hsize}c
>{\hsize=.1\hsize\linewidth=\hsize}c
>{\hsize=.1\hsize\linewidth=\hsize}c
>{\hsize=.1\hsize\linewidth=\hsize}c
}
%{ccXcXXXcccccc}
\toprule\
\textbf{Year} & \textbf{Ref.} & \textbf{Target} & \textbf{\# \glspl{HPC}} & \textbf{Classification} & \textbf{Learning} & \textbf{Latency} & \multicolumn{6}{c}{Evaluation Metrics} \\
 &  &  &  &  &  &  & \textbf{A} & \textbf{P} & \textbf{\gls{TPR}} & \textbf{S} & \textbf{F1} & \textbf{AUC}\\
\midrule\\
\rowcolor{shadecolor}
2013 & \cite{Demme_2013} & Android malware & 6 & Decision Tree & Offline
& NA & - & - & - & - & - & 0.83\\
\rowcolor{shadecolor}
& & Linux rootkits & 4 & KNN & Offline
& NA & - & - & 0.70\tnote{1} & - & - & -\\
\midrule
2014 & \cite{Tang_2014} & Internet Explorer exploitation & 4 & \gls{SVM} & Offline
& NA & - & - & - & - & - & 1.00\\
& & Adobe PDF Reader exploitation & 4 & \gls{SVM} & Offline
& NA & - & - & - & - & - & 1.00 \\
\midrule
\rowcolor{shadecolor}
2015 & \cite{Khasawneh_2015} & Ransomware & 5 & Logistic regression & Offline
& NA & - & - & - & - & - & 0.94 \\
\rowcolor{shadecolor}
& & Ransomware & 5  & Logistic regression (with Specialization) & Offline 
& NA & 0.87 & - & 0.81 & 0.96 & - & -\\
\midrule
2015 & \cite{Ozsoy_2015} & Viruses, worms, trojan horses, spyware, adware, and botnets & 5 & \gls{ann} & Offline
& NA & - & - & 1.00\tnote{1} & - & - & -\\
\midrule
\rowcolor{shadecolor}
2017 & \cite{Patel_2017} & Malware from various categories, sourced from VirusTotal~\cite{VirusTotal} dataset & 4 & BayesNet & Offline
& 0.624ms (SW) / 140ns (HW) & 0.85 & - & - & - & - & -\\
\midrule
2017 & \cite{Singh_2017} & Rootkits & 16 & \gls{SVM} &
& NA & 1.00 & 1.00 & 1.00 & - & 1.00 & - \\
\rowcolor{shadecolor}
\midrule
2018 & \cite{Sayadi_2018} & Malware from various categories, sourced from VirusTotal~\cite{VirusTotal} dataset & 4 & J48 (with Ensemble Learning) & Offline
& NA & 0.83 & - & - & - & - & 0.94 \\
\midrule
2018 & \cite{Zhou_2018} & Malware from various categories, sourced from VirusTotal~\cite{VirusTotal} dataset & 6 & Random Forest & Offline
& NA & - & 0.86 & 0.83 & - & 0.85 & 0.92\\
\midrule
\rowcolor{shadecolor}
2019 & \cite{Das_2019} & Malware from various categories, sourced from VX Heaven~\cite{Qiao:2016aa} dataset & 5 & J48 & Offline
& NA & - & 0.82 & 0.82 & - & 0.82 & 0.93 \\
\midrule
2019 & \cite{Sayadi_2019} & Backdoor & 4 & OneR & Offline
& NA & - & - & - & - & 0.94 & -\\
&  & Rookit & 4 & \gls{MLP} & Offline
& NA & - & - & - & - & 0.94 & -\\
&  & Virus & 4 & J48	 and ensemble learning (AdaBoostM1) & Offline
& NA & - & - & - & - & 0.96 & - \\
&  & Trojan & 4 & \gls{MLP} & Offline
& NA & - & - & - & - & 0.99 & -\\
\midrule
\rowcolor{shadecolor}
2021 & \cite{Gao_2021} & Trojan & 4 & JRIP  & Offline
& 20ns & - & 0.93 & - & - & - & -\\
\midrule
2021 & \cite{Sayadi_2021} & Stealthy rootkits & 4 & \gls{dnn} & Offline
& NA & 0.93 & 0.95 & 0.90 & - & 0.93 & 0.98 \\
\midrule
\rowcolor{shadecolor}
2022 & \cite{Torres_Liu_2022} & Data-only exploits~\cite{Hu_2015} & 50\tnote{2} & Two Classes-\gls{SVM} & Offline 
& NA & 0.99 & - & - & - & - & -\\
\rowcolor{shadecolor}
& & Data-only exploits~\cite{Hu_2015} & 6 & LZ78 & Offline
& NA & 0.84 & - & - & - & - & - \\
\midrule
2022 & \cite{Konstantinou:2022aa} & Stealthy attack on power grid & 6 & \gls{SVM} & Offline 
& 120s & 0.94 & - & - & - & - & - \\
\bottomrule\\
\end{tabularx}
\begin{tablenotes} [flushleft]
\item[1] Values extracted from ROC curves considering a false positive rate of 10\%.
\item[2] 50 is the whole set of features. This is why the authors also investigated a reduced set (in the following line).
\end{tablenotes}
%}
\end{threeparttable}
\end{table*}

Among all contributions reported in Table~\ref{tab:performance_analysis}, authors in \cite{Konstantinou:2022aa} showcase the effectiveness of \gls{HMD} on real scenarios: DARPA \gls{RADICS}, Intel \gls{TDT}, and Microsoft Defender. This is a tangible exploitation of \gls{HMD} into actual products. Still, using a single type of classifier (i.e., \gls{SVM}) leaves room for research and improvements. 

\begin{table*}
\begin{threeparttable}[b]
\caption{Reference studies including details on the full list of targets and classifications approaches tested and details on the reference systems.}
\label{tab:main_studies}
\centering
%\resizebox{\textwidth}{!}{%
\begin{tabularx}{\linewidth}{ccXXXX}
\toprule\
\textbf{Year} & \textbf{Ref.} & \textbf{Targets} & \textbf{Classification} & \textbf{Devices} & \textbf{OS}\\
\midrule\\
\rowcolor{shadecolor}
2013 & \cite{Demme_2013} & Android malware, Linux rootkits & Decision trees, \gls{ann}, \gls{KNN}, Random Forest & Arm Cortex-A9 OMAP4460, Intel Xeon X5550 & Android 4.1.1-1 (kernel 3.2), Linux kernel 2.6.32\\
\midrule
2014 & \cite{Tang_2014} & Exploitations on Internet Explorer and Adobe PDF Reader  & \gls{SVM} & Intel IvyBridge Core i7 & Windows XP\\
\midrule
\rowcolor{shadecolor}
2015 & \cite{Khasawneh_2015} & Ransomware, password stealers, trojan horses, backdoor, worms & Logistic regression	(w/o specialization) & Not specified & Windows 7\\
\midrule
2015 & \cite{Ozsoy_2015} & Viruses, worms, trojan horses, spyware, adware, and botnets & \gls{ann} &
Not specified, Altera EP4CE115 & Windows 7\\
\midrule
\rowcolor{shadecolor}
2017 & \cite{Patel_2017} & Malware from various categories, sourced from VirusTotal~\cite{VirusTotal} dataset & Logistic, SimpleLogistic, BayesNet, NaiveBayes, J48, PART, JRIP, OneR, MultiLayerPerceptron, SMO, SGD & Intel Haswell Core i5-4590, Xilinx Virtex 7 & Ubuntu 14.04 (kernel 4.4)\\
\midrule
2017 & \cite{Singh_2017} & Rootkits & \gls{SVM}, Decision tree, OC-SVM, Naive Bayes & Intel IvyBridge and Broadwell & Windows 7\\
\midrule
\rowcolor{shadecolor}
2018 & \cite{Sayadi_2018} & Malware from various categories, sourced from VirusTotal~\cite{VirusTotal} dataset & BayesNet, J48, REPTree, JRIP, OneR, MultiLayerPerceptron, SMO, SGD (w/o ensemble learning based on AdaBoostM1, Bagging) & Intel Xeon X5550, Xilinx Virtex 7 & Ubuntu 14.04 (kernel 4.4)\\
\midrule
2018 & \cite{Zhou_2018} & Malware from various categories, sourced from VirusTotal~\cite{VirusTotal} dataset & Decision trees, Naive Bayes, ANN, KNN, Random Forest (w/o ensemble learning AdaBoost) & AMD Bulldozer & Windows 7\\
\midrule
\rowcolor{shadecolor}
2019 & \cite{Das_2019} & Malware from various categories, sourced from VX Heaven~\cite{Qiao:2016aa} dataset & J48, IBk, SMO & Intel Sandy Bridge, Haswell, and Skylake & Ubuntu 16.04\\
\midrule
2019 & \cite{Sayadi_2019} & Backdoor, rootkits, viruses, trojan horses & J48, JRIP, OneR, \gls{MLP} (w/o AdaBoostM1) & Intel Xeon X5550, Xilinx Virtex 7 & Ubuntu 14.04 (kernel 4.4)\\
\midrule
\rowcolor{shadecolor}
2021 & \cite{Gao_2021} & Worms, rootkits, viruses, trojan horses & REPTree, JRIP, OneR, \gls{MLP}, SGD & Intel Xeon X5550, Xilinx Virtex 7 & Ubuntu 14.04 (kernel 4.4)\\
\midrule
2021 & \cite{Sayadi_2021} & Stealthy backdoor, rootkits, and trojan horses & \gls{dnn} &Intel Xeon X5550 &
Ubuntu 14.04 (kernel 4.4)\\
\midrule
\rowcolor{shadecolor}
2022 & \cite{Torres_Liu_2022} & Data-only exploitation & TC-SVM, OC-SVM, LZ78 & Intel Nehalem Core i7-920 & Ubuntu 16.04 (kernel 4.13)\\
\midrule
2022 & \cite{Konstantinou:2022aa} & Stealthy attack on power grid &\gls{SVM} & OpenPLC controller with Raspberry PI & 8-bus power grid in a PowerWorld simulator \\
\bottomrule\\
\end{tabularx}
%}
\end{threeparttable}
\end{table*}

As most of the current works on \gls{HMD} rely on \gls{ml} classifiers, the analysis conducted by Patel et al. \cite{Patel_2017}, summarized in Table \ref{tab:classifiers_analysis}, is particularly interesting. The authors thoroughly analyze eleven \gls{ml} classification algorithms (based on Weka\cite{Frank:2005aa} implementations). The goal was to understand the trade-offs between the design parameters offered by the algorithms. The chosen metric to evaluate performance was accuracy. The dataset used for training and testing the algorithms was extracted using the PERF tool in intervals of 10 ms executed in an Intel Haswell Core i5-4590 processor running Ubuntu 14.04 with Linux kernel 4.4. The baseline of benign application comprises the Mibench benchmark suite \cite{Guthaus_2001}, Linux system programs, browsers, text editors, and word processors. The malware came from the VirusTotal dataset. Since the \glspl{HPC} available in an Intel architecture are considerable, the accuracy of \gls{ml} algorithms covers different numbers (i.e., 32, 8, 4, 2, and 1) of hardware events. Table \ref{tab:classifiers_analysis} reports the accuracy for 4 hardware events, a reasonable quantity for concurrent monitoring in most modern processors, even in embedded scenarios~\cite{Dutto:2021aa}. JRIP (rule-based) presented the top accuracy, followed by four classifiers with the same top-two accuracy: J48 (decision-tree), OneR and PART (rule-based), and SGD. In this case, most classifiers have accuracy above 80\%. Another interesting observation is that reducing the hardware events below four significantly impacts the performance of most classifiers.

Similar findings are reported in Torres and Liu \cite{Torres_Liu_2022}. While the authors concentrated on a particular malware subclass (data-only exploits from \cite{Hu_2015}), they implemented two different experiments on different classifiers, distinguishing between using the complete set of 50 features or a smaller set of 6 features. The findings report a very high accuracy on the complete set of features (as seen on the first of the two rows dedicated to the paper in Table \ref{tab:performance_analysis}) and a degradation when only a subset is used. 


\begin{table}
\begin{threeparttable}[b]
%\renewcommand*{\arraystretch}{1}
%\scriptsize
\caption{Performance and efficiency of classifiers based on Weka implementations. Extracted from \cite{Patel_2017}.}
\label{tab:classifiers_analysis}
\centering
\begin{tabularx}{\linewidth}{
>{\hsize=.5\hsize\linewidth=\hsize}l
>{\hsize=.1\hsize\linewidth=\hsize}c
>{\hsize=.1\hsize\linewidth=\hsize}c
>{\hsize=.1\hsize\linewidth=\hsize}c
>{\hsize=.1\hsize\linewidth=\hsize}c
>{\hsize=.1\hsize\linewidth=\hsize}c
}
\toprule\
\multirow{3}{*}{Classifier} & \multirow{3}{*}{Accuracy} & SW  & \multicolumn{3}{c}{HW}                \\ %\cline{3-6} 
                            &                           & Latency  & Latency  & Power  & Area\tnote{1} \\ %\cline{3-6} 
                            &                           &    (ms)          &    (ns)                & (W) &       \\
 \midrule\\
\rowcolor{shadecolor}
BayesNet & 81.13 & 0.624 & 140 & 0.44 & 6794 \\
\midrule
J48 & 82.07 & 0.663 & 60 & 0.44 & 1801 \\
\midrule
\rowcolor{shadecolor}
JRIP & 83.96 & 0.653 & 90 & 0.44 & 1504 \\
\midrule
Logistic & 79.24 & 0.844 & 340 & 0.63 & 13041 \\
\midrule
\rowcolor{shadecolor}
\gls{MLP} & 81.13 & 0.870 & 40 & 1.03 & 36252 \\
\midrule
NaiveBayes & 78.30 & 0.802 & 10 & 1.34 & 58177 \\
\midrule
\rowcolor{shadecolor}
OneR & 82.07 & 0.653 & 220 & 0.32 & 1258 \\
\midrule
PART & 82.07 & 0.642 & 680 & 0.44 & 2131 \\
\midrule
\rowcolor{shadecolor}
SGD & 82.07 & 0.652 & 340 & 0.44 & 2556 \\
\midrule
SimpleLogistic & 79.24 & 0.648 & 3020 & 0.45 & 4721 \\
\midrule
\rowcolor{shadecolor}
SMO & 73.58 & 0.652 & 2330 & 0.44 & 2556 \\
\bottomrule\\
\end{tabularx}
\begin{tablenotes} [flushleft]
\item[1] The area is a function of total lookup tables, flip-flops, and DSP blocks.
\end{tablenotes}
\end{threeparttable}
\end{table}

\subsection{Efficiency}
\label{subsec:efficiency}

Alongside the detection quality, the \gls{HMD} aims to reduce the detectors cost in terms of resources. As the data required for the classification come from the hardware layer of the system stack, most studies evaluate FPGA-based implementations of \gls{ml} classifiers, providing measures for the power consumption and the area as the goal is to understand the trade-offs between the design parameters offered by the algorithms. When the classifier is software-based, the evaluation usually includes the latency, avoiding further monitoring of other resources. Unfortunately, as seen in Table~\ref{tab:performance_analysis}, not all works report the latency of the detection or, more in general, the costs of it. Generally, whenever the detection is performed at the software level, the latency is less than 1 ms. At the same time, more optimized hardware implementations can scale down to tens or hundreds of ns.

As reported in the previous section, the work from Patel et al. \cite{Patel_2017} covered a thorough analysis and, for this reason, is undoubtedly an excellent candidate to show the efficiency of the methodology. For hardware implementation, authors used the Xilinx Virtex 7 FPGA, implemented Weka models in C code, and used the Xilinx \gls{HLS} compiler to generate the final bitstream. The latency was evaluated both in software and hardware implementations. Authors implemented the classification algorithms in software at the \gls{OS} kernel level, which includes the time to read the \gls{HPC} and execute the classifiers. Eventually, the Intel Turbo Boost technology was disabled, as it might introduce errors in the time measurement, and the CPU governor was operating at a constant frequency of 800 MHz. The IP cores with the algorithms were synthesized in Vivado to estimate the power consumption, considering a 100 MHz clock. Power estimation contains both static power and dynamic power consumption of digital logic. %Their results are shown in the fifth column in Table \ref{tab:classifiers_analysis}.

Values in Table \ref{tab:classifiers_analysis} show the considerable difference between the latencies in software and hardware implementations. Software implementations have latencies almost in the order of milliseconds (ranging from 0.624ms to 0.870ms, best and worst cases). In contrast, hardware implementations are in the order of nanoseconds (ranging, in this case, from 10ns to 3020ns). The authors underlined that these slow profiles displayed by classifiers in the kernel space are three orders bigger than several malware executions (ranging in microseconds). Other findings related to latency are crucial to highlight. In software implementations, the latency for reading the \gls{HPC} is negligible when monitoring a single core but may increase significantly when monitoring multiple cores.
Moreover, the more \glspl{HPC} to read, the longer it takes. Concerning the classification algorithms, BayesNet (Bayesian network), PART (rule-based), and SimpleLogistic (logistic regression) showed the lowest latency values when implemented in software. Conversely, none of these three are on the list of the top three low latencies in hardware. NaiveBayes (Bayesian network), \gls{MLP} (\gls{ann}), and J48 (decision tree) are the three best hardware implementations. This paradox demonstrates the uncorrelation between the algorithms' latencies when comparing implementations at the kernel space and hardware.

%Table \ref{tab:classifiers_analysis} reports all latencies, showing how BayesNet (Bayesian network), PART (rule-based), and SimpleLogistic (logistic regression) has the lowest latency values when implemented in software. Some findings are crucial to highlight. At the kernel space, the latency for reading the \gls{HPC} is negligible when monitoring a single core but may increase significantly when monitoring multiple cores.  Moreover, while the more \glspl{HPC} to read, the longer it takes, authors underlined that the displayed latencies (milliseconds) were three orders of magnitude bigger than several malware (ranging in microseconds). The same table reports the findings of the hardware implementation. Latency is stated in several cycles, and each cycle is ten ns. NaiveBayes (Bayesian network), \gls{MLP} (\gls{ann}), and J48 (decision tree) are the three best implementations concerning the latency. Interestingly, none of the top three hardware latencies is in the list of a maximum of three software latencies.

%OneR (rule-based) presented the lowest power consumption, followed by J48 (decision tree), and JRIP and PART (rule-based) had the second lowest consumption. Eventually, the hardware area was evaluated, including the total number of lookup tables, flip-flops, and DSP blocks. According to Table \ref{tab:classifiers_analysis}, OneR, JRIP, and J48 algorithms present the top three lower hardware costs. %They made a comparison considering accuracy, latency, and power consumption. They calculated the Power Delay Product (PDP) using the latency and power consumption to account for power and latency together. PDP was later compared with accuracy in a functional analysis to find algorithms more suitable for battery-powered embedded devices. ML classifiers with lower PDP and higher accuracy are preferred. Classifiers OneR, JRIP, PART, and J48 perform better than the other classifiers. OneR, JRIP, and PART are rule-based classifiers and generate rules for the features that involve comparisons rather than computation. Therefore, they can run faster. This is also the case for the tree-based classifier, J48. Classifiers such as BayesNet and Logistic regression involve computation like probabilities and sigmoid functions, resulting in higher execution latency.
%Eventually, the hardware area was evaluated, including the total number of lookup tables, flip-flops, and DSP blocks. According to Table \ref{tab:classifiers_analysis}, OneR, JRIP, and J48 algorithms present the top three lower hardware costs. %They compared accuracy and area through the accuracy ratio over the area. Again, rule-based and tree-based classifiers performed significantly better in this analysis than highly accurate but complex Bayesnet, MultiLayerPerceptron, and logistic classifiers.

%The following sub-section (\ref{subsec:machine_learning_techniques}) continues the discussion of efficiency in the hardware-based detection approach but focuses on the efficiency overhead imposed by special machine-learning techniques to improve the performance.

\subsection{Machine learning techniques considerations}
\label{subsec:machine_learning_techniques}

Recent studies have explored various \gls{ml} methods to enhance the performance of \gls{HMD} detection approaches, especially in the last five years. These techniques aim to overcome the challenge of limited application characterization due to the concurrent capacity of \glspl{PMU} to monitor hardware events. While these methods show performance improvements, they often introduce increased complexity in classifiers, resulting in reduced efficiency, i.e., higher power consumption and increased area requirements. This section discusses ensemble learning, specialization, adaptive detection, and time series \gls{ml} techniques in \gls{HMD}.

In ensemble learning, multiple \gls{ml} algorithms are trained separately to create a classifier, combining their results to improve decision accuracy \cite{Dietterich_2000}. In \gls{HMD}, ensemble classifiers leverage the characteristics of individual algorithms to detect various types of malware while minimizing hardware events for runtime detection \cite{Khasawneh_2015, Sayadi_2018, Sayadi_2019}. However, the performance gains come with increased complexity and efficiency overhead \cite{Sayadi_2018, Gao_2021}.

Sayadi et al. \cite{Sayadi_2018} assessed the efficiency impact of ensemble learning in a malware detector on Xilinx Virtex 7 FPGA. Significant latency increases were observed when comparing a general classifier with 8 \glspl{HPC} to a Boosted classifier~\cite{Freund_Schapire_1997} with 4 \glspl{HPC}. When Boosted, the general \gls{MLP} algorithm passed from a latency of 3020ns to a latency of 5910ns. OneR increased from 10ns to 700ns, and J48  increased from 90ns to 670ns. In terms of hardware cost, the largest area increases were observed in OneR (from 2.1\% to 5.1\%), JRIP (from 2.5\% to 5.3\%), and BayesNet (from 11.5\% to 13.6\%). Conversely, J48, REPTree, and \gls{MLP} showed smaller area increases. The findings highlight substantial overhead in both latency and hardware costs. 

Another interesting \gls{ml} technique is the specialization. Instead of training a single multi-class classifier able to recognize several malware categories, different classifiers are trained, each specialized in detecting a specific malware. Authors in \cite{Khasawneh_2015} discuss and explore specialized detectors in \gls{HMD}. They used a logistic regression-based classifier for each malware class. As a result, the proposed detectors reduced the false positive rate by more than half compared to a single detector while increasing the detection rate. The authors proposed a two-level detector in the same paper, mixing a first level based on the hardware detection approach and a second level based on the software detection approach. The hardware detector was based on specialized ensemble techniques. The latency of this scheme was compared with malware detection purely based on software methods. As a result, they reported average latency reduced to 1/6.6 when the fraction of malware is low and latency reduced to 1/3.1 when 20\% of the programs are malware.

In 2019, Sayadi et al. \cite{Sayadi_2019} introduced a specialized two-stage malware detector, leveraging ensemble learning techniques, significantly improving accuracy. The first stage classifies applications into benign or malware categories (virus, rootkit, backdoor, and trojan horse). The second stage deploys an \gls{ml} classification algorithm that works best for each category of malware. Their 2021 work \cite{Sayadi_2021} continued using specialization for an accurate and run-time stealthy malware detector. They also evaluated the efficiency overhead of their specialized and ensemble learning malware detector, implemented on Xilinx Virtex 7 FPGA. A comparison of a general classifier with 4 \glspl{HPC} to a Boosted classifier with 4 \glspl{HPC} revealed notable latency increases for \gls{MLP} (from 1.020 to 5.910 ms), OneR (from 10 to 700 ns), J48 (from 30 to 670 ns), and JRIP (from 20 to 560 ns). \gls{MLP} (from 43.2\% to 61.7\%), JRIP (from 0.26\% to 5.3\%), OneR (from 0.49\% to 5.1\%) and J48 (from 0.93\% to 4.3\%) exhibited considerable increases regarding hardware cost. The findings emphasize substantial latency and hardware cost overhead.

Adaptive detection was proposed by Gao et al. \cite{Gao_2021} to optimize the performance versus cost. It targets higher or similar performance as ensemble learning, with a reduced cost. The technique leverages the concept that the \gls{ml} algorithm employed in the detector strongly correlates both the nature of the scrutinized malware and the overall performance metric. Adaptive detection involves a dynamic framework that assesses all underlying \gls{ml} algorithms in real time, opting for the optimal classifier to identify malicious patterns effectively. The implementation encompasses two primary online stages: (i) algorithm selection and (ii) malware detection. Consequently, only the most efficient ML-based detector is employed to differentiate malware from the benign class, eliminating the need to acquire results from individual base detectors and enhancing overall efficiency.

In the adaptive detector proposed by Gao et al. \cite{Gao_2021}, the algorithm selection step is done by a lightweight tree-based decision-making algorithm that accurately selects the most efficient model for inference. As a result, the scheme showed up to a 94\% detection rate while improving the cost-efficiency by more than 5X compared to existing ensemble-based malware detection methods.

%Time series are every day and everywhere. Both human activities and nature produce time series, like weather readings, financial recordings, physiological signals, and industrial observations. Earning representations and classifying time series are still attracting much attention. A time series classification problem does not treat each time point as a separate feature, nor ignore information contained in the time order of the data. In other words, the prediction would change if the feature order were scrambled in time series classification. There are several approaches for building time series classifiers: distance-based, shapelet-based, ensembled-based, dictionary-based, interval-based, and deep-learning-based. Particular focus has been given to deep neural networks, especially Fully Convolutional Neural Network (FCN) and Long Short-Term Memory (LSTM) \cite{Wang_2017,Karim_2018}. Alternative approaches have also been proposed, looking for less costly solutions. Sch\"{a}fer \cite{Schafer_2016}, and Li and Lin \cite{Li_Lin_2017} proposed a series of scalable time series classification approaches that are significantly faster than FCN and LSTM.

Eventually, time series classification is fundamental to understanding the key concept behind hardware-based malware detection. The intuition driving this technique stems from the program's phase behavior, transforming malware detection into a time series classification problem. In addressing this challenge, Sayadi et al., as outlined in \cite{Sayadi_2020} and \cite{Sayadi_2021}, introduced a time series \gls{ml} technique designed to identify stealthy malware in real time. In scenarios where attackers embed malicious files within benign programs on target hosts, executing both applications as a single thread, traditional signature-based antivirus tools falter. Embedded malware remains elusive even when the exact malware signature is in the detector database. The authors proposed a classifier based on a \glspl{FCNN} and exclusively utilized branch instructions as a low-level feature in their solution. The results demonstrated the efficacy of their technique, achieving a remarkable average detection performance of 94\% with only one \gls{HPC} feature, surpassing state-of-the-art detection methods. This enhanced performance, however, comes at a higher computational cost associated with employing a deep-learning-based solution.

While not explicitly implementing a time series technique, also \cite{Konstantinou:2022aa} reports similar results on the Intel \gls{TDT} use case. Although no specific numbers are provided, the paper compares the \gls{FFT} counting traces of the branch instructions and branch misprediction events for the WannaCry ransomware, underlining the significant difference with or without the ransomware. 

%Table \ref{tab:main_studies} summarizes the primary studies in the hardware-based malware detection approach. The columns "Devices" and "OS" (Operational Systems) refer to processors, FPGAs, and software in which the hardware-based detection framework was performed. The selection of studies was based on the following criteria: papers with the most citations, cited by these, and most recent publications.



%\begin{table*}
%\renewcommand*{\arraystretch}{1}
%\scriptsize
%\caption{Summary of main studies in the hardware-based malware detection approach. Compilation elaborated by the author.}
%\label{tab:main_studies}
%\centering
%\begin{tabularx}{\linewidth}{
%>{\hsize=.1\hsize\linewidth=\hsize}l
%>{\hsize=.1\hsize\linewidth=\hsize}l
%>{\hsize=.2\hsize\linewidth=\hsize}X
%>{\hsize=.4\hsize\linewidth=\hsize}X
%>{\hsize=.1\hsize\linewidth=\hsize}X
%>{\hsize=.1\hsize\linewidth=\hsize}X}
%\toprule
%\textbf{Year} & \textbf{Work} & \textbf{Topic} & \textbf{Focus} & \textbf{Devices} & \textbf{OS}\\
%\midrule
%\rowcolor{shadecolor}
%2002 &
%Sprunt \cite{Sprunt_2002} &
%Hardware monitoring units &
%The work is from the beginning of the hardware monitoring units. It discussed their basics and how they can be used for performance improvements &
%- &
%-\\
%
%2011 &
%Malone et al. \cite{Malone_2011} &
%Feasibility of hardware malware detection approach &
%As the first seminal work in the field, they introduced the approach and performed experiments to demonstrate the feasibility of detecting code modifications in programs based on the deviation of hardware events &
%Intel Yorkfield Q9400 &
%Ubuntu (kernel 2.6.38)\\
%
%\rowcolor{shadecolor}
%2013 &
%Demme et al. \cite{Demme_2013} &
%Feasibility of hardware malware detection approach &
%As the second seminal work in the field, they introduced the approach and performed two experiments to prove the feasibility, one detecting Android malware in ARM processors and the other detecting kernel rootkits in Intel processors &
%Arm Cortex-A9 OMAP4460, Intel Xeon X5550 &
%Android 4.1.1-1 (kernel 3.2), Linux kernel 2.6.32\\
%
%2013 &
%Wang and Karri \cite{Wang_Karri_2013} &
%Detection of kernel rootkits &
%The work proposed and tested an HPCs-based framework to run-time detection of kernel rootkits in a virtual machine &
%AMD K10 1356 &
%Redhat 7.3 (kernel 2.4.18), Fedora (kernel 2.6.11)\\
%
%\rowcolor{shadecolor}
%2014 &
%Tang et al. \cite{Tang_2014} &
%Feasibility of hardware malware detection approach and unsupervised learning &
%The work explored the feasibility of a hardware malware detection approach using unsupervised learning; the idea came from the anomaly-based detection approach &
%Intel IvyBridge Core i7 &
%Windows XP\\
%
%2015 &
%Garcia-Serrano \cite{Garcia-serrano_2015} &
%Feasibility of hardware malware detection approach and unsupervised learning &
%Based on Tang et al. \cite{Tang_2014}, the work investigated the use of unsupervised learning (anomaly-based detection approach) without the necessity of any other statistical construction like the rank-preserving power transform &
%- &
%-\\
%
%\rowcolor{shadecolor}
%2015 &
%Khasawneh et al. \cite{Khasawneh_2015} &
%ML classifiers, performance, and run-time detection &
%The work explored specialized detectors and ensemble learning techniques to improve the performance of hardware malware detectors &
%Not specified &
%Windows 7\\
%
%2015 &
%Ozsoy et al. \cite{Ozsoy_2015} &
%Feature selection, ML classifiers, performance, and run-time detection &
%The work was the first to address run-time in the hardware detection approach. They discussed malware detectors based on mixed approaches (hardware and software). They also implemented and tested a hardware-based malware detector in FPGA &
%Not specified, Altera EP4CE115 &
%Windows 7\\
%
%\rowcolor{shadecolor}
%2017 &
%Patel et al. \cite{Patel_2017} &
%ML classifiers, performance, efficiency, and run-time detection &
%The work analyzed various robust ML algorithms to help guide architectural decisions for better performance and efficiency. They analyzed the classifiers implemented in the software OS kernel and FPGA &
%Intel Haswell Core i5-4590, Xilinx Virtex 7 &
%Ubuntu 14.04 (kernel 4.4)\\
%
%2017 &
%Singh et al. \cite{Singh_2017} &
%Detection of kernel rootkits &
%The work discussed the difficulty in detecting rootkits because they execute in the context of other processes that access kernel information. They experiment with the detection of rootkits using the hardware detection approach &
%Intel IvyBridge and Broadwell &
%Windows 7\\
%
%\rowcolor{shadecolor}
%2018 &
%Sayadi et al. \cite{Sayadi_2018} &
%ML classifiers, performance, and run-time detection &
%The work proposed and tested ensemble learning techniques for the classifier to solve the small number of HPCs in run-time detection. They also developed and analyzed an FPGA implementation of ensemble classifiers &
%Intel Xeon X5550, Xilinx Virtex 7 &
%Ubuntu 14.04 (kernel 4.4)\\
%
%2018 &
%Zhou et al. \cite{Zhou_2018} &
%Feasibility of hardware malware detection approach &
%The work argued that detecting malware using HPCs is impossible. They reproduced preview works with good detailing and verified low detection rates, supporting their argument &
%AMD Bulldozer &
%Windows 7\\
%
%\rowcolor{shadecolor}
%2019 &
%Das et al. \cite{Das_2019} &
%Feature extraction and HPCs accuracy &
%The work presented the reasons for the inaccurate measurement of HPC and some suggestions/techniques to improve the process. They performed experiments related to the non-determinism of HPCs and the proposed techniques &
%Intel Sandy Bridge, Haswell, and Skylake &
%Ubuntu 16.04\\
%
%2019 &
%Sayadi et al. \cite{Sayadi_2019} &
%ML classifiers, performance, and run-time detection &
%The work explored specialized detectors and ensemble learning techniques for the problem of few performance counters in run-time. They implemented the proposed detector in FPGA &
%Intel Xeon X5550, Xilinx Virtex 7 &
%Ubuntu 14.04 (kernel 4.4)\\
%
%\rowcolor{shadecolor}
%2020 &
%Sayadi et al. \cite{Sayadi_2020} &
%ML classifiers, run-time detection, and embedded malware &
%The work explored a specialized time series machine learning approach to detect stealthy malware (embedded malware) at run-time accurately &
%Intel Xeon X5550 &
%Ubuntu 14.04 (kernel 4.4)\\
%
%2021 &
%Gao et al. \cite{Gao_2021} &
%ML classifiers, performance, efficiency, and run-time detection &
%The work is argued to be the first that addressed the challenge of performance vs. efficiency. They explored an adaptive detector without using ensemble learning techniques composed of a model selector and a set of ML classifiers. They also implemented and tested the scheme in an FPGA &
%Intel Xeon X5550, Xilinx Virtex 7 &
%Ubuntu 14.04 (kernel 4.4)\\
%
%\rowcolor{shadecolor}
%2021 &
%Sayadi et al. \cite{Sayadi_2021} &
%ML classifiers, run-time detection, and embedded malware &
%The work explored a specialized time series machine learning approach to accurately detect stealthy malware (embedded malware) at run-time. They deeply discussed embedded malware detection and time series classification &
%Intel Xeon X5550 &
%Ubuntu 14.04 (kernel 4.4)\\
%
%2022 &
%Sayadi et al. \cite{Sayadi_2022} &
%Overview of hardware malware detection approach and side-channel attacks &
%The work analyzed recent trends in AI-enabled hardware-based security and their challenges and opportunities, with a special section about side-channel attacks. They performed experiments with machine learning side-channel attacks in an AES-128 implemented in an MCU and an FPGA &
%STMicroelectronics STM32F415, Xilinx XC7A100T &
%-\\
%
%\rowcolor{shadecolor}
%2022 &
%Torres and Liu \cite{Torres_Liu_2022} &
%Detection of data-only attacks &
%The work discussed the problem of data-only attack detection and performed an experiment showing the high detection accuracy using the hardware-based detection approach. It also discusses the impact of feature selection algorithms &
%Intel Nehalem Core i7-920 &
%Ubuntu 16.04 (kernel 4.13)\\
%
%2022 &
%Botacin and Gr{\'e}gio \cite{Botacin_Gregio_2022} &
%Feasibility of hardware malware detection approach &
%The work argued that the hardware detection approach is only effective against some malware categories, specifically with attacks that exploit architectural side-effects. They claim the need for a theory of maliciousness to better state malware threats and evaluate proposed defenses &
%Not specified &
%Not specified\\
%\bottomrule
%\end{tabularx}
%\end{table*}
\section{Conclusion}\label{sec:conclusion}
In this work, we focus on addressing the fundamental challenge of OOD detection tasks, which is how to fully understand the semantic discrepancy between the ID/OOD samples. We reveal that the key to success in the realistic SCOOD task is to allocate as many ID samples in the unlabeled set correctly as possible. To this end, we propose a novel uncertainty-aware optimal transport scheme that introduces class-specific energy scores as guidance for effective label assignment. Experimental results show that our method achieves better performance than previous state-of-the-art methods on SCOOD benchmarks.

\textbf{Limitations.} In addition to temperature scaling, other techniques such as feature clipping applied in ReAct~\cite{sun2021react} also enhance the performance of energy score, so how to obtain an OOD score that best fits the SCOOD task can be further explored. Moreover, a setting highly related to SCOOD has been proposed in \cite{katz2022training} and formulated as a constrained optimization problem. We will also theoretically analyze these practical OOD settings in our feature work.

% \section*{Acknowledgments}
\textbf{Acknowledgments.} 
This work is supported by National Key R\&D Program of China under Grant 2020AAA0105701, National Natural Science Foundation of China (NSFC) under Grants 61872327, Major Special Science and Technology Project of Anhui, National Natural Science Foundation of China (62033012) and Ant Group through Ant Research Intern Program.

%
% %\newpage
\section{Alternative Definitions}\label{sec:other-definitions-short}
In this section, we discuss other potential definitions of Leximin approximation that might be considered intuitive.
\eden{removed ack. for anonymous submission}
% \footnote{We thank Sylvain Bouveret for suggesting definitions \ref{altDef:5} and \ref{altDef:6}.}.
For each alternative, we provide an example that illustrates why we believe it is inappropriate and a conclusion based on that example.
It should be noted that in order to avoid confusion, the error parameter $\gamma \in (0,1)$ is used in the alternative definitions (instead of $\beta$), to emphasize that these are only alternatives we do not use.


\begin{potentialDefinition}\label{altDef:2}
    A solution $x$ is a $(1-\gamma)$-approximately optimal if given a Leximin-optimal solution, $x^*$, there exists an integer $k \in [n]$ such that: 
    \begin{align*}
    \forall j < k: & \valBy{j}{x} \geq (1-\gamma) \cdot \valBy{j}{x^*}\\
    & \valBy{k}{x} > \valBy{k}{x^*}
    \end{align*}
\end{potentialDefinition}

\paragraph{Bad example def. \ref{altDef:2}:} Consider the following example with three objectives:
\begin{align*}
    \max \quad &\{f_i(x) = x_i \mid \forall 1 \leq i \leq 3 \} \\ \tag{E1}\label{eq:alt-def-eaxmple-1}
    s.t. \quad  & 99 x_1 + x_2 \leq 100\\
    &  x_3 \leq 100\\
    & x \in \mathbb{R}^3_{+}
\end{align*}
The Leximin optimal solution $x^*$ is $(1,1, 100)$ and therefore, by taking $k$ to be $2$, we get that any solution that its minimum objective value is at least $(1-\gamma)$ and its second-smallest objective value is more than $1$ is considered $(1-\gamma)$-approximately optimal Leximin solution.
For instance, consider $\gamma = 0.1$, the solution $(0.9, 1.1, 1.1)$ should be considered a $0.9$-approximately optimal according to this definition.
However, it is easy to see that this solution is quite bad for $f_2$ who can achieve $10.9$ (higher by a factor $> 9$) and very bad for $f_3$ who can achieve $100$ (higher by a factor $>90$).
And so, it seem reasonable to require that a good definition will consider as many objectives as possible.
% \erel{What objective values exactly? Do you mean: as many objectives as possible?}
% \eden{yes. To myself: this comment might be relevant to other places..}

\paragraph{Conclusion def. \ref{altDef:2}:} An appropriate definition should take into account as many objectives as possible.

\begin{potentialDefinition}\label{altDef:1}
    A solution $x$ is a $(1-\gamma)$-approximately optimal if for a Leximin-optimal solution, $x^*$, and for each $j = 1, \dots, n$ the following holds: 
    \begin{align*}
        \valBy{j}{x} \geq (1-\gamma) \cdot \valBy{j}{x^*} 
    \end{align*}
\end{potentialDefinition}

\paragraph{Bad example def. \ref{altDef:1}:} 
% An error in the first objective value might cause the other values to increase significantly.
Consider example \eqref{eq:alt-def-eaxmple-1} again.
Here, as the optimal solution is $(1,1, 100)$, any solution that yields at least $(1-\gamma,1-\gamma, (1-\gamma)\cdot 100)$.
However, considering $\gamma = 0.1$, $f_2$ can again achieve $9.1$ which is higher by a factor $> 100$ than the value it got $0.1$.

\paragraph{Conclusion def. \ref{altDef:1}:} An appropriate definition should consider the fact that an error in one objective might change the optimal value of other objectives.
As a consequence, another conclusion is that an appropriate definition should not consider the optimal solution at all.



\begin{potentialDefinition}\label{altDef:3}
    A solution $x$ is a $(1-\gamma)$-approximately optimal 
    if it satisfies the following requirements:
    \begin{enumerate}
        \item The objective-function with the smallest objective value achieves at least its maximum value times $(1-\gamma)$:
        \begin{align*}
            \valBy{1}{x} \geq (1-\gamma) \cdot \valBy{1}{x^*} 
        \end{align*}
        
        \item Given all the solutions that satisfies the first condition, let $m_2$ be the highest second-smallest objective value.
        The objective-function with the second-smallest objective value achieves at least the $m_2$ times $(1-\gamma)$.
        
        \item Given all the solutions that satisfies the former conditions, let $m_3$ be the highest third-smallest objective value.
        The objective-function with the third-smallest objective value achieves at least the $m_3$ times $(1-\gamma)$.
        
        \item and so on.
    \end{enumerate}
\end{potentialDefinition}

\paragraph{Bad example def. \ref{altDef:3}:}
Consider the following example with only two objectives:
\begin{align*}
    \max \quad &\{f_1(x) = x_1, f_2(x)=x_2\} \\
    s.t. \quad  & 99 x_1 + x_2 \leq 100\\
    & x \in \mathbb{R}^2_{+}
\end{align*}
The Leximin-optimal solution is $(1,1)$. Consider $\gamma = 0.1$, according to part (1) of this definition, all solutions in which the smallest objective value is at least $(1-\gamma)=0.9$ should be considered in order to determine $m_2$.
So, in this case, $m_2$ is determined to be $100 - 0.9 \cdot 99 = 10.9$.
Then, according to part (2), in order to be considered a $0.9$-approximately optimal, the second value must be at least $0.9 \cdot 10.9 = 9.81$.
But, even the exact Leximin optimal solution does not satisfy this requirement, so this cannot be considered an approximation to Leximin optimal.

In general, this definition has the disadvantage of favoring solutions that give the lowest bounds to the objective functions considered in the earlier steps,  since this may enable to increase the values of the higher objectives.
According to the Leximin nature, the most important thing is to make the worst-off player as happy as possible (and then the second worst-off and so on), therefore, we emphasize the importance of this characteristic also in the definition of the approximated version.

\paragraph{Conclusion def. \ref{altDef:3}:} An appropriate definition should also capture the Leximin optimal solutions, and maintain the Leximin nature whenever possible.

% \eden{I think this definition is actually equivalent to our current... need to think about it again}
% \begin{potentialDefinition}\label{altDef:4}
%     A solution $x$ is a $\gamma$-approximately-optimal Leximin solution if it can be viewed as the result of this process:
%     \begin{enumerate}
%         \item Choose a solution in which the objective-function with the smallest objective value achieves at least the maximum value minus $\gamma$:
%         \begin{align*}
%             \valBy{1}{x} \geq \valBy{1}{x^*} - \gamma
%         \end{align*}
%         Let $z_1$ be the value it achieves (i.e., $\valBy{1}{x})$.
        
%         \item Consider all the solutions in which the objective-function with the smallest objective value achieves at least $z_1$ and let $m_2$ be the highest second-smallest objective value.
%         Then, choose a solution in which the objective-function with the second-smallest objective value achieves at least the $m_2$ minus $\gamma$.
%         Let $z_2$ be the value it achieves.
%         \item Consider all the solutions in which the objective-function, the smallest objective value achieves at least $z_1$ and the second-smallest objective value achieves at least $z_2$, and let $m_3$ be the highest third-smallest objective value.
%         Then, choose a solution in which the objective-function with the third-smallest objective value achieves at least the $m_3$ minus $\gamma$.
        
%         \item and so on...
%     \end{enumerate}
% \end{potentialDefinition}

% \paragraph{Bad example def. \ref{altDef:3}:} Although in this definition, the Leximin optimal solution is also approximately-optimal as we wanted, another issue arises.

% \begin{itemize}
%     \item \textbf{Bad example:} two solutions that meet this definition, but one of them is strictly better (by more than $\gamma$) than the other from some point.
%     \item \textbf{Conclusion:} an appropriate (good?) definition should determine between two solutions if possible. 
% \end{itemize}

% -----------------------------------
% \subsection{others}
% (From the correspondence of Erel with Lemaitre and Bouveret)

%----------------------------------
% need to think about a corresponding def for mult...
\begin{potentialDefinition}\label{altDef:5}
    A solution $x$ is a $(1-\gamma)$-approximately optimal if for a Leximin-optimal solution, $x^*$, and for each $j = 1, \dots, n$: 
    % in the addive version was $$$
    \begin{align*}
        % |\valBy{j}{x} - \valBy{j}{x^*}| \leq \gamma\\
         \max\{\valBy{j}{y},\valBy{j}{x}\}  \leq \frac{1}{1-\gamma} \cdot \min\{\valBy{j}{y},\valBy{j}{x}\}
    \end{align*}
\end{potentialDefinition}

\paragraph{Bad example and conclusion def. \ref{altDef:5}:}
This definition is close to definition \ref{altDef:1} but weaker, still the same example and conclusion apply.

\begin{potentialDefinition}\label{altDef:6}
    A solution $x$ is a $(1-\gamma)$-approximately optimal if given a Leximin-optimal solution, $x^*$, there exists an integer $k \in [n]$ such that: 
    \begin{align*}
    \forall j < k: & \valBy{j}{x} = \valBy{j}{x^*}\\
    & \valBy{k}{x} > (1-\gamma) \cdot \valBy{k}{x^*}
    \end{align*}
\end{potentialDefinition}
% \eden{I'm not sure it is well defined, since by decreasing $\gamma$ (for example) the second value might become smaller than the first.}

\paragraph{Bad example def. \ref{altDef:6}:} As in the case of definition \ref{altDef:2}, by taking a small $k$, we cannot distinguish between two solutions that satisfy this definition, but one of them should be definitely preferred.
Consider again the following example with three objectives, where:
\begin{align*}
    \max \quad &\{f_i(x) = x_i \mid \forall 1 \leq i \leq 3 \} \\
    s.t. \quad  & 9 x_1 + x_2 \leq 10\\
    &  x_3 \leq 100\\
    & x \in \mathbb{R}^3_{+}
\end{align*}
The Leximin optimal solution $x^*$ is $(1,1, 100)$ and therefore, by taking $k$ to be $2$, we get that any solution that its minimum value is $1$ and its second-smallest objective value is more than $(1-\gamma)$ is considered $(1-\gamma)$-approximately optimal.
As an example, the solution $(1, 1, 1)$ is considered $(1-\gamma)$-approximately-optimal Leximin solution (as $(1-\gamma) < 1$).
But it is easy to see that this solution is quite bad for $f_3$ (who can achieve $100$).

\paragraph{Conclusion def. \ref{altDef:6}:} Same as for def. \ref{altDef:2}, an appropriate definition should take into account as many objectives as possible.

% \begin{potentialDefinition}
%     OWA.
% \end{potentialDefinition}

%--------------------------

\begin{potentialDefinition}\label{altDef:7}
    A solution $x$ is a $(1-\gamma)$-approximately optimal if there is no other solution $y$ that is $(1-\gamma)$-Leximin preferred over it, where this relation is defined as follows: $y$ is preferred over $x$ if  there exists an integer $k \in [n]$ such that:
    \begin{align*}
        \forall j < k \colon \quad &   \max\{\valBy{j}{y},\valBy{j}{x}\}  \leq \frac{1}{(1-\gamma)} \cdot \min\{\valBy{j}{y},\valBy{j}{x}\}\\
        & \valBy{k}{y} > \frac{1}{(1-\gamma)} \cdot 
\valBy{k}{x}
    \end{align*}
     [This relation is related to a one suggested in \cite{kalai_lexicographic_2012}, it is described in more detail in the Related work Section]. 
\end{potentialDefinition}

\paragraph{Bad example and conclusion def. \ref{altDef:7}:} As with definition \ref{altDef:3}, here also, the Leximin optimal solution is not optimal according to this relation and it might favor solutions with lower smallest objective values. 
Consider again the following example:
\begin{align*}
    \max \quad &\{f_1(x) = x_1, f_2(x)=x_2\} \\
    s.t. \quad  & 99 x_1 + x_2 \leq 100\\
    & x \in \mathbb{R}^2_{+}
\end{align*}
Assume that $\gamma = 0.1$, the Leximin-optimal solution is $(1,1)$, but the solution $(0.9,10.9)$ is preferred over it according to this relation (since for $k=2$ we get that $\max\{0.9,1\} \leq \frac{1}{0.9}\cdot\min\{0.9,1\}$ and $10.9 > \frac{1}{0.9} \cdot 1$) and therefore, it is not approximately-optimal.



\section{The Approximate Leximin Order}\label{sec:approx-order-is-strict-partial}

Unlike the leximin order, $\leximinPreferred$, which is a strict \textbf{total} order, the approximate leximin order, $\alphaBetaPreferred$ for $\DEFmultApprox\in (0,1]$ and $\DEFadditiveApprox \geq 0$ is a strict \textbf{partial} order.
The difference is that in partial orders, not all vectors are comparable.
Consider for example the sorted vectors $(1,2)$ and $(1, 3)$. 
According to the leximin order, $(1,3)$ is clearly preferred (as $3>2$), but according to many approximate leximin orders neither one is preferred over the other, for example according to the orders $\alphaBetaPreferredParams{0.6}{0}$,$ \alphaBetaPreferredParams{1}{1}$ or $\alphaBetaPreferredParams{0.8}{0.5}$.
% (irreflexive, asymmetric and transitive).

An order is a strict partial order if it is irreflexive, transitive and asymmetric.
Lemma \ref{lemma:order-is-irreflexive} proves that the order is irreflexive, Lemma \ref{lemma:order-is-transitive} proves it is transitive, and Lemma \ref{lemma:order-is-asymmetric} proves that it is asymmetric.
% \erel{It would be good to show an example why this is not a total order.}

% need to prove irreflexive, asymmetric (we have already proved that it is transitive).

% ***[I thought it would be better to prove it on vectors (rather than "solutions") to make it as general as possible]\\

Let $\DEFmultApprox\in (0,1]$ and $\DEFadditiveApprox \geq 0$. 

\begin{lemma}\label{lemma:order-is-irreflexive}
    The approximate leximin order $\alphaBetaPreferred$ is irreflexive.
\end{lemma}

\begin{proof}
    % \eden{I used $x$ only to remind the reader what irreflexive is, maybe it should simply be in the lemma description}
    Let $x$ be a solution. We will show that $x \nAlphaBetaPreferred x$.
    As the definition requires that one component be \emph{strictly greater} than the other, it is trivial.
\end{proof}

\begin{lemma}\label{lemma:order-is-transitive}
    The approximate leximin order $\alphaBetaPreferred$ is transitive.
\end{lemma}

\begin{proof}
    Let $x,y$ and $z$ be solutions such that $x \alphaBetaPreferred y$ and $y \alphaBetaPreferred z$.
    We will prove that $x \alphaBetaPreferred z$.

    
    Since $x \alphaBetaPreferred y$, there exists an integer $ k_1 \in [n]$ such that:
    \begin{align*}
        \forall j<k_1 \colon &  \valBy{j}{x} \geq \valBy{j}{y}\\
            & \valBy{k_1}{x} > \frac{1}{\DEFmultApprox} \left( \valBy{k_1}{y} + \DEFadditiveApprox \right)
    \end{align*}
    And since $y \alphaBetaPreferred z$, there exists an integer $k_2 \in [n]$ such that:
    \begin{align*}
        \forall j<k_2 \colon &  \valBy{j}{y} \geq \valBy{j}{z}\\
            & \valBy{k_2}{y} > \frac{1}{\DEFmultApprox} \left( \valBy{k_2}{z} + \DEFadditiveApprox \right) 
    \end{align*}

    As $\DEFmultApprox \in (0,1]$ and $\DEFadditiveApprox \geq 0$, it follows that:
    \begin{align}\label{eq:trans-k-s}
        \valBy{k_1}{x} > \valBy{k_1}{y}, \Hquad \valBy{k_2}{y} >  \valBy{k_2}{z}
    \end{align}

    % Accordingly, if $k_1=k_2$, then this integer, denoted by $k$, allows us to conclude that $x \alphaBetaPreferred z$. 
    % By the definitions of $k_1$ and $k_2$, for any $j<k_1=k_2$ the required holds as $\valBy{j}{x} \geq \valBy{j}{y} \geq \valBy{j}{z}$.
    % In addition, $\valBy{k_1}{x}> \valBy{k_1}{y}$ by equation \ref{eq:trans-k-s}
    % $ > \frac{1}{\DEFmultApprox} \left( \valBy{k_1}{y} + \DEFadditiveApprox \right)$ and nd  and 
    
    Let $k = \min\{k_1,k_2\}$.
    
    If $k = k_1$, by the definition of $k_1$, $\valBy{k}{x} > \frac{1}{\DEFmultApprox} \left( \valBy{k}{y} + \DEFadditiveApprox \right)$.
    However, $\valBy{k}{y} \geq \valBy{k}{z}$, by definition if $k<k_2$ and by equation \ref{eq:trans-k-s} if $k=k_2$. \ref{eq:transitive-k}
    Therefore, $\valBy{k}{x} > \frac{1}{\DEFmultApprox} \left( \valBy{k}{z} + \DEFadditiveApprox \right)$.
    
    Otherwise, if $k=k_2$, by the definition of $k_2$, $\valBy{k}{y} > \frac{1}{\DEFmultApprox} \left( \valBy{k}{z} + \DEFadditiveApprox \right)$. But, $\valBy{k}{x} \geq \valBy{k}{y}$, by definition if $k<k_1$ and by equation \ref{eq:trans-k-s} if $k=k_1$. Again, we can conclude that $\valBy{k}{x} > \frac{1}{\DEFmultApprox} \left( \valBy{k}{z} + \DEFadditiveApprox \right)$.

     In addition, for each $j<k$, since $j< k_1$ and $j < k_2$, by definition the following holds:
    \begin{align}\label{eq:transitive-k}
        \valBy{j}{x} \geq \valBy{j}{y} \geq \valBy{j}{z}
    \end{align}
    So, $k$ is an integer that satisfy all the requirements, and so, $x \alphaBetaPreferred z$.
    \end{proof}
    

    
    \begin{lemma}\label{lemma:order-is-asymmetric}
        The approximate leximin order $\alphaBetaPreferred$ is asymmetric.
    \end{lemma}
    
    \begin{proof}
        Let $x$ and $y$ be solutions such that $x \alphaBetaPreferred y$. We will show that $y \nAlphaBetaPreferred x$. 
        Assume by contradiction that $y \alphaBetaPreferred x$. 
        From Lemma \ref{lemma:order-is-transitive}, this relation is transitive. Therefore, since $x \alphaBetaPreferred y$ and $y \alphaBetaPreferred x$, also $x \alphaBetaPreferred x$.
        But, from Lemma \ref{lemma:order-is-irreflexive}, this relation is irreflexive --- a contradiction.
    \end{proof}
\section{Proof of Theorem \ref{th:main}}\label{sec:algo-sec-proofs}
\eden{should probably change the title}

This section is dedicated to proving Theorem \ref{th:main}.
To this end, we use another equivalent representation of \eqref{eq:sums-OP}, which was also introduced by \cite{Ogryczak_2006} 
(we provide the proof of equivalence in Appendix \ref{sec:equivalent-proofs}). 
\erel{Can't we just use it directly instead of P2?}
% (here also, the variables are $\ztVar{x}$ and $x$, and $z_1, \ldots z_{t-1}$ are constants)
\begin{align*}
    \max \quad &z_t \tag{P2-compact}\label{eq:compact-OP} \;\;
        s.t. &\quad  & (1) \quad x \in S\\
                    &&& (\Tilde{2}) \quad \sum_{i=1}^{\ell} \valBy{i}{x} \geq \sum_{i=1}^{\ell}  z_i && \ell = 1,\ldots, t-1 \nonumber\\
                    &&& (\Tilde{3}) \quad \sum_{i=1}^{t} \valBy{i}{x} \geq \sum_{i=1}^{t}  z_i
\end{align*}
In this problem, constraints $(\hat{2})$ and $(\hat{3})$ are replaced by  $(\Tilde{2})$ and $(\Tilde{3})$, respectively.  
The difference is that
$(\hat{2})$ gives, for each $\ell$, a lower bound on the sum for \emph{any} set of $\ell$ objective functions; whereas $(\Tilde{2})$ only considers the sum of the $\ell$ \emph{smallest} such values.  
% However, since the constraints set the same lower bound on this sum, the constraints are equivalent.  
Similarly for $(\hat{3})$ and $(\Tilde{3})$. 
Since  the problems are equivalent, a solver, either exact or approximate, for one can be used as a solver, with the same level of accuracy, for the other (Lemma \ref{lemma:solver-equivalent-prob}). 
Therefore, as \eqref{eq:compact-OP} is equivalent to \eqref{eq:sums-OP}, which, in turn, is equivalent to \eqref{eq:vsums-OP}, in proving the theorem we may assume that \textsf{OP} is an approximation procedure for \eqref{eq:compact-OP}.  
This will simplify the proofs. \eden{I added the line from the comment back, isn't it important to explain why we need this representation?}

% \erel{*** I do not understand. We say that P1 and P2 are equivalent with an exact solver, but not with an approximate solver. Here, we claim that P3 and P2-compact are equivalent, but this is true only with an exact solver. Don't we have to prove that they are equivalent also with an approximate solver? ***}

We denote $\retSol := x_n$ = the solution $x$ attained at the last iteration ($t=n$) of the algorithm. 

Following are some observations regarding the set of feasible solutions in each iteration, their objective values, and the solution $\retSol$ that will be useful later on.

% For any constants $z_1,\ldots, z_{t-1}$,
% any vector $x \in S$ that satisfies constraint $(\Tilde{2})$ of \eqref{eq:compact-OP} 
% is feasible to this problem.
% This is because any solution $x \in S$ can satisfy constraint $(\Tilde{3})$ with a small enough assignment to the variable $z_t$. \eden{I'm not sure how to explain it....}
\begin{observation}\label{obs:feasi-and-constraint2}
For any constants $z_1,\ldots, z_{t-1}$,
any vector $x \in S$ that satisfies constraint $(\Tilde{2})$ of \eqref{eq:compact-OP} 
can be a part of a feasible solution $(x,z_t)$ for any $z_t \leq \sum_{i=1}^{t} \valBy{i}{x} - \sum_{i=1}^{t-1} z_i$.
\end{observation}

Since $\retSol$ is a feasible solution of \eqref{eq:compact-OP} in iteration $n$, and as each
iteration only adds new constraints to $(\Tilde{2})$, it follows that $\retSol$ is also a feasible solution of \eqref{eq:compact-OP} in any iteration $1 \leq t\leq n$. 
\begin{observation}\label{obs:retSol-solves-any-t}
$\retSol$ is a feasible solution of \eqref{eq:compact-OP} in any iteration $1 \leq t\leq n$.
\end{observation}

Now, consider the problem \eqref{eq:compact-OP} that was solved in iteration $t$.
Here, $z_t$ is a \emph{variable} and $z_1, \ldots z_{t-1}$ are constants.
The objective of this problem is $\max z_t$, and the only constraint that includes the variable $z_t$ is  $(\Tilde{3})$.
Therefore, rearranging it to $\sum_{i=1}^{t} \valBy{i}{x} - \sum_{i=1}^{t-1}  z_i\geq z_t$, allows us to conclude that the objective value is determined by the left side of this inequality (as $z_t$ is maximized when the inequality turns to equality).
\begin{observation}\label{obs:obj-value}
The objective value obtained by a feasible solution $x$ to the problem \eqref{eq:compact-OP} that was solved in iteration $t$ is $\sum_{i=1}^{t} \valBy{i}{x} - \sum_{i=1}^{t-1}  z_i$.
\end{observation}

Lastly, as the value obtained as a $(\multApprox, \additiveApprox)$-approximation for this problem is the \emph{constant} $z_t$, the optimal value is at most $\frac{1}{\multApprox} (z_t+\additiveError)$. 
Consequently, the objective value of any feasible solution is at most this value.
Since $\retSol$ is feasible for any iteration $t$ (Observation \ref{obs:retSol-solves-any-t}) and since its objective is $\sum_{i=1}^t \valBy{i}{\retSol} - \sum_{i=1}^{t-1} z_i$ (Observation \ref{obs:obj-value}), we can conclude:

\begin{observation}\label{obs:obj-xt-to-zt}
    The objective value obtained by $\retSol$ to the problem \eqref{eq:compact-OP} that was solved in iteration $t$ is at most $\frac{1}{\multApprox} (z_t+\additiveError)$. That is:
    \begin{align*}
        \sum_{i=1}^t \valBy{i}{\retSol} - \sum_{i=1}^{t-1} z_i \leq \frac{1}{\multApprox} \left(z_t+\additiveError \right).
    \end{align*}
\end{observation}

% This conclusion also implies that for any $1 \leq t \leq n$, the solution $(x_t, z_t)$ that that was outputted for \eqref{eq:compact-OP} in iteration $t$, satisfies constraint $(\Tilde{3})$ as equality. That is:
% \begin{observation}\label{obs:equality-xt-zt}
% For any $1 \leq t \leq n$,  $\sum_{i=1}^{t} \valBy{i}{x_t} = \sum_{i=1}^{t}  z_i$.
% \end{observation}



%%%
% OVERALL EXPLANATION 
We start with Lemmas \ref{lemma:beta-vk}-\ref{lemma:fk-to-all}, which establish a relationship between the $k$-th least objective value obtained by $\retSol$ 
% ($\valBy{k}{\retSol}$) 
and the difference between the sum of the $(k-1)$ least objective values obtained by $\retSol$ and the sum of the $(k-1)$ first $z_i$ values.
% ($\sum_{i=1}^{k-1}\valBy{k}{\retSol} - \sum_{i=1}^{k-1}z_i$). 
Theorem \ref{th:main} then uses this relation to prove that the existence of another solution that would be $\left(\frac{\multApprox^2}{1-\multApprox + \multApprox^2}, \frac{\multApprox(2-\multApprox)\additiveApprox}{1-\multApprox +\multApprox^2}\right)$-preferred over $\retSol$ would lead to a contradiction.

For clarity, throughout the proofs, we denote the multiplicative error factor by $\multError = 1-\multApprox$.

% LEMMAS.
% BLAH BLAH.

\begin{lemma}\label{lemma:beta-vk}
    For all $k\in[n]$, 
    \begin{align*}
        \multError \valBy{k}{\retSol} \geq \left(\sum_{i=1}^k \valBy{i}{\retSol} - \sum_{i=1}^k z_i\right) -\multError \left(\sum_{i=1}^{k-1} \valBy{i}{\retSol} - \sum_{i=1}^{k-1} z_i\right) -\additiveError
    \end{align*}
\end{lemma}

\begin{proof}
By Observation \ref{obs:obj-xt-to-zt},
    \begin{align*}
         &\sum_{i=1}^k \valBy{i}{\retSol} - \sum_{i=1}^{k-1} z_i \leq \frac{1}{\multApprox} \left(z_k + \additiveError \right) = \frac{1}{1-\multError} \left(z_k + \additiveError \right)\\
         &\Rightarrow z_k +\additiveError \geq (1-\multError) \left(\sum_{i=1}^{k} \valBy{i}{\retSol} - \sum_{i=1}^{k-1}  z_i\right)\\
        &\Rightarrow z_k +\additiveError\geq \left(\sum_{i=1}^{k} \valBy{i}{\retSol} - \sum_{i=1}^{k-1}  z_i\right) - \multError \left(\sum_{i=1}^{k} \valBy{i}{\retSol} - \sum_{i=1}^{k-1}  z_i\right)\\
        &\Rightarrow \multError \valBy{k}{\retSol} \geq \left(\sum_{i=1}^k \valBy{i}{\retSol} - \sum_{i=1}^k z_i\right) -\multError \left(\sum_{i=1}^{k-1} \valBy{i}{\retSol} - \sum_{i=1}^{k-1} z_i\right) -\additiveError.
        \qedhere
    \end{align*}
\end{proof}


\begin{lemma}\label{lemma:beta-sums-to-diff}
    For all $k\in[n]$, 
    \begin{align*}
        \sum_{i=1}^k \multError^{i} \valBy{k-i+1}{\retSol} \geq \sum_{i=1}^k \valBy{i}{\retSol} - \sum_{i=1}^{k} z_i -\additiveError
    \end{align*}
\end{lemma}

\begin{proof}
    The proof is by induction on $k$.
    For $k=1$ the claim follows directly from Lemma \ref{lemma:beta-vk}.
    Assuming the claim is true for $1,\ldots k-1$, we show it is true for $k$:
    \begin{align*}
        &\sum_{i=1}^k \multError^{i} \valBy{k-i+1}{\retSol} = \multError \valBy{k}{\retSol} + \sum_{i=2}^k \multError^{i} \valBy{k-i+1}{\retSol}\\
        &= \multError \valBy{k}{\retSol} + \sum_{i=1}^{k-1} \multError^{i+1} \valBy{k-(i+1)+1}{\retSol} \\
        &= \multError \valBy{k}{\retSol} + \multError \sum_{i=1}^{k-1} \multError^{i} \valBy{(k-1) -i+1}{\retSol}\\
        &= \multError \valBy{k}{\retSol} + \multError \left(\sum_{i=1}^{k-1} \valBy{i}{\retSol} - \sum_{i=1}^{k-1} z_i\right) && \text{(by induction assumption)}\\
        &\geq \left(\sum_{i=1}^k \valBy{i}{\retSol} - \sum_{i=1}^k z_i\right) -\multError \left(\sum_{i=1}^{k-1} \valBy{i}{\retSol} - \sum_{i=1}^{k-1} z_i\right)-\additiveError  \\
        & \quad +  \multError \left(\sum_{i=1}^{k-1} \valBy{i}{\retSol} - \sum_{i=1}^{k-1} z_i\right) && \text{(by Lemma \ref{lemma:beta-vk})} \\
        &= \sum_{i=1}^k \valBy{i}{\retSol} - \sum_{i=1}^{k} z_i -\additiveError.
        \qed
    \end{align*}
\end{proof}


\begin{lemma}\label{lemma:fk-to-all}
    For all $1<k \leq n$, 
    \begin{align*}
        \frac{\multError}{1-\multError} \valBy{k}{\retSol} \geq \sum_{i=1}^{k-1}\valBy{i}{\retSol} - \sum_{i=1}^{k-1}z_i - \additiveError
    \end{align*}
\end{lemma}

\begin{proof}
    First, notice that since $k \geq (k-1)-i+1$ for any $1\leq i \leq k$ and as the function $\valBy{i}$ represents the $i$-th smallest objective value, also:
    \begin{align}\label{eq:increase-by-obj-size}
        \forall 1\leq i \leq k \colon \quad \valBy{k}{\retSol} \geq \valBy{(k-1)-i+1}{\retSol}
    \end{align}
    In addition, consider the geometric series with a first element $1$, a ratio $\multError$, and a length $(k-1)$. 
    As $\multError < 1$, its sum can be bounded in the following way:
    \begin{align}\label{eq:geometric-series-beta}
        \sum_{i=1}^{k-1} \multError^{i-1} = \frac{1-\multError^{k-1}}{1-\multError} < \lim_{k \to \infty}\frac{1-\multError^{k-1}}{1-\multError} = \frac{1}{1-\multError}
    \end{align}
    
    Now, the claim can be concluded as follows:
    \begin{align*}
        & \frac{\multError}{1-\multError}\valBy{k}{\retSol} = \multError \left(\frac{1}{1-\multError} \valBy{k}{\retSol} \right)\\
        & > \multError \left(\sum_{i=1}^{k-1} \multError^{i-1} \valBy{k}{\retSol} \right) && \text{(by Equation \eqref{eq:geometric-series-beta})}\\
        & \geq  \multError \left(\sum_{i=1}^{k-1} \multError^{i-1} \valBy{(k-1)-i+1}{\retSol} \right) && \text{(by Equation \eqref{eq:increase-by-obj-size})}\\
        &= \sum_{i=1}^{k-1} \multError^{i} \valBy{(k-1)-i+1}{\retSol} \\
        &\geq \sum_{i=1}^{k-1}\valBy{i}{\retSol} - \sum_{i=1}^{k-1}z_i - \additiveError && \text{(by Lemma \ref{lemma:beta-sums-to-diff})}
\end{align*}
\erel{Formally, Lemma \ref{lemma:beta-sums-to-diff} is for $k\geq 1$, and we apply it for $k-1$, which might be $0$.}\eden{I tried to fixed it, is it better?}
\end{proof}



%------
% thm.

We are now ready to prove the Theorem \ref{th:main}.
\begin{proof}[Proof of Theorem \ref{th:main}]
% \eden{I'm not sure if we should write again about the claim with $\multApprox$}
Recall that the claim is that $\retSol$ is a $\left(\frac{\multApprox^2}{1-\multApprox + \multApprox^2}, \frac{\multApprox(2-\multApprox)\additiveApprox}{1-\multApprox +\multApprox^2}\right)$-approximation.

For brevity, we define the following constants:
\begin{align*}
    \Delta^{mult} = \frac{\multApprox}{1-\multApprox + \multApprox^2}, \quad  \Delta^{add} = \frac{\multApprox(2-\multApprox)}{1-\multApprox +\multApprox^2}
\end{align*}
Accordingly, we need to prove that $\retSol$ is a $\left(\Delta^{mult} \cdot \multApprox, \Delta^{add}\cdot\additiveApprox\right)$-approximation.

We prove the following equation, that will be helpful later:
\begin{align}\label{equ:mu}
\frac{1}{\Delta^{mult} \cdot \multApprox} = \frac{1-\multError +\multError^2}{(1-\multError)^2}
\end{align}
This is true because
\begin{align*}
    &\Delta^{mult} \cdot \multApprox =   \frac{\multApprox^2}{1-\multApprox + \multApprox^2} && \text{(Definition of $\Delta^{mult}$)} \\
    &= \frac{(1-\multError)^2}{\multError +(1-\multError)^2} = \frac{(1-\multError)^2}{1-\multError +\multError^2} &&\text{(since $\multApprox = 1-\multError$)}\\
    & \Rightarrow \frac{1}{\Delta^{mult} \cdot \multApprox} = \frac{1-\multError +\multError^2}{(1-\multError)^2}
    \end{align*}
    Another equation that will be useful later is:
    \begin{align}\label{eq:additive-error}
        \frac{\Delta^{add}}{\Delta^{mult}\cdot \multApprox}  = \frac{1+\multError}{1-\multError}.
    \end{align}
    The reason for this is that
    \begin{align*}
        &\frac{\Delta^{add}}{\Delta^{mult}\cdot \multApprox} =\frac{1-\multApprox + \multApprox^2}{\multApprox^2} \cdot \frac{\multApprox(2-\multApprox)}{1-\multApprox +\multApprox^2}&& \text{(Definitions of $\Delta^{mult}$ and $\Delta^{add}$)}\\
        &=\frac{\multApprox(2-\multApprox)}{\multApprox^2} = \frac{(1-\multError)(1 + \multError)}{(1-\multError)^2} =\frac{1+\multError}{1-\multError}  &&\text{(since $\multApprox = 1-\multError$)}
    \end{align*}

    Now, suppose by contradiction that $\retSol$ is \emph{not} $\left(\Delta^{mult} \cdot \multApprox, \Delta^{add}\cdot\additiveApprox\right)$-approximately-optimal.
    By definition, this means there exists a solution $y \in S$  that is $\left(\Delta^{mult} \cdot \multApprox, \Delta^{add}\cdot\additiveApprox\right)$-preferred over it.
    That is, there exists an integer $1 \leq k \leq n$ such that:
    \begin{align*}
        \forall j < k \colon &\valBy{j}{y} \geq \valBy{j}{\retSol};\\
        & \valBy{k}{y} > \frac{1}{\Delta^{mult} \cdot\multApprox} \left(\valBy{k}{\retSol} + \Delta^{add} \cdot\additiveError \right).
    \end{align*}

    Since $\retSol$ was obtained in \eqref{eq:compact-OP} that was solved in the last iteration $n$, it is clear that $\sum_{i=1}^k \valBy{i}{\retSol} \geq \sum_{i=1}^{k} z_i$ (by constraint $(\Tilde{2})$ if $k<n$ and $(\Tilde{3})$ otherwise).
    Which implies:
    \begin{align}\label{eq:fk-to-zk}
        \sum_{i=1}^k \valBy{i}{\retSol} - \sum_{i=1}^{k-1} z_i \geq z_k
    \end{align}

    Now, consider \eqref{eq:compact-OP} that was solved in iteration $k$.
    By Observation \ref{obs:retSol-solves-any-t}, $\retSol$ is feasible to this problem.
    As the $(k-1)$ smallest objective values of $y$ are at least as high as those of $\retSol$, it is easy to conclude that $y$ also satisfies constraints $(\Tilde{2})$ of this problem; since, for any $\ell < k$:
    \begin{align*}
        \sum_{i=1}^{\ell} \valBy{i}{y} \geq\sum_{i=1}^{\ell} \valBy{i}{\retSol} \geq \sum_{i=1}^{\ell} z_i
    \end{align*}
    Therefore, by Observation \ref{obs:feasi-and-constraint2}, $y$ is also feasible to this problem. 

    If $k=1$, the objective value $y$ in this problem is $\valBy{1}{y}$ (Observation \ref{obs:obj-value}).
    In addition, $\valBy{1}{\retSol} \geq z_1$ by equation \ref{eq:fk-to-zk}. As $\Delta^{mult}\geq 0$ and $\Delta^{add}\geq 0$, it follows that:
    \begin{align*}
        \valBy{1}{y}> \frac{1}{\Delta^{mult} \cdot\multApprox} \left(\valBy{1}{\retSol} + \Delta^{add} \cdot\additiveError \right)\geq \frac{1}{\multApprox} \left(z_1 + \additiveError \right)
    \end{align*}
    But, $z_1$ was obtained as an approximation for this problem, therefore the optimal value is at most $\frac{1}{\multApprox}\left(z_1 + \additiveError \right)$ --- a contradiction.

    
    Otherwise, $k>1$, we shall now see that in this case $y$ also satisfies the following:
    \begin{align}\label{eq:yk-to-sum}
        \valBy{k}{y} > \frac{1}{1-\multError} \valBy{k}{\retSol} + \frac{\multError}{1-\multError}\sum_{i=1}^{k-1}\valBy{i}{\retSol} - \frac{\multError}{1-\multError} \sum_{i=1}^{k-1}z_i  +\frac{1}{1-\multError}\cdot\additiveError
    \end{align}
    this is true because
    \begin{align*}
        &\valBy{k}{y} > \frac{1}{ \Delta^{mult} \cdot\multApprox} \left(\valBy{k}{\retSol} + \Delta^{add}\cdot \additiveError \right) && \text{(Definition of $y$ for $k$)}\\
        &= \frac{1-\multError +\multError^2}{(1-\multError)^2} \valBy{k}{\retSol}+ \frac{\Delta^{add}}{\Delta^{mult} \multApprox}\cdot\additiveError && \text{(by Equation \ref{equ:mu})}\\
        &= \frac{1-\multError +\multError^2}{(1-\multError)^2} \valBy{k}{\retSol}+ \frac{1+\multError}{1-\multError}\cdot\additiveError && \text{(by Equation \ref{eq:additive-error})} \erel{???}\\
        &\geq\frac{1}{1-\multError} \valBy{k}{\retSol} + \frac{\multError}{1-\multError}\left(\sum_{i=1}^{k-1}\valBy{i}{\retSol} - \sum_{i=1}^{k-1}z_i-\additiveError\right) +\frac{1+\multError}{1-\multError}\cdot\additiveError && \text{(by Lemma \ref{lemma:fk-to-all} for $k>1$)}\\
        & = \frac{1}{1-\multError} \valBy{k}{\retSol} +\frac{\multError}{1-\multError}\sum_{i=1}^{k-1}\valBy{i}{\retSol} - \frac{\multError}{1-\multError} \sum_{i=1}^{k-1}z_i +\frac{1}{1-\multError}\cdot\additiveError &&\erel{???}\text{\eden{is it more clear?}}
    \end{align*}    
    
    We compute the objective value of $y$, which is $\sum_{i=1}^k \valBy{i}{y} - \sum_{i=1}^{k-1} z_i$ (by Observation \ref{obs:obj-value}):  
    \begin{align*}
        &\sum_{i=1}^k \valBy{i}{y} - \sum_{i=1}^{k-1} z_i=\sum_{i=1}^{k-1} \valBy{i}{y} - \sum_{i=1}^{k-1} z_i + \valBy{k}{y}\\
        &\geq \sum_{i=1}^{k-1} \valBy{i}{\retSol} - \sum_{i=1}^{k-1} z_i + \valBy{k}{y} && \text{(Definition of $y$ for $j<k$)}\\
        &> \sum_{i=1}^{k-1} \valBy{i}{\retSol} - \sum_{i=1}^{k-1} z_i + \frac{1}{1-\multError} \valBy{k}{\retSol} \\
        & \quad + \frac{\multError}{1-\multError}\sum_{i=1}^{k-1}\valBy{i}{\retSol} - \frac{\multError}{1-\multError}\sum_{i=1}^{k-1}z_i +\frac{1}{1-\multError}\cdot\additiveError && \text{(by Equation \ref{eq:yk-to-sum})}\\
        & = \frac{1}{1-\multError} \left(\sum_{i=1}^k \valBy{k}{\retSol} - \sum_{i=1}^{k-1}z_i + \additiveError\right) &&\text{(since  $1+\frac{\multError}{1-\multError} = \frac{1}{1-\multError}$)}\erel{???}\text{\eden{is it more clear?}}
        \\
        &\geq \frac{1}{1-\multError} \left(z_k +\additiveError\right) && \text{(by Equation \ref{eq:fk-to-zk}) }
    \end{align*}
    \eden{I'm not sure why to comment the lines, shouldn't we explain why it is a contradiction?how is the following?}
    % \emark{However, the approximately-optimal solution obtained for this problem during the algorithm run is $z_k$, so the optimal value is at most $\frac{1}{(1-\multError)}\left(z_k+\additiveError\right)$.
    % But, as we shall see, the objective $y$ yields in this problem, $\sum_{i=1}^k \valBy{i}{y} - \sum_{i=1}^{k-1} z_i$ (by Observation \ref{obs:obj-value}), is higher than this value, which is of course a contradiction:}
    However, the approximately-optimal value obtained for this problem during the algorithm run is $z_k$, so the optimal value is at most $\frac{1}{(1-\multError)}\left(z_k+\additiveError\right)$, which is, again, a contradiction.
    
\end{proof}

\section{Proof of Theorem \ref{th:app-main}}\label{sec:app-sec-proofs}
% \eden{should probably change the title}

% Agents are assumed to care only about their own share (allowing us to use the following abuse of notation in which $u_j$ takes a bundle $b$ of items), their utilities are assumed to be normalized ($u_j(\emptyset) = 0$), monotone ($u_j(b_1) \leq u_j(b_2)$ if $b_1 \subseteq b_2$), and submodular ($u_j(b_1) + u_j(b_2) \geq u_j(b_1 \cup b_2) + u_j(b_1 \cap b_2)$ for any bundles $b_1,b_2$).
% It is assumed that each agent assigns a positive utility to the set of all items.
% The utilities $(u_i)_{i=1}^n$ are assumed to be given in the \emph{value oracle model}, meaning that we do not have a direct access to them, but only to an oracle that indicates the value of an agent from a given simple allocation.
% % \eden{z1 > 0}

This section proves Theorem \ref{th:app-main}:
suppose we are given a randomized algorithm that returns a simple allocation that approximates the utilitarian welfare with multiplicative error $\multError$ (with success probability $p$).
Then, Algorithm \ref{alg:basic-ordered-Outcomes} can be used to obtain a stochastic allocation that approximates leximin with a multiplicative error of at most $\frac{\multError}{1-\multError +\multError^2}$ (with the same probability).

% title: the specific problem as P3
As we saw in Section \ref{sec:algo-short}, an approximation to leximin can be obtained by providing a procedure \textsf{OP} to approximate \eqref{eq:vsums-OP}  (Theorem \ref{th:main}), which, under these particular settings, becomes:
% \erel{Why do you call it "configuration LP"? I think this term refers to something else: \url{https://en.wikipedia.org/wiki/Configuration_linear_program}}
\begin{align}
&\max \quad z_t \quad s.t. \tag{\progAppFirst}\label{eq:app-vsums-OP}\\
& (\text{\progAppFirst.1.1}) \Hquad \sum_{A \in \mathcal{A}} p_d(A) = 1 \nonumber\\
& (\text{\progAppFirst.1.2}) \Hquad p_d(A) \geq 0  && \forall A \in \mathcal{A} \nonumber\\
& (\text{\progAppFirst.2}) \Hquad \ell y_{\ell} - \sum_{j=1}^n m_{\ell,j}\geq \sum_{i=1}^{\ell}  z_i && \forXinY{\ell}{t-1} \nonumber \\
& (\text{\progAppFirst.3}) \Hquad t y_t - \sum_{j=1}^{n} m_{t,j} \geq \sum_{i=1}^{t}  z_i \nonumber \\
& (\text{\progAppFirst.4}) \Hquad m_{\ell,j} \geq y_{\ell} - \sum_{A \in \mathcal{A}}p_d(A) \cdot u_j(A)  && \forXinY{\ell}{t},\Hquad \forXinY{j}{n} \nonumber \\
& (\text{\progAppFirst.5}) \Hquad m_{\ell,j} \geq 0  && \forXinY{\ell}{t},\Hquad \forXinY{j}{n} \nonumber
\end{align}
Here the variables are $p_d(A)$ for any simple allocation $A \in \mathcal{A}$, $\ztVar{}$, and $y_{\ell}$ and $m_{\ell,j}$ for all $\ell \in [t]$ and $ j\in [n]$; and the values $z_1, \ldots z_{t-1}$ are constants.
Notice that it is a \emph{linear program} that has a polynomial number of constraints thanks to \eqref{eq:vsums-OP} representation, but an exponential number of variables (since there is a variable $p_d(A)$ for each simple allocation).
So, it is unclear how to approach it directly in polynomial time.
% \eden{here?}
In addition, it means that the output size is exponential in $n$.
To deal with this issue, the solutions are considered in \emph{sparse form} --- a list of the variables with positive values, along with their values.
Accordingly, if a solution has only a polynomial number of variables with positive values it can be represented by a polynomial size.
We will later see that the procedure described in this section returns such a solution in polynomial time.
% \eden{should write something about the output size, as \cite{kawase_max-min_2020}}

% title: baseline
% \erel{I would move the following paragraph upwards}
With $t=1$, \eqref{eq:app-vsums-OP} can be viewed as the problem of egalitarian welfare maximization, indeed, Kawase and Sumita \cite{kawase_max-min_2020} who studied this problem, considered a slightly simpler representation. 
% After proving that approximating the optimal value to a factor better than $(1-\frac{1}{e})$ is NP-hard, they present a dual-based algorithm that achieves this accuracy \er{w.h.p (?)}.
We now show how their dual-based technique can be applied to approximate \eqref{eq:app-vsums-OP} for any $t\geq 1$ while maintaining the same approximation factor.


To begin, consider the following program \eqref{eq:app-ver2-vsums-OP}, which is the result of modifying \eqref{eq:app-vsums-OP} in three ways. 
First, changing the objective-function to $\min 1/z_t$ instead of $\max z_t$. 
Second, replacing all the original variables and constants, except $z_t$, with new ones that are smaller by a factor $z_t$ (that is, $p'_A = p_d(A)/z_t$ for all $A \in \mathcal{A}$, $,y'_{\ell} = y_{\ell}/z_t,m'_{\ell,j} = m_{\ell,j}/z_t$ for $\ell \in [t]$ and $ j\in [n]$,  and $z'_i = z_i/z_t$ for $i \in [t-1]$).
And third, dividing all the constraints by $z_t$ ($z_t > 0$ since $z_t \geq z_1$ for any $t \geq 1$ and  $z_1 >0$).
\eden{to myself: maybe to explain why $z_1>0$}

\begin{align}
& \min \quad 1/z_t \quad s.t. \tag{\progAppSecond}\label{eq:app-ver2-vsums-OP}\\
& (\text{\progAppSecond.1.1}) \Hquad \sum_{A \in \mathcal{A}} p'_A = 1/z_t \nonumber\\
& (\text{\progAppSecond.1.2}) \Hquad p'_A \geq 0  && \forall A \in \mathcal{A} \nonumber\\
& (\text{\progAppSecond.2}) \Hquad \ell y'_{\ell} - \sum_{j=1}^n m'_{\ell,j}\geq \sum_{i=1}^{\ell}  z'_i && \forXinY{\ell}{t-1} \nonumber \\
& (\text{\progAppSecond.3}) \Hquad t y'_t - \sum_{j=1}^{n} m'_{t,j} \geq \sum_{i=1}^{t-1}  z'_i + 1 \nonumber \\
& (\text{\progAppSecond.4}) \Hquad m'_{\ell,j} \geq y'_{\ell} - \sum_{A \in \mathcal{A}}p'_A \cdot u_j(A)  && \forXinY{\ell}{t},\Hquad \forXinY{j}{n} \nonumber \\
& (\text{\progAppSecond.5}) \Hquad m'_{\ell,j} \geq 0  && \forXinY{\ell}{t},\Hquad \forXinY{j}{n} \nonumber
\end{align}
The programs \eqref{eq:app-vsums-OP} and \eqref{eq:app-ver2-vsums-OP} are related in the following way:
% \erel{I would make this a lemma:}
\begin{lemma}\label{lemma:bijection}
There exists a bijection mapping each solution of 
\eqref{eq:app-vsums-OP} with objective value $V$ to a unique solution of 
\eqref{eq:app-ver2-vsums-OP} with objective value $1/V$.
\end{lemma}
\begin{proof}
Let $p_d(A)$ for $A \in \mathcal{A}$, $\ztVar{}$, and $y_{\ell}$ and $m_{\ell,j}$ for all $\ell \in [t]$ and $ j\in [n]$ be a feasible solution to the program \eqref{eq:app-vsums-OP} with objective value $V$.
It can be easily verified that $p'_A = p_d(A)/z_t$ for $A \in \mathcal{A}$, $z_t$, and $y'_{\ell} = y_{\ell}/z_t$ and $m'_{\ell,j} = m_{\ell,j}/z_t$ for all $\ell \in [t]$ and $ j\in [n]$ is a feasible solution to the program \eqref{eq:app-ver2-vsums-OP} with objective value $1/V$.
\end{proof}
% \eden{maybe to write something about why it is a bijection (or to write that it is straightforward)}

Denote this bijection by $\Psi$, this also implies the following:
\begin{lemma}\label{lemma:approx-acc-by-bijection}
    If a solution approximates the program \eqref{eq:app-ver2-vsums-OP} with a multiplicative error of $\frac{\multError}{1-\multError}$. Then the corresponding solution to \eqref{eq:app-vsums-OP} according to the bijection $\Psi$ approximates this program with a multiplicative error of $\multError$.
\end{lemma}

\begin{proof}
    Let $V^*$ be the optimal objective value of \eqref{eq:app-vsums-OP}. 
    By Lemma \ref{lemma:bijection}, there exists a solution to \eqref{eq:app-ver2-vsums-OP} with value $1/V^{*}$.
    This solution yields the optimal value for \eqref{eq:app-ver2-vsums-OP} --- if there was a solution that had a value \emph{lower} than $1/V^{*}$ (\eqref{eq:app-ver2-vsums-OP} is a minimization problem), then the corresponding solution to \eqref{eq:app-vsums-OP} (by the bijection $\Psi$) would have a value higher than the optimal value $V^*$.
    Now, let the value of the solution that approximates the program \eqref{eq:app-ver2-vsums-OP} with a multiplicative error of $\frac{\multError}{1-\multError}$ be $1/V$. 
    Since \eqref{eq:app-ver2-vsums-OP} is a minimization problem, assuming that $1/V$ approximates $1/V^*$ with a multiplicative error of $\frac{\multError}{1-\multError}$ means that:
    \begin{align*}
        \frac{1}{V} \leq \left(1+\frac{\multError}{1-\multError}\right)\frac{1}{V^*},
    \end{align*}
 which implies that $V \geq (1-\multError)V^*$.
    As \eqref{eq:app-vsums-OP} is a maximization problem, this means that $V$ approximates this problem with multiplicative error $\multError$.
    By Lemma \ref{lemma:bijection}, $V$ is the value of the corresponding solution to \eqref{eq:app-vsums-OP} by the bijection $\Psi$.
\end{proof}

Notice that the only constraint of \eqref{eq:app-ver2-vsums-OP} that includes the variable $z_t$, (\progAppSecond.1.1), says that $\sum_{A \in \mathcal{A}}p'_A = 1/z_t$, and also that its objective function is $\min 1/z_t$.
As a result, we can reduce the need for the variable $z_t$ by removing constraint (\progAppSecond.1.1) and changing the objective function to $\min \sum_{A \in \mathcal{A}}p'_A$.
This change makes \eqref{eq:app-ver2-vsums-OP} a \emph{linear} program.
This will allow us to approximate it using its dual, as we will see.

The following observation will be useful later:
\begin{observation}\label{obs:c2-to-c1-in-poly-time}
    If a solution to \eqref{eq:app-ver2-vsums-OP} is given in a sparse form --- a list of the variables with nonzero value and their values, then the corresponding solution to \eqref{eq:app-vsums-OP} in a sparse form can be computed in time polynomial to the number of nonzero variables.
\end{observation}
\noindent For completeness, we briefly outline the process. 
When given a list of variables with nonzero values, we first iterate the list and sum all variables of the form $p'_A$, and then set $z_t$ to be $1$ divided by this sum. 
After, for each variable $\nu'$ in the list, we set the corresponding variable, $\nu$, to $z_t \cdot \nu'$.


% title: dual 
Now, let us consider the dual program of \eqref{eq:app-ver2-vsums-OP}, which can be described as follows:
% \erel{When you present an LP, it can help the reader if you mention what exactly the variables of the LP are.}
\begin{align}
    \max &&& \sum_{\ell=1}^{t-1} q_{\ell} \sum_{i=1}^{\ell} z_i + q_t (\sum_{i=1}^{t-1} z_i +1) \tag{\progAppDual}\label{eq:app-dual}\\
        s.t. &&& (\text{\progAppDual.1}) \Hquad \sum_{j=1}^n u_j(A) \sum_{\ell=1}^t v_{\ell,j} \leq 1  && \forall A \in \mathcal{A} \nonumber\\
                    &&& (\text{\progAppDual.2}) \Hquad \ell q_{\ell} - \sum_{j=1}^n v_{\ell,j} = 0 && \forXinY{\ell}{t} \nonumber \\
                    &&& (\text{\progAppDual.3}) \Hquad q_{\ell} - v_{\ell,j} \leq 0  && \forXinY{\ell}{t},\Hquad \forXinY{j}{n} \nonumber \\
                    &&& (\text{\progAppDual.4}) \Hquad v_{\ell,j} \geq 0  && \forXinY{\ell}{t},\Hquad \forXinY{j}{n} \nonumber \\
                    &&& (\text{\progAppDual.5}) \Hquad q_{\ell} \geq 0  && \forXinY{\ell}{t} \nonumber
\end{align}
Here, the variables are $q_{\ell}$ and $v_{\ell,j}$ for any $\ell \in [t]$ and $j \in [n]$; and the constants are (as before) $z_i$ for $i \in [t-1]$.
Recall that $u_j(A)$ is the utility that agent $j$ assigns to simple allocation $A$, as given by the value oracle.
% title: ellipsoid variant
This problem has an exponential number of constraints --- a constraint for each allocation (in line (\progAppDual.1)) but only a polynomial number of variables.
Using the ellipsoid method \cite{grotschel_ellipsoid_1981}, it could be solved in polynomial time 
if we had a \emph{separation oracle} ---
an oracle that given a vector $\upsilon$ either determines that $\upsilon$ is infeasible and returns a violated constraint, or asserts that $\upsilon$ is feasible.
Unfortunately, as we shall now see, it is NP-hard to compute a separation oracle to this problem.
\begin{lemma}
    Computing a separation oracle to \eqref{eq:app-dual} is NP-hard.
\end{lemma}

% very similar to what they did in yonatan's paper..
\begin{proof}
We prove that a separation oracle for \eqref{eq:app-dual} would allow us to compute a leximin optimal stochastic allocation.
    As discussed previously, computing such an allocation is NP-hard, so the same applies for computing a separation oracle for \eqref{eq:app-dual}.

    First, we prove that such a separation oracle can be used to extract an optimal solution to \eqref{eq:app-ver2-vsums-OP}.
    Assume that the ellipsoid method was operated with the given oracle to solve \eqref{eq:app-dual}.
    Let $C$ be the set of constraints that the oracle determined as being violated.
    Since the ellipsoid method operates in polynomial time, the size of the set $C$ is also polynomial.
    Let $V_C$ be the set of variables of \eqref{eq:app-ver2-vsums-OP} associated with the constraints in $C$.
    By complementary slackness, the variables in $V_C$ are the only ones that may get a \emph{positive} value in the corresponding optimal solution to \eqref{eq:app-ver2-vsums-OP}.
    Therefore, the program \eqref{eq:app-ver2-vsums-OP} with only the variables in $V_C$ (and the other variables equal to zero) has a polynomial size, and therefore can be solved exactly.


    But, by Observation \ref{obs:c2-to-c1-in-poly-time}, this would allow us to find the corresponding optimal solution to \eqref{eq:app-vsums-OP} in polynomial time.
    % \erel{Did we say that $\psi$ can be computed in polynomial time?}\eden{in the way it is written now is not, it iterate over each variable of \eqref{eq:app-vsums-OP} and there are exponential number of them. I need to think how to write it appropriately. maybe "that can be computed in time equals to the number of positive variables"?}
    % \erel{If it is not polynomial, then the reduction is not polynomial, so it does not imply NP-hardness}
    This means the described process can be used as an approximation procedure to \eqref{eq:vsums-OP} (that became \eqref{eq:app-vsums-OP} under the settings of this problem) with $\multError = \additiveError = 0$.
    Therefore, by Theorem \ref{th:main}, this means we can use Algorithm \ref{alg:basic-ordered-Outcomes} to obtain a leximin optimal solution\footnote{Actually, Theorem \ref{th:main} says that Algorithm \ref{alg:basic-ordered-Outcomes} will output a $(1,0)$-leximin-approximation; But Lemma \ref{lemma:absence-of-errors} says that such a solution is, indeed, a leximin optimal solution.}.
\end{proof}


% --- it would allow us to compute a leximin optimal stochastic allocation, which is, as discussed previously, NP-hard.

In Appendix \ref{sec:mult-variant-ellipsoid}, we present another variant of the ellipsoid method, which allows us to approximate the program \eqref{eq:app-ver2-vsums-OP} given a \emph{half-randomized approximate separation oracle} to \eqref{eq:app-dual}.
That is, an oracle that, given a multiplicative error $\multError$, a success probability $p$, and a vector $\upsilon$, either determines that $\upsilon$ is infeasible and returns a violated constraint; or determines that $\upsilon$ is $\multError$-\textit{approximately-feasible}, which means that for any constraint $a \cdot x \leq b$, the vector $\upsilon$ satisfies $a \cdot \upsilon \leq (1+\multError)\cdot b$.
When the oracle says that $\upsilon$ is $\multError$-approximately-feasible, it is correct with probability at least $p$.
Given such an oracle for the dual program, the ellipsoid method variant can be used to output a solution to the primal, that approximates it to the same factor with probability at least $p^I$, where $I$ is an upper bound on the number of iterations in any execution of the ellipsoid method variant on the dual (if it is given a deterministic oracle).
We can therefore conclude the following result:
\begin{lemma}\label{lemma:approx-sep-oracle-to-goal}
    Given a half-randomized approximate separation oracle to the problem \eqref{eq:app-dual}, with a multiplicative error of $\frac{\beta}{1-\beta}$ and a success probability $p$, a stochastic allocation that approximates leximin to a multiplicative error $\frac{\multError}{1-\multError+\multError^2}$ can be obtained with probability $p^{nI}$.
\end{lemma}

\begin{proof}
    % To begin, assume that we are given a deterministic approximate separation oracle (i.e., with failure probability $p=0$).
    As described above, we can use the ellipsoid method variant of Appendix \ref{sec:mult-variant-ellipsoid} with the given oracle to \eqref{eq:app-dual} to obtain a solution to \eqref{eq:app-ver2-vsums-OP},  that approximates it with a multiplicative error of $\frac{\multError}{1-\multError}$ with probability $p^I$.
    Then, by Observation \ref{obs:c2-to-c1-in-poly-time}, this would allow us to find the corresponding solution to \eqref{eq:app-vsums-OP}, that, with probability $p^I$, approximates it with a multiplicative error of $\multError$.
    That is, the described process can be used as a randomized approximation procedure to \eqref{eq:vsums-OP} (that became \eqref{eq:app-vsums-OP} under the settings of this problem).
    % with $\multError = \additiveError = 0$.
    Therefore, by Theorem \ref{th:main}, Algorithm \ref{alg:basic-ordered-Outcomes} can be used to obtain a leximin approximation to the original problem with only a multiplicative error of $\frac{\multError}{1-\multError+\multError^2}$ with probability $p^{nI}$ (Corollary \ref{corollary:main-with-probability}).
\end{proof}

Now, we show that such an oracle can be designed given a randomized approximation algorithm for computing a simple allocation that approximates the utilitarian welfare. Specifically, 

\begin{lemma}\label{lemma:alg-for-utilitarian-to-sep-oracle}
    Given a randomized approximation algorithm for computing a simple allocation that approximates the utilitarian welfare with multiplicative error $\multError$ and a success probability $p$, a half-randomized approximate separation oracle to \eqref{eq:app-dual} can be designed with a multiplicative error of $\frac{\beta}{1-\beta}$ and a success probability at least $\left(1-\frac{1}{nI}(1-p)\right)$.
\end{lemma}

% \eden{should say somewhere that the oracle is polynomial time and therefore everything is?...}
% FROM HERE: https://tex.stackexchange.com/a/675333/20929
\algdef{SE}[REPEATN]{REPEATN}{ENDREP}[1]{\algorithmicrepeat\ #1 \textbf{times}}{\algorithmicend\ \algorithmicrepeat}
\begin{algorithm}[!tbp]
\caption{A Half-Randomized Approximate Separation Oracle to \eqref{eq:app-dual}}
\label{alg:sep-oracle}
INPUT: variables $q_{\ell}$ and $v_{\ell,j}$ for any $\ell \in [t]$ and $j \in [n]$, an $\multApprox$-approximation algorithm for the utilitarian welfare problem (\eqref{eq:utilitarian}) with success probability $p$.
\begin{algorithmic}[1] %[1] enables line numbers
\STATE Iterate over constraints (\progAppDual.2)-(\progAppDual.5). If one of them is  violated, stop and return it.
\STATE \textbf{If} $p=1$ then set $T:=1$; \textbf{else} set $T := 1 + \lceil-\log_{(1-p)}(nI)\rceil$.

\REPEATN{$T$}
    \STATE Operate the algorithm for the utilitarian welfare problem on $n,m,(u'_j)_{j=1}^n$ to obtain an allocation $\Tilde{A}$ with value $\nu$.
    \IF{$\nu > 1$}  
        \STATE Return the corresponding violated constraint $\sum_{j=1}^n u_j(\Tilde{A}) \sum_{\ell=1}^t v_{\ell,j} > 1$
    \ENDIF
\ENDREP
\STATE Return "the assignment is approximately-feasible".

\end{algorithmic}
\end{algorithm}


Algorithm \ref{alg:sep-oracle} describes the oracle.
It accepts as input an assignment to the variables of \eqref{eq:app-dual}, that is, $q_{\ell}$ and $v_{\ell,j}$ for any $\ell \in [t]$ and $j \in [n]$, and an algorithm for approximating the maximum utilitarian welfare.
It starts by verifying constraints (\progAppDual.2)-(\progAppDual.5) one by one (this is possible as their number is polynomial in $n$ and $m$). 
If a violated constraint was found, the oracle simply returns it. Otherwise, it proceeds to check constraints (\progAppDual.1).
Although the number of constraints described by (\progAppDual.1) is exponential in $n$, they can be treated collectively in polynomial time (as in \cite{kawase_max-min_2020}).
% \eden{here maybe to say something about the randomness}.\erel{Maybe mention that \textcite{kawase_max-min_2020} ignored this issue.}
First, notice that in order to determine whether the expression $\sum_{j=1}^n u_j(A) \sum_{\ell=1}^t v_{\ell,j}$ is at most $1$ for all simple allocations ($A \in \mathcal{A}$), it is sufficient to check the allocation that maximizes this expression and compare it to $1$.
Define new utility functions for all $j \in [n]$ and $A \in \mathcal{A}$, 
\begin{align*}
u'_j(A) := \sum_{\ell=1}^t v_{\ell,j} \cdot u_j(A) 
\end{align*}
The above expression can be simplified to $\sum_{j=1}^n u'_j(A)$. An allocation that maximizes this expression is an allocation that maximizes the utilitarian welfare (i.e., the sum of utilities) when the same sets of agents and items is considered but with different utilities%
\footnote{Notice that the utilities $u'_j$ are  normalized, monotone, submodular, and can be computed using $t\leq n$ calls to the value oracle of $u_j$}
($u'_j$ instead of $u_j$ for $j \in [n]$).
Such an allocation cannot be found in polynomial time since approximating the utilitarian welfare up to a factor better than $(1-\frac{1}{e})$ in the case of submodular utilities is known to be NP-hard \cite{khot_inapproximability_2008}.
However, the oracle is given an approximation algorithm to the utilitarian welfare problem as input.
Therefore, an allocation $\Tilde{A}$ with utilitarian value at least $(1-\multError)$ of the optimal can be obtained with probability $p$.
We shall now see that it is enough.


\begin{proof}[Proof of Lemma \ref{lemma:alg-for-utilitarian-to-sep-oracle}]
First, observe that when Algorithm \ref{alg:sep-oracle} returns a violated constraint, it is always correct.
This is obvious for constraints described by (\progAppDual.2)-(\progAppDual.5), since these constraints have been verified directly.
For constraints described by (\progAppDual.1), it means that the algorithm found an allocation $\Tilde{A}$ that satisfies $\sum_{j=1}^n u'_j(\Tilde{A}) > 1$.
    By the definition of $u'$, the constraint corresponding to this allocation is, indeed, violated:
    \begin{align*}
         \sum_{j=1}^n u_j(\Tilde{A}) \sum_{\ell=1}^t v_{\ell,j} = \sum_{j=1}^n u'_j(\Tilde{A}) > 1.
    \end{align*}
Let us assume that the given algorithm for the utilitarian welfare problem is deterministic (i.e., $p=1$) and then revisit the case $p<1$.
    Assume that the oracle said that the assignment is approximately-feasible.
    This means that the algorithm for the utilitarian welfare problem found an allocation $\Tilde{A}$ with value at most $1$.
    Since $\Tilde{A}$ is approximately-optimal, the optimal utilitarian value is at most $1/(1-\multError)\cdot 1$.
    As this is an upper bound of the utilitarian value of any allocation, it follows that all the constraints described bu (\progAppDual.1) are $\frac{\multError}{1-\multError}$-approximately maintained --- that is, for any allocation $A \in \mathcal{A}$ the following holds:
    \begin{align*}
        \sum_{j=1}^n u'_j(A) = \sum_{j=1}^n u_j(A) \sum_{\ell=1}^t v_{\ell,j} \leq \frac{1}{1-\multError}\cdot 1 = \left(1+\frac{\multError}{1-\multError}\right)\cdot1
    \end{align*}
    We get that, in this case, the oracle is also deterministic, and that the success probability is at least $\left(1-\frac{1}{nI}(1-p)\right) = 1$ for $p=1$.

    Assume now that $p<1$. Then, the oracle may be incorrect when it says the assignment is approximately feasible, but only if the algorithm for the utilitarian welfare problem did not return an appropriate approximation in all $T = \lceil-\log_{(1-p)}(nI)\rceil + 1$ operations, that is, with probability at most $(1-p)^T$.
    % as each operation of the oracle is independent
    Notice that $T>1$ since $\log_{(1-p)}(nI) < 0$\footnote{
    % The fact that  $\log_{(1-p)}(nI) < 0$ can be easily concluded 
    Since $(1-p)\in(0,1)$ and $nI>1$ by change of base: $\log_{(1-p)}(nI) = \log(nI)/\log(1-p)$, the numerator is positive and the denominator is negative.}.
    Now, as $T \geq -\log_{(1-p)}(nI) + 1$ and $(1-p)<1$ we get that:
    \begin{align*}
        &(1-p)^T \leq (1-p)\cdot(1-p)^{-\log_{(1-p)}(nI)} = (1-p)(nI)^{-1}
    \end{align*}
    So, the success probability is at least $\left(1-\frac{1}{nI}(1-p)\right)$.
\end{proof}

We can now prove Theorem \ref{th:app-main}.

\begin{proof}[Proof of Theorem \ref{th:app-main}]
    Assume we are given an algorithm that returns a simple allocation that approximates the utilitarian welfare with multiplicative error $\multError$ with success probability $p$.
    By Lemma \ref{lemma:alg-for-utilitarian-to-sep-oracle} this algorithm can be used to obtain an half-randomized approximate separation oracle to \eqref{eq:app-dual} with a multiplicative error $\frac{\multError}{1-\multError}$ with success probability $\left(1-\frac{1}{nI}(1-p)\right)$.
    By Lemma \ref{lemma:approx-sep-oracle-to-goal}, with such an oracle a stochastic allocation that approximates leximin to a multiplicative error of $\frac{\multError}{1-\multError+\multError^2}$ can be obtained with probability $\left(1-\frac{1}{nI}(1-p)\right)^{nI}$.
    If $p=1$ then the success probability is $1$ too (at least $\left(1-\frac{1}{nI}(1-p)\right)^{nI}= 1$).
    However, if $p<1$, then $\frac{1}{nI}(1-p) \in (0,1)$ and therefore the success probability is at least $p$\footnote{For any $\epsilon \in (0,1)$ and $k \in \mathbb{Z}_{+} \colon \Hquad (1 - \epsilon)^k \geq 1 - k \cdot \epsilon$}:
    \begin{align*}
        \left(1-\frac{1}{nI}(1-p)\right)^{nI} \geq \left(1-nI\cdot\frac{1}{nI}(1-p)\right) = p.   \end{align*}
\end{proof}

\section{Equivalent Single-objective Optimization Problems in the Presence of Errors}\label{sec:equivalent-proofs}

Many times, when referring to two optimization problems\footnote{In this section,  we consider only single-objective optimization problems.} as equivalent, one means that they have the same optimal value.
When two problems satisfy this relation, it is clear that in order to obtain an optimal \emph{value}, a solver\footnote{It is assumed that a solver (either approximate or exact) for a single-objective optimization problem returns a solution and its objective value.\eden{maybe to explain it better..}} for one can be used as a solver for the other. 
However, if we are interested in an optimal \emph{solution} that yields this value, a solver that returns an optimal solution for another problem with the same optimal value is not enough\eden{reduction to feasibility problem}.
Moreover, when it comes to approximation, even if we are only concerned about the objective value, an approximate solver for one can no longer be used for the other.
To illustrate, consider the following problems:
\begin{align*}
    (E1) \Hquad &\max\quad x                         &&& (E2)\Hquad &  \max\quad x\\
    &\Hquad s.t.\quad  x \in \{0.9,1\}       &&&& \Hquad s.t.\quad x \in \{0.95,1\} 
\end{align*}    
Both problems have the same optimal objective value $1$.
Now, assume that a multiplicative error of $0.1$ is acceptable.
An approximate solver for the problem $(E1)$ may return the objective value $0.9$, which is not a possible value of $(E2)$; similarly, an approximate solver for the problem $(E2)$ may return the objective value $0.95$, which is not a possible value of $(E2)$.
% Thus, although both problems have the same optimal value, an approximate solver for one problem \emph{cannot} be used as an approximate solver for the other.

In this appendix, we present a new definition of equivalent optimization problems, which requires a stronger relationship.
We prove that, according to our definition, when two optimization problems are equivalent, a solver for one, either exact or approximate, can also be used for the other.

\paragraph{Equivalent problems definition} We say that two (single-objective) optimization problems, $OP1 = (S_1,f_1)$ and $OP2 = (S_2,f_2)$, are \emph{equivalent} if they are from the same type --- either both are maximization problems or both are minimization problems; and there exists a bijection, $B \colon S_1 \to S_2$, mapping each solution of $OP1$, $x \in S_1$, to a unique solution of $OP2$, $B(x) \in S_2$, and they have the same objective value $f_1(x) = f_2(B(x))$.

% It is easy to conclude that this relation is symmetric, reflexive and transitive and therefore it is, indeed, an equivalence relation.
The following observation can be easily concluded by the definition:
\begin{observation}
    The equivalent relation between problems is transitive, reflexive and symmetric.
\end{observation}
% \begin{observation}
%     The equivalent relation between problems is transitive.
% \erel{maybe also argue that it is reflexive and symmetric, so it is an equivalence relation.}
% \end{observation}
% \eden{to explain }

If we are only concerned with the objective value, then the following lemma ensures that an approximate solver for one problem can be applied, as is, to the other (it is not necessary to know what the bijection is):

\begin{lemma}\label{lemma:approx-value-equivalent-prob}
    Let $OP1 = (S_1,f_1)$ and $OP2 = (S_2,f_2)$ be equivalent optimization problems, and let $v_1 \in \mathbb{R}$ be an $(\multApprox,\additiveApprox)$-approximation of the optimal objective value of $OP1$.
    Then, $v_1$ is also an $(\multApprox,\additiveApprox)$-approximation of the optimal objective value of $OP2$.
% \erel{Add that the approximation ratios ($\alpha$,$\epsilon$) is the same}
\end{lemma}

\begin{proof}
    For brevity, we prove the claim only for maximization problems, the proof for minimization problems is similar.
    
    Let $x^*\in S_1$ and $y^*\in S_2$ be optimal solutions of the problems $OP1$ and $OP2$ respectively.
    In order to prove that $v_1$ is an $(\multApprox,\additiveApprox)$-approximation of the optimal objective value of $OP2$, we will show that there is a solution $y \in S_2$ with objective value $v_1$, and also that $v_1 \geq \multApprox f_2(y^*) - \additiveApprox$.

    First, since $v_1$ is an $(\multApprox,\additiveApprox)$-approximation of the optimal objective value of $OP1$, there exists a solution $x \in S_1$ such that $f_1(x) = v_1$ and also $v_1 \geq \multApprox f_1(x^*) - \additiveApprox$.
    By definition of equivalent problems, the corresponding solution to $OP2$ by the bijection, $B(X) \in S_2$, has the same objective value $f_2(B(x)) = v_1$.
    
    In addition, we shall now see that both problems have the same optimal objective value.
    Let $B: S_1\to S_2$ be a bijection as described in the definition of equivalent problems.    
    So $f_1(x^*)=f_2(B(x^*))$, and $f_2(B(x^*))\leq f_2(y^*)$ by optimality of $y^*$, so $f_1(x^*)\leq f_2(y^*)$. By analogous arguments $f_2(y^*)\leq f_1(x^*)$, so in fact $f_1(x^*) = f_2(y^*)$.
    
    Therefore, we can conclude that:
    \begin{align*}
        f_2(B(x)) = v_1 \geq \multApprox f_1(x^*) - \additiveApprox =  \multApprox f_2(y^*) - \additiveApprox
    \end{align*}
    as required.
\end{proof}

% \begin{lemma}\label{lemma:solver-equivalent-prob}
%     Let $OP1 = (S_1,f_1)$ and $OP2 = (S_2,f_2)$ be equivalent optimization problems. Then, in order to approximate the optimal value, an $(\multApprox,\additiveApprox)$-approximate solver for one can be used as an $(\multApprox,\additiveApprox)$-approximate solver for the other.
% % \erel{Add that the approximation ratios ($\alpha$,$\epsilon$) is the same}
% \end{lemma}

% \begin{proof}
%     For brevity, we prove the claim only for maximization problems, the proof for minimization problems is similar.
%     Let $x^*\in S_1$ and $y^*\in S_2$ be optimal solutions of the problems $OP1$ and $OP2$ respectively.
%     Let $B: S_1\to S_2$ be a bijection as described in the definition of equivalent problems.    
%     So $f_1(x^*)=f_2(B(x^*))$, and $f_2(B(x^*))\leq f_2(y^*)$ by optimality of $y^*$, so $f_1(x^*)\leq f_2(y^*)$. By analogous arguments $f_2(y^*)\leq f_1(x^*)$, so in fact $f_1(x^*) = f_2(y^*)$.
%     % , since otherwise one of them is higher, and therefore the bijection can be used to obtain a solution to the second problem with value higher than optimal. \eden{need to rewrite it..}
%     Now, 
%     % without loss of generality, 
%     assume that we have an $(\multApprox, \additiveApprox)$-approximate solver for $OP1$, for some $\multApprox\in(0,1]$ and $\additiveError\geq 0$.
%     That is, the solver returns a solution $x \in S_1$ such that $f_1(x) \geq \multApprox \cdot f_1(x^*) - \additiveError$. 
%     Consider the corresponding solution to $OP2$ by the bijection, $B(X) \in S_2$, we know that $f_1(x_1) = f_2(B(x_1))$.
%     It follows that $B(x)$ is an $(\multApprox, \additiveApprox)$-approximation to $OP2$:
%     \begin{align*}
%         f_2(B(x)) = f_1(x) \geq \multApprox \cdot f_1(x^*) - \additiveError = \multApprox \cdot f_2(y^*) - \additiveError
%     \end{align*}
% \end{proof}

Notice that the approximation value is obtained by the corresponding solution ($B(X)$), and therefore, we can also conclude the following result:
% Therefore, if we also have access to procedures to calculate the bijection and its inverse, then we can use a solver for one problem to find the solution to the other, that is:
\begin{corollary}\label{corollary:solver-equivalent-prob}
    Let $OP1 = (S_1,f_1)$ and $OP2 = (S_2,f_2)$ be equivalent optimization problems, and let $P_{1\to 2}$ be a procedure that, given a solution to $OP1$, returns the corresponding solution to $OP2$.
    Then, an $(\multApprox, \additiveApprox)$-approximate solver for $OP1$ can be used to obtain a \emph{solution} that is an $(\multApprox, \additiveApprox)$-approximation for $OP2$.
\end{corollary}

If the procedure from $OP1$ to $OP2$ operates in polynomial time we say that $OP1$ is \emph{polynomial-time equivalent} to $OP2$. 

\eden{how is the name "polynomial-time equivalent"?}

% \eden{If we will have time: "Further, if the bijection is given and can be calculated in polynomial time, then ....}



\subsection{Relationships Between Single-Objective Problems for Leximin Optimization}
\eden{I'm not sure which title to give}

For clarity, descriptions of all the problems are provided here as well (table \ref{table:prob-des}).

\begin{table}[h!]
\begin{tabular}{l}
\hline
\\
$\begin{aligned}
     \text{(P1)}\Hquad \max \quad &\ztVar{x}  \;\;
        s.t. &\quad  & (1) \quad x \in S \\
              &     & & (2) \quad \valBy{\ell}{x}\geq z_{\ell} & \ell = 1,\ldots,t-1\nonumber \\
               &    & & (3) \quad \valBy{t}{x} \geq \ztVar{x} \nonumber  \\\\
    \text{(P2)}\Hquad\max \quad &\ztVar{x}  \;\;
        s.t. &\quad  & (1) \quad x \in S  \\
        &&& (\hat{2}) \quad \sum_{i \in F'} f_i(x) \geq \sum_{i=1}^{|F'|}  z_i & \forall F' \subseteq [n], |F'| < t \\
        &&& (\hat{3}) \quad \sum_{i \in F'} f_i(x) \geq \sum_{i=1}^{t}  z_i  & \forall F' \subseteq [n], |F'| = t\\\\
     \text{(P3)}\Hquad \max \quad &\ztVar{x}  \;\;
        s.t. &\quad  & (1) \quad x \in S  \\
                    &&& (2) \quad \ell y_{\ell} - \sum_{j=1}^n m_{\ell,j}\geq \sum_{i=1}^{\ell}  z_i & \ell = 1, \ldots,t-1 \nonumber \\
                    &&& (3) \quad t y_t - \sum_{j=1}^{n} m_{t,j} \geq \sum_{i=1}^{t}  z_i  \nonumber \\
                    &&& (4) \quad m_{\ell,j} \geq y_{\ell} - f_j(x)  & \ell = 1, \ldots,t,\Hquad j = 1, \ldots,n \nonumber \\
                    &&& (5) \quad m_{\ell,j} \geq 0  & \ell = 1, \ldots,t,\Hquad j = 1, \ldots,n \nonumber\\\\
    \text{(P2-compact)}& \\
    \max \quad &z_t  \;\;
        s.t. &\quad  & (1) \quad x \in S \\
                    &&& (\Tilde{2}) \quad \sum_{i=1}^{\ell} \valBy{i}{x} \geq \sum_{i=1}^{\ell}  z_i & \ell = 1,\ldots, t-1 \nonumber\\
                    &&& (\Tilde{3}) \quad \sum_{i=1}^{t} \valBy{i}{x} \geq \sum_{i=1}^{t}  z_i\\
\end{aligned}$\\
\\
\hline
\end{tabular}
\caption{Summary description of the problems.}
\label{table:prob-des}
\end{table}


\subsubsection{Equivalence of The Problems \eqref{eq:sums-OP} and \eqref{eq:compact-OP}}\label{sec:prob-sums-and-comp}
we prove that the \emph{identity function} is an appropriate bijection between \eqref{eq:sums-OP} and \eqref{eq:compact-OP}. Therefore, they are polynomial-time equivalent to each other. 

We start by proving the following lemma:
\begin{lemma}\label{lemma:sums-to-comp-constrants}
    For any $x \in S$, any $\ell \in [n]$ and a constant $c \in \mathbb{R}$ the following two conditions are equivalent:
    \begin{align}\label{eq:sums-to-comp-constrants}
         \forall F' \subseteq [n], |F'| = \ell \colon \sum_{i \in F'} f_i(x) &\geq c 
         \\
         \sum_{i=1}^{\ell} \valBy{i}{x}&\geq c 
    \end{align}
\end{lemma}

\begin{proof}
    For the first direction, recall that the values $ \valBy{1}{x}, \dots,  \valBy{\ell}{x}$ were obtained from $\ell$ objective functions (those who yield the smallest value).
    By the assumption, the sum of any set of function with size $\ell$ is at least $c$; therefore, it is true in particular for the functions corresponding to the values $ (\valBy{1}{x})_{i=1}^{\ell}$.
    For the second direction, assume that $\sum_{i=1}^{\ell} \valBy{i}{x}\geq c$.
    Since $ \valBy{1}{x}, \dots,  \valBy{\ell}{x}$ are the $\ell$ smallest values in $\allValues{x}$, we get that:
    \begin{align*}
       \forall F' \subseteq [n],\Hquad |F'| = \ell \colon \quad \sum_{i \in F'}f_i(x) \geq \sum_{i=1}^s \valBy{i}{x}\geq c.
    \end{align*}
\end{proof}

    Now, let $(x,z_t)$ be a solution to \eqref{eq:sums-OP}. 
    As $x$ satisfies constraint (1) of \eqref{eq:sums-OP}), it is also satisfies constraint (1) of \eqref{eq:compact-OP} (as both constraints are the same, $x \in S$).
    In addition, as $x$ satisfies constraint $(\hat{2})$ of \eqref{eq:sums-OP}, for any $\ell \in [t-1]$, 
    \begin{align*}
        \forall F' \subseteq [n], |F'| = \ell \colon \sum_{i \in F'} f_i(x) \geq \sum_{i=1}^{\ell} z_i
    \end{align*}
    by Lemma \ref{lemma:sums-to-comp-constrants}, also $\sum_{i=1}^{\ell} \valBy{i}{x} \geq \sum_{i=1}^{\ell} z_i$. Therefore, $x$ satisfies constraint $(\Tilde{2})$ of \eqref{eq:compact-OP}.
    Lastly, as $x$ and $z_t$ satisfy constraint $(\hat{3})$ of \eqref{eq:sums-OP}, 
    \begin{align*}
        \forall F' \subseteq [n], |F'| = t \colon \sum_{i \in F'} f_i(x) \geq \sum_{i=1}^{t} z_i
    \end{align*}
    again, by Lemma \ref{lemma:sums-to-comp-constrants} also $\sum_{i=1}^{t} \valBy{i}{x} \geq \sum_{i=1}^{t} z_i$.
    So, $x$ ans $z_t$   satisfy constraint $(\Tilde{3})$ of \eqref{eq:compact-OP}.
    Since we saw that $x$ ans $z_t$ satisfy all the constraints of \eqref{eq:compact-OP}, it is feasible to this problem.

    As in both problems the objective value is determined by $z_t$, it is clear that $(x,z_t)$ obtains the same objective value from both \eqref{eq:sums-OP} and \eqref{eq:compact-OP}.

    Therefore, the identity function (i.e., $B((x,z_t)) = (x,z_t)$) is an appropriate bijection and so, the problems are equivalent.



%--------------------------------------------------
\subsubsection{Equivalence of The  problems \eqref{eq:compact-OP} and \eqref{eq:vsums-OP}} We prove that these problems are equivalent by describing an appropriate bijection.
We will also see that this bijection and its inverse can be calculated in polynomial time and therefore, each problem is polynomial-time equivalent to the other.

We start with the following lemma:
\begin{lemma}\label{lemma:comp-to-p3-m-sums}
    For any $x \in S$ and any constant $c \in C$,
    \begin{align*}
        \sum_{j=1}^n \max(0, c - f_j(x) ) = \sum_{j=1}^n \max(0, c - \valBy{j}{x} )
    \end{align*}
\end{lemma}
\begin{proof}
     Let $(\pi_1, \ldots, \pi_n)$ be a permutation of $\{1,\ldots,n\}$ such that $f_{\pi_i}(x) = \valBy{i}{x}$ for any $i \in [n]$ (notice that such permutation exists by the definition of $\valBy{}{}$).
     That is, the value that $f_{\pi_i}$ obtains is the ${\pi_i}$-th smallest one in the multiset of all values $\allValues{x}$.
    Since each element in the sum $\sum_{j=1}^n \max(0, c - f_j(x))$ is affected by $j$ only through $f_j(x)$, the permutation $\pi$ allows us to conclude the following:
    \erel{This argument is not clear}\eden{better?}
    \begin{align*}
        \sum_{j=1}^n \max(0, c -f_j(x)) &= \sum_{j=\pi_1}^{\pi_n} \max(0,c -f_j(x)\\ 
        &= \sum_{j=1}^n \max(0,c -f_{\pi_i}(x)  = \sum_{j=1}^{n} \max(0,c -\valBy{j}{x})
    \end{align*}
\end{proof}



Following is Lemma \ref{lemma:comp-to-p3-mapping}, which describes a function $B$ and proves that it is a mapping from the feasible region of the problem \eqref{eq:compact-OP} to the feasible region of the problem \eqref{eq:vsums-OP}.
Then, Lemma \ref{lemma:comp-to-p3-is-bij} proves that this mapping is a bijection.
Lastly, Lemma \ref{lemma:comp-to-p3-obj} shows that the same objective value is obtained.

\begin{lemma}\label{lemma:comp-to-p3-mapping}
    Let $(x,z_t)$ be a feasible solution  to \eqref{eq:compact-OP}. Then $B((x,z_t)) = (x, z_t, (y_1,\ldots,y_n), (m_{1,1},\ldots,m_{n,n}))$ is a feasible solution to \eqref{eq:vsums-OP}, where
    \begin{align*}
        \quad y_{\ell} &:= \valBy{\ell}{x} \Hquad\forall \ell \in [n], 
        \\
        m_{\ell,j} &:= \max(0,y_{\ell} -f_j(x)) \Hquad \forall \ell \in [n], \Hquad \forall 1 \leq j \leq n 
    \end{align*}
\end{lemma}

\begin{proof}
    First, since $x$ satisfies constraint (1) of \eqref{eq:compact-OP}, it is also satisfies constraint (1) of \eqref{eq:vsums-OP} (as both constraints are the same).
    Also, as $m_{\ell,j} \geq 0$ and $m_{\ell,j} \geq y_{\ell} - f_j(x)$ for any $\ell \in [n]$ and $j \in [n]$, this assignment satisfies constraints (4) and (5) of \eqref{eq:vsums-OP}.
    
    To show that this assignment also satisfies constraints (2) and (3) of problem \eqref{eq:vsums-OP}, we first prove that for any $\ell \in [n]$ and any constant $c \in \mathbb{R}$ this assignment satisfies the following:
    \begin{align}\label{eq:comp-to-p3}
        \sum_{i=1}^{\ell} \valBy{i}{x}\geq c \Hquad \Longrightarrow \Hquad \ell y_{\ell} - \sum_{j=1}^n m_{\ell,j}\geq c
    \end{align}
    As $y_{\ell} = \valBy{\ell}{x}$, also  $m_{\ell,j} = \max(0,\valBy{\ell}{x} -f_j(x))$.
    % in this way it is easy to see that $j$ affects $m$ only through $f_j(x)$.
    And so, by Lemma \ref{lemma:comp-to-p3-m-sums}, it can also be described as $\sum_{j=1}^{n} \max(0,\valBy{\ell}{x} -\valBy{j}{x})$.
    Since $\valBy{\ell}{x}$ is the $\ell$-th smallest objective, it is clear that $\valBy{\ell}{x} - \valBy{j}{x} \leq 0$ for any $j > \ell$, and $\valBy{\ell}{x} - \valBy{j}{x} \geq 0$ for any $j \leq \ell$.
    We can now conclude that $\ell y_{\ell} - \sum_{j=1}^n m_{\ell,j}\geq c$:
    \begin{align*}
        &\ell y_{\ell} - \sum_{j=1}^n m_{\ell,j} = \ell \cdot \valBy{\ell}{x} - \sum_{j=1}^n \max(0,\valBy{\ell}{x} -\valBy{j}{x}) \\
        &= \ell \cdot \valBy{\ell}{x} - \sum_{j=1}^{\ell} \max(0,\valBy{\ell}{x} -\valBy{j}{x}) - \sum_{j=\ell+1}^n \max(0,\valBy{\ell}{x} -\valBy{j}{x}) \\
        &= \ell \cdot \valBy{\ell}{x} - \sum_{j=1}^{\ell} \left(\valBy{\ell}{x} -\valBy{j}{x}\right) - \sum_{j=\ell+1}^n 0 = \ell \cdot \valBy{\ell}{x} - \ell \cdot\valBy{\ell}{x} + \sum_{j=1}^{\ell} \valBy{j}{x}\\
        &= \sum_{j=1}^{\ell} \valBy{j}{x} \geq  c \text{~~~by assumption.}
    \end{align*}

    Now, since $x$ satisfies constraint $(\Tilde{2})$ of \eqref{eq:compact-OP}, for any $\ell \in [t-1]$, $\sum_{i=1}^{\ell} \valBy{i}{x} \geq \sum_{i=1}^{\ell} z_i$ and so by equation \ref{eq:comp-to-p3}, $\ell y_{\ell} - \sum_{j=1}^n m_{\ell,j}\geq  \sum_{i=1}^{\ell} z_i$
    Therefore, this assignment constraint (2) of problem \eqref{eq:vsums-OP}.
    In addition, as $x$ and $z_t$ satisfy constraint $(\Tilde{3})$ of \eqref{eq:compact-OP}, $\sum_{i=1}^{t} \valBy{i}{x} \geq \sum_{i=1}^{t} z_i$ and so by equation \ref{eq:comp-to-p3}, \ref{eq:comp-to-p3}, $t y_{t} - \sum_{j=1}^n m_{t,j}\geq  \sum_{i=1}^{t} z_i$.
    This means that also satisfies constraints (3) of problem \eqref{eq:vsums-OP}.
\end{proof}

\begin{lemma}\label{lemma:comp-to-p3-is-bij}
    The mapping $B$ is a bijection.
\end{lemma}

\begin{proof}
    Injective ($B(a) = B(b) \Rightarrow a = b$) is trivial since $x$ and $z_t$ are part of the solution.
    
    To prove that the mapping is surjective, we will show that for any feasible solution to \eqref{eq:vsums-OP}, that is,
    \begin{align*}
        (x \in S, z_t, y_1, \ldots, y_t, m_{1,1}, \ldots, m_{1,n}, m_{2,1}, \ldots, m_{2,n},\ldots, m_{t,1}, \ldots, m_{t,n})
    \end{align*}
    there is a feasible solution to \eqref{eq:compact-OP} that is  mapped to this solution.
    In fact, we prove that $(x,z_t)$ does.

    It is easy to see that since $x$ satisfies constraint (1) of \eqref{eq:vsums-OP}, it is also satisfies constraint (1) of \eqref{eq:compact-OP} (as both are the same).
    To show that it also satisfies constraints $(\Tilde{2})$ and $(\Tilde{3})$ of \eqref{eq:compact-OP}, we start by proving that for any $\ell \in [n]$ and any constant $c \in \mathbb{R}$:
    \begin{align}\label{eq:p3-to-comp}
         \ell y_{\ell} - \sum_{j=1}^n m_{\ell,j}\geq c
         \Hquad \Longrightarrow \Hquad \sum_{i=1}^{\ell} \valBy{i}{x}\geq c
    \end{align}
    Notice that, for any $j\in [n]$ and any $\ell \in [n]$, $m_{\ell,j} \geq y_{\ell} - f_j(x)$ by constraint (4) of \eqref{eq:vsums-OP}, and also $m_{\ell,j} \geq 0$ by constraint (5) of \eqref{eq:vsums-OP}.
    Therefore, $m_{\ell,j} \geq \max(0,y_{\ell} -f_j(x))$.
    And so, by Lemma \ref{lemma:comp-to-p3-m-sums}:
    \begin{align}\label{eq:p3-to-conp-m-sum}
        \sum_{j=1}^n m_{\ell,j} \geq  \sum_{j=1}^n \max(0,y_{\ell} -f_j(x)) = \sum_{j=1}^n \max(0,y_{\ell} -\valBy{j}{x})
    \end{align}
    Now, suppose by contradiction that $\ell y_{\ell} - \sum_{j=1}^n m_{\ell,j}\geq c$ but at the same time $\sum_{i=1}^{\ell} \valBy{i}{x}< c$ (equation \ref{eq:p3-to-comp}).
    Since $\ell y_{\ell} - \sum_{j=1}^n m_{\ell,j}\geq c$, by equation \ref{eq:p3-to-conp-m-sum} also:
    \begin{align*}
        c \leq \ell y_{\ell} - \sum_{j=1}^n m_{\ell,j} \leq \ell y_{\ell} - \sum_{j=1}^n \max(0,y_{\ell} -\valBy{j}{x})
    \end{align*}
    But, as $\sum_{i=1}^{\ell} \valBy{i}{x}< c$ this lead to contradiction:
\begin{align*}
       &\sum_{i=1}^{\ell} \valBy{i}{x} < c \leq \ell y_{\ell} - \sum_{j=1}^n \max(0,y_{\ell} -\valBy{j}{x})\\
       \Rightarrow \Hquad & \ell y_{\ell} - \sum_{i=1}^{\ell} \valBy{i}{x} - \sum_{j=1}^n \max(0,y_{\ell} -\valBy{j} {x}) > 0\\
       \Rightarrow \Hquad & \sum_{i=1}^{\ell} y_{\ell} - \sum_{i=1}^{\ell} \valBy{i}{x} - \sum_{j=1}^n \max(0,y_{\ell} -\valBy{j} {x}) > 0\\
       \Rightarrow \Hquad & \sum_{i=1}^{\ell}\left( y_{\ell} - \valBy{i}{x} \right) - \sum_{j=1}^{\ell} \max(0,y_{\ell} -\valBy{j} {x}) - \sum_{j=\ell+1}^n \max(0,y_{\ell} -\valBy{j} {x}) > 0\\
        \Rightarrow \Hquad &  \sum_{j=1}^{\ell} \underbrace{\left((y_{\ell} - \valBy{j}{x}) - \max(0,y_{\ell} -\valBy{j}{x})\right)}_{\text{each element } \leq 0} - \sum_{j=\ell+1}^n \underbrace{\max(0,y_{\ell} -\valBy{j}{x})}_{\text{each element } \geq 0} >  0\\
     \Rightarrow \Hquad & 0 > 0
   \end{align*}

    Now, as constraint (2) of problem \eqref{eq:vsums-OP} is satisfied, for any $\ell \in [t-1]$,  $\ell y_{\ell} - \sum_{j=1}^n m_{\ell,j}\geq  \sum_{i=1}^{\ell} z_i$, and so by equation \ref{eq:p3-to-comp}, also $\sum_{i=1}^{\ell} \valBy{i}{x} \geq \sum_{i=1}^{\ell} z_i$.
    This implies that $x$ satisfies constraint $(\Tilde{2})$ of \eqref{eq:compact-OP}.
    Similarly, as constraint (3) of problem \eqref{eq:vsums-OP} is satisfied,  $t y_{t} - \sum_{j=1}^n m_{t,j}\geq  \sum_{i=1}^{t} z_i$, and so by equation \ref{eq:p3-to-comp}, also $\sum_{i=1}^{t} \valBy{i}{x} \geq \sum_{i=1}^{t} z_i$.
    This implies that $x$ and $z_t$ satisfy constraint $(\Tilde{3})$ of \eqref{eq:compact-OP}.
\end{proof}


\begin{lemma}\label{lemma:comp-to-p3-obj}
    $(x,z_t)$ and $B((x,z_t))$ obtain the same objective value from the problems \eqref{eq:compact-OP} and \eqref{eq:vsums-OP} respectively.
\end{lemma}

\begin{proof}
    As in both problems the objective value is determined by $z_t$, by the definition of $B$ (the variable $z_t$ is mapped to itself), it is clear that $(x,z_t)$ and $B((x,z_t))$ obtains the same objective value from \eqref{eq:compact-OP} and \eqref{eq:vsums-OP} respectively.
\end{proof}


%--------------------------------------------------
\subsubsection{Relationship Between the Problems \eqref{eq:basic-OP} and \eqref{eq:sums-OP}} 
% Both problems are depended on a set of constants $z_1, \ldots, z_{t-1}$, 
We shall now prove Lemma \ref{lemma:alg-1-can-use-sums-exact} (Section \ref{sec:algo-short}), which says that in Algorithm \ref{alg:basic-ordered-Outcomes}, a solver for \eqref{eq:sums-OP} can be used (instead of for \eqref{eq:basic-OP}), and the algorithm will still output a leximin optimal solution.

\begin{proof}[Proof of Lemma \ref{lemma:alg-1-can-use-sums-exact}]
    Contrariwise, suppose that the returned solution, $x^*$, is not leximin optimal.
    This means that there exists a solution, $y \in S$, that leximin-preferred over it.
    That is, there exists an integer $k \in [n]$ such that:
    \begin{align*}
        \forall j < k \colon &\valBy{j}{y} = \valBy{j}{\retSol};\\
        & \valBy{k}{y} > \valBy{k}{\retSol}.
    \end{align*}
    In addition, since $x^*$ is the returned solution, it is the solution of \eqref{eq:sums-OP} that was solved in the last iteration and therefore $\sum_{i=1}^{s} \valBy{i}{s} \geq \sum_{i=1}^{s} z_i$ for any $s \in [n]$ (by constraint  $(\hat{2})$ for $s<n$ and constraint  $(\hat{3})$ for $s=n$).
    Now, consider \eqref{eq:sums-OP} that was solved in iteration $t$.
    Since $y$ is a solution ($y \in S$) it satisfies constraint (1).
    It is also easy to see that $y$ satisfies constraint $(\hat{2})$ --- for any $s \in [k-1]$:
    \begin{align*}
        &\sum_{i=1}^s \valBy{i}{y} = \sum_{i=1}^s \valBy{i}{\retSol}
        && \text{since } i\leq s<k \text{ and $y$'s def.}\\
        & \geq \sum_{i=1}^s z_i
    \end{align*}
    Moreover, since $z_t$ is a variable in this problem, it satisfies constraint $(\hat{3})$ with any $z_t \geq \sum_{i=1}^t \valBy{i}{y} - \sum_{i=1}^{t-1} z_i$.
    Therefore, it is feasible to this problem. 
    But, the objective value obtained by $y$ is higher than the optimal value, $z_t$, which is a contradiction:
    \begin{align*}
        \sum_{i=1}^t \valBy{i}{y} - \sum_{i=1}^{t-1} z_i > \sum_{i=1}^t \valBy{i}{x^*} - \sum_{i=1}^{t-1} z_i \geq \sum_{i=1}^t z_t - \sum_{i=1}^{t-1} z_i = z_t
    \end{align*}
\end{proof}
% We start by proving that, for $t \in [n]$, when the constants $z_1, \ldots, z_{t-1}$ represent the optimal values of \eqref{eq:basic-OP} in iterations $1, \ldots t$ respectively, the programs \eqref{eq:basic-OP} and \eqref{eq:sums-OP} are equivalent.

\eden{alternative: is this better?
\begin{proof}
    In Section \ref{sec:algo-sec-proofs}, it was proven that if Algorithm \ref{alg:basic-ordered-Outcomes} uses an $(\multApprox, \additiveApprox)$-approximate solver for \eqref{eq:compact-OP} as \textsf{OP}, then the returned solution is an $(\multApprox, \additiveApprox)$-approximation to leximin. 
    This means that, given an exact solver to \eqref{eq:compact-OP}, the algorithm will output a leximin optimal solution.
    However, we saw that \eqref{eq:sums-OP} and \eqref{eq:compact-OP} are equivalent and that the identity function is an appropriate bijection (Section \ref{sec:prob-sums-and-comp}).
    Therefore, in each iteration, a solver for \eqref{eq:sums-OP} will output the same solution and the same result will be obtained.
\end{proof}
}

\eden{in the next version we can also prove it in a maybe more interesting way.. that when the constants $z_1, \ldots, z_{t-1}$ represent the optimal values of \eqref{eq:basic-OP} the programs are equivalent}

\section{Ellipsoid Method Variant for Approximation}\label{sec:mult-variant-ellipsoid}
This Appendix describes a variant of the ellipsoid method that can be used to approximate  LPs that cannot be solved directly due to a large number of variables.
% It requires an approximate separation oracle for the dual program.
The method combines techniques presented in \cite{grotschel_geometric_1993,grotschel_ellipsoid_1981,karmarkar_efficient_1982}.

\subsection{Using Approximate Separation Oracles (multiple error)}
Our goal is to solve the following linear program (the primal):
\begin{align}
\tag{P}
\begin{split}
\min \quad &c^T \cdot x \\
s.t. \quad &A \cdot x \geq b, \quad x\geq 0;
\end{split}
\end{align}
We assume that (P) has a small number of constraints, but may have a huge number of variables, so we cannot solve (P) directly. We consider its \emph{dual}:
\begin{align}
\tag{D}
\begin{split}
\max \quad & b^T \cdot y \\
s.t. \quad &A^T \cdot y \leq c,\quad y\geq 0.
\end{split}
\end{align}
Assume that both problems have optimal solutions and denote the optimal solutions of (P) and (D) by $x^{*}$ and $y^{*}$ respectively. By the strong duality theorem:
\begin{align}
    c^T \cdot x^{*} = b^T \cdot y^{*}
\end{align}

While (D) has a small number of variables, it has a huge number of constraints, so
we cannot solve it directly either. 
In this Appendix, we show that it can be approximately using the following tool:

\begin{definition}
An \emph{approximate separation oracle} with multiplicative error (MASO) for the dual LP is an efficient function parameterized by a constant $\multError \geq 0$.
Given a vector $y$  it returns one of the following two answers:
\begin{enumerate}
\item "$y$ is infeasible". In this case, is returns a violated constraint, that is, a row $a_i^T \in A^T$ such that $a_i^T  y > c_i$.
\item "$y$ is \emph{approximately feasible}". 
That means that $A^T y \leq (1+\multError) \cdot c$
\end{enumerate}

\end{definition}
Given the MASO, we apply the ellipsoid method as follows (this is just a sketch
to illustrate the way we use the MASO; it omits some technical details):
\begin{itemize}
    \item Let $E_0$ be a large ellipsoid, that contains the entire feasible region, that is, all $y \geq 0$ for which $A^T y \leq c$.

    \item For $k = 0,1,\dots, K$ (where $K$ is a fixed constant, as will be explained later):
    \begin{itemize}
        \item Let $y_k$ be the centroid of ellipsoid $E_k$.
        
        \item Run the MASO on $y_k$.
        
        \item If the MASO returns "$y_k$ is infeasible" and a violated constraint $a_i^T$, then make a \emph{feasibility cut} --- keep in $E_{k+1}$ only those $y \in E_k$ for which $a_i^T y \leq c_i$.
        
        \item If the MASO returns "$y$ is approximately feasible", then make an \emph{optimality cut} --- keep in $E_{k+1}$ only those $y \in E_k$ for which $b^T y \geq b^T y_k$.
    \end{itemize}
    
    \item From the set $y_0, y_1, \dots, y_K$, choose the point with the highest $b^T \cdot y_k$ among all the approximately-feasible points.
\end{itemize}
Since both cuts are through the center of the ellipsoid, the ellipsoid dilates by a factor of at least $\frac{1}{r}$ at each iteration, where $r > 1$ is some constant (see \cite{grotschel_ellipsoid_1981} for computation of $r$). Therefore, by choosing $K := \log_2 r \cdot L$, where $L$ is the
number of bits in the binary representation of the input, the last ellipsoid $E_K$ is so small that all points in it can be considered equal (up to the accuracy of the binary representation).


The solution $y'$ returned by the above algorithm satisfies the following two conditions:
\begin{equation} \label{mult:y-star-is-approximetly-feasible}
     A^T y' \leq (1+\multError)\cdot c
\end{equation}
\begin{equation} \label{mult:y-star-obj-geq-opt}
     b^T y' \geq b^T y^{*}
\end{equation}
Inequality \ref{mult:y-star-is-approximetly-feasible} holds since, by definition, $y'$ is approximately-feasible.

To prove \ref{mult:y-star-obj-geq-opt}, suppose by contradiction that $b^T y^{*} > b^T y'$. 
Since $y^{*}$ is feasible for (D), it is in the initial ellipsoid. 
It remains in the ellipsoid throughout the algorithm: it is removed neither by a feasibility cut (since it is
feasible), nor by an optimality cut (since its value is at least as large as all values used for optimality cuts).
Therefore, it remains in the final ellipsoid, and it is chosen as the highest-valued feasible point rather than $y'$ --- a contradiction.

Now, we construct a reduced version of (D), where there are only at most $K$ constraints --- only the constraints used to make feasibility cuts.
Denote the reduced constraints by $A_{red}^T \cdot y \leq c_{red}$, where $A_{red}^T$ is a matrix containing a subset of at most $K$ rows of of $A^T$, and $c_{red}$ is a vector containing the corresponding subset of the elements of $c$. The reduced-dual LP is:
\begin{equation}
\tag{RD}
\begin{split}
\max  \quad & b^T y \\
s.t. \quad & A_{red}^T \cdot y \leq c_{red}, \quad y\geq 0
\end{split}
\end{equation}
Notice that it has the same number of variables as the program (D). Further, if we had run this ellipsoid method variant on (RD) (instead of (D)), then the result would have been exactly the same --- $y'$.
Therefore, (\ref{mult:y-star-obj-geq-opt}) holds for the (RD) too:
\begin{equation} \label{mult:y-star-to-y-redopt}
    b^T y' \geq b^T y^{*}_{red}
\end{equation}
where $y^{*}_{red}$ is the optimal value of (RD).


As $A_{red}^T$ contains a subset of at most $K$ rows of $A^T$, the matrix $A_{red}$ contains a subset of \emph{columns} of $A$.
Therefore, the dual of (RD) has only at most $K$ variables, which are those who correspond to the remaining columns of $A$:
\begin{equation}
	\tag{RP}
    \begin{split}
     \min \quad &c_{red}^T \cdot x_{red} \\
            s.t. \quad &A_{red} \cdot x_{red} \geq b, \quad x_{red}\geq 0
    \end{split}
\end{equation}
%  reduced-primal
%\er{Note that $A_{red}$ is a matrix with the same number of rows as $A$, but only at most $K$ columns.}
Since (RP) has a polynomial number of variables  and constraints, it can be solved exactly by any LP solver (not necessarily the ellipsoid method).
Denote the optimal solution by $x^{*}_{red}$. 

Let $x'$ be a vector which describes an assignment to the variables of (P), in which all variables that exist in (RP) have the same value as in $x^{*}_{red}$, and all other variables are set to $0$.
It follows that $A \cdot x' = A_{red} \cdot x^{*}_{red}$, therefore, since $x^{*}_{red}$ is feasible to RD, also $x'$ is a feasible solution to (P).
\erel{
In second reading, I think this should be made more formal.
Let $x'$ be a solution to (P), in which all variables that exist in (RP) have the same value as in $x^{*}_{red}$, and all other variables are set to 0.
We have to prove that 
(1) $x'$ is feasible for (P);
(2) $c^T x' \leq (1+\epsilon)\cdot c^T\cdot x^{*}$.
}
\eden{better?}
Similarly, $c^T \cdot x' = c^T_{red} \cdot x^{*}_{red}$.
We shall now see that this implies that the objective obtained by $x'$ approximates the objective obtained by $x^{*}$:
\begin{align*} 
&c^T \cdot x' = c^T_{red} \cdot x^{*}_{red} \\
&=  b^T \cdot y^{*}_{red} & \text{(by strong duality for the reduced LPs)} \\
                     &\leq  b^T\cdot y' & \text{(By (\ref{mult:y-star-to-y-redopt}))}\\
                     &\leq  (A \cdot x^{*})^T y' & \text{(definition of (P))} \\
                     &=  (x^{*})^T (A^T\cdot y') & \text{(properties of transpose and associativity of multiplication)} \\
                     &\leq  (x^{*})^T ((1+\multError)\cdot c) & \text{(by \ref{mult:y-star-is-approximetly-feasible})} \\
                     & = (1+\multError) \cdot (c^T x^{*}) & \text{(properties of transpose)}
\end{align*}
So, $x'$ ($x^{*}_{red}$ with all missing variables set to $0$) is an approximate solution to the primal LP (P) --- as required.

\subsection{Using Half-Randomized Approximate Separation Oracles}
Here, we allow the oracle to also be \emph{half-randomized}, that is, when it says that a solution is infeasible, it is always correct; however, when it says that a solution is approximately feasible, it is only correct with some probability $p \in [0,1]$.

Since the ellipsoid method variant is iterative, and since the oracle calls are independent, there is a probability $p^T$ that the oracle answers correctly in each iteration, and so, the overall process performs as before. 
We shall now explain why, using a half-randomized oracle, this ellipsoid method variant \emph{always} returns a feasible solution to the primal (even if the oracle was incorrect).

First, notice that the oracle is always correct when it determines that a solution is infeasible.
In addition, the construction of RD is only depended by these set of constraints.
Therefore, by the same arguments, $x'$ would still be a feasible solution to P (but not necessarily with an approximately-optimal objective value).

This means that given a half-randomized approximate separation oracle for the dual with error $\multError$ and success probability $p$, this ellipsoid method variant can be used as a randomized approximation algorithm for the primal with the same error and success probability $p^I$ (where $I$ is an upper bound on the number of iteration of the method on the given input). 
% \section{Saturation Algorithm}\label{sec:saturation-algorithm}
The following algorithm was independently proposed by different researchers for different problems \cite{willson,airiau_portioning_2019,nace_max-min_2008}.
% --- by Willson \cite{willson} for the  problem of fair allocation of divisible items, Airiau et al. \cite{airiau_portioning_2019} for the problem of portioning with ordinal preferences, Bei at el. \cite{bei_truthful_2022} for a variant of cake cutting that they called cake sharing and Nace and Pioro \cite{nace_max-min_2008} for multi-commodity flow problem 
But it can be generalized to capture the following case:
\begin{enumerate}
    \item The feasible region $S$ is \textit{convex}: for any two solutions $x, y \in S$ and for any $\lambda \in [0,1]$, the convex combination of $x$ and $y$ in relation to $\lambda$ is also a solution:
    \begin{align*}
        \forall x, y \in S, \quad \forall \lambda \in [0,1] \colon \quad  
        \bigl(\lambda x + (1-\lambda)y\bigl)\in S
    \end{align*}

    \item The size of the feasible region $S$ is polynomial with respect to $n$.
    % \eden{To myself: to check if this is accurate: i.e., it can be described with a number of variables and constraints that is polynomial to $n$.}

    \item The objective-functions are \textit{additive}: let $x,y,z \in S$ be solutions for which $\alpha,\beta \in \mathbb{R}$ exist such that $z = \alpha x + \beta y$, then for each objective-function $f_i \in \allObjFunc$:
    \begin{align*}
        f_i(z) &= f_i(\alpha x + \beta y) =\\
        &= \alpha f_i(x) + \beta f_i(y)
    \end{align*}
    \erel{ 
    For this condition, we must say that the solutions are vectors (we did not say this so far). Otherwise there is no meaning to adding or multiplying by scalars.
    }

    \item The objective-functions are \textit{concave}: for any objective-function $f_i \in \allObjFunc$ the set $\{f(x) \mid x \in S\}$ is concave (equivalently, the set $\{-f(x) \mid x \in S\}$ is convex). 

    \item There is a black-box for finding  the \textit{next maximin} value (denote by $OP1$): given a subset of objective-functions ($\mathcal{A}\subset \allObjFunc$) for which lower bounds have been set ($\forall f_i \in \mathcal{A} \colon z_i \in \mathbb{R}$), finds the highest value that all other objective functions can achieve simultaneously:
    \begin{align*}
        \max \quad &z\\
        s.t. \quad  & x \in S\\
                    & f_i(x) = z_i   & f_i \in \mathcal{A}\\
                    & f_i(x) \geq z   & f_i \notin \mathcal{A}
    \end{align*}

    \item There is a black-box for solving a saturation test (denote by $OP2$):
    % \eden{I think we should name this process, but I'm not sure if it is the best name...}: 
    For each objective-function $f_k \in \allObjFunc$, a single-objective optimization version of the problem with lower bounds on the values of the other objectives ($\forall f_i \in \mathcal{A} \colon z_i \in \mathbb{R}$ and $z \in \mathbb{R}$):
    \begin{align*}
    \max \quad &f_i(x)\\
    s.t. \quad  & x \in S\\
                    & f_i(x) = z_i   & f_i \in \mathcal{A}\\
                    & f_i(x) \geq z   & f_i \notin \mathcal{A}
    \end{align*}
\end{enumerate}
The algorithm is described in detail (in our terms and notations) in Algorithm \ref{alg:willson-leximin}. 


\begin{algorithm}[!htbp]
\caption{Saturation Algorithm--- for finding the Leximin optimal solution}
\label{alg:willson-leximin}
% \textbf{Input}: A black-box for OP1 and a black-box for OP2\\
% \textbf{Output}: The Lexical optimal solution
\begin{algorithmic}[1] %[1] enables line numbers 
\STATE Initialize the set of \textit{saturated} objective-functions $\mathcal{A} = \{\}$ and initialize $t=0$ (a step counter).

\STATE increase $t$ ($t = t+1$).

\STATE Use the black-box for $OP1$ to solve the following  problem, where the variables are $x$ (a vector) and $v$ (a scalar): 
\begin{align*}
\max \quad &v\\
        s.t. \quad  & x \in S\\
                    & f_i(x) \geq z_i   & f_i \in \mathcal{A}\\
                    & f_i(x) \geq v   & f_i \notin \mathcal{A}
\end{align*}
Let $x_t$ and $v_t$ be the optimal solution. 
    
\FOR{$f_k \notin \mathcal{A}$}
    \STATE Use the black-box for $OP2$ to solve the following problem, where the variables are $x$ (a vector) and $v$ (a scalar):
    \begin{align*}
    \max \quad & v\\
            s.t. \quad  & x \in S\\
                        & f_i(x) \geq z_i   & f_i \in \mathcal{A}\\
                        & f_i(x) \geq v_t   & f_i \notin \mathcal{A}\\
                        & f_k(x) \geq v
    \end{align*}
    Let $x_t^k$ and $v_t^k$ be the optimal solution. 

    \STATE \textbf{if} $v_t^k = v_t$ \textbf{then} set $f_k$ as \textit{saturated}: add it to $\mathcal{A}$ ($\mathcal{A} = \mathcal{A} \cup \{f_k\}$) and set its value to $v_t$ ($z_k = v_t$).
    % \IF{$z_{max}^k = z_{max}$}
        % \STATE Set $f_k$ as saturated: add it to $\mathcal{A}$ ($\mathcal{A} = \mathcal{A} \cup \{f_k\}$) and set its value to $z_{max}$ ($z_k = z_{max}$).
    % \ENDIF
\ENDFOR
\STATE \textbf{if} $|\mathcal{A}| = n$ \textbf{then} return $x_t$ \textbf{else} Goto line 2.
% \IF{$|\mathcal{A}| = n$}
    % \STATE return $x$ \eden{To myself: the return part of all algorithms should be explain better}
% \ENDIF
\end{algorithmic}
\end{algorithm}

The algorithm keeps a set of objective-functions that are saturated ($\mathcal{A}$) and lower bounds on their values ($\forall f_i \in \mathcal{A} \colon z_i \in \mathbb{R}$). 
The set is initially empty. 
At each iteration, at least one function becomes saturated and its lower bound is set.
When all functions become saturated, the algorithm terminates.
Each iteration of the algorithm can be divided into two parts.
In the first part, the first black-box is used to find the \textit{next max-min} value, which is the maximum value that all functions outside of $\mathcal{A}$ can achieve at the same time, given that all functions within $\mathcal{A}$ achieve their lower bounds.
In the second part, \textit{a saturation test} is made.
% \eden{I think we should name this process, but I'm not sure if it is the best name...}
For every function not in $\mathcal{A}$, the second black-box is used to find the maximum value of this function when all saturated functions ($f_i \in \mathcal{A}$) achieve their lower bounds and all other functions (outside of $\mathcal{A}$) achieve at least the max-min value from the first part.
This value is used to determine if this objective function is saturated, that is, if its maximal value from the saturation test is equal to the max-min value obtained in the first fart.
If so, we add it to the set of saturated objective-functions ($\mathcal{A}$) and set its lower bound to this value.

% \section{Additive Variant}\label{sec:additive}

\begin{theorem}\label{thm:leximin-approx-alg-leximin-opt}
    Let $\epsilon \in [0,1]$ and \textsf{OP} be a procedure that outputs a $\epsilon$ \emph{additive} approximation to \eqref{eq:vsums-OP}. Then Algorithm \ref{alg:basic-ordered-Outcomes} outputs a $\epsilon$ additive-approximate Leximin solution.  
\end{theorem}

\begin{proof}
    Contrariwise, suppose that $\retSol$ is not an $\epsilon$-approximately Leximin-optimal solution.
    This means that there exists a solution $y$ that is $\epsilon$-preferred over $\retSol$.
    That is, there exists an integer $k \in [n]$ such that:
    \begin{align*}
        \forall j < k \colon &\valBy{j}{y} \geq \valBy{j}{\retSol};\\
        & \valBy{k}{y} > \valBy{k}{\retSol} + \epsilon.
    \end{align*}
    We get that for all $s \in [k-1]$:
    \begin{align*}
         &\sum_{i=1}^s \valBy{i}{y} \geq \sum_{i=1}^s \valBy{i}{\retSol}
        && \text{since } i\leq s<k \text{ and $y$'s def.}\\
        & \geq \sum_{i=1}^s z_i && \text{constraint (2) for $t=n$.}
    \end{align*}
    Therefore, $y$ is a solution to the OP that was solved when $t = k$.
    \erel{You proved that $y$ satisfies constraint (2), but what about constraint (3)?}
    \eden{I'm not sure how to explain that constraint (3) is not a \textbf{standard} constraint. It determines the objective value $z$, so although it is not always optimal, it is always valid.}
    
    In addition, either $k<n$ or $k=n$. 
    If $k<n$ then constraint (2) for $t=n$ says that:
    \begin{align}\label{equ:approx-sum-k-geq-z-1}
        \sum_{i=1}^k \valBy{i}{\retSol} \geq \sum_{i=1}^k z_i
    \end{align}
    If $k=n$ then since $z=z_n$ constraint (3) says it.
    In both cases, we know that equation \ref{equ:approx-sum-k-geq-z-1} holds.
    
    And so, we get that:
    \begin{align*}
         \sum_{i=1}^k \valBy{i}{y} &= \sum_{i=1}^{k-1} \valBy{i}{y} + \valBy{k}{y}\\
         &  \geq\sum_{i=1}^{k-1}\valBy{i}{\retSol} + \valBy{k}{y} &&  \text{since } i \leq k-1 < k \text{ and $y$'s def.}\\
        & > \sum_{i=1}^{k-1}\valBy{i}{\retSol} + \valBy{k}{\retSol} + \epsilon &&  \text{$y$'s def. for } k
        \\
        & = \sum_{i=1}^{k}\valBy{i}{\retSol} + \epsilon \\
        & \geq \sum_{i=1}^{k} z_i + \epsilon &&  \text{equation } \ref{equ:approx-sum-k-geq-z-1}
    \end{align*}
    Which simply means that:
    \begin{align}\label{equ:sum-y-geq-sum-z-plus-eps}
         \sum_{i=1}^k \valBy{i}{y} > \sum_{i=1}^{k} z_i +\epsilon
    \end{align}
    That means that the $z$ achieved by the solution $y$ in the OP that was solved when $t = k$ is strictly more than the value we achieved $z_k$ plus $\epsilon$:
    \begin{align*}
        &\sum_{i=1}^k \valBy{i}{y} - \sum_{i=1}^{k-1} z_i && \text{insulated } z\\
        &> \sum_{i=1}^{k} z_i + \epsilon - \sum_{i=1}^{k-1} z_i  && \text{equation } \ref{equ:sum-y-geq-sum-z-plus-eps} \\
        &= z_k + \epsilon
    \end{align*}
    But we know that the error in this OP is at most $\epsilon$ --- a contradiction.
\end{proof}


\bibliographystyle{IEEEtran}
\bibliography{references}

\begin{IEEEbiography}[{\includegraphics[width=1in,height=1.25in,clip,keepaspectratio]{figures/bios/foto_maria}}]{Mar\'ia Leyva-Vallina} received her BSc in Software Engineering from the University of Oviedo in 2015 and her MSc in Artificial Intelligence from the Polytechnic University of Catalonia in 2017. She is currently pursuing her PhD with the Intelligent Systems group in the University of Groningen. Her main research interests are in representation learning for computer vision.\end{IEEEbiography}

% if you will not have a photo at all:
\begin{IEEEbiography}[{\includegraphics[width=1in,height=1.25in,clip,keepaspectratio]{figures/bios/nicola}}]{Nicola Strisciuglio}
received the Ph.D. degree (cum laude) in Computer Science from the University of Groningen (Netherlands) and the Ph.D. degree in Information
Engineering from the University of Salerno (Italy).  
He is currently an Assistant Professor at the Faculty of Electrical Engineering, Mathematics and Computer Science, University of Twente (Netherlands). He has been the
General Co-Chair of the 1st, 2nd and 3rd International Conference on Applications of Intelligent Systems (APPIS). His research interests include machine learning, signal processing and computer vision.
\end{IEEEbiography}

% insert where needed to balance the two columns on the last page with
% biographies
%\newpage

\begin{IEEEbiography}[{\includegraphics[width=1in,height=1.25in,clip,keepaspectratio]{figures/bios/petkov}}]{Nicolai Petkov} received the Dr. sc. techn. degree in computer engineering (informationstechnik) from the Dresden University of Technology, Dresden, Germany. Since 1991, he has been a Professor of computer science and the Head of the Intelligent Systems Group, University of Groningen. He has authored two monographs, and has authored or co-authored over 150 scientific papers. He holds four patents. His current research interests include pattern recognition, machine learning, data analytics, and brain-inspired computing, with applications in various areas. He is a member of the editorial boards of several journals.

\end{IEEEbiography}


\EOD

\end{document}