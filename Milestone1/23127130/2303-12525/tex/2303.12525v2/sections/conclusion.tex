\section{Conclusions and research challenges}
\label{sec:conclusion_challenges}

In summary, this paper provided a comprehensive overview of \gls{HMD} field, with a detailed analysis of hardware-based detection, harnessing the power of \glspl{HPC} and \gls{ml}. The advantages of \gls{HMD} include resilience to malware subverting the protection mechanism, adaptability to code variants and unknown malware, low complexity and overhead, potential for run-time detection, and cost reduction.

However, challenges persist in \gls{HMD}. The detection accuracy is the most significant challenge as classifiers have a statistical nature. Thus, their results are not deterministic, and ongoing research aims to minimize errors by exploring complex classifiers. In cases where high accuracy is unattainable, a potential solution combines software and hardware-based detectors concurrently, with hardware as the primary defense.
Moreover, ensuring consistency, accuracy, and standardization of hardware monitoring units (including \glspl{HPC}) is crucial for trustworthiness. Chip manufacturers can contribute by designing appropriate modules and providing comprehensive documentation. The limited number of \glspl{HPC} in mobile and \gls{iot} devices poses a feasibility challenge for this approach in these domains. Addressing these challenges will contribute to the continued advancement and effectiveness of \gls{HMD}.