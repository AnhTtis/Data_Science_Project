\section{Introduction}
\label{sec:introduction}

\PARstart{M}{alware}, short for malicious software, poses a significant threat to computer security. It includes any code modification within a software system aimed at causing harm or disrupting the system's intended function \cite{McGraw_Morrisett_2000, Alsmadi:2021aa}. Malware attacks cover spying, intrusive ads, email abuse, system damage, ransom demands, data release, slowdown, browser manipulation, and unauthorized access to sensitive information. Successful attacks lead to consequences that can be categorized into four groups: (i) unauthorized disclosure, where an authorized entity gains access to data; (ii) deception, where an authorized entity receives false data; (iii) disruption, causing interruptions in system services; and (iv) usurpation, resulting in unauthorized control of system services \cite{Stallings_Brown_2014}.
%The problem represented by the malware was first analyzed in the scientific community by Fred Cohen in 1987 \cite{Cohen_1987}, coinciding with the increase in data sharing due to the developing computer networks. In an allusion to biological processes, the paper discussed the primary attack mechanisms and the protection possibilities regarding computer viruses.
Computing systems, including personal computers, mobile phones, \gls{iot}, 5G devices, \glspl{CPS}, and enterprise-wide systems, are vulnerable to malware. The complexity and size of modern systems, often indicated by a rising number of lines of code, amplify the threat. Factors such as numerous bugs, unsafe programming languages, improper configuration, and the ease of concealing malicious code create potential vulnerabilities. Additionally, the increased network connectivity expands the security risks, making all devices potential targets for attackers. For example, cybercrimes have seen a 70\% increase in online fraud accomplished through mobile platforms, with a 30\% rise in \gls{iot} malware in 2020 \cite{SonicWall_2020}.

Globally, cybersecurity is paramount, with malware being a primary vehicle for cybercrimes. The World Economic Forum Global Risk Report 2023 ranks cyber insecurity eighth among top global risks, alongside threats like climate change and involuntary migration \cite{GlobalRisckReport2023}. Cybersecurity Ventures predicts a 15 percent annual growth in international cybercrime costs, reaching USD 8 trillion in 2023 and USD 10.5 trillion annually by 2025 \cite{GlobalRisckReport2023}. Global spending on cybersecurity products and services is expected to exceed USD 1.75 trillion from 2021 to 2025, growing 15 percent year-over-year \cite{cybersecurity_almanac_2023}. Ransomware, a prevalent malware threat, was predicted to cost USD 20 billion globally in 2021, with damage costs projected to exceed USD 265 billion annually by 2031 \cite{GlobalRisckReport2023}. 

Researchers have developed various malware detection methods in response to these alarming statistics, leveraging \gls{ml} and \gls{dl} techniques. Surveys have evaluated and categorized research in this domain, focusing on specific \glspl{OS}, such as Windows, or mobile platforms like Android.
Ye et al. \cite{Ye:2017aa} conducted a comprehensive survey on intelligent malware detection using data mining techniques, emphasizing the importance of \gls{FE} and algorithm selection. Subsequently, Ucci et al. \cite{Ucci:2019aa} provided an overview of machine learning-based malware analysis, focusing on analysis objectives, \gls{FE}, and \gls{ml} algorithms, albeit limited to \gls{PE} files.
Gibert et al. \cite{Gibert:2020aa} systematically reviewed \gls{ml} and \gls{dl} techniques for Windows malware detection, comparing input features, classification algorithms, and dataset characteristics. Similarly, Qiu et al. \cite{Qiu:2020aa} and Liu et al. \cite{Liu:2022aa} addressed deep Android malware detection, emphasizing supervised classification using \glspl{MLP} and \glspl{cnn} architectures.
Catal et al. \cite{Catal:2022aa} conducted an extensive literature review on \gls{dl} techniques for mobile malware detection, highlighting the prevalence of \gls{MLP} and \gls{cnn} architectures, with a focus on supervised learning and static features.
Furthermore, Deldar et al. \cite{Deldar:2023aa} proposed a survey on \gls{dl} techniques for zero-day malware detection, targeting features extracted at the software level to address emerging threats.

In the early 2010s, researchers initially proposed the idea of \gls{HMD} \cite{Malone_2011, Demme_2013}. \gls{HMD} involves dynamically analyzing micro-architecture events in a processor using \gls{ml} algorithms to differentiate between benign applications and malware. The shift towards \gls{HMD} is justified because of the potential of enhanced security by leveraging robust hardware monitoring infrastructures. This provides a more robust defense against sophisticated attacks that may exploit vulnerabilities in software-based approaches. Specifically, hardware features reflect phase behavior in the underlying hardware, as observed in prior studies \cite{Sherwood_2003, Isci_2006}. These phases often correspond to time-behavioral patterns in micro-architectural events, which vary significantly between programs, enabling the distinction between malicious and benign applications. Additionally, these hardware-based approaches address the zero-day issue, as demonstrated in \cite{He:2021aa}. To the best of our knowledge, a comprehensive overview of \gls{HMD} methods is still missing. This paper tries to cover this gap.

The structure is as follows: Section \ref{sec:malware_basics} covers the basics of malware, serving as a foundation for understanding the field. Section \ref{sec:malware_detection} presents a comprehensive overview of software and hardware-based malware detection solutions, with a detailed discussion of their strengths and weaknesses. Section \ref{sec:hw_based_detection_app} delves into crucial aspects of hardware-based detection. Lastly, Section \ref{sec:conclusion_challenges} provides conclusions and outlines research challenges. %Appendices \ref{appendix:feature_extraction_tools} and \ref{appendix:feature_selection_tools} detail common feature extraction and selection tools, respectively. Appendix \ref{appendix:malware_datasets_attack_tools} offers an overview of datasets and attack tools frequently used in experiments, serving as a helpful starting point for those entering the field.