\section{Opening Problems and Future Directions}
\label{sec:discuss}
 
The introduction of causal inference into recommender systems is still relatively recent, and there are still many promising but unexplored research directions, which we will discuss in this section.

\textbf{Causal assumptions in recommendations.} To extract causality knowledge from statistical data, causal inference-based methods are conducted with several causal assumptions. Although much of the existing work employ causal inference methods, these assumptions are often not explicitly clarified or even violated. To address this issue, existing work either ignores it or makes simple assumptions about data distributions. Therefore, estimating the impact of violations of causal assumptions on experimental results is crucial for bridging the gap between modern recommender system design and causal inference. For example, the positivity assumption is essential in PO-based approaches for the unbiased estimation of causal effect, but the data sparsity problem of recommender systems makes it difficult to satisfy. Considering that a small difference in recommendation accuracy may lead to a huge rise and fall in platform revenue, the effect of the violation of causal assumptions and that of artificial assumptions on predictions should be investigated.

% 尽管本文只关注推荐中的causal inference, 或者说causal estimation,但是笔者认为在推荐系统中,causal discovery与causal inference的结合是必然的。这是因为如前文提到的,前者是后者的基础,是正确性的保障。在没有causal discovery的情况下,我们会发现,即使是针对同样的问题,不同的工作基于自身的专家知识也会提出不同的因果图, 进而在这个基础上设计方法,而忽略了对因果图正确性的验证。所提的方法尽管都在实验中证明了有效性,但可能欠缺通用性,难以推广到其他数据集。这种现象在SCM-based method里面尤其明显。Causal Discovery可以从根源上减少基于人工专家经验设计因果图的情况,增强causal recommendation method的泛化性和在不同场景下的通用性。Causal Discovery可供参考的研究方向如下:1)发现变量之间的因果关系,如用户的年龄、客观经济条件、上下文地理位置等因素与交互decision之间的关联。也许聪明的读者已经发现了,涉及到的变量种类很多,可能不能观测,甚至专家并不知道这种变量的存在。2)发现物品之间的因果关系。例如,拥有打印机的用户可能同时购买纸张和墨盒,这使得纸张和墨盒的统计相关性很强。然而两者在因果上不相关,而有共同的原因item 打印机。正确发现物品之间的相关性对准确推荐帮助很大,但是物品的庞大数量使得building item-level causal graph intractable。一些工作尝试将物品进行聚类,转而构建cluster-level causal graph~\cite{wang2022sequential}。

\textbf{Causal Discovery in recommendations.} The integration of causal discovery and causal inference is inevitable in recommender systems. This belief stems from the fact that, as mentioned in Section \ref{sec:related}, the former serves as the foundation for the latter. In the absence of causal discovery, divergent causal graphs are often proposed for identical problems across different studies, leading to a plethora of methodologies, while overlooking the validation of the causal graph's correctness. Although these proposed methods have experimentally proven effective, they might lack generality and could be challenging to extrapolate to other datasets, a limitation notably prevalent in SCM-based methods. Causal discovery substantially reduces the reliance on manually designed causal graphs, enhancing the generalizability and applicability of causal recommendation methods across diverse scenarios. Possible research directions for Causal Discovery include: 1) Discovering causal relationships between variables. For example, exploring causal relationships between user attributes (e.g., age, economic status, and geographic context) and interaction decisions. 
%Astute readers might have already noted that there are numerous variables involved, some of which might be unobservable or even unknown to experts.
2) Discovering causal relationships between users and items. For example, paper and ink cartridges are always simultaneously observed. But they are causally irrelevant; instead, they share a common cause - the item “printer”~\cite{wang2022sequential}. Accurately identifying item-level causal relationships significantly enhances the precision of recommendations. 
In this direction, some exploratory work has been done~\cite{wang2022sequential, he2022causpref}. One potential solution involves combining causal discovery with GNNs. The ability of GNNs to explore structural information between nodes in a graph~\cite{gao2022graph} gives them a natural advantage in identifying causal relationships between users and items. Furthermore, the causal knowledge uncovered can be further incorporated into GNN-based recommendation algorithms to facilitate the learning of semantically meaningful and identifiable graph representations~\cite{jiang2023survey}.
% 在这个方向上,已经有一些探索的工作\cite{wang2022sequential, he2022causpref}。一个可能的解决方案将因果发现与GNN结合起来。因为GNN的优点可以发现图上节点之间的结构信息\cite{gao2022graph},与发掘causal relationships between users and items有天然的优势。此外,发现的因果知识还可以进一步灌输到 GNN-based  recommendation algorithms中,以促进有语义和可识别的图表示的学习\cite{jiang2023survey}。


\textbf{Transfer learning and Out-of-distribution recommendation.} Due to the data sparsity issue of recommendation systems, it will be a wise and practical choice to transfer user and item knowledge from other domains to improve prediction performance during cold start, offline evaluation, or online test, which is a transfer learning problem. Even in the same dataset, natural shifts of user preference or artificial bias also cause a violation of the IID hypothesis, which is an out-of-distribution (OOD) recommendation issue ~\cite{he2022causpref}. The core of both transfer learning and OOD recommendation is to transfer beneficial shared knowledge, such as users' inherent and unchanged preferences. Thus they can be formulated as invariant learning in some cases~\cite{wang2022invariant}. As we mentioned in Section \ref{sec:intro}, causal inference works to discover the unchangeable causal relationship in data, which can be reused in new domains. From this perspective, adopting causal inference to improve robustness and generalization ability to accomplish cross-domain or OOD recommendation is a promising direction, and some exciting attempts can be found recently~\cite{he2022causpref, wang2022out}.

\textbf{Dynamic recommendation.} Modern recommender systems usually involve feedback loops and dynamic updates. Therefore, it is crucial to incorporate loops into the causality-based methods to accurately model the dynamic and iterative data collection process for recommender systems~\cite{xu2022dynamic}. Some impressive work has also been proposed~\cite{chaney2018algorithmic, wang2021deconfounded, gupta2021causer, krauth2022breaking}. However, uncontrolled feedback loops may give rise to issues like the Matthew effect, echo chambers~\cite{chaney2018algorithmic, ge2020understanding, xu2022dynamic} and bias amplification~\cite{chaney2018algorithmic, wang2021deconfounded}. Original debiasing approaches (e.g., back-door adjustment) cannot be applied directly due to the change in the form of causal models. Therefore, deconfounding in multi-step and feedback loop-involved causal models is still an open research field.

\textbf{Causality-inspired foundation models for recommendation.}
The emergence of Large Language Models (LLMs) has sparked extensive exploration into the development of recommendation foundation models. These models, pre-trained on diverse language or interaction data, can be adapted for a wide array of downstream recommendation tasks~\cite{liu2023first, qiu2021u, hou2022towards, kang2023llms, lin2023can}. Integrating causality into these models is a promising direction~\cite{petrov2023generative}. However, it has been observed that bias in the pretraining corpus of foundation models can lead to unfairness in recommender systems from both user-side~\cite{hua2023up5, zhang2023chatgpt} and item-side~\cite{hou2023large}. Current studies primarily focus on the fairness issue in specific recommendation tasks. To address this, there is a growing interest in formulating novel pretraining tasks to evaluate the causal inference capabilities of recommendation foundation models~\cite{jin2023can}, aiming to mitigate bias issues at their root and enhance overall recommendation performance.
% 我们期待设计一个测试recommendation foundation models的causal inference能力的任务~\cite{jin2023can},使其在预训练的过程中就能通过因果推断来从根源上缓解推荐系统中的bias issue,并提升推荐性能。

\section{Conclusion}

In this survey paper, we have summarized the mechanisms and the strategies of causal inference for recommender systems, from the theoretical perspective: PO framework-based, SCM framework-based and general counterfactuals-based. The survey gives the clear description about the strengths of causal inference for recommendations and manages to use a unified symbol system to describe a large number of existing causal recommender approaches. We hope this survey can well help researchers in the recommendation field to utilize and innovate.