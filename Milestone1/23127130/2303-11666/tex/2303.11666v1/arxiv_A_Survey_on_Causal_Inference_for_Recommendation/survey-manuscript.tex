%%
%% This is file `sample-manuscript.tex',
%% generated with the docstrip utility.
%%
%% The original source files were:
%%
%% samples.dtx  (with options: `manuscript')
%% 
%% IMPORTANT NOTICE:
%% 
%% For the copyright see the source file.
%% 
%% Any modified versions of this file must be renamed
%% with new filenames distinct from sample-manuscript.tex.
%% 
%% For distribution of the original source see the terms
%% for copying and modification in the file samples.dtx.
%% 
%% This generated file may be distributed as long as the
%% original source files, as listed above, are part of the
%% same distribution. (The sources need not necessarily be
%% in the same archive or directory.)
%%
%% Commands for TeXCount
%TC:macro \cite [option:text,text]
%TC:macro \citep [option:text,text]
%TC:macro \citet [option:text,text]
%TC:envir table 0 1
%TC:envir table* 0 1
%TC:envir tabular [ignore] word
%TC:envir displaymath 0 word
%TC:envir math 0 word
%TC:envir comment 0 0
%%
%%
%% The first command in your LaTeX source must be the \documentclass command.
\documentclass[manuscript,screen]{acmart}
\usepackage{amsfonts}
\usepackage{amsmath}
\usepackage{bm}
\newcommand{\ind}{\perp\!\!\!\!\perp} 
\newcommand{\dep}{\not \! \perp \!\!\!\! \perp}
\newtheorem{definition}{Definition}
\newtheorem{assumption}{Assumption}
\usepackage{multirow}
\usepackage{tabularx}
\usepackage{color}
\usepackage{threeparttable}
\usepackage{subfig}
\usepackage{enumitem}
\usepackage{rotating}
\usepackage{makecell, booktabs}
\usepackage{array}
\usepackage{booktabs}

%%
%% \BibTeX command to typeset BibTeX logo in the docs
\AtBeginDocument{%
  \providecommand\BibTeX{{%
    \normalfont B\kern-0.5em{\scshape i\kern-0.25em b}\kern-0.8em\TeX}}}

%% Rights management information.  This information is sent to you
%% when you complete the rights form.  These commands have SAMPLE
%% values in them; it is your responsibility as an author to replace
%% the commands and values with those provided to you when you
%% complete the rights form.
\setcopyright{acmcopyright}
\copyrightyear{2018}
\acmYear{2018}
\acmDOI{XXXXXXX.XXXXXXX}

%% These commands are for a JOURNAL article.
\acmJournal{JACM}
\acmVolume{37}
\acmNumber{4}
\acmArticle{111}
\acmMonth{8}

\acmPrice{15.00}
\acmISBN{978-1-4503-XXXX-X/18/06}


%%
%% Submission ID.
%% Use this when submitting an article to a sponsored event. You'll
%% receive a unique submission ID from the organizers
%% of the event, and this ID should be used as the parameter to this command.
%%\acmSubmissionID{123-A56-BU3}

%%
%% For managing citations, it is recommended to use bibliography
%% files in BibTeX format.
%%
%% You can then either use BibTeX with the ACM-Reference-Format style,
%% or BibLaTeX with the acmnumeric or acmauthoryear sytles, that include
%% support for advanced citation of software artefact from the
%% biblatex-software package, also separately available on CTAN.
%%
%% Look at the sample-*-biblatex.tex files for templates showcasing
%% the biblatex styles.
%%

%%
%% The majority of ACM publications use numbered citations and
%% references.  The command \citestyle{authoryear} switches to the
%% "author year" style.
%%
%% If you are preparing content for an event
%% sponsored by ACM SIGGRAPH, you must use the "author year" style of
%% citations and references.
%% Uncommenting
%% the next command will enable that style.
%%\citestyle{acmauthoryear}

%%
%% end of the preamble, start of the body of the document source.
\begin{document}

%%
%% The "title" command has an optional parameter,
%% allowing the author to define a "short title" to be used in page headers.
\title{A Survey on Causal Inference for Recommendation}

%%
%% The "author" command and its associated commands are used to define
%% the authors and their affiliations.
%% Of note is the shared affiliation of the first two authors, and the
%% "authornote" and "authornotemark" commands
%% used to denote shared contribution to the research.
\author{Huishi Luo}
\email{hsluo2000@buaa.edu.cn}
\orcid{0000-0002-3553-2280}
\author{Fuzhen Zhuang}
\authornote{Corresponding author: zhuangfuzhen@buaa.edu.cn.}
\email{zhuangfuzhen@buaa.edu.cn}
\orcid{0000-0001-9170-7009}
\affiliation{%
  \institution{Institute of Artificial Intelligence, Beihang University}
  \city{Beijing}
  \country{China}
  \postcode{100191}
}

\author{Ruobing Xie}
\email{ruobingxie@tencent.com}
\orcid{0000-0003-3170-5647}
\affiliation{%
  \institution{WeChat Search Application Department, Tencent}
  \city{Beijing}
  \country{China}
  \postcode{100080}
}


\author{Hengshu Zhu}
\email{zhuhengshu@gmail.com}
\orcid{0000-0003-4570-643X}
\affiliation{%
  \institution{the Career Science Laboratory, BOSS Zhipin}
  \city{Beijing}
  \country{China}
  \postcode{100028}
}


\author{Deqing Wang}
\email{dqwang@buaa.edu.cn}
\orcid{0000-0001-6441-4390}
\affiliation{%
  \institution{SKLSDE, School of Computer Science, Beihang University}
  \city{Beijing}
  \country{China}
  \postcode{100191}
}

%%
%% By default, the full list of authors will be used in the page
%% headers. Often, this list is too long, and will overlap
%% other information printed in the page headers. This command allows
%% the author to define a more concise list
%% of authors' names for this purpose.
\renewcommand{\shortauthors}{Luo et al.}

%%
%% The abstract is a short summary of the work to be presented in the
%% article.
\begin{abstract}
Recently, causal inference has attracted increasing attention from researchers of recommender systems (RS), which analyzes the relationship between a cause and its effect and has a wide range of real-world applications in multiple fields. Causal inference can model the causality in recommender systems like confounding effects and deal with counterfactual problems such as offline policy evaluation and data augmentation. Although there are already some valuable surveys on causal recommendations, these surveys introduce approaches in a relatively isolated way and lack theoretical analysis of existing methods. Due to the unfamiliarity with causality to RS researchers, it is both necessary and challenging to comprehensively review the relevant studies from the perspective of causal theory, which might be instructive for the readers to propose new approaches in practice. This survey attempts to provide a systematic review of up-to-date papers in this area from a theoretical standpoint. Firstly, we introduce the fundamental concepts of causal inference as the basis of the following review. Then we propose a new taxonomy from the perspective of causal techniques and further discuss technical details about how existing methods apply causal inference to address specific recommender issues. Finally, we highlight some promising directions for future research in this field.
\end{abstract}

%%
%% The code below is generated by the tool at http://dl.acm.org/ccs.cfm.
%% Please copy and paste the code instead of the example below.
%%
\begin{CCSXML}
<ccs2012>
<concept>
<concept_id>10002951.10003317.10003347.10003350</concept_id>
<concept_desc>Information systems~Recommender systems</concept_desc>
<concept_significance>500</concept_significance>
</concept>
</ccs2012>
\end{CCSXML}

\ccsdesc[500]{Information systems~Recommender systems}

%%
%% Keywords. The author(s) should pick words that accurately describe
%% the work being presented. Separate the keywords with commas.
\keywords{Recommender Systems, Causal Inference, Causal Learning}

% \received{20 February 2007}
% \received[revised]{12 March 2009}
% \received[accepted]{5 June 2009}

%%
%% This command processes the author and affiliation and title
%% information and builds the first part of the formatted document.
\maketitle


\section{Introduction}
\label{sec:introduction}
% \begin{itemize}
%     % Diffusion of FL
%     \item {\st{Diffusion of FL}}
%     % Security threats to FL
%     \item {\st{Security threats to FL with particular focus on model poisoning}}
%     % Limitations of existing countermeasures
%     \item {\st{Current countermeasures (e.g., KRUM) and their limitations}}
%     % Proposed method and its advantages
%     \item {\st{Intuitive description of the proposed method and its difference (i.e., advantages) w.r.t. state of the art}}
%     % Main contributions
%     \item {\st{Summary of the main contributions of this work}}
%     % Paper's structure and organization
%     \item {\st{Paper's structure and organization}}
% \end{itemize}

% Diffusion of FL
Recently, {\em federated learning} (FL) has emerged as the leading paradigm for training distributed, large-scale, and privacy-preserving machine learning (ML) systems~\cite{mcmahan2017googleai,mcmahan2017aistats}. 
The core idea of FL is to allow multiple edge clients to collaboratively train a shared, global model without disclosing their local private training data.
%Specifically, an FL system consists of a central server and many edge clients; 
A typical FL round involves the following steps: {\em(i)} the server randomly picks some clients and sends them the current, global model; {\em(ii)} each selected client locally trains its model with its own private data; then, it sends the resulting local model to the server;\footnote{Whenever we refer to global/local model, we mean global/local model {\em parameters}.} {\em(iii)} the server updates the global model by computing an \emph{aggregation function}, usually the average (FedAvg), on the local models received from clients.
% \begin{enumerate}
%     \item[{\em(i)}] the server sends the current, global model to the clients and appoints some of them for training;
%     \item[{\em(ii)}] each selected client locally trains its copy of the global model with its own private data; then, it sends the resulting local model back to the server;\footnote{Whenever we refer to global/local model, we mean global/local model {\em parameters}.}
%     \item[{\em(iii)}] the server updates the global model by computing an \emph{aggregation function} on the local models received from clients (by default, the average, also referred to as FedAvg~\cite{mcmahan2017aistats}).
% \end{enumerate}
This process goes on until the global model converges. %(e.g., after a certain number of rounds or other similar stopping criteria).
%\\
% The advantages of FL over the traditional, centralized learning paradigm are undoubtedly clear in terms of flexibility/scalability (clients can join/disconnect from the FL network dynamically), network communications (only model weights\footnote{We will use \textit{parameters} and \textit{weights} interchangeably.} are exchanged between clients and server), and privacy (each client's private training data is kept local at the client's end and not uploaded to the server).
\\
% Security threats to FL
%However, the growing adoption of FL also raises security concerns~\cite{costa2022covert}, particularly about its confidentiality, integrity, and availability.
Although its advantages over standard ML, FL also raises security concerns~\cite{costa2022covert}. %, particularly about its confidentiality, integrity, and availability~\cite{costa2022covert}.
% OLD, LONG VERSION
% Indeed, some work deals with privacy leakage that may expose the local data of some clients~\cite{melis2019sp}. 
% A large body of work, instead, investigates attacks that usually aim to detriment the predictive accuracy of the learned global model. For instance, \emph{data poisoning} attacks achieve this goal by letting an adversary pollute the training set of some corrupt FL clients with maliciously crafted examples~\cite{jagielski2018sp}.
% Similarly, in \emph{model poisoning} the attacker attempts to tweak the global model weights~\cite{bhagoji2019pmlr} by directly perturbing the local model's weights of some infected FL clients before these are sent to the central server for aggregation, usually via so-called Byzantine attacks. 
% It turns out that Byzantine model poisoning attacks severely impact standard FedAvg; therefore, more robust aggregation functions must be designed to make FL systems secure.
Here, we focus on \emph{untargeted model poisoning} attacks~\cite{bhagoji2019pmlr}, where an adversary attempts to tweak the global model weights %\footnote{We will use the terms \textit{parameters} and \textit{weights} interchangeably.} 
by directly perturbing the local model's parameters of some infected clients before these are sent to the central server for aggregation.
In doing so, the adversary aims to jeopardize the global model \textit{indiscriminately} at inference time.
Such model poisoning attacks severely impact standard FedAvg; therefore, more robust aggregation functions must be designed to secure FL systems.
\\
% In this paper, we focus on designing a novel robust aggregation scheme at the server's end to contrast the effect of Byzantine model poisoning attacks.
%
% Current countermeasures and their limitations
%Several countermeasures have been proposed in the literature to combat model poisoning attacks on FL systems.
% Some methods use simple statistics more robust than plain average to smooth the impact of malicious updates (e.g., Trimmed Mean and FedMedian~\cite{yin2018icml}). 
% Other defenses implement outlier detection techniques to discard malicious updates from the aggregation performed at the server's end. Those are either based on heuristics (e.g., Krum/Multi-Krum~\cite{blanchard2017nips} and Bulyan~\cite{mhamdi2018pmlr}) or data-driven approaches (e.g., K-means clustering~\cite{shen2016acm} or DnC via spectral analysis~\cite{shejwalkar2021ndss}). 
% Finally, some strategies rely on a centralized ``source of trust'' to spot potential malicious updates (e.g., FLTrust~\cite{cao2020fltrust}).
% Several countermeasures have been proposed in the literature to combat model poisoning attacks on FL systems, i.e., to discard possible malicious local updates from the aggregation performed at the server's end. 
% These techniques range from simple statistics more robust than plain average (e.g., Trimmed Mean and FedMedian~\cite{yin2018icml}) to outlier detection heuristics (e.g., Krum/Multi-Krum~\cite{blanchard2017nips} and Bulyan~\cite{mhamdi2018pmlr}) or data-driven approaches (e.g., spectral analysis via K-means clustering~\cite{shen2016acm} or spectral analysis), or methods based on ``source of trust'' (e.g., FLTrust~\cite{cao2020fltrust}).
% OLD, LONG VERSION
%Several countermeasures have been proposed in the literature to combat Byzantine model poisoning attacks on FL systems.
% Descriptive statistics
% For example, Trimmed Mean and FedMedian aggregate local model updates using more robust statistics than standard average~\cite{yin2018icml}.
%
% % Heuristics for outlier detection
% Many existing Byzantine-resilient strategies implement some outlier detection heuristics to discard the model updates sent by potentially malicious clients from the input of the aggregation function.
% One of the most popular heuristics is Krum~\cite{blanchard2017nips}.
% This strategy tries to mitigate the impact of Byzantine attacks by selecting as a global model the local model with the smallest sum of Euclidean distances to {\em all} the other local models.
% Although powerful, Krum requires the server to know (or, at least, estimate) the number of malicious FL clients upfront, which is generally impossible in a realistic attack scenario. %
% Moreover, Krum may become ineffective for complex, high-dimensional model parameter spaces due to the curse of dimensionality.
% Bulyan~\cite{mhamdi2018pmlr} tries to overcome this issue by combining Krum with a variant of Trimmed Mean.
% % Data-driven outlier detection
% Other strategies use data-driven outlier detection techniques -- e.g., via K-means clustering~\cite{shen2016acm} -- to spot potential malicious local model updates. 
% %For instance, Shen et al. propose to cluster local model updates with K-means and thus identify outliers.
%
% % Other techniques
% As far as the server is concerned, any local model received can be from a potential malicious client. 
% FLTrust~\cite{cao2020fltrust} assumes the server acts as a client, i.e., trains a local model on an additional {\em trustworthy} dataset at the server's end and compares it against all the local models from other clients. 
% This way, the server can rely on some ``source of trust'' when discarding potentially malicious clients.
%\\
% Limitations of existing Byzantine-resilient strategies
Unfortunately, existing defense mechanisms either rely on simple heuristics (e.g., Trimmed Mean and FedMedian by~\cite{yin2018icml}) or need strong and unrealistic assumptions to work effectively (e.g., foreknowledge or estimation of the number of malicious clients in the FL system, as for Krum/Multi-Krum~\cite{blanchard2017nips} and Bulyan~\cite{mhamdi2018pmlr}, which, however, cannot exceed a fixed threshold).
Furthermore, outlier detection methods using K-means clustering~\cite{shen2016acm} or spectral analysis like DnC~\cite{shejwalkar2021ndss} do not directly consider the temporal evolution of local model updates received.
Finally, strategies like FLTrust~\cite{cao2020fltrust} require the server to collect its own dataset and act as a proper client, thereby altering the standard FL protocol.
\\
% OLD, LONG VERSION
% Overall, existing Byzantine-resilient strategies are either simple heuristics (e.g., FedMedian) or, if they are more complex, they rely on strong and unrealistic assumptions to work effectively (e.g., knowing the number of malicious clients in the FL system in advance, as for Krum and alike).
% Furthermore, data-driven outlier detection methods do not consider the temporary evolution of local model updates received (e.g., K-means clustering). 
% Finally, strategies like FLTrust requires the server to collect its own dataset and act as a proper client, thereby altering the standard FL protocol.
%
% Description of the proposed method
This work introduces a novel pre-aggregation \textit{filter} robust to untargeted model poisoning attacks. Notably, this filter $(i)$ operates without requiring prior knowledge or constraints on the number of malicious clients and $(ii)$ inherently integrates temporal dependencies. 
The FL server can employ this filter as a preprocessing step before applying \textit{any} aggregation function, be it standard like FedAvg or robust like Krum or Bulyan.
Specifically, we formulate the problem of identifying corrupted updates as a multidimensional (i.e., matrix-valued) time series anomaly detection task. 
The key idea is that legitimate local updates, resulting from well-calibrated iterative procedures like stochastic gradient descent (SGD) with an appropriate learning rate, show \textit{higher predictability} compared to malicious updates. This hypothesis stems from the fact that the sequence of gradients (thus, model parameters) observed during legitimate training exhibit regular patterns, as validated in Section~\ref{subsec:intuition}. %until convergence. 
%This regularity may be more pronounced for smooth convex loss functions, but it can still be captured within an appropriate time window, even for more complex and convoluted loss surfaces. 
%We provide evidence of this claim in Appendix~B, where we show that the average mutual information (i.e., ``predictability''), calculated over pairs of legitimate model updates sent at different FL rounds, is significantly higher than the corresponding computation for a malicious client.
\\
Inspired by the matrix autoregressive (MAR) framework for multidimensional time series forecasting~\cite{chen2021je}, we propose the FLANDERS ({\em \textbf{F}ederated \textbf{L}earning meets \textbf{AN}omaly \textbf{DE}tection for a \textbf{R}obust and \textbf{S}ecure}) filter.
The main advantages of FLANDERS over existing strategies like FLDetector~\cite{zhao2020multivariate} are its resilience to large-scale attacks, where $50\%$ or more FL participants are hostile, and the capability of working under realistic non-iid scenarios.
We attribute such a capability to two key factors: $(i)$ FLANDERS works without knowing a priori the ratio of corrupted clients, and $(ii)$ it embodies temporal dependencies between intra- and inter-client updates, quickly recognizing local model drifts caused by evil players. Below, we summarize our main contributions:

\begin{itemize}
\item[{\em(i)}]
We provide empirical evidence that the sequence of models sent by legitimate clients is more predictable than those of malicious participants performing untargeted model poisoning attacks.
\\
\item[{\em(ii)}] 
We introduce FLANDERS, the first pre-aggregation filter for FL robust to untargeted model poisoning based on multidimensional time series anomaly detection.
\\
\item[{\em(iii)}] 
We integrate FLANDERS into Flower,\footnote{\scriptsize{\url{https://flower.dev/}}} a popular FL simulation framework for reproducibility.
\\
\item[{\em(iv)}] 
We show that FLANDERS improves the robustness of the existing aggregation methods under multiple settings: different datasets, client's data distribution (non-iid), models, and attack scenarios.
\\
\item[{\em(v)}] 
We publicly release all the implementation code of FLANDERS along with our experiments.\footnote{\scriptsize{\url{https://anonymous.4open.science/r/flanders_exp-7EEB}}}
\end{itemize}

% Paper's structure and organization
The remainder of the paper is structured as follows. %some related work and the current state-of-the-art solutions to security issues that FL entails. 
Section~\ref{sec:background} covers background and preliminaries. 
In Section~\ref{sec:related}, we discuss related work.
Section~\ref{sec:problem} and Section~\ref{sec:method} describe the problem formulation and the method proposed. % to tackle it. 
Section~\ref{sec:experiments} gathers experimental results. %, and Section~\ref{sec:limitations} discusses some limitations of this work.
Finally, we conclude in Section~\ref{sec:conclusion}.
 %discusses the limitations of this work and draws future research directions.
%reports conclusions and draws perspectives for future research directions.

%%%%%%% OLD %%%%%%%
%to overcome the resilience of Byzantine failures in distributed Stochastic Gradient Descent computations. 
% The strength of Krum is its time complexity, which is linear in the gradient dimension. 
% However, the robustness of the approach is guaranteed for gradient-based learning applications only when the majority of the clients are not compromised. 
% Besides, the aggregation mechanism of Krum, as well as that of similar methods, is robust from a coarse-grained perspective and does not provide solutions to errors and perturbations that may occur at inference time.
%A related approach to~\cite{blanchard2017nips} is the work of Su et al.~\cite{su2016dc}. Here, the authors propose an iterated approximate agreement to tackle a multi-layer scenario attacked by Byzantine agents. 
%However, the method works efficiently on the sole discrete context and it is inapplicable to continuous state environments.
%\gabri{Maybe, we should just talk about the main limitations of existing countermeasures without digging into their details (or, we can just mention Krum as this is the most popular one). I will move the description of all these methods to the Related Work section.}
\section{Related work}
% There is extensive recent work on speeding up analytical queries due to the need for consistent execution times in the face of the explosive growth in the volume of available data.
% In this section, we divide existing work into two categories: maintaining data freshness in MVs (\Cref{sec:server_side}) and optimizations for minimizing ad-hoc query latency (\Cref{sec:client_side}).

% \subsection{Maintaining Data Freshness in MVs}
% \label{sec:server_side}
% There exists a variety of data warehousing applications aimed at supporting low-latency analytical queries on fresh data.
% In particular, these applications require efficiency in the propagation of newly ingested data into downstream MVs.
 
\mypara{Efficient MV Refresh}
Incremental view maintenance (IVM) aims to update MVs to reflect newly ingested data, taking advantage of already computed results to perform the update in a manner more efficient than computing from scratch (full refresh)
~\cite{ahmad2012dbtoaster,mcsherry2013differential,armbrust2013generalized,zeng2016iolap, palpanas2002incremental, griffin1995incremental, agiwal2021napa, braun2015analytics}. 
There is an abundance of work in IVM, including incremental updates on duplicate values~\cite{griffin1995incremental}, non-distributive aggregate functions~\cite{palpanas2002incremental}, higher-order views~\cite{ahmad2012dbtoaster}, and sliding windows~\cite{braun2015analytics}. 
More recent works also investigate the scalability aspect of IVM, proposing scale-independent updates~\cite{armbrust2013generalized} and sampled views~\cite{zeng2016iolap}. Since \system is applicable to arbitrary SQL statements, \system is orthogonal to and is fully compatible with existing IVM techniques.

\mypara{MV Refresh Scheduling}
There exist works on scheduling the refresh of a MV set focusing on resolving cyclic dependencies~\cite{folkert2005optimizing}, minimizing weighted average staleness~\cite{golab2009scheduling}, and prioritizing MVs with the highest speedups on predicted future queries~\cite{ahmed2020automated}.
\system's scheduling to speed up the end-to-end refresh of the MV set is not addressed in existing works.

\mypara{DAG Workflow Scheduling}
The execution of workloads consisting of individual jobs with acyclic dependencies is a well-studied topic~\cite{apacheoozie,sparkdag,marchal2018parallel,bathie2020revisiting,baruah2022ilp}; many of these techniques can be applied to MV refresh runs studied in this paper.
Existing workflow scheduling systems such as Apache Oozie~\cite{apacheoozie}, Apache Airflow~\cite{airflow}, and Spark DAG scheduler~\cite{sparkdag} automate the execution of user-defined workflows following a topological order.
There are recent works aimed at finding more optimal execution orders in terms of peak memory usage~\cite{marchal2018parallel, bathie2020revisiting} and execution time on parallel platforms~\cite{baruah2022ilp}.
While \system is designed for use with MV refresh runs/workloads, our technique on joint scheduling and optimization can be reasonably applied to general workloads as a possible future direction.

% \paragraph{Incremental MV indexing}
% Update-optimized indices such as the log-structured merge-trees (LSM)~\cite{o1996log} are used for indexing MVs due to frequent updates induced by data ingestion~\cite{gupta2016mesa,agiwal2021napa}.
% \system is orthogonal to indexing: \system is capable of efficiently performing MV refresh runs regardless of whether the individual MVs are indexed or not.

% \subsection{Ad-hoc Query Latency Reduction}
% \label{sec:client_side}

% The minimization of ad-hoc analytical query response times is a well-studied topic due to latency being negatively correlated with the productivity of a data analyst during a data analysis session~\cite{liu2014effects}.
% Sessions are commonly conducted within visualization systems that contain a variety of optimization techniques to ensure that query response times fall within a certain latency tolerance.

% \mypara{Data prefetching}
% Data is often loaded into memory on a by-need basis in visualization systems to minimize interference with user-issued query computations~\cite{mani2017effective,xin2021enhancing,galakatos2017revisiting, yan2020auto, battle2016dynamic, crotty2016case, jalaparti2018netco}. 
% Query-time data retrieval can be significantly expedited by anticipating the data usage of the user in future queries and pre-loading the data into memory during the downtime between user queries (`think time'). SMART~\cite{mani2017effective} prefetches data for modified versions of current user-issued queries with different filters and dimensions. A-WARE~\cite{crotty2016case} maintains a sample store constantly refined through ingesting data based on speculations of future plots.
% ForeCache~\cite{battle2016dynamic} uses an SVM to predict the user's current analysis phase and accordingly prefetches data tiles partitioned based on different numerical values. NetCo predicts future queries via log analysis, and solves an ILP formulation to prefetch data to maximize the number of SLO-meeting queries~\cite{jalaparti2018netco}.
% In the case of MV refresh workloads, `think time' is nonexistent as individual MVs are refreshed back-to-back, rendering data prefetching techniques non-applicable.

\mypara{Intermediate Data Caching}
Some existing data visualization systems cache user-defined variables to support the typical incremental construction of data visualizations~\cite{zgraggen2016progressive, eichmann2020idebench} during data analysis sessions~\cite{jupyter, rstudio, colab}. 
Recent work proposes a management scheme for these cached variables under a memory constraint that greedily keeps variables with the highest estimated time savings based on predicted future user behavior via neural networks~\cite{xin2021enhancing}.
While useful for data visualization, a greedy approach to memory management fails to achieve satisfactory results compared to \system.

\mypara{Intermediate Result Reuse}

There exist works on storing intermediate results from computations to speedup future computations~\cite{yang2018intermediate, dursun2017revisiting, nagel2013recycling, michiardi2019memory, galakatos2017revisiting}.
Studied topics include the identification of reuse opportunities by finding overlaps in computation graphs of successive jobs~\cite{yang2018intermediate, michiardi2019memory},
selective storage under a space constraint with heuristics such as reuse probability~\cite{dursun2017revisiting}, expected savings~\cite{yang2018intermediate}, and recompute-storage cost difference~\cite{nagel2013recycling},
and rewriting incoming jobs to take advantage of stored intermediates~\cite{galakatos2017revisiting}.
These works share similarity with \system in their selection of items to store under a memory constraint, however, \system's problem setting requires it to uniquely consider the joint (re)ordering of job executions along with the selection of items.

% work that considers both job execution (re)order as well as intermediate result caching with a bounded amount of memory. but notably lack the joint aspect of \system and cannot be used to achieve immediate speedup on an incoming MV refresh run if no intermediates are stored beforehand. 

\mypara{Incremental Query Processing} Incremental processing (IQP) is useful for cases where not all data required for a query is immediately available. Similar to online aggregation~\cite{hellerstein1997online}, initial results of a query are computed on a subset of required data and progressively refined as the rest of the required data arrives in a predictable pattern~\cite{tang2019intermittent,wangtempura}. Tang et al. propose a dynamic programming formulation to pick intermediate states to store in memory given a limited memory budget~\cite{tang2019intermittent}. Tempura rewrites the query plan for more efficient execution based on predicted data arrival patterns~\cite{wangtempura}. While similarities exist between the problem setting of IQP and \system, such as management of bounded memory, \system notably includes additional joint optimization for the order of MV updates.

% \paragraph{Sampling}
% Sampling has seen wide use in visualization systems for reducing the computation time of ad-hoc queries by computing an approximate result over a subset of data as exact results are not always required by the user~\cite{crotty2016case, mani2017effective, zgraggen2014panoramicdata, kraska2021northstar, galakatos2017revisiting, kandula2016quickr}. 
% Commonly studied topics in sampling for ad-hoc queries include complex query sampling~\cite{kandula2016quickr}, rare event aggregation~\cite{kraska2021northstar, galakatos2017revisiting}, and maintaining consistency between related sampled visualizations~\cite{zgraggen2014panoramicdata}.
% Sampling server-side at the MV level compromises the assumptions of downstream applications and is thus not considered in \system.

% \paragraph{Progressive visualization}
% The latency tolerance for time-consuming queries can be circumvented by presenting a partially-computed visualization to the user within the tolerance, which is then incrementally refined until it is fully accurate~\cite{rahman2017ve, zgraggen2016progressive, crotty2015vizdom, kraska2021northstar, kamat2017infiniviz}.
% Example plots which benefit from progressive visualization include bar charts~\cite{kamat2017infiniviz} and heatmaps~\cite{rahman2017ve}.
% Similar to sampling, study on this topic is orthogonal to \system as pushing out partially-updated MVs compromises downstream assumptions.
\section{Foundation}
\label{sec:foundation}

In this section, background information and several important concepts of causal inference and recommender systems are introduced to facilitate readers' understanding of the inter-study of the two research fields. The notations used in this survey are listed for convenience. At the end of this section, we set up categorizations of causal recommendations.

\subsection{Causal Inference}
In this part, we will give a brief review of two representative frameworks of causal inference, including the potential outcome (PO) framework by Rubin et al.~\cite{ splawa1990application, rubin1974estimating, imbens2015causal} and the structural causal models (SCM) framework by Pearl et al.~\cite{pearl1995causal, pearl1988probabilistic, pearl2009causality}
%, to provide RS researchers with a preliminary understanding
. Note that these two frameworks are logically equivalent~\cite{ pearl2009causality}.

\subsubsection{Potential Outcome Framework}
\label{sec:po}

The Potential Outcomes Framework (aka the Neyman-Rubin Causal Model) ~\cite{plawa1990application, rubin1974estimating, imbens2015causal} is the most widely used framework across many disciplines. With a hypothetical treatment (or manipulation, intervention), the causal effect, i.e., treatment effect, is defined as the difference between the potential outcomes under treatment and control for \textit{the same unit}~\cite{imbens2015causal}.

\begin{definition}[Unit]
A unit refers to the research object in the potential framework.
\end{definition}

A unit can be a physical object, an individual, or a collection of objects or persons, such as a classroom or a market, at a particular point in time~\cite{imbens2015causal}. In recommendation research, a user-item pair will usually be defined as a unit. It should be noticed that the same physical object or person at a different time is a different unit. This is a reasonable restriction, considering the same user will make different decisions at a different time even if exposed to the same item due to factors like preference shift, mood, occasion and so on.

\begin{definition}[Treatment]
Treatment can be defined as the action applied to a unit.
\end{definition}

This paper focuses on binary treatment (e.g., recommend or not), the most common setting in the recommendation field. In practice, we refer to the more active treatment simply as the “treatment” $T=1$ and the other treatment as the “control” $T=0$.

\textbf{Potential Outcome.} For each treatment-unit pair, the potential outcome is the outcome that the treatment is applied to the unit, denoted as $Y(T=t)$ (ignoring unit). For a unit, only the potential outcome corresponding to the treatment actually taken will be observed, denominated as observed outcome, while others are referred to as counterfactual outcomes. The \textit{fundamental problem of causal inference} in PO framework is that we can never obtain both observed and counterfactual outcomes for a unit: it is impossible to realize all treatments and observe the corresponding outcomes.

\textbf{Treatment Effect/ Causal effect.} Treatment effect is represented by the difference between the potential outcomes under treatment and control for the same unit, formulated as:

\begin{equation}
\label{eq:ITE}
    \textrm{TE} = Y(T=1) - Y(T=0),
\end{equation}

where $ {Y}(T=1)$ and $ {Y}(T=0)$ are the potential treated and control outcome of the unit, respectively. Treatment effect like Equation \ref{eq:ITE} is also called \textbf{Individual Treatment Effect}. Furthermore, the treatment effect can be defined at the population and subpopulation levels. At the population level, \textbf{ Average Treatment Effect (ATE) } is the expectation of ITE over the whole population~\cite{guo2020survey}, denoted as:

\begin{equation}
\label{eq:ATE}
    \textrm{ATE} = \mathbb{E}[Y(T=1) - Y(T=0)].
\end{equation}

The ATE on the subpopulation level is often of particular interest; thus we define \textbf{Conditional Average Treatment Effect (CATE)} on the units with the same features $X=x$ as:
\begin{equation}
\label{eq:CATE}
    \textrm{CATE} = \mathbb{E}[Y(T=1|X=x) - Y(T=0|X=x)].
\end{equation}

\textbf{Assumptions.} Despite the simple definition of the causal effect, the fundamental problem in causal inference, i.e., the \textit{missing data problem}, appear to be a major obstacle to the estimation of the causal effect. Therefore, it is critical to make additional assumptions.

\begin{assumption}[SUTVA]
\label{assumpt:sutva}
The potential outcomes for any unit do not vary with the treatments assigned to other units. For each unit, there are no different forms or versions of each treatment level, which lead to different potential outcomes.
\end{assumption}

The stable unit treatment value assumption, or SUTVA~\cite{imbens2015causal} is the most fundamental assumption in causal inference, incorporating both the \textit{No Interference} idea that treatments applied to one unit do not affect the outcome for another unit and the \textit{No Hidden Variations of Treatments} concept that for each unit there is only a single version of each treatment level. The second assumption, \textit{ignorability} or \textit{unconfoundedness}~\cite{rubin1990formal}, states that treatment assignment is free from dependence on the potential outcomes.

\begin{assumption}[Unconfoundedness / Ignorability]
\label{assumpt:uncon}
Treatment assignment $W$ is independent to the potential outcomes, i.e., $T \perp Y (T = 0),Y (T = 1)|X$, also written as $\textrm{Pr}(T=1|X, Y (T = 0),Y (T = 1)) = \textrm{Pr}(T=1|X)$, where $X$ denotes the background variables.
\end{assumption}

In other words, within subpopulations defined by the values of observed background variables, or covariates, the treatment assignment is random. The ignorability assumption rules out unmeasured confounders, which causally influences both the treatment $T$ and the outcome $Y(T)$. $\textrm{Pr}(T=1|X)$ is called the \textit{propensity score}~\cite{rosenbaum1983central}. The last assumption is positivity, or overlap:

\begin{assumption}[Positivity]
\label{assumpt:pos}
$0 < \textrm{Pr}(T=t|X=x) < 1, \forall t, x.$
\end{assumption}

In large data samples, positivity requires that there are both treated and control units for all values of the covariates. In contrast to the untestable ignorability assumption~\cite{ imbens2015causal}, positivity can be tested from observed data. The combination of unconfoundedness and positivity is referred to as “\textit{strong ignorability}~\cite{rosenbaum1983central}.”


\subsubsection{Structural Causal Models Framework}
\label{sec:scm}
Structural causal models (SCM)~\cite{pearl1995causal, pearl1988probabilistic, pearl2009causality} serve as a comprehensive causality framework, which unifies graphical models, nonparametric structural equations, and counterfactual and interventional logic. The most significant advantage of SCM is its intuitive structure of real-world causal dependencies based on graphical models as well as the wise and friendly symbiosis between counterfactual and graphical methods.

\textbf{Causal Graph.} A causal graph, or a causal diagram, is usually a Bayesian network, which describes the causal relations between variables by a Directed Acyclic Graph (DAG), where the nodes represent the variables and the edges record the causal relations. Causal graphs play an essential role in the SCM framework, for they provide a vivid representation of sets of variables that are relevant to each other in any given state of knowledge, and serves as a carrier of \textit{conditional independence} relationships along the order of construction, through which we can confirm whether it satisfies the criteria such that certain causal inference methods can be applied~\cite{pearl2009causality}. 

\begin{figure}[t!]
    \centering
    \vspace{-3mm}
    \subfloat[Chain]{
    	\centering
    	\includegraphics[width=0.2\textwidth, trim=-10 0 -10 0, clip]{pic/chain.pdf}
    }
    \centering
    \subfloat[Fork]{
    	\centering
    	\includegraphics[width=0.2\textwidth, trim=-10 0 -10 0, clip]{pic/fork.pdf}
    }
    \centering
        \subfloat[Collider]{
    	\centering
    	\includegraphics[width=0.2\textwidth, trim=-10 0 -10 0, clip]{pic/collider.pdf}
    }
    \centering    
    \vspace{-3mm}
    \caption{Graphical models of three typical types of causal structures.}
    \vspace{-6mm}
    \label{fig:junction}
\end{figure}

\textbf{\textit{d}-Separation.} We first review the concept of \textit{dependency-separation} (\textit{d}-Separation) as the knowledge base for conditional independence. There are three typical causal graphs of three disjoint sets of variables, shown in Figure~\ref{fig:junction}, with the help of which we can characterize any pattern of arrows in the network. In the \textit{chain} (Figure~\ref{fig:junction}(a)), $B$ is the $mediator$ that transmits the effect of $A$ to $C$. In the \textit{fork} (Figure~\ref{fig:junction}(b)), $B$ is often called a common cause or \textit{confounder} of $A$ and $C$. A confounder will make $A$ and $C$ statistically correlated even though there is no direct causal link between them, which may give rise to a so-called spurious correlation in the application. In the \textit{collider} (Figure~\ref{fig:junction}(c)), though $A$ and $C$ are independent to begin with, conditioning on (i.e., knowing the value of) $B$ will make them dependent. A good example is three features of Hollywood actors: Talent $\rightarrow$ Celebrity $\leftarrow$ Beauty ~\cite{elwert2014endogenous}. Although beauty and talent are completely unrelated to one another in the general population, an unanticipated negative correlation is found between talent and beauty if we only focus on famous actors: a celebrity is unattractive increases our belief that he or she is talented~\cite{pearl2018book}. This negative correlation is sometimes called collider bias or the “explain-away” effect. 
A \textit{path} means a sequence of consecutive edges (of any directionality) in the graph, and we regard stopping the flow of dependency between the variables that are connected by such paths as \textit{blocked}. In the chain and fork, the path between $A$ and $C$ will be blocked by conditioning on $B$, while in the collider, any conditioning on $B$ will introduce a correlation between them. The formal definition of \textit{d}-separation or blocking is defined as follows.

\begin{definition}[\textit{d}-Separation]
\label{def:d_s}
A path is said to be d-separated (or blocked) by conditioning on a set of nodes $\mathcal{Z}$ if and only if one of the two conditions is satisfied:
\begin{enumerate}
\item The path contains a chain $A \rightarrow B \rightarrow C$ or a fork $A \leftarrow B \rightarrow C$ such that the middle node $B$ is in $\mathcal{Z}$;
\item The path contains a collider such that the middle node $B$ is not in $\mathcal{Z}$ and such that no descendant of $B$ is in $\mathcal{Z}$.
\end{enumerate}
\end{definition}

\begin{figure}[t!]
    \centering
    \vspace{-3mm}
    \subfloat[]{
    	\centering
    	\includegraphics[width=0.20\textwidth, trim=-4 0 -4 0, clip]{pic/inter0.pdf}
    }
    \centering
    \subfloat[]{
    	\centering
    	\includegraphics[width=0.20\textwidth, trim=-4 0 -4 0, clip]{pic/inter1.pdf}
    }
    \centering
    \vspace{-3mm}
    \caption{Examples of the structural equation and intervention.}
    \vspace{-7mm}
    \label{fig:inter}
\end{figure}

\textbf{Structural Equations.} Beside causal graph, structural equation is another representation of causal information, where the former is an abstraction of the latter. In its general form, a structural equation of a variable $Y$ is defined as:

\begin{equation}
\label{eq:sem}
Y=f_Y(Pa, U),
\end{equation}

where $Pa$ (connoting parents) stands for the set of variables that directly determine the value of $Y$ and where $U$ represents exogenous variables or errors (or “disturbances”) due to omitted factors. For example, the causal graph in Fig.~\ref{fig:inter}(a) is associated with the structural model as:

\begin{align}
	\label{eq:structural_example}
	\begin{cases}
	a&=\ f_A(u_A), \\
	b&=\ f_B(a, u_B), \\
	c&=\ f_C(a, b, u_C),
	\end{cases}
\end{align}

where $U_A$, $U_B$ and $U_C$ represent exogenous variables.A set of equations in the form of Equation~\ref{eq:sem} is called a \textit{structural model}; if each variable has a distinct equation in which it appears on the left-hand side, then the model is called a \textit{structural causal model}.

\textbf{Intervention.} The \textit{do-calculus} allows researchers to complete \textit{intervention}, interpreted as controlling the value of a variable, by purely mathematical means instead of by carrying out a physical experiment, which is one of the outstanding contributions of Pearl’s SCM framework. The \textit{do}-calculus involves the do-operation, like $do(T=t)$, which denotes the intervention of setting the variable $T$ to $t$, realizing by blocking the effect of $T$’s parents on $T$ and set the value of $T$ as $t$. For example, if we $do(B=b_0)$ on the model in Fig.~\ref{fig:inter}(a), Equations~\ref{eq:structural_example} will be modified as:

\begin{align}
	\label{eq:intervention}
	\begin{cases}
	a&=\ f_A(u_A), \\
	b&=\ b_0, \\
	c&=\ f_C(a, b_0, u_C), 
	\end{cases}
\end{align}
the graphical description of which is shown in Fig.~\ref{fig:inter}(b). 



It is crucial to note that $\textrm{Pr}(Y=y|do(T=t))$ and $\textrm{Pr}(Y=y|T=t)$ are not the same. For example. the fork structure (Fig.\ref{fig:junction}(b))might represent the causal mechanism that connects the number of sales at a local ice cream shop on that day ($A$), a day’s temperature in a city ($B$), and the number of violent crimes in the city on that day ($C$)~\cite{glymour2016causal}. Because both ice cream sales and violent crime are more common in hot weather, a positive correlation might be found when estimate $P(C=c|A=a)$. However, as illustrated in manipulated graphical model of Figure~\ref{fig:inter}, crime rates $C$ are independent of ice cream sales $B$, which results in a different $\textrm{Pr}(C=c|do(A=a))$ from $\textrm{Pr}(C=c|A=a)$.

Although causal graph manipulation is the most fundamentalist approach to calculating $\textrm{Pr}(Y=y|do(T=t))$, it can be challenging and even impossible in reality. Fortunately, we can estimate $\textrm{Pr}(Y=y|do(T=t))$ from observed data with the following causal effect rule:
\begin{definition}[The Causal Effect Rule]
\label{def:ce_rule}
Given a graph $G$ in which a set of variables $PA$ are designated as the parents of $T$, the causal effect of $T$ on $Y$ is given by
\begin{equation}
\label{eq:ce_rule}
\textrm{Pr}(Y=y \mid do(T=t)) 
= \sum_{x} \textrm{Pr}(Y=y \mid T=t, PA=x) \textrm{Pr}(PA=x) 
= \sum_{x} \frac{\textrm{Pr}(T=t, Y=y, PA=t)}{\textrm{Pr}(T=t \mid PA=x)}, 
\end{equation}
where $x$ ranges over all the combinations of values that the variables in $PA$ can take.
\end{definition}
The most important benefit brought by the rule is that it enables us to finish the  \textit{do}-calculus purely on passive observational data~\cite{xu2021causal}. The factor $\textrm{Pr}(T=t \mid PA=x)$ is the propensity score, and Equation \ref{eq:ce_rule} is named \textit{inverse propensity score} (Section \ref{sec:ips}) in PO framework, which partly reflects the unity of the two frameworks.

%It is crucial to note that P $(Y=y|do(X=x))$ and $P (Y=y|X=x)$ are not the same. For the above example. the chain structure might represent the causal mechanism that connects a day’s temperature in a city ($A$), the number of sales at a local ice cream shop on that day ($B$), and the number of violent crimes in the city on that day ($C$)~\cite{glymour2016causal}. A spurious correlation is that an increase in ice cream sales is correlated with an increase in violent crime—not because ice cream causes crime, but because both ice cream sales and violent crime are more common in hot weather. If we were to intervene to make ice cream sales low, we would have the graphical model shown in Figure~\ref{fig:inter}, which shows that crime rates are independent of ice cream sales. Since it is impossible to carry out the physical experiment in the real world, we intervene with the \textit{do}-calculus: 

\textbf{Counterfactuals.} Counterfactuals are employed to emphasize our wish to compare two outcomes under the exact same conditions, differing only in one aspect: the \textit{antecedent}, or \textit{hypothetical condition}~cite{glymour2016causal}. For example, in the counterfactual question “What would be the user’s interaction if the recommended items had been different?” mentioned above in the ~\ref{sec:intro}, we would like to compare the user’s interaction under the same conditions except for the recommended item. Counterfactuals, situations which are non-existent in reality, cannot be inferred by do-calculus. Fortunately, Pearl~\cite{ pearl2009causality} proposed a new set of notations: $ \textrm{Pr}(Y(T=1)|T = 0, Y = Y(T=0))$ indicates the probability of the outcome $Y(T=1)$ would be if the observed treatment value is $T=0$, given the fact that we observe $Y=Y(T=0)$ in the data.


\subsubsection{Comparison between the two frameworks}
As mentioned above, the two frameworks are equivalent logically: an assumption or a theorem can be translated to its counterpart in the other,  and a problem solved in one framework would yield the same solution in another~\cite{pearl2009causality, glymour2016causal}. For example, $\textrm{Pr}(Y = y | do(T = t))$ in the SCM is equivalent to $\textrm{Pr}(Y(T=t)=y)$ in the PO, which the regular assessment in a controlled experiment, in which the distribution of $Y$ is estimated for each level $w$ of a random variable $T$. Causal effects that are measured between the results of the counterfactual world and the real world can be estimated conveniently in both frameworks. However, there are several important differences between PO and SCM. The most significant difference is that PO does not assume the causal relations between concerned variables, while SCM makes assumptions of causal mechanisms among a set of variables or searches for ones based on some assumptions. In other words, any given PO model corresponds to multiple causal graphs in SCM. For PO, it can be a strength for PO that causal effects can be reasoned without knowing the causal model, and be a weakness either. According to the unconfoundednes assumption, all confounders should be observed to infer a correct treatment effect since the mechanism is unknown, almost impossible in practice~\cite{aliprantis2015distinction}. In contrast, in SCM, causal diagrams allow us to work with causal effects by interventions on the fewest number of variables or the observed variables as much as possible.

\subsection{Recommender Systems}

Recommender systems predict users’ preferences and proactively recommend items users might like~\cite{ricci2015recommender, zhang2019deep} to alleviate information overload. 

\subsubsection{Recommendation Techniques}

RSs are usually classified into the following three categories~\cite{adomavicius2005toward, zhang2019deep}: content-based, collaborative filtering (CF), and hybrid. Content-based recommendation learns to recommend primarily based on comparisons across items’ and users’ auxiliary information~\cite{zhang2019deep}, such as items’ human-set tags, images, texts, and users’ sex. Collaborative filtering recommender systems recommend items according to user/item historical interactions, i.e., explicit (e.g., user’s previous ratings) or implicit feedback (e.g., click behavior)~\cite{zhang2019deep}. Hybrid approaches are those that combine collaborative filtering and content-based methods.

If we review the model structure, recommender systems can generally be divided into shallow models, neural network models, and GNN-based models~\cite{gao2022graph}. Shallow models involve methods that directly calculate the similarity of interactions and CF methods with matrix factorization (MF)~\cite{koren2009matrix} or factorization machine (FM) ~\cite{rendle2010factorization}, but suffer from insufficient learning of users’ complicated interest. Neural network models are proposed to solve this issue, with the advantage of high-order feature interactions~\cite{guo2017deepfm}. For example, Wide \& Deep~\cite{cheng2016wide} jointly trains linear models and deep neural networks to combine the benefits of memorization and generalization. Deep factorization machine (DeepFM)~\cite{guo2017deepfm} combines traditional  factorization machine (FM) with multi-layer perceptrons (MLP) in parallel. Graph neural networks (GNN) adopt embedding propagation to aggregate neighborhood embedding iteratively. GNN-based methods have become the new state-of-the-art approaches in recommender systems~\cite{gao2022graph}.

\subsubsection{Notation}

Considering a general recommender system, we assume $\mathcal{O}$ and $\mathcal{O}^{-}$ denote the observed dataset and unobserved dataset. Each observed sample includes the treatment $T$, background features $X$, and an interaction label $Y$. Background features $X$, aka., covariates are usually formulated as a high dimensional sparse vector containing information such as user ID, item ID, user profile, item category, etc. The interaction label $Y$, or the outcome, can be explicit feedback (e.g., rating) or implicit feedback (such as click and watch behavior). 
%$\bm{f}_u$ and $\bm{f}_i$ denote user features and item features, respectively, which is usually a high dimensional sparse vector with multi-fields~\cite{rendle2010factorization}, such as user ID, item ID, user profile, item category, etc.

In normal circumstances, researchers prefer choosing whether to recommend as the treatment. Therefore, the observed dataset can be denoted as ${\mathcal{O}} = \{(T=1,X,Y\}|_{1}^{|\mathcal{O}|} \in \mathcal{T} \times \mathcal{X} \times \mathcal{Y}$, where $\mathcal{T}$ means the treatment space, $\mathcal{X}$ is the feature spaces, and $\mathcal{Y}$ is the label space. In general, the observed dataset is obtained with the deployed recommender policy $\pi$; thus ${\mathcal{O}}$ will be specifically expressed as ${\mathcal{O}_\pi}$ if we are concerned about the policy. Note that settings of $T, X, Y$ vary slightly according to specific work. It would be better to understand with reference to the context.

\subsection{Existing Categorizations of Causal Recommendation}
\label{sec:exist_cate}
There are several categorization criteria for causal recommender systems. For example, similar to ~\cite{yao2021survey}, Yao et.al.~\cite{wu2022opportunity} divides biases in RS into three categories from the perspective of violating what causal assumptions are adopted in the standard PO framework. 1) Position bias and conformity bias can be seen as violations of the SUTVA assumption if recommender systems do not pay enough attention to the positions of items and users’ social networks. 2) Unconfoundedness and positivity are crucial assumptions in the recoverability of the target estimated. However, the former can be violated by popularity bias, and the latter can be violated by exposure bias, both of which result in the problem of missing not at random (MNAR). 3) The final bias violates some model-specific assumptions.
%Another categorization is based on the model specification. 





According to the survey~\cite{gao2022causal}, existing work of causal recommendation can be categorized into three groups: for addressing data bias, for addressing data missing and noise, and for beyond-accuracy objectives. 1) Causal debiasing work can be further divided into several subcategories based on the specific bias, such as popularity bias, clickbait bias, and exposure bias. 2) The problem of data missing refers to the usually-discussed data sparsity issue in RS, and data noise stems from unreliable implicit signals and delayed feedback. In order to alleviate these issues, researchers use the counterfactual technique to augment insufficient data and adjust sample weights. Besides, some causal recommender systems are designed for beyond-accuracy objectives like explainability, diversity and fairness. 

~\cite{zhu2023causal} summarizes different causal inference techniques with an emphasis on debiasing, explainability promotion, and generalization improvement. 

%For better understanding, Figure~\ref{fig:exist_cate} shows the above-mentioned categorizations of previous work of causal inference for recommendation.



The three studies mentioned above are pioneering efforts in this field, and each has a distinct focus on the causal frameworks it discussed. However, this paper will systematically classify causal inference for RS from a new perspective of the employed causal approach. We regard that, although it is quite convenient for researchers, especially those in RS industry, to speed up the implementation of causal RS techniques from the perspective of what application issues they aim to address, continuous research and significant breakthroughs of in-depth integration of causal inference and recommender systems require a comprehensive understanding of causal techniques, including their strengths as well as their limitations, which our work will have a greater contribution.



\section{PO-based Method}
\label{sec:po_based}


Many causal recommendation approaches, especially in early research, focus on applying the potential outcome framework proposed by Donald B. Rubin ~\cite{splawa1990application, rubin1974estimating, imbens2015causal} to the effect evaluation of the recommendation policy, more specifically, on the optimization functions in traditional deep-learning-based methods and on the reward functions in reinforcement-learning-based methods. Figure~\ref{fig:classification} provides the strategies, the objectives
%, and the common RS scenario of the approaches
concerning the PO framework. As shown in Figure~\ref{fig:classification}, There are generally two strategies in the potential outcome framework for RS, i.e., inverse propensity score and causal effect. It is worth mentioning that in this survey, a model estimating causal effects without a causal graph will be regarded as a PO-based model, while a model with a causal graph will be regarded as an SCM-based one, though both frameworks involve the estimation of causal effects. In this section, some related causal recommendation approaches are introduced in proper order according to the strategies shown in Figure~\ref{fig:classification}.

\begin{figure*}[t!]
\centering
\vspace{-3mm}
\includegraphics[height=0.56\textheight,trim=140 0 0 0 ,clip]{pic/classification.pdf} %l b r t
\vspace{-1mm}
\caption{Strategies of the causal inference for recommendation.}
\vspace{-6mm}
\label{fig:classification}
\end{figure*}

\subsection{Propensity Score Strategy}

Let's consider the process by which the recommendation system works, where given background variables $x \sim \textrm{Pr}(x)$, also referred to as pre-treatment variables or covariates~\cite{imbens2015causal}, (e.g., user and item features, time of the day, etc.), a recommender policy $\pi$ plays a role as a decision-making system, which makes a decision of whether to take an active treatment $t \sim \pi(t \mid x) $ (e.g., recommend an item), and the potential outcome $y \sim \textrm{Pr}(y \mid x, t) $, i.e., “reward” in the reinforcement learning context (e.g., click indicator), will be observed ~\cite{saito2022off}. For example, in online markets, information like user profile, historical consumptions, and products in the cart will be treated as context variables $x$, according to which the policy $\pi$ will produce a list of recommended items (i.e., treatment $t$), and the logged reward $y$ can be the click signal, conversions, or revenue, etc. The effectiveness of the policy $\pi$ can be evaluated through its running expected reward, formulated as:
\begin{equation}
R(\pi) :=\iiint y \textrm{Pr}(y \mid x, t) \pi(t \mid x) \textrm{Pr}(x) d x d t d y
=\mathbb{E}_{\textrm{Pr}(x) \pi(t \mid x) \textrm{Pr}(y \mid x, t)}[y].
\end{equation}

To learn the optimal policy 

\begin{equation}
\pi \in \underset{\pi \in \mathcal{\Pi}}{\arg \max } V(\pi),
\end{equation} 
where $\Pi$ means the policy class, an online A/B test will be the best choice~\cite{gomez2015netflix, kohavi2013online}, but   suffers from high expense. A substitute and common practice is offline evaluation,  by calculating an estimator $\hat{R}$ for the reward of a target policy $\pi$ using logged data $\mathcal{O}_{\pi_0}$ collected by a logging policy $\pi_0$ (which is different from $\pi$)~\cite{saito2022off}. However, like many other empirical sciences, offline evaluation is challenged with the problem of \emph{missing not at random} (MNAR). 

To address this issue, early approaches tend to predict the missing data directly~\cite{steck2010training} but have accentuated the problem of high bias~\cite{wang2019doubly, saito2021counterfactual}. Recently, many researchers have resorted to the \emph{propensity score} $e(X)$ in causality to recover the data distribution. For example, ExpoMF~\cite{liang2016modeling} first predicts the exposure matrix and then uses the exposures (i.e., propensity scores) to guide the model of the interaction matrix, which is inspired by the separation between propensity scores and potential outcomes in the PO framework. Similarly, Wang et al.~\cite{wang2018collaborative} propose SERec to integrate social exposure into collaborative filtering. A refreshing work is that Wang et al.~\cite{wang2020causal} aim to overcome the confounder issue with propensity score. They regard correlations among the interacted items as bringing indirect evidence for confounders and propose the deconfounded recommenders. They first build an exposure model to estimate the propensity score and then use this exposure model to estimate a substitute for the unobserved confounders, conditional on which the final outcome model (specifically in ~\cite{wang2020causal}, a rating model based on matrix factorization) is trained. In addition, inspired by ~\cite{joachims2017unbiased, fang2019intervention}, Chen et al.~\cite{chen2021adapting} propose IOBM (Interactional Observation-Based Model)to estimate propensity score in interaction settings, which learns low-dimensional embeddings as a substitute for unobservable confounders . Specifically, it learns individual embeddings to capture the potential outcome information from specific exposure events. Based on individual embeddings, the interactional embeddings, which uncovers the hidden relationship among single exposure events and utilizes query context information to apply attention, are learned through the bidirectional LSTM model. 

Propensity-based methods can be further divided into approaches based on inverse propensity score (IPS) and approaches based on doubly robust (DR) (Fig.\ref{fig:propensity}). One of the greatest strengths of applying propensity-based methods in RS is that most of them are unbiased and model-agnostic, simply deployed on the objective function for policy evaluation directly or for policy learning indirectly.




\subsubsection{Missing Not At Random}
\label{sec:MNAR}

In this part, we will introduce the phenomena and factors of missing not at random, to provide explanations and conclusions of challenges in recommender systems in a causal language to understand existing work better.

Recommendation algorithms often obey the missing at random (MAR)~\cite{rubin1976inference} assumption but may lead to biased prediction and suboptimal policy~\cite{little2019statistical, marlin2009collaborative}. The MAR condition essentially states that the probability that a potential outcome is missing does not depend on the value of that potential outcome and can be easily violated in recommender systems~\cite{marlin2009collaborative}. For example, on movie rating websites, movies with high ratings are less likely to be missing compared to movies with low ratings~\cite{pradel2012ranking}. The issue of missing not at random (MNAR) has been demonstrated by Marlin and Zemel ~\cite{marlin2009collaborative} and it is a phenomena stemming from \emph{selection bias} and \emph{confounding bias}~\cite{correa2019identification, wu2022opportunity}. 

\begin{figure}[t!]
    \centering
    \vspace{-3mm}
    \subfloat[User self-selection bias]{
    	\centering
    	\includegraphics[height=0.09\textheight, trim=-10 0 -10 0, clip]{pic/MNAR0.pdf}
    }
    \centering
    \subfloat[Confounding bias]{
    	\centering
    	\includegraphics[height=0.09\textheight, trim=-10 0 -10 0, clip]{pic/MNAR1.pdf}
    }
    \centering
    \vspace{-3mm}
    \caption{Causal explanation of user self-selection bias and confounding bias.}
    \vspace{-6mm}
    \label{fig:selection_confounding}
\end{figure}



Selection bias, or sampling bias, is usually discussed in the prediction task and can be further classified into model selection bias and user self-selection bias~\cite{wu2022opportunity}. For example, the case that the platform may systematically recommend pop music to younger users who may be more active on the service regardless of genre preferences~\cite{mcinerney2020counterfactual} will be regarded as model selection bias~\cite{yuan2019improving, mcinerney2020counterfactual} and can be eliminated by random recommendation. User self-selection bias~\cite{bareinboim2012controlling, elwert2014endogenous}, on the contrary, can not be removed by randomization of recommendation~\cite{correa2019identification}. It is caused by preferential exclusion of samples from the data~\cite{bareinboim2012controlling}. A typical example is a song recommender system, in which users usually rate songs they like or dislike and seldom rate what they feel neutral about~\cite{saito2020asymmetric}. Some of the most frequently discussed biases like popularity bias~\cite{zhang2021causal, wei2021model} and exposure bias~\cite{liang2016modeling, wang2018collaborative} will lead to model selection bias, while conformity bias~\cite{zhang2021causal, zheng2021disentangling} and clickbait bias~\cite{wang2021clicks} fall under user self-selection bias as a result of user preference.




Confounding bias~\cite{hernan2002causal, pearl2009causality} arises from the confounder described in Section \ref{sec:scm}, which affects both the treatment and the outcome, illuminated in Figure\ref{fig:selection_confounding}(b). Alternatively, it can be identified if the probabilistic distribution representing the statistical association is not always equivalent to the interventional distribution, i.e., $\textrm{Pr}(y \mid t) \neq \textrm{Pr}(y \mid do(t))$~\cite{guo2020survey}. A notable example of confounding bias is that a system trained with historical user interactions may over recommend items that the user used to like, and the user’s decision (i.e., outcome) is also affected by historical interactions~\cite{wang2021deconfounded}. 




Both biases can lead to invalid estimates of causality from the data, and they are not mutually exclusive because selection bias does not explicitly involve causality. Many model selection biases, including popularity bias and exposure bias, are also confounding biases. As for user self-selection bias, the model in Fig. \ref{fig:selection_confounding} (a) gives an illustration of its causal nature in which $S$ is a variable affected by both $T$ (treatment) and $Y$ (outcome), indicating entry into the data pool~\cite{bareinboim2012controlling}. Therefore, confounding bias is significantly different from user self-selection bias from the causal perspective. The former originates from common causes, whereas the latter originates from common outcomes~\cite{elwert2014endogenous}. The former stems from the systematic bias introduced during the treatment assignment, while the latter comes from the systematic bias during the collection of units into the sample~\cite{correa2019identification}.


\begin{figure*}[t!]
\centering
\vspace{-1mm}
\includegraphics[height=0.4\textheight,trim=0 0 0 0 ,clip]{pic/propensity_illustration.pdf} %l b r t
\vspace{-3mm}
\caption{Illustration of propensity score strategies for recommendation.}
\vspace{-6mm}
\label{fig:propensity}
\end{figure*}



\subsubsection{Inverse Propensity Score}
\label{sec:ips}
Inverse Propensity Score (IPS)~\cite{horvitz1952generalization, rosenbaum1987model, rosenbaum1983central,little2019statistical}, also named as inverse propensity weighting (IPW), or inverse propensity of treatment weighting (IPTW), is one of the favorite counterfactual techniques and has inspired a lot of causal inference methods in RS, especially for unbiased learning~\cite{joachims2017unbiased}. Propensity score is the probability of receiving the treatment given covariates $X$, formulated as:
\begin{equation}
e_{\pi}(X) = \textrm{Pr}_{\pi}(T=1 \mid X).
\end{equation}
IPS assigns a weight $w$ to each sample:
\begin{equation}
w=\frac{t}{e(x)}+\frac{1-t}{1-e(x)},
\end{equation}
which indicates the inverse probability of receiving the \emph{observed} treatment and control. The unbiasedness of IPS can be proven~\cite{rosenbaum1987model}. More specifically, for the reward estimation of recommendation policy, IPS adjusts the distribution of background features in the logged dataset to be consistent with that during $\pi$ tests online, formulated as:
\begin{equation}
\hat{R}_{\mathrm{IPS}}\left(\pi; \mathcal{O}_{\pi_0}\right):=\frac{1}{\mathcal{O}_{\pi_0}} \sum_{k=1}^{|\mathcal{O}_{\pi_0}|} \frac{e_{\pi}(X)}{e_{\pi_0}(X)} \cdot y_k = \frac{1}{\mathcal{O}_{\pi_0}} \sum_{k=1}^{|\mathcal{O}_{\pi_0}|} \frac{\textrm{Pr}_{\pi}(T=1 \mid X)}{\textrm{Pr}_{\pi_0}(T=1 \mid X)} \cdot y_k,
\end{equation}
where we assume that only positive feedback is taken into account, and $w= \frac{e_{\pi}(X)}{e_{\pi_0}(X)} $ is the ratio of the evaluation and logged policies. Note that in most applications in RS, IPS is model-agnostic, applied to the training objective function for policy evaluation directly or for policy learning indirectly.



\begin{table*}[pt]
  \small
  \centering
  \vspace{-1mm}
  \caption{\normalsize Summary of propensity score strategies for recommendation.}
  \vspace{-2mm}
  \setlength{\tabcolsep}{1.5mm}{
    \begin{tabular}{c|c|c|c|c|c}
    \toprule
    \textbf{Category} & \textbf{Model} & \textbf{Causal method} & \textbf{Backbone model} & \textbf{Issue of concern} & \textbf{Year} \\
    \midrule
    \multicolumn{1}{c|}{\multirow{5}[2]{*}{\makecell{Approach \\ Inspired by \\Propensity\\ Score}}} & ExpoMF~\cite{liang2016modeling} & Propensity score & MF  & Exposure bias & 2016 \\
          & SERec~\cite{wang2018collaborative} & Propensity score & MF  & Social recommendation & 2018 \\
          & Dcf~\cite{wang2020causal} & Propensity score & MF  & Unobserved confounding bias  & 2020 \\
          & CNFI~\cite{zhang2021causalneural} & Propensity score & MF    & Implicit feedback & 2021 \\
          & IOBM~\cite{chen2021adapting} & Propensity score & Bi-LSTM~\cite{graves2013speech}  & Interactional observation bias & 2021 \\
    \midrule
    \multirow{19}[2]{*}{\makecell{Approach\\ with Inverse\\ Propensity \\Score (IPS)}} & MF-IPS~\cite{schnabel2016recommendations} & IPS, SNIPS & MF  & Selection bias & 2016 \\
          & PBM~\cite{joachims2017unbiased} & IPS   & SVM-Rank~\cite{joachims2002optimizing, joachims2006training} & Position bias & 2017 \\
          & ~\cite{mehrotra2018towards} & IPS   & (reinforcement learning) & Fairness & 2018 \\
          & Multi-IPW~\cite{zhang2020large} & IPS   & Multi-task DNN & Selection bias & 2019 \\
          & CPBM~\cite{fang2019intervention} & IPS   & SVM-Rank & Selection bias & 2019 \\
          & ULRMF,ULBPR~\cite{sato2019uplift} & IPS, SNIPS, ATE   & MF  & Uplift & 2019 \\
          & DLCE~\cite{sato2020unbiased} & CIPS  & MF  & Unobserved confounding bias  & 2020 \\
          & Rel-MF~\cite{saito2020unbiased} & CIPS  & MF  & Unobserved confounding bias  & 2020 \\
          & ~\cite{christakopoulou2020deconfounding} & IPS   & Multi-task DNN & Observed confounding bias & 2020 \\
          & RIPS~\cite{mcinerney2020counterfactual} & RIPS  & (model-agnostic) & Slate recommendation & 2020 \\
          & ACL-~\cite{xu2020adversarial} & IPS   & \makecell{(adversarial learning)} & Identifiability & 2020 \\
          & UR-IPW~\cite{zhang2021user} & SNIPS & Multi-task DNN & \makecell{Post-click revisit effect\\ \&selection bias} & 2021 \\
          & ~\cite{li2021debiasing} & IPS   & (model-agnostic) & Domain bias & 2021 \\
          & CBDF~\cite{zhang2021counterfactual} & IPS   & (reinforcement learning) & Delayed feedback & 2021 \\
          & RD\&BRD~\cite{ding2022addressing} & \makecell{IPS/DR/\\ AutoDebias~\cite{chen2021autodebias}} & MF  & Unobserved confounding bias  & 2022 \\
          & DENC~\cite{li2022causal} & IPS   & (self-designed) & Selection bias & 2022 \\
          & CET~\cite{cai2022hard} & IPS   & BERT  & False negative  & 2022 \\
          & CAFL~\cite{krauth2022breaking} & IPS   & MF  & Feedback loop & 2022 \\
          & RIIPS~\cite{liu2022practical} & RIIPS  & Two-tower structure  & Selection bias & 2022 \\
    \midrule
    \multirow{6}[2]{*}{\makecell{Approach \\with Doubly\\ Robust}} & Propensity-free DR~\cite{yuan2019improving} & DR    & FFM~\cite{yuan2019one} & Selection bias & 2019 \\
          & Multi-DR~\cite{zhang2020large} & DR    & Multi-task DNN & Selection bias & 2019 \\
          & MRDR-DL~\cite{guo2021enhanced} & MRDR  & MF  & Selection bias & 2021 \\
          & Cascade-DR~\cite{kiyohara2022doubly} & Cascade-DR & MF  & High variance of RIPS & 2022 \\
          & ASPIRE~\cite{mondal2022aspire} & DR, ATE    & LightGBM~\cite{ke2017lightgbm} & Uplift & 2022 \\
          & DRIB~\cite{xiao2022towards} & DR    & MF  & Unobserved confounding bias  & 2022 \\
    \bottomrule
    \end{tabular}}%
    \label{tab:propensity}%
    \vspace{-6mm}
\end{table*}%


%A lot of IPS-based recommendation focuses on data debiasing in user interactions, mainly selection bias ~\cite{schnabel2016recommendations,saito2020unbiased,sato2020unbiased,zhang2021user,sato2021online, wu2021unbiased, li2022causal}. For example, ~\cite{schnabel2016recommendations} is a representative work adopting IPS to recommender system for the elimination of selection bias, in which the recommendation algorithm is based on matrix factorization and propensity scores are estimated via naive Bayes or logistic regression. Similarly account for selection bias, Saito et al. estimates the exposure propensity for each user-item pair~\cite{saito2020unbiased} and Sato et al. proposes the DLCE (Debiased Learning for the Causal Effect) model with IPS-based estimators to evaluating unbiased ranking uplift ~\cite{sato2020unbiased}. Unbiased IPS-based uplift is also concerned by ~\cite{sato2019uplift}. In addition, ~\cite{zhang2021user} proposes UR-IPW (User Retention Modeling with Inverse Propensity Weighting) to model revisit rate estimation accounting for the selection bias problem and ~\cite{li2021debiasing} adjusts domain weights based on IPS to reduce domain bias. Though IPS-based methods do not require an explicit analysis of the causal correlation between variables, some works~\cite{christakopoulou2020deconfounding, mcinerney2020counterfactual, ding2022addressing} still discuss causal graphs as a good guide to accurate model. For example, Ding et al.~\cite{ding2022addressing} leverage a causal graph to explain the risk of unmeasurable confounders on the accuracy of propensity estimation and propose RD (Robust Deconfounder) with the sensitivity analysis, obtaining the bound of propensity score to enhance the robustness of methods against unmeasured confounders. 

Much IPS-based recommendation focuses on data debiasing in user interactions, mainly selection bias ~\cite{schnabel2016recommendations,saito2020unbiased,sato2020unbiased,zhang2021user,sato2021online, zhang2021causalneural, wu2021unbiased, li2022causal}. For example, ~\cite{schnabel2016recommendations} is a representative work adopting IPS to recommender system for the elimination of selection bias, in which the recommendation algorithm is based on matrix factorization and propensity scores are estimated via naive Bayes or logistic regression. Similarly, Saito et al.~\cite{saito2020unbiased} estimate the exposure propensity for each user-item pair and Sato et al.~\cite{sato2020unbiased} propose the DLCE (Debiased Learning for the Causal Effect) model with IPS-based estimators to evaluating unbiased ranking uplift. Unbiased IPS-based uplift is also concerned by ~\cite{sato2019uplift}. In addition, ~\cite{zhang2021user} proposes UR-IPW (User Retention Modeling with Inverse Propensity Weighting) to model revisit rate estimation accounting for the selection bias problem and ~\cite{li2021debiasing} adjusts domain weights based on IPS to reduce domain bias. Though IPS-based methods do not require an explicit analysis of the causal correlation between variables, some works~\cite{christakopoulou2020deconfounding, mcinerney2020counterfactual, ding2022addressing} still discuss causal graphs as an excellent guide to accurate model. For example, Ding et al.~\cite{ding2022addressing} leverage a causal graph to explain the risk of unmeasurable confounders on the accuracy of propensity estimation and propose RD (Robust Deconfounder) with the sensitivity analysis, obtaining the bound of propensity score to enhance the robustness of methods against unmeasured confounders. Li et al.~\cite{li2022causal} construct the DENC (De-bias Network Confounding in Recommendation). This causal graph-based recommendation framework disentangles three determinants for the outcomes, including inherent factors, social network-based confounder and exposure, and estimates each of them with a specific component, respectively. 
%but Christakopoulou et al. still discuss the causal graph of user satisfaction and treat response rate as a confounding factor before using IPS to debias user satisfaction estimation from response bias~\cite{christakopoulou2020deconfounding}. 
By the way, there are some works~\cite{christakopoulou2020deconfounding, cai2022hard, zhang2021user} integrate multi-task models with IPS to learn propensity scores and user interactions simultaneously.





In addition to debiasing, some IPS-based methods are dedicated to addressing other issues that abound in RS~\cite{mehrotra2018towards, zhang2021counterfactual, krauth2022breaking}. For example, Mehrotra et al.~\cite{mehrotra2018towards} proposes an unbiased estimator of user satisfaction based on IPS to jointly optimize for supplier fairness and consumer relevance. Besides, the CBDF (Counterfactual Bandit with Delayed Feedback) algorithm ~\cite{zhang2021counterfactual} re-weights the observed feedback with importance sampling, which is determined by a survival model to deal with delayed feedbacks. The CAFL (causal adjustment for feedback loops) ~\cite{krauth2022breaking} extends the IPS estimator to break feedback loops. 

Despite the unbiasedness strength of IPS, the inaccurate estimation of the unknown propensity $e(x)$ or sample weight, which results in high variance~\cite{gilotte2018offline}, becomes the biggest obstacle to achieving it. To alleviate this problem, modified versions of IPS have been proposed to control variance and applied to RS, including Self Normalized IPS ~\cite{schnabel2016recommendations, zhang2021user}, Clipped IPS~\cite{saito2020unbiased, sato2020unbiased}, Reward interaction IPS~\cite{mcinerney2020counterfactual}, and Regularized per-Item IPS~\cite{liu2022practical}. Self Normalized Inverse Propensity Scoring (SNIPS)~\cite{swaminathan2015self} rescales the estimate of the original IPS without any parameters to reduce the high variance, which is:
\begin{equation}
\hat{R}_{\mathrm{SNIPS}}\left(\pi; \mathcal{O}_{\pi_0}\right):=
\left(\sum_{k=1}^{|\mathcal{O}_{\pi_0}|} \frac{e_{\pi}(X)}{e_{\pi_0}(X)}\right)^{-1}
\sum_{k=1}^{|\mathcal{O}_{\pi_0}|} \frac{e_{\pi}(X)}{e_{\pi_0}(X)} \cdot y_k,
\end{equation}
and is introduced to RS by works like ~\cite{schnabel2016recommendations} and ~\cite{zhang2021user} to alleviate selection bias. Clipped IPS (CIPS)~\cite{saito2020unbiased, sato2020unbiased} tightens the bound of the sample weight by introducing a scalar value hyperparameter $\lambda_{\mathrm{CIPS}}$, formulated as:
\begin{equation}
\hat{R}_{\mathrm{CIPS}}\left(\pi; \mathcal{O}_{\pi_0}\right):=
\frac{1}{\mathcal{O}_{\pi_0}}
\sum_{k=1}^{|\mathcal{O}_{\pi_0}|} \min \left\{ \frac{e_{\pi}(X)}{e_{\pi_0}(X)} , \lambda_{\mathrm{CIPS}} \right\}\cdot y_k,
\end{equation}
which has a lower variance but gives away its unbiasedness. McInerney et al.~\cite{mcinerney2020counterfactual} loosen the SUTVA assumption and propose Reward interaction IPS (RIPS) for sequential recommendations, which assumes a causal model in which users interact with a list of items from the top to the bottom. RIPS uses iterative normalization and lookback to estimate the average reward and achieves a better bias-variance trade-off than IPS. In addition to high variance, violation of the Unconfoundedness assumption is another challenge of utilizing IPS in RS. That is, the treatment mechanism is \emph{identifiable}~\cite{glymour2016causal, mohan2021graphical} from observed covariates due to the existence of unobserved ones, which leads to the inaccurate estimate of propensity score and the disagreement between the online and offline evaluations. To address the uncertainty brought by the identifiability issue, ~\cite{xu2020adversarial} proposes minimax empirical risk formulation, which can be converted to an adversarial game between two recommendation models via duality arguments and relaxations.

More recently, Liu et al.~\cite{liu2022practical} propose Regularized per-item IPS (RIIPS) with an additional penalty function that constrains the difference in recommended outcomes between the deployed system and the new system so that the explosion of propensity scores can be avoided.




\subsubsection{Doubly Robust}
Doubly Robust (DR)~\cite{funk2011doubly, dudik2014doubly, jiang2016doubly, wang2019doubly} is another powerful and effective causal method account for the MNAR issue. To understand DR, let us consider the two common-used approaches to mitigate against MNAR: direct method (DM)~\cite{beygelzimer2009offset} and IPS~\cite{saito2021evaluating}. The former designs a model (linear regression, deep neural network, etc.) to directly learn the missing outcomes based on the observed data, which has low variance due to the advantage of supervised learning but suffers from high bias caused by unmet IID assumptions, denoted as~\cite{saito2021evaluating}:
\begin{equation}
\hat{R}_{\mathrm{DM}}\left(\pi_{0} ; \mathcal{O}_{\pi_{0}}, \hat{y}\left(x_{k}, t\right) \right):=\frac{1}{|\mathcal{O}_{\pi_{0}}|} \sum_{k=1}^{|\mathcal{O}_{\pi_{0}}|}\textrm{Pr}_{\pi}\left(t=1 \mid x_{k}\right) \hat{y}\left(x_{k}, t\right),
\end{equation}
where $\hat{y}\left(x, t\right)$ is the estimated outcomes. The latter, though unbiased theoretically, often causes training losses to oscillate stemming from the inverse of propensity with high variance~\cite{thomas2016data}. What DR does is to combine the direct method and IPS, which takes advantage of both and overcomes their limitations:
\begin{equation}
\hat{R}_{\mathrm{DR}}\left(\pi ; \mathcal{O}_{\pi_0}, \hat{r}\right) := \hat{R}_{\mathrm{DM}}\left(\pi ; \mathcal{O}_{\pi_0}, \hat{y}\left(x_{k}, t\right) \right) + 
\frac{1}{|\mathcal{O}_{\pi_{0}}|} \sum_{k=1}^{|\mathcal{O}_{\pi_{0}}|} 
\frac{e_{\pi}(X)}{e_{\pi_0}(X)} \left(y_{k}-\hat{y}\left(x_{k}, t_{k}\right)\right). 
\end{equation}
DR uses the estimated outcomes to decrease the variance of IPS. It is also \emph{doubly robust} in that it is consistent with the policy reward value if either the propensity scores or the imputed outcomes are accurate for all user-item pairs~\cite{wang2019doubly, saito2021evaluating}. By the way, advanced versions like Switch-DR~\cite{wang2017optimal} and DRos (Doubly Robust with Optimistic Shrinkage)~\cite{su2020doubly} are proposed to further control the variance.

Based on the above advantages, DR has found an increasingly wide utilization in RS~\cite{yuan2019improving, zhang2020large, guo2021enhanced, kiyohara2022doubly, mondal2022aspire, xiao2022towards}. Yuan et al.~\cite{yuan2019improving} propose a propensity-free doubly robust method to address the issue that samples with low propensity scores are absent in the observed dataset. Zhang et al.~\cite{zhang2020large} propose Multi-DR based on a multi-task learning framework to address selection bias and data sparsity issues in CVR estimation. Gun et al.~\cite{guo2021enhanced} propose the MRDR (more robust doubly robust) estimator to further reduce the variance caused by inaccurate imputed outcomes in DR while retaining its double robustness. In addition, Kiyohara et al.~\cite{kiyohara2022doubly} expand previous RIPS to Cascade Doubly Robust estimator, which has the same user interaction assumption as RIPS. Xiao et al.~\cite{xiao2022towards} propose an information bottleneck-based approach to effectively learn the DR estimator for the estimation of recommendation uplift, with the hope of a better trade-off between the bias and variance of propensity scores. 

\subsection{Causal Effect Strategy}

The most critical and fundamental role of causal inference is to estimate the causal effects from observational data, which has a variety of applications in real-world recommender systems. Some works are dedicated to enhancing the causal effect of a recommender policy, i.e., uplift, and therefore deploy the causal effect as a direct or indirect optimization goal for higher platform benefits. Other works introduce treatment effects to recommender systems for beyond-uplift objectives.
%$ATE$ or $CATE$ to improve representation learning and alleviate selection bias. 
Note that in the PO framework, the causal relationship between variables will not be discussed while calculating causal effect, and all variables affecting potential outcomes except treatment will be treated as covariates.



\begin{table*}[pt]
  \small
  \centering
  % \renewcommand\arraystretch{1.5}
  \vspace{-3mm}
  \caption{\normalsize Summary of causal effect strategies for recommendation.}
  \vspace{-3mm}
  \setlength{\tabcolsep}{2.4mm}{
    \begin{tabular}{c|c|c|c|c|c}
    \toprule
    \textbf{Category} & \textbf{Model} & \textbf{Causal method} & \textbf{Backbone model} & \textbf{Issue of concern} & \textbf{Year} \\
    \midrule
    \multirow{5}[2]{*}{\makecell{Causal \\effect\\ for\\ Uplift}} & ULRMF, ULBPR~\cite{sato2019uplift} & IPS, SNIPS, ATE & MF  & \multirow{5}[2]{*}{Uplift} & 2019 \\
          & ~\cite{goldenberg2020free}  & CATE  & Xgboost~\cite{chen2016xgboost} &       & 2020 \\
          & AUUC-max~\cite{betlei2021uplift} & CATE  & \makecell{Linear\\/Wide \& Deep\\  }  &       & 2021 \\
          & CausCF~\cite{xie2021causcf} & CATE  & MF  &       & 2021 \\
          & ASPIRE~\cite{mondal2022aspire} & DR, ATE & LightGBM~\cite{ke2017lightgbm} &       & 2022 \\
    \midrule
    \multirow{6}[2]{*}{\makecell{Causal \\effect\\ beyond\\ Uplift}} & ~\cite{rosenfeld2017predicting} & ITE   & \makecell{Linear/regularized \\kernel methods} & Domain adaptation & 2017 \\
          & CausE~\cite{bonner2018causal} & ITE   & MF  & Domain adaptation & 2018 \\
          & ~\cite{mehrotra2020inferring} & TE    & \makecell{Structural \\state-space model~\cite{brodersen2015inferring}} & \makecell{Causal effect of\\a new track release} & 2020 \\
          & CACF~\cite{zhang2021causally} & ITE   & (self-designed) & Unobserved confounding bias & 2021 \\
          & MCRec~\cite{yao2022device} & CATE  & DIN~\cite{zhou2018deep} & \makecell{Device-cloud \\recommendation} & 2022 \\
          & LRIR~\cite{tran2022most} & ITE, ATE & (self-designed) & Disability employment & 2022 \\
    \bottomrule
    \end{tabular}}%
  \label{tab:effect}%
  \vspace{-6mm}
\end{table*}%


\subsubsection{Causal Effect for Uplift}

Uplift, in terms of the causal effect of recommendations, refers to the increase in user interactions purely caused by recommendations. Typical evaluations of recommender systems regard the positive user interactions as a success, such as clicks, purchases, and high rates. However, these interactions may have remained even without recommendations, which can be illustrated by the conclusion of~\cite{sharma2015estimating} that more than 75\% of click-throughs would still occur in the absence of recommendations. Therefore, the industry looks favorably on the uplift as the recommendation metric in expectation of higher reward.


%Let us consider the binary treatments of a user in the two given scenarios (recommended or not) to explain the concept of uplift in the PO framework, and the user  The causal effects will fell into four groups: 

%The Two-Model approach ~\cite{radcliffe2007using, nassif2013uplift} directly models $\mathbb{E}[Y(T=1|X=x)]$ and \mathbb{E}[Y(T=0|X=x)]$ separately using the treatment group data and the control group data, respectively, and $CATE$, the difference between them, is treated as uplift~\cite{gutierrez2017causal}. This approach is simple but may suffer from accumulative error in the subtraction of two separate predictions. Other causal approaches with traditional machine learning methods for uplift estimation are proposed in ~\cite{jaskowski2012uplift, jaskowski2012uplift, radcliffe2011real}.

It is a natural application to introduce the causality concepts such as ATE and CATE for uplift modeling since the definition of uplift is a counterfactual problem and consistent with the objective of causal effect estimation~\cite{yamane2018uplift, zhang2021unified}. Causal approaches with traditional machine learning methods for uplift estimation are proposed in~\cite{radcliffe2007using, nassif2013uplift, jaskowski2012uplift, jaskowski2012uplift, radcliffe2011real, gutierrez2017causal}. Regarding recommender systems, uplift estimation on online A/B testing suffers from the high expense and large fluctuations due to user self-selection bias~\cite{sato2021online}, while uplift estimated offline is bedeviled by a wide variety of biases that could lead to MNAR. In order to deal with these issues, much of the literature has been published. Sato et al.~\cite{sato2019uplift} utilize SNIPS-based ATE to accomplish offline uplift-based evaluation. Goldenberg et al.~\cite{goldenberg2020free} leverage the Retrospective Estimation technique that relies solely on data with positive outcomes for CATE-based uplift modeling, which makes it especially suited for many recommendation scenarios where only the treatment outcomes are observable. ~\cite{betlei2021uplift} learns a model that directly optimizes an upper bound on AUUC, a popular uplift metric based on the uplift curves and unified with ATE~\cite{yamane2018uplift}. In addition,  CausCF~\cite{xie2021causcf} extends the classical MF to the tensor factorization with three dimensions—user, item, and treatment effect for better uplift performance. It is worth mentioning that in the uplift modeling literature~\cite{diemert2018large,gutierrez2017causal, zhang2021unified}, there are two closely related metrics for uplift modeling, uplift and Qini curves, the latter of which is evaluated based on the ranking of conditional treatment effect estimations. 




\subsubsection{Causal Effect beyond Uplift}
%It is a natural application to introduce the concepts such as $ATE$ and $CATE$ for causal effect estimation as it is consistent with the objective of causal effect estimation. For example, early studies have explored the influence of social presence through causal effect estimation through small-scale questionaires.
%Causal effect has its natural supremacy of explainability and trusts over statistic-based techniques. For example, 
There are some other impressive recommendation works with causal effect~\cite{mehrotra2020inferring, zhang2021causally, rosenfeld2017predicting, bonner2018causal, yao2022device, tran2022most}. For example, ~\cite{mehrotra2020inferring} adapts a Bayesian model to infer the causal impact of new track releases, which may be an essential consideration in the design of music recommendation platforms. ~\cite{zhang2021causally} minimizes the distance between the traditional attention weights in the recommendation method and the ITE to reflect the true impact of the features on the interactions. ~\cite{rosenfeld2017predicting} and ~\cite{bonner2018causal} frames causal inference as a domain adaptation problem and leverages ITE with a large sample of biased data and a small sample of unbiased data to eliminate the bias problems, which are described in more detail in \ref{sec:domain_adaption}.


%~\cite{jesson2020identifying} incorporates uncertainty estimating into the design of systems for CATE inference to deal with covariate shift and the violation of the positivity assumption
\section{SCM-based Methods}
\label{sec:scm_based}



Unlike the PO framework, Structural Causal Model explicitly expresses the causal relationship between variables on a causal graph, based on the experiences, before analyzing the causal effect. Its intuitive features make it win undivided admiration among researchers in computer field. In this section, the corresponding strategies is classified according to their causal structures, i.e., collider, mediator, and confounder. We focus on how researchers abstract recommendation issues into causal problems with causal graphs and exploit tools in causal inference to cope with it.

\subsection{Causal Recommendation with Collider Structure}



As represented in Fig. ~\ref{fig:junction}(c), a collider node occurs when it receives effects from two or more other factors. Collider exists in recommender systems. For instance, item positions in the ranking list are influenced by user preference and item popularity. 

Analyzing the dependency between variables in collider structures will contribute to its utilization in recommender systems. Although $A$ and $C$ are independent, i.e., for all $a$ and $c$, $\textrm{Pr}(A=a|B=b)=\textrm{Pr}(A=a)$, conditioning on the collision node $C$ produces a dependence between the node’s parents, i.e., for some $a,b,c$, $\textrm{Pr}(A=a|B=b, C=c)=\textrm{Pr}(A=a|C=c)$.  %One of the simplest samples will suffice to understand the point. 
To understand the point, let us consider the most basic example where $C=A+B$, and 
$A$ and $B$ are independent variables~\cite{glymour2016causal}. In this case, given $C=10$, knowing $A=3$ means we can immediately calculate that $B=7$. Thus, $A$ and $B$ are dependent, given that $C=10$. This characteristic inspires us that in RS issues with collider structure, knowing the common effect and one of the causes would provide information for another effect ~\cite{zhang2022causal}.

Though collider structures permeate RSs, they are usually compounded by other causal relationships and are treated as other causal structures, which results in minor literature discussing purely colliders. A representative work is DICE~\cite{zheng2021disentangling}, which is proposed by Zheng et al. and tracks the popularity issue from the user’s perspective instead of eliminating popularity bias from the item’s perspective. Zheng et al. argue that users’ interactions are driven by individual interest as well as users’ conformity, which is independent of user interest and describes how users tend to follow other people, and provides a causal graph as shown in Fig. \ref{fig:collider} (a). From this point of view,  DICE splits user and item embeddings into interest and conformity embeddings, respectively, and learns disentangled representations with conformity-specific and interest-specific data, driven by the colliding effect: if a user interacts with a less popular item, not conforming to the mainstream, it usually indicates that the user is highly interested in the item itself, and vice versa. Further, ~\cite{ding2022causal} proposes CIGC (Causal Incremental Graph Convolution), which includes a new operator named CED (Colliding Effect Distillation), to efficiently retrain graph convolution network (GCN) based recommender models. CED frames the whole incremental training phase as a causal graph (see Fig. \ref{fig:collider} (b)) and create a collider $S_t$ between inactive nodes $R_{In,t}$ and new data $R_{Ac,t}$, which is represented as the pair-wise distance. Therefore, the incremental integration data $I_t $ can update both $R_{Ac,t}$ and $R_{In,t} $, since conditioning on the collider $S_t$ opens the path $I_t \rightarrow R_{Ac,t} \leftrightarrow  R_{In,t} $.

\begin{figure}[t!]
	\centering
    \vspace{-4mm}
	\subfloat[DICE]{
		%\begin{minipage}[t]{0.5\linewidth}
		\centering
		\includegraphics[width=0.23\textwidth, trim=0 -20 0 -10, clip]{pic/conformity.pdf}
		%\end{minipage}
	}
	\centering
	\subfloat[CIGC]{
		%\begin{minipage}[t]{0.5\linewidth}
		\centering
		\includegraphics[width=0.28\textwidth, trim=8 0 10 5, clip]{pic/CollidingEffectDistillation.pdf}
	}
	\centering
    \vspace{-3mm}
	\caption{Causal graphs of collider in recommender systems.}
    \vspace{-6mm}
	\label{fig:collider}
\end{figure}


\subsection{Causal Recommendation with Mediator Structure}




When one variable causes another, it may not do it directly but through a set of mediating variables instead. For example, an item purchased by your friends increases your purchase probability not only directly through the recommendation that integrates social network, but also indirectly through increased trust in the item.

The distinction between direct and indirect effects of the change of treatment on outcome is key to the utilization of the mediator structure, which can be done by conditioning on the mediating variable traditionally~\cite{glymour2016causal}. Specifically, as illustrated in Fig. \ref{fig:mediator} (a), the \textit{total effect} (ToE) of $I=i$ on $Y$ is defined as:
\begin{equation}
ToE=Y(I=i, K(I=i))-Y(I=i^*, K(I=i^*)),
\end{equation}
$I = i^{*}$ refers to the situation where the value of $I$ is different from the reality, i.e., counterfactual.
%typically set as null.
Total effect can be further decomposed into \textit{natural direct effect} (NDE) and \textit{total indirect effect} (TIE). NDE reflects the effect of $I$ on $Y$ through the direct path, i.e., $I \rightarrow Y$, while $K$ is set to the value when $I=i^*$:
\begin{equation}
    NDE = Y(I=i, K(I=i^*))-Y(I=i^*, K(I=i^*)).
\end{equation}
TIE is defined as the difference between TE and NIE, denoted as:
\begin{equation}
\label{eq:tie}
    TIE = ToE - NDE  = Y(I=i, K(I=i))-Y(I=i, K(I=i^*)),
\end{equation}
which represents the effect of $I$ on $Y$ through the indirect path $I \rightarrow K \rightarrow Y$. TE can also be decomposed into \textit{natural indirect effect} (NIE) and \textit{total direct effect} (TDE). NIE represents the effect of $I$ on $Y$ through the mediator, i.e., $I \rightarrow K \rightarrow  Y$, while the direct effect on $I \rightarrow Y$ is blocked by setting $I$ as $I*$, denoted as:
\begin{equation}
    NIE = Y(I=i*, K(I=i))-Y(I=i^*, K(I=i^*)).
\end{equation}
In linear systems, NIE and TIE have the same value, and NDE and TDE have the same value~\cite{glymour2016causal, pearl2022direct}.



However, if there are confounders of the mediator and the outcome, as the case of ~\cite{wei2021model} shown in Fig. \ref{fig:mediator} (b), conditioning on the mediator means conditioning on a collider, and thus indirect dependence will pass through the confounder to the outcome and misguide the calculation of indirect effect. To tackle the problem, we should intervene on the mediator, which involves counterfactuals. The controlled direct effect (CDE) on $Y$ of $I$ is defined as:
\begin{equation}
    CDE = Y(do(I=i), do(K=k))-Y(do(I=i^*), do(K=k)).
\end{equation}
The difference between NDE and CDE is explained in ~\cite{glymour2016causal}.





% Table generated from sheet 'collider_mediator'
\begin{table*}[pt]
  \small
  \centering
  % \renewcommand\arraystretch{1.5}
  \vspace{-3mm}
  \caption{\normalsize Summary of recommendation models with collider structure and mediator structure.}
  \vspace{-3mm}
  \setlength{\tabcolsep}{2.2mm}{
    \begin{tabular}{c|c|c|c|c|c}
    \toprule
    \textbf{Category} & \textbf{Model} & \textbf{Causal method} & \textbf{Backbone model} & \textbf{Issue of concern} & \textbf{Year} \\
    \midrule
    \multirow{2}[2]{*}{\makecell{Causal\\ recommendation\\ with collider structure}} & DICE~\cite{zheng2021disentangling} & \makecell{(causal view)} & MF(multi-task) & Popularity bias & 2021 \\
          & CIGC~\cite{ding2022causal} & \makecell{Intervention on \\the cause factor} & LightGCN~\cite{he2020lightgcn} & GCN model retraining & 2022 \\
    \midrule
    \multirow{6}[2]{*}{\makecell{Causal\\ recommendation \\with mediator \\structure}} & ~\cite{choi2011influence} & Mediation analysis & -     & Effect of social presence & 2011 \\
          & ~\cite{luo2013impact} & Mediation analysis & -     & Effect of informational factors & 2013 \\
          & CMA~\cite{yin2019identification} & NDE, TIE & -     & Effect of induced change & 2019 \\
          & MACR~\cite{wei2021model} & TIE   & (model-agnostic, multi-task) & Popularity bias & 2021 \\
          & CIRS~\cite{gao2022cirs} & \makecell{Intervention on \\the mediator} & PPO~\cite{schulman2017proximal} & Filter bubble~\cite{pariser2011filter} & 2022 \\
          & CCF~\cite{xu2021causal} & \makecell{Intervention on\\ the mediator, \\counterfactuals} & \makecell{NCF~\cite{he2017neural}, \\GRU4Rec~\cite{hidasi2015session}, etc.} & Historical bias & 2023 \\
    \bottomrule
    \end{tabular}}%
  \vspace{-3mm}
  \label{tab:collider_mediator}%
\end{table*}%

\begin{figure}[t!]
	\centering
    \vspace{-3mm}
	\subfloat[Simple mediator]{
		%\begin{minipage}[t]{0.5\linewidth}
		\centering
		\includegraphics[width=0.23\textwidth, trim=-20 -20 -20 0, clip]{pic/simple_mediator.pdf}
		%\end{minipage}
	}
	\centering
	\subfloat[Mediator with confounder]{
		%\begin{minipage}[t]{0.5\linewidth}
		\centering
		\includegraphics[width=0.23\textwidth, trim=-20 0 -20 0, clip]{pic/mediator_confounder.pdf}
	}
	\centering
    \vspace{-1mm}
	\caption{Causal graphs of mediators in recommender systems, where $I$, $Y$, $K$ and $U$ denote cause, effect, mediator and confounder variable of the mediator and the outcome, respectively.}
    \vspace{-2mm}
	\label{fig:mediator}
\end{figure}

Some works are generally interested in how much of the treatment’s causal effect on variable $Y$ is direct and how much is indirect, which is usually explored with the technique of \textit{mediation analysis}~\cite{kenny1979correlation, baron1986moderator}, similar to SCM but without exogenous variables and the introduction of counterfactuals. For example, in early studies, ~\cite{choi2011influence} conducts an experiment varying the level of social presence over hundreds of testers and examines the effect of social presence on users’ reuse intention and trust through mediation analysis. A similar structure is used to evaluate how electronic word-of-mouth affects user interactions in ~\cite{luo2013impact}. Further, Yin et al.~\cite{yin2019identification} aim to separate the direct effects of the change in user behaviors in the tested product from the effect of changes in user behaviors in other products, aka induced changes, for example, the effects of significant lifts in CTR on the recommendation list and of significant decreases in CTR on organic search results on the final insignificant lifts in the sitewide CVR during the A/B test of a new version of recommendation module. Therefore, they use causal mediation analysis (CMA) of potential outcome framework to estimate causal effects of the induced changes and also discuss the estimation under the situation that multiple unmeasured causally-dependent mediators exist with the help of a directed acyclic graph.


Some other works utilize Pearl’s counterfactual tool to cope with mediator structure in order to improve accuracy~\cite{wei2021model, xu2021causal, gao2022cirs}. Wei et al.~\cite{wei2021model} explore the popularity issue with the SCM framework and formulate the causal graph as Fig. \ref{fig:mediator} (b) shown, in which the probability of interaction $Y$ is influenced by three main factors: user-item matching ($K(U, I) \rightarrow Y$), item popularity ($I \rightarrow Y$) and user conformity ($U \rightarrow Y$), the last two of which are usually ignored by existing models and thus result in the terrible Matthew effect. Following this causal graph, Wei et al. propose MACR (Model-Agnostic Counterfactual Reasoning), a multi-task framework that consists of three modules to jointly learn the effects of $U \rightarrow Y$, $U\&I \rightarrow K \rightarrow Y$, and $I \rightarrow Y$,  respectively during recommender training and estimates TIE of $I$ on $Y$ in counterfactual inference:
\begin{equation}
\begin{aligned}
\label{eq:tie_macr}
 TIE= &ToE - NDE  \\
= &Y(U=u, I=i, K=K(U=u, I=i))-Y(U=u, I=i, do(K=K(U=u^*,I=i^*))) 	\\
= &Y_k(K(U=u, I=i))*Y_u(U=u)*Y_i(I=i)-Y_k(K(U=u^*, I=i^*)) *Y_u(U=u)*Y_i(I=i)\\
=&\hat{y}_{k} * \sigma\left(\hat{y}_{i}\right) * \sigma\left(\hat{y}_{u}\right)-c * \sigma\left(\hat{y}_{i}\right) * \sigma\left(\hat{y}_{u}\right),
\end{aligned}
\end{equation}
where $\sigma(\cdot)$ denotes the sigmoid function, and $c$ is a hyper-parameter that represents $Y_k(K(U=u*, I=I*))$, the reference situation of $Y_k(K(U=u, I=i))$. With counterfactual inference, MACR could rank items without popularity bias by reducing the direct effect from item properties to the ranking score.



\begin{figure}[t!]
	\centering
    \vspace{-3mm}
	\subfloat[]{
		%\begin{minipage}[t]{0.5\linewidth}
		\centering
		\includegraphics[width=0.23\textwidth, trim=-20 0 -20 0, clip]{pic/CCF0.pdf}
		%\end{minipage}
	}
	\centering
	\subfloat[]{
		%\begin{minipage}[t]{0.5\linewidth}
		\centering
		\includegraphics[width=0.23\textwidth, trim=-20 0 -20 0, clip]{pic/CCF1.pdf}
	}
	\centering
    \vspace{-2mm}
	\caption{(a-b) Causal graphs of the CCF model before and after intervention. $U$ and $I$ are user and item representation, respectively, $Y$ is preference score, and $H$ denotes user interaction history.}
    \vspace{-2mm}
	\label{fig:mediator_ccf}
\end{figure}




The work by Xu et al.~\cite{xu2021causal} regards the user interaction history $H$ as a mediator (Fig. \ref{fig:mediator_ccf} (a)) and proposes CCF (Causal Collaborative Filtering) to estimate $\textrm{Pr}(Y=y|U=u, do(I=i))$, where $u, i$ is a user-item pair and $y$ is the preference score for the pair. More specifically, $H=f_h(U=u)$ is a database retrieval operation that returns a user’s interaction history from the observational data, $I=f_0(U=u, H=h)$ means the recommended item $I$ returned from the already deployed recommendation system based on the user and the user’s interaction history, and $Y=f(U=u, I=i)$ represents the estimation of unbiased user preference on the item. $\textrm{Pr}(Y=y|U=u, do(I=i))$ adopts the conditional intervention to consider both observed and unobserved (counterfactual) interaction history, as presented in Fig. \ref{fig:mediator_ccf} (b). The derivation result of $\textrm{Pr}(Y=y|U=u,do(I=i))$ is given:
\begin{equation}
\begin{aligned}
\textrm{Pr}(Y=y \mid U=u,do(I=i)) 
\approx & \textrm{Pr}(Y=y \mid U=u,do(I=f_0(U=u, H=h))) \\
 = & \sum_{h} \textrm{Pr}(y \mid u, h, f_0(u, h))\textrm{Pr}(h\mid u)\end{aligned}
\end{equation}
It is tempting to conclude that if trained only with observed history $h$, f(U=u, I=i) would naturally degenerate to the original recommendation model $f_0(U=u, H=h)$. Therefore, Xu et al. adopt a heuristic-based approach to generate counterfactual history $h’$.


\subsection{Causal Recommendation with Confounder Structure}

There is a large volume of published studies investigating the confounding structures in recommendation since a lot of data biases widespread in recommender systems are, essentially, confounding biases mentioned in Section \ref{sec:MNAR}. Approaches to tackle confounder structures of existing literature can be categorized into four types: with back-door adjustment, with instrumental variables, with front-door adjustment, and with deep learning based intervention.
%, in which the last type accounts for the major part. 






\subsubsection{The Back-door-based Approach}

Before introducing the back-door adjustment approaches, let us briefly review the definitions of back-door path and back-door criterion~\cite{imbens2015causal}. 
\begin{definition}[Back-door Path]
\label{def:backdoor_path}
Given a pair of treatment $T$ and outcome variable $Y$, a path connecting $T$ and $Y$ is a back-door path for $(T, Y)$ if it satisfies that
\begin{enumerate}
\item it is not a directed path (it contains an arrow pointing into $T$); and
\item it is not blocked (it has no collider).
\end{enumerate}
\end{definition}
Back-door path help us to identify confounders, which is the central node of a fork on a back-door path of $(T, Y)$. The following two examples will help to illustrate it~\cite{pearl2018book}. In Fig. \ref{fig:confounder_back} (a), there is one back-door path from $T$ to $Y$, $T \leftarrow A \rightarrow Y$, indicating that $A$ is the confounder. For the estimation the effect of $T$ on $Y$, we should eliminate the confounding bias by either controlling $A$ to block the back-door path or running a randomized controlled experiment. Note that $T \rightarrow B \leftarrow A \rightarrow Y$ is blocked by the collider at $B$ and, therefore, not a back-door path. In Fig. \ref{fig:confounder_back} (b), we can control for $C$ to close the back-door path $T \leftarrow B \leftarrow C \rightarrow Y$. Here we present the formal definition of the back-door criterion to deal with the confounding effects.
\begin{definition}[Back-door Criterion]
\label{def:backdoor_criterion}
Given a pair of treatment $T$ and outcome variable $Y$, a set of variables $X$ satisfied the back-door criterion if $X$ blocks all back-door paths of $(T, Y)$.
\end{definition}

Based on the Back-door Criterion, we can further derive the Back-door Adjustment Theorem, which adjusts fewer variables compared to the Causal Effect Rule (Definition \ref{def:ce_rule}).

\begin{definition}[Back-door Adjustment]
\label{def:backdoor_adjustment}
If a set of variables $X$ satisfies the back-door criterion for $T$ and $Y$, the causal effect of $T$ on $Y$ is identifiable and given by the formula:
\begin{equation}
\textrm{Pr}(Y=y \mid do(T=t)) =\sum_{x} \textrm{Pr}(Y=y \mid T=t, X=x) \textrm{Pr}(X=x),
\end{equation}
\end{definition}

\begin{figure}[t!]
	\centering
    \vspace{-3mm}
	\subfloat[]{
		%\begin{minipage}[t]{0.5\linewidth}
		\centering
		\includegraphics[height=0.12\textheight, trim=-5 0 -5 0, clip]{pic/backdoor_path0.pdf}
		%\end{minipage}
	}
	\centering
	\subfloat[]{
		%\begin{minipage}[t]{0.5\linewidth}
		\centering
		\includegraphics[height=0.12\textheight, trim=-5 0 -5 0, clip]{pic/backdoor_path1.pdf}
	}
	\centering
    \vspace{-1mm}
	\caption{Causal graphs illustrating the back-door path.}
    \vspace{-4mm}
	\label{fig:confounder_back}
\end{figure}


\begin{figure}[t!]
\centering
\vspace{-1mm}
\includegraphics[height=0.12\textheight,trim=0 0 0 0 ,clip]{pic/backdoor_ex.pdf} %l b r t
\vspace{-1mm}
\caption{A causal graph representing the relationship between recommendation ($T$), click ($Y$), consuming desire ($A$), and number of recent interactions $B$. The dotted circle indicates this variable is unobservable.}
\vspace{-1mm}
\label{fig:backdoor_ex}
\end{figure}

\begin{figure}[t!]
\centering
\vspace{-1mm}
\includegraphics[height=0.13\textheight,trim=0 0 0 0 ,clip]{pic/iv.pdf} %l b r t
\vspace{-1mm}
\caption{A causal graph showing the relationships between a sudden shock in traffic 
($Z_{i}$), total exposure of the focal product $i$ ($T_{i}$), product demand $D_i$ and $D_j$, recommendation click-through of related product $j$ from the focal product $i$($Y_{ij}$). The focal product $i$ experiences an instantaneous shock $Z_{i}$ in traffic while the product $j$ recommended shown alongside does not. $V_j$ means direct exposure of product $j$ (e.g., through search or browsing), which is not influenced by recommendation.
%The exposure of the focal product $i$, $T_{i}$, is influenced by the instantaneous shock in traffic $Z_{i}$ and the unobserved inherent demand for the product $D_i$, while the exposure of product $j$ is further broken down into direct exposure (e.g., through search or browsing), $V_j$, recommendation click-throughs from the focal product, $Y_{ij}$, and click-throughs from other products that recommend product $j$. Note that product $i$ and product $j$ are usually related products, so their demand may be driven by some common factors.
}
\vspace{-3mm}
\label{fig:sharma2015estimating}
\end{figure}






To see what this means in practice, let us look at a concrete example, as presented in Fig \ref{fig:backdoor_ex}. Suppose we need to evaluate the effect of recommendation ($T$) on user’s click behavior ($Y$) of a newly deployed recommendation strategy on an online shopping platform. However, the time-varying consuming desire ($A$) makes it difficult to compare the effect with that of the existing one. For example, users might be more willing to spend due to the proximity of holidays, resulting in a seemly better recommendation effect of the tested policy. However, the consuming desire is unmeasurable for \textit{do}-calculation. Instead, we could control for an observed variable, the number of recent interactions $B$, that fits the back-door criterion from $T$ to $Y$. Therefore, adjusting for $B$ to block the back-door path $T \leftarrow A \rightarrow B \rightarrow  Y$ will give us the true causal effect of recommendation $T$ on click $Y$, formulated as:
\begin{equation}
\begin{aligned}
\textrm{Pr}(Y=y \mid do(T=t))=\sum_{x} \textrm{Pr}(Y=y \mid T=t, B=b) \textrm{Pr}(B=b).
\end{aligned}
\vspace{-2mm}
\end{equation}





Some literature on recommendation issues with confounder structures introduces the theory of back-door criterion~\cite{huang2012exploring, sharma2015estimating, tran2021recommending, wang2021clicks}. ~\cite{huang2012exploring} utilizes the back-door criterion to verify whether or not word-of-mouth recommendations can influence users’ evaluation of the recommended items. Sharma et al.~\cite{sharma2015estimating} treat an instantaneous shock in direct traffic as an \textit{instrumental variable} to answer the counterfactual question from purely observational data: how much interaction activity would there have been on the online shopping website if recommendations were absent, and apply the back-door criterion to block the possible unobserved confounding effect between the “exposure” $T_i$ and “click” $Y_{ij}$, as Fig. \ref{fig:sharma2015estimating} shown. Besides, Tran et al.~\cite{tran2021recommending} consider the job personal recommendation issue in Disability Employment Services and present a causality-based method to tackle the problem, in which the covariate set is determined by the back-door criterion.
%It is worth mentioning that the instrumental variable method is a popular method for learning causal effects with unobserved confounders, but it seems to find little application in recommender systems because of the difficulty of finding variables that satisfy the conditions of instrumental variables.



\begin{figure}[t!]
	\centering
    \vspace{-3mm}
	\subfloat[DecRS]{
		%\begin{minipage}[t]{0.5\linewidth}
		\centering
		\includegraphics[height=0.14\textheight, trim=-15 -12 -15 -12, clip]{pic/DecRS.pdf}
		%\end{minipage}
	}
	\centering
	\subfloat[PDA]{
		%\begin{minipage}[t]{0.5\linewidth}
		\centering
		\includegraphics[height=0.14\textheight, trim=-15 0 -15 0, clip]{pic/PDA.pdf}
	}
	\centering
    \vspace{-2mm}
	\caption{(a) The causal graph used in DecRS, where $U$ and $I$ denote user and item representation, $D$ represents the user historical distribution over item groups, $G$ is the group-level user representation, and $Y$ is the prediction score. (b) The causal graph of PDA, in which $U$ and $I$ denote user and item representation, $P$ is the item popularity, and $Y$ stands for user interactions.
}
    \vspace{-3mm}
	\label{fig:backpath}
\end{figure}


A multitude of studies employ back-door adjustment to block the back-door path by directly intervening on the treatment variable~\cite{wang2021deconfounded, zhang2021causal, he2022addressing, wang2022causal, zhan2022deconfounding, rajanala2022descover, xia2023user, zhang2023leveraging, yu2023causality, tsoumas2023evaluating}. For example, Wang et al.~\cite{wang2021deconfounded} propose the framework named DecRS (Deconfounded Recommender System) to eliminate bias amplification through intervention on the user representation $U$, which removes the effect of the historical user distribution over item groups $D$ on $U$, as Fig. \ref{fig:backpath} (a) shown. Zhang et al.~\cite{zhang2021causal} propose PDA (Popularity-bias Deconfounding and Adjusting) to eliminate the effect of item popularity $P$ through intervention on the item $I$ (see Fig. \ref{fig:backpath} (b)), denoted as:
\begin{equation}
\textrm{Pr}(Y=y \mid do(U=u, I=i))=\sum_{p} \textrm{Pr}(y \mid u, i, p) \textrm{Pr}(p \mid u,i)
=\textrm{Pr}(y \mid u, i, p) \textrm{Pr}(p),
\end{equation}
where $U$ denotes the user representation and $Y$ represents interactions. $\textrm{Pr}(y \mid u, i, p)$ and $\textrm{Pr}(p)$ are learned separately. It is worth mentioning that PDA can leverage popularity bias to enhance the recommendation performance by adjusting $\textrm{Pr}(p)$ in the inference stage, which can be regarded as counterfactual inference. More recently, Zhang et al.~\cite{zhang2023leveraging} address duration bias by identifying duration time as a confounder. Subsequently, they group data samples based on watch time feedback and craft novel duration supervision labels, thereby alleviating the confounding bias.




In the above literature elaboration, we may find a series of works that accomplish the integration of SCM-based causal inference and recommender systems with a similar pattern, as shown in Fig. \ref{fig:he_type}: they first analyze the causal relationship between the variables regarding the concern issue and formulate the causal graph based on it; after theoretical analysis, a multi-task or separated structure is adopted to learn the causal effects of the variables on the potential outcome in the training phase; once the training has been completed, appropriate variables are selected to intervene during the inference stage, i.e., they are set to counterfactual values directly or indirectly, and the outcome is estimated based on applicable causal rules (e.g., backdoor adjustment, TIE, etc.) to conduct counterfactual inference. 
%A representative research is CR~\cite{wang2021clicks} by Wang et al. They focus on the clickbait issue and build the cuasal graph as Fig. \ref{fig:clickbait} (a), where $Y$, $U$, $I$, $E$, and $P$ denote the prediction score, user features, item features, exposure features, and content features (post-click), respectively. They regard that users might click the items only because they are attracted by the exposure features, and thus there exists a direct effect from the exposure features to the click behavior $E \rightarrow Y$. As a result of ignoring the effect of $E \rightarrow Y$, the clickbait issue can be mitigating by reducing this effect. After the theoretical analysis, 


%应该说明的是,有一系列工作都是利用相似模式来完成SCm框架下的因果推断与推荐系统的融合,如图3所示:他们首先分析关心问题中涉及的变量之间的关系,并绘制出因果图;在训练阶段,用多任务或者分离的结构来学习这些变量对outcome的影响;在推断阶段,在这些变量里选择合适的变量去干预,即直接或间接地将其设置为其他值,在此基础上利用合适的规则(如后门调整、TIE等)估计outcome,从而完成反事实推断。一个典型的工作是【A】,该工作formulate the clickbait issue如图4所示的因果图,其中��,��,��,��,and�� denotethepredictionscore,user features, item features, exposure features, and content features, respectively。该文认为users might click the items only because they are attracted by the exposure features,因此存在a direct effect from the exposure features to the click behavior。忽略这个作用正是clickbait的来源。。。

%如前文所言,相比PO框架,SCM框架对因果图的依赖既是他的优点,也是他的缺点。因果图可以使得变量之间的关系变得直观,但是这些关系是现在主要是依赖于专家经验给出的,这使得可能存在对结果有影响但未被观察到的变量,甚至出现过于主观、与实际不符的因果关系假设。这一点可以由现有文献中轻易得知,因为不同的文献对于同一个issue往往会提出不同的因果图。为解决以上问题,Causal discovery显得至关重要。

\subsubsection{Instrumental Variable-based Approach}
The instrumental variable (IV) method is such a powerful approach for learning causal effects with confounders that it can be done even without controlling for, or collecting data on, the confounders~\cite{pearl2018book}. The instrumental variable causally influences the outcome only through the treatment (Fig. \ref{fig:iv_ex} (a)), defined as:
\begin{definition}[Instrumental Variable]
\label{def:iv}
Given an observed variable $Z$, covariates $X$, the treatment $T$ and the outcome $Y$, $Z$ is a valid instrumental variable (IV) for the causal effect of $T \rightarrow Y$ if $Z$ satisfies~\cite{angrist1996identification}: 
\begin{enumerate}
\item $Z \dep T \mid X$; and
\item $Z \ind Y \mid do(T), X$.
\end{enumerate}
\end{definition}
%The general set up for instrumental variable is shown in Fig. \ref{fig:iv} (a), where the effect of $Z \rightarrow T$ and $Z \rightarrow T \rightarrow Y$ can be learned without interference from covariates. Therefore, $\textrm{Pr}(Y \mid do(T))$ can be estimated though:
In practice, IV is often implemented in a two-stage lease squares (2SLS) procedure.

\begin{figure}[t!]
\centering
\vspace{-1mm}
\includegraphics[height=0.50\textheight,trim=0 10 0 0 ,clip]{pic/he_type.pdf} %l b r t
\vspace{-1mm}
\caption{Separate-learning-counterfactual-inference, a common pattern of SCM-based causal inference for recommender systems, learns causal effect with a separate structure or multi-task framework and performs counterfactual inference during testing.}
\vspace{-3mm}
\label{fig:he_type}
\end{figure}

\begin{figure}[t!]
	\centering
    \vspace{-3mm}
	\subfloat[]{
		%\begin{minipage}[t]{0.5\linewidth}
		\centering
		\includegraphics[height=0.09\textheight,trim=0 0 0 0 ,clip]{pic/iv_ex.pdf}
		%\end{minipage}
	}
	\centering
	\subfloat[]{
		%\begin{minipage}[t]{0.5\linewidth}
		\centering
		\includegraphics[height=0.09\textheight,trim=0 0 0 0 ,clip]{pic/proxy_variable.pdf}
	}
	\centering
    \vspace{-2mm}
	\caption{(a) The causal graph of a general setup for instrumental variables, where $Z$ is an instrumental variable. (b) Proxy variables is easier to be satisfied compared with the instrumental variables.
}
    \vspace{-3mm}
	\label{fig:iv_ex}
\end{figure}


% Table generated from sheet 'confounder'
\begin{table*}[htbp]
  \centering
  % \renewcommand\arraystretch{1.5}
  \small  
  \vspace{-2mm}
  \caption{\normalsize Summary of recommendation models with confounder structure.}
  \vspace{-2mm}
  \setlength{\tabcolsep}{2.0mm}{
    \begin{tabular}{c|c|c|c|c}
    \toprule
    \textbf{Model} & \textbf{Causal method} & \textbf{Backbone model} & \textbf{Issue of concern} & \textbf{Year} \\
    \midrule
    ~\cite{huang2012exploring} & Back-door criterion & MF    & Effect of WOM recommendation & 2012 \\
    ~\cite{sharma2015estimating} & Back-door adjustment, IV & -     & Effect of recommendations & 2015 \\
    ~\cite{chaney2018algorithmic} & -     & MF, etc. & Feedback loop bias & 2018 \\
    DEMER~\cite{shang2019environment} & -     & \makecell{(RL)} & Unobserved confounding bias & 2019 \\
    CPR~\cite{yang2021top} & Back-door adjustment & (model-agnostic) & Data insufficiency & 2021 \\
    CauSeR~\cite{gupta2021causer}. & Back-door adjustment & SR-GNN~\cite{wu2019session} & Popularity bias in SBRSs & 2021 \\
    MCT~\cite{tran2021recommending} & Back-door criterion, CATE & (custom-designed) & Disability employment & 2021 \\
    DecRS~\cite{wang2021deconfounded} & Back-door adjustment & FM, NFM~\cite{he2017neuralfactorization} & Bias amplification & 2021 \\
    PDA~\cite{zhang2021causal} & Back-door adjustment & MF    & Popularity bias & 2021 \\
    CR~\cite{wang2021clicks} & Back-door criterion, TIE & \makecell{MMGCN~\cite{wei2019mmgcn}\\(multi-task)} & Clickbait & 2021 \\
    D2Q~\cite{zhan2022deconfounding} & Back-door adjustment & (custom-designed) & Duration bias & 2022 \\
    DeSCoVeR~\cite{rajanala2022descover} & Back-door adjustment & (custom-designed) & Venue recommendation & 2022 \\
    IV4Rec~\cite{si2022model} & IV    & DIN, NRHUB~\cite{wu2019neural} & Recommendation using search data & 2022 \\
    HCR~\cite{zhu2022mitigating} & Front-door adjustment & MMGCN & Unobserved confounding bias & 2022 \\
    DCR~\cite{he2022addressing} & Back-door adjustment & NFM   & \multicolumn{1}{c|}{Observed confounding bias} & 2023 \\
    CaDSI~\cite{wang2022causal} & Back-door adjustment & (custom-designed) & Observed confounding bias & 2023 \\
    DecUCB~\cite{xia2023user} & Back-door adjustment & (custom-designed, bandit) & Observed confounding bias & 2023 \\
    iDCF~\cite{zhang2023debiasing} & Proxy variable & MF & Unobserved confounding bias & 2023 \\
    CVRDD~\cite{tang2023counterfactual} & TIE & MLP(model-agnostic) & Duration bias & 2023 \\
    DML~\cite{zhang2023leveraging} & Back-door adjustment & MMoE & Duration bias & 2023 \\
    CGSR~\cite{yu2023causality} & Back-door adjustment & (custom-designed) & Shortcut paths in SBRSs & 2023 \\
    ~\cite{tsoumas2023evaluating} & Back-door adjustment, IPS &	(custom-designed, knowledge-based RS)& Digital agriculture & 2023 \\
    DDCE~\cite{wang2023dual}& - & (custom-designed) & Popularity bias & 2023 \\
    \bottomrule
    \end{tabular}}%
    \label{tab:confounder}%
    \begin{tablenotes}
    \small
    \item *Here, WOM stands for word-of-mouth, RL for reinforcement learning, and SBRS for session-based recommender system.
    \end{tablenotes}
    \vspace{-4mm}
\end{table*}%

\begin{figure}[t!]
\centering
\vspace{-1mm}
\includegraphics[height=0.13\textheight,trim=0 0 0 0 ,clip]{pic/IV4Rec.pdf} %l b r t
\vspace{-1mm}
\caption{The causal graph of IV4Rec, which reconstructs treatment $T$ by leveraging search queries $Z$ as instrumental variables to decompose treatment into causal part  $T^{ca}$ and non-causal part $T^{no}$ and combining them with different weights. $X$ is a set of unmeasurable confounders and $Y$ represents users’ interaction.}
\vspace{-1mm}
\label{fig:iv4rec}
\end{figure}


Though a popular tool, instrumental variable seems to find little application in recommender systems because of the difficulty of finding variables that satisfy the conditions of instrumental variables. As already cited above,  Sharma et al.~\cite{sharma2015estimating} utilize an instantaneous shock in direct traffic as an instrumental variable to evaluate the recommendation effect. Si et al.~\cite{si2022model}  propose a model-agnostic framework named IV4Rec that effectively decomposes the embedding vectors into two parts: the causal part indicating a user’s personal preference for an item, and the non-causal part merely reflects the statistical dependencies between users and items such as exposure mechanism and display position, with users’ search behaviors as the instrumental variable. More specifically, it modifies the traditional IV method, using the residual of the least square regression as the causal embedding instead of discarding it. The causal graph is illustrated in Fig. \ref{fig:iv4rec}.

Considering the stringent conditions often associated with IVs, a recent theoretical advancement~\cite{miao2023identifying} has been proposed to estimate treatment effects utilizing an auxiliary variable, which requires less restrictive prerequisites compared to IVs. An example causal diagram for auxiliary variables is visually represented in Fig. \ref{fig:iv_ex}(b), where $Z$ serves as a proxy variable for the unmeasurable confounder. Building on this theory, Zhang et al.\cite{zhang2023debiasing} developed the iDCF (identifiable deconfounder) to account for the unmeasured user’s socio-economic status $X$ by employing the user's consumption level as a proxy variable $Z$, a descendant of the unobserved confounder $X$ yet not directly causally associated with either treatment or outcomes. Furthermore, they leverage iVAE~\cite{khemakhem2020variational} to infer the conditional distribution of the latent confounder, thus resolving the Non-Identification issue encountered in ~\cite{wang2020causal}.





\subsubsection{The Front-door-based Approach}

The front-door adjustment~\cite{imbens2015causal} is another popular method for learning causal effects with unobserved confounders, in which we condition on a set of variables $K$ that satisfies the front-door criterion.

\begin{definition}[Front-door Criterion]
\label{def:frontdoor_criterion}
Given a pair of treatment $T$ and outcome variable $Y$, a set of variables $K$ is said to satisfy the front-door criterion if:
\begin{enumerate}
\item $K$ intercepts all directed paths from $T$ to $Y$;
\item there is no back-door path from $T$ to $K$; and
\item all back-door paths from $K$ to $Y$ are blocked by $T$.
\end{enumerate}
\end{definition}

A graph depicting the front-door criterion is shown in Fig. \ref{fig:frontdoor} (a). In practice, $K$ is usually the mediator of the causal effect $T \rightarrow Y$. With the help of $K$, the causal effect of $T$ on $Y$ can be calculated as follows:

\begin{definition}[Front-Door Adjustment)]
\label{def:frontdoor_adjustment}
If $K$ satisfies the front-door criterion relative to $(T, Y)$ and $\textrm{Pr}(T, Y) > 0$, then the causal effect of $T$ on $Y$ is given by the formula
\begin{equation}
\textrm{Pr}(Y \mid do(T)) =\sum_{K} \textrm{Pr}(Y \mid do(K)) \textrm{Pr}(K \mid do(T)) =\sum_{K} \textrm{Pr}(K \mid T) \sum_{T^{\prime}} \textrm{Pr}\left(Y \mid T^{\prime}, K\right) \textrm{Pr}\left(T^{\prime}\right).
\end{equation}
\end{definition}

\begin{figure}[t!]
	\centering
    \vspace{-3mm}
	\subfloat[Front-door path]{
		\centering
		\includegraphics[width=0.25\textwidth, trim=-10 -40 -10 0, clip]{pic/frontdoor0.pdf}
	}
	\centering
	\subfloat[HCR]{
		\centering
		\includegraphics[width=0.25\textwidth, trim=-10 0 -10 0, clip]{pic/frontdoor1.pdf}
	}
	\centering
    \vspace{-3mm}
	\caption{(a) A graphical model representing the front-door path, in which $T$ denotes the treatment, and $Y$ denotes outcomes. Unobserved confounders $X$ exist in the causal effect $T \rightarrow Y$, and $K$ are variables that satisfy the front-door criterion. (b) The Causal graph for illustrating the relationship in the HCR framework. $U$: user features, $X$: hidden confounders, $I$: item features affected by $X$, $Y$: post-click interactions, $M$: click behaviors. }
    \vspace{-6mm}
	\label{fig:frontdoor}
\end{figure}




Zhu et al.~\cite{zhu2022mitigating} propose HCR (Hidden Confounder Removal) framework to mitigate hidden confounding effects by front-door adjustment, in which user and item feature $U$ and $I$ are treatments, post-click user behaviors $Y$ are the concerned outcome, and the click feedback $K$ acts as the mediator that satisfies the front-door criterion, as Fig. \ref{fig:frontdoor} (b) shown. However, in real-world recommendation scenarios, confounding bias also exists in the estimation of the click feedback, which means it is not competent to perform the front-door adjustment. In fact, the front-door adjustment, like the IV method, finds little application in recommender systems because of the lack of eligible variables.

\subsubsection{Deconfounded Recommender Algorithms}
%Deep learning is a core technique in mainstream models of recommender systems. Therefore, some literature expands traditional neural network-based recommendation algorithms to deal with confounders under the inspiration of causal inference. 

Instead of directly introducing causal technique, some literature expands sheer recommendation algorithms to deal with confounders under the inspiration of analysis from the perspective of causal inference. For example, ~\cite{chaney2018algorithmic} modifies several traditional recommendation algorithms to explore the impact of algorithmic confounding, which has found that the data-algorithm feedback loop amplifies the homogenization of user behavior without corresponding gains in utility and also amplifies the impact of recommendation systems on item consumption.



Some works integrate reinforcement learning-based recommender systems with causal inference to tackle the confounding issue. For example, DEMER (deconfounded multi-agent environment reconstruction)~\cite{shang2019environment} is proposed following the generative adversarial training framework to model the hidden confounder, which affects both actions and rewards as an agent interacts with the environment and thus obstructs an effective reconstruction of the environment, by treating the hidden confounder as a hidden policy. In ~\cite{yang2021top}, user representations $U$ are considered as a confounder of the recommendation lists $T$ and users’ interactions $Y$ on recommendation lists. To alleviate this confounding bias, CPR (counterfactual personalized ranking framework) builds the recommender simulator to generate new training samples based on the causal graph. 

As for session-based recommender systems (SBRSs), Gupta et al.~\cite{gupta2021causer} propose the CauSeR (Causal Session-based Recommendations) framework to perform deconfounded training to handle popularity bias. COCO-SBRS~\cite{song2023counterfactual} adopts a self-supervised approach to pre-train a recommendation model to learn the causalities in SBRSs, so as to eliminate confounding bias and make accurate next item recommendations. In terms of GNN-based recommendations, Gas et al. infer the unobserved confounders existing in representation learning with the CVAE model~\cite{sohn2015learning} and apply it to GNN-based strategy~\cite{gao2021deconfounding}.
\section{General Counterfactuals-based Methods}
\label{sec:general_counterfactual}
Some causal recommender approaches are established based on the general concept of counterfactuals, the world that does not exist but can be reasoned with some fundamental law and human intuition. In this section, we will introduce related strategies from the perspective of recommender issues they try to address, including domain adaptation, data augmentation, fairness, and explanation.

\subsection{Domain Adaptation}
\label{sec:domain_adaption}
RSs are trained and evaluated offline with the supervision of previously-collected data, which usually suffers from selection bias and confounding bias. It results in a gap between the training goal and the true recommendation objective, and, therefore, a sub-optimal recommender algorithm. To address this issue, we hope to evaluate the training policy on the unbiased data, which is collected from the randomized treatment policy. However, uniform data is always expensive and small-scale. To take full advantage of the uniform data, researchers train the recommender systems with a small amount of unbiased data and a large amount of biased data, with the hope of learning the counterfactual distribution of the biased data, which is both a counterfactual problem and a domain adaptation problem. 


~\cite{rosenfeld2017predicting} and ~\cite{bonner2018causal} train recommender policies on biased and unbiased data, and add regulation terms to the loss function so that the distance of parameters between the two policies in the inspiration of individual treatment effect is controllable. ~\cite{yuan2019improving} trains an unbiased imputation model to impute the labels of all observed and unobserved events in biased and unbiased data, and learns the final CTR model by combining the two data with the propensity-free doubly robust method. Further, ~\cite{liu2020general} propose KDCRec (Knowledge Distillation framework for Counterfactual Recommendation) in which the teacher network with unbiased data as input is used to guide the biased model via four approaches.


% Table generated from sheet 'general_counterfactual'
\begin{table*}[pt]
  \small
  \centering
  % \renewcommand\arraystretch{1.5}
  \vspace{-3mm}
  \caption{\normalsize Summary of recommendation models with general counterfactuals.}
  \vspace{-3mm}
  \setlength{\tabcolsep}{1.9mm}{
    \begin{tabular}{c|c|c|c|c}
    \toprule
    \textbf{Category} & \textbf{Model} & \textbf{Causal inference method} & \textbf{Backbone model} & \textbf{Year} \\
    \midrule
    \multirow{4}[1]{*}{\makecell{Domain\\adaptation}} & ~\cite{rosenfeld2017predicting} & ITE   & Linear/regularized kernel methods & 2017 \\
          & ~\cite{bonner2018causal} & ITE   & Matrix factorization  & 2018 \\
          & Propensity-free DR~\cite{yuan2019improving} & DR    & FFM   & 2019 \\
          & KDCRec~\cite{liu2020general} & ITE   & MF (knowledge distillation) & 2020 \\
    \midrule
    \multirow{4}[0]{*}{\makecell{Data\\augmentation}} & CF2~\cite{xiong2021counterfactual} & "Minimum" counterfactuals & (self-designed) & 2021 \\
          & CASR~\cite{wang2021counterfactual} & "Minimum" counterfactuals & NARM~\cite{li2017neural}, STAMP~\cite{liu2018stamp}, SASRec~\cite{kang2018self} & 2021 \\
          & CauseRec~\cite{zhang2021causerec} & Counterfactuals & \makecell{(self-designed, sequential recommendation)} & 2021 \\
          & POEM~\cite{liu2022modeling} & Counterfactuals & GCN   & 2022 \\
    \midrule
    \multirow{3}[0]{*}{Fairness} & ~\cite{li2021towards} & Counterfactuals & (self-designed) & 2021 \\
          & F-UCB~\cite{huang2022achieving} & Counterfactuals & UCB   & 2022 \\
          & CLOVER~\cite{wei2022comprehensive} & Counterfactuals & MELU~\cite{lee2019melu} & 2022 \\
    \midrule
    \multirow{2}[0]{*}{Explanation} & PRINCE~\cite{ghazimatin2020prince} & "Minimum" counterfactuals & HIN~\cite{shi2016survey} & 2020 \\
          & CountER~\cite{tan2021counterfactual} & "Minimum" counterfactuals & (black-box) & 2021 \\
          \bottomrule
    \end{tabular}}%
  \label{tab:fair_explain}%  
  \vspace{-6mm}
\end{table*}%

\subsection{Data Augmentation} 

Data augmentation is an uncontroversial counterfactual problem, such as answering the question: “what would be the user’s decision if a different item had been exposed?”. Therefore, some works are trying to integrate counterfactuals into the procedure of data augmentation.  

Xiong et al.~\cite{xiong2021counterfactual} generate new data samples by users’ feature-level preference for review-based recommendation. To generate more effective samples, they leverage the “minimum” idea in counterfactuals, learning the “minimum” change of the user feature-level preference that can “exactly” reverse the preference ranking of the user on a given item pair. For example, if slightly increasing the price attention of a user who had purchased an iPhone will make Xiaomi more attractive to her, this will be regarded as an effective counterfactual sample. Similarly, CASR (Counterfactual Data-Augmentation Sequential Recommendation)~\cite{wang2021counterfactual} generates the counterfactual sequence of items by “minimally” changing the user’s historical items, such that her currently interacted item can be “exactly” altered. 

The CauseRec (Counterfactual User Sequence Synthesis for Sequential Recommendation) proposed by Zhang et al.~\cite{zhang2021causerec} generates counterfactual data in a different way. It identifies indispensable and dispensable concepts in the historical behavior sequence. The former can represent a meaningful aspect of the user’s interest, while the latter indicates noisy behaviors that are less important in representing user interest. Therefore, it is reasonable to argue that replacing indispensable concepts in the original user sequence incurs a preference deviation of the original user representation, while replacing the dispensable ones still has a similar user representation, which CauseRec realizes through contrastive learning. Liu et al.~\cite{liu2022modeling} focus on the recommendation scenario where users are exposed with decision factor-based persuasion texts, i.e., persuasion factors, and generate new training samples by making simple but reasonable counterfactual assumptions about user behaviors, including:
\begin{itemize}
\vspace{-2mm}
    \item If a user clicks on an item without the existence of persuasion factors, the user will still be likely to click on it with a matching persuasion factor. 
    \item If a user does not click on an item with the existence of persuasion factors, the user will not click on it when the persuasion factor does not exist.
\vspace{-3mm}
\end{itemize}



\subsection{Fairness and Explanation}

The counterfactual technique is a natural tool for the evaluation of fairness since we can compare the outcome (ratings, recommendation lists, etc.) in the real world and in the counterfactual world in which only users’ sensitive features (e.g., gender and race) are intervened~\cite{huang2022achieving, wei2022comprehensive}. 
\begin{definition}[Counterfactual Fairness]
\vspace{-3mm}
A recommender model is counterfactually fair if, for any possible user $u$ with features $X=x$ and $Z=z$:
\begin{equation}
\textrm{Pr}(y \mid x, z) = \textrm{Pr}(y \mid x, do(z’))
\end{equation}
For any value $y$ and $z’$, where $Y$ denotes the potential outcome for user $u$. $Z$ are users’ sensitive features and $X$ are causally $Z$-independent features.
\vspace{-3mm}
\end{definition}
Based on the counterfactual fairness, Li et al. generate sensitive feature-independent user embeddings through adversary learning~\cite{li2021towards}. They train a predictor to learn the filtered embedding and an adversarial classifier to predict the sensitive features from the learned representation simultaneously. For the reinforcement learning-based recommendation, Huang et al. propose the F-UCB (fair causal bandit)~\cite{huang2022achieving}, picking arms from a subset of arms at each round in which all the arms satisfy counterfactual fairness constraint that users receive similar rewards regardless of their sensitive attributes. 

As for explanation, counterfactuals describe a dependency on the external facts that lead to certain outcomes, and thus allow researchers to reason about the behavior of a black-box algorithm~\cite{wachter2017counterfactual}. Literature on counterfactual explanation also resorts to the “minimum” idea in counterfactuals. For example, ~\cite{ghazimatin2020prince} presents PRINCE (Provider-side Interpretability with Counterfactual Evidence)  to search for a set of minimal actions performed by the user that, if removed, changes the recommendation to a different item, in a heterogeneous information network with users, items, and so on. To understand the point, consider the following example. If a user who has bought an iPhone and followed MacBook receives a recommendation about AirPods and would not have received it if she had not bought iPhone, PRINCE will regard the behavior “purchase of iPhone” as the explanation of the recommendation. Similarly, CountER (Counterfactual Explainable Recommendation) proposed by ~\cite{tan2021counterfactual} seeks the minimum changes of item features that exactly reverse the recommendation decision.
\section{Conclusion}\label{sec:conclusion}
In this work, we focus on addressing the fundamental challenge of OOD detection tasks, which is how to fully understand the semantic discrepancy between the ID/OOD samples. We reveal that the key to success in the realistic SCOOD task is to allocate as many ID samples in the unlabeled set correctly as possible. To this end, we propose a novel uncertainty-aware optimal transport scheme that introduces class-specific energy scores as guidance for effective label assignment. Experimental results show that our method achieves better performance than previous state-of-the-art methods on SCOOD benchmarks.

\textbf{Limitations.} In addition to temperature scaling, other techniques such as feature clipping applied in ReAct~\cite{sun2021react} also enhance the performance of energy score, so how to obtain an OOD score that best fits the SCOOD task can be further explored. Moreover, a setting highly related to SCOOD has been proposed in \cite{katz2022training} and formulated as a constrained optimization problem. We will also theoretically analyze these practical OOD settings in our feature work.

% \section*{Acknowledgments}
\textbf{Acknowledgments.} 
This work is supported by National Key R\&D Program of China under Grant 2020AAA0105701, National Natural Science Foundation of China (NSFC) under Grants 61872327, Major Special Science and Technology Project of Anhui, National Natural Science Foundation of China (62033012) and Ant Group through Ant Research Intern Program.


%%
%% The acknowledgments section is defined using the "acks" environment
%% (and NOT an unnumbered section). This ensures the proper
%% identification of the section in the article metadata, and the
%% consistent spelling of the heading.
\begin{acks}
This research work is supported by the National Key Research and Development Program of China under Grant No. 2021ZD0113602, the National Natural Science Foundation of China under Grant Nos. 62176014, 62276015, the Fundamental Research Funds for the Central Universities.
\end{acks}

%%
%% The next two lines define the bibliography style to be used, and
%% the bibliography file.
\bibliographystyle{ACM-Reference-Format}
\bibliography{ref}

%%
%% If your work has an appendix, this is the place to put it.
% \appendix

% \section{Research Methods}

% \subsection{Part One}

% Lorem ipsum dolor sit amet, consectetur adipiscing elit. Morbi
% malesuada, quam in pulvinar varius, metus nunc fermentum urna, id
% sollicitudin purus odio sit amet enim. Aliquam ullamcorper eu ipsum
% vel mollis. Curabitur quis dictum nisl. Phasellus vel semper risus, et
% lacinia dolor. Integer ultricies commodo sem nec semper.



\end{document}
\endinput
%%
%% End of file `sample-manuscript.tex'.
