\section{Opening problems and future direction}
\label{sec:discuss}
 
The introduction of causal inference into recommender systems is still relatively recent, and there are still many promising but unexplored research directions, which we will discuss in this section.

\textbf{Causal assumptions in recommendations.} To extract causality knowledge from statistical data, causal inference-based methods are conducted with several causal assumptions. Although much of the existing work employ causal inference methods, these assumptions are often not explicitly clarified or even violated. To address this issue, existing work either ignores it or makes simple assumptions about data distributions. Therefore, estimating the impact of violations of causal assumptions on experimental results is crucial for bridging the gap between modern recommender system design and causal inference. For example, the positivity assumption is essential in PO-based approaches for the unbiased estimation of causal effect, but the data sparsity problem of recommender systems makes it difficult to satisfy. Considering that a small difference in recommendation accuracy may lead to a huge rise and fall in platform revenue, the effect of the violation of causal assumptions and that of artificial assumptions on predictions should be investigated.


\textbf{Transfer learning and Out-of-distribution recommendation.} Due to the data sparsity issue of recommendation systems, it will be a wise and practical choice to transfer user and item knowledge from other domains to improve prediction performance during cold start, offline evaluation, or online test, which is a transfer learning problem. Even in the same dataset, natural shifts of user preference or artificial bias also cause a violation of the IID hypothesis, which is an out-of-distribution (OOD) recommendation issue ~\cite{he2022causpref}. The core of both transfer learning and OOD recommendation is to transfer beneficial shared knowledge, such as users' inherent and unchanged preferences. Thus they can be formulated as invariant learning in some cases~\cite{wang2022invariant}. As we mentioned in Section \ref{sec:intro}, causal inference works to discover the unchangeable causal relationship in data, which can be reused in new domains. From this perspective, adopting causal inference to improve robustness and generalization ability to accomplish cross-domain or OOD recommendation is a promising direction, and some exciting attempts can be found recently~\cite{he2022causpref, wang2022out}.

\textbf{Dynamic recommendation.} Modern recommender systems usually involve feedback loops and dynamic updates. Therefore, it is crucial to incorporate loops into the causality-based methods to accurately model the dynamic and iterative data collection process for recommender systems~\cite{xu2022dynamic}. Some impressive work has also been proposed~\cite{chaney2018algorithmic, wang2021deconfounded, gupta2021causer, krauth2022breaking}. However, uncontrolled feedback loops may give rise to issues like the Matthew effect, echo chambers~\cite{chaney2018algorithmic, ge2020understanding, xu2022dynamic} and bias amplification~\cite{chaney2018algorithmic, wang2021deconfounded}. Original debiasing approaches (e.g., back-door adjustment) cannot be applied directly due to the change in the form of causal models. Therefore, deconfounding in multi-step and feedback loop-involved causal models is still an open research field.


\section{Conclusion}

In this survey paper, we have summarized the mechanisms and the strategies of causal inference for recommender systems, from the theoretical perspective: PO framework-based, SCM framework-based and general counterfactuals-based. The survey gives the clear description about the strengths of causal inference for recommendations and manages to use a unified symbol system to describe a large number of existing causal recommender approaches. We hope this survey can well help researchers in the recommendation field to utilize and innovate.