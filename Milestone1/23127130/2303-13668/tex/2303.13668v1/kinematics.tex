\section{Kinematics}
\label{sectKin}

\subsection{Reference frames and momenta parametrization}\label{sec::frames}

In this section we describe the kinematics of the DDVCS process  \eqref{reaction}. The core of our evaluation is done in reference frames that coincide with those used in Ref. \cite{Belitsky:2002tf}. This choice allows us to stay consistent with the literature on the DDVCS topic, but also, in some cases, to facilitate comparison of obtained results. Relations to the Trento frame \cite{trento}, which is typically used to describe the DVCS process, and the frame popular in TCS analyses \cite{Berger:2001xd} are discussed in the following. The kinematical quantities are depicted in Fig. \ref{figure::DDVCSsetUp}.

\begin{figure}[!ht]
    \centering
    \includegraphics[trim={0 0 0 9px},clip]{plots/frameA.pdf}
    \includegraphics[trim={0 0 0 9px},clip]{plots/frameB.pdf}
    \caption{Kinematics of DDVCS process with the depiction of coordinate systems discussed in this article. (left)~Frame with initial proton at rest and transverse component of target polarisation vector, $\vec{S}_\perp$, defined with respect to the incoming virtual photon. (right)~Produced lepton pair center of mass frame.}
    \label{figure::DDVCSsetUp}
\end{figure}


We start the discussion by defining the ``target rest frame I'' (TRF-I), where the initial-state proton stays at rest, and where the $z-$axis is opposite to the incoming photon momentum. In this frame the four-momenta of the initial-state proton and of the incoming photon are:
\beq
\pI^\mu = (M, \vec{0}),\quad \qI^\mu = (\qI^0, 0, 0, \qI^3),
\eeq
where $M$ is the proton mass and $\qI^3 < 0$. With four-momenta of incoming electron, scattered electron, initial- and final-state protons denoted by $\kI$, $\kI'$, $\pI$, $\pI'$, respectively, we may write the incoming virtuality $Q^2$, Mandelstam variable $t$ and  Bj\"orken variable $x_B$ as:
\beq
Q^2 = -\qI^2 = -(\kI-\kI')^2,\quad t = \Delta^2 = (\pI' - \pI)^2,\quad x_B = Q^2/(2\pI\qI) \,.
\eeq
Four-momenta components of incoming virtual photon and final-state proton expressed in terms of these invariants read:
\begin{gather}\label{q0q3}
\qI^0 = \frac{Q}{\eps},\quad \qI^3 = -\frac{Q}{\eps}\sqrt{1 + \eps^2},\\
\pI^{\prime\mu} = \left( M - \frac{t}{2M},\, \modppI\sin\theta_N\cos\phi,\, \modppI\sin\theta_N\sin\phi,\, \modppI\cos\theta_N \right) \,.
\end{gather}
Here, $\eps = 2x_BM/Q$, while $\phi$ and $\theta_N$ are azimuthal and polar angles, respectively, given with respect to TRF-I axes. The angle $\theta_N$ is fixed by the momentum conservation and expressed in the following. We also define the angle $\vphiS$  between the $x$-axis and the transverse component of the initial proton polarization vector with respect to the incoming photon direction.

The momentum of the final-state proton is:
\beq
\modppI = \sqrt{-t\left( 1 - \frac{t}{4M^2} \right)}\,.
\eeq
The four-momentum of the outgoing photon with timelike virtuality $\qI'^2 = Q'^2$ is:
\beq 
\qI'^\mu = \qI'^0(1, \vec{v}),
\eeq
where
\beq\label{qp0v}
\qI'^0 = \frac{Q}{\eps} + \frac{t}{2M},\quad \modv = \sqrt{1 - \left( \frac{Q'}{\qI'^0} \right)^2}\,.
\eeq
The angle $\theta_N$ in terms of $\pI'$ and $\qI'$ reads:
\beq\label{cosN}
\cos\theta_N = -\frac{\eps^2(Q^2 + Q'^2 - t) - 2x_B t}{4x_B M \modppI \sqrt{1 + \eps^2}}\,.
\eeq
Four-momentum of the incoming electron in the massless limit is:
\begin{equation}
    \kI^\mu = E(1,\, \sin\theta_e,\, 0,\, \cos\theta_e)\,,
\end{equation}
where $E$ stands for the electron beam energy, while $\theta_e$ is the polar angle of electron's momentum in TRF-I. This angle is given by:
\begin{equation}
    \cos\theta_e = -\frac{1}{\sqrt{1+\eps^2}}\left( 1 + \frac{y\eps^2}{2} \right)\,,
\end{equation}
where $y = \pI\qI/(\pI\kI) = Q/(\eps E)$ is the inelasticity variable.

It is convenient to describe the produced leptons in their center-of-mass frame. In this work, again in correspondence to Ref. \cite{Belitsky:2002tf}, we define a new coordinate system called TRF-II and then, if required, we boost the relevant four-momenta along the direction of the outgoing photon momentum. TRF-II is rotated with respect to TRF-I by the angle $\thetaG$ between $z$-axis in TRF-I and $\vec{\qI'}$. This angle is given by:
\begin{equation}
    \cos\thetaG = -\frac{\eps (Q^2 - Q'^2 + t)/2  + Q\qI'^0}{Q\modv \qI'^0\sqrt{1+\eps^2}}\,.
\end{equation}
The transformation between TRF-I and TRF-II, without any boost, can be described by the Lorentz transformation matrix:
\begin{equation}
    \mathcal{R}_{\rm II\leftarrow I} = \left(\begin{array}{c|c}
        1 & 0_{1\times 3} \\
        \cline{1-2}
        0_{3\times 1} & (R_{\rm II\leftarrow I})_{3\times 3}
    \end{array}\right),\quad 
    (R_{\rm II\leftarrow I})_{3\times 3} = 
    \begin{pmatrix}
    -c_\gamma c_{\phi} & -c_\gamma s_{\phi} & -s_\gamma \\
    s_{\phi} & -c_{\phi} & 0 \\
    -s_\gamma c_{\phi} & -s_\gamma s_{\phi} & c_\gamma
    \end{pmatrix}\,,
\end{equation}
where short-hands $c_\gamma = \cos\thetaG$, $c_{\phi} = \cos\phi$, and likewise for sines, are used. With this transformation the momenta displayed previously in TRF-I in TRF-II read:
\begin{align}
    q'^\mu & = \qI'^0(1, 0, 0, \modv) \label{q'-TRFII} \,,\\
    q^\mu & = (\qI^0, -s_\gamma \qI^3, 0, c_\gamma \qI^3) \label{q-TRFII} \,,\\
    k^\mu & = E(1, -s_e c_\gamma c_\phi - c_e s_\gamma, s_e s_\phi, -s_e s_\gamma c_\phi + c_e c_\gamma ) \label{k-TRFII} \,,\\
    p^\mu & = \pI^\mu \,, \label{p-TRFII}
\end{align}
where $\qI^0, \qI^3$ and $\qI'^0$ are given in Eqs.~(\ref{q0q3}) and (\ref{qp0v}), and where $s_e = \sin\theta_e, c_e = \cos\theta_e$.

Since muon mass effects are of order $(m_\ell/Q')^2$ we neglect them by making four-momenta of the produced leptons light-like. In the center-of-mass frame of the lepton pair, i.e. in TRF-II frame boosted along its $z$-axis, these four-momenta are given by:
\begin{equation}
    \ell_{\mp}^\mu = \frac{Q'}{2}(1, \pm\vec{\beta}),\quad \vec{\beta} = (\sin\thetaL\cos\phiL, \sin\thetaL\sin\phiL, \cos\thetaL)\,.
\end{equation}
To revert the boost we use the velocity $\modv$ given in Eq.~(\ref{qp0v}), which gives:
\begin{equation}
    \ell_-^\mu = \left( \frac{1}{2}\qI'^0(1+\modv\cos\thetaL), \frac{1}{2}Q'\sin\thetaL\cos\phiL, \frac{1}{2}Q'\sin\thetaL\sin\phiL, \frac{1}{2}\qI'^0(\modv +  \cos\thetaL) \right)
\end{equation}
and $\ell_+$ is obtained by the substitution $(\phiL, \thetaL)\rightarrow (\pi+\phiL, \pi-\thetaL)$.

\subsection{Additional variables}

We introduce the following combinations of four-momenta:
\begin{equation}
    \Delta = p'-p = q-q',\quad \bq = \frac{q+q'}{2},\quad \bp = \frac{p+p'}{2}\,,
\end{equation}
which are used to define the skewness $\xi$ and ``generalized'' Bj\"orken variable $\bxB$ \cite{Diehl:2003ny}:
\begin{gather}
    \xi = -\frac{\Delta\bq}{2\bp\bq} = \frac{Q^2+Q'^2}{2Q^2/x_B - Q^2 - Q'^2 + t}\,,\\ \bxB = \frac{\bQ^2}{2\bp\bq} = \xi\frac{Q^2 - Q'^2 + t/2}{Q^2+Q'^2} \,,
\end{gather}
as well as the ``average'' virtuality:
\begin{equation}
    \bQ^2 = -\bq^2 = \frac{1}{2}\left(Q^2 - Q'^2 + \frac{t}{2}\right)\,.
\end{equation}
It can be shown that $\xi\in (0, 1]$, while $\bxB$ and $\bQ^2$ can be either positive or negative depending on the relative magnitude between the spacelike ($Q^2$) and timelike ($Q'^2$) virtualities:
\begin{equation}
    \bxB = \xi\frac{2\bQ^2}{Q^2+Q'^2}\,.
\end{equation}
For completeness we also specify the range of Mandelstam variable $t$  allowed by the kinematics of the process:
\begin{align}
    t_{0}(t_{1}) = & -\frac{1}{4x_B(1-x_B) + \eps^2}\Bigg\{ 
    2[(1-x_B)Q^2 - x_BQ'^2] + \eps^2(Q^2-Q'^2) \nonumber\\
    & \mp 2\sqrt{1+\eps^2}\sqrt{[(1-x_B)Q^2 - x_BQ'^2]^2 - \eps^2Q^2Q'^2}
    \Bigg\} \,, \label{t0t1def}
\end{align}
where $t_{0}(t_{1})$ corresponds to $-(+)$ sign and minimal (maximal) absolute value of $t$.

\subsection{Relations to Trento and BDP frames}

Throughout the text we consequently work with angles defined in TRF-I and boosted TRF-II frames. In phenomenological applications it is however desired to use Trento frame \cite{trento} instead of TRF-I, and BDP frame \cite{Berger:2001xd} instead of TRF-II, as they are used in modern phenomenology and measurements of DVCS and TCS processes. Here, we give a concise prescription how to translate angles between the frames:
\begin{equation}
    \phi = \begin{dcases}
       \pi - \phiTrento, & \mbox{if }\phiTrento\in [0, \pi] \\
       3\pi - \phiTrento, & \mbox{if }\phiTrento\in(\pi, 2\pi)
    \end{dcases}\,,
\end{equation}
\begin{equation}
    \varphi_{S} = \begin{dcases}
    \pi - \phiSTrento, & \mbox{if }\phiSTrento\in [0, \pi] \\
       3\pi - \phiSTrento, & \mbox{if }\phiSTrento\in(\pi, 2\pi)
    \end{dcases}\,,
\end{equation}
and 
\begin{align}
    \sin\thetaLBDP & = \sqrt{(\ci\sin\thetaL\cos\phiL + \si\cos\thetaL)^2 + \sin^2\thetaL\sin^2\phiL} \,,\\
    \sin\phiLBDP & = \sin\thetaL\sin\phiL/\sin\thetaLBDP \,,\\
    \cos\phiLBDP & = (\ci\sin\thetaL\cos\phiL + \si\cos\thetaL)/\sin\thetaLBDP \,,
\end{align}
where $\ci = \cos\chi$ and $\si = \sin\chi$. The angle $\chi$ accounts for the rotation between TRF-II and BDP frames so that:
\begin{align}
    \ci & = \frac{p'^0\sinh\zeta - p'^3\cosh\zeta}{\sqrt{(p'^1)^2 + (p'^0\sinh\zeta - p'^3\cosh\zeta)^2}} \,,\\
    \si & = \frac{p'^1}{\sqrt{(p'^1)^2 + (p'^0\sinh\zeta - p'^3\cosh\zeta)^2}} \,,
\end{align}
where $\zeta = \arctanh{\modv}$ is the rapidity for the boost from TRF-II to the muon-antimuon CM frame. The components of $p'$ can be computed with momenta in Eqs.~(\ref{q'-TRFII})-(\ref{p-TRFII}) as $p' = p + q - q'$.