\section{Monte Carlo studies}
\label{sectMC}

To make our study easy to use for the theory community, the cross-section formulae we obtained have been implemented in the open-source PARTONS framework \cite{Berthou:2015oaw}. The implementation of DDVCS in the EpIC Monte Carlo (MC) generator \cite{Aschenauer:2022aeb} has followed, making our work also accessible to experimentalists. This development is important to support the physics case of a new generation of experiments, like JLab12, JLab20+ and EIC, for which it is desirable to estimate the measurability of DDVCS. 

In this section we present first results obtained with EpIC for the DDVCS reaction, also checking the accuracy of this generator in reproducing the underlying cross-sections. Our results do not include any simulation of detector effects, and they are not affected by any efficiency one should take into account of in this kind of analysis. Therefore, the presented material should only be considered as a rough estimate and motivation for studying the measurability of DDVCS in more depth. 

The distribution of MC events we obtained as a function of $y$ is shown in Fig.~\ref{figure:MCHist}. The generation was done separately for four configurations of electron, $E_{e}$, and proton, $E_{p}$,  beam energies : i) $E_{e} = 10.6\ \mathrm{GeV}$ and fixed target, ii) $E_{e} = 22\ \mathrm{GeV}$ and fixed target, iii) $E_{e} = 5\ \mathrm{GeV}$ and $E_p = 41\ \mathrm{GeV}$, iv) $E_{e} = 10\ \mathrm{GeV}$ and $E_p = 100\ \mathrm{GeV}$; corresponding to JLab12, JLab20+ and EIC experiments. Additional conditions were used in the generation. The range of $Q^2$ variable was limited to $(0.15, 5)\ \mathrm{GeV}^2$. The lower value corresponds to the anticipated threshold for detection of scattered electrons, while the upper one is a reasonable limit for the observation of cross-section that is suppressed when $Q^2$ becomes large. The range of $Q'^2$ is $(2.25, 9)\ \mathrm{GeV}^2$ and it corresponds to the TCS analysis presented in Ref. \cite{CLAS:2021lky}. For $|t|$ we assumed $(0.1, 0.8)\ \mathrm{GeV}^2$ for JLab experiments and $(0.05, 1)\ \mathrm{GeV}^2$ for EIC ones. As for the angular dependencies, the ranges for $\phi$ and $\phiL$ angles are  $(0.1, 2\pi - 0.1)$, while for $\thetaL$ we have $(\pi/4, 3\pi/4)$. The limitations on angles help to suppress contributions coming from the BH sub-process. 

The total cross-section for the scattering (\ref{reaction}), including all  sub-processes, i.e.  BH, pure DDVCS and their interference, integrated in the aforementioned kinematic domain is tabulated in Table \ref{tab:MCCS}. In this table we also specify the integrated luminosity needed to record 10000 events presented in Fig.~\ref{figure:MCHist}, and the fraction of events recovered after cutting on the $y$ variable: $(0.1, 1)$ for JLab experiments and $(0.05, 1)$ for EIC ones.  

\begingroup
\begin{table}[!ht]
\begin{ruledtabular}
\begin{tabular}{@{}lcccccc@{}}
Experiment & Beam energies & Range of $|t|$ & $\sigma \rvert_{0<y<1}$ & $\mathcal{L}^{10\mathrm{k}}\rvert_{0<y<1}$ & $y_{\mathrm{min}}$ & $\sigma \rvert_{y_{\mathrm{min}} < y < 1} / \sigma \rvert_{0<y<1}$ \\ 
& $[\mathrm{GeV}]$ & $[\mathrm{GeV}^2]$ & $[\mathrm{pb}]$ & $[\mathrm{fb}^{-1}]$ & & \\[5pt] 
JLab12 & $E_{e} = 10.6$, $E_p = M$ & $(0.1, 0.8)$ & $0.14$ & $70$ & $0.1$ & $1$\\
JLab20+ & $E_{e} = 22$, $E_p = M$ & $(0.1, 0.8)$ & $0.46$ & $22$ & $0.1$ & $1$\\ 
EIC & $E_{e} = 5$, $E_p = 41$ & $(0.05, 1)$ & $3.9$ & $2.6$ & $0.05$ & $0.73$\\
EIC & $E_{e} = 10$, $E_p = 100$ & $(0.05, 1)$ & $4.7$ & $2.1$ & $0.05$ & $0.32$ \\
\end{tabular}
\end{ruledtabular}
\caption{Total DDVCS cross-section including all sub-processes, $\sigma \rvert_{0<y<1}$, obtained for given beam energies under the following conditions: $y \in (0, 1)$, $Q^2 \in (0.15, 5)\ \mathrm{GeV}^2$, $Q'^2 \in (2.25, 9)\ \mathrm{GeV}^2$, $\phi, \phiL \in(0.1, 2\pi - 0.1)$, $\thetaL \in (\pi/4, 3\pi/4)$ and $|t|$ range specified in 3th column. Corresponding integrated luminosity required to obtain 10000 events is denoted by $\mathcal{L}^{10\mathrm{k}}\rvert_{0<y<1}$. Fraction of events left after restricting the range of $y$ to $(y_{\mathrm{min}}, 1)$ is given in the last column.}
\label{tab:MCCS}
\end{table}
\endgroup

In Fig.~\ref{figure:MCHist} we also show the expected number of events, coming from a direct seven-fold integration of cross-section (\ref{xsec}) in the aforementioned kinematic domain and limits of $y$ specified by a given bin of the histogram. No free normalization factor is used here: integrated cross-section is multiplied by the luminosity given by EpIC. The comparison between obtained values and MC samples proves the correctness of the generation. 

An additional quantity shown in Fig.~\ref{figure:MCHist} is the fraction of pure DDVCS sub-process in the sample. As expected, this fraction is small, which stresses the need for measuring observables sensitive to the interference in the latter case. 

Additional information is provided in Fig. \ref{figure:MCxiVsRho}, where we show how pure DDVCS events populate the $\xi$ vs.~$\rho$ phase-space. Clearly, the $\xi \neq |\rho|$ domain is probed, proving the importance of the DDVCS reaction in the reconstruction of GPDs from experimental data. We note that because of the ranges choosen for $Q^2$ and $Q'^2$ we typically probe the negative, ``more'' time-like, domain~of~$\rho$. 

\begin{figure}
    \centering
    \includegraphics[width=0.45\textwidth]{plots/mc12.pdf}
    \includegraphics[width=0.45\textwidth]{plots/mc22.pdf}
    \includegraphics[width=0.45\textwidth]{plots/mc541.pdf}
    \includegraphics[width=0.45\textwidth]{plots/mc10100.pdf}
    \caption{Distributions of Monte Carlo events as a function of the inelasticity variable $y$. Each distribution is populated by 10000 events generated for the beam energies specified in the plots. Extra kinematic conditions are specified in the text. Black circle markers and gray histograms correspond  to the left axes.   Reference values for Monte Carlo distributions obtained with  a direct integration of differential cross-section. The fraction of  events coming from the VCS sub-process with respect to all Monte Carlo events is indicated by red square markers corresponding to the right axes.}
    \label{figure:MCHist}
\end{figure}

\begin{figure}
    \centering
    \includegraphics[width=0.45\textwidth]{plots/xiRho12.pdf}
    \includegraphics[width=0.45\textwidth]{plots/xiRho22.pdf}
    \includegraphics[width=0.45\textwidth]{plots/xiRho541.pdf}
    \includegraphics[width=0.45\textwidth]{plots/xiRho10100.pdf}
    \caption{Distribution of Monte Carlo events as a function of the skewness variable $\xi$ and the relative value of generalized Bj\"orken variable $\rho$. Each distribution is populated by 10000 events generated for the DDVCS sub-process at beam energies specified in the plot. Extra kinematical conditions, including cuts on the $y$ variable, are specified in the text.}
    \label{figure:MCxiVsRho}
\end{figure}
