\title{Phenomenology of double deeply virtual Compton scattering in the era of  new experiments}

\author{K.~Deja\,\orcidlink{0000-0002-9083-2382}}
\affiliation{National Centre for Nuclear Research (NCBJ), 02-093 Warsaw, Poland}

\author{V.~Mart\'inez-Fern\'andez\,\orcidlink{0000-0002-0581-7154}}
\affiliation{National Centre for Nuclear Research (NCBJ), 02-093 Warsaw, Poland}
\author{B.~Pire\,\orcidlink{0000-0003-4882-7800}}
\affiliation{Centre de Physique Th\'eorique, CNRS, École Polytechnique, I.P. Paris, 91128 Palaiseau, France  }

\author{P.~Sznajder\,\orcidlink{0000-0002-2684-803X}}
\affiliation{National Centre for Nuclear Research (NCBJ), 02-093 Warsaw, Poland}

\author{J.~Wagner\,\orcidlink{0000-0001-8335-7096}}
\affiliation{National Centre for Nuclear Research (NCBJ), 02-093 Warsaw, Poland}

\date{\today}

\begin{abstract}
We revisit the phenomenology of the deep exclusive electroproduction of a lepton pair, i.e. double deeply virtual Compton scattering (DDVCS), in view of new experiments planned in the near future. The importance of DDVCS in the reconstruction of generalized parton distributions (GPDs) in their full kinematic domain is emphasized. Using Kleiss-Stirling spinor techniques, we provide the leading order complex amplitudes for both DDVCS and Bethe-Heithler sub-processes. Such a formulation turns out to be convenient for practical implementation in the PARTONS framework and EpIC Monte Carlo generator that we use in simulation studies. 
\end{abstract}
\pacs{13.60.Fz, 12.38.Bx, 13.88.+e}

\maketitle

\section{Introduction}

The exclusive electroproduction of a lepton pair,
\begin{equation}
    e(k) + N(p)  \to e'(k') + N'(p') + \mu^+(\ell_+) + \mu^-(\ell_-) \,,
    \label{reaction}
\end{equation}
receives contributions from two sub-processes having the same initial and final states. One of them is purely QED Bethe-Heitler process (BH), while the other one is double deeply virtual Compton scattering (DDVCS):
\begin{equation}
    \gamma^*(q) + N(p)  \to \gamma^*(q') + N'(p') \,.
    \label{reaction2}
\end{equation}
 In the generalized Bjorken limit the amplitude of the latter is known \cite{Mueller:1998fv} to factorize into perturbatively calculable coefficient functions and generalized parton distributions (GPDs) that unravel the three-dimensional structure of the nucleon \cite{Diehl:2003ny,Belitsky:2005qn}. Conditions necessary to factorize DDVCS amplitude are the existence of a large scale, which may be either $Q^2 = -q^2 = -(k-k')^2$ or $Q'^2 = q'^2 = (\ell_+ +\ell_-)^2$, and relatively small squared four-momentum transfer to the nucleon, $-t = -(p'-p)^2$. Diagrams for both BH and DDVCS sub-processes are shown in Fig.~\ref{figure::RelevantDiagrams}. 

\begin{figure}[!ht]
    \centering
    \includegraphics{plots/tikz/1_tikz_0.pdf}
    \includegraphics{plots/tikz/1_tikz_1.pdf}
    \includegraphics{plots/tikz/1_tikz_2.pdf}
    \caption{(from left to right) The double deeply virtual Compton scattering (DDVCS) process at leading order  and the two types of Bethe-Heitler processes, which contribute to the electroproduction of a lepton pair. Complementary crossed diagrams  are not shown in this figure.}
    \label{figure::RelevantDiagrams}
\end{figure}
The DDVCS process has never been measured so far. Its importance comes from the fact that it is a most promising source of experimental information on GPDs. This can easily be seen in the fact that at the lowest order (LO) the deeply virtual Compton scattering (DVCS) \cite{Belitsky_2002} and timelike Compton scattering (TCS) \cite{Berger:2001xd} amplitudes, which are limits of DDVCS at $Q'^2 \to 0$ (DVCS) and $Q^2 \to 0$ (TCS), depend only on the $x=\xi$ domain, where $x$ is the average parton momentum, and $\xi$ describes the longitudinal momentum transfer. For the ongoing discussion at NLO see Ref.~\cite{Bertone:2021yyz}. The existence of two large scales in DDVCS, $Q^2$ and $Q'^2$, makes this process sensitive to the $x\neq\xi$ region, already at LO. We note here for completeness that another promising source of GPD information in the unexplored $x\neq\xi$ domain is lattice-QCD \cite{Bhattacharya:2022aob, Joo:2020spy}. 

The importance of DDVCS has  already been discussed some twenty years ago \cite{Belitsky:2002tf,guidal2003}. Although a quite detailed study of the phenomenological peculiarities of DDVCS already exists \cite{Belitsky:2003fj}, it is deemed appropriate to revisit the promises of this process, as it will be studied with great care in the near future at both fixed target facilities \cite{Chen:2014psa,Camsonne:2017yux,Zhao:2021zsm} and electron-ion colliders \cite{AbdulKhalek:2021gbh, Anderle:2021wcy}. Thanks to the use of Kleiss-Stirling techniques \cite{Kleiss:1984dp, Kleiss:1985yh}, we are allowed to present both the complete BH and the leading twist (LT) leading order DDVCS amplitudes as complex quantities. Such a formulation is convenient for practical implementation. We use the open-source PARTONS framework \cite{Berthou:2015oaw} to perform the numerical evaluation of DDVCS observables. Results of our work have also been implemented in the EpIC Monte Carlo generator \cite{Aschenauer:2022aeb} for a tentative estimate of DDVCS measurability in future experiments. 

The content of this article is as follows. In Sect.~\ref{sectKin} the kinematics of DDVCS is reviewed. In Sect.~\ref{sectAmp} we calculate the scattering amplitudes  for both Bethe-Heitler and leading order (in $\alpha_s$) DDVCS sub-processes using Kleiss-Stirling spinor techniques \cite{Kleiss:1984dp,Kleiss:1985yh}. In Sect.~\ref{sectLimits} both $Q^2 \to 0$ and $Q'^2 \to 0$ limits of our calculation are compared with independent results for TCS and DVCS. In Sect.~\ref{sectObs} we discuss a few specific observables for DDVCS, while in Sect.~\ref{sectMC} we present results obtained in our Monte Carlo study, which give some hints on the measurability of this process. Section~\ref{sectConc} provides the summary of this article.
