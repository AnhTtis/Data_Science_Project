\section{Amplitudes and cross-section}
\label{sectAmp}

Electroproduction of a massless lepton pair can be described in terms of seven variables, so that the differential cross-section reads \cite{Belitsky:2002tf, guidal2003}:
\begin{equation}\label{xsec}
    \frac{d^7\sigma}{dx_B dQ^2 dQ'^2 d|t| d\phi d\Omega_\ell} = \frac{\alpha_{\rm em}^4}{16(2\pi)^3} \frac{x_By^2}{Q^4\sqrt{1+\eps^2}}\left|\frac{\M}{e^4}\right|^2 \,,
\end{equation}
where $\M$ stands for the amplitude of the process. If the target is polarized transversely, cross-section becomes dependent also on $\vphiS$ angle defined in Fig.~\ref{figure::DDVCSsetUp}. In such case, one should consider an 8-fold differential cross-section: $d^8\sigma/(dx_B dQ^2 dQ'^2 d|t| d\phi d\Omega_\ell d\vphiS) = d^7\sigma/(dx_B dQ^2 dQ'^2 d|t| d\phi d\Omega_\ell)\times (2\pi d\vphiS)^{-1}$.

The amplitude $\M$ receives contributions from all sub-processes (and their crossed partners) depicted in Fig.~\ref{figure::RelevantDiagrams}:
\begin{equation}
    i\M = i\M_{\rm DDVCS} + i\M_{\rm BH1} + i\M_{\rm BH1X} + i\M_{\rm BH2} + i\M_{\rm BH2X} \,.
\end{equation}
In the following we will separately evaluate each sub-amplitude. For this purpose we will make use of techniques developed by Kleiss and Stirling (KS) in Refs.~\cite{Kleiss:1984dp, Kleiss:1985yh}.

\subsection{DDVCS amplitude}

\begin{figure}[ht!]
    \centering
    \includegraphics{plots/tikz/2_tikz_0.pdf}
    \includegraphics{plots/tikz/2_tikz_1.pdf}
    \caption{General DDVCS diagram in terms of Compton tensor $T^{\mu\nu}$ and its leading-twist (LT) approximation. The crossed diagram is not shown.} 
    \label{figure::ddvcsGeneral->LT}
\end{figure}

The amplitude of DDVCS contribution is the only one related to the internal distribution of partons in the hadron. It reads:
\begin{equation}
    i\M_{\rm DDVCS} = \frac{ie^4 \bar{u}(\ell_-,s_\ell)\gamma_\mu v(\ell_+,s_\ell)\bar{u}(k',s)\gamma_\nu u(k,s) }{(q^2+i0)(q'^2+i0)} T^{\mu\nu}_{s_2s_1}\,.
\end{equation}
Here, the Compton tensor $T^{\mu\nu}_{s_2s_1}$ is given by:
\begin{equation}\label{comptonTensor}
    T_{s_2s_1}^{\mu\nu} = i\int d^4z\ e^{i\bq z}\langle p',s_2|\mathcal{T}\{j^\mu(z/2)j^\nu(-z/2)\}|p,s_1\rangle,
\end{equation}
where $j^\mu(x) = \sum_f \frac{e_f}{e}\bar{\q}_f(x)\gamma^\mu\q_f(x)$ accounts for quark currents with flavors $f$ and electric charges $e_f$, and where $s_2, s_1 = \pm$ stand for hadron's helicity in final and initial state, respectively. 

To determine the leading terms in the Bj{\"o}rken regime it is convenient to define two light-like vectors, $n$ and $n^\star$, where $nn^\star = 1$. These light-like vectors are combinations of $\bp$ and $\bq$: 
\begin{align}
    n^\mu & = \frac{1}{\bp\bq R}\bq^\mu - \frac{1 - R}{ 2\bp\bq\bxB\delta^2 R}\bp^\mu \,,\\
    n^{\star\mu} & = -\frac{\bxB\delta^2}{R}\bq^\mu + \frac{1+R}{2R}\bp^\mu \,,
\end{align}
where $R = \sqrt{1+4\bxB^2\delta^2}$ and $\delta^2 = (M^2 - t/4)/(2\bp\bq\bxB)$. They allow for the decomposition of any four-vector in the following way: $v^\mu=v^-n^\mu + v^+n^{\star\mu} + v_\perp^\mu$. 

The Compton tensor introduced in Eq.~\eqref{comptonTensor} was computed to twist-3 accuracy in Ref.~\cite{Belitsky_2000}. For the purposes of our work it is enough to stay at leading twist (LT) and LO in $\alpha_s$, which corresponds to the  following decomposition:
\begin{equation}\label{comptonTensor_spinors}
    T^{\mu\nu}_{s_2s_2} = T^{(V)\mu\nu}\bar{u}(p',s_2)\left[ (\cffh + \cffe)\slashed{n} - \frac{\cffe}{M}\bp^+ \right] u(p, s_1) + T^{(A)\mu\nu}\bar{u}(p',s_2)\left[ \cffht\slashed{n} + \frac{\cffet}{2M}\Delta^+ \right]\gamma^5u(p,s_1)\,.
\end{equation}
Here, Compton form factors (CFFs) are defined as:
\begin{align}
    (\cffh,\cffe)(\bxB, \xi, t) & = \sum_{f=\{u, d, s\}}\int_{-1}^1 dx\ C_f^{(-)}(x, \bxB)(H_f, E_f)(x, \xi, t) \,, \label{cffHE} \\
    (\cffht,\cffet)(\bxB, \xi, t) & = \sum_{f=\{u, d, s\}}\int_{-1}^1 dx\ C_f^{(+)}(x, \bxB)(\widetilde{H}_f, \widetilde{E}_f)(x, \xi, t) \,, \label{cffHtEt}
\end{align}
where $C^{(\pm)}_f$ are hard scattering coefficient functions, which at LO read\footnote{This form of the LO coefficient function reflects the fact that, as for DVCS and TCS, only the charge conjugation even part of GPDs contribute to the DDVCS process. To access the complementary charge conjugation odd part of quark GPDs, one needs to study other processes like diphoton photo- or electroproduction \cite{Pedrak:2017cpp, Pedrak:2020mfm, Grocholski:2021man, Grocholski:2022rqj}. To access the chiral-odd quark GPDs, processes containing mesons in the final state are necessary \cite{Ivanov:2002jj, Boussarie:2016qop, Duplancic:2023kwe}.
}
\begin{equation}\label{C+-}
    C^{(\pm)}_f(x, \bxB) = \left(\frac{e_f}{e}\right)^2\left( \frac{1}{\bxB - x - i0} \pm \frac{1}{\bxB + x - i0} \right)\,,
\end{equation}
and where $H, E, \widetilde{H}$ and $\widetilde{E}$ are GPDs related to quark correlators (notice that $\bp^+ = 1$) as
%($n$ is a light-like vector fixing the ``plus'' direction)
\begin{align}
    \int \frac{d\lambda}{2\pi}\ e^{-i\lambda x}\langle p',s_2| \bar{\q}_f(\lambda n/2)\slashed{n}\q_f(-\lambda n/2) |p, s_1\rangle & =\bar{u}(p',s_2)\left[ (\cffh + \cffe)\slashed{n} - \frac{\cffe}{M}\bp^+ \right] u(p, s_1) \equiv \mathcal{J}^+_{s_2s_1} \label{Jcal}\,,\\
    \int \frac{d\lambda}{2\pi}\ e^{-i\lambda x}\langle p',s_2| \bar{\q}_f(\lambda n/2)\slashed{n}\gamma^5\q_f(-\lambda n/2) |p, s_1\rangle & = \bar{u}(p',s_2)\left[ \cffht\slashed{n} + \frac{\cffet}{2M}\Delta^+ \right]\gamma^5u(p,s_1) \equiv \mathcal{J}_{s_2s_1}^{(5)+}\,. \label{J5cal}
\end{align}
Finally, the Lorentz components have been isolated and concealed in the following tensors: 
\begin{align}
    T^{(V)\mu\nu} & = -\frac{1}{2}\left( g^{\mu\nu} - \frac{q^\mu q'^\nu}{qq'} \right) + \frac{\bxB}{\bp\bq}\left( \bp^\mu - \frac{\bp q'}{qq'}q^\mu \right)\left( \bp^\nu - \frac{\bp q}{qq'}q'^\nu \right) \label{T1}\,, \\
    T^{(A)\mu\nu} & = \frac{i}{2\bp\bq}\epsilon_{\theta\lambda\rho\sigma}\bp^\rho\bq^\sigma \left( g^{\mu\theta} - \frac{\bp^\mu q'^\theta}{\bp q'} \right) \left( g^{\nu\lambda} - \frac{\bp^\nu q^\lambda}{\bp q} \right)\,. \label{T2}
\end{align}

To LT, the dominant terms in the amplitude arise from keeping the following structures in tensors above:
\begin{align}
    T^{(V)\mu\nu} & = -\frac{1}{2}(g^{\mu\nu} - n^\mu n^{\star\nu} - n^\nu n^{\star\mu}) \equiv -\frac{1}{2}g_\perp^{\mu\nu}\,,\\
    T^{(A)\mu\nu} & = -\frac{i}{2}{\epsilon^{\mu\nu}}_{\rho\sigma} n^\rho n^{\star\sigma} \equiv -\frac{i}{2}\epsilon^{\mu\nu}_\perp\,,
\end{align}
which are used to define the vector and axial components of the DDVCS amplitude:
\begin{equation}
    i\M_{\rm DDVCS} = \frac{-ie^4}{(Q^2-i0)(Q'^2+i0)}\left( i\M^{(V)}_{\rm DDVCS} + i\M^{(A)}_{\rm DDVCS} \right)\,.
\end{equation}
Here, the first term (the vector contribution) corresponds to $T^{(V)}$, while the second one (the axial contribution) to $T^{(A)}$. In what follows we study these two contributions separately.

\subsubsection{Vector contribution to the DDVCS amplitude}

Up to photon propagators and factors $ie^4$, the vector amplitude may be written as:
\begin{equation}
    i\M_{\rm DDVCS} ^{(V)} = -\frac{g_\perp^{\mu\nu}}{2}\bar{u}(\ell_-,s_\ell)\gamma_\mu v(\ell_+,s_\ell)\bar{u}(k',s)\gamma_\nu u(k,s) \mathcal{J}^+_{s_2s_1}\,,
\end{equation}
where $s_\ell, s = \pm$ stand for muon's and electron's helicities, respectively. $\mathcal{J}^+$ is given by Eq.~\eqref{Jcal} and can be further decomposed into:
\begin{equation}\label{JcalDecomposition}
    \mathcal{J}^+_{s_2s_1} = (\cffh + \cffe) \Joplus_{s_2s_1} - \frac{\cffe}{M}\Jt_{s_2s_1}\,,
\end{equation}
where
\begin{equation}\label{Jcal1muJcal2}
    \mathcal{J}^{(1)\mu}_{s_2s_1} =\bar{u}(p',s_2)\gamma^\mu u(p, s_1)
    , \quad
    \mathcal{J}^{(2)}_{s_2s_1} =\bar{u}(p',s_2) u(p, s_1)\,.
\end{equation}
To use the KS methods for massive spinors, hadron momenta $p$ and $p'$ have to be decomposed by means of auxiliary light-like vectors, this is, $p = r_1+r_2$ and $p' = r'_1+r'_2$. The decomposition of nucleon spinors into massles ones reads:
\begin{align}
    u(p,+) & = \frac{s(r_1, r_2)}{M}u(r_1,+) + u(r_2,-) \,,\label{u+massive} \\
    u(p,-) & = \frac{t(r_1, r_2)}{m}u(r_1,-) + u(r_2,+) \,,\label{u-massive} 
\end{align}
where for two light-like vector $a$ and $b$:
\begin{align}
    s(a, b) & = \bar{u}(a,+)u(b,-) = -s(b, a)\,, \label{sKS_def}\\
    t(a, b) & = \bar{u}(a,-)u(b,+) = [s(b, a)]^*\,. \label{tKS_def}
\end{align}
Explicit computation of bilinears above shows that $s(a, b)$ acquires the simple form (see Eq.~(3.1) in Ref.~\cite{Kleiss:1985yh}):
\begin{equation}\label{sKS_expression}
    s(a, b) = (a^2 + ia^3)\sqrt{\frac{b^0 - b^1}{a^0 - a^1}} - (a\leftrightarrow b)\,,
\end{equation}
as long as $a\cdot\kappa_0\neq 0$ and $b\cdot\kappa_0 \neq 0$ with $\kappa^\mu_0 = (1, 1, 0, 0)$.

By making use of the above formulae we get the following expressions:
\begin{equation}\label{Jcal1mu}
    \mathcal{J}^{(1)\mu}_{s_2s_1} = Y_{s_2s_1}\bar{u}(r'_{s_2},+)\gamma^\mu u(r_{s_1},+) + Z_{s_2s_1}\bar{u}(r'_{-s_2},-)\gamma^\mu u(r_{-s_1},-)
\end{equation}
where $r'_{s_2} = r'_1\delta_{s_2+} + r'_2\delta_{s_2-}$ and $r_{s_1} = r_1\delta_{s_1+} + r_2\delta_{s_1-}$. Phases\begin{NoHyper}\footnote{$Y$ and $Z$ have unit modulus as $|s(r_1, r_2)|^2 = 2r_1r_2 = M^2$. Likewise, for $s\leftrightarrow t$ and/or $r_{1, 2}\leftrightarrow r'_{1, 2}$.}\end{NoHyper} $Y, Z$ read:
\begin{equation}
    Y_{s_2s_1} = \delta_{s_2+}\delta_{s_1+}\frac{t(r'_2,r'_1)s(r_1,r_2)}{M^2} + \delta_{s_2+}\delta_{s_1-}\frac{t(r'_2,r'_1)}{M} + \delta_{s_2-}\delta_{s_1+}\frac{s(r_1,r_2)}{M} + \delta_{s_2-}\delta_{s_1-}
\end{equation}
and
\begin{equation}
    Z_{s_2s_1} =
    \delta_{s_2-}\delta_{s_1-}\frac{s(r'_2,r'_1)t(r_1,r_2)}{M^2} + \delta_{s_2-}\delta_{s_1+}\frac{s(r'_2,r'_1)}{M} + \delta_{s_2+}\delta_{s_1-}\frac{t(r_1,r_2)}{M} + \delta_{s_2+}\delta_{s_1+}\,.
\end{equation}
A similar calculation for $\Jt_{s_2s_1}$ yields:
\begin{align}\label{Jcal2}
    \Jt_{s_2s_1} = &\phantom{+} \delta_{s_2+}\delta_{s_1+}
    \left[
    \frac{t(r'_2,r'_1)s(r'_1,r_2)}{M} + \frac{t(r'_2,r_1)s(r_1,r_2)}{M} 
    \right]
    \nonumber\\
    & + \delta_{s_2+}\delta_{s_1-}\left[ \frac{t(r'_2,r'_1)t(r_1,r_2)s(r'_1,r_1)}{M^2} + t(r'_2,r_2) \right] \nonumber\\
    & + \delta_{s_2-}\delta_{s_1+}\left[ \frac{s(r'_2,r'_1)s(r_1,r_2)t(r'_1,r_1)}{M^2} + s(r'_2,r_2) \right] \nonumber\\
    & +  \delta_{s_2-}\delta_{s_1-}
    \left[
    \frac{s(r'_2,r'_1)t(r'_1,r_2)}{M} + \frac{s(r'_2,r_1)t(r_1,r_2)}{M}
    \right]\,.
\end{align}
For further calculations it is useful to define two other scalars, namely the contraction of two currents: 
\begin{align}\label{function_f}
    f(\lambda, k_0, k_1; \lambda', k_2, k_3) = & \bar{u}(k_0,\lambda)\gamma^\mu u(k_1, \lambda)\bar{u}(k_2,\lambda')\gamma_\mu u(k_3,\lambda') \nonumber\\
    = & 2 [ s(k_2,k_1)t(k_0,k_3)\delta_{\lambda-}\delta_{\lambda'+} + t(k_2,k_1)s(k_0,k_3)\delta_{\lambda+}\delta_{\lambda'-} \nonumber\\
    & + s(k_2,k_0)t(k_1,k_3)\delta_{\lambda+}\delta_{\lambda'+} + t(k_2,k_0)s(k_1,k_3)\delta_{\lambda-}\delta_{\lambda'-} ]\,,
\end{align}
and the contraction of a current with a light-like vector $a$:
\begin{align}\label{function_g}
    g(s, \ell, a, k) = & \bar{u}(\ell, s)\slashed{a}u(k, s) \nonumber\\
    = & \delta_{s+}s(\ell,a)t(a,k) + \delta_{s-}t(\ell,a)s(a,k)\,.
\end{align}

By contracting Eq.~\eqref{Jcal1mu} with vector $n$ we arrive to:
\begin{equation}
    \Joplus_{s_2s_1}= \mathcal{J}^{(1)\mu}_{s_2s_1}n_\mu = Y_{s_2s_1}g(+,r'_{s_2},n,r_{s_1}) + Z_{s_2s_1}g(-,r'_{-s_2},n,r_{-s_1}) \,.
\end{equation}

Finally, the vector contribution to the DDVCS amplitude is:
\begin{align}\label{iM1ddvcs_final}
    i\M^{(V)}_{\rm DDVCS} = & -\frac{1}{2}\Bigg[ f(s_\ell, \ell_-, \ell_+; s, k', k) -  g(s_\ell,\ell_-,n^\star,\ell_+)g(s, k',n,k) - g(s_\ell,\ell_-,n,\ell_+)g(s, k',n^\star,k) \Bigg] \nonumber\\
    & \times \Bigg[ (\cffh + \cffe) [ Y_{s_2s_1}g(+,r'_{s_2},n,r_{s_1}) + Z_{s_2s_1}g(-,r'_{-s_2},n,r_{-s_1}) ] - \frac{\cffe}{M} \Jt_{s_2s_1} \Bigg]\,.
\end{align}
\subsubsection{Axial contribution to the DDVCS amplitude}
The axial amplitude $T^{(A)}$  may be written up to photon propagators and factors $ie^4$ as:
\begin{equation}\label{iM2ddvcs_initial}
    i\M^{(A)}_{\rm DDVCS} = \frac{-i\epsilon_\perp^{\mu\nu}}{2}\bar{u}(\ell_-,s_\ell)\gamma_\mu v(\ell_+,s_\ell)\bar{u}(k',s)\gamma_\nu u(k, s)\mathcal{J}^{(5)+}_{s_2s_1}\,,
\end{equation}
where $\mathcal{J}^{(5)+}$ was defined in Eq.~\eqref{J5cal}. We distinguish $\Jofplus$ and $\Jtfplus$:
\begin{equation}
    \mathcal{J}^{(5)+}_{s_2s_1} = \cffht\Jofplus_{s_2s_1} + \cffet\frac{\Delta^+}{2M}\Jtfplus_{s_2s_1}\,,
\end{equation}
such that
\begin{align}
    \Jofplus_{s_2s_1} & = \bar{u}(p',s_2)\slashed{n}\gamma^5u(p,s_1) \label{J15+}\,,\\
    \Jtfplus_{s_2s_1} & = \bar{u}(p',s_2)\gamma^5u(p,s_1) \,.\label{J25}
\end{align}

Similarly to the vector case, we can express the currents in terms of the scalar functions:
\begin{align}
    \Jofplus_{s_2s_1} = & 
    \phantom {-} \delta_{s_2+}\delta_{s_1+}\Bigg[ \frac{t(r'_2,r'_1)s(r_1,r_2)t(n,r_1)s(r'_1,n)}{M^2} - s(n,r_2)t(r'_2,n) \Bigg] 
    \nonumber \\  & 
    - \delta_{s_2-}\delta_{s_1-}
    \Bigg[ \frac{s(r'_2,r'_1)t(r_1,r_2)s(n,r_1)t(r'_1,n)}{M^2} - t(n,r_2)s(r'_2,n) \Bigg] 
    \nonumber \\  &
    + \delta_{s_2+}\delta_{s_1-} \frac{t(r'_2,r'_1)t(n,r_2)s(r'_1,n) - t(r_1,r_2)s(n,r_1)t(r'_2,n)}{M} 
    \nonumber \\  &
    - \delta_{s_2-}\delta_{s_1+}
    \frac{s(r'_2,r'_1)s(n,r_2)t(r'_1,n) - s(r_1,r_2)t(n,r_1)s(r'_2,n)}{M} 
\end{align}
and
\begin{align}
    \Jtfplus_{s_2s_1} =  & \phantom{-} \delta_{s_2+}\delta_{s_1+}\frac{ s(r_1,r_2)t(r'_2,r_1) - t(r'_2,r'_1)s(r'_1,r_2) }{M} 
     + \delta_{s_2+}\delta_{s_1-}\Bigg[ t(r'_2,r_2) - \frac{t(r'_2,r'_1)t(r_1,r_2)s(r'_1,r_1)}{M^2} \Bigg] \nonumber\\
    & - \delta_{s_2-}\delta_{s_1-}\frac{ t(r_1,r_2)s(r'_2,r_1) - s(r'_2,r'_1)t(r'_1,r_2) }{M} 
     - \delta_{s_2-}\delta_{s_1+} \Bigg[ s(r'_2,r_2) - \frac{s(r'_2,r'_1)s(r_1,r_2)t(r'_1,r_1)}{M^2} \Bigg]\,.
\end{align}

Because of $\epsilon_\perp$-structure in Eq.~\eqref{iM2ddvcs_initial} we cannot contract the above lepton currents, hence we need to compute them explicitly. The massless lepton current for muon-antimuon pair can be written as:
\begin{align}
    j_\mu(s_\ell, \ell_-, \ell_+) = & \bar{u}(\ell_-,s_\ell)\gamma_\mu v(\ell_+,s_\ell) \nonumber\\
    = & \frac{1}{2N_{\ell_-\ell_+}}{\rm tr}\left\{ \slashed{\ell}_-\gamma_\mu\slashed{\ell}_+\omega_{-s_\ell}\slashed{\K}_0 \right\} \nonumber\\
    = & \frac{1}{N_{\ell_-\ell_+}} \{ \ell_{-,\mu}(\ell_+\K_0) + \ell_{+,\mu}(\ell_-\K_0) - \K_{0,\mu}Q'^2/2 + is_\ell\epsilon_{\mu\alpha\beta\gamma}\ell_-^\alpha\ell_+^\beta\K_0^\gamma
    \}\,,
\end{align}
where $N_{\ell_-\ell_+} = \sqrt{(\ell_-\K_0)(\ell_+\K_0)}$.
Likewise, for the electron current ($N_{k'k} = \sqrt{(k'\K_0)(k\K_0)}$):
\begin{equation}
    j_\mu(s, k', k) = \frac{1}{N_{k'k}} \{ k'_{\mu}(k\K_0) + k_{\mu}(k'\K_0) - \K_{0,\mu}Q^2/2 + is\epsilon_{\mu\alpha\beta\gamma}k'^\alpha k^\beta\K_0^\gamma
    \}\,.
\end{equation}

Finally, the axial contribution is given by:
\begin{equation}\label{iM2ddvcs_final}
    i\M^{(A)}_{\rm DDVCS} = \frac{-i}{2} \epsilon^{\mu\nu}_\perp j_\mu(s_\ell,\ell_-,\ell_+)j_\nu(s, k', k)\left[ \cffht\Jofplus_{s_2s_1} + \cffet\frac{\Delta^+}{2M}\Jtfplus_{s_2s_1} \right]\,.
\end{equation}

Violation of gauge invariance due to a truncation of the twist expansion of the Compton tensor is a problem discussed in papers like \cite{Anikin_2000, Vanderhaeghen_1999} and more recently in \cite{Braun:2012hq, Braun:2014sta}. If any of the photon momenta carries a perpendicular component (as happens in TRF-II described in Sect.~\ref{sec::frames}), gauge invariance is violated by terms of order $O(\Delta_\perp/\sqrt{2\bp\bq})$, which can be considered twist-3 effects. In Ref.~\cite{Belitsky_2000} it was proven that by going to twist-3, violation is of order twist-4 and so on. Despite restoring gauge invariance is possible twist-by-twist, the existence of gauge symmetry-breaking terms affects predictions at LT. There are two ways to deal with it: 1) Lorentz transformation to a reference frame where photons do not carry perpendicular components, or 2) evaluation of hard part, which includes the Compton tensor's Lorentz structures, at $t = t_0$ (defined by Eq.~\eqref{t0t1def}). Choosing to evaluate at $t_0$ ensures that higher-twist corrections proportional to $\Delta_\perp^\mu$ vanish, avoiding violation of gauge invariance. This evaluation is consistent with the longitudinal factorization that is at the core of GPD description. In our phenomenological study in Sect.~\ref{sectObs}, we opt for this second option. As a consequence, the vector $n$, which defines the ``plus'' direction is also evaluated at $t = t_0$.

\subsection{The first Bethe-Heitler amplitude}

Now we describe the amplitude of BH1 and its crossed partner BH1X, both depicted in Fig.~\ref{figure::BH1}.

\begin{figure}[!ht]
    \centering
    \includegraphics{plots/tikz/3_tikz_0.pdf}
    \includegraphics{plots/tikz/4_tikz_0.pdf}
    \caption{Diagrams for Bethe-Heitler 1 (BH1) and its crossed partner (BH1X) for electroproduction of muon pairs.}
    \label{figure::BH1}
\end{figure}

The amplitude of BH1 reads:
\begin{equation}
    i\M_{\rm BH1} = \frac{ie^4 \bar{u}(\ell_-, s_\ell)\gamma^\beta v(\ell_+,s_\ell)\bar{u}(k',s)\gamma_\beta(\slashed{k}-\slashed{\Delta})\gamma_\alpha u(k, s) J^\alpha_{s_2s_1} }{(Q'^2+i0)(t+i0)((k-\Delta)^2+i0)}\,,
\end{equation}
where the electromagnetic hadronic current is defined by means of Dirac, $F_1(t)$, and Pauli, $F_2(t)$, elastic form factors (FFs):
\begin{align}
    J^\alpha_{s_2s_1} =~ & \langle p',s_2| \sum_f \frac{e_f}{e}\bar{\q}_f(0)\gamma^\alpha \q_f(0) |p,s_1\rangle 
    %\ \label{Jalpha_def} \\
    =~  \bar{u}(p',s_2)\left[ (F_1+F_2)\gamma^\alpha - \frac{F_2}{M}\bp^\alpha \right]u(p,s_1)\,. \label{Jalpha_spinors}
\end{align}

The structure of this current is the same as that of $\mathcal{J}^+$, see Eq.~\eqref{Jcal}. Therefore, up to CFF $\leftrightarrow$ FF replacement, one may apply the decomposition \eqref{JcalDecomposition}:
\begin{equation}\label{Jalpha}
    J^\alpha_{s_2s_1} = (F_1 + F_2) \mathcal{J}^{(1)\alpha}_{s_2s_1} - \frac{F_2}{M}\bp^\alpha\Jt_{s_2s_1}\,,
\end{equation}
where $\mathcal{J}^{(1)\alpha}$ and $\Jt$ are defined in Eqs.~\eqref{Jcal1mu} and \eqref{Jcal2}, respectively. 

With the decomposition \eqref{Jalpha} the evaluation of $i\M_{\rm BH1}$ can be separated into two parts related to $\mathcal{J}^{(1)}$ and $\Jt$:
\begin{equation}\label{iMbh1_splitting}
    i\M_{\rm BH1} = \frac{ie^4 \left( i\M^{(1)}_{\rm BH1} + i\M^{(2)}_{\rm BH1} \right) }{(Q'^2+i0)(t+i0)((k-\Delta)^2+i0)}\,.
\end{equation}

The first term in the numerator of Eq.~\eqref{iMbh1_splitting}, by means of the scalar function $f$ introduced in Eq.~\eqref{function_f}, can be expressed as:
\begin{equation}\label{eq:iM1BH1}
    i\M^{(1)}_{\rm BH1} = (F_1+F_2)\sum_Lf(s_\ell,\ell_-,\ell_+; s, k', L)\Big( Y_{s_2s_1}f(s,L,k; +,r'_{s_2}, r_{s_1}) + Z_{s_2s_1}f(s,L,k; -,r'_{-s_2}, r_{-s_1}) \Big)\,,
\end{equation}
where $L\in\{k',\ell_-,\ell_+\}$.

The second term in Eq.~(\ref{iMbh1_splitting}), after expanding $\bp$ in the sum of light-like vectors $R\in\{r_1, r_2, r'_1, r'_2\}$, has the following form:
\begin{equation}\label{eq:iM2BH1}
    i\M^{(2)}_{\rm BH1} = -\frac{F_2}{2M}\Jt_{s_2s_1}\sum_{L, R}f(s_\ell,\ell_-,\ell_+; s,k',L)g(s, L, R, k)\,.
\end{equation}

The amplitude of crossed BH1 reads:
\begin{equation}
    i\M_{\rm BH1X} = \frac{ie^4 \bar{u}(\ell_-,s_\ell)\gamma^\beta v(\ell_+,s_\ell)\bar{u}(k',s)\gamma_\alpha(\slashed{k}' + \slashed{\Delta})\gamma_\beta u(k,s)J^\alpha_{s_2s_1} }{(q'^2+i0)(t+i0)((k'+\Delta)^2+i0)}\,.
\end{equation}

Analogously to Eqs.~\eqref{iMbh1_splitting}, \eqref{eq:iM1BH1} and \eqref{eq:iM2BH1}:
\begin{equation}
    i\M_{\rm BH1X} = \frac{ie^4 \left( i\M^{(1)}_{\rm BH1X} + i\M^{(2)}_{\rm BH1X} \right) }{(Q'^2+i0)(t+i0)((k'+\Delta)^2+i0)}\,,
\end{equation}
and
\begin{align}
    i\M^{(1)}_{\rm BH1X} = & (F_1+F_2)\sum_L\sigma(L) f(s_\ell,\ell_-,\ell_+; s, L, k') \Big( Y_{s_2s_1}f(s, k', L; +, r'_{s_2},r_{s_1}) + Z_{s_2s_1}f(s, k', L;-,r'_{-s_2},r_{-s_1}) \Big) \,,\\
    i\M^{(2)}_{\rm BH1X} = & -\frac{F_2}{2M}\Jt_{s_2s_1}\sum_{L,R}\sigma(L)f(s_\ell,\ell_-,\ell_+; s, L, k)g(s,k',R,L)\,,
\end{align}
where $L\in\{k,\ell_-,\ell_+\}$, $R\in\{r_1,r_2,r'_1,r'_2\}$ and $\sigma(k) = +1$, $\sigma(\ell_-)= \sigma(\ell_+)=-1$.

\subsection{The second Bethe-Heitler amplitude}

\begin{figure}[!ht]
    \centering
    \includegraphics{plots/tikz/5_tikz_0.pdf}
    \includegraphics{plots/tikz/6_tikz_0.pdf}
    \caption{Diagrams for Bethe-Heitler 2 (BH2) and its crossed partner (BH2X) for electroproduction of muon pairs.}
    \label{figure::BH2}
\end{figure}

BH2 and its crossed partner are shown in Fig.~\ref{figure::BH2}. The amplitude of the former reads: 
\begin{equation}
    i\M_{\rm BH2} = \frac{ie^4 \bar{u}(k',s)\gamma^\beta u(k,s)\bar{u}(\ell_-,s_\ell)\gamma_\beta(\slashed{k} - \slashed{k}' - \slashed{\ell}_-)\gamma_\alpha v(\ell_+,s_\ell)J^\alpha_{s_2s_1} }{(Q^2-i0)(t+i0)((q-\ell_-)^2+i0)}\,,
\end{equation}
and, as for BH1, can be split into two terms corresponding to elements of Eq.~\eqref{Jalpha} as:
\begin{equation}
    i\M_{\rm BH2} = \frac{ie^4 \left( i\M^{(1)}_{\rm BH2} + i\M^{(2)}_{\rm BH2} \right) }{(Q^2-i0)(t+i0)((q-\ell_-)^2+i0)}\,.
\end{equation}

The same steps as presented in the previous sections lead to:
\begin{align}
    i\M^{(1)}_{\rm BH2} = & (F_1+F_2)\sum_L\sigma(L) f(s_\ell,\ell_-,L;s,k',k)\Big( Y_{s_2s_1}f(s_\ell,L,\ell_+;+,r'_{s_2},r_{s_1}) + Z_{s_2s_1}f(s_\ell,L,\ell_+;-,r'_{-s_2},r_{-s_1}) \Big) \,,\\
    i\M^{(2)}_{\rm BH2} = & \frac{-F_2}{2M}\Jt_{s_2s_1}\sum_{L, R}\sigma(L) f(s_\ell,\ell_-,L;s,k',k)g(s_\ell,L,R,\ell_+)\,,
\end{align}
where $L\in\{k,k',\ell_-\}$, $R\in\{r_1,r_2,r'_1,r'_2\}$, $\sigma(k) = +1$ and $\sigma(k') = \sigma(\ell_-) = -1$. 

The amplitude for the crossed partner of BH2 is given by:
\begin{equation}
    i\M_{\rm BH2X} = \frac{ie^4 \bar{u}(\ell_-,s_\ell)\gamma_\alpha(\slashed{k}- \slashed{k}' - \slashed{\ell_+})\gamma_\beta v(\ell_+,s_\ell)\bar{u}(k',s)\gamma^\beta u(k,s)J^\alpha_{s_2s_1} }{(Q'^2+i0)(t+i0)((q-\ell_+)^2 + i0)}\,.
\end{equation}
It can be expressed by:
\begin{equation}
    i\M_{\rm BH2X} =  \frac{-ie^4 \left( i\M^{(1)}_{\rm BH2X} + i\M^{(2)}_{\rm BH2X} \right) }{(Q^2-i0)(t+i0)((q-\ell_+)^2 + i0)}\,,
\end{equation}
for which
\begin{align}
    i\M^{(1)}_{\rm BH2X} = & (F_1+F_2)\sum_L\sigma(L)f(s_\ell,L,\ell_+;s, k',k)\Big( Y_{s_2s_1}f(s_\ell, \ell_-,L;+,r'_{s_2},r_{s_1}) + Z_{s_2s_1}f(s_\ell,\ell_-,L;-,r'_{-s_2},r_{-s_1}) \Big) \,,\\
    i\M^{(2)}_{\rm BH2X} = & -\frac{F_2}{2M}\Jt_{s_2s_1}\sum_{L,R}\sigma(L) f(s_\ell,L,\ell_+;s,k',k)g(s_\ell,\ell_-,R,L)\,,
\end{align}
with $L\in\{k,k',\ell_+\}$, $R\in\{r_1,r_2,r'_1,r'_2\}$ and $\sigma(k) = +1, \sigma(k')=\sigma(\ell_+)=-1$.

\subsection{Polarized target case}

Although we are not presenting the results for longitudinally and transversely polarized targets, for completeness we describe in the following how to address such cases within the Kleiss-Stirling approach. Thus far, hadron polarization denoted with index $s_1$ corresponds to the values $\pm$ for helicity with respect to the three-vector component of 
\begin{equation}
    s^\mu = (r_1^\mu - r_2^\mu)/M \,,
\end{equation}
with $M$ the target mass and $r_1,r_2$ two light-like vectors, such that hadron momentum can be written as $p = r_1+r_2$.

With respect to TRF-II axes, $s$ reads:
\begin{equation}
    s^\mu = (0, \hat{z}), \quad \hat{z} = (0, 0, 1)\,.
\end{equation}
Therefore, $\vec{s} = \hat{z}$ is parallel to the outgoing photon three-momentum $\vec{q'}$.
The relation between the quantization of helicity in direction $\hat{z}$ (denoted by $s_1=\pm$) and in another direction defined by three-vector\begin{NoHyper}\footnote{Angles $\phiS$ and $\thetaS$ are the azimuthal and polar orientations of $\vec{S}$ with respect to TRF-II.}\end{NoHyper} $\vec{S} = (\sin\thetaS\cos\phiS,\sin\thetaS\sin\phiS,\cos\thetaS)$ is:
\begin{align}
    | h_1 = + \rangle & = \cos(\thetaS/2)|s_1 = +\rangle + e^{i\phiS}\sin(\thetaS/2)|s_1 = -\rangle \,, \label{generalSpinStates+} \\
    | h_1 = - \rangle & = -e^{-i\phiS}\sin(\thetaS/2)|s_1 = +\rangle + \cos(\thetaS/2)|s_1 = -\rangle \,. \label{generalSpinStates-}
\end{align}

Introducing these spin states in the Compton tensor \eqref{comptonTensor} and the electromagnetic hadron current \eqref{Jalpha_spinors} is equivalent to relate spinors $u'(p,h_1)$ to $u(p,s_1)$ via:
\begin{equation}\label{generalSpinor}
     u'(p, h_1) = F_{h_1 +} u(p,+) + F_{h_1 -} u(p,-)\,,
\end{equation}
where matrix $F$ has been defined in accordance to \eqref{generalSpinStates+} and \eqref{generalSpinStates-} as
\begin{equation}
    F = \begin{pmatrix}
    \cos{\frac{\thetaS}{2}} & e^{i\phiS}\sin{\frac{\thetaS}{2}} \\
    -e^{-i\phiS}\sin{\frac{\thetaS}{2}} & \cos{\frac{\thetaS}{2}}
    \end{pmatrix} = \begin{pmatrix}
    F_{++} & F_{+-}\\
    F_{-+} & F_{--}
    \end{pmatrix}\,.
\end{equation}

Therefore, using (\ref{generalSpinor}) in Eqs.~(\ref{Jcal}), (\ref{J5cal}) and (\ref{Jalpha_spinors}), the correspondence between our current amplitudes, where target is polarized in direction $\hat{z}$ (index $s_1$), and the ones with a target polarized with respect to $\vec{S}$ (index $h_1$) reads:
\begin{equation}
    i\M(s_2,h_1) = F_{h_1+}i\M(s_2,s_1=+) + F_{h_1-}i\M(s_2,s_1=-)\,.
\end{equation}
We can orientate $\vec{S}$ in such a way that we define two types of target polarization: longitudinal ($\vec{S}\parallel\vec{k}$) and transverse  ($\vec{S}\perp\vec{k}$) to the electron beam.
