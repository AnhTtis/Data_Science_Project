\section{Conclusion}
\label{sectConc}

In this article we revisit the phenomenology of DDVCS process, which in the future may become one of the main sources of information on GPDs in the unexplored $\xi \neq x$ domain. The computation of BH and DDVCS amplitudes based on Kleiss-Stirling techniques allows us to express the results in the form of easy to implement complex quantities. The article also includes predictions for a new generation of experiments, including MC studies. 

Our preliminary impact studies provide promising results in terms of DDVCS measurability and encourage further analyses for future experimental facilities. The values of cross-sections integrated in reasonable kinematic ranges accessible to experiments (including $y$-cuts) are: $0.14~\mathrm{pb}$ for JLab12, $0.46~\mathrm{pb}$ for JLab20+, $2.8~\mathrm{pb}$ and $1.5~\mathrm{pb}$ for EIC experiments running with $5~\mathrm{GeV}\times41~\mathrm{GeV}$ and $10~\mathrm{GeV}\times100~\mathrm{GeV}$ beam energy configurations, respectively. We note that our study does not include any detector effects. We stress that a successful DDVCS programme will not only depend on high luminosity accessible to future machines, but also on effective muon reconstruction and possibility of running experiments with various beam and target polarization states. 
Our analysis should be complemented with NLO expressions of DDVCS CFFs \cite{Pire:2011st}. These NLO contributions include amplitudes proportional to gluon GPDs. Since the inclusion of these corrections turn out to be phenomenologically important in both DVCS and TCS cases \cite{Moutarde:2013qs}, we can anticipate that they will be also non-negligible in the DDVCS case. Moreover, they give access to the elusive gluon transversity GPDs \cite{Belitsky:2000jk}. The inclusion of kinematical higher-twist corrections, which has been demonstrated for the DVCS case in Ref.~\cite{Braun:2014sta}, should also be performed for DDVCS. We leave these improvements for a future study.