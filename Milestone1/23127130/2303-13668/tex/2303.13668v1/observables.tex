\section{DDVCS observables}
\label{sectObs}

In this section we present selected DDVCS observables in the kinematics of current and future experiments. We stress the usefulness of this process to constrain various types of GPDs by checking how DDVCS can be used to distinguish between different GPD models giving otherwise similar predictions for DVCS and TCS. For these purposes we use the GK \cite{Goloskokov_2007, Goloskokov_2007_2}, VGG \cite{guichon1998, guichon1999, Goeke_2001, guidal2005} and MMS \cite{Mezrag:2013mya} GPD models implemented in PARTONS. We note that the MMS model only differs from the GK one by the valence part described by either one-component \cite{Radyushkin:2011dh} (MMS) or two-component \cite{Radyushkin:1998es, Radyushkin:1998bz} (GK) double distributions. This difference in the choice of the double distribution makes MMS unique also with respect to VGG, and is responsible for a different behaviour of the model in $x \neq \xi$ domain. The three models are depicted in Fig.~\ref{fig:gpds} for the dominant distribution probed by DDVCS at leading order, $\sum_{q={\{u,d,s\}}} e_{q}^{2}H^{q(+)}(x, \xi, t)$, where $H^{q(+)}(x, \xi, t) = H^{q}(x, \xi, t) - H^{q}(-x, \xi, t)$. We note that all three models are similar in the DGLAP region ($|x| > \xi$), which is a consequence of the common PDF limit and a similar modelling method, but they differ significantly in the ERBL region ($|x| < \xi$). The latter one is directly probed by DDVCS at LO, making this process a convenient tool to distinguish between such various GPD models. 

\begin{figure}[!ht]
    \centering
    \includegraphics[width=0.31\textwidth]{plots/gpd_x.pdf}
    \includegraphics[width=0.31\textwidth]{plots/gpd_01.pdf}
    \includegraphics[width=0.31\textwidth]{plots/gpd_05.pdf}
    \caption{Distributions of $\sum_q e_{q}^{2}H^{q(+)}(x, \xi, t)$ at $t=-0.1~\mathrm{GeV}^{2}$, where $q=u,d,s$ flavours for (left) $\xi = x$, (middle) $\xi = 0.1$ and (right) $\xi = 0.5$. The solid black, dashed red and dotted green curves describe the GK, VGG and MMS GPD models, respectively. The C-even part of a given vector GPD is defined as: $H^{q(+)}(x, \xi, t) = H^{q}(x, \xi, t) - H^{q}(-x, \xi, t)$. The  scale is choosen as $\mu_{F}^{2} = 4~\mathrm{GeV}^{2}$.}
    \label{fig:gpds}
\end{figure}

The selected observables are unpolarized differential cross-sections:
\begin{equation}
    \sigma_{UU}(\phiLBDP) = \int_0^{2\pi} d\phi\int_{\pi/4}^{3\pi/4}d\thetaLBDP\ \sin\thetaLBDP \left( \frac{d^7\sigma^{\rightarrow}}{dx_B dQ^2 dQ'^2 d|t| d\phi d\OmegaLBDP} + \frac{d^7\sigma^{\leftarrow}}{dx_B dQ^2 dQ'^2 d|t| d\phi d\OmegaLBDP} \right)\,,
    \label{eq:ass1}
\end{equation}
and their cosine components:
\begin{equation}
\sigma_{UU}^{\cos(n \phiLBDP)}(\phiLBDP) = M_{UU}^{\cos(n \phiLBDP)}\cos(n \phiLBDP) \,.
\end{equation}
where
\begin{equation}
M_{UU}^{\cos(n \phiLBDP)} = \frac{1}{N}\int_0^{2\pi} d\phiLBDP \cos(n \phiLBDP)  \sigma_{UU}(\phiLBDP)  \,.
\end{equation}
Here, $N=2\pi$ for $n = 0$ and $N=\pi$ for $n > 0$.

We also consider asymmetries for longitudinally polarized electron beam: 
\begin{equation}
    A_{LU}(\phiLBDP) = \frac{\Delta\sigma_{LU}(\phiLBDP)}{\sigma_{UU}(\phiLBDP)}\,,
    \label{eq:ass2}
\end{equation}
where
\begin{equation}
    \Delta\sigma_{LU}(\phiLBDP) = \int_0^{2\pi} d\phi\int_{\pi/4}^{3\pi/4}d\thetaLBDP\ \sin\thetaLBDP \left( \frac{d^7\sigma^{\rightarrow}}{dx_B dQ^2 dQ'^2 d|t| d\phi d\OmegaLBDP} - \frac{d^7\sigma^{\leftarrow}}{dx_B dQ^2 dQ'^2 d|t| d\phi d\OmegaLBDP} \right)\,.
    \label{eq:ass3}
\end{equation}
Here, we omit the dependence on variables other than angles, while right and left arrows stand for positive and negative helicity of the incoming electron beam, respectively. In order to reduce the contribution coming from the pure BH sub-process, the integration over $\thetaLBDP$ angle is performed in the limited range $(\pi/4, 3\pi/4)$. The cross-section difference $\Delta\sigma_{LU}(\phiLBDP)$ is sensitive to $\sin \phiLBDP$ part of the interference, and therefore carries information on the imaginary part of CFFs \cite{Belitsky:2003fj}. If one neglects the $\phiLBDP$-dependence of $\sigma_{UU}(\phiLBDP)$ the same interpretation applies to the asymmetry $A_{LU}(\phiLBDP)$. This asymmetry in the $Q^2 \to 0$ limit can be related to the circular asymmetry in TCS~\cite{CLAS:2021lky, Grocholski_2020}. In the following we do not show results for $A_{LU}(\phi)$, i.e. the asymmetry obtained with $\phiLBDP \leftrightarrow \phi$ replacement in Eqs.~\eqref{eq:ass1}-\eqref{eq:ass3}, as its magnitude in the considered kinematics is much smaller than that for $A_{LU}(\phiLBDP)$ being a consequence of $Q'^2 \gg Q^2$. 

The predictions for JLab12, JLab20+ and EIC experiments are shown in Fig.~\ref{fig:denominators} for unpolarized cross-sections and their cosine components, and in Fig.~\ref{fig:asymetries} for the asymmetries. The plotted quantities are evaluated at kinematics specified in Table~\ref{tab:typicalValuesJLabEIC}. In our study the timelike virtuality, $Q'^2$, has been taken large enough to avoid resonances, as suggested by the recent measurement of TCS by CLAS \cite{CLAS:2021lky}. The selection of spacelike virtuality, $Q^2$, allows one to probe $|\rho|$ significantly smaller than $\xi$, still keeping a reasonable cross-section of DDVCS. Values of the inelasticity variable, $y$, have been taken in accordance to Monte Carlo simulations presented in Sect.~\ref{sectMC} and allow to achieve good statistics of collected events in the corresponding experiment. 

\begin{table}[!ht]
\begin{ruledtabular}
\begin{tabular}{@{}lccccc@{}}
Experiment & Beam energies & $y$ & $|t|$ & $Q^2$ & $Q'^2$ \\ 
& $[\mathrm{GeV}]$ &  & $[\mathrm{GeV^2}]$ & $[\mathrm{GeV^2}]$ & $[\mathrm{GeV^2}]$ \\[5pt] 
JLab12 & $E_{e} = 10.6$, $E_p = M$ & $0.5$ & $0.2$ & $0.6$ & $2.5$\\
JLab20+ & $E_{e} = 22$, $E_p = M$ & $0.3$ & $0.2$ & $0.6$ & $2.5$\\ 
EIC & $E_{e} = 5$, $E_p = 41$ & $0.15$ & $0.1$ & $0.6$ & $2.5$ \\
EIC & $E_{e} = 10$, $E_p = 100$ & $0.15$ & $0.1$ & $0.6$ & $2.5$\\
\end{tabular}
\end{ruledtabular}
\caption{DDVCS kinematics used for predictions of asymmetries presented in Fig.~\ref{fig:asymetries}.}
\label{tab:typicalValuesJLabEIC}
\end{table}

Figure~\ref{fig:denominators} for the unpolarized cross-sections show sizeable contributions of $\cos(n \phiLBDP)$ components for $n=0,1,2$. For the interpretation of these contributions we may use Ref.~\cite{Belitsky:2003fj}. The constant term, $\sigma_{UU}^{1}$, is dominated by BH, with a few percent contribution of pure DDVCS, mostly sensitive to the moduli of CFFs. The term $\sigma_{UU}^{\cos \phiLBDP}$ is particularly interesting, as it is induced by the interference between BH and pure DDVCS, and it carries information about the real parts of CFFs. Finally, $\sigma_{UU}^{\cos2 \phiLBDP}$ is only sensitive to the BH process. This term vanishes for $|\xi| = |\rho|$, i.e. is not observed in both DVCS and TCS limits.

Using Fig.~\ref{fig:asymetries} we conclude that the magnitude of the asymmetry $A_{LU}(\phiLBDP)$ is up to the order $20\%$ for JLab12, $15\%$ for JLab20+ and $3\%$-$7\%$ for EIC. Such sizable asymmetries and fairly large integrated cross-sections presented in Sect.~\ref{sectMC} strongly indicates the feasibility of meaningful DDVCS programmes at all considered facilities.

\begin{figure}[!ht]
    \centering
    \includegraphics[width=0.45\textwidth]{plots/den_12.pdf}
    \includegraphics[width=0.45\textwidth]{plots/den_22.pdf}
    \includegraphics[width=0.45\textwidth]{plots/den_541.pdf}
    \includegraphics[width=0.45\textwidth]{plots/den_10100.pdf}
    \caption{Unpolarized cross-section, $\sigma_{UU}(\phiLBDP)$, and its  $\sigma_{UU}^{\cos\phiLBDP}(\phiLBDP)$ and $\sigma_{UU}^{\cos2\phiLBDP}(\phiLBDP)$ components for beam energies specified in the plots and extra kinematic conditions given in Table~\ref{tab:typicalValuesJLabEIC}. The solid black, dashed red and dotted green curves are for GK, VGG and MMS GPD models, respectively.}
    \label{fig:denominators}
\end{figure}
\begin{figure}[!ht]
    \centering
    \includegraphics[width=0.45\textwidth]{plots/ass_12.pdf}
    \includegraphics[width=0.45\textwidth]{plots/ass_22.pdf}
    \includegraphics[width=0.45\textwidth]{plots/ass_541.pdf}
    \includegraphics[width=0.45\textwidth]{plots/ass_10100.pdf}
    \caption{Asymmetry $A_{LU}(\phiLBDP)$ for beam energies specified in the plots and extra kinematic conditions given in Table~\ref{tab:typicalValuesJLabEIC}. The solid black, dashed red and dotted green curves are for GK, VGG and MMS GPD models, respectively.}
    \label{fig:asymetries}
\end{figure}

The measurability of the $R$ ratio \cite{Berger:2001xd} and the forward-backward asymmetry \cite{CLAS:2021lky} should also be carefully studied. These two observables have been specifically designed in former TCS studies  to take advantage of the charge asymmetry of the final muons to isolate the interference of the Bethe-Heitler and the TCS processes. In the DDVCS reaction, they probe the interference of the BH2 amplitude with the sum of the BH1 and DDVCS amplitudes.