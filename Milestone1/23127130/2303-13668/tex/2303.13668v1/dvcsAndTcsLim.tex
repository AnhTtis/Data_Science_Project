\section{DVCS and TCS limits}\label{sectLimits}

In this section we numerically validate our results against DVCS and TCS limits, which were previously described in literature \cite{BELITSKY2014214,Berger:2001xd} and implemented in PARTONS framework. In these tests we utilize Goloskokov-Kroll GPD model (see for example \cite{Goloskokov_2007, Goloskokov_2007_2}), and the renormalization and factorization scales are $\mu_R^2 = \mu_F^2 = Q^2+Q'^2$. Furthermore, the skewness and generalized Bj{\"o}rken variables are evaluated at $t = 0$. The cross-sections are presented for Trento and BDP angles (see Sect. \ref{sectKin} for details).

The DVCS limit is obtained by taking $Q'^2 \to 0$, which for CFFs gives $\mathcal{F}(\rho, \xi, t)\xrightarrow{Q'^2\rightarrow 0}\mathcal{F}(\xi, \xi, t)$. To make DDVCS and DVCS cross-sections comparable, one must correct the former for the residual $Q'^2$ dependence and the splitting of the outgoing virtual-photon into the lepton pair:
\begin{equation}
    \int d\Omega_\ell \underbrace{\frac{d^7\sigma}{dx_B dQ^2 dQ^{\prime 2}d|t|d\phi d\Omega_\ell}}_{{\rm DDVCS}} \xrightarrow{Q^{\prime 2} \rightarrow 0} \underbrace{\left( \frac{d^4\sigma}{dx_B dQ^2 d|t|d\phi} \right)}_{{\rm DVCS}} \frac{\mathcal{N}}{Q^{\prime 2}} \,,
    \label{eq:limitPrescriptionDVCS}
\end{equation}
where $\mathcal{N} = \alpha_{\rm em}/(3\pi)$ \cite{Vanderhaeghen_1999}. 

The TCS limit, on the other hand, is obtained by taking $Q^2 \to 0$, which for Compton form factors gives $\mathcal{F}(\rho, \xi, t)\xrightarrow{Q^2\rightarrow 0}\mathcal{F}(-\xi, \xi, t)$. In term of cross-sections, we make the integration over $\phi$ angle and we include the photon flux, $\Gamma$, calculated within the equivalent photon approximation (EPA) \cite{kessler_epa, kessler_halArchives}: 
\begin{equation}
    \int d\phi \underbrace{\frac{d^7\sigma}{dx_B dQ^2 dQ^{\prime 2}d|t|d\phi d\Omega_\ell}}_{{\rm DDVCS}} \xrightarrow{Q^{2} \rightarrow 0} \underbrace{\left( \frac{d^4\sigma}{dQ^{\prime 2}d|t|d\Omega_\ell} \right)}_{{\rm TCS}} \frac{d^2\Gamma}{dx_{B}dQ^2} \,,
    \label{eq:limitPrescriptionTCS}
\end{equation}
where
\begin{equation}
    \frac{d^2\Gamma}{dx_{B}dQ^2} = \frac{\alpha_{\rm em}}{2\pi Q^2}\left( 1 + \frac{(1-y)^2}{y} - \frac{2(1-y)Q_{\rm min}^2}{yQ^2} \right)\frac{\nu}{Ex_B} \,.
\end{equation}
Here, 
\begin{equation}
    \nu = \frac{Q^2}{2 M x_B} 
\end{equation}
is the energy of the photon beam, while
\begin{equation}
    Q^2_{\rm min} = \frac{(y m_e)^2}{1-y}
\end{equation}
is the minimum value of the spacelike virtuality evaluated for the electron mass, $m_e$. We note that the prescriptions for both DVCS and TCS limits hold for each sub-process, i.e. BH, pure DDVCS and the interference. 

In Fig.~\ref{fig:limitCFFAsXi} we show how DDVCS CFFs plotted as a function of $\xi$ evolve as $Q^2 \to 0$ and $Q'^2 \to 0$. The curves for limits, $Q^2 = 0$ and $Q'^2 = 0$, are obtained with independent codes for DVCS and TCS processes available in PARTONS. One can conclude that the limits are reached without any discontinuities, hence DDVCS CFFs exhibit the proper reduction to DVCS and TCS counterparts when one of the two virtualities goes to zero. Although the presented quantity is only the imaginary part of CFF $\mathcal{H}$,  figures for real parts and other CFFs (not shown in this manuscript) lead to the same conclusions.
    
\begin{figure}[!ht]
    \centering
    \includegraphics[width=0.45\textwidth]{plots/cff_dvcs_limitXi.pdf}
    \includegraphics[width=0.45\textwidth]{plots/cff_tcs_limitXi.pdf}
    \caption{Imaginary part of DDVCS Compton form factor $\mathcal{H}$ as a function of $\xi$ approaching (left) DVCS and (right) TCS limits. The curves for DVCS and TCS limits, corresponding to $Q'^2 = 0$ and $Q^2 = 0$ values, respectively, are obtained with independent codes. Note that these curves nearly overlap with those for $Q'^2 = 0.001\ \mathrm{GeV}^2$ and $Q^2 = 0.001\ \mathrm{GeV}^2$. The left (right) plot is for $Q^2 = 1.5\ \mathrm{GeV}^2$ ($Q'^2 = 1.5\ \mathrm{GeV}^2$) and $t = -0.15\ \mathrm{GeV}^2$.}
    \label{fig:limitCFFAsXi}
\end{figure}

The comparison for cross-sections is shown in Fig.~\ref{fig:limitCSDDVCS} for pure VCS sub-processes and in Fig.~\ref{fig:limitCSBH} for BH. Also here DVCS and TCS limits are evaluated with independent codes available in PARTONS. These codes are numerical implementations of works published in Refs. \cite{BELITSKY2014214} and \cite{Berger:2001xd}. 

For pure VCS sub-processes shown in Fig.~\ref{fig:limitCSDDVCS} the comparison with the limits is presented for two kinematic configurations, which only differ by either $|t|/Q^2$ or $|t|/Q'^2$ ratios. We see that the relative difference between pure DDVCS and the limits is reduced as these ratios become smaller. This signals that the observed differences stem from kinematic higher-twist corrections, which are related to the choice of the frame used to describe a given process. The effect is expected, as DVCS and TCS are described in fixed target frames where virtual photons move along the $z$-axis. With two virtual photons in DDVCS case the frame must be different, resulting in differences in the twist expansion, see Sect.~\ref{sectAmp} for more details. As for BH we do not deal with this type of expansion, the agreement with DVCS and TCS limits is exact, as demonstrated in Fig. \ref{fig:limitCSBH}. 

\begin{figure}[!ht]
    \centering
    \includegraphics[width=0.45\textwidth]{plots/dvcsLim_VCS_t0p1_Q2is3.pdf}
    \includegraphics[width=0.45\textwidth]{plots/tcsLim_VCS_t0p1_Q2PrimeIs3.pdf}
    \includegraphics[width=0.45\textwidth]{plots/dvcsLim_VCS_t0p25_Q2is40.pdf}
    \includegraphics[width=0.45\textwidth]{plots/tcsLim_VCS_t0p25_Q2PrimeIs33.pdf}
    \caption{Comparison of DDVCS and (left) DVCS and (right) TCS cross-sections for pure VCS sub-process. Corresponding kinematic configurations are specified in the plots (all are for the fixed target). DDVCS cross-sections are modified according to Eqs. \eqref{eq:limitPrescriptionDVCS} and \eqref{eq:limitPrescriptionTCS}. Those for DVCS and TCS are evaluated with independent codes.}
    \label{fig:limitCSDDVCS}
\end{figure}

\begin{figure}[!ht]
    \centering
    \includegraphics[width=0.45\textwidth]{plots/dvcsLim_BH.pdf}
    \includegraphics[width=0.45\textwidth]{plots/tcsLim_BH.pdf}
    \caption{The same as Fig. \ref{fig:limitCSDDVCS} but for BH sub-process.}
    \label{fig:limitCSBH}
\end{figure}