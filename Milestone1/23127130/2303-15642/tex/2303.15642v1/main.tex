% This is samplepaper.tex, a sample chapter demonstrating the
% LLNCS macro package for Springer Computer Science proceedings;
% Version 2.20 of 2017/10/04
%
\documentclass[runningheads]{llncs}
%
\usepackage{graphicx}
% Used for displaying a sample figure. If possible, figure files should
% be included in EPS format.
%
% If you use the hyperref package, please uncomment the following line
% to display URLs in blue roman font according to Springer's eBook style:
% \renewcommand\UrlFont{\color{blue}\rmfamily}
\usepackage{multirow}
\usepackage{hyperref}
\usepackage{amsmath, bm}
\DeclareMathOperator*{\argmax}{arg\,max}

\usepackage{wrapfig}
\usepackage{booktabs} % mid/top/bottom rules for table

\usepackage{times}
% space after command
\usepackage{xspace}
\newcommand{\sine}{SInE\xspace}

\usepackage[nocompress]{cite}


%KK
\usepackage[textsize=scriptsize
%           ,disable
           ]{todonotes}

\newcommand{\KK}[2][]{\todo[color=purple!20,#1]{KK: #2}}

\begin{document}
%
\title{Graph Sequence Learning for Premise Selection}
%
%\titlerunning{Captioning Conjectures}
%\titlerunning{Captioning Conjectures}

% If the paper title is too long for the running head, you can set
% an abbreviated paper title here
%

\author{Edvard K. Holden
\and
Konstantin Korovin
}

\author{Edvard K. Holden
\and
Konstantin Korovin
}
%
\authorrunning{Holden \& Korovin}

\institute{The University of Manchester}

% First names are abbreviated in the running head.
% If there are more than two authors, 'et al.' is used.
%

\maketitle              % typeset the header of the contribution
%
\begin{abstract}

Premise selection is crucial for large theory reasoning as the sheer size of the problems quickly leads to resource starvation.
This paper proposes a premise selection approach inspired by the domain of image captioning, where language models automatically generate a suitable caption for a given image.
Likewise, we attempt to generate the sequence of axioms required to construct the proof of a given problem.
This is achieved by combining a pre-trained graph neural network with a language model.
We evaluated different configurations of our method and experience a 17.7\% improvement gain over the baseline. 



\keywords{Automated Theorem Proving  \and Machine Learning \and Premise Selection \and Sequence Learning \and Graph Neural Network}
\end{abstract}

\section{Introduction}
\section{Introduction}
\label{sec:introduction}
% \begin{itemize}
%     % Diffusion of FL
%     \item {\st{Diffusion of FL}}
%     % Security threats to FL
%     \item {\st{Security threats to FL with particular focus on model poisoning}}
%     % Limitations of existing countermeasures
%     \item {\st{Current countermeasures (e.g., KRUM) and their limitations}}
%     % Proposed method and its advantages
%     \item {\st{Intuitive description of the proposed method and its difference (i.e., advantages) w.r.t. state of the art}}
%     % Main contributions
%     \item {\st{Summary of the main contributions of this work}}
%     % Paper's structure and organization
%     \item {\st{Paper's structure and organization}}
% \end{itemize}

% Diffusion of FL
Recently, {\em federated learning} (FL) has emerged as the leading paradigm for training distributed, large-scale, and privacy-preserving machine learning (ML) systems~\cite{mcmahan2017googleai,mcmahan2017aistats}. 
The core idea of FL is to allow multiple edge clients to collaboratively train a shared, global model without disclosing their local private training data.
%Specifically, an FL system consists of a central server and many edge clients; 
A typical FL round involves the following steps: {\em(i)} the server randomly picks some clients and sends them the current, global model; {\em(ii)} each selected client locally trains its model with its own private data; then, it sends the resulting local model to the server;\footnote{Whenever we refer to global/local model, we mean global/local model {\em parameters}.} {\em(iii)} the server updates the global model by computing an \emph{aggregation function}, usually the average (FedAvg), on the local models received from clients.
% \begin{enumerate}
%     \item[{\em(i)}] the server sends the current, global model to the clients and appoints some of them for training;
%     \item[{\em(ii)}] each selected client locally trains its copy of the global model with its own private data; then, it sends the resulting local model back to the server;\footnote{Whenever we refer to global/local model, we mean global/local model {\em parameters}.}
%     \item[{\em(iii)}] the server updates the global model by computing an \emph{aggregation function} on the local models received from clients (by default, the average, also referred to as FedAvg~\cite{mcmahan2017aistats}).
% \end{enumerate}
This process goes on until the global model converges. %(e.g., after a certain number of rounds or other similar stopping criteria).
%\\
% The advantages of FL over the traditional, centralized learning paradigm are undoubtedly clear in terms of flexibility/scalability (clients can join/disconnect from the FL network dynamically), network communications (only model weights\footnote{We will use \textit{parameters} and \textit{weights} interchangeably.} are exchanged between clients and server), and privacy (each client's private training data is kept local at the client's end and not uploaded to the server).
\\
% Security threats to FL
%However, the growing adoption of FL also raises security concerns~\cite{costa2022covert}, particularly about its confidentiality, integrity, and availability.
Although its advantages over standard ML, FL also raises security concerns~\cite{costa2022covert}. %, particularly about its confidentiality, integrity, and availability~\cite{costa2022covert}.
% OLD, LONG VERSION
% Indeed, some work deals with privacy leakage that may expose the local data of some clients~\cite{melis2019sp}. 
% A large body of work, instead, investigates attacks that usually aim to detriment the predictive accuracy of the learned global model. For instance, \emph{data poisoning} attacks achieve this goal by letting an adversary pollute the training set of some corrupt FL clients with maliciously crafted examples~\cite{jagielski2018sp}.
% Similarly, in \emph{model poisoning} the attacker attempts to tweak the global model weights~\cite{bhagoji2019pmlr} by directly perturbing the local model's weights of some infected FL clients before these are sent to the central server for aggregation, usually via so-called Byzantine attacks. 
% It turns out that Byzantine model poisoning attacks severely impact standard FedAvg; therefore, more robust aggregation functions must be designed to make FL systems secure.
Here, we focus on \emph{untargeted model poisoning} attacks~\cite{bhagoji2019pmlr}, where an adversary attempts to tweak the global model weights %\footnote{We will use the terms \textit{parameters} and \textit{weights} interchangeably.} 
by directly perturbing the local model's parameters of some infected clients before these are sent to the central server for aggregation.
In doing so, the adversary aims to jeopardize the global model \textit{indiscriminately} at inference time.
Such model poisoning attacks severely impact standard FedAvg; therefore, more robust aggregation functions must be designed to secure FL systems.
\\
% In this paper, we focus on designing a novel robust aggregation scheme at the server's end to contrast the effect of Byzantine model poisoning attacks.
%
% Current countermeasures and their limitations
%Several countermeasures have been proposed in the literature to combat model poisoning attacks on FL systems.
% Some methods use simple statistics more robust than plain average to smooth the impact of malicious updates (e.g., Trimmed Mean and FedMedian~\cite{yin2018icml}). 
% Other defenses implement outlier detection techniques to discard malicious updates from the aggregation performed at the server's end. Those are either based on heuristics (e.g., Krum/Multi-Krum~\cite{blanchard2017nips} and Bulyan~\cite{mhamdi2018pmlr}) or data-driven approaches (e.g., K-means clustering~\cite{shen2016acm} or DnC via spectral analysis~\cite{shejwalkar2021ndss}). 
% Finally, some strategies rely on a centralized ``source of trust'' to spot potential malicious updates (e.g., FLTrust~\cite{cao2020fltrust}).
% Several countermeasures have been proposed in the literature to combat model poisoning attacks on FL systems, i.e., to discard possible malicious local updates from the aggregation performed at the server's end. 
% These techniques range from simple statistics more robust than plain average (e.g., Trimmed Mean and FedMedian~\cite{yin2018icml}) to outlier detection heuristics (e.g., Krum/Multi-Krum~\cite{blanchard2017nips} and Bulyan~\cite{mhamdi2018pmlr}) or data-driven approaches (e.g., spectral analysis via K-means clustering~\cite{shen2016acm} or spectral analysis), or methods based on ``source of trust'' (e.g., FLTrust~\cite{cao2020fltrust}).
% OLD, LONG VERSION
%Several countermeasures have been proposed in the literature to combat Byzantine model poisoning attacks on FL systems.
% Descriptive statistics
% For example, Trimmed Mean and FedMedian aggregate local model updates using more robust statistics than standard average~\cite{yin2018icml}.
%
% % Heuristics for outlier detection
% Many existing Byzantine-resilient strategies implement some outlier detection heuristics to discard the model updates sent by potentially malicious clients from the input of the aggregation function.
% One of the most popular heuristics is Krum~\cite{blanchard2017nips}.
% This strategy tries to mitigate the impact of Byzantine attacks by selecting as a global model the local model with the smallest sum of Euclidean distances to {\em all} the other local models.
% Although powerful, Krum requires the server to know (or, at least, estimate) the number of malicious FL clients upfront, which is generally impossible in a realistic attack scenario. %
% Moreover, Krum may become ineffective for complex, high-dimensional model parameter spaces due to the curse of dimensionality.
% Bulyan~\cite{mhamdi2018pmlr} tries to overcome this issue by combining Krum with a variant of Trimmed Mean.
% % Data-driven outlier detection
% Other strategies use data-driven outlier detection techniques -- e.g., via K-means clustering~\cite{shen2016acm} -- to spot potential malicious local model updates. 
% %For instance, Shen et al. propose to cluster local model updates with K-means and thus identify outliers.
%
% % Other techniques
% As far as the server is concerned, any local model received can be from a potential malicious client. 
% FLTrust~\cite{cao2020fltrust} assumes the server acts as a client, i.e., trains a local model on an additional {\em trustworthy} dataset at the server's end and compares it against all the local models from other clients. 
% This way, the server can rely on some ``source of trust'' when discarding potentially malicious clients.
%\\
% Limitations of existing Byzantine-resilient strategies
Unfortunately, existing defense mechanisms either rely on simple heuristics (e.g., Trimmed Mean and FedMedian by~\cite{yin2018icml}) or need strong and unrealistic assumptions to work effectively (e.g., foreknowledge or estimation of the number of malicious clients in the FL system, as for Krum/Multi-Krum~\cite{blanchard2017nips} and Bulyan~\cite{mhamdi2018pmlr}, which, however, cannot exceed a fixed threshold).
Furthermore, outlier detection methods using K-means clustering~\cite{shen2016acm} or spectral analysis like DnC~\cite{shejwalkar2021ndss} do not directly consider the temporal evolution of local model updates received.
Finally, strategies like FLTrust~\cite{cao2020fltrust} require the server to collect its own dataset and act as a proper client, thereby altering the standard FL protocol.
\\
% OLD, LONG VERSION
% Overall, existing Byzantine-resilient strategies are either simple heuristics (e.g., FedMedian) or, if they are more complex, they rely on strong and unrealistic assumptions to work effectively (e.g., knowing the number of malicious clients in the FL system in advance, as for Krum and alike).
% Furthermore, data-driven outlier detection methods do not consider the temporary evolution of local model updates received (e.g., K-means clustering). 
% Finally, strategies like FLTrust requires the server to collect its own dataset and act as a proper client, thereby altering the standard FL protocol.
%
% Description of the proposed method
This work introduces a novel pre-aggregation \textit{filter} robust to untargeted model poisoning attacks. Notably, this filter $(i)$ operates without requiring prior knowledge or constraints on the number of malicious clients and $(ii)$ inherently integrates temporal dependencies. 
The FL server can employ this filter as a preprocessing step before applying \textit{any} aggregation function, be it standard like FedAvg or robust like Krum or Bulyan.
Specifically, we formulate the problem of identifying corrupted updates as a multidimensional (i.e., matrix-valued) time series anomaly detection task. 
The key idea is that legitimate local updates, resulting from well-calibrated iterative procedures like stochastic gradient descent (SGD) with an appropriate learning rate, show \textit{higher predictability} compared to malicious updates. This hypothesis stems from the fact that the sequence of gradients (thus, model parameters) observed during legitimate training exhibit regular patterns, as validated in Section~\ref{subsec:intuition}. %until convergence. 
%This regularity may be more pronounced for smooth convex loss functions, but it can still be captured within an appropriate time window, even for more complex and convoluted loss surfaces. 
%We provide evidence of this claim in Appendix~B, where we show that the average mutual information (i.e., ``predictability''), calculated over pairs of legitimate model updates sent at different FL rounds, is significantly higher than the corresponding computation for a malicious client.
\\
Inspired by the matrix autoregressive (MAR) framework for multidimensional time series forecasting~\cite{chen2021je}, we propose the FLANDERS ({\em \textbf{F}ederated \textbf{L}earning meets \textbf{AN}omaly \textbf{DE}tection for a \textbf{R}obust and \textbf{S}ecure}) filter.
The main advantages of FLANDERS over existing strategies like FLDetector~\cite{zhao2020multivariate} are its resilience to large-scale attacks, where $50\%$ or more FL participants are hostile, and the capability of working under realistic non-iid scenarios.
We attribute such a capability to two key factors: $(i)$ FLANDERS works without knowing a priori the ratio of corrupted clients, and $(ii)$ it embodies temporal dependencies between intra- and inter-client updates, quickly recognizing local model drifts caused by evil players. Below, we summarize our main contributions:

\begin{itemize}
\item[{\em(i)}]
We provide empirical evidence that the sequence of models sent by legitimate clients is more predictable than those of malicious participants performing untargeted model poisoning attacks.
\\
\item[{\em(ii)}] 
We introduce FLANDERS, the first pre-aggregation filter for FL robust to untargeted model poisoning based on multidimensional time series anomaly detection.
\\
\item[{\em(iii)}] 
We integrate FLANDERS into Flower,\footnote{\scriptsize{\url{https://flower.dev/}}} a popular FL simulation framework for reproducibility.
\\
\item[{\em(iv)}] 
We show that FLANDERS improves the robustness of the existing aggregation methods under multiple settings: different datasets, client's data distribution (non-iid), models, and attack scenarios.
\\
\item[{\em(v)}] 
We publicly release all the implementation code of FLANDERS along with our experiments.\footnote{\scriptsize{\url{https://anonymous.4open.science/r/flanders_exp-7EEB}}}
\end{itemize}

% Paper's structure and organization
The remainder of the paper is structured as follows. %some related work and the current state-of-the-art solutions to security issues that FL entails. 
Section~\ref{sec:background} covers background and preliminaries. 
In Section~\ref{sec:related}, we discuss related work.
Section~\ref{sec:problem} and Section~\ref{sec:method} describe the problem formulation and the method proposed. % to tackle it. 
Section~\ref{sec:experiments} gathers experimental results. %, and Section~\ref{sec:limitations} discusses some limitations of this work.
Finally, we conclude in Section~\ref{sec:conclusion}.
 %discusses the limitations of this work and draws future research directions.
%reports conclusions and draws perspectives for future research directions.

%%%%%%% OLD %%%%%%%
%to overcome the resilience of Byzantine failures in distributed Stochastic Gradient Descent computations. 
% The strength of Krum is its time complexity, which is linear in the gradient dimension. 
% However, the robustness of the approach is guaranteed for gradient-based learning applications only when the majority of the clients are not compromised. 
% Besides, the aggregation mechanism of Krum, as well as that of similar methods, is robust from a coarse-grained perspective and does not provide solutions to errors and perturbations that may occur at inference time.
%A related approach to~\cite{blanchard2017nips} is the work of Su et al.~\cite{su2016dc}. Here, the authors propose an iterated approximate agreement to tackle a multi-layer scenario attacked by Byzantine agents. 
%However, the method works efficiently on the sole discrete context and it is inapplicable to continuous state environments.
%\gabri{Maybe, we should just talk about the main limitations of existing countermeasures without digging into their details (or, we can just mention Krum as this is the most popular one). I will move the description of all these methods to the Related Work section.}


% % % % % % % % % % % % % % % % % % % % % % % % % % % % % % % % % % % % % % % % % % % % % % % % 
\section{Related Works}\label{sec:related_works}
\setlength{\tabcolsep}{1.6mm}{
\renewcommand\arraystretch{1.1}
\begin{table}[ht]
  \centering
  \scalebox{0.9}{
  \begin{tabular}{llcccc}
    \toprule
    &\multirow{2}*{Methods} & \multirow{2}*{Sal.} &   \multicolumn{2}{c}{VOC} & MS~COCO \\
    \cmidrule(r){4-5}\cmidrule(r){6-6}
    &&&\texttt{val}&\texttt{test}&\texttt{val}\\
    \hline
    \multirow{13}*{\rotatebox{90}{ResNet-50}}
    &IRN~\cite{irn}          \tiny{CVPR'19}     &              & 63.5       & 64.8          & 42.0  \\
    &LayerCAM~\cite{layercam}\tiny{TIP'21}      &              & 63.0       & 64.5          & -     \\
    &AdvCAM~\cite{advcam}    \tiny{CVPR'21}     &              & 68.1       & 68.0          & 44.2  \\
    &RIB~\cite{rib}          \tiny{NeurIPS'21}  &              & 68.3       & 68.6          & 44.2  \\
    &ReCAM~\cite{recam}      \tiny{CVPR'22}     &              & 68.5       & 68.4          & 42.9  \\
    % \rowcolor{Gray}
    &\cellcolor{Gray}IRN+\texttt{LPCAM}    &\cellcolor{Gray} & \cellcolor{Gray}68.6    & \cellcolor{Gray}68.7      & \cellcolor{Gray}44.5  \\
    &SIPE~\cite{sipe}        \tiny{CVPR'22}     &              & 68.8       & 69.7          & 40.6  \\
    &OOD~\cite{ood}+Adv      \tiny{CVPR'22}     &              & 69.8       & 69.9          & -     \\
    &AMN~\cite{amn}          \tiny{CVPR'22}     &              & 69.5       & 69.6          & 44.7  \\
    &\cellcolor{Gray}AMN+\texttt{LPCAM}    &\cellcolor{Gray} & \cellcolor{Gray}70.1    &\cellcolor{Gray} 70.4      & \cellcolor{Gray}45.5  \\ 
    &ESOL~\cite{esol}        \tiny{NeurIPS'22}  &              & 69.9$^*$   & 69.3$^*$      & 42.6  \\
    &CLIMS~\cite{clims}      \tiny{CVPR'22}     &              & 70.4$^*$   & 70.0$^*$      & -     \\
    &EDAM~\cite{edam}        \tiny{CVPR'21}     &\checkmark    & 70.9$^*$   & 71.8$^*$      & -     \\
    &\cellcolor{Gray}EDAM+\texttt{LPCAM}  &\cellcolor{Gray}\checkmark & \cellcolor{Gray}71.8$^*$ &\cellcolor{Gray} 72.1$^*$& \cellcolor{Gray}42.1\\
    \hline
    \multirow{9}*{\rotatebox{90}{WideResNet-38}}
    &Spatial-BCE~\cite{sbce} \tiny{ECCV'22}     &              & 70.0       & 71.3      & 35.2  \\
    &BDM~\cite{bdm}          \tiny{ACMMM'22}    &\checkmark    & 71.0       & 71.0      & 36.7  \\ 
    &RCA~\cite{rca}+OOA      \tiny{CVPR'22}     &\checkmark    & 71.1       & 71.6      & 35.7  \\
    &RCA~\cite{rca}+EPS      \tiny{CVPR'22}     &\checkmark    & 72.2       & 72.8      & 36.8  \\
    &HGNN~\cite{hgnn}        \tiny{ACMMM'22}    &\checkmark         & 70.5$^*$   & 71.0$^*$  & 34.5  \\ 
    &EPS~\cite{eps}          \tiny{CVPR'21}     &\checkmark         & 70.9$^*$   & 70.8$^*$  & -     \\
    &RPIM~\cite{rpim}        \tiny{ACMMM'22}    &\checkmark         & 71.4$^*$   & 71.4$^*$  & -     \\ 
    &L2G~\cite{l2g}          \tiny{CVPR'22}     &\checkmark         & 72.1$^*$   & 71.7$^*$  & 44.2  \\
    \hline
    \multirow{2}*{\rotatebox{90}{\small{DeiT-S}}}
    &MCTformer~\cite{mctformer}    \tiny{CVPR'22}     &                 & 71.9$^{\dag}$  & 71.6$^{\dag}$   & 42.0  \\
    &\cellcolor{Gray}MCTformer+\texttt{LPCAM}      &\cellcolor{Gray} & \cellcolor{Gray}72.6$^{\dag}$  & \cellcolor{Gray}72.4$^{\dag}$  &\cellcolor{Gray} 42.8 \\
    \bottomrule
  \end{tabular}}
  \vspace{-2mm}
  \caption{The mIoU results (\%) based on DeepLabV2 on VOC and MS~COCO. The side column shows three backbones of multi-label classification model. ``Sal.'' denotes using saliency maps. * denotes the segmentation model is pre-trained on MS~COCO. $^\dag$ denotes the segmentation model is pre-trained on VOC.
  }
  \vspace{-6mm}
  \label{table_related}
\end{table}
}



% % % % % % % % % % % % % % % % % % % % % % % % % % % % % % % % % % % % % % % % % % % % % % % % 




% % % % % % % % % % % % % % % % % % % % % % % % % % % % % % % % % % % % % % % % % % % % % % % % 
% METHOD section
\section{Axiom Captioning}\label{sec:captioning}



The image captioning problem can be stated as follows: given an image $I$ and a dictionary of words $\Omega$, generate an accurate and grammatical caption $S$, consisting of words from $\Omega$.
This challenging problem goes beyond the already non-trivial task of identifying the image objects.
Rather, it requires identifying and comprehending: the objects, their attributes and their relation.
Moreover, this information must be decoded and represented as a grammatically correct sentence in the target language.


State-of-the-art image captioning approaches join the machine learning fields of image classification and language modelling.
An example of a captioning model based on the inject architecture is shown in Figure  \ref{fig:captioning_model}.
It consists of three components: an image encoder, a language model, and a dense output layer.
The image encoder extracts and embeds the image semantics as a feature vector.
The language model combines these salient features with an input word to produce an encoding of the current sequence.
Finally, the dense layer maps the encoding to a probability distribution over the vocabulary.


\begin{figure}
% https://docs.google.com/drawings/d/1emuPMr3dvgGAUhy2OM9iaIBYKOiIRq9SfijR7u7hyiw/edit
\centering
\includegraphics[scale=0.38]{diagrams/captioning_model.png} 
\caption{The inject architecture for image captioning.}
\label{fig:captioning_model}
\end{figure}


Despite the challenges of image captioning, the models produce appropriate and detailed image descriptions.
Due to their expressiveness, we believe these methods can be utilised for premise selection.
In the remaining parts of this section, we describe the sequence model.
%The embedding model is detailed in Section \ref{sec:problem_embedding}.





% % % % % % % % % % % % % % % % % % % % % % % % % % % % % % % % % % % % % % % % % % % % % % % % % % % % % % % % % % % % 


\subsection{Sequence Learning}

In the original task of image captioning, the model operates on pairs of images and captions in a target language.
In the context of premise selection, the images are replaced by problems and the captions are replaced with the axioms that appear in the proof of the problems.
Assume we have a problem $I$ with a corresponding proof $S^*$ and an axiom resource bank $\Omega$.
Next, we extract and impose an order on the $m$ axioms in $S^*$, resulting in $S =\langle s_1, \ldots , s_m \rangle$, $s_i \in \Omega$ for $1\leq i \leq m$. 
 %resulting in $S = \{s_1, \ldots , s_t \in \Omega^t \} $.
We describe the task of premise selection in the context of sequence learning as maximising the probability of producing the sequence of axioms used in the proof of a given problem.
Given the problem-axioms pair $(I, S)$ we can compute its log probability as:


$$
\log p (S | I) = \sum_{t=1}^m \log p(s_t | s_{t-1}, \ldots, s_1, I).
$$



We estimate $\log p(s_t|s_{t-1},\cdots, s_1, I)$ with the recurrent neural network (RNN) $\sigma$ with learnable parameters $\theta$.
RNNs exhibit a dynamic behaviour over a sequence of inputs due to their internal memory state $h_t$, which captures the previous inputs sequentially.
In particular, the output at step $t$ depends on the previous memory state $h_{t-1}$ and the input $s_{t-1}$.
The hidden state is defined as:

\[
% Should we start from t=1 and include s_0 as it is the start token?
h_t = 
\left\{
\begin{array}{r@{\quad}l}
     \sigma(I;\theta) & \text{if } t = 1,\\
    \sigma(h_{t-1}, s_{t-1};\theta)  & \text{otherwise}.\\
\end{array} \right.
\]

The RNN is trained to predict the next token in a sequence based on the previous token and the current memory state.
Over a training set of problem-axiom pairs  $\{(I^i, S^i)\}^N_{i=1}$, the model is trained to maximise the log probability of producing the correct sequence of axioms:
% https://arxiv.org/pdf/1609.06647v1.pdf 
$$
\theta^* = \argmax_\theta \sum_{I,S} \log p(S|I;\theta).
$$

Thus, the model predicts axioms based on the problem and the previously predicted axioms.
In our implementation, we use Long-Short Term Memory (LSTM)~\cite{lstm} cells as the underlying RNN.
LSTM is among the most popular RNN models due to its robustness towards vanishing and exploding gradients.





% % % % % % % % % % % % % % % % % % % % % % % % % % % % % % % % % % % % % % % % % % % % % % % % % % % % % % % % % % % % 

\subsection{Axiom Captioning}



% https://arxiv.org/pdf/1703.09137.pdf
The generative axiom prediction model is constructed using the par-inject architecture \cite{DBLP:journals/corr/TantiGC17}, as illustrated in Figure \ref{fig:rnn_sequence}.
This architecture takes a token embedding $\textbf{s}$ and a problem embedding $\textbf{I}$ at each time step.
The model is given the special start token $s_{start}$ to initialise the axiom generation process.
Likewise, a special end token, $s_{end}$, represents the end of a sequence. 
Consequently, start and end tokens are added to each axiom sequence such that the model is trained on the target sequence  $\langle s_{start} , s_1, \ldots , s_m, s_{end} \rangle$.
Axioms with few occurrences in the dataset are replaced by the Out-Of-Vocabulary token $s_{unkown}$.
These three special tokens are included in the dictionary $\Omega$.




%%%%%%%%%%%%%%%%%%%%%%%%%%%%%%%%%%%%%%%%%%%%%%%%%%%%%%%%%%%%%%%%%%%%%%%%%%%%%%%%%%%5
%%%%%%%%%%%%%%%%%%%%%%%%%%%%%%%%%%%%%%%%%%%%%%%%%%%%%%%%%%%%%%%%%%%%%%%%%%%%%%%%%%%5


\begin{figure}[!htpb]
% TODO: make dense fo the other way?
% https://docs.google.com/drawings/d/1wKfGzKwfxPLpEPpUQPa3VjB_9eKDOIEU6TEq8eZafhU/edit - General Overview
% https://docs.google.com/drawings/d/1QWBVa-MxMCS2uvMNYijSFhlXOkZsaNIPSS--Af1x1js/edit
\centering
%\includegraphics[scale=0.38]{diagrams/rnn_sequence.png}
\includegraphics[width=0.95\linewidth]{diagrams/rnn_sequence_hidden.png} 
\caption{Recurrent Neural Network predicting the next token in a sequence.}
\label{fig:rnn_sequence}
\end{figure}




%%%%%%%%%%%%%%%%%%%%%%%%%%%%%%%%%%%%%%%%%%%%%%%%%%%%%%%%%%%%


At training time, we apply teacher forcing., which feeds the next token of the training sequence to the model instead of its previous prediction.
This prevents the model from being unable to recover from poor predictions.
Hence, the prediction at each training step is expressed as:


\[
\hat{y_t} = 
\left\{
\begin{array}{r@{\quad}l}
      \sigma( \textbf{s}_{start}, \textbf{h}_{0}, \textbf{I} ; \theta ) & \text{if } t = 1,\\
    \sigma(\textbf{s}_{t-1}, \textbf{h}_{t-1}, \textbf{I} ; \theta)  & \text{otherwise.}\\
\end{array} \right.
\]

Where $\hat{y_t}$ is a probability distribution at time $t$ over the axioms in $\Omega$.






% % % % % % % % % % % % % % % % % % % % % % % % % % % % % % % % % % % % % % % % % % % % % % % % % % % % % % % % % % % % 

\subsection{Neural Attention Captioning} \label{sec:attention}




The captioning decoder is fed a static input entity at each time step, but it can be advantageous to emphasise different parts of the embedding based on the current model state~\cite{https://doi.org/10.48550/arxiv.1502.03044}.
This is achieved through a separate attention network that dynamically weighs some input according to a model state.
In other settings, the attention mechanism can emphasise particular words in a sequence or regions of an image with respect to the model state.
In this scenario, the incentive of attention is to emphasise particular sections and elements of the averaged graph representation to enhance the embedding.




The attention mechanism computes a context vector which is used as input to the next stage of the model.
It is a weighted sum of the $n$ embedding elements where each weight is the quantity of attention applied to the corresponding element:



\[
\bm{c}_t = \sum_{i=1}^n \alpha_{t,i} \bm{I}_i.
\]


The weights $\alpha_{t,i}$ are computed based on an alignment score function which measures how well each element matches the current state.
The scores are scaled by softmax into weights in the range of $[0, 1]$ where the sum of the weights equals to 1:
\begin{wrapfigure}[14]{}{0.35\textwidth}
    % https://docs.google.com/drawings/d/1eh82u7T8aDP7xJfElDQtM6h3W1nNxy-wxhJ-VI9aWnw/edit
\includegraphics[width=0.98\linewidth]{diagrams/attention.png} 
%\caption{Attention Neural Network}
\caption{Language model with Bahdanau attention.}
\label{fig:attention}
\end{wrapfigure}


\[
\alpha_{t,i} = 
\frac{
\exp(score(\bm{I}_i, \bm{h}_t))
}
{
\sum_{j=1}^n \exp(score(\bm{I}_j, \bm{h}_t))
}.
\]


In Section~\ref{exp:attention}, we experimented with both Loung attention~\cite{DBLP:journals/corr/LuongPM15} and Bahdanau attention~\cite{https://doi.org/10.48550/arxiv.1409.0473}.
The alignment function of all three attention types is shown in Table~\ref{tab:attention_alignment_functions}, where 
$W, W_1, W_2$ and $V$ are learnable attention parameters.
In the Bahdanau style, the context vector is concatenated with the token embedding and fed to the RNN decoder, as illustrated by Figure~\ref{fig:attention}.
This is a key difference from the Loung style, where the alignment scores are computed on the output of the RNN prior to the dense layer.




\begin{table}[!htpb]
% https://lilianweng.github.io/posts/2018-06-24-attention/
    \centering
    \begin{tabular}{@{}l c l l @{}} 
         Attention Style & &  & Alignment  \\ \toprule
         Bahdanau       & &  score($\bm{I_i}, \bm{h}_{t-1}$)  & = $V^\top \cdot tanh( W_1 \cdot \textbf{I}_i + W_2 \cdot \textbf{h}_{t-1}) $    \\ 
         Loung Dot      & &  score($\bm{I_i}, \bm{h}_{t}$)    & = $\bm{h}_{t}^\top \cdot \bm{I}_i  $ \\
         Loung Concat   & &  score($\bm{I_i}, \bm{h}_{t}$)    & = $V^\top \cdot tanh( W[\textbf{I}_i;\textbf{h}_{t}]) $   \\ \bottomrule
    \end{tabular}
    \caption{Overview of attention alignment functions.}
    \label{tab:attention_alignment_functions}
\end{table}




% Loung attention follows similar principles but the key difference is here it is implemented on the output of the RNN prior to the dense output layer.
% The alignment function of all three attention types is shown in Table~\ref{tab:attention_alignment_functions}.





% % % % % % % % % % % % % % % % % % % % % % % % % % % % % % % % % % % % % % % % % % % % % % % % 


%\section{Graph Neural Network Embeddings}
\section{Problem Embeddings}\label{sec:problem_embedding}\label{sec:embedding}




An embedding is a fixed-size, real-valued vector representation of an entity, where semantically similar entities ideally are close in the embedding space.
%The embedding function maps entries from an initial feature space to the more favoured embedding space.
In the original task of image captioning, image embeddings consist of low-level image features obtained from pre-trained convolutional neural networks over extensive image classification datasets.
%These salient features can be extracted as powerful embeddings for other downstream image-based learning tasks.


Computing problem embeddings in a similar fashion pose multiple challenges in the context of first-order problems.
Firstly, the problems have no natural fixed-size vector representation as they consist of unordered sets of tree-structured formulae.
Thus, encoding syntactic, structural and semantic properties as a vector is non-trivial.
Secondly, there is no immediate classification task for learning the semantics of first-order problems.
%This is in stark contrast to digital images, which have a natural vector representation and where identifying objects in images is a well-researched problem.
This paper attempts to overcome these challenges by producing embeddings via graph neural networks.






% % % % % % % % % % % % % % % % % % % % % % % % % % % % % % % % % % % % % % % % % % % % % % % % % % % % % % % % % % % %
\subsection{Problem Graph}


A first-order logic formula has an intrinsic tree-shaped structure and is naturally represented as a directed acyclic graph $G$ with vertices $V$ and edges $E$.
The vertices, also known as nodes, correspond to the types of elements occurring in the formula, such as function symbols and constants.
The edges denote a relationship between the vertices, e.g., an argument supplied to a function.
Figure~\ref{fig:graph_conjecture} illustrates the graph representation of a conjecture, spanning four different node types as visually represented by the colouring. 

This representation extends to sets of formulas by computing a global graph over the node elements in the formulae, as shown in Figure \ref{fig:graph_problem}.
The graph representation captures many aspects of the formulae while invariant to symbol renaming and encoding problems with previously unseen symbols.
This paper uses a graph encoding of 17 node types as described in~\cite{DBLP:conf/cade/RawsonR20}.


% I can only find 15 node types in the code??
% argument order in function and predicate application must be preserved in order to
% maintain a lossless representation. This is achieved by use of an auxiliary “argument node”
% for each argument in an application, connected by edges indicating the order of arguments, shown in Figure 3


\begin{figure}[!htpb]
    \centering
    \begin{minipage}{0.45\linewidth}
        \centering
        \includegraphics[width=0.95\linewidth]{diagrams/graph_conjecture.png}
        %\caption{Graph representation of the conjecture {\tt  fof(t6\_numbers, conjecture, r2\_xboole\_0(k4\_numbers, k2\_numbers))}}
        \caption{A conjecture graph.}

        \label{fig:graph_conjecture}
    \end{minipage}\hfill
    \begin{minipage}{0.45\linewidth}
        \centering
        \includegraphics[width=0.95\linewidth]{diagrams/graph_problem.png} 
        %\caption{Graph representation of problem {\tt t6\_numbers}, which consists of one conjecture and six axioms.}
        \caption{A problem graph of one conjecture and six axioms.}
        \label{fig:graph_problem}
    \end{minipage}
\end{figure}








%%%%%%%%%%%%%%%%%%%%%%%%%%%%%%%%%%%%%%%%%%%%%%%%%%%%%%%%%%%%%%%%%%%%%%%%%%%%%%%%%%%%%%%%%%%%%%%%%%%%%%%%%%%%%%%%%%%


\subsection{Graph Neural Networks}


The problem graph is embedded into an $n$-dimensional embedding space via a graph neural network.
A graph neural network is an optimisable transformation that operates on the attributes of a graph.
It utilises a ``graph-in, graph-out" methodology where it embeds the graph while preserving the structure and connectivity of the original graph.

A randomly initialised vector represents each node type $\Phi$ across all graphs in an $n$-dimensional embedding space.
Next, each node in a graph is assigned to its corresponding embedding vector $\bm{x}_\Phi$, resulting in the node feature matrix $X$.
The GNN embeds the type features of each node $\bm{x}_\Phi$ into the node feature embedding $\bm{x}_\Phi'$ through a node update function.
This effectively transforms the graph features $X$ into a more favourable embedding  $X'$.
Adjacent nodes are incorporated into the update of a node to encode the structure through message passing~\cite{DBLP:journals/corr/GilmerSRVD17}.


Message passing is accomplished through graph convolutional layers, and we utilise the operation described in~\cite{https://doi.org/10.48550/arxiv.1609.02907}.
The node-wise convolutional operation for the attributes $\bm{x}_i^{(k)}$ of node $i$ at step $k$ is described as:

% https://pytorch-geometric.readthedocs.io/en/latest/generated/torch_geometric.nn.conv.GCNConv.html#torch_geometric.nn.conv.GCNConv
\[ 
\bm{x}_i^{(k)} = W
\sum_{j \in \mathcal{N}(i) \bigcup \{i\}}
\frac{e_{j,i}}
{\sqrt{ \hat{d_j} \hat{d_i} } }
\bm{x}_j^{k-1}
%\cdot (\bm{W}^\top  \bm{x}_j^{k-1})
%+ \bm{b}
\]	

% https://pytorch-geometric.readthedocs.io/en/latest/generated/torch_geometric.nn.conv.GCNConv.html#torch_geometric.nn.conv.GCNConv
where $W$ is a learnable weight matrix, $\mathcal{N}(i)$ is the set of neighbouring nodes of $i$ and $\hat{d_i} = 1 + \sum_{j \in \mathcal{N}(i)}e_{j,i}$. 
The variable $e_{j,i}$ denotes the edge weight from $j$ to $i$.
 In this setting, all edge weights are 1. % and the layers scale linearly in the number of graph edges.
The convolutional operations are applied synchronously to all nodes in the graph and learn hidden layer representations that encode both local graph structure and nodes features.
% As a result, we obtain more semantically meaningful node features while retaining the graph structure.


%The convolutional operation for the attributes $x_i^{(k)}$ of node $i$ at step $k$ is described as;
% https://pytorch-geometric.readthedocs.io/en/latest/notes/create_gnn.html - just general message passing
% update with the formula we are using.
%\[ 
%\bm{x}_i^{(k)} = 
%\sum_{j \in \mathcal{N}(i) \bigcup \{i\}}
%\frac{1}{\sqrt{deg(i)} \cdot \sqrt{deg(j)} }
%\cdot (\bm{W}^\top  \bm{x}_j^{k-1})
%+ \bm{b}
%\]	




%The operation extracts the node $i$ and all its neighbours and performs a linear transformation on their attributes.
%Over a node $i$ and its adjacent node $j$, the transformation multiples the node attributes $\bm{x}_j^{k-1}$ with the weights $\bm{W}$ and scales the result according to the degree of the node pair.
%Further, the scaled attributes are aggregated with a sum function.
%Lastly, it applies the bias vector $\bm{b}$ before updating the attributes of node $i$.


%and is achieved through graph convolutional operators on the adjacency matrix $A$.


% https://docs.google.com/drawings/d/1O4xTRboNRjzMkgcBnc_9eddkvlg5l9FaApT0eO1DnlA/edit
%\begin{figure}[!htpb]
\begin{figure}
\centering
\includegraphics[scale=0.38]{diagrams/gnn_pipeline.png} 
\caption{Graph Neural Network for classification of graph or node properties.}
\label{fig:graph_workflow}
\end{figure}


After computing the graph embeddings, they are pooled and passed through the prediction layer, which produces the final model output.
We experiment with three different mean pooling approaches all nodes in the graph, only axiom nodes, and only the conjecture node.
An overview of the GNN pipeline is shown in Figure~\ref{fig:graph_workflow}.

In this approach, the GNN is pre-trained on auxiliary tasks, computing the embeddings before training the captioning model.
We experiment with supervised and unsupervised pre-training GNN approaches, as described below.




\subsection{Supervised Problem Embedding} 


In the supervised approach, the GNN is trained on the node level by performing binary premise selection over the axiom nodes, as described in~\cite{DBLP:conf/cade/RawsonR20}.
Based on their node embedding, the model learns to predict whether an axiom occurs in the proof of a problem.
%The model operates on a directional graph but performs message passing with bi-directional convolutional layers.
%This means that the node embeddings are a concatenation of one convolutional layer operating on the original adjacency matrix $A$ and another convolutional layer operating on the transposed adjacency matrix $A^T$.
%As a result, message passing occurs in both edge directions. 
During training, the resulting axiom node embeddings become increasingly valuable for modelling their contribution towards the proof.
Therefore, the node embeddings are expected to contain information crucial to premise selection.
Our experiments show that this information prevails through average pooling.

% However, whether this information will prevail when the nodes are pooled and injected into the captioning model is unclear.\KK{elaborate, do results form this paper clarify }



% % % % % % % % % % % % % % % % % % % % % % % % % % % % % % % % % % % % % % % % % % % % % % % % % % % % % % % % % % % %



\subsection{Unsupervised Problem Embedding}

The supervised learning task emphasises the axiom nodes, but it might be advantageous with a learning task encapsulating all the nodes in a graph.
Alas, no sensible labels are directly derivable from the problems to train a prediction model.
Thus, we employ unsupervised training through a synthetic dataset which utilises a relation property encapsulating all graph nodes.


The unsupervised training approach consists of training a matching model which learns the difference between two graphs according to some relational property, as described in~\cite{unsupervised_graph_classification}.
The model takes two graphs, $g_i$, $g_j$, as input and passes them through the Siamese GNN model, as illustrated in Figure \ref{fig:embedding_unsupervised}.
Next, the nodes of the embedded graphs are pooled into two graph embedding vectors.
The similarity of the two input graphs is approximated as the vector norm between the two graph embeddings:  $ || GNN(g_i) - GNN(g_j) || $. 
Training the GNN in this fashion enables it to produce embeddings encompassing structural similarities and dissimilarities.


% https://docs.google.com/drawings/d/1SN44rps-Ainpq6kzthcVkL0Xje2B75rge_1bWOk761k/edit
% https://docs.google.com/drawings/d/1DRFLgQJ-wf6uWvnD3OTRqU7tfMdhti7OQxk3KS7uKGk/edit
\begin{figure}[!htpb]
\centering
% \includegraphics[scale=0.45]{diagrams/GNN_unsupervised.png} 
\includegraphics[scale=0.37]{diagrams/gnn_unsupervised_detailed.png} 
\caption{Unsupervised GNN training based on pairwise graph similarity.}
\label{fig:embedding_unsupervised}
\end{figure}



The synthetic dataset consists of pairs of undirected graphs and a numeric property describing their relation.
The relational property utilised is the Laplacian spectrum distance~\cite{Wills_2020}, which can be defined as follows.
Given a  graph $G$, the adjacency matrix $A$ represents the node connections in the graph.
The diagonal degree matrix $D$ of $G$ represents the degree of each node, e.g. the number of neighbours.
Further, the Laplacian of the graph is defined as the degree matrix subtracted from the adjacency matrix:


\[
L = D - A
\]

\noindent
The eigenvalues $\lambda_1 \leq \ldots \lambda_i \ldots \leq \lambda_k$ of the Laplacian are given as $L \bm{x} = \lambda_i \bm{x}$.
Accordingly, the Laplacian spectrum distance $\pi$ of two graphs $G$ and $G'$, is defined as:

%\[
%\pi (G, G') = 
%\underset{k \in min(n, m)}{}
%\lVert \lambda_0 - \lambda_0' , \ldots \lambda_k - \lambda_k' \rVert
%\]

% Frobenius norm - np.linalg.norm
\[
\pi (G, G') = 
\sqrt{
\sum_{i=1}^{k} (\lambda_i - \lambda_i')^2
},
\]
%
where $k \in min(n, m)$, and $n$ and $m$ are the numbers of nodes in $G$ and $G'$.
 
The Laplacian spectrum distance is a computationally cheap metric, even for graphs of the magnitude required to represent first-order problems.
Although the metric encapsulates graph structure, it neither considers node types nor edge directions. % which could lead to the loss of valuable problem semantics.
Still, the distance provides an overall description of the structural similarity of the graphs and considers all graph nodes. 
%\KK{there are distances for weighted graphs which probably can represent labels}








% % % % % % % % % % % % % % % % % % % % % % % % % % % % % % % % % % % % % % % % % % % % % % % % 

\section{Experimental Evaluation}\label{sec:evaluation}

% Whether to use the 10K or the 6K results (automatic for tables, nothing else. Should it be?)
%\newcommand{\vocabsize}{vocab_10k}
\newcommand{\vocabsize}{vocab_6k}


We present in section~\ref{ssec:faces} an application of PnP-HVAE on face images, using a pretrained state-of-the-art hierarchical VAE. 
Next, we study the application of our framework to natural images. To that end, we introduce  in section~\ref{ssec:patchVDVAE}  a patch hierachical VAE architecture, that is able to model natural images of different resolutions. In section~\ref{ssec:app_nat}, we provide deblurring, super-resolution and inpainting experiments to demonstrate the relevance of the proposed method.

Additional results are presented in Appendix~\ref{app:add}. All experiments can be reproduced using the code available at \url{https://github.com/jprost76/PnP-HVAE}.



\subsection{Face Image restoration (FFHQ)}\label{ssec:faces}
We first demonstrate the effectiveness of PnP-HVAE on highly structured data, by performing face image restoration.
Latent variable generative models can accurately model structured images such as face images \cite{karras2019style,vahdat2020nvae,child2021very,kingma2018glow}, and then be used to produce high quality restoration of such data. 
In our experiments, we use the VDVAE model of~\cite{child2021very}, pre-trained on the FFHQ dataset~\cite{karras2019style}, as our hierarchical VAE prior.
VDVAE has $L=66$ latent variable groups in its hierarchy and generates images at resolution $256\times256$.

We compare PnP-HVAE with the intermediate layer optimization algorithm (ILO)~\cite{daras2021intermediate} that is based on a different class of generative models than HVAE. ILO is a GAN inversion method which optimizes the image latent code along with the intermediate layer representation of a StyleGAN to generate an image consistent with a degraded observation.
We use the official implementation of ILO, along with a StyleGAN2 model~\cite{karras2020analyzing, stylegan2pytorch}, that was trained for 550k iterations on images of resolution $256\times256$ from FFHQ.  
As VDVAE and StyleGAN models are not trained on the same train-test split of FFHQ, we chose to evaluate the methods on a subset of 100 images from the CelebA dataset~\cite{liu2018large}. 
For super-resolution, the degradation model corresponds to the application of a gaussian low-pass filter followed by a $\times 4$ sub-sampling, and the addition of a gaussian white noise with $\sigma=3$.
For the deblurring, we considered motion blur and  gaussian kernels, both with a noise level $\sigma=8$. %

We provide quantitative comparisons in table~\ref{table:comp_ILO}, along with a visual comparison of the results in figure~\ref{fig:face_restoration}.
PnP-HVAE has the best  PSNR and SSIM results for all the considered restoration tasks, while ILO provides better results  for the perceptual distance.
By jointly optimizing the image and its latent variable, PnP-HVAE provides  results that are both realistic and consistent with the degraded observation.
On the other hand,  ILO  only optimizes on an extended latent space. This method generates  sharp and realistic images with better LPIPS scores,   
but the results lack  of consistency with respect to the observation, which explains the overall lower PSNR performance. 






\subsection{PatchVDVAE: a HVAE for natural images}\label{ssec:patchVDVAE}
Available generative models in the literature operate on images of  fixed resolutions and
are either restrained to datasets of limited diversity, or even to registered face images~\cite{kingma2018glow,child2021very, vahdat2020nvae, karras2019style}, or requiring additional class information~\cite{brock2018large, dhariwal2021diffusion, song2020score, luhman2022optimizing}.
Fitting an unconditional model on natural images appears to be a more difficult task, as their resolution can change, and their content is highly diverse.
The complexity of the problem can be reduced by learning a prior model on patches of reduced dimension. 
For image restoration problems, the patch model can be reused on images of higher dimensions~\cite{zoran2011learning,prost2021learning,altekruger2022patchnr}. When the model is a full CNN, the prior on the set of the  patches can  be computed efficiently by applying the network on the full image~\cite{prost2021learning}.

We thus introduce  patchVDVAE, a fully convolutional hierarchical VAE.
Contrary to existing HVAE models whose resolution is constrained by the constant tensor at the input of the top-down block, patchVDVAE can generate images of different resolutions by controlling the dimension of the input latent. 
This amounts to defining a prior on patches whose dimension corresponds to the receptive field of the VAE. A similar model is used for image denoising in~\cite{prakash2021interpretable}.

 
For PatchVDVAE architecture, we use the same bottom-up and top-down blocks as VDVAE~\cite{child2021very}, and replace the constant trainable input in the first top-down block by a latent variable, to make the model fully convolutional (details on the  architecture are given in Appendix~\ref{app:details}). 
The training dataset is composed of $128\times 128$ patches extracted from a combination of DIV2K~\cite{agustsson2017ntire} and Flickr2K~\cite{Lim_2017_CVPR_workshops} datasets.
We perform data augmentation by extracting  patches at $3$ resolutions: HR-images and $\times 2$ and $\times 4$ downscaled images. 
The model is trained for $7.10^5$ iterations with a batch size of $64$. Following the recommendation of~\cite{hazami2022efficient}, we use Adamax optimizer with an exponential moving average and gradient smoothing of the variance.
We set the decoder model to be a gaussian with diagonal covariance, as in~\cite{luhman2022optimizing}.
PatchVDVAE is fully convolutional and can generate images of dimension that are multiples of $64$ as illustrated by
figure~\ref{fig:vdvae}.

\newlength{\patchwidth}
\setlength{\patchwidth}{0.135\columnwidth}
\begin{figure}[!ht]
    \centering
    \begin{subfigure}[t]{.34\columnwidth}\hspace{0.1cm}
        \setlength{\tabcolsep}{0.02pt}
\renewcommand{\arraystretch}{0}
        \begin{tabular}{*{2}{p{1.03\patchwidth}}}
            \includegraphics[width=\patchwidth]{figures_arxiv/patchVDVAE/samples/generated/64x64/setup-5-image-0018.png} &
            \includegraphics[width=\patchwidth]{figures_arxiv/patchVDVAE/samples/generated/64x64/setup-5-image-0016.png} \\
            \includegraphics[width=\patchwidth]{figures_arxiv/patchVDVAE/samples/generated/64x64/setup-5-image-0008.png} &
            \includegraphics[width=\patchwidth]{figures_arxiv/patchVDVAE/samples/generated/64x64/setup-5-image-0019.png}   
        \end{tabular}
    \end{subfigure}\hspace{-0.15cm}
    \begin{subfigure}[t]{.64\columnwidth}
\begin{tabular}{cc}\vspace{-0.1cm}
\includegraphics[width=2\patchwidth]{figures_arxiv/patchVDVAE/samples/generated/256x256/setup-2-image-0009.png}&
        \includegraphics[width=2\patchwidth]{figures_arxiv/patchVDVAE/samples/generated/256x256/setup-2-image-0002.png}\end{tabular}

    \end{subfigure}
    \caption{\label{fig:vdvae} Left: $64\times64$ patches samples from our patchVDVAE model trained on patches from natural images.
    Right: PatchVDVAE is fully convolutional and it can generate images of higher resolution (here: $128\times128$).\vspace{-0.2cm}}
\end{figure}

\subsection{Natural images restoration}\label{ssec:app_nat}
We  evaluate PnP-HVAE on natural image restoration.
For each task, we report the average value of the PSNR, the SSIM, and the LPIPS metrics on $20$ images from the test set of the BSD dataset~\cite{MartinFTM01}.\\


\noindent
{\bf Image deblurring.}
In the experiments, we consider $2$ gaussian kernels and $2$ motion blur kernels from~\cite{levin2009understanding}, with $3$ different noise levels 
$\sigma \in \{2.55, 7.65, 12.75\}$.
As a baseline we consider  EPLL~\cite{zoran2011learning}, which learns a prior on image patches with a gaussian mixture model.
We also compare PnP-HVAE  with PnP-MMO and GS-PnP, $2$ competing convergent Plug-and-Play methods based on CNN denoisers.
PnP-MMO~\cite{pesquet2021learning} restricts the denoiser to be contraction in order to guarantee the convergence of the PnP forward-backard algorithm. GS-PnP~\cite{hurault2022gradient} considers a gradient step denoiser and reaches state-of-the-art performances of non converging methods~\cite{zhang2021plug}.
We set the temperature $\tau$  in our method as $0.95$, $0.8$ and $0.6$ for noise levels $2.55$, $7.65$ and $12.75$ respectively, and we let it run for a maximum of $50$ iterations. 
For the three compared methods we use the official implementations and pre-trained models provided by the respective authors. 
Details on the choice of hyperparameters for the concurrent methods are provided in the Appendix~\ref{app:details}
Figure~\ref{fig:deblurring_bsd} illustrates that our method provides correct deblurring results. 

According to table~\ref{tab:deb}, the performance of PnP-HVAE is between those of EPLL and GS-PnP and it outperforms PnP-MMO for large noise levels.\\

\begin{table}
\begin{center}\footnotesize
    \begin{tabular}{>{\centering}m{.3cm}*{5}{c}}
    $\sigma$ &Method & PSNR$\uparrow$ & SSIM$\uparrow$ & LPIPS$\downarrow$  \\ 
    \hline
    \multirow{4}{*}{\vcell{$2.55$}}
    & PnP-HVAE & $27.75$ & $0.79$ & $0.31$\\
    & GS-PNP \cite{hurault2022gradient} & $\mathbf{29.59}$ & $\mathbf{0.84}$ & $\mathbf{0.22}$\\
    & EPLL \cite{zoran2011learning} & $26.49$ & $0.71$ & $0.36$\\ 
    & PnP-MMO \cite{pesquet2021learning} & $\underbar{29.50}$ & $\underbar{0.83}$ & $\underbar{0.20}$ \\ \hline
    \multirow{4}{*}{\vcell{$7.65$}}
    & PnP-HVAE & $\underbar{26.36}$ & $\underbar{0.72}$ & $\underbar{0.40}$\\
    & GS-PNP \cite{hurault2022gradient} & $\mathbf{27.33}$ & $\mathbf{0.77}$ & $\mathbf{0.31}$\\
    & EPLL \cite{zoran2011learning} & $24.04$ & $0.66$ & $0.45$ \\ 
    & PnP-MMO \cite{pesquet2021learning} & $25.34$ & $0.69$ & $0.34$\\
    \hline
    \multirow{4}{*}{\vcell{$12.75$}}
    & PnP-HVAE & $\underbar{25.12}$ & $\mathbf{0.73}$ & $\underbar{0.47}$\\
    & GS-PNP \cite{hurault2022gradient} & $\mathbf{26.32}$ & $\mathbf{0.73}$ & $\mathbf{0.37}$\\
    & EPLL \cite{zoran2011learning} & $23.28$ & $0.61$ & $0.51$ \\ 
    & PnP-MMO \cite{pesquet2021learning} & $22.42$ & $0.53$& $0.54$ \\
    \hline
    &\vspace*{-.3cm}\\
            \multicolumn{2}{c}{Blur and motion kernels}& \multicolumn{3}{c}{
        \includegraphics*[scale=1]{figures_arxiv/kernels/4.png}\;\includegraphics*[scale=1]{figures_arxiv/kernels/7.png}\;\includegraphics*[scale=1]{figures_arxiv/kernels/9.png}\;\includegraphics*[scale=1]{figures_arxiv/kernels/11.png}} 
    \end{tabular}
        \caption{\label{tab:deb}Comparison  of PnP-HVAE  and other restoration methods on deblurring. Results are averaged on $4$ kernels.\vspace{-0.2cm}}% on image deblurring.}
    \end{center}
\end{table}

\begin{figure}
    
    \begin{subfigure}[h]{\linewidth}
        \centering
        \includegraphics*[width=\columnwidth]{figures_arxiv/deb_s255_k7.pdf}\vspace{-0.1cm}
        \caption{Gaussian blur, $\sigma=2.55$}
    \end{subfigure}
    \begin{subfigure}[h]{\linewidth}
        \centering
        \includegraphics*[width=\columnwidth]{figures_arxiv/deb_s765_k11.pdf}\vspace{-0.1cm}
        \caption{Motion blur, $\sigma=7.65$}
    \end{subfigure}\vspace*{-0.1cm}
    \caption{\label{fig:deblurring_bsd} Natural image deblurring\vspace{-0.1cm}}
\end{figure}

\noindent {\bf Effect of the temperature.}
PnP-HVAE gives control on the temperature of the prior over the latent space.
In figure~\ref{fig:temp_effect}, we illustrate that reducing the temperature increases the strength of the regularization prior. In this example the tuning $\tau=0.7$ produces the best performance.\\
\begin{figure}[!ht]
   
    \includegraphics[width=\columnwidth]{figures_arxiv/demo_temp.pdf}\vspace{-0.15cm}
    \caption{ \label{fig:temp_effect} Effect of the temperature in PnP-VAE on a deblurring problem, with $\sigma=7.65$.\vspace{-0.15cm}}
\end{figure}


\noindent
{\bf Image inpainting.}
Next we consider the task of noisy image inpainting. 
We compose a test-set of 10 images from the validation set of BSD~\cite{MartinFTM01} and we create masks
  by occluding diverse objects of small size in the images. 
A gaussian white noise with $\sigma=3$ is added to the images.
As a comparaison, we still consider GS-PnP and EPLL.
For PnP-HVAE, the temperature is set to $\tau=0.6$, and the algorithm is run for a maximum of $200$ iterations, unless the residual $||\x_{k+1}-\x_k||$ is on a plateau.
We provide on Table~\ref{tab:inpainting_bsd} the distortion metrics with the ground truth, as well as a visual
\begin{table}



\begin{center}
    \begin{tabular}{cccc}
        & PSNR$\uparrow$ & SSIM$\uparrow$ &LPIPS$\downarrow$ \\\hline
        PnP-HVAE  & $\mathbf{29.54}$ & $\mathbf{0.93}$ & $\mathbf{0.06}$\\
        GS-PNP & $28.52$ & $\mathbf{0.93}$ & $0.09$\\
        EPLL & $\underline{29.16}$ & $\mathbf{0.93}$ & $\mathbf{0.06}$\\
    \end{tabular}
    \caption{\label{tab:inpainting_bsd}Quantitative evaluation for inpainting on BSD.}
    \end{center}
\end{table}
comparison on figure~\ref{fig:inpainting_bsd}. 
With its hierarchical structure,  PnP-HVAE outperforms the compared methods. \vspace{0.05cm}



\begin{figure}[!h]
    \includegraphics[width=\columnwidth]{figures_arxiv/demo_inp_bsd2.pdf}\vspace{-0.1cm}
    \caption{\label{fig:inpainting_bsd}Natural image inpainting\vspace{-0.3cm}}
\end{figure}













% % % % % % % % % % % % % % % % % % % % % % % % % % % % % % % % % % % % % % % % % % % % % % % % 



% % % % % % % % % % % % % % % % % % % % % % % % % % % % % % % % % % % % % % % % % % % % % % % % 
\section{Conclusion}\label{sec:conclusion}

In this paper, we presented a novel approach for performing premise selection.
It parallels image captioning, combining transfer learning on graph neural networks with sequence learning.
The graph representation provides a holistic view of the problem structure, while the sequence model uses this embedding to predict the sequence of axioms necessary for the proof.
Our evaluation found that the model performs better when the GNN is pre-trained on a related and supervised task with embeddings containing information of all the nodes in the graph.
Further, we observed that effective axiom captioning requires a fixed axiom order and a greedy decoder sampler.
Lastly, the proposed approach dramatically increases the number of solved problems when complemented with \sine and significantly outperforms related machine learning methods.


\subsection*{Acknowledgements}
%We thank Michael Rawson for providing the formula to graph translator utilised in this work.

We thank Michael Rawson for providing a translator from formulas to graphs, utilised in this work.


% % % % % % % % % % % % % % % % % % % % % % % % % % % % % % % % % % % % % % % % % % % % % % % % 


%
% ---- Bibliography ----
%
% BibTeX users should specify bibliography style 'splncs04'.
% References will then be sorted and formatted in the correct style.
%
% \bibliographystyle{splncs04}
% \bibliography{mybibliography}
%
%\begin{thebibliography}{8}
%\end{thebibliography}

\bibliographystyle{plain}
\bibliography{ref}



\end{document}
