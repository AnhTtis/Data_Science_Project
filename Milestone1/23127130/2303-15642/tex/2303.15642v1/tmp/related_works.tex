

Premise selection has previously been addressed with heuristic-based methods such as MePo~\cite{DBLP:journals/japll/MengP09} and the very successful \sine~\cite{DBLP:conf/cade/HoderV11} algorithm.
The core idea of \sine is that axioms are likely to contribute towards the proof if they contain symbols related to the symbols in the conjecture.
This is achieved by iteratively selecting axioms with symbols occurring in a set of selected axioms relative to how often the symbol appears globally. %in the overall problem.
The main limitation of the approach is a low specificity and not utilising any information from existing proofs.


The task of premise selection has also been approached with machine learning methods such as Naive Bayes~\cite{DBLP:journals/jar/Urban06}, kernel methods~\cite{DBLP:journals/jar/AlamaHKTU14}, K-NN~\cite{DBLP:journals/jar/KaliszykU15a} and binary classification~\cite{DBLP:journals/corr/AlemiCISU16, DBLP:journals/corr/abs-1807-10268,  https://doi.org/10.48550/arxiv.1802.03375, DBLP:conf/cade/RawsonR20}.
In the binary setting, the goal is to train a supervised model to score conjecture-axiom pairs.
A significant drawback of this method is that axioms are considered independent, and the problem sizes strongly skew predictions.
Instead, axioms should be treated as a collective entity, as all the axioms occurring in a proof must be selected to construct the proof.

The approaches most similar to our method are the sequence-to-sequence approach in~\cite{DBLP:conf/lpar/PiotrowskiU20} and its extension with a Transformer model~\cite{DBLP:conf/mkm/ProrokovicWS21}.
The sequence models treat the conjecture as a sequence of tokens and map it to a sequence of axioms.
Their main limitation is being unaware of how the conjecture relates to elements of the various axioms.
GNNs can model the relationship between formulae elements, as shown by the binary graph classification approach in~\cite{DBLP:conf/cade/RawsonR20}.
Meanwhile, our approach is aware of the relationship between the axioms occurring in the proof and the conjecture's connection to these axioms.


% Maybe relate back to SiNE which is more "context aware"

