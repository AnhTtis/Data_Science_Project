% ****** Start of file apssamp.tex ******
%
%   This file is part of the APS files in the REVTeX 4.2 distribution.
%   Version 4.2a of REVTeX, December 2014
%
%   Copyright (c) 2014 The American Physical Society.
%
%   See the REVTeX 4 README file for restrictions and more information.
%
% TeX'ing this file requires that you have AMS-LaTeX 2.0 installed
% as well as the rest of the prerequisites for REVTeX 4.2
%
% See the REVTeX 4 README file
% It also requires running BibTeX. The commands are as follows:
%
%  1)  latex apssamp.tex
%  2)  bibtex apssamp
%  3)  latex apssamp.tex
%  4)  latex apssamp.tex
%
\documentclass[%
 reprint,
%superscriptaddress,
%groupedaddress,
%unsortedaddress,
%runinaddress,
%frontmatterverbose, 
%preprint,
%preprintnumbers,
%nofootinbib,
%nobibnotes,
%bibnotes,
 amsmath,amssymb,
 aps,
%pra,
%prb,
prl,
%rmp,
%prstab,
%prstper,
floatfix,
]{revtex4-2}

\usepackage{graphicx}% Include figure files
\usepackage{dcolumn}% Align table columns on decimal point
\usepackage{bm}% bold math
%\usepackage{hyperref}% add hypertext capabilities
%\usepackage[mathlines]{lineno}% Enable numbering of text and display math
%\linenumbers\relax % Commence numbering lines
\usepackage[dvipsnames]{xcolor} % for colors
\newcommand{\David}[1]{\textcolor{blue}{[DSS:~#1]}}
%\usepackage[showframe,%Uncomment any one of the following lines to test 
%%scale=0.7, marginratio={1:1, 2:3}, ignoreall,% default settings
%%text={7in,10in},centering,
%%margin=1.5in,
%%total={6.5in,8.75in}, top=1.2in, left=0.9in, includefoot,
%%height=10in,a5paper,hmargin={3cm,0.8in},
%]{geometry}

%the below silences a warning per https://tex.stackexchange.com/questions/458544/revtex4-1-warnings-bibtex-jnrlst-dependency-not-reversed-set-1-and-bibtex
\bibliographystyle{apsrev4-1}
%should probably do (For \documentclass{revtex4-2), use \bibliographystyle{apsrev4-2}.) instead

\begin{document}

\preprint{APS/123-QED}

\title{
Is the Molecular Weight Dependence of the Glass Transition Temperature Driven by a Chain End Effect?
%\\with Forced Linebreak
}% Force line breaks with \\
%\thanks{A footnote to the article title}%

\author{William F. Drayer}
% \altaffiliation[Also at ]{Physics Department, XYZ University.}%Lines break automatically or can be forced with \\
\author{David S. Simmons}
 \email{dssimmons@usf.edu}
\affiliation{%
 Department of Chemical, Biological and Materials Engineering, University of South Florida, Tampa, FL 33620, USA
% Authors' institution and/or address\\
% This line break forced with \textbackslash\textbackslash
}%



\date{\today}% It is always \today, today,
             %  but any date may be explicitly specified

\begin{abstract}

The immense dependence of the glass transition temperature $T_g$ on molecular weight $M$ is one of the most fundamentally and practically important features of polymer glass formation. Here, we report on molecular dynamics simulation of multiple multiple model polymers demonstrating that the 70-year-old canonical explanation of this dependence - a simple chain end dilution effect - is likely incorrect at leading order. Instead, end effects, present only in relatively stiff polymers, become less significant on cooling, and $T_g(M)$ trends are dominated by whole-chain shifts in $T_g$ rather than an end effect.
These findings are more consistent with two-barrier models of $T_g$ and its $M$-dependence, suggesting a need to reassess the canonical understanding of this dependence and an opportunity to reveal new glass formation physics via renewed study of this trend.

\end{abstract}

%\keywords{Suggested keywords}%Use showkeys class option if keyword
                              %display desired
\maketitle

%\tableofcontents

%\section{\label{sec:level1}First-level heading:\protect
%\\ The line break was forced \lowercase{via} \textbackslash\textbackslash}

A diverse array of systems solidify on laboratory timescales through the glass transition,
a poorly understood phenomenon wherein relaxation times dramatically grow on cooling over a finite range of temperature $T$ \cite{debenedetti_supercooled_2001,cavagna_supercooled_2009,novikov_temperature_2022}.
One central feature of this transition
is a profound dependence on molecular weight $M$ or polymer degree of polymerization $N$: 
the glass transition temperature $T_g$ commonly differs over 200 K 
between the small molecule (e.g., monomer) and the infinite $M$ polymer limits \cite{novikov_correlation_2013}.

The canonical textbook explanation \cite{hiemenz_polymer_2007, rubinstein_polymer_2003, coleman_fundamentals_1998, rudin_elements_2012, mathot_calorimetry_1994, rosen_fundamental_1993} for this trend was established in the early 1950's by Fox and Flory (FF) \cite{fox_secondorder_1950} and Ueberreiter and Kanig (UK) \cite{ueberreiter_self-plasticization_1952},
depicting several scenarios wherein 
%the chain ends act as mobility-enhancing agents within the polymer. 
%Fox and Flory proposed a free volume scenario wherein end groups ``[disrupt] the local configuration order'' \cite{fox_secondorder_1950} %and thus enhance the mobility of their environment; 
%Ueberreiter and Kanig, in a similar manner, suggest that 
%``the end groups act as plasticizers and cause the `self\-plasticization' of the polymer'' \cite{ueberreiter_self-plasticization_1952}.
%Both cases can effectively be understood as a scenario wherein 
polymers are ``mixtures of end and middle groups'' \cite{ueberreiter_self-plasticization_1952}, 
with the ends exhibiting faster dynamics and/or higher free volume (lower $T_g$), and 
the middle groups exhibiting slower dynamics and/or lower free volume (higher $T_g$). 
In the high $M$ limit, one should then expect chain ends to exhibit enhanced mobility and free volume relative to
other chain segments,
with segments sufficiently far from chain ends exhibiting mobility characteristic of an infinite chain.

Despite the long-standing predominance of this perspective, 
%a debate 
questions have emerged over whether this represents the sole, 
or even the dominant, 
mechanism driving the $M$ dependence of $T_g$. 
Early work by Cown \cite{cown_general_1975} argued for the presence of 
%multiple
three regimes of $T_g(M)$ behavior,
a complexity not captured by the two-parameter FF and UK forms.
Novikov and R{\"o}ssler have suggested that the canonical scenario is missing a distinct mechanism that dominates in the low $M$ limit \cite{novikov_correlation_2013}. 
Other distinct scenarios have suggested that the $T_g(M)$ dependence is driven by the growth with $M$ of chain stiffness or intra\-molecular activation barriers,
neither directly mediated by any chain end effect
\cite{mirigian_dynamical_2015,baker_cooperative_2022}. Even the basic physical rationale for the FF and UK perspectives has been reconsidered, 
with Zaccone and Terentjev \cite{zaccone_disorder-assisted_2013} 
showing that the the FF equation can be derived by a chain connectivity rather than chain end dilution argument.

Data published by Miwa et al.\ adds to this complexity \cite{miwa_influence_2003}.
They reported that chain ends exhibit a local drop in the temperature of a spin transition
(which they argue is proportional to $T_g$)
for spin-labeled polystyrene, but that the chain end spin transition was \textit{itself} $M$-dependent, even at fairly high $M$. While the end-enhancement in mobility seems superficially consistent with canonical chain end dilution models, its local $M$ dependence is not anticipated by these models, 
which expect that chain ends should exhibit roughly fixed with $M$ (but enhanced relative to the mid-chain) mobility. The overall $M$ dependence is instead expected to emerge \textit{at the mean chain level} by averaging. 

To assess whether the $T_g(M)$ dependence is predominantly driven by chain end effects,
we probe local dynamics in molecular dynamics (MD) simulations of three established polymers models - 
a freely-jointed chain (FJC) \cite{kremer_dynamics_1990,bulacu_molecular-dynamics_2007}, 
a freely-rotating chain (FRC) \cite{bulacu_molecular-dynamics_2007}, 
and OPLS all-atom polystyrene (AAPS) \cite{jorgensen_development_1996,hung_universal_2019,hung_forecasting_2020}. 
%The first two of these are newly simulated for this study, while the third involves new analysis of simulations published in our prior work 
%\cite{hung_universal_2019,hung_forecasting_2020}.
These  models span a range of complexity and strength of intra\-molecular correlations,
which prior studies 
\cite{sokolov_why_2007,mirigian_dynamical_2015,baker_cooperative_2022,novikov_temperature_2022,zhou_activated_2022} 
have suggested play an important role in $M$ effects on $T_g$.
%note: the zhou is an updated paper from Schweizer that reformulates Tg(N)
Methodological details and supplementary dynamical data can be found in prior publications \cite{hung_universal_2019} and the SI.


\begin{figure}[t]
 \includegraphics[width=\linewidth]{plot_tg_inf_ends.png}
 \caption{\label{fig:relax_tg_n}
Mean system and chain end $T_g$, normalized by its value for the highest $M$ simulated for a given system, plotted for the systems shown in the legend.
%as a function of degree of polymerization $N$ for the FJC (blue circles), FRC (orange diamonds), and AAPS (green triangles).
% Pluses, crosses, and open triangles indicate the $T_g$ of chain ends for FJC, FRC, and AAPS, respectively.
 Error bars are standard errors on $T_g$ as a fit parameter for AAPS (standard errors are smaller than the data points for the FJC and FRC).
 }
\end{figure}

We begin by considering how varying $M$ impacts $T_g$ for these systems. 
We initially define $T_g$ for AAPS on the 100s experimental timescale via an extrapolation using the MYEGA model, which has been well-validated against experiment for AAPS over a wide range of chain lengths in prior work \cite{hung_forecasting_2020}. 
For the two coarse systems, consistent with prior work \cite{hsu_glass-transition_2016,xia_molecular_2015, mangalara_tuning_2015,ghanekarade_signature_2023} we define $T_g(M)$ on a computational timescale of $10^4 \tau_{LJ}$ (Lennard-Jones time units). As shown in Fig.\ \ref{fig:relax_tg_n}, 
all simulated systems exhibit an appreciable dependence of their mean system $T_g$ on chain length $N$, 
in a manner comparable to experiment.
This dependence is stronger in models with stronger intra\-molecular correlations 
%(AAPS $>$ FRC $>$ FJC).
(AAPS is the most stiff while the FJC is the most flexible), 
consistent with prior reasoning and common trends in experimental polymers 
\cite{sokolov_why_2007,novikov_correlation_2013,mirigian_dynamical_2015,novikov_temperature_2022,baker_cooperative_2022}.
%note: PDMS has 1 - 112/149 \approx 25% Tg reduction 

\begin{figure*}[t]
 \includegraphics[width=\linewidth]{fig2.png}
 \caption{\label{fig:chain_index}
 $T_g$ (a-c), $\langle u^2 \rangle$ (d-f), and $\log \tau$ (g-i) plotted as a function of bead index $i$:  chain ends are labeled $i=1$ (either bead for FJC and FRC or monomer for AAPS), $i=2$ indicates  repeat units bonded to chain ends, and so on until $i=N/2$, which indicates  pairs of middle-most repeat units for even $N$ (for odd AAPS chain lengths, the last data point is a single monomer).
 Each row corresponds to each model, as labeled inside each panel.
 Error bars are standard errors on $T_g$ as a fit parameter (within the data points for FRC).
 Note that error bars increase with chain length for AAPS due to reduced statistical sampling.
 }
\end{figure*}

As further shown in Fig.\ \ref{fig:relax_tg_n}, in all three models chain ends exhibit a negligible or weak local reduction in $T_g$, relative to the overall magnitude of the trend of mean $T_g$ with $M$. 
Essentially no $T_g$ end effect is observed in the FJC model and there is less than a 1\% end reduction for the FRC model ($\sim 4$ K in real units for typical glassy polymers). 
In AAPS the end effect is of order 30K, 
relative to a shift of nearly 200K in mean $T_g$ with $M$ (consistent with Miwa et al.'s experimental findings \cite{miwa_influence_2003}). 
Moreover, we show in \ref{fig:chain_index}a-c that local variations in $T_g$ along the backbone near the chain end are weak or absent in all three models.
Even in AAPS, where a modest $T_g$ end effect is present, it does not appreciably propagate to covalent\-ly connected segments. 
%Collectively, these results across three distinct models do not accord with the expectation that  systematically and appreciably more mobile chain ends drive the $T_g$ $M$ dependence.
%Instead, the variation of $T_g$ with $M$ is dominated by variations in $T_g$ that are nearly uniform along the chain.
These end effects are vastly too small to account for the much larger variation in mean-chain $T_g$ with $N$ observed in Fig.\ \ref{fig:relax_tg_n}.

Could a more pronounced end effect still be present in free volume, which is the underlying proposition of the FF model? 
To test this, we compute local values of the Debye-Waller factor, 
$\langle u^2 \rangle$,
which is a measure of dynamical free volume 
\cite{mckenzie-smith_relating_2022,
mckenzie-smith_explaining_2021,
hung_universal_2019,
puosi_fast_2019,
pazmino_betancourt_quantitative_2015,
ottochian_universal_2011,
ngai_why_2004,
starr_what_2002,
ngai_correlation_2001}, glassy elasticity \cite{van_zanten_brownian_2000,yang_glassy_2011}, and particle localization \cite{dyre_colloquium:_2006, mirigian_elastically_2014, mirigian_elastically_2014-2, hall_aperiodic_1987, buchenau_relation_1992}
that quantifies the space accessed by a segment within a cage of its transient neighbors. 
For each model we compute $\langle u^2 \rangle$ at a consistent $T$ near the lowest accessed by simulation, such that it is interpolated within our data for shorter chains and only mildly extrapolated (linearly) for the longest chains.

Fig.\ \ref{fig:chain_index}d-f indicate that trends in $\langle u^2 \rangle$ with $M$ and chain location are non\-universal. 
As with $T_g$, shifts in $\langle u^2 \rangle$ with $M$ in the FJC model are nearly uniform through the whole chain, with at most a very weak chain end gradient. 
In contrast, AAPS exhibits a strong chain end gradient of enhanced $\langle u^2 \rangle$. The FRC model exhibits a mix of these two scenarios. 
This trend is physically reasonable: intramolecular correlations play a larger role in setting the cage scale in stiffer polymers, such that the reduced bond connectivity near chain ends relieves caging more significantly in these systems. However, as shown by the FJC, it is evidently possible for a polymer to exhibit at least a 10\% drop in $T_g$ without a significant chain end effect in either $T_g$ or free volume. 
This indicates that chain end mobility or free volume effects cannot be the \textit{sole} driver of the $T_g(M)$ dependence. At the same time, results from the FRC and AAPS systems indicate that even the presence of an appreciable
enhancement in chain-end $\langle u^2 \rangle$ does not necessarily translate to a substantial suppression in chain end $T_g$.

Why does enhanced chain end $\langle u^2 \rangle$ not reliably lead to suppressed chain end $T_g$? In Fig.\ \ref{fig:chain_index}g-i we report relaxation time $\tau$ gradients along the chain backbone at the same $T$ for which we reported $\langle u^2 \rangle$ end gradients. Evidently, for $T$ well above that of the experimental-timescale $T_g$, the chain end $\tau$ is indeed considerably reduced in cases for which $\langle u^2 \rangle$ is enhanced at the chain end. 
The failure of this increase to lead to a chain-end $T_g$ enhancement lies in a subtle feature of its $T$ dependence. As shown by Fig.\ \ref{fig:extrapolate}, the chain-end mobility enhancement strengthens on cooling on an \emph{absolute} basis. 
However, this strengthening is insufficient to keep up with the growth in the overall activation barrier of relaxation on cooling, such that it becomes \emph{relatively} weaker in its implications for $T_g$. 
As seen here and in Fig.\ \ref{fig:timescale}, the magnitude of the $T_g$ end effect therefore shrinks as the conventional timescale $\tau_g$ that defines $T_g$ is increased towards experimental timescales.

\begin{figure}[t!]
 \includegraphics[width=\linewidth]{relax_ps_100_extrapolate.png}
 \caption{\label{fig:extrapolate}
 $\tau$ vs $T$ for $N=100$ AAPS, for chain ends (open symbols) and mean system (green symbols).
 Curves in corresponding colors are fits to the MYEGA functional form \cite{mauro_viscosity_2009}.
 Green vertical lines (rightmost of each pair) denote the $T$ at which the mean system $\tau$ equals $10^{-11}$s (dotted), $10^{-9}$s (dot-dashed), and $10^2$s (dashed), respectively. 
 Black vertical lines (leftmost of each pair) denote the $T$ at which the chain ends exhibit these same $\tau$.
 The $T$ difference, $\Delta T_g$, is reported for each of the three timescales.
 Heavy red vertical segments highlight the chain end $\tau$ reduction
 relative to the mean system at each timescale. 
 }
\end{figure}

This indicates that the chain end mobility effects intuited by FF and UK are present at high $T$s in semiflexible chains (although negligible in quite flexible chains) but diminish in significance upon cooling and become sub-dominant by the experimental $T_g$ timescale. 
Within the context of many classical theories of glass formation, this observation seems surprising. 
Many of these theories, including free volume theory \cite{doolittle_studies_1951,white_explaining_2017} and classical entropy theories \cite{adam_temperature_1965},
postulate the presence of a single dominant activation barrier to relaxation in glass-forming liquids. 
This barrier is postulated to grow on cooling in a manner that is essentially multiplicative. 
As an example, within the Adam-Gibbs theory of glass formation, $\log(\tau)$ goes as a high-$T$ barrier times a cooperativity factor over $k_BT$ \cite{adam_temperature_1965}. 
A reduction in either high-$T$ activation barrier or cooperativity at the chain ends would thus not be expected to diminish in importance on cooling. 
A similar intuition would seem to hold for free volume approaches given the inverse proportionality of the activation barrier to a single quantity (the free volume). 
It is this intuition, that the alteration in activation barriers near the chain end becomes highly important in the glass formation range, that drives the classical FF and UK viewpoints.

More recently, a distinct set of perspectives have emerged that view glassy super-Arrhenius behavior as emerging from an \emph{additive}, rather than multiplicative, growth in the barrier on cooling \cite{mirigian_elastically_2014,mirigian_elastically_2014-2, schmidtke_temperature_2015}. 
In particular, the Elastically Collective Nonlinear Langevin Equation (ECNLE) theory of glass formation formulates its activation barrier as a sum of a local barrier (which grows relatively weakly on cooling) and a collective elastic barrier (which is predicted to emerge and then grow relatively strongly on cooling towards $T_g$) \cite{mirigian_elastically_2014,mirigian_elastically_2014-2}.

\begin{figure}[b!]
 \includegraphics[width=\linewidth]{plot_temp_timescale.png}
 \caption{\label{fig:timescale}
 Magnitude of chain end $T_g$ effect vs conventional timescale $\tau_g$ for AAPS chain lengths indicated in the legend.
 Filled markers denote points for which both mid-chain and end-chain $T_g$ values are interpolated from simulation data; 
 grey markers are points for which the mid-chain $T_g$ is interpolated and end-chain values are mildly extrapolated; 
 open markers denote points for which $T_g$ values are extrapolated for all monomers.
 }
\end{figure}

We can understand our results within the context these newer perspectives. Consider a generic two-barrier model wherein the total activation barrier $F_{tot}^{mid}$ in the mid-chain is the sum of a local barrier $F_{loc}^{mid}$ and a collective barrier $F_{coll}^{mid}$. 
%The end effect is present far above $T_g$ and thus most consistent with an alteration to the local barrier (since the collective barrier is absent at high $T$); we also expect it to be roughly %$T$ invariant, since it impacts intra\-molecular barriers that are mainly ster\-ic and bonding in nature and therefore relatively a\-thermal. 
Given its largely local steric and bonding nature, we treat the end effect as a constant fractional reduction of the local barrier by a factor $\alpha^{end}$.  As derived in more detail in the SI, this leads to the prediction that the chain-end effect on relaxation time is given by
\begin{equation}\label{eq:tauratio}
\log \left( \frac{{{\tau }_{end}}\left( N,T \right)}{{{\tau }_{mid}}\left( N,T \right)} \right)=\frac{\left( 1-{{\alpha }^{end}} \right){{F}_{loc}^{mid}}\left( N,T \right)}{kT},
\end{equation}
while the chain-end effect on $T_g$ is given by
\begin{equation}\label{eq:Tgrad}
\frac{T_{g}^{end}}{T_{g}^{mid}}\approx \left[ 1-\left( 1-{{\alpha }^{end}} \right)x_{loc}^{mid}\left( N,T_{g}^{end} \right) \right],
\end{equation}
where 
$x_{loc}^{mid}$ is the fraction of the total barrier in the mid-chain that is contributed by the local barrier. Eq. \ref{eq:tauratio} predicts that the chain end effect on $\tau$ grows with reducing $T$ and with cooling-induced growth of $F_{loc}^{mid}$, while Eq. \ref{eq:Tgrad} predicts that the $T_g$ end effect shrinks on cooling, because the fractional reduction from the local barrier $x_{loc}^{mid}$ shrinks on cooling \cite{mirigian_elastically_2014} such that the end effects become diluted within the faster-growing overall barrier. 


Across three models, chain ends evidently do not exhibit sufficiently reduced $T_g$ values to account for the dependence of mean $T_g$ on chain length on the basis of a chain end dilution effect. This would appear to demand a reevaluation of the textbook understanding of the $T_g(M)$ dependence. 
Our data indicate that $T_g(M)$ variations are primarily driven by the $M$ dependence of the mid-chain (or whole-chain) activation barrier $F_{tot}^{mid}(N,T)$, which in turn may result from whole-chain trends in some combination of the local and collective barriers. 
This type of scenario has been predicted within the ECNLE theory, where both local and collective elastic contributions to this activation barrier grow with increasing size of the fundamental dynamical repeat unit. This unit is treated as the entire molecule in small rigid molecules \cite{mirigian_elastically_2014,mirigian_elastically_2014-2} and as the Kuhn segment in polymers \cite{mirigian_dynamical_2015,zhou_activated_2022}. In the small molecule case, this leads to the prediction that $T_g\sim \sqrt{M}$ \cite{mirigian_elastically_2014,mirigian_elastically_2014-2}, which is consistent with the small molecule limit identified by Novikov and R\"ossler \cite{novikov_correlation_2013}. In the polymer case, the $M$ dependence follows from the growth of the Kuhn segment with increasing $N$ (as measured by growth of the chain's characteristic ratio $C_N$) - a reflection of increases in effective chain stiffness with $M$. \cite{mirigian_dynamical_2015,zhou_activated_2022}. 

Indeed, prior studies have found that variation of $T_g$ with $M$ tracks with variations in $C_N$ for at least polystyrene, poly(methyl methacrylate), and polyethylene. This has also been reported in polydimethyl siloxane, although this is in dispute \cite{baker_cooperative_2022}. While we cannot add to the extant data for PS $T_g$ vs $C_N$ correlations due to the limited number of chains that can be simulated in an AA simulation accessing the glass formation range, Fig.\ \ref{fig:cn} illustrates that $T_g$ is proportional to $C_N$ for our two bead-spring systems, adding to evidence that $T_g(M)$ closely tracks with $C_N(M)$ as $M$ is varied for a given polymer, in line with the ECNLE scenario  \cite{mirigian_dynamical_2015,zhou_activated_2022}. 

\begin{figure}[bt!]
 \includegraphics[width=\linewidth]{plot_cn_tg.png}
 \caption{\label{fig:cn}
$T_g$ plotted as a function of normalized $C_N$, each normalized by their value for the longest chain of that type simulated,
 for the FJC (blue circles) and FRC (orange diamonds). Lines in corresponding colors are linear fits.
 }
\end{figure}

It is not clear whether this $C_N$ scenario alone fully accounts for the $T_g(M)$ dependence given suggestions that $T_g(M)$ exhibits multiple regimes, particularly in stiffer polymers. Baker et al. have argued that this may result from nontrivial variations in chain conformational statistics, combined with an intramolecular dynamical facilitation effect \cite{baker_cooperative_2022}. 
While not excluding that scenario, our results suggest a potential alternate scenario given that we find chain end effects on $T_g$ to be present, if weak, in our stiffer systems. 
It may be that the multiple regimes observed in some stiffer polymers reflect a combination of a leading order stiffness effect with a perhaps second-order end effect with parallels to FF and UK.
%that and exhibits a distinct N-scaling. 

Overall, these findings indicate a need to reopen the study of $M$ effects on $T_g$, with a focus on more recent theories wherein this trend is dominated by whole-chain effects rather than end effects. 
Indeed, the finding that the $T$-dependence of dynamical chain end effects can be understood based on a two-barrier model of the glass transition suggests that a renewed focus on this problem may have the potential to yield broader insights into the nature of glass formation itself.

This material is based upon work supported by the National Science Foundation under Grant No. DMR - 1849594. The authors acknowledge helpful discussions with Dr. Kenneth Schweizer.

% The \nocite command causes all entries in a bibliography to be printed out
% whether or not they are actually referenced in the text. This is appropriate
% for the sample file to show the different styles of references, but authors
% most likely will not want to use it.
% \nocite{*}

\bibliography{apssamp}% Produces the bibliography via BibTeX.

%
% ****** End of file apssamp.tex ******

\end{document}

Potential final note:
Fox-Flory in the limit of small $N$ has a divergent slope,
whereas Ueberreiter-Kanig has a finite slope.
Ueberreiter and Kanig in fact note that the Fox-Flory prediction of the "apparent energy of activation" is far too small in comparison to the accepted value.
This note/discussion may be best discussed (if at all) on the first page, alongside discussion of the Novikov and Rossler as they also highlight that UK performs better than FF.
