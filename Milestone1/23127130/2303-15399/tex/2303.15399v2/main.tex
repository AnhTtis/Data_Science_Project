% ****** Start of file apssamp.tex ******
%
%   This file is part of the APS files in the REVTeX 4.2 distribution.
%   Version 4.2a of REVTeX, December 2014
%
%   Copyright (c) 2014 The American Physical Society.
%
%   See the REVTeX 4 README file for restrictions and more information.
%
% TeX'ing this file requires that you have AMS-LaTeX 2.0 installed
% as well as the rest of the prerequisites for REVTeX 4.2
%
% See the REVTeX 4 README file
% It also requires running BibTeX. The commands are as follows:
%
%  1)  latex apssamp.tex
%  2)  bibtex apssamp
%  3)  latex apssamp.tex
%  4)  latex apssamp.tex
%
\documentclass[%
 reprint,
%superscriptaddress,
%groupedaddress,
%unsortedaddress,
%runinaddress,
%frontmatterverbose, 
%preprint,
%preprintnumbers,
%nofootinbib,
%nobibnotes,
%bibnotes,
 amsmath,amssymb,
 aps,
%pra,
%prb,
prl,
%rmp,
%prstab,
%prstper,
floatfix,
]{revtex4-2}

\usepackage{graphicx}% Include figure files
\usepackage{dcolumn}% Align table columns on decimal point
\usepackage{bm}% bold math
%\usepackage{hyperref}% add hypertext capabilities
%\usepackage[mathlines]{lineno}% Enable numbering of text and display math
%\linenumbers\relax % Commence numbering lines
\usepackage[dvipsnames]{xcolor} % for colors
\newcommand{\David}[1]{\textcolor{blue}{[DSS:~#1]}}
%\usepackage[showframe,%Uncomment any one of the following lines to test 
%%scale=0.7, marginratio={1:1, 2:3}, ignoreall,% default settings
%%text={7in,10in},centering,
%%margin=1.5in,
%%total={6.5in,8.75in}, top=1.2in, left=0.9in, includefoot,
%%height=10in,a5paper,hmargin={3cm,0.8in},
%]{geometry}

%the below silences a warning per https://tex.stackexchange.com/questions/458544/revtex4-1-warnings-bibtex-jnrlst-dependency-not-reversed-set-1-and-bibtex
\bibliographystyle{apsrev4-1}
%should probably do (For \documentclass{revtex4-2), use \bibliographystyle{apsrev4-2}.) instead

\begin{document}

\preprint{APS/123-QED}

\title{
Is the Molecular Weight Dependence of the Glass Transition Temperature Driven by a Chain End Effect?
%\\with Forced Linebreak
}% Force line breaks with \\
%\thanks{A footnote to the article title}%

\author{William F. Drayer}
% \altaffiliation[Also at ]{Physics Department, XYZ University.}%Lines break automatically or can be forced with \\
\author{David S. Simmons}
 \email{dssimmons@usf.edu}
\affiliation{%
 Department of Chemical, Biological and Materials Engineering, University of South Florida, Tampa, FL 33620, USA
% Authors' institution and/or address\\
% This line break forced with \textbackslash\textbackslash
}%



\date{\today}% It is always \today, today,
             %  but any date may be explicitly specified

\begin{abstract}

The immense dependence of the glass transition temperature $T_g$ on molecular weight $M$ is one of the most fundamentally and practically important features of polymer glass formation. Here, we report on molecular dynamics simulation of multiple multiple model polymers demonstrating that the 70-year-old canonical explanation of this dependence - a simple chain end dilution effect - is likely incorrect at leading order. Instead, end effects, present only in relatively stiff polymers, become less significant on cooling, and $T_g(M)$ trends are dominated by whole-chain shifts in $T_g$ rather than an end effect.
These findings are more consistent with two-barrier models of $T_g$ and its $M$-dependence, suggesting a need to reassess the canonical understanding of this dependence and an opportunity to reveal new glass formation physics via renewed study of this trend.

\end{abstract}
% Variations in molecular weight can dramatically tune the glass temperature at which many liquids practically solidify - an effect with both fundamental significance for the understanding of the glass transition and practical significant for materials design. For almost 70 years, this dependence has been predominantly understood as a chain-end dilution effect, wherein more-mobile chain ends represent a larger portion of the sample at lower $M$. 
% Here we report on findings from molecular dynamics simulations demonstrating that this proposed mechanism is inconsistent with local dynamics in multiple model polymers. 
% In particular, end effects are absent in some models that exhibit appreciable $T_g(M)$ dependencies, and even when present become less significant upon cooling towards $T_g$. We show that this can be understood based on a two-barrier scenario of glass formation wherein chain-end activation barrier reductions are concentrated in a local barrier contribution that becomes less important at low temperatures.
% Instead of end effects, we find that the $T_g(M)$ dependence is instead dominated by whole-chain shifts in dynamics with $M$ and is presaged by molecular weight variations of chain stiffness, as predicted by the two-barrier Elastically Collective Nonlinear Langevin Equation theory of glass formation.  
% These results suggest a need to reassess the canonical textbook understanding of the molecular-weight dependence of $T_g$ and suggest that renewed work on this problem has the potential to yield broader insights into the nature of glass formation.

%\keywords{Suggested keywords}%Use showkeys class option if keyword
                              %display desired
\maketitle

%\tableofcontents

%\section{\label{sec:level1}First-level heading:\protect
%\\ The line break was forced \lowercase{via} \textbackslash\textbackslash}

A diverse array of systems solidify on laboratory timescales through the glass transition,
a poorly understood phenomenon whereby relaxation times dramatically grow on cooling over a finite range of temperature.
One central feature of this transition
is a profound dependence on molecular weight: 
the glass temperature temperature $T_g$ commonly differs over 200 K 
between the small molecule (e.g., monomer) and the infinite molecular weight polymer limits \cite{novikov_correlation_2013}.

The canonical textbook explanation \cite{hiemenz_polymer_2007, rubinstein_polymer_2003, coleman_fundamentals_1998, rudin_elements_2012, mathot_calorimetry_1994, rosen_fundamental_1993} for this trend was established in the early 1950's by Fox and Flory (FF) \cite{fox_secondorder_1950} and Ueberreiter and Kanig (UK) \cite{ueberreiter_self-plasticization_1952},
depicting several scenarios wherein the chain ends act as mobility-enhancing agents within the polymer. 
Fox and Flory proposed a free volume scenario wherein end groups ``[disrupt] the local configuration order'' \cite{fox_secondorder_1950} and thus enhance the mobility of their environment; 
Ueberreiter and Kanig, in a similar manner, suggest that 
``the end groups act as plasticizers and cause the `self\-plasticization' of the polymer'' \cite{ueberreiter_self-plasticization_1952}.
Both cases can effectively be understood as a scenario wherein polymers are ``mixtures of end and middle groups'' \cite{ueberreiter_self-plasticization_1952}, 
with the ends exhibiting faster dynamics (lower $T_g$) and 
the middle groups exhibiting slower dynamics (higher $T_g$). 
In either case, in the high molecular weight limit,
one should expect chain ends to exhibit enhanced mobility and free volume relative to
%the topological middle of the 
other chain segments,
with segments sufficiently far from chain ends exhibiting mobility that is characteristic of an infinite molecular weight chain.

Despite the long-standing predominance of this chain-end-dilution perspective, 
%a debate 
questions have emerged over whether this genuinely represents the sole, 
or even the dominant, 
mechanism driving the molecular weight dependence of $T_g$. 
Early work by Cown \cite{cown_general_1975} argued for the presence of 
%multiple
three regimes of $T_g(M)$ behavior,
a complexity not captured by the two-parameter FF and UK forms.
Novikov and R{\"o}ssler have suggested that the canonical scenario is missing a distinct mechanism, connected to the Lindemann criterion of melting, that dominates in the low molecular weight limit \cite{novikov_correlation_2013}. 
Other distinct scenarios have suggested that the $T_g(M)$ dependence is driven by the growth with $M$ of chain stiffness or intra\-molecular activation barriers,
neither directly mediated by any chain end effect
\cite{mirigian_dynamical_2015,baker_cooperative_2022}. Even the basic physical rationale for the FF and UK perspectives has been reconsidered, 
with Zaccone and Terentjev \cite{zaccone_disorder-assisted_2013} 
showing that the the FF equation can be derived by a chain connectivity rather than chain end dilution argument.

Data published by Miwa et al.\ in particular adds to this complexity \cite{miwa_influence_2003}.
Specifically, 
they reported that chain ends exhibit a local drop in the temperature of a spin transition
(which they argue is proportional to $T_g$)
for spin-labeled polystyrene, but that the chain end spin transition was \textit{itself} molecular weight dependent, even at fairly high molecular weights. While the end-enhancement in mobility seems superficially consistent with canonical chain end dilution models, its local molecular weight dependence is not anticipated by these models, 
which expect that chain ends should exhibit roughly fixed with molecular weight (but enhanced relative to the mid-chain) mobility. The overall molecular weight dependence is instead expected to emerge \textit{at the mean chain level} by averaging. 

%Miwa et al. speculated that this anomalous observation was the result of ``cooperative motions of the chain end with surrounding segments,'' but this proposition remains untested and also appears to involve a substantial revision of the canonical scenario.

%Several central questions thus remain open.
%How general are chain end mobility enhancements? 
%Do such effects genuinely emerge from enhancements in chain end free volume?  
%Do these effects mechanistically drive the $T_g$ dependence on molecular weight?

%To answer these questions,
To assess whether the $T_g(M)$ dependence is predominantly driven by chain end effects,
we probe local dynamics and free volume in molecular dynamics (MD) simulations of three established polymers models - 
a freely-jointed chain (FJC) \cite{kremer_dynamics_1990,bulacu_molecular-dynamics_2007}, 
a freely-rotating chain (FRC) \cite{bulacu_molecular-dynamics_2007}, 
and OPLS all-atom polystyrene (AAPS) \cite{jorgensen_development_1996,hung_universal_2019,hung_forecasting_2020}. 
%The first two of these are newly simulated for this study, while the third involves new analysis of simulations published in our prior work 
%\cite{hung_universal_2019,hung_forecasting_2020}.
These  models span a range of complexity and strength of intra\-molecular correlations and activation barriers,
which prior studies 
\cite{sokolov_why_2007,mirigian_dynamical_2015,baker_cooperative_2022,novikov_temperature_2022,zhou_activated_2022} 
have suggested may play an important role in molecular weight effects on $T_g$.
%note: the zhou is an updated paper from Schweizer that reformulates Tg(N)
Details of the three models, methodology for quantifying their dynamics \cite{hung_universal_2019}, and supplementary dynamical data can be found in the SI.


\begin{figure}[t]
 \includegraphics[width=\linewidth]{plot_tg_inf_ends.png}
 \caption{\label{fig:relax_tg_n}
$T_g$, normalized by its value for the highest molecular weight simulated for a given system, plotted as a function of degree of polymerization $N$ for the FJC (blue circles), FRC (orange diamonds), and AAPS (green triangles).
 Pluses, crosses, and open triangles indicate the $T_g$ of chain ends for FJC, FRC, and AAPS, respectively.
 Error bars are standard errors on $T_g$ as a fit parameter for AAPS (standard errors are smaller than the data points for the FJC and FRC).
 }
\end{figure}

We begin by considering how varying molecular weight impacts $T_g$ for these systems. 
We initially define $T_g$ for AAPS on the 100s experimental timescale via an extrapolation using the MYEGA model, which has been well-validated against experiment for AAPS over a wide range of chain lengths in prior work \cite{hung_forecasting_2020}. 
For the two coarse systems, consistent with prior work \cite{hsu_glass-transition_2016,xia_molecular_2015, mangalara_tuning_2015,ghanekarade_signature_2023} we define $T_g(M)$ on a computational timescale of $10^4 \tau_{LJ}$ (Lennard-Jones time units). As shown in Figure \ref{fig:relax_tg_n}, 
all simulated systems exhibit an appreciable dependence of their mean system $T_g$ on chain length $N$, 
in a manner comparable to experiment.
This dependence is stronger in models with stronger intra\-molecular correlations 
%(AAPS $>$ FRC $>$ FJC).
(AAPS is the most stiff while the FJC is the most flexible), 
consistent with prior reasoning and common trends in experimental polymers 
\cite{sokolov_why_2007,novikov_correlation_2013,mirigian_dynamical_2015,novikov_temperature_2022,baker_cooperative_2022}.
%note: PDMS has 1 - 112/149 \approx 25% Tg reduction 

\begin{figure*}[t]
 \includegraphics[width=\linewidth]{fig2.png}
 \caption{\label{fig:chain_index}
 $T_g$ (a-c), $\langle u^2 \rangle$ (d-f), and $\log \tau$ (g-i) plotted as a function of bead index $i$:  chain ends are labeled $i=1$ (either bead for FJC and FRC or monomer for AAPS), $i=2$ indicates  repeat units bonded to chain ends, and so on until $i=N/2$, which indicates  pairs of middle-most repeat units for even $N$ (for odd AAPS chain lengths, the last data point is a single monomer).
 Each row corresponds to each model, as labeled inside each panel.
 Error bars are standard errors on $T_g$ as a fit parameter (within the data points for FRC).
 Note that error bars increase with chain length for AAPS due to reduced statistical sampling.
 }
\end{figure*}

Also shown in Figure \ref{fig:relax_tg_n} is that for all three models the local $T_g$ computed selectively for chain ends exhibits a negligible or weak reduction relative to the chain middle, as compared to the overall magnitude of the $T_g(M)$ trend. 
Essentially no $T_g$ end effect is observed in the FJC model and there is less than a 1\% end reduction for the FRC model ($\sim 4$ K in real units for typical glassy polymers). 
In AAPS the end effect is of order 30K, 
relative to a shift of nearly 200K in mean $T_g$ with $M$ (consistent with Miwa et al.'s experimental findings \cite{miwa_influence_2003}). 
Moreover, we show in \ref{fig:chain_index}a-c that local variations in $T_g$ along the backbone near the chain end are weak or absent in all three models.
Even in a AAPS, where a modest $T_g$ end effect is present, it does not appreciably propagate to covalent\-ly connected segments. 
%Collectively, these results across three distinct models do not accord with the expectation that  systematically and appreciably more mobile chain ends drive the $T_g$ molecular weight dependence.
%Instead, the variation of $T_g$ with molecular weight is dominated by variations in $T_g$ that are nearly uniform along the chain.

These weak or absent end effects are vastly too small to account for the much larger variation in mean-chain $T_g$ with $N$ observed in Figure \ref{fig:relax_tg_n}.
Could a more pronounced end effect still be present in free volume, which is the underlying proposition of the FF and UK models? 
To test this, we compute local values of the Debye-Waller factor, 
$\langle u^2 \rangle$,
which is a measure of dynamical free volume 
\cite{mckenzie-smith_relating_2022,
mckenzie-smith_explaining_2021,
hung_universal_2019,
puosi_fast_2019,
pazmino_betancourt_quantitative_2015,
ottochian_universal_2011,
ngai_why_2004,
starr_what_2002,
ngai_correlation_2001}, glassy elasticity, and particle localization \cite{dyre_colloquium:_2006, mirigian_elastically_2014, mirigian_elastically_2014-2, hall_aperiodic_1987, buchenau_relation_1992}
that quantifies the space accessed by a segment within a cage of its transient neighbors. 
For each model we compute this property at a consistent temperature near the lowest accessed by simulation for each model, such that it is interpolated within our data for shorter chains and only mildly extrapolated (linearly) for the longest chains.

Figure \ref{fig:chain_index}d-f indicate that trends in $\langle u^2 \rangle$ with molecular weight and chain location are non\-universal. 
As with $T_g$, shifts in $\langle u^2 \rangle$ with molecular weight in the FJC model are nearly uniform through the whole chain, with at most a very weak chain end gradient. 
In contrast, AAPS exhibits a strong chain end connectivity gradient of enhanced $\langle u^2 \rangle$. The FRC model exhibits a mix of these two scenarios. 
This trend is physically reasonable: intramolecular correlations play a larger role in setting the cage scale in stiffer polymers, such that the reduced bond connectivity near chain ends relieves caging more significantly in these systems. However, as shown by the FJC, it is evidently possible for a polymer to exhibit at least a 10\% drop in $T_g$ without a significant chain end effect in either $T_g$ or free volume. 
This indicates that chain end mobility or free volume effects cannot be the \textit{sole} driver of the $T_g$ molecular weight dependence. At the same time, results from the FRC and AAPS systems indicate that even the presence of an appreciable
enhancement in chain-end $\langle u^2 \rangle$ does not necessarily translate to a substantial suppression in chain end $T_g$.

Why does enhanced chain end $\langle u^2 \rangle$ not reliably lead to suppressed chain end $T_g$? In Figure \ref{fig:chain_index}g-i we report relaxation time gradients along the chain backbone at the same temperature for which we reported $\langle u^2 \rangle$ end gradients. Evidently, at temperatures well above that of the experimental-timescale $T_g$, the chain end relaxation time is indeed considerably reduced at the chain end in cases for which $\langle u^2 \rangle$ is enhanced at the chain end. Why does this not lead to a suppression in chain end $T_g$?

\begin{figure}[t]
 \includegraphics[width=\linewidth]{relax_ps_100_extrapolate.png}
 \caption{\label{fig:extrapolate}
 Log relaxation time as a function of $T$ for AAPS where $N=100$ for chain ends (open symbols) and mean system (green symbols).
 Curves in corresponding colors are fits to the MYEGA functional form \cite{mauro_viscosity_2009}.
 The green vertical lines (rightmost of each pair) denote the $T$ at which the mean system relaxation time is equal to $10^{-11}$ s (dotted), $10^{-9}$ s (dot-dashed), and $10^2$ s (dashed), respectively. 
 The black vertical lines (leftmost of each pair) denote the $T$ at which the chain ends exhibit these same relaxation times.
 The temperature difference, $\Delta T_g$, is reported for each of the three timescales.
 The shrinking $\Delta T_g$ with increasing timescale indicates that the end-chain $T_g$ suppression weakens with increasing timescale convention $\tau_g$. 
 Heavy red vertical segments highlight the chain end mobility enhancement
 relative to the mean system at each timescale, which grows on cooling. 
 }
\end{figure}

To understand this, in Figure \ref{fig:extrapolate} we plot relaxation time against $T$ for chain ends and the mean system for the $N=100$ AAPS chain. 
Highlighted in the vertical red segments, the chain-end mobility enhancement actually strengthens on cooling on an \emph{absolute} basis. 
However, this strengthening is insufficient to keep up with the growth in the overall activation barrier of relaxation on cooling, such that it becomes \emph{relatively} weaker in its implications for $T_g$. 
Thus, as the conventional timescale $\tau_g$ that defines $T_g$ is increased towards experimental timescales, 
the $T_g$ end effect actually weakens. 
Figure \ref{fig:timescale} further illustrates this, in that the magnitude of the $T_g$ end effect indeed shrinks with increasing timescale, across a range of molecular weights, with a saturation observed on approach to experimental timescales.

\begin{figure}[t]
 \includegraphics[width=\linewidth]{plot_temp_timescale.png}
 \caption{\label{fig:timescale}
 Magnitude of chain end $T_g$ effect relative to corresponding timescale convention $\tau_g$ for a subset of AAPS chain lengths indicated in the legend.
 Filled markers denote points for which both mid-chain and end-chain $T_g$ values are interpolated from simulation data; 
 grey markers are points for which the mid-chain $T_g$ is interpolated and end-chain values are mildly extrapolated; 
 open markers denote points for which $T_g$ values are extrapolated for all monomers.
 }
\end{figure}

This indicates that the chain end mobility effects intuited by Flory, Fox, Uberreiter, and Kanig are present at high temperatures in semiflexible chains (although negligible in quite flexible chains) but diminish in significance upon cooling towards experimental $T_g$. 
They ultimately do not dominate the $T_g(M)$ trend at experimental timescales even in relatively stiff PS. 
Within the context of many classical theories of glass formation, this observation seems surprising. 
Many of these theories, including free volume theory \cite{doolittle_studies_1951,white_explaining_2017} and classical entropy theories of glass formation \cite{adam_temperature_1965},
postulate the presence of a single dominant activation barrier to relaxation in glass-forming liquids. 
This barrier is postulated to grow on cooling in a manner that is essentially multiplicative. 
As an example, within the Adam-Gibbs theory of glass formation, the logarithm of the relaxation time goes as a high temperature barrier times a cooperativity factor over $k_BT$ \cite{adam_temperature_1965}. 
A reduction in either high-temperature activation barrier or cooperativity at the chain ends would thus not be expected to diminish in importance on cooling. 
A similar intuition would seem to hold for free volume approaches given the inverse proportionality of the activation barrier to a single quantity (the free volume). 
It is this intuition, that the alteration in activation barriers near the chain end becomes highly important in the glass formation range, that drives the classical FF and UK viewpoints.

More recently, a distinct set of perspectives on the glass transition have emerged that view glassy super-Arrhenius behavior as emerging from an \emph{additive}, rather than multiplicative, growth in the barrier on cooling \cite{mirigian_elastically_2014,mirigian_elastically_2014-2, schmidtke_temperature_2015}. 
In particular, the Elastically Collective Nonlinear Langevin Equation (ECNLE) theory of glass formation formulates its activation barrier as a sum of a local barrier (which is not constant but grows relatively weakly on cooling) and a collective elastic barrier (which is predicted to emerge and then grow relatively strongly on cooling towards $T_g$) \cite{mirigian_elastically_2014,mirigian_elastically_2014-2}.

We can understand our simulation results more clearly within the context these newer perspectives. 
Consider a 
%schematic
generic two-barrier model wherein the total activation barrier $F_{tot}^{mid}$ in the mid-chain is the sum of a local barrier $F_{loc}^{mid}$ and a collective barrier $F_{coll}^{mid}$:
\begin{equation}\label{eq:midF}
F_{tot}^{mid}\left( N,T \right)=F_{loc}^{mid}\left( N,T \right)+F_{coll}^{mid}\left( N,T \right).
\end{equation}
Alterations of this barrier at the chain end are rooted in alterations to intra\-molecular correlations that are intrinsically present arbitrarily far above $T_g$, and therefore most directly impact the local barrier (since the collective barrier is absent far above $T_g$). 
We thus model the end effect as a fractional reduction of the local barrier by a factor $\alpha^{end}$, which we model as roughly temperature-invariant (but expect to be chemistry dependent and larger for stiffer chains) since it reflects a truncation of intra\-molecular barriers that are mainly ster\-ic and bonding in nature and therefore relatively a\-thermal:
\begin{equation}\label{eq:endF}
F_{tot}^{end}\left( N,T \right)={{\alpha }^{end}}F_{loc}^{mid}\left( N,T \right)+F_{coll}^{mid}\left( N,T \right).
\end{equation}

One can quantify the expected temperature dependence of the chain-end relaxation time gradient within this perspective by employing the total barrier forms above within a generalized activation law to compute the ratio of chain-end to chain-mid relaxation times:
\begin{equation}\label{eq:tauratio}
\log \left( \frac{{{\tau }_{end}}\left( N,T \right)}{{{\tau }_{mid}}\left( N,T \right)} \right)=\frac{\left( 1-{{\alpha }^{end}} \right){{F}_{loc}^{mid}}\left( N,T \right)}{kT}.
\end{equation}
This equation anticipates that the enhanced chain end mobility (relative to the mid-chain) should grow on cooling, as we observe in Figure \ref{fig:extrapolate}, simply because of the reduction in temperature and any growth on cooling of $F_{loc}^{mid}$.

Why then does this not translate to a growth in the chain end $T_g$ effect? 
To understand this, we combine Equations \ref{eq:midF} and \ref{eq:endF} with a generalized activation law as above, rearrange them to solve for temperature, apply each at its corresponding local $T_g$,
%(end or mid), 
and take their ratio. This gives
\begin{equation}\label{eq:Tgrad}
\frac{T_{g}^{end}}{T_{g}^{mid}}=R\left[ 1-\left( 1-{{\alpha }^{end}} \right)x_{loc}^{mid}\left( N,T_{g}^{end} \right) \right],
\end{equation}
where 
$R={F_{tot}^{mid}\left( N,T_{g}^{end} \right)}/{F_{tot}^{mid}\left( N,T_{g}^{mid} \right)},\;$ 
and
$x_{loc}^{mid}={F_{loc}^{mid}\left( N,T_{g}^{end} \right)}/{F_{tot}^{mid}\left( N,T_{g}^{end} \right)}\;$
is the fraction of the total barrier in the mid-chain that is contributed by the local barrier. 
For a weak $T_{g}$ end effect as we observe, we can approximate the relaxation process as Arrhenius over the limited temperature range involved, giving $R\approx1$. 
Consistent with the results in Figure \ref{fig:timescale}, Equation \ref{eq:Tgrad} then predicts that the $T_g$ end effect shrinks on cooling, even as the $\tau$ end effect grows, because the fractional reduction from the local barrier $x_{loc}^{mid}$ shrinks on cooling such that the end effects become diluted within the faster-growing overall barrier. 

Collectively, these findings indicate that chain end mixing is not the leading-order mechanism driving molecular weight effects on $T_g$. 
Across three models, chain ends do not exhibit sufficiently reduced $T_g$ values to account for the dependence of mean $T_g$ on chain length on the basis of a chain end dilution effect. 
We find that the enhancements in chain-end mobility intuited by the FF and UK formulations are present at high temperature, but \emph{diminish} in importance on cooling. This behavior can be rationalized within a two-barrier model of glass formation in which end effects primarily reside in the high temperature barrier. 

This would appear to demand a reevaluation of the textbook understanding of the $T_g(M)$ dependence. 
Our data indicate that $T_g(M)$ variations are primarily driven by the molecular weight dependence of the mid-chain (or whole-chain) activation barrier $F_{tot}^{mid}(N,T)$. 
This type of scenario has been predicted within the ECNLE theory, where both local and collective elastic contributions to $F_{tot}^{mid}(N,T)$ grow with increasing $N$, because increases in molecular weight increase chain stiffness as measured by the chain's characteristic ratio $C_N$ \cite{mirigian_dynamical_2015}. 
Indeed, prior studies have found that variation of $T_g$ with $M$ tracks with variations in $C_N$ for at least polystyrene, poly(methyl methacrylate), and polyethylene. This has also been reported in polydimethyl siloxane, although this is in dispute \cite{baker_cooperative_2022}. 

\begin{figure}[t]
 \includegraphics[width=\linewidth]{plot_cn_tg.png}
 \caption{\label{fig:cn}
$T_g$ plotted as a function of normalized $C_N$, each normalized by their value for the longest chain of that type simulated,
 for the FJC (blue circles) and FRC (orange diamonds). Lines in corresponding colors are linear fits to the data.
 }
\end{figure}

We cannot fruitfully add to the extant data for PS $T_g$ vs $C_N$ correlations: the relatively small system sizes necessary to reach the glass formation range in an all-atom model preclude sufficient chain conformation\-al statistics. 
However, in Figure \ref{fig:cn} we plot the $T_g$ of each of our bead-spring systems vs $C_N$. 
Evidently, $T_g$ is proportional to $C_N$ for these two systems, adding to evidence that $T_g(M)$ closely tracks with $C_N(M)$ as $M$ is varied for a given polymer. 
This suggests that trends in dynamical stiffness (intra\-molecular contributions to activation barriers throughout the chain) may play a central role in the $T_g(M)$ dependence, as encoded in the ECNLE theory \cite{mirigian_dynamical_2015}. 
We emphasize that this does not imply that $C_N$ controls $T_g$. 
Instead, $C_N$ serves as a proxy for how $M$ changes intra\-molecular activation barriers and chain relaxation modes. 

It is also not clear whether this $C_N$ scenario alone fully accounts for the $T_g(M)$ dependence given prior suggestions that $T_g(M)$ exhibits multiple regimes, particularly in stiffer polymers. Baker et al. have argued that this may results from nontrivial variations in chain conformational statistics, combined with an intramolecular dynamical facilitation effect \cite{baker_cooperative_2022}. 
While not excluding that scenario, our results suggest a potential alternate scenario given that we find chain end effects on $T_g$ to be present, if weak, in our stiffer systems. 
It may be that the multiple regimes observed in some stiffer polymers reflect a combination of a leading order stiffness effect with a perhaps second-order end effect with parallels to FF and UK.
%that and exhibits a distinct N-scaling. 

Overall, these findings indicate a need to reopen the study of molecular weight effects on $T_g$, with a focus on more recent theories wherein this trend is dominated by whole-chain effects rather than end effects. 
Indeed, the finding that the temperature-dependence of dynamical chain end effects can be understood based on a two-barrier model of the glass transition suggests that a renewed focus on this problem may have the potential to yield broader insights into the nature of glass formation itself.

This material is based upon work supported by the National Science Foundation under Grant No. DMR - 1849594. The authors acknowledge helpful discussions with Dr. Kenneth Schweizer.

% The \nocite command causes all entries in a bibliography to be printed out
% whether or not they are actually referenced in the text. This is appropriate
% for the sample file to show the different styles of references, but authors
% most likely will not want to use it.
% \nocite{*}

\bibliography{apssamp}% Produces the bibliography via BibTeX.

%
% ****** End of file apssamp.tex ******

\end{document}

Potential final note:
Fox-Flory in the limit of small $N$ has a divergent slope,
whereas Ueberreiter-Kanig has a finite slope.
Ueberreiter and Kanig in fact note that the Fox-Flory prediction of the "apparent energy of activation" is far too small in comparison to the accepted value.
This note/discussion may be best discussed (if at all) on the first page, alongside discussion of the Novikov and Rossler as they also highlight that UK performs better than FF.
