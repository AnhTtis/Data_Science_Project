% ****** Start of file apssamp.tex ******
%
%   This file is part of the APS files in the REVTeX 4.2 distribution.
%   Version 4.2a of REVTeX, December 2014
%
%   Copyright (c) 2014 The American Physical Society.
%
%   See the REVTeX 4 README file for restrictions and more information.
%
% TeX'ing this file requires that you have AMS-LaTeX 2.0 installed
% as well as the rest of the prerequisites for REVTeX 4.2
%
% See the REVTeX 4 README file
% It also requires running BibTeX. The commands are as follows:
%
%  1)  latex apssamp.tex
%  2)  bibtex apssamp
%  3)  latex apssamp.tex
%  4)  latex apssamp.tex
%
\documentclass[%
 reprint,
%superscriptaddress,
%groupedaddress,
%unsortedaddress,
%runinaddress,
%frontmatterverbose, 
%preprint,
%preprintnumbers,
%nofootinbib,
%nobibnotes,
%bibnotes,
 amsmath,amssymb,
 aps,
%pra,
%prb,
prl,
%rmp,
%prstab,
%prstper,
floatfix,
]{revtex4-2}

\usepackage{graphicx}% Include figure files
\usepackage{dcolumn}% Align table columns on decimal point
\usepackage{bm}% bold math
%\usepackage{hyperref}% add hypertext capabilities
%\usepackage[mathlines]{lineno}% Enable numbering of text and display math
%\linenumbers\relax % Commence numbering lines

%\usepackage[showframe,%Uncomment any one of the following lines to test 
%%scale=0.7, marginratio={1:1, 2:3}, ignoreall,% default settings
%%text={7in,10in},centering,
%%margin=1.5in,
%%total={6.5in,8.75in}, top=1.2in, left=0.9in, includefoot,
%%height=10in,a5paper,hmargin={3cm,0.8in},
%]{geometry}

%the below silences a warning per https://tex.stackexchange.com/questions/458544/revtex4-1-warnings-bibtex-jnrlst-dependency-not-reversed-set-1-and-bibtex
\bibliographystyle{apsrev4-1}
%should probably do (For \documentclass{revtex4-2), use \bibliographystyle{apsrev4-2}.) instead

\begin{document}

\preprint{APS/123-QED}

\title{
Is the Molecular Weight Dependence of the Glass Transition Temperature Caused by a Chain End Effect?
%\\with Forced Linebreak
}% Force line breaks with \\
%\thanks{A footnote to the article title}%

\author{William F. Drayer}
% \altaffiliation[Also at ]{Physics Department, XYZ University.}%Lines break automatically or can be forced with \\
\author{David S. Simmons}%
 \email{dssimmons@usf.edu}
\affiliation{%
 Department of Chemical, Biological and Materials Engineering, University of South Florida, Tampa, FL 33620, USA
% Authors' institution and/or address\\
% This line break forced with \textbackslash\textbackslash
}%

\date{\today}% It is always \today, today,
             %  but any date may be explicitly specified

\begin{abstract}
For almost 70 years, the pronounced dependence of the glass transition temperature $T_g$ on molecular size has been understood as resulting from enhanced mobility and free volume of chain ends, combined with a simple dilution effect as chain length increases. Here we report on findings from molecular dynamics simulations demonstrating that this proposed mechanism is inconsistent with local dynamics and free volume behavior in multiple model polymers. Results indicate that free volume is not universally enhanced at the chain end; moreover, we find that chain end mobility enhancements are far too weak to account for the trend in $T_g$ with varying molecular weight. Results instead support a scenario wherein this trend is driven by a molecular size dependence of intra\-molecular correlations and barriers – an effect already present at temperatures far above the glass formation range. 

\end{abstract}

%\keywords{Suggested keywords}%Use showkeys class option if keyword
                              %display desired
\maketitle

%\tableofcontents

%\section{\label{sec:level1}First-level heading:\protect
%\\ The line break was forced \lowercase{via} \textbackslash\textbackslash}

A diverse array of systems solidify on laboratory timescales through the glass transition,
a poorly understood phenomenon whereby relaxation times dramatically grow on cooling over a finite range of temperature.
One central feature of this transition
is a profound dependence on molecular weight: 
the glass temperature temperature $T_g$ commonly differs over 200 K 
between the small molecule (e.g., monomer) and the infinite molecular weight limits \cite{novikov_correlation_2013}.

The modern canonical explanation \cite{rubinstein_polymer_2003,hiemenz_polymer_2007} for this trend was established in the early 1950's by Fox and Flory \cite{fox_secondorder_1950} and Ueberreiter and Kanig \cite{ueberreiter_self-plasticization_1952},
depicting several scenarios wherein the chain ends act essentially as mobility enhancing agents within the polymer. 
Fox and Flory proposed a free volume scenario wherein end groups ``[disrupt] the local configuration order'' and thus enhance the mobility of their environment; 
Ueberreiter and Kanig, in a similar manner, suggest that ``the end groups act as plasticizers and cause the `self\-plasticization' of the polymer''.
Both cases can effectively be understood as a scenario wherein polymers are ``mixtures of end and middle groups'' \cite{ueberreiter_self-plasticization_1952}, 
with the ends exhibiting faster dynamics (lower $T_g$) and 
the middle groups exhibiting slower dynamics (higher $T_g$). 
In either case, in the high molecular weight limit,
one should expect chain ends to exhibit enhanced mobility and free volume,
with segments sufficiently far from chain ends exhibiting mobility and free volume of an infinite molecular weight chain.

Despite the long-standing predominance of this chain-end-dilution perspective, 
over the last several years a debate has begun to emerge over whether this mechanism genuinely represents the sole, 
or even the dominant, 
mechanism driving the molecular weight dependence of $T_g$. 
Novikov and R{\"o}ssler \cite{novikov_correlation_2013} have suggested that this scenario is missing a distinct mechanism that dominates in the low molecular weight limit. 
They propose that
a power-law relationship holds between $T_g$ and molecular weight for small molecules as a combined consequence of a proportionality of $T_g$ to the melting temperature and the molecular-weight dependence of the Lindemann criterion of melting.
Two other distinct scenarios have suggested that this phenomenon occurs due to 
the growth of chain stiffness or intra\-molecular activation barriers,
neither directly mediated by any chain end effect
\cite{mirigian_dynamical_2015,baker_cooperative_2022}. 

Data published 20 years ago by Miwa et al. \cite{miwa_influence_2003} 
reporting on experimental local dynamics near polymer chain ends adds to this complexity.
Specifically, 
they reported that chain ends exhibit a drop in the temperature of a spin transition for spin-labeled polystyrene, relative to the average chain behavior.
At first glance, this suggests a local enhancement in mobility at polystyrene chain ends, 
potentially consistent with canonical chain end dilution models. 
In contrast, however, the chain end spin transition temperature was \textit{itself} reported to be molecular weight dependent, even at fairly high molecular weights. 
This phenomenon is not anticipated by classical standard chain end mobility models, 
which expect that chain ends should exhibit roughly fixed with molecular weight (but enhanced relative to the mid-chain) mobility, 
with a molecular weight dependence emerging \textit{at the mean chain level} via averaging. 
Miwa et al. speculated that this anomalous observation was the result of ``cooperative motions of the chain end with surrounding segments,'' 
but this proposition has remained untested and would also appear to 
%reflect a substantial re-imagining 
necessitate a substantial revision
of the classical scenario.

Several central questions thus remain open.
How general are chain end mobility enhancements? 
Do such effects genuinely emerge from enhancements in chain end free volume?  
Are these effects large enough to quantitatively account for the $T_g$ dependence on molecular weight?

To answer these questions,
we perform molecular dynamics (MD) simulations of three established polymers models:
a freely-jointed chain (FJC) \cite{kremer_dynamics_1990,bulacu_molecular-dynamics_2007}, freely-rotating chain (FRC) \cite{bulacu_molecular-dynamics_2007}, and all-atom polystyrene (AAPS) \cite{jorgensen_development_1996,hung_forecasting_2020}.
We select these three models because they span a range of both complexity and strength of intra\-molecular correlations and intra\-molecular activation barriers, which, as discussed above, may play an important role in molecular weight effects on $T_g$. 
Details regarding the three models may be found in the SI.
We utilize the Predictive Step\-wise Quenching Algorithm \cite{hung_universal_2019} to efficiently obtain relaxation data 
(as determined from %by...
the self intermediate scattering function) across at least four decades of time for the FJC and FRC models and three decades for AAPS.
Plots of relaxation time as a function of inverse temperature 
%are available in the SI 
for each of the three models
and the routine for determining
%we determine 
$T_g$ for each model from these data 
is described in detail in the SI.

\begin{figure}[b]
 \includegraphics[width=\linewidth]{plot_tg_inf_ends.png}
 \caption{\label{fig:relax_tg_n}
 Plot of $T_g$, normalized for each longest chain $T_g$ as denoted by $T_{g,\infty}$, as a function of chain length $N$ for FJC (blue circles), FRC (orange diamonds), and AAPS (green triangles).
 Pluses, crosses, and open triangles indicate the $T_g$ of chain ends for FJC, FRC, and AAPS, respectively.
 Error bars are standard errors on $T_g$ as a fit parameter for AAPS (unvertainties are smaller than the data points for FJC and FRC).
 }
\end{figure}

As can be seen in Figure \ref{fig:relax_tg_n}, 
all simulated systems exhibit an appreciable dependence of their mean system $T_g$ on chain length $N$, 
in a manner comparable to experiment 
(indeed, as shown in prior work \cite{hung_forecasting_2020},
the $T_g$ dependence of the AAPS model is in good quantitative agreement with experiment).
This dependence evidently becomes stronger in models with stronger intra\-molecular correlations 
%(AAPS $>$ FRC $>$ FJC).
(AAPS is the most stiff while the FJC is the most flexible).
%note: PDMS has 1 - 112/149 \approx 25% Tg reduction 

In order to begin understanding what role chain end effects may play in these trends, we incorporate in Figure \ref{fig:relax_tg_n} the local $T_g$ measured selectively for chain ends. As can be seen here, the local chain-end  $T_g$ is statistically indistinguishable from the mean-chain $T_g$ in both the FJC and FRC models. A modest statistically significant difference in $T_g$ at the chain end is observed for AAPS chains.
Given that this chain end $T_g$ drop is far weaker in magnitude than the dependence of $T_g$ on $N$  and that the chain end $T_g$ is strongly dependent on on $N$ 
(as previously reported by Miwa et al. in experiment), 
this result seems to be %difficult to square
at odds with canonical chain end mobility models. 
However, a natural question is whether this chain end $T_g$ effect
might initiate a longer-range $T_g$ gradient along the chain that could compensate for its relatively small magnitude, at least in AAPS where it is statistically significant. Such a scenario of longer-range transference of chain end $T_g$ effects would appear to accord with the proposition of Miwa et al.

To the contrary of this proposition, Figure \ref{fig:chain_index} demonstrates that local variations in $T_g$ along the backbone near the chain end are weak or absent in all three models.
Indeed, the FJC exhibits uniform $T_g$ along the backbone for all $N$.
The FRC exhibits a weak chain end $T_g$ gradient, visually extending several monomers from the chain end but with a magnitude of less than a 1\% alteration in $T_g$ relative to bulk 
(which can be mapped to less than \textasciitilde4 K in real units for typical glassy polymers).
In AAPS, the chain ends themselves exhibit a modest $T_g$ reduction in longer chains as reported above, but covalent\-ly connected segments near the chain end exhibit weak, if any, alterations in $T_g$ propagated from the chain ends. 
Collectively, these results across three distinct models do not accord with the expectation that  systematically and appreciably more mobile chain ends drive the $T_g$ molecular weight dependence.
Instead, the variation of $T_g$ with molecular weight is dominated by variations in $T_g$ that are nearly uniform along the chain.

\begin{figure}[t]
 \includegraphics[width=\linewidth]{plot_chain_ends.png}
 \caption{\label{fig:chain_index}
 Plot of $T_g$ as a function of bead index $i$:  chain ends are labeled $i=1$ (either bead for FJC and FRC or monomer for AAPS), $i=2$ indicates  repeat units bonded to chain ends, and so on until $i=N/2$, which indicates  pairs of middle-most repeat units for even $N$ (for odd AAPS chain lengths, the last data point is a single monomer).
 Models from top to bottom are FJC, FRC, and AAPS.
 Error bars are standard errors on $T_g$ as a fit parameter (within the data points for FRC).
 Note that error bars increase with chain length for AAPS due to reduced statistical sampling.
 }
\end{figure}

\begin{figure}[t]
 \includegraphics[width=\linewidth]{plot_dwf_i.png}
 \caption{\label{fig:chain_dwf}
 Plot of the Debye-Waller factor $\langle u^2 \rangle$ as a function of end index $i$: chain ends are labeled $i=1$ (either bead for FJC and FRC or monomer for AAPS) and $i=N/2$ denotes pairs of middle-most beads or monomers for even $N$.
 Models from top to bottom are FJC, FRC, and AAPS,
 with $\langle u^2 \rangle$ reported at $T = 0.4$, $T = 0.6$, and $T = 500$ K, respectively.
 }
 %note: update xlabel with i definition
\end{figure}

Within the Fox-Flory and Ueberreiter-Kanig models, enhanced chain end mobility is expected to emerge from enhanced chain end free volume. 
Could these perspectives still hold at the lower level of chain free volume effects? 
To answer this question, we report local values of the Debye-Waller factor, 
$\langle u^2 \rangle$,
an established measure of dynamical free volume 
\cite{mckenzie-smith_relating_2022,
mckenzie-smith_explaining_2021,
hung_universal_2019,
puosi_fast_2019,
pazmino_betancourt_quantitative_2015,
ottochian_universal_2011,
ngai_why_2004,
starr_what_2002,
ngai_correlation_2001}
that specifically quantifies the space accessed by a segment within a cage of its neighbors.

Figure \ref{fig:chain_dwf} indicates that trends in $\langle u^2 \rangle$ with molecular weight and chain location are highly non\-universal. 
As with $T_g$, shifts in $\langle u^2 \rangle$ with molecular weight in the FJC model are nearly uniform through the whole chain, with at most a very weak chain end gradient. 
In contrast, AAPS exhibits a strong chain end connectivity gradient of enhanced $\langle u^2 \rangle$, 
while a significant whole-chain $\langle u^2 \rangle$ shift for this system only emerges for the smallest chain $N=3$. The FRC model appears to exhibit a mix of these two scenarios. This trend can be rationalized based on the strength of intra\-molecular correlations. 
In the FJC, a relatively small portion of the caging potential emerges from bonded correlations, whereas relatively strong intra\-molecular constraints contribute appreciably to caging in AAPS. 
Thus, relaxing these constraints at the chain end (having one less bond) has a relatively weak effect on the cage scale in the FJC, an intermediate strength effect in the FRC, and a relatively strong effect in the AAPS.

As shown by the FJC, it is evidently possible for a polymer to exhibit at least a 10\% drop in $T_g$ without a significant chain end effect in either $T_g$ or free volume. 
This indicates that chain end mobility or free volume effects cannot be the \textit{sole} driver of the $T_g$ molecular weight dependence. At the same time, results from the FRC and AAPS systems indicate that even the presence of an appreciable
%substantial 
enhancement in chain end dynamic free volume does not necessarily translate to a substantial suppression in chain end $T_g$.
%Logically, 
Any mechanism in which local free volume enhancements at chain ends enhance mobility must logically involve some finite range of these effects 
(in a manner similar to recent proposed formulations for free volume effects on glass formation near interfaces \cite{white_dynamics_2021}). 
The absence of a substantial local $T_g$ reduction near chain ends in these systems (or an appreciable $T_g$ gradient along the chain backbone) would thus seem to be inconsistent with an origin of the $T_g$ molecular weight dependence grounded to leading order in chain end free volume enhancements.

These results appear to indicate that neither chain end mobility enhancement nor chain end free volume enhancement can account to leading order for the dependence of $T_g$ upon molecular weight.
How then can we understand this dependence, 
which appears to take place nearly uniformly through the chain 
(see Figure \ref{fig:chain_index}) rather than to emanate from chain ends? 
As discussed above, 
several more recent models 
attribute this trend primarily to a dependence of
intra\-molecular phenomena such as chain stiffness \cite{mirigian_dynamical_2015} and conformation\-al barriers \cite{baker_cooperative_2022} on chain length. 
Because these polymer intra\-molecular phenomena 
do not intrinsically emerge from glassy physics but instead from chain structural factors such as chain connectivity and ster\-ic effects, 
they should be present at high temperatures well above the glass formation range. 
This is a key distinction from intrinsically glassy physical mechanisms for $T_g$ and molecular weight (e.g. proposed free volume effects), 
which would be expected to emerge on cooling into the glass formation range.

\begin{figure}[t]
 \includegraphics[width=\linewidth]{plot_ea_norm_fit_d_coop.png}
 \caption{\label{fig:ea}
 Plot of the high-temperature activation energy $E_a$, normalized by the longest chain denoted by $E_{a,\infty}$ for each model, as a function of chain length $N$ for FJC (blue circles), FRC (orange diamonds), and AAPS (green triangles).
 Error bars are standard errors on the high-temperature limit of activation energy.
 }
\end{figure}

As such, in Figure \ref{fig:ea} we plot the (normalized) high temperature activation energy $E_a$ for the three models (details of calculation in SI). This barrier is measured and defined above the onset temperature of super\-cooled, non-Arrhenius dynamics, and it thus reflects the innate relaxation behavior of the liquid in the absence of `glassy' effects.
Evidently, $E_a$ exhibits a strong dependence on $N$,
as strong if not stronger than that of $T_g$ (cf. Figure \ref{fig:relax_tg_n}). 
This $E_a$ dependence on $N$ is consistent with prior experimental work by Schmidtke et al. \cite{schmidtke_temperature_2015} Even within an entirely Arrhenius model of dynamics, this trend would be expected to drive a strong trend in $T_g$, in the absence of any additional physics at low temperature. 
Indeed, theories of glass formation commonly cast dynamics in the glass formation range as reflecting either an additive or multiplicative enhancement of $E_a$. It follows that molecular weight effects on $T_g$ may actually be a result of
molecular weight effects on intra\-molecular factors, as encoded already in $E_a$ at high temperature. 
These findings thus appear to qualitatively accord with newer scenarios primarily invoking intra\-molecular correlations or barrier rather than end effects. 
Furthermore, these trends in $E_a$ may also cascade into the molecular weight dependence (or lack thereof) of the kinetic fragility of glass formation (a measure of the thermal abruptness of the glass transition), as discussed in Schmidtke et al. \cite{schmidtke_temperature_2015} (cf. Figure 11(b)),
as well as Hung et al. \cite{hung_forecasting_2020} (cf. Figure 7),
and Novikov and Sokolov \cite{novikov_temperature_2022} (cf. Figure 24).

Overall, the data presented here suggests that chain end mixing, 
which underpins the formulations of Fox-Flory and Ueberreiter-Kanig, 
is not generally the leading-order mechanism driving molecular weight effects on $T_g$.
Across three models, chain ends do not exhibit sufficiently reduced $T_g$ values to account for polymer $T_g$ depending upon chain length $N$ on the basis of a chain end dilution effect. 
Moreover, we find that even chain-end alterations in free volume, 
which underpin the Fox-Flory and Ueberreiter-Kanig models, 
are non\-universal, being negligible in an least one model (FJC) while still exhibiting an appreciable $T_g$ molecular weight dependence. 
Even in the case of stiffer chains that exhibit a chain-end free volume enhancement, we find that this enhanced chain end free volume does \textit{not} lead to chain ends that relax correspondingly faster. 
Instead, we find that the alterations in dynamics that drive $T_g$ trends with molecular weight originate at high temperatures well above the glass formation range. These findings provide new support for recently proposed alternate perspectives
in which $T_g$'s dependence on molecular weight is instead driven intra\-molecular and whole-chain effects such as the Lindemann criterion for melting \cite{novikov_correlation_2013}, 
chain stiffness \cite{mirigian_dynamical_2015}, and 
the competition between segmental packing and intra\-molecular barriers \cite{baker_cooperative_2022}.

% The \nocite command causes all entries in a bibliography to be printed out
% whether or not they are actually referenced in the text. This is appropriate
% for the sample file to show the different styles of references, but authors
% most likely will not want to use it.
% \nocite{*}

\bibliography{apssamp}% Produces the bibliography via BibTeX.

%
% ****** End of file apssamp.tex ******

\end{document}

Potential final note:
Fox-Flory in the limit of small $N$ has a divergent slope,
whereas Ueberreiter-Kanig has a finite slope.
Ueberreiter and Kanig in fact note that the Fox-Flory prediction of the "apparent energy of activation" is far too small in comparison to the accepted value.
This note/discussion may be best discussed (if at all) on the first page, alongside discussion of the Novikov and Rossler as they also highlight that UK performs better than FF.
