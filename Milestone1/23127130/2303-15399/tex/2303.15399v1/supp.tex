\documentclass{article}
\usepackage[utf8]{inputenc}
\usepackage{graphicx}
\usepackage[a4paper, 
%total={6in, 8in}, 
margin=1in]{geometry}
\usepackage[
backend=biber,
style=numeric-comp,
sorting=none,
]{biblatex}
\usepackage{nameref}

\title{
%kill leading whitespace for first figure
%\vspace{-30mm}
Supplementary Information for: \\
Is the Molecular Weight Dependence of the Glass Transition Temperature Caused by a Chain End Effect?
}
%Co\-polymer Sequence Effects on the Glass Transition}
\author{William F. Drayer and David S. Simmons*}
\date{%
%    $^1$\\
Department of Chemical, Biological, and Materials Engineering, University of South Florida, Tampa, FL, 33620\\%
\vspace{5mm}
    $^*$Corresponding Author: dssimmons@usf.edu\\[2ex]%
%    \today
}
\renewcommand{\thetable}{S\arabic{table}}
\renewcommand{\thefigure}{S\arabic{figure}}
\renewcommand{\theequation}{S\arabic{equation}}

\addbibresource{supp.bib}

\begin{document}

\maketitle

\section*{Model Details}
We present data for three models to span from a fully-flexible bead-spring chain to a chemically realistic polymer.
The simplest model, the freely-jointed chain (FJC), is the standard finite extensible nonlinear elastic (FENE) model
\cite{kremer_dynamics_1990}
with FENE elastic constant $k=30$ and maximum bond elongation $R_0 = 1.5$.
To increase chain stiffness, we employ an angle potential to model a freely-rotating chain (FRC).
We set the bending constant $k_\theta=25$ and bending equilibrium angle $\cos\theta_0 = -0.333$ (which corresponds to 109.5 degrees), as done in prior work
\cite{bulacu_molecular-dynamics_2007}.
Both the FJC and FRC span chain lengths of $4 \le N \le 60$ beads.
Finally, to analyze a model with realistic chemical structure, we perform additional analysis of all-atom polystyrene (AAPS) simulations published in prior work
\cite{hung_universal_2019,hung_forecasting_2020}. 
Noteworthy details include the application of hydrogen mass re\-partitioning
\cite{hopkins_long-time-step_2015}
and AAPS chain lengths ranging from $3 \le N \le 400$ chemical repeat units, with a total chemical repeat unit count per simulation of approximately 800, e.g., for $N=5$ there are 160 chains and for $N=400$ there are two chains.

\section*{Quantifying Relaxation and its Depedence on Temperature}
\label{choose_tg}
Consistent with in prior work
\cite{hung_universal_2019,hung_forecasting_2020}
we quantify relaxation via the self-part of the intermediate scattering function,
using a wave\-number near the first peak in the structure factor
(7.07196 for both the freely-jointed and freely-rotation chain models and 1.19952 for the all-atom polystyrene model). Below, we show representative intermediate scattering function data vs time for each model with a chain length of $N=10$ (along with $N=100$ for AAPS),
plotting both the mean system data and chain end data 
(cf. Figure 1 of the main letter).
The slow relaxation process within these relaxation functions is then fit to a stretched exponential and the alpha relaxation time defined (in a convention consistent with many prior works) as the time that this function decays to 0.2.
Upon obtaining these relaxation time and temperature pairs,
$T_g$ is then quantified by fitting to the functional form suggested by Mauro et al. \cite{mauro_viscosity_2009}, which may be written as
\begin{equation}
\label{eq:myega}
\log{\tau} = \log{\tau_\infty} + \frac{A}{T}\exp{\frac{B}{T}}.
\end{equation}

We rewrite this self-consistently in terms of the glass transition temperature, as
\begin{equation}
\label{eq:myega_tg}
\log{\tau} = \log{\tau_\infty} 
+ \left(\log\tau_{g} - \log{\tau_\infty}\right) 
\frac{T_g}{T} 
\exp{\left(B
\left(\frac{1}{T} - \frac{1}{T_g}\right)
\right)}
\end{equation}
in order to obtain standard errors on $T_g$ from a least-squares regression upon choosing a timescale $\tau_g$ for $T_g$.
We then define $T_g$ at an extrapolated experimental timescale of 100 seconds for AAPS, to allow for experimental comparability. 
For the FJC and FRC, we instead employ a computational timescale $T_g$ convention of $\tau_g = 10^{4}$ dimensionless Lennard-Jones time units, consistent with much prior work in bead models.
This process is used for both mean system (cf. Figure 1, filled markers) and repeat unit specific (cf. Figure 1, crosses and open markers, and Figure 2, all markers) analysis.

\begin{figure}[htbp]
    \centering
    \includegraphics[height=0.3\textheight]{isfs_fjc_10.png}
    \caption{Self-part of the intermediate scattering function for select temperatures
    (roughly a sixth of all temperatures simulated) of the freely-jointed chain de\-ca\-mer ($N=10$);
    the hottest (red circles) and coldest (blue triangles) temperatures simulated are included.
    Chain ends are denoted as open symbols with the corresponding symbol shape as the mean system data.
    }
    \label{fig:isfs_fjc_10}
\end{figure}

\begin{figure}[htbp]
    \centering
    \includegraphics[height=0.3\textheight]{isfs_frc_10.png}
    \caption{Self-part of the intermediate scattering function for select temperatures
    (roughly a sixth of all temperatures simulated) of the freely-rotating chain de\-ca\-mer ($N=10$);
    the hottest (red circles) and coldest (blue triangles) temperatures simulated are included.
    Chain ends are denoted as open symbols with the corresponding symbol shape as the mean system data.
    }
    \label{fig:isfs_frc_10}
\end{figure}

\begin{figure}[htbp]
    \centering
    \includegraphics[height=0.3\textheight]{isfs_ps_10.png}
    \caption{Self-part of the intermediate scattering function for select temperatures
    (roughly a tenth of all temperatures simulated) of the all-atom polystyrene de\-ca\-mer ($N=10$);
    the hottest (red circles) and coldest (blue triangles) temperatures simulated are included.
    Chain ends are denoted as open symbols with the corresponding symbol shape as the mean system data.
    }
    \label{fig:isfs_ps_10}
\end{figure}

\begin{figure}[htbp]
    \centering
    \includegraphics[height=0.3\textheight]{isfs_ps_100.png}
    \caption{Self-part of the intermediate scattering function for select temperatures
    (roughly a tenth of all temperatures simulated) of the all-atom polystyrene he\-ca\-to\-mer ($N=100$);
    the hottest (red circles) and coldest (blue triangles) temperatures simulated are included.
    Chain ends are denoted as open symbols with the corresponding symbol shape as the mean system data.
    }
    \label{fig:isfs_ps_100}
\end{figure}


\clearpage

\section*{Relaxation Time and Inverse Temperature Data}

\subsection*{System Mean and Chain End Relaxation Data}

Here we report figures that demonstrate the chain-end effect 
(or lack thereof)
for relaxation times as a function of inverse temperature for a chain length of $N=10$ for each model 
(along with $N=100$ for AAPS).

\begin{figure}[htbp]
    \centering
    \includegraphics[height=0.3\textheight]{relax_fjc_10.png}
    \caption{Relaxation time (base 10 logarithm) as a function of inverse temperature for the freely-jointed de\-ca\-mer ($N=10$).
    The mean relaxation time is shown in blue circles, 
    and the chain end relaxation times are in crosses.}
    \label{fig:fjc-10-ends}
\end{figure}

\begin{figure}[htbp]
    \centering
    \includegraphics[height=0.3\textheight]{relax_frc_10.png}
    \caption{Relaxation time (base 10 logarithm) as a function of inverse temperature for the freely-rotating de\-ca\-mer ($N=10$).
    The mean relaxation time is shown in orange diamonds, 
    and the chain end relaxation times are in crosses.}
    \label{fig:frc-10-ends}
\end{figure}

\begin{figure}[htbp]
    \centering
    \includegraphics[height=0.3\textheight]{relax_ps_10.png}
    \caption{Relaxation time (base 10 logarithm of time in fem\-to\-seconds) as a function of inverse temperature for the all-atom polystyrene de\-ca\-mer ($N=10$).
    The mean relaxation time is shown in green triangles, 
    and the chain end relaxation times are in open triangles.}
    \label{fig:ps-10-ends}
\end{figure}

\begin{figure}[htbp]
    \centering
    \includegraphics[height=0.3\textheight]{relax_ps_100.png}
    \caption{Relaxation time (base 10 logarithm of time in fem\-to\-seconds) as a function of inverse temperature for the all-atom polystyrene he\-ca\-to\-mer ($N=100$).
    The mean relaxation time is shown in green triangles, 
    and the chain end relaxation times are in open triangles.}
    \label{fig:ps-100-ends}
\end{figure}

\newpage

\subsection*{Mean System Relaxation Data}

Here we report mean system relaxation times as a function of inverse temperature for each model and all chain lengths studied.

\begin{figure}[htbp]
    \centering
    \includegraphics[height=0.3\textheight]{relax_fjc.png}
    \caption{Relaxation time (base 10 logarithm of time in Lennard-Jones  units)  as a function of inverse temperature for all studied chain lengths of the freely-jointed chain.}
    \label{fig:relax_fjc}
\end{figure}

\begin{figure}[htbp]
    \centering
    \includegraphics[height=0.3\textheight]{relax_frc.png}
    \caption{Relaxation time (base 10 logarithm of time in Lennard-Jones units)  as a function of inverse temperature for all studied chain lengths of the freely-rotating chain.}
    \label{fig:relax_frc}
\end{figure}

\begin{figure}[htbp]
    \centering
    \includegraphics[height=0.3\textheight]{relax_ps.png}
    \caption{Relaxation time (base 10 logarithm of time in fem\-to\-seconds) as a function of inverse temperature for all studied chain lengths of the all-atom polystyrene (cf. Figure 9b from \cite{hung_forecasting_2020})}
    \label{fig:relax_ps}
\end{figure}

\newpage

\section*{The Debye-Waller Factor and its Dependence on Temperature}
\subsection*{Definition of Debye-Waller Factor}

As in prior work \cite{hung_universal_2019},
we here define the Debye-Waller factor $\langle u^2 \rangle$ near the cage onset time for the models studied here;
specifically, 
we choose the mean-square displacement at a time delta of 0.275 (or $\approx 10^{-0.55}$) dimensionless time units for both the FJC and FRC and 
0.711 ps (or $\approx 10^{-0.15}$ ps) for AAPS.
These timescale choices are shown below as vertical black lines in Figures \ref{fig:msd_fjc_10}-\ref{fig:msd_ps_100}
(wherein the temperatures are the same as in Figures \ref{fig:isfs_fjc_10}-\ref{fig:isfs_ps_100}).

\begin{figure}[htbp]
    \centering
    \includegraphics[height=0.3\textheight]{msd_fjc_10.png}
    \caption{Mean-square displacement data for select temperatures
    (roughly a tenth of all temperatures simulated) of the freely-jointed chain de\-ca\-mer ($N=10$);
    the hottest (red circles) and coldest (blue triangles) temperatures simulated are included.
    Chain ends are denoted as open symbols;
    chain middles are denoted as filled symbols.
    }
    \label{fig:msd_fjc_10}
\end{figure}

\begin{figure}[htbp]
    \centering
    \includegraphics[height=0.3\textheight]{msd_frc_10.png}
    \caption{Mean-square displacement data for select temperatures
    (roughly a tenth of all temperatures simulated) of the freely-rotating chain de\-ca\-mer ($N=10$);
    the hottest (red circles) and coldest (blue triangles) temperatures simulated are included.
    Chain ends are denoted as open symbols;
    chain middles are denoted as filled symbols.
    }
    \label{fig:msd_frc_10}
\end{figure}

\begin{figure}[htbp]
    \centering
    \includegraphics[height=0.3\textheight]{msd_ps_10.png}
    \caption{Mean-square displacement data for select temperatures
    (roughly a tenth of all temperatures simulated) of the all-atom polystyrene de\-ca\-mer ($N=10$);
    the hottest (red circles) and coldest (blue triangles) temperatures simulated are included.
    Chain ends are denoted as open symbols;
    chain middles are denoted as filled symbols.
    }
    \label{fig:msd_ps_10}
\end{figure}

\begin{figure}[htbp]
    \centering
    \includegraphics[height=0.3\textheight]{msd_ps_100.png}
    \caption{Mean-square displacement data for select temperatures
    (roughly a tenth of all temperatures simulated) of the all-atom polystyrene he\-ca\-to\-mer ($N=100$);
    the hottest (red circles) and coldest (blue triangles) temperatures simulated are included.
    Chain ends are denoted as open symbols;
    chain middles are denoted as filled symbols.
    }
    \label{fig:msd_ps_100}
\end{figure}

\subsection*{Dependence of Debye-Waller Factor on Temperature}

Overall (see Figures \ref{fig:dwf_fjc}-\ref{fig:dwf_ps_100} below),
$\langle u^2 \rangle$ is linear with respect to temperature (as expected; see \cite{hung_universal_2019}).
To compare across chain lengths,
we choose a temperature for each model that is at worst 
(for the longest chains, which demonstrate the highest $T_g$ values) 
a slight extrapolation for any given chain length.
We then utilize the line-of-best-fit of $\langle u^2 \rangle$ vs temperature to evaluate $\langle u^2 \rangle$ for that specific temperature for each model: 0.4 for the FJC, 0.6 for the FRC, and 500 K for AAPS.
These resulting values are reported in Figure 3 of the main text.

\begin{figure}[htbp]
    \centering
    \includegraphics[height=0.3\textheight]{dwf_temp_fjc_10.png}
    \caption{Debye-Waller factor as a function of temperature for the freely-jointed chain of length 10. 
    The chain ends are blue circles ($i = 1$); 
    the pair of monomers bonded to chain ends are orange diamonds ($i = 2$);
    the next pair are green pentagons ($i = 3$);
    the next pair are purple octagons ($i = 4$);
    and the yellow hexagons are the middle pair of segments ($i=5$).}
    \label{fig:dwf_fjc}
\end{figure}

\begin{figure}[htbp]
    \centering
    \includegraphics[height=0.3\textheight]{dwf_temp_frc_10.png}
    \caption{Debye-Waller factor as a function of temperature for the freely-rotating chain of length 10. 
    The chain ends are blue circles ($i = 1$); 
    the pair of monomers bonded to chain ends are orange diamonds ($i = 2$);
    the next pair are green pentagons ($i = 3$);
    the next pair are purple octagons ($i = 4$);
    and the yellow hexagons are the middle pair of segments ($i=5$).}
    \label{fig:dwf_frc}
\end{figure}

\begin{figure}[htbp]
    \centering
    \includegraphics[height=0.3\textheight]{dwf_temp_ps_10.png}
    \caption{Debye-Waller factor as a function of temperature for the all-atom polystyrene of length 10. 
    The chain ends are blue circles ($i = 1$); 
    the pair of monomers bonded to chain ends are orange diamonds ($i = 2$);
    the next pair are green pentagons ($i = 3$);
    the next pair are purple octagons ($i = 4$);
    and the yellow hexagons are the middle pair of segments ($i=5$).}
    \label{fig:dwf_ps_10}
\end{figure}

\begin{figure}[htbp]
    \centering
    \includegraphics[height=0.3\textheight]{dwf_temp_ps_100.png}
    \caption{Debye-Waller factor as a function of temperature for the all-atom polystyrene of length 100. 
    The chain ends are blue circles ($i = 1$); 
    the pair of monomers bonded to chain ends are orange diamonds ($i = 2$);
    the next pair are green pentagons ($i = 3$);
    the next pair are purple octagons ($i = 4$);
    and the middle pair of monomers are yellow hexagons ($i = 25$).
    Note that most pairs of monomers ($5 \le i \le 24$) have been omitted for clarity,
    and that noise is increased relative to Figure \ref{fig:dwf_ps_10} due to sampling
    (as monomer count in each AAPS simulation is roughly constant, an order of magnitude increase in chain length corresponds to an order of magnitude less pairs of, e.g., chain ends).
    }
    \label{fig:dwf_ps_100}
\end{figure}

\clearpage

\section*{Determination of High-Temperature Activation Energies}

While we choose to fit relaxation data and thus define $T_g$ using the functional form from Mauro et al. \cite{mauro_viscosity_2009},
as discussed above in \textbf{\nameref{choose_tg}},
the functional form proposed by Schmidtke et al. \cite{schmidtke_temperature_2015},
\begin{equation}
\label{eq:coop}
\log{\tau} = \log{\tau_\infty} + 
\frac{E_\infty}{T} \left( 1 + \exp{ 
\left( -\mu \left( \frac{T}{E_\infty} -b \right) \right)
} \right),  
\end{equation}
with its additional fit parameter analytically recovers Arrhenius behavior in the high temperature limit.
As such, we utilize this form to robustly define high temperature activation energies.
Specifically,
we perform lines-of-best-fit to the relaxation data shown in Figures 
\ref{fig:relax_fjc}-\ref{fig:relax_ps}
and plot the slopes as a function of the width of the domain fitted,
shown below in Figures
\ref{fig:fjc_ea}-\ref{fig:ps_ea}.
In the high temperature limit, Arrhenius behavior should be recovered:
the slope $s \equiv \frac{d \log \tau}{d (1/T)}$ should plateau to the limiting value of $E_\infty$ as $\Delta (1/T) \rightarrow 0$.
We note that as this study focuses on $T_g$,
high temperature data is sparse and not the focus of this work.
Nevertheless,
we can fit the analytical derivative of Equation \ref{eq:coop},
\begin{equation}
\label{eq:d-coop}
s = E_\infty 
\left(
1 + 
\left(\frac{\mu}{E_\infty} T + 1\right)
\exp{\left( \mu b - \frac{\mu}{E_\infty} T \right)}
\right),  
\end{equation}
to the data in order to extract a high temperature plateau value from the data.
We further note that longer chain lengths typically have less high temperature data,
resulting in fit values that are systematically higher,
but we expect this does not dominate the trend shown in Figure 4 from the main letter.
In particular: 
the trend is strong and largely monotonic;
three smallest chain lengths for AAPS (Figure \ref{fig:ps_ea}) have high confidence in plateau values;
and the trend is reported experimentally by Schmidtke et al. \cite{schmidtke_temperature_2015} 
(cf. polymers poly(butadiene) (PI) and poly(dimethylsiloxane) (PDMS) in Figures 1 and 4).
Finally,
due to the ambiguity of the $\infty$ symbol as representing
either Arrhenius slope 
(as in Equations \ref{eq:coop} and \ref{eq:d-coop}) 
or 
the infinite molecular weight limit 
(such as how $T_{g,\infty}$ denotes $T_g$ for an infinite molecular weight polymer),
for the main letter and in particular Figure 4,
we use $E_a$ (instead of $E_\infty$ as shown above, which is consistent with \cite{schmidtke_temperature_2015}) to denote the high temperature activation energy
(viz., slope)
and $E_{a,\infty}$ to denote this value for the infinite molecular weight limit.

\begin{figure}[htbp]
    \centering
    \includegraphics[height=0.3\textheight]{plot_ea_deltas_fjc.png}
    \caption{Slope of Figure \ref{fig:relax_fjc} (the FJC model) as a function of domain width, 
    $\Delta (1/T)$, for chain lengths shown in the legend.
    Curves are fits to equation \ref{eq:d-coop}.
    }
    \label{fig:fjc_ea}
\end{figure}

\begin{figure}[htbp]
    \centering
    \includegraphics[height=0.3\textheight]{plot_ea_deltas_frc.png}
    \caption{Slope of Figure \ref{fig:relax_frc} (the FRC model) as a function of domain width, 
    $\Delta (1/T)$ , for chain lengths shown in the legend.
    Curves are fits to equation \ref{eq:d-coop}.
    }
    \label{fig:frc_ea}
\end{figure}

\begin{figure}[htbp]
    \centering
    \includegraphics[height=0.3\textheight]{plot_ea_deltas_ps.png}
    \caption{Slope of Figure \ref{fig:relax_ps} (the AAPS model) as a function of domain width,
    $\Delta (1/T)$ , for chain lengths shown in the legend.
    Curves are fits to equation \ref{eq:d-coop}.}
    \label{fig:ps_ea}
\end{figure}

\clearpage

\printbibliography

\end{document}
