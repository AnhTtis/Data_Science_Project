% This is file JF$\mathrm{M_2}$esam.tex
% first release v1.0, 20th October 1996
%       release v1.01, 29th October 1996
%       release v1.1, 25th June 1997
%       release v2.0, 27th July 2004
%       release v3.0, 16th July 2014
%   (based on JFMsampl.tex v1.3 for LaTeX2.09)
% Copyright (C) 1996, 1997, 2014 Cambridge University Press

\documentclass{jfm}
\usepackage{graphicx}
\usepackage{natbib}
\usepackage{color}
\usepackage{enumitem}
\usepackage{latexsym}
\usepackage{subfig}
\usepackage{natbib}
\usepackage{upgreek}
\usepackage{mathrsfs}
\usepackage{float}
\usepackage{caption}
\usepackage{soul}
\usepackage{textcomp}
\usepackage{stmaryrd}
\usepackage{amsmath}
\usepackage{mathtools}
\usepackage{amsfonts}
\usepackage{epstopdf, epsfig}
\usepackage{amsmath,etoolbox}
\usepackage{bigints}
\usepackage{lipsum}
\usepackage{physics}
\usepackage[normalem]{ulem}

%\usepackage{lipsum}                    % Dummytext
\usepackage{xargs}                      % Use more than one optional parameter in a new commands
\usepackage[pdftex,dvipsnames]{xcolor}  % Coloured text etc.
\usepackage[colorinlistoftodos,prependcaption,textsize=small]{todonotes}


\AtBeginEnvironment{pmatrix}{\setlength{\arraycolsep}{2pt}}
%\usepackage{todonotes}
%%% Example macros (some are not used in this sample file) %%%



\def\dd{{\rm d}}
\def\ee{{\rm e}}
\def\ii{{\rm i}}
%\def\mathrm{c.c}{{\rm c.\,c.}}

\def\norm#1{\left\|#1\right\|}
\newtheorem{lemma}{Lemma}
\newtheorem{corollary}{Corollary}


\title{5 wave interactions in internal gravity waves}

 \author{Saranraj Gururaj\aff{1}\corresp{\email{gmsaranraj@gmail.com}}  \and Anirban Guha\aff{1} }

\affiliation{
\aff{1} School of Science and Engineering, University of Dundee, DD1 4HN, U.K.
}
 

\begin{document}

\maketitle

\begin{abstract}
%Wave-wave interactions play a major role in the energy cascade of internal gravity waves in the oceans. 
% In this paper, we use multiple scale analysis to study wave-wave interactions of two plane parent waves with the same frequency and wavevector norm co-existing in a region of constant background stratification. Specifically, we explore the 5-wave system instability (5WSI) that contains two parent waves and three daughter waves. The five waves form two different triads, with one daughter wave being a part of both triads. 

% In this paper, we use multiple scale analysis to study a 5-wave system composed of two different internal gravity wave triads. Each of these triads consist of a parent wave and two daughter waves, with one daughter wave common between the two triads. The parent waves are assumed to have the same frequency and wavevector norm co-existing in a region of constant background stratification.[insert a statement about physical significance]. 
% %Such a 5-wave system can 
% We consider two cases where the parent waves are confined to the same vertical plane: Case 1(2) has parent waves with the same horizontal (vertical) wavenumber but with different vertical (horizontal) wavenumber. For both cases, the 5WSI is more dominant than triads for $f/\omega_1\gtrapprox0.3$, where $\omega_1$ and $f$ are the parent wave and the local Coriolis frequency, respectively. For $f/\omega_1\gtrapprox0.3$, the common daughter wave's frequency is $\approx \omega_1-f $ and $f$ for Cases 1 and 2, respectively. For oblique parent waves, 5WSIs become more dominant as the angle between the horizontal wavevectors of the parent waves (denoted by $\theta$) is decreased. Moreover, for any $\theta$, 5WSIs are more dominant than triads for $f/\omega_1\gtrapprox0.3$. 
% Numerical simulations match the theoretical growth rates of 5WSIs for a wide range of latitudes except when $f/\omega_1\approx0.5$ (critical latitude). More than three daughter waves are forced by the two parent waves when $f/\omega_1\approx0.5$. We formulate a reduced order model which shows that for any $\theta$, the maximum growth rate near the critical latitude is approximately twice the maximum growth rate of all triads.

%%%%%%%%%%%%%%%%%%%%%%%%%%%%%%%%%%%%%%%%%%%%%%%%%% 

We use multiple scale analysis to study a 5-wave system (5WS) composed of two different internal gravity wave triads.
Each of these triads consists of a parent wave and two daughter waves, with one daughter wave common between the two triads. The parent waves are assumed to have the same frequency and wavevector norm co-existing in a region of constant background stratification.  {Such 5-wave systems may emerge in oceans, for example, via tide-topography interactions, generating multiple parent internal waves that overlap.}
Two 2D cases are considered: Case 1(2) has parent waves with the same horizontal (vertical) wavenumber but with different vertical (horizontal) wavenumber. For both cases, the 5WS is more unstable than triads for $f/\omega_1\gtrapprox0.3$, where $\omega_1$ and $f$ are the parent wave and the local Coriolis frequency, respectively. For $f/\omega_1\gtrapprox0.3$, the common daughter wave's frequency is $\approx \omega_1-f $ and $f$ respectively for Cases 1 and 2. For 3D cases, 5WSs become more unstable as the angle ($\theta$) between the horizontal wavevectors of the parent waves  is decreased. Moreover, for any $\theta$, 5WSs have higher growth rates than triads for $f/\omega_1\gtrapprox0.3$. 
Numerical simulations match the theoretical growth rates of 5WSs for a wide range of latitudes, except when $f/\omega_1\approx0.5$ (critical latitude). More than three daughter waves are forced by the two parent waves when $f/\omega_1\approx0.5$. We formulate a reduced order model which shows that for any $\theta$, the maximum growth rate near the critical latitude is approximately twice the maximum growth rate of all triads.

%%%%%%%%%%%%%%%%%%%%%%%%%%%%%%%%%%%%%%%%%%%%%%%%%%%%%%%%%%%%

% For parent waves with wavevectors $(k_1,0,m_1)$ and $(k_1,0,-m_1)$, the 5WSI is only possible when $\omega_3 \approx \omega_1 - f$, where $\omega_1,\omega_3,f$ are the parent wave, common daughter wave, and the local Coriolis frequency, respectively. 5WSI is more dominant than triads for $f/\omega_1\gtrapprox0.3$. For parent waves with wavevectors $(k_1,0,m_1)$ and $(-k_1,0,m_1)$, once again triads are less dominant than the 5WSI for $f/\omega_1\gtrapprox0.3$, and $\omega_3\approx f$ is observed in the most unstable 5-wave system.
%Moreover, the reduced order model showed similar results for parent waves that are not on the same vertical plane. 

 

\end{abstract}


\begin{keywords} 

\end{keywords}
%\todo{not too keen about the word "oblique" in the abstract. Rather 3D makes more sense. Oblique can mean many things. One can think IGW is oblique in x-z plane=> sir i used oblique because akylas consistently seems to use it. hmm..Maybe 3D is better, i can then simply write, "For 3D cases"=> Yes=> }
\section{Introduction}

Internal waves play a major role in sustaining Meridional Overturning Circulation by causing diapycnal mixing \citep{MUNK1998,ferrari_2008}. Wave-wave interaction is estimated to be one of the most dominant mechanisms through which internal waves' energy cascades to small length scales \citep{DE_2019}, where it can cause mixing. As a result, understanding wave-wave interactions can be important to model the internal waves' energy cascade.


The stability of a plane internal gravity wave has been studied extensively. A primary/parent internal wave with small steepness is unstable to secondary (daughter) waves through triad interactions if the secondary waves' frequencies are lesser than the parent wave's frequency \citep{hasselmann_1967}. Moreover, the three waves' frequencies and wavevectors should also satisfy the resonant triad conditions: $\mathbf{k}_{1} = \mathbf{k}_{2} + \mathbf{k}_{3}$ and $\omega_1 = \omega_2 + \omega_3$ \citep{thorpe_1966,davis_acrivos_1967,hasselmann_1967}, where daughter waves are denoted by subscripts 2 and 3, while the parent wave by subscript 1. For parent waves with small steepness, a 2D stability analysis is sufficient to find the most dominant instability \citep{klostermeyer}, and the most unstable daughter wave combination depends on $\omega_1/N$ \citep{Sonmor_1997}, kinematic viscosity $(\nu)$ \citep{bourget,bourget_width_2014} and Coriolis frequency ($f$) \citep{young,maurer_joubaud_odier_2016}. Without any rotational effects and under inviscid conditions, for $\omega_1/N < 0.68$, the wavevectors of the most unstable daughter wave combination satisfy $|\mathbf{k}_{3}| < |\mathbf{k}_{1}| < |\mathbf{k}_{2}|$. However, for $\omega_1/N > 0.68$, the most unstable daughter waves' wavevectors satisfy $ |\mathbf{k}_{2}| \approx |\mathbf{k}_{3}| \gg |\mathbf{k}_{1}|$. This instability is called as Parametric Subharmonic Instability (PSI) \citep{Mackinnon_2005,young}. PSI is a special type of triad interaction where $\omega_2 \approx \omega_3 \approx \omega_1/2$. 
%However, instability of finite amplitude internal waves (steepness > $1$) is a 3-dimensional process  \citep{klostermeyer,riley1996}.
%In the oceans, viscosity has very little effect on large length scale waves. However, in laboratory conditions, viscosity can play a major role in deciding the specific combination of waves that drain energy from the parent wave \citep{bourget}. 

For internal wave triads, rotational effects can be very important for a wide range of latitudes and especially near the critical latitude. Near the critical latitude (where $f \approx \omega_1/2$), for any $\omega_1/N$, the primary wave gives its energy to waves whose frequency is close to the inertial frequency.  Moreover, the inertial waves have very small vertical length scales which can lead to increased kinetic energy dissipation \citep{richet_2018}. Semidiurnal mode-1 internal waves have been observed to lose a non-negligible portion of their energy as they pass through the critical latitude \citep{Mackinnon_2005,Alford_2007,Haze_2011}. Moreover, when semidiurnal internal wave modes interact with an ambient wave field that follows Garrett-Munk spectrum, their decay is fastest near the critical latitude \citep{hibiya_1998,decay_onuki,olbers_2020}.

%It was observed that even when the growth rate is slightly lower in the critical latitude (in comparison to equator), triad interactions causes increased dissipation. %Recently, stability of internal wave beams have been studied using analytical techniques \cite, simulations, and experiments.Unlike in the case of plane waves, the perturbations can grow to a significant amplitude only in cases where they can stay in contact with primary internal wave beam for long enough.

In this paper, we study the stability of two weakly nonlinear plane parent waves that coexist in a region. The motivation for this study stems from the fact that parent internal waves generated in different locations often meet/overlap in the oceans. For example, tide-topography interactions result in the generation of internal waves that propagate in horizontally opposite directions, and these waves overlap/coexist above the topography, c.f. \citet[figure 7]{sonya_nik_2011}.
% the topographic generation of internal waves often results in two internal wave beams intersecting each other. % For example, the topographic generation of internal waves often results in two internal wave beams intersecting each other. 
When two energetic parent waves meet in a region, they can resonantly interact with each other. Internal wave beam collision is an example of such direct interaction between the parent waves, and it has been studied extensively over the last few decades \citep{tabaei_akylas_lamb_2005,Jiang_Marcus_2009,akylas_karimi_2012}. 
%However, we do not focus on scenarios where the parent waves directly interact with each other. 
Parent waves, however, do not always resonantly interact with each other and form a triad. In the absence of direct interaction, each  parent wave would still be susceptible to triad interactions leading to the growth of daughter waves, and this is the setting explored in this paper. Specifically, we focus on the 5-wave system instability. In this instability, five waves (two parent waves and three daughter waves) are involved, and two distinct triads are formed between the five waves. Note that this implies one daughter wave is forced by both parent waves and is a part of two different triads. Some examples of parent waves overlapping are given in figures \ref{fig:5_wave_triad_schematic}(a)--\ref{fig:5_wave_triad_schematic}(b), and the examples shown can easily occur in the oceans when internal waves are generated by tide-topography interactions. In both figures, the region enclosed inside the green box would be a potential location for a 5-wave system. The wavevector and frequency conditions satisfied in a 5-wave system is given in figure \ref{fig:5_wave_triad_schematic}(c).

 % 5-wave interactions have been studied previously in the context of internal gravity waves \citep{common_parent_wave,5_wave_common_daughter_wave}. \cite{common_parent_wave} focuses on systems where the same parent wave generates four different daughter waves, which is not the focus here. \cite{5_wave_common_daughter_wave} explores the dynamical nature of the 5-wave interactions where a common daughter wave is present. They focus on a 2 dimensional system without including rotational effects and with the two parent waves having different frequencies. Also, the growth rate of 5-wave systems was not studied. In this chapter, we consider a 3-dimensional setting, and the primary focus is on the growth rates of the daughter waves in the 5-wave system. Conditions in which the 5-wave system is a faster instability than the 3-wave system are explored in detail. The parent waves' frequency is always assumed to be the same. All possible 5-wave systems are identified for different combination of parent wavevectors. Internal wave beams generated by a topography has the same frequency and often intersect other beams on their trajectory. Hence the assumption made has a geophysical significance.   Moreover, when internal waves get reflected on a surface, the overlap region between the incident and the reflected wave can potentially serve as a region where 5-wave interaction occurs.

\begin{figure}
 \centering{\includegraphics[width=1\textwidth]{Wavevector_beam_schematic_combination.png}}
 \caption{ Examples of different orientations of propagating parent waves: in (a) vertically 
 and (b) horizontally opposite directions, with the intersection region marked in green.  (c) Frequency and wavevector triad conditions that are satisfied between the 5 waves that are involved in the interaction region. Waves 1 and 5 are parent waves, while waves 2,3, and 4 are daughter waves, with wave-3 being the common daughter wave. }
  \label{fig:5_wave_triad_schematic}
\end{figure} 

In the context of internal gravity waves,  5-wave systems have been studied recently \citep{common_parent_wave,5_wave_common_daughter_wave}. \cite{common_parent_wave} focus on 5-wave systems where the same parent wave generates four different daughter waves, which is not the focus of this paper. \cite{5_wave_common_daughter_wave} explore 5-wave interactions that consist of two parent waves and three daughter waves, but their focus is on rogue wave generation. They study the 5-wave systems in a 2D setting without the rotational effects. Moreover, no detailed study was conducted on the growth rates. In this paper, we consider a 3D setting with rotational effects, which is observed to be important in our case. The primary focus is on the growth rates of the daughter waves and to understand scenarios in which the 5-wave system instability is faster than the 3-wave system instability (standard triads). In our study, the frequencies of the two parent waves are always assumed to be the same, and this assumption can be important in an oceanographic context since internal waves generated by the same tide have the same frequency. The paper is organized as follows. In \S \ref{Section:2}, we use multiple scale analysis to simplify the 3D, Boussinesq Navier-Stokes equations in the $f-$plane and derive the wave amplitude equations. Expressions for growth rates are provided.  In \S \ref{Section:3}, theoretical comparisons between the growth rates of 3-wave systems and 5-wave systems for different combinations of parent waves are provided. In \S \ref{Section:4}, numerical validations are provided for the 5-wave systems, and specific focus is also given to the fate of the parent waves near the critical latitude. The paper is summarized in \S \ref{Section:5}.
    
% In this chapter, we study the growth rates in a 5-wave triad system. The motivation for this study is that in the ocean, often times two different parent internal waves, that are generated in different locations cross/propagate past each other. This can be because two low mode internal waves that are generated from two different locations meet,or also topographic generation of internal waves result in multiple internal wave beams that intersect each other. This is shown in figure (). In such scenarios, the parent waves themselves can resonantly interact with each other and force a third wave, or a parent wave, without the influence of the other parent wave, can force two daughter waves on its own through wave-wave interactions.

% In such scenarios, significant studies focus on how the parent waves themselves can resonantly interact with each other and force a third wave. Moreover, a parent wave, without the influence of the other parent wave, can force two daughter waves on its own through wave-wave interactions which would behave similar to a classical/standard triad instability.

% However, a third scenario is also possible: when two parents meet each other, they can force a common daughter wave. Note that in this system, 5 waves (two parent waves and three daughter waves) would be involved in total. 5 wave interactions have been studied previously in the context of internal gravity waves \citep{common_parent_wave,5_wave_common_daughter_wave}. \cite{common_parent_wave} focuses on a system where the same parent wave generates four different daughter waves, which is not the focus here. \cite{5_wave_common_daughter_wave} explores the dynamical nature of the 5-wave interactions where a common daughter wave is present. They focus on a 2 dimensional system, without considering rotational effects, where the two parent waves' frequency is different. Moreover, no detailed study was done on the growth rates. In this study, a 3 dimensional setting is considered and the primary focus is on the growth rates of the daughter waves, and to see when the 5 wave system is susceptible to a faster instability in comparison to a three wave system. Moreover, the frequency of the two parent waves are assumed to be always the same, and all possible 5-wave systems are identified. Note this system is much more realisable in an oceanographic context: internal wave beams generated by a topography has the same frequency, and often intersect other beams on their trajectory. Moreover, when internal waves get reflected on a surface, the overlap region between the incident and the reflected wave can potentially serve as a region where 5-wave interaction occurs.

\section{{Governing equations}} \label{Section:2}

The 3D, incompressible, Boussinesq, Navier-Stokes equations in the $f-$plane in primitive variables are given by
% \begin{align}
%     \frac{\partial u}{\partial t} + u\frac{\partial u}{\partial x} + v\frac{\partial u}{\partial y} + w\frac{\partial u}{\partial z} - fv &= -\frac{1}{\rho_0}\frac{\partial p'}{\partial x} + \nu \Delta u \label{eqn:u1_S4} \\
%     \frac{\partial v}{\partial t} + u\frac{\partial v}{\partial x} + v\frac{\partial v}{\partial y}  + w\frac{\partial v}{\partial z} + fu &= -\frac{1}{\rho_0}\frac{\partial p'}{\partial y} +  \nu \Delta v \label{eqn:v1_S4} \\
%     \frac{\partial w}{\partial t} + u\frac{\partial w}{\partial x} + v\frac{\partial w}{\partial y}  + w\frac{\partial w}{\partial z}  &= -\frac{1}{\rho_0}\frac{\partial p'}{\partial z} + b + \nu \Delta w \label{eqn:w1_S4}\\ 
%     \frac{\partial b}{\partial t} + u\frac{\partial b}{\partial x} + v\frac{\partial b}{\partial y}  + w\frac{\partial b}{\partial z} + N^2w &=   \kappa \Delta b  \label{eqn:b1_S4}\\
%     \frac{\partial u}{\partial x} + \frac{\partial v}{\partial y}  + \frac{\partial w}{\partial z} &= 0
% \label{eqn:Continuity_S4}
%  \end{align}
\begin{subequations}
 \begin{align}
    \frac{ \textnormal{D} \textbf{u}}{ \textnormal{D} t}  + f\hat{z} \times \textbf{u} &= -\frac{1}{\rho_0} \nabla p  + b \hat{z} + \nu \Delta \textbf{u}, \label{eqn:total_mom_equation} \\
    \frac{ \textnormal{D} b}{\textnormal{D} t} + N^2w &=   \kappa \Delta b,  \label{eqn:boyancy_equation}\\
    \nabla.\textbf{u} &= 0.
\label{eqn:Continuity}
 \end{align}
 \end{subequations}
Here $\textbf{u} = (u,v,w)$, where the components respectively denote the zonal, meridional, and vertical velocities. Moreover,  $f$ is the local Coriolis frequency,  $\rho_0 \approx 1000 \,\textnormal{kg}\,\textnormal{m}^{-3}$ is the reference density, $p$  is the perturbation pressure,  $b$ is the buoyancy perturbation, $\nu$ is the kinematic viscosity, $N$ is the 
{background} buoyancy frequency, and $\kappa$ is the diffusion coefficient. 
%\todo{I have commented out the traditional approximation line=>}
%Moreover, the traditional approximation for the rotational effects is used in the above governing equations.
The operator $\Delta$ is defined as  $\Delta \equiv \partial^2/\partial x^2 + \partial^2/\partial y^2  + \partial^2/\partial z^2$, while ${ \textnormal{D} }/{ \textnormal{D} t}$ is the material derivative.

%\todo{why not use p as pert pressure=>sorry yeah..I have completed the subsections in that other paper, but i want to have one more look before you have a look. Also this paper results wise is done (Probably wont do 3d simulations), I just need to make writing is fine=>This paper is much better written than the scattering paper=>ok yeah..hmm i will go through the scattering paper more. I will let you know wjat part of this paper is ready as soon as i feel it is ready.=> sure..but make sure to read what youhave written multiple times=>yes sure.=>}
We intend to study wave-wave interactions using  multiple-scale analysis. To this end, we combine equations \eqref{eqn:total_mom_equation}--\eqref{eqn:Continuity} into a single equation. 
% To this end, ${\partial }/{\partial t} \eqref{eqn:w1_S4} + \eqref{eqn:b1_S4}$ is done, which leads to:
% \begin{equation}
%  \frac{\partial^2 w}{\partial t^2} + \frac{\partial}{\partial t} \left( \Vec{u}.\nabla w \right) + N^2w + \Vec{u}.\nabla b = \frac{\partial}{\partial t} \left( -\frac{1}{\rho_0}\frac{\partial p'}{\partial z} + \nu \Delta w\right) + \kappa \Delta b   \label{eqn:w1_b1_combo_S4}
% \end{equation}
% Now to cancel the pressure term, $\nabla^2_h \eqref{eqn:w1_b1_combo_S4} - \partial/\partial t(\partial/\partial x(\partial/\partial z \eqref{eqn:u1_S4} ))  - \partial/\partial t(\partial/\partial y(\partial/\partial z \eqref{eqn:v1_S4} ))$ leads to (after some simplification),
% \begin{align}
%  &\frac{\partial^2 (\nabla^2_h w)}{\partial t^2} + \frac{\partial}{\partial t} \nabla^2_h\left( \Vec{u}.\nabla w \right) + N^2(\nabla^2_h w) \hspace{0.4cm}= \hspace{0.4cm} \nu\frac{\partial}{\partial t} \left( \Delta \nabla^2_h w - (\Delta u)_{xz} - (\Delta v)_{yz} \right) \nonumber \\
%   & + \nabla^2_h (\Vec{u}.\nabla b) - (\Vec{u}.\nabla u)_{xzt} - (\Vec{u}.\nabla v)_{yzt} \hspace{1.4cm}   +\nabla^2_h(\kappa \Delta b) - f (v_{xzt} - u_{yzt})  \label{eqn:u1_v1_w1_b1_combo_S4}
% \end{align}
% where $\nabla_h^2 \equiv \partial^2/\partial x^2  + \partial^2/\partial y^2 $. Using \eqref{eqn:u1_S4}, and \eqref{eqn:v1_S4}, the term $- f (v_{xzt} - u_{yzt})$ can be re-written as,
% \begin{equation}
%  f ( u_{yzt} - v_{xzt} ) = -f(\Vec{u}.\nabla u)_{yz} + f(\Vec{u}.\nabla v)_{xz} + f^2v_{yz} + f^2u_{xz} + f \nu (\Delta u)_{yz} - f \nu (\Delta v)_{xz} 
% \end{equation}
% Note that $f^2v_{yz} + f^2u_{xz} =  -f^2w_{zz}$. As a result, \eqref{eqn:u1_v1_w1_b1_combo_S4} can be simplified further to,
After some simple manipulations, the single equation describing the evolution of vertical velocity is written in a compact form: 
% \begin{align}
%  &{ (\nabla^2_h w)}_{tt} + w_{zztt}  + N^2(\nabla^2_h w) + f^2w_{zz} + \textnormal{NLT} \hspace{0.4cm}= \hspace{0.4cm} \textnormal{VT}    \label{eqn:final_w_S4}
% \end{align}
\begin{equation}
     \frac{\partial^2 }{\partial t^2} (\Delta w)  + N^2(\nabla^2_h w) + f^2\frac{\partial^2 w}{\partial z^2} + \textnormal{NLT} \hspace{0.4cm}= \hspace{0.4cm} \textnormal{VT},    \label{eqn:final_w_S4}
\end{equation}
%  The notation $A_j \equiv \partial A/\partial j$ is used for convenience. $\textnormal{NLT}$ denotes all the nonlinear terms, and is given by,
where $\nabla_h^2 \equiv \partial^2/\partial x^2 + \partial^2/\partial y^2$. $\textnormal{NLT}$ denotes all the nonlinear terms:
% \begin{align}
%     \textnormal{NLT}=  \nabla^2_h\left( \textbf{u}.\nabla w \right)_t + \nabla^2_h (\textbf{u}.\nabla b) - (\textbf{u}.\nabla u)_{xzt} - (\textbf{u}.\nabla v)_{yzt} + f(\textbf{u}.\nabla u)_{yz} - f(\textbf{u}.\nabla v)_{xz}
%     \label{eqn:NLT_s4_definition}
% \end{align}
\begin{align}
    \textnormal{NLT} &=  \nabla^2_h\frac{\partial (\textbf{u}.\nabla w)}{\partial t } + \nabla^2_h (\textbf{u}.\nabla b) - \frac{\partial^3 (\textbf{u}.\nabla u)}{\partial x \partial z \partial t } \nonumber \\
    &+     f\frac{\partial^2 (\textbf{u}.\nabla u)}{\partial y \partial z } - f\frac{\partial^2 (\textbf{u}.\nabla v)}{\partial x \partial z }- \frac{\partial^3 (\textbf{u}.\nabla v)}{\partial y \partial z \partial t }. 
    \label{eqn:NLT_s4_definition}
\end{align}
Moreover, $\textnormal{VT}$ denotes viscous and molecular diffusion terms:
\begin{equation}
    \textnormal{VT}= \nu\frac{\partial}{\partial t} \left( \Delta^2 w  \right) + \nabla^2_h(\kappa \Delta b) + f \nu \frac{\partial^2 (\Delta u)}{\partial y \partial z }   - f \nu \frac{\partial^2 (\Delta v)}{\partial x \partial z }.
    \label{eqn:VT_s4_definition}
\end{equation}
For simplicity, we assume $\kappa=0$. Furthermore, we mainly focus on plane waves. Similar to the procedure used in \cite{bourget}, the vertical velocity of the $j-$th wave $(j=1,2,\ldots,5)$ is assumed to be a product of a rapidly varying phase and an amplitude that slowly varies in time. Mathematically this can be written as
\begin{equation}
    w_j(x,y,z,t)  =   a_j(\epsilon_t t) \exp{[\textnormal{i}(k_jx + l_jy + m_jz -\omega_j t)]} + \mathrm{c.c}.,
    \label{eqn:w_definition}
\end{equation}
where $k_j,l_j,m_j,$ and $\omega_j$ are respectively the zonal wavenumber, meridional wavenumber, vertical wavenumber, and frequency of the $j-$internal wave. {‘c.c.’ denotes the complex conjugate}. The amplitude is assumed to evolve on a slow time scale $\epsilon_t t$, where $\epsilon_t$ is a small parameter. 
%Moreover, $a_j$ itself is $\mathcal{O}(\epsilon_a)$, where \textcolor{blue}{$\epsilon_a \ll 1$, thereby enforcing wave steepness to remain a small quantity \citep{koudella_staquet_2006}.}
Moreover, $a_j$ itself is $\mathcal{O}(\epsilon_a)$, where  {$\epsilon_a \ll 1$. For weakly nonlinear wave-wave interactions, wave steepness should be a small quantity \citep{koudella_staquet_2006}, and $\epsilon_a$ is chosen accordingly.}
%Please note that here the amplitude only varies in time unlike the previous section. This is because the primary focus is here on growth rates, and the temporal variation is sufficient for that. Significant focus is not given to the finite width effects that can arise when the amplitude is also a function of space. 
On substituting \eqref{eqn:w_definition} in \eqref{eqn:final_w_S4}, at the leading order ($\mathcal{O}(\epsilon_a)$) we obtain the dispersion relation in 3D: 
% At leading order ($\mathcal{O}(\epsilon_a)$), this leads to the dispersion relation of internal waves in 3 dimensions, and it is given by,
\begin{equation}
    \omega_j^2 = \frac{N^2(k_j^2 + l_j^2) + f^2m_j^2}{k_j^2 + l_j^2 + m_j^2}.
    \label{eqn:3d_disp_relation}
\end{equation}
All 5 waves involved in the interaction must satisfy this dispersion relation.

Energy transfer between the waves due to weakly nonlinear wave-wave interactions occurs at $\mathcal{O}(\epsilon_a^2)$. For the $j$-th wave, the amplitude evolution equation reads
\begin{align}
   \mathcal{D}_j \frac{\partial a_j}{\partial t}    &= - \textnormal{NLT}_j + \textnormal{VT}_j, 
\label{eqn:o_epsilon_2_equation_j_wave} 
\end{align}
where $\mathcal{D}_j \equiv 2\textnormal{i} \omega_j (k_j^2+l_j^2+m_j^2)$ is defined for convenience. $\textnormal{NLT}_j$ and $\textnormal{VT}_j$ represent all the nonlinear and viscous terms with the phase of the $j-$th wave, respectively. The expression for $\textnormal{VT}_j$ is given by
\begin{equation}
    \textnormal{VT}_j = -\mathcal{D}_j{\nu}/{2}\left( \frac{f^2 m_j^2}{\omega_j^2} + m_j^2 + l_j^2 + k_j^2\right).
    \label{eqn:Viscous_term}
\end{equation}
$\textnormal{NLT}_j$ is obtained by substituting the fields $(u_j,v_j,w_j,b_j)$ in $\textnormal{NLT}$, and by retaining all the nonlinear terms that have the same phase as the $j-$th wave. Nonlinear terms that do not have the phase of any of the five waves are the `non-resonant terms' and are neglected. From $w_j$, we can obtain $u_j,v_j,$ and $b_j$ by using the polarisation relations:
%\todo{write (2.9) in matrix form such that $[u,v,b]=[U,V,B]w$ with U,V,B in the next line or on the right side=>}
% \begin{align}
%     % u_j  &=  U_j w_j, \hspace{2cm} v_j  =  V_j w_j, \hspace{1.95cm}   b_j = B_jw_j, \nonumber \\
%     &[u_j,v_j,b_j]^T  =  [U_j, V_j, B_j]^T w_j, \nonumber \\
%     &U_j  =  -\frac{m_j(\omega_j k_j + \textnormal{i} l_j f_j)}{\omega_j(k_j^2 + l_j^2)}, \hspace{0.2cm} V_j  =  -\frac{m_j(\omega_j l_j - \textnormal{i} k_j f_j)}{\omega_j(k_j^2 + l_j^2)}, \hspace{0.2cm} B_j = -\textnormal{i}\frac{N^2}{\omega_j}
%     \label{eqn:polarisation_relations}
% \end{align} 
\begin{align}
    \begin{bmatrix}
u_j\\ \\
v_j\\   \\
b_j   
\end{bmatrix} 
=    \begin{bmatrix}
U_j\\ \\
V_j\\   \\
B_j   
\end{bmatrix}
w_j
=    \begin{bmatrix}
- {m_j(\omega_j k_j + \textnormal{i} l_j f_j)}{/[\omega_j(k_j^2 + l_j^2)]}\\ \\
 - {m_j(\omega_j l_j - \textnormal{i} k_j f_j)}{/[\omega_j(k_j^2 + l_j^2)]}\\ \\  
 -\textnormal{i} {N^2}{/\omega_j}
\end{bmatrix}
w_j.
\label{eqn:polarisation_relations}
\end{align}
Polarisation expressions are also used to evaluate $\textnormal{NLT}_j$, the expressions for which are given in appendix \ref{App:A}. 

%\todo{you don't need to be tied up here, i will keep adding small todos, if there are major things, will let you know=>ok thanks..hmm I have written the matrix in a horizontal form now. I can write it in a column way but I think it wont be compact because the expressions for Uj are big. => use transpose to be mathematically correct=>done, thanks=> w is a scalar so it wouldn't have mattered anyways=> }

\subsection{ {Wave-amplitude equations and growth rates}}

The amplitude evolution of each of the $5$ waves can be obtained from
  \eqref{eqn:o_epsilon_2_equation_j_wave}:
\begin{equation}
    \frac{d a_{1}}{dt} = {\mathcal{M}}_1 a_2  a_3 - \mathcal{V}_1 a_1, \hspace{1.0cm}  \frac{d a_{2}}{dt} = {\mathcal{M}}_2 a_1  \bar{a}_3 - \mathcal{V}_2 a_2
    \label{eqn:triad_1_amp_evol}
\end{equation}
\begin{equation}
  \hspace{-0.4cm}  \frac{d a_{5}}{dt} = \mathcal{N}_5 a_4  a_3 - \mathcal{V}_5 a_5  \hspace{1.0cm}     \frac{d a_{4}}{dt} = \mathcal{N}_4 a_5  \bar{a}_3  - \mathcal{V}_4 a_4
    \label{eqn:wave_5_amp_evol}
\end{equation}
% \begin{equation}
%     \frac{d a_{2}}{dt} = {\mathcal{M}}_2 a_1  \bar{a}_3 - \mathcal{V}_2 a_2
%     \label{eqn:wave_2_amp_evol}
% \end{equation}
% \begin{equation}
%     \frac{d a_{4}}{dt} = \mathcal{N}_4 a_5  \bar{a}_3  - \mathcal{V}_4 a_4
%     \label{eqn:wave_4_amp_evol}
% \end{equation}
\begin{equation}
    \frac{d a_{3}}{dt} =  {\mathcal{M}}_{3} a_1  \bar{a}_2 + \mathcal{N}_{3} a_5  \bar{a}_4 - \mathcal{V}_3 a_3
    \label{eqn:wave_3_amp_evol}
\end{equation}
where $\mathcal{V}_j = {\nu}/{2}\left( {f^2 m_j^2}/{\omega_j^2} + m_j^2 + l_j^2 + k_j^2\right)$. As depicted in figure \ref{fig:5_wave_triad_schematic}(c),   wave-1,-2, and -3 form a triad, whose nonlinear coefficients are given by $\mathcal{M}_j$. Likewise,  wave-3, -4, and -5 also form a  triad, whose nonlinear coefficients are given by $\mathcal{N}_j$.  Expressions for $\mathcal{M}_j$ and $\mathcal{N}_j$ are given in  appendix \ref{App:A}. Wave-3, therefore, becomes  the common daughter wave in two different triads. Moreover, wave-3 can be thought of as  the 
daughter wave in a triad with 
 wave-1 and -5 as the two parent waves.

%The nonlinear coefficients of the triad comprising waves (1,2, and 3) is denoted by $\mathcal{M}_j$, while the nonlinear coefficients of the triad comprising waves (3,4, and 5) is denoted by $\mathcal{N}_j$. Expressions for $\mathcal{M}_j$ and $\mathcal{N}_j$ are given in the appendix \ref{App:A}.
Using pump wave approximation \citep{mcewan,young},  \eqref{eqn:triad_1_amp_evol}--\eqref{eqn:wave_3_amp_evol} can be simplified to a set of linear differential equations which are given below in a compact form:
% \begin{equation}
%     \frac{d a_{2}}{dt} = \mathcal{N}_2 A_1  \bar{a}_3 - \mathcal{V}_2 a_2
%     \label{eqn:wave_2_amp_evol_Pump}
% \end{equation}
% \begin{equation}
%     \frac{d a_{4}}{dt} = \mathcal{N}_4 A_5  \bar{a}_3  - \mathcal{V}_4 a_4
%     \label{eqn:wave_4_amp_evol_Pump}
% \end{equation}
% \begin{equation}
%     \frac{d a_{3}}{dt} = \mathcal{N}_{(3,1)} A_1  \bar{a}_2 + \mathcal{N}_{(3,2)} A_5  \bar{a}_4 - \mathcal{V}_3 a_3
%     \label{eqn:wave_3_amp_evol_Pump}
% \end{equation}
\begin{equation}
  \begin{bmatrix}
\dfrac{d \bar{a}_{2}}{dt}\\ \\
\dfrac{d \bar{a}_{4}}{dt}\\   \\
\dfrac{d {a}_{3}}{dt}   
\end{bmatrix} 
=  \begin{bmatrix}
- \mathcal{V}_2 & 0  & \bar{{\mathcal{M}}}_2 \bar{A}_1 \\ \\ \\
0 & - \mathcal{V}_4  & \bar{\mathcal{N}}_4 \bar{A}_5 \\  \\ \\
{\mathcal{M}}_{3} {A}_1& {\mathcal{N}}_{3} {A}_5  & -\mathcal{V}_3 
\end{bmatrix} 
\begin{bmatrix}
\bar{a}_{2}\\ \\ \\
\bar{a}_{4}\\  \\ \\
    {a}_{3}
\end{bmatrix}. 
\label{eq:scatter_matrix_scho}
\end{equation}
%\todo{Sir only A1 and A5. In pump wave, parent wave amplitude are assumed as constnats.=> sorry sorry my bad, yes yes of course=>sir cheers..thanks,=>thanks to you=>}
Note that $a_1(a_5)$ has been changed to $A_1(A_5)$ to denote the fact that they are now constants. By assuming $d a_j/d t = \sigma a_j$, we arrive at the equation
\begin{equation}
    (\sigma + \mathcal{V}_2)(\sigma + \mathcal{V}_3)(\sigma + \mathcal{V}_4) - \bar{\mathcal{N}}_{4}{\mathcal{N}}_{3} |A_5|^2(\sigma + \mathcal{V}_2) - \bar{\mathcal{M}}_{2}{\mathcal{M}}_{3} |A_1|^2(\sigma + \mathcal{V}_4) = 0,
    \label{eqn:cubic_equation_root}
\end{equation}
where $\sigma$ is the growth rate of the system of equations given in \eqref{eq:scatter_matrix_scho}. A real, positive $\sigma$ implies the daughter waves can extract energy from the parent wave. For $\nu = 0$ (inviscid flow), the growth rate has a simple expression given by
\begin{equation}
        \sigma = \sqrt{\bar{\mathcal{M}}_{2}{\mathcal{M}}_{3} |A_1|^2 + \bar{\mathcal{N}}_{4}{\mathcal{N}}_{3} |A_5|^2}.
        \label{eqn:growth_rate_inviscid}
\end{equation}
Note that by setting either $A_1=0$ or $A_5=0$, we arrive at the standard growth expression for triads (3-wave systems). Moreover, we can also obtain the condition
\begin{equation}
        \sqrt{\bar{\mathcal{M}}_{2}{\mathcal{M}}_{3} |A_1|^2 + \bar{\mathcal{N}}_{4}{\mathcal{N}}_{3} |A_5|^2} \leq \sqrt{2} \widehat{\sigma}_{1} \hspace{0.5cm} \textnormal{or} \hspace{0.5cm}         \sqrt{\bar{\mathcal{M}}_{2}{\mathcal{M}}_{3} |A_1|^2 + \bar{\mathcal{N}}_{4}{\mathcal{N}}_{3} |A_5|^2} \leq \sqrt{2} \widehat{\sigma}_{5}
        \label{eqn:growth_rate_inequality}
\end{equation}
where $\widehat{\sigma}_1(\widehat{\sigma}_5)$ is the maximum growth rate of all 3-wave systems of parent wave-1(5). If both the parent waves have the same amplitude $(A_1=A_5)$, frequency, and wavevector norm, then $\widehat{\sigma}_{1}=\widehat{\sigma}_{5}$. In such cases,  \eqref{eqn:growth_rate_inequality} implies  {that any 5-wave system's growth rate could, in principle, be higher (maximum being $\sqrt{2}$ times) than the maximum growth rate of all 3-wave systems.} { For all the parent wave combinations considered in this paper, $A_1=A_5$ is consistently taken for the analysis.}


\subsection{{5-wave system identification}} \label{sec:2.2}

%The objective is to identify all the possible solutions for 5-wave interactions where the two parent waves have the same frequency but different wavevectors.  
% For a given frequency and wavevector of the parent waves, the following equations have to be satisfied for a resonant 5-wave interaction.

For a resonant 5-wave system, all three daughter waves should satisfy the dispersion relation. This leads to 3 constraints, which are given below:
\begin{subequations}
\begin{align}
      \omega_3^2         &= \frac{N^2(k_3^2 + l_3^2) + f^2m_3^2}{k_3^2 + l_3^2 + m_3^2} ,  \label{eqn:Id_1} \\
    \omega_2^2         &= \frac{N^2(k_2^2 + l_2^2) + f^2m_2^2}{k_2^2 + l_2^2 + m_2^2}   ,  \label{eqn:Id_2} \\
    \omega_4^2         &= \frac{N^2(k_4^2 + l_4^2) + f^2m_4^2}{k_4^2 + l_4^2 + m_4^2}.    \label{eqn:Id_3} 
\end{align}
\end{subequations}
The following triad conditions also add additional constraints:
\begin{subequations}
 \begin{align}
    \omega_2 &= \omega_1 - \omega_3, \hspace{0.3cm} \textbf{k}_2 =  \textbf{k}_1 - \textbf{k}_3, \label{eqn:triad_cond_wv_2} \\
    \omega_4 &= \omega_5 - \omega_3, \hspace{0.3cm} \textbf{k}_4 =  \textbf{k}_5 - \textbf{k}_3, \label{eqn:triad_cond_wv_4} 
\end{align}   
\end{subequations}
where $\textbf{k}_j = (k_j,l_j,m_j)$ is the wavevector of the $j-$wave. Substitution of \eqref{eqn:triad_cond_wv_2} in \eqref{eqn:Id_2}, and  \eqref{eqn:triad_cond_wv_4} in \eqref{eqn:Id_3} lead to
\begin{subequations}
 \begin{align}
    \omega_3^2         &= \frac{N^2(k_3^2 + l_3^2) + f^2m_3^2}{k_3^2 + l_3^2 + m_3^2},   \label{eqn:Final_Id_1} \\
(\omega_1-\omega_3)^2 &= \frac{N^2((k_1-k_3)^2 + (l_1-l_3)^2) + f^2(m_1-m_3)^2}{(k_1-k_3)^2 + (l_1-l_3)^2 + (m_1-m_3)^2},   \label{eqn:Final_Id_2} \\
(\omega_5-\omega_3)^2 &= \frac{N^2((k_5-k_3)^2 + (l_5-l_3)^2) + f^2(m_5-m_3)^2}{(k_5-k_3)^2 + (l_5-l_3)^2 + (m_5-m_3)^2}.   \label{eqn:Final_Id_3} 
\end{align}   
\end{subequations}
Solutions for \eqref{eqn:Final_Id_1}-\eqref{eqn:Final_Id_3} would provide resonant 5-wave systems, and they are found by varying $(\omega_3,k_3,l_3,m_3)$. Hereafter we always assume $|\textbf{k}_1| =  |\textbf{k}_5|$, however, $\textbf{k}_1 \neq  \textbf{k}_5$, and $\omega_1 = \omega_5=0.1N$. Such small frequency values appear in many other studies,  for example,  \cite{sonya_nik_2011,Mathur_14}. 

\section{{Results from the reduced--order model}} \label{Section:3}

\subsection{Parent waves in the same vertical plane}


\subsubsection{ $\mathbf{k}_1 =  (k_1,0,m_1)$ and $\mathbf{k}_5 =  (k_1,0,-m_1)$} \label{sec:m_minus_m}

We first consider the scenario where the two parent waves have the same horizontal wavevector ($k,l$) but travel in vertically opposite directions, see figure \ref{fig:5_wave_triad_schematic}(a).  For simplicity,  the meridional wavenumbers  of the parent waves ($l_1$ and $l_5$) are assumed to be 0.
Internal waves propagating in vertically opposite directions are ubiquitous in the oceans. For example, internal wave beams getting reflected from the bottom surface of the ocean, or from the air-water interface, or even from the pycnocline, will result in scenarios where parent waves travelling in vertically opposite directions meet. 
%This is shown in figure () for different idealised cases.
%\todo{why these $f/\omega_1$ values are considered?=> Just to show we saw a wide values of f to make sure that this is the only 5-wave system. Do you want me to remove it?=> No, exactly. What I wanted is to write explicitly that you scanned a wide range of values, as seen that you have taken a small value, and then went 0.1, 0.2 etc. So, rather than just putting the numbers, what you wanted to achieve with these numbers are important=>ah ok..ok i will edit it now.=>} \textcolor{red}{ For $f/\omega_1 = (0.05,0.1,0.2,0.4)$, no resonant 5-wave systems were found for $\omega_3 \in (f+\delta \omega,\omega_1 - f - \delta \omega)$, where $\delta \omega = 0.01\omega_1$.}

For the given set of parent waves, a resonant 5-wave system is  possible only when $\omega_3\approxeq \omega_1-f$. No other resonant 5-wave systems were found for $0<f/\omega_1<0.5$. Hence, the 5-wave system always consists of (a) two parent waves, each with frequency $\omega_1$ (as per our assumption), (b) a common daughter wave with frequency $\omega_1-f$,  and (c) two inertial (frequency $f$) daughter waves, which also propagate in vertically opposite directions. 

Next we study the growth rates of the 5-wave system. First, we decide on the viscosity values in a non-dimensionalised form. In this regard we choose  $|A_1|/k_1\nu = 10^{4}$ and $|A_1|/k_1\nu = 10^{7}$, which  are used throughout the paper.  At $|A_1|/k_1\nu = 10^{7}$, viscous effects are usually negligible, hence $|A_1|/k_1\nu = 10^{4}$ is also considered to see what 5-wave systems are affected by the viscosity. We note in passing that   $|A_1|/k_1\nu \sim \mathcal{O}(10^{6})$ was used by \cite{bourget} to study triads with realistic oceanic parameters.  

%\textcolor{blue}{To evaluate growth rates for a range of $f$, }
%A comparison between the maximum growth rate of a 5-wave system and a 3-wave system for different $\nu$ is shown in figure \ref{fig:k_m_k_minus_m_max_growth_rate_1p_vs_2p}. 
% \todo{wavenumber=>Sir both are fine in this case i think. However, with wavevector it is slightly more general so i put it.=> I see that you wanted to mean (k1,0,0), can't we simply write that?=>ah ok..I will edit now.=>} 

Figure \ref{fig:k_m_k_minus_m_max_growth_rate_1p_vs_2p}(a) shows how the maximum growth rate of the 5-wave system and 3-wave systems vary with $f/\omega_1$ for different $\nu$. The growth rates are evaluated by fixing $k_1$ and $A_1$ as $f$ is varied. Figures \ref{fig:k_m_k_minus_m_max_growth_rate_1p_vs_2p}(b)--\ref{fig:k_m_k_minus_m_max_growth_rate_1p_vs_2p}(c)  respectively show the horizontal  and the vertical wavenumbers of the daughter waves involved in the 5-wave interaction. Note that the common daughter wave's horizontal wavevector  $(k_1,0)$ is always the same as that of the parent waves.  This is expected since the other two daughter waves are inertial waves. Moreover, the common daughter wave can have a positive or negative vertical wavenumber, and  both cases have the same growth rate. For low $f$ values, the 3-wave system has a higher growth rate, hence it is the dominant instability. {This is because the two 3-wave systems that combine to form the 5-wave system always contain inertial daughter waves. Moreover, the growth rate of 3-wave systems containing inertial waves is much smaller than the maximum possible growth rate \cite[figure 8]{richet_2018}. As a result, the resonant 5-wave system is of little significance at low latitudes.}

%\sout{This is because the conditions required  for a resonant 5-wave system (equations \eqref{eqn:Final_Id_1}--\eqref{eqn:Final_Id_3}) are more stringent in comparison to that of a 3-wave system. Note that a 5-wave system is essentially a combination of two different 3-wave systems (see figure \ref{fig:5_wave_triad_schematic}(c)), and in this case, both the 3-wave systems that combine to form the 5-wave system contain inertial daughter waves at low $f-$values. Such 3-wave systems have growth rates much smaller than the maximum possible growth rate, implying that the resultant 5-wave system is of little significance for low latitudes.}

%This results in the 5-wave system instability not being the dominant instability at low $f$ values because 3-wave systems which contain inertial waves have very low growth rates compared to the maximum possible growth rate, see for example figure 8 in \cite{richet_2018}. 

As $f$ is increased, the 5-wave system's growth rate becomes higher than the maximum growth rate of all 3-wave systems. The transition occurs near $f/\omega_1 \approx 0.3$, see figure \ref{fig:k_m_k_minus_m_max_growth_rate_1p_vs_2p}(a).  For high values of $f/\omega_1$, 5-wave systems may be faster in locations where an internal wave beam gets reflected from a flat bottom surface or from a nearly flat air-water surface. %However, for inclined reflecting surfaces, such a 5-wave system might not be the dominant instability since inclination results in a significant change in wavevector norm/magnitude \citep{phillips}. 
However, for inclined reflecting surfaces, the results presented here (which are based on the assumption that the two parent waves have the same wavevector norm) may not be valid since inclination results in a significant change in wavevector norm \citep{phillips}. 

%\todo{please check and see if you agree=> Sir so the 5-wave system may still be the dominant but it not possible to know from the results of this section, because the results are only for parent waves whose wavevector norm is the same. I have rewritten it now.=>yes, looks good=> sir thanks..was talkinh with Nick, seems like the cluster is nearly ready(hopefully).=>Wow, that's a very good news. Is Nick back?=>Yeah he is reemployed I think. yes he is workign with Mr.Derek.=> Great news.Thanks for updating. Also, I believe you had a fruitful discussion with Tom => Sir yeah, like I gave the data mainly and explained it. Today I also gave the linear stability analysis results (very cleaned up). I think the codes should be easy to use (saw your msg in teams as well). I have some more work to do for Dr. Tom (will take atleast a full day.)=>Ok great=> }
%Note that reflection of an internal wave can result in its wavevector norm/magnitude changing significantly if the reflecting surface is inclined \citep{phillips}. 
%In such cases, the above-mentioned solutions and growth rates may not be valid.
%at low $f$ values, daughter waves that are a part of the 3-wave system with maximum growth rate cannot be a part of a resonant 5-wave system. The resonant 5-wave system has to contain two inertial waves, since it is the only way equations \eqref{eqn:Final_Id_1}--\eqref{eqn:Final_Id_3} are satisfied.  
%For a resonant 5-wave system, more stringent conditions have to be satisfied as given in section \ref{sec:2.2}. For low $f$ values,
%triads with the highest growth rates in a 3-wave system with low $f$ values cannot combine to form a resonant 5-wave system.  


%\sout{Finally, we note that two parent waves with equal amplitude travelling in vertically opposite directions {in a constant $N$} \textcolor{red}{in the presence of constant stratification} produce a field that resembles an internal wave mode in a vertically bounded domain.}
{Finally, we note that the parent waves combination considered in this subsection produces a field that resembles an internal wave mode in a vertically bounded domain.} Hence, the predictions made in this section should also hold for modes in a bounded domain. However, in a vertically bounded domain, only a discrete set of vertical wavenumbers are allowed for a particular frequency. As a result, for a resonant 5-wave system to exist, the vertical wavenumbers of the daughter waves should be a part of the discrete vertical wavenumber spectrum. 

%Results obtained by direct numerical simulations provided in section \ref{} show that the theoretical growth rates of a 5-wave system match reasonably well with numerical results.

%\textcolor{blue}{Need to check for f = 0.48 also.}

%Moreover, for $f/\omega_1 = 0.45$, no 5-wave systems were found for $\omega_3 \in (f+\delta \omega,\omega_1 - f - \delta \omega)$, where $\delta \omega = 0.001\omega_1$. 

\begin{figure}
 \centering{\includegraphics[width=1.0\textwidth]{Nu_4_7_combined_F_variation_Single_Double_k_m_k_minus_m_With_wavenumber.png}}
 \caption{ (a) Variation of 5-wave system's (denoted by 5WS) growth rate  and 3-wave systems' (denoted by 3WS) maximum growth rate  with $f$ for $\mathbf{k}_1 =  (k_1,0,m_1)$ and $\mathbf{k}_5 =  (k_1,0,-m_1)$ for two different viscosity values. $\sigma_{\textnormal{ref}}$ is the maximum growth rate of a 3-wave system at $f/\omega_1=0.01$ and $|A_1|/k_1\nu = 10^{4}$.
 %\todo{where is k4 is (b)?=> both are zero, so in legend both are given by the same zero curve.=> vertical inertial waves?=>yeah. they satisfy the dispersion relation w^2 = (N^2k^2+f^2m^2)/(k^2+m^2). Please note if k=0, doesnt matter what is m. always w=f for whatever m.   => Yes that's brilliant=>}
 %Results for $|A_1|/k_1\nu=10^4$ and $10^7$ are shown. 
 The horizontal and vertical wavenumbers of the daughter waves in the 5-wave system are  respectively shown in (b) and (c). Note that in (b),  $k_2=k_4=0$.   }
  \label{fig:k_m_k_minus_m_max_growth_rate_1p_vs_2p}
\end{figure} 

% Equations  \eqref{eqn:wave_1_amp_evol}--\eqref{eqn:wave_3_amp_evol} is simulated to compare a 3-wave system and a 5-wave system at $f=0.45$. For the 5-wave system, triads are chosen such that $\omega_3 = 0.5499 \omega_1$ (nearly ). The wavevector for the wave-3 in the 5-wave system is: $k_3 = 1.0063, l_3 = -0.0188, m_3 = 31.7953$. This wave-vector leads to a detuning, that is non-dimensionlised with $f$, to ($\approx10^{-4}$). Hence the triad conditions are nearly satisfied. Both parent waves are assumed to have the same energy at the start of the simulation. For the 3-wave system, the triad which nearly has the maximum growth rate is chosen which occurs at $\omega_3 = 0.49\omega_1$. Viscosity is set to $0$. The wavevector for the wave-3 in the 3-wave system is: $k_3 = 0.5664, m_3 = 29.17$. Both systems are initialised with a very small (and equal) energy in wave-3 only. As a result, wave-2 and wave-4 start with zero energy in a 5-wave system, and wave-2 starts with zero energy in the 3-wave system ($a_2(0)=0$). 

%  \begin{figure}
%  \centering{\includegraphics[width=1.0\textwidth]{3ws_vs_5ws_k_m_k_minus_m_max_GR_F_045.png}}
%  \caption{ The figure shows the variation of the energy of the waves involved in a 3-wave system and a 5-wave system. (a) 5-wave system. (b) 3-wave system. }
%   \label{fig:Nonlinear_simulation_reduced_order}
% \end{figure} 

% Figure \ref{fig:Nonlinear_simulation_reduced_order} shows the evolution of the wave energy in a 3-wave system and a 5-wave system. It can be seen that a 5-wave system is faster than the 3-wave system where both parent wave lose all their energy in a shorter time in comparison. Moreover, it is also observed that both parent waves take lesser time to go from $90\%$ of their initial energy to $0\%$ of their initial energy. Hence even in the nonlinear stage, the rate of energy transfer is faster in the 5-wave system.

\subsubsection{   $\mathbf{k}_1 =  (k_1,0,m_1)$ and $\mathbf{k}_5 =  (-k_1,0,m_1)$ } 
\label{sec:horizontally_opposite_P_waves} 

Here we focus on the scenario where the two parent waves have the same vertical wavenumber but travel in horizontally opposite directions, as given in figure \ref{fig:5_wave_triad_schematic}(b). Moreover, $l_1=l_5=0$ is again assumed. For this particular combination of parent wavevectors, resonant 5-wave systems are possible for $\omega_3 \in (f, 0.53\omega_1)$. For $f/\omega_1 = 0.01$, resonant 5-wave systems exist up to $\omega_3 \approx 0.53\omega_1$. As $f$ increases, the maximum possible value of $\omega_3$ slowly reduces to $0.5\omega_1$. 
%Moreover, the systems where $\omega_3\approx0.5\omega_1$ do not have higher growth rates than the systems that are relatively further away from $\omega_3\approx0.5\omega_1$. 
%Two main branches are defined as follows:
We define two branches: 
5-wave systems where the common daughter wave has a positive (negative) vertical wave number is defined as Branch-1(2). 
%The growth rates are only considered up to $0.496\omega_1$ maximum.
% \textcolor{red}{For this system, for Map-3 and 4 (means the inertial wave is travelling opposite to the parent waves), the common daughter wave can have solutions that $>0.5\omega_1$ also.} However, as $f$ increases, the limit strictly becomes $0.5\omega_1$. Moreover, they do not higher growth rates than the triads for which $\omega_3<0.5\omega_1$. The growth rates are only considered up to $0.496\omega_1$ maximum.
Figures \ref{fig:k_m_minus_k_m_max_growth_rate_1p_vs_2p}(a)--\ref{fig:k_m_minus_k_m_max_growth_rate_1p_vs_2p}(b) show how the maximum growth rate for each of these two branches varies with $f$ for two different viscosity values. The maximum growth rate of 3-wave systems is once again plotted so as to provide a clear comparison between 5-wave and 3-wave systems. For lower $f$ values, resonant 5-wave systems have a lesser maximum growth rate than the maximum growth rate of 3-wave systems ($\sigma/\sigma_{\textnormal{ref}}<1$). However, the 5-wave instability is faster than the 3-wave instability for the higher $f$ values. The transition once again occurs near $f \approx 0.3\omega_1$.  All these observations are similar to that in figure \ref{fig:k_m_k_minus_m_max_growth_rate_1p_vs_2p}(a). {For high $f$ values, the maximum growth rate of both branches is almost the same.}
Viscosity has a non-negligible effect only when $f\approx\omega_1/2$, where the daughter waves have a high vertical wavenumber.

% For $f/\omega_1 > 0.3$, in the most unstable 5-wave system, the common daughter wave's frequency is always near or equal to the inertial frequency.
Figures \ref{fig:B12_k_m_minus_k_m}(a)--\ref{fig:B12_k_m_minus_k_m}(c) show how the growth rate of both the branches vary with $\omega_3/\omega_1$ for three different $f/\omega_1$ values that are greater than $0.3$. 
%growth rate variation with $\omega_3/\omega_1$ for both branches for three different $f/\omega_1$ values greater than $0.3$. 
%\textcolor{red}{Here, $\sigma_{\textnormal{ref}}$ is the maximum growth rate of 3-wave systems at a particular $f$ with  $|A_1|/k_1\nu = 10^{4}$. As a result, $\sigma_{\textnormal{ref}}$ is a function of $f/\omega_1$.}
Figure \ref{fig:B12_k_m_minus_k_m} reveals that growth rates always decrease as $\omega_3$ is increased, indicating the maximum growth rate is at $\omega_3=f$. For $f/\omega_1 > 0.3$, the common daughter wave is always an inertial wave in the most unstable 5-wave system. Interestingly, as $\omega_3$ is increased from $f$, the meridional wavenumber of the common daughter wave increases, hence making the instability 3D. Moreover, for the three $f$ values analysed in figure \ref{fig:B12_k_m_minus_k_m}, the zonal wavenumber of the common daughter wave $(k_3)$ is nearly zero for all the 5-wave systems. Note that the maximum growth rate occurs at $\omega_3 \approx f$ where $(k_3,l_3) \xrightarrow{} 0$. As a result, the system's most unstable mode can be studied/simulated by considering a 2D system. 
%Growth rates are plotted in the range $\omega_3/\omega_1 \in (f,0.49)$. 
%All the 5-wave systems have a higher growth rate than any 3-wave system. 
The effects of viscosity are more apparent for $|A_1|/k_1\nu = 10^{4}$ as expected, and Branch-1 is affected by viscous effects more than Branch-2.

%Viscous effects affect branch-1 more than branch-2 as $\omega_3 \xrightarrow{} 0.5\omega_1$. This is because the daughter waves' wavenumbers become very large hence making the viscous terms more important.  
%For $f\in(0.05\omega_1, 0.48\omega_1)$\todo{Can easily increase this range} , no Branch-1 solutions were found such that $\omega_3 \geq 0.501\omega_1$. This implies that a strong cutoff is at $\omega_3 = 0.5\omega_1$ for Branch-1 triads. However, the cutoff is not that strong for Branch 2 triads.  In general, we observe that as $f$ in increased, the possible solutions where $\omega_3 > 0.5\omega_1$ decreases. 
%Moreover, for Branch-2 triads, the maximum growth rate of the triads with $\omega_3 > 0.5\omega_1$ is lesser than the growth of triads with $\omega_3 < 0.5\omega_1$. This is consistent with the conclusion that for high $f$ the maximum growth rate occurs when $\omega_3 \xrightarrow{} f$.

For high $f$ values, inertial waves have been observed to be the daughter waves of a parent internal wave with semidiurnal frequency, see \cite{legg_2017_dissipation_generation,mean_current_richet_triad,richet_2018,GM_spectrum_tidal_forcing}. In topographic generation of internal waves, internal wave beams intersecting each other is quite common. The locations where internal wave beams intersect can serve as spots where a single inertial wave can extract energy from two different internal wave beams. 


\begin{figure}
 \centering{\includegraphics[width=1.0\textwidth]{Nu_4_7_combined_F_variation_Single_Double_k_m_minus_k_m_updated.png}}
 \caption{ Comparison of maximum growth rates of 5-wave systems and 3-wave systems for $\mathbf{k}_1 =  (k_1,0,m_1)$ and $\mathbf{k}_5 =  (-k_1,0,m_1)$.  (a) $|A_1|/k_1\nu = 10^{4}$, and (b) $|A_1|/k_1\nu = 10^{7}$.  }
  \label{fig:k_m_minus_k_m_max_growth_rate_1p_vs_2p}
\end{figure} 

\begin{figure}
 \centering{\includegraphics[width=1.0\textwidth]{k_m_minus_k_m_omega_variation_Main_Part_3.png}}
 \caption{ Growth rate variation with $\omega_3/\omega_1$ for Branch-1 and 2 for (a) $f/\omega_1 = 0.40$, (b) $f/\omega_1 = 0.45$, and (c) $f/\omega_1 = 0.48$. 
 %three different $f$ values. Every subplot shows the two branches for $|A_1|/k_1\nu=10^4$ and $10^7$.  
 }
  \label{fig:B12_k_m_minus_k_m}
\end{figure} 








%In figure (1e) of \cite{GM_spectrum_tidal_forcing}, internal wave beams with frequency $\omega_d-f$ seem to originate from the location where two semidiurnal internal wave beams collide. In that particular simulation $f = 0.35\omega_d$ was used, where a 5-wave system has a faster instability than a 3-wave system. If the $\omega_d-f$ beams originate from a location where the different parent waves meet, then the above mentioned mechanism seems likely to have occurred there. A similar phenomenon is also observed in \cite{richet_2018}, where the inertial waves are located just above the topography where the two semidiurnal internal wave beams meet.


% It can be seen that for lower $f$ values, the possible solutions for a 5-wave system has a lesser growth rate than a standard 3-wave system. However, as $f$ is increased, the 5-wave system instability is faster than the 3-wave system's instability. The change occurs near $f \approx 0.3\omega_1$. 

% Moreover, it can also be seen that near $\omega_3 = 0.5\omega_1$

\subsection{Oblique parent waves}

In the oceans, parent waves that are not on the same vertical plane can also propagate amidst each other. Here we study the maximum growth rate for 5-wave systems where the parent waves have a non-zero meridional wavenumber. The parent wavevectors are given by 
\begin{equation}
    \mathbf{k}_1 =  (k_1\sin{(\theta/2)},k_1\cos{(\theta/2)},m_1), \hspace{1cm}  \mathbf{k}_5 =  (-k_1\sin{(\theta/2)},k_1\cos{(\theta/2)},m_1),
    \label{eqn:theta_definition}
\end{equation}
where the parameter $\theta$ is used to vary the angle between the two parent wavevectors in the $(k,l)$ plane. Note that $\theta = \pi$ leads to the wavevector combination $\mathbf{k}_1 =  (k_1,0,m_1)$ and $\mathbf{k}_5 =  (-k_1,0,m_1)$ considered in \S \ref{sec:horizontally_opposite_P_waves}.  Following \eqref{eqn:theta_definition},  the condition $|\mathbf{k}_1| = |\mathbf{k}_5|$ will be automatically satisfied.
%\todo{or is it that ``we will still assume...''=> Sir since the wavevectors aare given ib the form of the above equation, I thought here we can write that " Note that |k1| = |k5|". Hmm..do you want me to change.=> Yes it is as per assumptioion},
The direction of the parent wavevectors can be changed by varying $\theta$, and how that impacts the growth rates of 5-wave systems  will be explored and  analysed.
%Moreover, $\theta = 0$ leads to single plane internal wave with twice the amplitude.
% \begin{itemize}
%  \item Case-1: $\mathbf{k}_1 =  (1/\sqrt{2}k_1,0,m_1)$ and $\mathbf{k}_5 =  (-k_1,0,m_1)$.
%     \item Case-2: $\mathbf{k}_1 =  (k_1,0,m_1)$ and $\mathbf{k}_5 =  (-k_1,0,m_1)$.
%     \item Case-3: $\mathbf{k}_1 =  (k_1,0,m_1)$ and $\mathbf{k}_5 =  (-k_1,0,m_1)$.
% \end{itemize}
%We will consider three cases: $\theta = \pi/4$, $\pi/2$, and $3\pi/4$.
Figures \ref{fig:max_growth_rate_oblique_1p_vs_2p}(a)--\ref{fig:max_growth_rate_oblique_1p_vs_2p}(c) show the variation of the maximum growth rate of 3-wave systems and 5-wave systems with $f$ respectively for  three different $\theta$ values:  $\pi/4$, $\pi/2$, and $3\pi/4$. 
%It can be seen that regardless of the horizontal wavevector orientation, near $f/\omega_1 \approx 0.5$, a 5-wave system is susceptible to a faster instability than a 3-wave system. %Moreover, near $f/\omega_1 \approx 0.5$, the 5-wave system's maximum growth rate is nearly $\sqrt{2}$ times the 5-wave system's maximum growth rate. 
Increasing $\theta$ results in 5-wave systems being less effective than 3-wave systems in the lower latitudes. For $\theta=\pi/4$, the 5-wave system is the dominant instability regardless of the latitude. A similar result is observed for $\theta=\pi/2$, however, the difference between the 5-wave and 3-wave systems is clearly reduced compared to $\theta=\pi/4$. For $\theta=3\pi/4$, the 5-wave system is the dominant instability only for $f/\omega_1 \gtrapprox 0.25$. Note that regardless of the $\theta$ value, 5-wave instability is expected to be faster than the 3-wave instability when $f/\omega_1 \gtrapprox 0.3$ considering the results from \S \ref{sec:horizontally_opposite_P_waves}.
%According to the results in this subsection and in \S \ref{sec:horizontally_opposite_P_waves}
%\todo{how is this section able to infer anything about oblique waves?that section is about theta=pi isn't it?=> yeah sir like this one is for the three theta values, and the most extreme one is theta = pi (of last section). For theta = pi, the 5-wave system is the faster instability after f/om1 = 0.3. In this section we note that decreasing theta results the 5 wave system becoming dominant in the lower latitudes (gradually).  That was the inference.=> "According to the results in this subsection and in \S \ref{sec:horizontally_opposite_P_waves}" is not what should be written..rather we stroll through the analysis (which is below) and then make a statement. Otherwise it is self-referencial. I agree with the \S \ref{sec:horizontally_opposite_P_waves} comment=> oh right..I worte according to the results in this section, here is the result in this section (kind of). I will rewrite it.=>}
 For $\theta = \pi$ and  $f/\omega_1 \gtrapprox 0.3$, the maximum growth rate for 5-wave systems occurs when $\omega_3 = f$.
 However, for $\theta = \pi/4$ and $\pi/2$, in the most unstable 5-wave system, wave-3 is not an inertial wave even for $f/\omega_1 = 0.45$. Hence, as $\theta$ is reduced, it is not necessary that the most unstable 5-wave system contains inertial waves. Note that the predictions for the 5-wave system will fail as $\theta\xrightarrow{}0$ since both parent waves will have the same wavevector. 

\begin{figure}
 \centering{\includegraphics[width=1.0\textwidth]{Oblique_Max_Growth_comparison_45_90_135_complete_paper_version.png}}
 \caption{ Variation of maximum growth rate  with $f$ for 5-wave systems and 3-wave systems for a oblique set of parent waves.  (a) $\theta = \pi/4$, (b) $\theta = \pi/2$, and (c) $\theta = 3\pi/4$. }
  \label{fig:max_growth_rate_oblique_1p_vs_2p}
\end{figure} 

\section{Numerical simulations} \label{Section:4}

Here we present results from numerical simulations conducted to validate the predictions from reduced-order analysis  presented in \S \ref{Section:3}, with the primary focus being on \S \ref{sec:m_minus_m} and \S \ref{sec:horizontally_opposite_P_waves}. Equations \eqref{eqn:total_mom_equation}--\eqref{eqn:Continuity} are solved with an open source pseudo-spectral code Dedalus \citep{Dedalus}. For numerical validations, we only consider 2D situations, i.e. $\partial/\partial y = 0$, implying $l_j=0$. The details of the simulations are as follows: we fix the parent waves' horizontal wavenumber at $k_1 = 1/H$, where $H=500$m. We consistently use $N = 10^{-3}\textnormal{s}^{-1}$ and $\omega_1/N = 0.1$. However $f/\omega_1$ is varied, and hence the vertical wavenumber of the parent waves ($m_1$) is a function of $f/\omega_1$. The amplitude of the parent waves is chosen such that the maximum zonal velocity ($u$) is always $0.001 \textnormal{ms}^{-1}$. Computational time is variable and depends on the simulation in question. For all simulations, $4$-th order Runge-Kutta method is used as the time-stepping scheme with a time step size of $(2\pi/\omega_1)/800$ (i.e. $800$ steps for one time period of the parent wave).
 {All the fields are expressed using Fourier modes in the horizontal direction, and either 64 or 128 modes are used per one horizontal wavelength of the parent wave. Moreover, the vertical direction is resolved using Chebyshev polynomials or Fourier modes, and the resolution is varied from a minimum of 96 to a maximum of 512 grid points per one vertical wavelength of the parent wave.} All simulations are initialised with a small amplitude noise, the spectrum of which is given by
\begin{equation}
    \mathcal{R}_{\textrm{noise}}(x,z) = \int_{0}^{k_{\textrm{noise}}}\int_{ m_{\textrm{lowest}}}^{m_{\textrm{noise}}} A_{\textrm{noise}} \sin( kx + mz + \phi_{\textrm{noise}}(k,m)) dm dk,
    \label{eqn:noise_definition}
\end{equation}
where $\phi_{\textrm{noise}}(k,m)\in [0,2\pi]$ is the random phase part, which   is generated using the `rand' function in Matlab for each $(k,m)$. 
Unless otherwise specified, $k_{\textrm{noise}} = 48k_1$ and $m_{\textrm{noise}} = 48m_1$. Moreover, $m_{\textrm{lowest}} = 2\pi/Lz$, where $Lz$ is the length of the domain in the $z$-direction. Equation \eqref{eqn:noise_definition} is added to the $b$ or $v$ field. The noise amplitude $A_{\textrm{noise}}$ is at least $10^{-3}$ times smaller than the primary waves' corresponding amplitude.  Unless otherwise mentioned, $\nu = 10^{-6} \textnormal{m}^2\textnormal{s}^{-1}$ is taken.
%Moreover, $m_{\textrm{lowest}} = 2\pi/Lz$, where $Lz$ is the length of the domain in the $z$-direction. 
Equation \eqref{eqn:noise_definition} is added to the $b$ or $v$ field. The noise amplitude $A_{\textrm{noise}}$ is at least $10^{-3}$ times smaller than the primary waves' corresponding amplitude.  Unless otherwise mentioned, $\nu = 10^{-6} \textnormal{m}^2\textnormal{s}^{-1}$ is taken.


% \begin{equation}
%     \mathcal{R}_{\textrm{noise}}(x,z) = \int_{0}^{k_{\textrm{noise}}}\int_{ m_{\textrm{lowest}}}^{m_{\textrm{noise}}} A_{\textrm{noise}} \sin( kx + mz + \phi_{\textrm{noise}}(k,m)) dm dk,
%     \label{eqn:noise_definition}
% \end{equation}
% where $\phi_{\textrm{noise}}(k,m)\in [0,2\pi]$ is the random phase part, which   is generated using the `rand' function in Matlab for each $(k,m)$. 
% Unless otherwise specified, $k_{\textrm{noise}} = 48k_1$ and $m_{\textrm{noise}} = 48m_1$. Moreover, $m_{\textrm{lowest}} = 2\pi/Lz$, where $Lz$ is the length of the domain in the $z$-direction. Equation \eqref{eqn:noise_definition} is added to the $b$ or $v$ field. The noise amplitude $A_{\textrm{noise}}$ is at least $10^{-3}$ times smaller than the primary waves' corresponding amplitude.  Unless otherwise mentioned, $\nu = 10^{-6} \textnormal{m}^2\textnormal{s}^{-1}$ is taken.



\subsection{{ $\mathbf{k}_1 =  (k_1,0,m_1)$ and $\mathbf{k}_5 =  (k_1,0,-m_1)$}} \label{sec_4_1}
We first focus on the parent wavevector combination $\mathbf{k}_1 =  (k_1,0,m_1)$ and $\mathbf{k}_5 =  (k_1,0,-m_1)$.  As mentioned previously, the combination of wavevectors $\mathbf{k}_1 =  (k_1,0,m_1)$ and $\mathbf{k}_5 =  (k_1,0,-m_1)$ leads to fields that are very similar to an internal wave mode in a vertically bounded domain. As a result, we also simulate low modes (modes-1 and 2) in a vertically bounded domain to observe whether there is an emergence of the `5-wave  instability'. The decay of the parent waves are simulated at specific latitudes where the daughter waves' vertical wavenumbers in the resonant 5-wave system are multiples of $m_1/3$ or $m_1/2$. This choice helps in reducing the computational resources required for the simulations. To estimate the energy in different wavevectors, we simply use 
 the Fast Fourier Transform (FFT) for both $x$ and $z$ directions in simulations where the parent waves are plane waves. In a vertically bounded domain, FFT is used only in the $x$ direction, while for the $z-$direction,  the orthogonal nature of the modes is exploited. As a measure of the energy contained in a wavevector, a non-dimensionalised energy $\widehat{E}$ is introduced:
\begin{equation}
    \widehat{E}(k,0,m,t) = \frac{|\hat{u}(k,0,m,t)|^2 + |\hat{w}(k,0,m,t)|^2 + |\hat{v}(k,0,m,t)|^2 + |\hat{b}(k,0,m,t)|^2/N^2}{ E_{\textnormal{ref}} } 
\end{equation}
%where $(\hat{u}(k,0,m),\hat{w}(k,m),\hat{v}(k,m),\hat{b}(k,m))$ denote the Fourier amplitude of wavevector $(k,0,m)$ of $(u,w,v,b)$, respectively. 
where the hat variables ($\hat{u},\hat{w},\hat{v},\hat{b}$) respectively denote the Fourier amplitudes of (${u},{w},{v},{b}$). $E_{\textnormal{ref}}$ serves as the measure of parent waves' energy at $t=0$ and is defined as 
\begin{equation}
   E_{\textnormal{ref}} =  \left(|\hat{u}(k_1,0,m_1)|^2 + |\hat{w}(k_1,0,m_1)|^2 + |\hat{v}(k_1,0,m_1)|^2 + |\hat{b}(k_1,0,m_1)|^2/N^2\right)\bigg|_{t=0}
   \label{eqn:E_ref_definition}
\end{equation} 
%Note that the maximum zonal velocity amplitude of a vertically bounded wave is $0.002 \textnormal{ms}^{-1}$.  
We simulate a total of 6 cases: 2 cases for parent waves in an unbounded domain (plane waves), and 2 cases each for mode-1 and mode-2 waves in a vertically bounded domain. For mode-1, $m_{\textrm{noise}}=96 m_1$ is chosen. For every simulation, a different $f$ value is used, and hence the resonant 5-wave system is different in each case. Figure \ref{fig:modes_log_energy} shows the exponential growth of daughter waves at  6 different latitudes due to 5-wave interactions.
%$\ln{(\widehat{E})}$ is plotted to show the exponential growth of the daughter waves.
Figures \ref{fig:modes_log_energy}(a)--\ref{fig:modes_log_energy}(e) plot
%In all the subplots except figure \ref{fig:modes_log_energy}(f), 
four different wavevectors. The wavevector $(|k_1|,0,|m_1|)$ contains the energy of both parent waves, while the other three wavevectors indicate the daughter waves. All three daughter waves grow exponentially, which provides clear evidence that this is a 5-wave system. In \ref{fig:modes_log_energy}(f), two different 5-wave systems emerge and both of them are plotted. Note that as the $f$ value increases, the daughter waves' vertical wavenumber also increases in the simulations (see the legend) which is in line with the theoretical predictions given in figure \ref{fig:k_m_k_minus_m_max_growth_rate_1p_vs_2p}(c). In all six simulations, inertial waves are present ($k=0$).  The growth rate of the daughter waves is calculated by estimating $d\ln{(\widehat{E})}/dt$. The comparison of growth rates from simulations and theory is presented in figure \ref{fig:Simulation_vs_theory_modes}, which shows a reasonably good agreement. For all the cases, the average of the three daughter waves' growth rate in a particular 5-wave system is taken. 
Moreover, figure \ref{fig:Simulation_vs_theory_modes} reveals that the growth rates are well above the maximum growth rate of all 3-wave systems. Note that the 5-wave interactions can happen for standing modes only at specific latitudes because the vertical wavenumbers are discrete, not continuous. However, for plane waves, there is no such constraint. As per the predictions in \S \ref{Section:3},  5-wave interactions should be faster than the 3-wave interactions provided $f/\omega_1 \gtrapprox 0.3$. 


It was observed that as $f/\omega_1 \xrightarrow[]{} 0.5$, multiple daughter wave combinations grow and extract a considerable amount of energy from the parent waves. This can even be seen in figure  \ref{fig:modes_log_energy}(f), where two different 5-wave systems emerge and extract a significant amount of energy. As $f/\omega_1 \xrightarrow[]{} 0.5$, multiple 5-wave systems can become coupled and grow at a rate which is faster  than any single 5-wave system (discussed in detail in \S \ref{sec_4_2}). Hence, the growth rates predicted from a single 5-wave interaction will not be accurate when $f \approx \omega_1/2$. As $f \xrightarrow[]{} \omega_1/2$, the growth rate for a mode-1 wave with zonal velocity amplitude $0.002\textnormal{ms}^{-1}$ will approach $2\sigma_{\textnormal{cl}}$ instead of $\sqrt{2}\sigma_{\textnormal{cl}}$, where $\sigma_{\textnormal{cl}}$ is the maximum growth rate for a plane wave with zonal velocity amplitude $0.001\textnormal{ms}^{-1}$ at the critical latitude \citep{young}.
%The growth rate for a mode-1 wave as $f \xrightarrow[]{} \omega_1/2$ will approach $2\sigma_{\textnormal{cl}}$ instead of $\sqrt{2}\sigma_{\textnormal{cl}}$, where $\sigma_{\textnormal{cl}}$ is the maximum growth rate for a plane wave with zonal velocity amplitude $0.001\textnormal{ms}^{-1}$ at the critical latitude \citep{young}.
We realise that in \cite{young}, the mode-1 wave was considered in the presence of a non-constant $N$. However, their prediction is still expected to hold in the present scenario (constant $N$). Our numerical simulations (results not shown here) also show the growth rates of the daughter waves being well above $\sqrt{2}\sigma_{\textnormal{cl}}$ for $f\approx\omega_1/2$. 

%When $f \approx \omega_1/2$, more than 2 triads become coupled leading to growth rate being more than the 5-wave prediction. 
%This is shown in figure \ref{} where the non-dimensionalised growth rates approach $2$ as $f \xrightarrow[]{} \omega_1/2$. 

\begin{figure}
 \centering{\includegraphics[width=1.0\textwidth]{Energy_Curves_plot_combined_Modes_Discrete_Part_4.png}}
 \caption{ 5-wave interactions for plane waves, mode-1, and mode-2 different $f$ values (i.e. latitudes), plotted in ascending order. $\widehat{t} \equiv t\omega_1/2\pi $. }
  \label{fig:modes_log_energy}
\end{figure} 

\begin{figure}
 \centering{\includegraphics[width=1.0\textwidth]{Simulation_vs_theory_modes.png}}
 \caption{ Comparison between theoretical growth rates and growth rates obtained from the simulations for $\mathbf{k}_1 =  (k_1,0,m_1)$ and $\mathbf{k}_5 =  (k_1,0,-m_1)$. Red (blue) markers indicate results from the simulations (theory), see legend. The black curve plots the variation of maximum growth rate of 3-wave systems with $f$. }
  \label{fig:Simulation_vs_theory_modes}
\end{figure} 

\begin{figure}
 \centering{\includegraphics[width=1.0\textwidth]{Energy_Curves_plot_combined_k_minus_k_PART_4.png}}
 \caption{ Four different 5-wave interactions for parent waves with wavevectors $\mathbf{k}_1 =  (k_1,0,m_1)$ and $\mathbf{k}_5 =  (-k_1,0,m_1)$. }
  \label{fig:k_minus_k_log_energy}
\end{figure} 

\begin{figure}
 \centering{\includegraphics[width=1.0\textwidth]{Simulations_vs_theory_k_minus_k.png}}
 \caption{ Comparison between theoretical growth rates and growth rates obtained from the simulations for $\mathbf{k}_1 =  (k_1,0,m_1)$ and $\mathbf{k}_5 =  (-k_1,0,m_1)$. Red markers indicate results from the simulations. Blue and green markers are predictions from the reduced order model. The black curve plots the variation of maximum growth rate of all 3-wave systems with $f$.  }
  \label{fig:Simulation_vs_theory_k_minus_k}
\end{figure} 

% \begin{figure}
%  \centering{\includegraphics[width=1.0\textwidth]{Simulations_vs_theory_Full_Part_2.png}}
%  \caption{ Comparison of theoretical growth rates to the growth rates obtained from the simulations. (a) $\mathbf{k}_1 =  (k_1,0,m_1)$ and $\mathbf{k}_5 =  (k_1,0,-m_1)$. (b) $\mathbf{k}_1 =  (k_1,0,m_1)$ and $\mathbf{k}_5 =  (-k_1,0,m_1)$. Red markers indicate results from the simulations. The black curve and the other markers are results from reduced order model. }
%   \label{fig:Simulation_vs_theory}
% \end{figure} 



\subsection{{$\mathbf{k}_1 =  (k_1,0,m_1)$ and $\mathbf{k}_5 =  (-k_1,0,m_1)$}} \label{sec_4_2}

We now validate 5-wave interactions for parent waves propagating in horizontally opposite directions. In this regard, we focus on latitudes where the daughter waves' vertical wavenumbers are multiples of $m_1/2$. Figure \ref{fig:k_minus_k_log_energy} shows the growth of daughter waves for four different $f/\omega_1$ values.  
%Figures \ref{fig:k_minus_k_log_energy}(a)--\ref{fig:k_minus_k_log_energy}(b) contain three waves that undergo exponential growth, hence all these waves must be daughter waves.
Figures \ref{fig:k_minus_k_log_energy}(a)--\ref{fig:k_minus_k_log_energy}(b) show energy in three wavevectors growing exponentially. The three wavevectors encompass both branch-1 and branch-2 daughter waves' wavevectors, and the simulation results are in line with theoretical predictions.
%Wavevectors predicted by the theoretical analysis for both 
%Here, the daughter waves of both Branch-1 and Branch-2 5-wave systems (predicted in the theoretical analysis) have the wavevectors shown in figure \ref{fig:k_minus_k_log_energy}(a)$\&$(b), see \S \ref{sec:horizontally_opposite_P_waves}. 
%Here, the daughter waves of both Branch-1 and Branch-2 5-wave systems (predicted in the theoretical analysis) have the wavevectors shown in figure \ref{fig:k_minus_k_log_energy}(a)$\&$(b), see \S \ref{sec:horizontally_opposite_P_waves}. 
%hence the daughter waves with the three wavevectors
%all these waves must be daughter waves.
%Here, the three wavevectors are actually the Branch-1 and Branch-2 5-wave systems predicted in the theoretical analysis, see \S \ref{sec:horizontally_opposite_P_waves}. 
The green curve (the wave with non-zero horizontal wavenumber) contains the energy of both leftward and rightward propagating waves. The growth rates estimated from the simulations are much higher than what is expected for a 3-wave interaction. For example, at $f=0.298\omega_1$, the growth rate of the daughter waves is $\approx30\%$ more than the growth rate of the individual 3-wave interactions that combine to form the 5-wave interaction. Figure \ref{fig:k_minus_k_log_energy}(c) shows only two daughter waves, which are part of the Branch-2 5-wave system. In this case, Branch-1 did not have a growth comparable to Branch-2. Finally, \ref{fig:k_minus_k_log_energy}(d) has three distinct 5-wave systems 
\begin{itemize}
    \item\, System-1 (daughter waves): $(k_1,0,4m_1)$, $(-k_1,0,4m_1)$, $(0,0,-3m_1)$,
    \item\, System-2 (daughter waves): $(k_1,0,4.5m_1)$, $(-k_1,0,4.5m_1)$, $(0,0,-3.5m_1)$,
    \item\, System-3 (daughter waves): $(k_1,0,-4.5m_1)$, $(-k_1,0,-4.5m_1)$, $(0,0,5.5m_1)$.
    \end{itemize}
System-1 is also present for $f/\omega_1=0.476$. This 5-wave system is present in both $f/\omega_1=0.476$ and $0.48$ because the change in $f$ is not that significant and hence the specific interaction is not expected to be detuned significantly. As a result, the system has an exponential growth. Even though the growth rates of System-2 and System-3 are observed to be higher than the growth rate of System-1, System-1 drains the largest amount of energy from the parent waves because the daughter waves in this system have a slightly higher energy at $t=0$. 
%For this combination of parent waves, the common daughter wave is the inertial wave. Moreover, the claim is validated by the fact that the growth rates are much more faster than the individual 3-wave systems' theoretical growth rates.
Growth rates obtained from the reduced order models are once again compared with the growth rates obtained from the numerical simulations, see figure \ref{fig:Simulation_vs_theory_k_minus_k}. When there are multiple branches growing,  the average growth rate of the (two) branches is taken since both branches have nearly the same growth rate. For $f/\omega_1=0.48$ in figure \ref{fig:k_minus_k_log_energy}(d), the average of system-2 and system-3's growth rates is compared with the theoretical growth rate since these are the two resonant Branch-1 and Branch-2 systems at $f/\omega_1=0.48$. It can be seen that theoretical predictions match reasonably well with the simulations. Moreover, similar to \S \ref{sec_4_1}, the growth rates of 5-wave systems are well above the maximum growth of 3-wave systems (shown by the black curve in figure \ref{fig:Simulation_vs_theory_k_minus_k}) for $f/\omega_1>0.4$.

%\todo{maybe a line on 3WS=>ok sure.=>}
%Note that once again the maximum growth rate among all 3-wave systems (denoted by the black curve in figure \ref{fig:Simulation_vs_theory_k_minus_k}) is well below the growth rate observed in the simulations showing that the interaction must be a 5-wave interaction.  

%%%%%%%%%%%%%%%%%%%%%%%%%%%%%%%%%%%%%%%%%%%%%%%%%%%%%%%%%%%%%%%%%%%%%%%%%%%%%%

% Growth rates obtained from the reduced order models are compared with the growth rates obtained from the numerical simulations. The results for both parent wave combinations are shown in figure \ref{fig:Simulation_vs_theory}. Figure \ref{fig:Simulation_vs_theory}(a) contains the growth rates of daughter waves for the six simulations shown in figure \ref{fig:modes_log_energy}. Figure\ref{fig:Simulation_vs_theory}(b) contains the growth rates of daughter waves for the four simulations shown in figure \ref{fig:k_minus_k_log_energy}. Figure \ref{fig:Simulation_vs_theory}(b) also contains growth rate data from two extra simulations that are not shown in figure \ref{fig:k_minus_k_log_energy}. For the cases shown in figure \ref{fig:Simulation_vs_theory}(a), the average of all daughter waves' growth rate present in a particular 5-wave system is plotted. For \ref{fig:Simulation_vs_theory}(b), the same approach is used for cases where there is only one 5-wave system present. If there are multiple Branches growing (as shown in figures \ref{fig:k_minus_k_log_energy}(b) and (d)), then the average of daughter waves' growth rate in both branches is taken. For $f/\omega_1=0.48$ in figure \ref{fig:k_minus_k_log_energy}(d), the average of system-2 and system-3's growth rate is compared with the theoretical growth rate since these are the two resonant systems (Branch-1 and Branch-2) at $f/\omega_1=0.48$. It can be seen that results match reasonably well for nearly all the simulations. The maximum growth rate among all 3-wave systems (denoted by the black curve in figure \ref{fig:Simulation_vs_theory}) is well below the growth rate observed in the simulations showing that the interaction is consistently a 5-wave interaction. This is observed for both parent wave combinations.

%%%%%%%%%%%%%%%%%%%%%%%%%%%%%%%%%%%%%%%%%%%%%%%%%%%%%%%%%%%%%%%%%%%%%%%%%%%%%%%

% Similar to section \ref{sec_4_1}, as $f \xrightarrow[]{} \omega_2$, multiple daughter wave systems grow at the expense of the parent wave. This can be seen in figure \ref{fig:k_minus_k_log_energy}(d) for $f=0.4834$. Here, six different 5-wave systems are present. Out of the 6 different systems, 3 systems are Branch-1, and 3 systems are Branch-2. In \ref{fig:k_minus_k_log_energy}(a), (b), and (c), a distinct 5-wave system emerged unlike the final case.

\subsection{Simulations and analysis for $f\approx \omega_1/2$}

% Similar to section \ref{sec_4_1}, as $f \xrightarrow[]{} \omega_1/2$, we observe that  growth rates from numerical simulations do not match the theoretical growth rates of 5-wave systems.
In \S \ref{sec_4_1}, we saw that the theoretical growth rates of 5-wave systems are not accurate for $f \approx \omega_1/2$. To test whether the 5-wave systems' growth rate holds near the critical latitude for $\mathbf{k}_1 =  (k_1,0,m_1)$ and $\mathbf{k}_5 =  (-k_1,0,m_1)$, we ran simulations for three different $f/\omega_1$ values: $f/\omega_1 = 0.496, 0.498$ and $0.499$. 
%For parent waves with $\mathbf{k}_1 =  (k_1,0,m_1)$ and $\mathbf{k}_5 =  (-k_1,0,m_1)$, to test whether the growth rates prediction of the 5-wave systems hold for $f \approx \omega_1/2$, we run simulations for three different $f/\omega_1$ values: $f/\omega_1 = 0.496, 0.498$ and $0.499$. 
Moreover, for each $f$, we ran three simulations: one  with $\nu = 10^{-6} \textnormal{m}^2\textnormal{s}^{-1}$, one with $\nu = 0.25 \times 10^{-6} \textnormal{m}^2\textnormal{s}^{-1}$, and finally one simulation with hyperviscous terms instead of viscous terms (i.e. by setting $\nu=0$). The hyperviscous operator $-\nu_{H}\Delta^4()$ is added to right hand side of  \eqref{eqn:total_mom_equation}--\eqref{eqn:boyancy_equation} with $\nu_H = 0.25 \times 10^{-6} \textnormal{m}^8\textnormal{s}^{-1}$. Hyperviscous terms are intended to make the simulation nearly inviscid, and they have been used previously to study PSI \citep{Haze_2011}. All simulations are run for $150$ time periods of the parent wave.  The simulations are stopped before the small-scale daughter waves attain energy comparable to the parent waves. The small-scale waves will break in such cases, and the ensuing turbulence is not resolved and is also not the focus of this study. We are only interested in the growth rate of the daughter waves. 

%Since the daughter waves can have very high vertical wavenumber near the critical latitude, to reduce the viscous damping, hyperviscous term is used for one simulation at each $f$ value.
%\todo{now you confuse the reader, you already mentioned that hyperv term is used for each f, what it means now?=> ah ok i can remove it. I have already mentioned that hyperviscous terms are intended to make the sims nearly inviscid. yes sorry=> The problem is that once you write, you don't read what you have written. So statements often are disconnected=> Sir to be honest I really read this part. My mistake was I wanted to say daughter waves have vertical wavenumbers and to reduce dissipation we use hyperviscous, but I have used too many words.=> Be more critical about your own writing, try to see if some other meaning, or confusion can result from the statements you make=>yes sorry. => I will edit this place.=> }

Figure \ref{fig:GR_contours_Simulations} shows the non-dimensionalised growth rates ($\sigma/\sigma_{\textnormal{cl}}$) of the daughter waves for all nine cases. In figure \ref{fig:GR_contours_Simulations}, each row is for a different $f$ value. Moreover, for each column, $\nu$ or $\nu_H$ is held constant. For the hyperviscous simulations and simulations with the lower viscosity, it can be seen that the non-dimensionalised growth rates are well above $\sqrt{2}$ for all three $f$-values (second and third column of figure \ref{fig:GR_contours_Simulations}). Daughter waves with $m=20-40m_1$ have $\sigma/\sigma_{\textnormal{cl}} \approx 1.85$ in the simulations with hyperviscous terms. For each $f$, simulations with $\nu=10^{-6}\textnormal{m}^2\textnormal{s}^{-1}$ have considerably lower growth rates (especially for higher wavenumbers) compared to the other simulations because of the viscous effects. 

We provide the reason for $\sigma/\sigma_{\textnormal{cl}}$ being well above $\sqrt{2}$ using the reduced order model. The dispersion relation for the daughter waves can be rewritten as
\begin{equation}
    (f+\delta\omega)^2 = \frac{N^2(nk_1)^2 + f^2m^2}{(nk_1)^2+m^2},
    \label{eqn:del_omega}
\end{equation}
where $\delta \omega$ is the difference between the wave's frequency ($f+\delta\omega$) and the inertial frequency ($f$), and $n$ is some constant {(but for our purposes, will  primarily be an integer)}.
Note that $k_1$ is the zonal wavenumber of the parent waves, but $m$ is \emph{not} the vertical wavenumber of parent waves.  Near the critical latitude, in a wave-wave interaction, any daughter wave's frequency would be approximately equal to the inertial frequency, implying $\delta \omega\!\ll\! f$. Hence \eqref{eqn:del_omega} leads to
%Assuming $\delta \omega \ll f$, equation \eqref{eqn:del_omega} would lead to
% \begin{equation}
%     \frac{\delta\omega}{f} = \frac{(N^2-f^2)(Ck_1)^2}{2f^2((Ck_1)^2+m^2)} \approx \frac{N^2(Ck_1)^2}{2f^2((Ck_1)^2+m^2)} \approx \frac{N^2(Ck_1)^2}{2f^2m^2}. 
%     \label{eqn:del_omega_simplification}
% \end{equation}
\begin{equation}
    \frac{\delta\omega}{f} \approx \frac{(N^2-f^2)(nk_1)^2}{2f^2[(nk_1)^2+m^2]} \ll 1.
    \label{eqn:del_omega_simplification}
\end{equation}
In scenarios where $N^2\gg f^2$, this yields
\begin{equation}
    m^2 \gg \frac{N^2(nk_1)^2}{2f^2}.
    \label{eqn:wavenumber_relation_crit}
\end{equation}
\begin{figure}
 \centering{\includegraphics[width=1.0\textwidth]{Critical_Growth_rates_Simulations_Part_2.png}}
 \caption{Growth rate contours ($\sigma/\sigma_{\textnormal{cl}}$) for the parent waves with wavevectors $\mathbf{k}_1 =  (k_1,0,m_1)$ and $\mathbf{k}_5 =  (-k_1,0,m_1)$ near the critical latitude ($f/\omega_1 \approx 0.5$). $f/\omega_1=0.496$ for Row-1 ((a), (b) and (c)), $f/\omega_1=0.498$ for Row-2 ((d), (e) and (f)), and $f/\omega_1=0.499$ for Row-3 ((g), (h) and (i)). Viscosity/hyperviscosity values used are as follows: $\nu = 10^{-6} \textnormal{m}^2\textnormal{s}^{-1}$ for Column-1 ((a), (d) and (g)), $\nu = 0.25 \times 10^{-6} \textnormal{m}^2\textnormal{s}^{-1}$ for Column-2 ((b), (e) and (h)), and $\nu_H = 0.25 \times 10^{-6} \textnormal{m}^8\textnormal{s}^{-1}$  for Column-3 ((c), (f) and (i)).
 %$f/\omega_1=0.496$, $f/\omega_1=0.498$, and $f/\omega_1=0.499$ is respectively used for the results in first row ((a), (b) and (c)), second row ((d), (e) and (f)), and the third row ((g), (h) and (i)). $\nu = 10^{-6} \textnormal{m}^2\textnormal{s}^{-1}$, $\nu = 0.25 \times 10^{-6} \textnormal{m}^2\textnormal{s}^{-1}$, and $\nu_H = 0.25 \times 10^{-6} \textnormal{m}^8\textnormal{s}^{-1}$ is respectively used for the results in the first, second, and the third column.
 }
  \label{fig:GR_contours_Simulations}
\end{figure} 
% \begin{figure}
%  \centering{\includegraphics[width=1.0\textwidth]{Del_omega_with_f_different_indices.png}}
%  \caption{ Variation of frequency mismatch with $n$ for different $f$ values. (a) $f/\omega_1 = 0.496$, (b) $f/\omega_1 = 0.498$, and (c) $f/\omega_1 = 0.499$. The number (a,b) in the horizontal axis implies the daughter waves in the specific interaction have horizontal wavenumbers $ak_1$ and $bk_1$.    }
%   \label{fig:f_omega_detuning}
% \end{figure} 
% $\delta\omega/f$ will be a very small quantity (provided $N^2\gg f^2$).
% \noindent The frequency of the waves whose wavenumbers satisfy equation \eqref{eqn:wavenumber_relation_crit} would be  $\approx f$. Moreover,

\noindent Near the critical latitude, $2f \approx \omega_1$. As a result, the sum of two daughter waves' frequencies would be $\approx\!\omega_1$ provided their wavenumbers satisfy  \eqref{eqn:wavenumber_relation_crit}.  As a consequence of this special scenario, a chain of coupled triads is possible as shown in figure \ref{fig:schematic_coupling}.  Every box contains the wavevector of a daughter wave. The absolute value of the horizontal wavenumber is lowest at the center of the chain, and it increases in either direction. However, the vertical wavenumber takes only two values. Note that $n$ would be an integer considering how the absolute value of the horizontal wavenumber increases in either direction of the central box $(0,0,m)$. Any two boxes that are connected by the same blue line add up to give a parent wave's wavevector. For example, $(2k_1,0,m)+(-k_1,0,m_1-m)$ gives $(k_1,0,m_1)$, which is the wavevector of one of the parent waves. Moreover, $(-k_1,0,m_1-m)+(0,0,m)$ gives $(-k_1,0,m_1)$, which is the other parent wave's wavevector. %For this chain to be resonant, the frequency of the wavevectors should be close to $\omega_1/2$. %Figure \ref{fig:f_omega_detuning} shows the variation in frequency mismatch with $n$ for different vertical wavenumber $(m)$ and $f/\omega_1$. For all three $f$ values, increasing $m$ decreases the mismatch in frequency. Moreover, as $n$ is increased the mismatch increases. 
%Note that \todo{n has to be defined in text. Also, why not use n in (4.5) ?=>}
Except for the daughter waves at the ends of the chain, every daughter wave would be forced by both parent waves. Assuming the wavenumbers of the daughter waves in the chain satisfy  \eqref{eqn:wavenumber_relation_crit},  the sum of any two waves' frequencies would be  $\approx \omega_1$, thus satisfying all the required triad conditions. For a fixed $m$, $\delta \omega$ would increase as $n$ is increased, which is evident from  \eqref{eqn:del_omega}. Hence for very large $n$, the daughter wave's frequency ($f+\delta\omega$) cannot be approximated by $f$ and the sum of two daughter waves' frequencies cannot be approximated by $\omega_1$ simply because $\delta\omega$ would be large. As a result, the triad conditions would not be satisfied for very large $n$. Assuming $\delta \omega$ is negligible up to some $n$,  the wave amplitude equations for the $2n+1$ daughter waves shown in figure \ref{fig:schematic_coupling} can be written in a compact way as follows:
%\todo{better to relate it to (2.13) as a generalization or something=>yes nice=>}
\begin{align}
\frac{d \mathbf{a}}{dt} &= \mathcal{H} \mathbf{\bar{a}} \label{eqn:triad_chain_amplitude_equation} \\
\mathbf{a} &= [a_{-n}\hspace{0.2cm} a_{1-n} \hspace{0.2cm} \dots \hspace{0.2cm} a_{n-1} \hspace{0.2cm} a_{n}]^T   \\
\mathcal{H} &= \begin{bmatrix}
 -\mathcal{V}_{-n} & \mathcal{M}_{(-n,1-n)} {A}_1  & 0  & 0  & 0  \\ 
\mathcal{M}_{(1-n,-n)} {A}_1 & -\mathcal{V}_{1-n}  & \mathcal{N}_{(1-n,2-n)} {A}_5  & 0  & 0  \\ 
\vdots & \ddots  & \ddots & \ddots  & \vdots  \\ 
 0  & 0  &\mathcal{M}_{(n-1,n-2)} {A}_1  & -\mathcal{V}_{n-1}  & \mathcal{N}_{(n-1,n)} {A}_5 \\ 
 0 & 0  & 0  & \mathcal{N}_{(n,n-1)} {A}_5   & -\mathcal{V}_{n} \\ 
\end{bmatrix} 
\label{eqn:Matrix_of_connection_of_waves}
\end{align}
where the coefficients $\mathcal{N}_{(i,j)}$ and $\mathcal{M}_{(i,j)}$ are given by
\begin{equation}
    \mathcal{N}_{(i,j)} = \frac{\mathfrak{N}_{(i,5,j)}}{\mathcal{D}_{i}}, \hspace{0.5cm}     \mathcal{M}_{(i,j)} = \frac{\mathfrak{N}_{(i,1,j)}}{\mathcal{D}_{i}}.
\end{equation}
The expression for $\mathfrak{N}_{(i,*,j)}$ is given in Appendix \ref{App:A}. Equation \eqref{eqn:triad_chain_amplitude_equation} is an extension of the system given in \eqref{eq:scatter_matrix_scho} to an arbitrary number of daughter waves. Note that using $n=1$ in \eqref{eqn:triad_chain_amplitude_equation} would result in equation \eqref{eq:scatter_matrix_scho}.
%is an extension of the 
The growth rate for the system given in  \eqref{eqn:triad_chain_amplitude_equation} can be found by calculating the eigenvalues of $\mathcal{H}$. In addition to  the $\mathbf{k}_1 =  (k_1,0,m_1)$ and $\mathbf{k}_5 =  (-k_1,0,m_1)$ case, we also analyze the theoretical growth rates for oblique parent waves near the critical latitude using  \eqref{eqn:triad_chain_amplitude_equation}. 
%Figure \ref{fig:f_critical_latitude} shows the gradual increase of the growth rate as $n$ increases for 4 different $\theta$ values: $\theta=\pi/4$, $\pi/2$, $3\pi/4$, and $\pi$ (see equation \eqref{eqn:theta_definition} for the definition of $\theta$). $\nu=0$ and $f=\omega_1(1/2 - 10^{-6})$ are taken. 
To this end, we consider four $\theta$ values: $\theta=\pi/4$, $\pi/2$, $3\pi/4$, and $\pi$ (see  \eqref{eqn:theta_definition} for the definition of $\theta$). For $\theta \neq \pi$, the parent waves have a non-zero meridional wavenumber ($l_1$). In such cases, the meridional wavenumber of all the daughter waves in the chain is simply assumed to be $l_1/2$. For all four $\theta$ values, figure \ref{fig:f_critical_latitude} shows the gradual increase of the growth rate as $n$ increases for two different $m$ values. The vertical wavenumbers are chosen to be a large value so that they satisfy  \eqref{eqn:wavenumber_relation_crit} up to $n=7$. For all the $\theta$ values, $\sigma/\sigma_{\textnormal{cl}} \approx 2$ for the higher $n$ values, which is what we observed in the simulation results shown in figure \ref{fig:GR_contours_Simulations}. Moreover, for $n=1$, $\sigma/\sigma_{\textnormal{cl}} \approx \sqrt{2}$  which is what we would expect for a 5-wave system with three daughter waves. 
%todo{exact location in this subsection=>}
% For $\nu=0$, and when $f/\omega_1 \approx 0.5$, the maximum growth rate of the triad chain shown in figure \ref{fig:schematic_coupling} approaches $2\sigma_{\textnormal{cl}}$ for high values of $n$.  \
%
Interestingly, for an oblique set of parent waves, the results are similar to the 2D case. Hence, 5-wave system growth rates do not apply near the critical latitude for an oblique set of parent waves as well. Note that even though high values of $m$ are used in the reduced order model, simulations show that the resonance can occur even at $m=20-40m_1$. As a result, near the critical latitude, regardless of the $\theta$ value, two parent waves force daughter waves as if they are a single wave with approximately twice the amplitude. 
\begin{figure}
 \centering{\includegraphics[width=1.0\textwidth]{K_downscale_energy.png}}
 \caption{ A simplified schematic showing how different daughter waves are coupled. Any two wavevectors (boxes) connected by the same blue line can act as a daughter wave combination for the wavevector $\textbf{k}_1 =  (k_1,0,m_1)$ or $\textbf{k}_5 =  (-k_1,0,m_1)$. }
  \label{fig:schematic_coupling}
\end{figure} 

 \begin{figure}
 \centering{\includegraphics[width=1.0\textwidth]{Theta_Variation_Growth_rate_critical_latitude.png}}
 \caption{  Variation of maximum growth rate with $n$ for triad chains near the critical latitude. (a) $\theta=\pi/4$, (b) $\theta=\pi/2$, (c) $\theta=3\pi/4$, and (d) $\theta=\pi$. Two different $m$ values are shown for each $\theta$. }
  \label{fig:f_critical_latitude}
\end{figure}

% Note that for a particular $n$, there would be $2n+1$ daughter waves present in the system (see figure \ref{fig:schematic_coupling}). 

% Results there for f = 0.498 and 0.496. nu = -6 and 0.25*-6. decision on 5 or7 or 9 wave. 
% If we take 7 or 9 wave then near m=20 it is not valid. m=40 or m=60 it may be valid.
% Inviscid: hypervisocsity simulations will help.
% \begin{equation}
% \mathbf{a} = [a_{-n}\hspace{0.2cm} a_{1-n} \hspace{0.2cm} \dots \hspace{0.2cm} a_{n-1} \hspace{0.2cm} a_{n}]^T
% \begin{bmatrix}
% \dot{a}_{-n}\\
% \dot{a}_{1-n}\\
%   \vdots    \\
% \dot{a}_{n-1}\\
% \dot{a}_{n}
% \end{bmatrix} 
% =  \begin{bmatrix}
%  0 & \mathcal{N}_{(-n,1-n)} {A}_5  & 0  & 0  & 0  \\ 
% \mathcal{N}_{(1-n,-n)} {A}_5 & 0  & \mathcal{M}_{(1-n,2-n)} {A}_1  & 0  & 0  \\ 
% \vdots & \ddots  & \ddots & \ddots  & \vdots  \\ 
%  0  & 0  &\mathcal{N}_{(n-1,n-2)} {A}_5  & 0  & \mathcal{M}_{(n-1,n)} {A}_1 \\ 
%  0 & 0  & 0  & \mathcal{M}_{(n,n-1)} {A}_1   & 0 \\ 
% \end{bmatrix} 
% \begin{bmatrix}
% \bar{a}_{-n}\\
% \bar{a}_{1-n}\\
%   \vdots    \\
% \bar{a}_{n-1}\\
% \bar{a}_{n}
% \end{bmatrix} 
% \label{eq:scatter_matrix_scho}
% \end{equation}
% \begin{equation}
%     \mathcal{N}_{i} = \frac{\mathfrak{N}_{(i,5,i+1)}}{\mathcal{D}_{i}}, \hspace{2cm} \mathcal{M}_{i} = \frac{\mathfrak{N}_{(i,1,i-1)}}{\mathcal{D}_{i}}
% \end{equation}
 
\section{Conclusions} \label{Section:5}

Wave-wave interactions play a major role in the energy cascade of internal gravity waves. In this paper, we use multiple scale analysis to study wave-wave interactions of two plane parent waves co-existing in a region. The main instability mechanism that is focused on is the 5-wave system instability that involves two parent waves and three daughter waves. The 5-wave system is composed of two different triads (3-wave systems) with one daughter wave being a part of both triads (see figure \ref{fig:5_wave_triad_schematic}(c)). For parent waves with wavevectors $(k_1,0,m_1)$ and $(k_1,0,-m_1)$, the 5-wave system is only possible when the common daughter wave's frequency is almost equal to $\omega_1 - f$ (where $\omega_1$ is the parent wave's frequency). The other two daughter waves are inertial waves that always propagate in vertically opposite directions. The growth rate of the above-mentioned 5-wave system is higher than the maximum growth rate of 3-wave systems for $f/\omega_1\gtrapprox0.3$. For parent waves with wavevectors $(k_1,0,m_1)$ and $(-k_1,0,m_1)$ (parent waves that propagate in horizontally opposite directions), similar to the previous parent wave combination, the maximum growth rate of 5-wave systems is higher than the maximum growth rate of 3-wave systems for $f/\omega_1\gtrapprox0.3$. For $f/\omega_1\gtrapprox0.3$, the common daughter wave's frequency is nearly equal to $f$ in the most unstable 5-wave system. Moreover, as the common daughter wave's frequency is increased from $f$, the meridional wavenumber increases significantly while the zonal wavenumber of the common daughter wave stays negligible. 

We also study 5-wave systems for cases where the two parent waves are not confined to the same vertical plane. In such scenarios, the dominance of the 5-wave systems increase as the angle between the horizontal wavevectors of the parent waves (denoted by $\theta$) is decreased. Moreover, for any $\theta$, the 5-wave system's instability is more dominant than the 3-wave system's instability for $f \gtrapprox 0.3\omega_1$.
%the maximum growth rate of 5-wave systems decreased in the lower latitudes as the angle between the horizontal wavevectors of the parent waves increased. Regardless of the angle between the parent waves' horizontal wavevector, fo, the 5-wave system instability is more dominant than the 3-wave system instability. 
Numerical simulations are conducted to test the theoretical predictions, and the theoretical growth rate of the 5-wave systems matches reasonably well with the results of the numerical simulations for a wide range of $f-$values. However, for all the 2D parent wave combinations considered,  the growth rates from the simulations do not match the theoretical 5-wave systems' growth rate near the critical latitude where $f\approx\omega_1/2$. Near the critical latitude, more than two triads become coupled, hence a chain of daughter waves is forced by the two parent waves. By modifying the reduced order model to account for a chain of daughter waves, the maximum growth rate is shown to be twice the maximum growth rate of all 3-wave systems. Moreover, the reduced order model showed similar results for parent waves that are not on the same vertical plane. Hence, near the critical latitude, the 5-wave systems' prediction is not expected to hold for oblique parent waves as well. \vspace{1cm}
%Numerical simulations show that the growth rates for a mode-1 internal wave near the critical latitude matches reasonably well with the theoretical 5-wave system's growth rates.
   
\noindent \textbf{Declaration of interests.} The authors report no conflict of interest. 
   
\appendix
\section{Nonlinear coupling coefficients} \label{App:A}

\noindent The quantities $\mathfrak{N}_{(j,p,d)}$ and $\mathfrak{O}_{(j,b,c)}$ are defined so that the nonlinear coefficients can be written in a compact form:
 % The expressions for $\textnormal{NLT}_j$ is provided below for the daughter waves in equation \eqref{eqn:daughter_NLT}. Hence equation \eqref{eqn:daughter_NLT} can be used only for $j=(2,3,4)$.

%%%%%%%%%%%%%%%%%%%%%%%%%%%%%%%%%%55 Original Version

% \begin{align}
%     \mathfrak{N}_{(j,p,d)} = &-{(k_p-k_d)(m_p-m_d)(\omega_p-\omega_d)} \left[ \left( U_p\bar{U}_d(k_p-k_d)  + U_p\bar{V}_d l_p - V_p\bar{U}_dl_d + U_p m_p - m_d\bar{U}_d  \right) \right] \nonumber \\
%    &- {(l_p-l_d)(m_p-m_d)(\omega_p-\omega_d)} \left[ \left( V_p\bar{V}_d(l_p-l_d)  - U_p\bar{V}_d k_d + V_p\bar{U}_d k_p + V_p m_p - m_d\bar{V}_d  \right) \right] \nonumber \\
%    &+ {(\omega_p-\omega_d)((l_p-l_d)^2+(k_p-k_d)^2)} \left[   \bar{V}_d l_p-l_dV_p  + m_p - m_d + k_p\bar{U}_d - U_p k_d \right] \nonumber \\
%    &+ \textnormal{i}{((l_p-l_d)^2+(k_p-k_d)^2)} \left[  \bar{U}_d B_p k_p  - U_p\bar{B}_d k_d + \bar{V}_d B_p l_p  - V_p\bar{B}_d l_d  + B_p m_p  - \bar{B}_d m_d      \right] \nonumber \\
%    &+ \textnormal{i}{f(l_p-l_d)(m_p-m_d)} \left[ \left( U_p\bar{U}_d(k_p-k_d)  + U_p\bar{V}_d l_p - V_p\bar{U}_dl_d + U_p m_p - m_d\bar{U}_d  \right) \right] \nonumber \\
%    &- \textnormal{i}{f(k_p-k_d)(m_p-m_d)} \left[ \left( V_p\bar{V}_d(l_p-l_d)  - U_p\bar{V}_d k_d + V_p\bar{U}_d k_p + V_p m_p - m_d\bar{V}_d  \right) \right] \nonumber \\
%    \label{eqn:daughter_NLT}
% \end{align}

% \begin{align}
%      \mathfrak{O}_{(j,b,c)} = &-{(k_b+k_c)(m_b+m_c)(\omega_b+\omega_c)} \left[ \left( U_b{U}_c(k_b+k_c)  + U_b{V}_c l_b + V_b{U}_cl_c + U_b m_b + {U}_cm_c  \right) \right] \nonumber \\
%    &- {(l_b+l_c)(m_b+m_c)(\omega_b+\omega_c)} \left[ \left( V_b{V}_c(l_b+l_c)  + U_b{V}_c k_c + V_b{U}_c k_b + V_b m_b + m_c{V}_c  \right) \right] \nonumber \\
%    &+ {((l_b+l_c)^2+(k_b+k_c)^2)(\omega_b+\omega_c)} \left[   {V}_c l_b+l_cV_b  + m_b + m_c + k_b{U}_c + U_b k_c \right] \nonumber \\
%    &+ \textnormal{i}{((l_b+l_c)^2+(k_b+k_c)^2)} \left[  {U}_c B_b k_b  + U_b{B}_c k_c + {V}_c B_b l_b  + V_b{B}_c l_c  + B_b m_b  + {B}_c m_c      \right] \nonumber \\
%    &+ \textnormal{i}{f(l_b+l_c)(m_b+m_c)} \left[ \left( U_b{U}_c(k_b+k_c)  + U_b{V}_c l_b + V_b{U}_cl_c + U_b m_b + m_c{U}_c  \right) \right] \nonumber \\
%    &- \textnormal{i}{f(k_b+k_c)(m_b+m_c)} \left[ \left( V_b{V}_c(l_b+l_c)  + U_b{V}_c k_c + V_b{U}_c k_b + V_b m_b + m_c{V}_c  \right) \right] \nonumber \\
%    \label{eqn:parent_NLT}
% \end{align} 

%%%%%%%%%%%%%%%%%%%%%%%%%%%%%%%%%%%%%%%%%%%%%%% Condensed Version


\begin{align}
    \mathfrak{N}_{(j,p,d)} = &-{(\omega_p-\omega_d)k_jm_j} \left[ \left( U_p\bar{U}_d(k_j)  + U_p\bar{V}_d l_p - V_p\bar{U}_dl_d + U_p m_p - m_d\bar{U}_d  \right) \right] \nonumber \\
   &- {(\omega_p-\omega_d)l_jm_j} \left[ \left( V_p\bar{V}_d(l_j)  - U_p\bar{V}_d k_d + V_p\bar{U}_d k_p + V_p m_p - m_d\bar{V}_d  \right) \right] \nonumber \\
   &+ {(\omega_p-\omega_d)(l_j^2+k_j^2)} \left[   \bar{V}_d l_p-l_dV_p  + m_j + k_p\bar{U}_d - U_p k_d \right] \nonumber \\
   &+ \textnormal{i}{(l_j^2+k_j^2)} \left[  \bar{U}_d B_p k_p  - U_p\bar{B}_d k_d + \bar{V}_d B_p l_p  - V_p\bar{B}_d l_d  + B_p m_p  - \bar{B}_d m_d      \right] \nonumber \\
   &+ \textnormal{i}{fl_jm_j} \left[ \left( U_p\bar{U}_d(k_j)  + U_p\bar{V}_d l_p - V_p\bar{U}_dl_d + U_p m_p - m_d\bar{U}_d  \right) \right] \nonumber \\
   &- \textnormal{i}{fk_jm_j} \left[ \left( V_p\bar{V}_d(l_j)  - U_p\bar{V}_d k_d + V_p\bar{U}_d k_p + V_p m_p - m_d\bar{V}_d  \right) \right],
   \label{eqn:daughter_NLT}
\end{align}
\begin{align}
     \mathfrak{O}_{(j,b,c)} = &-{(\omega_b+\omega_c)k_jm_j} \left[ \left( U_b{U}_c(k_j)  + U_b{V}_c l_b + V_b{U}_cl_c + U_b m_b + {U}_cm_c  \right) \right] \nonumber \\
   &- {(\omega_b+\omega_c)l_jm_j} \left[ \left( V_b{V}_cl_j  + U_b{V}_c k_c + V_b{U}_c k_b + V_b m_b + m_c{V}_c  \right) \right] \nonumber \\
   &+ {(\omega_b+\omega_c)(l_j^2+k_j^2)} \left[   {V}_c l_b+l_cV_b  + m_j + k_b{U}_c + U_b k_c \right] \nonumber \\
   &+ \textnormal{i}{(l_j^2+k_j^2)} \left[  {U}_c B_b k_b  + U_b{B}_c k_c + {V}_c B_b l_b  + V_b{B}_c l_c  + B_b m_b  + {B}_c m_c      \right] \nonumber \\
   &+ \textnormal{i}{fl_jm_j} \left[ \left( U_b{U}_ck_j  + U_b{V}_c l_b + V_b{U}_cl_c + U_b m_b + m_c{U}_c  \right) \right] \nonumber \\
   &- \textnormal{i}{fk_jm_j} \left[ \left( V_b{V}_cl_j  + U_b{V}_c k_c + V_b{U}_c k_b + V_b m_b + m_c{V}_c  \right) \right], 
   \label{eqn:parent_NLT}
\end{align} 

%%%%%%%%%%%%%%%%%%%%%%%%%%%%%%%%%%%%%%%%%%%%%%%%%%%%%%%%%%%%%%%%%%

% \begin{table}
% \centering
% \begin{tabular}{ c c c c c c c } 
%  \hline
%   & $\mathcal{M}_1$ & $\mathcal{M}_2$ & $\mathcal{M}_3$ & $\mathcal{N}_3$ & $\mathcal{N}_4$ & $\mathcal{N}_5$ \\ 
%  \hline
% \hspace{0.1cm} & 3/100 \hspace{0.1cm}& 1/100   &\hspace{0.1cm} 10   &\hspace{0.1cm} 0.30   & \hspace{0.1cm} 0.414   &\hspace{0.1cm}  \\ 
%  \hline
%  \hspace{0.1cm}&  2/100 \hspace{0.1cm}& 1/100   &\hspace{0.1cm}  8  \hspace{0.1cm}& \hspace{0.1cm} 0.20  &\hspace{0.1cm}  0.432  &\hspace{0.1cm}  \\ 
%  \hline
%  \hspace{0.1cm}&  3/100 \hspace{0.1cm}& 1/100   &\hspace{0.1cm}  14  \hspace{0.1cm}& \hspace{0.1cm} 0.35  &\hspace{0.1cm}  0.446  &\hspace{0.1cm}  \\  \hline
% \end{tabular}
% \caption{}
% \label{tab:1}
% \end{table}
\noindent where the indices $(j,p,d,b,c)$ are used to denote waves.
Using  \eqref{eqn:daughter_NLT} and \eqref{eqn:parent_NLT}, the nonlinear terms and coefficients used in wave amplitude equations can be written as
\begin{align}
    \mathcal{M}_1 &= \frac{\mathfrak{O}_{(1,2,3)}}{\mathcal{D}_{1}}, \hspace{0.5cm} \mathcal{M}_2 = \frac{\mathfrak{N}_{(2,1,3)}}{\mathcal{D}_{2}}, \hspace{0.5cm} \mathcal{M}_3 = \frac{\mathfrak{N}_{(3,1,2)}}{\mathcal{D}_{3}}, \\
    \mathcal{N}_5 &= \frac{\mathfrak{O}_{(5,3,4)}}{\mathcal{D}_{5}}, \hspace{0.5cm} \mathcal{N}_4= \frac{\mathfrak{N}_{(4,5,3)}}{\mathcal{D}_{4}}, \hspace{0.5cm} \mathcal{N}_3 = \frac{\mathfrak{N}_{(3,5,4)}}{\mathcal{D}_{3}}, \\
        \textnormal{NLT}_1 &= \mathcal{M}_1 \mathcal{D}_{1} a_2 a_3, \hspace{0.5cm}   \textnormal{NLT}_5 = \mathcal{N}_5 \mathcal{D}_{5} a_3 a_4, \\
     \textnormal{NLT}_4 &= \mathcal{N}_4 \mathcal{D}_{4} a_5 \bar{a}_3, \hspace{0.5cm}   \textnormal{NLT}_3 = \mathcal{N}_3 \mathcal{D}_{3} a_5 \bar{a}_4 + \mathcal{M}_3 \mathcal{D}_{3} a_1 \bar{a}_2, \hspace{0.5cm}  \textnormal{NLT}_2 = \mathcal{M}_2 \mathcal{D}_{2} a_1 \bar{a}_3. 
\end{align}


% \section{New Math}

          
% Pertubation anzatz: (Two daughter waves)

% \begin{align}
%     u'  &= Fu3(x) \exp(\ii (\omega_1/2 + \sigma) t)  \exp(s t)  \exp(\ii(l_3y+m_3z)) + \textnormal{c.c.} \nonumber \\
%        &+ Fu2(x) \exp(\ii (\omega_1/2 - \sigma) t)  \exp( \bar{s} t)  \exp(\ii( l_2y+m_2z)) + \textnormal{c.c.} \nonumber    
% \end{align}

% \begin{align}
%     w'  &= Fw3(x) \exp(\ii (\omega_1/2 + \sigma) t)  \exp(s t)  \exp(\ii(l_3y+m_3z)) + \textnormal{c.c.} \nonumber \\
%        &+ Fw2(x) \exp(\ii (\omega_1/2 - \sigma) t)  \exp(\bar{s} t)  \exp(\ii( l_2y+m_2z)) + \textnormal{c.c.} \nonumber    
% \end{align}


% \begin{align}
%     v'  &= Fv3(x) \exp(\ii (\omega_1/2 + \sigma) t)  \exp(s t)  \exp(\ii(l_3y+m_3z))+ \textnormal{c.c.} \nonumber \\
%        &+ Fv2(x) \exp(\ii (\omega_1/2 - \sigma) t)  \exp(\bar{s} t)  \exp(\ii( l_2y+m_2z)) + \textnormal{c.c.} \nonumber    
% \end{align}


% \begin{align}
%     b'  &= Fb3(x) \exp(\ii (\omega_1/2 + \sigma) t)  \exp(s t)  \exp(\ii(l_3y+m_3z))+ \textnormal{c.c.} \nonumber \\
%        &+ Fb2(x) \exp(\ii (\omega_1/2 - \sigma) t)  \exp(\bar{s} t)  \exp(\ii( l_2y+m_2z)) + \textnormal{c.c.} \nonumber    
% \end{align}

% \begin{align}
%     p'  &= Fp3(x) \exp(\ii (\omega_1/2 + \sigma) t)  \exp(s t)  \exp(\ii(l_3y+m_3z)) + \textnormal{c.c.} \nonumber \\
%        &+ Fp2(x) \exp(\ii (\omega_1/2 - \sigma) t)  \exp(\bar{s} t)  \exp(\ii( l_2y+m_2z)) + \textnormal{c.c.} \nonumber    
% \end{align}



% \begin{align}
%     \frac{\partial u}{\partial t} - fv + \frac{\partial p'}{\partial x} - \nu \Delta u &=  NLu \label{eqn:u1_S4} \\
%     \frac{\partial v}{\partial t} + fu + \frac{\partial p'}{\partial y} - \nu \Delta v  &= NLv  \label{eqn:v1_S4} \\
%     \frac{\partial w}{\partial t}  + \frac{\partial p'}{\partial z} - b - \nu \Delta w &=  NLw  \label{eqn:w1_S4}\\ 
%     \frac{\partial b}{\partial t} + N^2w &=  NLb \label{eqn:b1_S4}\\
%     \frac{\partial u}{\partial x} + \frac{\partial v}{\partial y}  + \frac{\partial w}{\partial z} &= 0
% \label{eqn:Continuity_S4}
%  \end{align}

% \begin{equation}
%  {m}_3 \equiv  \mathcal{V}m_1, \hspace{0.5cm} {m}_2 \equiv  (1-\mathcal{V})m_1, \hspace{0.5cm} \widehat{l}_3 \equiv  \mathcal{H}l_1, \hspace{0.5cm} \widehat{l}_2 \equiv (1-\mathcal{H})l_1,
% \end{equation}



% \begin{align}
%     &(\ii(\omega_1/2 + \sigma) + s - \nu{m}_3^2 - \nu l_3^2 )Fu3   - f Fv3 + \frac{d Fp3}{d x} - \nu \frac{d^2 Fu3}{d x^2}   =  NLu3 \label{eqn:u1_S4} \\
%     &(\ii(\omega_1/2 + \sigma) + s - \nu m_3^2 - \nu {l}_3^2 )Fv3   + f Fu3 +  \ii l_3 Fp3 - \nu \frac{d^2 Fv3}{d x^2}  = NLv3  \label{eqn:v1_S4} \\
%     &(\ii(\omega_1/2 + \sigma) + s - \nu{m}_3^2 - \nu{l}_3^2 )Fw3 - Fb3 + \ii {m}_3 Fp3  - \nu \frac{d^2 Fw3}{d x^2} =  NLw3  \label{eqn:w1_S4}\\ 
%     &(\ii(\omega_1/2 + \sigma) + s )Fb3 + N^2Fw3 =  NLb3 \label{eqn:b1_S4}\\
%    &\frac{d Fu3}{d x}  + \ii {l}_3Fv3  +  \ii{m}_3 Fw3 = 0
% \label{eqn:Continuity_S4}
%  \end{align}

% %%%%%%%%%%%%%%%%%%%%%%%%%%%% Conjugate version

% \begin{align}
%     &(-\ii(\omega_1/2 - \sigma) + s - \nu{m}_2^2 - \nu{l}_2^2 )\bar{Fu2}   - f \bar{Fv2} + \frac{d \bar{Fp2}}{d x} - \nu \frac{d^2 \bar{Fu2}}{d x^2}   =  \bar{NLu2} \label{eqn:u1_S4} \\
%     &(-\ii(\omega_1/2 - \sigma) + s - \nu{m}_2^2 - \nu{l}_2^2 )\bar{Fv2}   + f \bar{Fu2} - \ii {l}_2 \bar{{Fp2}} - \nu \frac{d^2 \bar{Fv2}}{d x^2}  = \bar{NLv2}  \label{eqn:v1_S4} \\
%     &(-\ii(\omega_1/2 - \sigma) + s - \nu{m}_2^2 - \nu{l}_2^2 )\bar{Fw2} - \bar{Fb2}  - \ii {m}_2 \bar{Fp2}  - \nu \frac{d^2 \bar{Fw2}}{d x^2} =  \bar{NLw2}  \label{eqn:w1_S4}\\ 
%     &(-\ii(\omega_1/2 - \sigma) + s )\bar{Fb2} + N^2\bar{Fw2} =  \bar{NLb2} \label{eqn:b1_S4}\\
%    &\frac{d \bar{Fu2}}{d x}  - \ii {l}_2 \bar{Fv2}  -  \ii{m}_2\bar{Fw2} = 0
% \label{eqn:Continuity_S4}
%  \end{align}

% %%%%%%%%%%%%%%%%%%%%%%%%%%%%% Correct Version
 
% % \begin{align}
% %     &(\ii(\omega_1/2 - \sigma) + s - \nu((1-\mathcal{V})m_1)^2 - \nu((1-\mathcal{H})k_1cos\theta/2)^2 )Fu2   - f Fv2 + \frac{d Fp2}{d x} - \nu \frac{d^2 Fu2}{d x^2}   =  NLu2 \label{eqn:u1_S4} \\
% %     &(\ii(\omega_1/2 - \sigma) + s - \nu((1-\mathcal{V})m_1)^2 - \nu((1-\mathcal{H})k_1cos\theta/2)^2 )Fv2   + f Fu2 +  \ii ((1-\mathcal{H})k_1cos\theta/2) Fp2 - \nu \frac{d^2 Fv2}{d x^2}  = NLv2  \label{eqn:v1_S4} \\
% %     &(\ii(\omega_1/2 - \sigma) + s - \nu((1-\mathcal{V})m_1)^2 - \nu((1-\mathcal{H})k_1cos\theta/2)^2 )Fw2 + \ii ((1-\mathcal{V})m_1) Fp2 - Fb2 - \nu \frac{d^2 Fw2}{d x^2} =  NLw2  \label{eqn:w1_S4}\\ 
% %     &(\ii(\omega_1/2 - \sigma) + s )Fb1 + N^2Fw2 =  NLb2 \label{eqn:b1_S4}\\
% %    &\frac{d Fu2}{d x}  + \ii ((1-\mathcal{H})k_1cos\theta/2)Fv2  +  \ii((1-\mathcal{V})m_1)Fw2 = 0
% % \label{eqn:Continuity_S4}
% %  \end{align}

% %%%%%%%%%%%%%%%%%%%%%%%%%%%%%%%%%

% \begin{align}
%  NLu &=  {U}u'_x + u'{U}_x + {V}u'_y + v'{U}_y + {W}u'_z + w'{U}_z  \label{eqn:u1} \\
%  NLv &=  {U}v'_x + u'{V}_x + {V}v'_y + v'{V}_y + {W}v'_z + w'{V}_z  \label{eqn:v1} \\
%  NLw &=  {U}w'_x + u'{W}_x + {V}w'_y + v'{W}_y + {W}w'_z + w'{W}_z  \label{eqn:w1} \\
%  NLb &=  {U}b'_x + u'{B}_x + {V}b'_y + v'{B}_y + {W}b'_z + w'{B}_z  \label{eqn:b1} 
%  \end{align} \vspace{0.5cm}


%  \noindent Base flow being used:

% \begin{align}
%     {W}  &= ( W_+ \exp(\ii k_1x) + W_-\exp(-\ii k_1x) ) \exp(\ii(l_1y+m_1z-\omega_1t)) + \textnormal{c.c.} \nonumber\\
%     {U}  &= ( U_+\exp(\ii k_1x) + U_-\exp(-\ii k_1x) ) \exp(\ii(l_1y+m_1z-\omega_1t)) + \textnormal{c.c.} \nonumber \\
%    {V}  &= ( V_+\exp(\ii k_1x) + V_-\exp(-\ii k_1x) ) \exp(\ii(l_1y+m_1z-\omega_1t)) + \textnormal{c.c.} \nonumber \\
%    {B}  &= ( B_+\exp(\ii k_1x) + B_-\exp(-\ii k_1x) ) \exp(\ii(l_1y+m_1z-\omega_1t)) + \textnormal{c.c.}  \nonumber       
% \end{align} \vspace{0.5cm}

% %%%%%%%%%%%%%%%%%%%%%%%%%%%%%%%%%%%%%%%%%%%%%%%%%%%%%% Original

% \begin{align}
%     u'  &=  \bar{Fu2} \exp(-\ii (\omega_1/2 - \sigma) t)  \exp(s t)  \exp(-\ii{l}_2y-\ii{m}_2z) + \textnormal{c.c.} \nonumber    \\
%     w'  &= \bar{Fw2} \exp(-\ii (\omega_1/2 - \sigma) t)  \exp(s t)  \exp(-\ii{l}_2y-\ii{m}_2z) + \textnormal{c.c.} \nonumber    \\
%     v'  &= \bar{Fv2} \exp(-\ii (\omega_1/2 - \sigma) t)  \exp(s t)  \exp(-\ii{l}_2y-\ii{m}_2z) + \textnormal{c.c.} \nonumber    \\
%     b'  &= \bar{Fb2} \exp(-\ii (\omega_1/2 - \sigma) t)  \exp(s t)  \exp(-\ii{l}_2y-\ii{m}_2z) + \textnormal{c.c.} \nonumber   \\
%     p'  &= \bar{Fp2} \exp(-\ii (\omega_1/2 - \sigma) t)  \exp(s t)  \exp(-\ii{l}_2y-\ii{m}_2z) + \textnormal{c.c.} \nonumber    
% \end{align}

% %%%%%%%%%%%%%%%%%%%%%%%%%%%%%%%%%%%%%%%%%%%%%%%%%%%%%%

% \begin{align}
%  NLu1 &=  \Bar{U}u'_x + u'\Bar{U}_x + \Bar{V}u'_y + v'\Bar{U}_y + \Bar{W}u'_z + w'\Bar{U}_z  \label{eqn:u1} \\
%  NLv1 &=  \Bar{U}v'_x + u'\Bar{V}_x + \Bar{V}v'_y + v'\Bar{V}_y + \Bar{W}v'_z + w'\Bar{V}_z  \label{eqn:v1} \\
%  NLw1 &=  \Bar{U}w'_x + u'\Bar{W}_x + \Bar{V}w'_y + v'\Bar{W}_y + \Bar{W}w'_z + w'\Bar{W}_z  \label{eqn:w1} \\
%  NLb1 &=  \Bar{U}b'_x + u'\Bar{B}_x + \Bar{V}b'_y + v'\Bar{B}_y + \Bar{W}b'_z + w'\Bar{B}_z  \label{eqn:b1} 
%  \end{align} \vspace{0.5cm}
 

% Assuming:


% \begin{align}
%     \bar{Fu2} &= \sum_{n=-TL}^{n=TL} \hat{u}(n) \exp{\ii nk_1 x}    \\
%     \bar{Fv2} &= \sum_{n=-TL}^{n=TL} \hat{v}(n) \exp{\ii nk_1 x}    \\
%     \bar{Fw2} &= \sum_{n=-TL}^{n=TL} \hat{w}(n) \exp{\ii nk_1 x}   \\  
%     \bar{Fb2} &= \sum_{n=-TL}^{n=TL} \hat{b}(n) \exp{\ii nk_1 x}   \\
%     \bar{Fp2} &= \sum_{n=-TL}^{n=TL} \hat{p}(n) \exp{\ii nk_1 x}     
% \end{align}

% \newpage

% %%%%%%%%%%%%%%%%%%%%%%%%%%%%%%%%%%%%%%%%%%%%%%%%%%%%%%

% $n-$th mode equations:

% \begin{align}
%     &(\ii(\omega_1/2 + \sigma) + s - \nu{m}_3^2 - \nu{l}_3^2 )\hat{u}(n)   - f \hat{v}(n) + \ii nk_1 \hat{p}(n)  - \nu n^2k_1^2\hat{u}(n)  =  NLu3 \label{eqn:u1_S4} \\
%     &(\ii(\omega_1/2 + \sigma) + s - \nu{m}_3^2 - \nu{l}_3^2 )\hat{v}(n)  + f \hat{u}(n) +  \ii {l}_3 \hat{p}(n) - \nu n^2k_1^2 \hat{v}(n)   = NLv3  \label{eqn:v1_S4} \\
%     &(\ii(\omega_1/2 + \sigma) + s - \nu{m}_3^2 - \nu{l}_3^2 )\hat{w}(n) - \hat{b}(n) + \ii {m}_3 \hat{p}(n)  - \nu n^2k_1^2 \hat{w}(n)=  NLw3  \label{eqn:w1_S4} \\
%     &(\ii(\omega_1/2 + \sigma) + s )\hat{b}(n)+ N^2\hat{w}(n)=  NLb3\label{eqn:b1_S4}\\
%    &\ii nk_1 \hat{u}(n)   + \ii {l}_3\hat{v}(n)  +  \ii{m}_3\hat{w}(n) = 0
% \label{eqn:Continuity_S4}
%  \end{align}

% \begin{align}
%  NLu3 &=   {U}u'_x  + u'{U}_x  -\ii{l}_2{V}u' + \ii{l}_1v'{U} - \ii{m}_2{W}u' +  \ii{m}_1 w'{U}  \label{eqn:u1} \\
%  NLv3 &=  {U}v'_x + u'{V}_x -\ii{l}_2{V}v' +\ii{l}_1v'{V} - \ii{m}_2{W}v' + \ii{m}_1w'{V}  \label{eqn:v1} \\
%  NLw3 &=  {U}w'_x + u'{W}_x -\ii{l}_2 {V}w' + \ii{l}_1v'{W} - \ii{m}_2 {W}w'  + \ii{m}_1w'{W}   \label{eqn:w1} \\
%  NLb3 &=  {U}b'_x + u'{B}_x -\ii{l}_2{V}b' + \ii{l}_1v'{B} - \ii{m}_2 {W}b' + \ii{m}_1w'{B}   \label{eqn:b1} 
%  \end{align} 

% %%%%%%%%%%%%%%%%%%%%%%%%%%%%%%%%%%%%%%%%

% %% Original arrangement - easy to cross check

%  \begin{align}
%  NLu3 &=  \ii{k}_1 ( U_+ (n-1) \hat{u}(n-1) + (n+1) U_- \hat{u}(n+1)) \\
%  &+ \ii{k}_1( U_+ \hat{u}(n-1) - U_- \hat{u}(n+1)) \\
%  &- \ii{l}_2( V_+ \hat{u}(n-1) + V_- \hat{u}(n+1)) \\ 
%  &+ \ii{l}_1 ( U_+ \hat{v}(n-1) + U_- \hat{v}(n+1)   ) \\
%  &- \ii{m}_2 ( W_+\hat{u}(n-1) +  W_-\hat{u}(n+1)) \\
%  &+  \ii{m}_1 ( U_+ \hat{w}(n-1) + U_- \hat{w}(n+1)   )  \label{eqn:u1} 
%  \end{align} 
 
%  \begin{align}
%  NLv3 &=  \ii{k}_1 ( U_+ (n-1) \hat{v}(n-1) + (n+1) U_- \hat{v}(n+1)) \\
%  &+ \ii{k}_1( V_+ \hat{u}(n-1) - V_- \hat{u}(n+1)) \\
%  &- \ii{l}_2( V_+ \hat{v}(n-1) + V_- \hat{v}(n+1)) \\ 
%  &+ \ii{l}_1 ( V_+ \hat{v}(n-1) + V_- \hat{v}(n+1)   ) \\
%  &- \ii{m}_2 (W_+\hat{v}(n-1) +  W_-\hat{v}(n+1)) \\
%  &+  \ii{m}_1 ( V_+ \hat{w}(n-1) + V_- \hat{w}(n+1)   )  \label{eqn:u1} 
%  \end{align} 
 
%  \begin{align}
%  NLw3 &=  \ii{k}_1 ( U_+ (n-1) \hat{w}(n-1) + (n+1) U_- \hat{w}(n+1)) \\
%  &+ \ii{k}_1( W_+  \hat{u}(n-1) - W_- \hat{u}(n+1)) \\
%  &- \ii{l}_2( V_+ \hat{w}(n-1) + V_- \hat{w}(n+1)) \\ 
%  &+ \ii{l}_1 ( W_+ \hat{v}(n-1) + W_- \hat{v}(n+1)   ) \\
%  &- \ii{m}_2 ( W_+ \hat{w}(n-1) +  W_-\hat{w}(n+1)) \\
%  &+  \ii{m}_1 ( W_+ \hat{w}(n-1) + W_- \hat{w}(n+1)   )  \label{eqn:u1} 
%  \end{align} 


%  \begin{align}
%  NLb3 &=  \ii{k}_1 ( U_+ (n-1) \hat{b}(n-1) + (n+1) U_- \hat{b}(n+1)) \\
%  &+ \ii{k}_1( B_+  \hat{u}(n-1) - B_- \hat{u}(n+1)) \\
%  &- \ii{l}_2( V_+ \hat{b}(n-1) + V_- \hat{b}(n+1)) \\ 
%  &+ \ii{l}_1 ( B_+ \hat{v}(n-1) + B_- \hat{v}(n+1)   ) \\
%  &- \ii{m}_2 ( W_+ \hat{b}(n-1) +  W_-\hat{b}(n+1)) \\
%  &+  \ii{m}_1 ( B_+ \hat{w}(n-1) + B_- \hat{w}(n+1)   )  \label{eqn:u1} 
%  \end{align} 


%  %%%%%%%%%%%%%%%%%%%%%%%%%%%%%%%%%%%%%%%%%%%%%%%%%%%%%%%%%%%%%%%%%%

%  %%%%%%%%%%%%%% Easy to code version

%   \begin{align}
%  NLu3 &=  \ii{k}_1 ( U_+ (n-1) \hat{u}(n-1) + (n+1) U_- \hat{u}(n+1)) \\
%  &+ \ii{k}_1( U_+ \hat{u}(n-1) - U_- \hat{u}(n+1)) \\
%  &- \ii{l}_2( V_+ \hat{u}(n-1) + V_- \hat{u}(n+1)) \\
%  &- \ii{m}_2 ( W_+\hat{u}(n-1) +  W_-\hat{u}(n+1)) \\
%  &+ \ii{l}_1 ( U_+ \hat{v}(n-1) + U_- \hat{v}(n+1)   ) \\
%  &+  \ii{m}_1 ( U_+ \hat{w}(n-1) + U_- \hat{w}(n+1)   )  \label{eqn:u1} 
%  \end{align} 
 
%  \begin{align}
%  NLv3 &=  \ii{k}_1 ( U_+ (n-1) \hat{v}(n-1) + (n+1) U_- \hat{v}(n+1)) \\
%   &- \ii{l}_2( V_+ \hat{v}(n-1) + V_- \hat{v}(n+1)) \\ 
%  &+ \ii{l}_1 ( V_+ \hat{v}(n-1) + V_- \hat{v}(n+1)   ) \\
%  &- \ii{m}_2 (W_+\hat{v}(n-1) +  W_-\hat{v}(n+1)) \\
%  &+ \ii{k}_1( V_+ \hat{u}(n-1) - V_- \hat{u}(n+1)) \\
%  &+  \ii{m}_1 ( V_+ \hat{w}(n-1) + V_- \hat{w}(n+1)   )  \label{eqn:u1} 
%  \end{align} 
 
%  \begin{align}
%  NLw3 &=  \ii{k}_1 ( U_+ (n-1) \hat{w}(n-1) + (n+1) U_- \hat{w}(n+1)) \\
%  &- \ii{l}_2( V_+ \hat{w}(n-1) + V_- \hat{w}(n+1)) \\ 
%  &- \ii{m}_2 ( W_+ \hat{w}(n-1) +  W_-\hat{w}(n+1)) \\
%  &+  \ii{m}_1 ( W_+ \hat{w}(n-1) + W_- \hat{w}(n+1)   ) \\
%  &+ \ii{l}_1 ( W_+ \hat{v}(n-1) + W_- \hat{v}(n+1)   ) \\
%  &+ \ii{k}_1( W_+  \hat{u}(n-1) - W_- \hat{u}(n+1)) \label{eqn:u1} 
%  \end{align} 


%  \begin{align}
%  NLb3 &=  \ii{k}_1 ( U_+ (n-1) \hat{b}(n-1) + (n+1) U_- \hat{b}(n+1)) \\
%  & - \ii{m}_2 ( W_+ \hat{b}(n-1) +  W_-\hat{b}(n+1)) \\
%  &- \ii{l}_2( V_+ \hat{b}(n-1) + V_- \hat{b}(n+1)) \\ 
%  &+ \ii{l}_1 ( B_+ \hat{v}(n-1) + B_- \hat{v}(n+1)   ) \\
%  &+ \ii{k}_1( B_+  \hat{u}(n-1) - B_- \hat{u}(n+1)) \\
%  &+  \ii{m}_1 ( B_+ \hat{w}(n-1) + B_- \hat{w}(n+1)   )  \label{eqn:u1} 
%  \end{align} 

% %%%%%%%%%%%%%%%%%%%%%%%%%%%%%%%%%%%%%%%%%%%%%%%


%  \begin{align}
%     &(-\ii(\omega_1/2 - \sigma) + s - \nu{m}_2^2  )\widehat{u}(n)   - f \widehat{v}(n)     =  \bar{NLu2} \label{eqn:u1_S4} \\
%     &(-\ii(\omega_1/2 - \sigma) + s - \nu{m}_2^2  )\widehat{v}(n)    + f \widehat{u}(n)    = \bar{NLv2}  \label{eqn:v1_S4} \\
%     & - \widehat{b}(n)  - \ii {m}_2 \widehat{p}(n)   =  \bar{NLw2}  \label{eqn:w1_S4}\\ 
%     &(-\ii(\omega_1/2 - \sigma) + s )\widehat{b}(n)   =  \bar{NLb2} \label{eqn:b1_S4}\\
%    & \widehat{w}(n) = 0
% \label{eqn:Continuity_S4}
%  \end{align} \vspace{0.5cm}


%  \noindent CONJUGATE Base flow terms used:

% \begin{align}
%     {W}  &= ( \bar{W}_+ \exp(-\ii k_1x) + \bar{W}_-\exp(\ii k_1x) ) \exp(-\ii(l_1y+m_1z-\omega_1t)) + \textnormal{c.c.} \nonumber\\
%     {U}  &= ( \bar{U}_+\exp(-\ii k_1x) + \bar{U}_-\exp(\ii k_1x) ) \exp(-\ii(l_1y+m_1z-\omega_1t)) + \textnormal{c.c.} \nonumber \\
%    {V}  &= ( \bar{V}_+\exp(-\ii k_1x) + \bar{V}_-\exp(\ii k_1x) ) \exp(-\ii(l_1y+m_1z-\omega_1t)) + \textnormal{c.c.} \nonumber \\
%    {B}  &= ( \bar{B}_+\exp(-\ii k_1x) + \bar{B}_-\exp(\ii k_1x) ) \exp(-\ii(l_1y+m_1z-\omega_1t)) + \textnormal{c.c.}  \nonumber       
% \end{align} \vspace{0.5cm}


% Schematic used:


% \begin{equation}
%     W1 = P Conj(W2), \hspace{1cm}  conj(W2) = conj(P) W1    
% \end{equation}


% \begin{align}
%  NLu2 &=   {U}u'_x  + u'{U}_x  -\ii{l}_3{V}u' + \ii{l}_1v'{U} - \ii{m}_3{W}u' +  \ii{m}_1 w'{U}  \label{eqn:u1} \\
%  NLv2 &=  {U}v'_x + u'{V}_x -\ii{l}_3{V}v' +\ii{l}_1v'{V} - \ii{m}_3{W}v' + \ii{m}_1w'{V}  \label{eqn:v1} \\
%  NLw2 &=  {U}w'_x + u'{W}_x -\ii{l}_3 {V}w' + \ii{l}_1v'{W} - \ii{m}_3 \Bar{W}w'  + \ii{m}_1w'{W}   \label{eqn:w1} \\
%  NLb2 &=  {U}b'_x + u'{B}_x -\ii{l}_3{V}b' + \ii{l}_1v'{B} - \ii{m}_3 \Bar{W}b' + \ii{m}_1w'{B}   \label{eqn:b1} 
%  \end{align} 


% %%%%%%%%%%%%%%%%%%%%%%%%%%%%%%%%%%%%%%%%%%%%%%%%%%%%%%%%%%%%%%%%%%%%%%%

% %%%%%%%%%% Normal version

%  \begin{align}
%  \bar{NLu2} &=  \ii{k}_1 ( U_+ (n-1) \hat{u}(n-1) + (n+1) U_- \hat{u}(n+1)) \\
%  &+ \ii{k}_1( U_+ \hat{u}(n-1) - U_- \hat{u}(n+1)) \\
%  &- \ii{l}_3( V_+ \hat{u}(n-1) + V_- \hat{u}(n+1)) \\ 
%  &+ \ii{l}_1 ( U_+ \hat{v}(n-1) + U_- \hat{v}(n+1)   ) \\
%  &- \ii{m}_3 ( W_+\hat{u}(n-1) +  W_-\hat{u}(n+1)) \\
%  &+  \ii{m}_1 ( U_+ \hat{w}(n-1) + U_- \hat{w}(n+1)   )  \label{eqn:u1} 
%  \end{align} 
 
%  \begin{align}
%  \bar{NLv2} &=  \ii{k}_1 ( U_+ (n-1) \hat{v}(n-1) + (n+1) U_- \hat{v}(n+1)) \\
%  &+ \ii{k}_1( V_+ \hat{u}(n-1) - V_- \hat{u}(n+1)) \\
%  &- \ii{l}_3( V_+ \hat{v}(n-1) + V_- \hat{v}(n+1)) \\ 
%  &+ \ii{l}_1 ( V_+ \hat{v}(n-1) + V_- \hat{v}(n+1)   ) \\
%  &- \ii{m}_3 (W_+\hat{v}(n-1) +  W_-\hat{v}(n+1)) \\
%  &+  \ii{m}_1 ( V_+ \hat{w}(n-1) + V_- \hat{w}(n+1)   )  \label{eqn:u1} 
%  \end{align} 
 
%  \begin{align}
%  \bar{NLw2} &=  \ii{k}_1 ( U_+ (n-1) \hat{w}(n-1) + (n+1) U_- \hat{w}(n+1)) \\
%  &+ \ii{k}_1( W_+  \hat{u}(n-1) - W_- \hat{u}(n+1)) \\
%  &- \ii{l}_3( V_+ \hat{w}(n-1) + V_- \hat{w}(n+1)) \\ 
%  &+ \ii{l}_1 ( W_+ \hat{v}(n-1) + W_- \hat{v}(n+1)   ) \\
%  &- \ii{m}_3 ( W_+ \hat{w}(n-1) +  W_-\hat{w}(n+1)) \\
%  &+  \ii{m}_1 ( W_+ \hat{w}(n-1) + W_- \hat{w}(n+1)   )  \label{eqn:u1} 
%  \end{align} 


%  \begin{align}
%  \bar{NLb2} &=  \ii{k}_1 ( U_+ (n-1) \hat{b}(n-1) + (n+1) U_- \hat{b}(n+1)) \\
%  &+ \ii{k}_1( B_+  \hat{u}(n-1) - B_- \hat{u}(n+1)) \\
%  &- \ii{l}_3( V_+ \hat{b}(n-1) + V_- \hat{b}(n+1)) \\ 
%  &+ \ii{l}_1 ( B_+ \hat{v}(n-1) + B_- \hat{v}(n+1)   ) \\
%  &- \ii{m}_3 ( W_+ \hat{b}(n-1) +  W_-\hat{b}(n+1)) \\
%  &+  \ii{m}_1 ( B_+ \hat{w}(n-1) + B_- \hat{w}(n+1)   )  \label{eqn:u1} 
%  \end{align} 


% %%%%%%%%%%%%%%%%%%%%%%%%%%%%%%%%%%%%%%%%%%%%%%%%%%%%%%%%%%%%%%%%%%55

% %%%%%%%%%%%%% Conjugate version

% \newpage

% \begin{align}
%     &(-\ii(\omega_1/2 - \sigma) + s - \nu{m}_2^2 - \nu{l}_2^2 )\widehat{u}(n)   - f \widehat{v}(n) + \ii n k_1 \widehat{p}(n) - \nu n^2k_1^2\widehat{u}(n)   =  \bar{NLu2} \label{eqn:u1_S4} \\
%     &(-\ii(\omega_1/2 - \sigma) + s - \nu{m}_2^2 - \nu{l}_2^2 )\widehat{v}(n)    + f \widehat{u}(n) - \ii {l}_2 \widehat{p}(n) - \nu n^2k_1^2\widehat{v}(n)  = \bar{NLv2}  \label{eqn:v1_S4} \\
%     &(-\ii(\omega_1/2 - \sigma) + s - \nu{m}_2^2 - \nu{l}_2^2 )\widehat{w}(n)  - \widehat{b}(n)  - \ii {m}_2 \widehat{p}(n)  - \nu n^2k_1^2\widehat{w}(n) =  \bar{NLw2}  \label{eqn:w1_S4}\\ 
%     &(-\ii(\omega_1/2 - \sigma) + s )\widehat{b}(n)  + N^2\widehat{w}(n)  =  \bar{NLb2} \label{eqn:b1_S4}\\
%    & \ii nk_1 \widehat{u}(n)   - \ii {l}_2 \widehat{v}(n)  -  \ii{m}_2\widehat{w}(n) = 0
% \label{eqn:Continuity_S4}
%  \end{align} \vspace{0.5cm}

%  \begin{align}
%  \bar{NLu2} &=  \ii{k}_1 ( \bar{U}_- (n-1) \hat{u}(n-1) + (n+1) \bar{U}_+ \hat{u}(n+1)) \\
%  &+ \ii{k}_1( -\bar{U}_+ \hat{u}(n+1) + \bar{U}_- \hat{u}(n-1)) \\
%  &+ \ii{l}_3( \bar{V}_+ \hat{u}(n+1) + \bar{V}_- \hat{u}(n-1)) \\ 
%  &- \ii{l}_1 ( \bar{U}_+ \hat{v}(n+1) + \bar{U}_- \hat{v}(n-1)   ) \\
%  &+ \ii{m}_3 ( \bar{W}_+\hat{u}(n+1) +  \bar{W}_-\hat{u}(n-1)) \\
%  &-  \ii{m}_1 ( \bar{U}_+ \hat{w}(n+1) + \bar{U}_- \hat{w}(n-1)   )  \label{eqn:u1} 
%  \end{align} 

%  \begin{align}
%  \bar{NLv2} &=  \ii{k}_1 ( \bar{U}_- (n-1) \hat{v}(n-1) + (n+1) \bar{U}_+ \hat{v}(n+1)) \\
%   &+ \ii{l}_3(  \bar{V}_- \hat{v}(n-1) +  \bar{V}_+ \hat{v}(n+1)) \\ 
%  &- \ii{l}_1 (  \bar{V}_- \hat{v}(n-1) +  \bar{V}_+ \hat{v}(n+1)   ) \\
%  &+ \ii{m}_3 ( \bar{W}_-\hat{v}(n-1) +   \bar{W}_+\hat{v}(n+1)) \\
%  &- \ii{k}_1(  \bar{V}_+ \hat{u}(n+1) -  \bar{V}_- \hat{u}(n-1)) \\
%  &-  \ii{m}_1 (  \bar{V}_+ \hat{w}(n+1) +  \bar{V}_- \hat{w}(n-1)   )  \label{eqn:u1} 
%  \end{align} 
 
%  \begin{align}
%  \bar{NLw2} &=  \ii{k}_1 ( \bar{U}_- (n-1) \hat{w}(n-1) + (n+1) \bar{U}_+ \hat{w}(n+1)) \\
%   &-  \ii{m}_1 ( \bar{W}_- \hat{w}(n-1) + \bar{W}_+ \hat{w}(n+1)   ) \\
%  &+ \ii{l}_3( \bar{V}_- \hat{w}(n-1) + \bar{V}_+ \hat{w}(n+1)) \\
%   &+ \ii{m}_3 ( \bar{W}_- \hat{w}(n-1) +  \bar{W}_+\hat{w}(n+1)) \\
% &+ \ii{k}_1( \bar{W}_- \hat{u}(n-1) - \bar{W}_+ \hat{u}(n+1)) \\
%  &- \ii{l}_1 ( \bar{W}_- \hat{v}(n-1) + \bar{W}_+ \hat{v}(n+1)   ) \label{eqn:u1} 
%  \end{align} 


%  \begin{align}
%  \bar{NLb2} &= \ii{k}_1 ( \bar{U}_- (n-1) \hat{b}(n-1) + (n+1) \bar{U}_+ \hat{b}(n+1)) \\
%   &+ \ii{l}_3( \bar{V}_- \hat{b}(n-1) + \bar{V}_+ \hat{b}(n+1)) \\ 
%  &+ \ii{m}_3 ( \bar{W}_- \hat{b}(n-1) +  \bar{W}_+\hat{b}(n+1)) \\
%   &- \ii{l}_1 ( \bar{B}_- \hat{v}(n-1) + \bar{B}_+ \hat{v}(n+1)   ) \\
%  &+ \ii{k}_1( \bar{B}_-  \hat{u}(n-1) - \bar{B}_+ \hat{u}(n+1)) \\
%  &-  \ii{m}_1 ( \bar{B}_- \hat{w}(n-1) + \bar{B}_+ \hat{w}(n+1)   )  \label{eqn:u1} 
%  \end{align} 


%%%%%%%%%%%%%%%%%%%%%%%%%%%%%%%%%%%%%%%%%%%%%%%%%%%%%%%%%%%%%%%%%%%%%%%%%%%%%%%%%%%%%%%%%%%%%%%%%
 
% \begin{subequations}
% \begin{align}
% \frac{\partial }{\partial t}\left(\nabla^{2}\psi\right)  + \frac{\partial b}{\partial x} - f\frac{\partial v}{\partial z} & = \nabla^{2}(\psi_z)\psi_x-\nabla^{2}(\psi_x)\psi_z, \label{eqn:NS_stream}\\
% \frac{\partial^{2} }{\partial t^{2}}\left(\nabla^{2}\psi\right) + N^{2}\frac{\partial^{2} \psi}{\partial x^{2}} + f^{2}\frac{\partial^{2} \psi}{\partial z^{2}} & = -\frac{\partial}{\partial t} \left(\{\nabla^{2}\psi,\psi\}\right) + \frac{\partial }{\partial x}\left(\{b,\psi\}\right) - f\frac{\partial }{\partial z}\left(\{v,\psi\}\right) , \label{eqn:NS_stream}\\
% \frac{\partial v}{\partial t} + f\frac{\partial \psi}{\partial z}  & = v_z\psi_x-v_x\psi_z, \label{eqn:corilios}\\
% \frac{\partial b}{\partial t} - N^{2}\frac{\partial \psi}{\partial x} & = b_z\psi_x-b_x\psi_z. \label{eqn:material_cons}
% \end{align}
% \end{subequations}

% $\{G_1,G_2\} \equiv  (\partial G_1/\partial x)(\partial G_2/\partial z) - (\partial G_1/\partial z)(\partial G_2/\partial x)$. $(\psi_z,-\psi_x) = (u,w)$.

% \begin{align}
%     {\Psi}  &= \psi_0(\exp(\ii k_1x) + \exp(-\ii k_1x) ) \exp(\ii(m_1z-\omega_1t)) + \textnormal{c.c.} \nonumber\\
%    {V}  &= \frac{fm_1}{\omega}\psi_0( \exp(\ii k_1x) + \exp(-\ii k_1x) ) \exp(\ii(m_1z-\omega_1t)) + \textnormal{c.c.} \nonumber \\
%    {B}  &= -\frac{N^2k_1}{\omega}\psi_0( \exp(\ii k_1x) - \exp(-\ii k_1x) ) \exp(\ii(m_1z-\omega_1t)) + \textnormal{c.c.}  \nonumber
%    \end{align}
% \begin{equation}
%      V = \frac{fm}{\omega}\Psi, \hspace{0.2cm} B = -\frac{N^2k}{\omega}\Psi
% \end{equation}
 
% \begin{align}
%     \psi_2 &=  \hat{\psi}_2(x,t)\exp(\ii(m_2z)) + c.c. \nonumber \\
%     \psi_3 &=  \hat{\psi}_3(x,t)\exp(\ii(m_3z)) + c.c. \nonumber \\
%      b_2 &=  \hat{b}_2(x,t)\exp(\ii(m_2z)) + c.c. \nonumber \\
%      b_3 &=  \hat{b}_3(x,t)\exp(\ii(m_3z)) + c.c. \nonumber \\
%      v_2 &=  \hat{v}_2(x,t)\exp(\ii(m_2z)) + c.c. \nonumber \\
%      v_3 &=  \hat{v}_3(x,t)\exp(\ii(m_3z)) + c.c. \nonumber 
% \end{align}

% \begin{align}
%     \hat{\psi}_2(x,t) = \sum_{n=-TN}^{n=TN} \Tilde{\psi}_{n2}(t)\exp{\ii k_n x} \nonumber \\
%     \hat{\psi}_3(x,t) = \sum_{n=-TN}^{n=TN} \Tilde{\psi}_{n3}(t)\exp{\ii k_n x} \nonumber \\
%     \hat{v}_2(x,t) = \sum_{n=-TN}^{n=TN} \Tilde{v}_{n2}(t)\exp{\ii k_n x} \nonumber \\
%     \hat{v}_3(x,t) = \sum_{n=-TN}^{n=TN} \Tilde{v}_{n3}(t)\exp{\ii k_n x} \nonumber \\
%     \hat{b}_2(x,t) = \sum_{n=-TN}^{n=TN} \Tilde{b}_{n2}(t)\exp{\ii k_n x} \nonumber \\
%     \hat{b}_3(x,t) = \sum_{n=-TN}^{n=TN} \Tilde{b}_{n3}(t)\exp{\ii k_n x} \nonumber 
% \end{align} 


% %%%%% n-th mode LHS are given by

% \begin{subequations}
% \begin{align}
% \textnormal{LHS}_j &= -(k_n^2+m_j^2)\frac{d \Tilde{\psi}_{nj}}{dt} + \ii k_n{\Tilde{b}_{nj}}  -\ii fm_j{\Tilde{v}_{nj}} , \nonumber \\
% \textnormal{LHS}_j &= \frac{d \Tilde{v}_{nj}}{dt} + \ii m_j{\Tilde{\psi}_{nj}}  , \nonumber \\
% \textnormal{LHS}_j &= \frac{d \Tilde{b}_{nj}}{dt} - \ii N^2 k_n{\Tilde{\psi}_{nj}}  , \nonumber  
% \end{align}
% \end{subequations}

 
% \begin{subequations}
% \begin{align}
% \textnormal{RHS}_3 = &-(k_1^2+m_1^2)(\ii m_1)\psi_{0+}(-\ii k_{1-n})  \Tilde{\psi}_{1-n} - (k_{1-n}^2+m_j^2)(-\ii m_j)\Tilde{\psi}_{1-n}\ii k_1 \psi_{0+}  , \nonumber \\
% &+ (k_1^2+m_1^2)(\ii k_1)\psi_{0+}(-\ii m_{j} )  \Tilde{\psi}_{1-n} + (k_{1-n}^2+m_2^2)(-\ii k_{1-n})\Tilde{\psi}_{1-n}\ii m_1 \psi_{0+}  , \nonumber \\
% \textnormal{RHS}_3 &= (\ii m_1) b_{0+} (-\ii k_{1-n}) \Tilde{\psi}_{1-n} + (-\ii m_j) \Tilde{b}_{1-n} (\ii k_{1})\psi_{0+} , \nonumber \\
% & - (\ii k_1) b_{0+} (-\ii m_{j}) \Tilde{\psi}_{1-n} - (-\ii k_{1-n}) \Tilde{b}_{1-n} (\ii m_1) {\psi}_{0+}  , \nonumber \\
% \textnormal{RHS}_3 &= (\ii m_1) v_{0+} (-\ii k_{1-n}) \Tilde{\psi}_{1-n} + (-\ii m_j) \Tilde{v}_{1-n} \ii k_{1}\psi_{0+}  , \nonumber \\
% & - (\ii k_1) v_{0+} (-\ii m_{j}) \Tilde{\psi}_{1-n} - (-\ii k_{1-n}) \Tilde{v}_{1-n} (\ii m_1) {\psi}_{0+}  , \nonumber
% \end{align}
% \end{subequations}

% \begin{subequations}
% \begin{align}
% \textnormal{RHS}_3 = &-(k_1^2+m_1^2)(\ii m_1)(-\ii k_{-1-n} ) \psi_{0-} \Tilde{\psi}_{-1-n} - (k_{-1-n}^2+m_j^2)(-\ii m_j) (-\ii k_1) \psi_{0-} \Tilde{\psi}_{-1-n} , \nonumber \\
% & +(k_1^2+m_1^2)(-\ii k_1)(-\ii m_{j} ) \psi_{0-} \Tilde{\psi}_{-1-n} + (k_{-1-n}^2+m_j^2)(-\ii k_{-1-n})(\ii m_1) \psi_{0-} \Tilde{\psi}_{-1-n} , \nonumber \\
% \textnormal{RHS}_3 &= (\ii m_1) b_{0-} (-\ii k_{-1-n}) \Tilde{\psi}_{-1-n} + (-\ii m_j) \Tilde{b}_{-1-n} (-\ii k_{1})\psi_{0-} , \nonumber \\
% & - (-\ii k_1) b_{0-} (-\ii m_{j}) \Tilde{\psi}_{-1-n} - (-\ii k_{-1-n}) \Tilde{b}_{-1-n} (\ii m_1) {\psi}_{0-}  , \nonumber \\
% \textnormal{RHS}_3 &= (\ii m_1) v_{0-} (-\ii k_{-1-n}) \Tilde{\psi}_{-1-n} + (-\ii m_j) \Tilde{v}_{-1-n} (-\ii k_{1})\psi_{0-} , \nonumber \\
% & - (-\ii k_1) v_{0-} (-\ii m_{j}) \Tilde{\psi}_{-1-n} - (-\ii k_{-1-n}) \Tilde{v}_{-1-n} (\ii m_1) {\psi}_{0-}  , \nonumber 
% \end{align}
% \end{subequations}


% \newpage

% \begin{subequations}
% \begin{align}
% \frac{\partial^{2} }{\partial t^{2}}\left(\nabla^{2}\psi\right) + N^{2}\frac{\partial^{2} \psi}{\partial x^{2}} + f^{2}\frac{\partial^{2} \psi}{\partial z^{2}} & = -\frac{\partial}{\partial t} \left(\{\nabla^{2}\psi,\psi\}\right) - \frac{\partial }{\partial x}\left(\{\psi,b\}\right) + f\frac{\partial }{\partial z}\left(\{\psi,v\}\right) , \label{eqn:NS_stream}\\
% \frac{\partial v}{\partial t} + f\frac{\partial \psi}{\partial z}  & = v_z\psi_x-v_x\psi_z, \label{eqn:corilios}\\
% \frac{\partial b}{\partial t} - N^{2}\frac{\partial \psi}{\partial x} & = b_z\psi_x-b_x\psi_z. \label{eqn:material_cons}
% \end{align}
% \end{subequations}

% \begin{subequations}
% \begin{align}
% \textnormal{LHS}_j &= -(k_n^2+m_j^2)(\frac{d \Tilde{\Psi}_{nj}}{dt}  ) - N^2 k^2_n {\Tilde{\psi}_{nj}}  - f^2 m^2_j{\Tilde{\psi}_{nj}} , \nonumber \\
% \textnormal{LHS}_j &\xrightarrow[]{}  \frac{d \Tilde{\psi}_{nj}}{dt} = \Psi_{nj} , \nonumber \\
% \textnormal{LHS}_j &= \frac{d \Tilde{v}_{nj}}{dt} + \ii m_j{\Tilde{\psi}_{nj}}  , \nonumber \\
% \textnormal{LHS}_j &= \frac{d \Tilde{b}_{nj}}{dt} - \ii N^2 k_n{\Tilde{\psi}_{nj}}  , \nonumber  
% \end{align}
% \end{subequations}

% \begin{subequations}
% \begin{align}
% \textnormal{RHS}_3 = &-(k_1^2+m_1^2)(\ii m_1)(-\ii k_{1-n}) \psi_{0+} \Tilde{\psi}_{1-n} - (k_{1-n}^2+m_j^2)(-\ii m_j)\ii k_1 \psi_{0+} \Tilde{\psi}_{1-n} , \nonumber \\
% &+ (k_1^2+m_1^2)(\ii k_1)(-\ii m_{j} ) \psi_{0+} \Tilde{\psi}_{1-n} + (k_{1-n}^2+m_2^2)(-\ii k_{1-n})\ii m_1 \psi_{0+} \Tilde{\psi}_{1-n} , \nonumber \\
% \textnormal{RHS}_3 &= (\ii m_1) b_{0+} (-\ii k_{1-n}) \Tilde{\psi}_{1-n} + (-\ii m_j) \Tilde{b}_{1-n} (\ii k_{1})\psi_{0+} , \nonumber \\
% & - (\ii k_1) b_{0+} (-\ii m_{j}) \Tilde{\psi}_{1-n} - (-\ii k_{1-n}) \Tilde{b}_{1-n} (\ii m_1) {\psi}_{0+}  , \nonumber \\
% \textnormal{RHS}_3 &= (\ii m_1) v_{0+} (-\ii k_{1-n}) \Tilde{\psi}_{1-n} + (-\ii m_j) \Tilde{v}_{1-n} \ii k_{1}\psi_{0+}  , \nonumber \\
% & - (\ii k_1) v_{0+} (-\ii m_{j}) \Tilde{\psi}_{1-n} - (-\ii k_{1-n}) \Tilde{v}_{1-n} (\ii m_1) {\psi}_{0+}  , \nonumber
% \end{align}
% \end{subequations}


% \begin{subequations}?>
% \begin{align}
% \textnormal{RHS}_3 = &-(k_1^2+m_1^2)(\ii m_1)(-\ii k_{-1-n} ) \psi_{0-} \Tilde{\psi}_{-1-n} - (k_{-1-n}^2+m_j^2)(-\ii m_j) (-\ii k_1) \psi_{0-} \Tilde{\psi}_{-1-n} , \nonumber \\
% & +(k_1^2+m_1^2)(-\ii k_1)(-\ii m_{j} ) \psi_{0-} \Tilde{\psi}_{-1-n} + (k_{-1-n}^2+m_j^2)(-\ii k_{-1-n})(\ii m_1) \psi_{0-} \Tilde{\psi}_{-1-n} , \nonumber \\
% \textnormal{RHS}_3 &= (\ii m_1) b_{0-} (-\ii k_{-1-n}) \Tilde{\psi}_{-1-n} + (-\ii m_j) \Tilde{b}_{-1-n} (-\ii k_{1})\psi_{0-} , \nonumber \\
% & - (-\ii k_1) b_{0-} (-\ii m_{j}) \Tilde{\psi}_{-1-n} - (-\ii k_{-1-n}) \Tilde{b}_{-1-n} (\ii m_1) {\psi}_{0-}  , \nonumber \\
% \textnormal{RHS}_3 &= (\ii m_1) v_{0-} (-\ii k_{-1-n}) \Tilde{\psi}_{-1-n} + (-\ii m_j) \Tilde{v}_{-1-n} (-\ii k_{1})\psi_{0-} , \nonumber \\
% & - (-\ii k_1) v_{0-} (-\ii m_{j}) \Tilde{\psi}_{-1-n} - (-\ii k_{-1-n}) \Tilde{v}_{-1-n} (\ii m_1) {\psi}_{0-}  , \nonumber 
% \end{align}
% \end{subequations}


\bibliographystyle{jfm}

\bibliography{references}


%%%%%% Cover letter

%%%%%%%%%%%%%%%%%%%%%%%%%%%%%%%%%
% \todo{remember to remove the cover letter from the paper. Also, I read the blue part before as well (on Friday) and made small edits to it. All good=>oh ok sir thanks a lot, I made the graphical abstract as well. I will go through the intro once again. But mostly I am also done with it. Yes I will remove the cover letter. Thnaks again => Cheers=> }
% Dear Prof. Oliver Bühler,

% We wish to submit a paper entitled ``5 wave interactions in internal gravity waves'' for publication in the Journal of Fluid Mechanics. We confirm that this work is original and has not been published elsewhere, nor is it currently under consideration for publication elsewhere. 

% In this paper, we study weakly nonlinear wave-wave interactions of internal gravity waves, which is one of the mechanisms by which the energy in internal waves cascades to small-length scales, where it causes ocean turbulence and mixing. Specifically, we have used multiple scale analysis to investigate 5-wave systems that are composed of two different internal gravity wave triads. Each of these triads consists of a parent wave (wave with a large amount of energy) and two daughter waves (waves with infinitesimal energy), and hence one daughter wave is shared between the two triads. Such 5-wave systems can emerge when two parent waves overlap/meet, which often occurs in the oceans. For example, when a tide interacts with topography, it generates multiple parent waves that overlap in the vicinity of the topography. We find and explain the scenarios where the growth rates of 5-wave systems are more than the growth rates of triads (3-wave systems). We also conduct numerical simulations to validate our theoretical predictions.

%  We have no conflicts of interest to disclose.

%  Thank you for considering this paper.

%  Sincerely,

%  Saranraj Gururaj and Anirban Guha
%%%%%%%%%%%%%%%%%%%%%%%%%%%%%%%%%

% Such 5-wave systems can emerge when two parent waves overlap, for example, (1) when a parent wave gets reflected from the seafloor, the upward propagating parent wave and the downward propagating parent wave overlap near the seafloor, or (2) tide-topography interactions generate multiple parent waves, and they overlap in the vicinity of the topography. 

 \end{document}







