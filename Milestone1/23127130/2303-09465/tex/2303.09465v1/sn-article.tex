





\documentclass[sn-mathphys]{sn-jnl}%

\usepackage{verbatim}

\jyear{2023}%

\theoremstyle{thmstyleone}%
\newtheorem{theorem}{Theorem}%
\newtheorem{proposition}[theorem]{Proposition}%

\theoremstyle{thmstyletwo}%
\newtheorem{example}{Example}%
\newtheorem{remark}{Remark}%

\theoremstyle{thmstylethree}%
\newtheorem{definition}{Definition}%

\raggedbottom

\begin{document}

\title{The chemistry at defects: Atomic-scale phase transformations of grain boundaries}


\author[1]{\fnm{Xuyang} \sur{Zhou}}\email{x.zhou@mpie.de}
\equalcont{These authors contributed equally to this work.}

\author[1]{\fnm{Prince} \sur{Mathews}}\email{mathews@mpie.de}
\equalcont{These authors contributed equally to this work.}

\author[2]{\fnm{Benjamin} \sur{Berkels}}\email{berkels@aices.rwth-aachen.de}

\author[1]{\fnm{Saba} \sur{Ahmad}}\email{saba@mpie.de}

\author[2]{\fnm{Amel Shamseldeen Ali} \sur{Alhassan}}\email{Alhassan@aices.rwth-aachen.de}

\author[1]{\fnm{Jörg} \sur{Neugebauer}}\email{neugebauer@mpie.de}

\author[1]{\fnm{Gerhard} \sur{Dehm}}\email{dehm@mpie.de}

\author[1,3]{\fnm{Tilmann} \sur{Hickel}}\email{hickel@mpie.de}

\author[1]{\fnm{Christina} \sur{Scheu}}\email{scheu@mpie.de}

\author*[1]{\fnm{Siyuan} \sur{Zhang}}\email{siyuan.zhang@mpie.de}

\affil[1]{\orgname{Max-Planck-Institut für Eisenforschung}, \orgaddress{\street{Max-Planck-Straße 1}, \city{Düsseldorf}, \postcode{40237}, \country{Germany}}}

\affil[2]{\orgdiv{Aachen Institute for Advanced Study in Computational Engineering Science (AICES)}, \orgname{RWTH Aachen University}, \orgaddress{\street{Schinkelstraße 2}, \city{Aachen}, \postcode{52062}, \country{Germany}}}

\affil[3]{\orgname{Federal Institute for Materials Research and Testing (BAM)}, \orgaddress{\street{Richard-Willstätter-Straße 11}, \city{Berlin}, \postcode{12489}, \country{Germany}}}

\abstract{Grain boundaries are two-dimensional defects that control many physical and chemical properties of materials. We build on the recent description of grain boundaries as defect phases and devise a way to trigger and observe the transformation of grain boundary phases. The atomistic structure of a symmetric tilt grain boundary in Mg was resolved by high resolution scanning transmission electron microscopy. By introducing Ga atoms to the grain boundary, transformations towards other grain boundary phases were tracked. Ab initio simulations were performed to construct the defect phase diagram for the grain boundary and understand the progress of their phase transformations. A pattern recognition tool was developed to compare the experimental and simulated structural units at the grain boundary and automatically trace the phase transformation. The ability to construct defect phase diagrams and tune their phase transformations without changing the geometric confinement of defects enables a new paradigm of material designs.}

\keywords{Grain boundary complexion, defect phase diagram, transmission electron microscopy, density functional theory, automatic pattern recognition}



\maketitle

\section{Introduction}\label{intro}

The development of materials has been a central theme and driving force in human civilization. 
The enormous combinations of elements to form compounds in various phases and structures have provided scientists and engineers an inexhaustible mine for innovations. 
Since the derivation of a rigorous thermodynamic framework, culminating in the works of J. W. Gibbs to construct phase diagrams \cite{gibbs1948collected}, systematic exploration on state variables (such as temperature, pressure, chemical compositions) and their influence on the phases and properties has ever expedited materials developments.
In particular, coexistence of multiple phases and their transformations have been central to tailor materials for their applications in infrastructure, transportation \cite{Tasan2015}, energy systems \cite{Sharma2009} and medical devices \cite{Raabe2007}. 


Besides the constituent phases, crystallographic defects within the materials often play a decisive role on their properties, e.g., dislocations on the mechanical behavior.
The chemical partitioning around defects can deviate significantly from the bulk phases, as has been described in concepts such as the Cottrell atmosphere around dislocations \cite{Cottrell1949,Yu2022}, the Suzuki effect around stacking faults \cite{Suzuki1962}, and elemental segregation at grain boundaries (GB) \cite{Raabe2014}.
Such effects have been increasingly employed, as materials design experiences a paradigm shift to master and exploit the chemical complexity around defects, either termed ``defect phases'' \cite{frolov2015phases,Korte-Kerzel2022} or ``complexions'' to make a clearer distinction from bulk phases \cite{Dillon2007,Kaplan2013,Cantwell2014}. 

The road map for the exploration and design of defect phases has been recently reviewed \cite{Korte-Kerzel2022}. 
In particular, there is a pending need to develop defect phase diagrams, which share most attributes with bulk phase diagrams, but are much more extensive due to the large variety of crystallographic defects observed in materials and available for their design.
For example, transformations of linear (dislocation) defect phases are being discovered in both experiments \cite{Kuzmina2015} and atomistic modelling \cite{Turlo2020}. %
Two-dimensional defect phases are omnipresent in materials, including surfaces, phase boundaries in multi-phase, and GBs in polycrystalline materials. 
GBs separate the regions of individual single-crystalline grains and can control functional properties such as electrical resistivity \cite{Bishara2021,BuenoVilloro2023,BuenoVilloro2023a}, magnetic coercivity \cite{Duerrschnabel2017}, as well as mechanical strength and ductility \cite{Hall1951,Krause2018,Cantwell2020,Dehm2022}.


The thermodynamic theory on GB phase transformations \cite{HART1968179,Cahn1982,rottman_1991,PhysRevB.73.024102} has been significantly developed in the last decade by Frolov and Mishin \cite{Frolov2012,Frolov2012a}. 
It has been recently revealed that different defect phases can coexist at GBs even in elemental metals with only a single bulk phase, such as Cu \cite{Frolov2013,Meiners2020,Frommeyer2022}. 
The GB phases are built up from a limited number of short period segments known as basic building blocks or structural units \cite{ASHBY19781647,Sutton.1983.0021,Pond:a17068,Sutton1989}, e.g. tetrahedrons, octahedrons, etc.. 
The structural unit model is based on the nature of metallic bonding, where atoms at GBs tend to bond with the largest possible number of neighboring atoms \cite{Brink2022}. 
Such a structural nature provides us with the opportunity to explore the behavior of GBs in a similar way as bulk materials to understand how and why phase transformation at GB occurs. 
The fundamental understanding of GBs is the base for tuning defect structures for advanced alloy design \cite{Schuh2012,Cantwell2020}. 

Compared to the local stress state, introducing alloying elements is a more flexible way to control GB phases. 
On the mesoscopic scale, GB faceting transformation was observed in Cu GBs, triggered by adding Bi \cite{Ference1988}. %
In another case, Ga was found to segregate to Al GBs \cite{Sigle2006}, leading to deleterious liquid metal embrittlement \cite{Senel2014}. 
With the development of high resolution imaging by aberration-corrected scanning transmission electron microscopy (STEM), ever increasing numbers of GB phases have been discovered from experiments. 
This paves the way to observe atomic-scale GB phase transformation, necessary for the construction of GB phase diagrams.

To fully explore the design space of defects, their phase diagrams with chemical axes need to be developed. 
Nevertheless, the common practice applied to construct bulk phase diagrams, by synthesizing individual samples at different compositions, is difficult to apply to defect phase diagrams, as the defect states among samples may not be comparable or reproducible.
A plethora of defect types (including many GBs with different rotation axes, angles, and GB planes) exist within materials, while they all partition the same chemical composition within the matrix bulk phase. 
As a result, the chemical potentials of each element have been proposed as the more useful state variables (compared to their concentrations) to construct defect phase diagrams, as they are transferable to treat different types of defects and can be directly compared to chemical potentials in bulk phases \cite{Korte-Kerzel2022}. 
The thorny question remains on designing ways to observe the same defects by changing the local chemical potential (composition) at otherwise identical conditions. 
Procedures need to be developed to observe the same defects as they experience local changes in chemical potentials (compositions).

In this article, we devise an in-situ approach to trigger GB phase transformation by ion implantation and monitor the atomic-scale structural transformation at the same GBs by STEM. 
Our proof-of-principle study shows three types of GB phase transformations, namely structural, disorder to order, and order to order transformations of a symmetric tilt GB in Mg throughout the process of quasi in-situ Ga ion implantation at identical GB regions. 
The GB structural units are detected by a novel automatic pattern recognition method for unbiased comparison with structures from density functional theory (DFT) calculations. 
We present the way to construct defect phase diagram for this GB, including transformations induced by shifting the local chemical potential (composition) and by leaving time to reach local equilibrium states. 
Our methodology of quasi in-situ STEM experiments with varying chemical potential and DFT calculations provide a blueprint to study defect phase transformations and construct defect phase diagrams, which can be employed to investigate many more nanoscopic defects crucial to the design of modern materials.

\section{Results and Discussion}\label{res}
\subsection{GB structural units and phase transformation}

In hexagonal closed pack (HCP) crystals, most low angle [0001] tilt GBs adopt a symmetric configuration, composed of arrays of edge dislocations with periodic spacing in between \cite{Zhang2022}. 
As the tilt angle increases, the spacing between GB dislocations becomes closer so that their cores start to overlap at higher tilt angles. 
Symmetric tilt $\Sigma$7 [0001]${\{3\overline{1}\overline{2}0\}}$ GBs (abbreviated as $\Sigma$7 GBs) have a misorientation angle of $\approx$22°, which is on the border of low angle and high angle GBs. 
On one hand, the dislocation cores are 0.85 nm apart and can be identified individually. 
On the other hand, the periodic occurrence of a simple structural unit makes it convenient to describe it using the coincidence site lattice model, as $\Sigma$7 suggests a high symmetry GB with one out of seven motifs overlapping between the neighboring grains \cite{Zhang2022}. 

The structure of the $\Sigma$7 GB in pure Mg is atomically resolved by STEM imaging (Fig.~\ref{0Ga}a). 
Each dislocation core has a compact structure forming a kite shape and leaving a Burgers vector of $\frac{1}{3}\langle2\overline{1}\overline{1}0\rangle$ (Fig.~\ref{loop}a). 
This structure is referred to as the T-type structural unit in the literature \cite{Wang1997}. 
The three-dimensional (3D) representation of such a T-type unit is a tetrahedron consisting of four triangular faces (see Fig.~\ref{0Ga}d), a common structural unit to build GBs \cite{ASHBY19781647,Sutton.1983.0021,Pond:a17068,Sutton1989}. 
In pure Mg, we have only observed T-type structural units as building blocks for $\Sigma$7 GBs. 
As shown in Fig.~\ref{0Ga}a, an arrow points to a GB disconnection, where the GB plane shifts by $\frac{1}{6}\langle2\overline{1}\overline{1}0\rangle$, half of the Burgers vector. 

\begin{figure}[h]%
\centering
\includegraphics[width=0.98\textwidth]{Images/Fig.STEM.png}
\caption {\textbf{Experimental observation of a GB phase transformation in Mg by Ga\textsuperscript{+}
implantation.} (a-c) High angle annular dark field (HAADF)-STEM images of a $\Sigma$7 GB in pure Mg, and after Ga\textsuperscript{+} implantation for (b) 1 and (c) 3 minutes. The images are overlaid with polygon grids to highlight the GB structural units (Section~\ref{sec:PatternRec}), while raw experimental images are presented in Fig.~\ref{0Ga_raw}. The dashed lines in (a) and (b) highlight the GB planes and the arrows point to regions of GB disconnections. (d) The T-type structural unit (green tetrahedron) and (e) the A-type structural unit (pink capped trigonal prism) in 3D and the nomenclature of their atomic columns.}\label{0Ga}
\end{figure}

After observation of the T-type $\Sigma$7 GB in pure Mg (Fig.~\ref{0Ga}a), we gradually introduced Ga atoms onto the sample by Ga\textsuperscript{+} implantation using a focused ion beam (FIB). 
As detailed in Section~\ref{sec:implantation}, the same GBs were observed after both sides of the lamella had been exposed to a Ga\textsuperscript{+} beam for 1 min and for 3 min, translating to an overall concentrations of 0.50 and 1.24 at.\% Ga in Mg. 
As shown in Fig.~\ref{0Ga_raw}, the implanted GB areas show a higher atomic number (Z) contrast than the surrounding grains, demonstrating the tendency of Ga atoms to segregate at GBs \cite{Zhang2022}. 
Closer examination reveals that some GB structural units have been transformed to the A-type unit\cite{Wang1997}, as highlighted in Fig.~\ref{0Ga}b,c. 
The A-type unit has a larger core structure and a capped trigonal prism shape in 3D (Fig.~\ref{0Ga}e), where the same Burgers vector $\frac{1}{3}\langle2\overline{1}\overline{1}0\rangle$ can be constructed (Fig.~\ref{loop}b).

As shown in Fig.~\ref{0Ga}d,e, the atomic sites for the structural units are labelled in alphabetical order with respect to the increasing distance to the GB plane. 
Our DFT calculations reveal that both structural units have similar formation energies, 0.309 and 0.311 J m\textsuperscript{-2} for T-type and A-type, respectively. 
Therefore, the T-type unit is the ground state configuration for Mg $\Sigma$7 GB, which is consistent with DFT literatures on Mg \cite{huber_Mg} and ZnO \cite{Sato2007}.
This correlates well to the experimental observation of only T-type units in $\Sigma$7 GBs of pure Mg. 

To understand the structural transformation of the GB, we simulated the FIB process in DFT by replacing one Mg column in the T-type structural unit by Ga and studying their structural relaxation. 
As shown in Fig.~\ref{T-A}a, the T-type structural unit with the \textbf{b} site occupied by Ga atoms is no longer stable, but spontaneously relaxes into an A-type structural unit with Ga atoms at the \textbf{a1} site. 
A similar structural transformation sequence with Ga occupying a neighboring site is presented in the Fig.~\ref{1Ga}. 
In each relaxation step, the identical DFT structure is viewed from two perspectives. 
The left column highlights the defect shape assuming the T-type (green tetrahedron) and the right column assuming the A-type (pink capped trigonal prism) structural units. 
From the top to the bottom images, the distortion of the green tetrahedron increases as a function of relaxation steps, while that of the pink polyhedron decreases. 
This visualization shows the shift of the local symmetric structure from the T-type to the A-type structural unit, completing the GB phase transformation. 

\begin{figure}[h]%
\centering
\includegraphics[width=0.98\textwidth]{Images/Fig.DFT+PR.png}
\caption{\textbf{Ga-induced phase transformation of a Mg $\Sigma7$ GB from T-type to A-type structural units.} (a) Snapshots of the structural transformation path obtained from DFT calculations going from the T-type (step 0) to the A-type (final step 52).
Starting with a T-type structure with Ga atoms occupying the \textbf{b} site, in each image the left and right columns highlight the perspectives of T-type (green tetrahedron) and A-type (pink capped trigonal prism) structural units, respectively. 
(b) Automatic pattern recognition to classify experimental images into T-type and A-type structural units. The displacement vectors between DFT and experimental structures are magnified by five times for visualization.}\label{T-A}
\end{figure}

In the early stage of Ga\textsuperscript{+} implantation, there is no clear preference of Ga atoms to occupy specific sites at the $\Sigma7$ GB (Fig.~\ref{0Ga}b), so that the GB phase is chemically disordered. 
A direct simulation of sufficiently large supercells to fully capture this disorder is
challenging for the computational demanding DFT approach. 
Instead, we used the DFT relaxation in Fig.~\ref{T-A}a and Fig.~\ref{1Ga} to understand the GB structural transformation from T-type to A-type units. 
Based on both sequences, we have developed automatic pattern recognition to classify experimental GB images into T-type and A-type units, as schematically shown in Fig.~\ref{T-A}b and detailed in Section~\ref{sec:PatternRec}. 
In short, DFT structures serve as inputs to locate GB structural units in the experimental images. 
Then the positions of the atomic columns are labelled and compared with the DFT structures to reach a decision on a better match to T-type or A-type units.
Moreover, it is observed from Fig.~\ref{T-A}a that after the structural transformation, the GB plane is shifted (in this case, to the right) by $\frac{1}{6}\langle2\overline{1}\overline{1}0\rangle$, half of the Burgers vector. 
This is experimentally captured in the middle T-A-T section of Fig.~\ref{0Ga}b. 
In this case, the GB plane of the A-type section is shifted to the left. 
It is also noteworthy that there is no shift in the GB plane between the T and A structural units at the two GB disconnections (yellow arrows in Fig.~\ref{0Ga}b).

\subsection{Construction of the GB defect phase diagram}

As shown in Fig.~\ref{DPD}a, Ga\textsuperscript{+} implantation into the sample causes an increase in the chemical potential of Ga, $\mu_{\rm Ga}$. 
Although A-type structural units with Ga atoms have higher energy than the T-type Ga-free units at Ga-poor conditions, their formation energies decrease with $\mu_{\rm Ga}$ at a slope proportional to the number of Ga atoms at the GB.
A coverage of the GB with 2.3 Ga atoms/nm$^2$, corresponding to one incorporated Ga atom per structural unit (Fig.~\ref{T-A}a), is sufficient to stabilize the A-type structure and starts to become favourable at $\mu_{\rm Ga}>-0.477$ eV. 
This finding is consistent with the experimental observation of T-type to A-type transformation upon Ga\textsuperscript{+} implantation.

\begin{figure}[h]%
\centering
\includegraphics[width=0.98\textwidth]{Images/Fig.DPD.png}
\caption{\textbf{Defect phase diagram of the Mg $\Sigma$7 [0001]${\{3\overline{1}\overline{2}0\}}$ symmetric tilt GB.} (a) The defect phase diagram as a function of chemical potential of Ga, $\mu_{\rm Ga}$, identifies four GB phases observed in experiments. The upper axis provides the corresponding Ga concentration in the Mg solid solution at 300 K. (b-e) The four experimental structures are stable in different ranges of $\mu_{\rm Ga}$. Ga implantation increases $\mu_{\rm Ga}$ and causes the structural transformation from T-type (b) to A-type structural units (c). After ion implantation, $\mu_{\rm Ga}$ decreases with increasing time so that the 6-Ga ordered GB phase (e) is transformed to the 3-Ga ordered GB phase (d).}\label{DPD}
\end{figure}

By continuing the Ga\textsuperscript{+} implantation, the chemical potential range can be substantially increased. 
A Ga-rich condition where the system is close to an equilibrium with a bulk Ga phase reservoir (i.e., $\mu_{\rm Ga}=0$ eV) was approached by thinning down the TEM sample completely with a focused Ga\textsuperscript{+} beam instead of using a Xe\textsuperscript{+} beam that led to the Ga-free GB state. 
Once Mg has been exposed to an overabundant amount of Ga, the Ga segregation to the GB becomes very apparent (Fig.~\ref{nGa}a). 
As shown in Fig.~\ref{nGa}b, the GB structure clearly follows the A-type unit. 
For each structural unit, there are six atomic columns showing brighter intensity, including pairs of \textbf{b}, \textbf{c}, and \textbf{e} sites.

\begin{figure}[h]%
\centering
\includegraphics[width=0.8\textwidth]{Images/Fig.Time.png}
\caption{\textbf{Transformation of chemically-ordered GB phases.} HAADF-STEM images of the same $\Sigma$7 GB (a-b) 1 day and (c-d) 20 months after Ga\textsuperscript{+} beam thinning. The long storage time enabled longer range Ga diffusion to form bulk Mg\textsubscript{5}Ga\textsubscript{2} precipitates, while the ordered 6-Ga unit transforms to a differently ordered 3-Ga unit. The T-type and A-type structural units are highlighted by green tetrahedrons and pink capped trigonal prisms, respectively. The corresponding STEM images without overlaid grids are presented in Fig.~\ref{nGa_raw}.}\label{nGa}
\end{figure}

According to the GB defect phase diagram (Fig.~\ref{DPD}a), the 6-Ga ordered GB phase is only stable at extreme Ga-rich conditions, $\mu_{\rm Ga}>-0.190$ eV. 
To put this chemical potential into context, $\mu_{\rm Ga}=-0.222$ eV was evaluated for a Ga atom inside the Mg phase, which is out of the stability range of the 6-Ga ordered GB phase. 
This indicates that additional phase transformations would occur at Mg GBs by changing $\mu_{\rm Ga}$. 
To explore the gap between the highest Ga loading (6-Ga configuration) and low coverages of Ga (approximated by the 1-Ga configuration), we exploited the time axis to allow an equilibration of the system and therewith a reduction of $\mu_{\rm Ga}$. 

As shown in Fig.~\ref{nGa}c, after 20 months storage in a desiccator, the same GB underwent a structural transformation to another ordered phase. 
Instead of six, only three bright Ga columns remained in each structural unit. The Ga atoms stay on the \textbf{b}, \textbf{c}, and \textbf{e} sites, except that they are no longer present in pairs. 
In the upper part of Fig.~\ref{nGa}d, the \textbf{b1} site is occupied by Ga along with the \textbf{c2} and \textbf{e2} sites on the right. 
Likewise, the lower part of the image shows the mirrored occupation on \textbf{b2}, \textbf{c1} and \textbf{e1} sites. 
Except for a structural unit at the disconnection, all other units remain to be classified as A-type. 
As shown in Fig.~\ref{nGa_raw}b-d, the excess Ga atoms have diffused to form bulk Mg\textsubscript{5}Ga\textsubscript{2} precipitates, belonging to the most Mg-rich intermetallic phase in the Mg-Ga system\cite{Feng2009}. 
This is another indication that the system has developed towards an equilibrium configuration during the 20 months storage. 

Energy dispersive X-ray spectroscopy (EDS) reveals a Ga content of 0.7 at.\% inside the Mg grains, which increases significantly at the $\Sigma$7 GB (Fig.~\ref{nGa_raw}g,h).
For this GB, the Gibbsian interfacial excess (shaded area in Fig.~\ref{nGa_raw}h) is 6.95 Ga atoms/nm$^2$, corresponding to a coverage of 3 Ga atoms per structural unit.
As there are exactly three bright columns in each structural unit, the ordered phase is simulated using full Ga occupancy at the corresponding sites.
As shown in Fig.~\ref{DPD}a, the ordered 3-Ga GB phase is stable in a range of chemical potentials between $-0.353$ and $-0.190$ eV. 

In the dilute limit, the chemical potential inside Mg grains can be related to the local concentration of Ga ($c_{\rm Ga}$) by $\mu_{\rm Ga}=-0.222$ eV + $k_BT \ln (c_{\rm Ga})$. 
At 300 K ($k_BT=26$ meV), dilution of Ga in the Mg matrix by one order of magnitude corresponds to a reduction of $\mu_{\rm Ga}$ by 60 meV. 
The chemical potential $\mu_{\rm Ga}$ is hence related to the local equilibrium with Ga solutes in Mg, as represented by the top axis of Fig.~\ref{DPD}a. 
The Ga concentration in the Mg solid solution of our sample is evaluated as $c_{\rm Ga}=0.007$. The corresponding chemical potential $\mu_{\rm Ga}=-0.351$ eV is marked on the Fig.~\ref{DPD}a, and falls in the stability range of the ordered 3-Ga GB phase. 
The defect phase diagram further shows that even until $\mu_{\rm Ga}=-0.477$ eV, corresponding to very low Ga concentrations of $c_{\rm Ga}=5\cdot10^{-5}$, the A-type structural unit with Ga decoration is still favored against the T-type. Since this concentration is well within the solubility limit, the driving force to lower $\mu_{\rm Ga}$ disappears, and hence the A-type units remain stable at the GB. 


\section{Conclusion}\label{conclusion}

In summary, we have demonstrated an effective way to construct defect phase diagrams by combining atomic-scale STEM characterization of defect structures, automatic pattern recognition and DFT modelling of their energetics. 
The chemical potential axis is sampled experimentally by successive ion implantation as well as allowing time for diffusion to transit local thermodynamic equilibrium to a more global scale. 
We have driven and monitored the phase transformation of the same Mg $\Sigma$7 GB from a T-type tetrahedron structural unit to A-type capped trigonal prism unit by Ga\textsuperscript{+} implantation. 
Different GB phases with ordered Ga atoms have been identified using STEM, including a 6-atom configuration in equilibrium with Ga and a 3-atom configuration in equilibrium with Mg-Ga solid solution. 
The experimental exploration covers a full range of chemical potentials relevant for the design of GB phases, and the structural information can be directly fed into atomistic modelling and narrows down the otherwise gargantuan space of configurations. 
DFT calculations not only provide the relative stability of different GB phases, but also connect them to the local equilibrium by the chemical potential. 
The developed methodology can be widely applied to study wide ranges of GBs and defects, and expedites the construction of defect phase diagrams for their usage in science and engineering. %



\bmhead{Supplementary information}

If your article has accompanying supplementary file/s please state so here. 

\bmhead{Acknowledgments}

This work was supported by the German research foundation (DFG) within the Collaborative Research Centre SFB 1394 “Structural and Chemical Atomic Complexity—From Defect Phase Diagrams to Materials Properties” (Project ID 409476157). The authors thank Philipp Keuter and Jochen M. Schneider for providing the Mg thin film sample. 



\begin{appendices}

\section{Methodology}\label{method}

\subsection{Synthesis}
The nanocrystalline Mg thin film was sputter-deposited onto a Si (100) substrate. Detailed description on the synthesis conditions are given in \cite{Zhang2022}, as well as the characterization of the sharp basal plane texture of Mg. The thin film hence contains numerous Mg [0001] tilt GBs with a random distribution of the misorientation angle \cite{Zhang2022}. The $\Sigma7$ GBs were selected by examining GBs with a misorientation of $\approx$22°.

\subsection{Ion implantation to trigger defect phase transformation}
\label{sec:implantation}
Ga\textsuperscript{+} implantation is applied to gradually change the local composition of GBs in Mg. First, the Ga-free specimen for STEM was prepared on a plasma (Thermo Fisher) FIB starting with Xe\textsuperscript{+} polishing at 30 kV and ending with a cleaning step a 8 kV. A Scios2 (Thermo Fisher) FIB was then employed to introduce Ga\textsuperscript{+} ions into the lamella. For that, cleaning cross-section patterns were used with samples tilted $\pm$8° towards the Ga\textsuperscript{+} beam at 5 kV and 7.7 pA. In one minute, 4.8 fmol of Ga\textsuperscript{+} ions are implanted on the sample surface. The sample has a volume of $6.5 \cdot 3 \cdot 0.2~ {\rm \mu m}^3$, amounting to 280 fmol of Mg atoms (24.305 g/mol, 1.74 g/cm\textsuperscript{3}). Suppose an implantation rate of 100\%, 1.7\% of Ga relative to Mg would be introduced to the sample. From the measured Ga concentrations in the Mg sample, 0.50\% and 1.24\% after 1 min and 3 min Ga\textsuperscript{+} implantation, respectively, the implantation rates are evaluated as 29\% (after 1 min) and 24\% (after 3 min). 

\subsection{Electron Microscopy}
High resolution STEM imaging was performed on a Titan Themis microscope (Thermo Fisher) operated at 300 kV. Using an aberration-corrected STEM probe of less than 0.1 nm size and 23.8 mrad convergence semi-angle, STEM images were acquired using the high angle annular dark field (HAADF) detector with collection semi-angles of 62-200 mrad. 
Energy dispersive X-ray spectroscopy (EDS) spectrum imaging was acquired using the SuperX detector. Multivariate statistical analysis was applied for noise reduction \cite{Zhang2018}, and subsequent elemental quantification was performed using the Cliff-Lorimer method. 
Precession electron diffraction 4D-STEM imaging was performed on a JEM2200 microscope (JEOL) operated at 200 kV. The beam size was $\approx$2 nm and a precession angle of 0.5° was used. 

\subsection{Computational Details}\label{DFT_calculations}
The density functional theory (DFT) calculations in this work have been carried out using the Vienna Ab initio Simulation Package (VASP) \cite{vasp_1,vasp_2} with the projected augmented wave method \cite{vasp_3} to describe the interaction between ionic cores and valence electrons. The Perdew-Burke-Ernzerhof form of parameterization of the generalized gradient approximation has been used to describe the exchange-correlation effects. A plane wave cutoff energy of 550 eV was used and based on the Monkhorst-Pack scheme \cite{monkhorst_pack}, the Brillouin zone was sampled with a k-point spacing of 0.12 nm\textsuperscript{-1} along all directions for all structures. The Methfessel-Paxton \cite{methfessel_paxton} smearing scheme was applied with the smearing width set to 0.15 eV.\par 

After the construction of supercells containing two GBs, they have been optimized by subjecting them to strains in the direction normal to the GB. A relaxation of atomic positions is performed in order to get to the equilibrium structure, thus preserving the lattice constants in the bulk regions of the supercell as well. The energy for this equilibrium structure is then used to calculate the GB energy ($\gamma_{GB}$) using the following equation:
\begin{equation} \label{GBE_eq}
    \gamma_{\text{GB}} = \frac{E_{\text{GB}} - E_{\text{bulk}}}{2A_{\text{GB}}}, 
\end{equation}
where $E_{\text{GB}}$ is the total energy of the supercell with the GBs, $E_{\text{bulk}}$ is the energy of hcp Mg bulk rescaled according to the number of atoms in the supercells for the GB and $A_{\text{GB}}$ is the area of cross-section of the GB. As each GB supercell contains two GBs, a factor of two is included in the denominator.

To analyze the competition of the defect phases with and without Ga addition, we look into the formation energy $E_{\text{f}}$ of the phases that are calculated using the following equation:
\begin{equation} \label{form_en}
    E_{\text{f}} = \frac{E_{\text{GB}} - N_{\text{Mg}}\mu_{\text{Mg}} - N_{\text{Ga}}\mu_{\text{Ga}}}{2A_{\text{GB}}},
\end{equation}
where $N$ represents the number of Mg and Ga atoms in the supercell, and $\mu$ represents their chemical potentials. %

\subsection{Automatic Pattern Recognition}
\label{sec:PatternRec}
To detect and distinguish the T-type and A-type structural units in STEM images, we use a novel mathematical framework that detects patterns in the form of atomic arrangements from simulations in atomic scale images and quantifies their differences by estimating the deformation between experimental and computational atomic arrangement.
This framework uses a multi-step procedure. First, a given simulated atomic arrangement is converted to a synthetic image patch with a sum of Gaussians centered at the simulated positions (``Structural units'' in Figure~\ref{T-A}). The potential occurrences of this patch in the STEM image are found with template matching using normalized correlation (see ``Experiments'' for example image patch matches in Figure~\ref{T-A}).
On each matching patch, the deformation between the simulated atomic column arrangement to the positions in the STEM image is estimated using bump fitting (``Comparison'' exemplifies the full deformation as shift for each atom in the structural unit in Figure~\ref{T-A}). The fitting of the positions from the simulated pattern to the experimental image is split into two steps: First the affine part of the deformation is determined, then the remaining nonlinear part. 

From the resulting fit, we derive feature descriptors that allow to determine which of the given patterns is present. Noting that near a candidate location for an A-type match, there are two neighboring candidates for a T-type match. %
We derive the same descriptors on these neighboring positions using the T-type arrangement. Only one of these neighboring matches is actually feasible, the infeasible one can be easily determined from the large deformation necessary to fit the simulation pattern to the image and is discarded (Figure~\ref{T-A} shows only the feasible match). 
Thus, for each A-type match, we have two sets of descriptors, one describing how well the A-type fits, the other one on the T-type fits. 

Three descriptors are employed for the decision making: 1. The standard deviation of the horizontal coordinates of the three atomic sites \textbf{a1}, \textbf{a2}, and \textbf{a3} (see Fig.~\ref{0Ga}d,e for the atomic site labels). A number close to zero corresponds to a straight GB plane; 2. The difference in $y$ coordinates between the \textbf{b1} and \textbf{b2} sites; 3. The difference in $y$ coordinates between the \textbf{d1} and \textbf{d2} sites. For the latter two descriptors, a number close to zero corresponds to a better mirror symmetry with respect to the GB plane. 
For a perfect T-type structure, all three descriptors have zero values, while the neighboring A-type motifs return bigger values and can be excluded. Likewise, for a perfect A-type structure, all three descriptors are zero, while the neighboring T-type motifs return bigger values and can be excluded. 
For decision making on the experimental images, the structural units with two or all of three descriptors returning smaller values (closer to zero) are designated (``Decision'' in Figure~\ref{T-A}).

The union of the atom sites of the chosen matches describes most atoms at the grain boundary. To visualize the entire atomic grid topology, positions of atomic columns in the neighboring grains are still needed. For this purpose, we applied the motif extraction approach described in \cite{AlZhBe23} to determine the unit cell motifs of the left and right grains in the STEM images. The hexagons from both grains are then constructed and merged with the four-sided polygons from the detected T-type units and the eight-sided polygons from the detected A-type units to describe the atomic grid topology.

\subsection{STEM multi-slice image simulation}
The purpose of the image simulation is to create simulated HAADF images based on the DFT calculated structures and compare them with the experimental measurements. We performed the STEM multi-slice simulations using the muSTEM (v5.2) software package \cite{ALLEN201511}. The microscope parameters for the simulations, such as the half-convergence angle (23.6 mard), primary electron energy (300 kV), and HAADF detector (62-200 mrad), were chosen to match the experimental measurements.   

\begin{figure}[h]%
\centering
\includegraphics[width=0.98\textwidth]{Images/Fig.4dstem.png}
\caption{(a) Orientation and GB maps reconstructed from the 4D-STEM data set. The thin film sample shows sharp (0001) texture (red color). Grains with a confidence index of less than 0.1 are shown in black. (b) Bright-field STEM image for the highlighted region in (a). White arrows in both figures point to the $\Sigma$7 GB for the high-resolution STEM study.}\label{4d}
\end{figure}

\begin{figure}[h]%
\centering
\includegraphics[width=0.98\textwidth]{Images/Fig.STEM_raw.png}
\caption {HAADF-STEM images without overlaid grids, corresponding to the ones presented in Fig.~\ref{0Ga}.}\label{0Ga_raw}
\end{figure}

\begin{figure}[h]%
\centering
\includegraphics[width=0.8\textwidth]{Images/Fig.Loop.png}
\caption{Burgers circuit analysis for the (a) T-type and (b) A-type structural units. The black arrows show pairs of $\frac{1}{3}\langle2\overline{1}\overline{1}0\rangle$ vectors that are closed by the Burgers vectors $\vec{b}=\frac{1}{3}[2\overline{1}\overline{1}0]$ (red arrows).}\label{loop}
\end{figure}

\begin{figure}[h]%
\centering
\includegraphics[width=0.98\textwidth]{Images/Fig.DFT1Ga.png}
\caption{Snapshots of DFT structural relaxation starting with a T-type unit with Ga atoms on the (a) \textbf{b} and (b) \textbf{e} sites, ending to an A-type unit with Ga atoms on the (a) \textbf{a1} and (b) \textbf{b} sites. For each image, the right two columns highlight the perspectives of T-type (green tetrahedron) and A-type (pink capped trigonal prism) structural units, respectively.}\label{1Ga}
\end{figure}

\begin{figure}[h]%
\centering
\includegraphics[width=0.98\textwidth]{Images/Fig.Mg5Ga2.png}
\caption{HAADF-STEM images of the same $\Sigma$7 GB (a) 1 day and (b) 20 months after Ga\textsuperscript{+} beam thinning. The long storage time enabled longer range Ga diffusion to form bulk precipitates characterized as Mg\textsubscript{5}Ga\textsubscript{2} by (c) high resolution HAADF-STEM image and (d) the corresponding fast Fourier transformation. (e,f) High resolution HAADF-STEM images of the same $\Sigma$7 GB (e) 1 day and (f) 20 months after Ga\textsuperscript{+} beam thinning. (g) EDS Ga maps of the corresponding area in (f) and (h) Ga composition profile across the GB.}\label{nGa_raw}
\end{figure}

\begin{figure}[h]%
\centering
\includegraphics[width=0.98\textwidth]{Images/Fig.Multislice.png}
\caption{Multislice STEM simulations for the structural models obtained from DFT calculations. Left: atomistic structural model. Right: Multislice STEM simulations. (a) T-type and (b) A-type pure Mg $\Sigma7$ GBs, and A-type units with chemical ordering of (c) three and (d) six Ga columns. The black arrows point to locations of the GBs.}\label{STEMsim}
\end{figure}

\end{appendices}



\bibliography{bibliography.bib}{}%


\end{document}
