\documentclass[10pt,twocolumn,letterpaper]{article}

\usepackage{iccv}
\usepackage{times}
\usepackage{epsfig}
\usepackage{graphicx}
\usepackage{amsmath}
\usepackage{amssymb}
\usepackage{comment}
\usepackage{multirow}
\usepackage{dsfont}
\usepackage{caption}
\usepackage{stfloats}
\usepackage{subcaption}

\usepackage[pagebackref=true,breaklinks=true,letterpaper=true,colorlinks,bookmarks=false]{hyperref}

\iccvfinalcopy

\def\httilde{\mbox{\tt\raisebox{-.5ex}{\symbol{126}}}}
\DeclareMathOperator*{\argmin}{arg\,min}


\newcommand{\OURS}{HyperDiffusion\xspace}
\newcommand{\MLP}{MLP\xspace}
\newcommand{\Indicator}{\mathds{1}}


\ificcvfinal\pagestyle{empty}\fi

\begin{document}

%%%%%%%%% TITLE
\title{\OURS: Generating Implicit Neural Fields with Weight-Space Diffusion}

\author{Ziya Erko\c{c}\textsuperscript{1}
\and
Fangchang Ma\textsuperscript{2}
\and
Qi Shan\textsuperscript{2}
\and
Matthias Nie{\ss}ner\textsuperscript{1}
\and
Angela Dai\textsuperscript{1}
\and
\textsuperscript{1}Technical University of Munich \, \textsuperscript{2}Apple\\ \\
\url{https://ziyaerkoc.com/hyperdiffusion}
}

\twocolumn[{%
	\renewcommand\twocolumn[1][]{#1}%
	\maketitle
	\begin{center}
		\includegraphics[width=\linewidth]{images/teaser.pdf}
		\captionof{figure}{
		\OURS{} enables a new paradigm in directly generating neural implicit fields by predicting their weight parameters. 
		We leverage implicit neural fields to optimize a set of MLPs that faithfully represent individual dataset instances (``Overfitting," top-left).
        Our network, based on a transformer architecture, then models a diffusion process directly on the optimized MLP weights (``Diffusion," bottom-left). 
        This enables synthesis of new implicit fields (``Synthesis," bottom-right).
		}
		\label{fig:teaser}
	\end{center}    
}]

\maketitle
\ificcvfinal\thispagestyle{empty}\fi

%%%%%%%%% ABSTRACT
\begin{abstract}
The current study investigated possible human-robot kinaesthetic interaction using a variational recurrent neural network model, called PV-RNN, which is based on the free energy principle.
Our prior robotic studies using PV-RNN showed that the nature of interactions between top-down expectation and bottom-up inference is strongly affected by a parameter, called the meta-prior, which regulates the complexity term in free energy.
% The current study examines how the behaviours of robots alter by changing the meta-prior $w$ in human-robot kinaesthetic interaction.
The current study examines how changing the meta-prior $w$ in the interaction phase affects the counter force generated when an experimenter attempts to induce movement pattern transitions familiar to the robot through its prior training.
The study also compares the counter force generated when trained transitions are induced by a human experimenter and when untrained transitions are induced.
Our experimental results indicated that (1) the human experimenter needs more/less force to induce trained transitions when $w$ is set with larger/smaller values, (2) the human experimenter needs more force to act on the robot when he attempts to induce untrained as opposed to trained movement pattern transitions.
Our analysis of time development of essential variables and values in PV-RNN during bodily interaction clarified the mechanism by which gaps in actional intentions between the human experimenter and the robot can be manifested as reaction forces between them.


%% Hiroki writing 2022-11-4
%Current study investigates the dynamics of the latent states during human-robot kinaesthetic interaction using PV-RNN.
%We have achieved to observe and analyse the internal state of an RNN model based on the free energy principle, during real-time human-robot interaction.
%Essential characteristics observed in the previous study of this variational recurrent neural network model, PV-RNN, is that by changing a meta prior $w$, the balance between the top-down intention and the bottom-up perceptual reality changes.
%In the current study, we examined how changing the weighting parameter $w$ between accuracy and complexity in free energy principle affects the humanoid robot's behaviour through human-robot interaction. We have conducted some human-robot kinaesthetic interaction experiments with various $w$ and quantitatively analysed the latent variable and the force applied to the humanoid robot. We have observed that the force required to change the robot's intention has increased, both when the top-down intention was strengthened by changing the $w$ and when corresponding switch of its primitive was against the experience of the RNN during its training. The study confirms through quantitative analysis that by increasing or decreasing the $w$ in PV-RNN, humanoid robot leads or follows the human counterpart during the human-robot kinaesthetic interaction.

\begin{comment}
Comment from Jun #2
・最後にQualitativeな結果(インパクト)が欲しい
・Current study investigates the problem on~と書き出すのが一般的
・最初の一文と最後の一文を対応させる
・最後の一文はもう少しAbstractかつ包括的に
\end{comment}

\begin{comment}
Comment from Jun #1
We investigated how the kinaesthetic human-robot interaction can affect the internal state of a model based on the free energy principle. 
=> how the internal state is affected is not the most important point in this study. This part should be rewritten.

The key function of this variational recurrent neural network model, PV-RNN, is that by changing a meta prior $w$, it takes a balance between the "complexity” term and the ”accuracy” term which corresponds to a top-down intention and a bottom-up perceptual reality in the free energy principle, respectively. 
=> This is not key function of PV-RNN. It is an essential characteristics observed in the previous study. The grammar after $w$ is something strange. Rewrite these.

This research has conducted a human-robot interaction experiment with a robotic agent in a kinaesthetic sense.
=> The sentence is not good. "in a kinaesthetic sense" is grammatically wrong.
MODIFIED => "In the current study human-robot interaction experiments using the kinaesthetic sense were conducted."

We investigated that when human forces the agent to switch primitives from one to another, larger force was required both when the human intention is conflictive against the top-down the intention of the agent and when the agent has a stronger top-down intention by modifying the $w$.
=> You should write the essential results of the experiments rather than what we investigated and also how these results could contribute to the studies on human-robot interaction.
\end{comment}

\end{abstract}

%%%%%%%%% BODY TEXT
\section{Introduction}
\label{sec:intro}
\begin{figure}[t]
\begin{center}
    \includegraphics[width=1\linewidth]{figures/teaser.pdf}
\end{center}
\vspace{-0.1in}
\caption{\textbf{{\em Foggy} vs {\em Clear} NeRF.} Our \ournerf gets rid of reconstruction errors manifested as foggy ``floaters" in the density volume without additional input or significant computational overhead. 
%
Below are density profiles along a given ray before and after our geometry correction procedure, where we discard density peaks corresponding to floaters.
}
\label{fig:teaser}
\vspace{-0.2in}
\end{figure}



%The emergence of 
Neural Radiance Fields (NeRFs)~\cite{mildenhall2020nerf}  %and its variants 
have made revolutionary contributions in %photo-realistic 
novel view synthesis~\cite{barron2021mip,barron2022mip}, 
autonomous driving~\cite{rematas2022urban,tancik2022block}, digital human~\cite{hong2022headnerf,zhao2022humannerf}, and 3D content generation~\cite{eg3d,poole2022dreamfusion,lin2022magic3d}.
%by leveraging a multi-layer perceptron (MLP) to implicitly model the mapping from input 5D coordinates (i.e., 3D coordinates $\mathbf{x} = (x,y,z)$ and 2D viewing directions $\mathbf{d}=(\theta,\phi)$) to volume density $\sigma$ and view-dependent emitted radiance color $\mathbf{c} = (r,g,b)$. 
%
%They then use traditional volume rendering mechanisms on the obtained continuous 5D function (i.e., MLP) to generate novel views. 
To date, unfortunately, most NeRF-based methods encounter challenges when tackling large-scale cluttered scenes (e.g., Fig.~\ref{fig:teaser}):
\begin{enumerate}[leftmargin=0.16in, topsep=2pt,itemsep=-1ex,partopsep=1ex,parsep=1ex]
\item Input observations used for NeRF are often too sparse  compared to forward-facing or synthetic looking-inward scenes;
%\item Recovering fine-grained objects within a large volume is challenging for NeRF; %in capturing details accurately.
\item View-dependent visual effects give rise to ambiguity, resulting in a ``foggy" density field as shown in Fig.~\ref{fig:teaser}. 
%
Such artifacts are particularly pronounced in indoor scenes strewn with view-dependent appearances, such as specular highlights, glossy surface reflections from man-made objects. 
\end{enumerate}

Despite attempts to enhance NeRF's rendering quality given suboptimal input, such as using 3D conical frustums~\cite{barron2021mip,barron2022mip}, physically-grounded augmentations~\cite{chen2022aug}, and misalignment correction~\cite{jiang2022alignerf},  these challenges have yet to be fully resolved.
%
Depth supervision~\cite{deng2022depth, wei2021nerfingmvs} or proxy geometry~\cite{xu2021scalable,wu2022scalable} images can help alleviate the challenges in handling large-scale with sparse input, at the expense of %but they come at the cost of requiring 
expensive pre-processing or additional input.
%
Another line of work~\cite{wang2021neus, oechsle2021unisurf, wang2022neuris} achieves better reconstruction of surface geometry by using signed distances instead of volume density as scene representation. However, they sacrifice the ability to synthesize photo-realistic novel views.

%We observe that NeRF has been suffering from foggy ``floater" artifacts in large-scale cluttered scenes.
%
%Such artifacts are particularly pronounced in indoor scenes strewn with view-dependent appearances from man-made objects. 
%
To address the above issues, we propose an extension to NeRF, dubbed as {\bf \ournerf}, which enforces effective {\em appearance} and {\em geometry} constraints conducive to accurate colors and 3D densities estimation. We believe \ournerf can contribute beyond novel view synthesis, such as NeRF object detection~\cite{hu2022nerf}, NeRF object segmentation~\cite{zhi2021place, liu2022unsupervised, fan2022nerf,ren2022neural}, and NeRF registration~\cite{goli2022nerf2nerf}, where the rooms for improvement are substantial if more accurate color and density estimation are available.

Correspondingly, there are two steps in \ournerf. First, for appearance correction, the view-independent and view-dependent color components are predicted from the underlying 3D scene, which is combined to produce the final color estimation (Fig.~\ref{fig:toaster}).
%
The view-independent component (diffuse color and shading) captures the overall scene color, while the view-dependent component (highlights or reflections) captures color variations due to changes in viewing angle.
%
\ournerf then discards these view-dependent appearances in the training views to prevent them from interfering with the density estimation.
%
Second, a simple and effective geometry correction procedure will be performed to further eliminate the foggy ``floaters" or density errors. This geometry correction procedure is based on an assumption in line with traditional ray tracing in computer graphics.
\begin{comment}
% xh: basically copying method
On the other hand, ClearNeRF performs a geometric correction procedure performed on each traced ray during inference to refine the density estimation and better tackle the floater artifacts. 
%
The geometry correction procedure assumes that there should only be one salient peak along each traced ray during NeRF inference. 
Only the salient peak closest to the ray origin (the camera center) corresponds to  true geometry while the others will be manifested as foggy floaters hovering in the density volume. 
%
This assumption is in line with traditional ray tracing in computer graphics where in the absence of noise, only one intersection per ray should be returned to indicate the closest ray-object intersection.
%
\end{comment}
%%%%%%%%%%%
%As shown in Fig.~\ref{fig:teaser}, when reconstructing an indoor scene with sparse input and highly view-dependent objects, NeRF produces severe floating artifacts due to its attempt to explain view-dependent appearances.
%
Experiments verify that our proposed \ournerf can effectively get rid of floater artifacts without additional input.% or significant computational overhead. 


In summary, our contributions include the following:
\begin{itemize}[leftmargin=0.16in, topsep=2pt,itemsep=-1ex,partopsep=1ex,parsep=1ex]
    \item We propose a concise method for decomposing view-independent and view-dependent appearance during NeRF training and eliminate the interference of view-dependent appearance.
    \item We propose a geometric correction procedure performed on each traced ray during inference to refine the density estimation and better tackle the floater artifacts.
    \item Extensive experiments and ablations verify the effectiveness of our core designs and results in improvements over the vanilla NeRF and other state-of-the-art alternatives.
    %without additional computational resources or other inputs.
\end{itemize}




\section{Related Work}
\label{sec:related_work}
\subsection{Co-Speech Gesture Synthesis}
The early approaches for generating co-speech gestures often involve creating linguistic rules to translate speech input into a sequence of pre-collected gesture segments, which are typically referred to as rule-based methods \cite{cassell1994rulefullbody,cassell2001beat,kipp2004gesture,kopp2006bml}. \citet{wagner2014rulereview} provide a comprehensive review of these methods. Rule-based methods produce interpretable and controllable results, but creating gesture datasets and rules requires significant effort. To alleviate the manual effort of designing rules in rule-based methods, data-driven approaches have gradually become predominant in this field. \citet{nyatsanga2023data_driven_gesture_survey} offer a thorough survey of these methods. Early data-driven approaches aim to directly learn mapping rules from data through statistical models \cite{neff2008videogesture,levine2009prosodygesture,levine2010gesturecontroller} and combine them with predefined gesture units for gesture generation. Later, the powerful modeling capability of deep neural networks makes it possible to train complex end-to-end models using raw speech-gesture data directly. One option is deterministic models, such as MLP \cite{kucherenko2020gesticulator}, CNN \cite{habibie2021videogesture}, RNN \cite{yoon2019robot,yoon2020trimodalgesture,bhattacharya2021affectivegesture,liu2022hierarchicalgesture}, and Transformer \cite{bhattacharya2021text2gestures}. Another choice is generative models, including flow-based models \cite{alexanderson2020stylegesture,ye2022styleflowgesture}, VAEs \cite{li2021audio2gesture,ghorbani2022zeroeggs}, and VQ-VAE \cite{yi2022talkshow,yazdian2022gesture2vec,liu2022vqgesturevideo}. Due to the inherent many-to-many relationship between speech and gesture, end-to-end models can generate natural-looking gestures but face challenges in ensuring content matching between speech and generated gestures \cite{yoon2022genea}. To address this issue, some neural systems aim to explicitly model both rhythm and semantics from the perspective of model structure \cite{kucherenko2021speech2properties2gestures,ao2022rhythmicgesticulator,liu2022disco} or training supervision strategy \cite{liang2022seeg}. Furthermore, hybrid systems, such as the combination of deep features and motion graphs \cite{zhou2022gesturemaster}, have been proposed to harness the advantages of different approaches. Recently, diffusion models \cite{sohldickstein2015diffusion,song2020improvedscore,ho2020ddpm} have demonstrated impressive results in image synthesis \cite{ramesh2022dalle2} and human motion generation \cite{tevet2022humanmotiondiffusion, zhang2022motiondiffuse}. Inspired by these works, our system adapts the latent diffusion model \cite{rombach2022latentdiffusion} for the co-speech gesture generation task and achieves appealing results.

\subsection{Style Control for Human Motion}
A typical approach to style control for human motion involves specifying a motion clip as a reference and transferring the reference clip's style to the source motion. This task is also known as \emph{style transfer}. Early works in motion style transfer integrate traditional machine learning techniques with manually defined features to infer motion styles \cite{hsu2005motion_style_translation,ma2010motion_style_transfer,xia2015realtime_motion_style_transfer,yumer2016spectral_motion_style_transfer}. Recently, deep learning-based methods have significantly enhanced motion quality. \citet{holden2016deepmotion} first propose a learning framework enabling motion style control through optimization in the motion manifold space. \citet{du2019stylemotioncvae} improve transfer efficiency by training a conditional VAE. \citet{mason2018few-shot_motion_style_transfer} use few-shot learning to generate stylized locomotion. \citet{aberman2020adain} employ a temporally invariant adaptive instance normalization (AdaIN) layer for target style injection, eliminating the need for paired data during training. \citet{wen2021stylemotionflow} achieve unsupervised style transfer using a flow model. \citet{jang2022motionpuzzle} introduce a method capable of controlling styles for individual body parts.

Previous co-speech gesture synthesis systems with style control can be categorized based on whether or not they require style labels. For methods needing labeled data, early works can only learn an individual style for one generator \cite{levine2010gesturecontroller,neff2008videogesture,ginosar2019stylegesture}. \citet{ahuja2022lowresource} propose a strategy that efficiently adapts the source generator to another speaker style using low-resource data. Some works learn a speaker style embedding space with labeled speaker-motion data, enabling gesture style control by sampling from this space \cite{ahuja2020stylegesture,yoon2020trimodalgesture,bhattacharya2021affectivegesture}. \citet{alexanderson2020stylegesture} aimat controlling fine-grained styles, such as gesturing speed and spatial scope, using preprocessed control signal-motion data. Their later work \cite{alexanderson2022diffusiongesture} utilizes a diffusion model for audio-driven motion synthesis, achieving label-based style control by training the model on labeled data. For methods not requiring style labels, \citet{habibie2022motionmatching} propose a motion matching framework to achieve flexible style control. Other studies achieve arbitrary style control by imitating an example given as a video \cite{liu2022hierarchicalgesture} or a motion clip \cite{ghorbani2022zeroeggs,ye2022styleflowgesture,kuriyama2022tokenizedgestures}.  In this work, we utilize a CLIP-based encoder to extract a style embedding from an arbitrary text prompt and incorporate it into the generator via an AdaIN layer, guiding the synthesis of stylized gestures. Our system supports fine-grained multimodal style prompts as opposed to label-based style control. It employs a self-supervised learning scheme and eliminates the need for labeled data. Additionally, we use an autoregressive model rather than a parallel model, making it potentially suitable for real-time applications.
\section{Method Overview}
\label{sec:overview}
\OURS is an unconditional generative model for implicit neural fields encoded by MLPs. 
We operate directly on MLP weights, enabling generation of new neural implicit fields characterized by synthesized MLP parameters. 
Our training paradigm encompasses a two-phase approach, as shown in Figure~\ref{fig:overview}. 

In the first MLP overfitting step, detailed in Section~\ref{sec:overfit}, we optimize a collection of MLPs such that each MLP represents a faithful neural occupancy field of a data sample (e.g., a 3D shape) from the training set.
This enables highly-accurate shape fitting due to the representation power of the neural fields.
The optimized MLP weights are flattened into 1D vectors and passed to a downstream diffusion process as ground truth signals. 

In the second step, detailed in Section~\ref{sec:diffusion}, the aforementioned optimized MLP weights are passed into a diffusion network for training. This diffusion network is domain-agnostic without any assumptions or prior knowledge on the dimensionality of the underlying signal, since its input is a set of flattened MLP weight vectors. After training is completed, new MLP weights, which correspond to a valid neural implicit field, can be synthesized through the reverse diffusion process on a randomly sampled noise signal. 

For 3D and 4D generation, the underlying meshes can be further extracted and visualized with Marching cubes~\cite{lorensen1987marching}. 
\section{Per-Sample MLP Overfitting}
\label{sec:overfit}
In this first step, we optimize MLPs for each data sample from the training set $\left\{S_i, i=1, \dots, N\right\}$ and save their optimized network weights.
Specifically, for a train sample $S_i$, the surface at iso-level $\tau$ is represented as
\begin{align}
\{ &\mathbf{x}\in\mathbb{R}^n,\quad f(\mathbf{x},\theta_i) = \tau \},
\end{align}
where $f$ is an MLP parametrized by $\theta_i \in \mathbb{R}^h$ (\ie, weights and biases in Fig.~\ref{fig:overview}) and $\mathbf{x}$ represents a spatial location. The goal is to minimize the binary cross entropy loss function
\begin{align}
L = \textrm{BCE}(f(\mathbf{x}, \theta_i), o^\textrm{gt}_i(\mathbf{x})),
\end{align}
where $o^\textrm{gt}_i(\mathbf{x})$ is the ground truth occupancy of $\mathbf{x}$ with respect to $S_i$.
Here, we exploit the representation power of neural fields, which can model high-dimensional surfaces with high accuracy.
As these optimized MLPs serve as ground truth for the following diffusion processes, their ability to model high-fidelity shapes is crucial.


Unlike prior methods, we do not require any auto-encoding networks~\cite{bagautdinov2018modeling, bloesch2018codeslam} nor auto-decoding networks (\eg, DeepSDF~\cite{chou2022diffusionsdf}), which typically share the same network parameters across an entire dataset. In contrast, although we use the same MLP architecture for different samples in the training dataset, one set of MLP weights is optimized specifically for each data sample. In other words, there is no parameter sharing, and this per-sample optimization attains the best fidelity possible for the implicit neural field representation.


\paragraph{MLP Architecture and Training}
Our MLP architecture is a standard multi-layer fully-connected network with ReLU activation functions and an input positional encoding~\cite{mildenhall2021nerf}. 
We use 3 hidden layers with 128 neurons each, finally outputting a scalar occupancy value. 
The same MLP architecture is shared across both 3D shape and 4D animation experiments, where each MLP encodes an occupancy field per-train sample. 
This enables dimension-agnostic paradigm for encoding of various data signals, in our case 3D and 4D shapes, where only the positional encoding is adapted for various-dimensional inputs.

To optimize a set of MLP weights and biases for an input 3D shape, we sample points randomly both inside and outside of the 3D surface. 
We normalize all train instances to $[-0.5, 0.5]^3$, and randomly sample $100k$ points within the space. 
To effectively characterize fine-scale surface detail, we further sample $100k$ points near the surface of the mesh. 
Both sets of points are combined and tested for inside/outside occupancy using generalized winding numbers \cite{barill2018fast}; these occupancies are used to supervise the overfitting process. 
We optimize each MLP with a mini-batch size of $2048$ points, trained with a BCE loss for 800 epochs until convergence, which takes $\approx 6$ minutes per shape.

MLP overfitting to 4D shapes is performed analogously to 3D. 
For each temporal frame, we sample $200k$ points and their occupancies, following the 3D shape sampling. 
The sampling process is repeated for each frame of an animation sequence. 
We optimize one set of MLP weights and biases for each animation sequence to represent each 4D shape.


\paragraph{Weight Initialization}
In order to encourage a smooth diffusion process over the set of optimized MLP weights, we guide the MLP optimization process with consistent weight initialization.
That is, we initially optimize one set of MLP weights and biases $\theta_1$ to represent the first train sample $S_1$, and use the optimized weights of $\theta_1$ to initialize optimization of the rest of the MLPs $\{\theta_2,...,\theta_N\}$.
\section{MLP Weight-Space Diffusion}
\label{sec:diffusion}

We then model the weight space of our optimized MLPs through a diffusion process.
We consider each set of optimized MLP weights and biases $\{\theta_i\}$ as a flattened 1D vector.
We use a transformer architecture $\mathcal{T}$, following \cite{peebles2022learning}, for our denoising network. 
As transformers have been shown to elegantly handle long vectors in the language domain, we find it to be a suitable choice for modeling the MLP weight space.
$\mathcal{T}$ predicts the denoised MLP weights directly, rather than the noise.
% 
Our $h$-dimensional vectors $\{\theta_i\}$ of MLP weights and biases are input to $\mathcal{T}$.
Each $\theta_i$ is divided into 8 tokens by MLP layers, to be encoded by $\mathcal{T}$.

Modeling the MLP weights as a 1D vector for diffusion enables a general formulation for modeling neural fields, as the MLP weights are agnostic to varying-dimensional data.
This makes \OURS{} flexible to a variety of neural field representations; in particular, we observe the neural field ability to compactly represent high-dimensional shape data and demonstrate  generative modeling of MLPs representing 3D and 4D shapes, respectively.

During diffusion modeling, we apply standard gaussian noise $t$ times to each vector $\theta$. 
The noisy vector, along with the sinusoidal embedding of $t$, are then input to a linear projection. The projections are then summed up with a learnable positional encoding vector.
Then, the transformer outputs denoised tokens that we pass through a final output projection to produce the predicted denoised MLP weights $w^*$.
We train with a Mean Squared Error (MSE) loss between the denoised weights $\theta^*$ and the input weights $\theta$.

We illustrate the denoising process for MLPs representing 3D shapes in Figure~\ref{fig:denoising_vis}. 
Noisy MLPs correspond to invalid shapes, which are denoised to MLPs that represent valid 3D surfaces. We employ Denoising Diffusion Implicit Models (DDIM) \cite{song2020denoising} to sample new MLPs from the diffusion process.


\begin{figure*}[hbtp]
    \includegraphics[width=\linewidth]{images/denoising.pdf}
    \caption{
    Denoising MLP parameters at various time steps, visualized with their corresponding shapes, from random noise (left) to fully denoised (right). The image shows the gradual change from 0 denoising steps, which are generated from random MLP weights, up to 500 steps corresponding to fully-denoised shape. More specifically, we leverage the DDIM \cite{song2020denoising} sampling strategy. Interestingly, noisy MLP weights do not necessarily correspond to a valid 3D shape; however, the iterative application of the denoising operator eventually converges to a quality output shape.  
    }
    \label{fig:denoising_vis}
\end{figure*}

\paragraph{Implementation Details}
Our 3-layer 128-dim MLPs  contain $\approx 36$k parameters, which are flattened and tokenized for diffusion.
We use an  AdamW~\cite{loshchilov2017decoupled} optimizer with batch size $32$ and initial   learning rate of $2e^{-4}$, which is reduced by $20$\% every $200$ epochs.
We train for $\approx 4000$ epochs until convergence, which takes $\approx 4$ days on a single A6000 GPU.


\section{Results}
\label{sec:results}


\subsection{Datasets}

For 3D shape generation, we use the car, chair, and airplane categories of the ShapeNet~\cite{chang2015shapenet} dataset. The car, chair and airplane categories have 3533, 6778, and 4045 shapes respectively. For the 4D shape generation task, we use 16-frame animal animation sequences from the  DeformingThings4D~\cite{li20214dcomplete} dataset, comprising  1772 sequences. 
For both datasets, we split the data into non-overlapping partitions, including training (80\%), validation (5\%) and testing (15\%) subsets. 



\subsection{Voxel-based Diffusion Baseline}
\label{sec:vox-baseline}
While several existing methods tackle 3D shape generation, unconditional 4D shape generation remains underexplored.
Thus, in addition to existing 3D baselines, we introduce a voxel-based diffusion model as a baseline for both 3D and 4D shape generation.

3D shapes are represented as dense occupancy grids, and a 3D UNet denoising network is applied on the 3D voxel grids.
4D shapes are represented similarly, with each frame of an animation sequence voxelized to a 3D occupancy grid, producing a 4D occupancy grid representing the full sequence.
We use the same 3D UNet as a denoising network to synthesize 4D animation, as a 4D UNet became computationally intractable.

For our experiments, we use a voxel resolution of $24^3$ for 3D shapes and $16\times 24^3$ for 4D shapes (the maximum spatial resolution such that 4D grids could be tractably trained).

\subsection{Evaluation Metrics}

Evaluation of unconditional synthesis of 3D and 4D shapes can be challenging due to lack of direct correspondence to ground truth data.
We thus follow prior works \cite{zeng2022lion, luo2021diffusion, zhou20213d} in evaluating  Minimum Matching Distance (MMD), Coverage (COV), and 1-Nearest-Neighbor Accuracy (1-NNA). For MMD, lower is better; for COV, higher is better; for 1-NNA, 50\% is the optimal.
\begin{align*}
\text{MMD}(S_g, S_r) &= \frac{1}{\vert S_r \vert} \sum_{Y \in S_r} \min_{X \in S_g} D(X, Y), \\
\text{COV}(S_g, S_r) &= \frac{\vert \{ \argmin_{Y \in S_r} D(X, Y) \vert X \in S_g \} \vert}{\vert S_r \vert}, \\
\text{1-NNA}(S_g, S_r) &= \frac{\sum_{X \in S_g} \Indicator[N_X \in S_g] + \sum_{Y \in S_r} \Indicator[N_Y \in S_r] }{\vert S_g \vert + \vert S_r \vert},
\end{align*}
where in the 1-NNA metric $N_X$ is a point cloud that is closest to $X$ in both generated and reference dataset, i.e., 
$$N_X = \argmin_{K \in S_r \cup S_g} D(X, K)$$

We use a Chamfer Distance (CD) distance measure $D(X,Y)$ for computing these metrics in 3D, and report CD values  multiplied by a constant $10^2$. 
To evaluate 4D shapes, we extend to the temporal dimension for $T$ frames:
$$D(X, Y) = \frac{1}{T}\sum_{t=0}^{T - 1} CD(X[t], Y[t]).$$
We note that the MMD metric has been discussed to be unreliable as a measure of generation quality~\cite{zeng2022lion}, and thus also consider perceptual metric for 3D shape generation.
In particular, we follow \cite{zhang20233dshape2vecset} and compute a Frechet Pointnet++ Distance~\cite{qi2017pointnet++} (FPD), analogous to the commonly used Frechet Inception Distance (FID score)~\cite{heusel2017gans} in the image domain. 
FPD instead uses a  3D Pointnet++ network (trained on the ModelNet~\cite{wu20153d}) for feature extraction from generated shapes. For FPD, lower scores are better.

For evaluation, each synthesized and ground truth shape is normalized by its mean and standard deviation on a per-shape basis.
To evaluate point-based measures, we sample 2048 points randomly from all baseline outputs; for our approach and the voxel baseline, points are sampled from the extracted mesh surface, and for point cloud baselines, points are sampled directly from the synthesized outputs. 
\section{RESULTS}

\begin{table*}[t]
\caption{Trajectories validated on the Valkyrie humanoid in simulation (all) and hardware (e, f).}

% \caption{Trajectories validated on the Valkyrie humanoid. All trajectories are validated in a physics simulation, (e) and (f) are additionally validated on hardware.}
\centering
\begin{tabular}{C{2em} C{15em} C{6em} C{4em} C{8em} C{7em} C{7em} C{7em}} 
 \hline
  & Description & Number of key frames & Duration (s) & Number of unique non-contact 6-dof anchors & Number of unique contact 6-dof anchors & Number of unique 1-dof anchors  & Number of unique CoM anchors\\ [0.5ex] 
 \hline
 (a) & Standing up from lying down on flat ground & 22 & 57.5 & 22 & 18 & 53 & 16  \\ 
 (b) & Stepping over a 45cm tall barrier with handholds & 18 & 59 & 14 & 38 & 38 & 17 \\
 (c) & Climbing up and standing on an 80cm tall ledge & 24 & 61.5 & 16 & 27 & 32 & 20 \\
 (d) & Rolling over from facing down to facing up & 6 & 19.5 & 3 & 7 & 45 & 0 \\
 (e) & Reaching forward and bracing against a wall to extend range of motion & 5 & 25 & 6 & 9 & 1 & 3 \\ 
 (f) & Crawling to kneeling with flat handholds & 21 & 84 & 24 & 27 & 47 & 18  \\ 
 \hline
\end{tabular}
\label{table:simulation_summary}
\end{table*}

\begin{figure*}
    \centering
    \includegraphics[width=0.95\textwidth]{Figures/DemoFigures3.PNG}
    \caption{Operator view while generating trajectories.}
    \label{fig:demo_trajectories}
\end{figure*}

We tested our framework by generating motions for a variety of multi-contact scenarios. Table \ref{table:simulation_summary} contains some of these scenarios along with key frame statistics. All trajectories were validated in simulation and trajectories (e) and (f) were validated on hardware. Key frame transitions have a default value of 2s for simulation and 4s for hardware but the operator can override this value to both shorten or extend transitions. We selected contact-rich scenarios to generate motions that are difficult to plan autonomously or through standard teleoperation. Many of the robot's limbs are used for contact, including feet (all), arms (all), knees (a, c, d, f), and chest (c, d). Additionally, many contact geometries are included. Planar contact occurs when the foot is in full contact with the environment. Point contacts are also common when the arms or knees are in contact with the environment, since these are modelled using curves meshes. Line contacts are also used when one edge of the foot is in contact, such as both feet in Fig. \ref{fig:demo_trajectories}(c). We find that the operator's use of anchors varies significantly based on the scenario. For example, when rolling over on flat ground (scenario d) the robot is often in a position where the CoM cannot be directly controlled and was not used as an anchor. In contrast, climbing onto a ledge (scenario c) requires careful positioning of the CoM while climbing and therefore is used in almost every key frame.

\subsection{CoM Constraint Regions}

To validate effectiveness of the CoM constraint region (Sec. \ref{sec:contact_mode}) for the generated motions, we compare it to a baseline flat-ground constraint. Figure \ref{fig:com_prox} shows this comparison performed for scenarios (e) and (f). The ``flat ground'' model computes the constraint region as the convex hull of the robot's contact points.  The ``multi-contact'' constraint region is computed using the friction- and actuation-aware model. The plotted quantity is the distance of the CoM to the nearest constraint edge of both regions. Both scenarios have key frames with substantial (multiple centimeter) difference in stability margin. Although the multi-contact constraint region is generally more restrictive, scenario (e) key frame 2 demonstrates this is not always the case. This key frame corresponds to a braced reaching motion, shown in Figure \ref{fig:hardware_demo} (left). In this situation, using the multi-contact constraint region enables a higher range of motion than would be possible if using the flat-ground model. Conversely, the multi-contact region is very restrictive for configurations in scenario (f) that require support from the arms. In scenario (f) key frame 7 (Fig. \ref{fig:hardware_demo}), the robot places significant weight on the right arm while lifting the left arm. The actuation limits of the right arm are reflected by the multi-contact constraint being 5cm higher than the flat ground model.

\begin{figure}
    \centering
    \includegraphics[width=0.8\columnwidth]{Figures/CoMStabilityMargins.png}
    \caption{CoM stability margins for scenarios (e) and (f).}
    \label{fig:com_prox}
\end{figure}

\begin{figure}
    \centering
    \includegraphics[width=\columnwidth]{Figures/hardware_demo.PNG}
    \caption{Valkyrie executing two multi-contact motions: (left) bracing against a wall with the right arm and reaching forward with the left arm and (right) placing the arms on cinder blocks while maneuvering to a kneel.}
    \label{fig:hardware_demo}
\end{figure}


% Timing and frequency of ''undo'' button
% other features that were useful. highlight cases where the actuation-feasible region is useful

\subsection{User Interface Operation}

We find there are two primary reasons for a generated key frame to be infeasible: controller failures and inverse kinematics failures. Controller failures often occur because the CoM trajectory is unstable or there is an unexpected collision while moving to a key frame. Our approach assumes that key frames are sufficiently close such that validating subsequent key frames serves as a validation of the trajectory between them. However in practice this does not always hold, particularly when a limb is moving near the environment such as the foot moving over the barrier in Fig. \ref{fig:demo_trajectories}(b). This could be addressed by incorporating a motion preview similar such as \cite{johnson2017team, marion2018director}. Inverse kinematics failures occur for two reasons: getting ``stuck'' and going unstable. Since our IK solver is based on local optimization, it is susceptible to getting stuck in local minima. Often the operator can guide the robot out of the minimum when aware that the problem is occurring. Solver instability can occur when inconsistent objectives are requested with high weight, such as contacts, collisions and CoM positioning. Such cases require halting the IK and reverting to the last key frame. For this reason, the operator may prefer disabling CoM and collision constraints in the solver and using visual cues as indication of feasibility, which can mitigate solver instability.

% Discuss different strategies, i.e. no com anchor for rolling on the ground, etc

% Emphasize how often the operator would press the abort button
% etc.

% Discuss the time taken for each one. ~5min per key frame.
% Discuss when and why the undo button is used.


\section{Conclusion}
\label{sec:conclusion}

We consider top-down attention by explaining from an Analysis-by-Synthesis (AbS) view of vision. Starting from previous work on the functional equivalence between visual attention and sparse reconstruction, we show that AbS optimizes a similar sparse reconstruction objective but modulates it with a goal-directed top-down modulation, thus simulating top-down attention. We propose \model, a top-down modulated ViT model that variationally approximates AbS. We show that \model achieves controllable top-down attention and improves over baselines on V\&L tasks as well as image classification and robustness.

\smallskip
\noindent \textbf{Acknowledgements.}
This work was supported by the Bavarian State Ministry of Science and the Arts coordinated by the Bavarian Research Institute for Digital Transformation (bidt), the ERC Starting Grant Scan2CAD (804724), and the German Research Foundation (DFG) Research Unit “Learning and Simulation in Visual Computing.” Apple was not involved in the evaluations and implementation of the code.


{\small
\bibliographystyle{ieee_fullname}
\bibliography{egbib}
}

\clearpage

\section{Appendix}

\subsection{Additional Qualitative Results}
We provide additional unconditional generation results on 3D and 4D generation in Figure~\ref{fig:supplementary_3d} and Figure~\ref{fig:supplementary_4d}. We can generate diverse sets of shapes in both 3D and 4D settings. Resulting meshes are clean, smooth, and can be readily used in any 3D design software and game engines. Although we can output 16 frames for animation sequences, we only show 3 frames in Figure~\ref{fig:supplementary_4d}. Full animation sequences are available in our website and video.

\subsection{Implementation Details}
We use the diffusion and transformer architecture implementations of \cite{peebles2022learning}, which are modified versions of OpenAI and minGPT implementations, respectively. We have a pre-determined MLP structure which consists of 3 hidden layers, each with 256 neurons. To process the MLPs with a transformer architecture, we first flatten the MLP weights into a 1D vector. Additionally, as a way to establish correspondence between components within the 1D vector and the MLP layers (\eg, first \textit{n} values are weights of the first layer), each layer is considered as two tokens, one for its weights and the other for its biases. Hence, in total we have 8 tokens coming from weights and biases. Thanks to this decomposition, transformer may figure out interaction between weights and biases across different layers during training. We also have one additional token representing the sinusoidal embedding of the timestep value. During synthesis of new samples, we again decompose the generated 1D vector into each layer's weights and biases, and load them into the same MLP structure.

Our transformer has 2880 hidden size (i.e., the size of each token after linear projection), 12 layers, and 16 self-attention heads. We use 500 diffusion timesteps in our implementation and a linear noise scheduler ranging between $1e^{-4}$ and $2e^{-2}$. For the sampling strategy, we used DDIM~\cite{song2020denoising} and do not skip any timesteps during sampling. In addition, our denoising network directly predicts the denoised version, following \cite{peebles2022learning}. We observe that $\approx 15\%$ of airplane, $\approx 16\%$ of chair and $\approx 51\%$ of car shapes in our train split of ShapeNet~\cite{chang2015shapenet} contain major self-intersections in their original shape mesh faces, and so we exclude them from the training set for both our approach as well as for all baselines. 

We also apply de-duplication to generated 3D shape results. Note that this has not been applied to baseline approaches, since their quantitative performance degraded with de-duplication. We achieve de-duplication by sampling twice the necessary amount and removing the ones that are very close to each other.
 
\begin{figure}
 \centering
 \includegraphics[width=0.98\linewidth]{images/supplementary_3d.png}
 \caption{Additional unconditional 3D shape generation results.}
 \label{fig:supplementary_3d}
\end{figure}
\begin{figure}
 \centering
 \includegraphics[width=\linewidth]{images/supplementary_4d.png}
 \caption{Additional unconditional 4D animation sequence generation results. We refer to our website for animated shape results.}
 \label{fig:supplementary_4d}
\end{figure}


\end{document}