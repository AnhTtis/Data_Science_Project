\documentclass[lettersize,journal]{IEEEtran}
\usepackage{authblk}
\usepackage{amsmath,amsfonts}
\usepackage{algorithmic}
\usepackage{array}
\usepackage{cuted}
\usepackage{amsmath}
\usepackage{soul}
%\usepackage{xcolor}

\usepackage[usenames,dvipsnames]{xcolor}

\newcommand{\hlc}[2][yellow]{ {\sethlcolor{#1} \hl{#2}} }

% \usepackage[caption=false,font=normalsize,labelfont=sf,textfont=sf]{subfig}
\usepackage{textcomp}
\usepackage{stfloats}
\usepackage[longend,ruled,linesnumbered]{algorithm2e}
\usepackage[font=footnotesize,labelfont=bf]{caption}
\usepackage{subcaption}
\usepackage{mathtools}
\newcommand{\var}{\texttt}
\usepackage{stfloats}
\usepackage{url}
\usepackage{verbatim}
\usepackage{graphicx}
\hyphenation{op-tical net-works semi-conduc-tor IEEE-Xplore}
\def\BibTeX{{\rm B\kern-.05em{\sc i\kern-.025em b}\kern-.08em
    T\kern-.1667em\lower.7ex\hbox{E}\kern-.125emX}}
\usepackage{balance}
\usepackage{array,multirow,graphicx}
\newcolumntype{L}{>{\centering\arraybackslash}m{1.8cm}}


\title{Energy Efficiency Analysis}
\author{}

\begin{document}

\maketitle

\RestyleAlgo{ruled}
\
\vspace{-1cm}
\section{UAV Energy Model} This section describes the UAV flight energy model and means to measure communication efficiency (w.r.t UAV flight energy). UAV energy comprises communication energy and flight energy. Energy used by UAVs for communication is negligible compared to that used for flight \cite{rodrigues2022energy}. As our movement action space allows the UAV to hover, climb, descend and propel in level flight, it is imperative to ascertain how much energy is used for each flight profile. Propulsion energy depends on velocity, acceleration and angle of attack. For simplification, we assume constant velocity to ignore the additional energy used to accelerate the UAV. Most notations and values are summarized in table[] and have been taken from \cite{bramwell2001bramwell} whereas some values typical to a UAV used as base station have been referred from \cite{8663615}. 

Total power, $P$, consumed by a UAV can be written as 
\begin{equation} 
P = P_i + P_p + P_{par} + P_{c/d}, \tag{1}\end{equation}
where $P_i$ is the induced power (work done against induced drag on blades), $P_p$ is the blade profile power (to overcome viscous drag losses due to rotor motion), $P_{par}$ is the parasitic power (to overcome drag on the fuselage) and $P_{c/d}$ is additional power required to  change the potential energy during climb or descend. These can be illustrated as 
\begin{figure}[b]


     \centering
     \includegraphics[width=3.2in]{Efficiency vs trajectory comparison.pdf}
     \caption{NOMA-assisted UAV network framework for Disaster recovery}\label{Efficiency vs trajectory comparison}

\vspace{-5mm}
\end{figure}

\begin{align*}&
P_p = \underbrace {\frac {\rho \delta A s (r\Omega)^3}{8}}_{\text {while hovering}} \times \underbrace {\left(\frac {3V^2}{(r\Omega)^2} +1\right)}_{\text {move factor}}, \tag{2a}\label{2a}\\
&P_i = \underbrace {\frac {T^{3/2}}{2 \rho A}}_{\text {while hovering}} \times \underbrace {\left(-\frac{V^2}{2v_{i}^{2}} + \sqrt{\frac{V^4}{4v_{i}^{4}}+1}\right)}_{\text {move factor}}, \tag{2b}\label{2b}\\
&P_{par} = \frac {\rho s A d_0 V^3}{2} \text{   and   } P_{c/d} = \pm TV_{c/d} , \tag{2c}\label{2c}
\end{align*}


\noindent where $\rho$, $\delta$, $\Omega$, $A$, $s$ and $r$ represent the air density, profile drag co-efficient, blade angular velocity, rotor area, solidity and rotor radius respectively. The product $r\Omega$ gives the blade tip velocity. $V$, $T$, $v_i$ and $d_0$ represent the airspeed, Thrust, induced velocity and fuselage drag ratio respectively. Velocity in the vertical plane is denoted as $V_{c/d}$.



\begin{table}[h]
\caption{UAV Energy Simulation Parameters.}
\centering
\begin{tabular}
{|m{0.17\textwidth}|>{\centering} m{0.08\textwidth}|>{\centering\arraybackslash}m{0.1\textwidth}|}
\hline
% {|m{0.1\textwidth}|>{\centering} m{0.2\textwidth}|  > {\centering}  |}

    \textbf{Parameter} & \textbf{Notation} & \textbf{Quantity} \\
    \hline
    Aircraft weight (weight = Thrust, $T$, in const velocity flights) & $W$ & 20 N \\  
    \hline
    Rotor disc Area & $A$ ($\pi R^2$) & 0.503 $m^2$ \\  
    \hline
    Rotor disc radius & $r$ & 0.4 m \\  
    \hline
    Density of Air & $\rho$ & 1.225 Kg/m$^3$ \\  
    \hline
    Angular speed (blade) & $\Omega$ & 300 rad/s \\  
    \hline
    Blade tip speed & $ $r$\Omega$  & 120 m/s \\  
    \hline
    Rotor solidity & $s$ & 0.05 \\  
    \hline
    Profile drag co-efficient & $\delta $ & 0.012 \\  
    \hline
    UAV velocity & $V$ & 5 m/s \\  
    \hline
    Induced velocity of rotor & $v_i$ = $\sqrt{\frac{W}{2\rho A}}$ & 4.028 m/s \\  
    \hline
    Fuselage drag ratio & $d_0$ & 0.6 \\  
   
    \hline

\end{tabular}
\label{Simulation parameters}

\end{table}

From \eqref{2c}, it is evident that $P_{par}$ and $P_{c/d}$ are 0 when hovering. Moreover, \eqref{2a} and \eqref{2b} suggest that $P_p$ increases quadratically with $V$ and $P_i$ decreases with $V$. The fact that $v_i$ is lower in hovering flight as compared to climb profiles or level flight \cite{bramwell2001bramwell} has been incorporated in the power calculations.

%one compilaiton warning in the 2nd last sentence of the upper paragraph. delete the last two sentences to find out

The total energy consumed, $P_{t}$, by all UAVs for each timestep, $t$, can be written as 

\begin{equation} 
\vspace*{-1mm}
P_{t} = \sum _{u=1}^{U} P_t^u, \tag{3}\end{equation}
where $P_t^u$ is the total energy expended by UAV $u$ for each time step, $t$. To calculate the system's energy efficiency for each episode, the total system throughput for the entire episode is divided by the total energy consumed by all UAVs. Therefore the mean energy efficiency, represented by $\gamma$, for each episode is given by 

\begin{equation} 
\gamma = \frac{\sum _{t=0}^{T}\mathcal{R}(t)}{\sum _{t=0}^{T}{P_t}}, \tag{4}\end{equation}
where the numerator provides the total system throughput for each episode and the denominator denotes the total power consumed by all UAVs during that episode.

\section{Simulation Results} 

Fig. \textbf{\ref{Efficiency vs trajectory comparison}} shows how the energy efficiency of various models varies with training episodes. Energy efficiency in case of SDQN and Mutual learning algorithms increases as the training matures because of two reasons - 1) the total system throughput increases as training progresses and 2) the trajectory gets increasingly smooth as compared to the initial random trajectory. As a result, the UAVs avoid unnecessary vacillations between climbing, descending and hovering as is evident in Fig. \textbf{6(b)}. Energy efficiency of a fully trained SDQN model is 13\% higher than that of Mutual learning algorithm. However, for the untrained models, Circular trajectory outperformed both SDQN and mutual learning algorithms because 1) UAVs following circular motion cover a larger area by default and yields a better throughput and 2) Circular UAV trajectory only comprises level flight profile and consumes lower energy than the climb and, even, hovering flight profile included in other models \cite{bramwell2001bramwell}.



\balance
\begin{thebibliography}{10}

\bibitem{rodrigues2022energy}Rodrigues, H., Coelho, A., Ricardo, M. \& Campos, R. Energy-aware relay positioning in flying networks. {\em International Journal Of Communication Systems}. \textbf{35}, e5233 (2022)

\bibitem{bramwell2001bramwell}Bramwell, A., Balmford, D. \& Done, G. Bramwell's helicopter dynamics. (Elsevier,2001)

\bibitem{8663615}
Y.~Zeng, J.~Xu, and R.~Zhang, ``Energy minimization for wireless communication
  with rotary-wing uav,'' {\em IEEE Transactions on Wireless Communications},
  vol.~18, no.~4, pp.~2329--2345, 2019.


\end{thebibliography}

\end{document}




