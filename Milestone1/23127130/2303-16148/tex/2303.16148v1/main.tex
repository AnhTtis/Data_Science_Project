% Template for producing ESWA-format journal articles using LaTeX    
% Written by Miha Ravber                
% Programming methodologies laboratory                    
% Faculty of Electrical Engineering and Computer Science 
% University of Maribor                              
% Koroška cesta 46, 2000 Maribor                                       
% E-mail: miha.ravber@um.si                           
% WWW: https://lpm.feri.um.si/en/members/ravber/    
% Created: November 20, 2020 by Miha Ravber                                          
% Modified: November 20, 2020 by Miha Ravber                     
% Use at your own risk :) 
% Please submit your issues on the github page: https://github.com/Ravby/eswa-template


\documentclass[review]{elsarticle}

%%%%%%%%%%%%%%%%%%%%%%%%%%%%%%%%
\newcommand{\quotes}[1]{``#1''}
\usepackage{amsmath}
\usepackage[export]{adjustbox}
\usepackage{caption}
\captionsetup[table]{skip=5pt}
\usepackage{multicol,multirow}
\usepackage{xcolor,colortbl}
\usepackage{subcaption}
\usepackage{graphicx}

%%%%%%%%%%%%%%%%%%%%%%%%%%%%%%%


\graphicspath{ {./figures/} }


\usepackage{hyperref}



\usepackage{float}
\usepackage{verbatim} %comments
\usepackage{apalike}
\restylefloat{figure}
\restylefloat{table}

% \journal{Expert Systems with Applications}

%% For ESWA journal you need to use APA style

\bibliographystyle{model5-names}\biboptions{authoryear}

\begin{document}
\begin{frontmatter}


% \begin{titlepage}
% \begin{center}
% \vspace*{1cm}

% \textbf{ \large Modelling Determinants of Cryptocurrency Prices: A Bayesian Network Approach}

% \vspace{1.5cm}

% % Author names and affiliations
% Rasoul Amirzadeh$^a$ (ramirzadeh@deakin.edu.au), Asef Nazari$^b$ (asef.nazari@deakin.edu.au), Dhananjay Thiruvady$^c$ (dhananjay.thiruvady@deakin.edu.au),  Mong Shan Ee$^d$ (mong.e@deakin.edu.au) \\

% \hspace{10pt}

% \begin{flushleft}
% \small  
% $^a$ School of  Information Technology, Deakin University, Geelong Waurn Ponds Campus, 3216, Australia \\
% $^b$ School of  Information Technology, Deakin University, Geelong Waurn Ponds Campus, 3216, Australia\\
% $^c$ School of  Information Technology, Deakin University, Geelong Waurn Ponds Campus, 3216, Australia \\
% $^d$ Business School, Deakin University, Burwood, Victoria, 3125, Australia

% \begin{comment}
% Clearly indicate who will handle correspondence at all stages of refereeing and publication, also post-publication. Ensure that phone numbers (with country and area code) are provided in addition to the e-mail address and the complete postal address. Contact details must be kept up to date by the corresponding author.
% \end{comment}

% \vspace{1cm}
% \textbf{Corresponding Author:} \\
% Rasoul Amirzadeh \\
% School of  Information Technology, Deakin University, Geelong Waurn Ponds Campus, 3216, Australia \\
% Tel: (+61) 413295108 \\
% Email: ramirzadeh@deakin.edu.au

% \end{flushleft}        
% \end{center}
% \end{titlepage}

\title{Modelling Determinants of Cryptocurrency Prices: A Bayesian Network Approach}

\author[label1]{Rasoul Amirzadeh \corref{cor1}}
\ead{ramirzadeh@deakin.edu.au}

\author[label1]{Asef Nazari}
\ead{asef.nazari@deakin.edu.au}

\author[label1]{Dhananjay Thiruvady}
\ead{dhananjay.thiruvady@deakin.edu.au}

\author[label4]{Mong Shan Ee}
\ead{mong.e@deakin.edu.au}

\cortext[cor1]{Corresponding author.}
\address[label1]{School of  Information Technology, Deakin University, Geelong Waurn Ponds Campus, 3216, Australia}
% \address[label2]{School of Information Technology, Deakin University, Victoria, 3220, Australia}
% \address[label3]{School of  Information Technology, Deakin University, Burwood, Victoria, 3125, Australia}
\address[label4]{Business School, Deakin University, Burwood, Victoria, 3125, Australia}

\begin{abstract}
The growth of market capitalisation and the number of altcoins (cryptocurrencies other than Bitcoin) provide investment opportunities and complicate the prediction of their price movements. A significant challenge in this volatile and relatively immature market is the problem of predicting cryptocurrency prices which needs to identify the factors influencing these prices. The focus of this study is to investigate the factors influencing altcoin prices, and these factors have been investigated from a causal analysis perspective using Bayesian networks. In particular, studying the nature of interactions between five leading altcoins, traditional financial assets including gold, oil, and S\&P 500, and social media is the research question. To provide an answer to the question, we create causal networks which are built from the historic price data of five traditional financial assets, social media data, and price data of altcoins. The ensuing networks are used for causal reasoning and diagnosis, and the results indicate that social media (in particular Twitter data in this study) is the most significant influencing factor of the prices of altcoins. Furthermore, it is not possible to generalise the coins' reactions against the changes in the factors. Consequently, the coins need to be studied separately for a particular price movement investigation.
\end{abstract}

\begin{keyword}
Cryptocurrencies \sep Altcoins \sep Bayesian networks \sep Twitter \sep Causal inference \sep Discretization
\end{keyword}

\end{frontmatter}

\section{Introduction}
The cryptocurrency market has quickly become an important part of global financial markets \citep{gajardo2018does}. The market capitalisation of cryptocurrencies was approximately 2.2 trillion dollars at the beginning of 2022,\footnote{www.coinmarketcap.com} leading to these markets being an excellent opportunity for potential investors. Compared to gold's market capitalisation of 10 trillion dollars and Apple's market capitalisation of 2.639 trillion dollars,\footnote{www.8marketcap.com} the cryptocurrency market has had significant growth, especially since it is relatively new within financial markets.\footnote{Initiated by Nakamoto's Bitcoin paper \quotes{Bitcoin A Peer-to-Peer Electronic Cash System} \citep{nakamoto2008peer}.
}

Despite Bitcoin being the centre of attention with the highest market capitalisation among cryptocurrencies, studies such as \citet{ji2019dynamic} are pointing to Bitcoin losing its dominance within the evolving cryptocurrency market. For example, Figure \ref{fig:capital_comparison} shows a snapshot of major cryptocurrencies by the total market capitalisation percentage (obtained from www.coinmarketcap.com). As can be seen, there is a significant decrease in the market capitalisation of Bitcoin in proportion to the market capitalisation of altcoins (cryptocurrencies other than Bitcoin). The total market capitalisation of altcoins  has increased from around 8\% in January 2017 to 67\% in January 2022. Moreover, there were around 50 different cryptocurrencies by the end of 2013, whereas there are over 17,000 altcoins in the market today. Some altcoins are created to overcome slow and complicated transaction deficiencies in Bitcoin or to introduce new features including security, privacy, and stability. Besides this evidence, new tradeable entities such as non-fungible tokens (NFTs) and metaverse tokens have attracted a significant number of new investors. In particular, altcoins are mainly used as core currencies in large marketplaces for NFT, such as OpenSea, which emphasises the growing importance of altcoins. Hence, the change in the landscape of cryptocurrencies has led  investors to explore alternatives to Bitcoin, with substantial opportunities and great potential \citep{ji2019dynamic}. 


\begin{figure*}[t]
  \centering
  % \includegraphics[width=.99\linewidth]
  \includegraphics[width=\linewidth,height=\textheight,keepaspectratio]{Figures/capital_comparison.jpg}
  \caption{The percentage of market capitalisation of Bitcoin and top performing altcoins.}
  \label{fig:capital_comparison}
\end{figure*}


Although there have been profitable investment opportunities in the cryptocurrency market, there were significant challenges impacting their profitability. Not only do cryptocurrencies face the same challenges of traditional financial assets such as accurate price prediction but also there are new challenges that ensue due to their particular ecosystem. These include the mining difficulty, security of wallets and cryptocurrency exchanges, blockchain-related energy consumption, the lack of international acceptance and legislation, etc. \citep{sabry2020cryptocurrencies}. 

An accurate price prediction mechanism is vital for a successful trading strategy in financial markets, however, developing such mechanisms is not straightforward. Although the financial asset price prediction task has attracted the attention of researchers, it is still a challenging problem because of the noisy and nonlinear attributes of financial data \citep{wang2021stock}. Furthermore, the ever-growing number of publications on designing new prediction algorithms~\citep{amirzadeh2022applying} shows the complexity of the prediction task. Besides, many factors can influence the price of a cryptocurrency, and investors need to analyse these factors to properly predict cryptocurrency prices. These factors could range from social media, trends in search engines, and declarations by political leaders to the interactions among cryptocurrencies, blockchain data, and classic financial asset prices~\citep{sabry2020cryptocurrencies}. Consequently, understanding these factors that influence cryptocurrency prices is the first step in developing an accurate price prediction algorithm. 

To tackle the challenge of price prediction, various studies \citep{ozer2022automated, mueller2020cryptocurrency, smuts2019drives} have attempted to explain the price movements of cryptocurrencies due to changes in the economic factors \citep{teker2019determinants}. These studies either focus on cryptocurrencies as an isolated market or in combination with other financial markets. They mainly concentrated on diversification opportunities  \citep{anyfantaki2018diversification}, the hedging capability \citep{wong2018cryptocurrency}, safe havens \citep{corbet2020any}, and financial contagion in cryptocurrencies \citep{ferreira2019contagion}. However, these studies are dominated by Bitcoin-focused research considering aspects such as market efficiency \citep{kurihara2017market}, price dynamics and bubbles \citep{geuder2019cryptocurrencies}, and portfolio management \citep{guesmi2019portfolio}. Further, studies bifurcate towards sentiment analyses \citep{inamdar2019predicting} or macroeconomic and financial indices \citep{poyser2019exploring} as the main price drivers.  Relatively few studies have examined altcoins \citep{wang2019cryptocurrency}. Therefore, investigating altcoins and the factors driving their prices can provide investors and fund managers with helpful insights  on investing in the cryptocurrency market, such as risk and portfolio management strategies \citep{amirzadeh2022applying}.  In particular, detecting the factors driving altcoin prices and gaining a better understanding of price movements are the first steps towards tackling challenges in the cryptocurrency market, including price prediction and portfolio management.

To study the driving factors of altcoin prices, we have adapted the framework of \citet{ciaian2016economics} for Bitcoin price formation and the study by \citep{poyser2019exploring} to categorise those factors represented in Figure~\ref{fig:price_drivers}. In this framework, price factors are divided into external and internal categories. The internal factors are directly derived from information on a decentralised peer-to-peer network (or blockchain) of cryptocurrencies, and they include supply, demand, transaction volume, mining difficulty, and so forth  \citep{iccelliouglu2019investigation}. On the other hand, external factors, which are the focus of this study, consist of macro-financial drivers and sentiment analysis of Twitter data in terms of  attractiveness and adoption. As parts of behavioural finance, these concepts are defined as the investment attractiveness of cryptocurrencies and investors' intention to use new financial products \citep{ricciardi2000behavioral}. It is important to note that the categorisation of the internal and external factors does not imply independence between them, as in a real-world situation they can interact dynamically. %%in a dynamic way they interact with each other. 

\begin{figure*}[h]
  \centering
  % \includegraphics[scale=0.6]
  \includegraphics[width=0.8\linewidth,height=0.8\textheight,keepaspectratio]
  {Figures/price_drivers.jpg}
  \caption{The categorisation of internal and external propelling factors driving cryptocurrency prices adopted from  \protect\citep{poyser2019exploring}.}
  \label{fig:price_drivers}
\end{figure*}

In order to investigate the macro-financial drivers of altcoin prices, we choose several assets from diverse financial classes, such as currencies, commodities, and stock indices. In choosing assets of potential influence, we have considered several similar studies in cryptocurrency literature including \citet{corbet2018exploring,charfeddine2020investigating, ji2018network}. We focus on the S\&P 500 index, an indicator of the performance of high-performing stocks in the US. It is considered a general indicator of the US economy and one of the most used indices to evaluate the strength of the stock markets \citep{balcilar2021dynamic}. We also focus on gold and oil as influential commodities in the world's economy. Gold's price is used as a general commodity index, and West Texas Intermediate (WTI) price is investigated as one of the major international crude oil price benchmarks \citep{jiang2021credit, zhu2016heterogeneity}. The effect of WTI is studied in the literature mainly due to the impact of energy requirements on mining cryptocurrencies \citep{hayes2017cryptocurrency, okorie2020crude}.
 Moreover, because major fiat currencies are the primary medium to trade altcoins in cryptocurrency exchanges, the US dollar index (USDX) is selected as another asset class. It measures the value of the US dollar relative to the value of six major currencies \citep{hasan2022exploring}. In addition, it indicates the appreciation and depreciation of the US dollar and allows for the monitoring of economic conditions of the world economy \citep{lee2010analysis}. Eventually, the MSCI World index is also included in our investigation. It is a market capitalisation weighted index of 1,546 companies all around the world and is considered one of the most widely known benchmarks for global markets to institutional investors and hedge funds \citep{miletic642013performance}. In summary, we study two commodities (gold and oil), two stock-related indices (MSCI and S\&P 500), and one currency index as a representative of major currencies (USDX) as potential micro-financial drivers of the altcoin price.

Behavioural finance, as a relatively new school of thought in finance, explains the psychological reasons behind investors' decision-making \citep{konigstorfer2020applications}. As opposed to standard (traditional)  finance, which is associated with the efficient market hypothesis,\footnote{The efficient market hypothesis is the consideration that a firm's share prices reflect all information regarding the business value \citep{rossi2018efficient}. In particular, based on the hypothesis, when new information is introduced to the market, it is immediately reflected in asset prices. Therefore, it is impossible to generate excess returns consistently \citep{malkiel2003efficient}.} behavioural finance attempts to understand and reason about the psychological patterns of buyers and sellers; this contains  emotional states involved and their degree of influence on the process of decision-making \citep{ricciardi2000behavioral}. Moreover, with the rapid growth of social media, more investors are influenced by social media in developing investment strategies \citep{gurdgiev2020herding,mai2018does}, where users who share their experiences and stories, influence others' perceptions and trading behaviours \citep{lund2018power}. As a result, and supported by narrative economics,\footnote{Narrative economics, introduced by Nobel winner Robert J. Shiller, studies the dynamics of popular narratives and stories and how people make economic decisions, such as investment in a volatile speculative asset, based on contagious stories \citep{shiller2017narrative}} these stories can potentially go viral and become contagious. Thus, the interest in analysing the relationship between financial markets, particularly cryptocurrencies, and investor attention has grown considerably recently \citep{smales2022investor, lin2021investor, zhu2021investor}, reflecting the importance of the impact of social media on investment activities.

As a representative of the social media platform, with around 330 million monthly active users, Twitter is one of the most widely used mediums. On this platform, there are ongoing public discussions on cryptocurrencies in various forms, such as hashtags, live spaces, and tweets. These conversations can be monitored for analysing cryptocurrency price movements based on public sentiment \citep{wolk2020advanced}. Hence, this study examines the relationship between the daily number of tweets associated with the altcoins and their intra-day prices as a measure of attractiveness and adoption.

In this paper, Bayesian networks (BNs) are applied as an efficient graphical probabilistic model to investigate the price drivers of altcoins. BNs provide a robust framework that is mathematically compatible with modelling uncertain, nonlinear, and complex systems, and can naturally deal with an increasing number of relevant features \citep{sevinc2020bayesian}. In particular, BNs are capable of pointing to the true root cause of factors influencing the price of altcoins in an explainable fashion. BNs have gained importance for modeling complex problems, and their applications are as diverse as evaluating water quality \citep{wepener2022application}, supply chain disruption during the recent pandemic \citep{hosseini2022multi},  risk assessment \citep{li2020risk}, and so forth. Yet, the number of studies regarding the application of BNs in cryptocurrency markets is relatively low. 
Therefore, this study focuses on factors that influence the price of cryptocurrencies by constructing BNs representing the interactions among those factors. These networks allow further investigation of the relationships between price determinants of cryptocurrencies from a causal analysis perspective.
Towards constructing BNs, we will assess several discretisation methods to convert continuous price data into price movement directions, as the first step in this study.

The contribution of our research can be summarised as follows.
\begin{itemize}
    \item Our study focuses on analysing the price drivers of altcoins despite the tendency of the literature towards Bitcoin in similar studies.
    \item This work simultaneously considers the impact of social media and micro-financial factors on the altcoin prices to provide a more realistic view of the cryptocurrency market.
    \item The research investigates the best-performing discretisation technique for constructing BNs in the context of cryptocurrencies.
    \item A Bayesian causal inference paradigm is used to detect the main influencing factors on the altcoin prices. Detecting the main influential factors on altcoin prices helps reduce the system's complexity in favour of proving a more explainable model to describe a complex system. In addition, inference and sensitivity analyses are performed using BNs to improve our understanding of cryptocurrency markets.   
\end{itemize}

The remainder of this paper is organised as follows. Section \ref{Background} provides a literature review on the studies of factors impacting cryptocurrency prices. An introduction to BNs is given in Section \ref{Material and Methods}. Section \ref{Experimental} outlines the approach for constructing BNs. Results are discussed in Section \ref{Results}, and Section \ref{Conclusions} conclude the paper with future research ideas.
\section{Related literature}  \label{Background}
The interactions between cryptocurrencies and traditional financial assets and their cross-asset dependencies are vastly studied from different points of view. In particular, the correlation between the prices of a set of cryptocurrencies and a set of conventional financial assets or the correlation between the prices of a set of cryptocurrencies among themselves have been investigated, as we cover in this section. The importance of correlational studies originates from the fact that the lack of correlation between two sets of financial entities would suggest diversification opportunities.  In rare cases, causal relationships are studied between the price of Bitcoin and a set of financial assets. Also, due to the failure in establishing a solid correlational relationship between cryptocurrencies and traditional financial assets, researchers examined the impact of non-financial factors such as social media on the price of cryptocurrencies.

To study the connectedness between a set of cryptocurrencies and traditional financial assets \citet{corbet2018exploring} examine the relationship between those assets  using the concept of spillovers. From an economic aspect, a spillover is the occurrence of an economic event due to another event in a seemingly unrelated domain \citep{lau2017return}. Their results show evidence of the relative isolation of those coins from the financial and economic assets. The isolation is not considered a negative attribute as it creates diversification opportunities. In another study, \citet{charfeddine2020investigating} investigate the dynamic relationship between Bitcoin and Ethereum with major financial assets and commodities including gold, oil, USD/YUAN, and S\&P 500. They conclude that the cross-correlation between cryptocurrencies and these assets is weak and is influenced by external shocks and events. To move beyond a correlational study, \citet{ji2018network} use a PC algorithm (a prototypical constraint-based algorithm for learning Bayesian networks) to uncover casual relationships and create directed acyclic graphs (DAGs) between Bitcoin and seven financial assets. Their results show that Bitcoin is an isolated market without being considerably influenced by other assets. However, there are still time-variant causal relationships between the markets, especially in periods of bear markets.

The correlational relationships among cryptocurrencies are the subject of several studies.  The cross-correlations among 119 cryptocurrencies are studied in \citep{stosic2018collective}. They find that the cross-correlation matrix shows a non-trivial hierarchical grouping of cryptocurrencies that is not supported by partial correlations. Their results also certify that the cryptocurrency market behaves differently from other financial markets. In addition, \citet{shi2020correlations} apply  the multivariate factor stochastic volatility model with the Bayesian estimation procedure to examine the dynamic correlations among six cryptocurrencies. Similar to \citet{stosic2018collective}, they find smaller groups of cryptocurrencies having similar price volatility. Furthermore, \citet{bouri2021return} study market integration of 12 cryptocurrencies using the dynamic equicorrelation model. They also conclude that trading volume and external uncertainties are two main determinants of market integration, and the connectedness among cryptocurrencies is time-varying.

The desire to establish a correlational relationship between cryptocurrencies and traditional financial assets leads the researchers to further investigate the price movements in both entities. At least from a portfolio management perspective, the correlation coefficient is an important metric in the construction of an efficient portfolio. The conditional correlation between cryptocurrencies, gold, and stock and bond indices is studied using a generalised dynamic conditional correlation model in a study by \citet{aslanidis2019analysis}. The results support the earlier conclusion that there is not a considerable correlation between cryptocurrencies and traditional financial assets. In addition, the correlation between cryptocurrencies themselves is positive and time-varying. In another study, \citet{giudici2021crypto} show that Bitcoin prices are not influenced by traditional financial assets. Furthermore, different cryptocurrencies are affected differently by the changes in the global stock markets and gold as investigated in \citep{malladi2021time}. They find that Bitcoin returns are independent of the stock market returns, but they are related to the returns of Ripple.

The lack of correlational relationships between cryptocurrencies and traditional financial assets forces researchers to find cryptocurrency price-changing factors outside of the financial markets. In particular, the impact of social media and social influencers on cryptocurrency price movements is investigated to obtain a better understanding of crypto assets. A study by \citet{lamon2017cryptocurrency} manages to determine the trend in cryptocurrency price using labelled daily news and social media data. Also, through sentiment analysis in  \citet{abraham2018cryptocurrency}, it is shown that the tweet sentiments are highly correlated with cryptocurrency price movements. In a  similar study, \citet{rouhani2019crypto} show that Ripple has the highest percentage of positive tweets (52\%), and Bitcoin receives the highest percentage of negative tweets (27\%) among all the cryptocurrencies by analysing around five million tweets. The predictive power of social media sentiment analysis is studied in \citet{kraaijeveld2020predictive} and \citet{kim2021dynamics}, and it is shown that Twitter sentiment has predictive power for price returns of Bitcoin, Bitcoin Cash, and Litecoin, and cryptocurrency markets are more responsive to positive social sentiment when it is in a downward trend.

%%%%%%%%%%%%%%%%%%%%%%%%%%%%%%%%%%%%%%%%%%%%%%%%%


% \section{Essential title page information}
% \label{title_page}

% \begin{enumerate}
%     \item \textbf{Title.} Concise and informative. Titles are often used in information-retrieval systems. Avoid abbreviations and formulae where possible.
%     \item \textbf{Author names and affiliations.} Where the family name may be ambiguous (e.g., a double name), please indicate this clearly. Present the authors' affiliation addresses (where the actual work was done) below the names. Indicate all affiliations with a lower-case superscript letter immediately after the author's name and in front of the appropriate address. Provide the full postal address of each affiliation, including the country name and, the e-mail address of each author.
%     \item \textbf{Corresponding author.} Clearly indicate who will handle correspondence at all stages of refereeing and publication, also post-publication. Ensure that phone numbers (with country and area code) are provided in addition to the e-mail address and the complete postal address. Contact details must be kept up to date by the corresponding author.
%     \item \textbf{Present/permanent address.} If an author has moved since the work described in the article was done, or was visiting at the time, a 'Present address' (or 'Permanent address') may be indicated as a footnote to that author's name. The address at which the author actually did the work must be retained as the main, affiliation address. Superscript Arabic numerals are used for such footnotes.
% \end{enumerate}

% % \section{Reference style and reference list}
% % \label{reference_style}

% % All paper submissions must completely comply with ESWA/ESWA X reference style and reference list (see details at
% % \href{https://www.elsevier.com/journals/expert-systems-with-applications/0957-4174/guide-for-authors}{https://www.elsevier.com/journals/\\expert-systems-with-applications/0957-4174/guide-for-authors}).

% % \subsection{Reference Style}
% % Citations in the text should follow the referencing style used by the American Psychological Association. You are referred to the Publication Manual of the American Psychological Association, Sixth Edition, ISBN 978-1-4338-0561-5.  APA's in-text citations require the author's last name and the year of publication. You should cite publications in the text, for example, (Smith, 2020).  However, you should not use [Smith, 2020].

% % \subsection{Reference List}
% % References should be arranged first alphabetically by the surname of the first author followed by initials of the author's given name, and then further sorted chronologically if necessary. More than one reference from the same author(s) in the same year must be identified by the letters 'a', 'b', 'c', etc., placed after the year of publication. For example, Van der Geer, J., Hanraads, J. A. J., \& Lupton, R. A. (2010). The art of writing a scientific article. Journal of Scientific Communications, 163, 51-59. https://doi.org/10.1016/j.Sc.2010.00372. There should be no [1], [2], [3], etc in your references list.


% % For textual citations use \textbackslash cite\{\}. Here is an example \citep{Feynman1963118,Dirac1953888}. For parenthetical citations use \textbackslash citep\{\}. Another example for citing \citep{Smith2012qr,Smith2013jd,art}: 


\section{An overview of Bayesian networks}
\label{Material and Methods}
Probabilistic graphical models (PGMs) use a graph, as a set of nodes and edges, to represent probability distributions of a complex probabilistic system \citep{larranaga2012review}. A graph allows visualising a system and the dependencies among its components. In addition, by inspecting the structure of the graph and using graph manipulation techniques, one can deal with the computational complexity of finding the best model mimicking the probabilistic system. Generally in graphical models for representing a probabilistic system, a node plays the role of a random variable (or a group of random variables), and edges represent the probabilistic relationships between those random variables. In this manner, the graph represents the joint probability distribution over all of the random variables. Also, based on the structure of the graph, the joint probability distribution can be decomposed into a product of factors such that the probability of each node only depends on the probability of its direct parents \citep{koller2009probabilistic}.

A BN is a member of PGMs with the property that its diagrammatic representation is a directed acyclic graph (DAG). The direction of an edge indicates the causal path from cause to effect, allowing casual inferential analysis. Theoretically, BNs are based on the Bayes theorem that describes the changes in the posterior probability in relation to the changes in the prior probability \citep{sidebotham2020most}. The nodes in a BN can represent both continuous and discrete random variables. However, the discrete case is more frequent, and continuous random variables are generally converted into discrete ones using a discretisation technique. For a discrete random variable (or node), there is a set of states. The probability of a random variable taking a particular state is defined as a conditional probability based on the states of its parents. For example, for a node representing the close price of a particular stock, which is a continuous random variable, the states may be defined as a binary choice between 'bullish' and 'bearish' or a choice between 'bullish', 'steady-state', and 'bearish' for a trinity case after a discretisation step. The joint probability distribution of a discrete node based on the probabilities of its parents' states is summarised in a conditional probability table (CPT).

BNs are widely utilised for solving problems in data mining and data exploration mainly by determining the causal relationships between variables and representing the dependencies in an explicit manner using DAGs. They have the capacity to incorporate expert knowledge along with empirical data to model uncertainties in modelling complicated real situations. BNs can also handle incomplete datasets and hidden variables and offer an efficient procedure for avoiding overfitting \citep{heckerman2008tutorial}. 

In BN terminology, for a node $v$ an ancestor node is a node from which a directed path reaches $v$  while a descendent node receives a directed path. Moreover, nodes with no descendent are referred to as leaf nodes in contrast to a root node that does not have ancestors \citep{yu2015modified}. The main idea behind the efficiency of BNs is that each variable is only influenced by its parents and ancestors. Formally, for a given set of random variables $\{X_1, X_2, \ldots, X_n\}$ the joint probability distribution $P(X_1, X_2, \ldots, X_n)$ needs $O(2^n)$ parameters when all $X_i$'s are discrete random variables with binary support sets which need a considerable amount of computational effort and space. However, following the main idea of nodes being influenced only by their ancestors, the joint probability distribution function can be represented as
$$P(X_1, X_2, \ldots, X_n)=\prod\limits_{i=1}^n P(X_i|\text{parents}(X_i))$$
where $\text{parents}(X_i)$ is the set of parents of $X_i$ in a DAG representation of a BN. The conditional probabilities $P(X_i|\text{parents}(X_i))$ that captures a localised dependency in a big network can be determined using $O(2^{|\text{parents}(X_i)|})$ that makes the computations more tractable \citep{heckerman2008tutorial}. 

A BN can be constructed manually using expert knowledge of the domain, automatically using large training data by means of computer programs (data-driven approach), or a combination of both \citep{zhang2016expert}.
Learning a BN structure from data is an area in ML that continues to grow; however, it is a challenging task even if complete data are available \citep{de2011efficient}. A BN is fully determined by a DAG, which is learned through a structure learning task \citep{amirkhani2016exploiting}, and the CPT of each node. The parameters of a BN model, which are  the table of conditional probabilities for discrete cases, are learned using the expectation-maximisation algorithm \citep{lu2015exploration} after fixing the DAG structure. To construct a CPT, one needs to consider all possible combinations of the states of a node and the states of its direct parents. For a node with $s$ states and $k$ parents each with $n$ states, the probability of $s \times n^k$ different combinations need to be calculated. Furthermore, Bayesian and constraint-based techniques are the two main approaches to performing structural learning to obtain a BN representation of a system. In the Bayesian approach, the user first builds a BN based on domain knowledge, then combines it with data to find the most likely structure. On the other hand, constraint-based algorithms build the model structure by searching conditional dependencies between each pair of variables \citep{uusitalo2007advantages}. It should be noted that it is essentially impossible to obtain a complete causal modal without considering a huge number of variables \citep{gheisari2016bnc}, and basically, learning the structure of a BN from data is an NP-hard problem \citep{kareem2020structure,kareem2021falcon}.


%%%%%%%%%%%%%%%%%%%%%%%%%%%%%%%%%%%%%%%%%%%%%%%%%



\section{Experimental design} \label{Experimental}
The main aim of this study is to find factors that influence the price movements of cryptocurrencies. We use BNs to investigate influential relationships between altcoins and a set of financial and non-financial factors. As the price data are continuous, various software packages for constructing BNs require a discretisation pre-processing step~\citep{death2015good}. An inferential DAG representing continuous data can be obtained without discretisation; however, those DAGs do not provide post-hoc tools including sensitivity analysis, diagnoses, and inference diagrams.\footnote{BN software packages, such as Genie, are capable of handling continuous data by applying PC algorithm for constructing DAG models; however, these networks do not provide post-hoc analysis tools} For carrying out the discretisation of price data, appropriate choices for the discretisation method and the number of intervals (bins) have to be made \citep{nojavan2017comparative}. To achieve this purpose, we create a procedure that builds various BNs based on commonly used discretisation methods with different numbers of intervals. The performances of BNs obtained from this procedure are further studied to analyse their competency in predicting the market movement directions. Since we use the data-driven approach to construct BNs, the procedure allows the evaluation  of multiple scenarios of building BNs from data and eventually detecting the best-performing BNs. Moreover,  another advantage of the procedure is incorporating k-means, one of the most effective clustering methods, which is generally unavailable in frequently used BNs software packages.

In this work, we consider three commonly used unsupervised machine learning (ML) methods \citep{saleh2018implementation,min2009global} to be combined with bin numbers for the discretisation task. A brief description of these methods follows.
\begin{itemize}
 \item Equal interval: the data range is divided into equal-length intervals based on the number of bins.
 \item Equal quantile: the data is sorted first to find different quantiles based on the number of bins. For example, in a 2-bin case, the data is divided based on the median, and in a 4-bin case, the division points are first, second, and third quantiles.
 \item K-means: k-means clustering is one of the most effective and popular methods in the literature \citep{ahmed2020k, sinaga2020unsupervised}. The algorithm iteratively groups data into clusters based on initially randomly selected centroids until it finds the optimal positions of the cluster centres. Each cluster then is treated as a bin at the termination of the algorithm. 
\end{itemize}

The main approach to extracting causal insights from data in this study is shown in the flowchart in Figure~\ref{fig:alg}. We keep the PC path in this flowchart only to provide the complete treatment process in dealing with data in a causal analysis. The BNs obtained from discretised data are later compared based on different prediction performance measures. In particular, through five-fold cross-validation, the models' predictive capabilities are assessed and compared using the area under the curve (AUC), and model accuracy as the percentage of correct predictions to the total number of observations in the test data. 

\begin{figure*}[t]
 \centering
 \includegraphics[scale=0.75]{Figures/flowchart.JPG}
 \caption{The flowchart demonstrating the construction of BNs in this study.}
 \label{fig:alg}
\end{figure*}

%%%%%%%%%%%%%%%%%%%%%%%%%%%%%%%%%%%%%%%%%%%%%%%%%
\subsection{Data}
The altcoins studied in this research are selected from the top 10 cryptocurrencies considering their market capitalisation and the minimum number of trading days (i.e., length $\geq$  900 days). The length of data is important in establishing stronger unpaired correlational studies between altcoins and traditional financial assets with decades of price data history. In particular, altcoins with more than four years of data are selected for performing statistically significant analysis supported by their historic price data. Hence, the altcoins we have chosen are Binance Coin, Ethereum, Litecoin, Ripple, and Tether. The total market capitalisation of these altcoins represents around 30\% of the cryptocurrency market in 2021. The daily close price data of the altcoins and traditional assets Gold, MSCI, S\&P 500, WTI, and USDX are extracted from \textit{www.investing.com}, and the daily tweet number of the altcoins are drawn from \textit{www.bitinfocharts.com} as time series data. Since the daily tweet data for Tether is unavailable, we exclude the analysis of the impact of social media data from this coin.

Descriptive statistics of the price data and the daily tweet number are presented in Tables~\ref{tab:coin describtion} and \ref{tab:tweet no describtion}. The columns in the tables are self-explanatory, and the last column in Table~\ref{tab:coin describtion}, {\it No. Obs.}, indicates the number of observations available at the time of extracting data. In addition, visual representations of changes in the prices of altcoins and the number of tweets are shown in Figure \ref{fig:tweet_price_all}.

\begin{table}[H]
 \centering
 \caption{Summary statistics for the price of altcoins.}
 \label{tab:coin describtion}
 %\resizebox{0.8\columnwidth}{!}{%
  \begin{adjustbox}{max width=\textwidth}
 \begin{tabular}{lcclccccccl|}
 \hline
  \textbf{} & \textbf{Mean} & \textbf{Std. Dev.} & \textbf{Min.} &\textbf{Median}&\textbf{Max.}&\textbf{No. Obs.}\\ \hline
   \textbf{Binance Coin} & 70.53 & 133.17 & 4.52 & 17.98& 676.56 & 925\\ 
 \textbf{Ethereum} & 542.76 & 696.69 & 8.09 & 294.26 &4141.1 & 1411 \\ 
 \textbf{Litecoin} & 66.56 & 66.17 &2.59 & 51.13& 373.35 & 1454\\  
 \textbf{Ripple} & 0.46 &	0.32 & 0.14&0.32 & 2.070 & 913\\ 
 \textbf{Tether}
 & 0.10 & 0.01 & 0.90 & 1.00 & 1.07 & 1129\\ \hline
  \end{tabular}
 %}
 \end{adjustbox}
\end{table}

\begin{table}[H]
 \centering
 \caption{Summary statistics for daily tweet number of altcoins.}
 \label{tab:tweet no describtion}
 %\resizebox{0.8\columnwidth}{!}
 \end{adjustbox}
\end{table}

To better understand the nature of the data used in this study, Figure~\ref{fig:tweet_price_all} shows a visual representation of the price variations of the four cryptocurrencies (in orange) along with the number of daily tweets associated with them (in blue) between January 2018 and July 2021. As mentioned before, the tweet data is not available for Tether, and hence its plot is not shown. The price plots demonstrate a steady start for Binance Coin and Ethereum, and there is a considerable steady period for Litecoin and Ripple. However, all the coins have a high level of variation in 2021. In particular, the variance of the close price data shows a significant difference before and after 2021. For example, the variance of BNB close price is 66.99 before 2021, and it is 28303.37 after 2021. Therefore, we expect a high level of errors in predicting price movement directions due to the increasing level of variation in data.  For the tweet numbers, although there are several spikes, their trends are relatively consistent in comparison to the price data.
\begin{figure*}[t]
  \centering
  \includegraphics[scale=0.55]{Figures/all_tweet_price.jpg}
  \caption{The price fluctuation and the number of daily tweets associated with Binance Coin, Ethereum, Litecoin, and Ripple.}
  \label{fig:tweet_price_all}
\end{figure*}






\section{Results and discussion} \label{Results}
The numerical results regarding detecting the factors affecting the price movements of altcoins derived using BNs are presented in this section. Moreover, we provide ad-hoc analysis including diagnosis and sensitivity analysis, which sheds further light on understanding the causal relationship between the influencing factors and altcoin prices. Before constructing BNs, we apply a discretisation pre-processing step  to raw data. There is a wide range of software packages for establishing causal networks, including Netica, Genie, Hugin, and Analytica.\footnote{For interested readers, a comparison of different BN software packages has been studied by \citet{mahjoub2011software}.} The Genie software package \citep{bayesfusion2017genie} is used to derive the BNs in this study, as it is frequently utilised in the literature and provides several computational facilities, including PC algorithm to build DAGs from continuous data. The computational work is performed on an Intel i7-Core(TM) CPU @ 1.90GHz, 2.11 GHz processor, and 16.0 GB RAM, which runs on Windows 10 Enterprise.

As explained earlier, the discretised data are fed into Genie. The discretisation step requires specifying an ML algorithm and the number of intervals (bins). Based on the number of intervals, a set of states are defined for and assigned to each node to build a CPT. Namely, the states for financial market-related nodes with two intervals can be defined as `Down' and `Up', in which `Down' represents the downward trend when today's closing price is lower than the previous day's closing price, and it is  `Up' for the other way round case. For three intervals, the states are `Down', `Steady', and `Up', and finally, for the four-interval case, the state set has four members of  `Strong Down', `Down', `Up', and `Strong Up'. After deciding on the states of each node, based on the proposed flowchart in Figure \ref{fig:alg}, the discretised data are fed into Genie (the BN box in the figure) for building the BNs. We construct 45 BNs for the combinations of choices considering five coins, three  bin numbers, namely, 2-bin, 3-bin, and 4-bin, and three  discretisation methods. We do not notice significant changes by considering more than four intervals in this study. It is worth mentioning that we experimented with both scaled and raw data, and we have not observed considerably different prediction accuracy among them.

To report the outcomes of numerical experiments, a comparison of the predictive performances of the 45 trained BNs is discussed in Section \ref{pred_acc} considering the AUC and accuracy measures. These measures are frequently used in assessing the predictive performance of ML models.  In addition, inference analysis and sensitivity analysis using BNs are presented in Section \ref{st_learn}. With inference analysis, we are interested in studying the influencing factors of the target node. Also, the aim of the sensitivity analysis is to study the changes in the probabilities of the target node (altcoins in this study) based on the changes in the factors influencing it.

\subsection{Model evaluation} \label{pred_acc}
The BNs are used to predict the price movement directions of altcoins. The predictions are obtained based on the posterior probabilities of the states of the target node, which is set as the price of an altcoin in a BN. The comparison of the performances of the BNs with the actual data is conducted by means of a five-fold cross-validation process in this study. The predictive performances of the BNs acquired from the three discretisation methods in combination with two, three, and four intervals for each altcoin are calculated based on accuracy and AUC measures. Table~\ref{tab:Comparison of predictive} summarises their predictive performances. For each altcoin, the results of prediction accuracy and AUC are reported considering the choice of an ML method and the number of intervals on different columns. In addition, the best results for each coin are boldfaced for further emphasis. 

\begin{table*}[h]
\caption{The predictive accuracy and AUC of the BNs obtained by choosing an unsupervised ML method and a bin number.}
\label{tab:Comparison of predictive}
 \begin{adjustbox}{max width=\textwidth}
\begin{tabular}{lc|ccc|ccc|ccc|}
\cline{3-11}
\multicolumn{2}{l|}{\multirow{2}{*}{}} &
  \multicolumn{3}{c|}{\textbf{Equal Interval}} &
  \multicolumn{3}{c|}{\textbf{Equal Quantile}} &
  \multicolumn{3}{c|}{\textbf{K-means}} \\ \cline{3-11} 
\multicolumn{2}{l|}{} &
  \multicolumn{1}{c|}{2 Intervals} &
  \multicolumn{1}{c|}{3 Intervals} &
  4 Intervals &
  \multicolumn{1}{c|}{2 Intervals} &
  \multicolumn{1}{c|}{3 Intervals} &
  4 Intervals &
  \multicolumn{1}{c|}{2 Intervals} &
  \multicolumn{1}{c|}{3 Intervals} &
  4 Intervals \\ \hline
\multicolumn{1}{|l|}{\multirow{2}{*}{\textbf{Binance Coin}}} &
  \textit{Acc.} &
  \multicolumn{1}{c|}{0.92757} &
  \multicolumn{1}{c|}{0.93297} &
  0.90487 &
  \multicolumn{1}{c|}{0.84108} &
  \multicolumn{1}{c|}{0.78811} &
  0.77514 &
  \multicolumn{1}{c|}{\textbf{0.98440}} &
  \multicolumn{1}{c|}{0.94703} &
  0.94270 \\ \cline{2-11} 
\multicolumn{1}{|l|}{} &
  \textit{AUC} &
  \multicolumn{1}{c|}{0.88362} &
  \multicolumn{1}{c|}{0.95636} &
  0.96090 &
  \multicolumn{1}{c|}{0.93523} &
  \multicolumn{1}{c|}{0.92638} &
  0.93921 &
  \multicolumn{1}{c|}{\textbf{0.96649}} &
  \multicolumn{1}{c|}{0.96448} &
  0.98228 \\ \hline
\multicolumn{1}{|l|}{\multirow{2}{*}{\textbf{Ethereum}}} &
  \textit{Acc.} &
  \multicolumn{1}{c|}{0.92757} &
  \multicolumn{1}{c|}{0.93297} &
  0.90487 &
  \multicolumn{1}{c|}{0.84108} &
  \multicolumn{1}{c|}{0.78811} &
  0.77514 &
  \multicolumn{1}{c|}{\textbf{0.96649}} &
  \multicolumn{1}{c|}{0.94703} &
  0.94270 \\ \cline{2-11} 
\multicolumn{1}{|l|}{} &
  \textit{AUC} &
  \multicolumn{1}{c|}{0.88362} &
  \multicolumn{1}{c|}{0.94980} &
  0.96090 &
  \multicolumn{1}{c|}{0.93523} &
  \multicolumn{1}{c|}{0.92638} &
  0.93921 &
  \multicolumn{1}{c|}{\textbf{0.98440}} &
  \multicolumn{1}{c|}{0.96448} &
  0.98228 \\ \hline
\multicolumn{1}{|l|}{\multirow{2}{*}{\textbf{Litecoin}}} &
  \textit{Acc.} &
  \multicolumn{1}{c|}{\textbf{0.93535}} &
  \multicolumn{1}{c|}{0.89752} &
  0.77304 &
  \multicolumn{1}{c|}{0.85213} &
  \multicolumn{1}{c|}{0.85282} &
  0.75585 &
  \multicolumn{1}{c|}{0.90096} &
  \multicolumn{1}{c|}{0.78404} &
  0.86176 \\ \cline{2-11} 
\multicolumn{1}{|l|}{} &
  \textit{AUC} &
  \multicolumn{1}{c|}{0.45643} &
  \multicolumn{1}{c|}{0.87317} &
  0.78070 &
  \multicolumn{1}{c|}{0.91294} &
  \multicolumn{1}{c|}{\textbf{0.95781}} &
  0.94389 &
  \multicolumn{1}{c|}{0.94055} &
  \multicolumn{1}{c|}{0.91143} &
  0.93683 \\ \hline
\multicolumn{1}{|l|}{\multirow{2}{*}{\textbf{Ripple}}} &
  \textit{Acc.} &
  \multicolumn{1}{c|}{\textbf{0.93647}} &
  \multicolumn{1}{c|}{0.89595} &
  0.89376 &
  \multicolumn{1}{c|}{0.85323} &
  \multicolumn{1}{c|}{0.77218} &
  0.70208 &
  \multicolumn{1}{c|}{0.88171} &
  \multicolumn{1}{c|}{0.80613} &
  0.81490 \\ \cline{2-11} 
\multicolumn{1}{|l|}{} &
  \textit{AUC} &
  \multicolumn{1}{c|}{0.78207} &
  \multicolumn{1}{c|}{0.92122} &
  \textbf{0.93868} &
  \multicolumn{1}{c|}{0.92043} &
  \multicolumn{1}{c|}{0.91182} &
  0.91274 &
  \multicolumn{1}{c|}{0.92287} &
  \multicolumn{1}{c|}{0.90505} &
  0.93104 \\ \hline
\multicolumn{1}{|l|}{\multirow{2}{*}{\textbf{Tether}}} &
  \textit{Acc.} &
  \multicolumn{1}{c|}{0.95748} &
  \multicolumn{1}{c|}{0.90788} &
  0.94686 &
  \multicolumn{1}{c|}{0.66253} &
  \multicolumn{1}{c|}{0.61736} &
  0.50044 &
  \multicolumn{1}{c|}{\textbf{0.96546}} &
  \multicolumn{1}{c|}{0.89105} &
  0.88663 \\ \cline{2-11} 
\multicolumn{1}{|l|}{} &
  \textit{AUC} &
  \multicolumn{1}{c|}{0.65604} &
  \multicolumn{1}{c|}{0.78437} &
  0.80167 &
  \multicolumn{1}{c|}{0.71815} &
  \multicolumn{1}{c|}{0.76704} &
  0.76908 &
  \multicolumn{1}{c|}{\textbf{0.82796}} &
  \multicolumn{1}{c|}{0.79505} &
  0.43255 \\ \hline
\multicolumn{2}{|l|}{\textbf{Number of winning}} &
  \multicolumn{1}{c|}{2} &
  \multicolumn{1}{c|}{0} &
  1 &
  \multicolumn{1}{c|}{0} &
  \multicolumn{1}{c|}{1} &
  0 &
  \multicolumn{1}{c|}{6} &
  \multicolumn{1}{c|}{0} &
  0 \\ \hline
\end{tabular}
 \end{adjustbox}
\end{table*}


\begin{figure*}[h]
 \centering
\includegraphics[width=\linewidth,height=0.7\textheight,keepaspectratio] {Figures/predication_acc.jpg}
 \caption{Prediction performance of BNs obtained from choosing an ML discretisation method and a bin number considering the accuracy measure.}
 \label{fig:prediction_acc}
\end{figure*}

A visual representation of the results presented in Table~\ref{tab:Comparison of predictive} is shown in Figures \ref{fig:prediction_acc} and \ref{tab:Comparison of predictive}, in which each bar represents the accuracy and AUC of each BN respectively. In these figures, ML methods are grouped separately, and each altcoin has a specific colour. Furthermore, the number of times that each combination of an ML technique and the number of bins has the highest performance is summarised in the bottom row of Table \ref{tab:Comparison of predictive}. 

\begin{figure*}[h]
 \centering
\includegraphics[width=\linewidth,height=0.7\textheight,keepaspectratio]
 {Figures/predication_AUC.jpg}
 \caption{Prediction performance of BNs obtained from choosing an ML discretisation method and a bin number considering the AUC measure.}
 \label{fig:prediction_AUC}
\end{figure*}

The results show that the k-means method with two bins performs better considering both the accuracy and AUC. In six cases k-means outperformed other ML methods, and in eight cases the 2-bins setting for the number of bins is successful. One explanation for the success of k-means with two bins is that a two-bin model provides a better representation of data in comparison to three-bin or four-bin representations. In other words, increasing the number of bins does not have a noticeable impact on the accuracy of the models due to the nature of the training data. To provide further evidence on this matter, Table \ref{tab:Comparison bin} shows the number of data points in each bin. In this table, it is evident that the number of data points in Bin1 changes negligibly after increasing the number of bins. Therefore, a 2-bin discretisation is a better representation of the data. In dealing with unbalanced classes of data, further binning generally does not improve the accuracy of predictions, as is seen in our analysis. 

In addition, k-means provide better accuracy and AUC in comparison to equal interval and quantile due to its clustering capacity, which provides a better learning capability from data. It is observed that k-means scores the highest performance measurement values for Binance Coin, Ethereum, and Tether due to the distribution of close price values in the dataset. However, for Litecoin and Ripple, due to a higher level of variations in the data, the equal interval and equal quantile methods achieve better results. Therefore, it is possible to conclude that for unseen cryptocurrency data choosing a combination of k-means with two bins discretesation approach is a good strategy for preprocessing data to construct BNs.



\begin{table*}[h]
\caption{The number of data points in each bin based on the choices of the number of intervals and the unsupervised ML method for discretisation.}
 
\label{tab:Comparison bin}
\begin{adjustbox}{max width=\textwidth}
\begin{tabular}{lc|ll|lll|llll|}
\cline{3-11}
\multicolumn{2}{l|}{} &
  \multicolumn{2}{c|}{\textbf{2 Intervals}} &
  \multicolumn{3}{c|}{\textbf{3 Intervals}} &
  \multicolumn{4}{c|}{\textbf{4 Intervals}} \\ \cline{3-11} 
\multicolumn{2}{l|}{\multirow{-2}{*}{}} &
  \multicolumn{1}{l|}{\textit{Bin1}} &
  \textit{Bin2} &
  \multicolumn{1}{l|}{\textit{Bin1}} &
  \multicolumn{1}{l|}{\textit{Bin2}} &
  \textit{Bin3} &
  \multicolumn{1}{l|}{\textit{Bin1}} &
  \multicolumn{1}{l|}{\textit{Bin2}} &
  \multicolumn{1}{l|}{\textit{Bin3}} &
  \textit{Bin4} \\ \hline
\multicolumn{1}{|l|}{} &
  Equal Interval &
  \multicolumn{1}{l|}{859} &
  66 &
  \multicolumn{1}{l|}{765} &
  \multicolumn{1}{l|}{96} &
  2 &
  \multicolumn{1}{l|}{792} &
  \multicolumn{1}{l|}{67} &
  \multicolumn{1}{l|}{44} &
  22 \\ \cline{2-11} 
\multicolumn{1}{|l|}{} &
  Equal Quantile &
  \multicolumn{1}{l|}{462} &
  463 &
  \multicolumn{1}{l|}{308} &
  \multicolumn{1}{l|}{309} &
  231 &
  \multicolumn{1}{l|}{231} &
  \multicolumn{1}{l|}{231} &
  \multicolumn{1}{l|}{231} &
  231 \\ \cline{2-11} 
\multicolumn{1}{|l|}{\multirow{-3}{*}{\textbf{Binance Coin}}} &
  K-menas &
  \multicolumn{1}{l|}{\textbf{793}} &
  \textbf{132} &
  \multicolumn{1}{l|}{792} &
  \multicolumn{1}{l|}{99} &
  34 &
  \multicolumn{1}{l|}{791} &
  \multicolumn{1}{l|}{66} &
  \multicolumn{1}{l|}{10} &
  28 \\ \hline
\multicolumn{1}{|l|}{} &
  Equal Interval &
  \multicolumn{1}{l|}{687} &
  724 &
  \multicolumn{1}{l|}{1259} &
  \multicolumn{1}{l|}{118} &
  34 &
  \multicolumn{1}{l|}{1128} &
  \multicolumn{1}{l|}{93} &
  \multicolumn{1}{l|}{65} &
  25 \\ \cline{2-11} 
\multicolumn{1}{|l|}{} &
  Equal Quantile &
  \multicolumn{1}{l|}{705} &
  706 &
  \multicolumn{1}{l|}{470} &
  \multicolumn{1}{l|}{470} &
  471 &
  \multicolumn{1}{l|}{353} &
  \multicolumn{1}{l|}{353} &
  \multicolumn{1}{l|}{353} &
  353 \\ \cline{2-11}  
\multicolumn{1}{|l|}{\multirow{-3}{*}{\textbf{Ethereum}}} &
  K-menas &
  \multicolumn{1}{l|}{\textbf{1253}} &
  \textbf{158} &
  \multicolumn{1}{l|}{1219} &
  \multicolumn{1}{l|}{115} &
  77 &
  \multicolumn{1}{l|}{1054} &
  \multicolumn{1}{l|}{205} &
  \multicolumn{1}{l|}{105} &
  47 \\ \hline 
\multicolumn{1}{|l|}{} &
  Equal Interval &
  \multicolumn{1}{l|}{\textbf{1362}} &
  \textbf{92} &
  \multicolumn{1}{l|}{1204} &
  \multicolumn{1}{l|}{215} &
  35 &
  \multicolumn{1}{l|}{1124} &
  \multicolumn{1}{l|}{238} &
  \multicolumn{1}{l|}{73} &
  19 \\ \cline{2-11} 

\multicolumn{1}{|l|}{} &
  Equal Quantile &
  \multicolumn{1}{l|}{727} &
  727 &
  \multicolumn{1}{l|}{\textbf{485}} &
  \multicolumn{1}{l|}{\textbf{484}} &
  \textbf{485} &
  \multicolumn{1}{l|}{364} &
  \multicolumn{1}{l|}{363} &
  \multicolumn{1}{l|}{361} &
  366 \\ \cline{2-11} 

\multicolumn{1}{|l|}{\multirow{-3}{*}{\textbf{Litecoin}}} &
  K-menas &
  \multicolumn{1}{l|}{1150} &
  304 &
  \multicolumn{1}{l|}{1049} &
  \multicolumn{1}{l|}{301} &
  104 &
  \multicolumn{1}{l|}{479} &
  \multicolumn{1}{l|}{671} &
  \multicolumn{1}{l|}{242} &
  62 \\ \hline

\multicolumn{1}{|l|}{} &
  Equal Interval &
  \multicolumn{1}{l|}{\textbf{855}} &
  \textbf{58} &
  \multicolumn{1}{l|}{789} &
  \multicolumn{1}{l|}{104} &
  20 &
  \multicolumn{1}{l|}{\textbf{738}} &
  \multicolumn{1}{l|}{\textbf{117}} &
  \multicolumn{1}{l|}{\textbf{46}} &
  \textbf{12} \\ \cline{2-11} 

\multicolumn{1}{|l|}{} &
  Equal Quantile &
  \multicolumn{1}{l|}{465} &
  457 &
  \multicolumn{1}{l|}{303} &
  \multicolumn{1}{l|}{305} &
  305 &
  \multicolumn{1}{l|}{228} &
  \multicolumn{1}{l|}{228} &
  \multicolumn{1}{l|}{228} &
  228 \\ \cline{2-11} 

\multicolumn{1}{|l|}{\multirow{-3}{*}{\textbf{Ripple}}} &
  K-means &
  \multicolumn{1}{l|}{748} &
  129 &
  \multicolumn{1}{l|}{656} &
  \multicolumn{1}{l|}{182} &
  75 &
  \multicolumn{1}{l|}{551} &
  \multicolumn{1}{l|}{237} &
  \multicolumn{1}{l|}{85} &
  40 \\ \hline

\multicolumn{1}{|l|}{} &
  Equal Interval &
  \multicolumn{1}{l|}{48} &
  1081 &
  \multicolumn{1}{l|}{15} &
  \multicolumn{1}{l|}{1026} &
  88 &
  \multicolumn{1}{l|}{10} &
  \multicolumn{1}{l|}{38} &
  \multicolumn{1}{l|}{1071} &
  10 \\ \cline{2-11} 

\multicolumn{1}{|l|}{} &
  Equal Quantile &
  \multicolumn{1}{l|}{530} &
  599 &
  \multicolumn{1}{l|}{337} &
  \multicolumn{1}{l|}{410} &
  382 &
  \multicolumn{1}{l|}{277} &
  \multicolumn{1}{l|}{253} &
  \multicolumn{1}{l|}{315} &
  284 \\ \cline{2-11} 

\multicolumn{1}{|l|}{\multirow{-3}{*}{\textbf{Tether}}} &
  K-means &
  \multicolumn{1}{l|}{\textbf{39}} &
  \textbf{1090} &
  \multicolumn{1}{l|}{31} &
  \multicolumn{1}{l|}{1007} &
  91 &
  \multicolumn{1}{l|}{19} &
  \multicolumn{1}{l|}{75} &
  \multicolumn{1}{l|}{1001} &
  34 \\ \hline
\end{tabular}
\end{adjustbox}
\end{table*}


\subsection{Price movement analyses}  \label{st_learn}
In this subsection, we further explain the best-performing BNs for each altcoin from the perspectives of inference and sensitivity analyses following a methodology used in \citet{asvija2021security} and \citet{hosseini2019development}. In certain cases, like Binance Coin, the best-performing networks coincide for both performance measures, and there is a single winner network. However, for the cases with different best models considering the accuracy and AUC measures, we follow the Occam’s razor principle~\footnote{According to Occam's razor principle, the simplest hypothesis among all correct hypotheses captures the best structure of the problem and holds the highest prediction accuracy \citep{gamberger1997conditions}.} and choose the network with a smaller number of parameters as the final winner. For each coin, the BN representation of the best-performing network in addition to a snapshot of typical inferential decision-making and the sensitivity analysis diagrams are depicted in Figures~\ref{fig: Inference} and \ref{fig: sensitivity}.


\subsubsection{Inference} 
This section deals with the probabilistic reasoning of the top-performing BNs that are represented in Figure~\ref{fig: Inference}. This kind of diagnostic inference analysis investigates the changes in the CPTs of other nodes when a particular node is fixed in one of its states. Inferences can be performed bi-directional either as prediction (from cause to effect) or as diagnosis (inferring from effect to cause) \citep{lu2020risk}. This analysis provides an important tool for decision-makers to investigate the impact of causal relationships among the nodes when the probability assigned to states in one particular node affects the probabilities of the states in other nodes.

\begin{figure*}[h]
 \centering
\begin{subfigure}{.5\textwidth}
  \centering
 % \includegraphics[scale=0.35]
\includegraphics[width=\linewidth,height=\textheight,keepaspectratio]
 {Figures/BNB_inf.JPG}
  \caption{Binance Coin inference diagram}
  \label{fig:BNB_inf}
\end{subfigure}%
\begin{subfigure}{.5\textwidth}
  \centering
\includegraphics[width=\linewidth,height=\textheight,keepaspectratio]{Figures/ETH_inf.JPG}
  \caption{Ethereum inference diagram}
  \label{fig: ETH_inf}
\end{subfigure}
\begin{subfigure}{.5\textwidth}
   \centering
\includegraphics[width=\linewidth,height=\textheight,keepaspectratio]{Figures/LTC_inf.JPG}
  \caption{Litecoin inference diagram}
  \label{fig:LTC_inf}
\end{subfigure}%
\begin{subfigure}{.5\textwidth}
   \centering
\includegraphics[width=\linewidth,height=\textheight,keepaspectratio]{Figures/XRP_inf.JPG}
  \caption{Ripple inference diagram}
  \label{fig:XRP_inf}
\end{subfigure}
\begin{subfigure}{.6\textwidth}
  \centering
\includegraphics[width=\linewidth,height=\textheight,keepaspectratio]{Figures/Tether_inf.JPG}
  \caption{Tether inference diagram}
  \label{fig:Tether_inf}
\end{subfigure}%

 \caption{A typical inference diagram in which the root of each BN is set as one of its states.} 
\label{fig: Inference}
\end{figure*}

For each altcoin, the best-performing BN is depicted in Figure~\ref{fig: Inference}. To facilitate the explanation of the probabilistic reasoning on the causal relationship between the nodes, the target node is fixed in one of its states in the figure. %In addition, the causal relationships between altcoins and the financial assets studied in this research are summarised in Table~\ref{tab:arc describtion}. 
The table shows that gold price only directly affects Tether and has the least number of direct influences among the financial factors. In addition, the number of tweets seems the most frequent influencing factor, considering the fact that we do not have tweet number data for Tether. 

Utilising graph terminology, the number of edges entering a node $v$, or the in-degree of the node represented as $\text{deg}^-(v)$, in a BN shows the number of influencing factors. With this analogy, $\text{deg}^-(\text{Litecoin})=0$ which indicates that Litecoin is isolated, and this fact is visually visible in Figure~\ref{fig:LTC_inf}. There is no direct edge entering the Litecoin node. Furthermore,  $\text{deg}^-(\text{Tether})=5$ indicates that Tether is affected by all the five factors, and it is the altcoin with the highest number of influencing factors in this study.

% \begin{table}[h]
%   \centering
%     \caption{The presence of causal relationships between the altcoins and traditional financial assets in addition to social media data in the best performing BNs. \DT{Don't see why this is necessary. What is wrong with the Bayesian networks?} \RAS{I believe it could be an informative table as it provides a comparison of all the BNs in one place. And we have two  paragraphs explaining it.  }}
%   \label{tab:arc describtion}
%   %\resizebox{0.8\columnwidth}{!}{%
%   \begin{tabular}{lcccccccc}
%   \hline
%    \textbf{} & \textbf{S\&P 500} & \textbf{MSCI} & \textbf{USDX} &\textbf{WTI}&\textbf{Gold}&\textbf{Tweet No.}\\ \hline
%      \textbf{Binance Coin} & \circled{} & \circled{} & & \circled{}& & \circled{}\\ \hline
%   \textbf{Ethereum} & & \circled{} & &\circled{} & & \\ \hline
%   \textbf{Litecoin}  & &  && & & \circled{}\\ \hline  
%   \textbf{Ripple}  &\circled{} &	 & \circled{}&& & \circled{}\\ \hline
%   \textbf{Tether} 
%  & \circled{} & \circled{} & \circled{} & \circled{} & \circled{} & \\ \hline
%     \end{tabular}
%   %}
% \end{table}

The computational results indicate that Tether, as the only stablecoin in the study, is highly connected to financial assets, and this fact shows a great deal of integration between Tether and other financial instruments. In addition, the causal relationships between the altcoins and associated social media data are confirmed, except for Ethereum. Our study finds that social media represent themselves as a strong determinant of price changes in cryptocurrencies. However, further investigation is necessary to quantify the strength of those relationships. Due to the lack of daily tweet number data for Tether, the final network does not show a connection between the coin and the social factor. However, the existence of a causal relationship between them is strongly predictable. Moreover, in relation to traditional financial assets, it is observed that MSCI and S\&P 500 have the highest number of causal connections to the altcoins in this study. With a similar analogy, USDX only influences Ripple and Tether. In particular, the energy index, WTI, has a causal relationship with Binance Coin and Ethereum due to a higher difficulty level of mining new coins. Despite Tether being a stablecoin without mining activities, the causal relationship with WTI can be attributed to a greater level of integration with traditional financial assets and recent trading virtual entities including NFTs and metaverse tokens. 

Furthermore, from the coins' perspective, the results show that Litcoin is the most isolated coin which has a causal connection with the daily tweet number, and as mentioned before, Tether has the highest number of connections. In summary, each altcoin has a different behaviour regarding relationships with traditional assets, and no general conclusions can  be drawn that are acceptable for all cryptocurrencies. This fact can be considered evidence of the distinct nature of each cryptocurrency and emphasise that investors require different investment strategies for each cryptocurrency.  

Energy price is one of the influential factors in the mining process of cryptocurrencies that determines the profitability of mining activities, affects the supply and demand equilibrium, and eventually, the market prices of the cryptocurrencies  \citep{hayes2017cryptocurrency, okorie2020crude}. In particular, as the difficulty of mining coins increases, more computational resources are needed. Therefore, the administrating bodies of cryptocurrencies have ongoing attempts to improve energy efficiency in the mining process. For example, Ethereum, the second-largest cryptocurrency by market capitalisation, has recently upgraded its blockchain network to reduce the energy consumption of its blockchain system. The substantial overhaul of Ethereum, known as the Merge, has decreased the electricity consumption and carbon footprint of the Ethereum network by over 99.9\%, based on a September 2022 report by the Crypto Carbon Ratings Institute.\footnote {https://carbon-ratings.com} Binance Coin and Ethereum, as the two mining-based altcoins of the study, have a causal relationship with the energy index.  This influential relationship can be attributed to the greater mining difficulty of these altcoins,~\footnote {For example, the mining difficulty of Ethereum was 13.3 Peta in April 2022 compared to 16.9 Mega of Litecoin.} which means more energy is needed to mine new coins. This leads to a higher mining cost, consequently influencing cryptocurrency price changes. 

The US economy is a significant influencer on volatility in financial markets, and its S\&P 500 and USDX indices are the subject of many studies as two of the most influential equity market indices \citep{kim2017dynamic}. To investigate the impact of this market on altcoins, we have incorporated data from S\&P 500 and USDX indices in this study. A comparative consideration of BNs of altcoins based on the diagnostic analysis, we find that the S\&P 500 index price changes have the highest impact only on Ethereum price changes. Furthermore, USDX has a causal relationship with Ripple and Tether as child-parent dependencies; however, the inference analysis reveals that its impact is not statistically significant (Figures~\ref{fig:XRP_inf} and~\ref{fig:Tether_inf}). 

Economic benchmarking is a technique in which a business's performance is examined against other businesses using holistic economic measures \citep{cunningham2014river}. While an economy benchmark represents the risk and returns of a class of assets, it can also help investors choose a passive or active trading strategy \citep{parikh2019emerging}. MSCI is a well-maintained benchmark in finance, and several fund-managing institutes use MSCI as a benchmark for portfolio management. Therefore, we investigate its suitability to be used as a benchmark for trading altcoins as well. With respect to the BNs that we obtain in this study, there is a directional edge between MSCI and the prices of Binance Coin, Ethereum, and Tether. Inferential analysis reveals that the influence of MSCI is significantly stronger than the impact of other financial assets when there is a causal relationship. This can be related to how the state of the world economy affects investment in the cryptocurrency market. 

Social media, particularly Twitter, is a major news source for cryptocurrency investors. Moreover, well-known and top influencers of the cryptocurrency market are more active on Twitter than on other social media. Therefore, daily tweet data is incorporated into our BN analysis. The BNs obtained from data in this research demonstrate that all mining-based altcoins, except Ethereum, have an edge  connected to their corresponding tweet data. It emphasises the significance of Twitter in the cryptocurrency market. In particular, possibly due to human psychology, both inference and diagnosis analyses show more significant changes in the number of tweets in the occurrence of an upward trend market. This result is in line with other studies such as \citep{kim2021dynamics} where an upward trend market  is highlighted as more influential in comparison to a  downward trend market in investigating the social sentiments of people's tweets. The fluctuation in the number of tweets is observed more in Binance Coin when the market status reverses from the `Down' state to the `Up' state, as understood in our inference and diagnosis analyses. 

In the list of altcoins analysed in this study, Tether is the only coin that is not mining-based. The coin is considered the first and the most recognised stablecoin. A stablecoin is a category of cryptocurrencies with price stabilisation mechanisms that they try to match the stability of another currency, such as a fiat currency, when directly backed, or match a third reserve asset like Ethereum, when indirectly backed \citep{clark2019sok,mita2019stablecoin}. %Considering the only stablecoin in this study, Tether is claimed to be backed by the US dollar. 
As it is represented in Figure~\ref{fig:Tether_inf}, Tether has the highest number of parents among all other altcoins, and its BN  contains a causal relationship with all traditional financial assets. Notably, USDX is one of the parents of Tether, which explains the US dollar as the fiat currency that Tether is pegged with. 


\subsubsection{Sensitivity analysis}
Sensitivity analysis in Bayesian networks deals with studying the relationship between influencing factors and the target variable  \citep{castillo1997sensitivity}. In other words, using sensitivity analysis, one typically studies how the uncertainty in model outputs can be apportioned to the variation of uncertainty in model inputs \citep{saltelli2019so}.  The analysis facilitates the shaping of causal inferences and leads to sanity checks, design improvements, and model debugging. This type of analysis is accomplished by sequentially selecting each state of the input variables, updating posterior probabilities, and recording the range of changes in output variables \citep{meurisse2022risk}. The sensitivity analysis is a significant part of any economic evaluation and helps analysts assess the reliability of outcomes \citep{walker2001allowing}. In this study, we performed a sensitivity analysis to identify and prioritise price-driving factors that significantly influence altcoins' prices. Information on the most important price determinants of altcoins is crucial for investors in designing investment strategies such as portfolio management.

\begin{figure*}[h!]
 \centering
\begin{subfigure}{.5\textwidth}
  \centering
\includegraphics[width=\linewidth,height=0.7\textheight,keepaspectratio]{Figures/BNB_Sen.JPG}
  \caption{Binance Coin sensitivity analysis}
  \label{fig:BNB_Sen}
\end{subfigure}%
\begin{subfigure}{.5\textwidth}
  \centering
\includegraphics[width=\linewidth,height=0.7\textheight,keepaspectratio]{Figures/ETH_Sen.jpg}
  \caption{Ethereum sensitivity analysis}
  \label{fig:ETH_Sen}
\end{subfigure}
\begin{subfigure}{.5\textwidth}
   \centering
\includegraphics[width=\linewidth,height=0.7\textheight,keepaspectratio]{Figures/LTC_sen.JPG}
  \caption{Litecoin sensitivity analysis}
  \label{fig:LT_Sen}
\end{subfigure}%
\begin{subfigure}{.5\textwidth}
   \centering
\includegraphics[width=\linewidth,height=0.7\textheight,keepaspectratio]{Figures/XRP_Sen.jpg}
  \caption{Ripple sensitivity analysis}
  \label{fig:XRP_Sen}
\end{subfigure}
\begin{subfigure}{0.5\textwidth}
  \centering
 \includegraphics[width=\linewidth,height=0.9\textheight,keepaspectratio]{Figures/Tether_Sen.jpg}
  \caption{Tether sensitivity analysis}
  \label{fig:Tether_Sen}
\end{subfigure}%

\label{fig:allCoins}
\caption{The sensitivity analysis results when the altcoin node is set as the target node.}
\label{fig: sensitivity}
\end{figure*}

The sensitivity analysis in this work is conducted using Genie based on the algorithm proposed in \citet{coupe2000computational}. The altcoin node is selected as the target node in each BN, and a visual output indicates the influencing factors with different shades of red proportional to the strength of influence. These outputs are demonstrated in Figure~\ref{fig: sensitivity} for each altcoin.

According to the results of sensitivity analysis  in Figure~\ref{fig:BNB_Sen}, Binance Coin is sensitive to changes in both S\&P 500 and WTI, where the influence of S\&P 500 is stronger.
Figure~\ref{fig:ETH_Sen} reveals that Ethereum is influenced by all the financial assets with the highest impact inserted by MSCI and gold. In addition, as can be seen in Figure~\ref{fig:XRP_Sen}, S\&P 500 and tweet numbers are causing factors affecting Ripple, in which the influence of tweet numbers is stronger than S\&P 500. Eventually, with sensitivity analysis in Figure~\ref{fig:Tether_Sen}, Tether is mainly driven by all five financial assets considered in this study, similar to Ethereum, and MSCI and gold have stronger influences than other factors. 
 Notably, for a coin that is not mining-based and does not require electricity to create new coins, the level of dependency between Tether and the energy index WTI is weak, and it is evidenced by the pale colour in the diagram.




\section{Conclusions and future work } \label{Conclusions}
The study aims to explore the factors that influence the price movements of various altcoins using BNs. For this purpose, the relationships between five popular altcoins with five financial assets and the number of daily tweets associated with each altcoin are contemplated. This research adopts a Bayesian network methodology to explore causal relationships between these elements due to the capacity of BNs in detecting such relationships and providing ad-hoc analysis possibilities. Besides, since cryptocurrency prices are continuous data and an appropriate discretisation of continuous data is one of the main challenges in constructing BNs for these data, we follow a procedure to create BNs from continuous data and investigate the best possible setting. Eventually, five top-performing BNs are selected for each altcoin based on the highest prediction accuracy and AUC  out of forty-five constructed BNs. Furthermore, we find that the BNs constructed using the discretisation approach consisting of  k-means and two bins generally provide better predictive performance than other discretisation approaches. 

As mentioned in the conclusion, extracting a trading strategy that is suitable for all cryptocurrencies is difficult. In addition, there are several price drivers that need particular attention and investigation. Geopolitical events and major economical disruptions can play an influential role in cryptocurrency markets. For example, China's policy on restricting cryptocurrency trading \citep{shahzad2018empirical}  can impact the volume of the trades due to its economic weight in the world. As another example, the recent tension between Russia and Ukraine not only impacts the energy and food crises all around the world but also potentially creates new opportunities for further utilisation of cryptocurrencies for all sorts of trades. The rise and fall of ISIS is another example of geopolitical events with considerable impact on cryptocurrency price changes. Therefore, further research is necessary to model and quantify the impact of such geopolitical, social, and economic phenomena on cryptocurrencies. 

The study of the impacts of the recent global pandemic is an important topic not only on cryptocurrencies but also on all financial markets  \citep{albulescu2021covid}. The disruption in the supply chain of goods, the limitation on international travel, and the resilience of the healthcare system impacted all aspects of human life. Further investigation is necessary to explore and analyse the effects of the pandemic on the cryptocurrency market in association with the impacts of the pandemic on other financial markets.

Another future research direction is to incorporate expert elicitation besides the BNs learned from data. As the cryptocurrency market is more recent in comparison to traditional financial markets, the combination of expert opinion and learning from data can enrich the influence networks towards extracting interesting trading strategies and decreasing the strong subjectivity of those opinions and the dependency on a large volume of data. In particular, the impacts of influencers on social media \citep{ante2021elon} and therefore on the cryptocurrency market are not straightforward to model, and interactions between the AI techniques and experts can produce more explainable networks.  

Our study in this work concentrates on the external price drivers of cryptocurrencies. However, the study of internal factors such as blockchain information about a particular coin is necessary to understand the inter-dependencies and their impacts on price changes and can reveal hidden interactions towards a better understanding of data. In addition, the investigation of some other external factors including sentiment analysis of social media content and incorporating Google trends would divulge the underlying motives of the price change and extract profitable trading strategies.


%%%%%%%%%%%%%%%%%%%%%%%%%%%%%%%%%%%%%%%%%%



% \bibliography{refs}
\begin{thebibliography}{99}
\expandafter\ifx\csname natexlab\endcsname\relax\def\natexlab#1{#1}\fi
\providecommand{\url}[1]{\texttt{#1}}
\providecommand{\href}[2]{#2}
\providecommand{\path}[1]{#1}
\providecommand{\DOIprefix}{doi:}
\providecommand{\ArXivprefix}{arXiv:}
\providecommand{\URLprefix}{URL: }
\providecommand{\Pubmedprefix}{pmid:}
\providecommand{\doi}[1]{\href{http://dx.doi.org/#1}{\path{#1}}}
\providecommand{\Pubmed}[1]{\href{pmid:#1}{\path{#1}}}
\providecommand{\bibinfo}[2]{#2}
\ifx\xfnm\relax \def\xfnm[#1]{\unskip,\space#1}\fi
%Type = Article
\bibitem[{Abraham et~al.(2018)Abraham, Higdon, Nelson \&
  Ibarra}]{abraham2018cryptocurrency}
\bibinfo{author}{Abraham, J.}, \bibinfo{author}{Higdon, D.},
  \bibinfo{author}{Nelson, J.}, \& \bibinfo{author}{Ibarra, J.}
  (\bibinfo{year}{2018}).
\newblock \bibinfo{title}{Cryptocurrency price prediction using tweet volumes
  and sentiment analysis}.
\newblock {\it \bibinfo{journal}{SMU Data Science Review}\/},  {\it
  \bibinfo{volume}{1}\/}, \bibinfo{pages}{1}.
%Type = Article
\bibitem[{Ahmed et~al.(2020)Ahmed, Seraj \& Islam}]{ahmed2020k}
\bibinfo{author}{Ahmed, M.}, \bibinfo{author}{Seraj, R.}, \&
  \bibinfo{author}{Islam, S. M.~S.} (\bibinfo{year}{2020}).
\newblock \bibinfo{title}{The k-means algorithm: A comprehensive survey and
  performance evaluation}.
\newblock {\it \bibinfo{journal}{Electronics}\/},  {\it \bibinfo{volume}{9}\/},
  \bibinfo{pages}{1295}.
%Type = Article
\bibitem[{Albulescu(2021)}]{albulescu2021covid}
\bibinfo{author}{Albulescu, C.~T.} (\bibinfo{year}{2021}).
\newblock \bibinfo{title}{Covid-19 and the united states financial markets’
  volatility}.
\newblock {\it \bibinfo{journal}{Finance Research Letters}\/},  {\it
  \bibinfo{volume}{38}\/}, \bibinfo{pages}{101699}.
%Type = Article
\bibitem[{Amirkhani et~al.(2016)Amirkhani, Rahmati, Lucas \&
  Hommersom}]{amirkhani2016exploiting}
\bibinfo{author}{Amirkhani, H.}, \bibinfo{author}{Rahmati, M.},
  \bibinfo{author}{Lucas, P.~J.}, \& \bibinfo{author}{Hommersom, A.}
  (\bibinfo{year}{2016}).
\newblock \bibinfo{title}{Exploiting experts’ knowledge for structure
  learning of bayesian networks}.
\newblock {\it \bibinfo{journal}{IEEE transactions on pattern analysis and
  machine intelligence}\/},  {\it \bibinfo{volume}{39}\/},
  \bibinfo{pages}{2154--2170}.
%Type = Article
\bibitem[{Amirzadeh et~al.(2022)Amirzadeh, Nazari \&
  Thiruvady}]{amirzadeh2022applying}
\bibinfo{author}{Amirzadeh, R.}, \bibinfo{author}{Nazari, A.}, \&
  \bibinfo{author}{Thiruvady, D.} (\bibinfo{year}{2022}).
\newblock \bibinfo{title}{Applying artificial intelligence in cryptocurrency
  markets: A survey}.
\newblock {\it \bibinfo{journal}{Algorithms}\/},  {\it \bibinfo{volume}{15}\/},
  \bibinfo{pages}{428}.
%Type = Article
\bibitem[{Ante(2021)}]{ante2021elon}
\bibinfo{author}{Ante, L.} (\bibinfo{year}{2021}).
\newblock \bibinfo{title}{How elon musk's twitter activity moves cryptocurrency
  markets}.
\newblock {\it \bibinfo{journal}{Available at SSRN 3778844}\/}, .
%Type = Article
\bibitem[{Anyfantaki et~al.(2018)Anyfantaki, Arvanitis \&
  Topaloglou}]{anyfantaki2018diversification}
\bibinfo{author}{Anyfantaki, S.}, \bibinfo{author}{Arvanitis, S.}, \&
  \bibinfo{author}{Topaloglou, N.} (\bibinfo{year}{2018}).
\newblock \bibinfo{title}{Diversification, integration and cryptocurrency
  market}, .
%Type = Article
\bibitem[{Aslanidis et~al.(2019)Aslanidis, Bariviera \&
  Mart{\'\i}nez-Iba{\~n}ez}]{aslanidis2019analysis}
\bibinfo{author}{Aslanidis, N.}, \bibinfo{author}{Bariviera, A.~F.}, \&
  \bibinfo{author}{Mart{\'\i}nez-Iba{\~n}ez, O.} (\bibinfo{year}{2019}).
\newblock \bibinfo{title}{An analysis of cryptocurrencies conditional cross
  correlations}.
\newblock {\it \bibinfo{journal}{Finance Research Letters}\/},  {\it
  \bibinfo{volume}{31}\/}, \bibinfo{pages}{130--137}.
%Type = Article
\bibitem[{Asvija et~al.(2021)Asvija, Eswari \& Bijoy}]{asvija2021security}
\bibinfo{author}{Asvija, B.}, \bibinfo{author}{Eswari, R.}, \&
  \bibinfo{author}{Bijoy, M.} (\bibinfo{year}{2021}).
\newblock \bibinfo{title}{Security threat modelling with bayesian networks and
  sensitivity analysis for iaas virtualization stack}.
\newblock {\it \bibinfo{journal}{Journal of Organizational and End User
  Computing (JOEUC)}\/},  {\it \bibinfo{volume}{33}\/},
  \bibinfo{pages}{44--69}.
%Type = Article
\bibitem[{Balcilar et~al.(2021)Balcilar, Ozdemir \&
  Ozdemir}]{balcilar2021dynamic}
\bibinfo{author}{Balcilar, M.}, \bibinfo{author}{Ozdemir, Z.~A.}, \&
  \bibinfo{author}{Ozdemir, H.} (\bibinfo{year}{2021}).
\newblock \bibinfo{title}{Dynamic return and volatility spillovers among s\&p
  500, crude oil, and gold}.
\newblock {\it \bibinfo{journal}{International Journal of Finance \&
  Economics}\/},  {\it \bibinfo{volume}{26}\/}, \bibinfo{pages}{153--170}.
%Type = Article
\bibitem[{BayesFusion(2017)}]{bayesfusion2017genie}
\bibinfo{author}{BayesFusion, L.} (\bibinfo{year}{2017}).
\newblock \bibinfo{title}{Genie modeler}.
\newblock {\it \bibinfo{journal}{User Manual. Available online:
  https://support. bayesfusion. com/docs/(accessed on 21 October 2019)}\/}, .
%Type = Article
\bibitem[{Bouri et~al.(2021)Bouri, Vo \& Saeed}]{bouri2021return}
\bibinfo{author}{Bouri, E.}, \bibinfo{author}{Vo, X.~V.}, \&
  \bibinfo{author}{Saeed, T.} (\bibinfo{year}{2021}).
\newblock \bibinfo{title}{Return equicorrelation in the cryptocurrency market:
  Analysis and determinants}.
\newblock {\it \bibinfo{journal}{Finance Research Letters}\/},  {\it
  \bibinfo{volume}{38}\/}, \bibinfo{pages}{101497}.
%Type = Article
\bibitem[{Castillo et~al.(1997)Castillo, Guti{\'e}rrez \&
  Hadi}]{castillo1997sensitivity}
\bibinfo{author}{Castillo, E.}, \bibinfo{author}{Guti{\'e}rrez, J.~M.}, \&
  \bibinfo{author}{Hadi, A.~S.} (\bibinfo{year}{1997}).
\newblock \bibinfo{title}{Sensitivity analysis in discrete bayesian networks}.
\newblock {\it \bibinfo{journal}{IEEE Transactions on Systems, Man, and
  Cybernetics-Part A: Systems and Humans}\/},  {\it \bibinfo{volume}{27}\/},
  \bibinfo{pages}{412--423}.
%Type = Article
\bibitem[{Charfeddine et~al.(2020)Charfeddine, Benlagha \&
  Maouchi}]{charfeddine2020investigating}
\bibinfo{author}{Charfeddine, L.}, \bibinfo{author}{Benlagha, N.}, \&
  \bibinfo{author}{Maouchi, Y.} (\bibinfo{year}{2020}).
\newblock \bibinfo{title}{Investigating the dynamic relationship between
  cryptocurrencies and conventional assets: Implications for financial
  investors}.
\newblock {\it \bibinfo{journal}{Economic Modelling}\/},  {\it
  \bibinfo{volume}{85}\/}, \bibinfo{pages}{198--217}.
%Type = Article
\bibitem[{Ciaian et~al.(2016)Ciaian, Rajcaniova \& Kancs}]{ciaian2016economics}
\bibinfo{author}{Ciaian, P.}, \bibinfo{author}{Rajcaniova, M.}, \&
  \bibinfo{author}{Kancs, d.} (\bibinfo{year}{2016}).
\newblock \bibinfo{title}{The economics of bitcoin price formation}.
\newblock {\it \bibinfo{journal}{Applied Economics}\/},  {\it
  \bibinfo{volume}{48}\/}, \bibinfo{pages}{1799--1815}.
%Type = Article
\bibitem[{Clark et~al.(2019)Clark, Demirag \& Moosavi}]{clark2019sok}
\bibinfo{author}{Clark, J.}, \bibinfo{author}{Demirag, D.}, \&
  \bibinfo{author}{Moosavi, M.} (\bibinfo{year}{2019}).
\newblock \bibinfo{title}{Sok: demystifying stablecoins}.
\newblock {\it \bibinfo{journal}{Communications of the ACM, Forthcoming}\/}, .
%Type = Article
\bibitem[{Corbet et~al.(2020)Corbet, Hou, Hu, Larkin \& Oxley}]{corbet2020any}
\bibinfo{author}{Corbet, S.}, \bibinfo{author}{Hou, Y.~G.},
  \bibinfo{author}{Hu, Y.}, \bibinfo{author}{Larkin, C.}, \&
  \bibinfo{author}{Oxley, L.} (\bibinfo{year}{2020}).
\newblock \bibinfo{title}{Any port in a storm: Cryptocurrency safe-havens
  during the covid-19 pandemic}.
\newblock {\it \bibinfo{journal}{Economics Letters}\/},  {\it
  \bibinfo{volume}{194}\/}, \bibinfo{pages}{109377}.
%Type = Article
\bibitem[{Corbet et~al.(2018)Corbet, Meegan, Larkin, Lucey \&
  Yarovaya}]{corbet2018exploring}
\bibinfo{author}{Corbet, S.}, \bibinfo{author}{Meegan, A.},
  \bibinfo{author}{Larkin, C.}, \bibinfo{author}{Lucey, B.}, \&
  \bibinfo{author}{Yarovaya, L.} (\bibinfo{year}{2018}).
\newblock \bibinfo{title}{Exploring the dynamic relationships between
  cryptocurrencies and other financial assets}.
\newblock {\it \bibinfo{journal}{Economics Letters}\/},  {\it
  \bibinfo{volume}{165}\/}, \bibinfo{pages}{28--34}.
%Type = Article
\bibitem[{Coup{\'e} et~al.(2000)Coup{\'e}, Jensen, Kj{\ae}rulff \& van~der
  Gaag}]{coupe2000computational}
\bibinfo{author}{Coup{\'e}, V.~M.}, \bibinfo{author}{Jensen, F.~V.},
  \bibinfo{author}{Kj{\ae}rulff, U.}, \& \bibinfo{author}{van~der Gaag, L.~C.}
  (\bibinfo{year}{2000}).
\newblock \bibinfo{title}{A computational architecture for n-way sensitivity
  analysis of bayesian networks}, .
%Type = Article
\bibitem[{Cunningham \& Lawrence(2014)}]{cunningham2014river}
\bibinfo{author}{Cunningham, M.}, \& \bibinfo{author}{Lawrence, D.}
  (\bibinfo{year}{2014}).
\newblock \bibinfo{title}{River murray operations economic benchmarking study},
  .
%Type = Article
\bibitem[{De~Campos \& Ji(2011)}]{de2011efficient}
\bibinfo{author}{De~Campos, C.~P.}, \& \bibinfo{author}{Ji, Q.}
  (\bibinfo{year}{2011}).
\newblock \bibinfo{title}{Efficient structure learning of bayesian networks
  using constraints}.
\newblock {\it \bibinfo{journal}{The Journal of Machine Learning Research}\/},
  {\it \bibinfo{volume}{12}\/}, \bibinfo{pages}{663--689}.
%Type = Article
\bibitem[{Death et~al.(2015)Death, Death, Stubbington, Joy \& van~den
  Belt}]{death2015good}
\bibinfo{author}{Death, R.~G.}, \bibinfo{author}{Death, F.},
  \bibinfo{author}{Stubbington, R.}, \bibinfo{author}{Joy, M.~K.}, \&
  \bibinfo{author}{van~den Belt, M.} (\bibinfo{year}{2015}).
\newblock \bibinfo{title}{How good are bayesian belief networks for
  environmental management? a test with data from an agricultural river
  catchment}.
\newblock {\it \bibinfo{journal}{Freshwater biology}\/},  {\it
  \bibinfo{volume}{60}\/}, \bibinfo{pages}{2297--2309}.
%Type = Article
\bibitem[{Ferreira \& Pereira(2019)}]{ferreira2019contagion}
\bibinfo{author}{Ferreira, P.}, \& \bibinfo{author}{Pereira, {\'E}.}
  (\bibinfo{year}{2019}).
\newblock \bibinfo{title}{Contagion effect in cryptocurrency market}.
\newblock {\it \bibinfo{journal}{Journal of Risk and Financial Management}\/},
  {\it \bibinfo{volume}{12}\/}, \bibinfo{pages}{115}.
%Type = Article
\bibitem[{Gajardo et~al.(2018)Gajardo, Kristjanpoller \&
  Minutolo}]{gajardo2018does}
\bibinfo{author}{Gajardo, G.}, \bibinfo{author}{Kristjanpoller, W.~D.}, \&
  \bibinfo{author}{Minutolo, M.} (\bibinfo{year}{2018}).
\newblock \bibinfo{title}{Does bitcoin exhibit the same asymmetric multifractal
  cross-correlations with crude oil, gold and djia as the euro, great british
  pound and yen?}
\newblock {\it \bibinfo{journal}{Chaos, Solitons \& Fractals}\/},  {\it
  \bibinfo{volume}{109}\/}, \bibinfo{pages}{195--205}.
%Type = Inproceedings
\bibitem[{Gamberger \& Lavra{\v{c}}(1997)}]{gamberger1997conditions}
\bibinfo{author}{Gamberger, D.}, \& \bibinfo{author}{Lavra{\v{c}}, N.}
  (\bibinfo{year}{1997}).
\newblock \bibinfo{title}{Conditions for occam's razor applicability and noise
  elimination}.
\newblock In {\it \bibinfo{booktitle}{European Conference on Machine
  Learning}\/} (pp. \bibinfo{pages}{108--123}).
\newblock \bibinfo{organization}{Springer}.
%Type = Article
\bibitem[{Geuder et~al.(2019)Geuder, Kinateder \&
  Wagner}]{geuder2019cryptocurrencies}
\bibinfo{author}{Geuder, J.}, \bibinfo{author}{Kinateder, H.}, \&
  \bibinfo{author}{Wagner, N.~F.} (\bibinfo{year}{2019}).
\newblock \bibinfo{title}{Cryptocurrencies as financial bubbles: The case of
  bitcoin}.
\newblock {\it \bibinfo{journal}{Finance Research Letters}\/},  {\it
  \bibinfo{volume}{31}\/}.
%Type = Article
\bibitem[{Gheisari \& Meybodi(2016)}]{gheisari2016bnc}
\bibinfo{author}{Gheisari, S.}, \& \bibinfo{author}{Meybodi, M.~R.}
  (\bibinfo{year}{2016}).
\newblock \bibinfo{title}{Bnc-pso: structure learning of bayesian networks by
  particle swarm optimization}.
\newblock {\it \bibinfo{journal}{Information Sciences}\/},  {\it
  \bibinfo{volume}{348}\/}, \bibinfo{pages}{272--289}.
%Type = Article
\bibitem[{Giudici \& Polinesi(2021)}]{giudici2021crypto}
\bibinfo{author}{Giudici, P.}, \& \bibinfo{author}{Polinesi, G.}
  (\bibinfo{year}{2021}).
\newblock \bibinfo{title}{Crypto price discovery through correlation networks}.
\newblock {\it \bibinfo{journal}{Annals of Operations Research}\/},  {\it
  \bibinfo{volume}{299}\/}, \bibinfo{pages}{443--457}.
%Type = Article
\bibitem[{Guesmi et~al.(2019)Guesmi, Saadi, Abid \&
  Ftiti}]{guesmi2019portfolio}
\bibinfo{author}{Guesmi, K.}, \bibinfo{author}{Saadi, S.},
  \bibinfo{author}{Abid, I.}, \& \bibinfo{author}{Ftiti, Z.}
  (\bibinfo{year}{2019}).
\newblock \bibinfo{title}{Portfolio diversification with virtual currency:
  Evidence from bitcoin}.
\newblock {\it \bibinfo{journal}{International Review of Financial
  Analysis}\/},  {\it \bibinfo{volume}{63}\/}, \bibinfo{pages}{431--437}.
%Type = Article
\bibitem[{Gurdgiev \& O’Loughlin(2020)}]{gurdgiev2020herding}
\bibinfo{author}{Gurdgiev, C.}, \& \bibinfo{author}{O’Loughlin, D.}
  (\bibinfo{year}{2020}).
\newblock \bibinfo{title}{Herding and anchoring in cryptocurrency markets:
  Investor reaction to fear and uncertainty}.
\newblock {\it \bibinfo{journal}{Journal of Behavioral and Experimental
  Finance}\/},  {\it \bibinfo{volume}{25}\/}, \bibinfo{pages}{100271}.
%Type = Article
\bibitem[{Hasan et~al.(2022)Hasan, Hassan, Karim \&
  Rashid}]{hasan2022exploring}
\bibinfo{author}{Hasan, M.~B.}, \bibinfo{author}{Hassan, M.~K.},
  \bibinfo{author}{Karim, Z.~A.}, \& \bibinfo{author}{Rashid, M.~M.}
  (\bibinfo{year}{2022}).
\newblock \bibinfo{title}{Exploring the hedge and safe haven properties of
  cryptocurrency in policy uncertainty}.
\newblock {\it \bibinfo{journal}{Finance Research Letters}\/},  {\it
  \bibinfo{volume}{46}\/}, \bibinfo{pages}{102272}.
%Type = Article
\bibitem[{Hayes(2017)}]{hayes2017cryptocurrency}
\bibinfo{author}{Hayes, A.~S.} (\bibinfo{year}{2017}).
\newblock \bibinfo{title}{Cryptocurrency value formation: An empirical study
  leading to a cost of production model for valuing bitcoin}.
\newblock {\it \bibinfo{journal}{Telematics and informatics}\/},  {\it
  \bibinfo{volume}{34}\/}, \bibinfo{pages}{1308--1321}.
%Type = Article
\bibitem[{Heckerman(2008)}]{heckerman2008tutorial}
\bibinfo{author}{Heckerman, D.} (\bibinfo{year}{2008}).
\newblock \bibinfo{title}{A tutorial on learning with bayesian networks}.
\newblock {\it \bibinfo{journal}{Innovations in Bayesian networks}\/},  (pp.
  \bibinfo{pages}{33--82}).
%Type = Article
\bibitem[{Hosseini \& Ivanov(2022)}]{hosseini2022multi}
\bibinfo{author}{Hosseini, S.}, \& \bibinfo{author}{Ivanov, D.}
  (\bibinfo{year}{2022}).
\newblock \bibinfo{title}{A multi-layer bayesian network method for supply
  chain disruption modelling in the wake of the covid-19 pandemic}.
\newblock {\it \bibinfo{journal}{International Journal of Production
  Research}\/},  {\it \bibinfo{volume}{60}\/}, \bibinfo{pages}{5258--5276}.
%Type = Article
\bibitem[{Hosseini \& Sarder(2019)}]{hosseini2019development}
\bibinfo{author}{Hosseini, S.}, \& \bibinfo{author}{Sarder, M.}
  (\bibinfo{year}{2019}).
\newblock \bibinfo{title}{Development of a bayesian network model for optimal
  site selection of electric vehicle charging station}.
\newblock {\it \bibinfo{journal}{International Journal of Electrical Power \&
  Energy Systems}\/},  {\it \bibinfo{volume}{105}\/},
  \bibinfo{pages}{110--122}.
%Type = Article
\bibitem[{{\.I}{\c{c}}ellio{\u{g}}lu \&
  {\"O}ner(2019)}]{iccelliouglu2019investigation}
\bibinfo{author}{{\.I}{\c{c}}ellio{\u{g}}lu, C.~{\c{S}}.}, \&
  \bibinfo{author}{{\"O}ner, S.} (\bibinfo{year}{2019}).
\newblock \bibinfo{title}{An investigation on the volatility of
  cryptocurrencies by means of heterogeneous panel data analysis}.
\newblock {\it \bibinfo{journal}{Procedia Computer Science}\/},  {\it
  \bibinfo{volume}{158}\/}, \bibinfo{pages}{913--920}.
%Type = Inproceedings
\bibitem[{Inamdar et~al.(2019)Inamdar, Bhagtani, Bhatt \&
  Shetty}]{inamdar2019predicting}
\bibinfo{author}{Inamdar, A.}, \bibinfo{author}{Bhagtani, A.},
  \bibinfo{author}{Bhatt, S.}, \& \bibinfo{author}{Shetty, P.~M.}
  (\bibinfo{year}{2019}).
\newblock \bibinfo{title}{Predicting cryptocurrency value using sentiment
  analysis}.
\newblock In {\it \bibinfo{booktitle}{2019 International Conference on
  Intelligent Computing and Control Systems (ICCS)}\/} (pp.
  \bibinfo{pages}{932--934}).
\newblock \bibinfo{organization}{IEEE}.
%Type = Article
\bibitem[{Ji et~al.(2018)Ji, Bouri, Gupta \& Roubaud}]{ji2018network}
\bibinfo{author}{Ji, Q.}, \bibinfo{author}{Bouri, E.}, \bibinfo{author}{Gupta,
  R.}, \& \bibinfo{author}{Roubaud, D.} (\bibinfo{year}{2018}).
\newblock \bibinfo{title}{Network causality structures among bitcoin and other
  financial assets: A directed acyclic graph approach}.
\newblock {\it \bibinfo{journal}{The Quarterly Review of Economics and
  Finance}\/},  {\it \bibinfo{volume}{70}\/}, \bibinfo{pages}{203--213}.
%Type = Article
\bibitem[{Ji et~al.(2019)Ji, Bouri, Lau \& Roubaud}]{ji2019dynamic}
\bibinfo{author}{Ji, Q.}, \bibinfo{author}{Bouri, E.}, \bibinfo{author}{Lau, C.
  K.~M.}, \& \bibinfo{author}{Roubaud, D.} (\bibinfo{year}{2019}).
\newblock \bibinfo{title}{Dynamic connectedness and integration in
  cryptocurrency markets}.
\newblock {\it \bibinfo{journal}{International Review of Financial
  Analysis}\/},  {\it \bibinfo{volume}{63}\/}, \bibinfo{pages}{257--272}.
%Type = Article
\bibitem[{Jiang et~al.(2021)Jiang, Wang, Ma \& Yang}]{jiang2021credit}
\bibinfo{author}{Jiang, Y.}, \bibinfo{author}{Wang, G.-J.},
  \bibinfo{author}{Ma, C.}, \& \bibinfo{author}{Yang, X.}
  (\bibinfo{year}{2021}).
\newblock \bibinfo{title}{Do credit conditions matter for the impact of oil
  price shocks on stock returns? evidence from a structural threshold var
  model}.
\newblock {\it \bibinfo{journal}{International Review of Economics \&
  Finance}\/},  {\it \bibinfo{volume}{72}\/}, \bibinfo{pages}{1--15}.
%Type = Article
\bibitem[{Kareem \& Okur(2020)}]{kareem2020structure}
\bibinfo{author}{Kareem, S.~W.}, \& \bibinfo{author}{Okur, M.~C.}
  (\bibinfo{year}{2020}).
\newblock \bibinfo{title}{Structure learning of bayesian networks using
  elephant swarm water search algorithm}.
\newblock {\it \bibinfo{journal}{International Journal of Swarm Intelligence
  Research (IJSIR)}\/},  {\it \bibinfo{volume}{11}\/}, \bibinfo{pages}{19--30}.
%Type = Article
\bibitem[{Kareem \& Okur(2021)}]{kareem2021falcon}
\bibinfo{author}{Kareem, S.~W.}, \& \bibinfo{author}{Okur, M.~C.}
  (\bibinfo{year}{2021}).
\newblock \bibinfo{title}{Falcon optimization algorithm for bayesian networks
  structure learning}.
\newblock {\it \bibinfo{journal}{Computer Science}\/},  {\it
  \bibinfo{volume}{22}\/}.
%Type = Article
\bibitem[{Kim et~al.(2021)Kim, Lee \& Assar}]{kim2021dynamics}
\bibinfo{author}{Kim, K.}, \bibinfo{author}{Lee, S.-Y.~T.}, \&
  \bibinfo{author}{Assar, S.} (\bibinfo{year}{2021}).
\newblock \bibinfo{title}{The dynamics of cryptocurrency market behavior:
  sentiment analysis using markov chains}.
\newblock {\it \bibinfo{journal}{Industrial Management \& Data Systems}\/}, .
%Type = Article
\bibitem[{Kim \& Sun(2017)}]{kim2017dynamic}
\bibinfo{author}{Kim, M.~H.}, \& \bibinfo{author}{Sun, L.}
  (\bibinfo{year}{2017}).
\newblock \bibinfo{title}{Dynamic conditional correlations between chinese
  sector returns and the s\&p 500 index: An interpretation based on investment
  shocks}.
\newblock {\it \bibinfo{journal}{International Review of Economics \&
  Finance}\/},  {\it \bibinfo{volume}{48}\/}, \bibinfo{pages}{309--325}.
%Type = Misc
\bibitem[{Koller \& Friedman(2009)}]{koller2009probabilistic}
\bibinfo{author}{Koller, D.}, \& \bibinfo{author}{Friedman, N.}
  (\bibinfo{year}{2009}).
\newblock \bibinfo{title}{Probabilistic graphical models, massachusetts}.
%Type = Article
\bibitem[{K{\"o}nigstorfer \& Thalmann(2020)}]{konigstorfer2020applications}
\bibinfo{author}{K{\"o}nigstorfer, F.}, \& \bibinfo{author}{Thalmann, S.}
  (\bibinfo{year}{2020}).
\newblock \bibinfo{title}{Applications of artificial intelligence in commercial
  banks--a research agenda for behavioral finance}.
\newblock {\it \bibinfo{journal}{Journal of behavioral and experimental
  finance}\/},  {\it \bibinfo{volume}{27}\/}, \bibinfo{pages}{100352}.
%Type = Article
\bibitem[{Kraaijeveld \& De~Smedt(2020)}]{kraaijeveld2020predictive}
\bibinfo{author}{Kraaijeveld, O.}, \& \bibinfo{author}{De~Smedt, J.}
  (\bibinfo{year}{2020}).
\newblock \bibinfo{title}{The predictive power of public twitter sentiment for
  forecasting cryptocurrency prices}.
\newblock {\it \bibinfo{journal}{Journal of International Financial Markets,
  Institutions and Money}\/},  {\it \bibinfo{volume}{65}\/},
  \bibinfo{pages}{101188}.
%Type = Article
\bibitem[{Kurihara \& Fukushima(2017)}]{kurihara2017market}
\bibinfo{author}{Kurihara, Y.}, \& \bibinfo{author}{Fukushima, A.}
  (\bibinfo{year}{2017}).
\newblock \bibinfo{title}{The market efficiency of bitcoin: a weekly anomaly
  perspective}.
\newblock {\it \bibinfo{journal}{Journal of Applied Finance and Banking}\/},
  {\it \bibinfo{volume}{7}\/}, \bibinfo{pages}{57}.
%Type = Article
\bibitem[{Lamon et~al.(2017)Lamon, Nielsen \&
  Redondo}]{lamon2017cryptocurrency}
\bibinfo{author}{Lamon, C.}, \bibinfo{author}{Nielsen, E.}, \&
  \bibinfo{author}{Redondo, E.} (\bibinfo{year}{2017}).
\newblock \bibinfo{title}{Cryptocurrency price prediction using news and social
  media sentiment}.
\newblock {\it \bibinfo{journal}{SMU Data Sci. Rev}\/},  {\it
  \bibinfo{volume}{1}\/}, \bibinfo{pages}{1--22}.
%Type = Article
\bibitem[{Larra{\~n}aga et~al.(2012)Larra{\~n}aga, Karshenas, Bielza \&
  Santana}]{larranaga2012review}
\bibinfo{author}{Larra{\~n}aga, P.}, \bibinfo{author}{Karshenas, H.},
  \bibinfo{author}{Bielza, C.}, \& \bibinfo{author}{Santana, R.}
  (\bibinfo{year}{2012}).
\newblock \bibinfo{title}{A review on probabilistic graphical models in
  evolutionary computation}.
\newblock {\it \bibinfo{journal}{Journal of Heuristics}\/},  {\it
  \bibinfo{volume}{18}\/}, \bibinfo{pages}{795--819}.
%Type = Article
\bibitem[{Lau et~al.(2017)Lau, Vigne, Wang \& Yarovaya}]{lau2017return}
\bibinfo{author}{Lau, M. C.~K.}, \bibinfo{author}{Vigne, S.~A.},
  \bibinfo{author}{Wang, S.}, \& \bibinfo{author}{Yarovaya, L.}
  (\bibinfo{year}{2017}).
\newblock \bibinfo{title}{Return spillovers between white precious metal etfs:
  The role of oil, gold, and global equity}.
\newblock {\it \bibinfo{journal}{International Review of Financial
  Analysis}\/},  {\it \bibinfo{volume}{52}\/}, \bibinfo{pages}{316--332}.
%Type = Article
\bibitem[{Lee \& Liu(2010)}]{lee2010analysis}
\bibinfo{author}{Lee, D.~S.}, \& \bibinfo{author}{Liu, H.}
  (\bibinfo{year}{2010}).
\newblock \bibinfo{title}{An analysis of several factors affecting the us
  dollar index}, .
%Type = Article
\bibitem[{Li et~al.(2020)Li, Wang, Wang, Shao \& He}]{li2020risk}
\bibinfo{author}{Li, M.}, \bibinfo{author}{Wang, H.}, \bibinfo{author}{Wang,
  D.}, \bibinfo{author}{Shao, Z.}, \& \bibinfo{author}{He, S.}
  (\bibinfo{year}{2020}).
\newblock \bibinfo{title}{Risk assessment of gas explosion in coal mines based
  on fuzzy ahp and bayesian network}.
\newblock {\it \bibinfo{journal}{Process Safety and Environmental
  Protection}\/},  {\it \bibinfo{volume}{135}\/}, \bibinfo{pages}{207--218}.
%Type = Article
\bibitem[{Lin(2021)}]{lin2021investor}
\bibinfo{author}{Lin, Z.-Y.} (\bibinfo{year}{2021}).
\newblock \bibinfo{title}{Investor attention and cryptocurrency performance}.
\newblock {\it \bibinfo{journal}{Finance Research Letters}\/},  {\it
  \bibinfo{volume}{40}\/}, \bibinfo{pages}{101702}.
%Type = Article
\bibitem[{Lu et~al.(2020)Lu, Zhong, Xu, Zhu, Ma, Wang \& Xu}]{lu2020risk}
\bibinfo{author}{Lu, Q.}, \bibinfo{author}{Zhong, P.-a.}, \bibinfo{author}{Xu,
  B.}, \bibinfo{author}{Zhu, F.}, \bibinfo{author}{Ma, Y.},
  \bibinfo{author}{Wang, H.}, \& \bibinfo{author}{Xu, S.}
  (\bibinfo{year}{2020}).
\newblock \bibinfo{title}{Risk analysis for reservoir flood control operation
  considering two-dimensional uncertainties based on bayesian network}.
\newblock {\it \bibinfo{journal}{Journal of Hydrology}\/},  {\it
  \bibinfo{volume}{589}\/}, \bibinfo{pages}{125353}.
%Type = Article
\bibitem[{Lu \& Zhou(2015)}]{lu2015exploration}
\bibinfo{author}{Lu, S.}, \& \bibinfo{author}{Zhou, G.} (\bibinfo{year}{2015}).
\newblock \bibinfo{title}{An exploration study on quality performance casual
  path model based on bn method}.
\newblock {\it \bibinfo{journal}{Metallurgical \& Mining Industry}\/}, .
%Type = Article
\bibitem[{Lund et~al.(2018)Lund, Cohen \& Scarles}]{lund2018power}
\bibinfo{author}{Lund, N.~F.}, \bibinfo{author}{Cohen, S.~A.}, \&
  \bibinfo{author}{Scarles, C.} (\bibinfo{year}{2018}).
\newblock \bibinfo{title}{The power of social media storytelling in destination
  branding}.
\newblock {\it \bibinfo{journal}{Journal of destination marketing \&
  management}\/},  {\it \bibinfo{volume}{8}\/}, \bibinfo{pages}{271--280}.
%Type = Inproceedings
\bibitem[{Mahjoub \& Kalti(2011)}]{mahjoub2011software}
\bibinfo{author}{Mahjoub, M.~A.}, \& \bibinfo{author}{Kalti, K.}
  (\bibinfo{year}{2011}).
\newblock \bibinfo{title}{Software comparison dealing with bayesian networks}.
\newblock In {\it \bibinfo{booktitle}{International Symposium on Neural
  Networks}\/} (pp. \bibinfo{pages}{168--177}).
\newblock \bibinfo{organization}{Springer}.
%Type = Article
\bibitem[{Mai et~al.(2018)Mai, Shan, Bai, Wang \& Chiang}]{mai2018does}
\bibinfo{author}{Mai, F.}, \bibinfo{author}{Shan, Z.}, \bibinfo{author}{Bai,
  Q.}, \bibinfo{author}{Wang, X.}, \& \bibinfo{author}{Chiang, R.~H.}
  (\bibinfo{year}{2018}).
\newblock \bibinfo{title}{How does social media impact bitcoin value? a test of
  the silent majority hypothesis}.
\newblock {\it \bibinfo{journal}{Journal of management information systems}\/},
   {\it \bibinfo{volume}{35}\/}, \bibinfo{pages}{19--52}.
%Type = Article
\bibitem[{Malkiel(2003)}]{malkiel2003efficient}
\bibinfo{author}{Malkiel, B.~G.} (\bibinfo{year}{2003}).
\newblock \bibinfo{title}{The efficient market hypothesis and its critics}.
\newblock {\it \bibinfo{journal}{Journal of economic perspectives}\/},  {\it
  \bibinfo{volume}{17}\/}, \bibinfo{pages}{59--82}.
%Type = Article
\bibitem[{Malladi \& Dheeriya(2021)}]{malladi2021time}
\bibinfo{author}{Malladi, R.~K.}, \& \bibinfo{author}{Dheeriya, P.~L.}
  (\bibinfo{year}{2021}).
\newblock \bibinfo{title}{Time series analysis of cryptocurrency returns and
  volatilities}.
\newblock {\it \bibinfo{journal}{Journal of Economics and Finance}\/},  {\it
  \bibinfo{volume}{45}\/}, \bibinfo{pages}{75--94}.
%Type = Article
\bibitem[{Meurisse et~al.(2022)Meurisse, Marcot, Woodberry, Barratt \&
  Todd}]{meurisse2022risk}
\bibinfo{author}{Meurisse, N.}, \bibinfo{author}{Marcot, B.~G.},
  \bibinfo{author}{Woodberry, O.}, \bibinfo{author}{Barratt, B.~I.}, \&
  \bibinfo{author}{Todd, J.~H.} (\bibinfo{year}{2022}).
\newblock \bibinfo{title}{Risk analysis frameworks used in biological control
  and introduction of a novel bayesian network tool}.
\newblock {\it \bibinfo{journal}{Risk Analysis}\/},  {\it
  \bibinfo{volume}{42}\/}, \bibinfo{pages}{1255--1276}.
%Type = Inproceedings
\bibitem[{Miletic et~al.(2013)Miletic, Korenak \&
  Ivanis}]{miletic642013performance}
\bibinfo{author}{Miletic, S.}, \bibinfo{author}{Korenak, B.}, \&
  \bibinfo{author}{Ivanis, I.} (\bibinfo{year}{2013}).
\newblock \bibinfo{title}{Performance of msci world index during the global
  financial crisis: Value-at-risk approach}.
\newblock In {\it \bibinfo{booktitle}{Employment, Education and
  Entrepreneurship (EEE 2013) Conference Belgrade-Serbia}\/} (pp.
  \bibinfo{pages}{419--436}).
%Type = Inproceedings
\bibitem[{Min(2009)}]{min2009global}
\bibinfo{author}{Min, H.} (\bibinfo{year}{2009}).
\newblock \bibinfo{title}{A global discretization and attribute reduction
  algorithm based on k-means clustering and rough sets theory}.
\newblock In {\it \bibinfo{booktitle}{2009 Second international symposium on
  knowledge acquisition and modeling}\/} (pp. \bibinfo{pages}{92--95}).
\newblock \bibinfo{organization}{IEEE} volume~\bibinfo{volume}{2}.
%Type = Inproceedings
\bibitem[{Mita et~al.(2019)Mita, Ito, Ohsawa \& Tanaka}]{mita2019stablecoin}
\bibinfo{author}{Mita, M.}, \bibinfo{author}{Ito, K.}, \bibinfo{author}{Ohsawa,
  S.}, \& \bibinfo{author}{Tanaka, H.} (\bibinfo{year}{2019}).
\newblock \bibinfo{title}{What is stablecoin?: A survey on price stabilization
  mechanisms for decentralized payment systems}.
\newblock In {\it \bibinfo{booktitle}{2019 8th International Congress on
  Advanced Applied Informatics (IIAI-AAI)}\/} (pp. \bibinfo{pages}{60--66}).
\newblock \bibinfo{organization}{IEEE}.
%Type = Article
\bibitem[{Mueller(2020)}]{mueller2020cryptocurrency}
\bibinfo{author}{Mueller, P.} (\bibinfo{year}{2020}).
\newblock \bibinfo{title}{Cryptocurrency mining: asymmetric response to price
  movement}.
\newblock {\it \bibinfo{journal}{Available at SSRN 3733026}\/}, .
%Type = Article
\bibitem[{Nakamoto \& Bitcoin(2008)}]{nakamoto2008peer}
\bibinfo{author}{Nakamoto, S.}, \& \bibinfo{author}{Bitcoin, A.}
  (\bibinfo{year}{2008}).
\newblock \bibinfo{title}{A peer-to-peer electronic cash system}.
\newblock {\it \bibinfo{journal}{Bitcoin.--URL: https://bitcoin. org/bitcoin.
  pdf}\/},  {\it \bibinfo{volume}{4}\/}.
%Type = Article
\bibitem[{Nojavan et~al.(2017)Nojavan, Qian \& Stow}]{nojavan2017comparative}
\bibinfo{author}{Nojavan, F.}, \bibinfo{author}{Qian, S.~S.}, \&
  \bibinfo{author}{Stow, C.~A.} (\bibinfo{year}{2017}).
\newblock \bibinfo{title}{Comparative analysis of discretization methods in
  bayesian networks}.
\newblock {\it \bibinfo{journal}{Environmental Modelling \& Software}\/},  {\it
  \bibinfo{volume}{87}\/}, \bibinfo{pages}{64--71}.
%Type = Article
\bibitem[{Okorie \& Lin(2020)}]{okorie2020crude}
\bibinfo{author}{Okorie, D.~I.}, \& \bibinfo{author}{Lin, B.}
  (\bibinfo{year}{2020}).
\newblock \bibinfo{title}{Crude oil price and cryptocurrencies: evidence of
  volatility connectedness and hedging strategy}.
\newblock {\it \bibinfo{journal}{Energy economics}\/},  {\it
  \bibinfo{volume}{87}\/}, \bibinfo{pages}{104703}.
%Type = Article
\bibitem[{Ozer \& Sakar(2022)}]{ozer2022automated}
\bibinfo{author}{Ozer, F.}, \& \bibinfo{author}{Sakar, C.~O.}
  (\bibinfo{year}{2022}).
\newblock \bibinfo{title}{An automated cryptocurrency trading system based on
  the detection of unusual price movements with a time-series clustering-based
  approach}.
\newblock {\it \bibinfo{journal}{Expert Systems with Applications}\/},  {\it
  \bibinfo{volume}{200}\/}, \bibinfo{pages}{117017}.
%Type = Article
\bibitem[{Parikh(2019)}]{parikh2019emerging}
\bibinfo{author}{Parikh, H.} (\bibinfo{year}{2019}).
\newblock \bibinfo{title}{Emerging market equity benchmarks for japanese
  investors: countries, sectors or styles?}
\newblock {\it \bibinfo{journal}{Journal of Asset Management}\/},  {\it
  \bibinfo{volume}{20}\/}, \bibinfo{pages}{289--300}.
%Type = Article
\bibitem[{Poyser(2019)}]{poyser2019exploring}
\bibinfo{author}{Poyser, O.} (\bibinfo{year}{2019}).
\newblock \bibinfo{title}{Exploring the dynamics of bitcoin’s price: a
  bayesian structural time series approach}.
\newblock {\it \bibinfo{journal}{Eurasian Economic Review}\/},  {\it
  \bibinfo{volume}{9}\/}, \bibinfo{pages}{29--60}.
%Type = Article
\bibitem[{Ricciardi \& Simon(2000)}]{ricciardi2000behavioral}
\bibinfo{author}{Ricciardi, V.}, \& \bibinfo{author}{Simon, H.~K.}
  (\bibinfo{year}{2000}).
\newblock \bibinfo{title}{What is behavioral finance?}
\newblock {\it \bibinfo{journal}{Business, Education \& Technology Journal}\/},
   {\it \bibinfo{volume}{2}\/}, \bibinfo{pages}{1--9}.
%Type = Article
\bibitem[{Rossi et~al.(2018)Rossi, Gunardi et~al.}]{rossi2018efficient}
\bibinfo{author}{Rossi, M.}, \bibinfo{author}{Gunardi, A.} et~al.
  (\bibinfo{year}{2018}).
\newblock \bibinfo{title}{Efficient market hypothesis and stock market
  anomalies: Empirical evidence in four european countries}.
\newblock {\it \bibinfo{journal}{Journal of Applied Business Research
  (JABR)}\/},  {\it \bibinfo{volume}{34}\/}, \bibinfo{pages}{183--192}.
%Type = Article
\bibitem[{Rouhani \& Abedin(2019)}]{rouhani2019crypto}
\bibinfo{author}{Rouhani, S.}, \& \bibinfo{author}{Abedin, E.}
  (\bibinfo{year}{2019}).
\newblock \bibinfo{title}{Crypto-currencies narrated on tweets: a sentiment
  analysis approach}.
\newblock {\it \bibinfo{journal}{International Journal of Ethics and
  Systems}\/}, .
%Type = Article
\bibitem[{Sabry et~al.(2020)Sabry, Labda, Erbad \&
  Malluhi}]{sabry2020cryptocurrencies}
\bibinfo{author}{Sabry, F.}, \bibinfo{author}{Labda, W.},
  \bibinfo{author}{Erbad, A.}, \& \bibinfo{author}{Malluhi, Q.}
  (\bibinfo{year}{2020}).
\newblock \bibinfo{title}{Cryptocurrencies and artificial intelligence:
  Challenges and opportunities}.
\newblock {\it \bibinfo{journal}{IEEE Access}\/},  {\it \bibinfo{volume}{8}\/},
  \bibinfo{pages}{175840--175858}.
%Type = Inproceedings
\bibitem[{Saleh et~al.(2018)Saleh, Puspita, Sanjaya
  et~al.}]{saleh2018implementation}
\bibinfo{author}{Saleh, A.}, \bibinfo{author}{Puspita, K.},
  \bibinfo{author}{Sanjaya, A.} et~al. (\bibinfo{year}{2018}).
\newblock \bibinfo{title}{Implementation of equal width interval discretization
  on smarter method for selecting computer laboratory assistant}.
\newblock In {\it \bibinfo{booktitle}{2018 6th International Conference on
  Cyber and IT Service Management (CITSM)}\/} (pp. \bibinfo{pages}{1--4}).
\newblock \bibinfo{organization}{IEEE}.
%Type = Article
\bibitem[{Saltelli et~al.(2019)Saltelli, Aleksankina, Becker, Fennell,
  Ferretti, Holst, Li \& Wu}]{saltelli2019so}
\bibinfo{author}{Saltelli, A.}, \bibinfo{author}{Aleksankina, K.},
  \bibinfo{author}{Becker, W.}, \bibinfo{author}{Fennell, P.},
  \bibinfo{author}{Ferretti, F.}, \bibinfo{author}{Holst, N.},
  \bibinfo{author}{Li, S.}, \& \bibinfo{author}{Wu, Q.} (\bibinfo{year}{2019}).
\newblock \bibinfo{title}{Why so many published sensitivity analyses are false:
  A systematic review of sensitivity analysis practices}.
\newblock {\it \bibinfo{journal}{Environmental modelling \& software}\/},  {\it
  \bibinfo{volume}{114}\/}, \bibinfo{pages}{29--39}.
%Type = Article
\bibitem[{Sevinc et~al.(2020)Sevinc, Kucuk \& Goltas}]{sevinc2020bayesian}
\bibinfo{author}{Sevinc, V.}, \bibinfo{author}{Kucuk, O.}, \&
  \bibinfo{author}{Goltas, M.} (\bibinfo{year}{2020}).
\newblock \bibinfo{title}{A bayesian network model for prediction and analysis
  of possible forest fire causes}.
\newblock {\it \bibinfo{journal}{Forest Ecology and Management}\/},  {\it
  \bibinfo{volume}{457}\/}, \bibinfo{pages}{117723}.
%Type = Article
\bibitem[{Shahzad et~al.(2018)Shahzad, Xiu, Wang \&
  Shahbaz}]{shahzad2018empirical}
\bibinfo{author}{Shahzad, F.}, \bibinfo{author}{Xiu, G.},
  \bibinfo{author}{Wang, J.}, \& \bibinfo{author}{Shahbaz, M.}
  (\bibinfo{year}{2018}).
\newblock \bibinfo{title}{An empirical investigation on the adoption of
  cryptocurrencies among the people of mainland china}.
\newblock {\it \bibinfo{journal}{Technology in Society}\/},  {\it
  \bibinfo{volume}{55}\/}, \bibinfo{pages}{33--40}.
%Type = Article
\bibitem[{Shi et~al.(2020)Shi, Tiwari, Gozgor \& Lu}]{shi2020correlations}
\bibinfo{author}{Shi, Y.}, \bibinfo{author}{Tiwari, A.~K.},
  \bibinfo{author}{Gozgor, G.}, \& \bibinfo{author}{Lu, Z.}
  (\bibinfo{year}{2020}).
\newblock \bibinfo{title}{Correlations among cryptocurrencies: Evidence from
  multivariate factor stochastic volatility model}.
\newblock {\it \bibinfo{journal}{Research in International Business and
  Finance}\/},  {\it \bibinfo{volume}{53}\/}, \bibinfo{pages}{101231}.
%Type = Article
\bibitem[{Shiller(2017)}]{shiller2017narrative}
\bibinfo{author}{Shiller, R.~J.} (\bibinfo{year}{2017}).
\newblock \bibinfo{title}{Narrative economics}.
\newblock {\it \bibinfo{journal}{American economic review}\/},  {\it
  \bibinfo{volume}{107}\/}, \bibinfo{pages}{967--1004}.
%Type = Article
\bibitem[{Sidebotham(2020)}]{sidebotham2020most}
\bibinfo{author}{Sidebotham, D.} (\bibinfo{year}{2020}).
\newblock \bibinfo{title}{Are most randomised trials in anaesthesia and
  critical care wrong? an analysis using bayes’ theorem}.
\newblock {\it \bibinfo{journal}{Anaesthesia}\/},  {\it
  \bibinfo{volume}{75}\/}, \bibinfo{pages}{1386--1393}.
%Type = Article
\bibitem[{Sinaga \& Yang(2020)}]{sinaga2020unsupervised}
\bibinfo{author}{Sinaga, K.~P.}, \& \bibinfo{author}{Yang, M.-S.}
  (\bibinfo{year}{2020}).
\newblock \bibinfo{title}{Unsupervised k-means clustering algorithm}.
\newblock {\it \bibinfo{journal}{IEEE access}\/},  {\it \bibinfo{volume}{8}\/},
  \bibinfo{pages}{80716--80727}.
%Type = Article
\bibitem[{Smales(2022)}]{smales2022investor}
\bibinfo{author}{Smales, L.~A.} (\bibinfo{year}{2022}).
\newblock \bibinfo{title}{Investor attention in cryptocurrency markets}.
\newblock {\it \bibinfo{journal}{International Review of Financial
  Analysis}\/},  {\it \bibinfo{volume}{79}\/}, \bibinfo{pages}{101972}.
%Type = Article
\bibitem[{Smuts(2019)}]{smuts2019drives}
\bibinfo{author}{Smuts, N.} (\bibinfo{year}{2019}).
\newblock \bibinfo{title}{What drives cryptocurrency prices? an investigation
  of google trends and telegram sentiment}.
\newblock {\it \bibinfo{journal}{ACM SIGMETRICS Performance Evaluation
  Review}\/},  {\it \bibinfo{volume}{46}\/}, \bibinfo{pages}{131--134}.
%Type = Article
\bibitem[{Stosic et~al.(2018)Stosic, Stosic, Ludermir \&
  Stosic}]{stosic2018collective}
\bibinfo{author}{Stosic, D.}, \bibinfo{author}{Stosic, D.},
  \bibinfo{author}{Ludermir, T.~B.}, \& \bibinfo{author}{Stosic, T.}
  (\bibinfo{year}{2018}).
\newblock \bibinfo{title}{Collective behavior of cryptocurrency price changes}.
\newblock {\it \bibinfo{journal}{Physica A: Statistical Mechanics and its
  Applications}\/},  {\it \bibinfo{volume}{507}\/}, \bibinfo{pages}{499--509}.
%Type = Inproceedings
\bibitem[{Teker et~al.(2019)Teker, Teker \& Ozyesil}]{teker2019determinants}
\bibinfo{author}{Teker, D.}, \bibinfo{author}{Teker, S.}, \&
  \bibinfo{author}{Ozyesil, M.} (\bibinfo{year}{2019}).
\newblock \bibinfo{title}{Determinants of cryptocurrency price movements}.
\newblock In {\it \bibinfo{booktitle}{14th Paris international conference on
  marketing, economics, education and interdisciplinary studies, MEEIS-19}\/}
  (pp. \bibinfo{pages}{12--14}).
%Type = Article
\bibitem[{Uusitalo(2007)}]{uusitalo2007advantages}
\bibinfo{author}{Uusitalo, L.} (\bibinfo{year}{2007}).
\newblock \bibinfo{title}{Advantages and challenges of bayesian networks in
  environmental modelling}.
\newblock {\it \bibinfo{journal}{Ecological modelling}\/},  {\it
  \bibinfo{volume}{203}\/}, \bibinfo{pages}{312--318}.
%Type = Article
\bibitem[{Walker \& Fox-Rushby(2001)}]{walker2001allowing}
\bibinfo{author}{Walker, D.}, \& \bibinfo{author}{Fox-Rushby, J.}
  (\bibinfo{year}{2001}).
\newblock \bibinfo{title}{Allowing for uncertainty in economic evaluations:
  qualitative sensitivity analysis}.
\newblock {\it \bibinfo{journal}{Health Policy and Planning}\/},  {\it
  \bibinfo{volume}{16}\/}, \bibinfo{pages}{435--443}.
%Type = Article
\bibitem[{Wang et~al.(2019)Wang, Zhang, Li \& Shen}]{wang2019cryptocurrency}
\bibinfo{author}{Wang, P.}, \bibinfo{author}{Zhang, W.}, \bibinfo{author}{Li,
  X.}, \& \bibinfo{author}{Shen, D.} (\bibinfo{year}{2019}).
\newblock \bibinfo{title}{Is cryptocurrency a hedge or a safe haven for
  international indices? a comprehensive and dynamic perspective}.
\newblock {\it \bibinfo{journal}{Finance Research Letters}\/},  {\it
  \bibinfo{volume}{31}\/}, \bibinfo{pages}{1--18}.
%Type = Article
\bibitem[{Wang et~al.(2021)Wang, Yang \& Liu}]{wang2021stock}
\bibinfo{author}{Wang, X.}, \bibinfo{author}{Yang, K.}, \&
  \bibinfo{author}{Liu, T.} (\bibinfo{year}{2021}).
\newblock \bibinfo{title}{Stock price prediction based on morphological
  similarity clustering and hierarchical temporal memory}.
\newblock {\it \bibinfo{journal}{IEEE Access}\/},  {\it \bibinfo{volume}{9}\/},
  \bibinfo{pages}{67241--67248}.
%Type = Article
\bibitem[{Wepener \& O’Brien(2022)}]{wepener2022application}
\bibinfo{author}{Wepener, V.}, \& \bibinfo{author}{O’Brien, G.}
  (\bibinfo{year}{2022}).
\newblock \bibinfo{title}{The application of bayesian networks to evaluate
  risks from multiple stressors to water quality of freshwater ecosystems}.
\newblock {\it \bibinfo{journal}{African Journal of Aquatic Science}\/},  {\it
  \bibinfo{volume}{47}\/}, \bibinfo{pages}{231--244}.
%Type = Article
\bibitem[{Wo{\l}k(2020)}]{wolk2020advanced}
\bibinfo{author}{Wo{\l}k, K.} (\bibinfo{year}{2020}).
\newblock \bibinfo{title}{Advanced social media sentiment analysis for
  short-term cryptocurrency price prediction}.
\newblock {\it \bibinfo{journal}{Expert Systems}\/},  {\it
  \bibinfo{volume}{37}\/}, \bibinfo{pages}{e12493}.
%Type = Article
\bibitem[{Wong et~al.(2018)Wong, Saerbeck \&
  Delgado~Silva}]{wong2018cryptocurrency}
\bibinfo{author}{Wong, W.~S.}, \bibinfo{author}{Saerbeck, D.}, \&
  \bibinfo{author}{Delgado~Silva, D.} (\bibinfo{year}{2018}).
\newblock \bibinfo{title}{Cryptocurrency: A new investment opportunity? an
  investigation of the hedging capability of cryptocurrencies and their
  influence on stock, bond and gold portfolios}.
\newblock {\it \bibinfo{journal}{An Investigation of the Hedging Capability of
  Cryptocurrencies and Their Influence on Stock, Bond and Gold Portfolios
  (January 29, 2018)}\/}, .
%Type = Article
\bibitem[{Yu et~al.(2015)Yu, Khan \& Garaniya}]{yu2015modified}
\bibinfo{author}{Yu, H.}, \bibinfo{author}{Khan, F.}, \&
  \bibinfo{author}{Garaniya, V.} (\bibinfo{year}{2015}).
\newblock \bibinfo{title}{Modified independent component analysis and bayesian
  network-based two-stage fault diagnosis of process operations}.
\newblock {\it \bibinfo{journal}{Industrial \& Engineering Chemistry
  Research}\/},  {\it \bibinfo{volume}{54}\/}, \bibinfo{pages}{2724--2742}.
%Type = Article
\bibitem[{Zhang \& Thai(2016)}]{zhang2016expert}
\bibinfo{author}{Zhang, G.}, \& \bibinfo{author}{Thai, V.~V.}
  (\bibinfo{year}{2016}).
\newblock \bibinfo{title}{Expert elicitation and bayesian network modeling for
  shipping accidents: A literature review}.
\newblock {\it \bibinfo{journal}{Safety science}\/},  {\it
  \bibinfo{volume}{87}\/}, \bibinfo{pages}{53--62}.
%Type = Article
\bibitem[{Zhu et~al.(2016)Zhu, Guo, You \& Xu}]{zhu2016heterogeneity}
\bibinfo{author}{Zhu, H.}, \bibinfo{author}{Guo, Y.}, \bibinfo{author}{You,
  W.}, \& \bibinfo{author}{Xu, Y.} (\bibinfo{year}{2016}).
\newblock \bibinfo{title}{The heterogeneity dependence between crude oil price
  changes and industry stock market returns in china: Evidence from a quantile
  regression approach}.
\newblock {\it \bibinfo{journal}{Energy Economics}\/},  {\it
  \bibinfo{volume}{55}\/}, \bibinfo{pages}{30--41}.
%Type = Article
\bibitem[{Zhu et~al.(2021)Zhu, Zhang, Wu, Zheng \& Zhang}]{zhu2021investor}
\bibinfo{author}{Zhu, P.}, \bibinfo{author}{Zhang, X.}, \bibinfo{author}{Wu,
  Y.}, \bibinfo{author}{Zheng, H.}, \& \bibinfo{author}{Zhang, Y.}
  (\bibinfo{year}{2021}).
\newblock \bibinfo{title}{Investor attention and cryptocurrency: Evidence from
  the bitcoin market}.
\newblock {\it \bibinfo{journal}{PLoS One}\/},  {\it \bibinfo{volume}{16}\/},
  \bibinfo{pages}{e0246331}.

\end{thebibliography}


\end{document}
