\documentclass[pdflatex,sn-aps]{sn-jnl}


\usepackage{graphicx}%
\usepackage{multirow}%
\usepackage{amsmath,amssymb,amsfonts}%
\usepackage{amsthm}%
\usepackage{mathrsfs}%
\usepackage[title]{appendix}%
\usepackage{xcolor}%
\usepackage{textcomp}%
% \usepackage{manyfoot}%
\usepackage{booktabs}%
\usepackage{algorithm}%
\usepackage{algorithmicx}%
\usepackage{algpseudocode}%
\usepackage{listings}%
\usepackage[normalem]{ulem}%
\usepackage{booktabs}%
\usepackage{longtable}
\usepackage{subcaption}
\usepackage{adjustbox}
\usepackage{array}
\usepackage{url} 
\usepackage{float}
\usepackage[section]{placeins}
\usepackage{colortbl}
\usepackage[normalem]{ulem}

\def\AN#1{{\color{red}{\bf [AN:} {\it{#1}}{\bf ]}}} 

% \bibliographystyle{apalike}

\begin{document}
\title[Article Title]{Causal Modelling of Cryptocurrency Price Movements Using Discretisation-Aware Bayesian Networks}

\author[1]{Rasoul Amirzadeh\thanks{Corresponding author: rasoul.amirzadeeh@gmail.com}}
\author[1]{Dhananjay Thiruvady}
\author[2]{Asef Nazari}
\author[1]{Mong Shan Ee}

\affil[1]{School of Information Technology, Deakin University,   Australia}
\affil[2]{Deakin Business School, Deakin University,  Australia}




\abstract{
This study identifies the key factors influencing the price movements of major cryptocurrencies, Bitcoin, Binance Coin, Ethereum, Litecoin, Ripple, and Tether, using Bayesian networks~(BNs). This study addresses two key challenges: modelling price movements in highly volatile cryptocurrency markets and enhancing predictive performance through discretisation-aware Bayesian Networks. It analyses both macro-financial indicators (gold, oil, MSCI, S\&P 500, USDX) and social media signals (tweet volume) as potential price drivers. Moreover, since discretisation is a critical step in the effectiveness of~BNs, we implement a structured procedure to build 54~BNs models by combining three discretisation methods (equal interval, equal quantile, and k-means) with several bin counts. These models are evaluated using four metrics, including balanced accuracy, F1 score, area under the ROC curve and a composite score. Results show that equal interval with two bins consistently yields the best predictive performance. We also provide deeper insights into each network’s structure through inference, sensitivity, and influence strength analyses. These analyses reveal distinct price-driving patterns for each cryptocurrency, underscore the importance of coin-specific analysis, and demonstrate the value of BNs for interpretable causal modelling in volatile cryptocurrency markets.}


% \AN{We need to have a sentence expressing the challenge that we aim to deal with in this paper. Possible options can be prediction challenges considering the complicated dynamics of cryptos and/or the challenge of sound discrimination in Bayesian networks. For example, ``This paper addresses the dual challenge of price direction forecasting within the highly volatile and complex dynamics of cryptocurrency markets, and achieving robust sound discrimination within Bayesian network frameworks.''}
% \AN{Also, in the text, be careful with using the term altcoins. If you include Bitcoin, bringing argument about altcoins does not make sense.}
% }

\keywords{Cryptocurrencies, Altcoins, Bayesian networks,  Social media, Causal inference, Discretisation,  Price prediction}
\maketitle

\section{Introduction}
Despite its relatively short history, the cryptocurrency market has become a significant component of global finance~\cite{gajardo2018does}. While Bitcoin remains the most well-known asset, its dominance is steadily declining. For instance, as shown in Figure~\ref{fig:capital_comparison}, its capital dropped from 85\% in February 2017 to 58\% in March 2025.\footnote{Obtained from coinstats.app in April 2025} Meanwhile, the number of altcoins has grown from around 50 in 2013 to around 10,000 by June 2025,\footnote{www.statista.com} offering innovations that address Bitcoin’s limitations in security, privacy, and stability. This shift underscores the growing importance of analysing altcoins as key players in the cryptocurrency domain, alongside Bitcoin, to better capture the evolving dynamics and investment behaviours in the cryptocurrency market.

\begin{figure*}[!htbp]
  \centering \includegraphics[width=\linewidth,height=0.3\textheight, keepaspectratio]{capital_comparison.jpg}
  \caption{The percentage of market capitalisation of Bitcoin from May 2015 to March 2025.}
  \label{fig:capital_comparison}
\end{figure*}

As financial assets, cryptocurrencies face challenges similar to those in traditional markets, such as the difficulty of accurate price prediction due to the noisy and nonlinear nature of financial data~\citep{wang2021stock}. However, these challenges are further complicated by the unique characteristics of the cryptocurrency ecosystem. Factors such as mining difficulty, wallet security, social media activity, search trends, interactions with traditional financial assets, and the lack of global acceptance and regulation significantly influence cryptocurrency price movements~\citep{griffith2023cryptocurrency}. Compared to traditional assets, the cryptocurrency market is less mature, and its dynamics are still evolving \cite{amirzadeh2024dynamic}. Therefore, understanding these diverse and interacting drivers is crucial for developing accurate models of cryptocurrency price behaviour.


While it is widely accepted that external factors, such as macroeconomic indicators and social media activity, can influence cryptocurrency prices~\cite{erzurumlu2020one}, most existing studies tend to examine these influences separately. The literature typically focuses either on macroeconomic variables or on the role of sentiment and public discourse~\citep{inamdar2019predicting}. As a result, there is a lack of comprehensive models that jointly evaluate these diverse influences. This separation of factors highlights a key gap in the literature: the need for integrated approaches that assess how financial and social signals collectively shape altcoin price dynamics~\citep{poyser2019exploring}.

In addition, most existing research remains heavily Bitcoin-centric, with studies addressing market efficiency, portfolio strategies, or financial contagion~\citep{kurihara2017market, guesmi2019portfolio, ferreira2019contagion}. Even when altcoins are included, analyses are often generalised rather than coin-specific, overlooking the unique behaviours and drivers of individual assets~\citep{wang2019cryptocurrency}.

Despite growing interest in understanding cryptocurrency price drivers and forecasting, existing modelling approaches often rely on traditional statistical methods or black-box machine learning techniques. These methods typically lack the ability to incorporate domain knowledge, primarily capture correlational rather than causal relationships, and rarely support transparent reasoning about the underlying drivers of price movements. To address these shortcomings, we propose the use of  BNs as a principled framework for modelling the interplay of financial and social factors. BNs identify causal relationships~\citep{sevinc2020bayesian} using directed acyclic graphs (DAGs), and offer key advantages such as handling missing data, integrating expert knowledge with empirical evidence, and enabling explainable inference~\citep{heckerman2008tutorial}, making them a compelling choice for tackling the complexities of cryptocurrency price prediction~\citep{amirzadeh2022applying}.

However, despite these advantages, the application of BNs on cryptocurrency remains limited. Existing studies often neglect the importance of discretisation preprocessing step, applying them without systematic evaluation, regardless of its major impact on BN structure and predictive performance~\cite{nojavan2017comparative}. To address this, we propose a structured discretisation pipeline using multiple methods and bin sizes to identify optimal configurations for cryptocurrency modelling, improving both accuracy and interpretability. Using this pipeline, we construct nine BNs per cryptocurrency considered in this study and evaluate their predictive performance using four different metrics to identify the most accurate models and uncover the key price-driving factors.

Building on these motivations and methodological gaps in the literature, this study makes the following contributions.
\begin{itemize}
    \item Provides a coin-specific modelling framework that uncovers the unique drivers of price movements for each cryptocurrency, allowing tailored analysis and interpretation.
    \item Presents a BN-based modelling framework that jointly incorporates macro-financial indicators and social media signals to capture external influences on cryptocurrency prices
    \item Proposes a structured discretisation pipeline, systematically comparing methods and bin settings to optimise BN construction on continuous time series data.
    \item Enhances interpretability through post-hoc BN analyses, including inference, sensitivity, and influence strength evaluations, offering transparent explanations of price dynamics.
\end{itemize}

The remainder of the paper is structured as follows: Section \ref{Background} reviews the literature on cryptocurrency price factors, while Section \ref{Material and Methods} introduces BNs.  Section~\ref{framwork} outlines the research framework, followed by BN construction in Section~\ref{Experimental}. Section~\ref{Results} presents results, and Section~\ref{Conclusions} concludes with future research directions.

% \AN{This is the best introduction you have written, Rasoul.}


\section{Related literature}  \label{Background}

Research on the relationship between cryptocurrencies and traditional financial assets has predominantly examined correlations, with a focus on identifying diversification opportunities. Corbet et al.~\cite{corbet2018exploring} study the connectedness between cryptocurrencies and traditional financial assets using the concept of spillovers, where an economic event affects seemingly unrelated assets~\cite{lau2017return}. Their findings suggest cryptocurrencies are relatively isolated from financial assets, which is beneficial for diversification opportunities. Charfeddine et al.~\cite{charfeddine2020investigating} examine the dynamic relationships between Bitcoin, Ethereum, and major financial assets like gold, oil, USD/YUAN, and S\&P 500, concluding that correlations are weak and influenced by external shocks. To explore causality, Ji et al.~\cite{ji2018network} use a PC algorithm to construct DAGs between Bitcoin and other financial assets. Their findings indicate that Bitcoin is largely isolated, although time-variant causal relationships appear during bear markets.


Several studies have examined correlations among cryptocurrencies. Stosic et al.~\cite{stosic2018collective} analyse cross-correlations among 119 cryptocurrencies, revealing a complex hierarchical structure that is not apparent when considering only partial correlations. In a similar study, Shi et al.~\cite{shi2020correlations} use a multivariate factor stochastic volatility model and observe smaller groups with similar price volatility among six cryptocurrencies. Bouri et al.~\cite{bouri2021return} study the market integration of 12 cryptocurrencies using a dynamic equicorrelation model. They conclude that trading volume and external uncertainties drive market integration, which varies over time.


Because correlation is important for portfolio optimisation, researchers have explored relationships between cryptocurrencies and traditional financial assets to better understand their price movements. Aslanidis et al.~\cite{aslanidis2019analysis} use a generalised dynamic conditional correlation model to examine relationships between cryptocurrencies, gold, and stock and bond indices, concluding that cryptocurrencies show limited correlation with traditional assets but positive, time-varying correlations among themselves. Similarly, Giudici et al.~\cite{giudici2021crypto} find Bitcoin price is not affected by traditional financial assets, while Malladi et al.~\cite{malladi2021time} demonstrate that different cryptocurrencies react differently to changes in global stock markets and gold. Specifically, they observe that Bitcoin returns are independent of stock market returns but linked to Ripple's return.



Researchers have extended their focus beyond traditional markets, examining additional factors that influence cryptocurrency prices, particularly the role of social media. Lamon et al.~\cite{lamon2017cryptocurrency} use labelled daily news and social media data to determine trends in cryptocurrency prices. Abraham et al.~\cite{abraham2018cryptocurrency} find that the sentiment of the tweet is highly correlated with price movements. Similarly,  Rouhani et al.~\cite{rouhani2019crypto} analyse five million tweets and find that Ripple had the highest percentage of positive tweets (52\%), while Bitcoin received the highest proportion of negative tweets~(27\%). The predictive power of social media sentiment has been further examined by \cite{kraaijeveld2020predictive} and \cite{kim2021dynamics}, showing that Twitter sentiment can predict price returns of Bitcoin, Bitcoin Cash, and Litecoin, with markets more responsive to positive sentiment during downward trends.

In summary, while previous studies have examined both financial correlations and social sentiment, there is still limited understanding of causal structures, particularly in the altcoin domain. Most existing research either focuses on macroeconomic factors or social media influences in isolation. This highlights the need for a more comprehensive approach that combines both dimensions to better explain altcoin price movements.



\section{An overview of Bayesian networks}
\label{Material and Methods}

BNs, a class of probabilistic graphical models, represent complex probability distributions using DAGs~\citep{larranaga2012review}. In a BN, nodes denote random variables and directed edges represent conditional dependencies through DAG, where edges have a defined direction (arrows) and do not form cycles (loops). Based on Bayes’ theorem, BNs support causal inference by factorising the joint distribution, where each node depends only on its direct parents, summarised in a conditional probability table (CPT)~\citep{koller2009probabilistic}.


In BN terminology, a node’s ancestors are nodes from which a directed path reaches it, while descendants are nodes that receive directed paths. Nodes with no descendants are leaf nodes, whereas nodes without ancestors are root nodes \citep{yu2015modified}. The efficiency of BNs stems from their structure, where each variable is conditionally dependent only on its immediate parents and ancestors in the network. For a given set of random variables $\{X_1, X_2, \ldots, X_n\}$ the joint probability distribution $P(X_1, X_2, \ldots, X_n)$ requires $O(2^n)$ parameters for binary variables, which can be computationally intensive. However, by considering only the local dependencies, it can be represented as:
$$P(X_1, X_2, \ldots, X_n)=\prod\limits_{i=1}^n P(X_i|\text{parents}(X_i))$$
where $\text{parents}(X_i)$ is the set of parents of $X_i$ in a DAG representation~\citep{heckerman2008tutorial}.

The nodes in a BN can be continuous or discrete; the discrete nodes are more common. Since most BN learning algorithms operate more efficiently on discrete variables, continuous data are typically discretised using common discretisation methods such as k-means and equal-width binning. The probability of each
discrete variable’s state is conditional on the states of its parents.


A BN can be constructed manually using expert knowledge, automatically using training data, or through a combination of both \citep{zhang2016expert}. Learning a BN structure from data is a challenging task, even with complete data, as it involves structure learning for DAG and parameter estimation for CPTs \citep{de2011efficient, amirkhani2016exploiting}.  These are constructed by calculating probabilities for all possible state combinations of a node and its parents, which can be computationally demanding.

Learning BN structures from data is an NP-hard problem, making it difficult to obtain a complete causal model without considering numerous variables \citep{gheisari2016bnc}.
Two main approaches to structure learning are Bayesian and constraint-based techniques. In the Bayesian approach, an initial structure is built using domain knowledge and then refined with data, whereas constraint-based algorithms search for conditional dependencies to determine relationships \citep{uusitalo2007advantages}.



\section{A Conceptual framework for external drivers of cryptocurrency prices }\label{framwork}

To study the factors driving cryptocurrency prices, we build on the findings of~\cite{ciaian2016economics} for Bitcoin price formation and~\cite{poyser2019exploring} to categorise these factors, as presented in Figure~\ref{fig:price_drivers}. The framework divides price determinants into internal and external categories. As the name implies, internal factors refer to elements that are directly linked to the cryptocurrency ecosystem itself, including supply, demand, transaction volume, and mining difficulty \citep{iccelliouglu2019investigation}. External factors, which are the focus of this study, include macro-financial conditions and investor behaviour—specifically, attractiveness and adoption. In behavioural finance, attractiveness refers to the attractiveness of new financial products, while adoption indicates investors' intention to use these products, such as cryptocurrencies~\citep{ricciardi2000behavioral}. It should be noted that internal and external factors are not completely independent; they dynamically interact in real-world scenarios.


\begin{figure*}[h]
  \centering
\includegraphics[width=0.6\linewidth,height=0.5\textheight,keepaspectratio]
  {price_drivers.jpg}
  \caption{Conceptual framework categorising internal and external factors influencing cryptocurrency prices. Adapted from \cite{poyser2019exploring}.}
  \label{fig:price_drivers}
\end{figure*}


To investigate macro-financial drivers of altcoin prices, we select representative assets from key financial classes: currencies, commodities, and stock indices. We include the S\&P 500 as a proxy for the performance of the US stock market \citep{balcilar2021dynamic}, and two major commodities, gold and West Texas Intermediate (WTI) oil. The WTI oil price is particularly relevant due to its impact on cryptocurrency mining costs \citep{hayes2017cryptocurrency, okorie2020crude}. The USD index (USDX) is also included for its role in altcoin trading, which captures the value of the dollar relative to a basket of major currencies \citep{hasan2022exploring}. Finally, the MSCI World Index serves as a global benchmark for macroeconomic conditions \citep{amirzadeh2023framework}. Thus, our analysis includes two commodities (gold and oil), two equity indices (MSCI and S\&P 500), and one currency index (USDX) as potential macro-financial drivers of altcoin prices.


Behavioural finance, particularly in the context of attractiveness and adoption, examines the psychological drivers of investor decision-making, with an emphasis on emotional influences~\citep{konigstorfer2020applications, ricciardi2000behavioral}. In contrast to traditional finance, which is based on the efficient market hypothesis assumes that asset prices fully reflect all available information, behavioural finance incorporates investor psychology. It accounts for how emotional states influence the investment decision-making process~\citep{rossi2018efficient, malkiel2003efficient}. With the growth of social media, investors are increasingly influenced by shared stories and experiences, shaping perceptions and trading behaviours~ \citep{gurdgiev2020herding, mai2018does, lund2018power}. According to narrative economics,\footnote{Narrative economics, introduced by Nobel winner Robert J. Shiller, studies the dynamics of popular narratives and stories and how people make economic decisions, such as investment in a volatile speculative asset, based on contagious stories~\citep{shiller2017narrative}} such narratives can rapidly gain attraction and significantly shape investment choices and influence financial decision-making~\citep{shiller2017narrative}. Consequently, recent research increasingly focuses on how social media influences financial markets, particularly cryptocurrencies, highlighting the role of investor attention in driving market dynamics \citep{smales2022investor, lin2021investor, zhu2021investor}.



With approximately 330 million monthly active users, X (formerly Twitter) serves as a major platform for public discussions on cryptocurrencies through tweets, hashtags, and live spaces. These conversations offer valuable data for analysing price movements based on public sentiment~\citep{wolk2020advanced}. In this study, we examine the relationship between daily tweet volumes related to altcoins and their intra-day prices as a proxy for investor attractiveness and adoption.



\section{Experimental design}  \label{Experimental}


This study aims to identify the causal relationships among factors influencing cryptocurrency price movements using BNs. As cryptocurrency price data are continuous, most BN software packages require the data to be discretised for model construction~\citep{death2015good}. Although DAGs can be constructed from continuous data using algorithms such as the PC algorithm, these models often lack post-hoc capabilities such as inference and sensitivity analyses.  Hence, selecting appropriate discretisation techniques and bin numbers is essential for accurate modelling~\citep{nojavan2017comparative}.


To address this, we implement a structured procedure that explores combinations of discretisation methods (equal interval, equal quantile, and k-means) and bin numbers (2, 3, and 4), resulting in a total of 54 BN models constructed across six cryptocurrencies. Each model is trained on 80\% of the data and tested on the remaining 20\%. Evaluation is based on three performance metrics: balanced accuracy, F1 score, and area under the ROC curve (AUC). Balanced accuracy is particularly useful for addressing class imbalance in the test sets, providing a more reliable measure of overall model performance than raw accuracy~\cite{guesne2024mind}.
 
Since each metric reflects different aspects of model quality, we adopt a composite evaluation strategy to identify the most effective discretisation configuration~\cite{hu2021multisite}. Following the method proposed in~\cite{lee2021neuroimaging}, each metric is scaled to a [0,100] range based on its distribution across all settings using the following formula: 

% \AN{I removed commas in the equation.}

\begin{equation}
P_{ij} = \left( \frac{S_{ij} - \min V_j}{\max V_j - \min V_j} \right) \times 100, \quad \text{where } V_j = \{ S_{ij} \mid i = 1, 2, \ldots, n \}
\end{equation}
This score provides a unified and interpretable basis for comparing models and is reported as the “Composite Score”.  The best-performing BN for each altcoin is selected for further inference and sensitivity analysis. The overall process is summarised in Figure~\ref{fig:alg}. 





\begin{figure}[h]
 \centering
\includegraphics[width=0.6\linewidth,height=0.6\textheight,keepaspectratio]{flowchart.JPG}
 \caption{Flowchart illustrating the pipeline for constructing BNs in this study, based on different discretisation methods and bin numbers.}
 \label{fig:alg}
\end{figure}


\noindent This process involves defining bin-specific state labels and numbers as follows:
\begin{itemize}
 \item \textbf{2-bin}: \textit{Down}, \textit{Up}
\item \textbf{3-bin}: \textit{Down}, \textit{Steady}, \textit{Up}
\item \textbf{4-bin}: \textit{Strong Down}, \textit{Down}, \textit{Up}, \textit{Strong Up}
  \footnote{Increasing the number of bins beyond four also showed no significant improvement in accuracy.}
\end{itemize}


We apply three commonly used  unsupervised ML methods as below~\citep{saleh2018implementation,min2009global}:
\begin{itemize}
  \item \textbf{Equal interval}: Splits the data range into intervals of equal width based on the number of bins. 
  \item \textbf{Equal quantile}: Divides sorted data into equal-sized groups using quantile cutoffs.
  \item \textbf{K-means}: Applies centroid-based clustering to group values into bins, offering flexible partitioning of data~\citep{ahmed2020k, sinaga2020unsupervised}. K-means is particularly noteworthy as it is not commonly included in most BN software packages, yet it often yields improved model performance.
\end{itemize}




Based on this procedure, we construct 54 BNs (6 altcoins × 3 methods × 3 bin counts). Both scaled and raw data are tested, with no significant difference in predictive performance.  Following this setup, all constructed BNs are evaluated using four key performance metrics.

% The detailed implementation steps are provided in Algorithm~\ref{alg}.

% \begin{algorithm}[!h]
% \caption{BN Construction and Evaluation for Altcoin Price Prediction} \label{alg}
% \begin{algorithmic}[1]
% \State \textbf{Start}
% \State Merge price, tweet volume, and macro-financial indicators into a unified dataset per cryptocurrency
% \For{each combination of discretisation method and bin number}
%     \State Discretise continuous features
%     \State Split data into 80\% training and 20\% testing
%     \State Train BN on training data
%     \State Evaluate on test data using Balanced Accuracy, F1 score, and AUC
%     \State Scale each metric to [0,100] and compute Composite Score
%     \State Store results
% \EndFor
% \State Select best-performing BN based on highest Composite Score
% \State \textbf{End}
% \end{algorithmic}
% \end{algorithm}




This study employs the GeNIe  software package~\citep{bayesfusion2017genie} to derive BNs. Genie is widely used in the literature for causal modelling and offers essential computational tools.\footnote{Several software packages are available for constructing causal networks, including Netica, Hugin, and Analytica. For a comparative overview, see \cite{mahjoub2011software}.}  We employ the Bayesian Search algorithm with GeNIe’s default configuration. The full set of parameters is listed in Table~\ref{tab:genie_config}. These defaults were selected to ensure consistency, ease of replication, and low computational overhead.
All computations are performed on a system with an Intel Core i7 CPU @ 1.90GHz (2.11 GHz), 16.0 GB RAM, running Windows 10 Enterprise.

\begin{table}[!h]
\centering
\caption{GeNIe parameter settings used for BNs construction.}
\label{tab:genie_config}
\begin{tabular}{ll}
\toprule
\textbf{Parameter}                  & \textbf{Value} \\
\midrule
Learning Algorithm                 & Bayesian Search \\
Discrete Threshold                 & 20 \\
Max Parent Count                   & 8 \\
Iterations                         & 20 \\
Sample Size                        & 50 \\
Seed                               & 0 \\
Link Probability                   & 0.1 \\
Prior Link Probability             & 0.001 \\
\bottomrule
\end{tabular}
\end{table}
 
\subsection{Dataset description and preprocessing}


The cryptocurrencies selected for this study are ranked among the top 10 cryptocurrencies by market capitalisation. Bitcoin, Binance Coin, Ethereum, Litecoin, Ripple, and Tether were selected based on their historical prominence, market capitalisation, and data availability. All selected coins had over 1,200 valid daily observations between January 2018 and April 2023, providing sufficient data length for robust statistical and time series analysis. This extensive period supports reliable correlation assessments with traditional financial, which typically span even longer time horizons. As a result, the selected cryptocurrencies include Bitcoin and five major altcoins, Binance Coin, Ethereum, Litecoin, Ripple, and Tether, which together accounted for approximately 70\% of the total cryptocurrency market capitalisation in 2023.\footnote{Based on the data on coingecko.com}




Daily closing price data for the selected altcoins and financial assets, including gold, MSCI, S\&P 500, WTI, and USDX, were obtained from \texttt{Yahoo Finance} using the \texttt{yfinance} Python library. Additionally, daily tweet volumes related to the altcoins were collected from \texttt{bitinfocharts.com}. Due to the absence of tweet data for Tether, it was excluded from the social media analysis. Summary statistics for both price and tweet data are presented in Appendix~\ref{Prices_Tweet_stats}.


To construct BNs for each cryptocurrency, we first create a merged dataset combining tweet volume, macro-financial indicators, and the corresponding coin’s price data, aligned by date availability. To address the challenges of time series data, we apply several preprocessing steps. This includes handling non-stationarity and distributional irregularities. Extreme outliers are removed using the interquartile range (IQR) method, while stationarity is tested using the Augmented Dickey-Fuller (ADF) and KPSS tests. When necessary, first-order differencing is applied to enforce stationarity. These preprocessing steps, detailed in Appendix~\ref{adf} and Appendix~\ref{Outlier}, are essential for ensuring reliable structure learning and inference in the BN models.

Furthermore, to reduce skewness and improve the consistency of binning during discretisation, we applied a logarithmic transformation to all price variables. This stabilises variance, mitigates extreme values, and supports more balanced binning across discretisation strategies~\citep{tessema2021enhancing}. Summary statistics of the raw data distributions are provided in Appendix~\ref{Outlier}, highlighting the skewness and outlier patterns that motivated this transformation.






\section{Results and discussion} \label{Results}

This section presents the results of identifying the key factors influencing cryptocurrency price movements using BNs, along with post-hoc analyses including inference and sensitivity analysis. Section~\ref{pred_acc} compares the predictive performance of the 54 trained BNs using four metrics. Section~\ref{st_learn} provides detailed results from inference,  sensitivity and the strength of influence analyses. Inference analysis highlights the most influential predictors of the target variable (cryptocurrency prices), while sensitivity analysis quantifies how variations in these predictors affect the probability distribution of the target node. Additionally, the strength of influence analysis measures the magnitude of causal effects between variables, offering further insights into the underlying price dynamics.


\subsection{Predictive performance of Bayesian networks across discretisation settings
} \label{pred_acc}


We evaluate the predictive performance of 54 BN models. Each model predicts directional price movements based on the posterior probabilities of the target node. Performance is assessed using balanced accuracy, F1 Score,  AUC, and Composite Score as reported in Table~\ref{tab:BN_prediction_metrics}. The configuration with the highest composite score for each cryptocurrency is highlighted in bold. The final row shows how often each discretisation–bin combination achieved the top score across all cryptocurrencies.





\begin{table*}[h]
\caption{Prediction performance of BNs using different discretisation strategies and bin settings, evaluated by balanced accuracy, AUC, F1-score, and a Composite Score. The highest value in each row is highlighted.}
\label{tab:BN_prediction_metrics}
\begin{adjustbox}{max width=\textwidth}
\renewcommand{\arraystretch}{1.1}
{\large
\begin{tabular}{|>{\raggedright\arraybackslash}p{2.5cm} >{\raggedright\arraybackslash}p{3.5cm} 
|>{\centering\arraybackslash}p{1.5cm} >{\centering\arraybackslash}p{1.5cm} >{\centering\arraybackslash}p{1.5cm} 
|>{\centering\arraybackslash}p{1.5cm} >{\centering\arraybackslash}p{1.5cm} >{\centering\arraybackslash}p{1.5cm} 
|>{\centering\arraybackslash}p{1.5cm} >{\centering\arraybackslash}p{1.5cm} >{\centering\arraybackslash}p{1.5cm}|}

\hline
\multicolumn{2}{|c|}{\multirow{2}{*}{\textbf{Coin / Metric}}} & \multicolumn{3}{c|}{\textbf{Equal Interval}} & \multicolumn{3}{c|}{\textbf{Equal Quantile}} & \multicolumn{3}{c|}{\textbf{K-means}} \\
\multicolumn{2}{|c|}{} & 2 bins & 3 bins & 4 bins & 2 bins & 3 bins & 4 bins & 2 bins & 3 bins & 4 bins \\
\hline
\multirow{4}{*}{\rotatebox{35}{\textbf{Bitcoin}}} 
& Balanced Acc. & {0.5000} & 0.3333 & 0.2357 & {0.5000} & 0.4938 & 0.2559 & 0.4917 & 0.3471 & 0.2630 \\
& AUC           & 0.5948 & {0.7825} & 0.4791 & 0.5000 & 0.7362 & 0.5227 & 0.4917 & 0.4909 & 0.5847 \\
& F1            & 0.0112 & 0.6442 & 0.5975 & 0.3312 & 0.4941 & 0.2157 & {0.7178} & 0.5807 & 0.1973 \\
& Composite Score & 138.13 & 226.51 & 82.97 & 152.18 & {\large\textit{\textbf{250.74}}} & 50.95 & 201.01 & 126.64 & 71.47 \\ \hline

\multirow{4}{*}{\rotatebox{35}{\textbf{Binance Coin}}} 
& Balanced Acc. & {0.6971} & 0.3241 & 0.2565 & 0.5086 & 0.2849 & 0.2500 & 0.4999 & 0.3333 & 0.2624 \\
& AUC           & 0.6438 & 0.6777 & 0.5423 & 0.5086 & 0.6262 & 0.5443 & 0.3635 & 0.4891 & 0.4334 \\
& F1            & 0.5454 & 0.2307 & 0.3923 & 0.4546 & 0.2631 & 0.0993 & {0.7098} & 0.0005 & 0.3614 \\
& Composite Score & {\large\textit{\textbf{266.03}}} & 149.03 & 113.60 & 168.04 & 128.44 & 71.47 & 155.89 & 58.61 & 75.90 \\ \hline
\multirow{4}{*}{\rotatebox{35}{\textbf{Ethereum}}} 
& Balanced Acc. & 0.4880 & 0.3317 & 0.2908 & {0.5000} & 0.4096 & 0.3438 & 0.4542 & 0.3510 & 0.2500 \\
& AUC           & 0.5371 & 0.1789 & 0.5871 & 0.5000 & {0.6120} & 0.4474 & 0.4429 & 0.4327 & 0.4609 \\
& F1            & 0.7134 & {0.7817} & 0.5784 & 0.3333 & 0.4047 & 0.3410 & 0.2805 & 0.4445 & 0.3731 \\
& Composite Score & {\large\textit{\textbf{246.22}}} & 107.25 & 206.63 & 170.93 & 215.63 & 132.44 & 166.15 & 139.82 & 121.80 \\ \hline


\multirow{4}{*}{\rotatebox{35}{\textbf{Litecoin}}}
& Balanced Acc. & {0.5000} & 0.3338 & 0.2639 & {0.5000} & 0.4804 & 0.3047 & {0.5000} & 0.3273 & 0.2723 \\
& AUC           & 0.6363 & 0.1789 & 0.5601 & 0.4531 & {0.7287} & 0.4583 & 0.6578 & 0.4857 & 0.4424 \\
& F1            & {0.8560} & 0.3787 & 0.6281 & 0.3333 & 0.4804 & 0.2500 & 0.4954 & 0.4661 & 0.4001 \\
& Composite Score & {\large\textit{\textbf{{283.19}}}} & 50.84 & 131.73 & 163.62 & 229.72 & 68.10 & 227.60 & 118.32 & 76.25 \\ \hline

\multirow{4}{*}{\rotatebox{35}{\textbf{Ripple}}}
& Balanced Acc. & 0.5000 & 0.3318 & 0.2518 & {0.5195} & 0.3362 & 0.2266 & 0.5000 & 0.3333 & 0.2718 \\
& AUC           & 0.6123 & 0.2371 & 0.4380 & 0.5195 & {0.6181} & 0.4638 & 0.4814 & 0.4982 & 0.4453 \\
& F1            & 0.6481 & {0.8493} & 0.6357 & 0.5133 & 0.3045 & 0.1927 & 0.2120 & 0.0187 & 0.3364 \\
& Composite Score & {\large\textit{\textbf{267.60}}} & 135.92 & 135.62 & 233.67 & 171.83 & 80.45 & 180.74 & 104.96 & 108.33 \\ \hline

\multirow{4}{*}{\rotatebox{35}{\textbf{Tether}}}
& Balanced Acc. & {0.5000} & 0.3333 & 0.2405 & {0.5000} & 0.3333 & 0.2500 & {0.5000} & 0.3333 & 0.2500 \\
& AUC           & {0.5000} & {0.5000} & 0.4778 & {0.5000} & {0.5000} & {0.5000} & {0.5000} & {0.5000} & {0.5000} \\
& F1            & 0.1930 & {0.9827} & 0.2359 & 0.3333 & 0.1689 & 0.0986 & {0.9886} & 0.0001 & 0.0000 \\
& Composite Score & 98.94 & 260.83 & 97.31 & 164.55 & 88.33 & 72.84 & {\large\textit{\textbf{261.75}}} & 50.00 & 50.00 \\ \hline
\multirow{1}{*}{\textbf{Number of winning}}                 &   & 4                & 0           & 0                & 0           &1                & 0           & 1                & 0           & 0           \\ \hline
\end{tabular}
}
\end{adjustbox}
\end{table*}

Based on the composite score, the equal interval method with two bins performs best, achieving the top score in four out of six cryptocurrencies—more than any other configuration. K-means and equal quantile with two or three bins show competitive but less consistent results, each producing a top score only once. These findings suggest that simpler discretisation strategies, especially Equal intervals with two bins, offer more reliable performance by maintaining class separability and avoiding over-fragmentation.

To compare overall model performance across cryptocurrencies, we calculate the average composite score for each coin by aggregating results across all discretisation methods and bin settings. As shown in Figure~\ref{fig:avg_composite_score}, Ethereum achieves the highest average score~(167.33), followed by Ripple and Litecoin, whereas Tether and Binance Coin yield the lowest. These results suggest that certain cryptocurrencies possess data structures that better align with the probabilistic assumptions of BNs.  

\begin{figure}[!h]
  \centering
\includegraphics[width=0.8\linewidth,height=0.4\textheight,keepaspectratio]{Composite_Score.png}
  \caption{Average Composite Score for each cryptocurrency, aggregated across all discretisation strategies and bin configurations, based on the results reported in Table~\ref{tab:BN_prediction_metrics}.}
  \label{fig:avg_composite_score}
\end{figure}



\subsection{Discretisation impact on cryptocurrency}
\label{sec:coin_discretisation}

This subsection provides a more detailed evaluation of discretisation performance at the individual cryptocurrency level. Based on model performance reported in Table~\ref{tab:BN_prediction_metrics}, Figure~\ref{fig:coin_specific_scores} illustrates results using balanced accuracy, which is selected due to its robustness to class imbalance and suitability for fair comparison. A hatched bar in each group marks the method’s average. This enables comparison across both bin settings and overall discretisation strategies. 





\begin{figure*}[h]
  \centering
\includegraphics[width=\linewidth]{coin_spec.png}
 \caption{Balanced accuracy scores of all discretisation configurations across cryptocurrencies. Bars represent scores for two, three, and four bins. Hatched bars indicate the average performance for each method.}  \label{fig:coin_specific_scores}
\end{figure*}

Equal interval with two bins yields the highest balanced accuracy for most coins, including Bitcoin, Binance Coin, Ripple, suggesting that simpler binning preserves signal better.
Notably, a general decline in accuracy is observed as the number of bins increases, indicating that finer granularity may introduce noise.
K-means shows greater variability and is more sensitive to bin count.  The hatched bars reinforce the robustness of equal interval across multiple cryptocurrencies.

At the individual coin level, equal interval with two bins yields the best performance for most cases, with Binance Coin achieving the highest balanced accuracy of 0.70. In contrast, the lowest scores often occur with four-bin configurations, especially under K-means for Ripple and Ethereum. While some coins such as Litecoin and Tether show more balanced results across methods, others exhibit clear preferences, highlighting method sensitivity at the coin level.

Increasing the number of bins generally reduces balanced accuracy, with the sharpest declines seen in K-means and equal interval. For example, Binance Coin drops from 0.6971 (Interval–2 bins) to 0.2565 (Interval–4 bins), and Ripple falls from 0.5195 (Quantile–2 bins) to 0.2266 (K-means–4 bins). Equal quantile shows more stable performance across bin sizes but does not reach the top scores. These results highlight the importance of carefully balancing bin granularity with model stability.

\subsection{Evaluating the effect of discretisation on Bayesian networks  performance}


To better understand how discretisation impacts model performance, we first examine the distribution of data points across different binning strategies and then evaluate their predictive outcomes. Table~\ref{tab:Comparison bin} presents the number of data points in each bin for various discretisation strategies and bin settings.



\begin{table*}[t]
\caption{The number of data points in each bin based on the number of intervals and the unsupervised ML method for discretisation. The bottom rows report average bin counts across all cryptocurrencies for each bin configuration.}
\label{tab:Comparison bin}
\begin{adjustbox}{max width=\textwidth}
\begin{tabular}{|c|c|rr|rrr|rrrr|}
\hline
\multirow{2}{*}{\textbf{Cryptocurrency}} & \multirow{2}{*}{\textbf{Method}} &
\multicolumn{2}{c|}{\textbf{2 Intervals}} &
\multicolumn{3}{c|}{\textbf{3 Intervals}} &
\multicolumn{4}{c|}{\textbf{4 Intervals}} \\
& & \multicolumn{1}{c|}{\textit{Bin1}} & \textit{Bin2} &
\multicolumn{1}{c|}{\textit{Bin1}} & \multicolumn{1}{c|}{\textit{Bin2}} & \textit{Bin3} &
\multicolumn{1}{c|}{\textit{Bin1}} & \multicolumn{1}{c|}{\textit{Bin2}} & \multicolumn{1}{c|}{\textit{Bin3}} & \textit{Bin4} \\ 
\hline

\multirow{3}{*}{\textbf{Bitcoin}} & EqualInterval & 20 & 237 & 3 & 193 & 61 & 2 & 18 & 223 & 14 \\
 & EqualQuantile & 128 & 129 & 86 & 85 & 86 & 64 & 64 & 64 & 65 \\
 & K-means & 225 & 32 & 176 & 64 & 17 & 179 & 46 & 30 & 2 \\ \arrayrulecolor{gray}\hline
\multirow{3}{*}{\textbf{Binance Coin}} & EqualInterval & 12 & 241 & 4 & 100 & 149 & 3 & 9 & 190 & 51 \\ 
 & EqualQuantile & 126 & 127 & 85 & 84 & 84 & 63 & 63 & 63 & 64 \\
 & K-means & 51 & 202 & 96 & 4 & 153 & 150 & 42 & 57 & 4 \\  \arrayrulecolor{gray}\hline
\multirow{3}{*}{\textbf{Ethereum}} & EqualInterval & 44 & 212 & 9 & 216 & 31 & 3 & 41 & 201 & 11 \\
 & EqualQuantile & 128 & 128 & 86 & 85 & 85 & 64 & 64 & 64 & 64 \\
 & K-means & 41 & 215 & 147 & 87 & 22 & 85 & 137 & 14 & 20 \\  \arrayrulecolor{gray}\hline
\multirow{3}{*}{\textbf{Litecoin}} & EqualInterval & 25 & 231 & 6 & 137 & 113 & 3 & 22 & 184 & 47 \\
 & EqualQuantile & 128 & 128 & 86 & 85 & 85 & 64 & 64 & 64 & 64 \\
 & K-means & 163 & 93 & 155 & 74 & 27 & 135 & 48 & 66 & 7 \\  \arrayrulecolor{gray}\hline
\multirow{3}{*}{\textbf{Ripple}} & EqualInterval & 63 & 193 & 10 & 224 & 22 & 4 & 59 & 184 & 9 \\
 & EqualQuantile & 128 & 128 & 86 & 85 & 85 & 64 & 64 & 64 & 64 \\
 & K-means & 98 & 158 & 26 & 31 & 199 & 135 & 10 & 21 & 90 \\  \arrayrulecolor{gray}\hline
\multirow{3}{*}{\textbf{Tether}} & EqualInterval & 167 & 95 & 2 & 259 & 1 & 1 & 166 & 94 & 1 \\
 & EqualQuantile & 131 & 131 & 88 & 86 & 88 & 66 & 65 & 65 & 66 \\
 & K-means & 260 & 2 & 259 & 2 & 1 & 51 & 2 & 1 & 208 \\
 \arrayrulecolor{black}\hline
\multirow{3}{*}{\textbf{Average}} & {EqualInterval} & 55.2 & 201.5 & 5.7 & 188.2 & 62.8 & 2.7 & 52.5 & 179.3 & 22.2 \\
                                  & {EqualQuantile} & 128.2 & 128.5 & 86.2 & 85.0 & 85.5 & 64.2 & 64.0 & 64.0 & 64.5 \\
                                  & {K-means}       & 139.7 & 117.0 & 143.2 & 43.7 & 69.8 & 122.5 & 47.5 & 31.5 & 55.2 \\
\hline

\end{tabular}
\end{adjustbox}
\end{table*}


Equal interval tends to produce skewed distributions, particularly toward the \textit{Up} class in the 2-bin setting (201.5 vs 55.2) and the \textit{Steady} class in the 3-bin setting (188.2 on average), which shows its sensitivity to how the data is spread numerically. As expected from its design, the equal quantile method yields nearly uniform distributions across bins, with the 2-bin configuration averaging 128.2 and 128.5 for \textit{Down} and \textit{Up}, respectively. The 3-bin and 4-bin settings also show similarly balanced splits. 

K-means discretisation reflects the underlying cluster structure but often results in significant class imbalance. For instance, the 3-bin configuration assigns an average of 143.2 instances to \textit{Down} compared to 43.7 to \textit{Steady}. These results suggest that while K-means may capture some hidden patterns in the data, it can also create dominant classes that reduce model flexibility.

Furthermore, Table~\ref{tab:Comparison bin} also reveals notable differences in bin balance across cryptocurrencies. Tether consistently shows the most imbalanced distributions, particularly under K-means and equal interval methods, where one bin dominates (e.g., 260 vs 2 in the 2-bin K-means case). In contrast, Bitcoin and Ethereum tend to exhibit more balanced distributions—especially under the equal quantile method, which yields near-uniform splits across all bin counts. This suggests that some coins are better suited to balanced discretisation due to their underlying distribution.





\subsection{Bayesian networks reasoning for cryptocurrency price movements  
}  \label{st_learn}



This section presents inference and sensitivity analyses for the top-performing BNs identified for each cryptocurrency based on the highest composite score, following methodologies used by \cite{asvija2021security, hosseini2019development}.   Each selected BN is visualised alongside its inference and sensitivity analysis outputs in Figures~\ref{fig:Inference} and~\ref{fig:sensitivity}, respectively.






\subsubsection{Inference analysis} 


 Inference in BNs is a form of probabilistic reasoning that examines how changes in one variable propagate through the network. Specifically, we observe changes in  CPTs of other nodes when a particular node is fixed in one of its states. Figure~\ref{fig:Inference} presents inference analysis for the top-performing BNs. In our analysis, each cryptocurrency is set as the target node to illustrate how its associated factors respond. Inference can be bi-directional, used for prediction (cause to effect) or diagnosis (effect to cause)~\citep{lu2020risk}.

\begin{figure*}[h]
\centering
\begin{minipage}{\textwidth}
\centering

% Row 1
\begin{subfigure}[t]{0.48\textwidth}
  \centering
  \fbox{\includegraphics[width=\linewidth, height=0.12\textheight]{BTC_inf.jpg}}
 \caption{Bitcoin (Equal quantile, 3 bins)}

  \label{fig:BTC_inf}
\end{subfigure}
\hfill
\begin{subfigure}[t]{0.48\textwidth}
  \centering
   \fbox{\includegraphics[width=\linewidth, height=0.12\textheight]{BNB_inf.jpg}}
  \caption{Binance Coin (Equal interval, 2 bins)}
  \label{fig:BNB_inf}
\end{subfigure}

\vspace{0.2cm}

% Row 2
\begin{subfigure}[t]{0.48\textwidth}
  \centering
  \fbox{\includegraphics[width=\linewidth, height=0.12\textheight]{ETH_inf.jpg}}
  \caption{Ethereum (Equal interval, 2 bins)}
  \label{fig:ETH_inf}
\end{subfigure}
\hfill
\begin{subfigure}[t]{0.48\textwidth}
  \centering
  \fbox{\includegraphics[width=\linewidth, height=0.12\textheight]{LTC_inf.jpg}}
 \caption{Litecoin (Equal interval, 2 bins)}
  \label{fig:LTC_inf}
\end{subfigure}

\vspace{0.2cm}

% Row 3
\begin{subfigure}[t]{0.48\textwidth}
  \centering
  \fbox{\includegraphics[width=\linewidth, height=0.12\textheight]{XRP_inf.jpg}}
 \caption{Ripple (Equal interval, 2 bins)}
  \label{fig:XRP_inf}
\end{subfigure}
\hfill
\begin{subfigure}[t]{0.48\textwidth}
  \centering
  \fbox{\includegraphics[width=\linewidth, height=0.12\textheight]{Tether_inf.jpg}}
 \caption{Tether (K-means, 2 bins)}
  \label{fig:Tether_inf}
\end{subfigure}

\caption{Inference diagrams for the best-performing BNs of each cryptocurrency. In each network, the root node representing the target variable (price direction) is fixed to the `Down' state, and the resulting posterior beliefs of connected variables are observed. This illustrates how evidence propagates through the network via probabilistic reasoning.}

\label{fig:Inference}
\end{minipage}
\end{figure*}

From the perspective of individual cryptocurrencies, BNs reveal various levels of connectivity. Using graph terminology,\footnote{The number of directed edges entering a node $v$, known as the in-degree $\text{deg}^-(v)$, quantifies how many variables directly influence that node. Similarly, the out-degree $\text{deg}^+(v)$ represents how many variables the node influences. The total degree is given by $\text{deg}(v) = \text{deg}^-(v) + \text{deg}^+(v)$, reflecting the overall connectivity of the node within the graph.} based on Figure~\ref{fig:BTC_inf}, Bitcoin has an in-degree of $\text{deg}^-(\text{Bitcoin}) = 1$. This dependency suggests that global equity changes~(MSCI) may play a key role in shaping Bitcoin's directional movements. Notably, Bitcoin also acts as a parent node, exerting influence on Tweet No., implying that its price behaviour may drive changes in social media activity and sentiment.


As depicted in Figure~\ref{fig:BNB_inf}, $\text{deg}-(\text{Binance\ Coin}) = 3$, with edges connecting to MSCI, S\&P 500, and Tweet No. Similar to Bitcoin, Binance Coin has a direct incoming edge from MSCI. Additionally, the presence of an edge from S\&P 500 indicates a broader financial sensitivity compared to Bitcoin. This combination suggests that Binance Coin's directional movements are more strongly influenced by traditional financial assets. Moreover, Binance Coin also influences Tweet No., implying that its price behaviour may drive social media sentiment, as observed in the Bitcoin network.



Unlike other networks, Ethereum~(Figure~\ref{fig:ETH_inf}) appears structurally more isolated from financial assets, with only a single incoming edge to MSCI. Notably, Gold and USDX are completely disconnected and do not influence Ethereum's price movements. Moreover, Ethereum is also a parent of Tweet No., similar to Bitcoin and Binance Coin, indicating that its directional movements may play a dominant role in shaping social sentiment.



As shown in Figure~\ref{fig:LTC_inf}, Litecoin receives direct inputs from both MSCI and S\&P 500, resulting in an in-degree of $\text{deg}^-(\text{Litecoin}) = 2$. Considering the network state is set to \texttt{Down}, this reflects a relatively high level of connectivity. Litecoin remains well integrated with these financial assets and shows a highly confident posterior prediction, with a 99\% probability for an upward movement. Notably, Litecoin is isolated from several variables in the network, including USDX, WTI, and Tweet No.



As shown in Figure~\ref{fig:XRP_inf}, $\text{deg}^-(\text{Ripple}) = 1$, with a direct edge from MSCI. This structure reflects a linked path from commodity prices through equity markets to Ripple’s price direction. Notably, Ripple is completely disconnected from social sentiment (Tweet No.) as well as from Gold and USDX under this configuration.


Tether~(Figure~\ref{fig:Tether_inf}) has no incoming edges, where $\text{deg}^-(\text{Tether}) = 0$, and it is completely isolated in this network, with no edge influence from any financial. As a stablecoin, this may reflect its distinct role within the cryptocurrency ecosystem, unlike mining-based assets, Tether is typically pegged to fiat currency and designed to maintain price stability, which may reduce its sensitivity to broader market signals under this configuration.


Across the networks, a consistent pattern emerges regarding social sentiment. For Bitcoin, Ethereum, and Binance Coin, the price direction nodes act as parents of Tweet No., indicating that their price movements may influence tweet volume or sentiment, rather than being influenced by it. This suggests that social media activity tends to lag behind market movements, reflecting public reactions to price changes rather than driving them. In contrast, Litecoin, Ripple, and Tether have no edge to Tweet No., suggesting that their price movements may not be directly influenced by social media activity.


Considering the financial assets, MSCI appears as a parent node in four of the BNs, indicating its significant role as a causal factor in cryptocurrency price movements. This highlights its importance as a global equity indicator. S\&P 500 also frequently has direct edges to cryptocurrencies, reinforcing the relevance of traditional equity markets in shaping crypto price dynamics. In contrast, Gold and USDX often appear as isolated nodes across several BNs, suggesting their limited influence.




% In contrast, Tether has the highest in-degree with $\text{deg}^-(\text{Tether})=5$, indicating that it is influenced by all five financial factors considered (Figure~\ref{fig:Tether_inf}). Binance Coin and Ethereum show multiple financial and social influences, whereas Ripple displays strong dependency on both S\&P 500 and tweet activity. These differences underscore the distinct market dynamics of each altcoin.

% Considering financial assets, the MSCI World Index and S\&P 500 emerge as the most frequently connected traditional financial assets across altcoins. For example, MSCI influences Binance Coin, Ethereum, and Tether, while S\&P 500 connects with Ripple and Ethereum. The US Dollar Index (USDX) plays a more limited role, influencing only Ripple and Tether. 

% Energy prices are a crucial factor in the economics of cryptocurrency mining, as they directly affect profitability and influence supply-demand dynamics~\citep{hayes2017cryptocurrency, okorie2020crude}. Accordingly, there have been ongoing efforts to improve energy efficiency in blockchain networks.\footnote{For example, Ethereum’s transition to a proof-of-stake mechanism—known as “The Merge”—reduced its energy consumption by over 99.9\%, according to the Crypto Carbon Ratings Institute (2022). See: \url{https://carbon-ratings.com}} Despite this shift, Ethereum’s historically high mining complexity (e.g., 13.3 Peta for Ethereum vs. 16.9 Mega for Litecoin in April 2022) helps explain the strong impact of energy prices (WTI) on its price behaviour. WTI is causally linked to Binance Coin and Ethereum—both of which are mining-based—due to their higher mining difficulty. Tether's connection to WTI, despite its stablecoin nature, may reflect broader integration with financial markets and its engagement in emerging virtual asset ecosystems such as NFTs and metaverse tokens. Among all financial assets, Gold is the least influential among financial factors, only affecting Tether.


% Social media—especially Twitter—plays a prominent role in cryptocurrency price dynamics. Our BN analysis shows that all mining-based coins, except Ethereum, are directly influenced by daily tweet volume. Notably, upward market trends generate larger fluctuations in tweet activity, a phenomenon aligned with findings by~\citet{kim2021dynamics}, which highlight stronger sentiment shifts during bullish phases.

% Tweet activity has the strongest impact on Binance Coin, particularly during reversals from a “Down” to “Up” market state. This aligns with behavioral patterns in investor sentiment—our inference results show a stronger reaction in tweet activity during upward price trends, likely reflecting increased optimism and engagement during bullish markets.

% A stablecoin is a cryptocurrency designed to maintain price stability by pegging its value to an underlying asset—either directly to fiat currencies like the US dollar or indirectly through reserve assets such as Ethereumt~\citep{clark2019sok, mita2019stablecoin}. Tether, the only stablecoin in this study, is the most connected altcoin in our BNs~(Figure~\ref{fig:Tether_inf}), showing causal relationships with all five financial variables. Notably, USDX is one of its parents, consistent with Tether’s pegging to the US dollar.



\subsubsection{Sensitivity and influence strength}

Sensitivity analysis in BNs examines how variations in input factors influence changes in the target variable~\citep{castillo1997sensitivity}. It quantifies the extent to which uncertainty in the model's output stems from uncertainty in its inputs~\citep{saltelli2019so}, offering insights into causal relationships and the model’s internal logic~\citep{meurisse2022risk}. As a common tool in economic evaluations, it helps analysts to assess the reliability of outcomes , and aids in evaluating decision reliability~\citep{walker2001allowing}. In this study, we use sensitivity analysis to identify and prioritise key price-driving factors for each altcoin, guiding investment strategies and portfolio management.


% \begin{figure*}[!t]
%  \centering
% \begin{subfigure}{0.48\textwidth}
% \vspace{-5pt}
%   \centering
% \includegraphics[width=\linewidth,height=0.9\textheight,keepaspectratio]{BNB_Sen.JPG}
%   \caption{Binance Coin sensitivity analysis}
%   \label{fig:BNB_Sen}
% \end{subfigure}%
% \begin{subfigure}{.5\textwidth}
% \vspace{-5pt}
%   \centering
% \includegraphics[width=\linewidth,height=0.7\textheight,keepaspectratio]{ETH_Sen.jpg}
%   \caption{Ethereum sensitivity analysis}
%   \label{fig:ETH_Sen}
% \end{subfigure}
% \begin{subfigure}{0.48\textwidth}
% \vspace{-3pt}
%    \centering
% \includegraphics[width=\linewidth,height=0.7\textheight,keepaspectratio]{LTC_sen.JPG}
%   \caption{Litecoin sensitivity analysis}
%   \label{fig:LT_Sen}
% \end{subfigure}%
% \begin{subfigure}{0.48\textwidth}
% \vspace{-3pt}
%    \centering
% \includegraphics[width=\linewidth,height=0.7\textheight,keepaspectratio]{XRP_Sen.jpg}
%   \caption{Ripple sensitivity analysis}
%   \label{fig:XRP_Sen}
% \end{subfigure}
% \begin{subfigure}{0.48\textwidth}
% \vspace{-2pt}
%   \centering
% \includegraphics[width=\linewidth,height=0.9\textheight,keepaspectratio]{Tether_Sen.jpg}
%   \caption{Tether sensitivity analysis}
%   \label{fig:Tether_Sen}
% \end{subfigure}%

% \label{fig:allCoins}
% \caption{The sensitivity analysis results when the altcoin node is set as the target node.}
% \label{fig: sensitivity}
% \end{figure*}


\begin{figure*}[h]
\centering
\begin{minipage}{\textwidth}
\centering

% Row 1
\begin{subfigure}[t]{0.48\textwidth}
  \centering
 \fbox{\includegraphics[width=\linewidth, height=0.12\textheight]{BTC_sen.jpg}}
 \caption{Bitcoin}

  \label{fig:BTC_sen}
\end{subfigure}
\hfill
\begin{subfigure}[t]{0.48\textwidth}
  \centering
  \fbox{\includegraphics[width=\linewidth, height=0.12\textheight]{BNB_sen.jpg}}
  \caption{Binance Coin }
  \label{fig:BNB_sen}
\end{subfigure}

\vspace{0.2cm}

% Row 2
\begin{subfigure}[t]{0.48\textwidth}
  \centering
\fbox{\includegraphics[width=\linewidth, height=0.12\textheight]{ETH_sen.jpg}}
  \caption{Ethereum }
  \label{fig:ETH_sen}
\end{subfigure}
\hfill
\begin{subfigure}[t]{0.48\textwidth}
  \centering
 \fbox{\includegraphics[width=\linewidth, height=0.12\textheight]{LTC_sen.jpg}}
 \caption{Litecoin}
  \label{fig:LTC_sen}
\end{subfigure}

\vspace{0.2cm}

% Row 3
\begin{subfigure}[t]{0.48\textwidth}
  \centering
  \fbox{\includegraphics[width=\linewidth, height=0.12\textheight]{XRP_sen.jpg}}
 \caption{Ripple}
  \label{fig:XRP_sen}
\end{subfigure}
\hfill
\begin{subfigure}[t]{0.48\textwidth}
  \centering
 \fbox{\includegraphics[width=\linewidth, height=0.12\textheight]{Tether_sen.jpg}}
 \caption{Tether}
  \label{fig:Tether_sen}
\end{subfigure}
\caption{Sensitivity and strength-of-influence analysis for each cryptocurrency. Each altcoin is set as the target node, and red shading indicates the relative sensitivity of input variables. Arc thickness represents the strength of influence between connected variables, with thicker lines denoting stronger causal relationships.}

\label{fig:sensitivity}
\end{minipage}
\end{figure*}


The sensitivity analysis was conducted using Genie, implementing the algorithm described in \cite{coupe2000computational}. For each cryptocurrency, the corresponding node was set as the target variable within its best-performing BN. The resulting visual outputs highlight the influence of each input factor, with the intensity of red shading representing the relative strength of impact. These visualisations are presented in Figure~\ref{fig:sensitivity} for all six cryptocurrencies.


According to the results of the sensitivity analysis (Figure~\ref{fig:BTC_sen}), Bitcoin is primarily influenced by S\&P 500 and MSCI, with S\&P 500 having the strongest effect. In contrast, Binance Coin, Ethereum, and Litecoin exhibit low sensitivity to other variables, as reflected by the minimal presence of red shading in their diagrams (Figure~\ref{fig:BNB_sen}, \ref{fig:ETH_sen}, and \ref{fig:LTC_sen}).

For Ripple (Figure~\ref{fig:XRP_sen}), MSCI emerges as the dominant influencer, with S\&P 500 contributing more weakly, as can be seen from the lighter colour intensity. This suggests that Ripple’s direction is more closely tied to global equity performance than commodity-driven movements. Furthemore, Tether (Figure~\ref{fig:Tether_sen}) remains fully insensitive to external variables, consistent with its structural isolation observed in the BN inference.


While sensitivity analysis highlights which inputs most affect the target variable, strength of influence further clarifies the magnitude of these effects along each arc, providing a complementary layer of causal interpretation.
To further quantify the directional dependencies within the BNs, we used the Strength of Influence feature available in GeNIe.  This analysis captures the extent to which a parent node affects the probability distribution of its child node, and measures the average change in the child’s posterior distribution when the parent transitions between its states where higher values indicate stronger influence. This complements the structural analysis by prioritizing relationships not only by presence but also by magnitude of effect. In the visualisations, the thickness of each arc reflects the corresponding strength of influence, and greater thickness indicates a stronger relationship between the connected nodes.   


The strength of influence analysis highlights distinct roles for each cryptocurrency within its respective BN. As shown in Figure~\ref{fig:BTC_sen}, Bitcoin exerts a stronger influence on tweet activity than it receives from financial assets. This is evident from the thicker outgoing arc to Tweet No. compared to the thinner arc from MSCI to Bitcoin. Moreover, Binance Coin (Figure~\ref{fig:BNB_sen}) receives input from MSCI and has connections to both Tweet No. and the S\&P 500. The arc toward the S\&P 500 is thicker than the one to Tweet No., which may suggest a stronger connection to financial variables than to social sentiment. Moreover, in Figure~\ref{fig:ETH_sen}, the arc from Ethereum to Tweet No. is clearly visible and thicker than any other in its network, indicating that Ethereum may play a stronger role in shaping sentiment compared to other cryptocurrencies.

Both Ripple and Litecoin exhibit low influence strengths with their connected nodes. Although they are structurally linked to financial variables such as MSCI and the S\&P 500, the arcs are narrow and less prominent (Figures~\ref{fig:LTC_sen} and \ref{fig:XRP_sen}), suggesting that these inputs have limited impact under the current model.


% Ethereum (Figure~\ref{fig:ETH_sen}) shows sensitivity to all five financial variables, with MSCI and gold having the greatest impact.  Tether (Figure~\ref{fig:Tether_Sen}) is also highly sensitive to all five financial factors, particularly MSCI and gold. Notably, its relationship with the energy index WTI is weak, and it is evidenced by the pale colour in the diagram.




% According to the results of sensitivity analysis  in Figure~\ref{fig:BNB_Sen}, Binance Coin is primarily influenced by S\&P 500 and WTI, with S\&P 500 having the stronger effect. Ethereum (Figure~\ref{fig:ETH_Sen}) shows sensitivity to all five financial variables, with MSCI and gold having the greatest impact. For Ripple (Figure~\ref{fig:XRP_Sen}), tweet volume and S\&P 500 emerge as the dominant influencers, with tweet volume having a stronger effect. Tether (Figure~\ref{fig:Tether_Sen}) is also highly sensitive to all five financial factors, particularly MSCI and gold. Notably, its relationship with the energy index WTI is weak, and it is evidenced by the pale colour in the diagram.


\section{Conclusions and Future Work} \label{Conclusions}

This study investigates the drivers of cryptocurrency price movements using Bayesian networks, modelling relationships among six major cryptocurrencies, five traditional financial assets, and daily tweet volumes. BNs are selected for their ability to uncover causal dependencies and support probabilistic reasoning. Given the continuous nature of price data, the study systematically evaluates multiple discretisation strategies and proposes a structured BN construction pipeline. Using this pipeline, nine BNs are constructed for each cryptocurrency based on different discretisation methods and bin settings. Their predictive performance is evaluated using four metrics to identify the most accurate models and uncover key price-driving factors.


From a total of 54 constructed BNs, the top-performing BN for each cryptocurrency is selected based on a composite score. Results indicate that equal interval discretisation with two bins yields the most accurate and robust performance across metrics. Analysis of these metrics reveals that different cryptocurrencies respond differently to the same model configurations, underscoring the importance of coin-specific modelling. This suggests that a single investment strategy is unlikely to be effective across all cryptocurrencies due to their distinct behaviours and varying sensitivities to external drivers. Notably, equal interval discretisation with two bins emerged as the most robust configuration overall, though its effectiveness varied across coins.

% \AN{which finding actually? The first sentence only tells that equal intervals with two bins is the best discrimination setting. Your findings may come from the analysis of metrics.} 

% underscore the importance of coin-specific modelling, highlighting that a single investment strategy is unlikely to be effective across all cryptocurrencies due to their distinct behaviours and sensitivities to external drivers.

In addition, post-hoc analyses, including sensitivity analysis and strength of influence evaluation, highlight which variables exert the greatest impact on price dynamics. These methods enhance model transparency and help investors and researchers prioritise the most influential economic and social factors in the cryptocurrency market. The findings reveal that, although there is a connection between social media activity and cryptocurrency prices, the direction of influence often flows from price movement to social media engagement, suggesting that price changes tend to drive public discourse rather than result from it. Furthermore, the relationship between traditional financial assets and cryptocurrency prices is coin-dependent, reinforcing the need for tailored, coin-specific modelling approaches.


Building on these insights, several future directions are suggested. First, major geopolitical and macroeconomic events, such as China’s crypto bans~\citep{shahzad2018empirical}, the Russia–Ukraine conflict, and COVID-19~\citep{albulescu2021covid}
% \AN{possibly a new pandemic! I think COVID has already passed!}
, warrant deeper analysis for their effects on cryptocurrencies and global markets, particularly in detecting regime shifts. Second, incorporating expert knowledge into BN learning could enhance causal inference and model interpretability, especially when assessing the impact of influential figures~\citep{ante2021elon}. Third, as an important direction, to improve the robustness of causal inference in dynamic markets, a promising direction is to extend the current approach using dynamic Bayesian networks to account for temporal dependencies among influencing price factors~\cite{amirzadeh2023dynamic}.
Lastly, examining internal blockchain data, such as transactions, hash rates,  and integrating sentiment or search trends, including Google Trends, may improve prediction and support more adaptive investment strategies.

% \AN{You only cited two of your papers. Please cite all your cryptocurrency papers.}

\section*{Declarations}
\subsection*{Conflict of interest}
The authors have no competing interests to declare that are relevant to the content of this article.
\subsection*{Funding}
No funding was received to assist with the preparation of this manuscript.

\subsection*{Contributions}
All authors contributed equally to this study.


\subsection*{Ethical approval}
This article does not contain any studies with human participants and Animals performed by authors.

\subsection*{Data availability }
Data were derived from the following resource available in the public domain:
https://finance.yahoo.com/


% \bibliography{sn-article}

\newpage
\begin{appendix}


\section{Descriptive Analysis of Data Frames}\label{Descriptive}

In this section, we provide a statistical analysis of the cryptocurrency datasets used in this study, focusing on price and tweet volume across six major coins over the period from January 2018 to April 2023.

\subsection{Summary Statistics of Prices and Tweet Volumes}\label{Prices_Tweet_stats}


Tables~\ref{tab:coin_describtion} summarise the descriptive statistics of the daily closing prices after merging with traditional asset data and tweet volume. The dataset was constructed by aligning daily records from all available sources. The columns are self-explanatory, and the {\it No. Obs.} column indicates the number of valid observations in the merged dataset. While missing values in the price and macroeconomic variables were negligible, the Twitter data exhibited more frequent gaps. Consequently, the variation in observation counts across assets is primarily due to missing entries in the tweet data.



\begin{table*}[h]
 \centering
 \caption{Summary statistics for the daily close prices of cryptocurrencies.}
 \label{tab:coin_describtion}
 \begin{adjustbox}{max width=\textwidth}
 \begin{tabular}{lllllll}
 \hline
 \textbf{Altcoin} & \textbf{Mean} & \textbf{Std. Dev.} & \textbf{Min.} & \textbf{Median} & \textbf{Max.} & \textbf{No. Obs.} \\
 \hline
 Bitcoin       & 20664.35 & 16771.39 & 3242.48  & 11570.15 & 67566.83 & 1282 \\
 Binance Coin  & 161.88   & 183.92   & 4.53     & 29.23    & 675.68   & 1264 \\
 Ethereum      & 1169.65  & 1173.12  & 84.31    & 586.01   & 4151.87  & 1281 \\
 Litecoin      & 96.91    & 56.65    & 23.46    & 75.54    & 250.49   & 1281 \\
 Ripple        & 0.49     & 0.28     & 0.14     & 0.39     & 1.15     & 1277 \\
 Tether        & 1.0010   & 0.0022   & 0.9968   & 1.0004   & 1.0053   & 1307 \\
 \hline
 \end{tabular}
 \end{adjustbox}
\end{table*}


Tables~\ref{tab:coin_describtion} and~\ref{tab:tweet_no_describtion} show the descriptive statistics of daily tweet counts for each cryptocurrency. 
The table shows notable differences in tweet activity across altcoins. Bitcoin has the highest average and variability in tweet volume, indicating its central role in market discussions. Ethereum and Ripple also receive considerable attention, reflected in their relatively high tweet counts. In contrast, Binance Coin and Litecoin have lower and more stable tweet volumes, suggesting more limited or consistent public engagement. These differences highlight the varying levels of market attention each cryptocurrency attracts on social media.
\begin{table*}[h]
 \centering
 \caption{Summary statistics for daily tweet volumes related to altcoins.}
 \label{tab:tweet_no_describtion}
 \begin{adjustbox}{max width=\textwidth}
 \begin{tabular}{llllll}
 \hline
 \textbf{Altcoin} & \textbf{Mean} & \textbf{Std. Dev.} & \textbf{Min.} & \textbf{Median} & \textbf{Max.} \\
 \hline
 Bitcoin       & 75916.23  & 56546.05  & 445     & 50454.00  & 252186 \\
 Binance Coin  & 95.75     & 89.95     & 1       & 63.00     & 280 \\
 Ethereum      & 18146.17  & 14133.48  & 2418    & 14007.00  & 60569 \\
 Litecoin      & 1388.96   & 772.13    & 360     & 1154.00   & 3175 \\
 Ripple        & 12675.62  & 9730.45   & 2362    & 8479.00   & 35596 \\
 \hline
 \end{tabular}
 \end{adjustbox}
\end{table*}


To better illustrate the relationship between price trends and social media activity, Figure~\ref{fig:tweet_price_all} shows the daily prices and tweet volumes for each cryptocurrency. As seen in the figure, price fluctuations are most noticeable during 2021, coinciding with a surge in tweet activity. This period reflects a market momentum and investor attention, possibly linked to broader market regime shifts such as responses to COVID-19. Notably, Bitcoin and Ethereum exhibit consistently high tweet volumes throughout the period, suggesting ongoing investor engagement. In contrast, altcoins such as Binance Coin and Ripple display more spikes in tweet activity, typically aligned with sharp price movements. This indicates that discourse around some assets is more reactive and event-driven, whereas others maintain broader and more continuous public attention.

\begin{figure*}[h]
  \centering
\includegraphics[width=\linewidth,height=\textheight,keepaspectratio]{all_tweet_price.jpg}
\caption{Daily close prices (orange) and tweet volumes (blue) for each altcoin from 2018 to 2023. To reduce the impact of extreme outliers and enhance readability, values above the 90th percentile are removed.}
  \label{fig:tweet_price_all}
\end{figure*}


\subsection{Distributional Properties and Outlier Detection}\label{Outlier}


The descriptive statistics in Table~\ref{tab:coin_stats_descriptive} presents summary statistics on the distributional characteristics and outlier counts for each cryptocurrency. As can be seen, Bitcoin and Binance Coin exhibit moderate right skewness with light tails and no detected outliers. Ethereum shows a right-skewed distribution with moderate tail thickness and a moderate number of outliers, reflecting occasional deviations from central trends. Litecoin and Ripple display stronger positive skewness and heavier tails, consistent with more volatile behaviour and higher frequencies of extreme price changes. Ripple, in particular, shows the greatest asymmetry and volatility, as indicated by its high number of outliers. Tether, despite having relatively low skewness, exhibits extremely high kurtosis due to its narrow price range, where even small fluctuations are statistically identified as outliers.

\begin{table*}[!ht]
\centering
\caption{Descriptive statistics for each cryptocurrency: number of outliers, skewness, and kurtosis.}
\label{tab:coin_stats_descriptive}
\begin{adjustbox}{max width=\textwidth}
\begin{tabular}{lccc}
\hline
\textbf{Coin} & \textbf{Outliers} & \textbf{Skewness} & \textbf{Kurtosis} \\
\hline
Bitcoin       & 0   & 1.00 & -0.27 \\
Binance Coin  & 0   & 0.82 & -0.65 \\
Ethereum      & 29  & 1.10 &  0.16 \\
Litecoin      & 32  & 1.34 &  1.66 \\
Ripple        & 62  & 1.69 &  3.32 \\
Tether        & 223 & 0.73 & 16.97 \\
\hline
\end{tabular}
\end{adjustbox}
\end{table*}










\subsection{Stationarity Evaluation for Time Series Modelling}\label{adf}


To assess the stationarity of the data for time series modelling, we applied both the Augmented Dickey-Fuller (ADF) and Kwiatkowski–Phillips–Schmidt–Shin (KPSS) tests. A time series is considered stationary if it yields a p-value below 0.05 in the ADF test and above 0.05 in the KPSS test. As shown in Table~\ref{tab:stationarity_tests}, most cryptocurrency price series failed to satisfy both tests. 



\begin{table*}[h]
\centering
\caption{Results of ADF and KPSS tests for stationarity.}
\label{tab:stationarity_tests}
\begin{adjustbox}{max width=\textwidth}
\begin{tabular}{lcccc}
\toprule
\textbf{Altcoin} & \textbf{ADF Test Statistic} & \textbf{ADF p-value} & \textbf{KPSS Test Statistic} & \textbf{KPSS p-value} \\
\midrule
Bitcoin       & -1.7294 & 0.4160 & 3.0818 & 0.0100 \\
Binance Coin  & -1.8062 & 0.3774 & 3.9603 & 0.0100 \\
Ethereum      & -1.5446 & 0.5113 & 3.2603 & 0.0100 \\
Litecoin      & -2.9096 & 0.0442 & 0.6457 & 0.0185 \\
Ripple        & -2.9066 & 0.0446 & 0.8434 & 0.0100 \\
Tether        & -5.0250 & 0.0000 & 1.2232 & 0.0100 \\
\bottomrule
\end{tabular}
\end{adjustbox}
\end{table*}


Among the cryptocurrencies, only Litecoin and Ripple partially satisfied the stationarity criteria by rejecting the ADF null hypothesis but failing the KPSS test, suggesting weak stationarity. Tether, being a stablecoin, was the only asset that met both criteria and exhibited stationarity without transformation. Consequently, all other series were transformed using first-order differencing prior to modeling to ensure stationarity.






\end{appendix}


\begin{thebibliography}{10}
\providecommand{\url}[1]{{#1}}
\providecommand{\urlprefix}{URL }
\providecommand{\doi}[1]{\url{https://doi.org/#1}}


\bibitem{gajardo2018does}
G.~Gajardo, W.D. Kristjanpoller, M.~Minutolo (2018), Does bitcoin exhibit the
  same asymmetric multifractal cross-correlations with crude oil, gold and djia
  as the euro, great british pound and yen?
\newblock Chaos, Solitons \& Fractals \textbf{109}, 195--205

\bibitem{wang2021stock}
X.~Wang, K.~Yang, T.~Liu (2021), Stock price prediction based on morphological
  similarity clustering and hierarchical temporal memory.
\newblock IEEE Access \textbf{9}, 67241--67248

\bibitem{griffith2023cryptocurrency}
T.~Griffith, D.~Clancey-Shang (2023), Cryptocurrency regulation and market
  quality.
\newblock Journal of International Financial Markets, Institutions and Money
  \textbf{84}, 101744

\bibitem{amirzadeh2024dynamic}
R.~Amirzadeh, D.~Thiruvady, A.~Nazari, M.S. Ee (2024), Dynamic evolution of
  causal relationships among cryptocurrencies: an analysis via bayesian
  networks.
\newblock Knowledge and Information Systems pp. 1--16

\bibitem{erzurumlu2020one}
Y.O. Erzurumlu, T.~Oygur, A.~Kirik (2020), One size does not fit all: external
  driver of the cryptocurrency world.
\newblock Studies in Economics and Finance \textbf{37}(3), 545--560

\bibitem{inamdar2019predicting}
A.~Inamdar, A.~Bhagtani, S.~Bhatt, P.M. Shetty, \emph{Predicting cryptocurrency
  value using sentiment analysis}, in \emph{2019 International Conference on
  Intelligent Computing and Control Systems (ICCS)} (IEEE, 2019), pp. 932--934

\bibitem{poyser2019exploring}
O.~Poyser (2019), Exploring the dynamics of bitcoin’s price: a bayesian
  structural time series approach.
\newblock Eurasian Economic Review \textbf{9}(1), 29--60

\bibitem{kurihara2017market}
Y.~Kurihara, A.~Fukushima (2017), The market efficiency of bitcoin: a weekly
  anomaly perspective.
\newblock Journal of Applied Finance and Banking \textbf{7}(3), 57

\bibitem{guesmi2019portfolio}
K.~Guesmi, S.~Saadi, I.~Abid, Z.~Ftiti (2019), Portfolio diversification with
  virtual currency: Evidence from bitcoin.
\newblock International Review of Financial Analysis \textbf{63}, 431--437

\bibitem{ferreira2019contagion}
P.~Ferreira, {\'E}.~Pereira (2019), Contagion effect in cryptocurrency market.
\newblock Journal of Risk and Financial Management \textbf{12}(3), 115

\bibitem{wang2019cryptocurrency}
P.~Wang, W.~Zhang, X.~Li, D.~Shen (2019), Is cryptocurrency a hedge or a safe
  haven for international indices? a comprehensive and dynamic perspective.
\newblock Finance Research Letters \textbf{31}, 1--18

\bibitem{sevinc2020bayesian}
V.~Sevinc, O.~Kucuk, M.~Goltas (2020), A bayesian network model for prediction
  and analysis of possible forest fire causes.
\newblock Forest Ecology and Management \textbf{457}, 117723

\bibitem{heckerman2008tutorial}
D.~Heckerman (2008), A tutorial on learning with bayesian networks.
\newblock Innovations in Bayesian networks pp. 33--82

\bibitem{amirzadeh2022applying}
R.~Amirzadeh, A.~Nazari, D.~Thiruvady (2022), Applying artificial intelligence
  in cryptocurrency markets: A survey.
\newblock Algorithms \textbf{15}(11), 428

\bibitem{nojavan2017comparative}
F.~Nojavan, S.S. Qian, C.A. Stow (2017), Comparative analysis of discretization
  methods in bayesian networks.
\newblock Environmental Modelling \& Software \textbf{87}, 64--71

\bibitem{corbet2018exploring}
S.~Corbet, A.~Meegan, C.~Larkin, B.~Lucey, L.~Yarovaya (2018), Exploring the
  dynamic relationships between cryptocurrencies and other financial assets.
\newblock Economics Letters \textbf{165}, 28--34

\bibitem{lau2017return}
M.C.K. Lau, S.A. Vigne, S.~Wang, L.~Yarovaya (2017), Return spillovers between
  white precious metal etfs: The role of oil, gold, and global equity.
\newblock International Review of Financial Analysis \textbf{52}, 316--332

\bibitem{charfeddine2020investigating}
L.~Charfeddine, N.~Benlagha, Y.~Maouchi (2020), Investigating the dynamic
  relationship between cryptocurrencies and conventional assets: Implications
  for financial investors.
\newblock Economic Modelling \textbf{85}, 198--217

\bibitem{ji2018network}
Q.~Ji, E.~Bouri, R.~Gupta, D.~Roubaud (2018), Network causality structures
  among bitcoin and other financial assets: A directed acyclic graph approach.
\newblock The Quarterly Review of Economics and Finance \textbf{70}, 203--213

\bibitem{stosic2018collective}
D.~Stosic, D.~Stosic, T.B. Ludermir, T.~Stosic (2018), Collective behavior of
  cryptocurrency price changes.
\newblock Physica A: Statistical Mechanics and its Applications \textbf{507},
  499--509

\bibitem{shi2020correlations}
Y.~Shi, A.K. Tiwari, G.~Gozgor, Z.~Lu (2020), Correlations among
  cryptocurrencies: Evidence from multivariate factor stochastic volatility
  model.
\newblock Research in International Business and Finance \textbf{53}, 101231

\bibitem{bouri2021return}
E.~Bouri, X.V. Vo, T.~Saeed (2021), Return equicorrelation in the
  cryptocurrency market: Analysis and determinants.
\newblock Finance Research Letters \textbf{38}, 101497

\bibitem{aslanidis2019analysis}
N.~Aslanidis, A.F. Bariviera, O.~Mart{\'\i}nez-Iba{\~n}ez (2019), An analysis
  of cryptocurrencies conditional cross correlations.
\newblock Finance Research Letters \textbf{31}, 130--137

\bibitem{giudici2021crypto}
P.~Giudici, G.~Polinesi (2021), Crypto price discovery through correlation
  networks.
\newblock Annals of Operations Research \textbf{299}(1), 443--457

\bibitem{malladi2021time}
R.K. Malladi, P.L. Dheeriya (2021), Time series analysis of cryptocurrency
  returns and volatilities.
\newblock Journal of Economics and Finance \textbf{45}(1), 75--94

\bibitem{lamon2017cryptocurrency}
C.~Lamon, E.~Nielsen, E.~Redondo (2017), Cryptocurrency price prediction using
  news and social media sentiment.
\newblock SMU Data Sci. Rev \textbf{1}(3), 1--22

\bibitem{abraham2018cryptocurrency}
J.~Abraham, D.~Higdon, J.~Nelson, J.~Ibarra (2018), Cryptocurrency price
  prediction using tweet volumes and sentiment analysis.
\newblock SMU Data Science Review \textbf{1}(3), 1

\bibitem{rouhani2019crypto}
S.~Rouhani, E.~Abedin (2019), Crypto-currencies narrated on tweets: a sentiment
  analysis approach.
\newblock International Journal of Ethics and Systems

\bibitem{kraaijeveld2020predictive}
O.~Kraaijeveld, J.~De~Smedt (2020), The predictive power of public twitter
  sentiment for forecasting cryptocurrency prices.
\newblock Journal of International Financial Markets, Institutions and Money
  \textbf{65}, 101188

\bibitem{kim2021dynamics}
K.~Kim, S.Y.T. Lee, S.~Assar (2021), The dynamics of cryptocurrency market
  behavior: sentiment analysis using markov chains.
\newblock Industrial Management \& Data Systems

\bibitem{larranaga2012review}
P.~Larra{\~n}aga, H.~Karshenas, C.~Bielza, R.~Santana (2012), A review on
  probabilistic graphical models in evolutionary computation.
\newblock Journal of Heuristics \textbf{18}(5), 795--819

\bibitem{koller2009probabilistic}
D.~Koller, N.~Friedman.
\newblock Probabilistic graphical models, massachusetts (2009)

\bibitem{yu2015modified}
H.~Yu, F.~Khan, V.~Garaniya (2015), Modified independent component analysis and
  bayesian network-based two-stage fault diagnosis of process operations.
\newblock Industrial \& Engineering Chemistry Research \textbf{54}(10),
  2724--2742

\bibitem{zhang2016expert}
G.~Zhang, V.V. Thai (2016), Expert elicitation and bayesian network modeling
  for shipping accidents: A literature review.
\newblock Safety science \textbf{87}, 53--62

\bibitem{de2011efficient}
C.P. De~Campos, Q.~Ji (2011), Efficient structure learning of bayesian networks
  using constraints.
\newblock The Journal of Machine Learning Research \textbf{12}, 663--689

\bibitem{amirkhani2016exploiting}
H.~Amirkhani, M.~Rahmati, P.J. Lucas, A.~Hommersom (2016), Exploiting
  experts’ knowledge for structure learning of bayesian networks.
\newblock IEEE transactions on pattern analysis and machine intelligence
  \textbf{39}(11), 2154--2170

\bibitem{gheisari2016bnc}
S.~Gheisari, M.R. Meybodi (2016), Bnc-pso: structure learning of bayesian
  networks by particle swarm optimization.
\newblock Information Sciences \textbf{348}, 272--289

\bibitem{uusitalo2007advantages}
L.~Uusitalo (2007), Advantages and challenges of bayesian networks in
  environmental modelling.
\newblock Ecological modelling \textbf{203}(3-4), 312--318

\bibitem{ciaian2016economics}
P.~Ciaian, M.~Rajcaniova, d.~Kancs (2016), The economics of bitcoin price
  formation.
\newblock Applied Economics \textbf{48}(19), 1799--1815

\bibitem{iccelliouglu2019investigation}
C.{\c{S}}. {\.I}{\c{c}}ellio{\u{g}}lu, S.~{\"O}ner (2019), An investigation on
  the volatility of cryptocurrencies by means of heterogeneous panel data
  analysis.
\newblock Procedia Computer Science \textbf{158}, 913--920

\bibitem{ricciardi2000behavioral}
V.~Ricciardi, H.K. Simon (2000), What is behavioral finance?
\newblock Business, Education \& Technology Journal \textbf{2}(2), 1--9

\bibitem{balcilar2021dynamic}
M.~Balcilar, Z.A. Ozdemir, H.~Ozdemir (2021), Dynamic return and volatility
  spillovers among s\&p 500, crude oil, and gold.
\newblock International Journal of Finance \& Economics \textbf{26}(1),
  153--170

\bibitem{hayes2017cryptocurrency}
A.S. Hayes (2017), Cryptocurrency value formation: An empirical study leading
  to a cost of production model for valuing bitcoin.
\newblock Telematics and informatics \textbf{34}(7), 1308--1321

\bibitem{okorie2020crude}
D.I. Okorie, B.~Lin (2020), Crude oil price and cryptocurrencies: evidence of
  volatility connectedness and hedging strategy.
\newblock Energy economics \textbf{87}, 104703

\bibitem{hasan2022exploring}
M.B. Hasan, M.K. Hassan, Z.A. Karim, M.M. Rashid (2022), Exploring the hedge
  and safe haven properties of cryptocurrency in policy uncertainty.
\newblock Finance Research Letters \textbf{46}, 102272

\bibitem{amirzadeh2023framework}
R.~Amirzadeh, D.~Thiruvady, A.~Nazari, M.S. Ee (2023), A framework for
  empowering reinforcement learning agents with causal analysis: Enhancing
  automated cryptocurrency trading.
\newblock arXiv preprint arXiv:2310.09462

\bibitem{konigstorfer2020applications}
F.~K{\"o}nigstorfer, S.~Thalmann (2020), Applications of artificial
  intelligence in commercial banks--a research agenda for behavioral finance.
\newblock Journal of behavioral and experimental finance \textbf{27}, 100352

\bibitem{rossi2018efficient}
M.~Rossi, A.~Gunardi, et~al. (2018), Efficient market hypothesis and stock
  market anomalies: Empirical evidence in four european countries.
\newblock Journal of Applied Business Research (JABR) \textbf{34}(1), 183--192

\bibitem{malkiel2003efficient}
B.G. Malkiel (2003), The efficient market hypothesis and its critics.
\newblock Journal of economic perspectives \textbf{17}(1), 59--82

\bibitem{gurdgiev2020herding}
C.~Gurdgiev, D.~O’Loughlin (2020), Herding and anchoring in cryptocurrency
  markets: Investor reaction to fear and uncertainty.
\newblock Journal of Behavioral and Experimental Finance \textbf{25}, 100271

\bibitem{mai2018does}
F.~Mai, Z.~Shan, Q.~Bai, X.~Wang, R.H. Chiang (2018), How does social media
  impact bitcoin value? a test of the silent majority hypothesis.
\newblock Journal of management information systems \textbf{35}(1), 19--52

\bibitem{lund2018power}
N.F. Lund, S.A. Cohen, C.~Scarles (2018), The power of social media
  storytelling in destination branding.
\newblock Journal of destination marketing \& management \textbf{8}, 271--280

\bibitem{shiller2017narrative}
R.J. Shiller (2017), Narrative economics.
\newblock American economic review \textbf{107}(4), 967--1004

\bibitem{smales2022investor}
L.A. Smales (2022), Investor attention in cryptocurrency markets.
\newblock International Review of Financial Analysis \textbf{79}, 101972

\bibitem{lin2021investor}
Z.Y. Lin (2021), Investor attention and cryptocurrency performance.
\newblock Finance Research Letters \textbf{40}, 101702

\bibitem{zhu2021investor}
P.~Zhu, X.~Zhang, Y.~Wu, H.~Zheng, Y.~Zhang (2021), Investor attention and
  cryptocurrency: Evidence from the bitcoin market.
\newblock PLoS One \textbf{16}(2), e0246331

\bibitem{wolk2020advanced}
K.~Wo{\l}k (2020), Advanced social media sentiment analysis for short-term
  cryptocurrency price prediction.
\newblock Expert Systems \textbf{37}(2), e12493

\bibitem{death2015good}
R.G. Death, F.~Death, R.~Stubbington, M.K. Joy, M.~van~den Belt (2015), How
  good are bayesian belief networks for environmental management? a test with
  data from an agricultural river catchment.
\newblock Freshwater biology \textbf{60}(11), 2297--2309

\bibitem{guesne2024mind}
S.J. Guesn{\'e}, T.~Hanser, S.~Werner, S.~Boobier, S.~Scott (2024), Mind your
  prevalence!
\newblock Journal of Cheminformatics \textbf{16}(1), 43

\bibitem{hu2021multisite}
K.~Hu, M.~Wang, Y.~Liu, H.~Yan, M.~Song, J.~Chen, Y.~Chen, H.~Wang, H.~Guo,
  P.~Wan, et~al. (2021), Multisite schizophrenia classification by integrating
  structural magnetic resonance imaging data with polygenic risk score.
\newblock NeuroImage: Clinical \textbf{32}, 102860

\bibitem{lee2021neuroimaging}
J.J. Lee, H.J. Kim, M.~{\v{C}}eko, B.y. Park, S.A. Lee, H.~Park, M.~Roy, S.G.
  Kim, T.D. Wager, C.W. Woo (2021), A neuroimaging biomarker for sustained
  experimental and clinical pain.
\newblock Nature medicine \textbf{27}(1), 174--182

\bibitem{saleh2018implementation}
A.~Saleh, K.~Puspita, A.~Sanjaya, et~al., \emph{Implementation of equal width
  interval discretization on SMARTER method for selecting computer laboratory
  assistant}, in \emph{2018 6th International Conference on Cyber and IT
  Service Management (CITSM)} (IEEE, 2018), pp. 1--4

\bibitem{min2009global}
H.~Min, \emph{A global discretization and attribute reduction algorithm based
  on k-means clustering and rough sets theory}, in \emph{2009 Second
  international symposium on knowledge acquisition and modeling}, vol.~2 (IEEE,
  2009), pp. 92--95

\bibitem{ahmed2020k}
M.~Ahmed, R.~Seraj, S.M.S. Islam (2020), The k-means algorithm: A comprehensive
  survey and performance evaluation.
\newblock Electronics \textbf{9}(8), 1295

\bibitem{sinaga2020unsupervised}
K.P. Sinaga, M.S. Yang (2020), Unsupervised k-means clustering algorithm.
\newblock IEEE access \textbf{8}, 80716--80727

\bibitem{bayesfusion2017genie}
L.~BayesFusion (2017), Genie modeler.
\newblock User Manual. Available online: https://support. bayesfusion.
  com/docs/(accessed on 21 October 2019)

\bibitem{mahjoub2011software}
M.A. Mahjoub, K.~Kalti, \emph{Software comparison dealing with Bayesian
  networks}, in \emph{International Symposium on Neural Networks} (Springer,
  2011), pp. 168--177

\bibitem{tessema2021enhancing}
H.D. Tessema, S.L. Abebe, \emph{Enhancing just-in-time defect prediction using
  change request-based metrics}, in \emph{2021 IEEE International Conference on
  Software Analysis, Evolution and Reengineering (SANER)} (IEEE, 2021), pp.
  511--515

\bibitem{asvija2021security}
B.~Asvija, R.~Eswari, M.~Bijoy (2021), Security threat modelling with bayesian
  networks and sensitivity analysis for iaas virtualization stack.
\newblock Journal of Organizational and End User Computing (JOEUC)
  \textbf{33}(4), 44--69

\bibitem{hosseini2019development}
S.~Hosseini, M.~Sarder (2019), Development of a bayesian network model for
  optimal site selection of electric vehicle charging station.
\newblock International Journal of Electrical Power \& Energy Systems
  \textbf{105}, 110--122

\bibitem{lu2020risk}
Q.~Lu, P.a. Zhong, B.~Xu, F.~Zhu, Y.~Ma, H.~Wang, S.~Xu (2020), Risk analysis
  for reservoir flood control operation considering two-dimensional
  uncertainties based on bayesian network.
\newblock Journal of Hydrology \textbf{589}, 125353

\bibitem{castillo1997sensitivity}
E.~Castillo, J.M. Guti{\'e}rrez, A.S. Hadi (1997), Sensitivity analysis in
  discrete bayesian networks.
\newblock IEEE Transactions on Systems, Man, and Cybernetics-Part A: Systems
  and Humans \textbf{27}(4), 412--423

\bibitem{saltelli2019so}
A.~Saltelli, K.~Aleksankina, W.~Becker, P.~Fennell, F.~Ferretti, N.~Holst,
  S.~Li, Q.~Wu (2019), Why so many published sensitivity analyses are false: A
  systematic review of sensitivity analysis practices.
\newblock Environmental modelling \& software \textbf{114}, 29--39

\bibitem{meurisse2022risk}
N.~Meurisse, B.G. Marcot, O.~Woodberry, B.I. Barratt, J.H. Todd (2022), Risk
  analysis frameworks used in biological control and introduction of a novel
  bayesian network tool.
\newblock Risk Analysis \textbf{42}(6), 1255--1276

\bibitem{walker2001allowing}
D.~Walker, J.~Fox-Rushby (2001), Allowing for uncertainty in economic
  evaluations: qualitative sensitivity analysis.
\newblock Health Policy and Planning \textbf{16}(4), 435--443

\bibitem{coupe2000computational}
V.M. Coup{\'e}, F.V. Jensen, U.~Kj{\ae}rulff, L.C. van~der Gaag (2000), A
  computational architecture for n-way sensitivity analysis of bayesian
  networks

\bibitem{shahzad2018empirical}
F.~Shahzad, G.~Xiu, J.~Wang, M.~Shahbaz (2018), An empirical investigation on
  the adoption of cryptocurrencies among the people of mainland china.
\newblock Technology in Society \textbf{55}, 33--40

\bibitem{albulescu2021covid}
C.T. Albulescu (2021), Covid-19 and the united states financial markets’
  volatility.
\newblock Finance Research Letters \textbf{38}, 101699

\bibitem{ante2021elon}
L.~Ante (2021), How elon musk's twitter activity moves cryptocurrency markets.
\newblock Available at SSRN 3778844

\bibitem{amirzadeh2023dynamic}
R.~Amirzadeh, D.~Thiruvady, A.~Nazari, M.S. Ee (2023), Dynamic bayesian
  networks for predicting cryptocurrency price directions: Uncovering causal
  relationships.
\newblock arXiv preprint arXiv:2306.08157

\end{thebibliography}

\end{document}
