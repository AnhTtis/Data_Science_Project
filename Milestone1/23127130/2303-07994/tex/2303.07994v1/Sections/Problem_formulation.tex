\section{Control Problem}
\label{sec:problem_formulation}
\subsection{Interventional X-ray System}
The considered interventional X-ray system is depicted in Fig. \ref{fig:CLEA}. While it has 3 degrees of freedom, this paper considers only the roll axis. The roll axis body rotates in a roller-based guidance attached to the sleeve, thus positioning the X-ray source and detector in this dimension. It is driven by a permanent magnet DC motor and an amplifier with a maximum input of 5 [V] through a transmission consisting of gears and belts. The rotation is measured using an incremental encoder with an effective resolution of $0.0119$ [deg]. Encoder measurements and power for the X-ray source are supplied through a cable at the side of the setup, which is acting as a dynamic link. The software runs on a Speedgoat system with $T_s= \frac{1}{500}$ [s].

\begin{figure}[t]
	\centering
	\includegraphics[width=\linewidth]{./Figures/CLEA.pdf}
	\caption{Interventional X-ray system with roll axis with orientation $\theta$ positioning the X-ray source and detector. The configuration-dependent cable forces (\protect \drawrectanglelegend{darkmagenta}) and friction characteristics of the guidance (\protect \drawrectanglelegend{darkcyan}) limit the effectiveness of physical-model-based feedforward control.}
	\label{fig:CLEA}
\end{figure}

This mechanical design, motivated and constrained by the use around medical personnel, introduces the following hard-to-model nonlinear dynamics.
\begin{enumerate}
	\item The mass distribution is unbalanced, resulting in configuration-dependent gravitational forces.
	\item The cable acts as a configuration-dependent inertia, and is tensioned for large negative $\theta$, acting as a one-sided spring.
	\item The friction characteristics in the guidance depend on the normal forces acting on the contact surface of the rollers, and are thus configuration-dependent. 
\end{enumerate} 
The configuration-dependent gravitational forces can be quantitatively captured by a physical model. However, the exact way these parasitic cable and friction forces depend on the configuration $\theta$ is exceptionally hard to model from a physics-based perspective, limiting the tracking performance of physical-model-based feedforward control. 


\subsection{Control Approach}
The main dynamics of the interventional X-ray system, i.e., the configuration-dependent gravitational forces, can be described by physics, but the cable and friction forces are only qualitatively understood and hard to model. Therefore, the physical model $\mathcal{M}_\zeta$ with parameters $\zeta$ is complemented by a neural network $\mathcal{C}_\phi$ to learn these hard-to-model dynamics, resulting in a parallel PGNN feedforward controller $\mathcal{F}_{\zeta,\phi}$, see Fig. \ref{fig:control_setup}. Additionally, a simple PD feedback controller is employed to compensate both unknown external disturbances as well as dynamics uncompensated by the PGNN feedforward controller.

The aim of this paper is to learn the parameters $\zeta,\phi$ of the PGNN feedforward controller (defined in the next section) that compensate the hard-to-model cable and friction force, thereby increasing tracking performance, in such a way that $\mathcal{M}_\zeta$ stays interpretable and $\mathcal{C}_\phi$ increases performance by learning only unmodeled dynamics, i.e., while ensuring complementarity. In this paper, an inverse system identification approach is taken, in which $\zeta,\phi$ are learned based on a dataset $\mathcal{D}$ describing the system's dynamics consisting of inputs $\hat{u}(k)$ corresponding to outputs $\theta(k)$ with $k=1,\ldots,N$ that has been obtained in advance, e.g., feedback or iterative learning control data. 
\begin{figure}[t]
	\centering
	\includegraphics[width=\linewidth]{./Figures/Control_setup.pdf}
	\caption{Two-degree-of-freedom control configuration for roll axis of an interventional X-ray system with feedback controller $C$ and feedforward controller $\mathcal{F}_{\zeta,\phi}$ consisting of physical model $\mathcal{M}_\zeta$ and neural network $\mathcal{C}_\phi$. }
	\label{fig:control_setup}
\end{figure}
