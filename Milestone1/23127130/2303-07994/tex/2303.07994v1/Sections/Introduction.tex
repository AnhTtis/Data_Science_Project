\section{Introduction}
\label{sec:introduction}
Image-guided therapy (IGT) in general, and interventional X-rays specifically, are a key technology in healthcare that directly improve treatment quality by enabling minimally invasive therapies through visualization of patient tissue \citep{Jolesz2014}. The interventional X-ray is one of the main IGT systems and is able to create 3D images of the relevant tissue by combining a sequence of 2D X-ray snapshots \citep{Pelc2014}. It is specifically geared toward use during surgery to generate real-time images of the relevant tissue, enabling small surgical tools as opposed to making large incisions, resulting in faster patient recovery.

Accurate feedforward control \citep{Clayton2009, Butterworth2009} is essential during operation of an interventional X-ray system to guarantee both high imagine quality as well as patient and operator safety. First, accurate feedforward control allows for compensation of the system's dynamics before errors occur, resulting in accurate tracking of the desired setpoint for the imaging sequence, minimizing visual artefacts such as motion blur. Second, mismatch between motor torques predicted by feedforward control and actual applied torques, i.e., the feedback controller contribution, can be used as a basis to distinguish nominal operating conditions from anomalies such as collisions, increasing safety.

Feedforward controllers based on physical models \citep{Devasia2002, Zou2004, 489285} have limited performance due to hard-to-model or unknown dynamics present in an interventional X-ray: dynamics not included in the physical model are not compensated through feedforward control, resulting in reproducible tracking errors. At the same time, these physical models are highly flexible, i.e., result in the same performance for different trajectories \citep{6837472}, and can describe the majority of the dynamics in terms of simple expressions using a few interpretable parameters, such as mass and snap coefficients \citep{Boerlage2003}.
% \citep{Lambrechts2005}

In sharp contrast to a physical-model-based approach, neural network feedforward controllers can compensate all predictable dynamics of any system given a sufficiently rich (recurrent) parametrization \citep{Goodfellow2016, Schafer2006} and have been successfully applied in feedforward control to improve tracking performance \citep{HUNT19921083, Sorensen1999, Otten1997}. However, neural networks are challenging to optimize \citep{80336}, uninterpretable, and lack the ability to extrapolate based on physical prior knowledge \citep{Schoukens2019}. %, i.e., lack task flexibility

Physics-guided neural networks (PGNNs) aim to reconcile the performance of neural networks with the flexibility and interpretability of physical models by explicitly introducing the physics into the structure \citep{7959606} or optimization criterion \citep{2017arXiv171011431K}. These PGNNs can be shown to increase performance over physical-model-based feedforward controllers \citep{Bolderman2021}. Despite their improvement, the contributions of the physical model and neural network are not always well distinguished. Complementarity is obtained by explicitly separating the physical model and neural network contribution through orthogonal projection-based regularization \citep{Kon2022PhysicsGuided}, resulting in physically meaningful model coefficients.

%Although important improvements have been made in motion feedforward for systems with hard-to-model dynamics such as interventional X-rays, at present the physical models do not integrate well with data-driven neural network-based models. The aim of this paper is to develop and implement an approach that integrates physical models and neural network models on an industrial interventional X-ray system. To this end, the recent developments in \citep{Kon2022PhysicsGuided} are employed, where .

The main contribution of this paper is the development and experimental validation of the PGNN feedforward framework to learn and subsequently compensate the hard-to-model dynamics present in the interventional X-ray. This consists of the following subcontributions.
\begin{enumerate}[label=C\arabic*)]
	\item A PGNN feedforward parametrization for the interventional X-ray system, consisting of a physical model describing its equations of motion and a suitable neural network (Section \ref{sec:physics_guided_neural_network}).
	\item An orthogonal projection-based regularizer to ensure complementarity of the physical model and neural network (Section \ref{sec:orthogonal_projection_based_regularization}).
	\item Experimental validation of the PGNN feedforward controller on an interventional X-ray system, illustrating its superior tracking performance over physical-model-based feedforward control (Section \ref{sec:experimental_validation}).
\end{enumerate}
%\begin{enumerate}[label=C\arabic*)]
%	\item A PGNN feedforward parametrization for the interventional X-ray (Section \ref{sec:physics_guided_neural_network}).
%	\item An orthogonal projection-based regularization to ensure complementarity of the physical model and neural network (Section \ref{sec:orthogonal_projection_based_regularization}).
%	\item Experimental validation of the PGNN feedforward controller on an interventional X-ray (Section \ref{sec:experimental_validation}).
%\end{enumerate}