\section{Conclusion}

In this work, we present our framework of multi remote energy hubs with capture of CO2 enabled in an energy demand center and its valorization by synthesizing methane in remote renewable energy hubs. We demonstrate the feasibility of serving the energy demand at the horizon 2050 of an entire country with only renewable energy and gas power plant fueled by synthetic methane while decarbonizing the energy and industry sectors on a case study implying Belgium as energy demand center and two RREHs: Greenland and Algeria. Our reference scenario exhibits a gas price of 136.0€/MWh instead of 149.7€/MWh in \cite{Berger2021} where only direct air capture was available in the RREH in order to feed CO2 into the methanation process. This shows the potential of Post Combustion Carbon Capture installations in the context of remote renewable energy hubs supply chains. We also derive a cost of CO2 of 163€ per ton in order to avoid any emission in the industrial and energy sector in Belgium. Finally, our model effectively captures the "competition" between different RREHs and is able to select exactly in which investments should be prioritized. In our simulations, the investments were made only for the RREH located in Algeria. In this respect, it would be interesting to study further how the different devices structuring the RREH in Greenland should be modified to become competitive with the RREH located in Algeria. This could be done for example by modifying the wind turbines selected for the Greenland hub so that they can operate with  higher nominal wind speeds and higher cut-off speeds in order to better exploit the strong winds in this area.