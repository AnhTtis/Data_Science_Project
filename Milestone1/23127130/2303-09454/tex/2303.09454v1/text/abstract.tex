\begin{abstract}

In this paper, we propose a multi-RREH (Remote Renewable Energy Hub) based optimization framework. This framework allows a valorization of CO2 using carbon capture technologies. This valorization is grounded on the idea that CO2 gathered from the atmosphere or post combustion can be combined with hydrogen to produce synthetic methane. The hydrogen is obtained from water electrolysis using renewable energy (RE). Such renewable energy is generated in RREHs, which are locations where RE is cheap and abundant (e.g., solar PV in the Sahara Desert, or wind in Greenland). We instantiate our framework on a case study focusing on Belgium and 2 RREHs, and we conduct a techno-economic analysis. This analysis highlights the key role played by the cost of the two main carbon capture technologies: Post Combustion Carbon Capture (PCCC) and Direct Air Capture (DAC). In addition, we use our framework to derive a carbon price threshold above which carbon capture technologies may start playing a pivotal role in the decarbonation process of our industries. For example, this price threshold may give relevant information for calibrating the EU Emission Trading System so as to trigger the emergence of the multi-RREH.

% \textcolor{blue}{Raph: should we mention ENS / CO2 cap as well here? I do not think personally Victor}


% \textbf{Remarks}: we say solar \textbf{PV} in Sahara but just "wind" for Greenland, should we say solar and wind or solar pv and wind turbines?
% Also, either we put a capital letter on words of abbreviation or we do not, but we have to be consistent (e.g. RE - renewable energy)?

\end{abstract}