\section{A Multi-RREH Approach}

The Multi Remote Renewable Energy Hub approach amounts in modelling several RREHs in order to set up a competitive framework between CO2 exploitation channels. Each hub comes with its own characteristics, such as renewable energy type and potential or distance to the demand energy centre that may impact its competitiveness.

\begin{figure}
    \centering
    \includegraphics[scale=0.5]{MultiRemoteEnergyHub.png}
    \caption{A schematic illustration of the multi remote renewable energy hub. Several CO2 exploitation channels compete for CO2 resource.}
    \label{fig:multi_REH_scheme}
\end{figure}

The methodology proposed in this paper encompass the possibility to model several RREH.


 \textcolor{red}{Raph: one possible idea would be to develop a multi-agent approach for CO2 emitters as well: some CO2 emitter may have cheaper CO2 sources, such as cement work. They could, in theory, have and advantage over other CO2 sources.}
