\documentclass{ecos_2023}

% \title{ECOS 2023: Manuscript Template}
\title{Towards CO2 valorization  in a multi remote renewable energy hub framework}


\author{V. Dachet\textsuperscript{a}, A. Benzerga\textsuperscript{b}, R. Fonteneau\textsuperscript{c}, D. Ernst\textsuperscript{d}\textsuperscript{e}}

\address{
\textsuperscript{a} University of Liège,
Liège,
Belgium,
victor.dachet@uliege.be
\textbf{CA}
\and
\textsuperscript{b} University of Liège,
Liège,
Belgium,
abenzerga@uliege.be
\and
\textsuperscript{c} University of Liège,
Liège,
Belgium,
raphael.fonteneau@uliege.be
\and
\textsuperscript{d} University of Liège,
Liège,
Belgium,
dernst@uliege.be
\and
\textsuperscript{e} Telecom Paris, Institut Polytechnique de Paris,
Paris,
France,
damien.ernst@telecom-paris.fr}



\keywords{\normalsize
CO2 Valorization, Energy Hub, Multi-Energy Systems, Optimization of Energy Systems, Sector Coupling.
}


\usepackage[utf8]{inputenc}
\usepackage{amsmath}
\usepackage{amsfonts}
\usepackage{amssymb}
\usepackage{graphics}
\usepackage{graphicx}
\usepackage{url}
\usepackage{xcolor}


\usepackage{hyperref}

\usepackage[title]{appendix}

% Bibliography
% \usepackage{cite}
\usepackage[numbers]{natbib} % ECOS FORMAT with use of \citet{}
\bibliographystyle{plainnat}
% \usepackage[style=authoryear]{biblatex}
% \addbibresource{bib.bib}

\abstract{\normalsize


Over the past few years, there has been a significant amount of research focused on studying the ReLU activation function, with the aim of achieving neural network convergence through over-parametrization. However, recent developments in the field of Large Language Models (LLMs) have sparked interest in the use of exponential activation functions, specifically in the attention mechanism.

Mathematically, we define the neural function $F: \R^{d \times m} \times  \mathbb{R}^d \rightarrow \mathbb{R}$ using an exponential activation function. Given a set of data points with labels $\{(x_1, y_1), (x_2, y_2), \dots, (x_n, y_n)\} \subset \mathbb{R}^d \times \mathbb{R}$ where $n$ denotes the number of the data. Here $F(W(t),x)$ can be expressed as $F(W(t),x) := \sum_{r=1}^m a_r \exp(\langle w_r, x \rangle)$, where $m$ represents the number of neurons, and $w_r(t)$ are weights at time $t$. It's standard in literature that $a_r$ are the fixed weights and it's never changed during the training. We initialize the weights $W(0) \in \mathbb{R}^{d \times m}$ with random Gaussian distributions, such that $w_r(0) \sim \mathcal{N}(0, I_d)$ and initialize $a_r$ from random sign distribution for each $r \in [m]$.

Using the gradient descent algorithm, we can find a weight $W(T)$ such that $\| F(W(T), X) - y \|_2 \leq \epsilon$ holds with probability $1-\delta$, where $\epsilon \in (0,0.1)$ and $m = \Omega(n^{2+o(1)}\log(n/\delta))$. To optimize the over-parametrization bound $m$, we employ several tight analysis techniques from previous studies [Song and Yang arXiv 2019, Munteanu, Omlor, Song and Woodruff ICML 2022]. 

 

}

% Packages to define the euro symbol in equations
\usepackage{siunitx}
\usepackage{eurosym}

\DeclareSIUnit{\EUR}{\text{\euro}}
\sisetup{
  per-mode = fraction,
  inter-unit-product = \ensuremath{{}\cdot{}},
}

% package to remove for final version
\usepackage[normalem]{ulem} % allows to strikes through a text
% \usepackage{todonotes} % allows to do notes


% TO REMOVE REMAINING DOTS IN AUTOREF
\renewcommand\thesection{\arabic{section}.}

\renewcommand*{\figureautorefname}{Fig.}

\def\equationautorefname#1#2\null{%
  #1(#2\null)%x
}


\begin{document}




\maketitle



\section{Introduction}
\label{sec:introduction}
% \begin{itemize}
%     % Diffusion of FL
%     \item {\st{Diffusion of FL}}
%     % Security threats to FL
%     \item {\st{Security threats to FL with particular focus on model poisoning}}
%     % Limitations of existing countermeasures
%     \item {\st{Current countermeasures (e.g., KRUM) and their limitations}}
%     % Proposed method and its advantages
%     \item {\st{Intuitive description of the proposed method and its difference (i.e., advantages) w.r.t. state of the art}}
%     % Main contributions
%     \item {\st{Summary of the main contributions of this work}}
%     % Paper's structure and organization
%     \item {\st{Paper's structure and organization}}
% \end{itemize}

% Diffusion of FL
Recently, {\em federated learning} (FL) has emerged as the leading paradigm for training distributed, large-scale, and privacy-preserving machine learning (ML) systems~\cite{mcmahan2017googleai,mcmahan2017aistats}. 
The core idea of FL is to allow multiple edge clients to collaboratively train a shared, global model without disclosing their local private training data.
%Specifically, an FL system consists of a central server and many edge clients; 
A typical FL round involves the following steps: {\em(i)} the server randomly picks some clients and sends them the current, global model; {\em(ii)} each selected client locally trains its model with its own private data; then, it sends the resulting local model to the server;\footnote{Whenever we refer to global/local model, we mean global/local model {\em parameters}.} {\em(iii)} the server updates the global model by computing an \emph{aggregation function}, usually the average (FedAvg), on the local models received from clients.
% \begin{enumerate}
%     \item[{\em(i)}] the server sends the current, global model to the clients and appoints some of them for training;
%     \item[{\em(ii)}] each selected client locally trains its copy of the global model with its own private data; then, it sends the resulting local model back to the server;\footnote{Whenever we refer to global/local model, we mean global/local model {\em parameters}.}
%     \item[{\em(iii)}] the server updates the global model by computing an \emph{aggregation function} on the local models received from clients (by default, the average, also referred to as FedAvg~\cite{mcmahan2017aistats}).
% \end{enumerate}
This process goes on until the global model converges. %(e.g., after a certain number of rounds or other similar stopping criteria).
%\\
% The advantages of FL over the traditional, centralized learning paradigm are undoubtedly clear in terms of flexibility/scalability (clients can join/disconnect from the FL network dynamically), network communications (only model weights\footnote{We will use \textit{parameters} and \textit{weights} interchangeably.} are exchanged between clients and server), and privacy (each client's private training data is kept local at the client's end and not uploaded to the server).
\\
% Security threats to FL
%However, the growing adoption of FL also raises security concerns~\cite{costa2022covert}, particularly about its confidentiality, integrity, and availability.
Although its advantages over standard ML, FL also raises security concerns~\cite{costa2022covert}. %, particularly about its confidentiality, integrity, and availability~\cite{costa2022covert}.
% OLD, LONG VERSION
% Indeed, some work deals with privacy leakage that may expose the local data of some clients~\cite{melis2019sp}. 
% A large body of work, instead, investigates attacks that usually aim to detriment the predictive accuracy of the learned global model. For instance, \emph{data poisoning} attacks achieve this goal by letting an adversary pollute the training set of some corrupt FL clients with maliciously crafted examples~\cite{jagielski2018sp}.
% Similarly, in \emph{model poisoning} the attacker attempts to tweak the global model weights~\cite{bhagoji2019pmlr} by directly perturbing the local model's weights of some infected FL clients before these are sent to the central server for aggregation, usually via so-called Byzantine attacks. 
% It turns out that Byzantine model poisoning attacks severely impact standard FedAvg; therefore, more robust aggregation functions must be designed to make FL systems secure.
Here, we focus on \emph{untargeted model poisoning} attacks~\cite{bhagoji2019pmlr}, where an adversary attempts to tweak the global model weights %\footnote{We will use the terms \textit{parameters} and \textit{weights} interchangeably.} 
by directly perturbing the local model's parameters of some infected clients before these are sent to the central server for aggregation.
In doing so, the adversary aims to jeopardize the global model \textit{indiscriminately} at inference time.
Such model poisoning attacks severely impact standard FedAvg; therefore, more robust aggregation functions must be designed to secure FL systems.
\\
% In this paper, we focus on designing a novel robust aggregation scheme at the server's end to contrast the effect of Byzantine model poisoning attacks.
%
% Current countermeasures and their limitations
%Several countermeasures have been proposed in the literature to combat model poisoning attacks on FL systems.
% Some methods use simple statistics more robust than plain average to smooth the impact of malicious updates (e.g., Trimmed Mean and FedMedian~\cite{yin2018icml}). 
% Other defenses implement outlier detection techniques to discard malicious updates from the aggregation performed at the server's end. Those are either based on heuristics (e.g., Krum/Multi-Krum~\cite{blanchard2017nips} and Bulyan~\cite{mhamdi2018pmlr}) or data-driven approaches (e.g., K-means clustering~\cite{shen2016acm} or DnC via spectral analysis~\cite{shejwalkar2021ndss}). 
% Finally, some strategies rely on a centralized ``source of trust'' to spot potential malicious updates (e.g., FLTrust~\cite{cao2020fltrust}).
% Several countermeasures have been proposed in the literature to combat model poisoning attacks on FL systems, i.e., to discard possible malicious local updates from the aggregation performed at the server's end. 
% These techniques range from simple statistics more robust than plain average (e.g., Trimmed Mean and FedMedian~\cite{yin2018icml}) to outlier detection heuristics (e.g., Krum/Multi-Krum~\cite{blanchard2017nips} and Bulyan~\cite{mhamdi2018pmlr}) or data-driven approaches (e.g., spectral analysis via K-means clustering~\cite{shen2016acm} or spectral analysis), or methods based on ``source of trust'' (e.g., FLTrust~\cite{cao2020fltrust}).
% OLD, LONG VERSION
%Several countermeasures have been proposed in the literature to combat Byzantine model poisoning attacks on FL systems.
% Descriptive statistics
% For example, Trimmed Mean and FedMedian aggregate local model updates using more robust statistics than standard average~\cite{yin2018icml}.
%
% % Heuristics for outlier detection
% Many existing Byzantine-resilient strategies implement some outlier detection heuristics to discard the model updates sent by potentially malicious clients from the input of the aggregation function.
% One of the most popular heuristics is Krum~\cite{blanchard2017nips}.
% This strategy tries to mitigate the impact of Byzantine attacks by selecting as a global model the local model with the smallest sum of Euclidean distances to {\em all} the other local models.
% Although powerful, Krum requires the server to know (or, at least, estimate) the number of malicious FL clients upfront, which is generally impossible in a realistic attack scenario. %
% Moreover, Krum may become ineffective for complex, high-dimensional model parameter spaces due to the curse of dimensionality.
% Bulyan~\cite{mhamdi2018pmlr} tries to overcome this issue by combining Krum with a variant of Trimmed Mean.
% % Data-driven outlier detection
% Other strategies use data-driven outlier detection techniques -- e.g., via K-means clustering~\cite{shen2016acm} -- to spot potential malicious local model updates. 
% %For instance, Shen et al. propose to cluster local model updates with K-means and thus identify outliers.
%
% % Other techniques
% As far as the server is concerned, any local model received can be from a potential malicious client. 
% FLTrust~\cite{cao2020fltrust} assumes the server acts as a client, i.e., trains a local model on an additional {\em trustworthy} dataset at the server's end and compares it against all the local models from other clients. 
% This way, the server can rely on some ``source of trust'' when discarding potentially malicious clients.
%\\
% Limitations of existing Byzantine-resilient strategies
Unfortunately, existing defense mechanisms either rely on simple heuristics (e.g., Trimmed Mean and FedMedian by~\cite{yin2018icml}) or need strong and unrealistic assumptions to work effectively (e.g., foreknowledge or estimation of the number of malicious clients in the FL system, as for Krum/Multi-Krum~\cite{blanchard2017nips} and Bulyan~\cite{mhamdi2018pmlr}, which, however, cannot exceed a fixed threshold).
Furthermore, outlier detection methods using K-means clustering~\cite{shen2016acm} or spectral analysis like DnC~\cite{shejwalkar2021ndss} do not directly consider the temporal evolution of local model updates received.
Finally, strategies like FLTrust~\cite{cao2020fltrust} require the server to collect its own dataset and act as a proper client, thereby altering the standard FL protocol.
\\
% OLD, LONG VERSION
% Overall, existing Byzantine-resilient strategies are either simple heuristics (e.g., FedMedian) or, if they are more complex, they rely on strong and unrealistic assumptions to work effectively (e.g., knowing the number of malicious clients in the FL system in advance, as for Krum and alike).
% Furthermore, data-driven outlier detection methods do not consider the temporary evolution of local model updates received (e.g., K-means clustering). 
% Finally, strategies like FLTrust requires the server to collect its own dataset and act as a proper client, thereby altering the standard FL protocol.
%
% Description of the proposed method
This work introduces a novel pre-aggregation \textit{filter} robust to untargeted model poisoning attacks. Notably, this filter $(i)$ operates without requiring prior knowledge or constraints on the number of malicious clients and $(ii)$ inherently integrates temporal dependencies. 
The FL server can employ this filter as a preprocessing step before applying \textit{any} aggregation function, be it standard like FedAvg or robust like Krum or Bulyan.
Specifically, we formulate the problem of identifying corrupted updates as a multidimensional (i.e., matrix-valued) time series anomaly detection task. 
The key idea is that legitimate local updates, resulting from well-calibrated iterative procedures like stochastic gradient descent (SGD) with an appropriate learning rate, show \textit{higher predictability} compared to malicious updates. This hypothesis stems from the fact that the sequence of gradients (thus, model parameters) observed during legitimate training exhibit regular patterns, as validated in Section~\ref{subsec:intuition}. %until convergence. 
%This regularity may be more pronounced for smooth convex loss functions, but it can still be captured within an appropriate time window, even for more complex and convoluted loss surfaces. 
%We provide evidence of this claim in Appendix~B, where we show that the average mutual information (i.e., ``predictability''), calculated over pairs of legitimate model updates sent at different FL rounds, is significantly higher than the corresponding computation for a malicious client.
\\
Inspired by the matrix autoregressive (MAR) framework for multidimensional time series forecasting~\cite{chen2021je}, we propose the FLANDERS ({\em \textbf{F}ederated \textbf{L}earning meets \textbf{AN}omaly \textbf{DE}tection for a \textbf{R}obust and \textbf{S}ecure}) filter.
The main advantages of FLANDERS over existing strategies like FLDetector~\cite{zhao2020multivariate} are its resilience to large-scale attacks, where $50\%$ or more FL participants are hostile, and the capability of working under realistic non-iid scenarios.
We attribute such a capability to two key factors: $(i)$ FLANDERS works without knowing a priori the ratio of corrupted clients, and $(ii)$ it embodies temporal dependencies between intra- and inter-client updates, quickly recognizing local model drifts caused by evil players. Below, we summarize our main contributions:

\begin{itemize}
\item[{\em(i)}]
We provide empirical evidence that the sequence of models sent by legitimate clients is more predictable than those of malicious participants performing untargeted model poisoning attacks.
\\
\item[{\em(ii)}] 
We introduce FLANDERS, the first pre-aggregation filter for FL robust to untargeted model poisoning based on multidimensional time series anomaly detection.
\\
\item[{\em(iii)}] 
We integrate FLANDERS into Flower,\footnote{\scriptsize{\url{https://flower.dev/}}} a popular FL simulation framework for reproducibility.
\\
\item[{\em(iv)}] 
We show that FLANDERS improves the robustness of the existing aggregation methods under multiple settings: different datasets, client's data distribution (non-iid), models, and attack scenarios.
\\
\item[{\em(v)}] 
We publicly release all the implementation code of FLANDERS along with our experiments.\footnote{\scriptsize{\url{https://anonymous.4open.science/r/flanders_exp-7EEB}}}
\end{itemize}

% Paper's structure and organization
The remainder of the paper is structured as follows. %some related work and the current state-of-the-art solutions to security issues that FL entails. 
Section~\ref{sec:background} covers background and preliminaries. 
In Section~\ref{sec:related}, we discuss related work.
Section~\ref{sec:problem} and Section~\ref{sec:method} describe the problem formulation and the method proposed. % to tackle it. 
Section~\ref{sec:experiments} gathers experimental results. %, and Section~\ref{sec:limitations} discusses some limitations of this work.
Finally, we conclude in Section~\ref{sec:conclusion}.
 %discusses the limitations of this work and draws future research directions.
%reports conclusions and draws perspectives for future research directions.

%%%%%%% OLD %%%%%%%
%to overcome the resilience of Byzantine failures in distributed Stochastic Gradient Descent computations. 
% The strength of Krum is its time complexity, which is linear in the gradient dimension. 
% However, the robustness of the approach is guaranteed for gradient-based learning applications only when the majority of the clients are not compromised. 
% Besides, the aggregation mechanism of Krum, as well as that of similar methods, is robust from a coarse-grained perspective and does not provide solutions to errors and perturbations that may occur at inference time.
%A related approach to~\cite{blanchard2017nips} is the work of Su et al.~\cite{su2016dc}. Here, the authors propose an iterated approximate agreement to tackle a multi-layer scenario attacked by Byzantine agents. 
%However, the method works efficiently on the sole discrete context and it is inapplicable to continuous state environments.
%\gabri{Maybe, we should just talk about the main limitations of existing countermeasures without digging into their details (or, we can just mention Krum as this is the most popular one). I will move the description of all these methods to the Related Work section.}

\section{Related work}
% There is extensive recent work on speeding up analytical queries due to the need for consistent execution times in the face of the explosive growth in the volume of available data.
% In this section, we divide existing work into two categories: maintaining data freshness in MVs (\Cref{sec:server_side}) and optimizations for minimizing ad-hoc query latency (\Cref{sec:client_side}).

% \subsection{Maintaining Data Freshness in MVs}
% \label{sec:server_side}
% There exists a variety of data warehousing applications aimed at supporting low-latency analytical queries on fresh data.
% In particular, these applications require efficiency in the propagation of newly ingested data into downstream MVs.
 
\mypara{Efficient MV Refresh}
Incremental view maintenance (IVM) aims to update MVs to reflect newly ingested data, taking advantage of already computed results to perform the update in a manner more efficient than computing from scratch (full refresh)
~\cite{ahmad2012dbtoaster,mcsherry2013differential,armbrust2013generalized,zeng2016iolap, palpanas2002incremental, griffin1995incremental, agiwal2021napa, braun2015analytics}. 
There is an abundance of work in IVM, including incremental updates on duplicate values~\cite{griffin1995incremental}, non-distributive aggregate functions~\cite{palpanas2002incremental}, higher-order views~\cite{ahmad2012dbtoaster}, and sliding windows~\cite{braun2015analytics}. 
More recent works also investigate the scalability aspect of IVM, proposing scale-independent updates~\cite{armbrust2013generalized} and sampled views~\cite{zeng2016iolap}. Since \system is applicable to arbitrary SQL statements, \system is orthogonal to and is fully compatible with existing IVM techniques.

\mypara{MV Refresh Scheduling}
There exist works on scheduling the refresh of a MV set focusing on resolving cyclic dependencies~\cite{folkert2005optimizing}, minimizing weighted average staleness~\cite{golab2009scheduling}, and prioritizing MVs with the highest speedups on predicted future queries~\cite{ahmed2020automated}.
\system's scheduling to speed up the end-to-end refresh of the MV set is not addressed in existing works.

\mypara{DAG Workflow Scheduling}
The execution of workloads consisting of individual jobs with acyclic dependencies is a well-studied topic~\cite{apacheoozie,sparkdag,marchal2018parallel,bathie2020revisiting,baruah2022ilp}; many of these techniques can be applied to MV refresh runs studied in this paper.
Existing workflow scheduling systems such as Apache Oozie~\cite{apacheoozie}, Apache Airflow~\cite{airflow}, and Spark DAG scheduler~\cite{sparkdag} automate the execution of user-defined workflows following a topological order.
There are recent works aimed at finding more optimal execution orders in terms of peak memory usage~\cite{marchal2018parallel, bathie2020revisiting} and execution time on parallel platforms~\cite{baruah2022ilp}.
While \system is designed for use with MV refresh runs/workloads, our technique on joint scheduling and optimization can be reasonably applied to general workloads as a possible future direction.

% \paragraph{Incremental MV indexing}
% Update-optimized indices such as the log-structured merge-trees (LSM)~\cite{o1996log} are used for indexing MVs due to frequent updates induced by data ingestion~\cite{gupta2016mesa,agiwal2021napa}.
% \system is orthogonal to indexing: \system is capable of efficiently performing MV refresh runs regardless of whether the individual MVs are indexed or not.

% \subsection{Ad-hoc Query Latency Reduction}
% \label{sec:client_side}

% The minimization of ad-hoc analytical query response times is a well-studied topic due to latency being negatively correlated with the productivity of a data analyst during a data analysis session~\cite{liu2014effects}.
% Sessions are commonly conducted within visualization systems that contain a variety of optimization techniques to ensure that query response times fall within a certain latency tolerance.

% \mypara{Data prefetching}
% Data is often loaded into memory on a by-need basis in visualization systems to minimize interference with user-issued query computations~\cite{mani2017effective,xin2021enhancing,galakatos2017revisiting, yan2020auto, battle2016dynamic, crotty2016case, jalaparti2018netco}. 
% Query-time data retrieval can be significantly expedited by anticipating the data usage of the user in future queries and pre-loading the data into memory during the downtime between user queries (`think time'). SMART~\cite{mani2017effective} prefetches data for modified versions of current user-issued queries with different filters and dimensions. A-WARE~\cite{crotty2016case} maintains a sample store constantly refined through ingesting data based on speculations of future plots.
% ForeCache~\cite{battle2016dynamic} uses an SVM to predict the user's current analysis phase and accordingly prefetches data tiles partitioned based on different numerical values. NetCo predicts future queries via log analysis, and solves an ILP formulation to prefetch data to maximize the number of SLO-meeting queries~\cite{jalaparti2018netco}.
% In the case of MV refresh workloads, `think time' is nonexistent as individual MVs are refreshed back-to-back, rendering data prefetching techniques non-applicable.

\mypara{Intermediate Data Caching}
Some existing data visualization systems cache user-defined variables to support the typical incremental construction of data visualizations~\cite{zgraggen2016progressive, eichmann2020idebench} during data analysis sessions~\cite{jupyter, rstudio, colab}. 
Recent work proposes a management scheme for these cached variables under a memory constraint that greedily keeps variables with the highest estimated time savings based on predicted future user behavior via neural networks~\cite{xin2021enhancing}.
While useful for data visualization, a greedy approach to memory management fails to achieve satisfactory results compared to \system.

\mypara{Intermediate Result Reuse}

There exist works on storing intermediate results from computations to speedup future computations~\cite{yang2018intermediate, dursun2017revisiting, nagel2013recycling, michiardi2019memory, galakatos2017revisiting}.
Studied topics include the identification of reuse opportunities by finding overlaps in computation graphs of successive jobs~\cite{yang2018intermediate, michiardi2019memory},
selective storage under a space constraint with heuristics such as reuse probability~\cite{dursun2017revisiting}, expected savings~\cite{yang2018intermediate}, and recompute-storage cost difference~\cite{nagel2013recycling},
and rewriting incoming jobs to take advantage of stored intermediates~\cite{galakatos2017revisiting}.
These works share similarity with \system in their selection of items to store under a memory constraint, however, \system's problem setting requires it to uniquely consider the joint (re)ordering of job executions along with the selection of items.

% work that considers both job execution (re)order as well as intermediate result caching with a bounded amount of memory. but notably lack the joint aspect of \system and cannot be used to achieve immediate speedup on an incoming MV refresh run if no intermediates are stored beforehand. 

\mypara{Incremental Query Processing} Incremental processing (IQP) is useful for cases where not all data required for a query is immediately available. Similar to online aggregation~\cite{hellerstein1997online}, initial results of a query are computed on a subset of required data and progressively refined as the rest of the required data arrives in a predictable pattern~\cite{tang2019intermittent,wangtempura}. Tang et al. propose a dynamic programming formulation to pick intermediate states to store in memory given a limited memory budget~\cite{tang2019intermittent}. Tempura rewrites the query plan for more efficient execution based on predicted data arrival patterns~\cite{wangtempura}. While similarities exist between the problem setting of IQP and \system, such as management of bounded memory, \system notably includes additional joint optimization for the order of MV updates.

% \paragraph{Sampling}
% Sampling has seen wide use in visualization systems for reducing the computation time of ad-hoc queries by computing an approximate result over a subset of data as exact results are not always required by the user~\cite{crotty2016case, mani2017effective, zgraggen2014panoramicdata, kraska2021northstar, galakatos2017revisiting, kandula2016quickr}. 
% Commonly studied topics in sampling for ad-hoc queries include complex query sampling~\cite{kandula2016quickr}, rare event aggregation~\cite{kraska2021northstar, galakatos2017revisiting}, and maintaining consistency between related sampled visualizations~\cite{zgraggen2014panoramicdata}.
% Sampling server-side at the MV level compromises the assumptions of downstream applications and is thus not considered in \system.

% \paragraph{Progressive visualization}
% The latency tolerance for time-consuming queries can be circumvented by presenting a partially-computed visualization to the user within the tolerance, which is then incrementally refined until it is fully accurate~\cite{rahman2017ve, zgraggen2016progressive, crotty2015vizdom, kraska2021northstar, kamat2017infiniviz}.
% Example plots which benefit from progressive visualization include bar charts~\cite{kamat2017infiniviz} and heatmaps~\cite{rahman2017ve}.
% Similar to sampling, study on this topic is orthogonal to \system as pushing out partially-updated MVs compromises downstream assumptions.

%\input{text/RREH.tex}

%\section{A Multi-RREH Approach}

The Multi Remote Renewable Energy Hub approach amounts in modelling several RREHs in order to set up a competitive framework between CO2 exploitation channels. Each hub comes with its own characteristics, such as renewable energy type and potential or distance to the demand energy centre that may impact its competitiveness.

\begin{figure}
    \centering
    \includegraphics[scale=0.5]{MultiRemoteEnergyHub.png}
    \caption{A schematic illustration of the multi remote renewable energy hub. Several CO2 exploitation channels compete for CO2 resource.}
    \label{fig:multi_REH_scheme}
\end{figure}

The methodology proposed in this paper encompass the possibility to model several RREH.


 \textcolor{red}{Raph: one possible idea would be to develop a multi-agent approach for CO2 emitters as well: some CO2 emitter may have cheaper CO2 sources, such as cement work. They could, in theory, have and advantage over other CO2 sources.}


\section{CO2 Valorisation in a Multi-Remote Renewable Energy Hubs Approach}
\label{sec:Multi-RREH}

The Remote Renewable Energy Hub concept was first introduced in \cite{Berger2021} where the authors proposed a hub for synthesizing CH4 based on hydrogen and CO2 captured from the air thanks to a methanation unit. This concept has emerged within the context of global grid \cite{chatzivasileiadis2013global} and multi-energy systems approaches. These approaches aim at optimising the generation and utilisation of renewable energy (RE) by both (i) looking for abundant and cheap RE fields, (ii) taking advantage of daily/seasonal complementary of RE, as well as (iii) using power-to-gas technologies for better addressing RE generation fluctuations and meet e-fuels demand to act as a substitute for molecules derived nowadays from fossil fuels.


% \textcolor{blue}{While in the original article \cite{Berger2021} the methanation unit has access to CO2 by a Direct Air Capture Unit and the demand is satisfied by one unique RREH located in Algeria, in this paper, we propose to explore the possibility to valorize CO2 captured by Post Combustion Capture techniques at the energy demand center (EDC). Regarding the RREH, we also differ from the original paper by directly proposing a multi-RREH approach. This results in a framework where the EDC plays the role of a CO2 provider that can serve a set of multiple RREHs  $\{RREH_1, \ldots, RREH_h  \}$. Each hub $RREH_i ( 1 \leq  i  \leq h )$ comes with its own characteristics, such as renewable energy type, potential and distance to the EDC that may impact its competitiveness and the way it will be provided with CO2 by the EDC.}

In the original article \cite{Berger2021}, the methanation unit was supplied with CO2 by a Direct Air Capture unit, and the energy demand was fulfilled by a single RREH located in Algeria. However, in this paper, we propose to investigate the feasibility of valorizing CO2 captured through Post Combustion Capture techniques at the energy demand center (EDC). Additionally, we deviate from the original paper by introducing a multi-RREH approach, wherein the EDC serves as a CO2 provider to a set of multiple RREHs, denoted as ${RREH_1, \ldots, RREH_h }$. Each hub $RREH_i ( 1 \leq i \leq h )$ has its unique characteristics, such as renewable energy type, potential, distance from the EDC, and means of CO2 transport from the EDC, which can affect its competitiveness.



% We carry out a full analysis of this model on a realistic case study with Belgium as EDC and two RREHs: one located in Greenland and one in Algeria. In \autoref{fig:RREH_model}, you have access to a schematic view of the system fully described below in \autoref{subsec:model_config}.


% The system was modelled using GBOML, as was done in \cite{Berger2021}, and all model code has been made available online.

% Furthermore, this multi-RREH framework can be easily extended to include multiple EDCs.

In order to illustrate the concepts discussed above, we have developed a model for a multi-RREH system based on the following assumptions: (i) the EDC is Belgium, encompassing its gas and electricity demands as well as its CO2 emissions, (ii) there are two RREHs: one situated in the Sahara desert with access to solar and wind resources, and another in Greenland benefiting from the high-quality wind fields in the region. A detailed schematic of the resulting system is shown in \autoref{fig:RREH_model}. Similar to \cite{Berger2021}, we employed the GBOML language \cite{Miftari2022}, a recently developed language tailored for energy system optimization (refer to \autoref{sec:modelling} for more information), to model the system.

We note that the GBOML model code with two RREHs and one EDC system is available online\footnote{\url{https://gitlab.uliege.be/smart_grids/public/gboml/-/tree/master/examples}} and can  be easily extended to add additional RREHs and EDCs.

% In \cite{Berger2021}, the methanation unit has access to CO2 by a Direct Air Capture Unit and the demand is satisfied by one unique RREH located in Algeria. 

% In this paper, we further investigate this RREH concept by introducing CO2 capture by Post Combustion Carbon Capture where CO2 is abundant \textit{i.e.} in an energy demand center (EDC) and transport it, for example by boats, to the methanation unit located in the RREH. Synthetic methane is generated at the hub, and shipped back to the EDC. 

% Moreover, we extend the concept to multiple RREHs in order to set up a competitive framework between methane producers. Each hub comes with its own characteristics, such as renewable energy type, potential and distance to the EDC that may impact its competitiveness.

% \textcolor{red}{
% traditional approach to generate CH4 in remote renewable energy hub was based on direct air capture technologies. In this work, we propose to investigate the cost effectiveness of capturing CO2 in places where CO2 is abundant due to heavy industries by post combustion carbon capture technologies (PCCC). Those places are often linked with an important energy demand. Therefore, capturing CO2 in this places, then transporting it towards the RREH. In this RREH, producing CH4 and transporting it back to the energy demand center. In this paper, we 
% }
% \textcolor{red}{
% \begin{itemize}
%     \item Initially generating ch4 in RREH is based on DAC
%     \item Investigation wether capturing co2 in smokes would be interesting or at least complementary to DAC + Transport e.g. shipped
%     \item Case study: one energy demand center and 2 RREHS.
%     \item full modelling with gboml describe below as it was the case in \cite{Berger2021}.
%     \item Easy extension to multiple energy demand centers. 
%     \item All models required have been put online. 
% \end{itemize}
% }

% The traditional approach for generating ch4 in RREH is based on DAC see .... 
% In this paper we would like to investigate wehter capturing co2 in smod-kes would be interesting or at least complementary. 
% This CO2 will be for example shipped. 

% IN this paper, in the particular in one energy demand center and 2 RREHS: one in .. 

% we propose a full modelling with gboml describe below as it was the case in [3].
% this methology can be easily extended to multiple energy demand centers. 

% All models required have been put online. 





% In this paper, we further investigate this RREH concept by introducing Post Combustion Carbon Capture as well as CO2 transportation. CO2 is captured in Belgium, and transported to a RREH by carrier boats. Synthetic methane is generated at the hub, and shipped back to Belgium. A schematic is provided in \autoref{fig:RREH_model}. Examples of RREH include solar energy in the Sahara desert \cite{Berger2021}, as well as wind energy in Greenland \cite{Radu2019393}.

\begin{figure}[p]
    \vspace{-3ex}
    \centering   \includegraphics[width=\textwidth, height=\textheight,keepaspectratio]{Figures/20230313_GR-BE-DZ.pdf}
    \caption{A schematic illustration of the remote energy hub. CO2 being captured, it may be used to synthesize fuel either locally either in a remote energy hub where renewable energy may be cheaper and more abundant.}
    \label{fig:RREH_model}
\end{figure}



% The Multi Remote Renewable Energy Hub approach amounts in modelling several RREHs in order to set up a competitive framework between CO2 exploitation channels. Each hub comes with its own characteristics, such as renewable energy type, potential and distance to the demand energy centre that may impact its competitiveness.


% CHatGPT REFORMULATION:

% The Remote Renewable Energy Hub (RREH) concept was introduced by Berger et al. (2021) within the context of global grid and multi-energy systems approaches. These approaches aim to optimize renewable energy (RE) generation and utilization by identifying abundant and cost-effective RE fields, exploiting daily/seasonal complementarity, and employing power-to-gas technologies to address RE generation fluctuations and fossil fuels demand.

% n their study, Berger et al. (2021) proposed a unique RREH located in Algeria, where the methanation unit has access to CO2 via a Direct Air Capture Unit. In this paper, we further investigate the RREH concept by introducing Post Combustion Carbon Capture technology to capture CO2 from energy demand centers and transport it to the methanation unit located in the RREH, where synthetic methane is produced and shipped back to the energy demand centers. We also extend the concept to multiple RREHs, which allows for the comparison of the competitiveness of different CO2 exploitation channels based on their renewable energy potential and distance to the energy demand centers.

% To demonstrate the feasibility of our approach, we present a full analysis of a realistic case study involving Belgium as the energy demand center and two RREHs located in Greenland and Algeria. The system is modeled in GBOML, and the models are available online. The proposed framework is also easily extendable to multiple energy demand centers. The system's schematic view is provided in Figure \ref{fig:RREH_model}, and its complete description is presented in Section \ref{subsec:model_config}.

% \section{Modelling}
\label{fig:Modelling}

This section describes the core of our methodology, a linear program (LP). The LP is modelled using a recently proposed framework, the Graph-Based Optimization Modeling Language (GBOML) that we present in section \ref{subsec:GBOML}. The LP formalization is provided in section \ref{subsec:modellingAssumptions}.

\subsection{GBOML}
\label{subsec:GBOML}

    In this work, we rely on a the Graph-Based Optimization Modelling Language (GBOML), a recently developed framework for enhancing the modelling of energy systems \cite{Miftari2022}. One key aspect when using GBOML is to divide the model into nodes and clusters; in our case, we consider the following nodes: on-shore Belgium, off-shore Belgium, and as many RRH as desired.
    
    The Belgian on-shore node models energy demands, CO2 emissions, PCCC and export of CO2. The RREH handles imports of CO2, production of H$_2$ from renewable energy, transformation of CO2 into CO2 neutral synthetic CH$_4$.


\subsection{Main modelling Assumptions}
\label{subsec:modellingAssumptions}


\subsubsection{Objective}
\label{subsub:objective}

    The objective function of the model is obtained by summing costs associated with all technologies and carriers involved in the model.
    
    Minimize the overall costs function (Capex, Opex, Energy not served, $CO_2$ emissions).

\subsection{CO2 emitters}
    \label{sub:CO2_emitters}
    
    The energy demand center includes several $CO_2$ emitters, mainly gas fired power plants and industries. Such quantities of $CO2_2$ are captured using PCCC technologies. 
    
    We denote by $\mathcal{P}_{CO_2}$ the set of all technologies that emit CO2 and that may be equiped with PCCC technologies.
    
    Each CO2 emitter is modelled according to the following characteristics:
    \begin{itemize}
        \item CO2 emission curve, in CO2 kt/h, $q^{CO_2,(i)}_t$, where $i \in \{ 1, \ldots , N^{CO_2}  \}$ indexes CO2 emitters, \textcolor{red}{Victor note: in \cite{BERGER_power_to_gas_2020106039} the notation is $Q^{p}_{CO_2, t}$ with $p \in \mathcal{P}_{CO_2}$ is there a reason to not follow the notation of Mathias? }
        \item For each CO2 emitter , there is the possibility to capture CO2 using PCCC technologies.
    \end{itemize}



\subsection{CO2 capture}

    Two main carbon capture processes are considered in this paper: Post Combustion Carbon Capture (PCCC) and Direct Air Capture (DAC).
    
    Different PCCC technologies exist, but absorption solvent-based methods is currently the leading one \cite{madejski2022methods}. It is based on a reaction between CO2 and a solvent in aqueous solution. The capture is achieved in a two-step process. In the first step, the post-combustion gas reacts with the solvent in order to catch the carbon dioxyde. In the second step, the $CO_2$ is regenerated at high temperature in a stripper.
    
    PCCC is modelled using the following characteristics:
    \begin{itemize}
        \item $\Phi^{p,CC}$: Electrical energy required per unit mass of CO2 captured via PCCC for technology $p \in \mathcal{P}_{CO_2}$.
        \item $Q^{c, CC}_{CO_2,t}$: fraction of CO2 mass flow of technology $p \in \mathcal{P}_{CO_2}$ captured via PCCC at time $t \in \{ 0, \ldots, T-1  \}$.
    \end{itemize}
    
    
    Direct Air Capture technologies are mainly divided into two categories: (i) high temperature aqueous solutions and (ii) low temperature solid sorbent systems \cite{fasihi2019techno}, the latter showing lower heat supply costs.
    
    DAC is modelled using the following characteristics:
    
    \begin{itemize}
        \item $\Phi^{DAC}_E$: Electrical energy required per unit mass of CO2 captured using DAC [GWh/kt]. 
        \item $\Phi^{DAC}_{NG}$: Natural gas energy required per unit mass of CO2 captured using DAC [GWh/kt].
        \item $Q^{DAC}_{CO_2, t}$: CO2 mass flow exiting DAC units at time $t \in \{ 0, \ldots, T-1  \}$ [kt/h].
        \item $Q^{DAC, A}_{CO_2, t}$: CO2 mass flow captured from the atmosphere via DAC at time $t \in \{ 0, \ldots, T-1  \}$ [kt/h].
    \end{itemize}
    


\subsubsection{Remote Renewable Energy Hubs}
\label{subsub:RREH}

    We denote by $\mathcal H$ the set of all RREH. Each hub $h \in \mathcal{H}$ is characterised using the following characteristics:
    \begin{itemize}
        \item Renewable energy resources, modelled in the form of capacity factor time series $\pi^{PV,h}_t$ (for solar PV) and $\pi^{W,h}_t$ (for wind energy) 
        \item Electrolysis and methanation plants. \textcolor{red}{Victor Note: cfr equations 17 to 21 and eq 28 to 31 in \cite{BERGER_power_to_gas_2020106039}}
        \item Geographical characteristics, including distance to the Energy Demand Center  $d^{h} \in \mathbb{R}$.
    \end{itemize}
    
    
    
    
    Many constraints associated with renewable energy generation, 


\subsubsection{CO2 and Synthetic methane transportation}
    Carriers : Pipes and boats.
    
    
    In this first version of our model, we assume a simplified version of the CO2 transportation system by considering a continuous flow between the RREHs and the energy demand center.

\subsubsection{Decision variables}
\label{subsub:carriers}

    Among variables to be optimised, there are both sizing and operation variables. Sizing variables relates with investment decision variables.
    
    how large generation and transport capacities should be sized. 

\subsubsection{Main cost assumptions}

    PCCC and DAC costs, as well as CO2  transportation costs, play a key role in our results.

\section{Modelling}\label{sec:modelling}
    % This section gives insight into the optimization framework this work relies on. 
    % The multi-energy system model proposed in this work is built using the GBOML language introduced in \cite{Miftari2022}, a recently developed language dedicated to the modelling graph based optimisation of multi-energy systems. This class of problem can be seen as optimisation on graphs, in a sense that multi-energy system can be seen as a set of nodes $\mathcal{N}$ that contribute to the (linear) objective, while also including local constraints, as well as hyperedges $\mathcal{E}$ for modelling constraints between nodes, e.g., in our setting, between RREHs and the EDC. 
    
    % The formalism in this work follows the one introduced in \cite{Berger2021}.
    % The entire system is defined by a set of nodes $\mathcal{N}$ and a set of hyperedeges $\mathcal{E}$. The optimisation horizon is denoted by $T$, and time-steps are indexed by $t \in \mathcal{T}$ with $\mathcal{T} =  \{1, \ldots, T\}$.
    
    % A node $n \in \mathcal{N}$ is defined by internal $X^{n}$ and external $Z^{n}$ variables. Internal variables describe the specific characteristics of the unit. For example, the nominal power capacity installed of the asset.
    
    % Moreover, equality constraints $h_i(X^{n}, Z^{n}, t)=0$ with $i \in \mathcal{I}$ and inequality constraints $g_j(X^{n}, Z^{n}, t) \le 0$ with $j \in \mathcal{J}$, for each $t \in \mathcal{T}$. Those constraints enable the modelling of the operational constraints. 
    
    % Each node $n$ has a cost function associated $F^{n}(X^{n}, Z^{n}) = \sum_{t=1}^{T} f^{n}(X^{n},Z^{n},t) $. This cost function typically represents the so called capital expenditure and operational expenditure: CAPEX and OPEX respectively.
    
    % Finally, equality and inequality constraints on hyperedges can be defined as $H^{e}(Z^{e}) = 0$ and $G^{e}(Z^{e}) \le 0$ with $ e \in \mathcal{E}$. Those constraints enable the modelling of laws of conservation as well as cap on given commodity. 

This section provides insight into the optimization framework that underlies the multi-energy system model proposed in this work. The GBOML language introduced in \cite{Miftari2022}, a recently developed language dedicated to modeling graph-based optimization of multi-energy systems, is utilized to build this model. The optimization problem can be viewed as optimization on graphs, where a multi-energy system is considered as a set of nodes $\mathcal{N}$ that contribute to the (linear) objective and local constraints, and hyperedges $\mathcal{E}$ are used to model the constraints between nodes, such as those between RREHs and the EDC in our context.

The formalism employed in this work follows that introduced in \cite{Berger2021}. The entire system is defined by sets of nodes $\mathcal{N}$ and hyperedges $\mathcal{E}$. The optimization horizon is denoted by $T$, with time-steps indexed by $t \in \mathcal{T}$, where $\mathcal{T} = \{1, \ldots, T\}$.

A node $n \in \mathcal{N}$ is defined by internal $X^{n}$ and external $Z^{n}$ variables, where internal variables describe the specific characteristics of the unit, such as the nominal power capacity installed in the asset. Equality constraints $h_i(X^{n}, Z^{n}, t)=0$ with $i \in \mathcal{I}$ and inequality constraints $g_j(X^{n}, Z^{n}, t) \le 0$ with $j \in \mathcal{J}$, are employed for each $t \in \mathcal{T}$ to model operational constraints.

Each node $n$ has an associated cost function $F^{n}(X^{n}, Z^{n}) = \sum_{t=1}^{T} f^{n}(X^{n},Z^{n},t)$ that typically represents the capital expenditure and operational expenditure, i.e., CAPEX and OPEX, respectively.

Finally, equality and inequality constraints on hyperedges can be defined as $H^{e}(Z^{e}) = 0$ and $G^{e}(Z^{e}) \le 0$ with $e \in \mathcal{E}$ to model the laws of conservation and caps on given commodities.

    One can read this type of problem as:
     \begin{equation}\label{eq:prob_statement}
    \begin{aligned}
    \min \quad &  \sum_{n=1}^{N} F^{n}(X^{n}, Z^{n})\\
    \textrm{s.t.} \quad & h_i(X^{n}, Z^{n},t) = 0, \forall n \in \mathcal{N}, \forall t \in \mathcal{T}, \forall i \in \mathcal{I}\\
      \quad & g_j(X^{n},Z^{n},t) \le 0, \forall n \in \mathcal{N}, \forall t \in \mathcal{T}, \forall j \in \mathcal{J}\\
      \quad & H^{e}(Z^{e}) = 0, \forall e \in \mathcal{E}\\
      \quad & G^{e}(Z^{e}) \le 0, \forall e \in \mathcal{E}. \\
    \end{aligned}
    \end{equation}

%\subsection{Assumptions}\label{subsec:assumptions}

The main assumptions underlying our model are the following:
\begin{itemize}
    \item Centralised planning and operation: In this framework, a single entity is responsible for making all investment and operation decisions.
    \item Perfect forecast and knowledge: It is assumed that the demand curves, as well as weather time series, are available and known \textit{in advance} for the entire optimisation horizon, i.e., $\forall t \in \{ 1, \ldots , T \}$.
    \item Permanence of investment decisions: Investment decisions result in the sizing of installation capacities at the beginning of the time horizon. Capacities remain fixed throughout the entire optimisation period, i.e., $\forall t \in \{ 1, \ldots , T \}$.
    \item Linear modelling of technologies: All technologies and their interactions are modelled using linear equations within this framework.
    \item Spatial aggregation: The energy demands and generation at each node are represented by single points. The topology of the embedded network required to serve this demand locally is not modelled in this approach. This can be viewed as an extension of the copper plate modelling approach used in electrical power systems.
\end{itemize}

     In our problem, all cost functions and constraints are affine transformation of the inputs. More details on the constraints of each technology can be found in \cite{BERGER_power_to_gas_2020106039}, \cite{Berger2021}.
    Additionnaly, the local objective function corresponding to the CAPEX is modelled with a uniform weighted average cost of capital (WACC) of $7\%$ for each technology. Thus, the CAPEX is computed using the following formula:
    \begin{equation}
        \zeta^n = \mbox{CAPEX}_n \times \frac{\mbox{w}}{(1 - (1 + \mbox{w})^{-\mbox{L}_n})}
        \label{waccformula}
    \end{equation}
    with $L_n$ the lifetime of technology n and w the WACC. Hence, $\zeta^n$ represents the annualised cost of investing in technology n.

    Moreover, a cap on the net CO2 emissions (\textit{i.e.} release in minus captured from the atmosphere) is added to the model. This latter is defined as 
    \begin{equation}\label{eq:cap_co2}
        \sum_{t \in \mathcal{T}} ( \sum_{a \in \mathcal{A}} q_{co2, t}^{a} - \sum_{c \in \mathcal{C}} q_{co2, t}^{c} ) \le \kappa_{co2} \nu
    \end{equation}
    with $\mathcal{A}$ and $\mathcal{C}$ representing the sets of technologies that release CO2 into the atmosphere and those that capture CO2 directly from the atmosphere, respectively, $\kappa_{co2}$ represents the CO2 cap in kilotons per year, and $\nu$ represents the number of years covered by the optimization horizon. The shadow price, or marginal cost, which is the dual variable associated with \autoref{eq:cap_co2} allows for the derivation of a CO2 cost in €/t. A detailed explanation of dual variables as marginal costs in linear programming can be found in \cite[Chapter 4]{linearOptim}.

    
    % Since the cap on CO2 is set to zero, the cost associated with the emission of one more unit will be given by the optimal dual vector, which is the solution to the dual problem of \autoref{eq:prob_statement}. A detailed explanation of the dual problem can be found in \cite[Chapter 4]{linearOptim}. Each element of this dual vector can be interpreted as a shadow price associated with one constraint, i.e., a cost per unit increase of the right-hand side of the constraint \cite[Chapter 4, p.155]{linearOptim}. In the case where the constraint is a less-than-or-equal-to constraint ($\le$), the associated dual variable $p$ is negative. Therefore, emitting one more ton of CO2 will decrease the objective function by $p \times 1$ €. However, we should not emit this ton of CO2. Consequently, a cost of $p$ €/t (i.e., the shadow price) must be associated with the emission of any additional ton of CO2 in our system.
    
    % According to \cite[p.155]{linearOptim}, the optimal dual vector can be interpreted as the shadow price or marginal cost per unit increase of the right-hand side of the constraints. In our case, the dual variable $p$ associated with a constraint $\le$ is negative. Therefore, emitting one more unit of CO2 will decrease the objective function by the marginal price, i.e., $p$€, the dual variable value. In other words, since our CO2 cap is set at zero, emitting one more ton of CO2 will cost you $p \times 1$ (\textit{e.g.} looking at the units, €$/t \times t =$€). However, since we do not want to emit this ton of CO2, a cost of $p$ should be associated with this emission of one more unit.

    






In this section, we revisit the full problem (introduced in \cref{s.model}) and conduct numerical experiments to evaluate the performance of $\DPI$ for its intended use case. Our analysis is motivated by our partner organization, Keheala, which operates a digital health platform to support medication adherence among TB patients in highly resource-constraint settings. Here, we first summarize the state of the global TB epidemic and the Keheala behavioral intervention (\cref{ss.keheala}). We then describe our data sources and the validated simulation model we have developed to test outreach policies (\cref{ss.data} and \cref{ss.sim}). Next, we discuss how our policy, as well as some benchmark policies, can be implemented using Keheala's data (\cref{ss.policies}), before presenting our numerical results (\cref{ss.results}).

\subsection{The global TB epidemic and the Keheala intervention}\label{ss.keheala}
TB remains one of the deadliest communicable diseases in the world, causing 1.6 million deaths in 2021. This is remarkable in light of the fact that effective treatment has been available for over 80 years, with the current WHO guidelines recommending a 6 month regimen of antibiotics for drug-susceptible TB and a more intense regiment for drug-resistant strains \citep{World22Global}. A key limiting factor for curbing the epidemic is lack of patient adherence to these treatment regiments, which increases the probability of infection spreading, drug resistance, and poor health outcomes \citep{garfein2019synchronous}. 

Keheala was designed to provide treatment adherence support to TB patients in resource-limited settings. Their platform operates on the Unstructured Supplementary Service Data (USSD) mobile phone protocol, which importantly allows phones without smart capabilities to access the service. Once a patient has enrolled with Keheala, they are meant to self-verify treatment adherence every day, using their mobile phone. In addition, they have access to a range of services. Some are on-demand, for example educational material about TB or leaderboards for verification rates. Others are automatic, such as adherence reminders, which are sent to patients daily (at their pre-determined medication time) in the absence of verification. In addition, the Keheala protocol is to escalate outreach interventions when patients do not self-verify adherence. It states that after one day of non-adherence patients should receive a customized message to encourage resumed adherence and after two days of non-adherence patients should receive a phone call from a support sponsor. While these support sponsors are full-time employees, they are not healthcare professionals. They are members of the local community who have experience with TB treatment and are therefore familiar with the many contributing factors associated with low treatment adherence, such as side-effects, societal stigma against TB patients, and challenges with refilling prescriptions.

The overall effectiveness of Keheala's combination of services was evaluated in a randomized controlled trial (RCT) in Nairobi, Kenya. The trial demonstrated that Keheala reduced unsuccessful TB treatment outcomes—a composite of loss to follow-up, treatment failure, and death—by roughly two-thirds, as compared to a control group that received the standard of care \citep{Yoeli19Digital}. Given this success, Keheala's primary practical objective is to ensure that enrolled patients remain engaged with the platform through adherence verification.

In this paper, we focus on the final level in Keheala's escalation protocol---support sponsors making phone calls to patients. This part of the outreach was organized through populating a daily list of patients who had not verified treatment adherence for 48 hours. Support sponsors had many responsibilities in operating the platform, but would make phone calls to as many patients on the list as possible on a given day. Since hiring support sponsors is a costly aspect of operating the service, Keheala is interested in implementing a more personalized and targeted approach for prioritizing which patients should receive a phone call on a given day. Being able to maintain a similar performance with fewer support sponsors (or equivalently, serve more patients with the same number of support sponsors) is desirable for any future scale-up of the system.








\subsection{Data sources.}\label{ss.data}
Based on the success of the first RCT, the effectiveness of Keheala was further evaluated in a second RCT\footnote{The trial was approved by the institutional review board of Kenyatta National Hospital and the University of Nairobi. Trial participants or their parents or guardians provided written informed consent. The trial’s protocol and statistical analysis plan were registered in advance with ClinicalTrials.gov (\#NCT04119375).} during 2018-2020. The RCT was conducted in partnership with 902 health clinics distributed across each of Kenya's eight regions, representing a mix of rural and urban clinics. The study included four treatment arms and enrolled over 15,000 patients. We obtained data for 5,433 patients enrolled in the Keheala intervention arm (other arms aimed to independently test specific components of the Keheala intervention). 

As part of the RCT, the study team collected socio-demographic information from all patients. This information includes static covariates such as age, gender, language preferences, location, as well as limited clinical history (see \cref{s.app.list_features} for a full list). In addition, Keheala collected engagement data about each patient during their enrollment in the service. This includes whether a patient verified on a given day, how many reminders they received, and whether they were contacted by a support sponsor. 

After filtering out patients with missing information or not enough data, we conducted all our analysis on 3594 patients. The average patient was enrolled on the platform for 118 days. On an average day, 608 patients were enrolled and 210 of those were eligible for a support sponsor call according to the protocol (i.e., having not verified treatment adherence for the preceding 48 hours). The support sponsors, employed by Keheala, had a range of responsibilities in operating the platform, including making outreach phone calls to the eligible patients. 
The average number of calls made per day was 25.5. Hence, in our analysis, we use a budget of $B$ = 26 as our main point of comparison.
	

\subsection{Simulation Model.}\label{ss.sim}
We build a simulation model that we use to estimate the counterfactual outcomes of different outreach approaches.
The simulator is effectively represented by a single function, $f(S, A) \in [0, 1]$, which denotes the probability that a patient in state $S$ with action $A$ verifies in the next time step.
This function is used to simulate one step transitions for every patient.
We first describe the state space of the patients, describe the exact simulation procedure, and then discuss how we learn $f$ from data and validate the simulator.

\subsubsection{Patient state space.}\label{ss.simstatespace}
For patient $i$, let $X_i \in \bR^{13}$ be their static covariates. 
Let $V_{it} \in \{0, 1\}$ denote whether patient $i$ verified at time $t$, and let $A_{it}\in \{0, 1\}$ denote whether the patient received the intervention at time $t$.
Let $H_{it} = (V_{i1}, A_{i1}, \dots, V_{i, t-1}, A_{i,t-1}, V_{it}) \in \bR^{2t-1}$ be the history of verifications and interventions up to time $t$.
We define a \textit{condensed} history $\tH_{it} \in \bR^{21}$ by summarizing the history $H_{it}$ into 21 features, aiming to capture as much relevant information as possible.
The condensed history contains the patient's recent and overall behavior.
For statistics such as the number of times the patient verified and the number of interventions they have received, we aggregate them over the past week, as well as in total.
We also include information on their verification and non-verification streaks, as well as how long they have been in the platform.
See \cref{s.app.list_features} for a full list of these features.
Then, we define the state of patient $i$ at time $t$ to be $S_{it} = (X_i, \tH_{it}) \in \bR^{34}$.

\subsubsection{Simulation procedure.} 
Given $f$ and an intervention policy $\pi$, we `mimic' the RCT by simulating patient behavior day by day. 
In total, we simulate $T=700$ time steps, where each $t \in [T]$ corresponds to one day between April 2018 to March 2020. We let $T_{s}(i)$ and $T_{e}(i)$ denote the starting and ending time steps that patient $i$ was enrolled in Keheala. Each patient $i$ is then introduced into the system at time $t=T_s(i)$, and removed at time $t=T_e(i)$. We use the observed data from the RCT for their first 7 days in the system to initialize their state. 
Then, in each time period, given the set of patients that were active in the RCT for more than 7 days, we use a policy $\pi$ on these patients to determine who receives sponsor outreach. If a patient $i$ was in state $S_{it}$ at time $t$ and the policy $\pi$ chose action $A_{it}$, we let $V_{i, t+1}$ be 1 with probability $f(S_{it}, A_{it})$, and 0 otherwise (where the randomness is independent across patients and time steps).
Finally, we use $V_{i, t+1}$ to update their state for the next time step. 

\subsubsection{Estimating the $f$ function.} \label{s.learnf} 
Using the state space $\cS$ as described above, we construct the function $f: \cS \times \{0, 1\} \to [0, 1]$ using data from the RCT. Specifically, we learn the two functions $f(S, 0)$ and $f(S, 1)$ separately. For $f(S, 0)$, we train a gradient boosting classifier on the dataset $\{(S_{it}, V_{i,t+1})\}_{i \in N, t \in [T] : A_{it} = 0}$, using $V_{it}$ as the outcome variable. For $f(S, 1)$, we write the function as $f(S, 1) = f(S, 0) + \tau(S)$ and we learn $\tau(S)$ using the double machine learning method of estimating heterogeneous treatment effects \citep{chernozhukov2018double}. See \cref{s.app.simulation} for details on implementing this method.

\subsubsection{Train and test split.} \label{s.traintestsplit}
Importantly, we use a different set of patients to estimate the $f$ function (and to run our simulations) from the set of patients we use to train our policies. In particular, we randomly split all patients from the RCT into two groups, which we call \textit{train} and \textit{test}. We use the \textit{test} set to estimate the $f$ function that forms the basis for the simulation model. We keep the \textit{train} set of patients separate and use it to train policies (see \cref{ss.policies}). This ensures that the policies we evaluate are not learned off of the same dataset that was used to learn the simulator. 
The simulation itself uses the test patients, and we duplicated each patient so that we maintain a similar total number of patients as in the original study.


\subsubsection{Simulation validation.}
We validate the performance of the simulator on a \textit{different} intervention policy than the simulator was trained on, by leveraging the fact that there was variability in the number of interventions given throughout the RCT.
In particular, the average number of interventions given during the first half of the RCT was around double of that of the latter half (45.9 vs. 21.4),
and this variation induces a change in the intervention assignment policy.
Then, dividing the data into halves produces two datasets that are generated using effectively different intervention policies.

To validate the simulator, we use the method from \cref{s.learnf} to learn $f$ using the first dataset, and then validate its performance on the second dataset. 
Using this procedure, the AUCs on the second dataset for $f(S, 0)$ and $f(S, 1)$ were 0.918 and 0.745, respectively.
We also check the calibration of both of these functions, by grouping the samples into bins based on their predicted probability of verifying the next day, and checking whether their actual verification rates.


We group the samples based on the simulated probability of verification into bins with a 10\% range, and we compute the expected calibration error (ECE) \citep{naeini2015obtaining}. For bin $i$, let $o_i$ be the true fraction of positive instances in bin $i$, $e_i$ be the mean of the predicted probabilities of the instances in bin $i$, and $N_i$ be the number of samples in bin $i$. Then, the ECE is defined as
\begin{align}
\text{ECE}	 = \frac{1}{N} \sum_{i=1}^{10} N_i |o_i - e_i|,
\end{align}
where $N$ is the total number of samples.
The expected calibration error was 0.0066 for $f(S, 0)$ and 0.0308 for $f(S, 1)$.
Figure~\ref{f.calibration} displays these bins.

These results demonstrate that the simulator has good performance in mimicking patient behavior. 
As expected, the AUC and the ECE is worse for $f(S, 1)$ compared to $f(S, 0)$; 
this is due to the \textit{significantly} fewer number samples with an intervention in the training data, as well as the increase in variance of doing off-policy estimation.
The training data used for $f(S, 0)$ had 300K samples, while the one used for $f(S, 1)$ had 4.5K samples.
For $f(S, 1)$, the calibration is slightly off for samples with a high probability of verification (bins 0.7-0.9); however, we note that the 0.7-0.9 bins only contain 11.3\% of all samples for $f(S, 1)$.


\begin{figure}[h]%
	\centering
	\subfloat[Calibration for $f(S, 0)$]{{\includegraphics[width=0.48\linewidth]{figs/bar_pred0} }}%
	\subfloat[Calibration for $f(S, 1)$]{{\includegraphics[width=0.48\linewidth]{figs/bar_pred1} }}%
	\vspace{2mm}
	\caption{Calibration plots for $f(S, 0)$ and $f(S, 1)$ for simulation validation. 
		We group the samples based on the simulated probability of verification
		into bins with a 10\% range, which we label by the lower number. 
		For example, the 0.3 bin on the x-axis represents the samples whose probability of verification according to $f$ is in $[0.3, 0.4)$; hence we should expect the actual number of verifications of those samples to be close to 0.35.}
	\label{f.calibration}%
	\vspace{-2mm}
\end{figure}



\subsection{Outreach Policies and Experimental Design}\label{ss.policies}
Using the simulation model described above, we compare the performance of three main policies. For each policy, we vary the budget for outreach interventions per day between 10 and 40. As mentioned before, the average number of sponsor outreaches during a given day of the trial was 26. Importantly, we restrict all policies so that they can only provide an outreach to patients who have not verified for at least two days in a row. This is because that was what was done in the RCT, hence there is no data for how an outreach affects behaviors for those who do not meet this criterion (thus we would not be able to accurately evaluate policies that do not follow this restriction).

We note that attaining improved performance with lower outreach capacity is particularly important for the resource-limited setting at hand as it speaks to the performance achievable during a future scale-up of the system, in which the ratio of patients to support sponsors is likely to be much higher. 
Next, we describe the implemented policies.

\subsubsection{$\DPI$ for Keheala.} 
The first step in operationalizing $\DPI$ is defining the state space for each patient. For this, we use the same condensed state space as described in \cref{ss.simstatespace}, i.e., we define the state of patient $i$ at time $t$ to be $S_{it} = (X_i, \tH_{it}) \in \bR^{34}$ (a full list of these features is included in \ref{s.app.list_features}). Importantly, we note that all of the features of this state space are observable to Keheala at any time $t$, once a patient has been enrolled on the platform for seven days. 

The second step is estimating the $\hz_{it}(S_{it})$ score for each patient at each time period, which is ultimately used to prioritize patients. 
As before, we let $T_{s}(i)$ and $T_{e}(i)$ be the starting and ending time steps that patient $i$ was enrolled in Keheala. Using this notation, we can represent the future verification \emph{rate} for patient $i$ at time $t$ by $y_{it} = \frac{1}{T_{\text{e}}(i)-t} \sum_{r=t+1}^T V_{ir}$.
Then, the data from the RCT can be written in the form $\{(S_{it}, A_{it}, y_{it})\}_{i \in [N], t \in \{T_{s}(i), \dots, T_{e}(i)\}}$, and we can estimate the function $q_{it}^{\baseline}(S, A)$ using this data.
In our implementation, we use a linear function approximation for the verification rate, assuming the form 
\begin{align*}
q_{it}^{\baseline}(S, A) = \langle \theta_A, S \rangle \cdot (T_{\text{e}}(i)-t),
\end{align*}
for each of the two actions $A \in \{0, 1\}$.
The $\langle \theta_A, S \rangle$ term represents the future verification rate, and $T_{\text{e}}(i)-t$ represents the number of days left; combined, $q_{it}^{\baseline}(S, A)$ represents the total number of future verifications.
We note that the state contains information regarding the number of days the patient has been enrolled in Keheala, hence the verification rate is also a function of the time step.

We estimate $\theta_a$ using least squares with an $\ell_2$ regularizer:
\begin{align} \label{eq:leastsquares}
	\hat{\theta}_a &\in \argmin_{\theta \in \bR^{34}} \bigg( \sum_{i \in N} \sum_{t=T_{\text{s}}(i)}^{T_{\text{e}}(i)}  \bI(A_{it} = A)(y_{it} - \theta^\top S_{it})^2 + ||\theta||^2_2 \bigg).
\end{align}

Finally, we compute a patient's estimate of their intervention value at time $t$ as
\begin{align*}
	\hz_{it}(S_{it}) = \langle \htheta_1 - \htheta_0,S_{it} \rangle \cdot (T_{\text{e}}(i)-t), 
\end{align*}
and the resulting policy is to give the intervention to up to $B$ patients with the highest positive $\hz_{it}(S_{it})$ values.





\subsubsection{Bandit.}
The bandit policy aims to choose patients with the highest increase in the probability of next-day verification, using a linear contextual bandit model. In terms of the two-state model from \cref{ss.2statemodel}, the goal is to choose patients with the highest value of $\tau$.
We essentially use the same state space and linear model as was used for $\DPI$, except that the outcome variable is next-day verification, rather than total future verifications.
We first learn a prior using the offline data, and then we run a Thompson sampling style policy, which continually updates the policy with online data.

Specifically, we assume the linear form $V_{i,t+1} = \langle \beta_a, S_{it} \rangle$ for action $a \in \{0, 1\}$, with unknown parameters $\beta_0, \beta_1 \in \bR^{34}$.
The prior on $(\beta_0, \beta_1)$ is initialized as the output of a least-squares regression using the offline data, the same data that was used to train $\DPI$.
At each time step, $(\tilde{\beta_0}, \tilde{\beta_1})$ is sampled from the posterior. 
Then, the policy chooses the $B$ patients with the highest value of $\langle \tilde{\beta_1}, S_{it} \rangle - \langle \tilde{\beta_0}, S_{it} \rangle$.
After the outcome is observed at each time step, the posterior is updated accordingly.
The detailed description on the algorithm can be found in Section~\ref{sec:app:bandit}.

This policy makes use of strictly more data than $\DPI$, since $\DPI$ only uses the offline data. 
In the results, we confirm that this policy indeed learns myopic rewards correctly.
Therefore, this is a very strong benchmark algorithm for optimizing myopic rewards.


\subsubsection{Whittle's index (QWI).} \label{sec:qwi_description}
The next benchmark is the Whittle's index.
The advantage of this method compared to the bandit benchmark is that is is non-myopic.
However, the downside is that computing the Whittle's index requires the model to be known.
To implement Whittle's index in our setting where the model is unknown, we leverage the recent work of \cite{avrachenkov2022whittle} who propose a Q-learning approach to learn the Whittle's index, which we refer to as $\QWI$.
$\QWI$ is an online learning method that simultaneously learns the Q-values as well as the Whittle's index for each state.

There are two main challenges in implementing $\QWI$ in our setting. The first is that the algorithm is an online learning method, and the second is that it requires a finite state space as it learns the Whittle's index for each state separately. 
For the first point, we adapt the algorithm from \cite{avrachenkov2022whittle} to an offline setting so that it can use the same data that is used to train $\DPI$.
For the second point, we define a smaller, finite state space so that $\QWI$ can be implemented.
We define a patient's state at a point in time to be a 3-tuple $(s_1, s_2, s_3)$, where $s_1$ represents the number of times the patient verified in the last week, $s_2$ is the patient's historical total verification rate, and $s_3$ is the number of times that the patient received an intervention in the last week. The values of each of these terms are bucketed into a small number of bins (3 bins for $s_1$ and $s_3$, 5 bins for $s_2$), resulting in 45 states in total. Specifically, the bins for both $s_1$ and $s_2$ are $0, 1$ and $2-7$. For $s_2$, the bins are $0-1\%$, $1-5\%$, $5-20\%$, $20-45\%$, and $45-100\%$. The bin values were chosen to balance the number of samples in each bin. 
This results in 45 states in total.


Based on this state space, we learn the Whittle's index, $\lambda(s) \in \bR$, for each state $s$.
Then, at each point in time, $\QWI$ chooses the patients in states with the highest Whittle's index to give the intervention to.
Further details of the learning algorithm is deferred to Appendix~\ref{sec:app:qwi}. 

\added{
We note that the state space for $\QWI$ is different from that of $\DPI$, due to the computational limitation of $\QWI$. In \cref{ss.state_space_results}, we run simulations where we modify the state space for $\DPI$ to be the same as $\QWI$, so that we can isolate the performance difference to the algorithm rather than the state space.
That said, we believe that the ease of working with an infinite state space is a substantial advantage of $\DPI$.
}




\subsubsection{Baseline.}
The $\baseline$ policy approximates the heuristic followed by Keheala in the two RCTs that have been implemented. In both cases, the protocol was that patients were added to the support sponsor call queue after not verifying for 48 hours. As a result, the order of patients in the queue is effectively random, determined by a combination of their designated medication time (which prompts automated reminders to take the medicine and verify) and the timing of their self-verification. We approximate the resulting outreach policy by selecting $B$ patients out of all those who have not verified for 48 hours, at random. 

\added{
The $\RAND$ policy used in the theoretical results is aimed to be an approximation of $\baseline$. 
The discrepancy between these policies is solely for technical convenience, 
as $\RAND$ is easier to analyze due to the independence across patients.
}


\subsubsection{Null policy.}
\added{
Lastly, we simulate the $\NULL$ policy which does not give any interventions. 
We note that this policy does not depend on the budget parameter $B$.
}

\subsection{Results}\label{ss.results}

The results are shown in Figure~\ref{fig:main_results}. 

\begin{figure}[h]
\begin{center}
\vspace{-3mm}
  \includegraphics[width=1\linewidth]{figs/results_June14_whittle}
  \caption{Average overall verification rate over 50 runs for each policy and budget. 
  The overall verification rate for the $\NULL$ policy was 54.2\%.
  The shaded region indicates a 95\% confidence interval. The star represents the operating point for Keheala.}
  \label{fig:main_results}
\vspace{-6mm}
\end{center}
\end{figure}

\subsubsection{Overall performance.}
The average performance for each budget and policy over 50 runs are shown in Figure~\ref{fig:main_results}, which shows that $\DPI$ clearly outperforms the other policies over a wide range of budget values.
For a practical interpretation of the results, consider \textsf{Baseline} at a budget of 26, the policy and budget that Keheala was operating during the RCT, which results in an overall verification rate of 62.0\%.
By using less than \textit{half} of the budget, $B=12$, $\DPI$ achieves the same verification rate at 62.2\%.
As the costliest aspect of Keheala's system is in hiring staff to provide the interventions, these results imply that they can cut these costs by half to achieve the same outcome.
\added{
The $\NULL$ policy (no interventions) results in a verification rate of 54.2\% (we did not plot this for readability of the figure).
One can interpret this number as a reference benchmark to compare the effectiveness of interventions.
When the budget is 26, $\baseline$ improves over $\NULL$ by 14.6\%, while $\DPI$ improves over $\NULL$ by 20.3\%. 
Therefore, $\DPI$ improves the effectiveness of the interventions over $\baseline$ by 38.3\%.
}

Additionally, we observe that the improvement of $\DPI$ compared to the other policies is especially substantial for smaller budgets. 
This implies that when the number of patients that can be targeted is small, $\DPI$ can correctly identify the set of patients to target that result in the largest gains.
This is especially valuable for scaling up the system.
Indeed, if Keheala wanted to expand to include more patients without linearly increasing their staff costs, then the ratio of budget to the number of patients would decrease, resulting in the regime where $\DPI$ offers major improvements.

The fact that the performance of $\bandit$ policy improves over $\baseline$ as the budget increases is caused by the increase in relevant data.
Note that the number of interventions is small ($\sim 26$) relative to the number of patients in the system at once ($\sim 600$), implying that the number of data points with $A=1$ is much smaller than that of $A=0$. 
Therefore, the main bottleneck in estimation is learning patient behaviors after receiving an intervention.
As the budget increases, the $\bandit$ has access to more data from patients with an intervention, and hence is able to improve its learning. 

$\QWI$ has a slightly inconsistent performance curve relative to the other policies.
Its performance is always better than or similar to $\baseline$, but compared to $\bandit$, it over-performs in the mid-budget regime, but under-performs as the budget increases.
We dive deeper into the types of patients each policy targets in \cref{sss.targetedpatients}, where we provide an explanation for this behavior.  


\added{
One factor that may be contributing to the poor performance of $\QWI$ is the state space that is used.
$\QWI$ uses a discretized state space as described in \cref{sec:qwi_description}, different than the infinitely-sized state space used by $\DPI$.
In \cref{ss.state_space_results}, we run additional experiments where we run $\DPI$ with the same state space as $\QWI$, so that the performance differences can be purely attributed to the algorithm rather than the state space. 
}


\subsubsection{Patient-level verification rates.} 
The overall number of verifications increases under $\DPI$, but how do these rates get impacted at the patient-level?
Fixing the budget to be 26, we compute the verification rate of \textit{each} patient, and we examine the distribution of these patient-level rates. \added{
In Figure~\ref{fig:diff_vrates_all}, we plot how the distribution of patient verification rates shift under the $\DPI$, $\bandit$, and $\QWI$ algorithms, compared to $\baseline$.
We see that under $\DPI$, the distribution shifts in a way that there are fewer patients with verification rates under 50\%, and more patients with a verification rate higher than 50\%.
We see a similar phenomenon for $\QWI$, but the magnitude of the shift is smaller.
$\bandit$ also observes an increase in $>50\%$ verification rates, but there is also an increase of those with very low (0-10\%) verification rates.
These results for $\DPI$ represent a desirable type of shift, where main improvement of $\DPI$ comes from an increase in the number of patients with a high verification rate.
We also provide absolute numbers in \cref{tab.verification_rates}, where we show the percentage of patients whose verification rate is higher than 50\% and 70\% for each of the four algorithms.
Under $\DPI$, the number of patients whose verification rate is above 50\% and 70\% increased relatively by 6.7\% and 5.2\% respectively compared to $\baseline$.
}


\begin{figure}
\centering
\begin{subfigure}{.47\textwidth}
  \centering
  \includegraphics[width=1\linewidth]{figs/2024_dpi-keheala}
  \caption{Comparing $\DPI$ to $\baseline$.}
\end{subfigure}%
\begin{subfigure}{.47\textwidth}
  \centering
  \includegraphics[width=1\linewidth]{figs/2024_bandit-keheala}
  \caption{Comparing $\bandit$ to $\baseline$.}
\end{subfigure} \\
\begin{subfigure}{.47\textwidth}
  \centering
  \includegraphics[width=1\linewidth]{figs/2024_qwi-keheala}
  \caption{Comparing $\QWI$ to $\baseline$.}
\end{subfigure}
\caption{
\added{
Differences in the distribution of patient verification rates compared to $\baseline$. 
  The bins represent the difference in the number of patients whose overall verification rate is between $0-10\%$, $10-20\%, \dots, 90-100\%$.
For example, the first bin in (a) shows that there were 28 fewer patients whose verification rate was between 0 and 10\% under $\DPI$, compared to $\baseline$.
 There were 3594 patients in total, and the budget was fixed at 26.
 }
}
  \label{fig:diff_vrates_all}
\end{figure}


\begin{table}[h]
\TableSpaced %
\caption{\added{
The average percentage of patients whose verification rate was over 50\% and 70\% across the four algorithms.
 There were 3594 patients in total, and the budget was fixed at 26.
}} \label{tab.verification_rates}
\vspace{2mm}
\begin{center}
\begin{tabular}{@{}c|cccc@{}}
\toprule
\% patients with \\ verification rate      & \quad $\baseline$ \;  & $\DPI$ & $\bandit$& $\QWI$ \\ \midrule
$\ge 50\%$  &  61.7\%   & 66.1\%  & 63.8\% & 63.9\% \\ 
$\ge 70\%$  &  38.2\%  & 40.2\%  & 39.7\%  & 39.5\% \\ \bottomrule
\end{tabular}
\end{center}
\end{table}



\subsubsection{Description of the targeted patients.} \label{sss.targetedpatients}
In \cref{tab.stats}, we fix the budget to be 26 and we show statistics regarding the state of the targeted patients for each of the four policies.
For example, under $\baseline$, on average, the patient that received an intervention had a treatment effect of 8.8\% with respect to the probability that they will verify the next day.
8.8\% is the `true' average treatment effect, in the sense that the numbers that are averaged are taken directly from the simulation model.


\begin{table}[h]
\TableSpaced %
\caption{
Average statistics of the state of patients who were given an intervention, across the three policies that were run for $B=26$.
(a) is the average value of $f(x, 1) - f(x, 0)$, the increase in probability that the patient verifies the next day when they are given an intervention. 
(b) is the average $f(x, 0)$, the probability that a patient verifies the next day \textit{without} an intervention. 
(c) is the average number of remaining days the patient will be on TB treatment for.
} \label{tab.stats}
\vspace{2mm}
\begin{center}
\begin{tabular}{@{}ccccc@{}}
\toprule
                                                 & $\baseline$ & $\DPI$ & $\bandit$ & $\QWI$ \\ \midrule
\multicolumn{1}{l}{(a) \added{$f(x, 1) - f(x, 0)$}}  & 8.8\%   & 10.6\%    & 13.2\%    & 6.9\%    \\ 
\multicolumn{1}{l}{(b) \added{$f(x, 0)$}}   & 18.2\%  & 22.2\%  & 35.2\%   & 12.2\%      \\ 
\multicolumn{1}{l}{(c) Days on TB treatment remaining} & 68.3     & 109.3   & 92.4    & 69.7      \\ \bottomrule
\end{tabular}
\end{center}
\end{table}



Statistic (a) represents exactly what the $\bandit$ policy optimizes for, the increase in probability of the patient verifying the next day.
The fact that $\bandit$ yields the highest value confirms that indeed, the policy correctly learns what it is supposed to learn.
$\DPI$ chooses patients with a higher one-step treatment effect than $\baseline$, but lower than that of $\bandit$.
Then, the fact $\DPI$ outperforms $\bandit$ in terms of overall verification implies that a myopic strategy of looking only one step ahead is not sufficient.
The next two statistics shed light on why this may be.


Statistic (b) represents the probability that the targeted patient would have verified anyway without an intervention, and we see that the $\bandit$ targets patients with a much higher verify probability than the other policies. 
We plot the entire distribution of this quantity in Figure~\ref{fig:base_probs}, where we see that $\bandit$ often targets those with a relatively high probability ($>45\%$), while $\DPI$ targets those with a relatively low probability ($<15\%$).
This may contribute to the improved performance of $\DPI$, and the reasoning for this can be seen through the two-state model from \cref{ss.2statemodel}, where statistic (b) corresponds to the parameter $p$.
If two patients have the same values of the parameters $g$ and $\tau$ but differing values for $p$, the intervention value is higher when $p$ is smaller (see \cref{prop:z_clean_form}).
This is because the patient with a high value of $p$ is more likely to switch to state 1 at the current time step as well as all future time steps.
As an extreme example, for a patient with $p=0$, they \textit{need} an intervention to switch to state 1, whereas a patient with $p>0$ may switch to state 1 (either now or in the future), without an intervention.
Therefore, an intervention is more likely to be helpful for those with a smaller value of $p$, which $\DPI$ targets.

On the other hand, $\QWI$ takes the above strategy to an extreme, where it targets those with a very low probability of verifying (12.2\%), but on average these patients also do not have a high next-day treatment effect (6.9\%).
This may explain the inconsistent behavior of $\QWI$ as the budget increases. 
The strategy of targeting these patients with a low verify probability and a low treatment effect is reasonably effective in the mid-budget regime; however, as the budget increases, one may also need to judiciously target other types of patients, which $\QWI$ does not do. 





\begin{figure}[h]
\begin{center}
\vspace{-5mm}
  \includegraphics[width=0.54\linewidth]{figs/base_probs_June2024_2}
\vspace{-2mm}
  \caption{
  Histogram of the value of $f(x, 0)$ of targeted patients, the probability that the patients would verify without an intervention.
  This is the entire distribution of the statistic (b) in \cref{tab.stats} for $\bandit$ and $\DPI$.}
  \label{fig:base_probs}
\vspace{-6mm}
\end{center}
\end{figure}

Lastly, statistic (c) is the average number of days a targeted patient has remaining on the platform.
If an intervention positively affects patients for all of their future time steps, then targeting those with longer time left in the system would result in higher benefits. 
The results show that $\DPI$ targets those with the longest days of treatment left.

\subsubsection{Prominent features for $\DPI$.}
\cref{tab.coefs} displays the five most predictive features of the intervention values that $\DPI$ uses to target patients.
These features were found by using Lasso regression with a tuned parameter -- see Section~\ref{s.app.coefficients} for details on the method used.
The results show that the intervention value is lower when the number of previous interventions is higher (first two features), which is intuitive since patients may become fatigued and less receptive when there are too many interventions.
The intervention value is lower when the patient's past verifications is higher (third and fourth features). This is consistent with the analysis in \cref{tab.stats}, where $\DPI$ targets those with a smaller value of $f(x, 0)$.
Lastly, the intervention value increases for patients who are older.


\begin{table}[h]
\TableSpaced %
\caption{
The most predictive features of higher intervention values for $\DPI$, as well as the sign of their coefficient. 
} \label{tab.coefs}
\vspace{2mm}
\begin{center}
\begin{tabular}{@{}lc@{}}
\toprule
\; Feature & Sign of Coefficient  \\ \midrule
\; Interventions: total number & $-$ \\ 
\; Interventions: \# previous week &  $-$ \\
\; Verifications: overall percentage &  $-$ \\
\; Verifications: \# previous week & $-$ \\
\; Age & $+$ \\
\bottomrule
\end{tabular}
\end{center}
\end{table}



\subsection{\added{Robustness of State Space Representation}} \label{ss.state_space_results}
\added{One of the factors attributing to the performance gap between $\DPI$ and $\QWI$ is the differences in state space representation. $\QWI$ requires a finite state space and its computation time scales with the number of states. Hence, we use a relatively small state space for $\QWI$ for our numerical experiments. The ease of using a larger and infinite size state space is an inherent advantage of $\DPI$ over $\QWI$; however, in order to isolate the performance difference caused by the algorithm itself, we run $\DPI$ using the same state space as $\QWI$.

We try two variants of $\DPI$ that differ in the state space used:
\begin{itemize}
	\item \textsf{DecompPI-3}: This uses the same three features used for $\QWI$ (number of times verified in the last week, total historical verification rate, and number of interventions received in the last week), but these features are \textit{not discretized}, and hence the size of the state space is still infinite.
	\item \textsf{DecompPI-3-discrete}: This uses the exact same state space as $\QWI$ --- three features that are discretized in the same way, resulting in 45 states.
\end{itemize}


\begin{figure}[h]
\begin{center}
\vspace{-3mm}
  \includegraphics[width=1\linewidth]{figs/2024_May13_results3D}
  \caption{Average overall verification rate over 50 runs for each policy and budget. The shaded region indicates a 95\% confidence interval. The star represents the operating point for Keheala.}
  \label{fig:DPI3}
\vspace{-6mm}
\end{center}
\end{figure}

The performance of these two algorithms, along with the original $\DPI$ and $\QWI$ policies are shown in Figure~\ref{fig:DPI3}.
The policies \textsf{DecompPI-3-discrete} and $\QWI$ use the exact same state space, and we see that the former consistently outperforms the latter.
This comparison isolates the performance gap induced by the \emph{algorithms}, and the results provide robust evidence on the strength of $\DPI$.
Lastly, we see that \textsf{DecompPI-3} consistently has a strong performance, comparable to that of $\DPI$.
This demonstrates the robustness of $\DPI$ with respect to the feature space, and it also exemplifies the benefit of employing an infinite state space compared to a discretized one.  
}















\section{Conclusion}\label{sec:conclusion}
In this work, we focus on addressing the fundamental challenge of OOD detection tasks, which is how to fully understand the semantic discrepancy between the ID/OOD samples. We reveal that the key to success in the realistic SCOOD task is to allocate as many ID samples in the unlabeled set correctly as possible. To this end, we propose a novel uncertainty-aware optimal transport scheme that introduces class-specific energy scores as guidance for effective label assignment. Experimental results show that our method achieves better performance than previous state-of-the-art methods on SCOOD benchmarks.

\textbf{Limitations.} In addition to temperature scaling, other techniques such as feature clipping applied in ReAct~\cite{sun2021react} also enhance the performance of energy score, so how to obtain an OOD score that best fits the SCOOD task can be further explored. Moreover, a setting highly related to SCOOD has been proposed in \cite{katz2022training} and formulated as a constrained optimization problem. We will also theoretically analyze these practical OOD settings in our feature work.

% \section*{Acknowledgments}
\textbf{Acknowledgments.} 
This work is supported by National Key R\&D Program of China under Grant 2020AAA0105701, National Natural Science Foundation of China (NSFC) under Grants 61872327, Major Special Science and Technology Project of Anhui, National Natural Science Foundation of China (62033012) and Ant Group through Ant Research Intern Program.


\section{Acknowledgements}
    The authors would like to thank Jocelyn Mbenoun for the templates and the useful conversations about energy, as well as Bardhyl Miftari and Guillaume Derval for their useful help with shadow pricing. The authors extend also their thanks to Julien Confetti for his precious help in the elaboration of script for generating the multi-hub picture. This research is supported by the public service of the Belgium federal government (SPF Économie, P.M.E., Classes moyennes et Energie) within the framework of the DRIVER project. Victor Dachet was supported by the Walloon region (Service Public de Wallonie Recherche, Belgium) under grant n°2010235 – ARIAC by
    \hyperlink{https://digitalwallonia4.ai/}{digitalwallonia4.ai}.

% \section{Appendices}\label{sec:annex}
% \subsection{Previous scenari}

\begin{table}[h]
    \centering
    \begin{tabular}{|c|c|c|c|c|c|c|c|}
            \hline
            scenario & time horizon & Cap on CO2 & Cost of CO2 & ENS ALLOWED & Cost ENS & Pipe and/or Boat & Objective \\ \hline
            1 & 8760 & 0.0 & 0.0 & False & - & pipe and boat & 39992.71 \\
            2 & 8760 & None & 0.08 & True & 3.0 & pipe and boat & 37515.28 \\
            3 & 8760 & None & 0.0 & True & 3.0 & pipe and boat & 36107.64 \\
            4 & 8760 & None & 0.08 & True & 3.0 & only pipe & 37552.58 \\
            5 & 8760 & None & 0.08 & True & 3.0 & only carrier & 37539.88 \\
            6 & 8760 & 0.0 & 0.0 & False & - & pipe and boat & 52345.09 \\
            7 & 8760 & None & 0.08 & False & - & pipe and boat & 39050.49 \\
            8 & 8760 & None & 0.1648981 & False & - & pipe and boat & 39992.71 \\
            \hline
            
    \end{tabular}
    \caption{Scenari parameters}
    \label{tab:scenario_parameters_prev}
\end{table}


\begin{table}[h]
    \centering
    \begin{tabular}{|c|c|c|c|c|c|c|c|}
         \hline
            scenario & wind on & wind off & solar\_be & ccgt\_be & wind\_gl & wind\_nz & solar\_nz \\ \hline
            1 & 8.40 & 7.71 & 9.56 & 22.41 & 0.00 & 98.13 & 91.39 \\ 
            2 & 8.40 & 8.00 & 13.83 & 17.39 & 0.00 & 87.31 & 81.09 \\ 
            3 & 8.40 & 8.00 & 13.29 & 17.32 & 0.00 & 86.46 & 80.49 \\ 
            4 & 8.40 & 8.00 & 13.42 & 17.30 & 0.00 & 86.42 & 80.45 \\ 
            5 & 8.40 & 8.00 & 13.88 & 17.50 & 0.00 & 87.93 & 81.63 \\ 
            6 & 8.40 & 8.00 & 15.68 & 19.58 & 126.54 & 0.00 & 0.00 \\ 
            7 & 8.40 & 7.19 & 8.95 & 22.06 & 0.00 & 93.04 & 86.40 \\ 
            8 & 8.40 & 7.71 & 9.56 & 22.41 & 0.00 & 98.18 & 91.43 \\  
            \hline
            
    \end{tabular}
    \caption{Total Power installation in GW}
    \label{tab:power_prev}
\end{table}



\begin{table}[h]
    \centering
    \begin{tabular}{|c|c|c|c|c|c|}
            \hline
            scenario & PCCC & PCCC CCGT & DAC NZ & DAC GR \\ \hline
            1 & 4.11 & 2.59 & 1.26 & 0.00 \\ 
            2 & 4.11 & 1.55 & 0.00 & 0.00 \\ 
            3 & 5.00 & 0.44 & 0.00 & 0.00 \\ 
            4 & 4.11 & 1.33 & 0.00 & 0.00 \\ 
            5 & 4.11 & 1.59 & 0.00 & 0.00 \\ 
            6 & 4.11 & 2.59 & - & 1.22 \\ 
            7 & 4.11 & 1.76 & 0.00 & 0.00 \\ 
            8 & 4.11 & 2.59 & 1.27 & 0.00 \\
            \hline
            
    \end{tabular}
    \caption{Technology of capture with capacity in kt/h}
    \label{tab:capture_co2_prev}
\end{table}


\begin{table}[h]
    \centering
    \begin{tabular}{|c|c|c|c|c|}
            \hline
            Scenario & pipe nz & carrier nz & pipe gr & carrier gr \\
            \hline
            1 & 3.196 & 2.363 & 0.000 & 0.000 \\
            2 & 2.158 & 5.626 & 0.000 & 0.000 \\
            3 & 5.445 & 0.000 & 0.000 & 0.000 \\
            4 & 5.442 &  - & 0.000 &  - \\
            5 &  - & 9.280 &  - & 0.000 \\
            6 & 0.000 & 0.000 & 0.000 & 7.518 \\
            7 & 2.374 & 5.833 & 0.000 & 0.000 \\
            8 & 3.196 & 2.344 & 0.000 & 0.000 \\
            \hline
    \end{tabular}
    \caption{Transport technology with capacity in kt/h}
    \label{tab:transport_co2_prev}
\end{table}

\subsection{Technologies}\label{subsec:technologies}
In this sub-section, the different types of technologies are described, namely conversion and storage technology and balances (respectively represented by nodes and hyperedges in \autoref{sec:modelling}). The different constraints are also thouroughly described. The notation follows the convention taken in \cite{Berger2021} where the Latin letters denote optimisation variables and indices, while Greek letters indicate parameters. 

\subsubsection{Nodes}

% \textbf{Variable Energy Sources}: A set $\mathcal{P}_R = \{PV, W_{on}, W_{off}\}$:
    % $$
    % \begin{aligned}
    % & P_{E, t}^p \leq \pi_t^p\left(\kappa_0^p+K_E^p\right), \quad \forall t \in \mathcal{T}, \quad \forall p \in \mathcal{P}_R \\
    % & K_E^p \leq \kappa_{\max }^p, \quad \forall p \in \mathcal{P}_R
    % \end{aligned}
    % $$
    % with $\kappa_0^p$ represents the pre-installed capacity and $K_E^p$ the newly installed capacity. The parameter $\pi_t^p$ represents the load factor at timestep $t$. 
    % Investment and operating costs are described as
    % $$
    % C^p=\left(\zeta^p+\theta_f^p\right) K_E^p+\sum_{t \in \mathcal{T}} \theta_v^p P_{E, t}^p \delta t, \quad \forall p \in \mathcal{P}_R
    % $$
    % with 

% \textbf{Dispatchable Technologies}
    % \begin{equation}
        % \begin{aligned}
        % & P_{E, t}^p \leq \kappa_0^p+K_E^p, \quad \forall p \in \mathcal{P}_D \\
        % & P_{E, t}^p-P_{E, t-1}^p \leq \Delta_{+}^p\left(\kappa_0^p+K_E^p\right), \quad \forall p \in \mathcal{P}_D \\
        % & P_{E, t}^p-P_{E, t-1}^p \geq-\Delta_{-}^p\left(\kappa_0^p+K_E^p\right), \quad \forall p \in \mathcal{P}_D \\
        % & \mu^p\left(\kappa_0^p+K_E^p\right) \leq P_{E, t}^p, \quad \forall p \in \mathcal{P}_D \\
        % & Q_{\mathrm{CO}_2, t}^p=\frac{\nu^p P_{E, t}^p}{\eta^p}, \quad p \in\{\mathrm{BM}, % \mathrm{WS}\}
        % \end{aligned}
    % \end{equation}

\textbf{Conversion technologies}: 
Let us consider a given node $n \in \mathcal{N}$ wich is a conversion technology \textit{i.e.} which takes as input a given commodity $i$ flow and outputs another commodity $r$ flow. A conversion technology is described by an internal variable representing the capacity of a guven technology associated with a given commodity and by external variables linking the flows in and out. These relations are expressed as 

\begin{equation}
q_{rt}^n - \phi_{i}^n q_{i(t+\tau_{i}^n)}^n = 0, \mbox{ } \forall i \in \mathcal{I}^n\setminus\{r\}, \mbox{ } \forall t \in \mathcal{T}^n,
\label{eq:conversion}
\end{equation}
where $q_{it}^n \in \mathbb{R}^{+}$ is the flow of a given commodity $i$, $\phi_{i}^n$ is the conversion factor from $i$ to $r$ and $\tau_{i}^n$ is the time for the process to occur. Other constraints occur in conversion technology such as the maximum capacity expressed as

\begin{equation}
K_0^n - K_t^n = 0, \mbox{ } \forall t \in \mathcal{T}\setminus\{0\},
\label{eq:conversioncapacitystatic}
\end{equation}
where $K_t^n$ is the capacity at each time step of the time horizon considered. The maximum capacity remains constant over the entire horizon because we use a static investment policy. In the following $K^{n}$ will be a shorthand for $K_0^n$ and represents the new capacity installed. 

Some conversion technology are not dispatchable (\textit{e.g.} variable renewable energy). Therefore, the availability of such a technology is expressed as 

\begin{equation}
q_{r't}^n - \pi_{t}^n (\underbar{$\kappa$}^{n} + K^{n}) \le 0, \mbox{ } \forall t \in \mathcal{T},
\label{eq:sizing}
\end{equation}
where $\pi_{t}^n \in [0,1]$ is the normalized capacity factor of the technology $n$ at timestep $t$, $\underbar{$\kappa$}^{n}$ is the pre-installed capacity. 

Some technology capacities are bounded by a maximum potential capacity. This constraint reads as
\begin{equation}
(\underbar{$\kappa$}^{n} + K^{n}) - \bar{\kappa}^{n} \le 0,
\label{eq:potential}
\end{equation}
where $\bar{\kappa}^{n}$ is the maximum capacity of technology $n$.

To run a technology $n$ a minimal input flow may be needed and is expressed as
\begin{equation}
\mu^{n} (\underbar{$\kappa$}^{n} + K^{n}) - \frac{\phi_{i}^n}{\phi_{r'}^n}q_{it}^n \le 0, \mbox{ } \forall t \in \mathcal{T},
\label{eq:mustrun}
\end{equation}
where $\mu^{n} \in [0,1]$ represents the minimum operating level (as a fraction of the installed capacity). \textcolor{red}{Since the technology is sized with respect to the flow of commodity r', the flow of a commodity $i \neq r'$ must be scaled by the ratio of conversion factors in \ref{eq:mustrun}. }

Some conversion technologies are limited in the rate at which they can change a commodity flow. These are the so-called ramping up constraint defined as
\begin{equation}
\frac{\phi_{i}^n}{\phi_{r'}^n}(q_{it}^n - q_{i(t-1)}^n) - \Delta_{i,+}^{n} (\underbar{$\kappa$}^{n} + K^{n}) \le 0, \mbox{ } \forall t \in \mathcal{T}\setminus\{0\},
\label{eq:rampup}
\end{equation}

and ramping down constraint defined as
\begin{equation}
\frac{\phi_{i}^n}{\phi_{r'}^n}(q_{i(t-1)}^n - q_{it}^n) - \Delta_{i,-}^{n} (\underbar{$\kappa$}^{n} + K^{n}) \le 0, \mbox{ } \forall t \in \mathcal{T}\setminus\{0\},
\label{eq:rampdown}
\end{equation}
where $\Delta_{i,+}^{n} \in [0,1]$ and $\Delta_{i,-}^{n} \in [0,1]$ the maximum rates at which flows can be ramped up and down.

Finally, we get the local objective function for technology $n$ written as
\begin{equation}
F_n = \nu (\zeta^{n} + \theta_f^n) K^{n} + \sum_{t \in \mathcal{T}} \theta_{t,v}^n q_{r't}^n \delta t,\label{eq:objectiveconversion}
\end{equation}
where $\nu \in \mathbb{N}$ is the number of years spanned by the optimisation horizon, $\zeta^n \in \mathbb{R}_+$ represents the (annualised) investment cost (also known as capital expenditure, CAPEX), $\theta_f^n \in \mathbb{R}_+$ models fixed operation and maintenance (FOM) costs and $\theta_{t,v}^n \in \mathbb{R}_+$ represents variable operation and maintenance (VOM) costs, which may be time-dependent.


%%%%%%%%%%%%%%%
%   STORAGE   %
%%%%%%%%%%%%%%%

\textbf{Storage}: Let $n \in \mathcal{N}$ be a node representing a given storage technology. This later internal variable represents the current amount of commodity stored while the external variables are the flows in (charge) and out (discharge) of this technology as well as other commodity input flow needed. The dynamics of charge and discharge is described as follows

\begin{equation}
e_{t+1}^n - (1-\eta_{S}^n) e_{t}^n - \eta_{+}^{n} q_{it}^n + \frac{1}{\eta_{-}^{n}} q_{jt}^n = 0, \mbox{ } \forall t \in \mathcal{T} \setminus\{T-1\},
\label{eq:storagedynamics}
\end{equation}
where $e_{t}^n \in \mathbb{R}_+$ is the inventory level at time $t$, $q_{i_ut}^n \in \mathbb{R}_+$ and $q_{i_yt}^n \in \mathbb{R}_+$ represent commodity in and outflows at time $t$, respectively, $\eta_S^n \in [0, 1]$ is the self-discharge rate, $\eta_+^n \in [0, 1]$ is the charge efficiency and $\eta_-^n \in [0, 1]$ is the discharge efficiency. The consumption of other commodity may be modelled with

\begin{equation}
q_{lt}^n - \phi_{i}^n q_{it}^n = 0, \mbox{ } \forall t \in \mathcal{T}.
\label{eq:conversionstorage}
\end{equation}

To avoid board effect, we impose that the stock at the begininng of the time horizon is equal at the end of the time horzion. Thsi prevents the model to use freely a commodity stored. The constraint can be written as
\begin{equation}
e_{0}^n - e_{T-1}^n = 0.
\label{eq:storagecyclicity}
\end{equation}
where $e_{0}^n$ is the inventory level at timestep zero and $e_{T-1}^n$ is the inventory level at the end of the spanned time horizon. 

As for conversion technology (\textit{cfr} \autoref{eq:conversioncapacitystatic}) the policy of investment is static. Thus it can be expressed as 
\begin{equation}
E_0^n - E_t^n = 0, \mbox{ } \forall t \in \mathcal{T}\setminus\{0\},
\label{eq:storagestockstatic}
\end{equation}
where $E_0^n$ is the newly installed capacity. In the later, the shorthand $E^n$ to denote $E_t^n$ will be used. 
The total storage capacity is defined as

\begin{equation}
e_{t}^n - (\underbar{$\epsilon$}^{n} + E^{n}) \le 0, \mbox{ } \forall t \in \mathcal{T},
\label{eq:storagestocksizing}
\end{equation}

As stated in \autoref{eq:potential} for conversion technologies, storage technology might have a maximum storage capacity. This is expressed as 
\begin{equation}
(\underbar{$\epsilon$}^{n} + E^{n}) - \bar{\epsilon}^{n} \le 0,
\label{eq:storagepotential}
\end{equation}
where $\bar{\epsilon}^n \in \mathbb{R}_+$ represents the maximum stock capacity that may be deployed. 

A minimum inventory level may be used and described as follows


\begin{equation}
\sigma^{n}(\underbar{$\epsilon$}^{n} + E^{n}) - e_{t}^n \le 0, \mbox{ } t \in \mathcal{T},
\label{eq:storageminimumsoc}
\end{equation}
where $\sigma^{n} \in [0, 1]$ is the mimum level of the inventory as a proportion of the installed capacity. 

The maximum inflow is modelled using an internal variable that is also constant throughout the time horizon considered, as in Eq. \eqref{eq:conversioncapacitystatic}. It is defined as follows
\begin{equation}
q_{it}^n - (\underbar{$\kappa$}^{n} + K^{n}) \le 0, \mbox{ } \forall t \in \mathcal{T},
\label{eq:storageflowplussizing}
\end{equation}
where $\underbar{$\kappa$}^n \in \mathbb{R}_+$ denotes the existing flow capacity and $K^n \in \mathbb{R}_+$ is used as shorthand for the new capacity. The maximum in- and outflows may be asymmetric, depending on the properties of the underlying technology, which is modelled via
\textcolor{red}{The maximum in- and outflows may be asymmetric, depending on the properties of the underlying technology, which is modelled via}
\begin{equation}
q_{jt}^n - \rho^{n} (\underbar{$\kappa$}^{n} + K^{n}) \le 0, \mbox{ } \forall t \in \mathcal{T},
\label{eq:storageflowminussizing}
\end{equation}
where $\rho^n \in \mathbb{R}_+$ represents the maximum discharge-to-charge ratio.
Finally the local objective for node $n$ can be written as follows
\begin{equation}
F^n = \Big[\nu (\varsigma^{n} + \vartheta^{n}_{f}) E^{n} + \sum_{t \in \mathcal{T}} \vartheta_{t,v}^n e_{t}^n\Big] + \Big[\nu (\zeta^{n} + \theta^n_f) K^{n} + \sum_{t \in \mathcal{T}} \theta_{t,v}^n q_{it}^n \delta t \Big].
\label{eq:objectivestorage}
\end{equation}
where $\varsigma^n \in \mathbb{R}_+$ and $\zeta^n \in \mathbb{R}_+$ represent the stock and flow components of CAPEX, $\vartheta_f^n \in \mathbb{R}_+$ and $\theta_f^n \in \mathbb{R}_+$ model the stock and flow components of FOM costs, while $\vartheta_{t,v}^n \in \mathbb{R}_+$ and $\theta_{t,v}^n \in \mathbb{R}_+$ represent the stock and flow components of VOM costs, which may be time-dependent.

\subsubsection{Hyperedges}
\textbf{Conservation}: Let $e\in \mathcal{E}$ be a conservation hyperedge which ensure that all the flows going in and out of a given commodity are respected. 
\begin{equation}
\sum_{n \in e_T} q_{it}^n - \sum_{n \in e_H} q_{it}^n - \lambda_{t}^e = 0, \mbox{ } \forall t \in \mathcal{T},
\label{eq:conservationhyperedge}
\end{equation}


\vspace{2ex}

\begin{appendices}    
    \section{Glossary}
    \begin{tabbing}
        xxxxxxxxx\= xxxxxxxxxxxxxxxxxxxxxxxxxx \kill
        BE \> Belgium\\
        CAPEX \> Capital Expenditure\\
        CCGT \> Combined Cycle Gas Turbine\\
        DAC \> Direct Air Capture\\
        DZ \> Algeria\\
        EDC \> Energy Demand Center\\
        ENS \> Energy Not Served \\
        ETS \> Emission Trading System\\
        GBOML \> Graph Based Optimzation Modeling Language\\
        GL \> Greenland\\
        HHV \> Higher Heating Value\\
        OPEX \> Operational Expenditure\\
        PCCC \> Post Combustion Carbon Capture\\
        PV \> Photovoltaic \\
        RE \> Renewable Energy \\
        RREH \> Remote Renewable Energy Hub \\
        RES \> Renewable Energy Sources \\
    \end{tabbing}
\end{appendices}

%\begin{table}[tbp]
%\caption{\label{tab-nomenclature} Nomenclature.}
%\centering

%\begin{table*}
\section*{Nomenclature}
\small
%\footnotesize
%\fontsize{8pt}{9.6pt}
%\begin{longtable}{cclccl}
\begin{tabularx}{\textwidth}{cclccl}
\toprule
Symbol                      & Units                &  Description                                & Symbol                       & Units                &  Description                                  \\
\midrule
$\bol{A}$                   & \si{N/A}             &  Magnetic potential                         &  $u_\text{jet}$              & \si{m/s}             &  Jet velocity                               \\
$\bol{B}$                   & \si{N/(A m)}         &  Magnetic flux density                      &  $\bol{U}$                   & N.D.                 &  Nondimensional velocity                    \\
$B_0$                       & \si{N/(A m)}         &  Scale for magnetic flux density            &  $W(x)$                      & \si{1/m^2}           &  Spatial distribution of laser spot         \\
$\widehat{\bol{B}}$         & N.D.                 &  Nondimensional magnetic flux density       &  $x,y,z$                     & \si{m}               &  Components of coordinates                  \\
$c_p$                       & \si{J/(kg K)}        &  Specific heat                              &  $\bol{x}$                   & \si{m}               &  Spatial coordinates                        \\
$C_D$                       & N.D.                 &  Drag coefficient                           &  $\bol{X}$                   & N.D.                 &  Nondimensional coordinates                 \\
$d$                         & \si{m}               &  Droplet diameter                           &  $z_c$                       & \si{m}               &  Axial position of droplet center           \\
$d_\text{noz}$              & \si{m}               &  Nozzle outlet diameter                     &  $\alpha$                    & \si{m^2/s}           &  Thermal diffusivity                        \\
$\bol{D}$                   & \si{s^{-1}}          &  Strain rate tensor                         &  $\varepsilon$               & N.D.                 &  Emissivity                                 \\
$\bol{e}_g$                 & N.D.                 &  Unit vector along gravity direction        &  $\eta$                      & \si{1/m}             &  Attenuation coefficient                    \\
$\bol{e}_z$                 & N.D.                 &  Unit vector along axial direction          &  $\Theta$                    & N.D                  &  Nondimensional temperature                 \\
$\bol{f}_m$                 & \si{N/m^3}           &  Lorentz force per unit volume              &  $\lambda$                   & \si{W/(m K)}         &  Thermal conductivity                       \\
$F_m$                       & \si{N}               &  Levitation force                           &  $\mu$                       & \si{Pa s}            &  Viscosity                                  \\
$\bol{g}$                   & \si{m/s^2}           &  Gravitational acceleration                 &  $\mu_0$                     & \si{N/A^2}           &  Permeability of free space                 \\
$h$                         & \si{W/(m^2 K)}       &  Heat transfer coefficient                  &  $\nu$                       & \si{m^2/s}           &  Kinematic  viscosity                       \\
$i$                         & N.D.                 &  Imaginary unit                             &  $\nu_\ast$                  & N.D.                 &  Viscosity ratio                            \\
$I_0$                       & \si{W}               &  Power of the laser heat source             &  $\rho$                      & \si{kg/m^3}          &  Density                                    \\
$I_s$                       & \si{A}               &  Electrical current amplitude               &  $\sigma_e$                  & \si{S/m}             &  Electrical conductivity                    \\
$\bol{J}$                   & \si{A/m^2}           &  Electric current density                   &  $\sigma_\text{SB}$          & \si{W/(m^2 K^4)}     &  Stefan Boltzmann constant                  \\
$\widehat{\bol{J}}$         & N.D.                 &  Nondimensional current density             &  $\sigma_T$                  & \si{N/(m K)}         &  Temperature coefficient of surface tension \\
$\mathcal{J}$               & N.D.                 &  Objective function                         &  $\tau$                      & N.D.                 &  Nondimensional time                        \\
$L$                         & \si{m}               &  Representative length                      &  $\tau_w$                    & \si{Pa}              &  Surface shear stress                       \\
$\bol{n}$                   & N.D.                 &  Unit normal vector                         &  $\phi_\text{jet}$           & \si{m^3/s}           &  Volumetric flow rate                       \\
$p$                         & \si{Pa}              &  Pressure                                   &  $\varphi$                   & \si{rad}             &  Azimuthal angle                            \\
$P$                         & N.D                  &  Nondimensional pressure                    &  $\omega$                    & \si{rad/s}           &  Angular frequency of electric current      \\
$p_d$                       & \si{Pa}              &  Dynamic pressure                           &  $\Bi$                       & N.D.                 &  Biot number                                \\
$q_m$                       & \si{W/m^3}           &  Joule heat                                 &  $\Ec$                       & N.D.                 &  Eckert number                              \\
$r$                         & \si{m}               &  Radial coordinate                          &  $\Ga$                       & N.D.                 &  Galilei number                             \\
$R_0$                       & \si{m}               &  Droplet radius                             &  $\Pl$                       & N.D.                 &  Planck number                              \\
$R_L$                       & \si{m}               &  Radius of laser spot                       &  $\La$                       & N.D.                 &  Laser power number                         \\
$R_s, \theta_s$             & \si{m}, \si{rad}     &  Coil position in spherical coordinate      &  $\Ma$                       & N.D.                 &  Marangoni number                           \\
$s$                         & \si{m}               &  Distance from the axis of laser spot       &  $\Mg$                       & N.D.                 &  Magnetic number                            \\
$t$                         & \si{s}               &  Time                                       &  $\Nu$                       & N.D.                 &  Nusselt number                             \\
$T$                         & \si{K}               &  Temperature                                &  $\Pm$                       & N.D.                 &  Magnetic Prandtl number                    \\
$T_a$                       & \si{K}               &  Ambient temperature                        &  $\PR$                       & N.D.                 &  Prandtl number                             \\
$T_\ast$                    & \si{K}               &  Melting point                              &  $\Rey$                      & N.D.                 &  Reynolds number                            \\
$\bol{u}$                   & \si{m/s}             &  Velocity                                   &  $\Rey_\text{jet}$           & N.D.                 &  Jet Reynolds number                        \\
$u_\text{max}$              & \si{m/s}             &  Maximum velocity                           &  $\Sp$                       & N.D.                 &  Shielding parameter                        \\
\bottomrule
\end{tabularx}
%\end{table*}


% \bibliographystyle{plainnat}
\bibliography{bib.bib}
% \printbibliography

\end{document}
