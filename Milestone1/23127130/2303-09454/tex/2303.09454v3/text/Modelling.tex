\section{Modelling}
\label{fig:Modelling}

This section describes the core of our methodology, a linear program (LP). The LP is modelled using a recently proposed framework, the Graph-Based Optimization Modeling Language (GBOML) that we present in section \ref{subsec:GBOML}. The LP formalization is provided in section \ref{subsec:modellingAssumptions}.

\subsection{GBOML}
\label{subsec:GBOML}

    In this work, we rely on a the Graph-Based Optimization Modelling Language (GBOML), a recently developed framework for enhancing the modelling of energy systems \cite{Miftari2022}. One key aspect when using GBOML is to divide the model into nodes and clusters; in our case, we consider the following nodes: on-shore Belgium, off-shore Belgium, and as many RRH as desired.
    
    The Belgian on-shore node models energy demands, CO2 emissions, PCCC and export of CO2. The RREH handles imports of CO2, production of H$_2$ from renewable energy, transformation of CO2 into CO2 neutral synthetic CH$_4$.


\subsection{Main modelling Assumptions}
\label{subsec:modellingAssumptions}


\subsubsection{Objective}
\label{subsub:objective}

    The objective function of the model is obtained by summing costs associated with all technologies and carriers involved in the model.
    
    Minimize the overall costs function (Capex, Opex, Energy not served, $CO_2$ emissions).

\subsection{CO2 emitters}
    \label{sub:CO2_emitters}
    
    The energy demand center includes several $CO_2$ emitters, mainly gas fired power plants and industries. Such quantities of $CO2_2$ are captured using PCCC technologies. 
    
    We denote by $\mathcal{P}_{CO_2}$ the set of all technologies that emit CO2 and that may be equiped with PCCC technologies.
    
    Each CO2 emitter is modelled according to the following characteristics:
    \begin{itemize}
        \item CO2 emission curve, in CO2 kt/h, $q^{CO_2,(i)}_t$, where $i \in \{ 1, \ldots , N^{CO_2}  \}$ indexes CO2 emitters, \textcolor{red}{Victor note: in \cite{BERGER_power_to_gas_2020106039} the notation is $Q^{p}_{CO_2, t}$ with $p \in \mathcal{P}_{CO_2}$ is there a reason to not follow the notation of Mathias? }
        \item For each CO2 emitter , there is the possibility to capture CO2 using PCCC technologies.
    \end{itemize}



\subsection{CO2 capture}

    Two main carbon capture processes are considered in this paper: Post Combustion Carbon Capture (PCCC) and Direct Air Capture (DAC).
    
    Different PCCC technologies exist, but absorption solvent-based methods is currently the leading one \cite{madejski2022methods}. It is based on a reaction between CO2 and a solvent in aqueous solution. The capture is achieved in a two-step process. In the first step, the post-combustion gas reacts with the solvent in order to catch the carbon dioxyde. In the second step, the $CO_2$ is regenerated at high temperature in a stripper.
    
    PCCC is modelled using the following characteristics:
    \begin{itemize}
        \item $\Phi^{p,CC}$: Electrical energy required per unit mass of CO2 captured via PCCC for technology $p \in \mathcal{P}_{CO_2}$.
        \item $Q^{c, CC}_{CO_2,t}$: fraction of CO2 mass flow of technology $p \in \mathcal{P}_{CO_2}$ captured via PCCC at time $t \in \{ 0, \ldots, T-1  \}$.
    \end{itemize}
    
    
    Direct Air Capture technologies are mainly divided into two categories: (i) high temperature aqueous solutions and (ii) low temperature solid sorbent systems \cite{fasihi2019techno}, the latter showing lower heat supply costs.
    
    DAC is modelled using the following characteristics:
    
    \begin{itemize}
        \item $\Phi^{DAC}_E$: Electrical energy required per unit mass of CO2 captured using DAC [GWh/kt]. 
        \item $\Phi^{DAC}_{NG}$: Natural gas energy required per unit mass of CO2 captured using DAC [GWh/kt].
        \item $Q^{DAC}_{CO_2, t}$: CO2 mass flow exiting DAC units at time $t \in \{ 0, \ldots, T-1  \}$ [kt/h].
        \item $Q^{DAC, A}_{CO_2, t}$: CO2 mass flow captured from the atmosphere via DAC at time $t \in \{ 0, \ldots, T-1  \}$ [kt/h].
    \end{itemize}
    


\subsubsection{Remote Renewable Energy Hubs}
\label{subsub:RREH}

    We denote by $\mathcal H$ the set of all RREH. Each hub $h \in \mathcal{H}$ is characterised using the following characteristics:
    \begin{itemize}
        \item Renewable energy resources, modelled in the form of capacity factor time series $\pi^{PV,h}_t$ (for solar PV) and $\pi^{W,h}_t$ (for wind energy) 
        \item Electrolysis and methanation plants. \textcolor{red}{Victor Note: cfr equations 17 to 21 and eq 28 to 31 in \cite{BERGER_power_to_gas_2020106039}}
        \item Geographical characteristics, including distance to the Energy Demand Center  $d^{h} \in \mathbb{R}$.
    \end{itemize}
    
    
    
    
    Many constraints associated with renewable energy generation, 


\subsubsection{CO2 and Synthetic methane transportation}
    Carriers : Pipes and boats.
    
    
    In this first version of our model, we assume a simplified version of the CO2 transportation system by considering a continuous flow between the RREHs and the energy demand center.

\subsubsection{Decision variables}
\label{subsub:carriers}

    Among variables to be optimised, there are both sizing and operation variables. Sizing variables relates with investment decision variables.
    
    how large generation and transport capacities should be sized. 

\subsubsection{Main cost assumptions}

    PCCC and DAC costs, as well as CO2  transportation costs, play a key role in our results.