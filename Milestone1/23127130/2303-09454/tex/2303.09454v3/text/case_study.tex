\section{Case Study: Belgium}

This case study is focused on Belgium with two remote renewable energy hubs: one located in Algeria and another one located in Greenland. We will analyse the techno-economic feasibilty of the system while responding to an energy demand composed only of electricity and gas in Belgium.

\vspace{1ex}
\subsection{Data}
The data cover 2 years: 2015 and 2016. It is used to characterize energy demand as well as load factors for renewable energy sources. 

\textbf{Renewable generation profiles}

In order to determine the generation profiles of variable energy sources in Belgium we use the data from the transmission system operator (TSO) of Belgium \cite{Elia_power}. The profiles for the RREH located in Algeria are extracted with the same methodology as in \cite{Berger2021}. For the RREH situated in Greenland, the profiles of renewable energy are extracted thanks to the MAR model \cite{MAR} and given a power curve for an offsore wind turbine MHI Vestas Offshore V164-9.5MW.


\textbf{Energy consumption}

The energy consumption data is collected for two energy vectors: gas (\cite{Fluxys}) and electricity (\cite{Elia_load}) with the same methodology as in \cite{BERGER_power_to_gas_2020106039}. In \autoref{fig:energy_demand}, the data corresponding to the two years is represented, where the signal is daily aggregated. In some cases, gas usage is shifted towards electricity needs, as described in \cite[section 4.2.2]{BERGER_power_to_gas_2020106039}. This shift is due to the use of heat pumps, which can help decarbonize heating in Europe. For both energy vectors, industrial and heating demands are taken into account. 

The peak power demand is equal to 60.13 GWh/h for both gas and electricity. The energy demand for electricity ranges from 6.42 to 20.29 GWh/h, while that for gas ranges from 5.51 to 39.84 GWh/h. The total energy demand is on average 106.45 TWh/year and 132.65 TWh/year for electricity and gas, respectively.

\begin{figure}[h]
    \centering
    \includegraphics[width=0.95\textwidth]{Figures/energy_demand.pdf}
    \caption{Daily aggregated profiles of electricity and natural gas demand covering the years 2015 and 2016 spanned by the optimisation.}
    \label{fig:energy_demand}
\end{figure}

\vspace{1ex}
\subsection{Model Configuration}\label{subsec:model_config}

Our model consists of three main components (see \autoref{fig:RREH_model}): the energy demand center located in Belgium and two Remote Renewable Energy Hubs (RREHs) situated in Algeria and Greenland. The RREH in Algeria is modeled as described in \cite{Berger2021} with the same techno-economic parameters. The distinction is made with the inclusion of the CO2 connection between Belgium and Algeria. The RREH in Greenland is similarly modeled, with the exception of the removal of the photovoltaic potential and the modification of the high-voltage direct current (HVDC) line to a length of 100 km rather than 1000 km. 

The transportation of CO2 is achieved through the use of boats, which have a CAPEX of 5M€/kt, a lifespan of 40 years, and an average daily energy consumption of 0.0150 GWh/day. CO2 transport data was obtained from \cite{DanishEnergyAgency}. The loading and traveling time for these boats are assumed identical to those for liquefied methane carriers \cite{Berger2021}, \textit{i.e.} 24 and 116 hours, respectively. In order to fill the tank of CO2 carriers with fuel (liquefied methane), these tanks are loaded when unloading the CO2 at the RREH. Indeed, at the RREH, synthetic CH4 is available without having undergone any additional transport-related losses. Except for the storage facilities, liquefaction of CO2 has been excluded from the model. Sideways analyses have confirmed that this assumption has a negligible impact on the optimal objective.

Belgium is modeled with an electricity and gas demand as depicted in \autoref{fig:energy_demand}, with various means of production, including wind power, solar power, and a combined cycle gas turbine. The solar potential is limited to 40GW. The wind potential is equal to 8.4 GW and 8 GW for onshore and offshore capacities, respectively. The techno-economic parameters of each technology deployed in Belgium follow those in \cite{BERGER_power_to_gas_2020106039}.

We have also added a CO2 source that is equivalent to 40Mt CO2/year, which corresponds to the energy sectors and industrial processes greenhouse gases in Belgium in 2019 \cite[Table 4.1.1 (pp. 165- 166)]{EU-commission}. We assume that we can install post-carbon capture technologies (PCCC) in these sectors.

In terms of carbon capture technologies, the model has access to direct air capture installed at the RREHs, as well as a PCCC in Belgium on the 40Mt of CO2 per year and a PCCC installation on the CCGT.

As stated in \cite{BERGER_power_to_gas_2020106039}, the cost of PCCC is 3150M€/kt/h of CAPEX. The variable operating and maintenance costs (VOM and FOM) have been neglected in this analysis. However, a demand of $0.4125 GWh_{el}/kt_{CO2}$ of electricity is required. The expected lifetime is assumed to be 20 years.

Similarly, according to \cite{Berger2021}, the cost of DAC is equal to 4801.4 M€/kt/h of CAPEX. Similar to PCCC, VOM and FOM are ignored. The operational requirements for DAC are $0.1091 GWh_{el}/kt_{CO2}$ of electricity, $0.0438 kt_{H2}/kt_{CO2}$ of di-hydrogen, and $5.0 kt_{H20}/kt_{CO2}$ of water. The expected lifetime is assumed to be 30 years.

\vspace{1ex}
\subsection{Results}


In this section, we explore several scenari. We describe the variables that are used to differentiate the scenari
\begin{enumerate}
    \item Cost or Cap on CO2: either a cap is set of 0 t/year or a price at 80€/t or 0€/t
    \item Cost of energy not served (ENS): either energy not served is not allowed or a penalty of 3000€/MWh is imposed for each unit of unproduced energy.
    \item Forcing or not the use of a given RREH.
\end{enumerate}

\newpage
The results are generated with 5 scenari:

\textbf{Scenario 1}: This scenario seeks to avoid energy scarcity, whatever the cost. Therefore, no ENS is allowed. In addition, a hard constraint is set on CO2 emissions: a cap on CO2 is set.

\textbf{Scenario 2}:  This scenario follows the same assumptions as scenario 1 except that it leverages the constraint on energy not served. The cost associated to electricity not served is equal to 3000€/MWh, which is a standard value in the electricity context \cite{Voll}.

\textbf{Scenario 3}: This scenario leverages the constraint on CO2 emissions, and does not force the avoidance of energy not served but is penalized by 3000€/MWh not served. A penalty is associated with any CO2 emission in the atmosphere in the form of a fee equal to 80€/t - a value that reflects the current price of CO2 in the EU-ETS trading system \cite{CO2_price}.

\textbf{Scenario 4}: This scenario follows the same assumptions as scenario 3, with the difference that the cost of CO2 is equal to 0€/MWh. The aim is to showcase the system's configuration in the absence of any considerations for CO2 emissions.

\textbf{Scenario 5}: This scenario follows the same assumptions as scenario 1, with the difference that the only available RREH is in Greenland. 

\begin{table}[t] % 83745.0
    \centering
    \begin{tabular}{|c|c|c|c|c|c|}
            \hline
            Scenario & Cap on CO2  & Cost of CO2  & ENS & Cost ENS  & Objective \\ 
              & (kt) & (M€/kt) &  & (k€/MWh) & (M€) \\
            \hline
            1 & 0.0 & 0.0 & No & - & 83742.61 \\
            2 & 0.0 & 0.0 & Yes & 3.0 & 80778.02 \\
            3 & No & 0.08 & Yes & 3.0 & 78872.94 \\
            4 & No & 0.0 & Yes & 3.0 & 76323.94 \\
            5 & 0.0 & 0.0 & No & - & 111209.95 \\
            \hline
            
    \end{tabular}
    \caption{Scenari parameters.}
    \label{tab:scenario_parameters}
\end{table}

These scenari summarized in \autoref{tab:scenario_parameters} vary in their degree of constraint. Scenario 1 is the most restrictive, with a cap on CO2 emissions and no allowance for energy not served. Scenario 2 allows for energy not served, while scenarios 3 and 4 remove the cap and replace it with CO2 prices of 80€ and 0€ per ton, respectively. Finally, scenario 5 requires the use of the RREH in Greenland, with parameters identical to those of scenario 1.

\vspace{1cm}
\begin{figure}[h]
    \centering
    \includegraphics[width=\textwidth]{Figures/costs_subplot.pdf}
    \caption{(a): Breakdown of costs per scenario and per cluster (Belgium (BE), Algeria (DZ), and Greenland (GL)). (b): Breakdown of costs per scenario per asset function. Flexibility covers storage capacities, CO2 Infra covers CO2 capture, storage, and transport, Power covers means of electricity production, Conversion covers all assets that convert one commodity into another and Transport HVDC lines and CH4 carriers.}
    \label{fig:cost_per_scenario_cluster}
\end{figure}

\vspace{1ex}
\subsection{Analyses and Discussion}
In this section, we introduce and discuss the results in detail. We choose to present a cross-scenario analysis in the light of key indicators and statistics extracted from the model.



\textbf{Total cost.}

The results indicate that the costs associated with enabling the hub in Algeria are substantially lower than those in Greenland, as depicted in  \autoref{fig:cost_per_scenario_cluster} (a) where nothing is built in the Greenland hub from scenarios 1 to 4, despite it being available for use. This disparity in costs can be attributed to the over-dimensioning of flexibility assets, particularly the storage capacities, as illustrated in \autoref{fig:cost_per_scenario_cluster} (b). This is mainly explained to electricity generated solely through wind available in Greenland, whereas both solar and wind electricity are obtainable in Algeria. This implies that the flexibility assets have to take the lead in maintaining a minimum of electricity delivery required in the electrolysis power plant. 

Furthermore, a reduction in total costs is observed in the first four scenarios with respect to the objective. This is explained with the order on the scenari based on their degree of constraint with scenario 1 being the most constrained and scenario 4 being the least.



\textbf{Power installation capacities.}

All power capacities installations are displayed in \autoref{tab:power}.

The potential in Belgium of solar energy is never reached while for both wind offshore and onshore the potential is reached in all scenari. 

From scenario 1 to scenario 2, the only difference being the allowance of ENS, there is an increase in the installation of controllable energy production assets. Indeed, there is a shift in capacity from CCGT to solar energy in Belgium between the first scenario and the second.

Comparing scenario 1 and 5, solar energy in Belgium is more expensive than importing CH4 from the RREH in Algeria. Importing from Greenland is more expensive and leads to an increase in power capacity installation in Belgium for solar, but it does not reach the maximum potential.

Another interesting comparison can be made with the work of \cite{Berger2021}, where the capacity installation in the hub for the reference scenario is 4.3GW of solar and 4.4GW of wind. In our case, the reference scenario 1 displays 100.51GW and 103.62GW, respectively. The power installation capacity is multiplied by approximately 23 while providing, on average, 282TWh/year of gas (HHV) to serve the gas demand and part of the electricity demand in Belgium, which is 28.2 times the gas production in the original paper.

\begin{table}[h]
    \centering
    \begin{tabular}{|c|c|c|c|c|c|c|c|}
         \hline
            Scenario & Wind onshore & Wind offshore & Solar & CCGT & Wind & Wind & Solar \\
             & BE & BE & BE & BE & GL & DZ & DZ \\\hline
            % 0 & 8.40 & 8.00 & 10.57 & 22.40 & 0.00 & 101.51 & 98.45 \\ 
            1 & 8.40 & 8.00 & 10.56 & 22.69 & 0.00 & 103.62 & 100.51 \\ 
            2 & 8.40 & 8.00 & 15.35 & 17.95 & 0.00 & 98.43 & 95.47 \\ 
            3 & 8.40 & 8.00 & 14.95 & 17.83 & 0.00 & 93.32 & 90.32 \\ 
            4 & 8.40 & 8.00 & 14.72 & 17.82 & 0.00 & 93.28 & 90.28 \\ 
            5 & 8.40 & 8.00 & 17.48 & 19.58 & 129.43 & 0.00 & 0.00 \\ 
            \hline
            
    \end{tabular}
    \caption{Total Power installation in GW per scenario.}
    \label{tab:power}
\end{table}


\textbf{CO2 installations (transport, capture).} 

% Why scenario 4 has not installed the max capacity of PCCC in Belgium? It seems that a mixture of both PCCC is better than with

In \autoref{tab:capture_co2}, the capacities of the CO2 capture units and the installations of transport capacity per scenario are displayed. Each time PCCC is activated, we recall that capturing CO2 is the only means to create gas in our system, and thus a minimum installation is required to support the demand. On the other hand, the DAC is only activated when a CO2 cap is set. PCCC has an efficiency of CO2 capture set to 90\%, which means that a direct air capture technology asset is necessary to recover the remaining 10\% of emissions in the atmosphere. This leads to a direct consequence, which is that when the DAC is available, the capacity of transport decreases because CO2 is locally available in the hub. However, the cost of CO2 capture by PCCC added to transport of CO2 is cheaper than the cost of DAC in the RREH. The only way to put PCCC out of business would be to have a distance between the hub and the energy demand center so long that the transport cost would increase too much.

Due to the higher concentration of CO2 in manufacturing smoke compared to the air, PCCC will likely always be cheaper than DAC, even with significant improvements in the DAC process. As a result, the operational costs associated with the energy required for PCCC will be lower than those of DAC.

\begin{table}[h]
    \centering
    \begin{tabular}{|c|c|c|c|c|c|c|c|}
            \hline
            Scenario & PCCC & PCCC CCGT & DAC DZ & DAC GL & Carrier DZ & Carrier GL \\ \hline
            1 & 4.11 & 2.62 & 1.30 & 0.00 & 8.030 & 0.000 \\
            2 & 4.11 & 2.07 & 1.47 & 0.00 & 7.142 & 0.000 \\
            3 & 4.11 & 1.80 & 0.00 & 0.00 & 9.694 & 0.000 \\
            4 & 3.76 & 2.06 & 0.00 & 0.00 & 9.701 & 0.000 \\
            5 & 4.11 & 2.40 & 0.00 & 1.35 & 0.000 & 7.564 \\
            \hline
            
    \end{tabular}
    \caption{Capacity, in kt/h, of CO2 capture technology and transport by hub and per scenario.}
    \label{tab:capture_co2}
\end{table}

% \begin{table}[h]
%     \centering
%     \begin{tabular}{|c|c|c|}
%             \hline
%             Scenario & DZ & GL \\
%             \hline
%             1 & 8.030 & 0.000 \\
%             2 & 7.142 & 0.000 \\
%             3 & 9.694 & 0.000 \\
%             4 & 9.701 & 0.000 \\
%             5 & 0.000 & 7.564 \\
%             \hline
%     \end{tabular}
%     \caption{Transport CO2 carrier capacity in kt/h, by hub.}
%     \label{tab:transport_co2}
% \end{table}


\textbf{Cost of CO2 derived and Cap of CO2.}

From the first, second, and fifth scenarios, we are able to derive a shadow price thanks to the CO2 cap constraint. These correspond to approximately 162.77€/tCO2 for the first and second scenarios and 235.65€/tCO2 for the fifth scenario. This shows that given the system considered, i.e., Belgium and RREHs, putting a price of CO2 equal to 162.77€ would avoid these emissions in the atmosphere and activate the export of CO2 to Norway for storage purposes. In scenario 3, where a price of 80€/tCO2 is set, there is a net balance in the atmosphere of approximately 15Mt/year. In scenario 4, where no price is fixed, there is a net balance in the atmosphere which is equivalent to 16Mt/year. 

We would like to emphasize that the CO2 cap in our model only considers the emissions from the industrial and energy sectors, which are fully modeled. It does not account for a part of the emissions resulting from the gas demand served. Of this demand, 32\% is attributed to industrial needs, which are included in the statistics of the 40 Mt of CO2 emitted per year (see \autoref{subsec:model_config}), while the remaining 68\% is due to heating and is not covered by our cap. This heating gas demand translates to approximately 12.3 Mt of CO2 emitted per year.


% 30419.178/2 kt/year, which is equivalent to 15Mt/year. In scenario 4, where no price is fixed, there is a net balance in the atmosphere of 31974.940/2 kt/year, which is equivalent to 16Mt/year. 

% Shadow price 1st scenario: 162€/t.
% Scenario 1: -0.16276696569145419 M€/kt
% Scenario 2: -0.16276711070181626
% Scenario 5: -0.23565231232553557

\textbf{Cost of CH4 derived}

To estimate the cost of CH4 production, we first subtract from the optimal objective function the cost of the means of electricity production in Belgium (PV, on/off shore wind, CCGT), the cost of unserved energy (when applicable), and the cost related to export of CO2 for sequestration. All of these costs are substracted because they do not refer directly to the cost of producing synthetic methane. Then, we divide the obtained cost by the total energy content (HHV) in CH4 produced at the output of the regasification power plant in Belgium.

These methane costs, listed in \autoref{tab:ch4_price}, are compared to the price of 147.9€/MWh of methane (HHV) obtained by \cite{Berger2021}. Our scenarios achieve a lower cost for gas production (except for Greenland). This demonstrates that PCCC, which uses smoke with a high concentration of CO2 combined with transport, is more cost-effective than having only access to a DAC unit, as previously mentioned.

In our system, no fossil gas is available for import to Belgium; only synthetic gas produced from CO2 capture is used. If fossil gas were still available for import, our model would seek to minimize costs and import as much cheap gas as possible while staying within our carbon budget.

\begin{table}[h]
    \centering
    \begin{tabular}{|c|c|c|c|c|c|}
        \hline
        Scenario & 1 & 2 & 3 & 4 & 5 \\
        \hline
        [€/MWh] & 136.00 & 137.19 & 133.89 & 129.27 & 192.00 \\
        \hline
    \end{tabular}
    \caption{Estimation of methane price by retrieving the costs of power installations in Belgium, costs of unserved energy, and costs of exporting CO2 for storage purposes.}
    \label{tab:ch4_price}
\end{table}



% \textbf{Robustness and Competition}


% One may argue that a supply chain system with a single RREH is not resilient to perturbations. In our linear model, once the capital expenditure (capex) is paid, the only expense is operational costs. Due to higher operational costs in Greenland, the capex is paid only once in Algeria. Multiple RREHs may increase robustness.

% Our model effectively captures competition by identifying a low-cost RREH, such as Algeria, and determining if a new RREH can enter the market. However, this is not possible for Greenland with the current configuration. Therefore, further analysis is required to optimize the RREH configuration in Greenland and improve its effectiveness compared to the RREH in Algeria.
\begin{figure}[h]
    \centering
    \includegraphics[width=0.47\textwidth]{Figures/blackout.pdf}
    \caption{Evening of January 18th leading to the maximum shadow price associated with the hard constraint on energy not served in scenarios 1 and 5.}
    \label{fig:blackout}
\end{figure}

\textbf{ENS cost discussion}

The cost of unserved energy is a fixed parameter in scenarios 2, 3, and 4, but not in scenarios 1 and 5. Instead, a hard constraint is imposed to ensure that electricity demand is always met, resulting in a shadow price associated with the constraint. The maximum shadow price values for scenarios 1 and 5 are 913,640€/MWh and 1,075,913€/MWh, respectively. This is attributed to the peak in electricity and gas demand observed on January 18th at 18:00 (as shown in Figure \ref{fig:blackout}), where renewable energy load factors were low. Thus, all energy demand had to be supplied by the Combined Cycle Gas Turbine (CCGT) and gas resources.





%\begin{table}[]
%    \centering
%    \begin{tabular}{|c|c|c|c|c|}
%         \hline
%        scenario & wind on & Wind off & Solar & CCGT \\
%        \hline
%        1 & 0.110 & 0.100 & 0.073 & 0.717 \\
%         2 & 0.114 & 0.130 & 0.106 & 0.650 \\
%         3 & 0.115 & 0.132 & 0.104 & 0.649 \\
%         4 & 0.115 & 0.131 & 0.103 & 0.650 \\
%         5 & 0.109 & 0.116 & 0.109 & 0.666 \\
%        \hline
%     \end{tabular}
%     \caption{Share of electricity production in Belgium TO REMOVE }
%     \label{tab:share_elec_be}
% \end{table}









% \begin{enumerate}
%     \item PCCC VS DAC: which is the cheapiest ? PCCC would be out of business for 2 main reasons:
%     \begin{itemize}
%         \item Cost of transport: e.g. Distance of PCCC to methanation unit or cost of carriers vs pipes.
%         \item Highlight the fact that phisically speaking the PCCC will be always cheaper in OPEX than DAC. 
 %    \end{itemize}    
%     \item Derive a CO2 cost to enable CO2 capture and storage with no usage. Done scenario 1 shadow price
 %    \item Scenario with the price of the shadow price . Done
 %    \item Scenario without energy not served DONE
 %    \item A word on the gas if improrted as today in Belgium. + Price of gas in our model  + Discuss the CO2 emitted by the gas demand.
% \end{enumerate}