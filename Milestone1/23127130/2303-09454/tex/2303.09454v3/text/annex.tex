\subsection{Previous scenari}

\begin{table}[h]
    \centering
    \begin{tabular}{|c|c|c|c|c|c|c|c|}
            \hline
            scenario & time horizon & Cap on CO2 & Cost of CO2 & ENS ALLOWED & Cost ENS & Pipe and/or Boat & Objective \\ \hline
            1 & 8760 & 0.0 & 0.0 & False & - & pipe and boat & 39992.71 \\
            2 & 8760 & None & 0.08 & True & 3.0 & pipe and boat & 37515.28 \\
            3 & 8760 & None & 0.0 & True & 3.0 & pipe and boat & 36107.64 \\
            4 & 8760 & None & 0.08 & True & 3.0 & only pipe & 37552.58 \\
            5 & 8760 & None & 0.08 & True & 3.0 & only carrier & 37539.88 \\
            6 & 8760 & 0.0 & 0.0 & False & - & pipe and boat & 52345.09 \\
            7 & 8760 & None & 0.08 & False & - & pipe and boat & 39050.49 \\
            8 & 8760 & None & 0.1648981 & False & - & pipe and boat & 39992.71 \\
            \hline
            
    \end{tabular}
    \caption{Scenari parameters}
    \label{tab:scenario_parameters_prev}
\end{table}


\begin{table}[h]
    \centering
    \begin{tabular}{|c|c|c|c|c|c|c|c|}
         \hline
            scenario & wind on & wind off & solar\_be & ccgt\_be & wind\_gl & wind\_nz & solar\_nz \\ \hline
            1 & 8.40 & 7.71 & 9.56 & 22.41 & 0.00 & 98.13 & 91.39 \\ 
            2 & 8.40 & 8.00 & 13.83 & 17.39 & 0.00 & 87.31 & 81.09 \\ 
            3 & 8.40 & 8.00 & 13.29 & 17.32 & 0.00 & 86.46 & 80.49 \\ 
            4 & 8.40 & 8.00 & 13.42 & 17.30 & 0.00 & 86.42 & 80.45 \\ 
            5 & 8.40 & 8.00 & 13.88 & 17.50 & 0.00 & 87.93 & 81.63 \\ 
            6 & 8.40 & 8.00 & 15.68 & 19.58 & 126.54 & 0.00 & 0.00 \\ 
            7 & 8.40 & 7.19 & 8.95 & 22.06 & 0.00 & 93.04 & 86.40 \\ 
            8 & 8.40 & 7.71 & 9.56 & 22.41 & 0.00 & 98.18 & 91.43 \\  
            \hline
            
    \end{tabular}
    \caption{Total Power installation in GW}
    \label{tab:power_prev}
\end{table}



\begin{table}[h]
    \centering
    \begin{tabular}{|c|c|c|c|c|c|}
            \hline
            scenario & PCCC & PCCC CCGT & DAC NZ & DAC GR \\ \hline
            1 & 4.11 & 2.59 & 1.26 & 0.00 \\ 
            2 & 4.11 & 1.55 & 0.00 & 0.00 \\ 
            3 & 5.00 & 0.44 & 0.00 & 0.00 \\ 
            4 & 4.11 & 1.33 & 0.00 & 0.00 \\ 
            5 & 4.11 & 1.59 & 0.00 & 0.00 \\ 
            6 & 4.11 & 2.59 & - & 1.22 \\ 
            7 & 4.11 & 1.76 & 0.00 & 0.00 \\ 
            8 & 4.11 & 2.59 & 1.27 & 0.00 \\
            \hline
            
    \end{tabular}
    \caption{Technology of capture with capacity in kt/h}
    \label{tab:capture_co2_prev}
\end{table}


\begin{table}[h]
    \centering
    \begin{tabular}{|c|c|c|c|c|}
            \hline
            Scenario & pipe nz & carrier nz & pipe gr & carrier gr \\
            \hline
            1 & 3.196 & 2.363 & 0.000 & 0.000 \\
            2 & 2.158 & 5.626 & 0.000 & 0.000 \\
            3 & 5.445 & 0.000 & 0.000 & 0.000 \\
            4 & 5.442 &  - & 0.000 &  - \\
            5 &  - & 9.280 &  - & 0.000 \\
            6 & 0.000 & 0.000 & 0.000 & 7.518 \\
            7 & 2.374 & 5.833 & 0.000 & 0.000 \\
            8 & 3.196 & 2.344 & 0.000 & 0.000 \\
            \hline
    \end{tabular}
    \caption{Transport technology with capacity in kt/h}
    \label{tab:transport_co2_prev}
\end{table}

\subsection{Technologies}\label{subsec:technologies}
In this sub-section, the different types of technologies are described, namely conversion and storage technology and balances (respectively represented by nodes and hyperedges in \autoref{sec:modelling}). The different constraints are also thouroughly described. The notation follows the convention taken in \cite{Berger2021} where the Latin letters denote optimisation variables and indices, while Greek letters indicate parameters. 

\subsubsection{Nodes}

% \textbf{Variable Energy Sources}: A set $\mathcal{P}_R = \{PV, W_{on}, W_{off}\}$:
    % $$
    % \begin{aligned}
    % & P_{E, t}^p \leq \pi_t^p\left(\kappa_0^p+K_E^p\right), \quad \forall t \in \mathcal{T}, \quad \forall p \in \mathcal{P}_R \\
    % & K_E^p \leq \kappa_{\max }^p, \quad \forall p \in \mathcal{P}_R
    % \end{aligned}
    % $$
    % with $\kappa_0^p$ represents the pre-installed capacity and $K_E^p$ the newly installed capacity. The parameter $\pi_t^p$ represents the load factor at timestep $t$. 
    % Investment and operating costs are described as
    % $$
    % C^p=\left(\zeta^p+\theta_f^p\right) K_E^p+\sum_{t \in \mathcal{T}} \theta_v^p P_{E, t}^p \delta t, \quad \forall p \in \mathcal{P}_R
    % $$
    % with 

% \textbf{Dispatchable Technologies}
    % \begin{equation}
        % \begin{aligned}
        % & P_{E, t}^p \leq \kappa_0^p+K_E^p, \quad \forall p \in \mathcal{P}_D \\
        % & P_{E, t}^p-P_{E, t-1}^p \leq \Delta_{+}^p\left(\kappa_0^p+K_E^p\right), \quad \forall p \in \mathcal{P}_D \\
        % & P_{E, t}^p-P_{E, t-1}^p \geq-\Delta_{-}^p\left(\kappa_0^p+K_E^p\right), \quad \forall p \in \mathcal{P}_D \\
        % & \mu^p\left(\kappa_0^p+K_E^p\right) \leq P_{E, t}^p, \quad \forall p \in \mathcal{P}_D \\
        % & Q_{\mathrm{CO}_2, t}^p=\frac{\nu^p P_{E, t}^p}{\eta^p}, \quad p \in\{\mathrm{BM}, % \mathrm{WS}\}
        % \end{aligned}
    % \end{equation}

\textbf{Conversion technologies}: 
Let us consider a given node $n \in \mathcal{N}$ wich is a conversion technology \textit{i.e.} which takes as input a given commodity $i$ flow and outputs another commodity $r$ flow. A conversion technology is described by an internal variable representing the capacity of a guven technology associated with a given commodity and by external variables linking the flows in and out. These relations are expressed as 

\begin{equation}
q_{rt}^n - \phi_{i}^n q_{i(t+\tau_{i}^n)}^n = 0, \mbox{ } \forall i \in \mathcal{I}^n\setminus\{r\}, \mbox{ } \forall t \in \mathcal{T}^n,
\label{eq:conversion}
\end{equation}
where $q_{it}^n \in \mathbb{R}^{+}$ is the flow of a given commodity $i$, $\phi_{i}^n$ is the conversion factor from $i$ to $r$ and $\tau_{i}^n$ is the time for the process to occur. Other constraints occur in conversion technology such as the maximum capacity expressed as

\begin{equation}
K_0^n - K_t^n = 0, \mbox{ } \forall t \in \mathcal{T}\setminus\{0\},
\label{eq:conversioncapacitystatic}
\end{equation}
where $K_t^n$ is the capacity at each time step of the time horizon considered. The maximum capacity remains constant over the entire horizon because we use a static investment policy. In the following $K^{n}$ will be a shorthand for $K_0^n$ and represents the new capacity installed. 

Some conversion technology are not dispatchable (\textit{e.g.} variable renewable energy). Therefore, the availability of such a technology is expressed as 

\begin{equation}
q_{r't}^n - \pi_{t}^n (\underbar{$\kappa$}^{n} + K^{n}) \le 0, \mbox{ } \forall t \in \mathcal{T},
\label{eq:sizing}
\end{equation}
where $\pi_{t}^n \in [0,1]$ is the normalized capacity factor of the technology $n$ at timestep $t$, $\underbar{$\kappa$}^{n}$ is the pre-installed capacity. 

Some technology capacities are bounded by a maximum potential capacity. This constraint reads as
\begin{equation}
(\underbar{$\kappa$}^{n} + K^{n}) - \bar{\kappa}^{n} \le 0,
\label{eq:potential}
\end{equation}
where $\bar{\kappa}^{n}$ is the maximum capacity of technology $n$.

To run a technology $n$ a minimal input flow may be needed and is expressed as
\begin{equation}
\mu^{n} (\underbar{$\kappa$}^{n} + K^{n}) - \frac{\phi_{i}^n}{\phi_{r'}^n}q_{it}^n \le 0, \mbox{ } \forall t \in \mathcal{T},
\label{eq:mustrun}
\end{equation}
where $\mu^{n} \in [0,1]$ represents the minimum operating level (as a fraction of the installed capacity). \textcolor{red}{Since the technology is sized with respect to the flow of commodity r', the flow of a commodity $i \neq r'$ must be scaled by the ratio of conversion factors in \ref{eq:mustrun}. }

Some conversion technologies are limited in the rate at which they can change a commodity flow. These are the so-called ramping up constraint defined as
\begin{equation}
\frac{\phi_{i}^n}{\phi_{r'}^n}(q_{it}^n - q_{i(t-1)}^n) - \Delta_{i,+}^{n} (\underbar{$\kappa$}^{n} + K^{n}) \le 0, \mbox{ } \forall t \in \mathcal{T}\setminus\{0\},
\label{eq:rampup}
\end{equation}

and ramping down constraint defined as
\begin{equation}
\frac{\phi_{i}^n}{\phi_{r'}^n}(q_{i(t-1)}^n - q_{it}^n) - \Delta_{i,-}^{n} (\underbar{$\kappa$}^{n} + K^{n}) \le 0, \mbox{ } \forall t \in \mathcal{T}\setminus\{0\},
\label{eq:rampdown}
\end{equation}
where $\Delta_{i,+}^{n} \in [0,1]$ and $\Delta_{i,-}^{n} \in [0,1]$ the maximum rates at which flows can be ramped up and down.

Finally, we get the local objective function for technology $n$ written as
\begin{equation}
F_n = \nu (\zeta^{n} + \theta_f^n) K^{n} + \sum_{t \in \mathcal{T}} \theta_{t,v}^n q_{r't}^n \delta t,\label{eq:objectiveconversion}
\end{equation}
where $\nu \in \mathbb{N}$ is the number of years spanned by the optimisation horizon, $\zeta^n \in \mathbb{R}_+$ represents the (annualised) investment cost (also known as capital expenditure, CAPEX), $\theta_f^n \in \mathbb{R}_+$ models fixed operation and maintenance (FOM) costs and $\theta_{t,v}^n \in \mathbb{R}_+$ represents variable operation and maintenance (VOM) costs, which may be time-dependent.


%%%%%%%%%%%%%%%
%   STORAGE   %
%%%%%%%%%%%%%%%

\textbf{Storage}: Let $n \in \mathcal{N}$ be a node representing a given storage technology. This later internal variable represents the current amount of commodity stored while the external variables are the flows in (charge) and out (discharge) of this technology as well as other commodity input flow needed. The dynamics of charge and discharge is described as follows

\begin{equation}
e_{t+1}^n - (1-\eta_{S}^n) e_{t}^n - \eta_{+}^{n} q_{it}^n + \frac{1}{\eta_{-}^{n}} q_{jt}^n = 0, \mbox{ } \forall t \in \mathcal{T} \setminus\{T-1\},
\label{eq:storagedynamics}
\end{equation}
where $e_{t}^n \in \mathbb{R}_+$ is the inventory level at time $t$, $q_{i_ut}^n \in \mathbb{R}_+$ and $q_{i_yt}^n \in \mathbb{R}_+$ represent commodity in and outflows at time $t$, respectively, $\eta_S^n \in [0, 1]$ is the self-discharge rate, $\eta_+^n \in [0, 1]$ is the charge efficiency and $\eta_-^n \in [0, 1]$ is the discharge efficiency. The consumption of other commodity may be modelled with

\begin{equation}
q_{lt}^n - \phi_{i}^n q_{it}^n = 0, \mbox{ } \forall t \in \mathcal{T}.
\label{eq:conversionstorage}
\end{equation}

To avoid board effect, we impose that the stock at the begininng of the time horizon is equal at the end of the time horzion. Thsi prevents the model to use freely a commodity stored. The constraint can be written as
\begin{equation}
e_{0}^n - e_{T-1}^n = 0.
\label{eq:storagecyclicity}
\end{equation}
where $e_{0}^n$ is the inventory level at timestep zero and $e_{T-1}^n$ is the inventory level at the end of the spanned time horizon. 

As for conversion technology (\textit{cfr} \autoref{eq:conversioncapacitystatic}) the policy of investment is static. Thus it can be expressed as 
\begin{equation}
E_0^n - E_t^n = 0, \mbox{ } \forall t \in \mathcal{T}\setminus\{0\},
\label{eq:storagestockstatic}
\end{equation}
where $E_0^n$ is the newly installed capacity. In the later, the shorthand $E^n$ to denote $E_t^n$ will be used. 
The total storage capacity is defined as

\begin{equation}
e_{t}^n - (\underbar{$\epsilon$}^{n} + E^{n}) \le 0, \mbox{ } \forall t \in \mathcal{T},
\label{eq:storagestocksizing}
\end{equation}

As stated in \autoref{eq:potential} for conversion technologies, storage technology might have a maximum storage capacity. This is expressed as 
\begin{equation}
(\underbar{$\epsilon$}^{n} + E^{n}) - \bar{\epsilon}^{n} \le 0,
\label{eq:storagepotential}
\end{equation}
where $\bar{\epsilon}^n \in \mathbb{R}_+$ represents the maximum stock capacity that may be deployed. 

A minimum inventory level may be used and described as follows


\begin{equation}
\sigma^{n}(\underbar{$\epsilon$}^{n} + E^{n}) - e_{t}^n \le 0, \mbox{ } t \in \mathcal{T},
\label{eq:storageminimumsoc}
\end{equation}
where $\sigma^{n} \in [0, 1]$ is the mimum level of the inventory as a proportion of the installed capacity. 

The maximum inflow is modelled using an internal variable that is also constant throughout the time horizon considered, as in Eq. \eqref{eq:conversioncapacitystatic}. It is defined as follows
\begin{equation}
q_{it}^n - (\underbar{$\kappa$}^{n} + K^{n}) \le 0, \mbox{ } \forall t \in \mathcal{T},
\label{eq:storageflowplussizing}
\end{equation}
where $\underbar{$\kappa$}^n \in \mathbb{R}_+$ denotes the existing flow capacity and $K^n \in \mathbb{R}_+$ is used as shorthand for the new capacity. The maximum in- and outflows may be asymmetric, depending on the properties of the underlying technology, which is modelled via
\textcolor{red}{The maximum in- and outflows may be asymmetric, depending on the properties of the underlying technology, which is modelled via}
\begin{equation}
q_{jt}^n - \rho^{n} (\underbar{$\kappa$}^{n} + K^{n}) \le 0, \mbox{ } \forall t \in \mathcal{T},
\label{eq:storageflowminussizing}
\end{equation}
where $\rho^n \in \mathbb{R}_+$ represents the maximum discharge-to-charge ratio.
Finally the local objective for node $n$ can be written as follows
\begin{equation}
F^n = \Big[\nu (\varsigma^{n} + \vartheta^{n}_{f}) E^{n} + \sum_{t \in \mathcal{T}} \vartheta_{t,v}^n e_{t}^n\Big] + \Big[\nu (\zeta^{n} + \theta^n_f) K^{n} + \sum_{t \in \mathcal{T}} \theta_{t,v}^n q_{it}^n \delta t \Big].
\label{eq:objectivestorage}
\end{equation}
where $\varsigma^n \in \mathbb{R}_+$ and $\zeta^n \in \mathbb{R}_+$ represent the stock and flow components of CAPEX, $\vartheta_f^n \in \mathbb{R}_+$ and $\theta_f^n \in \mathbb{R}_+$ model the stock and flow components of FOM costs, while $\vartheta_{t,v}^n \in \mathbb{R}_+$ and $\theta_{t,v}^n \in \mathbb{R}_+$ represent the stock and flow components of VOM costs, which may be time-dependent.

\subsubsection{Hyperedges}
\textbf{Conservation}: Let $e\in \mathcal{E}$ be a conservation hyperedge which ensure that all the flows going in and out of a given commodity are respected. 
\begin{equation}
\sum_{n \in e_T} q_{it}^n - \sum_{n \in e_H} q_{it}^n - \lambda_{t}^e = 0, \mbox{ } \forall t \in \mathcal{T},
\label{eq:conservationhyperedge}
\end{equation}

