\section{Modelling}\label{sec:modelling}
    % This section gives insight into the optimization framework this work relies on. 
    % The multi-energy system model proposed in this work is built using the GBOML language introduced in \cite{Miftari2022}, a recently developed language dedicated to the modelling graph based optimisation of multi-energy systems. This class of problem can be seen as optimisation on graphs, in a sense that multi-energy system can be seen as a set of nodes $\mathcal{N}$ that contribute to the (linear) objective, while also including local constraints, as well as hyperedges $\mathcal{E}$ for modelling constraints between nodes, e.g., in our setting, between RREHs and the EDC. 
    
    % The formalism in this work follows the one introduced in \cite{Berger2021}.
    % The entire system is defined by a set of nodes $\mathcal{N}$ and a set of hyperedeges $\mathcal{E}$. The optimisation horizon is denoted by $T$, and time-steps are indexed by $t \in \mathcal{T}$ with $\mathcal{T} =  \{1, \ldots, T\}$.
    
    % A node $n \in \mathcal{N}$ is defined by internal $X^{n}$ and external $Z^{n}$ variables. Internal variables describe the specific characteristics of the unit. For example, the nominal power capacity installed of the asset.
    
    % Moreover, equality constraints $h_i(X^{n}, Z^{n}, t)=0$ with $i \in \mathcal{I}$ and inequality constraints $g_j(X^{n}, Z^{n}, t) \le 0$ with $j \in \mathcal{J}$, for each $t \in \mathcal{T}$. Those constraints enable the modelling of the operational constraints. 
    
    % Each node $n$ has a cost function associated $F^{n}(X^{n}, Z^{n}) = \sum_{t=1}^{T} f^{n}(X^{n},Z^{n},t) $. This cost function typically represents the so called capital expenditure and operational expenditure: CAPEX and OPEX respectively.
    
    % Finally, equality and inequality constraints on hyperedges can be defined as $H^{e}(Z^{e}) = 0$ and $G^{e}(Z^{e}) \le 0$ with $ e \in \mathcal{E}$. Those constraints enable the modelling of laws of conservation as well as cap on given commodity. 

This section provides insight into the optimization framework that underlies the multi-energy system model proposed in this work. The GBOML language introduced in \cite{Miftari2022}, a recently developed language dedicated to modeling graph-based optimization of multi-energy systems, is utilized to build this model. The optimization problem can be viewed as optimization on graphs, where a multi-energy system is considered as a set of nodes $\mathcal{N}$ that contribute to the (linear) objective and local constraints, and hyperedges $\mathcal{E}$ are used to model the constraints between nodes, such as those between RREHs and the EDC in our context.

The formalism employed in this work follows that introduced in \cite{Berger2021}. The entire system is defined by sets of nodes $\mathcal{N}$ and hyperedges $\mathcal{E}$. The optimization horizon is denoted by $T$, with time-steps indexed by $t \in \mathcal{T}$, where $\mathcal{T} = \{1, \ldots, T\}$.

A node $n \in \mathcal{N}$ is defined by internal $X^{n}$ and external $Z^{n}$ variables, where internal variables describe the specific characteristics of the unit, such as the nominal power capacity installed in the asset. Equality constraints $h_i(X^{n}, Z^{n}, t)=0$ with $i \in \mathcal{I}$ and inequality constraints $g_j(X^{n}, Z^{n}, t) \le 0$ with $j \in \mathcal{J}$, are employed for each $t \in \mathcal{T}$ to model operational constraints.

Each node $n$ has an associated cost function $F^{n}(X^{n}, Z^{n}) = \sum_{t=1}^{T} f^{n}(X^{n},Z^{n},t)$ that typically represents the capital expenditure and operational expenditure, i.e., CAPEX and OPEX, respectively.

Finally, equality and inequality constraints on hyperedges can be defined as $H^{e}(Z^{e}) = 0$ and $G^{e}(Z^{e}) \le 0$ with $e \in \mathcal{E}$ to model the laws of conservation and caps on given commodities.

    One can read this type of problem as:
     \begin{equation}\label{eq:prob_statement}
    \begin{aligned}
    \min \quad &  \sum_{n=1}^{N} F^{n}(X^{n}, Z^{n})\\
    \textrm{s.t.} \quad & h_i(X^{n}, Z^{n},t) = 0, \forall n \in \mathcal{N}, \forall t \in \mathcal{T}, \forall i \in \mathcal{I}\\
      \quad & g_j(X^{n},Z^{n},t) \le 0, \forall n \in \mathcal{N}, \forall t \in \mathcal{T}, \forall j \in \mathcal{J}\\
      \quad & H^{e}(Z^{e}) = 0, \forall e \in \mathcal{E}\\
      \quad & G^{e}(Z^{e}) \le 0, \forall e \in \mathcal{E}. \\
    \end{aligned}
    \end{equation}

%\subsection{Assumptions}\label{subsec:assumptions}

The main assumptions underlying our model are the following:
\begin{itemize}
    \item Centralised planning and operation: In this framework, a single entity is responsible for making all investment and operation decisions.
    \item Perfect forecast and knowledge: It is assumed that the demand curves, as well as weather time series, are available and known \textit{in advance} for the entire optimisation horizon, i.e., $\forall t \in \{ 1, \ldots , T \}$.
    \item Permanence of investment decisions: Investment decisions result in the sizing of installation capacities at the beginning of the time horizon. Capacities remain fixed throughout the entire optimisation period, i.e., $\forall t \in \{ 1, \ldots , T \}$.
    \item Linear modelling of technologies: All technologies and their interactions are modelled using linear equations within this framework.
    \item Spatial aggregation: The energy demands and generation at each node are represented by single points. The topology of the embedded network required to serve this demand locally is not modelled in this approach. This can be viewed as an extension of the copper plate modelling approach used in electrical power systems.
\end{itemize}

     In our problem, all cost functions and constraints are affine transformation of the inputs. More details on the constraints of each technology can be found in \cite{BERGER_power_to_gas_2020106039}, \cite{Berger2021}.
    Additionnaly, the local objective function corresponding to the CAPEX is modelled with a uniform weighted average cost of capital (WACC) of $7\%$ for each technology. Thus, the CAPEX is computed using the following formula:
    \begin{equation}
        \zeta^n = \mbox{CAPEX}_n \times \frac{\mbox{w}}{(1 - (1 + \mbox{w})^{-\mbox{L}_n})}
        \label{waccformula}
    \end{equation}
    with $L_n$ the lifetime of technology n and w the WACC. Hence, $\zeta^n$ represents the annualised cost of investing in technology n.

    Moreover, a cap on the net CO2 emissions (\textit{i.e.} release in minus captured from the atmosphere) is added to the model. This latter is defined as 
    \begin{equation}\label{eq:cap_co2}
        \sum_{t \in \mathcal{T}} ( \sum_{a \in \mathcal{A}} q_{co2, t}^{a} - \sum_{c \in \mathcal{C}} q_{co2, t}^{c} ) \le \kappa_{co2} \nu
    \end{equation}
    with $\mathcal{A}$ and $\mathcal{C}$ representing the sets of technologies that release CO2 into the atmosphere and those that capture CO2 directly from the atmosphere, respectively, $\kappa_{co2}$ represents the CO2 cap in kilotons per year, and $\nu$ represents the number of years covered by the optimization horizon. The shadow price, or marginal cost, which is the dual variable associated with \autoref{eq:cap_co2} allows for the derivation of a CO2 cost in €/t. A detailed explanation of dual variables as marginal costs in linear programming can be found in \cite[Chapter 4]{linearOptim}.

    
    % Since the cap on CO2 is set to zero, the cost associated with the emission of one more unit will be given by the optimal dual vector, which is the solution to the dual problem of \autoref{eq:prob_statement}. A detailed explanation of the dual problem can be found in \cite[Chapter 4]{linearOptim}. Each element of this dual vector can be interpreted as a shadow price associated with one constraint, i.e., a cost per unit increase of the right-hand side of the constraint \cite[Chapter 4, p.155]{linearOptim}. In the case where the constraint is a less-than-or-equal-to constraint ($\le$), the associated dual variable $p$ is negative. Therefore, emitting one more ton of CO2 will decrease the objective function by $p \times 1$ €. However, we should not emit this ton of CO2. Consequently, a cost of $p$ €/t (i.e., the shadow price) must be associated with the emission of any additional ton of CO2 in our system.
    
    % According to \cite[p.155]{linearOptim}, the optimal dual vector can be interpreted as the shadow price or marginal cost per unit increase of the right-hand side of the constraints. In our case, the dual variable $p$ associated with a constraint $\le$ is negative. Therefore, emitting one more unit of CO2 will decrease the objective function by the marginal price, i.e., $p$€, the dual variable value. In other words, since our CO2 cap is set at zero, emitting one more ton of CO2 will cost you $p \times 1$ (\textit{e.g.} looking at the units, €$/t \times t =$€). However, since we do not want to emit this ton of CO2, a cost of $p$ should be associated with this emission of one more unit.

    




