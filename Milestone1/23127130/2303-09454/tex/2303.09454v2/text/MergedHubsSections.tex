\section{CO2 Valorisation in a Multi-Remote Renewable Energy Hubs Approach}
\label{sec:Multi-RREH}

The Remote Renewable Energy Hub concept was first introduced in \cite{Berger2021} where the authors proposed a hub for synthesizing CH4 based on hydrogen and CO2 captured from the air thanks to a methanation unit. This concept has emerged within the context of global grid \cite{chatzivasileiadis2013global} and multi-energy systems approaches. These approaches aim at optimising the generation and utilisation of renewable energy (RE) by both (i) looking for abundant and cheap RE fields, (ii) taking advantage of daily/seasonal complementary of RE, as well as (iii) using power-to-gas technologies for better addressing RE generation fluctuations and meet e-fuels demand to act as a substitute for molecules derived nowadays from fossil fuels.


% \textcolor{blue}{While in the original article \cite{Berger2021} the methanation unit has access to CO2 by a Direct Air Capture Unit and the demand is satisfied by one unique RREH located in Algeria, in this paper, we propose to explore the possibility to valorize CO2 captured by Post Combustion Capture techniques at the energy demand center (EDC). Regarding the RREH, we also differ from the original paper by directly proposing a multi-RREH approach. This results in a framework where the EDC plays the role of a CO2 provider that can serve a set of multiple RREHs  $\{RREH_1, \ldots, RREH_h  \}$. Each hub $RREH_i ( 1 \leq  i  \leq h )$ comes with its own characteristics, such as renewable energy type, potential and distance to the EDC that may impact its competitiveness and the way it will be provided with CO2 by the EDC.}

In the original article \cite{Berger2021}, the methanation unit was supplied with CO2 by a Direct Air Capture unit, and the energy demand was fulfilled by a single RREH located in Algeria. However, in this paper, we propose to investigate the feasibility of valorizing CO2 captured through Post Combustion Capture techniques at the energy demand center (EDC). Additionally, we deviate from the original paper by introducing a multi-RREH approach, wherein the EDC serves as a CO2 provider to a set of multiple RREHs, denoted as ${RREH_1, \ldots, RREH_h }$. Each hub $RREH_i ( 1 \leq i \leq h )$ has its unique characteristics, such as renewable energy type, potential, distance from the EDC, and means of CO2 transport from the EDC, which can affect its competitiveness.



% We carry out a full analysis of this model on a realistic case study with Belgium as EDC and two RREHs: one located in Greenland and one in Algeria. In \autoref{fig:RREH_model}, you have access to a schematic view of the system fully described below in \autoref{subsec:model_config}.


% The system was modelled using GBOML, as was done in \cite{Berger2021}, and all model code has been made available online.

% Furthermore, this multi-RREH framework can be easily extended to include multiple EDCs.

In order to illustrate the concepts discussed above, we have developed a model for a multi-RREH system based on the following assumptions: (i) the EDC is Belgium, encompassing its gas and electricity demands as well as its CO2 emissions, (ii) there are two RREHs: one situated in the Sahara desert with access to solar and wind resources, and another in Greenland benefiting from the high-quality wind fields in the region. A detailed schematic of the resulting system is shown in \autoref{fig:RREH_model}. Similar to \cite{Berger2021}, we employed the GBOML language \cite{Miftari2022}, a recently developed language tailored for energy system optimization (refer to \autoref{sec:modelling} for more information), to model the system.

We note that the GBOML model code with two RREHs and one EDC system is available online\footnote{\url{https://gitlab.uliege.be/smart_grids/public/gboml/-/tree/master/examples}} and can  be easily extended to add additional RREHs and EDCs.

% In \cite{Berger2021}, the methanation unit has access to CO2 by a Direct Air Capture Unit and the demand is satisfied by one unique RREH located in Algeria. 

% In this paper, we further investigate this RREH concept by introducing CO2 capture by Post Combustion Carbon Capture where CO2 is abundant \textit{i.e.} in an energy demand center (EDC) and transport it, for example by boats, to the methanation unit located in the RREH. Synthetic methane is generated at the hub, and shipped back to the EDC. 

% Moreover, we extend the concept to multiple RREHs in order to set up a competitive framework between methane producers. Each hub comes with its own characteristics, such as renewable energy type, potential and distance to the EDC that may impact its competitiveness.

% \textcolor{red}{
% traditional approach to generate CH4 in remote renewable energy hub was based on direct air capture technologies. In this work, we propose to investigate the cost effectiveness of capturing CO2 in places where CO2 is abundant due to heavy industries by post combustion carbon capture technologies (PCCC). Those places are often linked with an important energy demand. Therefore, capturing CO2 in this places, then transporting it towards the RREH. In this RREH, producing CH4 and transporting it back to the energy demand center. In this paper, we 
% }
% \textcolor{red}{
% \begin{itemize}
%     \item Initially generating ch4 in RREH is based on DAC
%     \item Investigation wether capturing co2 in smokes would be interesting or at least complementary to DAC + Transport e.g. shipped
%     \item Case study: one energy demand center and 2 RREHS.
%     \item full modelling with gboml describe below as it was the case in \cite{Berger2021}.
%     \item Easy extension to multiple energy demand centers. 
%     \item All models required have been put online. 
% \end{itemize}
% }

% The traditional approach for generating ch4 in RREH is based on DAC see .... 
% In this paper we would like to investigate wehter capturing co2 in smod-kes would be interesting or at least complementary. 
% This CO2 will be for example shipped. 

% IN this paper, in the particular in one energy demand center and 2 RREHS: one in .. 

% we propose a full modelling with gboml describe below as it was the case in [3].
% this methology can be easily extended to multiple energy demand centers. 

% All models required have been put online. 





% In this paper, we further investigate this RREH concept by introducing Post Combustion Carbon Capture as well as CO2 transportation. CO2 is captured in Belgium, and transported to a RREH by carrier boats. Synthetic methane is generated at the hub, and shipped back to Belgium. A schematic is provided in \autoref{fig:RREH_model}. Examples of RREH include solar energy in the Sahara desert \cite{Berger2021}, as well as wind energy in Greenland \cite{Radu2019393}.

\begin{figure}[p]
    \vspace{-3ex}
    \centering   \includegraphics[width=\textwidth, height=\textheight,keepaspectratio]{Figures/20230313_GR-BE-DZ.pdf}
    \caption{A schematic illustration of the remote energy hub. CO2 being captured, it may be used to synthesize fuel either locally either in a remote energy hub where renewable energy may be cheaper and more abundant.}
    \label{fig:RREH_model}
\end{figure}



% The Multi Remote Renewable Energy Hub approach amounts in modelling several RREHs in order to set up a competitive framework between CO2 exploitation channels. Each hub comes with its own characteristics, such as renewable energy type, potential and distance to the demand energy centre that may impact its competitiveness.


% CHatGPT REFORMULATION:

% The Remote Renewable Energy Hub (RREH) concept was introduced by Berger et al. (2021) within the context of global grid and multi-energy systems approaches. These approaches aim to optimize renewable energy (RE) generation and utilization by identifying abundant and cost-effective RE fields, exploiting daily/seasonal complementarity, and employing power-to-gas technologies to address RE generation fluctuations and fossil fuels demand.

% n their study, Berger et al. (2021) proposed a unique RREH located in Algeria, where the methanation unit has access to CO2 via a Direct Air Capture Unit. In this paper, we further investigate the RREH concept by introducing Post Combustion Carbon Capture technology to capture CO2 from energy demand centers and transport it to the methanation unit located in the RREH, where synthetic methane is produced and shipped back to the energy demand centers. We also extend the concept to multiple RREHs, which allows for the comparison of the competitiveness of different CO2 exploitation channels based on their renewable energy potential and distance to the energy demand centers.

% To demonstrate the feasibility of our approach, we present a full analysis of a realistic case study involving Belgium as the energy demand center and two RREHs located in Greenland and Algeria. The system is modeled in GBOML, and the models are available online. The proposed framework is also easily extendable to multiple energy demand centers. The system's schematic view is provided in Figure \ref{fig:RREH_model}, and its complete description is presented in Section \ref{subsec:model_config}.