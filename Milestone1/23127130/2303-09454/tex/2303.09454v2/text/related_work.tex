% % % % % % % % % % % % % % % % % % % % % % % % % % % % % % % % % % % % 
\section{Related Work}
\label{sec:related_work}

This work is mainly related with the following topics that may play an important role in the deep decarbonation of our societies: (i) global grid  approaches, (ii) power-to-X technologies, multi-energy systems and and energy hub approaches, and (iii) CO2 quotas markets.

Global Grid (GG) approaches \cite{chatzivasileiadis2013global}, \cite{yu2019global}, sometimes referred to as Global Energy Interconnection approaches \cite{liu2015global},  are related with the idea of harvesting renewable energy from abundant and potentially remote renewable energy fields to feed the electricity demand in high demand centres. 
These approaches have mainly been oriented towards solutions using the electricity vector to repatriate energy from energy hubs, and have received a growing interest starting from the DESERTEC concept \cite{samus2013assessing} that focuses on Sahara solar energy resources from the Sahara desert to serve the European electricity demand. More recently, wind from Northern Europe and Greenland has also been identified as a promising resource to be valued within the GG context \cite{radu2022assessing}. Resource and demand configurations combining several types of resources as well as demand time zones show better results \cite{yu2019global}.

Multi-energy systems approaches \cite{munster2020sector,o2016energy} exploit the benefits of integrating energy demand and generation, as well as infrastructure. Power-to-X technologies, in particular power-to-CH4 technologies using hydrolysis and renewable energy for producing H2 \cite{GOTZ20161371}, offer a CO2 neutral solution to serve gas demand, but also a way to store vast quantities of energy issues from renewable sources \cite{BLANCO20181049}. Recently, Berger et al. have proposed a modeling framework \cite{Berger2021} for assessing the techno-economics viability of carbon-neutral synthetic fuel production from renewable electricity in remote areas where high-quality renewable resources are abundant. Let us mention that the idea of energy hubs was preexisting the work of Berger et al. \cite{mohammadi2017energy}, however the contribution of Berger et al. is the introduction of remote energy production, far from the demand. Our contribution is in line with the latter.


% Because it seeks to valorise CO2, this work is also related to the European Union Emissions Trading System (EU ETS). According to the European Commission website
% \footnote{\url{https://climate.ec.europa.eu/eu-action/eu-emissions-trading-system-eu-ets_en}} and \cite{carbonMarkets}, the EU ETS system is a 'cap and trade' system : a cap is set on the total amount of certain greenhouse gases that can be emitted by the installations covered by the ETS system. Within the cap, installations receive emissions allowances, which can be traded with one another. The limit on the total number of allowances available ensures that they have a value. The cap is reduced over time so that total emissions fall. If an installation does not fully cover its emissions, it suffers heavy fines. On the other hand, if an installation manages to reduce its emissions, it can either keep the spare allowance for covering future needs, either sell it to another installation which does not succeed in covering its own emissions. Such a trading mechanism should ensure that GHG emissions are avoided as soon as this is the cheapest solution, and also guide investments into low GHG emissions solutions.

As this work aims to enhance the value of CO2, it is closely linked to the European Union Emissions Trading System (EU ETS). The EU ETS system, which is described on the European Commission's website \footnote{\url{https://climate.ec.europa.eu/eu-action/eu-emissions-trading-system-eu-ets_en}} and in \cite{carbonMarkets}, is a 'cap and trade' program. The system sets a cap on the total amount of certain greenhouse gases (GHG) that can be emitted by the facilities covered by the ETS. Within the cap, facilities are given emissions allowances, which can be traded with one another. The total number of allowances available is limited to ensure that they have value, and the cap is gradually reduced over time to lower total emissions. If a facility fails to cover its emissions fully, it faces substantial fines. Conversely, if a facility reduces its emissions, it can either retain the surplus allowance for future use or sell it to another facility that has not succeeded in covering its own emissions. This trading mechanism aims to reduce GHG emissions as soon as it becomes the most cost-effective solution and encourage investments in low GHG emissions solutions.
