\section{Introduction}
\label{sec:intro}

While the whole world is engaged in a process to decrease greenhouse gas emissions, capturing CO2 appears more and more as a crucial element to limit global warming. Once it is captured, CO2 may be either stored (CCS - Carbon Capture and Storage), either valorized (CCU - Carbon Capture and Utilisation), for instance through synthetic methane generation. In this article, we focus on CCU, where CO2 is seen as a required ingredient in the process of generating synthetic methane, together with \textit{green} hydrogen, i.e. hydrogen obtained from renewable energy-based electrolysis.

In this paper, we build on top of the Remote Renewable Energy Hub (RREH) approach  \cite{Berger2021} to propose a multi-hub, multi CO2 sources approach. CO2 is captured using both Post-Combustion Carbon Capture (PCCC) and Direct Air Capture (DAC) technologies. Hydrogen is produced from electrolysis using renewable energy in a RREH which is particularly well-suited for producing cheap and abundant renewable energy (e.g., solar energy in the Sahara desert, or wind energy in Greenland). The RREH concept also relies on the following idea: some locations show large amount of energy consumption while not having lots of renewable energy resources (e.g., Europe). On the opposite, some places have abundant renewable energy while having almost no energy demand. In its original formulation, the RREH concept suggests to use DAC technologies to feed the CO2 demand at the RREH. In this paper, we include PCCC technologies as an alternative to DAC technologies: in addition or replacement to being captured in the atmosphere, CO2 emitted in energy intensive locations may be transported to the RREHs to be combined with green hydrogen for producing neutral synthetic methane.


We propose a methodology for assessing the technico-economic feasibility of exporting CO2 into RREH where synthetic CO2-neutral methane would be generated using locally produced green H2. We formalise an optimisation problem where CO2 sources are in "competition" to provide CO2 to the methanation units in the RREHs. This methodology is based on a linear program modelling of Belgium energy system, including gas and electricity demand, and main CO2 emitters. We rely on previously published approaches to develop our approach Berger et al. \cite{Berger2021}, and, in particular, we use the GBOML language Miftari et al. \cite{Miftari2022} to model the energy system and to optimize it.

Our methodology is evaluated in the Belgian context: we consider Belgian CO2 emissions and Belgian gas and electricity demand. CO2 may be captured using Post Combustion Carbon Capture (PCCC) in Belgium or DAC in RREH locations. CO2 neutral synthetic methane will be produced in a remote energy hub from where it would be shipped back to serve the Belgian gas demand. We derive a CO2 emission cost in order to have a neutral emission system. We also determine a value of lost load (\textit{i.e.} a price associated with a lack of energy service) in order to serve the energy demand at all times. Several scenarios are studied with different prices of CO2 emissions, allocation or not of unserved energy and forcing of a given RREH.
