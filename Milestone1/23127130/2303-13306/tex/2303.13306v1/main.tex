% ****** Start of file apssamp.tex ******
%
%   This file is part of the APS files in the REVTeX 4.2 distribution.
%   Version 4.2a of REVTeX, December 2014
%
%   Copyright (c) 2014 The American Physical Society.
%
%   See the REVTeX 4 README file for restrictions and more information.
%
% TeX'ing this file requires that you have AMS-LaTeX 2.0 installed
% as well as the rest of the prerequisites for REVTeX 4.2
%
% See the REVTeX 4 README file
% It also requires running BibTeX. The commands are as follows:
%
%  1)  latex apssamp.tex
%  2)  bibtex apssamp
%  3)  latex apssamp.tex
%  4)  latex apssamp.tex
%
\documentclass[%
 reprint,
%superscriptaddress,
%groupedaddress,
%unsortedaddress,
%runinaddress,
%frontmatterverbose, 
%preprint,
%preprintnumbers,
%nofootinbib,
%nobibnotes,
%bibnotes,
 amsmath,amssymb,
 aps,
 prl
%pra,
%prb,
%rmp,
%prstab,
%prstper,
%floatfix,
]{revtex4-2}


\usepackage{graphicx}% Include figure files
\usepackage{dcolumn}% Align table columns on decimal point
\usepackage{bm}% bold math
%\usepackage{hyperref}% add hypertext capabilities
%\usepackage[mathlines]{lineno}% Enable numbering of text and display math
%\linenumbers\relax % Commence numbering lines

%\usepackage[showframe,%Uncomment any one of the following lines to test 
%%scale=0.7, marginratio={1:1, 2:3}, ignoreall,% default settings
%%text={7in,10in},centering,
%%margin=1.5in,
%%total={6.5in,8.75in}, top=1.2in, left=0.9in, includefoot,
%%height=10in,a5paper,hmargin={3cm,0.8in},
%]{geometry}

\usepackage{mathtools} % for '\underbracket' macro

\usepackage{xcolor} % For colorful text.


% Some symbols we use
\def\ar{{ \mathrm{\textit{\textsf r}} }}
\def\kay{{ \mathrm{\textit{\textsf k}} }}
\def\dbu{{ \mathrm{\textit{\textsf w}} }}
\def\yu{{ \mathrm{\textit{\textsf u}} }}
\def\ess{{ \mathrm{\textit{\textsf s}} }}%\:\!
%\def\sspsi{{ \mathrm{  \scriptsize{\textit{\textsf X }}}  }}
\def\chy{{ \mathrm{  \large{\textit{\textsf x\,}}}  }}
\def\htch{{ \mathrm{\textit{\textsf h}} }}
\def\zed{{ \mathrm{\textit{\textsf z}} }}
\def\eff{{ \mathrm{\textit{\textsf f}} }}
\def\ex{{ \mathrm{\textit{\textsf x}} }}

\def\half{{\textstyle \frac{1}{2}}}

% Following https://felix11h.github.io/blog/greek-sansmath  and  https://tex.stackexchange.com/questions/195832/replace-several-letters-in-math-font/195867#195867
\DeclareSymbolFont{greekletters}{OML}{iwona}{m}{it}
\DeclareMathSymbol{\sspsi}{\mathord}{greekletters}{"20}


\usepackage{verbatim}
\makeatletter
\newcommand{\detailtexcount}[1]{%
\immediate\write18{texcount -merge -sum -q #1.tex output.aux > #1.wcdetail }%
\verbatiminput{#1.wcdetail}%
}
\makeatother

\begin{document}

\preprint{APS/123-QED}


\title{Lifting, loading, and buckling in conical shells}

\author{Daniel Duffy} % ORCID 0000-0002-0383-5527
 %\altaffiliation[Also at ]{Physics Department, XYZ University.}%Lines break automatically or can be forced with \\
\author{John S.~Biggins}% ORCID 0000-0002-7452-2421
 \email{jsb56@cam.ac.uk}
\affiliation{Department of Engineering, University of Cambridge, Trumpington St., Cambridge CB2 1PZ, UK}%

\author{Joselle M. McCracken} % 0000-0002-0411-3542
\author{Tayler S. Hebner} % ORCID 0000-0003-2723-3835
\author{Timothy J. White} % ORCID 0000-0001-8006-7173
\affiliation{
Department of Chemical and Biological Engineering, University of Colorado Boulder, 596 UCB, Boulder, Colorado, 80309, USA
}%

\date{\today}% It is always \today, today,
             %  but any date may be explicitly specified

\begin{abstract}
Liquid crystal elastomer films that morph into cones are strikingly capable lifters. Thus motivated, we combine theory, numerics, and experiments to reexamine the load-bearing capacity of conical shells. We show that a cone squashed between frictionless surfaces buckles at a smaller load, even in scaling, than the classical Seide/Koiter result. Such buckling begins in a region of greatly amplified azimuthal compression  generated in an outer boundary layer with oscillatory bend. Experimentally and numerically, buckling then grows sub-critically over the full cone. We derive a new thin-limit formula for the critical load, $\propto t^{5/2}$, and validate it numerically. We also investigate deep post-buckling, finding further instabilities producing intricate states with multiple Pogorelov-type curved ridges arranged in concentric-circles or Archimedean spirals. Finally, we investigate the forces exerted by such states, which limit
% limit, render, confer, provide, supply, afford, deliver, 
 lifting performance in active cones.
\end{abstract}

%\keywords{Suggested keywords}
%Use showkeys class option if keyword display desired

\maketitle

%\section{Main text}
%\tableofcontents


%\section{Main text}
Liquid crystal elastomers (LCEs) are muscular actuating solids that contract uniaxially along their director on heating \cite{kupfer1991nematic, warnerbook}. Flat LCE sheets containing concentric circle directors (+1~defects) correspondingly morph into conical shells \cite{modes2011gaussian, de2012engineering}, which can spectacularly lift thousands of times their own weight~\cite{guin2018layered} (Fig.~\ref{fig:intro}a). Continuing the trend of soft materials reinvigorating shell mechanics~\cite{holmesConfined, vaziriReisLocalization, hutchinsonReisGeometricRole, bertoldiReisBuckliball, zhangWrinkling, aharoniSmecticWrinkles, vellaFrustrating, calladineWithoutImperfections}, we investigate the buckling load of a conical shell as a fundamental limit on lifting performance.
%Load-bearing conical shells have become a subject of renewed interest, due to their excellent performance as lifters when constructed from active soft materials such as liquid crystal elastomers (LCEs)~\cite{warnerbook}. Patterned LCE films of small thickness \(t$ can morph from flat into cones upon heating, lifting loads thousands of times their own weight (see fig.~\ref{fig:intro}a), with many potential applications in soft machines~\cite{guin2018layered}. 

In thin sheets, the prohibitive energetic cost of stretch (\({\propto t}\)) relative to bend (\({\propto t^3}\)) strongly favours almost-isometric deformations. Accordingly, an LCE cone's strength is usually attributed to the tip's singular Gauss curvature \cite{modes2011gaussian}, which, via the \textit{Theorema Egregium} \cite{gauss1828disquisitiones, o2014elementary}, guarantees it cannot be flattened isometrically, naively suggesting that buckling requires a stretch-scale load \({f_* \propto t}\). However, if bend were cost-free, the cone could buckle under zero load via tip-inversion, which is isometric but requires a perfectly sharp ridge. 
Finite-threshold buckling therefore occurs via a short-wavelength mode where stretch and bend compete.  
Indeed, the classical result \cite{seide1956} predicts wavelengths $\propto \sqrt{t}$ and accordingly a stretch-bend load scaling ${f_* \propto t^2}$, similar to compressed cylinders~\cite{koiter1945} and pressurized spheres~\cite{zoelly1915, hutchinson2016buckling}.

\begin{figure}[t]
\centering
\includegraphics[width=\linewidth]{"fig1".pdf}
\caption{(a) A $2 \times 2$ array of heat-activated LCE cones supporting a load $1100 \times$ their own weight~\cite{guin2018layered}. (b) Compressing a conical shell between frictionless slides. (c)
Controlled-force compression of experimental (upper images) and numerical (lower images, force-compression plot) LCE cones (${\alpha \approx 60^\circ}$, ${R = 200 t}$, $\nu = 1/2$), both exhibiting sub-critical buckling. Using numerical and experimental (Sec.~S9A) modulus values, both buckling thresholds are $\sim 20\%$ of the classical Seide threshold ${\, \sim 3\, \mathrm{mN}}$. }
\label{fig:intro}
\end{figure}



Here we combine theory, numerics, and experiments to investigate the buckling and post-buckling of conical LCE shells. 
We find compressed cones deform predominantly in an outer boundary layer, which instigates buckling at much smaller loads than predicted classically.
% at sub-classical loads
Shells are usually frustratingly weaker than their theoretical idealizations, but this is usually attributed to acute imperfection sensitivity \cite{koiter1945, hutchinsonImperfections2018}. Some previous works have also highlighted boundary conditions \cite{steinInfluence, steinAdvances, HoffNachbar, HoffSoong, AlmrothInfluence, gormanEvan-Iwanowski, shenBook,  calladineWithoutImperfections, kobayashi_influence, tovstikBook}, but clarity on the key physics hasn't emerged. Cones provide a clear-cut, analytically tractable example where the boundary layer's influence is profound, even yielding a new thin-limit scaling: ${f_* \propto t^{5/2}}$. 

%We anticipate that boundary layers in other shell problems may yield similarly consequential scaling results.\\ 
%squashed cones exhibit a previously unscrutinized boundary effect that can be analyzed decisively to reveal its paramount importance\textcolor{red}{/salience?}. 
%We hope this springboards deeper understanding of such effects in shell mechanics moving forward.

%A great hindrance in structural mechanics is that loaded shells, although ubiquitous, are often far weaker than classical theory suggests. The usual diagnosis is sensitivity to the imperfections found in any real shell~\cite{koiter1945, hutchinsonImperfections2018, hutchinsonReisGeometricRole}. However, we observe in numerics that even perfect cones buckle at a force far below the classical result when squashed. The largest deformations, both at the moment of buckling and before, occur in a boundary layer near the outer rim. The classical analysis neglects this boundary layer entirely, hence the discrepancy. Other authors have discussed boundary layers in shells: leveraging them in the design of active shells\cite{holmesEfficientSnap}, ... santangelo+hayward halftone????... noting their likely importance for shape selection\cite{holmesCurvatureInducedPRL, holmesVellaBistability}, and investigating their influence on load-bearing capacity \cite{steinInfluence, steinAdvances, gormanEvan-Iwanowski, shenBook}. However, we feel their potential importance in the latter area is still not fully appreciated or understood, and a squashed cone provides a premier example. Using theory and numerics, we elucidate the mechanism by which a boundary layer instigates buckling, and find that it has a profound effect, yielding a new scaling law in the thin limit: \({f_* \propto t^{5/2}}\). We propose that such boundary effects will in general play a key role in the weakness afflicting many shells, and be of equal or greater importance than imperfections in some cases.

We begin our investigation by using surface alignment to fabricate $30\, \mu \mathrm{m}$ thick LCE sheets with circular director patterns (following \cite{mccracken_sharpen}, Sec.~S8A-B), which morph into cones with semi-angle $\approx 60^{\circ}$ on heating to 145$^{\circ}$C. Actuated cones were then squashed under a glass slide of controlled weight, and their deformations tracked with an optical profilometer (Sec.~S9B). Corresponding numerics were conducted using  MorphoShell~\cite{defective_nematogenesis} to minimize a nonlinear shell energy for a cone squashed between frictionless slides. Both experimental and numerical cones buckle sub-critically, popping into a nose-like shape at a load far below that classically expected (Fig.~\ref{fig:intro}c).

To investigate further, we consider a conical shell with semi-angle $\alpha$ and radius $R$, and denote arc-length along generators by $s$ and perpendicular distance to the cone axis by $r$. Compressing the shell by $\Delta h$ under a vertical force $f$ induces a membrane strain $\varepsilon$  and a bending strain $\beta = \kappa - \bar{\kappa}$,  $\kappa$ and $\bar \kappa$ being the deformed and undeformed curvature tensors. The deformed state will then minimize the standard energy $E = \int W \, \mathrm{d}A - f \Delta h $. For small strains (but large rotations) the appropriate energy density is~\cite{niordsonbook}
\begin{equation}
W(\varepsilon, \beta) = \frac{Y}{2(1-\nu^2)} \, Q(\varepsilon)   +  \frac{D}{2} \, Q(\beta) ,
\label{eq:energyDensity}
\end{equation}
where $Y = \mathtt{E} t $ and $D = \mathtt{E} t^3 / (12(1-\nu^2))$ are the stretching and bending moduli for Young's modulus $\mathtt{E} $ and Poisson ratio $\nu$, while $Q(\tau) \equiv \nu \,  \mathrm{tr}(\tau)^2 + {(1-\nu)}\,  \mathrm{tr}(\tau \cdot \tau)$.

\begin{figure*}[t]
\centering
\includegraphics[width=\linewidth]{"fig2".pdf}
\caption{(a)~Stress profiles shortly before buckling for the cone in Fig.~\ref{fig:intro}, with (beige) MorphoShell and (dashed) boundary-layer (\ref{eq:thetaEnergyDimless}) results agreeing well. The region of large compressive azimuthal stress near the boundary drives boundary-layer buckling at $f_\star \propto t^{5/2}$, but is absent from the membrane base state (dotted). (b)~Transitory MorphoShell topographies just after boundary buckling, labelled by $t/R$, colored by elevation, with smaller thicknesses yielding larger azimuthal mode numbers. (c)~Buckling mode shapes $\dbu(\ess)$ from numerical linear stability analyses (dashed, $\alpha = 60^\circ$) using our boundary-layer base state, converging to our thin-limit theory (solid) as $t/R$ decreases by factors of 10. (d)~Buckling force against $t/R$ for $\nu=1/2$ and various $\alpha$, with numerics converging to our thin-limit result (\ref{eq:thinlim_fstar}).}
\label{fig:2}
\end{figure*}

To probe the stability of an axisymmetric pre-buckled base state, $\varepsilon_0$, $\kappa_0$, we consider small additional displacements, $ \bm{u}$ tangentially and $w$ normally. Splitting the associated changes $\delta \varepsilon$ and $\delta \beta$ by order in displacement, retaining only terms important for short-wavelength perturbations gives~\cite{niordsonbook} \vspace{-0.2cm}
\begin{align}
\delta \varepsilon &=  \overbracket[0.140ex][0.3ex]{ \half \left(\bm{\nabla}  \bm{u} + \bm{\nabla}  \bm{u}^\mathrm{T} \right) + \kappa_0 \,  w }^{\delta \varepsilon_1} \, + \, \overbracket[0.140ex][0.3ex]{ \half \left( \bm{\nabla}  w \otimes \bm{\nabla}  w \right) }^{\delta \varepsilon_2} , \label{eq:strain} \notag \\
\delta \beta &=  - \mathrm{Hessian}\left(  w \right) = \delta \beta_1. 
%\label{eq:deltakappa}
\end{align}
% The \delta \beta does NOT include all linear terms. There is another term like w \kappa^2, and also terms linear in u. I did check however that all the terms neglected from both \delta \varepsilon and \delta \beta are scaling-smaller (given the u, w, and base state scalings we go on to find at least!). You could do the local membrane case with all terms included if you wanted to.
The $\kappa_0 w$ strain is characteristic of shells,  while the sole nonlinearity, $\delta \varepsilon_2$, allows inhomogeneous $w$ to relieve tangential compression: the basic mechanism of compressive buckling. Since the base state is an equilibrium, the leading energy change is quadratic in displacements:
%
% The way I derive this is to realise first that the stretching part of W can be written exactly as (1/2)tr[N(eps) . eps]. Then sub in eps = eps_0 + delta eps_1 + delta eps_2. Then, remembering that N is a linear func of its argument, we see that this equals W_stretch(eps_0) + (1/2) tr[N(delta eps_1) . delta eps_1] + (1/2) tr[N(eps_0) . delta eps_2]. The middle term can be written exactly as Y Q(delta eps_1) / (2(1-nu^2)), and thus we have the first and third terms in delta E below. To get the bending term, notice that the procedure for bend would be exactly the same except for a different overall prefactor, and the fact that delta beta_2 = 0 in our model. Thus you can write the bend term down almost immediately. Then we're done.
%
\begin{equation}
%\delta E = \int \left[ \frac{Y Q(\delta \varepsilon_1)}{2(1-\nu^2)} + \frac{D Q(\delta \beta_1)}{2} + \mathrm{tr}(N\left(\varepsilon_0\right) \cdot \delta \varepsilon_2)  \right] \mathrm{d}A,
\delta E = \begingroup\textstyle \int\endgroup \left[ W(\delta \varepsilon_1, \delta \beta_1)+ \mathrm{tr}(N\left(\varepsilon_0\right) \cdot \delta \varepsilon_2)  \right] \ \mathrm{d}A,
\label{eq:deltaEnergy}
\end{equation}
where $N(\varepsilon) = \partial W(\varepsilon, \beta) / \partial \varepsilon$ is the membrane stress.
%$ = \frac{Y}{1-\nu^2}\left(\nu \,  \mathrm{tr}(\varepsilon) I  + (1-\nu) \, \varepsilon \right)$
% The N is a Cauchy-type stress (see my Cauchy_and_Piola-Kirchoff_for_shells notes). Also note; ``membrane'' here encodes the fact that actually in-plane forces come from varying the energy under an infinitesimal tangential perturbation $\bm u$, and the bend energy actually also has changes that are O(u) which N does not account for. So there are actually in-plane bending forces, though they are usually small and neglected, hence the use of N. 
%The first term in this energy could be written $\frac{1}{2} \mathrm{tr}\left( N \left( \delta\varepsilon_1 \right) \cdot \delta \varepsilon_1 \right)$.
% The stability question is first and foremost `can we find a BC-permitted non-zero u, w that makes this energy = 0?'. The answer in initial loading is `no', and instability is the moment that the answer first becomes `yes'. However, if you draw a quadratic function of 2 variables with no linear terms for illustration, you see that this moment is the moment at which the initial-loading parabaloid as one of its principal curvatures drop to zero, such that it develops a flat-bottomed valley in one direction. A special property of such quadratics, evident from that geometric picture, is that at this moment not only do you have non-zero u,w that have quadratic energy \delta E = 0, those u,w are in fact also stationary points of the energy, because any point on the line defining the bottom of the valley is a stationary point. This is important because it means you should look at (in our notation) \delta \delta E and require it to be zero for all u,w. This `for all' condition then means you can make the usual kinds of arguments about the boundary terms, so you get E-L equations and BCs, whereas if you just look at \delta E = 0, you don't have this `for all' condition so can't make the same arguments I think. So the fact that we're looking at the *first* moment the quadratic parabaloid develops a flat valley is doing some real work I think. I think the same idea can easily be shown to apply for arbitrarily many degrees of freedom in the finite-dimensional case. I think as usual the continuum case is mathematically much harder under the hood but will all work out fine (i.e. the same); for one thing we're happy (FEM etc) that the continuum problem can be arbitrarily well-approximated by a discretized energy, so if that's true it really must work out the same I think! In finite dimensions I suppose reading off the eigenproblem from the quadratic form is only ok if the sandwiched operator is symmetric, so presumably self-adjointness or similar is key in the continuum case.

In general, (\ref{eq:deltaEnergy}) involves covariant derivatives and integration over the entire curved shell. However, following ref.~\cite{paulose2013buckling}, the anticipated short wavelength allows us to just consider a patch small compared to $r$ but large compared to wavelength. In this patch we may neglect \textit{all} covariant considerations and use Cartesian coordinates $(x,\, y)$, aligning $x$ with $s$. Minimizing $\delta E$ variationally over displacements yields the expected tangential and normal force-balance equations, linear in displacements:
\begin{align}
    \nabla \cdot N(\delta \varepsilon_1) &= 0 , \label{eq:divN=0} \\
    D \nabla^4 w + \mathrm{tr} \left( \kappa_0 \cdot N(\delta \varepsilon_1) \right) - \nabla \cdot{\left( N(\varepsilon_0) \cdot \nabla w\right)} &= 0 . \label{eq:normal_force_balance} 
\end{align}
We satisfy (\ref{eq:divN=0})  with a scalar Airy stress function $\psi$ such that $N(\delta \varepsilon_1) =  ( I \nabla^2 - \mathrm{Hessian})\psi \equiv \Lambda \psi$. Geometric compatibility of the strain then requires~\cite{paulose2013buckling}
\begin{equation}
\nabla^4 \psi - Y \, \mathrm{tr} \left( \Lambda  (\kappa_0 w) \right) = 0 .\label{eq:compatability}
\end{equation}
%To derive: first invert the stress-strain relation to get $Y \delta \varepsilon_1 = (1+\nu)N(\delta \varepsilon_1) - \nu \mathrm{tr}(N(\delta \varepsilon_1)) I$. To do so slickly, take the trace of the expression $N(\delta \varepsilon_1)$ to find $tr(\delta \varepsilon_1) = (1-\nu) tr(N)/Y$, and sub that back in to the $N(\delta \varepsilon_1)$ expression, then rearrange for the strain. Now, $\Lambda$ kills the $u$-parts of the strain, so applying $\Lambda$ to the strain and tracing gives one of the terms in (6) immediately. The other follows quickly from applying $\Lambda$ to the inverted stress-strain relation we found, given $div N = 0$.

Traditionally one considers a membrane base state, with $\kappa_0 = \cos(\alpha)/r \, \bm{\hat{y}}\otimes \bm{\hat{y}}$ matching the undeformed cone, and $N(\varepsilon_0) = -f / \left( 2\pi r \cos\alpha \right) \bm{\hat{x}}\otimes \bm{\hat{x}}$ following from vertical force balance. Both vary slowly over the cone, and so are effectively constant over the patch. We then search for oscillatory buckling solutions, substituting $ (w, \psi) = (a, b)\,   \mathrm{exp}({i (k_x x + k_y y)})$ into (\ref{eq:normal_force_balance}, \ref{eq:compatability}). As expected, the exponentials cancel, leaving algebraic equations that we solve for the ratio $a/b$ and buckling force $f$. Interestingly, force only depends on wavevector via $(k_x^2+k_y^2)^2 /k_x^2 \equiv k_{\circ}^2$ and, minimizing over $k_\circ$, we find buckling commences at the classical Seide $f_* = 4 \pi \sqrt{Y D} \cos^2\! \alpha \propto t^2$ \cite{seide1956}, with $k_{\circ}^{4} = Y / (D r^2) \cos^2\! \alpha$, corresponding to a `Koiter circle' of wavevectors with wavelengths $\propto \sqrt{r t}$, as is familiar from cylinders~\cite{koiter1945}. The threshold is radius independent, so buckling occurs over the entire cone simultaneously. 
% Koiter circle has been looked at for cones, though only really makes sense for local buckling which is not really understood by the author~\cite{spagnoliReinterpreting}.

When squashing between slides we instead observe buckling at $\sim20\%$ of Seide's value (Fig.~\ref{fig:intro}c). Moreover, the dominant pre-buckling deformations are localized within a  boundary layer of width $l\sim \sqrt{Rt}$ that is qualitatively different from the bulk's membrane state. Informatively, if the membrane state is artificially imposed at the boundaries, MorphoShell reproduces Seide's threshold. We thus focus on this axisymmetric boundary layer, described by the local angle $\vartheta(s)$ (Fig.~\ref{fig:2}a) and the radial and vertical displacements $\Delta r(s), \, \Delta z(s)$, and with effectively constant $r\approx R$. Subtly, the rim's radial freedom allows near-complete $s$-stress relaxation, $N_{ss} \sim l N_{\phi \phi}/ R$,  consistent with stress equilibrium $N_{\phi \phi}=R N'_{ss}$. Consequently, $s$-strain is a pure Poisson effect of hoop strain, $\varepsilon_{ss} = -\nu \Delta r / R$, and  the dominant balance between stretch and bend in (\ref{eq:energyDensity}) is simply
\begin{equation}
    W \approx \half  Y (\Delta r/R)^2+ \half D \vartheta'^2. \label{eq:energy:dim}
\end{equation}
Comparing these terms confirms the characteristic boundary-layer length scale, $l \equiv (R^2 D / Y )^{1/4} \sim \sqrt{R t}$. These terms and length scale also emerge from a conventional linear-elasticity treatment, but (\ref{eq:energy:dim}) also encompasses large rotations. Scaling all lengths by $l$ ($\ess \equiv s / l $, $\Delta r(s) \equiv l \Delta \ar (\ess)$ etc.) then leads to the dimensionless boundary-layer energy:
\begin{equation}
\frac{E}{\pi l \sqrt{Y D}} =  \int \! \left[ \Delta \ar^2 +  \theta'^{\,2} - \frac{f }{  \pi \sqrt{Y D}}  \Delta \zed'\right] \mathrm{d}\ess .
\label{eq:thetaEnergyDimless}
\end{equation}
The small-strain relations $\Delta \ar' = \sin \theta - \sin \alpha$ and $\Delta \zed' = \cos \alpha - \cos \theta$ allow us to write $E$ in terms of a  single variable, $\Delta \ar$. Minimization via standard variational calculus then yields a nonlinear shape equation and boundary conditions (Sec.~S1). Solving numerically using SciPy's solve\_bvp~\cite{scipy} reveals a universal scaling form $\Delta \ar(\ess)$ for the boundary layer in all thin cones of given semi-angle and dimensionless load $f/\sqrt{YD}$. The resultant shapes naturally include an outward flare, but oscillate into the cone, creating a band of inward displacement and large compressive hoop stress $N_{\phi \phi}=Y \Delta r/r$, which ultimately precipitates buckling. The smaller $N_{ss}$ follows from vertical force balance, and both shape and stresses  agree well with full MorphoShell simulations at realistic LCE thicknesses~ (Fig.~\ref{fig:2}a).

Eq.~(\ref{eq:thetaEnergyDimless}) clarifies that geometrically large rotations in the boundary layer require $f \sim \sqrt{Y D}\propto t^2 $, the Seide buckling scale.
%\textcolor{red}{Limit point vs bifurcation (=jump in controlled displacement experiment rather than only jump in controlled force experiment?)? John says not really; more to do with whether instability breaks a symmetry or not. Though it is an interesting feature that the lip instability gives a shape jump while the thin-limit one doesn't, though that's only going to be true for slide squashing which gives inequality constraints rather than more traditional controlled displacement}
% Note that a flare with Delta alpha = O(1) over an arbitrarily small flare length scale has arbitrarily large Gauss curvature even though strains are arbitrarily small everywhere! And this is actually what you get in the thin limit if axisymmetry is enforced.
In sufficiently thick simulated cones ($R \sim 10 t$), a pronounced flared shape indeed forms under such loads before violent buckling eventually occurs via tip inversion (though accurate tip mechanics are beyond shell theory, so in practice other $f \sim \sqrt{Y D}$ instabilities might occur instead, e.g.~Seide buckling or `sleeve-rolling' boundary inversion).

However, instability in thinner cones (Fig.~\ref{fig:intro}) is different, occurring at much smaller loads, and breaking azimuthal symmetry with a mode number $m$ that increases with $R/t$ (Fig.~\ref{fig:2}b), suggesting a competing instability with a higher thickness scaling.  We thus use the incremental energy (\ref{eq:deltaEnergy}) to investigate the linear stability of the boundary-layer base state to azimuthally varying perturbations $w= l \dbu(\ess) \cos{(m \phi)}$, and similarly for $\bm{u}$. Initially, we retain the full geometric nonlinearity of the boundary layer, requiring (\ref{eq:deltaEnergy}) to be understood covariantly on the base-state surface. Variational minimization then yields straightforward but cumbersome analogues of (\ref{eq:divN=0}, \ref{eq:normal_force_balance}), and associated boundary conditions at the rim (Sec.~S2), which we again solve using solve\_bvp. We indeed uncover an instability with  $\dbu(\ess)$ localized to the rim and maximal in the compressive region, but with a surprisingly long decay into the cone over $\sim 100 l$ (Fig~\ref{fig:2}c). The resulting $f_*$ values (minimized over $m$)  agree well with MorphoShell for realistic  thicknesses, and present a clear thin asymptote to $f_* \propto t^{5/2}$ (Fig.~\ref{fig:2}d) with a correspondingly increasing azimuthal mode number and convergent (scaled) mode shape.

These numerical investigations clarify that, in the thin limit, $\Delta \ar$, $\Delta \zed$, and $\Delta \theta \equiv \theta - \alpha$ are all asymptotically small at buckling, suggesting a much simpler geometrically linear treatment of the base state will suffice. Linearizing the small-strain relations and substituting $\Delta \ar$ for $\theta$ in (\ref{eq:energy:dim}), we see that the small-amplitude boundary layer is better characterised by an $\alpha$-dependent length scale $\ell \equiv l \sqrt{2 \sec \alpha}$. Using this modified length for non-dimensionalization gives $W\propto4 \Delta \ar^2 +  \Delta \ar''^2$, and hence, minimizing, the linear Euler-Lagrange equation $\Delta \ar'''' + 4 \Delta \ar = 0$, as for a plate on a foundation. The solutions are $\Delta \ar \propto e^{\eta \ess}$ for $\eta ^4 = -4$, whose oscillations produce the regions of hoop compression. Imposing the natural boundary conditions $\Delta \ar'' = 0$, $\pi \sqrt{Y D} \Delta \ar''' = - f \sin\alpha$ and discarding the growing solutions gives the thin-limit base state:
\begin{equation}
\Delta \ar = f e^\ess \cos(\ess) \sin(\alpha) 
/ \left( 2 \pi \sqrt{Y D}  \right).
\end{equation}
% This might look alarming, because you might think that there should be a radial force = 0 BC too that would say $\Delta \ar''' = 0$. But the point is that actually in this picture based on eqs 7 and 8, we've hard coded in the relationship between \Delta \ar and \Delta \zed, so the profile really only has 1 degree of freedom, which you can choose to be either \Delta \ar or \Delta \zed or some linear combo like displacement-in-normal-direction. But you shouldn't try and interpret the model or the BCs too physically in terms of forces etc; forces are not going to appear in an intuitive way when you've hard coded in a constraint which isn't really a constraint physically, rather something that would turn out to hold approximately if you solved the problem without it. For example, suppose in eq.8 we applied the force directly in the generator direction instead of vertically. Then a solution would just be \Delta \ar = 0, for any force. But there would be no way of seeing or calculating from that eq.8 picture the physical Nss stress required to maintain such a scenario. Thus the \Delta \ar equations/boundary conditions/solutions should be interpreted merely as a good way to find the *shape*. To think about physical forces you *have* to do something else, as we do to find Nss etc.
Reassuringly, this base state can be verified as the thin limit of a lengthy but routine linear elastic treatment (Sec.~S3). 

A further simplification arises because the base-state  displacements are small, and localized within $\sim \ell$ of the rim. We may thus consider a  patch at the rim, small compared to $r$ but large compared to $\ell$, and address stability with  an Airy stress and the Cartesian  equations (\ref{eq:normal_force_balance}, \ref{eq:compatability}). In the $(x,y)$ basis, the base state has
%Since all significant deformations are localized to within $\sim l$ of the boundary,
% I think the careful version here is this: To construct local coords throughout our buckling patch, we project from the tangent plane at the patch centre. The surface is described locally by a height function above that tangent plane: R(x,y) = (x, y, h(x,y))......
%we search for buckling with wavelength $\sim l$, performing a local stability analysis within a small patch at the boundary. For sufficiently \textcolor{red}{asymptotically??} small $f/ \sqrt{Y D}$, we can leverage the separation of scales $l \ll \text{patch size} \ll 1/\text{base state curvature}$ to again neglect all covariant complications within the patch and use our Cartesian $(x,y)$ basis, JOHN THINKS ACTUALLY THINKS PATCH MUCH SMALLER THAN 1/CURV MIGHT BE SUFFICIENT BUT NOT NECESSARY, AND THAT EVEN IF CURVATURE WAS LARGE, WE MIGHT BE OK AS LONG PATCH NOWHERE HAS LARGE NORMAL DISPLACEMENTS FROM THE TANGENT PLANE DEFINED AT THE PATCH CENTRE.
 $\kappa_0 = \mathrm{diag}(-\Delta \theta' / \ell,\, \cos\alpha / R)$ and $N(\varepsilon_0) = \mathrm{diag}(0,\, Y \ell \Delta \ar / R ) $ to leading order which, unlike the classical case, vary over a scale $\ell$ and hence are inhomogeneous even within the patch.  We find the equations take their simplest dimensionless form if we scale the force as ${ f \equiv 2 \pi D  \eff / (\ell \tan \alpha) }$, and the fields as
\begin{equation}
(w,\, \psi)=\left( \ell \dbu(\ess),\, Y \ell^3 / R \,  \cos(\alpha)  \sspsi(\ess)\right) \, \cos(\kay y / \ell),
\end{equation}
with $\ess=x/\ell $. This scaling retains many features of Seide buckling, including the scale of the wavevector, the natural stress scale, the relative sizes of $w$ and $\psi$, and the correspondingly small in-plane displacement $u \sim (\ell/R) w$. Remarkably, substituting into (\ref{eq:normal_force_balance}, \ref{eq:compatability}) gives equations that are not only dimensionless, but lack explicit dependence on $\alpha$ and $\nu$:
\begin{align}
&\dbu\,'''' - 2 \kay^2 \dbu\,'' + \kay^4 \dbu + 4 \sspsi''  = 2 \eff \, \kay^2 e^{\ess} (2 \sspsi \sin\ess - \dbu  \cos\ess ) , %\label{eq:dimlessLinstabNormal}
\notag \\
&\sspsi'''' - 2 \kay^2 \sspsi'' + \kay^4 \sspsi  - \dbu\,''   = -\eff \, \kay^2  \dbu \,  e^{\ess} \sin\ess.
\label{eq:dimlessLinstabCompat}
\end{align}

Since the patch extends to the rim, we also require boundary conditions: To leading order, given the anticipated scalings, $\dbu = 0$ (zero vertical displacement) and $\dbu\,'' = \sspsi = \sspsi' = 0$ (natural conditions from varying (\ref{eq:deltaEnergy})).
% What we're doing here is thinking about controlled displacement essentially, and feeling ok about the idea that one could do more (very) complex things to carefully consider the fact that `controlled displacement' slide squashing is really an inequality constraint because motion away from the slide is uninihibited. Force control is similarly hard to be careful about; what we've found is an instability that can occur in force control under which the boundary behaves as it does in displacement control (which is what we think probably happens). However, you could instead e.g. imagine the boundary of the cone is charged and is in a uniform field to apply controlled force; then there's no reason to assume vert disp = 0 on the boundary and one should use different BCs (which would allow e.g. some kind of frilly mode extending all the way to the boundary), though it may end up working out the same. Or if you wanted to think about the slide with a weight on, again the inequality constraint raises its head, because really the weight position is like the minimum z coord over a frilly boundary if axisymmetry is broken. If you assume axisymmetry then indeed I think it's simple and the BCs should not be vert disp = 0. But we're deliberately looking for non-axisymmetric modes because that's what we see, and ones where the outer rim remains in a plane (which we're moderately relaxed about in the same kind of way as the inequality constraint business), and those then require that vert disp = 0 at the outer boundary. In the bulk we can pick anything that results in decaying into the bulk (which we'd expect for essentially any choice).
We again solve (\ref{eq:dimlessLinstabCompat}) with solve\_bvp, sweeping through $\kay$ to find the mode that becomes unstable at the smallest $\eff$. The resultant mode shapes reproduce the thin limit of our previous approach (Fig.~\ref{fig:2}c), with first instability at $\eff_*=62.7\ldots$, $\kay_*= 0.252\ldots$ (Fig.~S4).  Restoring dimensions yields the thin-limit buckling threshold
\begin{equation}
    f_* = (278.4\ldots) (Y D^3 / R^2)^{1/4} \sqrt{\cos\alpha}\cot{\alpha} \propto t^{5/2},
\label{eq:thinlim_fstar}
\end{equation}
agreeing with Fig.~\ref{fig:2}d, with azimuthal mode number $m_*  =(0.178\ldots)(R^2 Y / D)^{1/4} \sqrt{\cos \alpha} \propto \sqrt{R/t}$. Thus, larger cones are in fact weaker for a given $t$.
Conversely, smaller cones are stronger, though this is ultimately limited by the alternative $f_* \propto t^2$ modes noted earlier.

Fig.~\ref{fig:2}d reveals that $f_*$ asymptotes remarkably slowly, perhaps due to the surprisingly long-ranged mode shape: a reminder to be cautious when exploiting `thinness' in shells. Although our experiments and MorphoShell simulations ($R \approx 100 t$) are pre-asymptotic, they nevertheless exhibit the same mechanical character: a region of compressive azimuthal boundary-layer stress initiates azimuthal buckling long before bulk instability. The cone's load-bearing capacity then drops drastically, and large post-buckling deformations immediately propagate deep into the bulk. The near-collapse of curves for different $\alpha$ in Fig.~\ref{fig:2}d furthermore shows that, surprisingly, the $\alpha$-dependence of our asymptotic result pertains even far from the asymptote. 
%We have *found* that there's no Koiter circle for our azimuthal boundary buckling; if there were we'd have found multiple $\kay$ values going unstable at exactly the same $\eff_*$. This is not very surprising, in that the Koiter circle is rather magic in the first place, exhibiting more symmetry than the actual base state! We think the key magic is that as you change viewing direction in the tangent plane the stress and `stiffness' both change by factors that exactly cancel in the resulting buckling load. Another big difference to the boundary buckling case is that the membrane base state is translationally symmetric over the local patch; this difference may also be part of the magic.

%Note:if you have radial clamp rather than radial freedom, base state scaling properties are different, hoop stress doesn't dominate over s stress, our 1D energies invalid etc.

A compelling feature of soft solids is that buckling need not precipitate failure, allowing creative use of the resultant morphing~\cite{reisReprogrammableMemory, vellaPassiveControl, holmesSnappingSurfaces, bertoldiOrigamiMetre, bertoldiUniversallyBistable, bertoldiJumper, bertoldiNavigating, reisBuckliphilia}. In this spirit, we now squash cones far beyond their initial instability, using displacement control to explore full hysteretic cycles (Fig.~\ref{fig:3}). We consider cones formed by  concentric-circle directors on both disks and squares, since both are used as LCE actuators~\cite{guin2018layered, Ware_Boothby}. Numerically, MorphoShell finds multiple successive instabilities, producing striking shapes:
%with many motifs also seen in experiment. 
disk-type cones yield increasing numbers of concentric circular ridges (Fig.~\ref{fig:3}a-b), evoking the exact isometries of a cone, with multiple sharp inversions, although blunted in the spirit of Pogorelov~\cite{PogorelovMonograph, VellaPogo, seffeninvertedcones}. 
%which can be thought of as successive inversions in the spirit of Pogorelov~\cite{PogorelovMonograph, VellaPogo, seffeninvertedcones}, reflecting the existence of an underlying multiply-inverted isometry. 
%For the (disk-type) example pictured in Fig.~\ref{fig:3}a-b, patterns of concentric circular ridges eventually form, which can be thought of as successive inversions in the spirit of Pogorelov~\cite{PogorelovMonograph, VellaPogo, seffeninvertedcones}, reflecting the existence of an underlying isometry. 
In squashing, subsequent ridges form via violent rim inversions, while in unsquashing they are annihilated centrally by tip inversion (Movies~M1-3). Simulated square-type cones tend to instead exhibit spiral ridges, which (un)wind continuously in (un)squashing (Fig.~S5, Movies M4-6). Experimentally, square-type samples were fabricated with a slightly different chemistry~\cite{mccracken_sharpen, hebner_intermolecular, mccraken_below_ambient}, and displacement control was implemented by using a digital caliper to control and measure the height of a slide (Sec.~S8C/9B). These cones switched between concentric-circle and spiral forms across successive load cycles (Fig.~\ref{fig:3}c), suggesting a delicate balance: a topic for future work. 


%Alternatively, spiral ridges may form instead, which (un)wind continuously in (un)squashing. We see the latter in the pictured square-type example, though both ridge forms are seen in experiment, and can even occur simulataneously within a sample (Fig.~\ref{fig:3}c). Issues of shape selection between the two forms are thus rather delicate.
%The shell returns to its undeformed conical shape.

\begin{figure}[h]
\centering
\includegraphics[trim={0cm 0.3cm 0.1cm 0cm},clip,width=\linewidth]{"fig3".pdf}
\caption{(a) Vertical force against height ratio (initial/squashed) for an LCE cone compressed quasistatically in MorphoShell ($\alpha = 60^\circ$, $R = 100 t$), with (un)squashing in (teal) purple. (b) Enlarged view of the small-compression region, where the unsquashing single-ridge force agrees well with our theory (dotted). (c) Height-colored experimental topographies of a square-type LCE cone, with two successive compressions of the same sample (left, middle) exhibiting both concentric-circle and spiral ridges, and (right) a spiral ridge in a similar simulated cone.}
\label{fig:3}
\end{figure}

% Remember that finding the ridge force from the dimensionful energy without any virtual work argument is rather subtle; see my Mathematica.
Unsquashing concentric ridges culminates in a single circular ridge moving towards the tip (Fig.~\ref{fig:3}b), yielding the cone's weakest states. The form and strength of such a low-force ridge can be understood by formulating an axisymmetric shape equation, as used to describe the boundary layer. Indeed, a sharp circular ridge would be an isometry, with zero stretch but divergent bend, leading instead to transversely blunted ridges, where stretch and bend compete
~\cite{seffeninvertedcones}. The dominant energies are exactly those in (\ref{eq:thetaEnergyDimless}), with blunting over the same scale $l$, except with ridge radius $\rho$ replacing $R$. The right-hand side of (\ref{eq:thetaEnergyDimless}) is dimensionless so, substituting $l$, the energy of a ridge must be $E = 2 g(\alpha) (Y D^3 \rho^2)^{1/4}$, where $g(\alpha)$ is a dimensionless geometric factor. Recognizing that the height of a ridge is $h=(R-\rho)\cot{\alpha}$, virtual work shows that the ridge exerts a force $f = -\partial E/\partial h= g(\alpha) \tan(\alpha) (Y D^3 / \rho^2)^{1/4} \sim t^2 \sqrt{t/\rho}$. Interestingly, this force \textit{decreases} with $\rho$, opposite to the spherical case~\cite{PogorelovMonograph}.

We calculate $g(\alpha)$ for a ridge held between two distant circular clamps by again minimizing (\ref{eq:thetaEnergyDimless}) numerically, confirming that ridges in steeper cones cost more energy (Fig.~S6). The resultant force agrees well with MorphoShell (Fig.~\ref{fig:3}b). Interestingly this thickness scaling matches our asymptotic $f_*$, suggesting that, in scaling terms, boundary-layer buckling weakens the cone to the greatest possible extent. At sufficiently large radius a Pogorelov ridge in a sphere buckles into a polygon, under compressive azimuthal stress generated by the blunting deformations~\cite{KnocheSecondary, VellaPogo, vaziriMaha, vaziriReisLocalization}: a cousin of boundary buckling. Cone ridges can also exhibit polygonality (Fig.~\ref{fig:intro}c), but the presence of multiple ridges appears stabilizing, hence the circular ridges we observe at the deepest compressions. 

A motivating question for  LCE cones is what load  they can lift, rather than merely support. Actuation starts with mild, weak cones, which will buckle into concentrically ridged states, and must unbuckle to lift. 
We therefore simulate a cone that activates from flat  under a slide of fixed weight (initially supported by a small spacer so actuation commences), and explore how far it lifts. During unsquashing, ridges remain in contact with the slides, so the shell's cross-section is zigzag-like (Fig.~\ref{fig:4}a).  Concentric ridges are thus equispaced, and spirals approximately Archimedean (Fig.~\ref{fig:4}b-c). Since each ridge's force decreases with radius, the lowest-force state with $N$ ridges has all ridges at their largest possible radii. If the load exceeds this state's lifting force, the cone is stuck; otherwise it can  lift all the way to the next such state. 
%A many-ridged shell can thus lift a slide of given weight until it reaches such a state whose force is less than the weight. 
%For example, we expect a return to an un-ridged cone only when the slide weight is below the \textit{weakest} force exerted by a single ridge. 
We thus predict a staircase of lifting heights as a function of weight. Assigning the single-ridge energy to each concentric ridge, a second virtual work argument (Sec.~S7) yields a prediction for this ladder, in good agreement with MorphoShell for small $N$ (Fig.~\ref{fig:4}d). 

We conclude that only loads $\propto t^{5/2}$ can be lifted by LCE cones to large heights. This force is frustratingly close in scaling to the weak $\propto t^3$ forces offered by pure benders, and matches the scaling of boundary buckling. However, the unusual quantized height-load relationship allows  for realization of discrete actuation strokes, which may be highly desirable in robust soft mechanisms or soft computation. At larger weights we observe increasing deviations from our staircase, with each ridge bearing load $\sim \sqrt{Y D}$. This is unsurprising given the zigzag model must  fail when  ridge spacing and blunting become commensurate, but is good news for lifters: deeply ridged states can exert large forces, with potential for powerful small-stroke soft actuators.
%in which ridges were assumed to be lightly loaded and interactions between them were neglected.
%ridge energy was calculated without a flat slide pressing directly on the ridge; we took the cross-section geometry to be exactly isometric in calculating virtual work; and interactions between ridges were neglected. \textcolor{red}{ Note the plateaus in the unloading curve in fig 3 have the 'wrong' sign of slope for larger numbers of ridges; clear evidence of the zig-zag model becoming quickly inadequate}. 
% I think ridges rammed up against the slide can exert $\sim \mu t^2$ forces (fig 4), by changing their shape away from our simple model (contact mechanics at slide surface etc).

Overall, our study reveals that, for frictionless boundary conditions, 
%the strength of cones is limited by 
boundary-layer buckling limits the strength of cones, leading to a load-bearing capacity $\propto t^{5/2}$. This represents a significant knockdown from the classical result, \textit{without} imperfections, and may contribute broadly to the practical weakness of shells. Active lifting to large heights via ridged states exhibits the same scaling. Future investigations will focus on using  boundary conditions, geometry, and material parameters to optimize lifting performance and extract something closer to the full $\propto t$ energy budget of an actuating sheet.



\begin{figure}[h]
\centering
\includegraphics[trim={0cm 0cm 0cm 0cm},clip,width=0.99\linewidth]{"fig4".pdf}
\caption{(a) Profile sketch of a concentric-ridged lifter. (b) MorphoShell confirms that concentric ridges are roughly equispaced while (c) spiral ridges are Archimedean. (d) Final height of a concentric-ridged lifter ($\alpha \approx 60^\circ$, $R \approx 100 t$) against weight lifted from a many-ridged state. Our simple theory (line) agrees well with MorphoShell (markers) for small weights, but significantly underestimates lifting performance at large weights.
% The deviations in height for very small N are ~t so we're really not concerned about them. In our analytic formula, the sum scales as N^{3/2} for large N, and h \approx h0/N for large N, so f \propto N^2 for large N. Thus indeed as N increases, the formula predicts that the force per ridge grows something like \propto N, and thus each ridge ends up bearing a larger and larger force. Thus, high loads of many multiples of \sqrt{Y D} are supported NOT just by making loads of ridges each bearing a small force \propto t^{5/2}; rather more ridges are formed as load increases, but the load-per-ridge increases also. Thus our toy model certainly ceases to apply, and looking at the t-scaling of the formula, we expect this to happen when N \sim \sqrt{h/t} such that each ridge bears a force ~\sqrt{Y D}. When instead N^2 \sim \sqrt{h/t}, the force-per-ridge is still small but overall a load \sim \sqrt{Y D} can be supported. For our numerics \sqrt{h/t}\sim 2 or 3 or so, so it makes sense that we see even 2 or 3 ridges supporting multiples of \sqrt{Y D}.
}
\label{fig:4}
\end{figure}

\begin{acknowledgments}
J.S.B.~was supported by a UKRI ‘future leaders fellowship’ grant (grant no.~MR/S017186/1). D.D.~was supported by the EPSRC Centre for Doctoral Training in Computational Methods for Materials Science (grant no.~EP/L015552/1). T.S.H.~and T.J.W.~acknowledge support from a Graduate Research Fellowship and DMR 2105369 from the National Science Foundation.
\end{acknowledgments}

%\bibliography{references.bib}% Produces the bibliography via BibTeX.

%apsrev4-2.bst 2019-01-14 (MD) hand-edited version of apsrev4-1.bst
%Control: key (0)
%Control: author (8) initials jnrlst
%Control: editor formatted (1) identically to author
%Control: production of article title (0) allowed
%Control: page (0) single
%Control: year (1) truncated
%Control: production of eprint (0) enabled
\begin{thebibliography}{50}%
\makeatletter
\providecommand \@ifxundefined [1]{%
 \@ifx{#1\undefined}
}%
\providecommand \@ifnum [1]{%
 \ifnum #1\expandafter \@firstoftwo
 \else \expandafter \@secondoftwo
 \fi
}%
\providecommand \@ifx [1]{%
 \ifx #1\expandafter \@firstoftwo
 \else \expandafter \@secondoftwo
 \fi
}%
\providecommand \natexlab [1]{#1}%
\providecommand \enquote  [1]{``#1''}%
\providecommand \bibnamefont  [1]{#1}%
\providecommand \bibfnamefont [1]{#1}%
\providecommand \citenamefont [1]{#1}%
\providecommand \href@noop [0]{\@secondoftwo}%
\providecommand \href [0]{\begingroup \@sanitize@url \@href}%
\providecommand \@href[1]{\@@startlink{#1}\@@href}%
\providecommand \@@href[1]{\endgroup#1\@@endlink}%
\providecommand \@sanitize@url [0]{\catcode `\\12\catcode `\$12\catcode
  `\&12\catcode `\#12\catcode `\^12\catcode `\_12\catcode `\%12\relax}%
\providecommand \@@startlink[1]{}%
\providecommand \@@endlink[0]{}%
\providecommand \url  [0]{\begingroup\@sanitize@url \@url }%
\providecommand \@url [1]{\endgroup\@href {#1}{\urlprefix }}%
\providecommand \urlprefix  [0]{URL }%
\providecommand \Eprint [0]{\href }%
\providecommand \doibase [0]{https://doi.org/}%
\providecommand \selectlanguage [0]{\@gobble}%
\providecommand \bibinfo  [0]{\@secondoftwo}%
\providecommand \bibfield  [0]{\@secondoftwo}%
\providecommand \translation [1]{[#1]}%
\providecommand \BibitemOpen [0]{}%
\providecommand \bibitemStop [0]{}%
\providecommand \bibitemNoStop [0]{.\EOS\space}%
\providecommand \EOS [0]{\spacefactor3000\relax}%
\providecommand \BibitemShut  [1]{\csname bibitem#1\endcsname}%
\let\auto@bib@innerbib\@empty
%</preamble>
\bibitem [{\citenamefont {K{\"u}pfer}\ and\ \citenamefont
  {Finkelmann}(1991)}]{kupfer1991nematic}%
  \BibitemOpen
  \bibfield  {author} {\bibinfo {author} {\bibfnamefont {J.}~\bibnamefont
  {K{\"u}pfer}}\ and\ \bibinfo {author} {\bibfnamefont {H.}~\bibnamefont
  {Finkelmann}},\ }\bibfield  {title} {\bibinfo {title} {Nematic liquid single
  crystal elastomers},\ }\href@noop {} {\bibfield  {journal} {\bibinfo
  {journal} {Die Makromolekulare Chemie, Rapid Communications}\ }\textbf
  {\bibinfo {volume} {12}},\ \bibinfo {pages} {717} (\bibinfo {year}
  {1991})}\BibitemShut {NoStop}%
\bibitem [{\citenamefont {Warner}\ and\ \citenamefont
  {Terentjev}(2007)}]{warnerbook}%
  \BibitemOpen
  \bibfield  {author} {\bibinfo {author} {\bibfnamefont {M.}~\bibnamefont
  {Warner}}\ and\ \bibinfo {author} {\bibfnamefont {E.~M.}\ \bibnamefont
  {Terentjev}},\ }\href@noop {} {\emph {\bibinfo {title} {Liquid crystal
  elastomers}}},\ Vol.\ \bibinfo {volume} {120}\ (\bibinfo  {publisher} {Oxford
  University Press},\ \bibinfo {year} {2007})\BibitemShut {NoStop}%
\bibitem [{\citenamefont {Modes}\ \emph {et~al.}(2011)\citenamefont {Modes},
  \citenamefont {Bhattacharya},\ and\ \citenamefont
  {Warner}}]{modes2011gaussian}%
  \BibitemOpen
  \bibfield  {author} {\bibinfo {author} {\bibfnamefont {C.~D.}\ \bibnamefont
  {Modes}}, \bibinfo {author} {\bibfnamefont {K.}~\bibnamefont
  {Bhattacharya}},\ and\ \bibinfo {author} {\bibfnamefont {M.}~\bibnamefont
  {Warner}},\ }\bibfield  {title} {\bibinfo {title} {Gaussian curvature from
  flat elastica sheets},\ }\href@noop {} {\bibfield  {journal} {\bibinfo
  {journal} {Proceedings of the Royal Society A: Mathematical, Physical and
  Engineering Sciences}\ }\textbf {\bibinfo {volume} {467}},\ \bibinfo {pages}
  {1121} (\bibinfo {year} {2011})}\BibitemShut {NoStop}%
\bibitem [{\citenamefont {de~Haan}\ \emph {et~al.}(2012)\citenamefont
  {de~Haan}, \citenamefont {S{\'a}nchez-Somolinos}, \citenamefont
  {Bastiaansen}, \citenamefont {Schenning},\ and\ \citenamefont
  {Broer}}]{de2012engineering}%
  \BibitemOpen
  \bibfield  {author} {\bibinfo {author} {\bibfnamefont {L.~T.}\ \bibnamefont
  {de~Haan}}, \bibinfo {author} {\bibfnamefont {C.}~\bibnamefont
  {S{\'a}nchez-Somolinos}}, \bibinfo {author} {\bibfnamefont {C.~M.}\
  \bibnamefont {Bastiaansen}}, \bibinfo {author} {\bibfnamefont {A.~P.}\
  \bibnamefont {Schenning}},\ and\ \bibinfo {author} {\bibfnamefont {D.~J.}\
  \bibnamefont {Broer}},\ }\bibfield  {title} {\bibinfo {title} {Engineering of
  complex order and the macroscopic deformation of liquid crystal polymer
  networks},\ }\href@noop {} {\bibfield  {journal} {\bibinfo  {journal}
  {Angewandte Chemie International Edition}\ }\textbf {\bibinfo {volume}
  {51}},\ \bibinfo {pages} {12469} (\bibinfo {year} {2012})}\BibitemShut
  {NoStop}%
\bibitem [{\citenamefont {Guin}\ \emph {et~al.}(2018)\citenamefont {Guin},
  \citenamefont {Settle}, \citenamefont {Kowalski}, \citenamefont {Auguste},
  \citenamefont {Beblo}, \citenamefont {Reich},\ and\ \citenamefont
  {White}}]{guin2018layered}%
  \BibitemOpen
  \bibfield  {author} {\bibinfo {author} {\bibfnamefont {T.}~\bibnamefont
  {Guin}}, \bibinfo {author} {\bibfnamefont {M.~J.}\ \bibnamefont {Settle}},
  \bibinfo {author} {\bibfnamefont {B.~A.}\ \bibnamefont {Kowalski}}, \bibinfo
  {author} {\bibfnamefont {A.~D.}\ \bibnamefont {Auguste}}, \bibinfo {author}
  {\bibfnamefont {R.~V.}\ \bibnamefont {Beblo}}, \bibinfo {author}
  {\bibfnamefont {G.~W.}\ \bibnamefont {Reich}},\ and\ \bibinfo {author}
  {\bibfnamefont {T.~J.}\ \bibnamefont {White}},\ }\bibfield  {title} {\bibinfo
  {title} {Layered liquid crystal elastomer actuators},\ }\href
  {https://doi.org/10.1038/s41467-018-04911-4} {\bibfield  {journal} {\bibinfo
  {journal} {Nature communications}\ }\textbf {\bibinfo {volume} {9}},\
  \bibinfo {pages} {2531} (\bibinfo {year} {2018})},\ \bibinfo {note}
  {http://creativecommons.org/licenses/by/4.0/}\BibitemShut {NoStop}%
\bibitem [{\citenamefont {Stein-Montalvo}\ \emph {et~al.}(2019)\citenamefont
  {Stein-Montalvo}, \citenamefont {Costa}, \citenamefont {Pezzulla},\ and\
  \citenamefont {Holmes}}]{holmesConfined}%
  \BibitemOpen
  \bibfield  {author} {\bibinfo {author} {\bibfnamefont {L.}~\bibnamefont
  {Stein-Montalvo}}, \bibinfo {author} {\bibfnamefont {P.}~\bibnamefont
  {Costa}}, \bibinfo {author} {\bibfnamefont {M.}~\bibnamefont {Pezzulla}},\
  and\ \bibinfo {author} {\bibfnamefont {D.~P.}\ \bibnamefont {Holmes}},\
  }\bibfield  {title} {\bibinfo {title} {Buckling of geometrically confined
  shells},\ }\href {https://doi.org/10.1039/C8SM02035C} {\bibfield  {journal}
  {\bibinfo  {journal} {Soft Matter}\ }\textbf {\bibinfo {volume} {15}},\
  \bibinfo {pages} {1215} (\bibinfo {year} {2019})}\BibitemShut {NoStop}%
\bibitem [{\citenamefont {Nasto}\ \emph {et~al.}(2013)\citenamefont {Nasto},
  \citenamefont {Ajdari}, \citenamefont {Lazarus}, \citenamefont {Vaziri},\
  and\ \citenamefont {Reis}}]{vaziriReisLocalization}%
  \BibitemOpen
  \bibfield  {author} {\bibinfo {author} {\bibfnamefont {A.}~\bibnamefont
  {Nasto}}, \bibinfo {author} {\bibfnamefont {A.}~\bibnamefont {Ajdari}},
  \bibinfo {author} {\bibfnamefont {A.}~\bibnamefont {Lazarus}}, \bibinfo
  {author} {\bibfnamefont {A.}~\bibnamefont {Vaziri}},\ and\ \bibinfo {author}
  {\bibfnamefont {P.~M.}\ \bibnamefont {Reis}},\ }\bibfield  {title} {\bibinfo
  {title} {Localization of deformation in thin shells under indentation},\
  }\href {https://doi.org/10.1039/C3SM50279A} {\bibfield  {journal} {\bibinfo
  {journal} {Soft Matter}\ }\textbf {\bibinfo {volume} {9}},\ \bibinfo {pages}
  {6796} (\bibinfo {year} {2013})}\BibitemShut {NoStop}%
\bibitem [{\citenamefont {Lee}\ \emph {et~al.}(2016)\citenamefont {Lee},
  \citenamefont {López~Jiménez}, \citenamefont {Marthelot}, \citenamefont
  {Hutchinson},\ and\ \citenamefont {Reis}}]{hutchinsonReisGeometricRole}%
  \BibitemOpen
  \bibfield  {author} {\bibinfo {author} {\bibfnamefont {A.}~\bibnamefont
  {Lee}}, \bibinfo {author} {\bibfnamefont {F.}~\bibnamefont
  {López~Jiménez}}, \bibinfo {author} {\bibfnamefont {J.}~\bibnamefont
  {Marthelot}}, \bibinfo {author} {\bibfnamefont {J.~W.}\ \bibnamefont
  {Hutchinson}},\ and\ \bibinfo {author} {\bibfnamefont {P.~M.}\ \bibnamefont
  {Reis}},\ }\bibfield  {title} {\bibinfo {title} {{The Geometric Role of
  Precisely Engineered Imperfections on the Critical Buckling Load of Spherical
  Elastic Shells}},\ }\bibfield  {journal} {\bibinfo  {journal} {Journal of
  Applied Mechanics}\ }\textbf {\bibinfo {volume} {83}},\ \href
  {https://doi.org/10.1115/1.4034431} {10.1115/1.4034431} (\bibinfo {year}
  {2016}),\ \bibinfo {note} {111005}\BibitemShut {NoStop}%
\bibitem [{\citenamefont {Shim}\ \emph {et~al.}(2012)\citenamefont {Shim},
  \citenamefont {Perdigou}, \citenamefont {Chen}, \citenamefont {Bertoldi},\
  and\ \citenamefont {Reis}}]{bertoldiReisBuckliball}%
  \BibitemOpen
  \bibfield  {author} {\bibinfo {author} {\bibfnamefont {J.}~\bibnamefont
  {Shim}}, \bibinfo {author} {\bibfnamefont {C.}~\bibnamefont {Perdigou}},
  \bibinfo {author} {\bibfnamefont {E.~R.}\ \bibnamefont {Chen}}, \bibinfo
  {author} {\bibfnamefont {K.}~\bibnamefont {Bertoldi}},\ and\ \bibinfo
  {author} {\bibfnamefont {P.~M.}\ \bibnamefont {Reis}},\ }\bibfield  {title}
  {\bibinfo {title} {Buckling-induced encapsulation of structured elastic
  shells under pressure},\ }\href {https://doi.org/10.1073/pnas.1115674109}
  {\bibfield  {journal} {\bibinfo  {journal} {Proceedings of the National
  Academy of Sciences}\ }\textbf {\bibinfo {volume} {109}},\ \bibinfo {pages}
  {5978} (\bibinfo {year} {2012})}\BibitemShut {NoStop}%
\bibitem [{\citenamefont {Zhang}\ \emph {et~al.}(2018)\citenamefont {Zhang},
  \citenamefont {Hao}, \citenamefont {Li}, \citenamefont {Feng},\ and\
  \citenamefont {Gao}}]{zhangWrinkling}%
  \BibitemOpen
  \bibfield  {author} {\bibinfo {author} {\bibfnamefont {C.}~\bibnamefont
  {Zhang}}, \bibinfo {author} {\bibfnamefont {Y.-K.}\ \bibnamefont {Hao}},
  \bibinfo {author} {\bibfnamefont {B.}~\bibnamefont {Li}}, \bibinfo {author}
  {\bibfnamefont {X.-Q.}\ \bibnamefont {Feng}},\ and\ \bibinfo {author}
  {\bibfnamefont {H.}~\bibnamefont {Gao}},\ }\bibfield  {title} {\bibinfo
  {title} {Wrinkling patterns in soft shells},\ }\href
  {https://doi.org/10.1039/C7SM02261A} {\bibfield  {journal} {\bibinfo
  {journal} {Soft Matter}\ }\textbf {\bibinfo {volume} {14}},\ \bibinfo {pages}
  {1681} (\bibinfo {year} {2018})}\BibitemShut {NoStop}%
\bibitem [{\citenamefont {Aharoni}\ \emph {et~al.}(2017)\citenamefont
  {Aharoni}, \citenamefont {Todorova}, \citenamefont {Albarr{\'a}n},
  \citenamefont {Goehring}, \citenamefont {Kamien},\ and\ \citenamefont
  {Katifori}}]{aharoniSmecticWrinkles}%
  \BibitemOpen
  \bibfield  {author} {\bibinfo {author} {\bibfnamefont {H.}~\bibnamefont
  {Aharoni}}, \bibinfo {author} {\bibfnamefont {D.~V.}\ \bibnamefont
  {Todorova}}, \bibinfo {author} {\bibfnamefont {O.}~\bibnamefont
  {Albarr{\'a}n}}, \bibinfo {author} {\bibfnamefont {L.}~\bibnamefont
  {Goehring}}, \bibinfo {author} {\bibfnamefont {R.~D.}\ \bibnamefont
  {Kamien}},\ and\ \bibinfo {author} {\bibfnamefont {E.}~\bibnamefont
  {Katifori}},\ }\bibfield  {title} {\bibinfo {title} {The smectic order of
  wrinkles},\ }\href {https://doi.org/10.1038/ncomms15809} {\bibfield
  {journal} {\bibinfo  {journal} {Nature Communications}\ }\textbf {\bibinfo
  {volume} {8}},\ \bibinfo {pages} {15809} (\bibinfo {year}
  {2017})}\BibitemShut {NoStop}%
\bibitem [{\citenamefont {Liu}\ \emph {et~al.}(2021)\citenamefont {Liu},
  \citenamefont {Domino}, \citenamefont {de~Dinechin}, \citenamefont
  {Taffetani},\ and\ \citenamefont {Vella}}]{vellaFrustrating}%
  \BibitemOpen
  \bibfield  {author} {\bibinfo {author} {\bibfnamefont {M.}~\bibnamefont
  {Liu}}, \bibinfo {author} {\bibfnamefont {L.}~\bibnamefont {Domino}},
  \bibinfo {author} {\bibfnamefont {I.~D.}\ \bibnamefont {de~Dinechin}},
  \bibinfo {author} {\bibfnamefont {M.}~\bibnamefont {Taffetani}},\ and\
  \bibinfo {author} {\bibfnamefont {D.}~\bibnamefont {Vella}},\ }\href
  {https://doi.org/10.48550/ARXIV.2108.06499} {\bibinfo {title} {Frustrating
  frustration in snap-induced morphing}} (\bibinfo {year} {2021})\BibitemShut
  {NoStop}%
\bibitem [{\citenamefont {Calladine}(2018)}]{calladineWithoutImperfections}%
  \BibitemOpen
  \bibfield  {author} {\bibinfo {author} {\bibfnamefont {C.~R.}\ \bibnamefont
  {Calladine}},\ }\bibfield  {title} {\bibinfo {title} {Shell buckling, without
  ‘imperfections’},\ }\href {https://doi.org/10.1177/1369433217751585}
  {\bibfield  {journal} {\bibinfo  {journal} {Advances in Structural
  Engineering}\ }\textbf {\bibinfo {volume} {21}},\ \bibinfo {pages} {2393}
  (\bibinfo {year} {2018})}\BibitemShut {NoStop}%
\bibitem [{\citenamefont {Gauss}(1828)}]{gauss1828disquisitiones}%
  \BibitemOpen
  \bibfield  {author} {\bibinfo {author} {\bibfnamefont {C.~F.}\ \bibnamefont
  {Gauss}},\ }\href@noop {} {\emph {\bibinfo {title} {Disquisitiones generales
  circa superficies curvas}}},\ Vol.~\bibinfo {volume} {1}\ (\bibinfo
  {publisher} {Typis Dieterichianis},\ \bibinfo {year} {1828})\BibitemShut
  {NoStop}%
\bibitem [{\citenamefont {O'{N}eill}(2014)}]{o2014elementary}%
  \BibitemOpen
  \bibfield  {author} {\bibinfo {author} {\bibfnamefont {B.}~\bibnamefont
  {O'{N}eill}},\ }\href@noop {} {\emph {\bibinfo {title} {Elementary
  differential geometry}}}\ (\bibinfo  {publisher} {Academic press},\ \bibinfo
  {year} {2014})\BibitemShut {NoStop}%
\bibitem [{\citenamefont {Seide}(1956)}]{seide1956}%
  \BibitemOpen
  \bibfield  {author} {\bibinfo {author} {\bibfnamefont {P.}~\bibnamefont
  {Seide}},\ }\bibfield  {title} {\bibinfo {title} {{Axisymmetrical Buckling of
  Circular Cones Under Axial Compression}},\ }\href
  {https://doi.org/10.1115/1.4011410} {\bibfield  {journal} {\bibinfo
  {journal} {Journal of Applied Mechanics}\ }\textbf {\bibinfo {volume} {23}},\
  \bibinfo {pages} {625} (\bibinfo {year} {1956})}\BibitemShut {NoStop}%
\bibitem [{\citenamefont {Koiter}(1945)}]{koiter1945}%
  \BibitemOpen
  \bibfield  {author} {\bibinfo {author} {\bibfnamefont {W.~T.}\ \bibnamefont
  {Koiter}},\ }\emph {\bibinfo {title} {On the stability of elastic
  equilibrium}},\ \href@noop {} {Ph.D. thesis},\ \bibinfo  {school} {Techische
  Hooge School, Delft} (\bibinfo {year} {1945})\BibitemShut {NoStop}%
\bibitem [{\citenamefont {Zoelly}(1915)}]{zoelly1915}%
  \BibitemOpen
  \bibfield  {author} {\bibinfo {author} {\bibfnamefont {R.}~\bibnamefont
  {Zoelly}},\ }\emph {\bibinfo {title} {{\"U}ber ein Knickungsproblem an der
  Kugelschale}},\ \href@noop {} {Ph.D. thesis},\ \bibinfo  {school} {ETH
  Zurich} (\bibinfo {year} {1915})\BibitemShut {NoStop}%
\bibitem [{\citenamefont {Hutchinson}(2016)}]{hutchinson2016buckling}%
  \BibitemOpen
  \bibfield  {author} {\bibinfo {author} {\bibfnamefont {J.~W.}\ \bibnamefont
  {Hutchinson}},\ }\bibfield  {title} {\bibinfo {title} {Buckling of spherical
  shells revisited},\ }\href@noop {} {\bibfield  {journal} {\bibinfo  {journal}
  {Proceedings of the Royal Society A: Mathematical, Physical and Engineering
  Sciences}\ }\textbf {\bibinfo {volume} {472}},\ \bibinfo {pages} {20160577}
  (\bibinfo {year} {2016})}\BibitemShut {NoStop}%
\bibitem [{\citenamefont {Hutchinson}\ and\ \citenamefont
  {Thompson}(2018)}]{hutchinsonImperfections2018}%
  \BibitemOpen
  \bibfield  {author} {\bibinfo {author} {\bibfnamefont {J.~W.}\ \bibnamefont
  {Hutchinson}}\ and\ \bibinfo {author} {\bibfnamefont {J.~M.~T.}\ \bibnamefont
  {Thompson}},\ }\bibfield  {title} {\bibinfo {title} {Imperfections and energy
  barriers in shell buckling},\ }\href
  {https://doi.org/https://doi.org/10.1016/j.ijsolstr.2018.01.030} {\bibfield
  {journal} {\bibinfo  {journal} {International Journal of Solids and
  Structures}\ }\textbf {\bibinfo {volume} {148-149}},\ \bibinfo {pages} {157}
  (\bibinfo {year} {2018})},\ \bibinfo {note} {special Issue Dedicated to the
  Memory of George Simitses}\BibitemShut {NoStop}%
\bibitem [{\citenamefont {Stein}(1964)}]{steinInfluence}%
  \BibitemOpen
  \bibfield  {author} {\bibinfo {author} {\bibfnamefont {M.}~\bibnamefont
  {Stein}},\ }\bibfield  {title} {\bibinfo {title} {The influence of
  prebuckling deformations and stresses in the buckling of perfect cylinders},\
  }\href@noop {} {\bibfield  {journal} {\bibinfo  {journal} {NASA Technical
  Report}\ }\textbf {\bibinfo {volume} {TR-R-190}} (\bibinfo {year}
  {1964})}\BibitemShut {NoStop}%
\bibitem [{\citenamefont {Stein}(1968)}]{steinAdvances}%
  \BibitemOpen
  \bibfield  {author} {\bibinfo {author} {\bibfnamefont {M.}~\bibnamefont
  {Stein}},\ }\bibinfo {title} {Recent advances in shell buckling},\ in\ \href
  {https://doi.org/10.2514/6.1968-103} {\emph {\bibinfo {booktitle} {6th
  Aerospace Sciences Meeting}}}\ (\bibinfo  {publisher} {American Institute of
  Aeronautics and Astronautics},\ \bibinfo {year} {1968})\ Chap.\ \bibinfo
  {chapter} {103}\BibitemShut {NoStop}%
\bibitem [{\citenamefont {Hoff}\ and\ \citenamefont
  {Nachbar}(1962)}]{HoffNachbar}%
  \BibitemOpen
  \bibfield  {author} {\bibinfo {author} {\bibfnamefont {N.~J.}\ \bibnamefont
  {Hoff}}\ and\ \bibinfo {author} {\bibfnamefont {W.}~\bibnamefont {Nachbar}},\
  }\bibfield  {title} {\bibinfo {title} {The buckling of a free edge of an
  axially-compressed circular cylindrical shell},\ }\href@noop {} {\bibfield
  {journal} {\bibinfo  {journal} {Quart. Appl. Math.}\ }\textbf {\bibinfo
  {volume} {20}},\ \bibinfo {pages} {267} (\bibinfo {year} {1962})}\BibitemShut
  {NoStop}%
\bibitem [{\citenamefont {Hoff}\ and\ \citenamefont {Soong}(1965)}]{HoffSoong}%
  \BibitemOpen
  \bibfield  {author} {\bibinfo {author} {\bibfnamefont {N.~J.}\ \bibnamefont
  {Hoff}}\ and\ \bibinfo {author} {\bibfnamefont {T.-C.}\ \bibnamefont
  {Soong}},\ }\bibfield  {title} {\bibinfo {title} {Buckling of circular
  cylindrical shells in axial compression},\ }\href
  {https://doi.org/10.1016/0020-7403(65)90050-0} {\bibfield  {journal}
  {\bibinfo  {journal} {International Journal of Mechanical Sciences}\ }\textbf
  {\bibinfo {volume} {7}},\ \bibinfo {pages} {489} (\bibinfo {year}
  {1965})}\BibitemShut {NoStop}%
\bibitem [{\citenamefont {Almroth}(1966)}]{AlmrothInfluence}%
  \BibitemOpen
  \bibfield  {author} {\bibinfo {author} {\bibfnamefont {B.~O.}\ \bibnamefont
  {Almroth}},\ }\bibfield  {title} {\bibinfo {title} {Influence of edge
  conditions on the stability of axially compressed cylindrical shells},\
  }\href {https://doi.org/10.2514/3.3396} {\bibfield  {journal} {\bibinfo
  {journal} {AIAA Journal}\ }\textbf {\bibinfo {volume} {4}},\ \bibinfo {pages}
  {134} (\bibinfo {year} {1966})}\BibitemShut {NoStop}%
\bibitem [{\citenamefont {Gorman}\ and\ \citenamefont
  {Evan-Iwanowski}(1970)}]{gormanEvan-Iwanowski}%
  \BibitemOpen
  \bibfield  {author} {\bibinfo {author} {\bibfnamefont {D.}~\bibnamefont
  {Gorman}}\ and\ \bibinfo {author} {\bibfnamefont {R.}~\bibnamefont
  {Evan-Iwanowski}},\ }\bibinfo {title} {An analytical and experimental
  investigation of the effect of large prebuckling deformations on the buckling
  of clamped thin walled circular cylindrical shells subjected to axial loading
  and internal pressure},\ in\ \href@noop {} {\emph {\bibinfo {booktitle}
  {Developments in theoretical and applied mechanics}}},\ Vol.~\bibinfo
  {volume} {4}\ (\bibinfo  {publisher} {Pergamon},\ \bibinfo {year} {1970})\
  pp.\ \bibinfo {pages} {415--426}\BibitemShut {NoStop}%
\bibitem [{\citenamefont {Shen}(2017)}]{shenBook}%
  \BibitemOpen
  \bibfield  {author} {\bibinfo {author} {\bibfnamefont {H.-S.}\ \bibnamefont
  {Shen}},\ }\href {https://doi.org/10.1142/10208} {\emph {\bibinfo {title}
  {Postbuckling Behavior of Plates and Shells}}}\ (\bibinfo  {publisher} {World
  Scientific},\ \bibinfo {year} {2017})\ Chap.~\bibinfo {chapter}
  {5}\BibitemShut {NoStop}%
\bibitem [{\citenamefont {Kobayashi}(1967)}]{kobayashi_influence}%
  \BibitemOpen
  \bibfield  {author} {\bibinfo {author} {\bibfnamefont {S.}~\bibnamefont
  {Kobayashi}},\ }\bibfield  {title} {\bibinfo {title} {The influence of
  prebuckling deformation on the buckling load of truncated conical shells
  under axial compression},\ }\href@noop {} {\bibfield  {journal} {\bibinfo
  {journal} {NASA Contractor Report}\ }\textbf {\bibinfo {volume} {CR-707}}
  (\bibinfo {year} {1967})}\BibitemShut {NoStop}%
\bibitem [{\citenamefont {Tovstik}\ and\ \citenamefont
  {Smirnov}(2001)}]{tovstikBook}%
  \BibitemOpen
  \bibfield  {author} {\bibinfo {author} {\bibfnamefont {P.}~\bibnamefont
  {Tovstik}}\ and\ \bibinfo {author} {\bibfnamefont {A.}~\bibnamefont
  {Smirnov}},\ }\href {https://doi.org/10.1142/4790} {\emph {\bibinfo {title}
  {Asymptotic Methods in the Buckling Theory of Elastic Shells}}}\ (\bibinfo
  {publisher} {World Scientific},\ \bibinfo {year} {2001})\ Chap.\ \bibinfo
  {chapter} {14.3}\BibitemShut {NoStop}%
\bibitem [{\citenamefont {McCracken}\ \emph {et~al.}(2021)\citenamefont
  {McCracken}, \citenamefont {Donovan}, \citenamefont {Lynch},\ and\
  \citenamefont {White}}]{mccracken_sharpen}%
  \BibitemOpen
  \bibfield  {author} {\bibinfo {author} {\bibfnamefont {J.~M.}\ \bibnamefont
  {McCracken}}, \bibinfo {author} {\bibfnamefont {B.~R.}\ \bibnamefont
  {Donovan}}, \bibinfo {author} {\bibfnamefont {K.~M.}\ \bibnamefont {Lynch}},\
  and\ \bibinfo {author} {\bibfnamefont {T.~J.}\ \bibnamefont {White}},\
  }\bibfield  {title} {\bibinfo {title} {Molecular engineering of mesogenic
  constituents within liquid crystalline elastomers to sharpen thermotropic
  actuation},\ }\href {https://doi.org/10.1002/adfm.202100564} {\bibfield
  {journal} {\bibinfo  {journal} {Advanced Functional Materials}\ }\textbf
  {\bibinfo {volume} {31}},\ \bibinfo {pages} {2100564} (\bibinfo {year}
  {2021})}\BibitemShut {NoStop}%
\bibitem [{\citenamefont {Duffy}\ and\ \citenamefont
  {Biggins}(2020)}]{defective_nematogenesis}%
  \BibitemOpen
  \bibfield  {author} {\bibinfo {author} {\bibfnamefont {D.}~\bibnamefont
  {Duffy}}\ and\ \bibinfo {author} {\bibfnamefont {J.~S.}\ \bibnamefont
  {Biggins}},\ }\bibfield  {title} {\bibinfo {title} {Defective nematogenesis:
  Gauss curvature in programmable shape-responsive sheets with topological
  defects},\ }\href {https://doi.org/10.1039/D0SM01192D} {\bibfield  {journal}
  {\bibinfo  {journal} {Soft Matter}\ }\textbf {\bibinfo {volume} {16}},\
  \bibinfo {pages} {10935} (\bibinfo {year} {2020})}\BibitemShut {NoStop}%
\bibitem [{\citenamefont {Niordson}(1985)}]{niordsonbook}%
  \BibitemOpen
  \bibfield  {author} {\bibinfo {author} {\bibfnamefont {F.~I.}\ \bibnamefont
  {Niordson}},\ }\href@noop {} {\emph {\bibinfo {title} {Shell theory}}},\
  \bibinfo {series} {North-Holland series in applied mathematics and
  mechanics}, Vol.~\bibinfo {volume} {29}\ (\bibinfo  {publisher}
  {North-Holland},\ \bibinfo {year} {1985})\BibitemShut {NoStop}%
\bibitem [{\citenamefont {Paulose}\ and\ \citenamefont
  {Nelson}(2013)}]{paulose2013buckling}%
  \BibitemOpen
  \bibfield  {author} {\bibinfo {author} {\bibfnamefont {J.}~\bibnamefont
  {Paulose}}\ and\ \bibinfo {author} {\bibfnamefont {D.~R.}\ \bibnamefont
  {Nelson}},\ }\bibfield  {title} {\bibinfo {title} {Buckling pathways in
  spherical shells with soft spots},\ }\href@noop {} {\bibfield  {journal}
  {\bibinfo  {journal} {Soft Matter}\ }\textbf {\bibinfo {volume} {9}},\
  \bibinfo {pages} {8227} (\bibinfo {year} {2013})}\BibitemShut {NoStop}%
\bibitem [{\citenamefont {Virtanen}\ \emph {et~al.}(2020)\citenamefont
  {Virtanen}, \citenamefont {Gommers}, \citenamefont {Oliphant}, \citenamefont
  {Haberland}, \citenamefont {Reddy}, \citenamefont {Cournapeau}, \citenamefont
  {Burovski}, \citenamefont {Peterson}, \citenamefont {Weckesser},
  \citenamefont {Bright}, \citenamefont {{van der Walt}}, \citenamefont
  {Brett}, \citenamefont {Wilson}, \citenamefont {Millman}, \citenamefont
  {Mayorov}, \citenamefont {Nelson}, \citenamefont {Jones}, \citenamefont
  {Kern}, \citenamefont {Larson}, \citenamefont {Carey}, \citenamefont {Polat},
  \citenamefont {Feng}, \citenamefont {Moore}, \citenamefont {{VanderPlas}},
  \citenamefont {Laxalde}, \citenamefont {Perktold}, \citenamefont {Cimrman},
  \citenamefont {Henriksen}, \citenamefont {Quintero}, \citenamefont {Harris},
  \citenamefont {Archibald}, \citenamefont {Ribeiro}, \citenamefont
  {Pedregosa}, \citenamefont {{van Mulbregt}},\ and\ \citenamefont {{SciPy 1.0
  Contributors}}}]{scipy}%
  \BibitemOpen
  \bibfield  {author} {\bibinfo {author} {\bibfnamefont {P.}~\bibnamefont
  {Virtanen}}, \bibinfo {author} {\bibfnamefont {R.}~\bibnamefont {Gommers}},
  \bibinfo {author} {\bibfnamefont {T.~E.}\ \bibnamefont {Oliphant}}, \bibinfo
  {author} {\bibfnamefont {M.}~\bibnamefont {Haberland}}, \bibinfo {author}
  {\bibfnamefont {T.}~\bibnamefont {Reddy}}, \bibinfo {author} {\bibfnamefont
  {D.}~\bibnamefont {Cournapeau}}, \bibinfo {author} {\bibfnamefont
  {E.}~\bibnamefont {Burovski}}, \bibinfo {author} {\bibfnamefont
  {P.}~\bibnamefont {Peterson}}, \bibinfo {author} {\bibfnamefont
  {W.}~\bibnamefont {Weckesser}}, \bibinfo {author} {\bibfnamefont
  {J.}~\bibnamefont {Bright}}, \bibinfo {author} {\bibfnamefont {S.~J.}\
  \bibnamefont {{van der Walt}}}, \bibinfo {author} {\bibfnamefont
  {M.}~\bibnamefont {Brett}}, \bibinfo {author} {\bibfnamefont
  {J.}~\bibnamefont {Wilson}}, \bibinfo {author} {\bibfnamefont {K.~J.}\
  \bibnamefont {Millman}}, \bibinfo {author} {\bibfnamefont {N.}~\bibnamefont
  {Mayorov}}, \bibinfo {author} {\bibfnamefont {A.~R.~J.}\ \bibnamefont
  {Nelson}}, \bibinfo {author} {\bibfnamefont {E.}~\bibnamefont {Jones}},
  \bibinfo {author} {\bibfnamefont {R.}~\bibnamefont {Kern}}, \bibinfo {author}
  {\bibfnamefont {E.}~\bibnamefont {Larson}}, \bibinfo {author} {\bibfnamefont
  {C.~J.}\ \bibnamefont {Carey}}, \bibinfo {author} {\bibfnamefont
  {{\.I}.}~\bibnamefont {Polat}}, \bibinfo {author} {\bibfnamefont
  {Y.}~\bibnamefont {Feng}}, \bibinfo {author} {\bibfnamefont {E.~W.}\
  \bibnamefont {Moore}}, \bibinfo {author} {\bibfnamefont {J.}~\bibnamefont
  {{VanderPlas}}}, \bibinfo {author} {\bibfnamefont {D.}~\bibnamefont
  {Laxalde}}, \bibinfo {author} {\bibfnamefont {J.}~\bibnamefont {Perktold}},
  \bibinfo {author} {\bibfnamefont {R.}~\bibnamefont {Cimrman}}, \bibinfo
  {author} {\bibfnamefont {I.}~\bibnamefont {Henriksen}}, \bibinfo {author}
  {\bibfnamefont {E.~A.}\ \bibnamefont {Quintero}}, \bibinfo {author}
  {\bibfnamefont {C.~R.}\ \bibnamefont {Harris}}, \bibinfo {author}
  {\bibfnamefont {A.~M.}\ \bibnamefont {Archibald}}, \bibinfo {author}
  {\bibfnamefont {A.~H.}\ \bibnamefont {Ribeiro}}, \bibinfo {author}
  {\bibfnamefont {F.}~\bibnamefont {Pedregosa}}, \bibinfo {author}
  {\bibfnamefont {P.}~\bibnamefont {{van Mulbregt}}},\ and\ \bibinfo {author}
  {\bibnamefont {{SciPy 1.0 Contributors}}},\ }\bibfield  {title} {\bibinfo
  {title} {{{SciPy} 1.0: Fundamental Algorithms for Scientific Computing in
  Python}},\ }\href {https://doi.org/10.1038/s41592-019-0686-2} {\bibfield
  {journal} {\bibinfo  {journal} {Nature Methods}\ }\textbf {\bibinfo {volume}
  {17}},\ \bibinfo {pages} {261} (\bibinfo {year} {2020})}\BibitemShut
  {NoStop}%
\bibitem [{\citenamefont {Chen}\ \emph {et~al.}(2021)\citenamefont {Chen},
  \citenamefont {Pauly},\ and\ \citenamefont
  {Reis}}]{reisReprogrammableMemory}%
  \BibitemOpen
  \bibfield  {author} {\bibinfo {author} {\bibfnamefont {T.}~\bibnamefont
  {Chen}}, \bibinfo {author} {\bibfnamefont {M.}~\bibnamefont {Pauly}},\ and\
  \bibinfo {author} {\bibfnamefont {P.~M.}\ \bibnamefont {Reis}},\ }\bibfield
  {title} {\bibinfo {title} {A reprogrammable mechanical metamaterial with
  stable memory},\ }\href {https://doi.org/10.1038/s41586-020-03123-5}
  {\bibfield  {journal} {\bibinfo  {journal} {Nature}\ }\textbf {\bibinfo
  {volume} {589}},\ \bibinfo {pages} {386} (\bibinfo {year}
  {2021})}\BibitemShut {NoStop}%
\bibitem [{\citenamefont {Gomez}\ \emph {et~al.}(2017)\citenamefont {Gomez},
  \citenamefont {Moulton},\ and\ \citenamefont {Vella}}]{vellaPassiveControl}%
  \BibitemOpen
  \bibfield  {author} {\bibinfo {author} {\bibfnamefont {M.}~\bibnamefont
  {Gomez}}, \bibinfo {author} {\bibfnamefont {D.~E.}\ \bibnamefont {Moulton}},\
  and\ \bibinfo {author} {\bibfnamefont {D.}~\bibnamefont {Vella}},\ }\bibfield
   {title} {\bibinfo {title} {Passive control of viscous flow via elastic
  snap-through},\ }\href {https://doi.org/10.1103/PhysRevLett.119.144502}
  {\bibfield  {journal} {\bibinfo  {journal} {Phys. Rev. Lett.}\ }\textbf
  {\bibinfo {volume} {119}},\ \bibinfo {pages} {144502} (\bibinfo {year}
  {2017})}\BibitemShut {NoStop}%
\bibitem [{\citenamefont {Holmes}\ and\ \citenamefont
  {Crosby}(2007)}]{holmesSnappingSurfaces}%
  \BibitemOpen
  \bibfield  {author} {\bibinfo {author} {\bibfnamefont {D.}~\bibnamefont
  {Holmes}}\ and\ \bibinfo {author} {\bibfnamefont {A.}~\bibnamefont
  {Crosby}},\ }\bibfield  {title} {\bibinfo {title} {Snapping surfaces},\
  }\href {https://doi.org/https://doi.org/10.1002/adma.200700584} {\bibfield
  {journal} {\bibinfo  {journal} {Advanced Materials}\ }\textbf {\bibinfo
  {volume} {19}},\ \bibinfo {pages} {3589} (\bibinfo {year}
  {2007})}\BibitemShut {NoStop}%
\bibitem [{\citenamefont {Melancon}\ \emph {et~al.}(2021)\citenamefont
  {Melancon}, \citenamefont {Gorissen}, \citenamefont {Garc{\'i}a-Mora},
  \citenamefont {Hoberman},\ and\ \citenamefont
  {Bertoldi}}]{bertoldiOrigamiMetre}%
  \BibitemOpen
  \bibfield  {author} {\bibinfo {author} {\bibfnamefont {D.}~\bibnamefont
  {Melancon}}, \bibinfo {author} {\bibfnamefont {B.}~\bibnamefont {Gorissen}},
  \bibinfo {author} {\bibfnamefont {C.~J.}\ \bibnamefont {Garc{\'i}a-Mora}},
  \bibinfo {author} {\bibfnamefont {C.}~\bibnamefont {Hoberman}},\ and\
  \bibinfo {author} {\bibfnamefont {K.}~\bibnamefont {Bertoldi}},\ }\bibfield
  {title} {\bibinfo {title} {Multistable inflatable origami structures at the
  metre scale},\ }\href {https://doi.org/10.1038/s41586-021-03407-4} {\bibfield
   {journal} {\bibinfo  {journal} {Nature}\ }\textbf {\bibinfo {volume}
  {592}},\ \bibinfo {pages} {545} (\bibinfo {year} {2021})}\BibitemShut
  {NoStop}%
\bibitem [{\citenamefont {Vasios}\ \emph {et~al.}(2021)\citenamefont {Vasios},
  \citenamefont {Deng}, \citenamefont {Gorissen},\ and\ \citenamefont
  {Bertoldi}}]{bertoldiUniversallyBistable}%
  \BibitemOpen
  \bibfield  {author} {\bibinfo {author} {\bibfnamefont {N.}~\bibnamefont
  {Vasios}}, \bibinfo {author} {\bibfnamefont {B.}~\bibnamefont {Deng}},
  \bibinfo {author} {\bibfnamefont {B.}~\bibnamefont {Gorissen}},\ and\
  \bibinfo {author} {\bibfnamefont {K.}~\bibnamefont {Bertoldi}},\ }\bibfield
  {title} {\bibinfo {title} {Universally bistable shells with nonzero gaussian
  curvature for two-way transition waves},\ }\href
  {https://doi.org/10.1038/s41467-020-20698-9} {\bibfield  {journal} {\bibinfo
  {journal} {Nature Communications}\ }\textbf {\bibinfo {volume} {12}},\
  \bibinfo {pages} {695} (\bibinfo {year} {2021})}\BibitemShut {NoStop}%
\bibitem [{\citenamefont {Gorissen}\ \emph {et~al.}(2020)\citenamefont
  {Gorissen}, \citenamefont {Melancon}, \citenamefont {Vasios}, \citenamefont
  {Torbati},\ and\ \citenamefont {Bertoldi}}]{bertoldiJumper}%
  \BibitemOpen
  \bibfield  {author} {\bibinfo {author} {\bibfnamefont {B.}~\bibnamefont
  {Gorissen}}, \bibinfo {author} {\bibfnamefont {D.}~\bibnamefont {Melancon}},
  \bibinfo {author} {\bibfnamefont {N.}~\bibnamefont {Vasios}}, \bibinfo
  {author} {\bibfnamefont {M.}~\bibnamefont {Torbati}},\ and\ \bibinfo {author}
  {\bibfnamefont {K.}~\bibnamefont {Bertoldi}},\ }\bibfield  {title} {\bibinfo
  {title} {Inflatable soft jumper inspired by shell snapping},\ }\href
  {https://doi.org/10.1126/scirobotics.abb1967} {\bibfield  {journal} {\bibinfo
   {journal} {Science Robotics}\ }\textbf {\bibinfo {volume} {5}},\ \bibinfo
  {pages} {eabb1967} (\bibinfo {year} {2020})}\BibitemShut {NoStop}%
\bibitem [{\citenamefont {Medina}\ \emph {et~al.}(2020)\citenamefont {Medina},
  \citenamefont {Farrell}, \citenamefont {Bertoldi},\ and\ \citenamefont
  {Rycroft}}]{bertoldiNavigating}%
  \BibitemOpen
  \bibfield  {author} {\bibinfo {author} {\bibfnamefont {E.}~\bibnamefont
  {Medina}}, \bibinfo {author} {\bibfnamefont {P.~E.}\ \bibnamefont {Farrell}},
  \bibinfo {author} {\bibfnamefont {K.}~\bibnamefont {Bertoldi}},\ and\
  \bibinfo {author} {\bibfnamefont {C.~H.}\ \bibnamefont {Rycroft}},\
  }\bibfield  {title} {\bibinfo {title} {Navigating the landscape of nonlinear
  mechanical metamaterials for advanced programmability},\ }\href
  {https://doi.org/10.1103/PhysRevB.101.064101} {\bibfield  {journal} {\bibinfo
   {journal} {Phys. Rev. B}\ }\textbf {\bibinfo {volume} {101}},\ \bibinfo
  {pages} {064101} (\bibinfo {year} {2020})}\BibitemShut {NoStop}%
\bibitem [{\citenamefont {Reis}(2015)}]{reisBuckliphilia}%
  \BibitemOpen
  \bibfield  {author} {\bibinfo {author} {\bibfnamefont {P.~M.}\ \bibnamefont
  {Reis}},\ }\bibfield  {title} {\bibinfo {title} {{A Perspective on the
  Revival of Structural (In)Stability With Novel Opportunities for Function:
  From Buckliphobia to Buckliphilia}},\ }\bibfield  {journal} {\bibinfo
  {journal} {Journal of Applied Mechanics}\ }\textbf {\bibinfo {volume} {82}},\
  \href {https://doi.org/10.1115/1.4031456} {10.1115/1.4031456} (\bibinfo
  {year} {2015}),\ \bibinfo {note} {111001}\BibitemShut {NoStop}%
\bibitem [{\citenamefont {Ambulo}\ \emph {et~al.}(2017)\citenamefont {Ambulo},
  \citenamefont {Burroughs}, \citenamefont {Boothby}, \citenamefont {Kim},
  \citenamefont {Shankar},\ and\ \citenamefont {Ware}}]{Ware_Boothby}%
  \BibitemOpen
  \bibfield  {author} {\bibinfo {author} {\bibfnamefont {C.~P.}\ \bibnamefont
  {Ambulo}}, \bibinfo {author} {\bibfnamefont {J.~J.}\ \bibnamefont
  {Burroughs}}, \bibinfo {author} {\bibfnamefont {J.~M.}\ \bibnamefont
  {Boothby}}, \bibinfo {author} {\bibfnamefont {H.}~\bibnamefont {Kim}},
  \bibinfo {author} {\bibfnamefont {M.~R.}\ \bibnamefont {Shankar}},\ and\
  \bibinfo {author} {\bibfnamefont {T.~H.}\ \bibnamefont {Ware}},\ }\bibfield
  {title} {\bibinfo {title} {Four-dimensional printing of liquid crystal
  elastomers},\ }\href {https://doi.org/10.1021/acsami.7b11851} {\bibfield
  {journal} {\bibinfo  {journal} {ACS Applied Materials \& Interfaces}\
  }\textbf {\bibinfo {volume} {9}},\ \bibinfo {pages} {37332} (\bibinfo {year}
  {2017})}\BibitemShut {NoStop}%
\bibitem [{\citenamefont {Pogorelov}(1988)}]{PogorelovMonograph}%
  \BibitemOpen
  \bibfield  {author} {\bibinfo {author} {\bibfnamefont {A.~V.}\ \bibnamefont
  {Pogorelov}},\ }\href@noop {} {\emph {\bibinfo {title} {Bendings of surfaces
  and stability of shells}}},\ edited by\ \bibinfo {editor} {\bibfnamefont
  {B.}~\bibnamefont {Silver}},\ Vol.~\bibinfo {volume} {72}\ (\bibinfo
  {publisher} {American Mathematical Soc.},\ \bibinfo {year}
  {1988})\BibitemShut {NoStop}%
\bibitem [{\citenamefont {Gomez}\ \emph {et~al.}(2016)\citenamefont {Gomez},
  \citenamefont {Moulton},\ and\ \citenamefont {Vella}}]{VellaPogo}%
  \BibitemOpen
  \bibfield  {author} {\bibinfo {author} {\bibfnamefont {M.}~\bibnamefont
  {Gomez}}, \bibinfo {author} {\bibfnamefont {D.~E.}\ \bibnamefont {Moulton}},\
  and\ \bibinfo {author} {\bibfnamefont {D.}~\bibnamefont {Vella}},\ }\bibfield
   {title} {\bibinfo {title} {The shallow shell approach to pogorelov's problem
  and the breakdown of ‘mirror buckling’},\ }\href@noop {} {\bibfield
  {journal} {\bibinfo  {journal} {Proceedings of the Royal Society A:
  Mathematical, Physical and Engineering Sciences}\ }\textbf {\bibinfo {volume}
  {472}},\ \bibinfo {pages} {20150732} (\bibinfo {year} {2016})}\BibitemShut
  {NoStop}%
\bibitem [{\citenamefont {Seffen}(2016)}]{seffeninvertedcones}%
  \BibitemOpen
  \bibfield  {author} {\bibinfo {author} {\bibfnamefont {K.~A.}\ \bibnamefont
  {Seffen}},\ }\bibfield  {title} {\bibinfo {title} {Inverted cones and their
  elastic creases},\ }\href {https://doi.org/10.1103/PhysRevE.94.063002}
  {\bibfield  {journal} {\bibinfo  {journal} {Phys. Rev. E}\ }\textbf {\bibinfo
  {volume} {94}},\ \bibinfo {pages} {063002} (\bibinfo {year}
  {2016})}\BibitemShut {NoStop}%
\bibitem [{\citenamefont {Hebner}\ \emph {et~al.}(2021)\citenamefont {Hebner},
  \citenamefont {Bowman},\ and\ \citenamefont {White}}]{hebner_intermolecular}%
  \BibitemOpen
  \bibfield  {author} {\bibinfo {author} {\bibfnamefont {T.~S.}\ \bibnamefont
  {Hebner}}, \bibinfo {author} {\bibfnamefont {C.~N.}\ \bibnamefont {Bowman}},\
  and\ \bibinfo {author} {\bibfnamefont {T.~J.}\ \bibnamefont {White}},\
  }\bibfield  {title} {\bibinfo {title} {The contribution of intermolecular
  forces to phototropic actuation of liquid crystalline elastomers},\ }\href
  {https://doi.org/10.1039/D1PY00028D} {\bibfield  {journal} {\bibinfo
  {journal} {Polym. Chem.}\ }\textbf {\bibinfo {volume} {12}},\ \bibinfo
  {pages} {1581} (\bibinfo {year} {2021})}\BibitemShut {NoStop}%
\bibitem [{\citenamefont {Bauman}\ \emph {et~al.}(2022)\citenamefont {Bauman},
  \citenamefont {McCracken},\ and\ \citenamefont
  {White}}]{mccraken_below_ambient}%
  \BibitemOpen
  \bibfield  {author} {\bibinfo {author} {\bibfnamefont {G.~E.}\ \bibnamefont
  {Bauman}}, \bibinfo {author} {\bibfnamefont {J.~M.}\ \bibnamefont
  {McCracken}},\ and\ \bibinfo {author} {\bibfnamefont {T.~J.}\ \bibnamefont
  {White}},\ }\bibfield  {title} {\bibinfo {title} {Actuation of liquid
  crystalline elastomers at or below ambient temperature},\ }\href
  {https://doi.org/https://doi.org/10.1002/anie.202202577} {\bibfield
  {journal} {\bibinfo  {journal} {Angewandte Chemie International Edition}\
  }\textbf {\bibinfo {volume} {61}},\ \bibinfo {pages} {e202202577} (\bibinfo
  {year} {2022})}\BibitemShut {NoStop}%
\bibitem [{\citenamefont {Knoche}\ and\ \citenamefont
  {Kierfeld}(2014)}]{KnocheSecondary}%
  \BibitemOpen
  \bibfield  {author} {\bibinfo {author} {\bibfnamefont {S.}~\bibnamefont
  {Knoche}}\ and\ \bibinfo {author} {\bibfnamefont {J.}~\bibnamefont
  {Kierfeld}},\ }\bibfield  {title} {\bibinfo {title} {The secondary buckling
  transition: Wrinkling of buckled spherical shells},\ }\href
  {https://doi.org/10.1140/epje/i2014-14062-9} {\bibfield  {journal} {\bibinfo
  {journal} {The European Physical Journal E}\ }\textbf {\bibinfo {volume}
  {37}},\ \bibinfo {pages} {62} (\bibinfo {year} {2014})}\BibitemShut {NoStop}%
\bibitem [{\citenamefont {Vaziri}\ and\ \citenamefont
  {Mahadevan}(2008)}]{vaziriMaha}%
  \BibitemOpen
  \bibfield  {author} {\bibinfo {author} {\bibfnamefont {A.}~\bibnamefont
  {Vaziri}}\ and\ \bibinfo {author} {\bibfnamefont {L.}~\bibnamefont
  {Mahadevan}},\ }\bibfield  {title} {\bibinfo {title} {Localized and extended
  deformations of elastic shells},\ }\href
  {https://doi.org/10.1073/pnas.0707364105} {\bibfield  {journal} {\bibinfo
  {journal} {Proceedings of the National Academy of Sciences}\ }\textbf
  {\bibinfo {volume} {105}},\ \bibinfo {pages} {7913} (\bibinfo {year}
  {2008})}\BibitemShut {NoStop}%
\end{thebibliography}%


%\newpage
% \detailtexcount{main}

\end{document}
