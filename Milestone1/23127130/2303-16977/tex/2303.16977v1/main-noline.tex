\documentclass[twocolumn]{aastex62}
\usepackage{CJK}
% \usepackage{lineno}
% \linenumbers

%% The default is a single spaced, 10 point font, single spaced article.
%% There are 5 other style options available via an optional argument. They
%% can be envoked like this:
%%
%% \documentclass[argument]{aastex61}
%% 
%% where the arguement options are:
%%
%%  twocolumn   : two text columns, 10 point font, single spaced article.
%%                This is the most compact and represent the final published
%%                derived PDF copy of the accepted manuscript from the publisher
%%  manuscript  : one text column, 12 point font, double spaced article.
%%  preprint    : one text column, 12 point font, single spaced article.  
%%  preprint2   : two text columns, 12 point font, single spaced article.
%%  modern      : a stylish, single text column, 12 point font, article with
%% 		  wider left and right margins. This uses the Daniel
%% 		  Foreman-Mackey and David Hogg design.
%%
%% Note that you can submit to the AAS Journals in any of these 6 styles.
%%
%% There are other optional arguments one can envoke to allow other stylistic
%% actions. The available options are:
%%
%%  astrosymb    : Loads Astrosymb font and define \astrocommands. 
%%  tighten      : Makes baselineskip slightly smaller, only works with 
%%                 the twocolumn substyle.
%%  times        : uses times font instead of the default
%%  linenumbers  : turn on lineno package.
%%  trackchanges : required to see the revision mark up and print its output
%%  longauthor   : Do not use the more compressed footnote style (default) for 
%%                 the author/collaboration/affiliations. Instead print all
%%                 affiliation information after each name. Creates a much
%%                 long author list but may be desirable for short author papers
%%
%% these can be used in any combination, e.g.
%%
%% \documentclass[twocolumn,linenumbers,trackchanges]{aastex61}

%% AASTeX v6.* now includes \hyperref support. While we have built in specific
%% defaults into the classfile you can manually override them with the
%% \hypersetup command. For example,
%%
%%\hypersetup{linkcolor=red,citecolor=green,filecolor=cyan,urlcolor=magenta}
%%
%% will change the color of the internal links to red, the links to the
%% bibliography to green, the file links to cyan, and the external links to
%% magenta. Additional information on \hyperref options can be found here:
%% https://www.tug.org/applications/hyperref/manual.html#x1-40003

%% If you want to create your own macros, you can do so
%% using \newcommand. Your macros should appear before
%% the \begin{document} command.
%%
\newcommand{\vdag}{(v)^\dagger}
\newcommand\aastex{AAS\TeX}
\newcommand\latex{La\TeX}
\newcommand{\etal}{et al.\,}
\newcommand{\secpoint}{\mbox{$''\mskip-7.6mu.\,$}}
\newcommand{\secnopoint}{\mbox{$''\mskip-7.6mu\,$}}
\newcommand{\lya}{Ly$\alpha$}
\newcommand{\brg}{Br$\gamma \:$}
\newcommand{\microns}{$\mu$m}
\newcommand{\kprime}{\textit{K}' }
\newcommand{\chisquare}{$\chi^{2}$ }
\newcommand\tna{\,\tablenotemark{a}}
\newcommand\tnb{\,\tablenotemark{b}}
\newcommand\tnc{\,\tablenotemark{c}}
\newcommand\tnd{\,\tablenotemark{d}}
\newcommand\tne{\,\tablenotemark{e}}
\newcommand\tnf{\,\tablenotemark{f}}
\newcommand{\perday}{days$^{-1}$}
\newcommand{\kms}{km s$^{-1}$}
\newcommand{\msun}{M$_{\odot}$ }
\newcommand{\vmax}{$v_{max}$}
\newcommand{\mcomp}{$M_{\text{comp}}$}
\newcommand{\mcompsini}{$M_{\text{comp}}\sin{i}$}
\newcommand{\vsini}{$v\sin{i}$}
\newcommand{\vtick}{\textsc{\char13}}
\newcommand{\mtot}{$M_{\text{tot}}$}

%% Reintroduced the \received and \accepted commands from AASTeX v5.2
% \received{July 1, 2016}
% \revised{September 27, 2016}
% \accepted{\today}
%% Command to document which AAS Journal the manuscript was submitted to.
%% Adds "Submitted to " the arguement.
% \submitjournal{ApJ}

%% Mark up commands to limit the number of authors on the front page.
%% Note that in AASTeX v6.1 a \collaboration call (see below) counts as
%% an author in this case.
%
%\AuthorCollaborationLimit=3
%
%% Will only show Schwarz, Muench and "the AAS Journals Data Scientist 
%% collaboration" on the front page of this example manuscript.
%%
%% Note that all of the author will be shown in the published article.
%% This feature is meant to be used prior to acceptance to make the
%% front end of a long author article more manageable. Please do not use
%% this functionality for manuscripts with less than 20 authors. Conversely,
%% please do use this when the number of authors exceeds 40.
%%
%% Use \allauthors at the manuscript end to show the full author list.
%% This command should only be used with \AuthorCollaborationLimit is used.

%% The following command can be used to set the latex table counters.  It
%% is needed in this document because it uses a mix of latex tabular and
%% AASTeX deluxetables.  In general it should not be needed.
%\setcounter{table}{1}

%%%%%%%%%%%%%%%%%%%%%%%%%%%%%%%%%%%%%%%%%%%%%%%%%%%%%%%%%%%%%%%%%%%%%%%%%%%%%%%%
%%
%% The following section outlines numerous optional output that
%% can be displayed in the front matter or as running meta-data.
%%
%% If you wish, you may supply running head information, although
%% this information may be modified by the editorial offices.
\shorttitle{S-star Binary Search}
\shortauthors{Chu et al.}
%%
%% You can add a light gray and diagonal water-mark to the first page 
%% with this command:
% \watermark{text}
%% where "text", e.g. DRAFT, is the text to appear.  If the text is 
%% long you can control the water-mark size with:
%  \setwatermarkfontsize{dimension}
%% where dimension is any recognized LaTeX dimension, e.g. pt, in, etc.
%%
%%%%%%%%%%%%%%%%%%%%%%%%%%%%%%%%%%%%%%%%%%%%%%%%%%%%%%%%%%%%%%%%%%%%%%%%%%%%%%%%

%% This is the end of the preamble.  Indicate the beginning of the
%% manuscript itself with \begin{document}.

\begin{document}
\begin{CJK*}{UTF8}{gbsn}

\title{Evidence of a decreased binary fraction for massive stars within 20 milliparsecs of the supermassive black hole at the Galactic center}

\correspondingauthor{Devin Chu}
\email{dchu@astro.ucla.edu}

\author[0000-0003-3765-8001]{Devin S. Chu}
\affil{Department of Physics and Astronomy \\
UCLA\\
Los Angeles, CA 90095-1547, USA}
% \collaboration{(UCLA Galactic Center Group)}

\author[0000-0001-9554-6062]{Tuan Do}
\affiliation{Department of Physics and Astronomy \\
UCLA\\
Los Angeles, CA 90095-1547, USA}
% \collaboration{(UCLA Galactic Center Group)}

\author[0000-0003-3230-5055]{Andrea Ghez}
\affiliation{Department of Physics and Astronomy \\
UCLA\\
Los Angeles, CA 90095-1547, USA}
% \collaboration{(UCLA Galactic Center Group)}

\author[0000-0002-2836-117X]{Abhimat K. Gautam}
\affiliation{Department of Physics and Astronomy \\
UCLA\\
Los Angeles, CA 90095-1547, USA}
% \collaboration{(UCLA Galactic Center Group)}

\author[0000-0001-5800-3093]{Anna Ciurlo}
\affiliation{Department of Physics and Astronomy \\
UCLA\\
Los Angeles, CA 90095-1547, USA}
% \collaboration{(UCLA Galactic Center Group)}

\author[0000-0003-2400-7322]{Kelly Kosmo O'neil}
\affiliation{Department of Physics and Astronomy \\
UCLA\\
Los Angeles, CA 90095-1547, USA}
% \collaboration{(UCLA Galactic Center Group)}

\author[0000-0003-2874-1196]{Matthew W. Hosek Jr.}
\altaffiliation{Brinson Prize Fellow}
\affiliation{Department of Physics and Astronomy \\
UCLA\\
Los Angeles, CA 90095-1547, USA}

\author[0000-0002-2186-644X]{Aur\'elien Hees}
\affiliation{SYRTE \\
Observatoire de Paris\\
Universit\'e PSL, CNRS, Sorbonne Universit\'e, \\
Paris, France}

\author[0000-0002-9802-9279]{Smadar Naoz}
\affiliation{Department of Physics and Astronomy \\
UCLA\\
Los Angeles, CA 90095-1547, USA}
\affiliation{Mani L. Bhaumik Institute for Theoretical Physics \\
Department of Physics and Astronomy \\
UCLA\\
Los Angeles, CA 90095-1547, USA}

\author[0000-0001-5972-663X]{Shoko Sakai}
\affiliation{Department of Physics and Astronomy \\
UCLA\\
Los Angeles, CA 90095-1547, USA}

\author[0000-0001-9611-0009]{Jessica R. Lu}
\affiliation{Astronomy Department\\
University of California, Berkeley\\
Berkeley, CA 94720, USA}


\author[0000-0002-3038-3896]{Zhuo Chen (陈卓)}
\affiliation{Department of Physics and Astronomy \\
UCLA\\
Los Angeles, CA 90095-1547, USA}

\author[0000-0001-7017-8582]{Rory O. Bentley}
\affiliation{Department of Physics and Astronomy \\
UCLA\\
Los Angeles, CA 90095-1547, USA}

% \author[0000-0002-6753-2066]{Mark R. Morris}
% \affiliation{Department of Physics and Astronomy \\
% UCLA\\
% Los Angeles, CA 90095-1547, USA}

\author{Eric E. Becklin}
\affiliation{Department of Physics and Astronomy \\
UCLA\\
Los Angeles, CA 90095-1547, USA}

\author{Keith Matthews}
\affiliation{Department of Physics and Astronomy \\
Caltech\\
Pasadena, CA 91125, USA}

%% Note that the \and command from previous versions of AASTeX is now
%% depreciated in this version as it is no longer necessary. AASTeX 
%% automatically takes care of all commas and "and"s between authors names.

%% AASTeX 6.1 has the new \collaboration and \nocollaboration commands to
%% provide the collaboration status of a group of authors. These commands 
%% can be used either before or after the list of corresponding authors. The
%% argument for \collaboration is the collaboration identifier. Authors are
%% encouraged to surround collaboration identifiers with ()s. The 
%% \nocollaboration command takes no argument and exists to indicate that
%% the nearby authors are not part of surrounding collaborations.

%% Mark off the abstract in the ``abstract'' environment. 
\begin{abstract}

% While binaries are important for understanding the young star cluster at the Galactic center, there have been limited surveys for binaries amongst the S-stars. 
We present the results of the first systematic search for spectroscopic binaries within the central 2 x 3 arcsec$^2$ around the supermassive black hole at the center of the Milky Way galaxy. This survey is based primarily on over a decade of adaptive optics-fed integral-field spectroscopy (R$\sim$4000), obtained as part of the \textit{Galactic Center Orbits Initiative} at Keck Observatory, and has a limiting $K$'-band magnitude of 15.8, which is at least 4 magnitudes deeper than previous spectroscopic searches for binaries at larger radii within the central nuclear star cluster. From this primary dataset, over 600 new radial velocities are extracted and reported, increasing by a factor of 3 the number of such measurements.  We find no significant periodic signals in our sample of 28 stars, of which 16 are massive, young (main-sequence B) stars and 12 are low-mass, old (M and K giant) stars. Using Monte Carlo simulations, we derive upper limits on the intrinsic binary star fraction for the young star population at 47\% (at 95\% confidence) located $\sim$20 mpc from the black hole. The young star binary fraction is significantly lower than that observed in the field (70\%). This result is consistent with a scenario in which the central supermassive black hole drives nearby stellar binaries to merge or be disrupted and may have important implications for the production of gravitational waves and hypervelocity stars.

%this can also explain the presence of the so-called G-objects, which show spatially resolved tidal interactions.
% (0.1 $pc^2$)

%, providing important insight into the formation and evolution of the paradoxically young massive stars located within an arcsecond (0.04 pc) of the central supermassive black hole

%With over two decades of integral field spectroscopy data, advanced tools for fitting near-infrared stellar spectra, we conduct a spectroscopic binary search of 28 stars at the Galactic center. After subtracting a star's motion around the supermassive black hole, we search for a periodic signal using a Lomb-Scargle analysis and fitting the residual radial velocity curve to a binary system radial velocity curve. 
%%- Give radial distance of sample, magnitude cut off and how it improves work, more limits on companion masses

%%old sentence

\end{abstract}

%% Keywords should appear after the \end{abstract} command. 
%% See the online documentation for the full list of available subject
%% keywords and the rules for their use.
\keywords{Infrared spectroscopy (2285), Galactic center (565), Adaptive optics (2281)}

%% From the front matter, we move on to the body of the paper.
%% Sections are demarcated by \section and \subsection, respectively.
%% Observe the use of the LaTeX \label
%% command after the \subsection to give a symbolic KEY to the
%% subsection for cross-referencing in a \ref command.
%% You can use LaTeX's \ref and \label commands to keep track of
%% cross-references to sections, equations, tables, and figures.
%% That way, if you change the order of any elements, LaTeX will
%% automatically renumber them.

%% We recommend that authors also use the natbib \citep
%% and \citet commands to identify citations.  The citations are
%% tied to the reference list via symbolic KEYs. The KEY corresponds
%% to the KEY in the \bibitem in the reference list below. 

%%%%%%%%%%%%%%%%%%%%%%%%%%%%%%%%%%%%%%%%%%%%%%%%%% 
%INTRO
%%%%%%%%%%%%%%%%%%%%%%%%%%%%%%%%%%%%%%%%%%%%%%%%%% 

\section{Introduction} \label{sec:intro}

% \section{Introduction}

The increasing complexity of source code poses a key challenge to the reliability of large-scale software systems. Software bugs in these systems can lead to safety issues~\cite{bug_safety} for users around the world as well as cause non-negligible financial losses~\cite{bug_loss}. As such, developers have to spend a large amount of time and effort on bug fixing. Consequently, \aprfull (\apr), designed to automatically generate patches to fix software bugs, has attracted wide attention from both academia and industry~\cite{long2016prophet, legoues2012genprog, long2015spr, lou2020can, tufano2018empstudy}. 


To achieve \apr, one popular approach is known as Generate-and-Validate (G\&V)~\cite{qi2015gv, ghanbari2019prapr, lou2020can, le2016hdrepair, legoues2012genprog, wen2018capgen, hua2018sketchfix, martinez2016astor, koyuncu2020fixminder, liu2019tbar, liu2019avatar}, which is typically based on the following pipeline: First, fault localization techniques~\cite{wong2016fl, abreu2007ochiai, zhang2013injecting, papadakis2015metallaxis, li2019deepfl, li2017transforming} are applied to determine the suspicious locations in programs where bugs are likely to exist. Then, the buggy locations are used by the \apr tools to generate a list of patches that replace buggy lines with correct lines. Afterward, each patch is validated against the original test suite to identify any \emph{plausible patches} (i.e., passing all tests in the test suite). Finally, to determine the \emph{correct patches}, developers examine the list of plausible patches to see if any of them can correctly fix the bug. 

Traditional \apr tools can mainly be categorized into heuristic-based~\cite{legoues2012genprog, le2016hdrepair, wen2018capgen}, constraint-based~\cite{mechtaev2016angelix, le2017s3, demacro2014nopol, long2015spr} and \template~\cite{ghanbari2019prapr, hua2018sketchfix, martinez2016astor, liu2019tbar, liu2019avatar}. Among these traditional tools, \template \apr tools~\cite{ghanbari2019prapr, liu2019tbar, benton2020effectiveness} have been able to achieve state-of-the-art results. \Template \apr tools typically leverage pre-defined templates (e.g., adding a nullness check) for bug fixing. However, since these fix templates are typically handcrafted, the number and types of bugs they are able to fix can be limited. 



To address the limitations of traditional \apr, researchers have proposed various \learning \apr tools~\cite{li2020dlfix, chen2018sequencer, jiang2021cure, lutellier2020coconut, zhu2021recoder, ye2022rewardrepair} based on the \nmtfull (\nmt) architecture~\cite{sutskever2014mt} where the input is the buggy code snippets and the goal is to translate the buggy code snippets into a fixed version. To accomplish this, \learning \apr tools require supervised training datasets with pairs of both buggy and fixed code snippets in order to learn how to perform this translation step. These training data are usually obtained by mining historical bug fixes using heuristics/keywords~\cite{dallmeier2007benchmark}, which can be imprecise for identifying bug-fixing commits; even the actual bug-fixing commits can include irrelevant code changes, leading to further pollution in the dataset~\cite{xia2022alpharepair}.
% 
Moreover, it can be hard for such \apr tools to generalize and fix bug types unseen during training. 



To better leverage recent advances in \plmfull{s} (\plm{s}), researchers~\cite{xia2022alpharepair, xia2023repairstudy, kolak2022patch, prenner2021codexws} have directly applied \plm{s} to generate patches without bug-fixing datasets. These \llm-based \apr tools work by either directly generating a complete code function~\cite{prenner2021codexws, xia2023repairstudy} or predict/infill the correct code snippet given its surrounding context~\cite{xia2022alpharepair, xia2023repairstudy}. By directly using \llm{s} that are pre-trained on billions of open-source code snippets, \llm-based \apr tools can achieve state-of-the-art performance on many repair datasets~\cite{xia2022alpharepair}. 


% 
%
%

Traditional \apr tools have long used the insight of the \emph{plastic surgery hypothesis}~\cite{barr2014plastic} where it states that the code ingredients to fix a bug already exist within the same project. Traditional \apr tools have manually designed pattern-~\cite{ghanbari2019prapr, saha2017elixir} or heuristic-based~\cite{jiang2018simfix, legoues2012genprog} approaches to finding and using such relevant code ingredients to generate fixes for bugs. However, the plastic surgery hypothesis has been largely ignored in \llm-based \apr. In fact, \llm provides a unique opportunity to fully automate the plastic surgery hypothesis idea via fine-tuning (learning project-specific information via model updates from the buggy project) and prompting (directly providing relevant code ingredients to the model), and make it directly applicable to different languages (since the \llm{s} are typically multi-lingual).%
Moreover, despite the intensive manual efforts involved, traditional \apr tools still cannot fully leverage project-specific information due to large search space for leveraging/composing existing code ingredients. In contrast, the project-specific information can effectively leveraged by \llm{s} due to their power in code understanding/vectorization, e.g., even partial/imprecise information may still guide \llm{s} in correct patch generation!
 To this end, we ask the question: \emph{How useful is the plastic surgery hypothesis in the era of \plm{s}}?








\mypara{Our Work.} To answer the question, we present \ourtech{\xspace} -- a \llm-based approach that automatically utilizes the plastic surgery hypothesis by systematically combining multiple fine-tuning and prompting strategies for \apr. \ourtech fine-tunes \plm{s} using two novel domain-specific training strategies: \textbf{\epfinetune} -- we fine-tune using the original buggy project by aggressively masking out a high percentage of tokens, which allows \plm to learn project-specific code tokens and programming styles; and \textbf{\rofinetune} -- which only masks out a single continuous code sequence per training sample, allowing the model to get used to the final \csapr task of predicting a single continuous code sequence. Furthermore, we directly leverage the ability for \plm{s} to understand natural language instructions and introduce a novel prompting strategy, \textbf{\idprompting}, which uses information retrieval and static analysis to obtain a list of relevant identifiers for the buggy lines. While such relevant identifiers are critical for fixing some difficult bugs, they may not be seen by the \llm during inference due to limited context window size. Through the use of prompting, we directly tell the model to use these extracted identifiers (relevant code ingredients) to generate the correct code. Finally, to perform repair, we combine all four model variants (including the base model, both fine-tuned models and the base model with prompting) for the final repair.





While our insight of leveraging the plastic surgery hypothesis for \llm-based \apr is generalizable across different types of \plm{s}, to implement \ourtech, we choose a recent \plm{\xspace}, \ctfive~\cite{wang2021codet5}, which is pre-trained on millions of open-source code snippets. \ctfive is an encoder-decoder model trained using \mspfull (\msp) objective where a percentage of tokens are masked out and each continuous masked token sequence is referred to as a masked span. Also, although we only extract relevant identifiers from the current buggy project (since this paper focuses on the plastic surgery hypothesis), our work can be easily extended to obtain other code information (such as relevant statements or functions) from other sources, such as  the massive pre-training corpora~\cite{husain2020codesearchnet} or historical bug-fixing datasets~\cite{jiang2019infer}, which can provide more coding knowledge for \llm{s}. Besides, although we mainly focus on using traditional string comparison algorithms for information retrieval in this paper, these techniques can be easily replaced by other frequency-based retrieval~\cite{robertson2009probabilistic} and neural search (or embedding-based search)~\cite{reimers2019sentence}.
  In summary, this paper makes the following contributions:


%


\begin{itemize}[noitemsep, leftmargin=*, topsep=0pt]
    \item \textbf{Dimension.} This paper is the first to revisit the important plastic surgery hypothesis in the era of \llm{s}. It opens up a new dimension for \llm-based \apr to incorporate previously neglected information from the buggy project itself to boost \apr performance. Furthermore, it demonstrates the promising future of retrieval-based prompting for modern \llm-based \apr.
    \item \textbf{Implementation.} We implement \ourtech based on the recent \ctfive model. We augment the model using two novel fine-tuning strategies: \epfinetune and \rofinetune, along with a novel prompting strategy based on information retrieval and static analysis: \idprompting. We combine the patches generated by all four models together and perform patch ranking to speed up \apr.% 
    \item \textbf{Evaluation Study.} We conduct an extensive evaluation against state-of-the-art \apr tools. On the widely studied \dfj 1.2 and 2.0 datasets~\cite{just2014dfj}, \ourtech is able to achieve the new state-of-the-art results of 89 and 44 correct bug fixes (15 and 8 more than best baseline) respectively.  Furthermore, we perform a broad ablation study to justify our design. \ourtech demonstrates for the first time that the plastic surgery hypothesis can substantially boost \llm-based \apr and advance state-of-the-art \apr, while being fully automated and general. Moreover, even partial/imprecise code ingredients may still effectively guide \llm{s} for \apr!
\end{itemize}



The closest known stars to the Milky Way's supermassive black hole (SMBH) comprise the so-called ``S-star'' cluster, where S stands for Sgr A*, the emissive source associated with the SMBH. This population is both distinct dynamically and spectroscopically from the surrounding stellar population. Spectroscopic observations have also revealed that most of these stars are main-sequence B stars \citep{Ghez:2003iw,Eisenhauer:2005gh,Habibi:2017}. Unlike their cousins outside the central radius of 0.04 parsecs, this population lacks Wolf-Rayet stars, suggesting the S-stars have formed within the last 20 million years. Their young ages raise questions about their formation mechanism, since traditional star formation would be disrupted by the tidal forces of the black hole \citep{Morris:1993fp}.

%and are one of the most intriguing star populations at the Galactic center. Their orbits have proved the existence of a supermassive black hole  with a mass of $4 \times 10^{6}$ \msun \ \citep{Ghez:2008ty,Gillessen:2009fg,Gillessen:2017fa} and their motions have exhibited post-Newtonian effects \citep[e.g.][and references therein]{GRAVITY:2018A, GRAVITY:2020precession, Do:2019}. 

Numerous investigations have been done to postulate the formation of these S-stars. General mechanisms include: (1) binary star systems scattered from outside the region and then tidally disrupted, leaving behind one component of the original binary while the other is ejected as a hypervelocity star \citep[e.g.][]{Hills:1988br,Perets:2007fo,Generozov2020ApJ}, (2) S-stars formed in the clockwise disk located just outside 1 arcsec of the SMBH and then migrated to the SMBH \citep[e.g.][]{Levin:2007ez,Lockmann:2008be,Merritt:2009ab}, and (3) merger of binary stars at the Galactic centers caused by the Kozai-Lidov mechanism, with the product appearing as a main-sequence B-star \citep[e.g.][]{witzel2014,Stephan:2016eh,Fragione2019MNRAS,Ciurlo:2020Nature}.

Binary stars provide crucial roles in these formation mechanisms, and the discovery of binary stars amongst the S-stars may attest to particular formation mechanisms. Additionally, massive stars in the field have high multiplicity fractions \citep{Sana:2012,Duchene:2013}, so it is reasonable to expect these main-sequence B-stars at least started out in multiple systems \citep{Naoz2018}. Previous studies have identified 3 binary systems \citep{Ott:1999gn,Martins:2006,Rafelski:2007gu,Pfuhl:2014eba,Gautam:2019}, but none amongst the S-stars. \citet{Chu:2018} performed the first spectroscopic search for binaries amongst the S-stars and focused on the well-studied star S0-2 (also known as S2). Through radial velocity monitoring, \citet{Chu:2018} did not find significant evidence for S0-2 being a binary and placed a hypothetical companion mass upper limit at 1.6 \msun, which is below current detection limits. The dataset that was used to analyze S0-2 can be used to perform a more comprehensive survey.

% The lack of a binary companion agreed with \citet{GravCollaboration:2017}, which did not detect any stars with $K$ magnitude of 17.1 mag within the field of view of S0-2 and Sgr A*.

% With nearly two decades of spectroscopic measurements, there is an opportunity to expand the search for spectroscopic binaries to the rest of the S-star cluster. \citet{Chu:2018} established a framework for searching for spectroscopic binaries of S-star in radial velocity data. Improvements to tools for fitting mid-infrared stellar spectra \citep[1.9-2.3 \microns,][]{starkit:2015,BOSZ:2017} and better understanding of radial velocity measurement systematic uncertainties \citep{Do:2019} have also improved our sensitivity to other spectroscopic binaries.

In this work, we use the Galactic Center Orbits Initiative (GCOI, PI Ghez, W. M. Keck Observatory 1995 - present) long-term monitoring of this region with W. M. Keck Observatory to conduct a systematic search for binary stars using radial velocities of the S-stars. This paper is organized as follows. Section \ref{sec:Sample} details the sample selection process for this search. Section \ref{sec:data} describes the radial velocity data used in searching for spectroscopic binaries. Section \ref{sec:motion} details the process of modeling the long-term motion of the sample stars around the central black hole. Section \ref{sec:per_search} describes the search methodology for looking for companion stars in the stellar sample. Section \ref{sec:derive_limits} provides an overview of placing an upper limit on the binary star fraction. Section \ref{sec:discussion} discusses how these limits carry implications for the evolution of the S-stars.

%%%%%%%%%%%%%%%%%%%%%%%%%%%%%%%%%%%%%%%%%%%%%%%%%% 
%SAMPLE
%%%%%%%%%%%%%%%%%%%%%%%%%%%%%%%%%%%%%%%%%%%%%%%%%%

\section{Sample Selection} \label{sec:Sample}

The broadest criteria of the star sample used in the analysis presented in this paper is that the star must be brighter than $K'$ $>$ 16 magnitude and located within the field of view of this study's primary dataset, which is centered on S0-2 and covers 3" $\times$ 2" at a PA of 285 degrees (see Figure \ref{fig:S-star_OSIRIS}, Table \ref{tab:all_spec_obs}, and Section \ref{sec:data}). Our magnitude limit stems from what can be measured with adequate signal-to-noise from a single night of observations ($\sim$3-4 hours of integration). These initial criteria yield an intermediate sample of 62 stars . From here, we make several other cuts. First,  Wolf-Rayet emission line sources (IRS16C and IRS16SW) are excluded, because measuring their radial velocities is complicated due to their stellar winds. Similarly, the seven main-sequence O stars are not included since they are featureless across the spectral range studied (2.121 -- 2.229 $\mu$m). We further omit 25 stars, for which source confusion prevents their radial velocities from being extracted without bias in our primary dataset (see Appendix \ref{sect:confusion_gas_appendix} for details). We also omit the star S0-28 because it only has 2 radial velocity measurements, which is too few to conduct a periodicity search. This leads to a final sample 28 stars, of which 16 are early-type stars and 12 are late-type stars (see Table \ref{tab:full_rv_sample}).

% and two early-type stars (S1-2 and S1-33) whose radial velocity measurements were biased by the background gas emission in the region at their locations .

%%Can move this table to later in the paper if needed because it contains kinematic fit information

\startlongtable
\begin{deluxetable*}{lccccccccc}
\tablecolumns{10} 
\tablewidth{0pc} 
\tablecaption{Summary of Spectroscopic Observations with New Radial Velocities\label{tab:all_spec_obs}}
\tablehead{ 
    \multicolumn{3}{c}{Date\tna}  & 
    \colhead{Instrument} &
	\colhead{$N_{\text{frames}} \times t_{\text{int}}$} &
	\colhead{Filter} &
	\colhead{Scale} &
	\colhead{FWHM\tnb} &
	\colhead{SNR\tnb} &
	\colhead{New RVs} \\
	\cline{1-3}
    \colhead{(UT)} &
	\colhead{(MJD)} &
    \colhead{(Epoch)} &
	\colhead{} &
	\colhead{(s)} &
	\colhead{} &
	\colhead{(mas)} &
	\colhead{(mas)} &
	\colhead{} &
	\colhead{This Work\tnc} 
}
\startdata
2005-07-03$^{1}$		&  	53554.50	&  	2005.503	&	OSIRIS	&  	7  $\times$ 900   	&	Kbb 	&	20	&	58	  	&	44	&	5\\
2006-06-18$^{1}$		&  	53904.50	&  	2006.461	&	OSIRIS	&  	9  $\times$ 900   	&	Kn3 	&	35	&	81	  	&	39	&	10\\
2006-06-30$^{1}$		&  	53916.50	&  	2006.494	&	OSIRIS	&  	9  $\times$ 900   	&	Kn3 	&	35	&	77	  	&	42	&	12\\
2006-07-01$^{1}$		&  	53917.50	&  	2006.497	&	OSIRIS	&  	9  $\times$ 900   	&	Kn3 	&	35	&	64	  	&	46	&	8\\
2007-05-21$^{1}$		&  	54241.50	&  	2007.384	&	OSIRIS	&  	2 $\times$ 900   	&	Kn3 	&	35	&	86		&	16	&	5\\
2007-07-18$^{1}$		&  	54299.29	&  	2007.542	&	OSIRIS	&  	2 $\times$ 900   	&	Kn3 	&	35	&	66$^{5}$		&	33$^{5}$	&	4\\
2007-07-19$^{1}$		&  	54300.29	&  	2007.545	&	OSIRIS	&  	2 $\times$ 900   	&	Kn3 	&	35	&	56	  	&	32	&	6\\
2008-05-16$^{2}$		&  	54602.50	&  	2008.372	&	OSIRIS	&  	11 $\times$ 900   	&	Kn3 	&	35	&	57	  	&	66	&	17\\
2008-07-25$^{2}$		&  	54672.28	&  	2008.563	&	OSIRIS	&  	9  $\times$ 900 	&	Kn3 	&	35	&	81	  	&	57	&	17\\
2009-05-05$^{2}$		&  	54956.50	&  	2009.342	&	OSIRIS	&  	7   $\times$ 900	&	Kn3 	&	35	&	70	  	&	58	&	18\\
2009-05-06$^{2}$		&  	54957.50	&  	2009.344	&	OSIRIS	&  	12  $\times$ 900	&	Kn3 	&	35	&	81	  	&	74	&	9\\
2010-05-05$^{2}$		&  	55321.50	&  	2010.341	&	OSIRIS	&  	6  $\times$ 900 	&	Kn3 	&	35	&	70	  	&	26	&	8\\
2010-05-08$^{2}$		&  	55324.50	&  	2010.349	&	OSIRIS	&  	11  $\times$ 900 	&	Kn3 	&	35	&	79	  	&	43	&	19\\
2011-07-10$^{2}$		&  	55752.33	&  	2011.520	&	OSIRIS	&  	6   $\times$ 900   	&	Kn3 	&	35	&	71	  	&	29	&	20\\
2011-07-19$^{2}$		&  	55761.31	&  	2011.545	&	OSIRIS	&  	6 $\times$ 900   	&	Kn3 	&	35	&	96$^{6}$		&	27$^{6}$	&	7\\
2012-06-11$^{2}$		&  	56089.50	&  	2012.444	&	OSIRIS	&  	7  	$\times$ 900	&	Kn3 	&	20	&	64	  	&	40	&	4\\
2012-07-22$^{2}$		&  	56130.31	&  	2012.555	&	OSIRIS	&  	7   $\times$ 900   	&	Kn3 	&	35	&	92	 	&	37	&	10\\
2012-08-12$^{2}$		&  	56151.33	&  	2012.613	&	OSIRIS	&  	6   $\times$ 900   	&	Kn3 	&	35	&	56	  	&	66	&	6\\
2012-08-13$^{2}$		&  	56152.27	&  	2012.615	&	OSIRIS	&  	7   $\times$ 900	&	Kn3 	&	35	&	99	  	&	41	&	8\\
2013-05-11$^{2}$		&  	56423.50	&  	2013.358	&	OSIRIS	&  	11  $\times$ 900	&	Kbb		&	35	&	73	  	&	41	&	7\\
2013-05-12$^{2}$		&  	56424.50	&  	2013.361	&	OSIRIS	&  	11  $\times$ 900  	&	Kbb 	&	35	&	62	  	&	45	&	3\\
2013-05-13$^{2}$		&  	56425.50	&  	2013.363	&	OSIRIS	&  	12  $\times$ 900	&	Kbb	 	&	35	&	61	  	&	33	&	3\\
2013-05-14$^{2}$		&  	56426.50	&  	2013.366	&	OSIRIS	&  	11  $\times$ 900   	&	Kn3	 	&	35	&	67	  	&	72	&	21\\
2013-05-16$^{2}$		&  	56428.50	&  	2013.372	&	OSIRIS	&  	7  $\times$ 900		&	Kn3	 	&	20	&	98	  	&	53	&	5\\
2013-05-17$^{2}$		&  	56429.50	&  	2013.374	&	OSIRIS	&  	7  $\times$ 900 	&	Kn3	 	&	20	&	64   	&	43	&	5\\
2013-07-25$^{2}$		&  	56498.33	&  	2013.563	&	OSIRIS	&  	11  $\times$ 900 	&	Kn3	 	&	35	&	79   	&	35	&	8\\
2013-07-26$^{2}$		&  	56499.34	&  	2013.566	&	OSIRIS	&  	6  $\times$ 900 	&	Kn3	 	&	35	&	73   	&	22	&	5\\
2013-07-27$^{2}$		&  	56500.33	&  	2013.568	&	OSIRIS	&  	11  $\times$ 900 	&	Kn3	 	&	35	&	72   	&	49	&	18\\
2013-08-10$^{2}$		&  	56514.29	&  	2013.607	&	OSIRIS	&  	7 $\times$ 900		&	Kn3 	&	35	&	62   	&	50	&	13\\
2013-08-11$^{2}$		&  	56515.31	&  	2013.609	&	OSIRIS	&  	9 $\times$ 900		&	Kn3 	&	35	&	69   	&	24	&	14\\
2013-08-13$^{2}$		&  	56517.29	&  	2013.615	&	OSIRIS	&  	12 $\times$ 900		&	Kn3 	&	35	&	67   	&	54	&	11\\
2014-05-17				&  	56794.51	&  	2014.374	&	OSIRIS	&  	6 $\times$ 900   	&	Kn3 	&	35	&	69$^{6}$		&	41$^{6}$	&	6\\
2014-05-18$^{3}$		&	56795.50	& 	2014.376	&	OSIRIS	& 	13 $\times$ 900		&	Kn3		&	35	& 	66 		&	53	&	18\\
2014-05-19				&  	56796.51	&  	2014.379	&	OSIRIS	&  	13 $\times$ 900   	&	Kn3 	&	35	&	65$^{6}$		&	62$^{6}$	&	20\\
2014-05-22				&  	56799.51	&  	2014.387	&	OSIRIS	&  	7 $\times$ 900   	&	Kn3 	&	35	&	82$^{6}$		&	26$^{6}$	&	6\\
2014-05-23$^{3}$		&	56800.50	& 	2014.390	&	OSIRIS	& 	10 $\times$ 900		&	Kn3		&	35	& 	76 		&	42	&	8\\
2014-07-03$^{3}$		&	56841.36	& 	2014.502	&	OSIRIS	& 	8 $\times$ 900		&	Kn3		&	35	& 	66 		&	57	&	22\\
2015-05-04$^{3}$		&	57146.50	& 	2015.337	&	OSIRIS	& 	5 $\times$ 900		&	Kn3		&	35	& 	77 		&	49	&	4\\
2015-07-21$^{3}$		&	57224.35	& 	2015.551	&	OSIRIS	& 	5 $\times$ 900		&	Kn3		&	35	& 	56 		&	51	&	24\\
2015-08-07				&	57241.33	& 	2015.597	&	OSIRIS	& 	2 $\times$ 900		&	Kn3		&	35	& 	84$^{7}$ 		&	13$^{7}$	&	1\\
2016-05-14$^{3}$		&	57522.50	& 	2016.367	&	OSIRIS	& 	8 $\times$ 900		&	Kbb		&	35	& 	78 		&	58	&	1\\
2016-05-15$^{3}$		&	57523.50	& 	2016.370	&	OSIRIS	& 	4 $\times$ 900		&	Kbb		&	35	& 	80 		&	36	&	2\\
2016-05-16$^{3}$		&	57524.50	& 	2016.372	&	OSIRIS	& 	8 $\times$ 900		&	Kbb		&	35	& 	84 		&	63	&	2\\
2016-07-11				&	57580.35	& 	2016.525	&	OSIRIS	& 	8 $\times$ 900		&	Kbb		&	35	& 	69$^{8}$		&	42$^{8}$ &	6\\
2016-07-12				&	57581.33	& 	2016.528	&	OSIRIS	& 	7 $\times$ 900		&	Kbb		&	35	& 	115$^{9}$		&	30$^{9}$ &	1\\
2017-05-17$^{4}$		&	57890.52	&	2017.374	&	OSIRIS	&	11 $\times$ 900		&	Kn3		&	35 	&	73		&	101	&	20\\
2017-05-18$^{4}$		&	57891.51	&	2017.377	&	OSIRIS	&	9 $\times$ 900		&	Kn3		&	35 	&	94		&	49	&	20\\
2017-05-19$^{4}$		&	57892.50	&	2017.379	&	OSIRIS	&	6 $\times$ 900		&	Kn3		&	35 	&	86		&	77	&	8\\
2017-07-19$^{4}$		&	57953.33	&	2017.546	&	OSIRIS	&	12 $\times$ 900		&	Kn3		&	35 	&	77		&	55	&	10\\
2017-07-27$^{4}$		&	57961.32	&	2017.568	&	OSIRIS	&	13 $\times$ 900		&	Kn3		&	35 	&	89		&	76	&	20\\
2017-08-14$^{4}$		&	57979.28	&	2017.617	&	OSIRIS	&	8 $\times$ 900		&	Kn3		&	35 	&	75		&	71	&	21\\
2018-03-17$^{4}$		&	58194.64	&	2018.207	&	OSIRIS	&	2 $\times$ 900		&	Kn3		&	35 	&	70		&	30	&	6\\
2018-04-24$^{4}$		&	58232.57	&	2018.310	&	OSIRIS	&	7 $\times$ 900		&	Kn3		&	35 	&	73		&	67	&	12\\
2018-05-13$^{4}$		&	58251.51	&	2018.362	&	NIFS	&	12 $\times$ 600		&	K		&	50 $\times$ 100 	&		&	84	&	7\\
2018-05-22$^{4}$		&	58260.49	&	2018.387	&	NIFS	&	7 $\times$ 600		&	K		&	50 $\times$ 100 	&		&	66	&	4\\
2018-05-23$^{4}$		&	58261.50	&	2018.390	&	OSIRIS	&	14 $\times$ 900		&	Kn3		&	35 	&	91		&	97	&	19\\
2018-06-05$^{4}$		&	58274.47	&	2018.425	&	OSIRIS	&	10 $\times$ 900		&	Kn3		&	35 	&	108		&	44	&	4\\
2018-07-22$^{4}$		&	58321.33	&	2018.554	&	OSIRIS	&	11 $\times$ 900		&	Kn3		&	35 	&	77		&	113	&	18\\
2018-07-31$^{4}$		&	58330.32	&	2018.578	&	OSIRIS	&	11 $\times$ 900		&	Kn3		&	35 	&	73		&	121	&	16\\
2018-08-11$^{4}$		&	58341.31	&	2018.608	&	OSIRIS	&	9 $\times$ 900		&	Kn3		&	35 	&	79		&	121	&	19\\
\enddata
\tablenotetext{a}{These observations, where noted, were first reported for studies of S0-2 alone in the following references: 1) \citet{Ghez:2008ty}, 2) \citet{Boehle:2016ko}, 3) \citet{Chu:2018}, 4) \citet{Do:2019}.}
\tablenotetext{b}{The reported values are assessed on S0-2 unless otherwise noted as follows: 5) S1-15, 6) S0-14, 7) S1-13, 8) S0-3, 9) S0-12. All stars used for characterization have $K'$ mag of 13.5-14.5.}
\tablenotetext{c}{This includes only RVs for the final sample.}
% \tablenotetext{d}{Modfied Julian Date}
%Average FWHM of S0-2 (unless otherwise noted) in the mosaic made of all frames, measured by fitting a two-dimensional Gaussian to the source.}
% \tablenotetext{b}{Spectral SNR of S0-2 unless otherwise noted}
% \tablenotetext{c}{S1-15}
% \tablenotetext{d}{S0-14}
% \tablenotetext{e}{S1-13}
% \tablenotetext{f}{S0-3}
% \tablenotetext{g}{S0-12}
% \tablenotetext{f}{Observation first reported in following reference: $f$) \citet{Ghez:2008ty}, $g$) \citet{Boehle:2016ko}, $h$) \citet{Chu:2018}, $i$) \citet{Do:2019}.}
\tablecomments{Col 1-3: date of observation given in UT, modified Julian date, and Julian Year, Col 4: instrument name, Col 5: number of frames combined times exposure time of each frame, Col 6: instrument filter, Col 7: pixel scale used, Col 8: FWHM of reference star, Col 9: spectral signal-to-noise ratio of reference star, Col 10: new radial velocity measurements reported.}
% \tablenotetext{c}{Spectral $R$ of each instrument: NIRSPEC $\sim$ 2600, NIRC2 $\sim$ 4000, OSIRIS $\sim$ 4000, IRCS $\sim$ 20000, NIFS $\sim$ 5000.}
% \label{tab:all_spec_obs}
\end{deluxetable*}


\begin{deluxetable*}{llcrrrrrrcrr}
\tablecolumns{11}
\tablewidth{0pc} 
\tablecaption{S-star RV Sample\label{tab:full_rv_sample}}
\tablehead{
\colhead{Star}	&
\colhead{$K'$}	&
\colhead{Spectral}	&
\colhead{RA$\Delta$\tna}	&
\colhead{Dec$\Delta$\tna}	&
\colhead{R2D\tna}	&
\colhead{RV}	&
\colhead{[RV $\sigma$]}	&
\colhead{RV Baseline}	&
% \colhead{Astrometry}	&
\colhead{RV Long-term Trend}	&
\colhead{Semimajor Axis}		\\
\colhead{}	&
\colhead{(mag)}	&
\colhead{Type}	&
\colhead{(\arcsec)}	&
\colhead{(\arcsec)}	&
\colhead{(\arcsec)}	&
\colhead{Points}	&
\colhead{(\kms)}	&
\colhead{(Years)}	&
% \colhead{Accel.}	&
\colhead{Method}	&
\colhead{(mpc)}
}
\startdata
S0-1	&	14.7	&	Early	&	0.04	&	-0.26	&	0.264   &	50	&	53	&	15	&	   Orbit	&	24.43 $\pm$ 0.46   \\
S0-2	&	14.0	&	Early	&	-0.01	&	0.17	&	0.172   &	115	&	23	&	18	&	   Orbit	&	4.885 $\pm$ 0.024    \\
S0-3	&	14.5	&	Early	&	0.34	&	0.12	&	0.356   &	59	&	37	&	14	&	   Orbit	&	14.082 $\pm$ 0.082    \\
S0-4	&	14.1	&	Early	&	0.45	&	-0.33	&	0.558   &	52	&	46	&	15	&	   Orbit	&	16.39 $\pm$ 0.66    \\
S0-5	&	15.0	&	Early	&	0.17	&	-0.36	&	0.408   &	42	&	61	&	14	&	   Orbit	&	10.678 $\pm$ 0.067    \\
S0-7	&	15.1	&	Early	&	0.51	&	0.10	&	0.524   &	23	&	30	&	12	&   Polynomial	&	39.9 $\pm$ 8.0     \\
S0-8	&	15.8	&	Early	&	-0.23	&	0.16	&	0.274   &	45	&	95	&	14	&	   Orbit	&	16.612 $\pm$ 0.089    \\
S0-9	&	14.2	&	Early	&	0.22	&	-0.60	&	0.625   &	33	&	36	&	13	&   Polynomial	&	69 $\pm$ 14     \\
S0-11	&	15.1	&	Early	&	0.49	&	-0.06	&	0.505   &	28	&	32	&	12	&   Polynomial	&	103 $\pm$ 21     \\
S0-14	&	13.5	&	Early	&	-0.76	&	-0.28	&	0.811   &	41	&	18	&	12	&   Polynomial	&	48.7 $\pm$ 9.7     \\
S0-15	&	13.5	&	Early	&	-0.97	&	0.18	&	0.984   &	31	&	32	&	12	&   Polynomial	&	54 $\pm$ 11     \\
S0-16	&	15.3	&	Early	&	0.23	&	0.17	&	0.284   &	24	&	76	&	14	&	   Orbit	&	11.611 $\pm$ 0.062    \\
S0-19	&	15.3	&	Early	&	-0.08	&	0.40	&	0.404   &	39	&	120	&	15	&	   Orbit	&	11.581 $\pm$ 0.040    \\
S0-20	&	15.8	&	Early	&	0.05	&	0.14	&	0.153   &	32	&	200	&	14	&	   Orbit	&	10.260 $\pm$ 0.033    \\
S0-31	&	14.9	&	Early	&	0.57	&	0.45	&	0.711   &	9	&	41	&	11	&   Polynomial	&	57 $\pm$ 11     \\
S1-8	&	14.0	&	Early	&	-0.58	&	-0.92	&	1.088   &	16	&	33	&	11	&	   Polynomial	&	89 $\pm$ 18     \\
\hline
S0-6	&	14.0	&	Late	&	0.02	&	-0.36	&	0.356   &	47	&	2.9	&	13	&   Polynomial	&	102 $\pm$ 20     \\
S0-12	&	14.3	&	Late	&	-0.55	&	0.41	&	0.689   &	48	&	3.1	&	12	&   Polynomial	&	115 $\pm$ 23     \\
S0-13	&	13.2	&	Late	&	0.56	&	-0.41	&	0.691   &	48	&	2.9	&	12	&   Polynomial	&	82 $\pm$ 16     \\
S0-17	&	15.9	&	Late	&	0.05	&	0.008	&	0.048   &	44	&	90	&	15	&	   Orbit	&	13.639 $\pm$ 0.090    \\
S0-18	&	14.9	&	Late	&	-0.12	&	-0.42	&	0.441   &	18	&	4.0	&	12	&   Polynomial	&	73 $\pm$ 15     \\
S0-27	&	15.5	&	Late	&	0.15	&	0.55	&	0.566   &	12	&	5.6	&	12	&   Polynomial	&	52 $\pm$ 10     \\
S1-5	&	12.4	&	Late	&	0.32	&	-0.89	&	0.943   &	27	&	2.6	&	12	&   Polynomial	&	150 $\pm$ 30     \\
S1-6	&	15.4	&	Late	&	-0.96	&	0.74	&	1.217   &	19	&	7.7	&	10	&   Polynomial	&	163 $\pm$ 33     \\
S1-10	&	14.7	&	Late	&	-1.10	&	-0.02	&	1.099   &	22	&	5.2	&	12	&   Polynomial	&	116 $\pm$ 23     \\
S1-13	&	14.0	&	Late	&	-1.14	&	-0.97	&	1.501   &	7	&	5.2	&	12	&   Polynomial	&	[158]     \\
S1-15	&	14.0	&	Late	&	-1.36	&	0.49	&	1.443   &	23	&	3.9	&	11	&   Polynomial	&	[152]     \\
S1-31	&	15.6	&	Late	&	-0.99	&	0.54	&	1.125   &	16	&	7.6	&	11	&   Polynomial	&   [118]	 \\
\enddata
\tablenotetext{a}{From Sgr A*.}
% \tablenotetext{b}{RV extraction is affected by background gas and is excluded from further analysis.}
\tablecomments{Col 1: star name, Col 2: magnitude in $K'$, Col 3-5: projected distance from SgrA*, Col 6: total radial velocity points used in analysis, Col 7: median radial velocity uncertainty, Col 8: baseline of radial velocity measurements, Col 9: method for subtracting long-term RV trend, Col 10: semi-major axes estimates (values in brackets come from averaging comparably large separations).}
\end{deluxetable*}

%%IF we make a .7 arcsecond cut, the sample changes
%% 6 young stars are removed - 18 young stars to 12 young stars
%% 6 old stars removed -  12 old stars to 6 old stars

%In addition to the 693 RV measurements from reported observations in Table \ref{tab:all_spec_obs}, we also include 273 radial velocities from \citet{Gillessen:2017fa} for the sample stars.

%and which are listed in Table \ref{tab:full_rv_sample} (and the omitted stars are summarized  in Table \ref{tab:excluded_stars}).

%. We note, however, that IRS16SW is a known binary \citep{Ott:1999gn, Martins:2007fw}

%in image and/or spectral space, meaning we cannot distinguish their spectra and therefore their radial velocities,

%%%%%%%%%%%%%%%%%%%%%%%%%%%%%%%%%%%%%%%%%%%%%%%%%% 
%OBSERVATIONS AND DATA
%%%%%%%%%%%%%%%%%%%%%%%%%%%%%%%%%%%%%%%%%%%%%%%%%% 

\section{Radial Velocities} \label{sec:data}

\subsection{New Radial Velocities} \label{subsec:datasets}

% The GCOI has used the 10-m telescopes on the W.M. Keck Observatory to monitor the center of the Galaxy since 1995. This GCOI dataset primarily utilizes the laser guide star adaptive optics (LGSAO) \citep{Wizinowich:2006,vanDam:2006} system to achieve near near-diffraction limited image quality in the near infrared wavelengths. The GCOI primary science focus is to study the central SMBH properties with the use of stellar orbits \citep[e.g.][]{Ghez:2008ty, Do:2019}. The experimental design of this project has required extensive monitoring of the S-stars, most notably S0-2. This dataset can also be used to look for spectroscopic binaries amongst the other S-stars near the black hole.

% \subsection{Spectroscopy Data} \label{subsec:spec_data}

The primary starting point for the radial velocity analysis is the spectrally calibrated datasets that have been used by the Galactic Center Orbits Initiative (GCOI) to study S0-2 (see references in Table \ref{tab:all_spec_obs}). The majority of these observations were taken with the OSIRIS spectrograph \citep[$R \sim$ 4000,][]{Larkin:2006} on the W. M. Keck 10-m Telescope using the laser guide star adaptive optics (LGSAO) system \citep{Wizinowich:2006,vanDam:2006} and reduced via the OSIRIS reduction pipeline \citep{OSIRIS_pipeline:2017, Lockart:2019}. These 45 datasets were taken in the 35mas pixel scale and through the Kn3 (2.121 -- 2.229 $\mu$m) filter, which covers the \brg absorption line ($\lambda =$ 2.166 \microns) for the young stars and Na for the old stars. With a 2 x 2 dither pattern that keeps the central 1" $\times$ 1" in the field of view, a total view of 3" $\times$ 2" is achieved (see Figure \ref{fig:S-star_OSIRIS}). Over the course of the reported observations, OSIRIS has gone through the following two upgrades: 1) a grating upgrade in December 2012 \citep{Mieda2014PASP} and 2) a detector upgrade in April 2016 \citep{Boehle2016SPIE}. Appendix \ref{sect:instrument} shows the impact of the detector upgrade on our dataset (the grating upgrade had no significant effect).
Of the 45 datasets with newly reported radial velocities, four observations are newly reported here: 05-17-2014, 05-19-2014, 05-22-2014, and 08-07-2015 UT. The three nights in May 2014 do not contain S0-2 radial velocity measurements due to noise spikes affecting the spectra of that star. For the night of 08-07-2015, a field south of the central pointing was observed, which meant S0-2 was not in the field of view, but the star S1-13 was in the dither, which is on the southern edge of the central pointing. On average, the observations have a FWHM of 76 mas and a spectral signal-to-noise ratio of 53 for a 14 mag star.

Several datasets supplement the above core data. Thirteen of these datasets are also taken with OSIRIS, at a different plate scale (20 mas) and/or through the broader Kbb filter (1.965 -- 2.381 $\mu$m). Of these, two are new observations that have not been previously published. These two Kbb observations were taken on 07-11-2016 and 07-12-2016 UT. On these nights, weather and dithering problems prevented S0-2 from being observed, but other stars in the sample were still in the field of view. We also include a number of previously published datasets taken with other instruments, including Keck NIRSPEC ($R \sim$ 2800 in low resolution mode), Keck NIRC2 ($R \sim$ 4000), Gemini North NIFS ($R \sim$ 5000), and Subaru IRCS ($R \sim$ 20000). A summary of all the spectroscopic observations with new radial velocities is reported in Table \ref{tab:all_spec_obs}.

%%code to generate the plot below is jupyter notebook%%
\begin{figure*}
\centering
\includegraphics[width=\linewidth]{Sstar_sample_plot_arrow.pdf}
\caption{
Finding chart for stars included in the periodicity search. The GCOI's OSIRIS 4-pointing outer dither pattern is overlaid on an adaptive optics 2.2 $\mu m$ Keck NIRC2 image of the Galactic center. Most of the sample is located in the central pointing dither pattern (blue circles: early-type, orange squars: late-type). The central square is covered by all dither points, meaning more spectra are taken of these stars throughout the night. We include all stars brighter than $K = 16$ mag, except Wolf-Rayet stars (green diamond), stars lacking absorption lines (red triangle), or are confused with other stellar sources or gas features (purple triangle). 
\label{fig:S-star_OSIRIS}
}
\end{figure*}

% More statements on data quality for detector, t-int

% For 9 observations, the Kbb (1.965 -- 2.381 $\mu$m) filter was used. Out of these 9 observations, eight were taken with the 35mas scale and one with the 20mas scale. 

% This filter trades increased wavelength coverage for a reduced field of view. 

% To do: majority taken in 30mas scale and spectral resolution. and Kbb (1.965 -- 2.381 $\mu$m)

%As a result, stars closest to the SMBH such as S0-2 have the most complete coverage, while stars further from the SMBH have fewer frames of coverage on a given night. The dither pattern completeness is also affected by factors such as seeing conditions, AO and telescope performance, so sometimes the dither pattern is not consistently completed. Despite the outer dither regions not having the same coverage as the inner region, these regions still provide valuable spectral information for the GCOI.

%These data have been reduced with the latest version of the OSIRIS data reduction pipeline \citep{OSIRIS_pipeline:2017,Lockart:2019} to make the data cubes. Blank skies and telluric standard stars were also observed during the night to correct for sky and atmospheric lines. Further details on these processes can be found in \citet{Do:2013fn}. 


% \textit{Include subsection if there are new observations.}

\subsection{Extracting Radial Velocities} \label{subsec:rvs}

Previous papers from the UCLA GCOI reported the radial velocities of S0-2 and S0-38. In this work, we also extracted the radial velocities of other known stars located in the central pointing. The methods and calibrations used to measure radial velocities are reported in \citet{Do:2019}. 

To summarize, a star's spectrum is extracted from the individual data cubes from a given epoch using a circular aperture, with an annulus around the star to estimate the sky background. The spectra are then averaged into a combined spectrum. The star's combined spectrum is then modeled using the Bayesian inference tool \textit{Starkit} \citep{starkit:2015} and compared to spectra in the BOSZ spectral grid \citep{BOSZ:2017}. The radial velocity and its uncertainty are derived using the median and 1 sigma central credible interval of the marginalized posterior. \cite{Do:2019} showed that this technique of spectral fitting reduced uncertainties and systematic bias compared to fitting a Gaussian to the \brg line for the star S0-2. Example spectra and their model fits for both early and late-type stars are shown in Figure \ref{fig:example_spec}. Early-type stars are main-sequence B stars, with the \brg absorption line being the major spectral feature in the Kn3 filter. Late-type stars are M and K giants with many absorption lines in Kn3, noteably the Na doublet lines around 22100 angstroms. The measured radial velocity is then corrected for the local standard of rest with respect to the Galactic center\footnote{We use the IRAF procedure \textit{rvcorrect}. This correction uses a velocity of 20 \kms \ for the solar motion with respect to the local standard of rest in the direction $\alpha = 18^{h}, \delta = +30\deg$ for epoch 1900 \cite{Kerr:1986}, corresponding to $(u,v,w) = (10, 15.4, 7.8)$ \kms.}.

%%code to generate plot below is jupyter notebook starkit_work/example stellar spec plot
%%use starkit environment
\begin{figure}
\centering
\includegraphics[width=\linewidth]{S0-14_S0-6_example_spec_with_lines.pdf}
\caption{
Example spectra and model fits for the early-type star S0-14 (top) and the late-type star S0-6 (bottom). \brg absorption is the primary spectral feature for early-type stars in the OSIRIS Kn3 filter, while the late-type stars have more features. The multiple absorption line features in late-type stars lead to increased precision for their radial velocity measurements compared to early-type stars.
\label{fig:example_spec}
}
\end{figure}

We apply this same technique to all the other stars in the OSIRIS data when a star's radial velocity can be measured. The number and quality of radial velocity measurements extracted for each epoch depends greatly on a number of factors, such as weather conditions, adaptive optics performance, and position in the OSIRIS dither pattern. Stellar crowding and confusion can also lead to difficulties when extracting a radial velocity measurement. Even though a star may be identified in an OSIRIS cube, its spectrum may not be of adequate quality to measure its radial velocity. We perform a quality inspection of the extracted spectra to ensure their radial velocities can be measured.

In this work, we report 626 new radial velocity measurements. To this, we add the 344 radial velocities from the literature \citep{Gillessen:2017fa, Do:2019}, the majority of which are for S0-2. As Figure \ref{fig:rv_semimajor} shows, the new RV measurements dramatically increase ($\sim$3 times) the coverage for other stars in this region, enabling the first binary star population study.

% increase in radial velocity points done in this work. (semimajor axis discussion - put in the next section where I describe orbit and polynomial fits?)

% From the listed observations observations in Table \ref{tab:all_spec_obs}
% (NEED TO INCORPORATE) In addition to the 695 RV measurements from reported observations in Table \ref{tab:all_spec_obs}, we also include 273 radial velocities from \citet{Gillessen:2017fa} for the sample stars.

%%rv_periodicity.semimajor_rv_plot()
\begin{figure}
\centering
\includegraphics[width=\linewidth]{semimajor_rvpoints_3-9-2023.pdf}
\caption{
Number of radial velocity points for a star versus its estimated semi-major axis. The number of previously published points is given by the unfilled gray circles, while the number of points used in this work is given in red. The dotted lines connect the points for visual increase. S0-2 is the star with the most radial velocity points because of its brightness and close proximity to the black hole. This work reports 626 new radial velocity measurements.
\label{fig:rv_semimajor}
}
\end{figure}

%\subsection{Imaging and Astrometric Data} \label{subsec:img_data}

%Information about the imaging observations taken with the NIRC2 instrument \citep{Matthews:1994} on the Keck Telescope are described in depth in \citet{Do:2019}, and references therein. \citet{Jia:2019, Gautam:2019} describe in depth the processes used to determine the stars\vtick \ astrometric positions and photometric magnitudes in the reduced imaging data. \citet{Jia:2019} also details the process of deriving the stellar positions for each epoch and aligning them to a common reference frame reported in \citet{Sakai:2019}. This process results in astrometric positions of each star in a common reference frame over the course of the observations. We use the astrometric positions reported in \citet{Do:2019}, which spans from 1995-2018.  


% \section{Sample Selection} \label{sec:sample}
% \subsection{Sample Selection} \label{subsec:sample}

% \citet{Chu:2018} investigated the star S0-2 as a binary star due to its precise radial velocity measurements, well known orbit, and to quantify the potential bias a spectroscopic binary would have on the relativistic redshift measured by \citet{Do:2019,GRAVITY:2018A}. This work expands this investigation to 29 stars in a systematic way. The investigation of S0-2's relativistic redshift required plentiful observations \citep{Do:2019,Hees:2019a} and add increased sensitivity to radial velocity variations for the other stars close to the SMBH.

% The broadest criterion of the star sample is that the star must be brighter than $K'$ $>$ 16 magnitude and located within the field of view of the primary instrument that has been used by the GCOI to study S0-2's RV variation. This field is centered on S0-2 and covers 3" $\times$ 2" at a PA of 285 degrees. Our magnitude limit stems from what can be measured with adequate signal-to-noise from a single night of observations. These initial criteria creates an intermediate sample of 62 stars (see Figure \ref{fig:S-star_OSIRIS}). From here, we make several other cuts. First,  Wolf-Rayet emission line sources (IRS16C and IRS16SW) are excluded, because measuring their radial velocities is complicated due to their stellar winds. We note, however, that IRS16SW is a known binary \citep{Ott:1999gn, Martins:2007fw}. Similarly, the 7 main-sequence O stars are not included since they are featureless across the primary filter used for the GCOI spectroscopic observations (Kn3, 2.121 -- 2.229 $\mu$m). We omit 23 stars confused in image and/or spectral space, meaning we cannot distinguish their spectra and therefore their radial velocities, and two early-type stars (S1-2 and S1-33) whose radial velocity measurements were biased by the background gas emission in the region at their locations. We also omit the star S0-28 because it only has 2 radial velocity measurements, which is too few to conduct a periodicity search. This leads to a final sample 28 stars, of which 16 are early-type stars and 12 are late-type stars and which are listed in Table \ref{tab:full_rv_sample} (and the omitted stars are summarized  in Table \ref{tab:excluded_stars}).  In addition to the 693 RV measurements from reported observations in Table \ref{tab:all_spec_obs}, we also include 273 radial velocities from \citet{Gillessen:2017fa} for the sample stars.

% (We also omit the star S0-28 because it only has 2 radial velocity measurements, which is too few to conduct a periodicity search for - move this to later when S1-2 and S1-33 are excluded) 

% \input{sample_cut_test_with_gautam_kp.tex}

%OSIRIS Kn3 35 mas central pointing, which has been described in numerous works such as \citet{Boehle:2016ko, Chu:2018, Do:2019}. The primary objective of this configuration was to monitor the star S0-2, and it places S0-2 at the center of the dither pattern. The OSIRIS field of view contains additional stars around S0-2, and we include these additional stars in our search. 

% Because we are focused on adequately sampling the parameter space of periods on the order of a few days, we do not combine measurements taken over multiple nights. Stars with $K'$ $>$ 16 magnitude do not achieve adequate signal-to-noise to measure their radial velocities for a single night (usually a half-night). Therefore, we only consider stars with $K'$ $<$ 16 mag. 


% Figure \ref{fig:S-star_OSIRIS} shows the sample of stars for the following analysis. It also shows examples of stars with $K'$ $<$ 16 that are omitted because they are emission line sources, featureless in Kn3, or confused with other sources. Stars are classified as early-type if they exhibit \brg absorption, which is consistent of main-sequence B-stars. Stars are classified as late-type if they possess absorption lines consistent with red giants, primarily Na I and CO \citep{Do:2013fn}. 


%\textcolor{red}{How to introduce/mention Yin et al in prep. work?}

% We report the radial velocities for these stars in Section \ref{sect:RV_tables}.


%%%%%%%%%%%%%%%%%%%%%%%%%%%%%%%%%%%%%%%%%%%%%%%%%% 
%Modeling the long term motion motion around SMBH
%%%%%%%%%%%%%%%%%%%%%%%%%%%%%%%%%%%%%%%%%%%%%%%%%% 

\section{Modeling the long-term motion around the SMBH} \label{sec:motion}

% \subsection{Producing Residual Radial Velocity Curves} \label{subsec:rv_resid}

Some of the stars in the selected sample have significant long-term motion from their orbits around the SMBH, which we model and remove as a necessary initial step for conducting a periodic search for binary stars. For the 10 stars with orbital periods around the SMBH of less than 180 years (a semi-major axis less than 24 mpc) and which have gone through a turning point during our observations, we have enough astrometry and radial velocities in the GCOI database to directly model their orbital motions. For the shortest period and best-studied star, S0-2, we use the reported orbital model in \citet{Do:2019}. This model also provides us with the black hole parameters (position, proper motion, mass, and distance to Earth), which we use as fixed values in our orbital model for the other short-period stars. For the short-period stars beyond S0-2 with measurable orbits, we also fix the astrometric correlation length from source confusion to 30 mas and the radial velocity offset between the Keck and VLT measurements to 0 \kms \citep[see][]{Do:2019,Ciurlo:2020Nature}. Leaving these values free has no impact on modeling the radial velocity curves or residuals. The seven modeled parameters for these stars are the six standard stellar orbital parameters (period, eccentricity, inclination, longitude of the ascending node, argument of periapse, epoch of closest approach) plus the astrometric mixing parameter. The star's semi-major axis and its uncertainty can also be obtained from its orbital period (and uncertainty) and black hole mass using Kepler's Third Law. Once the star's orbital fit is performed, a model for its radial velocity is generated and subtracted to create a residual curve. The last column of Table \ref{tab:full_rv_sample} provides the estimated semi-major axes for these short-period stars' motions around the SMBH.

% which includes a relativistic redshift and Romer time delay. These stars are not expected to show observable relativistic redshift effects, so we fix the relativistic redshift parameter to be true. We note that leaving the black hole parameters free did not have a significant effect on the radial velocity models of these stars. We do not fit for an offset and additive error between Keck and VLT data points. Finally, we fix an astrometric correlation length of 20 mas \citep[which comes from S0-2 in ][]{Do:2019} and fit for a mixing parameter. 

% While interesting, investigating the black hole and resulting orbital parameters favored by these stars are beyond the scope of this work. 

%visible motions due to their proximity to the SMBH. We are interested in studying their intrinsic motion to search for a potential periodic signal, which may be evidence for a spectroscopic binary. The methodology for subtracting out this motion depends on the star.

% Stars in subset 2 have significant accelerations in both astrometry and radial velocity due to their orbits around the SMBH. They have also been monitored for years and have had their orbits previously reported \citep[e.g.][]{Gillessen:2009fn,Gillessen:2017fa}. For these 10 stars, we perform an orbital fit of their motion around the SMBH to create an orbital model of their radial velocities. In this method, we performed a more robust fit to their radial velocities compared to a polynomial model. Stars in subset 1 do not possess enough kinematic information to constrain their orbital parameters, which is why their radial velocities are fit with a polynomial instead.

% Orbit stars have their motions fit in the same way as described in works such as \citet{Do:2019}, and we use the orbital solution reported in \citet{Do:2019} for S0-2. For stars other than S0-2, we make a few changes. We still fit these stars with Newtonian plus relativistic redshift model, which also accounts for the Romer time delay. These stars are not expected to show observable relativistic redshift effects, so we fix the relativistic redshift parameter to be true. 

% Additionally, these stars do not individually constrain the mass of the black hole ($M_{BH}$) or distance to the black hole ($R_{0}$), but these particular parameters are well constrained by S0-2 \citep[e.g.][]{GRAVITY:2018A,GRAVITY:2019a,Do:2019}. Therefore, fix these of $M_{BH}$ and $R_{0}$ to the values reported in \citet{Do:2019}, while leaving the other parameters free. 



% We fix the black hole parameters (mass, distance, position, and velocity) to the values reported in \citet{Do:2019}, which come from the S0-2 orbital fit. S0-2 has the greatest orbital phase coverage of this sample and provides strong constraints on these black hole parameters.

% An example of a star's radial velocity data, orbital model, and residual are shown in Figure \ref{fig:S0-2_rv_res}.

For stars with longer orbital periods around the black hole, we performed a polynomial fit to their radial velocities. The degree of the polynomial fit is determined by the F-test, where a higher degree polynomial must pass the F-test of the lower degree polynomial with a 95\% significance. Of the 18 stars fit with a polynomial, all but one are best fit with a constant radial velocity model, and one (S0-6) is well fit with a constant acceleration model. These polynomial fits are reported in Table \ref{tab:polynomial_table_test}. The polynomial fit is then subtracted from the radial velocity points to create residual points, which are reported in Table \ref{tab:S0-14_example_rv}. For these longer period stars, their semi-major axes are estimated using the same orbit fitting method described above. These estimates have formal uncertainties from 1-20\%, and the lower value is most likely an underestimate due to the small orbital phase coverage; we therefore assign a 20\% semi-major axis uncertainty for these stars. For three late-type stars on the edge of our sample, we were unable to obtain orbital solutions. To estimate their semi-major axes, we take the average of comparably large separations from the SMBH. These semi-major axes estimates are also reported in Table \ref{tab:full_rv_sample}.

% following assumptions: These stars are isotropically distributed, which appears the case for the S-star stars \citep[e.g.][]{Gillessen:2017fa}. The stars are also assumed to be at apoapse, since that is where stars spend the most amount of time on their orbits. With these assumptions, their semi-major axes are estimated from the following relationship: 

% take their projected separation (R2D$_{proj}$) from the SMBH and convert it to a physical separation (R2D) using the black hole distance from \citet{Do:2019}. The star's current $z$ distance is:

% \begin{equation}
%     z = \frac{R2D}{\sqrt{2}}
% \end{equation}

% From this $z$ value, we then assume that the star is at apoapse, where a star spends the most amount of time relative to other orbital phase positions. With this assumption, we get their semi-major axes using the following equation:

% \begin{equation} \label{eq:a_position}
%     a = \frac{\sqrt{1.5} \ R2D_{phys}}{(1 + e)},
% \end{equation}
% \noindent where $R2D_{phys}$ is the physical projected separation from the SMBH using their angular projected distance and Earth's distance to the SMBH of $7.9 \times 10^{3}$ pc from \citet{Do:2019}. \textcolor{red}{Statement on how apoapse assumption affects this.} \textcolor{red}{Statement on how eccentricity assumption affects this.} We also assume the stars have an $e = 0.27\pm0.07$ \citep{Yelda:2014by} The uncertainties in the semi-major axes for these stars come from the 0.07 uncertainty in $e$ and applying it to Equation \ref{eq:a_position}. 



% \subsection{Semi-major Axes for Stellar Sample} \label{subsec:semimajor}

% (Need to introduce the semi-major axis component/motivation)

% Gathering the semi-major axes of the stars enables us to consider important scientific implications. For the orbit star subset, the semi-major axes comes from the orbital fits. We take their orbital periods, the fixed black hole mass, and solve for the semi-major axes using Kepler's Third Law.



% Accelerations in astrometry were determined using the same methodology as listed in \citet{Jia:2019}. The determination of accelerations and polynomial degree of the radial velocity fits are reported in Table \ref{tab:full_rv_sample}. The sample of stars were then divided into two subsets:
% \begin{enumerate}
% 	\item Polynomial Stars
% 	\item Orbit Stars
% \end{enumerate}

% We began by determining which stars have significant accelerations in both radial velocity and astrometry. To determine significant accelerations along the line of sight, we performed a polynomial fit to the stars' radial velocities. 

\begin{deluxetable*}{lrrcc}
\tablecolumns{5}
\tablewidth{0pc} 
\tablecaption{Polynomial Fit Results \label{tab:polynomial_table_test}}
\tablehead{
\colhead{Star}	&
\colhead{$t_{0}$}   &
\colhead{$v_{z0}$}	&
\colhead{$a_{z}$\tna}	&
\colhead{$\chi^{2}_{red}$}  \\
% \colhead{Semimajor Axis}\\
\colhead{}	&
\colhead{(Epoch)}	&
\colhead{(\kms)}	&
\colhead{(\kms yr$^{-1}$)}	&
\colhead{}  
% \colhead{(mpc)}
}
\startdata
S0-7	&   2014.4978   &	105.3	$\pm 5.9	$		&	[6.9]				&	3.3		\\
S0-9	&   2014.3799   &	114.5	$\pm 4.8	$		&	[4.1]				&	2.1		\\
S0-11	&   2014.2696   &	-22.2	$\pm 4.6	$		&	[4.7]				&	1.5		\\
S0-14	&   2013.8235   &	-31.3	$\pm 2.8	$		&	[2.2]				&	1.0		\\
S0-15	&   2013.6498   &	-552.5	$\pm 4.8	$		&	[4.1]				&	2.9		\\
S0-31	&   2012.5771   &	-119	$\pm 11	$		&	[9.3]				&	1.7		\\
S1-8	&   2012.2612   &	-112.0	$\pm 7.2	$		&	[5.9]				&	1.9		\\
S0-6	&   2013.6093   &	90.38	$\pm 0.42	$		&	$0.83 \pm 0.12$	&	1.7			\\
S0-12	&   2013.9746   &	-39.26	$\pm 0.48	$		&	[0.44]				&	2.0		\\
S0-13	&   2013.9545   &	-45.11	$\pm 0.41	$		&	[0.35]				&	2.2		\\
S0-18	&   2014.4520   &	-289.3	$\pm 1.0	$		&	[1.0]				&	3.5		\\
S0-27	&   2013.8851   &	-121.2	$\pm 1.5	$		&	[1.3]				&	4.3		\\
S1-5	&   2014.4119   &	11.20	$\pm 0.50	$		&	[0.38]				&	1.9		\\
S1-6	&   2013.8535   &	-42.0	$\pm 1.5	$		&	[1.7]				&	1.1		\\
S1-10	&   2013.7375   &	-33.6	$\pm 1.0	$		&	[0.77]				&	1.4		\\
S1-13	&   2012.4453   &	-749.1	$\pm 1.6	$		&	[1.2]				&	6.8		\\
S1-15	&   2011.8983   &	-120.40	$\pm 0.82	$		&	[0.65]				&	2.1		\\
S1-31	&   2014.0749  	&	182.9	$\pm 1.6	$		&	[1.7]				&	1.9		\\
% S0-1	&     	&				&					&			&   0.024    \\
% S0-2	&     	&				&					&			&   0.004    \\
% S0-3	&     	&				&					&			&   0.014    \\
% S0-4	&     	&				&					&			&   0.016    \\
% S0-5	&     	&				&					&			&   0.011    \\
% S0-8	&     	&				&					&			&   0.017    \\
% S0-16	&     	&				&					&			&   0.012    \\
% S0-17	&     	&				&					&			&   0.014    \\
% S0-19	&     	&				&					&			&   0.012    \\
% S0-20	&     	&				&					&			&   0.010    \\
\enddata
\tablenotetext{a}{3$\sigma$ limits are given in brackets. The value is reported with its 1$\sigma$}
\tablecomments{Col 1: star name, Col 2: $t_0$ epoch from polynomial radial velocity fit, Col 3: constant velocity offset, Col 4: acceleration parameter, Col 5: $\chi^{2}_{red}$ of the fit.}
\end{deluxetable*}

\begin{deluxetable*}{ccrrrcc}
\tablecolumns{7}
\tablewidth{0pc}
\tablecaption{Radial Velocities and Residuals \label{tab:S0-14_example_rv}}
\tablehead{
\colhead{Epoch}	&
\colhead{MJD}   &
\colhead{RV}	&
\colhead{RV $\sigma$}	&
\colhead{Residual}  &
\colhead{Source}    &
\colhead{Reference}\\
\colhead{(Year)}	&
\colhead{}	&
\colhead{(\kms)}	&
\colhead{(\kms)}	&
\colhead{(\kms)}  &
\colhead{}  &
\colhead{}
}
\startdata
2006.461 & 53904.50 & -81 & 23 & -50 & S0-14 \\
2006.497 & 53917.50 & -46 & 16 & -15 & S0-14 \\
2007.384 & 54241.50 & -4 & 20 & 26 & S0-14 \\
... & ... & ... & ... & ... & ... \\
% 2007.545 & 54300.29 & -21.84 & 18.16 & 9.52 & Kn3 \\
% 2008.372 & 54602.50 & -48.53 & 14.32 & -17.17 & Kn3 \\
% 2008.563 & 54672.28 & -18.0 & 15.03 & 13.36 & Kn3 \\
% 2009.342 & 54956.50 & -23.44 & 14.17 & 7.92 & Kn3 \\
% 2009.344 & 54957.50 & -50.37 & 16.79 & -19.01 & Kn3 \\
% 2010.341 & 55321.50 & 40.45 & 21.81 & 71.81 & Kn3 \\
% 2010.349 & 55324.53 & 0.51 & 17.94 & 31.87 & Kn3 \\
% 2011.520 & 55752.33 & -58.99 & 17.9 & -27.63 & Kn3 \\
% 2011.545 & 55761.32 & -29.06 & 24.74 & 2.3 & Kn3 \\
% 2012.555 & 56130.31 & -30.03 & 18.33 & 1.33 & Kn3 \\
% 2013.366 & 56426.50 & -43.05 & 19.38 & -11.69 & Kn3 \\
% 2013.563 & 56498.32 & -28.11 & 18.23 & 3.25 & Kn3 \\
% 2013.566 & 56499.34 & -20.11 & 24.08 & 11.25 & Kn3 \\
% 2013.568 & 56500.31 & -36.82 & 16.3 & -5.46 & Kn3 \\
% 2013.607 & 56514.28 & -9.79 & 27.01 & 21.57 & Kn3 \\
% 2013.609 & 56515.31 & -63.57 & 27.14 & -32.21 & Kn3 \\
% 2013.615 & 56517.30 & -26.2 & 19.12 & 5.16 & Kn3 \\
% 2014.374 & 56794.52 & -50.5 & 19.73 & -19.14 & Kn3 \\
% 2014.376 & 56795.51 & -19.41 & 14.02 & 11.95 & Kn3 \\
% 2014.379 & 56796.51 & -31.23 & 15.24 & 0.13 & Kn3 \\
% 2014.387 & 56799.52 & -36.95 & 35.67 & -5.59 & Kn3 \\
% 2014.390 & 56800.53 & 1.65 & 35.77 & 33.01 & Kn3 \\
% 2014.502 & 56841.33 & -11.36 & 14.63 & 20.0 & Kn3 \\
% 2015.338 & 57146.55 & -40.43 & 32.27 & -9.07 & Kn3 \\
% 2015.551 & 57224.35 & -18.03 & 17.45 & 13.33 & Kn3 \\
% 2017.374 & 57890.51 & -49.65 & 13.03 & -18.29 & Kn3 \\
% 2017.377 & 57891.50 & -37.99 & 14.49 & -6.63 & Kn3 \\
% 2017.380 & 57892.50 & -51.31 & 18.59 & -19.95 & Kn3 \\
% 2017.568 & 57961.31 & -27.66 & 14.3 & 3.7 & Kn3 \\
% 2017.617 & 57979.29 & -33.71 & 13.53 & -2.35 & Kn3 \\
% 2018.207 & 58194.64 & -17.11 & 23.4 & 14.25 & Kn3 \\
% 2018.311 & 58232.57 & -35.01 & 22.63 & -3.65 & Kn3 \\
% 2018.363 & 58251.52 & -15.65 & 19.66 & 15.71 & NIFS \\
% 2018.390 & 58261.47 & -32.52 & 16.78 & -1.16 & Kn3 \\
% 2018.426 & 58274.45 & -98.74 & 136.22 & -67.38 & Kn3 \\
% 2018.554 & 58321.32 & -45.37 & 14.67 & -14.01 & Kn3 \\
% 2018.579 & 58330.32 & -31.29 & 15.03 & 0.07 & Kn3 \\
% 2018.609 & 58341.31 & -30.43 & 51.43 & 0.93 & Kn3 \\
\enddata
\tablecomments{A full electronic version will be published in the journal. Col 1: Julian Year, Col 2: modified Julian date, Col 3: radial velocity corrected to local standard of rest, Col 4: radial velocity uncertainty, Col 5: residual, Col 6: star, Col 7: if noted, radial velocity reported in following reference.}
\end{deluxetable*}


% For stars in subset 1, we use the favored polynomial fit to their radial velocities to account for their motion around the SMBH. We take the fit and subtract it from the data to produce a residual radial velocity curve. An example of a star fit with a polynomial, S0-14, is shown in Figure \ref{fig:S0-14_rv_res}. The favored polynomial fits for each star in the polynomial sample are reported in Table \ref{tab:polynomial_table_test}. Ultimately, 20 out of the 28 stars are part of the polynomial star sample, with only one star (S0-6) favoring a constant acceleration fit (1st order polynomial). The remaining 18 stars favor a constant velocity fit (0th order polynomial).

%%code is S014_plot, but can be modified for other stars with other polynomial fit

\begin{figure}
\centering
\includegraphics[width=\linewidth]{S0-14_rv_resdidual.pdf}
\caption{
Top: Radial velocity of S0-14 over time. A model fit of polynomial order 0 is fit to the data. Bottom: Difference of measured radial velocity points from the model for S0-14. Uncertainty of the model fit is incorporated into the shaded region.
\label{fig:S0-14_rv_res}
}
\end{figure}



% \begin{figure}
% \centering
% \includegraphics[width=\linewidth]{S0-2_rv_resdidual.pdf}
% \caption{
% Top: Radial velocity of S0-2 over time. An orbital model fit is fit to the data is shown by the line. Bottom: Difference of measured radial velocity points from the model for S0-2. Uncertainty of the orbital model fit is incorporated into the shaded region. This orbital fit was first reported in \citet{Do:2019}.
% \label{fig:S0-2_rv_res}
% }
% \end{figure}



% \begin{itemize}
%     \item semi-major axis comes from the orbit fit. (for most complete description: Take the given period value from fit then use Kepler's law to get semi-major axis. Mbh is fixed for the fit, already as described in section above)
%     \item Polynomial sample: take the R2D, convert this to a physical distance assuming R0 from S0-2 fit. Then we assume z distance is R2D / sqrt(2). Finally, by making the assumption that most stars have an e = 0.3 and that most stars are at appopase, we can assume a = z / 1.3
% \end{itemize}

%%%%%%%%%%%%%%%%%%%%%%%%%%%%%%%%%%%%%%%%%%%%%%%%%% 
%PERIODICITY SEARCH
%%%%%%%%%%%%%%%%%%%%%%%%%%%%%%%%%%%%%%%%%%%%%%%%%% 

\section{Companion Star Searches} \label{sec:per_search}

% \input{sstar_binary_paper/Sections/periodicity_search}

The following section describes the process for detecting periodic signals in the radial velocity residual curves using two types of periodicity searches: a Lomb-Scargle analysis \citep{Lomb:1976,Scargle:1982eu,VanderPlas:2018} and a Bayesian fit for potential binary systems. A similar methodology was done for the star S0-2 in \citet{Chu:2018}. The Lomb-Scargle analysis provides a computationally efficient method for detecting periodic signals in unevenly spaced data. The Bayesian fitting method provides a more complete and robust approach and allows us to derive upper limits on the orbital parameters of hypothetical binary companions to these S-stars.

% The results of these two processes are summarized in Section \ref{subsec:per_results}.

\subsection{Lomb-Scargle Analysis} \label{subsec:lombscargle}

Once a star's residual radial velocity curve is made, it is run through a Lomb-Scargle package, gatspy \citep{VanderPlas:2015,gatspy:2016} to search for a periodic signal. When running this periodic search, a range of periods from 2 to 10000 days are sampled. The lower period sampling limit of 2 day comes from the Nyquist sampling limit of taking data on consecutive days. The upper period limit of 10000 days ensures that the entire time baseline of the dataset is covered. To ensure that potential signals are not missed due to uniform sampling, the ``$N$ samples per peak''feature of the gatspy package is used. We specify $N=10$ samples per peak, which carries out 10 additional, finer samples around a peak in Lomb-Scargle power \citep[see][for details]{VanderPlas:2015,gatspy:2016}. The periods are uniformly spaced at $1/N\Delta t$, where $N$ is the samples per peak and $\Delta t$ is the maximum time baseline of observations. Spacing the sampled periods at $1/N\Delta t$ ensures proper sampling of a dataset \citep{VanderPlas:2018}. \citet{Chu:2018} calculated an upper limit of 119 days for the longest period for an S0-2 binary star system, as systems with longer periods would separated at S0-2's closest approach to the SMBH. The stars in this sample are not expected to pass as close to the SMBH, hence would have longer maximum periods, which is why we decided to increase the upper period sampled. After this step, a Lomb-Scargle power spectrum is obtained, containing power values for every sampled period. An example Lomb-Scargle periodogram for S0-14 is shown in Figure \ref{fig:LS_S0-14}. Periodograms for the entire sample are given in Appendix \ref{sect:ls_k_results}

The significance of the peak Lomb-Scargle powers are determined in three ways: i) comparing the Lomg-Scargle amplitude to the star's radial velocity uncertainty ii) Monte Carlo simulation and iii) bootstrap false alarm probability (FAP) test significance. Each of these methods produces a significance value between 0 to 100\%, with the higher percentage corresponding to a higher significance. These three methods compliment one another, and a higher significance value means a higher likelihood that a binary system as been detected.

One way to evaluate the significance of the Lomb-Scarlge power for each star is to look at the fit amplitude from the Lomb-Scargle model. The Lomb-Scargle analysis returns an amplitude of the sinusoid model fit to the residual curve. This amplitude is divided by the median radial velocity uncertainty for that star to determine the amplitude significance in terms of sigma. If a star's residual has a very large amplitude of variation relative to its radial velocity uncertainty, the variation can be considered significant relative to noise.

The second way to determine significance is to use a Monte Carlo simulation. This Monte Carlo simulation is conducted in the same way as described in \citet{Chu:2018}. To summarize, 100,000 simulated residual curves with no periodic signal are generated for each star. Each simulated curve has the same observation times and uncertainties as the data. Every data point is drawn from a Gaussian distribution centered around 0 \kms, meaning the simulated data contained no periodic signal and only noise. These simulated curves are then run through the Lomb-Scargle process described above, and the maximum Lomb-Scargle power for each run is recorded. A cumulative distribution function for the 100,000 simulations is compiled. The peak Lomb-Scargle power from the data is then compared to the cumulative distribution function. This approach allows us to quantify the significance of our peak signal relative to a non-periodic data set taken at the same observation sampling and uncertainties.

%%a power value corresponding to the 3$\sigma$ confidence level is determined, and this value is compared to the power values from the residual curve. We define a significant detection to be anything greater than the 3$\sigma$ confidence level.

%%rv_periodicity.plot_S014_for_paper()
\begin{figure}
\centering
\includegraphics[width=\linewidth]{S0-14_ls_kamp_for_paper.pdf}
\caption{
Top: Lomb-Scargle periodogram of S0-14's residual radial velocity curve. Every period value sample has a corresponding Lomb-Scargle power. The significance of the peak power value is then calculated to determine if this period represents a detection. Bottom: 95\% upper confidence limit on the amplitude of RV variations induced by a binary system ($K$) as a function of the binary orbital period for S0-14.
\label{fig:LS_S0-14}
}
\end{figure}

%%May be worth just using the S0-14 periodogram without any significance - why highlight one but not another?
% \begin{figure}
% \centering
% \includegraphics[width=\linewidth]{S0-14_deg0_lombscargle_no_limit.pdf}
% \caption{
% Lomb-Scargle periodogram of S0-14's residual radial velocity curve. Every period value sample has a corresponding Lomb-Scargle power. The significance of the peak power value is then calculated to determine if this period represents a detection. 
% \label{fig:LS_S0-14}
% }
% \end{figure}

% \begin{figure}
% \centering
% \includegraphics[width=\linewidth]{S0-2_instrument_offset_fit_lombscargle_with_95limit.pdf}
% \caption{
% Lomb-Scargle periodogram of S0-2's residual radial velocity curve. The black dash-dotted line is the 95\% confidence level detection value. No power reaches the 95\% confidence level detection value implying that no significant periodic signal is found in the observations.
% \label{fig:LS_S0-2}
% }
% \end{figure}

The bootstrap false alarm probability (FAP) is another way to test the significance of the signal in a slightly different view than the Monte Carlo simulations. \citet{VanderPlas:2018} explains how the FAP addresses the probability that a signal with no periodic component would lead to a peak of a given magnitude. We choose the bootstrap algorithm method in gatspy because it is the most robust estimate specifically of the FAP compared to other given methods in gatspy \citep[see][for details]{VanderPlas:2018}. A similar bootstrap method was implemented by \citet{Gautam:2019} to determine the FAP with Galactic center photometry data. To conduct this test, 10,000 Lomb-Scargle periodograms are simulated. Each of these periodograms is obtained by keeping the observation times and by drawing residual values randomly with replacement from the residual curve. The maxima of each resulting periodogram are computed. The peak periodogram power values from the data are compared to the distribution of power values from the bootstrap to determine the FAP. The bootstrap false alarm test significance is then defined as 1 - FAP and reported. This way, a higher value of 1 - FAP (a lower FAP value) corresponds to a higher significance of a binary.

None of the stars' residual curves have periodic variations beyond the 3$\sigma$ (99.7\%) confidence limit using all three methods. While there is some variation in the significance values for some stars, the main importance is how the star performs in all three tests. A true binary system is expected to show high significance with all three tests. No star exhibits high significance across all methods, suggesting that no binary stars have been detected. The detailed results from the Lomb-Scargle periodicity search are presented in Table \ref{tab:periodicity_results}. 

% The average test significance of the three tests for each star in the sample are calculated, and none of the stars have significant periodic signals.

%and the histogram of these values are shown in Figure \ref{fig:period_hist}. This histogram shows that none of the stars have significant periodic signals.

\begin{deluxetable*}{llccccc}
\tablecolumns{7}
\tablewidth{0pc} 
\tablecaption{Companion Star Search Results\label{tab:periodicity_results}}
\tablehead{
\colhead{Star}	&
\colhead{Spectral}	&
\colhead{LS Amp}	&
\colhead{Monte Carlo}	&
\colhead{Bootstrap False Alarm}	&
\colhead{Has Detected}	&
\colhead{$K$ Limit}	\\
\colhead{}	&
\colhead{Type}	&
\colhead{Significance\tna (\%)}	&
\colhead{Significance\tna (\%)}	&
\colhead{Significance\tna (\%)}	&
\colhead{Companion}	&
\colhead{(\kms)}
}
\startdata
S0-1 & Early & 89.64 & 92.34 & 99.94 & No & 44.2 \\
S0-2 & Early & 69.30 & 72.00 & 80.32 & No & 8.5 \\
S0-3 & Early & 75.80 & 94.70 & 90.03 & No & 23.9 \\
S0-4 & Early & 82.73 & 92.50 & 98.66 & No & 56.4 \\
S0-5 & Early & 65.42 & 94.88 & 98.52 & No & 38.4 \\
S0-7 & Early & 96.11 & 93.80 & 92.38 & No & 62.9 \\
S0-8 & Early & 75.54 & 96.40 & 99.79 & No & 50.7 \\
S0-9 & Early & 85.06 & 95.90 & 92.38 & No & 41.5 \\
S0-11 & Early & 79.26 & 90.20 & 86.80 & No & 38.7 \\
S0-14 & Early & 82.03 & 98.60 & 70.99 & No & 15.7 \\
S0-15 & Early & 85.95 & 93.50 & 33.37 & No & 56.5 \\
S0-16 & Early & 90.52 & 67.00 & 99.27 & No & 82.4 \\
S0-19 & Early & 56.10 & 64.00 & 99.73 & No & 50.1 \\
S0-20 & Early & 67.17 & 96.60 & 52.88 & No & 90.4 \\
S0-31 & Early & 79.38 & 100.00 & 42.34 & No & 75.7 \\
S0-6 & Late & 76.83 & 97.18 & 44.18 & No & 3.1 \\
S0-12 & Late & 85.69 & 99.00 & 94.57 & No & 3.9 \\
S0-13 & Late & 84.13 & 90.50 & 86.34 & No & 3.4 \\
S0-17 & Late & 60.64 & 75.00 & 99.90 & No & 21.1 \\
S0-18 & Late & 99.57 & 97.80 & 73.03 & No & 21.1 \\
S0-27 & Late & 97.65 & 90.30 & 52.30 & No & 20.5 \\
S1-5 & Late & 90.88 & 92.70 & 4.47 & No & 4.75 \\
S1-6 & Late & 95.54 & 96.30 & 63.66 & No & 11.7 \\
S1-8 & Early & 96.92 & 90.40 & 29.82 & No & 59.2 \\
S1-10 & Late & 85.08 & 99.00 & 95.40 & No & 6.9 \\
S1-13 & Late & 99.74 & 100.00 & 45.61 & No & 41.4 \\
S1-15 & Late & 92.29 & 93.20 & 46.69 & No & 7.9 \\
S1-31 & Late & 88.97 & 82.50 & 51.89 & No & 17.7 \\
\enddata
\tablenotetext{a}{For the period with maximum Lomb-Scargle power}
\tablecomments{Col 1: star name, Col 2: spectral type, Col 3: Lomb-Scargle model amplitude significance, Col 4: Monte Carlo simulation significance, Col 5: bootstrap false alarm significance, Col 6: average significance from columns 3-5, Col 7: $K$ amplitude limit.}
\end{deluxetable*}

% \begin{figure}
% \centering
% \includegraphics[width=\linewidth]{periodicity_histogram.pdf}
% \caption{
% Histogram of the average test significance percentage of the three periodicity tests for the stars in the sample. A star has a significant periodic detection if the significance values for all tests are greater than 3$\sigma$ (99.7\%). This region is highlighted in green. No stars meet the criteria for significant periodicity.
% \label{fig:period_hist}
% }
% \end{figure}

\subsection{Binary Curve Fitting} \label{subsec:binary_fit}

Another approach to search for a companion is through a Bayesian fit of the residual curve to the binary system curve. This is the same as the method described in \citet{Chu:2018}. The residual curves are fit with a binary star radial velocity model plus a constant. The following equation was used to model the radial velocity curve of an eccentric binary system~\citep{Hilditch:2001aa}
\begin{equation}
	RV = K \frac{\sqrt{1 - e^{2}}\cos{E}\cos{\omega} - \sin{E}\sin{\omega}}{1 - e \cos{E}}\ + O,
	\label{eq:eccentric_rv}
\end{equation}
with
\begin{equation}
	K = \frac{2\pi a\sin{i}}{P}\, ,
	\label{eq:Kdef}
\end{equation}
and where $e$ is the binary eccentricity, $\omega$ the argument of periastron, $E$ the eccentric anomaly determined by solving the Kepler equation, $i$ the inclination, $P$ the period  and $a$ the semi-major axis. This model is parametrized using the following 5 variables: the constant offset $O$, the radial velocity amplitude $K$, the eccentricity $e$, the argument of periastron $\omega$ and the mean longitude at J2000 (noted $L_0$). The use of the mean longitude at J2000 is preferred to the usual time of closest approach which is not bounded and not defined in case of circular orbits~\citep{Hilditch:2001aa}. For different fixed binary orbital periods $P$, this model is fit to the radial velocity residuals using a MultiNest sampler~\citep{Feroz:2008fi,Feroz:2009aa,Feroz:2013aa}. A strong periodic signal at a given period would lead to a large, peaked value of $K$ in the posterior. This method takes into account parameters such as eccentricity, which changes the shape of the curve from a perfect sinusoid wave. Periods from 2 to 500 days are uniformly sampled in log space. For S0-2, we followed the same methodology as \citet{Chu:2018}, where we evenly spaced at 0.05 days for periods from 2 to 150 days since periods beyond 119 days are excluded by the binary stability criteria. Because of the more computationally expensive nature of this method, we did not sample periods as long as the Lomb-Scargle method. An example output of this methodology is shown for S0-14 in Figure \ref{fig:LS_S0-14} and the complete set of the $K$ amplitude limit figures for the full sample is provided in Appendix \ref{sect:ls_k_results}.

% It should be noted that we did not detect any significant periods with the Lomb-Scargle method. 

% \begin{figure}
% \centering
% \includegraphics[width=\linewidth]{S0-14_deg0_period_kamp.pdf}
% \caption{
% 95\% upper confidence limit on the amplitude of RV variations induced by a binary system ($K$) as a function of the binary orbital period for S0-14.
% \label{fig:S014_K}
% }
% \end{figure}

After calculating $K$ upper limits for every sampled period, the median $K$ value is then taken as a summary upper limit for the star. The limits on $K$ amplitude are reported in Table \ref{tab:periodicity_results}. The $K$ amplitude results for the sample are also shown in Figure \ref{fig:LS_S0-14}. The $K$ amplitude limits can be used to derive hypothetical companion mass limits and are reported in Appendix \ref{sec:comp_mass_limits}. We do not report any detection of a binary system from this method, and these limits reflect our sensitivities to detecting binaries.

% \begin{figure}
% \centering
% \includegraphics[width=\linewidth]{S0-2_instrument_offset_fit_period_kamp.pdf}
% \caption{95\% upper confidence limit on the amplitude of RV variations induced by a binary system ($K$) as a function of the binary orbital period.
% \label{fig:S02_K}
% }
% \end{figure}

% \subsection{Results from the Companion Star Search} \label{subsec:per_results}

% he Monte Carlo method, nor do they have Bootstrap False Alarm significance greater than 3$\sigma$, as were . Additionally, we look at the fit amplitude from the Lomb-Scargle analysis of the highest-power period for each star. We take the amplitude of the sinusoid fit and divide it by the median RV uncertainty for that star to determine the amplitude significance in terms of sigma. Ultimately, none of these stars have model amplitudes greater than 3$\sigma$. 

%We plot these significance values in Figure \ref{fig:period_hist}.

% \begin{figure}
% \centering
% \includegraphics[width=\linewidth]{mc_falsealarm_zoom.pdf}
% \caption{
% Significance of Monte Carlo simulation and significance of bootstrap false alarm test, focusing on stars with the highest significance. Stars shown in this diagram have highest values for both significance values. A star is considered periodic if the significance values are greater than 3$\sigma$ for both tests, a region shown by the green shaded square. No stars meet the criteria for significant periodicity.
% \label{fig:MC_FA}
% }
% \end{figure}

% \begin{figure}
% \centering
% \includegraphics[width=\linewidth]{amplitude_falsealarm_zoom.pdf}
% \caption{
% Significance of Lomb-Scargle fit amplitude divided by median RV error and significance of bootstrap false alarm test, focusing on stars with the highest significance. Stars shown in this diagram have highest values for both significance values. A star is considered periodic if the significance values are greater than 3$\sigma$ for both tests, a region shown by the green shaded square. No stars meet the criteria for significant periodicity.
% \label{fig:Amp_FA}
% }
% \end{figure}

% Because we found no periodic signal that passed our criteria, we conclude that the S-stars in the sample do not contain detectable signals of being part of a spectroscopic binary system. 

% \begin{figure}
% \centering
% \includegraphics[width=\linewidth]{kp_kamp.pdf}
% \caption{
% Median $K$ amplitude value plotted with the $K'$ magnitude for each star, color coded by their spectral types. The median $K$ amplitude value comes from marginalizing the $K$ amplitude limits over all sampled periods.
% \label{fig:Kp_Kamp}
% }
% \end{figure}

\begin{figure}
\centering
\includegraphics[width=\linewidth]{kamp_rvsig.pdf}
\caption{
Median radial velocity uncertainty plotted with the $K$ amplitude limit value for each star, color coded by their spectral types. The $K$ amplitude value comes from marginalizing the $K$ amplitude limits over all sampled periods. The dashed lines represent the median values of the limits and radial velocity uncertainties for both the early-type and late-type stars.
\label{fig:Kamp_RVsig}
}
\end{figure}




%%%%%%%%%%%%%%%%%%%%%%%%%%%%%%%%%%%%%%%%%%%%%%%%%% 
%DERIVING THE LIMITS
%%%%%%%%%%%%%%%%%%%%%%%%%%%%%%%%%%%%%%%%%%%%%%%%%%

\section{Binary Star Fraction Limits} \label{sec:derive_limits}

% \section{Discussion} \label{sec:discussion}

% We provide some comments on the growth conditions which constituted the majority of our analysis in sections \ref{sec:Hmixing} and \ref{sec:Hsigma}. In the simplest cases of Lemma \ref{lemma:unstableGrowth}, growth was established in an analogous fashion to the old one-step expansion condition (\ref{eq:oldOneStepExpansion}), finding the relevant Jacobians $M_j$ and checking that their expansion factors $K(M_j)$ satisfy
\begin{equation}
    \label{eq:discussionOneStep}
    \sum_j \frac{1}{K(M_j)} <1.
\end{equation}
For the more complicated cases, the inductive method used to establish growth near the accumulation points in Lemma \ref{lemma:unstableGrowth} and the weakened one-step expansion condition (\ref{eq:oneStep}) both address the same fundamental issue: the splitting of unstable curves by singularities into an unbounded number of small components. They circumvent this obstacle in rather different ways, however. While (\ref{eq:oneStep}) generalises (\ref{eq:discussionOneStep}) to ensure an growth of unstable curves `on average' (see \cite{chernov_statistical_2009} for a precise statement), our inductive method is a more direct adaptation of (\ref{eq:discussionOneStep}), using it to generate contradictory geometric conditions which a hypothetical non-growing unstable curve must satisfy. It may be possible to prove Theorem \ref{sec:Hmixing} using (\ref{eq:oneStep}) as the basis for growth. Since we required (\ref{eq:oneStep}) anyway for proving Theorem \ref{thm:HsigmaExp}, this could potentially condense our analysis, but only to a minor extent. A convenience of the method used in section \ref{sec:Hmixing} is that, by way of the `simple intersection' property, it naturally gives geometric information on the images of manifolds, useful for proving the property \textbf{(M)} of Theorem \ref{thm:katok-strelcyn}.

We expect that essentially analogous analysis can be applied to establish mixing properties in a wide class of piecewise linear non-uniformly hyperbolic maps, including those (like the OTM) which sit on the boundary of ergodicity and beyond. While we have relied on the precise partition structure of $H_\sigma$, its fundamental feature (self-similar sequences of elements $A^k$, sharing boundaries with its neighbours $A^{k-1},A^{k+1}$ and accumulating onto some point $p$) is quite typical to return map systems. See, for example, those of various stadium billiards \cite{chernov_chaotic_2006,chernov_improved_2008,chernov_statistical_2009} and LTMs \cite{springham_polynomial_2014}. Indeed, the same method can be used to prove the Bernoulli property for non-monotonic LTMs \cite{myers_hill_mixing_2022}, where monotonicity of the manifold images cannot be assumed and the classical argument \cite{sturman_mathematical_2006} fails. The OTM is the pointwise limit of these maps as the boundary shrinks to null measure. It further has utility in proving growth conditions for maps which are uniformly hyperbolic but possess regions $A_j$ where the hyperbolicity is very weak, signified by $K(M_j) \approx 1$, so that (\ref{eq:discussionOneStep}) fails. Typically this leads to suboptimal bounds on mixing windows, see e.g. \cite{wojtkowski_model_1981,przytycki_ergodicity_1983,myers_hill_family_2022}. The map $H_{(\eta,\eta)}$ for $\eta \approx 1/2$ is another example, possessing weak hyperbolicity over $A_2, A_3$. Letting $\varepsilon = |\eta-1/2|>0$, there is an upper bound $N = N(\varepsilon)$ on escape times from the intersections $A_2\cap \sigma, A_3 \cap \sigma$. The growth lemma then follows by applying the inductive step roughly $N$ times and can be established for arbitrarily small $\varepsilon$, opening the door to establishing optimal mixing windows.

The above gives two examples of piecewise linear perturbations to $H$ where mixing with respect to Lebesgue is preserved and our methods can be applied. Nonlinear perturbations to the shear profiles complicate the analysis in several ways. Firstly as the map's Jacobians takes on a broader range of values, cone invariance becomes an increasingly harder condition to establish. Cones must be widened, giving looser bounds on expansion factors, which may already be weak due to new regions of weaker stretching. This, together with the change from polygonal to curvilinear return time partition elements and nonlinear local manifolds, adds some complexity to showing growth conditions. This does not rule out certain (small) nonlinear perturbations however. There is some leeway in the inequalities which govern cone invariance and growth of local manifolds, the latter of which is not too dissimilar from the piecewise linear setting (see Lemmas \ref{lemma:piecewiseApprox}, \ref{lemma:componentLength}). Certain small perturbations would not alter the \emph{topological} structure of the return time partition, i.e. which elements share boundaries, the key information needed for setting up the induction. Finally while the partition elements would no longer be polygonal, only coarse geometric information is required for verifying each inductive step. Following the above, a potential perturbation could be to replace the linear portions of each shear by a cubic, perturbing the tent profile
\[  f(t) = \begin{cases} 2t & 0 \leq t \leq 1/2, \\ 2(1-t) & 1/2 \leq t \leq 1 ,\end{cases} \]
of the OTM shears to
\[  f_a(t) = \begin{cases} \frac{1}{8} t \left(16 - a + 6at - 8at^{2} \right) & 0 \leq t \leq 1/2, \\ \frac{1}{8}\left(1-t\right)\left( 16 - a + 6a\left(1-t\right) - 8a\left(1-t\right)^{2}\right)  & 1/2 \leq t \leq 1, \end{cases}   \]
for $a>0$. For small enough $a$ the gradient range $f'(t)$ is restricted to small neighbourhoods of $\{ 2, -2\}$ and the escape time partition retains a similar structure. We illustrate this in Figure \ref{fig:perturbations}, showing escapes from the square $S_3$ under the map $G \circ F$, equivalent to escapes from the perturbed $A_3$ under the $G \circ F$, but with a cleaner geometry for comparison. When $a$ is too large the analogy to the OTM breaks down. At $a=16$ the map is twice differentiable everywhere and features a new source of slowed mixing, the Jacobian is the identity at the corner points $x,y \in \{  0, 1/2 \}$ giving locally parabolic behaviour (visible in the escape time partition). 

\begin{figure}
    \centering
    \includegraphics[width=0.24 \linewidth]{0.png}
    \includegraphics[width=0.24 \linewidth]{4.png}
    \includegraphics[width=0.24 \linewidth]{8.png}
    \includegraphics[width=0.24 \linewidth]{16.png}
    \caption{Partition of escape times from $S_3$ under the mapping $F \circ G$ for $a= 0,4,8,16$. }
    \label{fig:perturbations}
\end{figure}

% \subsection{Placing the Lack of Detections into Context - Early-Type Stars} \label{subsec:placing_limits_young}

Performing this systematic search for spectroscopic binaries has yielded no candidates, and we can use this result to place limits on the intrinsic binary population. To do this, one needs to make assumptions about the underlying binary star population. For the young, massive stars, we make use of the \citet{Sana:2012} distributions of binary system parameters (mass ratios $q$, eccentricities, periods). For the late-type stars, which are expected to be around 1\msun, we pull from the distributions reported by \citet{Raghavan:2010}. These distributions are used to create an initial estimate of the $K$ amplitude distributions for both the massive star and solar mass star binary populations and we later explore variations in Appendix \ref{sec:period_simulations}, which shows no impact for the early-type stars and a very modest impact for the late-type stars. 
% The goal is to create $K$ amplitude distributions for both the massive star and solar mass star binary populations. 

% (e.g., Binney & Tremaine, 2002; Ross et al. 2020; see latter for unbinding of eccentric binaries) truncate the binaries separation's distribution leaving a peak at ~10au (see Stephan et al. 2016;2019). Furthermore, stability and the Hills process tend to truncate the distribution in a similar manner.  [Devin, (1) I assume that this is what you did - or that the original distribution was modulated in a different way - sorry I don't remember the details - even if not, if you assumed Sana's distribution it's fine because it favors short-period binaries, see below (2) the two previous sentences can be combined together or written better, sorry this is a "first draft" kind of a thing.] Thus, it is often estimated as log-normal in the literature (e.g., Fragione et al. 2019 [a great opportunity to cite more of his papers ]).  Note that Sana et al. 2012 distribution, adopted here, favors short-period distribution, consistent with the unbinding processes (as noted in Hoang et al. 2018). [see even if you didn't take those processes into account it wouldn't matter so much; this was shown by Bao-Minh].  Further, the eccentricity distribution was adopted as a uniform distribution [right?] following Raghavan et al. (2010) which seems to be consistent with frequent dynamical encounters (Geller et al. 2019). Overall we do not expect that the distribution will affect our results as a test we... [here you can maybe outline the difference of having log-normal Vs Sana (with the relevant cut-off) I wouldn't suggested Kroupa 1995]

Parameters are drawn from the given distributions of log P, $e$, and mass ratio $q$ from the given distributions. This is done 100,000 times to create a population of 100,000 binary systems. Using the binary mass equation:

\begin{equation}
	M_{\text{comp}}\sin{i}  = \left(\frac{PM_{tot}^{2}}{2\pi G}\right)^{1/3} \ K\, ,
	\label{eq:Binary_mass_s-star_paper}
\end{equation}

\noindent and inserting the drawn parameters, a distribution of $K$ amplitudes are calculated for this simulated binary star population. When generating a binary system, we also make sure that the system does not result in a merger by calculating the minimum separation and ensuring it does not fall below the radius of the star ($\sim$ 6 R$_{\odot}$). With these simulated distributions for the two populations (see Figure \ref{fig:simulated_k_dist}), we can then use our $K$ amplitude limits - and zero detections - to derive their binary fractions. 

% The same process is done for the solar mass star population, except that parameters are drawn from \citet{Raghavan:2010}, with the exception of the period distribution. Instead, we use the same binary separation distribution as \citet{Stephan:2016eh,Stephan:2019}, which peaks around 10 AU and truncates at 100 AU to account for the unbinding processes at the Galactic center. We note that the uniform eccentricity distribution following \citet{Raghavan:2010} seems to be consistent with frequent dynamical encounters \citep{Geller2019ApJ}.

%The entire period distribution for this analysis for consistency in approach with the massive stars, but it is incredibly unlikely that solar mass binary systems with these long periods would have survived at the GC.
% which focused on systems around a solar mass (see Figure \ref{fig:binary_simulation_parameters_late}).

\begin{figure}
\centering
\includegraphics[width=\linewidth]{joint_k_dist_logx.pdf}
\caption{
Top: Normalized $K$ amplitude distributions for the simulated massive star binary population using \citet{Sana:2012} parameters, along with the median $K$ amplitude limit from the early-type sample (blue dotted line). Bottom: Normalized $K$ amplitude distributions for the simulated solar mass binary population using \citet{Raghavan:2010} parameters, along with the median $K$ amplitude limit from the late-type sample (orange dotted line). While the late-type sample has smaller $K$ amplitude limits, they remain higher than the distribution of $K$ amplitudes for the corresponding simulated binary star population.
\label{fig:simulated_k_dist}
}
\end{figure}

The calculated $K$ amplitude distributions for massive and solar mass stars is for a population made completely of binaries (a binary fraction of 100\%). To make $K$ amplitude distributions for binary fractions less than 100\%, the corresponding percentage of $K$ values are replaced with 0 \kms, representing the single star population. For example, a population with a binary fraction of 50\% will have 50,000 values of 0 \kms, and 50,000 values randomly drawn from the original simulated distribution. $K$ amplitude distributions for different populations with binary fractions ranging from 10 - 100\%, spaced evenly at 10\%, are created. We also conduct finer sampling at binary fractions between 30-50\%.

Once the adjusted $K$ amplitude distribution is established, a simulation is run to determine how many simulated binary star systems would be detected based on our $K$ amplitude limits. For the early-type stars, the $K$ limit from each of our 16 stars in Table \ref{tab:periodicity_results} are compared to a randomly drawn $K$ value from our massive star distribution adjusted for binary fraction. If the drawn $K$ value from the simulated population is higher than the limit from the sample star, we consider it a detection. For each simulation for the massive star population, there can be a minimum of zero detections and a maximum of 16 detections. This simulation is repeated 100,000 times, for each different $K$ amplitude distribution adjusted for binary fraction. The same process is done for the late-type stars using the 12 late-type stars and solar mass $K$ amplitude distributions.

The fraction of simulations with zero detections for each adjusted $K$ amplitude distribution are shown in Figure \ref{fig:binary_compare}. For a massive star population with a 47\% binary fraction, 5\% of the of simulations yielded zero detections. Based on this simulation and our zero binary detections, we can exclude a binary fraction greater than 47\% for this population with a 95\% confidence limit. For the solar mass star populations, a constraint cannot be obtained, with even a 100\% binary fraction only excluded at a 70\% confidence limit.

\begin{figure}
\centering
\includegraphics[width=\linewidth]{detection_population_compare_v3.pdf}
\caption{
The simulated binary fraction populations versus the fraction of Monte Carlo simulations with zero detections for each population of binary fractions, for both early-type and late-type stars. The binary fraction where we can place an upper limit at 95\% confidence is the black arrow. The early-type star binary fraction limit is 47\%. We cannot place a limit for the late-type stars. Our constraining power lies with the early-type stars, as the $K$ amplitude distribution contains higher values of $K$ compared to the late-type distribution.
\label{fig:binary_compare}
}
\end{figure}

% Even for binary fractions of 100\%, around 30\% of the simulations had 0 detections.

%%%%%%%%%%%%%%%%%%%%%%%%%%%%%%%%%%%%%%%%%%%%%%%%%% 
%DISCUSSION
%%%%%%%%%%%%%%%%%%%%%%%%%%%%%%%%%%%%%%%%%%%%%%%%%%

\section{Discussion} \label{sec:discussion}

% \begin{itemize}
%     \item Young star limit
%     \item Old star limit and discussion
%     \item Merger implications
%     \item Hypervelocity
%     \item Other star formation items
% \end{itemize}

Our simulations have enabled us to place a limit on the young star population binary fraction at 47\% (with 95\% confidence). This is well below the binary fraction ($70\pm 9\%$) for massive stars larger galactic radii \citep{Sana:2012}. \citet{Stephan:2016eh} have estimated the decrease in the binary star fraction from evaporation and mergers via three-body interactions with the central black hole through the eccentric Kozai-Lidov effect. Figure \ref{fig:stephan_compare} shows their simulation results at an age of 6 Myr, which we normalize to the observed binary star fraction of 70\% at large radii (shaded region). These predictions are consistent with our observations. We also note that the eclipsing binary fraction of stars outside the central arcsecond ($\sim 0.4$ pc) reported by \citet{Gautam:2019} is consistent with the field star binary fraction. The low binary fraction within $\sim$20 mpc appears to be well-explained by a scenario in which the central SMBH drives binary star mergers near its proximity. The process has important implications for the production of gravitational wave sources \citep{LIGO2016_offical}. Additional observations will further improve limits on the multiplicity of these stars closest to the SMBH.

% (HOW TO INCORPORATE INTO ABSTRACT?)
This result of a low binary fraction is also consistent with the binary star disruption mechanism. In this evolution mechanism, a binary star system is tidally disrupted by the SMBH, leaving one single component bound to the SMBH \citep[e.g.][]{Hills:1988br,Perets:2007fo, Fragione2017MNRAS, Generozov2020ApJ}. The other component is ejected as a hypervelocity star, which have been observed in the Milky Way \citep[see][for a review]{Brown:2015}. It is also possible that a triple system may be disrupted by the SMBH and leave behind a captured binary S-star, so the discovery an S-star binary could support a disrupted triple system hypothesis \citep{Fragione2018MNRAS}.

% (DISK STAR BINARY FRACTION POINT/STAR FORMATION?) 
\citet{Naoz2018} explain that unaccounted binary stars can bias the inferred kinematic properties of the nearby clockwise disk of young stars. While the stars in this work are not members of the clockwise disk, it is interesting to compare the young S-stars to the disk population \citep[e.g.][]{Madigan:2014fp}. Given the closer proximity to the SMBH compared to the disk, the S-stars would be more sensitive to the effects of the SMBH. This closer proximity could lead to binary mergers and binary disruptions. Therefore, the S-star binary star fraction can be lower than the disk binary fraction. 

It is not surprising that our binary fraction limit for the late-type stars is not as constraining as the limit for the early-type stars. The late-type stars' $K$ amplitude distribution is dominated by very low values due to the binary population having longer periods and lower stellar masses. Even though we can place lower $K$ amplitude limits for the individual late-type stars given our better radial velocity precision, these lower limits do not outweigh the population's distribution of $K$ amplitudes. Additionally, not identifying binary candidates among the late-type stars is unsurprising. \citet{Stephan:2016eh,Stephan:2019} reports that the evaporation timescale for a binary system with a total mass of 2\msun \ and separated by 3 AU (P $\sim 1300$ days) evaporates in under $10^{6}$ years. Since these late-type stars are $\sim$ 1 Gyr old, these stars have had sufficient time to evaporate, if they were previously part of binary star systems. After a Gyr, \citet{Stephan:2016eh} explains that there has been more time for mergers to take place, so even though binary star systems can survive longer than the evaporation time due to hardening interactions, these hardened, close binary stars can merge as they evolve off the main-sequence\footnote{These merged stars would also appear younger by comparison.}. Nevertheless, discovering binary star systems among the late-type star population would provide a strong constraint for the density of objects at the Galactic center \citep{Rose:2020}, and continued monitoring will provide improved sensitivity for the late-type star population.

%%%%%%%%%%%%%%%%%%%%%%%%%%%%%%%%%%%%%%%%%%%%%%%%%% 
%DISCUSSION and Conclusion
%%%%%%%%%%%%%%%%%%%%%%%%%%%%%%%%%%%%%%%%%%%%%%%%%%

% \section{Summary and Conclusion} \label{sec:conclusion}

% \subsection{Apparent lower binary fraction compared to field stars}

% INSERT REVISED CONCLUSION - should we keep this in?

% We have conducted the first systematic search for binary stars within the central 6 $arcsec^2$ (0.1 $pc^2$) of the central supermassive black hole. There were no significant periodic detections in our sample of 28 stars. These lack of detections enables us to place a limit on the young star population binary fraction at 47\% (with 95\% confidence). This is well below the binary fraction ($70\pm 9\%$) for massive stars larger galactic radii \citep{Sana:2012}. \citet{Stephan:2016eh} have estimated the decrease in the binary star fraction from evaporation and mergers via three-body interactions with the central black hole through the eccentric Kozai-Lidov effect. Figure \ref{fig:stephan_compare} shows their simulation results at an age of 6 Myr, which we normalize to the observed binary star fraction of 70\% at large radii (shaded region). These predictions are consistent with our observations. We also note that the eclipsing binary fraction of stars outside the central arcsecond ($\sim 0.4$ pc) reported by \citet{Gautam:2019} is consistent with the field star binary fraction. The low binary fraction within $\sim$20 mpc appears to be well-explained by a scenario in which the central SMBH drives binary star mergers near its proximity. The process has important implications for the production of gravitational wave sources \citep{LIGO2016_offical}. Additional observations will further improve limits on the multiplicity of these stars closest to the SMBH.

% and evaporation the dynamical evolution of an initial binary star population as a function of time.

% which disagrees with the binary fraction for massive field stars and is consistent with models for mergers at this distance from the black hole. This result is consistent with the scenario that the central SMBH drives binary star mergers near its proximity. The process has important implications for the production of gravitational wave sources. Additional observations will further improve limits on the multiplicity of these stars closest to the supermassive black hole.

% Our binary star limit is well below that detected for massive young stars. Figure \ref{fig:stephan_compare} shows how our derived binary fraction limit of 47\% compares to the survived binary fraction limit as a function of semi-major axis distance from the black hole from \citet{Stephan:2016eh}, normalized to the field binary star fraction of $70\pm 9\%$ \citep{Sana:2012}. Our limit is consistent with the \citet{Stephan:2016eh} prediction that a fraction of the initial binary star population has interacted via mergers and evaporation, leading to a reduced binary fraction closer to the SMBH. This binary star merging mechanism has important implications for producing gravitational wave sources \citep[e.g.][]{LIGO2016_offical}. The binary fraction limit from this work also suggests a discrepancy between the binary fractions of the young S-star sample and the field stars.  The difference in binary fraction could attest to the important role the SMBH plays in the evolution of the S-star cluster. Continued monitoring of these stars and the identification of a spectroscopic binary will have major implications in determining the evolutionary nature of the S-stars.

% The 47\% limit for the massive stars is well below the $70\pm 9\%$ binary fraction value reported in \citet{Sana:2012}. 

\begin{figure}
\centering
\includegraphics[width=\linewidth]{stephan_2016_compare_med_semimajor_bin_average-3-9-23.pdf}
\caption{
Binary fraction upper limit of 47\% for the early-type star sample (red). The x-axis bar shows the range of the semi-major axis distribution. This limit is compared to the binary fraction model from \citet{Stephan:2016eh} for a given semi-major axis from Sgr A*, normalized to a starting binary fraction limit and uncertainty of 70\%$\pm$9\% for massive stars from \citet{Sana:2012} (gray shaded region). The binary fraction limit from this work for the young stars is consistent with the binary merger model and inconsistent with the binary fraction for massive stars in the solar neighborhood.
\label{fig:stephan_compare}
}
\end{figure}

% Not identifying binary candidates among the late-type stars is unsurprising. \citet{Stephan:2016eh,Stephan:2019} reports that the evaporation timescale for a binary system with a total mass of 2\msun \ and separated by 3 AU (P $\sim 1300$ days) evaporates in under $10^{6}$ years. Since these late-type stars are $\sim$ 1 Gyr old, these stars have had sufficient time to evaporate, if they were previously part of binary star systems. After a Gyr, \citet{Stephan:2016eh} explains that there has been more time for mergers to take place, so even though binary star systems can survive longer than the evaporation time due to hardening interactions, these hardened, close binary stars can merge as they evolve off the main-sequence\footnote{These merged stars would also appear younger by comparison.}. Nevertheless, discovering binary star systems among the late-type star population would provide a strong constraint for the density of objects at the Galactic center \citep{Rose:2020}, and continued monitoring will provide improved sensitivity for this population.

% A merger rate of 13\% and evaporation rate of 27\% for all young binaries at the Galactic Center may help explain this discrepancy \citep{Stephan:2016eh}. Starting with a binary fraction of 70\% from \citet{Sana:2012} and having 60\% of young binary systems survive (40\% of binaries merge or evaporate) according to \citet{Stephan:2016eh}, the resulting binary fraction would be 42\%. We note that these limits are dependent on the binary system separation and mass.

% The fact that we do not find any binary stars among the young S-stars attests to evolution mechanisms that result in a low binary fraction. Additionally, these S-stars may be merger products that appear as main-sequence B-stars \citep{Stephan:2016eh,Stephan:2019}. This includes the binary disruption scenario, where the S-star we see is a tidally captured star originally part of binary system \citep{Hills:1988br,Perets:2009ke}.

% \subsection{Placing the Lack of Detections into Context - Late-Type Stars}

% Discovering binary star systems among the late-type star population would provide a strong constraint for the density of objects at the Galactic center \citep{Rose:2020}. We apply a similar simulation approach as described in Section \ref{subsec:placing_limits_young} to the late-type stars. We draw from the distribution of binary properties reported in \citet{Raghavan:2010} to create our $K$ amplitude distributions, which focused on systems around a solar mass (see Figure \ref{fig:binary_simulation_parameters_late}).

% Following the same steps as in \ref{subsec:placing_limits_young}, we obtain the fraction of simulations with 0 detections for each population of binary fractions. We found that even for binary fractions of 100\%, around 30\% of our simulations had 0 detections \ref{fig:binary_compare}. This mostly has to do with the fact that the $K$ amplitude distribution for this population is dominated by small values. This is expected, as $K$ is $\propto$ $P^{-1/3}$, so $K$ decreases with long period binaries. Since the \citet{Raghavan:2010} distribution is dominated by long period binaries ($P = 10^{5}$ days), it is expected that the $K$ amplitude distribution would mostly contain small values.

% Not identifying binary candidates among the late-type stars is not surprising. \citet{Stephan:2016eh,Stephan:2019} reports that the evaporation timescale for a binary system with a total mass of 2\msun \ and separated by 3 AU (P $\sim 1300$ days) evaporates in under $10^{6}$ years. Since these late-type stars are $\sim$ 1 Gyr old, these stars have had sufficient time to evaporate, if they were previously part of binary star systems. After a Gyr, \citet{Stephan:2016eh} explains that there has been more time for mergers to take place, so even though binary star systems can survive longer than the evaporation time due to hardening interactions, these hardened, close binary stars can merge as they evolve off the main-sequence\footnote{These merged stars would also appear younger by comparison.}. These compounding effects means it is unsurprising that we do not find binaries among this population.

% \subsection{Apparent lower binary fraction compared to field stars}

% Figure \ref{fig:stephan_compare} shows how our derived binary fraction limit of 47\% compares to the survived binary fraction limit as a function of semi-major axis distance from the black hole from \citet{Stephan:2016eh}. We also compare our limit to the field binary star limit from \citet{Sana:2012}. Our limit of 47\% is consistent with the \citet{Stephan:2016eh} prediction that a fraction of the initial binary star population has interacted via mergers and evaporation, leading to a reduced binary fraction closer to the SMBH. The binary fraction limit from this work also suggests a discrepancy between the binary fractions of the young S-star sample and the field stars. We also note that the eclipsing binary fraction of stars outside the central arcsecond ($\sim 0.4$ pc) reported by \citet{Gautam:2019} is consistent with the field star binary fraction. The difference in binary fraction could attest to the important role the SMBH plays in the evolution of the S-star cluster. Continued monitoring of these stars and the identification of a spectroscopic binary will have major implications in determining the evolutionary nature of the S-stars.

\section{Acknowledgements}

 
 We are grateful for the helpful and constructive comments from the referee. We thank M. R. Morris for his comments and long-term efforts on the Galactic Center Orbits Initiative. The primary data for this work was collected with the W. M. Keck Observatory, which is operated as a scientific partnership among the California Institute of Technology, the University of California, and the National Aeronautics and Space Administration. We wish to recognize that the summit of Maunakea has always held a very significant cultural role for the indigenous Hawaiian community. We are most fortunate to have the opportunity to observe from this mountain. We also thank the staff of the Keck Observatory, especially Jim Lyke, Randy Campbell, Gary Puniwai, Heather Hershey, Hien Tran, Scott Dahm, Jason McIlroy, Joel Hicock, and Terry Stickel, for all their help in obtaining the new observations.  Finally, we are grateful for the financial support for this work provided by NSF AST grants 1412615 and 1909554, the Gordon \& Betty Moore Foundation, the Levine-Leichtman Family Foundation, Ken and Eileen Kaplan Student Support Fund, the Galactic Center Board of Advisors, and the Janet Marott Student Travel Awards. S.N. acknowledges the partial support from NASA ATP 80NSSC20K0505 and thanks Howard and Astrid Preston for their generous support. 



\facility{W. M. Keck Observatory, Gemini North Observatory}
\software{Numpy \citep{numpy2011CSE....13b..22V,numpyharris2020array}, Astropy \citep{2013Astropy,Astropy2018AJ}, Starkit \citep{starkit:2015}, gatspy \citep{VanderPlas:2015, VanderPlas:2018}, IRAF \citep{IRAF1986SPIE,IRAF1993ASPC}, Multinest \citep{Feroz:2008fi,Feroz:2009aa,Feroz:2013aa}, SPISEA \citep{spisea2020,Hosek:2020AJ}, Scipy \citep{2020SciPy}, OSIRIS Data Reduction Pipeline \citep{OSIRIS_pipeline:2017, Lockart:2019}}

%% Similar to \facility{}, there is the optional \software command to allow 
%% authors a place to specify which programs were used during the creation of 
%% the manusscript. Authors should list each code and include either a
%% citation or url to the code inside ()s when available.

%%%%%%%%%%%%%%%%%%%%%%%%%%%%%%%%%%%%%%%%%%%%%%%%%% 
%APPENDIX
%%%%%%%%%%%%%%%%%%%%%%%%%%%%%%%%%%%%%%%%%%%%%%%%%% 

\appendix

% \section{Sample Selection}
% \label{sect:Sample Selection}

% As discussed in Section \ref{subsec:sample}, we detailed our sample selection criteria. Table \ref{tab:sample_cut} lists all of the 62 stars in the region with $K'$ $<$ 16. Stars that were excluded from the radial velocity sample are listed in this table.

% \input{sample_cut_test_with_gautam_kp.tex}

\section{Source Confusion}	\label{sect:confusion_gas_appendix}

We took extra care to ensure that radial velocity measurements were not affected by either stellar or gaseous source confusion. Stellar source confusion affects 23 stars, which are therefore removed from the sample. Local gas can also affect the measurement of the \brg absorption line, since it not only emits \brg, but it does so at different velocities. One of the checks we conducted was to look at the strength of the gas emission at the star's radial velocity in the subtracted background. This led to the removal of two further stars, S1-2 and S1-33, since they were identified as having potentially biased radial velocity measurements based on their subtracted gas backgrounds. Table \ref{tab:excluded_stars} lists the complete list of stars that were excluded from this analysis for all the reasons discussed in Section \ref{sec:Sample}.

% The GC contains dynamic gas that emits \brg at different velocities, creating a multiple peaked Gaussian line. It is possible that background subtraction may imprint artifact \brg absorption lines on stellar spectra. This artifact becomes a problem if the measured stellar velocity similar to the gas velocity.

% As a further check, we compared their measured radial velocities to the modeled gas velocity for a corresponding epoch (Anna's work, more detail if necessary) and found their values to be similar. These conclusions convinced us that S1-2 and S1-33's radial velocities are biased by the background gas, and we excluded these stars from the periodicity analysis.

% (how much is worth saying about modeling the gas for an epoch is computationally expensive, so it can't be done with every epoch)

\startlongtable
\begin{deluxetable*}{lrcrrrl}
\tablecolumns{7}
\tablewidth{0pc} 
\tablecaption{Excluded Stars\label{tab:excluded_stars}}
\tablehead{
\colhead{Name}	&
\colhead{$K'$}	&
\colhead{Spectral}	&
\colhead{RA$\Delta$\tna}	&
\colhead{Dec$\Delta$\tna}	&
\colhead{R2D\tna}	&
\colhead{Exclusion}	\\
\colhead{}	&
\colhead{(mag)}	&
\colhead{Type}	&
\colhead{(\arcsec)}	&
\colhead{(\arcsec)}	&
\colhead{(\arcsec)}	&
\colhead{Reason}	
}
\startdata
S0-24 & 15.58 & Late & 0.20 & 0.09 & 0.22 & Confused \\
S0-26 & 15.20 & Early & 0.33 & 0.21 & 0.40 & Confused \\
S0-53 & 15.50 & Unknown & 0.35 & 0.20 & 0.40 & Confused \\
S0-28 & 15.45 & Late & -0.14 & -0.49 & 0.51 & Too Few RVs \\
S0-62 & 15.37 & Late & 0.16 & -0.54 & 0.57 & Confused \\
S0-29 & 15.45 & Late & 0.37 & -0.44 & 0.58 & Confused \\
S0-67 & 15.49 & Late & 0.25 & -0.54 & 0.59 & Confused \\
S0-33 & 15.95 & Unknown & 0.65 & -0.53 & 0.83 & Confused \\
S0-32 & 14.08 & Unknown & 0.32 & 0.79 & 0.85 & Foreground Star \\
S0-35 & 15.20 & Unknown & 0.02 & 0.88 & 0.88 & Confused \\
S1-3 & 12.09 & Early & 0.32 & 0.88 & 0.94 & Featureless \\
S1-26 & 15.41 & Late & -0.88 & 0.39 & 0.96 & Confused \\
S0-108 & 15.67 & Unknown & 0.45 & -0.90 & 1.01 &  Confused \\
S1-2 & 14.64 & Early & 0.08 & -1.02 & 1.02 &  Background Gas \\
S1-1 & 13.02 & Early & 1.04 & 0.03 & 1.04 &  Featureless \\
S1-27 & 15.80 & Early & -1.03 & 0.19 & 1.05 &  Confused \\
S1-29 & 15.26 & Early & 1.07 & 0.16 & 1.08 &  Confused \\
S1-4 & 12.43 & Early & 0.88 & -0.66 & 1.10 &  Featureless \\
S1-28 & 15.92 & Late & -0.37 & -1.05 & 1.12 &  Confused \\
irs16C & 9.91 & Early & 1.05 & 0.55 & 1.18 &  Wolf-Rayet \\
S1-32 & 15.15 & Late & -0.99 & -0.66 & 1.19 &  Confused \\
S1-7 & 15.73 & Late & -1.05 & -0.58 & 1.20 &  Confused \\
S1-85 & 15.50 & Unknown & 0.92 & -0.83 & 1.24 &  Confused \\
S1-33 & 14.94 & Early & -1.25 & -0.00 & 1.25 &  Background Gas \\
S1-86 & 15.30 & Unknown & 1.02 & 0.74 & 1.26 &  Confused \\
S1-12 & 13.41 & Early & -0.75 & -1.03 & 1.27 &  Featureless \\
S1-34 & 12.91 & Late & 0.87 & -0.99 & 1.32 &  Confused \\
S1-14 & 12.90 & Early & -1.32 & -0.37 & 1.37 &  Featureless \\
irs16SW & 9.98 & Early & 1.11 & -0.95 & 1.46 &  Wolf-Rayet \\
S1-40 & 15.63 & Unknown & -1.41 & -0.61 & 1.54 &  Confused \\
S1-21 & 13.21 & Early & -1.64 & 0.09 & 1.64 &  Featureless \\
S1-22 & 12.52 & Early & -1.57 & -0.52 & 1.65 &  Featureless \\
S1-51 & 14.91 & Unknown & -1.66 & -0.17 & 1.67 &  Confused \\
S1-45 & 15.19 & Unknown & -1.28 & 1.10 & 1.69 &  Confused \\
%% ADD S1-2 AND S1-33
\enddata
\tablenotetext{a}{From Sgr A*.}

\end{deluxetable*}

\section{Impact of OSIRIS detector upgrade}
\label{sect:instrument}

Figure \ref{fig:tint_fwhm} compares the performance of the old and new detector for a K $\sim$14 star from our standard Galactic Center observational set-up. The new detector has enabled improved spectral signal-to-noise for data for a given total integration time and FWHM.

%%code to generate plot below is rv_periodicity.int_snr_plot()
\begin{figure}
\centering
\includegraphics[width=\linewidth]{tint_scalefwhm_snr_w_fit.pdf}
\caption{
Spectral signal-to-noise ratio of a $K'$ $\sim$14 mag star for a dataset's total integration time scaled by the dataset's FWHM relative to the average FWHM of 76 mas. Data taken with the newest OSIRIS detector and previous detector are plotted in red and gray, respectively. The dashed lines are fits to the data subsets. The steeper slope of the new detector data fit ($9.00\pm 0.69\times10^{-3}$) versus the old detector data ($5.42 \pm 0.29\times10^{-3}$) shows the improved spectral signal-to-noise for a given integration time and FWHM.
\label{fig:tint_fwhm}
}
\end{figure}

\section{Lomb-Scargle and $K$ Amplitude Limits}
\label{sect:ls_k_results}

This appendix section presents the results discussed in Section \ref{sec:per_search}. Figure \ref{fig:all_ls} shows the Lomb-Scargle periodograms for all stars used in the analysis, and Figure \ref{fig:all_k} shows the $K$ amplitude limits per period.

\begin{figure*}
\centering
\includegraphics[width=\linewidth]{all_ls.pdf}
\caption{
Lomb-Scargle periodograms for all 28 stars in the sample. For each plot, period in days is plotted on the x-axis, and the Lomb-Scargle power is on the y-axis.
\label{fig:all_ls}
}
\end{figure*}

\begin{figure*}
\centering
\includegraphics[width=\linewidth]{all_kamp.pdf}
\caption{
$K$ amplitude limits for all 28 stars in the sample. For each plot, period in days is plotted on the x-axis, and the $K$ amplitude limit in \kms is on the y-axis.
\label{fig:all_k}
}
\end{figure*}

% \section{Orbital Fitting Methodology For Orbit S-star Sample}	\label{sect:orbit_fit_sys}

% For the short period stars beyond S0-2 with measurable orbits, we use a seven parameter model. The following black hole parameters are fixed to S0-2's values: mass ($M_{BH}$), distance ($R_{0}$), the position ($x,y$), and velocity ($v_{x},v_{y}, v_{z}$) of the black hole. Similarly, we fix the astrometric correlation length from source confusion to 30 mas and the radial velocity offset between the Keck and VLT measurements to 0 \citep[see][]{Do:2019,Ciurlo:2020Nature}. Leaving these values free has no impact on modeling the radial velocity curves or residuals. The seven model values are the six standard stellar orbital parameters plus the astrometric correlation strength.

% We also investigated potential systematic effects of orbit fitting by running numerous combinations of orbital fits for each star. Specifically, we wanted to analyze the influence of the following sets of parameters:

% \begin{enumerate}
%     \item SMBH parameters
%     \item Offset and additive error between Keck and VLT data points
%     \item Mixing and correlation in the uncertainty of the astrometric measurements
% \end{enumerate}

% We wanted to ensure that changes in the black hole parameters and resulting orbital fit would not introduce a signal into the radial velocity investigations. We investigated two combinations: we left all parameters free, and we fixed all parameters to values from \citet{Do:2019}.

% We also fit for a radial velocity offset and additive error between Keck and VLT data points in order to account for potential systematic between the radial velocity values reported from different instruments. This procedure was also conducted in the orbital fit method used in \citet{Do:2019}. We wanted to ensure that there was no significant offset between Keck and VLT data, because otherwise this offset would produce an artificial signal in the periodicity search. We investigated two combinations: leaving both of these parameters free, and not including them at all.

% We also account for potential correlations in the uncertainty of astrometric measurements. This involves two parameters: a correlation length scale $\Lambda$ and mixing parameter $p$. These parameters are discussed in greater depth in \citet{Do:2019} and used in orbital fits in \citet{Ciurlo:2020Nature}. We investigated two combinations: fixing $\Lambda$ = 30 mas \citep{Do:2019,Ciurlo:2020Nature} and keeping $p$ free, and not including them at all.

% To summarize, each of the 3 sets of parameters (SMBH parameters, RV offset and RV additive error, and correlated astrometric uncertainty) had two combination of fits. This meant that it was possible to have 8 different permutations of all of the set combinations. We fit 8 different fit combinations for each of the stars in our orbit sample and used Bayesian information criteria to investigate the importance of each parameter set, as done in \citet{Do:2019}. Generally, free black hole parameters were favored, but the black hole parameters were not well constrained. A more detailed investigation into these parameter values is beyond the scope of this work. The offset and additive error were not favored. Finally, a fixed $\Lambda$ and fitting for $p$ were favored.

% Ultimately, we found that the different orbital fits did not impact the radial velocity models and residuals. We concluded that we would do the following for the different orbital fits for each star: fix the black hole parameters to those reported in \citet{Do:2019}; not include an instrumental offset and additive error; include a fixed $\Lambda$ and fit for $p$.

% \section{Using Photometry to Place Constraints} \label{sec:photo_constraints}

%%%%%%%%%%%%%%%%%%%%%%%%%%%%%%%%%%%%%%%%%%%%%%%%%% 
%MASS LIMITS
%%%%%%%%%%%%%%%%%%%%%%%%%%%%%%%%%%%%%%%%%%%%%%%%%% 

% \section{Placing limits on companion masses}
% \label{sec:comp_mass_limits}

\section{Placing limits on companion masses}
\label{sec:comp_mass_limits}

% \input{sstar_binary_paper/Sections/companion_mass_limits}

With the results from the binary curve fitting, in particular our limits on the $K$ amplitude, we move to place limits on hypothetical companion masses of binary systems using the same methodology as \citet{Chu:2018}. For each period $P$, there is a limit on $K$, and the binary mass equation (Equation \ref{eq:Binary_mass_s-star_paper} can be solved assuming for a total mass, a limit for the companion mass for each period can be calculated. In order to determine the total mass for a star, we use its $K'$ photometry reported \citet{Gautam:2019} and an isochrone generated with the SPISEA software \citep{spisea2020,Hosek:2020AJ}. A 6.78 Myr isochrone is used for the early-type stars and a 1 Gyr isochrone is used for the late-type stars. These isochrones use the MIST stellar evolution models \citep{MIST2016ApJ}, and each isochrone is corrected for extinction to the Galactic Center with a value of $A_{K'} = 2.46$ \citep{schodel:2010}. Solar metalicities are used for both isochrones. These isochrones are shown in Figure \ref{fig:isochrones}. The total mass used for each star is given in Table \ref{tab:mass_limits}. \citet{Habibi:2017} reported masses for early-type S-stars stars in their analysis. For stars that overlap with our sample, their reported mass values are lower than the isochrone mass values but still consistent within 2$\sigma$. We report the median upper limits for the companion masses for all periods in Table \ref{tab:mass_limits} and Figure \ref{fig:Kp_mass}.

% \begin{equation}
% 	M_{\text{comp}}\sin{i}  = \left(\frac{PM_{tot}^{2}}{2\pi G}\right)^{1/3} \ K\, ,
% 	\label{eq:Binary_mass_s-star_paper}
% \end{equation}

\begin{deluxetable*}{lrrrrrr}
\tablecolumns{7}
\tablewidth{0pc}
\tablecaption{Companion Mass Limits\label{tab:mass_limits}}
\tablehead{
\colhead{Star} &
\colhead{Mean Mag} &
\colhead{Spectral} &
\colhead{Isochrone} &
\colhead{Upper Limit $M_{comp}$} &
\colhead{Upper Limit Mass} &
\colhead{Equal Mass}    \\
\colhead{}  &
\colhead{($K'$)}  &
\colhead{Type}  &
\colhead{Mass (\msun)}  &
\colhead{Mass (\msun)}  &
\colhead{Ratio} &
\colhead{Binary (\msun)}
}
\startdata
S1-5	 & 12.48	 & Late		& 1.2 		 & 	0.10 	&	 0.083	 &	1.2 	\\
S0-13	 & 13.24	 & Late		& 1.2 		 & 	0.10 	&	 0.083	 &	1.2 	\\
S0-15	 & 13.55	 & Early	& 20.2 		 & 	5.4 	&	 0.27	 &	16.7 	\\
S0-14	 & 13.57	 & Early 	& 20.0 		 & 	1.7 	&	 0.085	 &	16.3 	\\
S0-6	 & 13.95	 & Late 	& 1.2 		 & 	0.10 	&	 0.083	 &	1.2 	\\
S1-13	 & 13.96	 & Late 	& 1.2 		 & 	0.70 	&	 0.58	 &	1.2 	\\
S0-2	 & 14.02	 & Early 	& 17.5 		 & 	1.1 	&	 0.063	 &	14.2 	\\
S1-15	 & 14.04	 & Late 	& 1.2 		 & 	0.10 	&	 0.083	 &	1.2 	\\
S1-8	 & 14.08	 & Early 	& 17.0 		 & 	5.6 	&	 0.33	 &	13.9 	\\
S0-4	 & 14.15	 & Early 	& 16.7 		 & 	5.7 	&	 0.34	 &	13.7 	\\
S0-9	 & 14.24	 & Early 	& 16.3 		 & 	3.8 	&	 0.23	 &	13.4 	\\
S0-12	 & 14.27	 & Late 	& 1.2 		 & 	0.10 	&	 0.083	 &	1.2 	\\
S0-3	 & 14.53	 & Early 	& 14.9 		 & 	2.2 	&	 0.15	 &	12.1 	\\
S1-10	 & 14.66	 & Late 	& 1.2 		 & 	0.10 	&	 0.083	 &	11.7 	\\
S0-1	 & 14.68	 & Early 	& 13.9 		 & 	3.1 	&	 0.22	 &	1.2 	\\
S0-18	 & 14.92	 & Late 	& 1.2 		 & 	0.40 	&	 0.33	 &	11.4 	\\
S0-5	 & 14.97	 & Early 	& 12.6 		 & 	2.2 	&	 0.17	 &	1.2 	\\
S0-31	 & 15.03	 & Early 	& 12.3 		 & 	4.9 	&	 0.48	 &	10.3 	\\
S0-7	 & 15.12	 & Early 	& 11.9 		 & 	4.6 	&	 0.38	 &	10.1 	\\
S0-11	 & 15.13	 & Early 	& 11.9 		 & 	3.0 	&	 0.25	 &	9.7 	\\
S0-16	 & 15.3		 & Early 	& 11.0 		 & 	5.4 	&	 0.49	 &	9.7 	\\
S0-19	 & 15.36	 & Early 	& 10.9 		 & 	3.6 	&	 0.33	 &	9.0 	\\
S1-6	 & 15.38	 & Late 	& 1.2 		 & 	0.20 	&	 0.16	 &	8.9 	\\
S0-27	 & 15.54	 & Late 	& 1.2 		 & 	0.30 	&	 0.25	 &	1.2 	\\
S1-31	 & 15.59	 & Late 	& 1.2 		 & 	0.30 	&	 0.25	 &	1.2 	\\
S0-8	 & 15.79	 & Early 	& 9.0 		 & 	3.1 	&	 0.34	 &	1.2 	\\
S0-20	 & 15.85	 & Early 	& 8.8 		 & 	5.8 	&	 0.64	 &	7.4 	\\
S0-17	 & 15.85	 & Late 	& 1.2 		 & 	0.30 	&	 0.25	 &	7.2 	\\
\enddata
\tablecomments{Col 1: star name, Col 2: mean magnitude in $K'$, Col 3: spectral type, Col 4: mass from the isochrone, Col 5: upper limit on companion mass, Col 6: upper limit on the mass ratio, Col 7: mass of each component of an equal mass binary system that would emit same flux as the star's photometry.}
\end{deluxetable*}

%binary_mass.plot_isochrone()
\begin{figure}
\centering
\includegraphics[width=\linewidth]{isochrone_sample_plot.pdf}
\caption{
Two SPISEA isochronese used for determining the mass of a star based on its $K'$ magnitude. The MIST stellar evolution models \citep{MIST2016ApJ} and extinction law from \citep{schodel:2010} are applied to these isochrones. The 6Myr isochrone was used for the early-type stars, while the 1 Gyr isochrone was used for the late-type stars.
\label{fig:isochrones}
}
\end{figure}

%binary_mass.plot_kp_mass_limit()
\begin{figure}
\centering
\includegraphics[width=\linewidth]{kp_mass_sample_joint_v2.pdf}
\caption{
Left: Companion mass limits for each star plotted with their projected distance from Sgr A*. These limits come from marginalizing the mass limits over all sampled periods. Right: Companion mass limits for each star plotted with their $K'$ magnitude.
\label{fig:Kp_mass}
}
\end{figure}

The photometric information from \citet{Gautam:2019} of each star is used to place conservative limits on the masses of an equal mass binary system. To do this, the total flux from the star is divided in half. The SPISEA isochrone is then searched to find the mass of a star that would contribute the equivalent amount of flux. This places a limit on the components of a face-on binary system composed of equal mass stars. This can be thought of as a conservative limit, as the spectral differences between different mass stars are not considered in this part of the analysis. These limits are reported in Table \ref{tab:mass_limits}.

% \input{photo_mass_limits}

%%%%%%%%%%%%%%%%%%%%%%%%%%%%%%%%%%%%%%%%%%%%%%%%%% 
%SIMULATION APPENDIX
%%%%%%%%%%%%%%%%%%%%%%%%%%%%%%%%%%%%%%%%%%%%%%%%%% 

% \section{Placing limits on companion masses}
% \label{sec:comp_mass_limits}

\section{Effect of Period Distributions and Sample Size on Simulations}
\label{sec:period_simulations}

We note that while we adopt field-like distributions for the binaries, there are processes that truncate long period binaries in the dense environment of the Galactic Center compared to the field. For example, flyby stars unbind wide separated binaries \citep[e.g.][see latter for unbinding of eccentric binaries]{BinneyTremaine2008, Rose:2020} which produces a distribution described by \citet{Stephan:2016eh,Stephan:2019}. Furthermore, stability and the Hills process tend to truncate the distribution in a similar manner. Thus, it can be estimated as loguniform in the literature \citep[e.g.][]{Fragione2019MNRAS}. Note that the \citet{Sana:2012} distribution, adopted here for the early-type stars, favors short-period distribution, consistent with the unbinding processes \citep[as noted in][]{Hoang2018}. We examine the impact of these effects on our results by testing two period distributions for the young stars: (1) the \citet{Sana:2012} period distribution truncated at $9.8 \times 10^{4}$ days, (2) a lognormal distribution truncated at the same length. We find that both of these period distributions produce no impact on our inference about the binary fraction. The binary fraction limit changes by less than 3\%. The effects of these period distributions are shown in Figure \ref{fig:period_investigation}.

For late-type stars, the binary distribution makes a modest difference in the resulting binary fraction limit because the \citet{Raghavan:2010} distribution extends to very long periods that these dynamical processes will truncate. To approximate these effects, we use the results of the simulations from \citet{Stephan:2016eh} which shows a truncated period distribution at $3.3 \times 10^{5}$ days for 1.2 \msun stars. If the late-type star binary population is truncated, then our data has a modest constraint on the binary fraction. We would infer an upper limit of the binary fraction to be less than 93\% at 95\% confidence. 

%binary_simulations.compare_cdfs_per_cut()
\begin{figure}
\centering
\includegraphics[width=\linewidth]{period_investigation_detection_population_compare_joint_plot_v2.pdf}
\caption{
The simulated binary fraction populations versus the fraction of Monte Carlo simulations with zero detections for each population of binary fractions with different period distributions. Left: The early-type star simulations when using the following period distributions: \citet{Sana:2012}, truncated \citet{Sana:2012}, and loguniform. The binary fraction limit changes by less than 3\%. Right: The late-type star simulations when using the following period distributions: \citet{Raghavan:2010} and \citet{Stephan:2016eh}. The \citet{Stephan:2016eh} distribution truncates the longer period binaries compared to \citet{Raghavan:2010}. We would infer a binary fraction less than 93\% at 95\% confidence with the \citet{Stephan:2016eh} distribution.
\label{fig:period_investigation}
}
\end{figure}

We have also explored our sensitivity to sample size for our simulations. To account for an increased sample size, we double-count stars from our observational sample and compare them to the simulated populations. For the young stars, it would take an increase of eight stars (from 16 to 24) to decrease the binary fraction limit by 10\%. Changing the sample by one or two stars for the early-type stars does not dramatically affect the limit. 

% Increased time baseline monitoring of our current sample will presumably improve our K amplitude limit, which can then lead to a lower binary fraction limit.

% In order to decrease the late-type binary fraction limit by at least 10\% for the late-type stars, it would take an increase of three stars (from 12 to 15).

% Therefore, we explore a period distribution that truncates at (input period) using a model from \citet{Stephan:2016eh,Stephan:2019}. We also explore a loguniform distribution for the early-type stars that also truncates at the same period.

% Furthermore, stability and the Hills process tend to truncate the distribution in a similar manner. Thus, it can be estimated as loguniform in the literature \citep[e.g.][]{Fragione2019MNRAS}. Note that the \citet{Sana:2012} distribution, adopted here for the early-type stars, favors short-period distribution, consistent with the unbinding processes \citep[as noted in][]{Hoang2018}.

% The same process is done for the solar mass star population, except that parameters are drawn from \citet{Raghavan:2010}, with the exception of the period distribution. Instead, we use the same binary separation distribution as \citet{Stephan:2016eh,Stephan:2019}, which peaks around 10 AU and truncates at 100 AU to account for the unbinding processes at the Galactic center. We note that the uniform eccentricity distribution following \citet{Raghavan:2010} seems to be consistent with frequent dynamical encounters \citep{Geller2019ApJ}.

% We explored the effects of having a period cutoff on our simulations. This investigation was done because the SMBH will disrupt widely separated binary systems.

% \begin{itemize}
%     \item Affect on early-type stars is small
%     \item Loguniform investigation, effect is minimal
%     \item Bigger affect on late-type stars
% \end{itemize}

%% Appendix material should be preceded with a single \appendix command.
%% There should be a \section command for each appendix. Mark appendix
%% subsections with the same markup you use in the main body of the paper.

%% Each Appendix (indicated with \section) will be lettered A, B, C, etc.
%% The equation counter will reset when it encounters the \appendix
%% command and will number appendix equations (A1), (A2), etc. The
%% Figure and Table counter will not reset.


%% The reference list follows the main body and any appendices.
%% Use LaTeX's thebibliography environment to mark up your reference list.
%% Note \begin{thebibliography} is followed by an empty set of
%% curly braces.  If you forget this, LaTeX will generate the error
%% "Perhaps a missing \item?".
%%
%% thebibliography produces citations in the text using \bibitem-\cite
%% cross-referencing. Each reference is preceded by a
%% \bibitem command that defines in curly braces the KEY that corresponds
%% to the KEY in the \cite commands (see the first section above).
%% Make sure that you provide a unique KEY for every \bibitem or else the
%% paper will not LaTeX. The square brackets should contain
%% the citation text that LaTeX will insert in
%% place of the \cite commands.

%% We have used macros to produce journal name abbreviations.
%% \aastex provides a number of these for the more frequently-cited journals.
%% See the Author Guide for a list of them.

%% Note that the style of the \bibitem labels (in []) is slightly
%% different from previous examples.  The natbib system solves a host
%% of citation expression problems, but it is necessary to clearly
%% delimit the year from the author name used in the citation.
%% See the natbib documentation for more details and options.

\bibliography{apj-jour,sstar_paper.bib}

%% This command is needed to show the entire author+affilation list when
%% the collaboration and author truncation commands are used.  It has to
%% go at the end of the manuscript.
%\allauthors

%% Include this line if you are using the \added, \replaced, \deleted
%% commands to see a summary list of all changes at the end of the article.
%\listofchanges
\end{CJK*}
\end{document}

% End of file `sample61.tex'.
