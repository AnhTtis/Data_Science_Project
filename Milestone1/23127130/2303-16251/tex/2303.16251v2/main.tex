\documentclass[letterpaper, 10 pt, conference]{ieeeconf}  % Comment this line out
                                                          % if you need a4paper
%\documentclass[a4paper, 10pt, conference]{ieeeconf}      % Use this line for a4
                                                          % paper

\IEEEoverridecommandlockouts                              % This command is only
%                                                          % needed if you want to
%                                                          % use the \thanks command
\overrideIEEEmargins
% See the \addtolength command later in the file to balance the column lengths
% on the last page of the document

% Set to 0 for CDC compilation
% Set to 1 for ArXiv compilation
\def\ARXIV{1}


\usepackage[style=ieee,backend=bibtex]{biblatex}
\addbibresource{../../CoOL-bib/cool-refs}

%\usepackage{amsthm}
\usepackage{amsmath}
\usepackage{amssymb}
\usepackage{mathabx} % for \widecheck
\usepackage[scr=esstix]{mathalpha}
\usepackage{upgreek}
\usepackage{bm}
\usepackage{hyperref}
\hypersetup{colorlinks,
	linkcolor=blue,
	citecolor=blue,
	urlcolor=blue,
	linktocpage,
	plainpages=false}
\usepackage{cancel}
\usepackage{graphicx}
\usepackage{mathtools}
\usepackage{float}
\usepackage{tikz}
\usetikzlibrary{shapes,matrix,arrows,calc,positioning,fit,bayesnet,angles,patterns}

\tikzset{%
  every neuron/.style={
    circle,
    draw,
    minimum size=0.5cm
  },
  neuron missing/.style={
    draw=none, 
    scale=2,
    text height=0.333cm,
    execute at begin node=\color{black}$\vdots$
  },
}


%\theoremstyle{plain}
\newtheorem{theorem}{Theorem}
\newtheorem{corollary}{Corollary}
\newtheorem{lemma}{Lemma}
\newtheorem{proposition}{Proposition}
\newtheorem{remark}{Remark}
\newtheorem{assumption}{Assumption}

\definecolor{purple}{rgb}{1, 0, 1}

\newcommand{\ie}{\emph{i.e.,}\xspace}
\newcommand{\eg}{\emph{e.g.,}\xspace}
\newcommand{\abr}{\emph{abbr.}\xspace}
\newcommand{\ea}{\emph{et al.}\xspace}
\newcommand{\gensync}{\emph{GenSync}\xspace}
\newcommand{\colosseum}{\emph{Colosseum}\xspace}
\newcommand{\srep}{\emph{SREP}\xspace} % Set Reconciliation Enhances
\newcommand{\srepsim}{\emph{SREPSim}\xspace}
% Propagation
\newcommand{\esrep}{\emph{E-SREP}\xspace}
\newcommand{\epsrep}{\emph{EP-SREP}\xspace}
\newcommand{\mesrep}{\emph{ME-SREP}\xspace}
\newcommand{\mempoolsync}{\emph{MempoolSync}}

\newcommand{\fref}[1]{Fig.~\ref{#1}}
\newcommand{\tref}[1]{Table~\ref{#1}}
\newcommand{\aref}[1]{Algorithm~\ref{#1}}
\newcommand{\procref}[1]{Procedure~\ref{#1}}
\newcommand{\sref}[1]{Section~\ref{#1}}
\newcommand{\lineref}[1]{line~\ref{#1}}
\newcommand{\appref}[1]{Appendix~\ref{#1}}

% Change \eqref
\LetLtxMacro{\originaleqref}{\eqref}
\renewcommand{\eqref}{Eq.~\originaleqref}

% Theorems and corollaries
\newcounter{theoremcount}
\setcounter{theoremcount}{0}
\DeclareRobustCommand{\theorem}[1]{%
  \refstepcounter{theoremcount}%
  \noindent\textit{\textbf{Theorem \thetheoremcount\label{theorem:#1}: }}%
}
\DeclareRobustCommand{\theoremref}[1]{Theorem~\ref{theorem:#1}}

\DeclareRobustCommand{\proof}{\emph{Proof:}\xspace}
\DeclareRobustCommand{\qqed}{\hfill$\blacksquare$}

\newcounter{corollcount}
\setcounter{corollcount}{0}
\DeclareRobustCommand{\coroll}[1]{%
  \refstepcounter{corollcount}%
  \noindent\textit{\textbf{Corollary \thecorollcount\label{coroll:#1}: }}%
}
\DeclareRobustCommand{\corollref}[1]{Corollary~\ref{coroll:#1}}

\newcounter{lemmacount}
\setcounter{lemmacount}{0}
\DeclareRobustCommand{\lemma}[1]{%
  \refstepcounter{lemmacount}%
  \noindent\textit{\textbf{Lemma \thelemmacount\label{lemma:#1}: }}%
}
\DeclareRobustCommand{\lemmaref}[1]{Lemma~\ref{lemma:#1}}

\newcounter{definitioncount}
\setcounter{definitioncount}{0}
\DeclareRobustCommand{\definition}[1]{%
  \refstepcounter{definitioncount}%
  \noindent\textit{\textbf{Definition \thedefinitioncount\label{definition:#1}: }}%
}
\DeclareRobustCommand{\defref}[1]{Definition~\ref{definition:#1}}

%notes of different authors
\newif\ifnotes
\notestrue
\notesfalse

\newif\ifdiff
\difftrue
\difffalse

\newcommand{\anote}[1]{\ifnotes $\ll$\textsf{\textcolor{purple}{Ari: {#1}}}$\gg$ \fi}
\newcommand{\nnote}[1]{\ifnotes $\ll$\textsf{\textcolor{orange}{Novak: {#1}}}$\gg$ \fi}
\newcommand{\diff}[1]{\ifdiff\textcolor{orange}{#1}\else#1\fi}

%%% Local Variables:
%%% mode: latex
%%% TeX-master: "main"
%%% End:



\begin{document}

\title{\LARGE \bf
  Function Approximation with Randomly Initialized Neural Networks for Approximate Model 
  Reference Adaptive Control
}


\author{Tyler Lekang and Andrew Lamperski
  % <-this % stops a space
%  \thanks{This work was supported in part by NSF CMMI-2122856}% <-this % stops a space
      \thanks{T. Lekang and A. Lamperski are with the department of Electrical and
        Computer Engineering, University of Minnesota, Minneapolis,
        MN 55455, USA 
        {\tt\small lekang@umn.edu, alampers@umn.edu}}
}



\maketitle
\thispagestyle{empty}
\pagestyle{empty}

\begin{abstract}
Classical results in neural network approximation theory show how arbitrary continuous functions 
can be approximated by networks 
with a single hidden layer, under mild assumptions on the activation function. However, the 
classical theory does not give a constructive means to generate the network parameters that 
achieve a desired accuracy. Recent results have demonstrated that for specialized activation 
functions, such as ReLUs, high accuracy can be achieved via linear combinations of 
\textit{randomly initialized} activations. 
%
These recent works utilize specialized integral representations of target functions that depend 
on the specific activation functions used. 
%
This paper defines \textit{mollified integral representations}, which provide a means to form 
integral representations of target functions using activations for which no direct integral 
representation is currently known.
%
The new construction enables approximation guarantees for randomly initialized networks using 
any activation for which there exists an established base approximation which may not be 
constructive.
We extend the results to the supremum norm and show how this enables application to an extended, 
approximate version of (linear) model reference adaptive control.
\end{abstract}

\section{Introduction}

%Machine learning is a wide-ranging umbrella topic that encompasses many different 
%methods and models, which can be used to bestow onto a machine or system the ability to 
%complete 
%various tasks without specific experience being hard coded or otherwise already possessed by 
%the 
%system \cite{mahesh2020machine}.
Adaptive control lies at the intersection of machine learning methods and the control of 
dynamics systems \cite{gaudio2019connections}, and has considerable development in the controls 
literature \cite{slotine1991applied,sastry1989adaptive,khalil2017high}. Deep neural network 
based machine learning approaches have achieved unprecedented levels of performance and utility
in areas such as language processing \cite{otter2020survey}, medical computer vision
\cite{esteva2021deep}, and reinforcement learning \cite{degrave2022magnetic}. In this paper, we
aim to contribute novel ideas to function approximation techniques and apply them to an 
approximate extension of Model Reference Adaptive Control (MRAC) established in 
\cite{lavretsky2013robust}.

A key challenge in function approximation work is that many theorems describe the existence of 
approximations that are linear combinations of some activation function (such as a sigmoid), but 
are not constructive. In \cite{pinkus1999approximation}, many of these results are reviewed, 
including classic results on the universal approximation properties of continuous 
sigmoidal activations 
\cite{cybenko1989approximation,hornik1989multilayer,funahashi1989approximate}, as well as 
another classic result by \cite{barron1993universal} for arbitrary sigmoidal
activations, which was expanded to arbitrary hinge functions by \cite{breiman1993hinging}.

Random approximation provides a key way to avoid the non-constructiveness by randomly sampling 
the internal parameters of the activations from a known set, if the target function of interest 
has an integral representation. However, determining these representations for general classes 
of target functions and activations is a nontrivial task.
\cite{irie1988capabilities} gives a constructive
method, but only for target and activation functions in $L_1$. In \cite{kainen2010integral} and 
\cite{petrosyan2020neural}, they propose constructive
methods for a class of target functions with unit step and ReLU activations respectively. In 
\cite{hsu2021approximation}, functions are approximated using trigonometric polynomial ridge 
functions, which can then be shown in expectation to be equivalent to randomly initialized ReLU 
activations.

There are several interesting and important results in the literature having to do with
neural networks with random (typically Gaussian) parameter initialization, sometimes as a 
consequence of using randomly initialized gradient descent for training the network. A classic
result by \cite{neal1994priors} shows that the output of a single hidden-layer network with
Gaussian randomly initialized parameters goes to a Gaussian Process as the width goes to
infinity. Similar results are then achieved in \cite{lee2017deep} for deep fully-connected
networks as all hidden layer widths go to infinity. Also for deep fully-connected networks, in
\cite{jacot2018neural} the authors define the Neural Tangent Kernel and propose that its limit,
as the hidden layer widths go to infinity, can be used to study the timestep evolution and
dynamics of the parameters, and the corresponding network output function, in gradient descent.
In \cite{poole2016exponential}, the authors show that single hidden-layer networks cannot achieve
the same rates of increase in a measure of curvature produced by the network output, as deep
networks can, with parameters Gaussian randomly initialized and bounded activation functions. In
\cite{arora2019fine}, the authors show that single hidden-layer networks of a sufficient width
can use Gaussian randomly initialized gradient descent on values of a target function, and
achieve guaranteed generalization to the entire function. And in \cite{du2019gradient}, the
authors show that deep fully-connected networks, where each hidden layer width meets a
sufficient size, can be trained with Gaussian randomly initialized gradient descent and be
guaranteed to reach the global minimum at a linear rate.

Our primary contributions in this paper are: developing a novel method for bridging convex 
combinations of activation functions (eg. step and ReLU) with unknown parameters to
approximations using randomly initialized parameters by using a mollified integral 
representation, obtaining a main overall result bounding the approximation error for the random 
approximations in the $L_2$ norm, extending these results to the $L_\infty$ norm under certain 
conditions, and then applying those results to an approximate extension of MRAC adaptive control.


The organization of the remaining parts of the paper are as follows. Section~\ref{sec:notation}
presents preliminary notation. Section~\ref{sec:apx:background} presents background on function 
approximation with neural networks. Section~\ref{sec:apx:mollApx} presents theoretical results 
on 
mollified approximations, while Section~\ref{sec:apx:random} gives an overall approximation 
error 
bound in $L_2$ norm for randomly initialized approximations. Section~\ref{sec:apx:application}
presents an extension to $L_\infty$ norm bounds and an application on approximate MRAC, and we 
provide closing remarks in Section~\ref{sec:conclusion}.


\section{Notation}
\label{sec:notation}

We use $\R,\N$ to denote the real and natural numbers. $\Ind$ denotes the indicator 
function, while $\Exp$ denotes the expected value. We use $\Bc,\Kc$ to denote bounded and 
compact convex subsets of $\R^n$. The integer set $\{1,\dots,k\}$ is denoted by $[k]$. 
$B_x(r)\subset\R^n$ denotes the radius $r$ Euclidean ball centered at $x\in\R^n$.
%For any $n,m\in\N$, we interpret $x,y\in \R^n$ as length $n$ column vectors and their transposes
%$x^\top,y^\top$ as length $n$ row vectors. $\norm{x}_2=\sqrt{x^\top\!x}$ will always denote the 
%standard Euclidean vector norm and inner product, with $x^\top\!y$ the dot product of two equal 
%length vectors. $0_n$, $1_n$ denote the length $n$ vector of zeros and ones respectively. 
%The notation that $A$ is a $n\times m$ matrix assumes real entries, thus $A\in\R^{n\times m}$ 
%and its transpose $A^\top\in\R^{m\times n}$.
%If $A$ is a square matrix its trace is denoted by $\tr(A)$. $I_n$ denotes the $n\times n$
%identity matrix. $A\prec B$ ($\preceq$) denotes that the matrix $B-A$ is positive 
%(semi)definite. And so, $aI_n\preceq A\preceq bI_n$ denotes that $\text{eig}(A)\in[a,b]$.
%
%Index subscripts on vectors and matrices denote the row index, for example $A_i^\top$ is the
%$i$th row of $A^\top$. The Frobenius matrix norm is denoted $\|A\|_F$. We denote the product of
%matrix $A\in\R^{n\times m}$ with vector $x\in\R^m$ as $Ax$,
%which is a length $n$ (column) 
%vector where the $i$th (row) element is the inner product $(Ax)_i$. If $A\in\R^{n\times m}$ and 
%$x\in\R^n$, then the matrix product $A^\top\!x$ results in a length $m$ vector where we denote 
%the $i$th element $(A^\top\!x)_i$ equivalently as the inner product $A_i^\top\!x$.
%
%We denote time indices using subscripts, for all (scalar, vector, or matrix valued) functions 
%of 
%time. For example, $x_t$ or $A_t$, instead of $x(t)$ or $A(t)$. Random variables are denoted in 
%bold, eg. $\xb$, and so stochastic processes also have a time subscript, eg. $\xb_t$. The $i$th 
%row of a time-varying vector or matrix is denoted with $t,i$ subscript.
We interpret $w,x\in\R^n$ as column vectors and denote their inner product as $w^\top\!x$.
Similarly, we denote the product of matrix $W\in\R^{n\times N}$ with vector $x$ as $W^\top\!x$,
which is a length $N$ vector where the $i$th element is the inner product $W_i^\top\!x$.
Subscripts on vectors and matrices denote the row index, for example $W_i^\top$ is the $i$th row
of $W^\top$. The standard Euclidean norm is denoted
$\|w\|$.
% The integer set $\{1,\dots,k\}$ is denoted by $[k]$.
We use subscripts on
time-dependent variables to reduce parentheses, for example $\phi(x(t))$ is instead denoted
$\phi(x_t)$. The $i$th row of a time-varying vector or matrix is then $x_{t,i}$.


\section{Background on Network Calculus}
\label{sec: background}


\begin{figure*}[tbh]
\centering
\begin{subfigure}[b]{0.3\textwidth}
    \centering
    \includegraphics[width=\linewidth]{images/in-out.png}
    \caption{Arrival and departure data and their relation with delay $d(t)$ and backlog $b(t)$. For a FIFO system, the delay is the horizontal distance between $R(t)$ and $R^*(t)$ but some other multiplexing techniques may shift the data to a later priority, causing a longer delay.}
    \label{fig: data in-out}
\end{subfigure}
\hfill
\begin{subfigure}[b]{0.35\textwidth}
    \centering
    \includegraphics[width=\linewidth]{images/arrival-service.png}
    \caption{Characteristics of an arrival curve and a service curve. From any point of observation, the arriving data never exceeds its arrival curve; the departure data is also never less than the service curve with respect to the data arrival.}
    \label{fig: arrival-service curves}
\end{subfigure}
\hfill
\begin{subfigure}[b]{0.33\textwidth}
    \centering
    \includegraphics[width=\linewidth]{images/bound.png}
    \caption{Delay and backlog bounds of a system. Backlog is the maximum vertical distance between $\alpha(t)$ and $\beta(t)$; FIFO delay is their maximum horizontal distance; but for arbitrary multiplexing, the delay guarantee is when the system clears its buffer, thus it's the intersection of $\alpha(t)$ and $\beta(t)$.}
    \label{fig: system bounds}
\end{subfigure}
\caption{Network calculus framework. We let $R(t)$ and $R^*(t)$ be the arrival and departure data flow of a system; $\alpha(t)$ be the piecewise linear concave arrival curve and $\beta(t)$ be the piecewise linear convex service curve of a system.}
% \hossein{Better to show piece-wise linear concave arrival curve and piece-wise linear convex service curve instead of token-bucket and rate-latency.}}
\end{figure*}

We recall some of the network calculus essentials for a better understanding of the framework used in Saihu. In the following context, we use the following notation: $\mbb{R}^+$ is the set of non-negative real numbers; $[x]_+$ denotes $\max(0, x)$

The data flow is by convention modeled as a left-continuous wide-sense increasing function $R(t): \mbb{R}^+ \mapsto \mbb{R}^+$ with respect to time $t$~\cite{ncbook2001leboudec}. 

A system $\mcal{S}$ receives arrival data described as a cumulative function $R(t)$ and delivers departure data as another cumulative function $R^*(t)$. Figure~\ref{fig: data in-out} illustrates such a system $\mcal{S}$. The benefit of representing a system like this is that we can observe system backlog and delay with such a model. 

\begin{definition}[Backlog and Delay~\cite{ncbook2001leboudec}]
    The backlog of a system at time~$t$ is
    \begin{equation}
        b(t) = R(t) - R^*(t)
    \end{equation}
    
    The virtual delay of a FIFO system at time $t$ is
    \begin{equation}
        d_{FIFO}(t) = \inf \lbp \tau \geq 0 : R(t) \leq R^*(t+\tau) \rbp
    \end{equation}
\end{definition}



The backlog of a system can be viewed as the vertical distance between $R$ and $R^*$. The FIFO (\textit{First-in First-out}) delay is the horizontal distance between $R$ and $R^*$. One may obtain other delay values if the multiplexing technique is not FIFO.

% \begin{figure}
%     \centering
%     \includegraphics[width=0.9\linewidth]{images/in-out.png}
%     \caption{In/out data flow; delay and backlog}
%     \label{fig: data in-out}
% \end{figure}

Since we are interested in the system guarantee instead of a single instance of data flow, we would like to have general bounds to the arrival and departure data flows. Therefore, we define \textit{arrival curve} and \textit{service curve} as the bounds of arrival and departure data flows.

\begin{definition}[Arrival Curve~\cite{ncbook2001leboudec}]
    Given a wide-sense increasing function $\alpha: \mbb{R}^+ \mapsto \mbb{R}^+$, we say that a flow $R(t)$ is $\alpha$-constrained if and only if for all $s \leq t$:
    \begin{equation}
        R(t) - R(s) \leq \alpha(t-s)
    \end{equation}
    We say $R(t)$ has $\alpha$ as an arrival curve.
\end{definition}

\begin{definition}[Service Curve~\cite{ncbook2001leboudec}]
    Given a wide-sense increasing function $\beta: \mbb{R}^+ \mapsto \mbb{R}^+$ and $\beta(0) = 0$. A system $\mcal{S}$ having $R(t)$ and $R^*(t)$ as its arrival and departure flows. We say $\mcal{S}$ offers a service curve $\beta$ if and only if
    \begin{equation}
        R^*(t) \geq (R \otimes \beta)(t) =: \inf_{s \leq t} \lbp R(s) + \beta(t-s) \rbp
    \end{equation}
    where $\otimes$ denotes the min-plus convolution
\end{definition}

Figure~\ref{fig: arrival-service curves} illustrates the arrival and service curves. Any segment of arrival flow $R(t)$ is constrained by arrival curve $\alpha$ and the output curve $R^*(t)$ is always no less than the curve $R\otimes\beta$. As a result, an arrival curve upper bounds the incoming traffic, and a service curve lower bounds the outgoing traffic.

% \begin{figure}
%     \centering
%     \includegraphics[width=\linewidth]{images/arrival-service.png}
%     \caption{Arrival/Service curve}
%     \label{fig: arrival-service curves}
% \end{figure}

We consider 2 special types of curves throughout this paper, \textit{token-bucket} (or sometimes called \textit{leaky-bucket}) curve and \textit{rate-Latency} curve.

\begin{definition}[Token-bucket and Rate-latency~\cite{ncbook2001leboudec}]
    A token-bucket curve $\gamma_{r,b}$ with arrival rate $r$ and burst $b$ is defined as
    \begin{equation}
        \gamma_{r,b}(t) = b + rt
    \end{equation}

    A rate-latency curve $\beta_{R,T}$ with service rate $R$ and latency $T$ is defined as
    \begin{equation}
        \beta_{R,T}(t) = R \lb t - T \rb_+
    \end{equation}
\end{definition}

A token-bucket curve is determined by a burst $b$ and an arrival rate~$r$. Burst represents the maximum possible data volume that can arrive simultaneously, and arrival rate represents the maximum long-term data rate~\cite{bouillard2022tradeoff}.
A rate-latency curve is determined by a latency~$T$ and a service rate~$R$. Latency represents the time a server needs before starting to process the incoming data, and service rate represents the minimum rate to process data after the initial latency.

With the help of arrival and service curves, we can derive delay and backlog bounds for a system $\mcal{S}$ illustrated in Figure~\ref{fig: system bounds}. Suppose a system $\mcal{S}$ has arrival curve $\alpha$ and service curve~$\beta$, its worst-case backlog $b^*$ is the maximum vertical distance between~$\alpha$ and~$\beta$. Similarly, depending on the multiplexing technique applied to the system, its worst-case delay bound $d^*$ is the maximum horizontal distance between $\alpha$ and $\beta$ if $\mcal{S}$ is a FIFO system. If we don't have any information about its multiplexing technique, referred to as arbitrary multiplexing, the best we can say is that when $\alpha$ and $\beta$ intersect each other, where all data has been delivered out of the system. Consequently, the worst-case delay bound for arbitrary multiplexing is the time required for $\mcal{S}$ to clear its buffer.

% \begin{figure}
%     \centering
%     \includegraphics[width=\linewidth]{images/bound.png}
%     \caption{System delay/backlog bounds}
%     \label{fig: system bounds}
% \end{figure}

While a service curve captures the slowest possible output speed of a system, a link's transmission capacity limits the speed as well. Hence, we model this phenomenon using a \textit{greedy shaper} with a sub-additive function $\sigma: \mbb{R}^+ \mapsto \mbb{R}^+$ concatenated with a server. We consider a concatenation as shown in Figure \ref{fig: system}. By convention we assume $\sigma(0) = 0$ and $\beta(t) \leq \sigma(t), \forall t \in \mbb{R}^+$, meaning that the buffer is cleared at the beginning and the service never exceed its physical limitation. With the above definition, such greedy shaper conserves the service provided by the system due to theorem \ref{thm: shaping}.

\begin{figure}[thb]
    \centering
    \includegraphics[width=0.7\linewidth]{images/system.png}
    \caption{Shaping of departure data. A flow that has an arrival curve $\alpha$ feeds into a server with an arrival data flow $R(t)$. The server having service curve $\beta$ takes $R(t)$ and gives a departure data flow $R^*(t)$ to a shaper with shaping function $\sigma$. The shaper takes $R^*(t)$ and shape the data flow as another departure $D(t)$.}
    \label{fig: system}
\end{figure}


\begin{theorem}[Shaping conserves service \cite{ncbook2001leboudec}]
\label{thm: shaping}
Following the system shown in Figure \ref{fig: system}, we have
\begin{equation}
     D = R^* \otimes \sigma \geq \lp R \otimes \beta \rp \otimes \sigma = R \otimes \lp \beta \otimes \sigma \rp = R \otimes \beta
\end{equation}
\end{theorem}

In the following context, we model the shaping function $\sigma$ as a token-bucket curve $\gamma_{C,L}$ with transmission capacity $C$ and the packet size $L$ to capture the link capacity and packetization~\cite{bouillard2022tradeoff}.


\section{Mollified Approximations}
\label{sec:apx:mollApx}
In this section, we define the \textit{mollified approximation} $\fhlN$ using a novel method of 
forming a \textit{mollified integral representation} for a base approximation $\fhN$ to a 
target function $f\in\Cs(\Bc)$, and bound their difference in the $L_2(\Bc,\muB)$ norm.


\subsection{Functions with Integral Representation}
If a target function $f:\R^n\to\R$ has an \textit{integral representation} over the parameters
set $\Upsilon$ with some activation function $\sigma:\R\to\R$, then
\begin{equation}\label{eq:apx:intRep}
f(x) \ = \ \int_\Upsilon \gs(\upgamma)\,\sigma(\parX)\,\d\upgamma
% \hspace{40pt} \forall x\in\Bc
\end{equation}
holds with some bounded, continuous \textit{coefficient function} \mbox{$\gs:\Upsilon\to\R$}. 
This is a version of \eqref{eq:apx:baseApx} that is continuous in the parameters $\upgamma_i$ 
over $\Upsilon$. Indeed, if we define $\upgamma_1,\dots,\upgamma_N$ to be a sequence of points 
that uniformly grids $\Upsilon$ with volume $\d_N$ between any neighboring points, then as
$N\to\infty$ (and $\d_N\to0$) we have
$$
\lim\limits_{N\to\infty}\sum_{i=1}^{N}\gs(\upgamma_i)\,\sigma(\pariX)\,\d_N 
= \int_\Upsilon \gs(\upgamma)\,\sigma(\parX)\,\d\upgamma = f(x)
%\forall x\in\Bc
\ .
$$
This implies that the (infinite) sum of the absolute values of the coefficients is finite, with
$$
%C_\infty := 
\lim\limits_{N\to\infty}\sum_{i=1}^{N}|\gs(\upgamma_i)|\,\d_N \ 
= \ \int_\Upsilon |\gs(\upgamma)|\,\d\upgamma \ < \ \infty
$$
holding since $\gs$ is bounded and the integral is over the bounded set $\Upsilon$.

However, finding such integral representations \eqref{eq:apx:intRep} for general classes of
functions $f$ and activation functions $\sigma$ is a nontrivial task. If \textit{both}
are $L_1$ (absolutely integrable over $\R^n$ and $\R$ respectively), then Theorem 1 in
\cite{irie1988capabilities} shows that the integral representation can always be constructed
with $\gs$ related to the Fourier transforms of $f$ and $\sigma$. But this is a limiting
restriction, since none of the typical activation functions like sigmoids, steps, ReLU's, and 
their variants meet this requirement. Even so, in \cite{funahashi1989approximate} this was used 
to prove the universal approximation properties of sigmoid activations.
%In
%\cite{hsu2021approximation}, it is shown that a particular definition of a trigonometric ridge
%function has an integral representation with ReLU activations, and then those ridge functions
%can be the basis functions as in \eqref{eq:apx:genApx} for target functions in $L_2([-1,1]^n)$.
%But again, this is a limited class of target functions compared to $\Cs(\Bc)$.


\subsection{Mollified Integral Representation}
\label{sec:apx:moll}
Now we will propose a novel method of forming an integral representation for
\textit{any} function $f\in\Cs(\Bc)$, so long as a base approximation $\fhN$ of the form 
\eqref{eq:apx:baseApx} exists and the bounded parameter set $\Upsilon\subset\R^{n+1}$ and a
bound $\Sc_N$ on the coefficient size $\sum_{i=1}^N|\theta_i|$ are known.

For any fixed parameters $\upgamma_1,\dots,\upgamma_N\in\Upsilon$, this approximation $\fhN$ 
can be defined as
\begin{align}\nonumber
\fhN(x) \ =& \  \sum_{i=1}^{N} \theta_i\,\sigma(\pariX) \\\label{eq:apx:baseIntRep}
=& \ 
\int_\Upsilon \,\sum_{i=1}^N\theta_i\,\deltano(\upgamma-\upgamma_i)\ \sigma(\parX)\,\d\upgamma \ 
\ ,
\end{align}
where $\deltano$ is the $(n+1)$-dimensional Dirac delta which satisfies for any
$\upgamma_i\in\Upsilon$ that
$\int_\Upsilon\deltano(\upgamma-\upgamma_i)\,\sigma(\parX)\,\d\upgamma =\sigma(\pariX)\ $.

Thus, the RHS of \eqref{eq:apx:baseIntRep} is like an integral representation 
\eqref{eq:apx:intRep} of $\fhN$ with a coefficient mapping \mbox{$\gs(\upgamma) = \sum_{i=1}^N 
\theta_i\,\deltano(\upgamma-\upgamma_i)$}.
However, this mapping is not continuous and bounded as required, and the Dirac delta is not even
a function in the regular sense. Indeed, if the overall coefficients of a random approximation 
$\fhR$ were set as $\vartheta_j=\gs(\upgammab_j)$ for each randomly sampled 
$\upgammab_1,\dots,\upgammab_R\in\Upsilon$, then in this case they almost surely would all be 
zero.

We propose the \textit{Mollified integral representation} to solve this issue, where the Dirac
delta in \eqref{eq:apx:baseIntRep} is replaced with the \textit{mollified} delta function
$\deltalno$. Recalling that $\upgamma=[w_1\ \cdots\ w_n\ b\,]$, this is defined as the 
$(n+1)$-dimensional bump function
\begin{align}\label{eq:apx:mollDel}
&\deltalno(\upgamma) \ := \ \\\nonumber
&\begin{cases}
(\eta\lambda)^{n+1}\prod_{i=1}^n
\exp\left(\dfrac{-1}{1-\lambda^2w_i^2}\right)
%\cdots\,
%\exp\left(\dfrac{-1}{1-\lambda^2w_n^2}\right)\,
\exp\left(\dfrac{-1}{1-\lambda^2b^2}\right)\\[8pt]
\hspace{160pt} \upgamma\in(-\frac{1}{\lambda},\frac{1}{\lambda})^{n+1} \\
0 \hspace{155pt} \text{otherwise}
\end{cases}
\end{align}
with $\eta^{-1}=\int_{-1}^{1}\exp\left(\frac{-1}{1-y^2}\right)\d y \approx0.444$ thus giving
$\int\deltalno(\upgamma-\upgamma_i)\,\d\upgamma = \int\deltano(\upgamma-\upgamma_i)\,\d\upgamma
= 1$ for any $n\geq1$ and $\upgamma_i\in\R^{n+1}$.

We term the positive scalar $\lambda>0$ as the \textit{mollification factor}. This controls the
height and width of the \mbox{mollified} delta $\deltalno$. Since the height scales as
$\lambda^{n+1}$ and the integral remains constant, this requires the width to shrink. 
Figure~\ref{fig:apx:mollFactor} plots the 1-dimensional case for different $\lambda$ values.
%\begin{figure}[H]
\begin{figure}
\centering
\includegraphics[width=\columnwidth]{mollFactor}
\caption{\label{fig:apx:mollFactor}The mollified delta $\deltal$ for different $\lambda$ values.
In each case, $\int_{-1/\lambda}^{1/\lambda}\deltal(\upgamma)\d\upgamma=1$.}
\end{figure}

The support of $\deltalno(\upgamma-\upgamma_i)$ is the $(n+1)$-dimensional open cube set
$(-\frac{1}{\lambda},\frac{1}{\lambda})^{n+1}$ centered about any $\upgamma_i\in\Upsilon$. And
so, if $\upgamma_i$ is sufficiently close to (or on) the boundary of $\Upsilon$, then it no
longer holds that the entire support of $\deltalno(\upgamma-\upgamma_i)$ is within $\Upsilon$.
Thus, we define the \textit{$\lambda$-expanded} parameters set $\Upsilonl$ such that the support
is always fully contained, up to and include the boundary of $\Upsilon$. For example, if
$\Upsilon$ is rectangular product of intervals $[\alpha_1,\beta_1]\times\cdots\times
[\alpha_n,\beta_n]$, then each interval is expanded as 
$[\alpha_i-\frac{1}{\lambda},\beta_i+\frac{1}{\lambda}]$ to obtain $\Upsilonl$.
With this $\lambda$-expanded parameters set, we always maintain that
$\int_\Upsilonl\deltalno(\upgamma-\upgamma_i)\d\upgamma=1$ for all $\lambda>0$, $n\geq1$, and any
$\upgamma_i\in\Upsilon$.

Since $\Upsilon\subset\Upsilonl$ for all $\lambda>0$, then \eqref{eq:apx:baseIntRep} is 
equivalently
\begin{equation}\label{eq:apx:baseIntRepLam}
\fhN(x) = \int_\Upsilonl \,\sum_{i=1}^N\theta_i\,\deltano(\upgamma-\upgamma_i)\ 
\sigma(\parX)\,\d\upgamma \ \ .
\end{equation}
And so, by replacing the Dirac delta in \eqref{eq:apx:baseIntRepLam} with the mollified delta
$\deltalno$, for any choice of $\lambda>0$, we obtain the corresponding \textit{mollified 
approximation}
\begin{equation}\label{eq:apx:mollApx}
\fhlN(x) := \int_\Upsilonl \,\sum_{i=1}^N\theta_i\,\deltalno(\upgamma-\upgamma_i)\ 
\sigma(\parX)\,\d\upgamma
\end{equation}
using a \textit{mollified integral representation}.
This now gives the required continuous, bounded coefficient mapping for any fixed
$\upgamma_1,\dots,\upgamma_N\in\Upsilon$ as
\begin{equation}\label{eq:apx:gLam}
\gl(\upgamma) := \sum_{i=1}^N\theta_i\,\deltalno(\upgamma-\upgamma_i)
%\ \leq \ \left(\frac{\eta\lambda}{e}\right)^{n+1}\sum_{i=1}^N|\theta_i|
\ \ .
\end{equation}
We illustrate the transformation for the 1-dimensional case conceptually in 
Figure~\ref{fig:apx:moll}.
%\begin{figure}[H]
\begin{figure}
\centering
\includegraphics[width=.9\columnwidth]{moll}
\caption{\label{fig:apx:moll}The base approximation $\fhN$ is transformed into the mollified
approximation $\fhlN$ by replacing Dirac deltas $\delta$ with the mollified deltas $\deltal$.}
\end{figure}

\subsection{Main Result}
Let us define $E(x):=\|\,[x_1\ \cdots\ x_n\ 1]\,\|_2=\|X\|_2$ as the Euclidean norm mapping from 
$\R^n\times\{1\}$ to $\R$, and then define
\begin{equation}\label{eq:apx:EB}
%\Esup := \normSupB{E(x)}
%\qquad\text{and}\qquad
\EB := \normB{E(x)}
\end{equation}
as the $L_2(\Bc,\muB)$ norm of this mapping over the Euclidean distances of $x\in\Bc$. We can now
state the main result, which bounds the difference between a base approximation and its 
corresponding mollified approximation, for a chosen mollifcation factor $\lambda$, in the 
$L_2(\Bc,\muB)$ norm.

\begin{theorem}\label{thm:apx:moll}
\textit{
Let there exist a base approximation $\fhN$ of the form \eqref{eq:apx:baseApx} using any 
activation function $\sigma$ satisfying Assumption~\ref{sigAsmpt}, for any $N\in\N$, with some 
unknown parameters $\upgamma_1,\dots,\upgamma_N\in\Upsilon$ and coefficients $\theta\in\R^N$, 
but having a known bound on the coefficient size of $\sum_{i=1}^N|\theta_i|\leq \Sc_N$ and the 
parameters set $\Upsilon$ known. Then, the mollified approximation $\fhlN$ over the 
$\lambda$-expanded parameters set $\Upsilonl$ in \eqref{eq:apx:mollApx} satisfies either of:}
\begin{align}\label{eq:apx:epslSupBound}
i)\hspace{55.5pt} \normB{\fhN-\fhlN} \ &\leq \ \Sc_N \, \Dsig \\\label{eq:apx:epslL2Bound}
ii)\hspace{52pt} \normB{\fhN-\fhlN} \ &\leq \ \Sc_N\, \frac{\Lsig\,\sqrt{n+1}}{\lambda}\ \EB \ .
\end{align}
\end{theorem}
\begin{proof}
\if\ARXIV1
Given in Appendix~\ref{app:apx:mollProof}.
\fi
\if\ARXIV0
Given in Appendix I of \cite{lekang2023functionfull}.
\fi
\end{proof}

\begin{remark}
The results of Theorem~\ref{thm:apx:moll} show that if the activation function $\sigma$ is 
Lipschitz over its feasible inputs, then as $\lambda\to\infty$ we have $\fhlN(x)\to\fhN(x)$ in 
the sense of the $L_2(\Bc,\muB)$ norm.
\end{remark}

\section{Randomly Initialized Activations with M.I.R.}\label{sec:random}
In this section we motivate and prove error bounds in the $\normB{\cdot}$ norm for an
approximation of the form \eqref{singleHidden} with internal parameters randomly initialized
over the set $\Thetal$, using the mollified integral representation previously defined in
Section~\ref{sec:mollified}. \\\\
Let us denote $\ftr(x,\Thetabr)$ as an approximation of the form \eqref{singleHidden} which uses
$\Thetabr = \{\thetab_1,\dots,\thetab_R\}\in\Thetalr$ sampled iid according to some nonzero
probability density $\Pl$, and define $\pmin=\min_{\theta\in\Thetal}\Pl(\theta)>0$. The key property is that $P_\lambda$ is an arbitrary positive density. For example, we could take $P_{\lambda}$ to be uniform over $\Thetal$.
\\\\
The proof in this section then closely follows Lemma 4 in \cite{rahimi2008weighted}.\\\\

The main result for this section is shown next. 
\begin{theorem}\label{thRand}
For any target function $f:\R^n\to\R$ that is continuous on a bounded set $B\subset\R^n$,
parameters set $\Theta$, mollification $\lambda>0$, and activation function $\sigma$ satisfying
Assumption~\ref{sigAsmpt}, let $\fhlN$ be a mollified approximation defined in \eqref{molliFunc}.
% the random approximation
% $\ftr(x,\Thetabr)$ can be bounded as follows.
With probability greater than 1-$\nu$, for any
$\nu\in(0,1)$, over the iid draw $\Thetabr\simi\Pl$, there is a random approximation
$\ftr(x,\Thetabr)$ satisfying
% \begin{align*}
% &\normB{\fhlN(x)-\ftr(x,\Thetabr)} \\\nonumber
% &\hspace{20pt} \leq \frac{\Lc\,\Cgl}{\pmin \sqrt{R}}
% \bigg(\norm{\thmax}\ZB + \frac{|\sigma(0)|}{\Lc} \\ \nonumber
% & \hspace{100pt} + \Dt\ZB\sqrt{\tfrac{1}{2}\log\left(\tfrac{1}{\nu}\right)}\,\bigg)
% \end{align*}
\begin{multline}
  \nonumber
\normB{\fhlN(x)-\ftr(x,\Thetabr)} 
 \leq \frac{\Lc\,\Cgl}{\pmin \sqrt{R}}\\
\cdot\bigg(\norm{\thmax}\ZB + \frac{|\sigma(0)|}{\Lc} %\\ 
 + \Dt\ZB\sqrt{\tfrac{1}{2}\log\left(\tfrac{1}{\nu}\right)}\,\bigg)
\end{multline}
with $|c_j|\leq\frac{\Cgl}{\pmin\,R}$ for all $j\in[R]$, where \\
$\Cgl = \sum_{i=1}^N|c_i| \left(\frac{\eta\lambda}{e}\right)^{n+1}$.
\end{theorem}
\begin{proof}
First, we observe that \eqref{molliFunc} can be written as
\begin{align}\label{fhlNgRep}
\fhlN(x) &= \int_\Thetal \sum_{i=1}^{N} c_i\,\deltalno(\theta-\theta_i)\,\sigma(\theta\cdot z)\,\d\theta \\\nonumber
&= \int_\Thetal\gl(\theta)\,\sigma(\theta\cdot z)\,\d\theta \ \ ,
\end{align}
with $\gl:\Thetal\to\R$ being such that\\
$\max_{\theta\in\Thetal}\limits|\gl(\theta)|\leq \sum_{i=1}^N
|c_i|\left(\frac{\eta\lambda}{e}\right)^{n+1} = \Cgl$. \\\\
Next, we explicitly set the coefficients of $\ftr$ to
\begin{equation}\label{ujDef2}
c_j = \frac{\gl(\theta_j)}{\Pl(\theta_j)\,R}
\end{equation}
for all $j\in[R]$, which immediately satisfies
\begin{equation}\label{uMax}
|c_j|=\frac{|\gl(\theta_j)|}{\Pl(\theta_j)\,R}\leq\frac{\Cgl}{\pmin\,R} \ \ .
\end{equation}
Denoting $\d\Pl(\theta_j)=\Pl(\theta_j)\,\d\theta_j$ for all $j\in[R]$, then the iid assumption on $\thetab_i$ implies that 
\begin{align*}
&\Exp_{\Thetabr\simi\Pl}\limits\left[\ftr(x,\Thetabr)\right] \\
%&\, = \int_\Thetal\cdots\int_\Thetal \sum_{j=1}^R\frac{\gl(\theta_j)}{\Pl(\theta_j)\,R}
%\sigma(\theta_j\cdot z)\d\Pl(\theta_1)\cdots\d\Pl(\theta_R) \\
%&\overset{(a)}{=} \int_\Thetal\cdots\int_\Thetal \frac{\gl(\theta_1)}{\Pl(\theta_1)\,R}\,
%\sigma(\theta_1\cdot z)\,\d\Pl(\theta_1)\cdots\d\Pl(\theta_R) + \\
%& \cdots + \int_\Thetal\cdots\int_\Thetal \frac{\gl(\theta_1)}{\Pl(\theta_1)\,R}\,
%\sigma(\theta_1\cdot z)\,\d\Pl(\theta_1)\cdots\d\Pl(\theta_R) \\
%&\overset{(a)}{=} \int_\Thetal\frac{\gl(\theta_1)}{R}\,
%\sigma(\theta_1\cdot z)\,\frac{\d\Pl(\theta_1)}{\Pl(\theta_1)}\ \ 1^{R-1} \ + \cdots \\
%& \hspace{50pt} + \int_\Thetal\frac{\gl(\theta_R)}{R}\,
%\sigma(\theta_R\cdot z)\,\frac{\d\Pl(\theta_R)}{\Pl(\theta_R)}\ \ 1^{R-1} \\
&= \frac{1}{R}\Bigg(R\cdot \int_\Thetal \gl(\theta)\,
\sigma(\theta\cdot z)\,\d\theta \Bigg) 
= \fhlN(x) \ \ .
\end{align*}
% Here (a) is because each $\theta_j$ only belongs to
% the corresponding $\d\Pl(\theta_j)$ integral, while the remaining $R-1$ integrate to $1$,
% (b) is because there are $R$ copies of the same integral, and (c) is by \eqref{fhlNgRep}.\\\\
Next we define the function $h:\Thetalr\to\R$ as
\begin{align}\nonumber
h(\Thetabr) &= \normB{\ftr(x,\Thetabr) - \Exp_{\Thetabr\simi\Pl}\limits
\left[\ftr(x,\Thetabr)\right]} \\ \label{hDef}
&= \ \normB{\fhlN(x) - \ftr(x,\Thetabr)} \ \ .
\end{align}
Thus, for all $(\theta_1,\dots,\theta_R),(\thetaw_1,\dots,\thetaw_R)\in\Thetalr$ such that
$\theta_j=\thetaw_j$ for all $j\neq i$, we have that 
\begin{align}\nonumber
&\big|h(\theta_1,\dots,\theta_R)-h(\thetaw_1,\dots,\thetaw_R)\big| \\ \nonumber
&= \Big|\,\normB{\fhlN(x) - \ftr(x,\theta_1,\dots,\theta_R)} \\ \nonumber
&\hspace{40pt} - \normB{\fhlN(x) - \ftr(x,\thetaw_1,\dots,\thetaw_R)}\,\Big|\\ \nonumber
&\overset{(a)}{\leq} \big\|\fhlN(x) - \ftr(x,\theta_1,\dots,\theta_R) \\ \nonumber
&\hspace{40pt} - \fhlN(x) + \ftr(x,\thetaw_1,\dots,\thetaw_R) \big\|_B \\\nonumber
&\overset{(b)}{=} \normB{\frac{\gl(\theta_i)}{\Pl(\theta_i)\,R}\,
\sigma(\theta_i\cdot z) - \frac{\gl(\thetaw_i)}{\Pl(\thetaw_i)\,R}\,
\sigma(\thetaw_i\cdot z) } \\ \nonumber
&\overset{(c)}{\leq} \frac{\Cgl}{\pmin\,R}\,\normB{\sigma(\theta_i\cdot z) -
                                          \sigma(\thetaw_i\cdot z) } \\
  %\nonumber
%&\overset{(d)}{=} \frac{\Cgl}{\pmin\,R}\,\normB{\sigmaw(\theta_i\cdot z) -
%\sigmaw(\thetaw_i\cdot z) } \\ \nonumber
&\overset{(d)}{\leq} \frac{\Lc\Cgl}{\pmin\,R}\,\normB{(\theta_i -
                                            \thetaw_i)\cdot z }
  %\\
  \label{absHbound}
  %&
    \overset{(e)}{\leq} \frac{\Lc\Cgl}{\pmin\,R}\,\Dt\,\ZB
\end{align}
holds for all $i\in[R]$. Here (a) is by the reverse triangle inequality, (b) is by
\eqref{ujDef2}, (c) is by \eqref{uMax}, (d) is by the $\Lc$-Lipschitz property of $\sigma$, and
(e) is by Cauchy-Schwarz and the definitions of $\Dt$ and $\ZB$. \\\\
To bound the expectation of $h(\Thetabr)$, we begin with
\begin{align}\label{varNormK}
&\Exp_{\Thetabr\simi\Pl}\limits \normB{\ftr(x,\Thetabr) - \Exp_{\Thetabr\simi\Pl}\limits
\left[\ftr(x,\Thetabr)\right]}^2 \\ \nonumber
&\overset{(a)}{\leq} \frac{1}{R}\Exp_{\thetab\sim\Pl}\limits
\normB{\frac{\gl(\thetab)}{\Pl(\thetab)}\sigma(\thetab\cdot z)}^2 \\ \nonumber
&\leq \frac{\Cgl^2}{\pmin^2\,R}\,\Exp_{\thetab\sim\Pl}\limits
\normB{\sigma(\thetab\cdot z) }^2 \\\nonumber
&= \frac{\Cgl^2}{\pmin^2\,R}\,\Exp_{\thetab\sim\Pl}\limits\normB{\sigmaw(\thetab\cdot z) +
                                                \sigma(0)}^2
  %\\
%  \label{frBound}
%&\overset{(b)}{\leq} \frac{\Lc^2\,\Cgl^2}{\pmin^2\,R}\,\Exp_{\thetab\sim\Pl}\limits
%\normB{\thetab\cdot z + \frac{\sigma(0)}{\Lc}}^2 \ \ .
\end{align}
Here (a) is the identity \eqref{eq:varBnd} (defined in the appendix), and $\sigmaw(y)=\sigma(y)-\sigma(0)$.  

Now note that $\sqrt{\Exp_{\thetab\sim\Pl}\|f(x,\thetab)\|_B^2}$ defines a norm over functions, $f(x,\thetab)$.
% and (b) is by the
% $\Lc$-Lipschitz property of $\sigmaw(y)=\sigma(y)-\sigma(0)$ and that $\sigmaw(0)=0$, thus
% we have $\sigmaw(y)\leq\Lc y$. 
\\\\
The bound is completed by using the triangle inequality, followed by the Lipschitz property of $\sigmaw(y)$, and finally the upper bounds on $\theta$: 
\begin{align}\nonumber
&\Exp_{\Thetabr\simi\Pl}\limits\left[h(\Thetabr)\right]
= \Exp_{\Thetabr\simi\Pl}\limits\left[\sqrt{h^2(\Thetabr)\,}\,\right] \\\nonumber
%&\leq \sqrt{\Exp_{\Thetabr\simi\Pl}\limits\left[h^2(\Thetabr)\,\right]} \\\nonumber
%&= \sqrt{\Exp_{\Thetabr\simi\Pl}\limits \normB{\ftr(x,\Thetabr) - \Exp_{\Thetabr\simi\Pl}\limits
%\left[\ftr(x,\Thetabr)\right]}^2 } \\\nonumber
&\leq \frac{\Cgl}{\pmin\,\sqrt{R}\,}\,\sqrt{\Exp_{\thetab\sim\Pl}\limits 
                                                  \normB{\sigmaw(\thetab\cdot z) + \sigma(0)}^2} \\
  \nonumber
  &\leq \frac{\Cgl}{\pmin\,\sqrt{R}\,}\left( \sqrt{\Exp_{\thetab\sim\Pl}\limits 
                                                  \normB{\sigmaw(\thetab\cdot z)}} + |\sigma(0)|\right)
  \\
  \nonumber
  &\le
 \frac{\Cgl}{\pmin\,\sqrt{R}\,}\left(\Lc \sqrt{\Exp_{\thetab\sim\Pl}\limits 
                                                  \normB{\thetab\cdot z}} + |\sigma(0)|\right)
    \\
%  \nonumber
%&\overset{(a)}{\leq} \frac{\Lc\,\Cgl}{\pmin\,R}\,\sqrt{
%\normB{\norm{\thmax}\norm{z} + \frac{\sigma(0)}{\Lc}}^2} \\\nonumber
%&\overset{(b)}{\leq} \frac{\Lc\,\Cgl}{\pmin\,\sqrt{R}\,}\,\,\left(\sqrt{
%\Exp_{\thetab\sim\Pl}\limits \int_B (\thetab\cdot z)^2\,\muBdx} 
%+ \frac{\sigma(0)}{\Lc}\right) \\\nonumber
%&\overset{(c)}{\leq} \frac{\Lc\,\Cgl}{\pmin\,\sqrt{R}\,}\,\left( 
%\sqrt{\int_B \norm{\thmax}^2\norm{z}^2 \,\muBdx} + \frac{\sigma(0)}{\Lc} \ \right)
%   \nonumber
%&\leq \frac{\Lc\,\Cgl}{\pmin\,\sqrt{R}\,}\,
%\left(\Big\|\norm{\thmax}\norm{z}\Big\|_B + \frac{|\sigma(0)|}{\Lc}\right) 
  %\\
  \label{expH}
&= \frac{\Lc\,\Cgl}{\pmin\,\sqrt{R}\,}\,
\left(\norm{\thmax}\ZB + \frac{|\sigma(0)|}{\Lc}\right) \ \ .
\end{align}
% Here (a) is by Cauchy-Schwarz and definition of $\thmax$, (b) is by the triangle inequality, and
% (c) is by the definition of $\ZB$ and $\normB{\cdot}$.\\\\
Finally, we can apply McDiarmid's inequality \eqref{McDIneq} (see appendix) for any $t>0$ to
get that
\begin{equation}
\P\Big[\,h(\Thetabr)-\E[h(\Thetabr)] \geq t\Big]\leq 
\exp\left(\frac{-2\,t^2}{\sum_{i=1}^r k_i^2}\right) \ .
\end{equation}
In our case, from \eqref{absHbound} we have that $k_i=\frac{\Lc\Cgl}{\pmin\,R}\,\Dt\ZB$ for
all $i\in[R]$ and we bounded $\E[h(\Thetabr)]$ in \eqref{expH}. Therefore, for any $t>0$ we have
\begin{align}\nonumber
\P\Bigg[\,h(\Thetabr)-&\frac{\Lc\,\Cgl}{\pmin\,\sqrt{R}\,}\,
\left(\norm{\thmax}\ZB + \tfrac{|\sigma(0)|}{\Lc}\right) \geq t \Bigg] \\ \nonumber
&\leq \ \P\Big[\,h(\Thetabr)-\E[h(\Thetabr)] \geq t\Big] \\ \label{McD}
& \leq \ 
\exp\left(\frac{-2\,\pmin^2\,R\,t^2}{\Lc^2\,\Cgl^2\,\Dt^2\,\ZB^2} \right) \ .
\end{align}
Setting the RHS of \eqref{McD} to $\nu\in(0,1)$ and solving for $t$ gives
\begin{align}\nonumber
\log&\left(\frac{1}{\nu}\right) = 
\frac{2\,\pmin^2\,R\,t^2}{\Lc^2\,\Cgl^2\,\Dt^2\,\ZB^2} \\ \label{tVal}
& \implies \quad t \ = \ \frac{\Lc\,\Cgl\,\Dt\,\ZB}{\pmin\,\sqrt{R}}\,
\sqrt{\frac{1}{2}\log\left(\frac{1}{\nu}\right)} \ \ .
\end{align}
Combining \eqref{McD} and \eqref{tVal} then gives the result, for any $\nu\in(0,1)$, as the
complement of
\begin{align*}
\P\Bigg[\,h(\Thetabr) \ \geq \ &\frac{\Lc\,\Cgl}{\pmin\,\sqrt{R}\,}\,
\left(\norm{\thmax}\ZB + \frac{|\sigma(0)|}{\Lc}\right) \\ \nonumber 
&+ \frac{\Lc\,\Cgl\,\Dt\,\ZB}{\pmin\,\sqrt{R}}\,
\sqrt{\frac{1}{2}\log\left(\frac{1}{\nu}\right)} \ \Bigg] \ \leq \ \nu \ \ .
\end{align*}
\end{proof}

%%% Local Variables:
%%% mode: latex
%%% TeX-master: "main"
%%% End:


\section{Application}
\label{sec:apx:application}

In this section, we will first show how he $L_2$ norm bounds of the previous sections can be 
extended to $L_\infty$ (supremum) norm bounds when the target function $f\in\Cs(\Kc)$ is 
differentiable at the origin and Lipschitz on a convex compact set $\Kc\subset\R^n$ including 
the origin. Then we will apply the theory to an approximate extension of the Model Reference 
Adaptive Control (MRAC) setup in \cite{lavretsky2013robust}.

\subsection{Extending the Error Bound to $L_\infty$ for Lipschitz Functions on Convex Compact
Sets}
\label{sec:apx:extLinf}

We assume that $\Kc$ is full dimension, with nonzero Lebesgue measure (volume) 
$\Vc=\mu(\Kc)>0$ and diameter $\Dc>0$. It also must hold that $\Kc\subseteq B_0(\rho)$ for a 
sufficiently large radius $0<\rho\leq\Dc$.

Then, for any approximation $\fhN$ of the form \eqref{eq:apx:baseApx} using an activation
function $\sigma$ satisfying Assumption~\ref{sigAsmpt}, we have
%=\sum_{i=1}^N\theta_i\,\sigma(\pariX)$,
that
\begin{align*}
&\abs{\sum_{i=1}^N \theta_i\,\sigma(\pariX) - \sum_{i=1}^N \theta_i\,\sigma(\pariY)} \\ 
&\leq \sum_{i=1}^N \abs{\theta_i\,\sigma(\pariX) - \theta_i\,\sigma(\pariY)} 
\\
&= \sum_{i=1}^N |\theta_i|\!\abs{\sigma(\pariX) - \sigma(\pariY)}
\leq
\sum_{i=1}^N |\theta_i|\Lsig\abs{\pariX - \pariY}\\
& =  
\sum_{i=1}^N |\theta_i|\Lsig\abs{w_i^\top\!(x-y)} 
\ \leq \  
\sum_{i=1}^N |\theta_i|\!\norm{w_i}_2\,\norm{x-y}_2
\end{align*}
holds for all $x,y\in\R^n$, recalling that $X:=[x_1\ \cdots\ x_n\ 1]^\top$ and $Y:=[y_1\ \cdots\ 
y_n\ 1]^\top$, since $\abs{\sigma(u)-\sigma(v)}\leq\Lsig\abs{u-v}$ for any $u,v\in\R$ by the 
Lipschitz condition of Assumption~\ref{sigAsmpt}. Thus, such approximations are 
$\widehat\Lc$-Lipschitz with constant
\begin{equation}\label{eq:apx:Lhat}
\widehat\Lc \ = \ \Lsig\sum_{i=1}^N |\theta_i|\!\norm{w_i}_2 \ \ .
\end{equation}

A similar calculation gives that such approximations are $\widehat\Ds$-bounded with constant
\begin{equation}\label{eq:apx:Dhat}
\widehat\Ds \ = \ \Dsig\sum_{i=1}^N |\theta_i|
\end{equation}
if $\abs{\sigma(u)-\sigma(v)}\leq\Dsig$ for any $u,v\in\R$ by the bounded condition of 
Assumption~\ref{sigAsmpt}.

Let an affine shift to function $f$ be defined as
\begin{equation}\label{eq:apx:fshift}
\fch(x) \ := \ f(x) - \as^\top x - \bs
\end{equation}
for any $\as\in\R^n$ and $\bs\in\R$.
Thus, if $f$ is $\Lc$-Lipschitz, then
\begin{align*}
&\abs{\fch(x)-\fch(y)} \ = \ 
%\abs{f(x) - \as^\top x - f(0_n) - f(y) + \as^\top y + f(0_n)} \\
\abs{f(x)-f(y) - \as^\top(x-y)} \\
&\leq \ 
\abs{f(x)-f(y)} + \abs{\as^\top(x-y)} \\
&\leq\ \Lc\norm{x-y}_2 + \norm{\as}_2\norm{x-y}_2
\end{align*}
holds for all $x,y\in\R^n$. And so, $\fch$ is also Lipschitz, with constant
$\widecheck\Lc = \Lc + \norm{\as}_2$.


\begin{lemma}\label{lem:apx:extLinf}
\textit{
Let $f:\R^n\to\R$ be $\Lc$-Lipschitz on a convex compact set $\Kc\subset\R^n$ containing the 
origin and differentiable at the origin. Let there also be an approximation $\fh:\R^n\to\R$
which is $\widehat\Ds$-bounded or $\widehat{\Lc}$-Lipschitz on $\Kc$ and such that the 
approximation error $\fw(x)=\fch(x)-\fh(x)$, with $\fch$ as in \eqref{eq:apx:fshift}, satisfies 
the $L_2(\Kc,\mubKc)$ norm bound}
$$
\normK{\fw} = \sqrt{\int_\Kc\abs{\fw(x)}^2\mubKcdx} \ \leq \ \epsbase
$$\noindent
\textit{
where the bound $\epsbase>0$ is a finite constant and $\mubKcdx$ is the uniform probability
measure on $\Kc$ with $\int_\Kc\mubKcdx=1$.}

\textit{
Assume $\Kc$ is full dimension with diameter $\Dc>0$, and it holds that $\Kc\subseteq 
B_0(\rho)$ for a ball of sufficiently large radius $0<\rho\leq\Dc$.}
\textit{
Then, the approximation error also satisfies the $L_\infty(\Kc)$ norm bound}
\begin{align}\label{eq:apx:lemDLinfBound}
&\sup_{x\in\Kc}\abs{\fw(x)} \ \leq \ 
2\,
\left(\frac{r}{\Dc(r+\sqrt{r^2+\Dc^2})}\right)^{\!\frac{-n}{n+2}}\,K(\epsbase)
\end{align}
\textit{where}
\begin{align*}
K(\epsbase) \ = \\
i)& \hspace{30pt} \left(\left(\Lc + \norm{\as}_2\right)^n
\epsbase^{\,2}\right)^{\frac{1}{n+2}}+\widehat{\Ds}
\\
ii)& \hspace{30pt}
\left(\left(\Lc + \norm{\as}_2 + \widehat{\Lc}\right)^n
\epsbase^{\,2}\right)^{\frac{1}{n+2}}
\end{align*}
\textit{
and $r$ is the largest radius ball within $\Kc$ centered at its centroid.}
\end{lemma}
\begin{proof}
\if\ARXIV1
Given in Appendix~\ref{app:apx:lemLinfProof}.
\fi
\if\ARXIV0
Given in Appendix III of \cite{lekang2023functionfull}.
\fi
\end{proof}

\subsection{Application to MRAC}
\label{sec:apx:LMRAC}

In Chapter 12 of \cite{lavretsky2013robust}, an approximate extension to the general linear MRAC 
setup (see Chapter 9) is introduced which allows for nonlinearities $f:\R^n\to\R^\ell$ as
$
f(x) \ = \ \Theta^\top\!\Psi(x) + \epsilon_f(x)
$
such that the plant is given by
\begin{align}\nonumber
\dot{x}_t \ &= \ A\,x_t + B\big(u_t + f(x)) \\\label{eq:apx:plant}
&= \ A\,x_t + B\big(u_t + \Theta^\top\!\Psi(x_t) + \epsilon_f(x)\big) \ \ ,
\end{align}
where $A$ is a known $n\times n$ state matrix for the plant state $x_t\in\R^n$, $B$ is a known
$n\times\ell$ input matrix for the input $u_t\in\R^\ell$, and $\Theta$ is an \underline{unknown}
$N\times\ell$ matrix which linearly parameterizes the known vector function $\Psi:\R^n\to\R^N$.
We assume $(A,B)$ is controllable. The general setup also includes an unknown diagonal scaling 
matrix $\Lambda$, such that the overall input matrix is $B\Lambda$, and assumes that $A$ is 
unknown. For simplicity, we assume $A$ is known and omit $\Lambda$.

It is assumed that there exists an $n\times\ell$ matrix of feedback gains $K_x$ and an
$\ell\times\ell$ matrix of feedforward gains $K_r$ satisfying the \textit{matching conditions}
\begin{align*}\nonumber
A + BK_x^\top = A_r \\
BK_r^\top = B_r
\end{align*}
to a controllable, linear reference model
\begin{equation*}
\dot{x}_t^r = A_r\,x_t^r + B_r\,r_t \ \ ,
\end{equation*}
where $A_r$ is a known Hurwitz $n\times n$ reference state matrix for the reference state
$x^r_t\in\R^n$, $B_r$ is a known $n\times\ell$ reference input matrix, and $r_t\in\R^\ell$ is a
bounded reference input. Here, we assume that $K_x$ and $K_r$ can be directly calculated 
from known $A$ and $B$, and used directly in the control law.

It is required that the nonlinearity satisfy the bound
\begin{equation}\label{eq:apx:epsf}
\norm{\epsilon_f(x)}_2 \ \leq \ \bar\varepsilon
\end{equation}
for some constant $\bar\varepsilon>0$ and for all $x\in B_0(r)$ with some radius $r>0$. Then,
the adaptive control law
$$
u_t \ = \ K_x^\top x_t - \Thetah_t^\top\!\Psi(x_t) + \left(1-\mu(x)\right)K_r^\top r_t + 
\mu(x)\uw(x_t)
$$
is shown to stabilize the state tracking error $e_t = x_t-x_t^r$ down to a compact set about the 
origin, where the $N\times\ell$ matrix of parameter estimates $\Thetah_t$ is dynamically updated 
with the update rule
\begin{equation*}
\dot{\Thetah}_t = \Gamma\,\Psi(x_t)\,e_t^\top P_xB \ \ .
\end{equation*}
%(see Chapter 9 of \cite{lavretsky2013robust} for further details).
The scalar function $\mu:\R^n\to[0,1]$
transitions the control law from tracking the reference input to simply returning the plant 
state $x_t$ to the bounded region $B_0(r)$ within which \eqref{eq:apx:epsf} is valid, using an 
appropriately defined $\uw(x_t)$ based on assumptions about the growth of $\epsilon_f(x)$ 
outside of $B_0(r)$. (See Chapter~12 of \cite{lavretsky2013robust} for details.)

And so, if the vector function $\Psi(x)$ is constructed as
\begin{equation}\label{eq:pe:Psi}
\Psi(x)
= 
\begin{bmatrix}
\sigma(\upgamma_1^\top X) \\
\vdots \\
\sigma(\upgamma_N^\top X)
\end{bmatrix} \ ,
\end{equation}
with an activation function $\sigma$ satisfying the bounded (growth) conditions of 
Assumption~\ref{sigAsmpt} with constant $\Dsig$ or $\Lsig$, then the above setup is valid for
any nonlinearity $f = [f_1\ \cdots\ f_\ell]$ where we 
can show that
$$
\sup_{x\in\Kc}\abs{f_i(x)-\Theta_i^\top\Psi(x)} \ \leq \ \bar\varepsilon_i
$$
holds for some constants $\bar\varepsilon_1,\dots,\bar\varepsilon_\ell>0$. Here, 
$f_1,\dots,f_\ell:\R^n\to\R$ and each $\Theta_i^\top\in\R^N$ is the corresponding row of the 
true parameters $\Theta^\top$.


\section{Conclusion}\label{sec:conclusion}
In this work, we focus on addressing the fundamental challenge of OOD detection tasks, which is how to fully understand the semantic discrepancy between the ID/OOD samples. We reveal that the key to success in the realistic SCOOD task is to allocate as many ID samples in the unlabeled set correctly as possible. To this end, we propose a novel uncertainty-aware optimal transport scheme that introduces class-specific energy scores as guidance for effective label assignment. Experimental results show that our method achieves better performance than previous state-of-the-art methods on SCOOD benchmarks.

\textbf{Limitations.} In addition to temperature scaling, other techniques such as feature clipping applied in ReAct~\cite{sun2021react} also enhance the performance of energy score, so how to obtain an OOD score that best fits the SCOOD task can be further explored. Moreover, a setting highly related to SCOOD has been proposed in \cite{katz2022training} and formulated as a constrained optimization problem. We will also theoretically analyze these practical OOD settings in our feature work.

% \section*{Acknowledgments}
\textbf{Acknowledgments.} 
This work is supported by National Key R\&D Program of China under Grant 2020AAA0105701, National Natural Science Foundation of China (NSFC) under Grants 61872327, Major Special Science and Technology Project of Anhui, National Natural Science Foundation of China (62033012) and Ant Group through Ant Research Intern Program.


\newpage
\printbibliography

\if\ARXIV1
\newpage
\appendices
\onecolumn
\section{Appendix for Proofs}

\paragraph{Proof of Theorem \ref{thm:main}.}

\begin{proof}
\label{proof:main}
Our proof has two steps. In Step 1, we will show that SimCLR is equivalent to minimizing the cross entropy loss defined in Eqn.~(\ref{eqn:cross-entropy}). 
In Step 2, we will show  that minimizing the cross-entropy loss 
is equivalent to spectral clustering on $\bfpi$. 
Combining the two steps together, we have proved our theorem. 

\textbf{Step 1: } SimCLR is equivalent to minimizing the cross entropy loss.

The cross-entropy loss takes expectation over 
$\bfW_\bfX\sim \mathbb{P}(\cdot ; \bfpi)$, 
which means $\bfW_\bfX$ has exactly one non-zero entry in each row $i$. By Lemma~\ref{lem:multinomial}, we know every row $i$ of $\bfW_\bfX$ is independent of other rows. Moreover, 
$\bfW_{\bfX,i}\sim \mathcal{M}(1, \bfpi_i/\sum_j \bfpi_{i,j})=\mathcal{M}(1, \bfpi_i)$, because $\bfpi_i$ itself is a probability distribution.
Similarly, we know $\bfW_\bfZ$ also has the row-independent property by sampling over $\mathbb{P}(\cdot;\bfK_\bfZ)$.
Therefore, by Lemma~\ref{lem:cross_split}, we know Eqn.~(\ref{eqn:cross-entropy}) is equivalent to:
\[
 -\sum_{i=1}^n \mathbb{E}_{\bfW_{\bfX,i}}[\log \mathbb{P}(\bfW_{\bfZ,i}=\bfW_{\bfX,i};\bfK_\bfZ)],
\]

This expression takes expectation over $\bfW_{\bfX,i}$ for the given row $i$. Notice that 
$\bfW_{\bfX,i}$ has exactly one non-zero entry, which equals $1$ (same for $\bfW_{\bfZ,i}$). 
As a result
we expand the above expression to be:
\begin{equation}
 -\sum_{i=1}^n \sum_{j\neq i} \Pr(\bfW_{\bfX,i,j}=1)\log \Pr(\bfW_{\bfZ,i,j}=1).
\label{eqn:detailed-expansion}    
\end{equation}


By Lemma~\ref{lem:multinomial}, $\Pr(\bfW_{\bfZ,i,j}=1)=\bfK_{\bfZ,i,j}/\|\bfK_{\bfZ,i}\|_1$ for $j\neq i$. Recall that $\bfK_\bfZ=(k(\bfZ_i-\bfZ_j))_{(i,j)\in[n]^2}$, which means 
$\bfK_{\bfZ,i,j}/\|\bfK_{\bfZ,i}\|_1=\frac{\exp(-\|\bfZ_i-\bfZ_j\|^2/{2\tau})}{\sum_{k\neq i}
\exp(-\|\bfZ_i-\bfZ_k\|^2/{2\tau})
}$ for $j\neq i$, when $k$ is the Gaussian kernel with variance $\tau$. 

Notice that $\bfZ_i=f(\bfX_i)$, so we know
\begin{equation}
-\log \Pr(\bfW_{\bfZ,i,j}=1)=
-\log \frac{\exp(-\|f(\bfX_i)-f(\bfX_j)\|^2/{2\tau})}{\sum_{k\neq i}
\exp(-\|f(\bfX_i)-f(\bfX_k)\|^2/{2\tau}),
}
\label{eqn:infonce-equivalence}    
\end{equation}


The right hand side is exactly the InfoNCE loss defined in Eqn.~(\ref{eqn:infonce}).
Inserting Eqn.~(\ref{eqn:infonce-equivalence}) into Eqn.~(\ref{eqn:detailed-expansion}), we get the SimCLR algorithm, which first samples augmentation pairs $(i,j)$ with $\Pr(\bfW_{\bfX,i,j}=1)$ for each row $i$, and then optimize the InfoNCE loss. 

\textbf{Step 2: } minimizing the cross entropy loss 
is equivalent to spectral clustering on $\bfpi$.


By Lemma~\ref{lem:convert_to_spectral}, we may further convert the loss to 
\begin{equation}
\label{eqn:main-theorem-repul-attr}
\min_{\bfZ}
-\sum_{(i,j)\in [n]^2} \mathbf{P}_{i,j}
\log k (\bfZ_i-\bfZ_j)+\log \mathbf{R}(\bfZ).
\end{equation}
Since $k$ is the Gaussian kernel, this reduces to \[
\min_\bfZ \mathrm{tr}(\bfZ^\top \mathbf{L}(\bfpi) \bfZ)
+\log \mathbf{R}(\bfZ),
\]

where we use the fact that $\mathbb{E}_{\bfW_\bfX\sim \mathbb{P}(\cdot; \bfpi)}[\mathbf{L}(\bfW_\bfX)]
=\mathbf{L}(\bfpi)
$, because the Laplacian operator is linear and $
\mathbb{E}_{\bfW_\bfX\sim \mathbb{P}(\cdot; \bfpi)}(\bfW_\bfX)=\bfpi
$.
\end{proof}

\paragraph{Proof of Theorem \ref{thm:clip}.}
\begin{proof}
Since $\bfW_\bfX\sim \mathbb{P}(\cdot;\bfpi_{\mathbf{A}, \mathbf{B}})$, we know 
$\bfW_\bfX$ has exactly one non-zero entry in each row, denoting the pair that got sampled. 
A notable difference compared to the previous proof is we now have $n_\mathcal{A}+n_\mathcal{B}$ objects in our graph. CLIP deals with this by taking a mini-batch of size $2N$, 
such that $n_\mathcal{A}=n_\mathcal{B}=N$, and adding the $2N$ InfoNCE losses together. We label the objects in $\mathcal{A}$ as $[n_\mathcal{A}]$, and the objects in $\mathcal{B}$ as $\{n_\mathcal{A}+1, \cdots, n_\mathcal{A}+n_\mathcal{B}\}$. 

Notice that $\bfpi_{\mathbf{A}, \mathbf{B}}$ is a bipartite graph, so the edges of objects in $\mathcal{A}$ will only connect to object in $\mathcal{B}$ and vice versa. We can define the similarity matrix in $\cZ$ as $\bfK_\bfZ$, 
where $\bfK_\bfZ(i, j+n_\mathcal{A})=\bfK_\bfZ(j+n_\mathcal{A},i)= k(\bfZ_i-\bfZ_j)$ for $i\in [n_\mathcal{A}], j\in [n_\mathcal{B}]$, and otherwise we set $\bfK_\bfZ(i,j)=0$. 
The rest is same as the previous proof. 
\end{proof}

\paragraph{Proof of Theorem \ref{thm:exponential}.}

\begin{proof}
\label{proof:exponential}
Since the objective function consists of a linear term combined with an entropy regularization, which is a strongly concave function, the maximization problem is a convex optimization problem. Owing to the implicit constraints provided by the entropy function, the problem is equivalent to having only the equality constraint. We then introduce the Lagrangian multiplier $\lambda$ and obtain the following relaxed problem:

$$
\widetilde{E}(\boldsymbol{\alpha})=\psi_{1}-\sum_{i=1}^n \alpha_{i} \psi_{i}+\tau \sum_{i=1}^n \alpha_{i}\log \alpha_{i}+\lambda\left(\boldsymbol{\alpha}^{\top} \mathbf{1}_n-1\right).
$$

As the relaxed problem is unconstrained, taking the derivative with respect to $\alpha_{i}$ yields

$$
\frac{\partial \widetilde{E}(\boldsymbol{\alpha})}{\partial \alpha_{i}}=-\psi_{i}+\tau\left(\log \alpha_{i}+\alpha_{i} \frac{1}{\alpha_{i}}\right)+\lambda=0.
$$

Solving the above equation implies that $\alpha_{i}$ takes the form
$
\alpha_{i}=\exp \left(\frac{1}{\tau} \psi_{i}\right) \exp \left(\frac{-\lambda}{\tau}-1\right).
$ Since $\alpha_{i}$ lies on the probability simplex, the optimal $\alpha_{i}$ is explicitly given by
$
\alpha^{*}_{i}=\frac{\exp \left(\frac{1}{\tau} \psi_{i}\right)}{\sum_{i^{\prime}=1}^n \exp \left(\frac{1}{\tau} \psi_{i^{\prime}}\right)} .
$ Substituting the optimal point into the objective function, we obtain
$$
\begin{aligned}
E\left(\boldsymbol{\alpha}^*\right)  &=\psi_1-\sum_{i=1}^n \frac{\exp \left(\frac{1}{\tau} \psi_{i}\right)}{\sum_{i^{\prime}=1}^n \exp \left(\frac{1}{\tau} \psi_{i^{\prime}}\right)} \psi_{i}+\tau \sum_{i=1}^n \frac{\exp \left(\frac{1}{\tau} \psi_{i}\right)}{\sum_{i^{\prime}=1}^n \exp \left(\frac{1}{\tau} \psi_{i^{\prime}}\right)}\log \frac{\exp \left(\frac{1}{\tau} \psi_{i}\right)}{\sum_{i^{\prime}=1}^n \exp \left(\frac{1}{\tau} \psi_{i^{\prime}}\right)} \\
& =\psi_1 - \tau \log \left(\sum_{i=1}^n \exp \left(\frac{1}{\tau} \psi_{i}\right)\right).
\end{aligned}
$$
Thus, the Lagrangian dual function is given by
\begin{equation*}
-E\left(\boldsymbol{\alpha}^*\right)= -\tau \log \frac{\exp \left(\frac{1}{\tau} \psi_{1}\right)}{\sum_{i=1}^n \exp \left(\frac{1}{\tau} \psi_{i}\right)}.\qedhere
\end{equation*}
\end{proof}



\section{More on Experiments} \label{section: experiment_details}

\paragraph{CIFAR-10 and CIFAR-100} CIFAR-10 ~\citep{krizhevsky2009learning} and CIFAR-100 ~\citep{krizhevsky2009learning} are well-known classic image classification datasets. Both CIFAR-10 and CIFAR-100 contain a total of 60k $32 \times 32$ labeled images of different classes, with 50k for training and 10k for testing. CIFAR-10 is similar to CIFAR-100, except there are 10 different classes in CIFAR-10 and 100 classes in CIFAR-100.

\paragraph{TinyImageNet} TinyImageNet ~\citep{le2015tiny} is a subset of ImageNet ~\citep{deng2009imagenet}. There are 200 different object classes in TinyImageNet, with 500 training images, 50 validation images, and 50 test images for each class. All the images in TinyImageNet are colored and labeled with a size of $64 \times 64$.

\textbf{Pseudo-code.} Algorithm \ref{alg:Training Procedure} presents the pseudo-code for our empirical training procedure.

\begin{algorithm}[!htbp]
\caption{Training Procedure}
\label{alg:Training Procedure}
\begin{algorithmic}[1]
\REQUIRE trainable encoder network $f$, batch size $N$, augmentation strategy \textit{aug}, loss function $L$ with hyperparameters \textit{args}
\FOR {sampled minibatch ${x_i}_{i=1}^N$}
\FORALL{$i \in { 1, ..., N }$}
\STATE draw two augmentations $t_i = \textit{aug}\left(x_i\right) $, $t_i' = \textit{aug}\left(x_i\right) $
\STATE $z_i = f\left(t_i\right)$, $z_i' = f\left(t_i'\right)$
\ENDFOR
\STATE compute loss $\mathcal{L} = L(N, z, z', \textit{args})$
\STATE update encoder network $f$ to minimize $\mathcal{L}$
\ENDFOR
\STATE \textbf{Return} encoder network $f$
\end{algorithmic}
\end{algorithm}

We also provide the pseudo-code for our core loss function used in the training procedure in Algorithm \ref{alg:Core loss}. The pseudo-code is almost identical to SimCLR's loss function, with the exception of an extra parameter $\gamma$.

\begin{algorithm}[!htbp]
\caption{Core loss function $\mathcal{C}$}
\label{alg:Core loss}
\begin{algorithmic}[1]
\REQUIRE batch size $N$, two encoded minibatches $z_1, z_2$, $\gamma$, temperature $\tau$
\STATE $z = \textit{concat}\left(z_1, z_2\right)$
\FOR {$i \in {1, ..., 2N }, j \in {1, ..., 2N}$ }
\STATE $s_{i,j} = \Vert z_i - z_j \Vert_2^{\gamma}$
\ENDFOR
\STATE \textbf{define} $l(i, j)$ \textbf{as} $l(i, j) = - \log \frac{exp\left(s_{i,j}/\tau \right)}{\sum_{k=1}^{2N} \mathbf{1}{[k \ne i]} exp\left(s{i, j} / \tau \right)} $
\STATE \textbf{Return} $\frac{1}{2N} \sum_{k=1}^N\left[l(i, i+N) + l(i+N, i)\right]$
\end{algorithmic}
\end{algorithm}

Utilizing the core loss function $\mathcal{C}$, we can define all kernel loss functions used in our experiments in Table \ref{table: loss definition}. For all $z_i \in z$ with even dimensions $n$, we define $z_{L_i} = z_i\left[0:n/2\right]$ and $z_{R_i} = z_i\left[n/2:n\right]$.

\begin{table}[ht]
\centering
\begin{tabular}{{@{}l|l@{}}}
Kernel  &  Loss function \\ \midrule
Laplacian & $\mathcal{C}\left(N, z, z', \gamma=1, \tau\right)$\\ \midrule
Sum       & $\lambda * \mathcal{C}\left(N, z, z', \gamma=1, \tau_1\right) + (1-\lambda) * \mathcal{C}\left(N, z, z', \gamma=2, \tau_2\right)$  \\ \midrule
Concatenation Sum&$\lambda * \mathcal{C}\left(N, z_L, z'_L, \gamma=1, \tau_1\right) + (1-\lambda) * \mathcal{C}\left(N, z_R, z'_R, \gamma=2, \tau_2\right)$\\ \midrule
$\gamma = 0.5$ & $\mathcal{C}\left(N, z, z', \gamma=0.5, \tau\right)$          \\ 

\end{tabular}

\caption{Definition of kernel loss functions in our experiments}
\label {table: loss definition}
\end{table}

\textbf{Baselines.} We reproduce the SimCLR algorithm using PyTorch Lightning~\citep{PytorchLightning}.

\textbf{Encoder details.}
The encoder $f$ consists of a backbone network and a projection network. We employ ResNet50~\citep{ResNet} as the backbone and a 2-layer MLP (connected by a batch normalization~\citep{ioffe2015batch} layer and a ReLU \cite{nair2010rectified} layer) with hidden dimensions 2048 and output dimensions 128 (or 256 in the concatenation kernel case).

\textbf{Encoder hyperparameter tuning.}
For each encoder training case, we randomly sample 500 hyperparameter groups (sample details are shown in Table \ref{table: Hyperparameter sample}) and train these samples simultaneously using Ray Tune ~\citep{RayTune}, with the ASHA scheduler~\citep{li2018massively}. Ultimately, the hyperparameter group that maximizes the online validation accuracy (integrated in PyTorch Lightning) within 5000 validation steps is chosen for the given encoder training case.

\begin{table}[ht]
\centering

\begin{tabular}{@{}l|l|l@{}}
\midrule
Hyperparameter  & Sample Range & Sample Strategy \\ \midrule
start learning rate & $\left[10^{-2}, 10\right]$ & log uniform \\ \midrule
$\lambda$       & $\left[0, 1\right]$ & uniform \\ \midrule
$\tau$, $\tau_1$, $\tau_2$ & $\left[0, 1\right]$ & log uniform \\ \midrule
\end{tabular}

\caption{Hyperparameters sample strategy}
\label {table: Hyperparameter sample}
\end{table}

\textbf{Encoder training.} 
We train each encoder using the LARS optimizer~\citep{LARSOptimizer}, LambdaLR Scheduler in PyTorch, momentum 0.9, weight decay $10^{-6}$, batch size 256, and the aforementioned hyperparameters for 400 epochs on a single A-100 GPU.

\textbf{Image transformation.} The image transformation strategy, including augmentation, is identical to the default transformation strategy provided by PyTorch Lightning.

\textbf{Linear evaluation.}
The linear head is trained using the SGD optimizer with a cosine learning rate scheduler, batch size 64, and weight decay $10^{-6}$ for 100 epochs. The learning rate starts at $0.3$ and ends at $0$.

\textbf{Moco Experiments.} We also tested our method based on MoCo~\citep{he2019moco}. The results are summarized in Table \ref{tab:results-moco}. Here we choose ResNet18~\citep{ResNet} as the backbone and set a temperature of $0.1$ as default. For our simple sum kernel, we set $\lambda=0.8$. The results show that our method outperforms the original MoCo method.

\begin{table}[thb]
\centering
\caption{MoCo Experiment Results on CIFAR-10 and CIFAR-100.}
\label{tab:results-moco}
\resizebox{\textwidth}{!}{%
\begin{tabular}{@{}c|ccc|ccc@{}}
\toprule
\multirow{3}{*}{Method} & \multicolumn{3}{c|}{CIFAR-10} & \multicolumn{3}{c}{CIFAR-100} \\ \cmidrule(lr){2-4} \cmidrule(lr){5-7} 
                        & 200 epochs & 400 epochs    & 1000 epochs   & 200 epochs & 400 epochs & 1000 epochs         \\ \midrule
MoCo (repro.)         & $76.41 \pm 0.12$    & $80.01 \pm 0.15$          & $84.45 \pm 0.08$    & $\mathbf{47.02 \pm 0.11}$ & $52.50 \pm 0.07$ & $57.62 \pm 0.15$            \\
\midrule
Laplacian Kernel        & ${78.09 \pm 0.10}$    & $\mathbf{83.85 \pm 0.09}$          & $\mathbf{88.34 \pm 0.16}$    & $46.12 \pm 0.22$   & $53.44 \pm 0.17$ & $59.10 \pm 0.14$        \\
Simple Sum Kernel & $\mathbf{78.12 \pm 0.15}$   & $83.23 \pm 0.18$ & $87.50 \pm 0.20$ & $46.65 \pm 0.06$ & $\mathbf{53.62 \pm 0.19}$ & $\mathbf{59.83 \pm 0.12}$\\
\bottomrule
\end{tabular}
}
\end{table}



\section{More Experiments on Synthetic Data}


Consider a scenario with $n$ clusters, each containing $k$ vertices. Let the probability of vertices $u$ and $v$ from the same cluster belonging to $\bfpi$ be $p$. Conversely, for vertices $u$ and $v$ from different clusters, let the probability of belonging to $\pi$ be $q$. We generate the graph $\bfpi$ randomly, based on $p$ and $q$. We experiment with values of $k=100$ and $n=6$ for ease of visualization, embedding all points in a two-dimensional space. Each vertex's initial position originates from a normal distribution. In each iteration, we sample a subgraph of $\bfpi$ uniformly, ensuring each vertex has an out-degree of $1$. We then optimize the corresponding vectors using InfoNCE loss with an SGD optimizer and iterate until convergence. Our experimental setup consists of an SGD learning rate of $1$, an InfoNCE loss temperature of $0.5$, and a batch size of $50$. We evaluate two scenarios with different $p$ and $q$ values: $p=1$, $q=0$, and $p=0.75$, $q=0.2$. The results of these experiments are visualized in Figure \ref{fig:vis-spectral-cluster}. The obtained embeddings exhibit the hallmark pattern of spectral clustering of graph $\bfpi$.

\begin{figure}[!tb]
\centering
\subfigure{
\includegraphics[width=1\textwidth]{Figures/cluster_pi.png}
\label{fig:vis-cluster}
}
\subfigure{
\includegraphics[width=1\textwidth]{Figures/noised_cluster_pi.png}
\label{fig:vis-noised-cluster}
}
\caption{Visualizations of the optimization process using InfoNCE Loss on the vectors corresponding to $\bfpi$. Points of identical color belong to the same cluster within $\bfpi$. To showcase the internal structure of $\bfpi$, we randomly select 10 vertices from each cluster to display the edge distribution of $\bfpi$.}
\label{fig:vis-spectral-cluster}
\end{figure}


\fi

\end{document}