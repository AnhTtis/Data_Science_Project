\section{Mollified Approximations}
\label{sec:apx:mollApx}
In this section, we define the \textit{mollified approximation} $\fhlN$ using a novel method of 
forming a \textit{mollified integral representation} for a base approximation $\fhN$ to a 
target function $f\in\Cs(\Bc)$, and bound their difference in the $L_2(\Bc,\muB)$ norm.


\subsection{Functions with Integral Representation}
If a target function $f:\R^n\to\R$ has an \textit{integral representation} over the parameters
set $\Upsilon$ with some activation function $\sigma:\R\to\R$, then
\begin{equation}\label{eq:apx:intRep}
f(x) \ = \ \int_\Upsilon \gs(\upgamma)\,\sigma(\parX)\,\d\upgamma
% \hspace{40pt} \forall x\in\Bc
\end{equation}
holds with some bounded, continuous \textit{coefficient function} \mbox{$\gs:\Upsilon\to\R$}. 
This is a version of \eqref{eq:apx:baseApx} that is continuous in the parameters $\upgamma_i$ 
over $\Upsilon$. Indeed, if we define $\upgamma_1,\dots,\upgamma_N$ to be a sequence of points 
that uniformly grids $\Upsilon$ with volume $\d_N$ between any neighboring points, then as
$N\to\infty$ (and $\d_N\to0$) we have
$$
\lim\limits_{N\to\infty}\sum_{i=1}^{N}\gs(\upgamma_i)\,\sigma(\pariX)\,\d_N 
= \int_\Upsilon \gs(\upgamma)\,\sigma(\parX)\,\d\upgamma = f(x)
%\forall x\in\Bc
\ .
$$
This implies that the (infinite) sum of the absolute values of the coefficients is finite, with
$$
%C_\infty := 
\lim\limits_{N\to\infty}\sum_{i=1}^{N}|\gs(\upgamma_i)|\,\d_N \ 
= \ \int_\Upsilon |\gs(\upgamma)|\,\d\upgamma \ < \ \infty
$$
holding since $\gs$ is bounded and the integral is over the bounded set $\Upsilon$.

However, finding such integral representations \eqref{eq:apx:intRep} for general classes of
functions $f$ and activation functions $\sigma$ is a nontrivial task. If \textit{both}
are $L_1$ (absolutely integrable over $\R^n$ and $\R$ respectively), then Theorem 1 in
\cite{irie1988capabilities} shows that the integral representation can always be constructed
with $\gs$ related to the Fourier transforms of $f$ and $\sigma$. But this is a limiting
restriction, since none of the typical activation functions like sigmoids, steps, ReLU's, and 
their variants meet this requirement. Even so, in \cite{funahashi1989approximate} this was used 
to prove the universal approximation properties of sigmoid activations.
%In
%\cite{hsu2021approximation}, it is shown that a particular definition of a trigonometric ridge
%function has an integral representation with ReLU activations, and then those ridge functions
%can be the basis functions as in \eqref{eq:apx:genApx} for target functions in $L_2([-1,1]^n)$.
%But again, this is a limited class of target functions compared to $\Cs(\Bc)$.


\subsection{Mollified Integral Representation}
\label{sec:apx:moll}
Now we will propose a novel method of forming an integral representation for
\textit{any} function $f\in\Cs(\Bc)$, so long as a base approximation $\fhN$ of the form 
\eqref{eq:apx:baseApx} exists and the bounded parameter set $\Upsilon\subset\R^{n+1}$ and a
bound $\Sc_N$ on the coefficient size $\sum_{i=1}^N|\theta_i|$ are known.

For any fixed parameters $\upgamma_1,\dots,\upgamma_N\in\Upsilon$, this approximation $\fhN$ 
can be defined as
\begin{align}\nonumber
\fhN(x) \ =& \  \sum_{i=1}^{N} \theta_i\,\sigma(\pariX) \\\label{eq:apx:baseIntRep}
=& \ 
\int_\Upsilon \,\sum_{i=1}^N\theta_i\,\deltano(\upgamma-\upgamma_i)\ \sigma(\parX)\,\d\upgamma \ 
\ ,
\end{align}
where $\deltano$ is the $(n+1)$-dimensional Dirac delta which satisfies for any
$\upgamma_i\in\Upsilon$ that
$\int_\Upsilon\deltano(\upgamma-\upgamma_i)\,\sigma(\parX)\,\d\upgamma =\sigma(\pariX)\ $.

Thus, the RHS of \eqref{eq:apx:baseIntRep} is like an integral representation 
\eqref{eq:apx:intRep} of $\fhN$ with a coefficient mapping \mbox{$\gs(\upgamma) = \sum_{i=1}^N 
\theta_i\,\deltano(\upgamma-\upgamma_i)$}.
However, this mapping is not continuous and bounded as required, and the Dirac delta is not even
a function in the regular sense. Indeed, if the overall coefficients of a random approximation 
$\fhR$ were set as $\vartheta_j=\gs(\upgammab_j)$ for each randomly sampled 
$\upgammab_1,\dots,\upgammab_R\in\Upsilon$, then in this case they almost surely would all be 
zero.

We propose the \textit{Mollified integral representation} to solve this issue, where the Dirac
delta in \eqref{eq:apx:baseIntRep} is replaced with the \textit{mollified} delta function
$\deltalno$. Recalling that $\upgamma=[w_1\ \cdots\ w_n\ b\,]$, this is defined as the 
$(n+1)$-dimensional bump function
\begin{align}\label{eq:apx:mollDel}
&\deltalno(\upgamma) \ := \ \\\nonumber
&\begin{cases}
(\eta\lambda)^{n+1}\prod_{i=1}^n
\exp\left(\dfrac{-1}{1-\lambda^2w_i^2}\right)
%\cdots\,
%\exp\left(\dfrac{-1}{1-\lambda^2w_n^2}\right)\,
\exp\left(\dfrac{-1}{1-\lambda^2b^2}\right)\\[8pt]
\hspace{160pt} \upgamma\in(-\frac{1}{\lambda},\frac{1}{\lambda})^{n+1} \\
0 \hspace{155pt} \text{otherwise}
\end{cases}
\end{align}
with $\eta^{-1}=\int_{-1}^{1}\exp\left(\frac{-1}{1-y^2}\right)\d y \approx0.444$ thus giving
$\int\deltalno(\upgamma-\upgamma_i)\,\d\upgamma = \int\deltano(\upgamma-\upgamma_i)\,\d\upgamma
= 1$ for any $n\geq1$ and $\upgamma_i\in\R^{n+1}$.

We term the positive scalar $\lambda>0$ as the \textit{mollification factor}. This controls the
height and width of the \mbox{mollified} delta $\deltalno$. Since the height scales as
$\lambda^{n+1}$ and the integral remains constant, this requires the width to shrink. 
Figure~\ref{fig:apx:mollFactor} plots the 1-dimensional case for different $\lambda$ values.
%\begin{figure}[H]
\begin{figure}
\centering
\includegraphics[width=\columnwidth]{mollFactor}
\caption{\label{fig:apx:mollFactor}The mollified delta $\deltal$ for different $\lambda$ values.
In each case, $\int_{-1/\lambda}^{1/\lambda}\deltal(\upgamma)\d\upgamma=1$.}
\end{figure}

The support of $\deltalno(\upgamma-\upgamma_i)$ is the $(n+1)$-dimensional open cube set
$(-\frac{1}{\lambda},\frac{1}{\lambda})^{n+1}$ centered about any $\upgamma_i\in\Upsilon$. And
so, if $\upgamma_i$ is sufficiently close to (or on) the boundary of $\Upsilon$, then it no
longer holds that the entire support of $\deltalno(\upgamma-\upgamma_i)$ is within $\Upsilon$.
Thus, we define the \textit{$\lambda$-expanded} parameters set $\Upsilonl$ such that the support
is always fully contained, up to and include the boundary of $\Upsilon$. For example, if
$\Upsilon$ is rectangular product of intervals $[\alpha_1,\beta_1]\times\cdots\times
[\alpha_n,\beta_n]$, then each interval is expanded as 
$[\alpha_i-\frac{1}{\lambda},\beta_i+\frac{1}{\lambda}]$ to obtain $\Upsilonl$.
With this $\lambda$-expanded parameters set, we always maintain that
$\int_\Upsilonl\deltalno(\upgamma-\upgamma_i)\d\upgamma=1$ for all $\lambda>0$, $n\geq1$, and any
$\upgamma_i\in\Upsilon$.

Since $\Upsilon\subset\Upsilonl$ for all $\lambda>0$, then \eqref{eq:apx:baseIntRep} is 
equivalently
\begin{equation}\label{eq:apx:baseIntRepLam}
\fhN(x) = \int_\Upsilonl \,\sum_{i=1}^N\theta_i\,\deltano(\upgamma-\upgamma_i)\ 
\sigma(\parX)\,\d\upgamma \ \ .
\end{equation}
And so, by replacing the Dirac delta in \eqref{eq:apx:baseIntRepLam} with the mollified delta
$\deltalno$, for any choice of $\lambda>0$, we obtain the corresponding \textit{mollified 
approximation}
\begin{equation}\label{eq:apx:mollApx}
\fhlN(x) := \int_\Upsilonl \,\sum_{i=1}^N\theta_i\,\deltalno(\upgamma-\upgamma_i)\ 
\sigma(\parX)\,\d\upgamma
\end{equation}
using a \textit{mollified integral representation}.
This now gives the required continuous, bounded coefficient mapping for any fixed
$\upgamma_1,\dots,\upgamma_N\in\Upsilon$ as
\begin{equation}\label{eq:apx:gLam}
\gl(\upgamma) := \sum_{i=1}^N\theta_i\,\deltalno(\upgamma-\upgamma_i)
%\ \leq \ \left(\frac{\eta\lambda}{e}\right)^{n+1}\sum_{i=1}^N|\theta_i|
\ \ .
\end{equation}
We illustrate the transformation for the 1-dimensional case conceptually in 
Figure~\ref{fig:apx:moll}.
%\begin{figure}[H]
\begin{figure}
\centering
\includegraphics[width=.9\columnwidth]{moll}
\caption{\label{fig:apx:moll}The base approximation $\fhN$ is transformed into the mollified
approximation $\fhlN$ by replacing Dirac deltas $\delta$ with the mollified deltas $\deltal$.}
\end{figure}

\subsection{Main Result}
Let us define $E(x):=\|\,[x_1\ \cdots\ x_n\ 1]\,\|_2=\|X\|_2$ as the Euclidean norm mapping from 
$\R^n\times\{1\}$ to $\R$, and then define
\begin{equation}\label{eq:apx:EB}
%\Esup := \normSupB{E(x)}
%\qquad\text{and}\qquad
\EB := \normB{E(x)}
\end{equation}
as the $L_2(\Bc,\muB)$ norm of this mapping over the Euclidean distances of $x\in\Bc$. We can now
state the main result, which bounds the difference between a base approximation and its 
corresponding mollified approximation, for a chosen mollifcation factor $\lambda$, in the 
$L_2(\Bc,\muB)$ norm.

\begin{theorem}\label{thm:apx:moll}
\textit{
Let there exist a base approximation $\fhN$ of the form \eqref{eq:apx:baseApx} using any 
activation function $\sigma$ satisfying Assumption~\ref{sigAsmpt}, for any $N\in\N$, with some 
unknown parameters $\upgamma_1,\dots,\upgamma_N\in\Upsilon$ and coefficients $\theta\in\R^N$, 
but having a known bound on the coefficient size of $\sum_{i=1}^N|\theta_i|\leq \Sc_N$ and the 
parameters set $\Upsilon$ known. Then, the mollified approximation $\fhlN$ over the 
$\lambda$-expanded parameters set $\Upsilonl$ in \eqref{eq:apx:mollApx} satisfies either of:}
\begin{align}\label{eq:apx:epslSupBound}
i)\hspace{55.5pt} \normB{\fhN-\fhlN} \ &\leq \ \Sc_N \, \Dsig \\\label{eq:apx:epslL2Bound}
ii)\hspace{52pt} \normB{\fhN-\fhlN} \ &\leq \ \Sc_N\, \frac{\Lsig\,\sqrt{n+1}}{\lambda}\ \EB \ .
\end{align}
\end{theorem}
\begin{proof}
\if\ARXIV1
Given in Appendix~\ref{app:apx:mollProof}.
\fi
\if\ARXIV0
Given in Appendix I of \cite{lekang2023functionfull}.
\fi
\end{proof}

\begin{remark}
The results of Theorem~\ref{thm:apx:moll} show that if the activation function $\sigma$ is 
Lipschitz over its feasible inputs, then as $\lambda\to\infty$ we have $\fhlN(x)\to\fhN(x)$ in 
the sense of the $L_2(\Bc,\muB)$ norm.
\end{remark}