\onecolumn
\section*{Appendix}

\subsubsection*{Useful Relations}

The following are used in multiple places and thus we detail them here for convenience. First,
we bound the absolute value of the integrand in \eqref{eq:funcReq1} as
\begin{align}\label{eq:intgBnd1}
|\cos(\wdx+b)-\cos(b)| \ \leq \ |e^{i\wdx}-1| = \ 2|\sin(\tfrac{\wdx}{2})| \ \leq
\ 2|\tfrac{\wdx}{2}| \ = \ |\wdx| \ \ ,
\end{align}
and then similarly bound the absolute value of the integrand in \eqref{eq:funcReq2} as
\begin{align}\nonumber
|\cos(\wdx+b)+\wdx\sin(b)-\cos(b)| \ \leq \ |e^{i\wdx}-i\wdx-1| &= \sqrt{(\cos(\wdx)-1)^2 +
(\sin(\wdx)-\wdx)^2} \\ \nonumber
&\leq |\cos(\wdx)-1| + |\sin(\wdx)-\wdx| \\ \nonumber
&= 2|\sin(\tfrac{\wdx}{2})|^2 + |\sin(\wdx)-\wdx| \\ \nonumber
&\leq 2|\tfrac{\wdx}{2}|^2 + \tfrac{1}{2}|\wdx|^2 \\ \label{eq:intgBnd2}
&= |\wdx|^2 \ \ .
\end{align}
Note that the first inequalities are because $\big|\real{Ze^{ib}}\big|\leq|Ze^{ib}|=|Z|$ holds
for all $Z\in\C$ and $b\in\R$, and the second term in the second inequality of
\eqref{eq:intgBnd2} is because
\begin{align*}
|\sin(y)-y| = \left|\int_0^{|y|}\cos(y)-1\,\d y\right| \leq
\int_0^{|y|}|\cos(y)-1|\,\d y \leq \int_0^{|y|}y\,\d y = \tfrac{1}{2}|y|^2 \ \ ,
\end{align*}
since $|\sin(0)-0|=0$, $\tfrac{\d}{\d y}(\sin(y)-y)=\cos(y)-1$, and $|\cos(y)-1|\leq|y|$ is
established in \eqref{eq:intgBnd1} with $b=0$.\\\\
Next, we show that the expected squared $\normB{\cdot}$ norm of the difference between the
sample average function $\fbbR=\frac{1}{R}\sum_{i=1}^R\fb_i$ and its mean, is bounded by
$\frac{1}{R}$ times the expected squared norm of $\fb$, where $\fb_1,\dots,\fb_R$ and $\fb$ are
all drawn iid from a function space $\Fc$, as
\begin{align}\nonumber
\E\normB{\fbbR - \E\fbbR}^2 \ &= \ \int_B \left(\fbbR - \E\fbbR\right)^2\,\muBdx \\ \nonumber
&\overset{(a)}{=} \ \int_B \left(\E\left[\fbbR^{\,2}\right]
-2\,\E\left[\fbbR\right]\E\left[\fbbR\right]
+ \E\left[\fbbR\right]^2 \right)\,\muBdx \\ \nonumber
&\overset{(b)}{=} \ \int_B \left(\E\left[\fbbR^{\,2}\right] -
\E\left[\fbbR\right]^2 \right)\,\muBdx \ 
= \int_B \text{var}\left(\fbbR\right)\,\muBdx \\ \nonumber
&\overset{(c)}{=} \ \int_B \frac{1}{R}\,\text{var}(\fb)\,\muBdx \ 
= \ \frac{1}{R} \int_B \left(\E\left[\fb^2\right]-\E[\fb]^2\right)\,\muBdx \\ \label{eq:varBnd}
&= \ \frac{1}{R}\left(\E\normB{\fb}^2-\normB{\E[\fb]}^2\right) \ 
\overset{(d)}{\leq} \ \frac{1}{R}\,\E\normB{\fb}^2 \ \ .
\end{align}
Here (a) is by using Fubini's theorem to bring the expectation inside the integral, the
linearity of expectation, and the outer expectation then being redundant for the last term, (b)
is from combining the middle and last terms, (c) is by the facts of variance of averaged iid
random variables (covariance terms are zero, while means and variances are equal), and (d) is
because the second term is nonpositive and therefore it can be dropped with inequality. 

\subsubsection*{Integrability of $\FCBo$ and $\FCBt$ Classes}

We note here that the integrability of \eqref{eq:funcReq1} and \eqref{eq:funcReq2}, given using
\eqref{eq:intgBnd1} and \eqref{eq:intgBnd2} respectively with Cauchy-Schwarz as
\begin{align*}
\int_{\R^n} |e^{i\wdx}-1|\,G(\dw) &\leq \int_{\R^n} |\wdx|\,G(\dw) \leq 
\norm{x}\int_{\R^n} \norm{w}\,G(\dw) \leq \infty \\
\int_{\R^n} |e^{i\wdx}-i\wdx-1|\,G(\dw) &\leq \int_{\R^n} |\wdx|^2\,G(\dw) \leq
\norm{x}^2\int_{\R^n} \norm{w}^2\,G(\dw) \leq \infty \ \ ,
\end{align*}
is satisfied for any $x\in\R^n$ by the definitions of $C_{f,1}$ and $C_{f,2}$, because it always
holds that $B\subseteq \{x\in\R^n\ |\ \norm{x}\leq r\}$ for sufficiently large $r$ and we then
have $\absB{w}\leq r\norm{w}$.

\subsubsection*{Lemma on Approximating Functions of Convex Combinations}

The following lemma is credited to Maurey in \cite{pisier1981remarques}.

%\begin{lemma*}
\begin{customlemma}{A.1}\label{lem:convComb}
If $\fch$ is in the closure of the convex hull of a bounded set of functions $\Fc$ in a Hilbert
space $\Hc$, such that $\norm{f}_{\Hc}\leq b$ holds for all $f\in\Fc$, then for every $N\geq1$,
there exists an $\fbN$ in the convex hull of $N$ points in $\Fc$ such that
\begin{equation*}
\norm{\fch-\fbN}_{\Hc}^2 \ \leq \ \frac{\bar{c}}{N}
\end{equation*}
holds for any $\bar{c} > b^2 - \norm{\fch}_{\Hc}^2 \ $.
\end{customlemma}
%\end{lemma*}
\begin{proof}
First define $\fst$ as a finite but arbitrarily large convex combination of $m$ points
$\fst_i\in\Fc$, meaning it has the form $\fst = \sum_{i=1}^m\gamma_i\fst_i$ with $\gamma_i\geq0$, $\sum_{i=1}^m\gamma_i=1$, and such that
$\norm{\fch-\fst}_{\Hc}\leq\frac{\delta}{\sqrt{N}}$ for some $\delta>0$. Randomly sample
functions $\fb,\fb_1,\dots,\fb_N$ iid from the set $\{\fst_i,\dots,\fst_m\}$ according to
$\P[\fb=\fst_i]=\gamma_i$. Thus, we have $\E[\fbbN]=\E[\frac{1}{N}\sum_{i=1}^N\fb_i]=\E[\fb]
=\fst$, and then using \eqref{eq:varBnd} gives
$\E\norm{\fbbN-\fst}_{\Hc}^2=\frac{1}{N}(\E\norm{\fb}_{\Hc}^2 - \norm{\fst}_{\Hc}^2) \leq
\frac{1}{N}(b^2 - \norm{\fst}_{\Hc}^2)$. Since the expectation is bounded this way, there must
be a realization $\fbN$ that also satisfies this bound. Then by the triangle inequality, we have
$\norm{\fch-\fbN}_{\Hc} \leq \norm{\fch-\fst}_{\Hc} + \norm{\fbN-\fst}_{\Hc} =
\frac{1}{\sqrt{N}}(\delta + \sqrt{b^2 - \norm{\fst}_{\Hc}^2})$. And since $\fst$ can get
arbitrarily close to $\fch$ by letting $m\to\infty$, and thus $\delta\to0$ and
$\norm{\fst}_{\Hc}\to\norm{\fch}_{\Hc}$, this proves the result.
\end{proof}
In our setting, we use the $\normB{\cdot}$ norm for the $L_2(B)$ space. We note that
\cite{makovoz1996random} (Thm. 1) gives the same $O(\frac{1}{\sqrt{N}})$ result, but with a
superior constant. The proof there uses the same method as above, but breaks $\Fc$ into $N$
subsets of diameter $\epsilon$, and then uses that to bound the variance terms. This gives the
bound constant in terms of an inverse covering number of $\Fc$ (denoted $\epsilon_N(\Fc)$),
which goes to zero as $N\to\infty$. However, this requires quantifying $\epsilon_N(\Fc)$, which
may be nontrivial.


\subsubsection*{Proofs of Theorem~\ref{thm:barron93} and Theorem~\ref{thm:brei93}}

We closely follow the proof methods in \cite{barron1993universal} and \cite{breiman1993hinging},
while keeping the setup consistent.\\\\
Recalling that $g(x)=f(x)$ for $x\in B$, then taking the real part of both sides of
\eqref{eq:funcReq1} and \eqref{eq:funcReq2} gives
\begin{align*}
\real{\fch_1(x)} &= \real{\int_{\R^n} (e^{i\wdx}-1)\,\Gw_1(\dw)} &
\real{\fch_2(x)} &= \real{\int_{\R^n} (e^{i\wdx}-i\wdx-1)\,\Gw_2(\dw)} \\
\fch_1(x) &= \real{\int_{\R^n} (e^{i\wdx}-1)\,e^{i\theta_1(w)}G_1(\dw)} &
\fch_2(x) &= \real{\int_{\R^n} (e^{i\wdx}-i\wdx-1)\,e^{i\theta_2(w)}G_2(\dw)} \\
&= \int_{\R^n} \big(\cos(\wdx + \theta_1(w)) &
&= \int_{\R^n} \big(\cos(\wdx + \theta_2(w))  \\
&\hspace{60pt} - \cos(\theta_1(w))\big)\,G_1(\dw) & 
&\hspace{45pt} + \wdx\,\sin(\theta_2(w)) - \cos(\theta_2(w))\big)\,G_2(\dw) \\
&= \int_{\Omega} h_1(x,w)\,\Lambda_1(\dw) &
&= \int_{\Omega} h_2(x,w)\,\Lambda_2(\dw)
\end{align*} 
where $\Omega=\R^n\setminus\{0\}$, with equality maintained because the integrand at $w=0$ is
zero in both cases, and
\begin{align*}
h_1(x,w) &= \frac{C_{f,1}}{\absB{w}}\big(\cos(\wdx + \theta_1(w)) &
h_2(x,w) &= \frac{C_{f,2}}{\absB{w}^2}\big(\cos(\wdx +\theta_2(w)) \\
& \hspace{70pt} -\cos(\theta_1(w))\big) & &\hspace{50pt} +\wdx\,\sin(\theta_2(w)) 
-\cos(\theta_2(w))\big) \\
\Lambda_1(\dw) &= \frac{\absB{w}}{C_{f,1}}\,G_1(\dw) &
\Lambda_2(\dw) &= \frac{\absB{w}^2}{C_{f,2}}\,G_2(\dw) \ \ .
\end{align*}
Then $\Lambda_1$ and $\Lambda_2$ are probability measures over $\Omega$ by the definitions of
$C_{f,1}$ and $C_{f,2}$ respectively, such that $\int_{\Omega}\Lambda_1(\dw)=
\int_{\Omega}\Lambda_2(\dw)=1$, and we have
$\E_{\Lambda_1}[\frac{1}{N}\sum_{i=1}^Nh_1(x,\wb_i)]=\fch_1$ and
$\E_{\Lambda_2}[\frac{1}{N}\sum_{i=1}^Nh_2(x,\wb_i)]=\fch_2$ for $\wb_1,\dots,\wb_N$ sampled iid
from $\Lambda_1$ and $\Lambda_2$ respectively. Then by \eqref{eq:intgBnd1} and
\eqref{eq:intgBnd2}, it holds for all $x\in B$ and $w\in\Omega$ that
\begin{align*}
|h_1(x,w)| \ &\leq \ \frac{C_{f,1}}{\absB{w}}\,|\wdx| \ \leq\ C\frac{|\wdx|}{\absB{w}}
\ \leq\ C &
|h_2(x,w)| \ &\leq \ \frac{C_{f,2}}{\absB{w}^2}\,|\wdx|^2 \ \leq\ 
C\frac{|\wdx|^2}{\absB{w}^2} \ \leq\ C \ \ .
\end{align*}
And so, by using Fubini's theorem it holds that
\begin{align*}
&\E_{\Lambda_1}\left[\int_B \left(\fch_1(x) -  \frac{1}{N}\sum_{i=1}^Nh_1(x,\wb_i)\right)^2
\muBdx\right] & 
&\E_{\Lambda_2}\left[\int_B \left(\fch_2(x) -  \frac{1}{N}\sum_{i=1}^Nh_2(x,\wb_i)\right)^2
\muBdx\right]\\
&= \int_B \E_{\Lambda_1}\left(\fch_1(x) -  \frac{1}{N}\sum_{i=1}^Nh_1(x,\wb_i)\right)^2
\muBdx &
&= \int_B \E_{\Lambda_2}\left(\fch_2(x) -  \frac{1}{N}\sum_{i=1}^Nh_2(x,\wb_i)\right)^2
\muBdx\\ 
&= \frac{1}{N} \int_B \E_{\Lambda_1}\left(\fch_1(x) -  h_1(x,\wb)\right)^2
\muBdx &
&= \frac{1}{N} \int_B \E_{\Lambda_2}\left(\fch_2(x) -  h_2(x,\wb)\right)^2
\muBdx\\
&\leq \frac{1}{N} \int_B \E_{\Lambda_1}|h_1(x,\wb)|^2 \muBdx \ \leq \ \frac{C^2}{N} &
&\leq \frac{1}{N} \int_B \E_{\Lambda_2}|h_2(x,\wb)|^2 \muBdx \ \leq \ \frac{C^2}{N} \ \ .
\end{align*}
This proves that for every $f_1\in\FCBo$ and $f_2\in\FCBt$, it holds over
$x\in B$ that
\begin{align*}
\fch_1(x) &\in \covcl\left\{\frac{\gamma}{\absB{w}}\,\big(\cos(\wdx+b) - \cos(b)\big)
\ \ \Big|\ \ w\in\Omega , \ |\gamma|\leq C , \ b\in\R \right\} = \covcl\,\Hc_{C,B}^{\cos,1}
\\[6pt]
\fch_2(x) &\in \covcl\left\{\frac{\gamma}{\absB{w}^2}\,\big(\cos(\wdx+b) + \wdx\,
\sin(b)- \cos(b)\big)
\ \ \Big| \ \ w\in\Omega , \ |\gamma|\leq C , \ b\in\R \right\} = \covcl\,\Hc_{C,B}^{\cos,2}
\end{align*}
with closure in the sense of $L_2(B)$ as $N\to\infty$.\\\\
Next we set $y=\frac{w}{\absB{w}}\cdot x$ and then note that for any $w\in\Omega$, this gives
$y\in[-1,1]$ over $x\in B$. Thus, points in $\Hc_{C,B}^{\cos,1}$ and $\Hc_{C,B}^{\cos,2}$ are
smooth univariate functions of $y$, respectively of the form
\begin{align*}
h_1(y) &= \frac{\gamma}{\absB{w}}\,\big(\cos(\absB{w}y+b) - \cos(b)\big) &
h_2(y) &= \frac{\gamma}{\absB{w}^2}\,\big(\cos(\absB{w}y+b) + \absB{w}y\,\sin(b)- \cos(b)\big)
\end{align*}
for any $w\in\Omega$, $|\gamma|\leq C$, and $b\in\R$. These functions are uniformly continuous
and bounded with $h_1(0)=h_2(0)=0$, $|h_1(y)|\leq C$, and $|h_2(y)|\leq C$ on $y\in[-1,1]$, thus
they can be uniformly approximated by piecewise constant and piecewise linear functions. And so,
letting $y_0=0 < y_1 < \cdots < y_{N-1} < y_N=1$ be a uniform partition of $[0,1]$ such that
$y_i-y_{i-1}=\frac{1}{N}$ holds for all $i\in[N]$, we define the piecewise functions
\begin{align*}
h_{1,N}(y) &= \sum_{i=0}^{N-1}\big(h_1(-y_i)-h_1(-y_{i-1})\big)\,\step(-y-y_i) \ 
+ \ \sum_{i=0}^{N-1}\big(h_1(y_i)-h_1(y_{i-1})\big)\,\step(y-y_i)\\
h_{2,N}(y) &= \sum_{i=0}^{N-1}\frac{h_2(-y_{i+1})-2h_2(-y_i)+h_2(-y_{i-1})}{1/N}\,\relu(-y-y_i)
+ \sum_{i=0}^{N-1}\frac{h_2(y_{i+1})-2h_2(y_i)+h_2(y_{i-1})}{1/N}\,\relu(y-y_i)
\end{align*}
where $y_j=0$ for any negative $j$. This gives that
\begin{align*}
h_{1,N}(y) &= 
\begin{cases}
h_1(y_{i-1}) & \text{if} \ \ y\in[y_{i-1},y_i] \\
h_1(-y_{i-1}) & \text{if} \ \ y\in[-y_i,-y_{i-1}]
\end{cases} &
h_{2,N}(y) &= 
\begin{cases}
h_2^y(y_{i-1},y_i) & \text{if} \ \ y\in[y_{i-1},y_i] \\
h_2^y(-y_i,-y_{i-1}) & \text{if} \ \ y\in[-y_i,-y_{i-1}]
\end{cases}
\end{align*}
holds for all $i\in[N]$, where $h_2^y(y_1,y_2)$ is the secant line between $h_2(y_1)$ and
$h_2(y_2)$. Then as $N\to\infty$, we get that $h_{1,N}\to h_1$ and $h_{2,N}\to h_2$ uniformly,
and that the coefficients of $h_{1,N}$ go to the values of $h_1$ and the coefficients of
$h_{2,N}$ go to the values of the derivative
$h_2'(y)=\frac{\gamma}{\absB{w}}\,(-\sin(\absB{w}y+b) + \sin(b))$.\\\\
Thus, the sums of the absolute values of the coefficients over $y\in[0,1]$ are bounded by the
respective total variation of $h_1(y)$ and $h_2'(y)$ over that interval. Since we have
$|h_1'(y)|=|\gamma||(-\sin(\absB{w}y+b))|\leq C$ and
$|h_2''(y)|=|\gamma||(-\cos(\absB{w}y+b))|\leq C$, then
\begin{align*}
\sum_{i=0}^N\big|h_1(y_i)-h_1(y_{i-1})\big| \leq \int_0^1|h_1'(y)|\d y &\leq C &
\sum_{i=0}^{N-1}\left|\frac{h_2(y_{i+1})-2h_2(y_i)+h_2(y_{i-1})}{1/N}\right|
\leq \int_0^1|h_2''(y)|\d y &\leq C \ \ .
\end{align*}
The same arguments then hold for the interval $y\in[-1,0]$, and so the total variation over the
whole interval $y\in[-1,1]$, and thus the sum of the absolute values of the coefficients over
that interval, is bounded by $2C$ in both cases. \\\\
This proves that for every $f_1\in\FCBo$ and $f_2\in\FCBt$, it holds over $x\in B$ that
\begin{align*}
\fch_1(x) &\in \covcl\big\{\gamma\,\step(a\cdot x+b) \ \ \big| \ \ |\gamma|\leq2C ,
\absB{a}=1 , |b|\leq1\big\}  = \covcl\,\Hc_{C,B}^S \\
\fch_2(x) &\in \covcl\big\{\gamma\,\relu(a\cdot x+b) \ \ \big|
\ \ |\gamma|\leq2C , \absB{a}=1 , |b|\leq1\big\}  = \covcl\,\Hc_{C,B}^R
\end{align*}
with closure in the supremum norm over $x\in B$, and thus also in the sense of $L_2(B)$, as
$N\to\infty$. \\\\
To bound $\normB{h}$ for all $h$ in $\Hc_{C,B}^S$ and $\Hc_{C,B}^R$ respectively, note that in
both cases the largest norms are for points with $b=1$, thus giving
\begin{align*}
\normB{h}^2 &= \int_B (\gamma\,\step(a\cdot x+b))^2\,\muBdx \leq \gamma^2\int_{-1}^1
1^2\,\muy(\d y) \leq (2C)^2 \qquad \forall h\in\Hc_{C,B}^S \\
\normB{h}^2 &= \int_B (\gamma\,\relu(a\cdot x+b))^2\,\muBdx \leq \gamma^2\int_{-1}^1
(y+1)^2\,\muy(\d y) \leq \tfrac{4}{3}(2C)^2 \qquad \forall h\in\Hc_{C,B}^R
\end{align*}
where $\muy(\d y)=\frac{1}{2}\d y$ is the uniform measure over $y\in[-1,1]$.\\\\
The proofs are then completed by using Lemma~\ref{lem:convComb} with $\bar{c}=(2C)^2$ and
$\bar{c}=\frac{4}{3}(2C)^2$ respectively.\\
\vphantom{a}\hfill$\blacksquare$


\subsubsection*{McDiarmid's Inequality}

Let $\wb_1,\dots,\wb_d$ be independent random variables in $\Wc_1,\dots,\Wc_d$ and let the
function $h:\Wc_1\times\dots\times\Wc_d\to\R$ satisfy, for some $k_1,\dots,k_d>0$, that
\begin{equation*}
|h(w_1,\dots,w_d)-h(\ww_1,\dots,\ww_d)|\leq k_i
\end{equation*}
holds for all $(w_1,\dots,w_d),(\ww_1,\dots,\ww_d)\in\Wc_1,\times\dots\times\Wc_d$ such that $w_j=\ww_j$ for all $j\neq i$, for all $i\in[d]$. Then, for any $t\geq0$, it holds that
\begin{equation}\label{McDIneq}
\P\Big[\,h(\wb_1,\dots,\wb_d)-\E[h(\wb_1,\dots,\wb_d)]\,\geq t\,\Big]
\ \leq \ \exp\left(\frac{-2\,t^2}{\sum_{i=1}^dk_i^2}\right) \ \ .
\end{equation}

%%% Local Variables:
%%% mode: latex
%%% TeX-master: "main"
%%% End:
