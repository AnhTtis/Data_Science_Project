\section{Randomly Initialized Activations with M.I.R.}\label{sec:random}
In this section we motivate and prove error bounds in the $\normB{\cdot}$ norm for an
approximation of the form \eqref{singleHidden} with internal parameters randomly initialized
over the set $\Thetal$, using the mollified integral representation previously defined in
Section~\ref{sec:mollified}. \\\\
Let us denote $\ftr(x,\Thetabr)$ as an approximation of the form \eqref{singleHidden} which uses
$\Thetabr = \{\thetab_1,\dots,\thetab_R\}\in\Thetalr$ sampled iid according to some nonzero
probability density $\Pl$, and define $\pmin=\min_{\theta\in\Thetal}\Pl(\theta)>0$. The key property is that $P_\lambda$ is an arbitrary positive density. For example, we could take $P_{\lambda}$ to be uniform over $\Thetal$.
\\\\
The proof in this section then closely follows Lemma 4 in \cite{rahimi2008weighted}.\\\\

The main result for this section is shown next. 
\begin{theorem}\label{thRand}
For any target function $f:\R^n\to\R$ that is continuous on a bounded set $B\subset\R^n$,
parameters set $\Theta$, mollification $\lambda>0$, and activation function $\sigma$ satisfying
Assumption~\ref{sigAsmpt}, let $\fhlN$ be a mollified approximation defined in \eqref{molliFunc}.
% the random approximation
% $\ftr(x,\Thetabr)$ can be bounded as follows.
With probability greater than 1-$\nu$, for any
$\nu\in(0,1)$, over the iid draw $\Thetabr\simi\Pl$, there is a random approximation
$\ftr(x,\Thetabr)$ satisfying
% \begin{align*}
% &\normB{\fhlN(x)-\ftr(x,\Thetabr)} \\\nonumber
% &\hspace{20pt} \leq \frac{\Lc\,\Cgl}{\pmin \sqrt{R}}
% \bigg(\norm{\thmax}\ZB + \frac{|\sigma(0)|}{\Lc} \\ \nonumber
% & \hspace{100pt} + \Dt\ZB\sqrt{\tfrac{1}{2}\log\left(\tfrac{1}{\nu}\right)}\,\bigg)
% \end{align*}
\begin{multline}
  \nonumber
\normB{\fhlN(x)-\ftr(x,\Thetabr)} 
 \leq \frac{\Lc\,\Cgl}{\pmin \sqrt{R}}\\
\cdot\bigg(\norm{\thmax}\ZB + \frac{|\sigma(0)|}{\Lc} %\\ 
 + \Dt\ZB\sqrt{\tfrac{1}{2}\log\left(\tfrac{1}{\nu}\right)}\,\bigg)
\end{multline}
with $|c_j|\leq\frac{\Cgl}{\pmin\,R}$ for all $j\in[R]$, where \\
$\Cgl = \sum_{i=1}^N|c_i| \left(\frac{\eta\lambda}{e}\right)^{n+1}$.
\end{theorem}
\begin{proof}
First, we observe that \eqref{molliFunc} can be written as
\begin{align}\label{fhlNgRep}
\fhlN(x) &= \int_\Thetal \sum_{i=1}^{N} c_i\,\deltalno(\theta-\theta_i)\,\sigma(\theta\cdot z)\,\d\theta \\\nonumber
&= \int_\Thetal\gl(\theta)\,\sigma(\theta\cdot z)\,\d\theta \ \ ,
\end{align}
with $\gl:\Thetal\to\R$ being such that\\
$\max_{\theta\in\Thetal}\limits|\gl(\theta)|\leq \sum_{i=1}^N
|c_i|\left(\frac{\eta\lambda}{e}\right)^{n+1} = \Cgl$. \\\\
Next, we explicitly set the coefficients of $\ftr$ to
\begin{equation}\label{ujDef2}
c_j = \frac{\gl(\theta_j)}{\Pl(\theta_j)\,R}
\end{equation}
for all $j\in[R]$, which immediately satisfies
\begin{equation}\label{uMax}
|c_j|=\frac{|\gl(\theta_j)|}{\Pl(\theta_j)\,R}\leq\frac{\Cgl}{\pmin\,R} \ \ .
\end{equation}
Denoting $\d\Pl(\theta_j)=\Pl(\theta_j)\,\d\theta_j$ for all $j\in[R]$, then the iid assumption on $\thetab_i$ implies that 
\begin{align*}
&\Exp_{\Thetabr\simi\Pl}\limits\left[\ftr(x,\Thetabr)\right] \\
%&\, = \int_\Thetal\cdots\int_\Thetal \sum_{j=1}^R\frac{\gl(\theta_j)}{\Pl(\theta_j)\,R}
%\sigma(\theta_j\cdot z)\d\Pl(\theta_1)\cdots\d\Pl(\theta_R) \\
%&\overset{(a)}{=} \int_\Thetal\cdots\int_\Thetal \frac{\gl(\theta_1)}{\Pl(\theta_1)\,R}\,
%\sigma(\theta_1\cdot z)\,\d\Pl(\theta_1)\cdots\d\Pl(\theta_R) + \\
%& \cdots + \int_\Thetal\cdots\int_\Thetal \frac{\gl(\theta_1)}{\Pl(\theta_1)\,R}\,
%\sigma(\theta_1\cdot z)\,\d\Pl(\theta_1)\cdots\d\Pl(\theta_R) \\
%&\overset{(a)}{=} \int_\Thetal\frac{\gl(\theta_1)}{R}\,
%\sigma(\theta_1\cdot z)\,\frac{\d\Pl(\theta_1)}{\Pl(\theta_1)}\ \ 1^{R-1} \ + \cdots \\
%& \hspace{50pt} + \int_\Thetal\frac{\gl(\theta_R)}{R}\,
%\sigma(\theta_R\cdot z)\,\frac{\d\Pl(\theta_R)}{\Pl(\theta_R)}\ \ 1^{R-1} \\
&= \frac{1}{R}\Bigg(R\cdot \int_\Thetal \gl(\theta)\,
\sigma(\theta\cdot z)\,\d\theta \Bigg) 
= \fhlN(x) \ \ .
\end{align*}
% Here (a) is because each $\theta_j$ only belongs to
% the corresponding $\d\Pl(\theta_j)$ integral, while the remaining $R-1$ integrate to $1$,
% (b) is because there are $R$ copies of the same integral, and (c) is by \eqref{fhlNgRep}.\\\\
Next we define the function $h:\Thetalr\to\R$ as
\begin{align}\nonumber
h(\Thetabr) &= \normB{\ftr(x,\Thetabr) - \Exp_{\Thetabr\simi\Pl}\limits
\left[\ftr(x,\Thetabr)\right]} \\ \label{hDef}
&= \ \normB{\fhlN(x) - \ftr(x,\Thetabr)} \ \ .
\end{align}
Thus, for all $(\theta_1,\dots,\theta_R),(\thetaw_1,\dots,\thetaw_R)\in\Thetalr$ such that
$\theta_j=\thetaw_j$ for all $j\neq i$, we have that 
\begin{align}\nonumber
&\big|h(\theta_1,\dots,\theta_R)-h(\thetaw_1,\dots,\thetaw_R)\big| \\ \nonumber
&= \Big|\,\normB{\fhlN(x) - \ftr(x,\theta_1,\dots,\theta_R)} \\ \nonumber
&\hspace{40pt} - \normB{\fhlN(x) - \ftr(x,\thetaw_1,\dots,\thetaw_R)}\,\Big|\\ \nonumber
&\overset{(a)}{\leq} \big\|\fhlN(x) - \ftr(x,\theta_1,\dots,\theta_R) \\ \nonumber
&\hspace{40pt} - \fhlN(x) + \ftr(x,\thetaw_1,\dots,\thetaw_R) \big\|_B \\\nonumber
&\overset{(b)}{=} \normB{\frac{\gl(\theta_i)}{\Pl(\theta_i)\,R}\,
\sigma(\theta_i\cdot z) - \frac{\gl(\thetaw_i)}{\Pl(\thetaw_i)\,R}\,
\sigma(\thetaw_i\cdot z) } \\ \nonumber
&\overset{(c)}{\leq} \frac{\Cgl}{\pmin\,R}\,\normB{\sigma(\theta_i\cdot z) -
                                          \sigma(\thetaw_i\cdot z) } \\
  %\nonumber
%&\overset{(d)}{=} \frac{\Cgl}{\pmin\,R}\,\normB{\sigmaw(\theta_i\cdot z) -
%\sigmaw(\thetaw_i\cdot z) } \\ \nonumber
&\overset{(d)}{\leq} \frac{\Lc\Cgl}{\pmin\,R}\,\normB{(\theta_i -
                                            \thetaw_i)\cdot z }
  %\\
  \label{absHbound}
  %&
    \overset{(e)}{\leq} \frac{\Lc\Cgl}{\pmin\,R}\,\Dt\,\ZB
\end{align}
holds for all $i\in[R]$. Here (a) is by the reverse triangle inequality, (b) is by
\eqref{ujDef2}, (c) is by \eqref{uMax}, (d) is by the $\Lc$-Lipschitz property of $\sigma$, and
(e) is by Cauchy-Schwarz and the definitions of $\Dt$ and $\ZB$. \\\\
To bound the expectation of $h(\Thetabr)$, we begin with
\begin{align}\label{varNormK}
&\Exp_{\Thetabr\simi\Pl}\limits \normB{\ftr(x,\Thetabr) - \Exp_{\Thetabr\simi\Pl}\limits
\left[\ftr(x,\Thetabr)\right]}^2 \\ \nonumber
&\overset{(a)}{\leq} \frac{1}{R}\Exp_{\thetab\sim\Pl}\limits
\normB{\frac{\gl(\thetab)}{\Pl(\thetab)}\sigma(\thetab\cdot z)}^2 \\ \nonumber
&\leq \frac{\Cgl^2}{\pmin^2\,R}\,\Exp_{\thetab\sim\Pl}\limits
\normB{\sigma(\thetab\cdot z) }^2 \\\nonumber
&= \frac{\Cgl^2}{\pmin^2\,R}\,\Exp_{\thetab\sim\Pl}\limits\normB{\sigmaw(\thetab\cdot z) +
                                                \sigma(0)}^2
  %\\
%  \label{frBound}
%&\overset{(b)}{\leq} \frac{\Lc^2\,\Cgl^2}{\pmin^2\,R}\,\Exp_{\thetab\sim\Pl}\limits
%\normB{\thetab\cdot z + \frac{\sigma(0)}{\Lc}}^2 \ \ .
\end{align}
Here (a) is the identity \eqref{eq:varBnd} (defined in the appendix), and $\sigmaw(y)=\sigma(y)-\sigma(0)$.  

Now note that $\sqrt{\Exp_{\thetab\sim\Pl}\|f(x,\thetab)\|_B^2}$ defines a norm over functions, $f(x,\thetab)$.
% and (b) is by the
% $\Lc$-Lipschitz property of $\sigmaw(y)=\sigma(y)-\sigma(0)$ and that $\sigmaw(0)=0$, thus
% we have $\sigmaw(y)\leq\Lc y$. 
\\\\
The bound is completed by using the triangle inequality, followed by the Lipschitz property of $\sigmaw(y)$, and finally the upper bounds on $\theta$: 
\begin{align}\nonumber
&\Exp_{\Thetabr\simi\Pl}\limits\left[h(\Thetabr)\right]
= \Exp_{\Thetabr\simi\Pl}\limits\left[\sqrt{h^2(\Thetabr)\,}\,\right] \\\nonumber
%&\leq \sqrt{\Exp_{\Thetabr\simi\Pl}\limits\left[h^2(\Thetabr)\,\right]} \\\nonumber
%&= \sqrt{\Exp_{\Thetabr\simi\Pl}\limits \normB{\ftr(x,\Thetabr) - \Exp_{\Thetabr\simi\Pl}\limits
%\left[\ftr(x,\Thetabr)\right]}^2 } \\\nonumber
&\leq \frac{\Cgl}{\pmin\,\sqrt{R}\,}\,\sqrt{\Exp_{\thetab\sim\Pl}\limits 
                                                  \normB{\sigmaw(\thetab\cdot z) + \sigma(0)}^2} \\
  \nonumber
  &\leq \frac{\Cgl}{\pmin\,\sqrt{R}\,}\left( \sqrt{\Exp_{\thetab\sim\Pl}\limits 
                                                  \normB{\sigmaw(\thetab\cdot z)}} + |\sigma(0)|\right)
  \\
  \nonumber
  &\le
 \frac{\Cgl}{\pmin\,\sqrt{R}\,}\left(\Lc \sqrt{\Exp_{\thetab\sim\Pl}\limits 
                                                  \normB{\thetab\cdot z}} + |\sigma(0)|\right)
    \\
%  \nonumber
%&\overset{(a)}{\leq} \frac{\Lc\,\Cgl}{\pmin\,R}\,\sqrt{
%\normB{\norm{\thmax}\norm{z} + \frac{\sigma(0)}{\Lc}}^2} \\\nonumber
%&\overset{(b)}{\leq} \frac{\Lc\,\Cgl}{\pmin\,\sqrt{R}\,}\,\,\left(\sqrt{
%\Exp_{\thetab\sim\Pl}\limits \int_B (\thetab\cdot z)^2\,\muBdx} 
%+ \frac{\sigma(0)}{\Lc}\right) \\\nonumber
%&\overset{(c)}{\leq} \frac{\Lc\,\Cgl}{\pmin\,\sqrt{R}\,}\,\left( 
%\sqrt{\int_B \norm{\thmax}^2\norm{z}^2 \,\muBdx} + \frac{\sigma(0)}{\Lc} \ \right)
%   \nonumber
%&\leq \frac{\Lc\,\Cgl}{\pmin\,\sqrt{R}\,}\,
%\left(\Big\|\norm{\thmax}\norm{z}\Big\|_B + \frac{|\sigma(0)|}{\Lc}\right) 
  %\\
  \label{expH}
&= \frac{\Lc\,\Cgl}{\pmin\,\sqrt{R}\,}\,
\left(\norm{\thmax}\ZB + \frac{|\sigma(0)|}{\Lc}\right) \ \ .
\end{align}
% Here (a) is by Cauchy-Schwarz and definition of $\thmax$, (b) is by the triangle inequality, and
% (c) is by the definition of $\ZB$ and $\normB{\cdot}$.\\\\
Finally, we can apply McDiarmid's inequality \eqref{McDIneq} (see appendix) for any $t>0$ to
get that
\begin{equation}
\P\Big[\,h(\Thetabr)-\E[h(\Thetabr)] \geq t\Big]\leq 
\exp\left(\frac{-2\,t^2}{\sum_{i=1}^r k_i^2}\right) \ .
\end{equation}
In our case, from \eqref{absHbound} we have that $k_i=\frac{\Lc\Cgl}{\pmin\,R}\,\Dt\ZB$ for
all $i\in[R]$ and we bounded $\E[h(\Thetabr)]$ in \eqref{expH}. Therefore, for any $t>0$ we have
\begin{align}\nonumber
\P\Bigg[\,h(\Thetabr)-&\frac{\Lc\,\Cgl}{\pmin\,\sqrt{R}\,}\,
\left(\norm{\thmax}\ZB + \tfrac{|\sigma(0)|}{\Lc}\right) \geq t \Bigg] \\ \nonumber
&\leq \ \P\Big[\,h(\Thetabr)-\E[h(\Thetabr)] \geq t\Big] \\ \label{McD}
& \leq \ 
\exp\left(\frac{-2\,\pmin^2\,R\,t^2}{\Lc^2\,\Cgl^2\,\Dt^2\,\ZB^2} \right) \ .
\end{align}
Setting the RHS of \eqref{McD} to $\nu\in(0,1)$ and solving for $t$ gives
\begin{align}\nonumber
\log&\left(\frac{1}{\nu}\right) = 
\frac{2\,\pmin^2\,R\,t^2}{\Lc^2\,\Cgl^2\,\Dt^2\,\ZB^2} \\ \label{tVal}
& \implies \quad t \ = \ \frac{\Lc\,\Cgl\,\Dt\,\ZB}{\pmin\,\sqrt{R}}\,
\sqrt{\frac{1}{2}\log\left(\frac{1}{\nu}\right)} \ \ .
\end{align}
Combining \eqref{McD} and \eqref{tVal} then gives the result, for any $\nu\in(0,1)$, as the
complement of
\begin{align*}
\P\Bigg[\,h(\Thetabr) \ \geq \ &\frac{\Lc\,\Cgl}{\pmin\,\sqrt{R}\,}\,
\left(\norm{\thmax}\ZB + \frac{|\sigma(0)|}{\Lc}\right) \\ \nonumber 
&+ \frac{\Lc\,\Cgl\,\Dt\,\ZB}{\pmin\,\sqrt{R}}\,
\sqrt{\frac{1}{2}\log\left(\frac{1}{\nu}\right)} \ \Bigg] \ \leq \ \nu \ \ .
\end{align*}
\end{proof}

%%% Local Variables:
%%% mode: latex
%%% TeX-master: "main"
%%% End:
