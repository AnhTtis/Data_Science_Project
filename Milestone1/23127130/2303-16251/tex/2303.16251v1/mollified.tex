\section{Mollified Integral Representations}\label{sec:mollified}

In this section we develop \textit{mollified} versions of the expectation integral
representations \eqref{intRep} needed for randomly initialization approximations of the form
\eqref{linCombBasis}.\\\\
First note that a base approximation $\fbN$ \mbox{of the form \eqref{singleHidden}} always has
an exact integral representation as
\begin{align}\nonumber
&\fbN(x) = \sum_{i=1}^{N} c_i\,\sigma(\theta_i\cdot z)\,
\int_\Theta \deltano(\theta-\theta_i)\,\d\theta \\ \label{diracExact}
&= \sum_{i=1}^{N} c_i \int_\Theta \deltano(\theta-\theta_i)\,\sigma(\theta\cdot z)\,\d\theta \\
\nonumber
&= \sum_{i=1}^{N} c_i
\int_{\alpha_n}^{\beta_n}\cdots\int_{\alpha_1}^{\beta_1}\int_{\alpha_0}^{\beta_0}
\delta(w_1-w_{i,1})\cdots\delta(w_n-w_{i,n})\\\nonumber
&\hspace{80pt}\delta(b-b_i)\,\sigma(w\cdot x + b)\,
\d b\,\d w_1\cdots\,\d w_n
\end{align}
with $\delta$ as the Dirac delta, and so $\int_\Theta \deltano(\theta-\theta_i)\,\d\theta = 1$
and $\int_\Theta \deltano(\theta-\theta_i)\,\sigma(\theta\cdot z)\,\d\theta =
\sigma(\theta_i\cdot z)$ hold for all $i\in[N]$. And so, we define a mollified approximation
to \eqref{diracExact} as
\begin{equation}\label{molliFunc}
\fhlN(x) = \sum_{i=1}^{N} c_i \int_\Thetal \deltalno(\theta-\theta_i)\,\sigma(\theta\cdot z)\,\d\theta \ ,
\end{equation}
where $\deltal$ is a mollified version of the Dirac delta, defined for any
\textit{mollification} $\lambda>0$ as
\begin{equation}
\deltal(y) =
\begin{cases}
\eta\lambda\exp\left(\dfrac{-1}{1-\lambda^2y^2}\right) \hspace{20pt} y\in(-\frac{1}{\lambda},
\frac{1}{\lambda})\\
0 \hspace{100pt} o.w.
\end{cases}
\end{equation}
and thus
\begin{equation}
\deltal^n(x) =
\begin{cases}
(\eta\lambda)^n\exp\left(\dfrac{-1}{1-\lambda^2x_1^2}\right)\cdots
\exp\left(\dfrac{-1}{1-\lambda^2x_n^2}\right)\\[8pt]
\hspace{135pt} x\in(-\frac{1}{\lambda},\frac{1}{\lambda})^n\\
0 \hspace{130pt} o.w.
\end{cases}
\end{equation}
with $\eta^{-1}=\int_{-1}^{1}\exp\left(\frac{-1}{1-y^2}\right)\d y \approx0.444$ ensuring that
$\int_{(-\frac{1}{\lambda},\frac{1}{\lambda})^n}\deltal^n(x)\d x=1$ holds for all $\lambda>0$
and $n\geq1$.\\\\
To allow for the support of $\deltalno(\theta)$, which is
$\theta\in(-\frac{1}{\lambda},\frac{1}{\lambda})^{n+1}$, to be centered at any
$\theta_i\in\Theta$ including the boundary, we define the \textit{$\lambda$-expanded} parameters
set $\Thetal = [\alpha_1-\frac{1}{\lambda},\beta_1+\frac{1}{\lambda}]\times\cdots
\times[\alpha_n-\frac{1}{\lambda},\beta_n+\frac{1}{\lambda}]
\times[\alpha_0-\frac{1}{\lambda},\beta_0+\frac{1}{\lambda}]$. \\\\
%We assume going forward that this expansion of $\Theta$ is feasible. Otherwise, the original
%definition of $\Theta$ can be reduced by the $\frac{1}{\lambda}$ factor, at least for some
%$\lambda>1$, thus gaining back the original definition with the expansion.\\\\
Thus, we have $\theta_i\in\Theta\Rightarrow\theta_i\in\Thetal$ for all $i\in[N]$ and
$\lambda>0$, and so \eqref{diracExact} is equivalently
\begin{equation}\label{diracExactLam}
\fbN(x) = \sum_{i=1}^{N} c_i \int_\Thetal \deltano(\theta-\theta_i)\,
\sigma(\theta\cdot z)\,\d\theta \ .
\end{equation}
Let us denote the $\lambda$-expanded set of inputs to $\sigma$ as
$\Il=\{\theta\cdot z \ | \ \theta\in\Thetal, z\in K\times\{1\}\}$, as well as
$\Dt=\diam(\Thetal)$ and $\thmax=\argmax_{\theta\in\Thetal}\!\norm{\theta}$.\\\\
The main result for this section is now stated and proven as follows.
\begin{theorem}\label{th:moll}
For any target function $f:\R^n\to\R$ that is continuous on a bounded set $B\subset\R^n$,
parameter set $\Theta$, mollification $\lambda>0$, and activation function $\sigma$ satisfying
Assumption~\ref{sigAsmpt}, an approximation $\fbN$ of the form \eqref{singleHidden} and the corresponding mollified
$\fhlN$ defined by \eqref{molliFunc} satisfy
\begin{equation}\label{epslBound}
\normB{\fbN(x)-\fhlN(x)} \ \leq \ \sum_{i=1}^N|c_i| \,\frac{\Lc\sqrt{n+1}}{\lambda}\,\ZB \ \ .
\end{equation}
\end{theorem}
\begin{proof}
First, we observe that \eqref{diracExactLam} can be written as
\begin{align}\nonumber
\fbN(x) &= \sum_{i=1}^{N} c_i\,\sigma(\theta_i\cdot z)\,
\int_\Thetal \deltalno(\theta-\theta_i)\,\d\theta \\
&= \sum_{i=1}^{N} c_i\,\int_\Thetal \deltalno(\theta-\theta_i)\,\sigma(\theta_i\cdot z)\,
\d\theta \ ,
\end{align}
where $\int_\Thetal \deltalno(\theta-\theta_i)\,\d\theta=1$ holds for all $\lambda>0$, $n\geq1$,
and $\theta_i\in\Theta$, and $\sigma(\theta_i\cdot z)$ is a constant with respect to
$\d\theta$.\\\\
Then we define the set $\Thetail =
[w_{i,1}-\frac{1}{\lambda},w_{i,1}+\frac{1}{\lambda}]\times\cdots
\times[w_{i,n}-\frac{1}{\lambda},w_{i,n}+\frac{1}{\lambda}]
\times[b_i-\frac{1}{\lambda},b_i+\frac{1}{\lambda}]$ as the points
$\theta\in(-\frac{1}{\lambda},\frac{1}{\lambda})^{n+1}$ centered at $\theta_i$. Therefore, we
have
\begin{align}\nonumber
&|\fbN(x) - \fhlN(x)| = \\ \nonumber
%&\bigg| \, \sum_{i=1}^{N} c_i\,\int_\Thetal\deltalno(\theta-\theta_i)
%\,\sigma(\theta_i\cdot z)\,\d\theta \ - \\ \nonumber
%&\hspace{70pt} \sum_{i=1}^{N} c_i\,\int_\Thetal \deltalno(\theta-\theta_i)\,\sigma(\theta\cdot
%z)\,\d\theta \, \bigg| \\\nonumber
&= \left| \, \sum_{i=1}^{N} c_i\,\int_\Thetal\deltalno(\theta-\theta_i)
\,\Big(\sigma(\theta_i\cdot z) - \sigma(\theta\cdot z)\Big)\,\d\theta \, \right| \\\nonumber
&\overset{(a)}{\leq} \sum_{i=1}^{N} \, |c_i| \, \left| \int_\Thetal\deltalno(\theta-\theta_i)
\,\Big(\sigma(\theta_i\cdot z) - \sigma(\theta\cdot z)\Big)\,\d\theta \, \right| \\\nonumber
&\overset{(b)}{\leq} \sum_{i=1}^{N} \, |c_i| \, \int_\Thetal\deltalno(\theta-\theta_i)
\,\left|\sigma(\theta_i\cdot z) - \sigma(\theta\cdot z)\right|\,\d\theta \\\nonumber
&\overset{(c)}{\leq} \sum_{i=1}^{N} |c_i| \int_\Thetal\deltalno(\theta-\theta_i)
\max_{\phi\in\Thetail}\left|\sigma(\theta_i\cdot z) - \sigma(\phi\cdot z)\right|\d\theta
\\\nonumber
&= \sum_{i=1}^{N} |c_i| \max_{\phi\in\Thetail}\left|\sigma(\theta_i\cdot z) - \sigma(\phi\cdot z)
\right| \int_\Thetal\deltalno(\theta-\theta_i)\d\theta \\\nonumber
&\overset{(d)}{\leq} \sum_{i=1}^{N} \, |c_i| \, \Lc
\,\max_{\phi\in\Thetail}\left|(\theta_i -\phi)\cdot z)\right| \\ \label{absDifffNflN}
&\overset{(e)}{=} \ \sum_{i=1}^{N} \, |c_i| \, \Lc \, \frac{\lVert z\rVert \sqrt{n+1}}{\lambda} \ \ .
\end{align}
Here (a) is by the triangle inequality, (b) is because $\left|\int
f(x)\,\d x\right|\leq\int|f(x)|\,\d x$ and $\deltalno$ is positive, (c) is because the
integrand is only nonzero on $\theta\in\Thetail$ due to the support of $\deltalno$, (d) is the
$\Lc$-Lipschitz property of $\sigma$ and that $\int_\Thetal\deltalno(\theta-\theta_i)\,
\d\theta=1$ for all $i\in[N]$, and (e) uses the Cauchy-Schwarz inequality, combined with the fact that $\|\theta_i-\phi\|\le \frac{\sqrt{n+1}}{\lambda}$. 
\\\\
The definitions of $\normB{\cdot}$ and $\ZB$ together with \eqref{absDifffNflN} then give the
result as
\begin{align*}
&\normB{\fbN(x)-\fhlN(x)}\! =\! \sqrt{\int_B |\fbN(x) - \fhlN(x)|^2 \muBdx} \\
&\leq \sum_{i=1}^{N} \, |c_i| \, \frac{\Lc \sqrt{n+1}}{\lambda} \,
                                                                                 \sqrt{\int_B \lVert z\rVert^2\,\muBdx}\\
  &= \sum_{i=1}^N|c_i| \,\frac{\Lc \sqrt{n+1}}{\lambda}\,\ZB\ \ .
\end{align*}
\end{proof}

%%% Local Variables:
%%% mode: latex
%%% TeX-master: "main"
%%% End:
