\section{Conclusion}\label{sec:conc}
We have shown that the error between a target function $f$ that is continuous on a
bounded set $B\subset\R^n$ and a randomly initialized approximation $\ftr$ of the form
\eqref{linCombBasis} can be bounded using the triangle inequality and three separate but related
upper bounds: 1) from $f$ to a base approximation $\fbN:\R^n\to\R$ of the form
\eqref{singleHidden} with an arbitrary number of terms, 2) from $\fbN$ to the \textit{mollified}
approximation $\fhlN$, and 3) from $\fhlN$ to $\ftr$.\\\\
%There is a very clear path forward for improving this method: skip step 2 altogether! But this
%requires the existence of integral representations with the activation function of interest to
%either the target function $f$ directly, or to some class of basis functions which approximates
%$f$ in linear combination.\\\\
In \cite{pinkus1999approximation} Section 3, density results are proven for a very general class
of activation functions. Essentially proving that any target function $f$ that is continuous on a
compact set $B\subset\R^n$ is in the closure of the linear span of an activation in that general
class. It would seem then that such infinite combinations would naturally tend towards integral
representations. But continuing to find those corresponding bounded coefficient functions
$g:\Theta\to\R$, for a general class of target functions and activations, is the most obvious
direction for future efforts.