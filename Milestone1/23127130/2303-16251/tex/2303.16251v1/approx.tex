\section{Function Approximation}\label{sec:approx}
In this section, we cite and extend results for the existence of base approximations $\fbN$ of
the form \eqref{singleHidden} for various activation functions, to provide concrete and explicit
examples for use in Theorem~\ref{th:overall}.\\\\
Classic results from \cite{barron1993universal} and \cite{breiman1993hinging}, prove the
existence of convex combinations of scaled and shifted unit step functions and ReLU functions
which approximate functions in particular classes to order $O(\frac{1}{\sqrt{N}})$ with a hidden layer of width $N$. Since $N$ is arbitrary in our theorems, these two results
are very useful, as we can let $N\to\infty$ and obtain arbitrarily small $\epsapprox$ while
keeping the same $K_N$, since they are \textit{convex} combinations.\\\\
For any $C>0$, let $\FCBo$ be the class of functions $f$ which are continuous on $B$ and such
that there is some complex measure $\Gw(\dw)=e^{i\theta(w)}G(\dw)$ satisfying the following two
requirements:
\begin{enumerate}
\item for some real function $g$, which is equal to $f$ on $B$,
\begin{equation}\label{eq:funcReq1}
g(x)-g(0) = \displaystyle\int_{\R^n} (e^{i\wdx}-1)\,\Gw(\dw)
\end{equation}
holds for all $x\in\R^n$, such that
\item $C_{f,1} = \displaystyle\int_{\R^n} \absB{w} G(\dw) \leq C \ \ $.
\end{enumerate}
Then let $\FCBt$ be defined in the same way, except that the requirements are instead:
\begin{enumerate}
\item $g(x)-\nabla g(0)\cdot x-g(0) =$
\begin{equation}\label{eq:funcReq2}
\vphantom{a}\hspace{40pt} \displaystyle\int_{\R^n} (e^{i\wdx}-i\wdx-1)\,\Gw(\dw) \ \ .
\end{equation}
\item $C_{f,2} = \displaystyle\int_{\R^n} \absB{w}^2 G(\dw) \leq C \ \ $.
\end{enumerate}
Note the second requirement establishes the integrability of the first requirement, for both
classes. This is shown in the appendix. An example of such a $\Gw(\dw)$ is the Fourier
transform of a real function $g$, equal to $f$ on $B$, which has sufficient decay in the frequency domain to meet the second requirement. Also note that $g$ can be modified outside of
$B$ in order to assist with this. Finally, note that the definitions of $C_{f,1}$ and $C_{f,2}$
are masking the effect of the \textit{curse of dimensionality}. In general, there is nothing
assumed about $\Gw(\dw)$ that allows these constants to avoid inherent dependence on the
spatial dimension $n$.

\subsection{Unit Step and ReLU}
\begin{theorem}[\cite{barron1993universal} Thm. 1]\label{thm:barron93}
For any $C>0$ and bounded set $B\subset\R^n$ including the origin, there exists an approximation
$\fbN$ of the form \eqref{singleHidden} using unit step functions, for any $N\geq1$, to every
function $f\in\FCBo$, which satisfies
\begin{equation*}
\normB{\fch-\fbN} \ \leq \ \frac{2C}{\sqrt{N}}
\end{equation*}
with $\sum_{i=1}^N|c_i|\leq2C$ and $\fch(x)=f(x)-f(0)$.
\end{theorem}

\begin{theorem}[\cite{breiman1993hinging} Thm. 3]\label{thm:brei93}
For any $C>0$ and bounded set $B\subset\R^n$ including the origin, there exists an approximation
$\fbN$ of the form \eqref{singleHidden} using ReLU functions, for any $N\geq1$, to every
function $f\in\FCBt$, which satisfies
\begin{equation*}
\normB{\fch-\fbN} \ \leq \ \frac{\sqrt{\tfrac{16}{3}}\,C}{\sqrt{N}}
\end{equation*}
with $\sum_{i=1}^N|c_i|\leq2C$ and $\fch(x)=f(x)-\nabla f(0)\cdot x-f(0)$.
\end{theorem}

As we build upon their proof methods below, short proofs of these results are given in the appendix.


\subsection{Sign and Leaky ReLU}

Let us now define arbitrarily shifted and scaled versions of the unit step and ReLU functions
over $y\in\R$ as
\begin{align}\label{stepSS}
\steph(y) = \kh(\step(y)-\bh\,) \\ \label{reluSS}
\reluh(y) = \kh(\relu(y)-\wh y-\bh\,)
\end{align}
for any $\kh\in\R\setminus\{0\}$ and $\wh,\bh\in\R$. Note that \eqref{reluSS} is also known as a
\textit{hinge function}, which is defined as either the $\max$ or $\min$ of any two intersecting
hyperplanes $y=a\cdot x+b$ in $\R^n$.\\\\
In the proofs of Theorems~\ref{thm:barron93} and \ref{thm:brei93} it is shown that for every
$f_1\in\FCBo$ and $f_2\in\FCBt$, there respectively exists an (infinite) convex combination of
unit step or ReLU functions such that
\begin{align*}
f_1(x)-f_1(0) &= \sum_i u_i\,\gamma_i\,\step(a_i\cdot x + b_i) \\
f_2(x)-\nabla f_2(0)\cdot x-f_2(0) &= \sum_i u_i\,\gamma_i\,\relu(a_i\cdot x + b_i)
\end{align*}
hold with $|\gamma_i|\leq2C$, $\absB{a_i}=1$, $|b_i|\leq1$, $u_i\geq0$, and $\sum_iu_i=1$. And
so, if we then simply apply \eqref{stepSS} and \eqref{reluSS} to these relations, we get that
\begin{align*}
f_1(x)-\bh_1 &= \sum_i u_i\,\gammah_i\,\steph(a_i\cdot x + b_i) \\
f_2(x)-\wh_2\cdot x-\bh_2 &= \sum_i u_i\,\gammah_i\,\reluh(a_i\cdot x + b_i)
\end{align*}
hold with $\bh_1=f_1(0)+\sum_iu_i\gamma_i\,\bh\ $, $\bh_2=f_2(0)+\sum_iu_i\gamma_i\,\bh$,\\[6pt]
$\wh_2=\nabla f_2(0)+\sum_iu_i\gamma_i\,\wh$,
$|\gammah_i|=\big|\frac{\gamma_i}{\kh}\big|\leq\frac{2C}{|\kh|}$, and $u_i$, $a_i$, and $b_i$
the same as before.\\\\
Since the sign function $\sign$ satisfies \eqref{stepSS} for $\kh=2$ and $\bh=\frac{1}{2}$, and
the leaky ReLU function \cite{maas13rectifier} $\Lrelu$ satisfies \eqref{reluSS} for
$\kh=1-\alpha$, $\wh=\frac{-\alpha}{1-\alpha}$, and $\bh=0$ with any $\alpha\in(0,1)$ (typically
$\alpha=0.01$), we have the following extensions of the previous results.

\begin{theorem}\label{thm:sign}
For any $C>0$ and bounded set $B\subset\R^n$ including the origin, there exists an approximation
$\fbN$ of the form \eqref{singleHidden} using sign functions, for any $N\geq1$, to every
function $f\in\FCBo$, which satisfies
\begin{equation*}
\normB{\fch-\fbN} \ \leq \ \frac{C}{\sqrt{N}}
\end{equation*}
with $\sum_{i=1}^N|c_i|\leq C$ and $\fch(x)=f(x)-f(0)-2\sum_{i=1}^Nc_i$.
\end{theorem}
\begin{proof}
Applying \eqref{stepSS} to the proof of Theorem~\ref{thm:barron93} gives that for every
$f\in\FCBo$ we have $\fch(x)\in\covcl\,\Hc_{C,B}^{Si}$, where
\begin{align*}
\Hc_{C,B}^{Si}& = \\
&\big\{\gamma\,\sign(a\cdot x+b) \ \ \big| \ \ |\gamma|\leq\tfrac{2C}{2} , \absB{a}=1 ,
|b|\leq1\big\}
\end{align*}
and $\fch(x)=f(x)-f(0)-\sum_{i=1}^N2u_i\gamma_i$ with $u_i\geq0$ and $\sum_iu_i=1$. Note then
that $2u_1\gamma_1,\dots,2u_N\gamma_N$ are the corresponding coefficients for
Theorem~\ref{thm:barron93} using the unit step. We then set $c_i=u_i\gamma_i$ as the new
coefficients for all $i\in[N]$, and thus have $\sum_{i=1}^N|c_i|=\sum_{i=1}^Nu_i|\gamma_i|
\leq C$.\\\\
As in that proof, we note that the largest norm $\normB{h}$, for all $h$ in $\Hc_{C,B}^{Si}$, is
with $b=1$, which gives
\begin{align*}
\normB{h}^2 \leq \gamma^2\int_{-1}^1 1^2\,\muy(\d y) \leq C^2 \qquad \forall
h\in\Hc_{C,B}^{Si}
\end{align*}
with $\muy(\d y)=\frac{1}{2}\d y$ the uniform measure over $y\in[-1,1]$. Thus, using
Lemma~\ref{lem:convComb} with $\bar{c}=C^2$ completes this proof.
\end{proof}

\begin{theorem}\label{thm:Lrelu}
For any $C>0$ and bounded set $B\subset\R^n$ including the origin, there exists an approximation
$\fbN$ of the form \eqref{singleHidden} using leaky ReLU functions with any $\alpha\in(0,1)$,
for any $N\geq1$, to every function $f\in\FCBt$, which satisfies
\begin{equation*}
\normB{\fch-\fbN} \ \leq \ \frac{\sqrt{\tfrac{16}{3}}\,C}{(1-\alpha)\sqrt{N}}
\end{equation*}
with $\sum_{i=1}^N|c_i|\leq \frac{2C}{1-\alpha}$ and $\fch(x)=f(x)-\nabla f(0)\cdot x
+\big(\alpha\sum_{i=1}^N c_i\big)\cdot x -f(0)$.
\end{theorem}
\begin{proof}
Fix any $\alpha\in(0,1)$ and then applying \eqref{reluSS} to the proof of
Theorem~\ref{thm:brei93} gives that for every $f\in\FCBt$ we have
$\fch(x)\in\covcl\,\Hc_{C,B}^{LR}$, where
\begin{align*}
\Hc_{C,B}^{LR}& = \\
&\big\{\gamma\,\Lrelu(a\cdot x+b) \ \ \big| \ \ |\gamma|\leq\tfrac{2C}{1-\alpha} , \absB{a}=1 ,
|b|\leq1\big\}
\end{align*}
and $\fch(x)=f(x)-\nabla f(0)\cdot x-\big(\sum_{i=1}^N (-\alpha)u_i\gamma_i\big)\cdot x -f(0)$
with $u_i\geq0$ and $\sum_iu_i=1$. Note then that $(1-\alpha)u_1\gamma_1,
\dots,(1-\alpha)u_N\gamma_N$ are the \mbox{corresponding} coefficients for
Theorem~\ref{thm:brei93} using the ReLU. We then set $c_i=u_i\gamma_i$ as the new coefficients
for all $i\in[N]$, and thus have $\sum_{i=1}^N|c_i|=\sum_{i=1}^Nu_i|\gamma_i|\leq
\frac{2C}{1-\alpha}$.\\\\
As in that proof, we note that the largest norm $\normB{h}$, for all $h$ in $\Hc_{C,B}^{LR}$, is
with $b=1$, which gives
\begin{align*}
\normB{h}^2 \leq \gamma^2\int_{-1}^1 (y+1)^2\,\muy(\d y) \leq \frac{\frac{4}{3}(2C)^2}
{(1-\alpha)^2} \quad \forall h\in\Hc_{C,B}^{Si}
\end{align*}
with $\muy(\d y)=\frac{1}{2}\d y$ the uniform measure over $y\in[-1,1]$. Thus, using
Lemma~\ref{lem:convComb} with $\bar{c}=\frac{4}{3}\frac{(2C)^2}{(1-\alpha)^2}$ completes this
proof.
\end{proof}


\subsection{Approximates}
In this section we consider the following \textit{approximate} \mbox{activation} functions: the
logistic sigmoid $\sigmoid(y)=\frac{1}{1+e^{-y}}$ which approximates the unit step $\step(y)$,
the hyperbolic tangent  $\tanhy(y)=\frac{e^y-e^{-y}}{e^y+e^{-y}}$ which approximates the sign
function $\sign(y)$, and several $\elu(y)$ which approximate the ReLU $\relu(y)$, such as the ELU
\cite{clevert2015fast}, SELU \cite{klambauer2017self}, CELU \cite{barron2017continuously},
GELU \cite{hendrycks2016gaussian} and SiLU \cite{elfwing2018sigmoid}. The first three are equal
to the ReLU on the positive side ($y$ for $y>0$), while having various exponential curves for
$y\leq0$ tending to some $-c$. The latter two are simply products of $y$ with standard CDF
functions, the Gaussian (normal) $\Phi(y)$ and logistic $\sigmoid(y)$ distributions
respectively. \\\\
An important observation in \cite{barron1993universal} is that if there is a function
$f\in\covcl\, \Hc_C^\sigma$, where 
\begin{align*}
\Hc_{C,B}^\sigma = \big\{\gamma\,\sigma(a\cdot x+b) \ \ \big| \ \ |\gamma|\leq 2C ,
\absB{a}=1 , |b|\leq1 \big\}
\end{align*}
with closure in $L_2(B)$, and if there is a scalar function $g$ such that $\lim_{\beta \to\infty}g(\beta)\sigmaw(\beta y)=\sigma(y)$
$\muy$-almost everywhere to $\sigma(y)$, with
$\muy$ the uniform measure on $y\in[-1,1]$, then we also have that
$f\in\covcl\,\Hc_C^{\sigmaw}$, where 
\begin{align*}
\Hc_C^{\sigmaw} = 
\big\{\gamma\,g(t)\,&\sigmaw(a\cdot x+b) \ \ \big|\\
& |\gamma|\leq 2C , a\in\R^n, b\in\R, t>0 \big\}
\end{align*}
with closure in $L_2(B)$, by using the dominated convergence theorem. \\\\
Note then that as $\beta \to \infty$ logistic sigmoids have $\lim_{\beta\to\infty}\sigmoid(\beta y)=\step(y)$, $\muy$-almost everywhere, where $\step(y)$ is the unit
step. This follows since $\sigmoid(y)\to1$ as $y\to\infty$ and
$\sigmoid(y)\to0$ as $y\to-\infty$. The same reasoning shows that if $\tanhy$ is the hyperbolic tangent, then $\tanhy(\beta y)$ converges pointwise $\muy$-almost everywhere to the sign function $\sign(y)$ on $y\in[1,1]$.
This follows since $\tanhy(y)\to1$ as $y\to\infty$ and $\tanhy(y)\to-1$ as $y\to-\infty$.\\\\
For each of the previously mentioned \mbox{approximate} ReLU functions $\elu$, the rescaling $\frac{1}{\beta}\elu(\beta y)$
converges pointwise $\muy$-almost everywhere to the ReLU function $\relu(y)$ on $y\in[1,1]$.
Note that in all cases we have the identity $\frac{1}{\beta}\beta y=y$ for any $\beta>0$. Thus,
the convergence follows in the first three cases because
$\frac{1}{\beta}\elu(y)\to\frac{-c}{\beta}$ as $y\to-\infty$ and $\frac{-c}{\beta}\to0$ as
$\beta\to\infty$, while in the latter two cases CDF$(y)\to1$ as $y\to\infty$ and CDF$(y)\to0$ as
$y\to-\infty$.\\\\
However, we have assumed that $a,b$ are constrained within the bounded set
$\Theta\subset\R^{n+1}$. And so, let us define
\begin{align*}
\Hc_{C,B}^{\sigmaw,\beta} =
\big\{\gamma\,&g(t)\,\sigmaw(\beta(a\cdot x+b)) \ \ \big|\\
& |\gamma|\leq 2C , \absB{a}\leq1 , |b|\leq1, 0<t\leq\beta \big\}
\end{align*}
for any fixed $\beta>0$, thus giving arguments to $\sigmaw$ of $|\beta a\cdot x|\leq\beta$ over
$x\in B$ and $|\beta b|\leq \beta$. Clearly then, $\Hc_{C,B}^{\sigmaw,\infty}=\Hc_C^{\sigmaw}$,
and so any $f\in\covcl\,\Hc_{C,B}^\sigma$ is also in $\covcl\,\Hc_{C,B}^{\sigmaw,\infty}$ if the
above described convergence holds for $g\sigmaw\to\sigma$.\\\\
Let
$$
\Delta^{\sigmaw,\beta}=\sup_{h\in\covcl\Hc_{C,B}^{\sigma}}\inf_{\bar h\in\covcl\Hc_{C,B}^{\sigmaw,\beta}}\|h-\bar h\|_B.
$$
As long as $\sigma$ has only finitely many discontinuities, this quantity goes to $0$ as $\beta\to\infty$. Indeed, for any $\epsilon>0$, a finite collection of functions of the form $\{\gamma_i \sigma(a_i\cdot x+b_i)|i\in [M]\}\in \Hc_{C,B}^{\sigma}$ can be chosen such every $\min_{i\in [M]}\|\gamma \sigma(a\cdot x+b) -\gamma_i \sigma(a_i\cdot x+b_i)\|_B\le \epsilon$ for all $|\gamma|\le 2C$, $|a|_B=1$, and $|b|\le 1$. Then, we can choose $\beta$ sufficiently high so that $\|\gamma_i \sigma(a_i\cdot x+b_i)-\gamma_i g(\beta)\sigmaw(\beta(a_i\cdot x+b_i))\|_B\le \epsilon $ for each $i\in [M]$. 
% bound the maximum $\|h_N-\bar{h}_N\|_B$ over all $h_N$ convex
% combinations of $N$ points in $\Hc_{C,B}^\sigma$ and all $\bar{h}_N$ convex combinations of $N$
% points in $\Hc_{C,B}^{\sigmaw,\beta}$, for any fixed $\beta>0$, $N\geq1$, and approximate
% $\sigmaw$ that converges to $\sigma$ in the above sense.

The following corollary implies that if $\sigma$ can be used to produce accurate approximations of target function, $f$, than the approximation, $\sigmaw$ can give similar errors. 

\begin{corollary}
  \label{cor:scalingApprox}
For any $C>0$, bounded set $B\subset\R^n$ including the origin, parameter set
$\Theta\subset\R^{n+1}$, and $\beta>0$, let the following hold:
\begin{enumerate}
\item there exists an approximation $\fbN$ of the form \eqref{singleHidden} using activation
function $\sigma$, for any $N\geq1$, to every function $f$ in some set $\Fc$, which satisfies
$\normB{\fch-\fbN} \leq \epsilon_N$ for some $\epsilon_N>0$ that depends on $N$, some affine shift $\fch$ of $f$, and $\sum_{i=1}^N|c_i|\leq K_C$ for some $K_C>0$ that depends on $C$,
\item for some $g:\R\to\R$ and $\sigmaw$ approximate activation function, $g(\beta)
\sigmaw(\beta y)$ converges to $\sigma(y)$ pointwise $\muy$-almost everywhere as
$\beta\to\infty$, with $\muy$ the uniform measure on $y\in[-1,1]$, and
\item \label{cond:betaDiam}
  $(\beta a,\beta b)\in\Theta$ for all $\absB{a}\leq1$ and all $|b|\leq1$.
\end{enumerate}
Then, there also exists an approximation $\bar{h}_N$ of the form \eqref{singleHidden} using the
approximate activation function $\sigmaw$, for any $N\geq1$, to every function $f\in\Fc$, which
satisfies
\begin{equation*}
\normB{\fch-\bar{h}_N} \ \leq \ \epsilon_N + \Delta_N^{\sigmaw,\beta}
\end{equation*}
with $\sum_{i=1}^N|c_i|\leq K_C|g(\beta)|$ and the same $\fch(x)$ definition.
\end{corollary}
\begin{proof}
We have $\normB{\fch-\bar{h}_N} \leq \normB{\fch-\fbN} + \normB{\fbN-\bar{h}_N}$ by the triangle
inequality, with the former assumed bounded by $\epsilon_N$ and the latter bounded
by $\Delta^{\sigmaw,\beta}$ as defined above for $\bar{h}_N$ as a convex combination of $N$
points in $\Hc_{C,B}^{\sigmaw,\beta}$. Then the coefficients are
$u_1\gamma_1g(t_1),\dots,u_N\gamma_Ng(t_N)$ which gives the bound on $\sum_{i=1}^N|c_i|=
\sum_{i=1}^N|u_i\gamma_ig(t_i)|$ with $|g(t_i)|\leq|g(\beta)|$, $u_i\geq0$, and
$\sum_{i=1}^Nu_i=1$.
\end{proof}

Corollary~\ref{cor:scalingApprox} can be combined with Theorem~\ref{th:overall} to derive approximation bounds for any of the sigmoidal and hinge-like activation functions described above. Note, however, that the diameter of $\Theta$ may need to be suitably increased to ensure that Condition \ref{cond:betaDiam} of Corollary~\ref{cor:scalingApprox} holds. As a result, a greater number of samples, $R$, may be required to counteract the effect of the higher diameter in the bound from Theorem~\ref{th:overall}.

%%%%%%%%%%%%%% CAN TRY TO ADD THIS PART IN IF WE HAVE TIME
%Lastly, note that \cite{barron1993universal} (Lemma 5) gives a method of quantifying
%$\Delta_N^{\sigmaw,\beta}$ for specific $\sigmaw$. We do this in the appendix.
%\textbf{Remark:} for the ReLU approximates $\elu$, this means that $\sum_{i=1}^N|c_i|\to0$ as
%$\beta\to\infty$. This is because of the previously mentioned identity that
%$\frac{1}{\beta}\beta y=y$ for any $\beta>0$. Practically, we get to trade a $\frac{1}{\beta}$
%reduction of the outer coefficients for the $\beta$ increase of the inner coefficients, in order
%to maintain the same positive linear characteristic as the ReLU.

%%% Local Variables:
%%% mode: latex
%%% TeX-master: "main"
%%% End:
