\documentclass[10pt,journal,compsoc]{IEEEtran}
\usepackage{hyperref}
\usepackage{enumitem}
\usepackage{bbm}
\usepackage{xcolor}
\usepackage{stackengine}
\usepackage[export]{adjustbox}
% \usepackage[algo2e, ruled, linesnumbered]{algorithm2e} 
\usepackage{algorithm}
\usepackage{algorithmic}
\usepackage{ragged2e}

% *** CITATION PACKAGES ***
%
\ifCLASSOPTIONcompsoc
  % IEEE Computer Society needs nocompress option
  % requires cite.sty v4.0 or later (November 2003)
  \usepackage[nocompress]{cite}
\else
  % normal IEEE
  \usepackage{cite}
\fi

% *** GRAPHICS RELATED PACKAGES ***
%
\ifCLASSINFOpdf
  % \usepackage[pdftex]{graphicx}
  % declare the path(s) where your graphic files are
  % \graphicspath{{../pdf/}{../jpeg/}}
  % and their extensions so you won't have to specify these with
  % every instance of \includegraphics
  % \DeclareGraphicsExtensions{.pdf,.jpeg,.png}
\else
  % or other class option (dvipsone, dvipdf, if not using dvips). graphicx
  % will default to the driver specified in the system graphics.cfg if no
  % driver is specified.
  % \usepackage[dvips]{graphicx}
  % declare the path(s) where your graphic files are
  % \graphicspath{{../eps/}}
  % and their extensions so you won't have to specify these with
  % every instance of \includegraphics
  % \DeclareGraphicsExtensions{.eps}
\fi

% correct bad hyphenation here
\hyphenation{op-tical net-works semi-conduc-tor}

\usepackage{graphicx} %插入图片的宏包
\usepackage{float} %设置图片浮动位置的宏包
\usepackage{subfigure} %插入多图时用子图显示的宏包
\usepackage{amsmath}
\usepackage{booktabs}
\usepackage{algorithm}
\usepackage{algorithmic}
\newcommand{\daniel}[1]{\textcolor{black}{#1}}
\newcommand{\meng}[1]{\textcolor{black}{#1}}

\begin{document}
%
% paper title
% Titles are generally capitalized except for words such as a, an, and, as,
% at, but, by, for, in, nor, of, on, or, the, to and up, which are usually
% not capitalized unless they are the first or last word of the title.
% Linebreaks \\ can be used within to get better formatting as desired.
% Do not put math or special symbols in the title.
% \title{Attention-Aware Anime Face\\ Line Drawing Colorization}
\title{AnimeDiffusion: Anime Face Line Drawing Colorization via Diffusion Models}
 
%
%
% author names and IEEE memberships
% note positions of commas and nonbreaking spaces ( ~ ) LaTeX will not break
% a structure at a ~ so this keeps an author's name from being broken across
% two lines.
% use \thanks{} to gain access to the first footnote area
% a separate \thanks must be used for each paragraph as LaTeX2e's \thanks
% was not built to handle multiple paragraphs
%
%
%\IEEEcompsocitemizethanks is a special \thanks that produces the bulleted
% lists the Computer Society journals use for "first footnote" author
% affiliations. Use \IEEEcompsocthanksitem which works much like \item
% for each affiliation group. When not in compsoc mode,
% \IEEEcompsocitemizethanks becomes like \thanks and
% \IEEEcompsocthanksitem becomes a line break with idention. This
% facilitates dual compilation, although admittedly the differences in the
% desired content of \author between the different types of papers makes a
% one-size-fits-all approach a daunting prospect. For instance, compsoc 
% journal papers have the author affiliations above the "Manuscript
% received ..."  text while in non-compsoc journals this is reversed. Sigh.

\author{Yu~Cao$^\dag$,~\IEEEmembership{Student Member,~IEEE},
        Xiangqiao~Meng$^\dag$,
        P.Y.~Mok,~\IEEEmembership{Member,~IEEE},\\
        Xueting~Liu,
        Tong-Yee~Lee,~\IEEEmembership{Senior Member,~IEEE}
        and~Ping~Li,~\IEEEmembership{Member,~IEEE}% <-this % stops a space
\IEEEcompsocitemizethanks{
\IEEEcompsocthanksitem $^\dag$ indicates equal contribution.
\IEEEcompsocthanksitem Y.~Cao, X.~Meng, P.Y.~Mok and P.~Li are with The Hong Kong Polytechnic University, Hong Kong SAR, China. E-mail: \protect\\
\{yu-daniel.cao, xiangqiao.meng\}@connect.polyu.hk,\\ \{tracy.mok, p.li\}@polyu.edu.hk.
\IEEEcompsocthanksitem X.~Liu is with Caritas Institute of Higher Education, Hong Kong SAR, China. \protect E-mail: tliu@cihe.edu.hk.
\IEEEcompsocthanksitem T.-Y. Lee is with National Cheng Kung University, Tainan, Taiwan. \protect \\E-mail: tonylee@mail.ncku.edu.tw.
}% <-this % stops an unwanted space
\thanks{Manuscript received April 19, 2005; revised August 26, 2015.}}

% note the % following the last \IEEEmembership and also \thanks - 
% these prevent an unwanted space from occurring between the last author name
% and the end of the author line. i.e., if you had this:
% 
% \author{....lastname \thanks{...} \thanks{...} }
%                     ^------------^------------^----Do not want these spaces!
%
% a space would be appended to the last name and could cause every name on that
% line to be shifted left slightly. This is one of those "LaTeX things". For
% instance, "\textbf{A} \textbf{B}" will typeset as "A B" not "AB". To get
% "AB" then you have to do: "\textbf{A}\textbf{B}"
% \thanks is no different in this regard, so shield the last } of each \thanks
% that ends a line with a % and do not let a space in before the next \thanks.
% Spaces after \IEEEmembership other than the last one are OK (and needed) as
% you are supposed to have spaces between the names. For what it is worth,
% this is a minor point as most people would not even notice if the said evil
% space somehow managed to creep in.



% The paper headers
\markboth{Journal of \LaTeX\ Class Files,~Vol.~14, No.~8, August~2015}%
{Shell \MakeLowercase{\textit{et al.}}: Bare Demo of IEEEtran.cls for Computer Society Journals}

% in the abstract or keywords.
\IEEEtitleabstractindextext{%
\begin{abstract}
\justifying  
It is a time-consuming and tedious work for manually colorizing anime line drawing images, which is an essential stage in \meng{cartoon animation} creation pipeline. Reference-based line drawing colorization is a challenging task that relies on the precise cross-domain long-range dependency modelling between the line drawing and reference image. Existing \meng{learning methods} still utilize generative adversarial networks (GANs) as one key module of their model architecture.
In this paper, we propose a novel method called AnimeDiffusion using diffusion models that performs anime face line drawing colorization automatically. To the best of our knowledge, this is the first diffusion model tailored for anime content creation. 
% Due to the huge training consumption of the diffusion models, 
\meng{In order to solve the huge training consumption problem of diffusion models}, we design a hybrid training strategy, first pre-training a diffusion model with classifier-free guidance and then fine-tuning it with image reconstruction guidance. We find that with a few iterations of fine-tuning, the model shows wonderful colorization performance, as illustrated in Fig.~\ref{fig:teaser}. For training AnimeDiffusion, we conduct an anime face line drawing colorization benchmark dataset, which contains 31696 training data and 579 testing data. We hope this dataset can fill the gap of no available high resolution anime face dataset for colorization method evaluation.
Through multiple quantitative metrics evaluated on our dataset and a user study, we demonstrate AnimeDiffusion outperforms state-of-the-art GANs-based models for anime face line drawing colorization. 
We also collaborate with professional artists to test and apply our AnimeDiffusion for their creation work.
We release our code on \href{https://github.com/xq-meng/AnimeDiffusion}{https://github.com/xq-meng/AnimeDiffusion}.
\end{abstract}

% Note that keywords are not normally used for peerreview papers.
\begin{IEEEkeywords}
Line drawing colorization, diffusion models, conditional generation
\end{IEEEkeywords}}


% make the title area
\maketitle


% To allow for easy dual compilation without having to reenter the
% abstract/keywords data, the \IEEEtitleabstractindextext text will
% not be used in maketitle, but will appear (i.e., to be "transported")
% here as \IEEEdisplaynontitleabstractindextext when the compsoc 
% or transmag modes are not selected <OR> if conference mode is selected 
% - because all conference papers position the abstract like regular
% papers do.
\IEEEdisplaynontitleabstractindextext
% \IEEEdisplaynontitleabstractindextext has no effect when using
% compsoc or transmag under a non-conference mode.



% For peer review papers, you can put extra information on the cover
% page as needed:
% \ifCLASSOPTIONpeerreview
% \begin{center} \bfseries EDICS Category: 3-BBND \end{center}
% \fi
%
% For peerreview papers, this IEEEtran command inserts a page break and
% creates the second title. It will be ignored for other modes.
\IEEEpeerreviewmaketitle



\IEEEraisesectionheading{\section{Introduction}\label{sec:introduction}}
% Computer Society journal (but not conference!) papers do something unusual
% with the very first section heading (almost always called "Introduction").
% They place it ABOVE the main text! IEEEtran.cls does not automatically do
% this for you, but you can achieve this effect with the provided
% \IEEEraisesectionheading{} command. Note the need to keep any \label that
% is to refer to the section immediately after \section in the above as
% \IEEEraisesectionheading puts \section within a raised box.




% The very first letter is a 2 line initial drop letter followed
% by the rest of the first word in caps (small caps for compsoc).
% 
% form to use if the first word consists of a single letter:
% \IEEEPARstart{A}{demo} file is ....
% 
% form to use if you need the single drop letter followed by
% normal text (unknown if ever used by the IEEE):
% \IEEEPARstart{A}{}demo file is ....
% 
% Some journals put the first two words in caps:
% \IEEEPARstart{T}{his demo} file is ....
% 
% Here we have the typical use of a "T" for an initial drop letter
% and "HIS" in caps to complete the first word.
\IEEEPARstart
% {T}{his} demo file is intended to serve as a ``starter file''
% for IEEE Computer Society journal papers produced under \LaTeX\ using
% IEEEtran.cls version 1.8b and later.
% You must have at least 2 lines in the paragraph with the drop letter
% (should never be an issue)
% \hfill mds
% \hfill August 26, 2015
{L}{ine} drawing colorization is an essential process in the animation industry, however, manually colorizing is time consuming, especially for the line drawings with complex structure content. So, it is necessary and valuable to design a kind of automatic line drawing colorization system.
\meng{Line drawing colorization is challenging, because line drawings, different from grayscale images\cite{9188002, 9186041, 9904484, 8676327}, only contain structure content composing of a series of lines without any luminance or texture information.}
This question has greatly attracted attention of researchers in the field of Computer Graphics, therefore many approaches~\cite{qu2006manga,furusawa2017comicolorization,sykora2009lazybrush,9143503} are proposed for mange and cartoon line drawing colorization during the past time.

Early work~\cite{varga2017automatic} utilized neural network to automatically colorize the cartoon images with random color, which is the first deep learning-based cartoon colorization method. Nevertheless, many interactions are usually needed to refine the colored results to satisfy what the user specified. In order to effectively control the color of the colored result, many user-hint based methods have been proposed successively, such as scribble colors~\cite{ci2018user}, point colors~\cite{zhang2018two}, text-hint~\cite{kim2019tag2pix}, and language-based~\cite{zou2019language}. While these user-hint based methods are still not convenient and intuitive, especially for amateur users without aesthetic judgement. Reference-based colorization methods, such as \cite{lee2020reference,li2022eliminating,zhang2017style,sun2019adversarial,liu2022reference,cao2022attention} provide 
a more convenient way. Users only need to prepare a line drawing and a corresponding reference color image, and the algorithm can automatically complete the colorizing process without other manual intervention.

\begin{figure*}[ht]
\centering  %图片全局居中
\includegraphics[width=\linewidth]{./teaser.pdf}
\caption{We propose AnimeDiffusion that performs reference-based line drawing colorization. Give one reference color image (the left side) and four line drawings of the same character (on the top), AnimeDiffusion generates four colored results (on the bottom) with accurate color and semantic correspondence. In particular, we can generate precise eye color and surprising hair details. The anime character is Nishikino Maki of LoveLive and line drawings are drawn by Ms. Xiao Meng.}
\label{fig:teaser}
\end{figure*}

Reference-base line drawing colorization can be formulated as a conditional image generation task. Since generative adversarial networks (GANs) has become the mainstream model for many generation tasks in the last decade, especially using images as generation conditions. Many previous work for line drawing colorization utilized GANs as one of the most important module of their model architecture design. However, this kind of approach mainly focuses on the improvement of feature aggregation module of two extracted deep features. For example, Lee et al.~\cite{lee2020reference} proposed an attention based Spatial Correspondence Feature Transfer (SCFT) module. Li et al.~\cite{li2022eliminating} eliminated the gradient conflict among attention branches by using Stop-Gradient Attention (SGA) module. Cao et al.~\cite{cao2022attention} designed an attention-aware model for generating high quality colored anime line drawing images. Otherwise, GANs-based models require to deployment of multiple losses, which also increases the instability of the training process.

Along with the diffusion probabilistic models~\cite{sohl2015deep} (“diffusion models” for brevity) has been proved to be an excellent model that is capable of generating high quality images, many algorithms based on diffusion models have been proposed in recent years. As a novel generation algorithm, it has greatly \meng{promoted} the progress of AI-Generated Content (AIGC) technology. Inspired by this, we propose the first diffusion model called AnimeDiffusion tailored for anime face line drawing colorization. Since the diffusion models \meng{usually have} the problem of high computing consumption, we design a hybrid training strategy that consists of classifier-free guidance pre-training stage and image reconstruction guidance fine-tuning stage. 
During the AnimeDiffusion training procedure, we train an U-Net which regards the line drawings and reference images as conditional denoising input. In order to make the model learn semantic correspondence ability, the reference image is a geometry distorted version of original reference image by applying Thin-Plate Splines (TPS) transformation. 
The original reference image is added to a Gaussian noise and concatenated with line drawing and reference image together. The U-Net uses the concatenated images as input and predicts the noise that added onto the original reference feature map. 
This pre-training process mainly makes AnimeDiffusion learn the denoising ability.
In the fine-tuning stage of AnimeDiffusion, we calculate the MSE loss between the reconstructed image and the original reference image and update the parameters of AnimeDiffusion when performing reverse sampling task.
\meng{It is worth to note that our fine-tuning is different from the one applied in other approaches. Existing methods mainly fine-tune the pre-trained image-to-image\cite{meng2021sdedit},\cite{kim2022diffusionclip} or text-to-image\cite{rombach2022high},\cite{zhang2023adding} diffusion models for various kinds of downstream tasks. However, our fine-tuning allows us to train a diffusion model from sketch to perform colorization more efficiently.}
Experimental results demonstrate that AnimeDiffusion generates better results than the state-of-the-art GANs-based line drawing colorization models both qualitatively and quantitatively. In order to train AnimeDiffusion and fill the gap of no available high resolution anime face dataset, we conduct a novel benchmark dataset for academic research purpose. All original images are chosen from %Danbooru202
\cite{danbooru2020}.

Our main contributions can be summarized as follows:
\begin{itemize}[itemsep=0pt, topsep=0pt, parsep=0pt]
    \item
    We propose the first AnimeDiffusion model tailored for anime face line drawing colorization. Experiments demonstrate that AnimeDiffusion notably outperforms the GANs-based counterparts and achieves the state-of-the-art anime face line drawing colorization results.
    \item
    We design a hybrid training strategy for AnimeDiffusion in order to tackle the problem of high computing consumption of diffusion models. The proposed strategy can accelerate the network convergence and improve colorization performance.
    \item
    We conduct a new anime face line drawing colorization benchmark dataset, which contains 31696 training data and 579 testing data. Our dataset aims to fill the gap of no available high resolution ($256\times256$) anime face dataset for training and evaluation.
\end{itemize}

% \subsection{Subsection Heading Here}
% Subsection text here.

% % needed in second column of first page if using \IEEEpubid
% %\IEEEpubidadjcol

% \subsubsection{Subsubsection Heading Here}
% Subsubsection text here.


% An example of a floating figure using the graphicx package.
% Note that \label must occur AFTER (or within) \caption.
% For figures, \caption should occur after the \includegraphics.
% Note that IEEEtran v1.7 and later has special internal code that
% is designed to preserve the operation of \label within \caption
% even when the captionsoff option is in effect. However, because
% of issues like this, it may be the safest practice to put all your
% \label just after \caption rather than within \caption{}.
%
% Reminder: the "draftcls" or "draftclsnofoot", not "draft", class
% option should be used if it is desired that the figures are to be
% displayed while in draft mode.
%
%\begin{figure}[!t]
%\centering
%\includegraphics[width=2.5in]{myfigure}
% where an .eps filename suffix will be assumed under latex, 
% and a .pdf suffix will be assumed for pdflatex; or what has been declared
% via \DeclareGraphicsExtensions.
%\caption{Simulation results for the network.}
%\label{fig_sim}
%\end{figure}

% Note that the IEEE typically puts floats only at the top, even when this
% results in a large percentage of a column being occupied by floats.
% However, the Computer Society has been known to put floats at the bottom.


% An example of a double column floating figure using two subfigures.
% (The subfig.sty package must be loaded for this to work.)
% The subfigure \label commands are set within each subfloat command,
% and the \label for the overall figure must come after \caption.
% \hfil is used as a separator to get equal spacing.
% Watch out that the combined width of all the subfigures on a 
% line do not exceed the text width or a line break will occur.
%
%\begin{figure*}[!t]
%\centering
%\subfloat[Case I]{\includegraphics[width=2.5in]{box}%
%\label{fig_first_case}}
%\hfil
%\subfloat[Case II]{\includegraphics[width=2.5in]{box}%
%\label{fig_second_case}}
%\caption{Simulation results for the network.}
%\label{fig_sim}
%\end{figure*}
%
% Note that often IEEE papers with subfigures do not employ subfigure
% captions (using the optional argument to \subfloat[]), but instead will
% reference/describe all of them (a), (b), etc., within the main caption.
% Be aware that for subfig.sty to generate the (a), (b), etc., subfigure
% labels, the optional argument to \subfloat must be present. If a
% subcaption is not desired, just leave its contents blank,
% e.g., \subfloat[].


% An example of a floating table. Note that, for IEEE style tables, the
% \caption command should come BEFORE the table and, given that table
% captions serve much like titles, are usually capitalized except for words
% such as a, an, and, as, at, but, by, for, in, nor, of, on, or, the, to
% and up, which are usually not capitalized unless they are the first or
% last word of the caption. Table text will default to \footnotesize as
% the IEEE normally uses this smaller font for tables.
% The \label must come after \caption as always.
%
%\begin{table}[!t]
%% increase table row spacing, adjust to taste
%\renewcommand{\arraystretch}{1.3}
% if using array.sty, it might be a good idea to tweak the value of
% \extrarowheight as needed to properly center the text within the cells
%\caption{An Example of a Table}
%\label{table_example}
%\centering
%% Some packages, such as MDW tools, offer better commands for making tables
%% than the plain LaTeX2e tabular which is used here.
%\begin{tabular}{|c||c|}
%\hline
%One & Two\\
%\hline
%Three & Four\\
%\hline
%\end{tabular}
%\end{table}


% Note that the IEEE does not put floats in the very first column
% - or typically anywhere on the first page for that matter. Also,
% in-text middle ("here") positioning is typically not used, but it
% is allowed and encouraged for Computer Society conferences (but
% not Computer Society journals). Most IEEE journals/conferences use
% top floats exclusively. 
% Note that, LaTeX2e, unlike IEEE journals/conferences, places
% footnotes above bottom floats. This can be corrected via the
% \fnbelowfloat command of the stfloats package.

\section{Related Work}
\subsection{Line Drawing Colorization}
Since line drawing contains only structure information with sparse line sets, existing colorization methods for gray-scale images cannot directly be used. Many colorization methods tailored for line drawings have been developed. Traditional line drawing colorization approaches~\cite{qu2006manga}, \cite{sykora2009lazybrush} are commonly optimized-based which allow users to use brushes to inject desired color into specific \meng{regions.} With the advancement of deep learning technology and for the better control of color, many user-hint colorization methods spring up. The color hints are usually concatenated with line drawing and encoded as the input for neural network in many deep learning based methods. Ci et al.~\cite{ci2018user} proposed a conditional GAN model to colorize the anime line drawing using color scribbles, which can generate colored results with accurate shading. Zhang et al.~\cite{zhang2018two} developed a color points hint two-stage colorization method, which divided the complex colorization task into two simpler and goal-clearer subtasks. Kim et al.~\cite{kim2019tag2pix} utilized their SECat module to generate illustrations with quality details using text tags as their hints. Zou et al.~\cite{zou2019language} for the first time presented a language-based system for interactive colorization of scene sketches. However, the complexity of such user-hints methods will become more labor-intensive as the number of line drawings increase, many interactions are usually needed to refine the colored results and these methods are not user-friendly for amateur users without aesthetic judgement, especially for preparing appropriate color hints. Therefore, many reference based colorization methods have been proposed, and they are very suitable for colorizing line drawing sets or videos of anime characters, which need to keep the same characters with consistent colors during each frame. Sato et al.~\cite{sato2014reference} segmented the target and reference image into different regions, then represented regions as nodes of a graph structure and colorized the monochrome target image by matching the graphs of the target and reference images. Furusawa et al.~\cite{furusawa2017comicolorization} proposed the first semi-automatic system to colorize an entire manga with color features extracted from the input reference image. Chen et al.~\cite{9143503} proposed an active learning based framework to match local regions between line arts and reference color image, followed by mixed-integer quadratic programming (MIQP) which considers the spatial contexts to further refine matching results. Shi et al.~\cite{shi2022reference} proposed a new line art video colorization method using 3D convolutional module to refine the temporal consistency of the colored result. Dou et al.~\cite{dou2021dual} is the first work that utilizes the HSV color space for anime sketch colorization. Maejima et al.~\cite{maejima2021anime} proposed colorization method for anime character using few-shot learning. Sun et al.~\cite{sun2019adversarial} trained a dual conditional GAN to colorize contours in different styles which helps designers create icons. Li et al.~\cite{li2022style} presented an icon colorization system that is composed of an encoder-decoder network and a conditional normalizing flow. \meng{Our AnimeDiffusion} is a novel reference-based colorization tailored for anime face line drawing colorization. Compared with previous GANs-based methods, AnimeDiffusion can generate better results both in visual quality and quantitative metrics.

\subsection{Semantic Correspondence}
Semantic correspondence~\cite{xiao2022learning} is one of the fundamental problems in computer vision which goal is to establish dense correspondences across images containing the targets of the same category or with similar semantic information. In computer graphics, it is also very important for exemplar-based image colorization task, and there usually exists the same semantic information between the target image and exemplar image in practical colorization usage. For line art colorization, this can also be viewed as cross-domain correspondence since the texture difference between line drawing and reference color image. Zhang et al.~\cite{zhang2020cross} proposed an exemplar-based image translation system based on cross-domain correspondence learning. Lee et al.~\cite{lee2020reference} proposed an attention-based module to spatially match and aggregate the sketch feature and reference color image feature. Li et al.~\cite{li2021globally} designed a model to colorize grayscale natural image, even if the exemplar image has no similar semantic information of the target grayscale image. He et al.~\cite{he2019progressive} \meng{presented} a novel progressive color transfer model, which jointly optimizes dense semantic correspondences in the deep feature domain and the local color transfer in the image domain. Zhang et al.~\cite{zhang2019deep} proposed the first end-to-end exemplar-based video colorization algorithm, which unified the semantic correspondence and colorization into a single network. Lu et al.~\cite{lu2020gray2colornet} proposed an unified semantic color transfer system from reference image to the target grayscale image. Most of these semantic correspondence learning methods are designed for grayscale image or video colorization. However, we design AnimeDiffusion to perform anime face line drawing colorization, which can generate results with clear and accurate semantic colors.


\subsection{Diffusion Models}
Diffusion models such as denoising diffusion probabilistic models (DDPM)~\cite{ho2020denoising} have achieved great success in image generation tasks. It is shown that image generation models based on diffusion models have better performance in terms of training stability and generation quality~\cite{dhariwal2021diffusion, ho2022cascaded}. 
Denoising diffusion implicit models (DDIM)~\cite{song2020denoising} accelerates the sampling procedure and enables a determined generation process with given Gaussian noise. In addition to generating high quality images based on random noise, diffusion models also show good performance in conditional image-to-image translation tasks.
An image-to-image diffusion models (Palette)~\cite{saharia2022palette} offers a versatile and general framework for image manipulation. Stochastic Differential Editing (SDEdit)~\cite{meng2021sdedit} is a guided image editing and synthesis method, which synthesizes realistic images by iteratively denoising through a stochastic differential equation (SDE). Combined with the contrastive language-image pre-training (CLIP)~\cite{radford2021learning} model, it can also be used for multimodal generation tasks. A text-guided diffusion models (DiffusionCLIP)~\cite{kim2022diffusionclip} shows the flexibility to add text guidance conditions. Latent diffusion
models (LDMs)~\cite{rombach2022high} can be trained on limited computational
resources using powerful pre-trained autoencoders in the latent space. 
Compared with GANs-based models, image generation based on diffusion models can easily add a variety of guidance, such as texts, strokes, and reference images. 
Since existing diffusion models are designed for natural image generation with random noise or text prompt. Some diffusion models can perform natural image colorization based on the prior knowledge of color in real world. They cannot be directly used for our task.
\meng{Very recently, Zhang et al.\cite{zhang2023adding} proposed ControlNet which can generate diversity colored cartoon images according to the sketch input and text prompt. Since it used diffusion models pre-trained on natural image dataset, it sometimes produced some distorted results compared to input sketches.
Our AnimeDiffusion is the first one to perform reference-based anime face line drawing colorization with accurate color information using diffusion models.}

\begin{figure*}[ht]
\centering  %图片全局居中
\includegraphics[width=\linewidth]{./framework.pdf}
\caption{The flowchart of AnimeDiffusion.}
\label{fig:framework}
\end{figure*}
\section{Overview}
Given a reference image, we aim to colorize the line drawing with clear geometry structure and accurate semantic color. The core problem to solve is how to inject color from the corresponding position of reference image into line drawing. Previous GANs-based methods usually design two encoders to extract line drawing features and reference image features respectively, and then use feature aggregation block to integrate two cross-domain features in the latent space. This operation can make model learn semantic correspondence ability to generate colored results, and one discriminator is needed to distinguish the generated results from real colored images which makes the colored results more realistic. However, these GANs-based methods \meng{suffer from} two problems, one is that line drawing features and reference images features are integrated in the latent space using complicated feature aggregation module. This part of network design is usually stacked with various novel neural network modules and it is not intuitive and explainable for algorithm designers, \meng{such as \cite{lee2020reference}, \cite{li2022eliminating}, \cite{liu2022reference}.}
The other one is that the weighted sum of multiple loss is employed\meng{\cite{johnson2016perceptual}, \cite{huang2018multimodal}} combined with adversarial loss, this makes it difficult to train the network and the training process is more unstable. Additional normalization techniques \meng{\cite{huang2017arbitrary},\cite{miyato2018spectral}} or improvements of the GAN model itself \meng{\cite{arjovsky2017wasserstein, gulrajani2017improved, mao2017least}} are needed to solve this problem.


% Our proposed AnimeDiffusion model is based on diffusion models which can fundamentally solve the above two problems. 
\meng{However, diffusion models can fundamentally solve the above two problems.}
We formulate our training procedure as conditional noise prediction task, therefore, multiple condition images including line drawing and reference are directly concatenated in the pixel space. It is easier to extend diffusion models for other conditional generation tasks without the need to design special feature extractors. In addition, the loss function of diffusion models is simple and closely related to the training task. Combined with our designed hybrid training strategy, our model finally has excellent coloring ability. Without introducing additional discriminator network, the quality of the colorized images is extremely close to that of real colored images.

\section{AnimeDiffusion}
\subsection{Model Architecture}
As illustrated in Fig.~\ref{fig:framework}(a), assuming $I_{gt}$ is an original colored image, and an line drawing $I_{l}$ is extracted using XDoG~\cite{winnemoller2012xdog} extractor. The more detailed information about our data preparation will be introduced in section 5.1. Since there are usually large spatial structure discrepancy between the line drawing and reference image, in order to make AnimeDiffusion learn the accurate semantic correspondence ability during the training process, we apply \meng{TPS transformation\cite{24792}} to convert $I_{gt}$ to a geometry distorted version $I_{r}$. The forward diffusion goes from $I_{gt}$ to $I_{noise}^{(t)}$ with random $t$ step in the range of $T$. Then $I_{noise}^{(t)}$, $I_{l}$ and $I_{r}$ are concatenated together to comprise $I_{concate}$ with 7 channels. We build an U-Net to predict the noise with 3 channels that added onto the $I_{gt}$. \meng{We propose a novel conditional noise prediction proxy task for the pre-training stage by introducing 
$I_{l}$ and $I_{r}$ as additional conditional inputs.}
The information of $t$ is embedded using the time embedding function and is transmitted into all convolutional blocks of both encoder and decoder in the U-Net.
As is shown in Fig.~\ref{fig:framework}(b) and (c), the convolutional blocks in encoder and decoder have the same structure containing a residual block followed by an attention block. It is worth to note that we use dotted box to represent attention block is a selective usage. At the shallow layers of the encoder, we do not use attention block due to the large dimension of feature maps.
\meng{The encoder and decoder equipped with multi-head self-attention make our model efficiently capture global and local features in different convolutional layers.}
There is a linear projection module to map the embedded time information $t_{emb}$ to the one with the same size as the feature map after the first convolution operation. We use the common add operation to encode the time information into the convolutional block. The attention block is not only used as a sub-block in the encoder and decoder of U-Net, but also as an independent block in the bottleneck of U-Net. The detailed structure of attention block is illustrated in Fig.~\ref{fig:framework}(d). The use of attention block can make model learn long-range features and multi-scale features which is essential for our colorization task. Then we use denoising function to transfer the predicted noised $N_{pred}^{(t)}$ combined with $I_{noise}^{(t)}$ to the generated colored image $I_{gen}$.

\subsection{Line Extraction}
Due to the lack of a large amount of hand-drawn line data, it is quite time-consuming and laborious to expand the data volume by hand-drawn method. We use \meng{XDoG\cite{winnemoller2012xdog}} line style as the intermediate representation of line drawings during the training and inference stage. Any input line drawings created by artist will be automatically converted to XDoG style to fit the model. For this reason, we build a line extraction module integrated in AnimeDiffusion. During the training stage, reference color images are used as input to extract the line drawings, and during the inference stage, hand-drawn line drawings are used as input to transform the line draft style into XDoG style.

For given image $x$, the line extractor is described in the form\cite{winnemoller2012xdog}

\begin{equation}
\label{equ:xdog1}
    S_{\sigma, k, p}(x) = (1 + p) \cdot G_{\sigma}(x) - p \cdot G_{k\sigma}(x)
\end{equation}
where $G_{\sigma}$ and $G_{k\sigma}$ is Gaussian convolution operation,  $\sigma$ is the variance of Gaussian convolution kernel, $k$ is scaling ratio of the variance between two convolution, and $p$ is used to control the edge emphasis lines. 

We need a line extractor \meng{in which} line extraction results are as close as possible to the effect of a painter's hand-drawn line.
The variance $\sigma$ of the Gaussian convolution has a significant effect on the line thickness of the line drawing, and we choose $\sigma$ to be 0.3 to get a reasonable line width.
We hired a professional artist to draw line art for some of the images in our dataset. 
For a color image $I_c$, its line drawing extraction \meng{result} is $S_{k, p}(I_c)$, and the hand-drawn image by the painter is represented as $H(I_c)$.
The objective of parameters for the line extractor is

\begin{equation}
\label{equ:xdog2}
    \arg\mathop{\min}_{k, p}\ \sum_i \lVert S_{k, p}(I_i) - H(I_i)  \rVert_2^2
\end{equation}

\subsection{Training Strategy}
We design a hybrid training strategy to train AnimeDiffusion, which consists of classifier-free guidance pre-training stage and an image reconstruction guidance fine-tuning stage. This training strategy separates the denoising task from the image reconstruction task, makes the network learning a specific task at each stage, and is more beneficial to network training and weight updating. Each training step will be introduced in detail in \meng{section 4.3.1 and 4.3.2.}

\subsubsection{Pre-training Stage}

During the classifier-free guidance pre-training stage, AnimeDiffusion mainly learns denoising ability. 
As shown in Fig. \ref{fig:framework}, the original image $I_{gt}$ goes through a forward diffusion process which is a Markov chain since it adds Gaussian noise to $I_{gt}$ and obtains noisy image $I_{noise}^{t}$ for time step $t$ iteratively.
Each step of the forward process is a Gaussian transition.

\begin{equation}
\label{equ:1}
     q({I}_{noise}^{(t)} | {I}_{noise}^{(t-1)}) = \mathcal{N}({I}_{noise}^{(t)}; \sqrt{1 - \beta_t}{I}_{noise}^{(t-1)}, \beta_t \mathbf{I})
\end{equation}
where $\beta_t$ is variance schedule at time step $t$. 
The forward process of the diffusion model represents the addition of noise from step $0$ to $t$.
For the cumulative $t$ steps of noise addition $I_{noise}^{(1:T)}$, the marginal distribution is

\begin{equation}
\label{equ:2}
    q({I}_{noise}^{(1:T)} | {I}_{gt}) = \Pi_{t = 1}^T q({I}_{noise}^{(t)} | {I}_{noise}^{(t-1)}) 
\end{equation}
Under the condition of equation \ref{equ:1}, the marginal distribution of each forward step is a standard Gaussian distribution

\begin{equation}
\label{equ:3}
    q({I}_{noise}^{(t)} | {I}_{gt}) = \mathcal{N}({I}_{noise}^{(t)}; \sqrt{\bar{\alpha}_t} {I}_{gt}, (1 - \bar{\alpha}_t)\mathbf{I})
\end{equation}
where $\bar{\alpha}_t = \Pi_{i = 1}^t (1 - \beta_i)$. After $t$ times iteration, the result latent variable $I_{noise}^{(t)}$ can be simplified as

\begin{equation}
\label{equ:4}
    {I}_{noise}^{(t)} = \sqrt{\bar{\alpha}_t} {I}_{gt} + \sqrt{1 - \bar{\alpha}_t} \epsilon, \epsilon \sim \mathcal{N}(\mathbf{0}, \mathbf{I})
\end{equation}
The training objective of the model is to predict the noise $N_{pred}$=$\epsilon_\theta({I}_{l},{I}_{r},{I}_{noise}^{(t)}, t)$ with given noised data point ${I}_{noise}^{(t)}$, time step $t$ and condition $I_{l}$ and $I_{r}$, and optimizing the objective

\begin{equation}
    \label{equ:5}
    \mathbbm{E}_{{I}_{gt} \sim q({I}_{gt}), \epsilon \sim \mathcal{N}(\mathbf{0}, \mathbf{I}), {I}_{l},{I}_{r}, t} \Vert \epsilon - \epsilon_\theta({I}_{l},{I}_{r}, {I}_{noise}^{(t)}, t) \Vert_p^p
\end{equation}

For a simplified description, we subsequently use $\epsilon_\theta$ represent $\epsilon_\theta({I}_{l},{I}_{r}, {I}_{noise}^{(t)}, t)$.
Palette\cite{saharia2022palette} indicates that the L2 norm can capture the output distribution more faithfully, and we adopt $p = 2$ in our pre-training stage.

We used a model of U-Net with attention blocks to predict the noise $\epsilon$ added in equation \ref{equ:4}.
The U-Net needs to accept line drawing $I_l$, reference image $I_r$, and noisy image $I_{noise}^{t}$ as inputs. 
Considering the high spatial consistency between line drawing and color images, we concatenate the above three in the channel dimension.
And the subsequent experimental in section \ref{sec:exp} results demonstrate that without a complex feature fusion mechanism, the model we proposed can achieve the semantic correspondence between the line drawings and the reference maps. 
\meng{Our previous work\cite{cao2022attention} for the first time emphasized the importance of semantic correspondence for reference-based line drawing colorization. The method in this paper uses clever algorithm design to achieve more amazing results in accurate semantic correspondence, especially in the region of anime character face.}

\subsubsection{Fine-tuning Stage}
After training $\epsilon_\theta({I}_{l},{I}_{r},{I}_{noise}^{(t)}, t)$, diffusion models inference through the learned reverse process. Since the result distribution of forward process $p({I}_{noise}^{(T)})$ approximates a standard Gaussian distribution $\mathcal{N}(\mathbf{0}, \mathbf{I})$, the sampling process starts from pure Gaussian noise, followed by $T$ rounds of denoising.
On the one hand, training diffusion models often requires a large batch size and long iteration rounds with large computing consumption.
On the other hand, we want to strike a balance between the diversity and accuracy of generated results. 
Based on the pre-trained model already having some denoising ability, we introduce the image reconstruction guidance fine-tuning stage to improve the generation ability of AnimeDiffusion. 
Since the great diversity of the generated results of the diffusion model, as the training iterations grow, the quality of generated images is affected less by the input noise and more by the guidance condition.
Therefore we input the noise generated according to the reference image instead of random noise in the fine-tuning and inference process of the model.
According to equation \ref{equ:4}, $I_{gt}$ is estimated as

\begin{equation}
\label{equ:inf_x0}
    \widetilde{I}_{gt} = \frac{1}{\sqrt{\bar{\alpha}_t}}({I}_{noise}^{(t)} - \sqrt{1 - \bar{\alpha}_t} \epsilon_\theta({I}_{l},{I}_{r},{I}_{noise}^{(t)}, t))
\end{equation}
The mean value of reverse process $p_\theta({I}_{noise}^{(t-1)} | {I}_{noise}^{(t)}, {I}_{l}, {I}_{r})$ is parameterize as

\begin{equation}
\label{equ:inf_mu}
    \widetilde{\mu}_\theta({I}_{noise}^{(t)}, t) = \frac{\sqrt{\bar{\alpha}_{t - 1}} \beta_t}{1 - \bar{\alpha}_t} \cdot \widetilde{I}_{gt} + \frac{(1 - \bar{\alpha}_{t - 1}) \sqrt{\alpha_t}}{1 - \bar{\alpha}_t} \cdot {I}_{noise}^{(t)}
\end{equation}
With the estimation of $\mu_\theta({I}_{noise}^{(t-1)}, t)$, each iteration of reverse process is

\begin{equation}
\label{equ:inf_ddpm}
    {I}_{noise}^{(t-1)} = \widetilde{\mu}_\theta({I}_{noise}^{(t)}, t) + \sigma_t^2 \epsilon, \epsilon \sim \mathcal{N}(\mathbf{0}, \mathbf{I})
\end{equation}
where $\sigma_t$ is the sampling variance with $\sigma_t^2 = \frac{1 - \bar{\alpha}_{t - 1}}{1 - \bar{\alpha}_t}\beta_t$.

The time consumption and space consumption for fine-tuning by directly adopting the reverse process of DDPM is vast, so we use DDIM\cite{song2020denoising} as our denoising function, which is an alternative non-Markov chain denoising process with different sampling or reverse process

\begin{equation}
\label{equ:inf_ddim}
    {I}_{noise}^{(t' - 1)} = \sqrt{\bar{\alpha}_{t' - 1}}\widetilde{I}_{gt} + \sqrt{1 - \bar{\alpha}_{t' - 1} - \eta \sigma_{t'}^2}\epsilon_\theta + \eta \sigma_{t'} \epsilon
\end{equation}
DDIM obtains a sub-sequence of time $[0, T)$, where $t'$ is sampled time sequence\cite{song2020denoising}. 
$\eta$ is a hyper-parameter that controls whether noise is added during the reverse process. If $\eta$ is set to $0$, the process of image generation is deterministic.

Since both the forward and reverse processes of DDPM are random, the colorization results are different for the same sample. 
DDIM provides a deterministic reverse sampling strategy, but due to the different initial noise, there is no guarantee that the image reconstruction can be completed with the original image as the reference image.
To fully utilize the image synthesis performance of the diffusion model for image processing purposes, we borrow the deterministic forward process from DiffusionCLIP\cite{kim2022diffusionclip} in our fine-tuning stage.
According to equation \ref{equ:inf_x0} and \ref{equ:inf_ddim}, DDIM is condidered as an Euler method to solve an \meng{ordinary differential equation (ODE)}

\begin{equation}
    \label{equ:ode}
    \mathrm{d}\frac{{I}_{noise}^{(t)}}{\sqrt{\bar{\alpha}_t}} = \epsilon_\theta \cdot \mathrm{d}\sqrt{\frac{1 - \bar{\alpha}_t}{\bar{\alpha}_t}}
\end{equation}

The above ODE holds in a finite number of steps, and to obtain an accelerated forward process, we use the same sampling time series $t'$ as the reverse process of DDIM.
The recursive relation from $I_{noise}^{(t')}$ to $I_{noise}^{(t' + 1)}$ is simplified as

\begin{equation}
    I_{noise}^{(t' + 1)} = \sqrt{\bar{\alpha}_{t' + 1}}\widetilde{I}_{gt}(I_{noise}^{(t')}) + \sqrt{1 - \bar{\alpha}_{t' + 1}}\epsilon_\theta
\end{equation}
$\widetilde{I}_{gt}(I_{noise}^{(t')})$ represents the ground truth estimation function of equation \ref{equ:inf_x0}.
Accordingly, to obtain the deterministic reverse process, we apply $\eta$ as 0 in the reverse process of DDIM.

After several iterations, the denoising function finally generates the colored image $I_{gen}$. We calculate the mean square error (MSE) between $I_{gen}$ and $I_{gt}$, in order to constrain the generated image as close to the original image as possible in pixel level. It is worth noted that the dotted arrow in Fig. \ref{fig:framework} represents reverse gradient propagation to update the parameters of U-Net during the fine-tuning stage.
\begin{eqnarray}
    L_{rec} = \mathbbm{E}_{{I}_{gt} \sim q({I}_{gt}), \epsilon \sim \mathcal{N}(\mathbf{0}, \mathbf{I})}\parallel I_{gen} - I_{gt} \parallel_p^p
\end{eqnarray}

% \begin{algorithm}
% \label{alg:fine_tuning}
% \SetAlgoLined
% \KwData{\, $\epsilon_\theta$ pretrained noise prediction model, \\
% \qquad\quad $\widetilde{I}_{gt}(\ )$ ground truth estimation function, \\
% \qquad\quad $\mathcal{T}' = \{t'_{0}, \cdots T'\}$ sampled time sequence, \\
% \qquad\quad $\mathcal{I} = \{I_{gt, 0}, \cdots I_{gt, n}\}$ images for fine-tuning.
% }
% \KwResult{$\hat{\epsilon}_\theta$ fine-tuned noise prediction model}

% \texttt{\# Forward process}\;

% $\hat{\mathcal{I}} \leftarrow \{\ \}$ \;

% \For{$I_{gt}$ \texttt{in} $\mathcal{I}$}{
%     \For{$t'$ \texttt{in} $T'$}{
%         $\widetilde{\epsilon} \leftarrow \epsilon_\theta(I_l, I_r, I_{noise}^{t'}, t')$\;
        
%         $I_{noise}^{(t' + 1)} \leftarrow \sqrt{\bar{\alpha}_{t' + 1}}\cdot \widetilde{I}_{gt}(I_{noise}^{t'}) + \sqrt{1 - \bar{\alpha}_{t' + 1}}\cdot \widetilde{\epsilon} $ \;
%     }\;
%     $\hat{\mathcal{I}} \leftarrow \hat{\mathcal{I}} \cup \{I_{noise}^{T'}\}$\; 
% }

% \texttt{\# Reverse process for fine-tuning}\;

% $\hat{\epsilon}_\theta \leftarrow \epsilon_\theta$\;

% \For{$I_{noise}^{T'}$ \texttt{in} $\mathcal{\hat{I}}$}{
%     \For{$t'$ \texttt{in} \texttt{reverse}($\mathcal{T}'$)}{
%         $\widetilde{\epsilon} \leftarrow \epsilon_\theta(I_l, I_r, I_{noise}^{t'}, t')$\;

%         $I_{noise}^{(t' - 1)} \leftarrow \sqrt{\bar{\alpha}_{t' - 1}}\cdot \widetilde{I}_{gt}(I_{noise}^{t'}) + \sqrt{1 - \bar{\alpha}_{t' - 1}}\cdot \widetilde{\epsilon} $ \;
%     }
%     $I_{gen} \leftarrow I_{noise}^{(t'_{0})}$\;
    
%     $\mathcal{L} \leftarrow \mathbbm{E}_{{I}_{gt} \sim q({I}_{gt}), \epsilon \sim \mathcal{N}(\mathbf{0}, \mathbf{I})}\parallel I_{gen} - I_{gt} \parallel_p^p$\;
    
%     \texttt{Gradient step:} $\nabla_{\hat{\epsilon_\theta}} \mathcal{L}$
% }

% \caption{Anime Diffusion Fine-tuning}
% \end{algorithm}

\renewcommand{\algorithmicrequire}{\textbf{Input:}}
\renewcommand{\algorithmicensure}{\textbf{Output:}}

\begin{algorithm}
\caption{Anime Diffusion Fine-tuning}
\begin{algorithmic}[1]
\REQUIRE{\, $\epsilon_\theta$ pretrained noise prediction model, \\
\qquad $\mathcal{I}$ training anime face images, \\
\qquad $S$ number of sampling steps 
}
\ENSURE{$\hat{\epsilon}_\theta$ fine-tuned noise prediction model}
\STATE Initialize list of forward noisy images $\hat{\mathcal{I}}$;
\STATE Sampling a increasing sub-sequence $\mathcal{T}' = \{t'_{0}, \cdots T'\}$ of length $S$;
\FOR{$I_{gt}$ \textbf{in} $\mathcal{I}$}
    \FOR{$t'$ \textbf{in} $\mathcal{T}'$}
        \STATE Calculate $\widetilde{\epsilon} \leftarrow \epsilon_\theta(I_l, I_r, I_{noise}^{t'}, t')$;
        \STATE Predict the ground truth $\widetilde{I}_{gt}(I_{noise}^{t'}, t')$;
        \STATE Forward step $I_{noise}^{(t' + 1)} \leftarrow \sqrt{\bar{\alpha}_{t' + 1}} \widetilde{I}_{gt} + \sqrt{1 - \bar{\alpha}_{t' + 1}}\widetilde{\epsilon}$;
    \ENDFOR
    \STATE Update $\hat{\mathcal{I}}$;
\ENDFOR
\FOR{$I_{noise}^{T'}$ \textbf{in} $\mathcal{\hat{I}}$}
    \FOR{$t'$ \textbf{in} \texttt{reverse}($\mathcal{T}'$)}
        \STATE Calculate $\widetilde{\epsilon} \leftarrow \epsilon_\theta(I_l, I_r, I_{noise}^{t'}, t')$;
        \STATE Predict the ground truth $\widetilde{I}_{gt}(I_{noise}^{t'}, t')$;
        \STATE Reverse step $I_{noise}^{(t' - 1)} \leftarrow \sqrt{\bar{\alpha}_{t' - 1}}\widetilde{I}_{gt} + \sqrt{1 - \bar{\alpha}_{t' - 1}}\widetilde{\epsilon}$;
    \ENDFOR
    \STATE Update reconstruction loss $L_{rec}$;
    \STATE Gradient step $\nabla_{\hat{\epsilon}_\theta} L_{rec}$
\ENDFOR
\end{algorithmic}
\end{algorithm}

We perform a small number of steps of fine-tuning based on the pre-trained model, as shown in algorithm 1. 
\meng{In summary, we perform our proposed hybrid training strategy to reduce computation consumption of training and inference significantly in comparison with the traditional manner of diffusion method.}
\daniel{The training objective of the pre-training stage is to obtain the solution of equation \ref{equ:ode}, the derivative of the path from the initial distribution to the target distribution at time step $t$.
When applying equation \ref{equ:inf_ddim} for inference, the distribution of the sub-sequences of time step $\{t'\}$ will affect the colorization results due to the deviation between the predicted derivatives $\epsilon_\theta$ and the true path from time step $t'$ to $t'-1$.
Therefore, we fix the sub-sequence $t'$ in the fine-tuning stage and select a smaller number of sampling steps to make our model save inference time.
Fine-tuning the pre-trained model so that the noise prediction at time step $t'$ is closer to the direction pointing to the next time step $t' - 1$, rather than accurately predicting the derivative.}

\subsection{User Interface}

\begin{figure}[htb]
\centering  %图片全局居中
\includegraphics[width=0.95\linewidth]{./ui.pdf}
\caption{User interface of AnimeDiffusion for colorizing anime face line drawings.}
\label{fig:ui}
\end{figure}

As is shown in Fig. \ref{fig:ui}, a user interface is developed for users to perform line drawing colorization by our AnimeDiffusion. Users only need to provide the line drawing and the reference image as two inputs of AnimeDiffusion, then it can one-key automatically complete the coloring process to generate colored results without additional human intervention. The elapsed time of the colorizing operation is printed at the bottom of the interface.
\daniel{Owing to the adoption of the DDIM acceleration sampling strategy and the fine-tuning stage, our method generated high-quality colorization results with only a small number of sampling steps.
The colorization time consumption for a line drawing can be controled within 0.2s on a machine equipped with RTX 4090, excluding the initialization period.}
Our end-to-end AnimeDiffusion model can be directly integrated into the practical colorization pipeline in the animation creation industry. 
\meng{In contrast to other diffusion methods\cite{zhang2023adding},\cite{saharia2022palette}, our method can accurately edit face line drawings according to reference images. Especially as shown in Fig.\ref{fig:teaser}, the artist added a lot of detail lines to the hair, our AnimeDiffusion can also add detailed textures instead of just flat color to the hair after coloring.}

\section{Experiments}
\label{sec:exp}

\subsection{Dataset}
In this paper, we focus on the anime face line drawing colorization task. In order to train AnimeDiffusion, a large dataset of anime face images is necessary, and the image resolution should not be too low in order to adequately express the color and detail information of the face. However, there is no dataset that meets our needs and can be directly used for training. So we build a benchmark dataset for anime face line drawing colorization. All anime character images are collected from Danbooru2020\cite{danbooru2020}, which is a large-scale anime image database with 4.2m+ images.  According to our task requirements, we only cut out the face part. After simple manual alignment and denoising operation, a total of 31696 training data and 579 testing data are produced. Due to limitations in GPU memory and model computing efficiency, all images are resized to $256 \times 256$ resolution. To simulate the manual line drawing style by artists and generate paired line drawing images, we use XDoG\cite{winnemoller2012xdog} to extract line drawings from colored anime images and set the parameters of XDoG algorithm with $\phi = 1 \times 10^9$ to keep a step transition at the border of lines in line drawings. We randomly set $\sigma$ to be $0.3/0.4/0.5$ to get different levels of line thickness, which generalizes AnimeDiffusion on various line widths to avoid overfitting. And we set $p=9, k=4.5, \epsilon=0.01$ in XDoG. 
\meng{As mentioned before, in the practical colorization scenario, there are large space discrepancies between the target line drawing and reference image.}
In order to make AnimeDiffusion learn accurate semantic correspondence ability during training and \meng{avoid learning trivial solution by directly using pixel aligned training data, we randomly set the parameters of TPS transformation on colored reference images when loading training data, i.e. each image of one batch data will have different geometry distortions. To some extent, this is a data augmentation trick.}

\subsection{Implementation Details}
We implement our AnimeDiffusion model based on the PyTorch framework, and it is trained on 1 NVIDIA A100 GPU. All input image size is fixed at $256 \times 256$. 
For the diffusion hyper-parameters setting, we use a linear noise schedule of $(1e^{-6}, 1e^{-2})$ with 1000 time steps. 
We pre-train the model with a batch size of 32 for 300 epochs, and we don't find overfitting, and we fine-tune the model with a batch size of 4 for 1 epoch. 
\daniel{On our devices, the pre-training stage \daniel{takes} 40 hours and the fine-tuning stage \daniel{takes} 110 minutes.}
We apply the Adam optimizer with a learning rate of $1e^{-5}$ for both of the above stages. Besides, we have no other hyper-parameters to adjust, like learning rate decay or warm-up schedule.

\subsection{Qualitative Evaluation}
We compare our AnimeDiffusion with another three state-of-the-art GANs-based methods. %, Lee et al.\cite{lee2020reference}, Li et al.\cite{li2022eliminating} and Cao et al.\cite{cao2022attention}. 
Lee et al.~\cite{lee2020reference} proposed the self-augmented supervised training strategy and designed a model with an attention based Spatial Correspondence Feature Transfer (SCFT) module. We regard it as the baseline for line drawing colorization task. 
Li et al.~\cite{li2022eliminating} designed a Stop-Gradient Attention (SGA) module to  eliminate the gradient conflict among attention branches. 
Cao et al.~\cite{cao2022attention} proposed an attention-aware improved method based on \cite{lee2020reference}, which focus on the anime line drawing colorization task. 
Since the variety of anime characters' faces and combine with the actual needs of anime character creation, we set two anime cases separately including anime face with homochromatic pupils and anime face with heterochromatic pupils. The latter is a very challenging case, which needs model require a high precision extraction of local features and semantic correspondence. To the best of our knowledge, we are the first learning based work can generate results with accurate color in pupils according to the reference image with no extra eyes segmentation label~\cite{9143503} or pupil position estimation network~\cite{akita2020colorization}.

For homochromatic pupils case, we show detailed comparison results in Fig.~\ref{fig:detail_single_color}. Yellow region shows that AnimeDiffusion can recognize the ear semantic information from line drawing and inject the right color the same as face. Green region shows that AnimeDiffusion can maintain the light-reflecting effect compared with the original color image (Fig.~\ref{fig:detail_single_color}(c)). Blue region indicates that AnimeDiffusion can accurately transfer the color information from reference image (Fig.~\ref{fig:detail_single_color}(a)) into the line drawing (Fig.~\ref{fig:detail_single_color}(b)) in the eyes part. However, the other three methods have flaws in the areas we have marked in three colors.
\meng{Since the diffusion models are based on maximum likelihood estimation method that can estimate the probability density more accurately than GANs-based method. So our results are much clearer than the other three \daniel{methods}. The performance of feature aggregation modules in other three models is not the best, so the coloring effect is defective.
Combined with our proposed training strategy, denoising and reconstruction tasks are separated during the training procedure, making the network training more stable. Therefore, our model acquires better detailed features capture ability.}

\begin{figure}[ht]
    \centering
    \subfigure[]{
    \begin{minipage}[b]{0.31\linewidth}
        \includegraphics[width=0.95\linewidth, cframe={black 0.75pt}]{./500.pdf}
    \end{minipage}
    }\hspace{-3pt}
    \subfigure[]{
    \begin{minipage}[b]{0.31\linewidth}
        \includegraphics[width=0.95\linewidth, cframe={black 0.75pt}]{./l_28137.pdf}
    \end{minipage}
    }\hspace{-3pt}
    \subfigure[]{
    \begin{minipage}[b]{0.31\linewidth}
        \includegraphics[width=0.95\linewidth, cframe={black 0.75pt}]{./gt_28137.pdf}
    \end{minipage}
    } \\
    \subfigure[]{
    \begin{minipage}[b]{0.24\linewidth}
        \includegraphics[width=0.95\linewidth, cframe={black 0.75pt}]{./scft.pdf}\\
        \includegraphics[width=0.95\linewidth, cframe={green 0.75pt}]{./hair_scft.pdf}\\
        \includegraphics[width=0.95\linewidth, cframe={blue 0.75pt}]{./eye_scft.pdf}
    \end{minipage}
    }\hspace{-8pt}
    \subfigure[]{
    \begin{minipage}[b]{0.24\linewidth}
        \includegraphics[width=0.95\linewidth, cframe={black 0.75pt}]{./sga.pdf}\\
        \includegraphics[width=0.95\linewidth, cframe={green 0.75pt}]{./hair_sga.pdf}\\
        \includegraphics[width=0.95\linewidth, cframe={blue 0.75pt}]{./eye_sga.pdf}
    \end{minipage}
    }\hspace{-8pt}
    \subfigure[]{
    \begin{minipage}[b]{0.24\linewidth}
        \includegraphics[width=0.95\linewidth, cframe={black 0.75pt}]{./icme.pdf}\\
        \includegraphics[width=0.95\linewidth, cframe={green 0.75pt}]{./hair_icme.pdf}\\
        \includegraphics[width=0.95\linewidth, cframe={blue 0.75pt}]{./eye_icme.pdf}
    \end{minipage}
    }\hspace{-8pt}
    \subfigure[]{
    \begin{minipage}[b]{0.24\linewidth}
        \includegraphics[width=0.95\linewidth, cframe={black 0.75pt}]{./ours.pdf}\\
        \includegraphics[width=0.95\linewidth, cframe={green 0.75pt}]{./hair_ours.pdf}\\
        \includegraphics[width=0.95\linewidth, cframe={blue 0.75pt}]{./eye_ours.pdf}
    \end{minipage}
    }
    \caption{Detailed comparison of colorization results. (a) reference image, (b) line drawing, (c) original color image, (d) Lee et al.\cite{lee2020reference}, (e) Li et al.\cite{li2022eliminating}, (f) Cao et al.\cite{cao2022attention}, (g) AnimeDiffusion }
    \label{fig:detail_single_color}
\end{figure}
\begin{figure*}[p]
    \centering
    \subfigure[]{
    \begin{minipage}[b]{0.125\linewidth}
        \includegraphics[width=\linewidth]{./r_265.pdf}\vspace{4pt}
        \includegraphics[width=\linewidth]{./r_568.pdf}\vspace{4pt}
        \includegraphics[width=\linewidth]{./r_573.pdf}\vspace{4pt}
        \includegraphics[width=\linewidth]{./r_252.pdf}\vspace{4pt}
        \includegraphics[width=\linewidth]{./r_229.pdf}\vspace{4pt}
        \includegraphics[width=\linewidth]{./r_179.pdf}\vspace{4pt}
        \includegraphics[width=\linewidth]{./r_145.pdf}\vspace{4pt}
        \includegraphics[width=\linewidth]{./r_578.pdf}\vspace{4pt}
        \includegraphics[width=\linewidth]{./r_304.pdf}\vspace{4pt}
    \end{minipage}
    }
    \subfigure[]{
    \begin{minipage}[b]{0.125\linewidth}
        \includegraphics[width=\linewidth]{./l_622.pdf}\vspace{4pt}
        \includegraphics[width=\linewidth]{./l_1962.pdf}\vspace{4pt}
        \includegraphics[width=\linewidth]{./l_576.pdf}\vspace{4pt}
        \includegraphics[width=\linewidth]{./l_995.pdf}\vspace{4pt}
        \includegraphics[width=\linewidth]{./l_1697.pdf}\vspace{4pt}
        \includegraphics[width=\linewidth]{./l_404.pdf}\vspace{4pt}
        \includegraphics[width=\linewidth]{./l_1022.pdf}\vspace{4pt}
        \includegraphics[width=\linewidth]{./l_496.pdf}\vspace{4pt}
        \includegraphics[width=\linewidth]{./l_1604.pdf}\vspace{4pt}
    \end{minipage}
    }
    \subfigure[]{
    \begin{minipage}[b]{0.125\linewidth}
        \includegraphics[width=\linewidth]{./scft_265_00622.pdf}\vspace{4pt}
        \includegraphics[width=\linewidth]{./scft_568_01962.pdf}\vspace{4pt}
        \includegraphics[width=\linewidth]{./scft_573_576.pdf}\vspace{4pt}
        \includegraphics[width=\linewidth]{./scft_252_00995.pdf}\vspace{4pt}
        \includegraphics[width=\linewidth]{./scft_229_01697.pdf}\vspace{4pt}
        \includegraphics[width=\linewidth]{./scft_179_00404.pdf}\vspace{4pt}
        \includegraphics[width=\linewidth]{./scft_145_01022.pdf}\vspace{4pt}
        \includegraphics[width=\linewidth]{./scft_578_496.pdf}\vspace{4pt}
        \includegraphics[width=\linewidth]{./scft_304_01604.pdf}\vspace{4pt}
    \end{minipage}
    }
    \subfigure[]{
    \begin{minipage}[b]{0.125\linewidth}
        \includegraphics[width=\linewidth]{./sga_265_00622.pdf}\vspace{4pt}
        \includegraphics[width=\linewidth]{./sga_568_01962.pdf}\vspace{4pt}
        \includegraphics[width=\linewidth]{./sga_573_576.pdf}\vspace{4pt}
        \includegraphics[width=\linewidth]{./sga_252_00995.pdf}\vspace{4pt}
        \includegraphics[width=\linewidth]{./sga_229_01697.pdf}\vspace{4pt}
        \includegraphics[width=\linewidth]{./sga_179_00404.pdf}\vspace{4pt}
        \includegraphics[width=\linewidth]{./sga_145_01022.pdf}\vspace{4pt}
        \includegraphics[width=\linewidth]{./sga_578_496.pdf}\vspace{4pt}
        \includegraphics[width=\linewidth]{./sga_304_01604.pdf}\vspace{4pt}
    \end{minipage}
    }
    \subfigure[]{
    \begin{minipage}[b]{0.125\linewidth}
        \includegraphics[width=\linewidth]{./icme_265_00622.pdf}\vspace{4pt}
        \includegraphics[width=\linewidth]{./icme_568_01962.pdf}\vspace{4pt}
        \includegraphics[width=\linewidth]{./icme_573_576.pdf}\vspace{4pt}
        \includegraphics[width=\linewidth]{./icme_252_00995.pdf}\vspace{4pt}
        \includegraphics[width=\linewidth]{./icme_229_01697.pdf}\vspace{4pt}
        \includegraphics[width=\linewidth]{./icme_179_00404.pdf}\vspace{4pt}
        \includegraphics[width=\linewidth]{./icme_145_01022.pdf}\vspace{4pt}
        \includegraphics[width=\linewidth]{./icme_578_496.pdf}\vspace{4pt}
        \includegraphics[width=\linewidth]{./icme_304_01604.pdf}\vspace{4pt}
    \end{minipage}
    }
    \subfigure[]{
    \begin{minipage}[b]{0.125\linewidth}
        \includegraphics[width=\linewidth]{./our_265_00622.pdf}\vspace{4pt}
        \includegraphics[width=\linewidth]{./our_568_01962.pdf}\vspace{4pt}
        \includegraphics[width=\linewidth]{./our_573_576.pdf}\vspace{4pt}
        \includegraphics[width=\linewidth]{./our_252_00995.pdf}\vspace{4pt}
        \includegraphics[width=\linewidth]{./our_229_01697.pdf}\vspace{4pt}
        \includegraphics[width=\linewidth]{./our_179_00404.pdf}\vspace{4pt}
        \includegraphics[width=\linewidth]{./our_145_01022.pdf}\vspace{4pt}
        \includegraphics[width=\linewidth]{./our_578_496.pdf}\vspace{4pt}
        \includegraphics[width=\linewidth]{./our_304_01604.pdf}\vspace{4pt}
    \end{minipage}
    }
    \subfigure[]{
    \begin{minipage}[b]{0.125\linewidth}
        \includegraphics[width=\linewidth]{./gt_622.pdf}\vspace{4pt}
        \includegraphics[width=\linewidth]{./gt_1962.pdf}\vspace{4pt}
        \includegraphics[width=\linewidth]{./gt_576.pdf}\vspace{4pt}
        \includegraphics[width=\linewidth]{./gt_995.pdf}\vspace{4pt}
        \includegraphics[width=\linewidth]{./gt_1697.pdf}\vspace{4pt}
        \includegraphics[width=\linewidth]{./gt_404.pdf}\vspace{4pt}
        \includegraphics[width=\linewidth]{./gt_1022.pdf}\vspace{4pt}
        \includegraphics[width=\linewidth]{./gt_496.pdf}\vspace{4pt}
        \includegraphics[width=\linewidth]{./gt_1604.pdf}\vspace{4pt}
    \end{minipage}
    }
    \caption{Qualitative comparison for anime face with homochromatic pupils. (a) reference images, (b) line drawings, (c) Lee et al.\cite{lee2020reference}, (d) Li et al.\cite{li2022eliminating}, (e) Cao et al.\cite{cao2022attention}, (f) AnimeDiffusion, and (g) original color images.}
    \label{fig:single_qualitative}
\end{figure*}
\begin{figure*}[hbt]
    \centering
    \subfigure[]{
    \begin{minipage}[b]{0.125\linewidth}
        \includegraphics[width=\linewidth]{./r_2725.pdf}\vspace{4pt}
        \includegraphics[width=\linewidth]{./r_8717.pdf}\vspace{4pt}
        \includegraphics[width=\linewidth]{./r_1975.pdf}\vspace{4pt}
        \includegraphics[width=\linewidth]{./r_577.pdf}\vspace{4pt}
        \includegraphics[width=\linewidth]{./r_7526.pdf}\vspace{4pt}
    \end{minipage}
    }
    \subfigure[]{
    \begin{minipage}[b]{0.125\linewidth}
        \includegraphics[width=\linewidth]{./l_48.pdf}\vspace{4pt}
        \includegraphics[width=\linewidth]{./l_114.pdf}\vspace{4pt}
        \includegraphics[width=\linewidth]{./l_278.pdf}\vspace{4pt}
        \includegraphics[width=\linewidth]{./l_527.pdf}\vspace{4pt}
        \includegraphics[width=\linewidth]{./l_573.pdf}\vspace{4pt}
    \end{minipage}
    }
    \subfigure[]{
    \begin{minipage}[b]{0.125\linewidth}
        \includegraphics[width=\linewidth]{./scft_02725_048.pdf}\vspace{4pt}
        \includegraphics[width=\linewidth]{./scft_08717_114.pdf}\vspace{4pt}
        \includegraphics[width=\linewidth]{./scft_01975_278.pdf}\vspace{4pt}
        \includegraphics[width=\linewidth]{./scft_577_527.pdf}\vspace{4pt}
        \includegraphics[width=\linewidth]{./scft_07526_573.pdf}\vspace{4pt}
    \end{minipage}
    }
    \subfigure[]{
    \begin{minipage}[b]{0.125\linewidth}
        \includegraphics[width=\linewidth]{./sga_02725_048.pdf}\vspace{4pt}
        \includegraphics[width=\linewidth]{./sga_08717_114.pdf}\vspace{4pt}
        \includegraphics[width=\linewidth]{./sga_01975_278.pdf}\vspace{4pt}
        \includegraphics[width=\linewidth]{./sga_577_527.pdf}\vspace{4pt}
        \includegraphics[width=\linewidth]{./sga_07526_573.pdf}\vspace{4pt}
    \end{minipage}
    }
    \subfigure[]{
    \begin{minipage}[b]{0.125\linewidth}
        \includegraphics[width=\linewidth]{./icme_02725_048.pdf}\vspace{4pt}
        \includegraphics[width=\linewidth]{./icme_08717_114.pdf}\vspace{4pt}
        \includegraphics[width=\linewidth]{./icme_01975_278.pdf}\vspace{4pt}
        \includegraphics[width=\linewidth]{./icme_577_527.pdf}\vspace{4pt}
        \includegraphics[width=\linewidth]{./icme_07526_573.pdf}\vspace{4pt}
    \end{minipage}
    }
    \subfigure[]{
    \begin{minipage}[b]{0.125\linewidth}
        \includegraphics[width=\linewidth]{./our_02725_048.pdf}\vspace{4pt}
        \includegraphics[width=\linewidth]{./our_08717_114.pdf}\vspace{4pt}
        \includegraphics[width=\linewidth]{./our_01975_278.pdf}\vspace{4pt}
        \includegraphics[width=\linewidth]{./our_577_527.pdf}\vspace{4pt}
        \includegraphics[width=\linewidth]{./our_07526_573.pdf}\vspace{4pt}
    \end{minipage}
    }
    \subfigure[]{
    \begin{minipage}[b]{0.125\linewidth}
        \includegraphics[width=\linewidth]{./gt_48.pdf}\vspace{4pt}
        \includegraphics[width=\linewidth]{./gt_114.pdf}\vspace{4pt}
        \includegraphics[width=\linewidth]{./gt_278.pdf}\vspace{4pt}
        \includegraphics[width=\linewidth]{./gt_527.pdf}\vspace{4pt}
        \includegraphics[width=\linewidth]{./gt_573.pdf}\vspace{4pt}
    \end{minipage}
    }
    \caption{Qualitative comparison for anime face with heterchromatic pupils. (a) reference images, (b) line drawings, (c) Lee et al.\cite{lee2020reference}, (d) Li et al.\cite{li2022eliminating}, (e) Cao et al.\cite{cao2022attention}, (f) AnimeDiffusion, and (g) original color images.}
    \label{fig:double_qualitative}
\end{figure*}

More comparison results are illustrated in Fig.~\ref{fig:single_qualitative}, given line drawings and reference images, AnimeDiffusion generates colored results with accurate color and good semantic correspondence. The image is clear without noise, and the color texture is smooth and soft. Especially in the eyes, the color is very precise, and the sense of light in the eyes is kept very good and full of charm. By contrast, Lee et al.~\cite{lee2020reference} generates results with color bleeding and wrong semantic correspondence, the color texture is rough and detail information is unclear. Li et al.~\cite{li2022eliminating} generates results with inaccurate color. Cao et al.~\cite{cao2022attention} generates results with sharpen image quality and wrong color in eyes. In general, the quality of images produced by GANs-based methods is not very stable and random flaw sometimes occurs. Our results are of high quality with beautiful color and rich details. Compared with the other three GANs-based methods, the image texture quality has been significantly improved.

For heterochromatic pupils case which is a very challenging task of anime face line drawing colorization. As is illustrated in Fig.~\ref{fig:double_qualitative}, AnimeDiffusion can generate results with accurate color in different pupils according to the reference image. Although Lee et al.~\cite{lee2020reference} produces some results with heterochromatic pupils, but the overall quality of results is not high, there are still color bleeding and inaccurate semantic correspondence. Li et al.~\cite{li2022eliminating} generates results with distorted color globally and inaccurate color in pupils.
Cao et al.~\cite{cao2022attention} fail to handle the heterochromatic pupils case, but the image sharpness and semantic information are still accurate.
\meng{As heterochromatic pupil is a fine-grained feature in the image space, and GAN is not accurate in data distribution modeling, the three SOTA GANs-based methods\daniel{\cite{lee2020reference, li2022eliminating, cao2022attention}} cannot handle it well. In contrast, our diffusion-based solution takes advantage of its precise data modeling property, combined with the use of multi-scale feature self-attention modules. Therefore, pupils can be colorized accurately according to the reference images without introducing additional processing modules.}

\subsection{Quantitative Evaluation}
\subsubsection{Evaluation Metrics}
We mainly use three evaluation metrics for quantitative comparison AnimeDiffusion with other methods. The popular Fréchet Inception Distance (FID) is used to assess the generation ability of algorithms in perceptual level. Besides measuring the perceptual credibility, we also adopt Peak Signal-to-Noise Ratio (PSNR) and Multi-Scale Structural Similarity Index
Measure (MS-SSIM) to evaluate the image reconstruction ability of algorithms in pixel level.
We design two kinds of colorization tasks respectively including self-reference reconstruction and random-reference colorization to analysize the colorization performance of AnimeDiffusion.  

\subsubsection{Self-reference Reconstruction}
For self-reference colorization, the line drawing and reference image are paired, ideally the colorized output should be exactly the same as the reference image. We directly use our paired testing data for conducting self-reference colorization. In fact, during the training phase, the main task of AnimeDiffusion is to do the image self-reconstruction, through this proxy task, the network can learn the colorization ability. For fairness, we train AnimeDiffusion and other three GANs-based methods sufficiently to compute PSNR and MS-SSIM. As is shown in Table~\ref{tab:quantitative_compar}, AnimeDiffusion acquires the best image reconstruction performance.

\subsubsection{Random-reference Colorization}
For the random-reference colorization, it is more like the common practical usage when using reference-based colorization method. We shuffle all the reference images in our testing data, then use unpaired line drawings and reference images to perform random-reference colorization. Using the total 579 generated images and 579 reference images to compute FID score. A smaller FID indicates that the distribution of the colored images is closer to the reference images and indicates that the model with wonderful generation ability. As is shown in Table~\ref{tab:quantitative_compar}, AnimeDiffusion shows better generation ability than other three GANs-based methods. One thing needs to note is that although Cao et al.~\cite{cao2022attention} shows little poor image reconstruction performance than Lee et al.~\cite{lee2020reference}, but shows better generation ability than Lee et al.\cite{lee2020reference} and Li et al.~\cite{li2022eliminating}, that is because Lee et al.~\cite{lee2020reference} just learns a trivial solution. This point is also discovered and discussed in Li et al.~\cite{li2022eliminating}.

\begin{table}[ht]
    \centering
    \caption{Quantitative Comparison between AnimeDiffusion and Other Three SOTA GANs-based Methods}
    \label{tab:quantitative_compar}
    \begin{tabular}{lccc}
    \toprule
        Method & PSNR$\uparrow$  &  MS-SSIM$\uparrow$ & FID$\downarrow$  \\
    \midrule
        Lee et al.\cite{lee2020reference} & 23.8901 & 0.9224 & 57.19\\
        Li et al.\cite{li2022eliminating} & 18.6347 & 0.8209 & 49.33 \\
        Cao et al.\cite{cao2022attention}   & 19.7746 & 0.8388 & 46.39 \\
        AnimeDiffusion     & \textbf{25.4658} & \textbf{0.9596} & \textbf{44.19} \\
    \bottomrule
    \end{tabular}
\end{table}

\subsection{Ablation Study}
We perform extensive ablation experiments to verify the effectiveness of our designed fine-tuning strategy when training AnimeDiffusion. 
We find that the denoising model obtained by classifier-free guidance pre-training can generate images with high diversity, but this diversity also means that the colorization results are unstable since randomness is introduced by Gaussian noise. 

We believe that the pre-trained model already acquires the ability to capture the line structure and can inject different colors in the corresponding area according to the reference image. 
Fine-tuning is only to eliminate color gaps due to insufficient pre-training.
To validate our idea, we perform image reconstruction test and reference-based line drawing colorization test respectively. We fine-tune AnimeDiffusion with 1 epoch and 10 epochs with a batch size of 4 for comparison.


\begin{figure}[tb]
    \centering
    \subfigure[]{
    \begin{minipage}[b]{0.23\linewidth}
        \includegraphics[width=\linewidth]{./rec_91_gt.pdf}\vspace{4pt}
        \includegraphics[width=\linewidth]{./rec_94_gt.pdf}\vspace{4pt}
        \includegraphics[width=\linewidth]{./rec_99_gt.pdf}
    \end{minipage}
    }\hspace{-3pt}
    \subfigure[]{
    \begin{minipage}[b]{0.23\linewidth}
        \includegraphics[width=\linewidth]{./rec_91_ft00.pdf}\vspace{4pt}
        \includegraphics[width=\linewidth]{./rec_94_ft00.pdf}\vspace{4pt}
        \includegraphics[width=\linewidth]{./rec_99_ft00.pdf}
    \end{minipage}
    }\hspace{-3pt}
    \subfigure[]{
    \begin{minipage}[b]{0.23\linewidth}
        \includegraphics[width=\linewidth]{./rec_91_ft01.pdf}\vspace{4pt}
        \includegraphics[width=\linewidth]{./rec_94_ft01.pdf}\vspace{4pt}
        \includegraphics[width=\linewidth]{./rec_99_ft01.pdf}
    \end{minipage}
    }\hspace{-3pt}
    \subfigure[]{
    \begin{minipage}[b]{0.23\linewidth}
        \includegraphics[width=\linewidth]{./rec_91_ft10.pdf}\vspace{4pt}
        \includegraphics[width=\linewidth]{./rec_94_ft10.pdf}\vspace{4pt}
        \includegraphics[width=\linewidth]{./rec_99_ft10.pdf}
    \end{minipage}
    }
    \caption{Image reconstruction test for ablation study. (a) Ground truth images, (b) Results without fine-tuning, (c) Results with fine-tuning for 1 epoch, (d) Results with fine-tuning for 10 epochs.}
    \label{fig:ablation_reconstruction}
\end{figure}
\begin{figure}[htb]
    \centering
    \subfigure[]{
    \begin{minipage}[b]{0.23\linewidth}
        \includegraphics[width=\linewidth]{./014_ref.pdf}\vspace{4pt}
        \includegraphics[width=\linewidth]{./085_ref.pdf}\vspace{4pt}
        \includegraphics[width=\linewidth]{./373_ref.pdf}
    \end{minipage}
    }\hspace{-4pt}
    % \subfigure[]{
    % \begin{minipage}[b]{0.18\linewidth}
    %     \includegraphics[width=\linewidth]{./433_line.pdf}\vspace{4pt}
    %     \includegraphics[width=\linewidth]{./380_line.pdf}\vspace{4pt}
    %     \includegraphics[width=\linewidth]{./256_line.pdf}
    % \end{minipage}
    % }\hspace{-4pt}
    \subfigure[]{
    \begin{minipage}[b]{0.23\linewidth}
        \includegraphics[width=\linewidth]{./ret_014_433_ft00.pdf}\vspace{4pt}
        \includegraphics[width=\linewidth]{./ret_085_380_ft00.pdf}\vspace{4pt}
        \includegraphics[width=\linewidth]{./ret_373_256_ft00.pdf}
    \end{minipage}
    }\hspace{-4pt}
    \subfigure[]{
    \begin{minipage}[b]{0.23\linewidth}
        \includegraphics[width=\linewidth]{./ret_014_433_ft01.pdf}\vspace{4pt}
        \includegraphics[width=\linewidth]{./ret_085_380_ft01.pdf}\vspace{4pt}
        \includegraphics[width=\linewidth]{./ret_373_256_ft01.pdf}
    \end{minipage}
    }\hspace{-4pt}
    \subfigure[]{
    \begin{minipage}[b]{0.23\linewidth}
        \includegraphics[width=\linewidth]{./ret_014_433_ft10.pdf}\vspace{4pt}
        \includegraphics[width=\linewidth]{./ret_085_380_ft10.pdf}\vspace{4pt}
        \includegraphics[width=\linewidth]{./ret_373_256_ft10.pdf}
    \end{minipage}
    }
    \caption{Reference-based line drawing colorization test for ablation study. (a) Reference images, (b) Results without fine-tuning, (c) Results with fine-tuning for 1 epoch, (d) Results with fine-tuning for 10 epochs.}
    \label{fig:ablation_colorization}
\end{figure}

We perform the image reconstruction test by distorting the ground truth image that serves as the reference image. Comparison results are shown in Fig.~\ref{fig:ablation_reconstruction}. The results without fine-tuning can reconstruct the overall structure of original image, but the color difference is obvious. After adding the image reconstruction loss to the fine tuning, the effect is significantly improved. The results of different fine-tuning schemes are not obvious in terms of visual differences

For reference-based line drawing colorization test, we show results in Fig.~\ref{fig:ablation_colorization}. Although the model without fine-tuning can distinguish regions that need different colors, there is still a color gap between the generated images and the reference images, and the result colored image looks dimmer. We think the model needs more training to generate results with accurate color information. However, training the diffusion model with classifier-free guidance is time-consuming, so we briefly fine-tune the model and get much better results.

We also compute PSNR, MS-SSIM and FID quantitative index for quantitative comparison. 
Results are shown in Table~\ref{tab:quantitative_ft}. After 10 epochs of fine-tuning, AnimeDiffusion continues to gain in FID score but little improvement in PSNR and MS-SSIM score. In fact, fine-tuning the model for 1 epoch is enough to have good colorization performance.
This also validates that our fine-tuning is mainly to fix color bias based on the pre-trained model with fundamental generation ability. Our designed hybrid training strategy can make model learn better colorization ability and save the training time cost.
% \begin{figure*}[htbp] %H为当前位置,!htb为忽略美学标准,htbp为浮动图形
% \centering %图片居中
% \subfigure[Input]{
% \label{subfig6.1}
% \includegraphics[width=0.15\textwidth]{./figures/6_line.jpg}}
% \subfigure[Reference]{
% \label{subfig6.2}
% \includegraphics[width=0.15\textwidth]{./figures/6_ref.jpg}}
% \subfigure[Ours]{
% \label{subfig6.3}
% \includegraphics[width=0.15\textwidth]{./figures/6_our.jpg}}
% \subfigure[\cite{li2022eliminating}]{
% \label{subfig6.4}
% \includegraphics[width=0.15\textwidth]{./figures/6_sga.jpg}}
% \subfigure[\cite{lee2020reference}]{
% \label{subfig6.5}
% \includegraphics[width=0.15\textwidth]{./figures/6_scft.jpg}}
% \subfigure[GT]{
% \label{subfig6.6}
% \includegraphics[width=0.15\textwidth]{./figures/6_gt.jpg}}
% \caption{More qualitative comparisons of colorization results between ours and \cite{li2022eliminating}, \cite{lee2020reference}.} %最终文档中希望显示的图片标题
% \label{Fig.6} %用于文内引用的标签
% \end{figure*}
\begin{table}[ht]
    \centering
    \caption{Quantitative Evaluation for Ablation Study Results}
    \label{tab:quantitative_ft}
    \begin{tabular}{lccc}
    \toprule
        AnimeDiffusion  & PSNR$\uparrow$ & MS-SSIM$\uparrow$ & FID$\downarrow$ \\
    \midrule
        Without Fine-tuning     & 12.4234 & 0.8079 & 55.1841 \\
        Fine-tuning (1 epoch)   & 25.4658 & 0.9596 & 44.1876 \\
        Fine-tuning (10 epochs) & \textbf{25.8992} & \textbf{0.9600} & \textbf{40.4392} \\
    \bottomrule
    \end{tabular}
\end{table}
\subsection{User Study}
It is generally challenging to evaluate the visual quality of images, in particular for line drawing colorization. We randomly select 50 line drawings and 50 reference images to perform reference-based line drawing colorization using AnimeDiffusion compared with Lee et al.~\cite{lee2020reference}, Li et al.~\cite{li2022eliminating} and Cao et al.\cite{cao2022attention}. Then we conduct a user study to let participants compare colored results of these four methods, participants need subjectively evaluate the colored results according to the reference images and original color images of line drawings to choose the best one from four choices, along with a brief description of why they think this result is the visually best. A user interface of our user study is shown in Fig.~\ref{fig:user_study}.
20 participants take part in the user study and the percentage of each method chosen as the best is shown in Table~\ref{tab:user_study}. It is indicated that AnimeDiffusion has absolute advantages in human visual evaluation. 
\begin{table}[ht]
    \centering
    \caption{User Study Result}
    \label{tab:user_study}
    \begin{tabular}{lc}
    \toprule
        Methods & Percentage of chosen as best\\
    \midrule
        Lee et al.\cite{lee2020reference} & 13.0\% \\
        Li  et al.\cite{li2022eliminating} &17.4\% \\ 
        Cao et al.\cite{cao2022attention} & 21.0\% \\
        AnimeDiffusion  & \textbf{48.6\%} \\
    \bottomrule
    \end{tabular}
\end{table}

\begin{figure}[ht]
\centering  %图片全局居中
\includegraphics[width=\linewidth]{./user_study.pdf}
\caption{A user interface of our user study.}
\label{fig:user_study}
\end{figure}

\section{Application}
\subsection{Famous Anime Character Recolorization}
Although we aim to train AnimeDiffusion to perform single line drawing colorization, however, in animation creation industry, reference-based colorization method can also be used to recolorize a series of images or even consecutive video frames of the same anime character.
Sometimes during the creation process, the same character appears different colors in different images, and using the recolorization technique, it is convenient to unify the same character's colors based on one reference image.
We apply AnimeDiffusion to perform recolorization task. In Fig.~\ref{fig:recolorization}, for each anime character, the top row is original colored image and the bottom row is our recolorized results.
In fact, we first convert original colored images to line drawings using XDoG extractor, then they are combined with the reference image together and are fed into AnimeDiffusion to generate colored results.
As is shown in Fig.~\ref{fig:recolorization}, according to one reference image, the other images of the same character can be recolorized with the same color style and accurate semantic information. 

\subsection{Original Anime Character Colorization}
We collaborate with professional artists and use our AnimeDiffusion to assist their work for original line drawings colorization. We invite professional artist to drawing line drawings attached with one reference color image. Since AnimeDiffusion pre-trained on our collected dataset has learned the semantic information of anime character face, it can generalize well to other hand-drawn characters, as is demontrated in Fig.~\ref{fig:original}. With our developed user interface, artist can easily do batch colorization of the same character according to one reference image. This greatly saves the artists' creation time and helps them to complete the creation more efficiently.
\begin{figure}[htb]
    \centering
    \subfigure{
    \begin{minipage}[c]{0.23\linewidth}
    \stackunder[4pt]{\includegraphics[width=\linewidth]{./ref_1.pdf}}{Anime 1}
    \end{minipage}
    \begin{minipage}[c]{0.71\linewidth}
    \includegraphics[width=0.325\linewidth]{./a1_gt1.pdf}
    \includegraphics[width=0.325\linewidth]{./a1_gt2.pdf}
    \includegraphics[width=0.325\linewidth]{./a1_gt3.pdf}\vspace{4pt}
    \includegraphics[width=0.325\linewidth]{./a1_g1.pdf}
    \includegraphics[width=0.325\linewidth]{./a1_g2.pdf}
    \includegraphics[width=0.325\linewidth]{./a1_g3.pdf}
    \end{minipage}
    % }\vspace{2pt}
    % \subfigure{
    % \begin{minipage}[c]{0.23\linewidth}
    % \stackunder[4pt]{\includegraphics[width=\linewidth]{./ref_2.pdf}}{Anime 2}
    % \end{minipage}
    % \begin{minipage}[c]{0.71\linewidth}
    % \includegraphics[width=0.325\linewidth]{./a2_l1.pdf}
    % \includegraphics[width=0.325\linewidth]{./a2_l2.pdf}
    % \includegraphics[width=0.325\linewidth]{./a2_l3.pdf}\vspace{4pt}
    % \includegraphics[width=0.325\linewidth]{./a2_g1.pdf}
    % \includegraphics[width=0.325\linewidth]{./a2_g2.pdf}
    % \includegraphics[width=0.325\linewidth]{./a2_g3.pdf}
    % \end{minipage}
    }\vspace{2pt}
    \subfigure{
    \begin{minipage}[c]{0.23\linewidth}
    \stackunder[4pt]{\includegraphics[width=\linewidth]{./ref_3.pdf}}{Anime 2}
    \end{minipage}
    \begin{minipage}[c]{0.71\linewidth}
    \includegraphics[width=0.325\linewidth]{./a3_gt1.pdf}
    \includegraphics[width=0.325\linewidth]{./a3_gt2.pdf}
    \includegraphics[width=0.325\linewidth]{./a3_gt3.pdf}\vspace{4pt}
    \includegraphics[width=0.325\linewidth]{./a3_g1.pdf}
    \includegraphics[width=0.325\linewidth]{./a3_g2.pdf}
    \includegraphics[width=0.325\linewidth]{./a3_g3.pdf}
    \end{minipage}
    }
    \caption{Illustration of famous anime character recolorization. Recolorization results have a uniform color style. The two anime characters are Hoshizora Rin and Sonoda Umi of LoveLive.}
    \label{fig:recolorization}
\end{figure}
\begin{figure}[htb]
    \centering
    \subfigure{
    \begin{minipage}[c]{0.23\linewidth}
    \stackunder[4pt]{\includegraphics[width=\linewidth]{./reference.pdf}}{Reference}
    \end{minipage}
    \begin{minipage}[c]{0.71\linewidth}
    \includegraphics[width=0.325\linewidth]{./l1.pdf}
    \includegraphics[width=0.325\linewidth]{./l2.pdf}
    \includegraphics[width=0.325\linewidth]{./l3.pdf}\vspace{4pt}
    \includegraphics[width=0.325\linewidth]{./g1.pdf}
    \includegraphics[width=0.325\linewidth]{./g2.pdf}
    \includegraphics[width=0.325\linewidth]{./g3.pdf}
    \end{minipage}
    }
    \caption{Illustration of original anime character colorization. The anime character with an exaggerated hairstyle is Little Yuyuan, which is created by Ms. Yuwen Wang.}
    \label{fig:original}
\end{figure}


\subsection{Fashion Illustration Sketch Colorization}
We also extend AnimeDiffusion to colorize fashion illustration sketches. Since fashion illustration is the same as animation, it is also the first to outline the line draft, and then fill in the color. We regard fashion illustation as another type of animation. As is shown in Fig.~\ref{fig:fashion}, given one color illustration and one sketch, AnimeDiffusion can generate colorization results which extend the range of accurate semantic correspondence to half of the body. We can not only keep the accurate color of the face, but also have good control over the clothing and torso, even the color of skin can be accurately distinguished.
Fashion designers can use our AnimeDiffusion user interface to easily colorize hand-drawn fashion sketches, which are used for the follow-up process of garment pattern making.

\begin{figure}[htb]
    \centering
    \subfigure[]{
    \begin{minipage}[b]{0.31\linewidth}
        \includegraphics[width=\linewidth]{./ref_109.pdf}\vspace{4pt}
        \includegraphics[width=\linewidth]{./ref_135.pdf}\vspace{4pt}
        \includegraphics[width=\linewidth]{./ref_187.pdf}
    \end{minipage}
    }\hspace{-3pt}
    \subfigure[]{
    \begin{minipage}[b]{0.31\linewidth}
        \includegraphics[width=\linewidth]{./con_131.pdf}\vspace{4pt}
        \includegraphics[width=\linewidth]{./con_020.pdf}\vspace{4pt}
        \includegraphics[width=\linewidth]{./con_191.pdf}
    \end{minipage}
    }\hspace{-3pt}
    \subfigure[]{
    \begin{minipage}[b]{0.31\linewidth}
        \includegraphics[width=\linewidth]{./ret_109_131.pdf}\vspace{4pt}
        \includegraphics[width=\linewidth]{./ret_135_020.pdf}\vspace{4pt}
        \includegraphics[width=\linewidth]{./ret_187_191.pdf}
    \end{minipage}
    }
    \caption{Illustration of fashion illustration sketch colorization. (a) Reference images, (b) Sketches, (c) Colorization results} 
    \label{fig:fashion}
\end{figure}

\section{Generalization, Limitation and Future Work}
The Matcha framework exhibits a high degree of generalizability thanks to the commonsense knowledge inside LLMs.
Without LLMs, a control algorithm, e.g. one trained with reinforcement learning \cite{Li23InternallyRewarded, Singh20COGConnecting}, may require massive datasets/interactions to learn
the common sense \cite{Singh20COGConnecting} of collaborating different modalities, yet being less efficient and generalizable.

However, interpreting the real world with language can be limited to the complexity of the task and the environment dynamics.
For example, advanced reasoning techniques such as decomposing may be required to deal with a complicated task,
where the task is decomposed into several sub-tasks to tackle separately. 
This automatic operation highlights the flexibility of LLMs but also poses challenges to the static language expression of a complex world
--- The vision-to-language module should be called multiple times with flexible queries.
This brings the requirement of vision-enabled LLMs \cite{Zhu23MiniGPT4Enhancing, Brohan23RT2Visionlanguageaction}, 
built on which the reasoning can be malleable. But multimodal LLMs are yet less controllable and accurate in terms of describing the scene
compared with a templated module.

Despite current limitations, multimodal LLMs gain increasing attention due to their great potential and flexibility.
Future work will explore the multimodal models \cite{Tong22VideoMAEMasked, Brohan23RT2Visionlanguageaction} to leverage unified features.

\section{Conclusion And Future Work}
In this paper, we propose AnimeDiffusion, the first diffusion model tailored for anime face line drawing colorization. 
In order to train AnimeDiffusion we build a benchmark dataset for research purpose and also fill the gap of no available high resolution anime face dataset to evaluate line drawing colorization algorithms. 
To handle the high computation consumption problem of diffusion models, we design a novel hybrid training strategy which separates the image denoising task and image reconstruction task. 
Through extensive experiments and a user study, AnimeDiffusion has demonstrated better performance both qualitatively and quantitatively, outperforming other state-of-the-art GANs-based methods, with higher image quality and semantic color information. 
To the best of our knowledge, AnimeDiffusion is the first learning based work can accurately colorize anime face line drawing with heterochromatic pupils according to reference color image, without other special module for processing eyes or pupils in anime face. 

\meng{However, there is a limitation in out method. 
Our model uses paired training data in training, and there is some style correlation between the reference image and the line drawings. 
For special style line drawings, such as Chibi cartoons as shown in Fig \ref{fig:limitation}(c), if the corresponding semantic information does not exist in the reference image, the colorization result of the line drawing may appear to be inconsistent with the real image.}
In the future work, We will work on multi-modal input line drawing colorization such as combining text information and reference image together to make the interactive way of colorization more rich. 
This will greatly reduce the manual tasks of animators and improve the creation efficiency \meng{and colorization effect} of the animation creation industry.

% if have a single appendix:
%\appendix[Proof of the Zonklar Equations]
% or
%\appendix  % for no appendix heading
% do not use \section anymore after \appendix, only \section*
% is possibly needed

% use appendices with more than one appendix
% then use \section to start each appendix
% you must declare a \section before using any
% \subsection or using \label (\appendices by itself
% starts a section numbered zero.)
%


% \appendices
% \section{Proof of the First Zonklar Equation}
% Appendix one text goes here.

% % you can choose not to have a title for an appendix
% % if you want by leaving the argument blank
% \section{}
% Appendix two text goes here.


% use section* for acknowledgment
\ifCLASSOPTIONcompsoc
  % The Computer Society usually uses the plural form
  \section*{Acknowledgments}
  We thank Mr. Henry Tian, Ms. Mandy Wong, Ms. Rachel Liu and Mr. Leonard Chen for their help with anime knowledge and image samples selection. We thank Ms. Xiao Meng and Ms. Yuwen Wang for helping us create wonderful hand-painted line drawings.
\else
  % regular IEEE prefers the singular form
  \section*{Acknowledgment}
\fi



% Can use something like this to put references on a page
% by themselves when using endfloat and the captionsoff option.
\ifCLASSOPTIONcaptionsoff
  \newpage
\fi



% trigger a \newpage just before the given reference
% number - used to balance the columns on the last page
% adjust value as needed - may need to be readjusted if
% the document is modified later
%\IEEEtriggeratref{8}
% The "triggered" command can be changed if desired:
%\IEEEtriggercmd{\enlargethispage{-5in}}

% references section

% can use a bibliography generated by BibTeX as a .bbl file
% BibTeX documentation can be easily obtained at:
% http://mirror.ctan.org/biblio/bibtex/contrib/doc/
% The IEEEtran BibTeX style support page is at:
% http://www.michaelshell.org/tex/ieeetran/bibtex/
%\bibliographystyle{IEEEtran}
% argument is your BibTeX string definitions and bibliography database(s)
%\bibliography{IEEEabrv,../bib/paper}
%
% <OR> manually copy in the resultant .bbl file
% set second argument of \begin to the number of references
% (used to reserve space for the reference number labels box)
%% 下面註釋部分是IEEE官方的參考文獻模板
% \begin{thebibliography}{1}
% \bibitem{IEEEhowto:kopka}
% H.~Kopka and P.~W. Daly, \emph{A Guide to \LaTeX}, 3rd~ed.\hskip 1em plus
%   0.5em minus 0.4em\relax Harlow, England: Addison-Wesley, 1999.
  
% \end{thebibliography}

\bibliographystyle{IEEEtran}
\bibliography{reference}

% biography section
% 
% If you have an EPS/PDF photo (graphicx package needed) extra braces are
% needed around the contents of the optional argument to biography to prevent
% the LaTeX parser from getting confused when it sees the complicated
% \includegraphics command within an optional argument. (You could create
% your own custom macro containing the \includegraphics command to make things
% simpler here.)
%\begin{IEEEbiography}[{\includegraphics[width=1in,height=1.25in,clip,keepaspectratio]{mshell}}]{Michael Shell}
% or if you just want to reserve a space for a photo:
% \begin{IEEEbiography}[{\includegraphics[width=1in,height=1.25in,clip,keepaspectratio]{mshell}}]{Michael Shell}
% \end{IEEEbiography}

% \begin{IEEEbiography}[{\includegraphics[width=1in,height=1.25in,clip,keepaspectratio]{bio_photo/Daniel.jpg}}]{Yu~Cao}
% (Student Member, IEEE)
% received the B.Eng degree in communication engineering from Qingdao Institute of Technology and M.Eng degree in communication and information system from Xidian University.
% He is currently working toward the Ph.D. degree in The Hong Kong Polytechnic University. His research interests include computer vision and computer graphics.
% \end{IEEEbiography}

% \begin{IEEEbiography}
% [{\includegraphics[width=1in,height=1.25in,clip,keepaspectratio]{bio_photo/Meng_Xiangqiao.jpg}}]{Xiangqiao~Meng}
% received his B.Eng and M.S degree from Zhejiang University. He is currently working toward the Ph.D. degree with the Department of Computing, the Hong Kong Polytechnic University. His research interests include computer graphics, machine learning, and computer vision.
% \end{IEEEbiography}

% \begin{IEEEbiography}{P.Y.~Mok}
% Biography text here.
% \end{IEEEbiography}

% \begin{IEEEbiography}{Xueting~Liu}
% received her B.Eng. degree from Tsinghua University and Ph.D. degree from The
% Chinese University of Hong Kong in 2009 and 2014 respectively. She is currently an Assistant Professor in the School of Computing and Information Sciences, Caritas Institute of Higher Education. Her research interests include computer graphics, computer vision, machine learning, computational manga and anime, and nonphotorealistic rendering.
% \end{IEEEbiography}

% \begin{IEEEbiography}{Tong-Yee~Lee}
% (Senior Member, IEEE) received the Ph.D. degree in computer engineering from
% Washington State University, Pullman, in 1995. He is currently a Chair Professor with the Department of Computer Science and Information
% Engineering, National Cheng-Kung University (NCKU), Tainan, Taiwan. 
% He leads the Computer Graphics Group, Visual System Laboratory,
% NCKU \href{http://graphics.csie.ncku.edu.tw}{(http://graphics.csie.ncku.edu.tw)}.
% His current research interests include computer graphics, non-photorealistic rendering, medical visualization, virtual reality, and media resizing. He is a Senior Member of the IEEE and a Member of the ACM. He is an Associate Editor of the IEEE Transactions on Visualization and Computer Graphics.
% \end{IEEEbiography}

% \begin{IEEEbiography}{Ping~Li}
% (Member, IEEE) received the Ph.D. degree in computer science and engineering from The Chinese University of Hong Kong, Hong Kong, in 2013. He is currently an Assistant Professor with the Department of Computing and an Assistant Professor with the School of Design, The Hong Kong Polytechnic University, Hong
% Kong. He has published over 180 top-tier scholarly research articles (e.g., TVCG, TIP, TNNLS, TMI, TMM, TCSVT, TCYB, TBME, TSMC, TII,
% AAAI, CVPR, NeurIPS), pioneered several new research directions, and made a series of landmark contributions in his areas. He has an excellent research project reported by the ACM TechNews, which only reports the top breakthrough news in computer science worldwide. More importantly, however, many of his research outcomes have strong impacts to research fields, addressing societal
% needs and contributed tremendously to the people concerned. His current research interests include image/video stylization, colorization,
% artistic rendering and synthesis, realism in non-photorealistic rendering,
% computational art, and creative media.
% \end{IEEEbiography}

% % if you will not have a photo at all:
% \begin{IEEEbiographynophoto}{John Doe}
% Biography text here.
% \end{IEEEbiographynophoto}

% insert where needed to balance the two columns on the last page with
% biographies
%\newpage

% You can push biographies down or up by placing
% a \vfill before or after them. The appropriate
% use of \vfill depends on what kind of text is
% on the last page and whether or not the columns
% are being equalized.

%\vfill

% Can be used to pull up biographies so that the bottom of the last one
% is flush with the other column.
%\enlargethispage{-5in}



% that's all folks
\end{document}


