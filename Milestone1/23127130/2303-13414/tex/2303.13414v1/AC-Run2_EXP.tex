
\documentclass[ALICE,manyauthors]{cernphprep}

\usepackage[comma,square,numbers,sort&compress]{natbib}
\usepackage{hyperref}
\usepackage[T1]{fontenc}
\usepackage{color}
\usepackage{xspace}
\usepackage{lineno}
\usepackage{enumerate}   
\usepackage{orcidlink} 


%%%%%%%%%%%%%%%%%%% DOCUMENT %%%%%%%%%%%%%%%%%%%%%%%
\begin{document}
\DeclareMathOperator*{\veccat}{%
    \mathchoice%
        {\Bigg\Vert}%
        {\Big\Vert}%
        {\Vert}%
        {\Vert}%
}%


%\linenumbers

%%%%%%%%%%%%%%%  Title page %%%%%%%%%%%%%%%%%%%%%%%%
\begin{titlepage}
% the dates below correspond to CERN approval
% please don't touch: EB chairs will take care
\PHyear{2023}       % required, will be obtained from CERN
\PHnumber{047}      % required, will be obtained from CERN
\PHdate{20 March}  % required, will be obtained from CERN
%%%%%%%%%%%%%%%%%%%%%%%%%%%%%%%%%%%%%%%%%%%%%%%%%%%%

%%% Put your own title + short title here:
\title{Higher-order correlations between different moments of two flow amplitudes in \PbPb collisions at $\mathbf{\sqrt{\textit{s}_{\text{NN}}}=5.02}$~\TeV}
\ShortTitle{Higher-order correlations between different moments...}   % appears on left page headers

%%% Do not change the next lines
\Collaboration{ALICE Collaboration\thanks{See Appendix~\ref{app:collab} for the list of collaboration members}}
\ShortAuthor{ALICE Collaboration} % appears on right page headers, do not change

\begin{abstract}
The correlations between different moments of two flow amplitudes, extracted with the recently developed asymmetric cumulants, are measured in \PbPb collisions at $\snn = 5.02\;\text{TeV}$ recorded by the ALICE detector at the LHC.
The magnitudes of the measured observables show a dependence on the different moments as well as on the collision centrality, indicating the presence of non-linear response in all even moments up to the eighth.
Furthermore, the higher-order asymmetric cumulants show different signatures than the symmetric and lower-order asymmetric cumulants.
Comparisons with state-of-the-art event generators using two different parameterizations obtained from Bayesian optimization show differences between data and simulations in many of the studied observables, indicating a need for further tuning of the models behind those event generators.
These results provide new and independent constraints on the initial conditions and transport properties of the system created in heavy-ion collisions.

\end{abstract}
\end{titlepage}

\setcounter{page}{2} %please do not remove this line

%%%%%%%%%%%%%%%%%%%%%%%%%%%%%%%%
% Begin main text
%%%%%%%%%%%%%%%%%%%%%%%%%%%%%%%%
\section{Introduction}
An important ongoing program in high-energy nuclear physics is the exploration of the phase diagram of quantum chromodynamics (QCD) at large values of temperature and/or density. In that regime, quark--gluon plasma (QGP), an extreme state of nuclear matter in which quarks and gluons are deconfined, is formed. Ultrarelativistic heavy-ion collisions at RHIC and LHC recreate the conditions in which the QGP can be produced and, consequently, studied.
Properties of the QGP are typically portrayed by specifying its transport properties (e.g., shear or bulk viscosity), or its equation of state. To date, the QGP remains the state of matter with the smallest ratio of shear viscosity to entropy density ($\eta/s$) ever discovered.
Besides scrutinizing the details of individual states of nuclear matter in the QCD phase diagram, it is equally important to constrain the nature of the transition between different states, i.e., whether it is a smooth crossover, 1st, or 2nd order phase transition, etc.
Another extensive effort is allocated to the search for the critical point in the QCD phase diagram, whose existence is still not confirmed experimentally. Recent reviews can be found in Refs.~\cite{Heinz:2013th, Braun-Munzinger:2015hba, Busza:2018rrf, ALICE:2022wpn}.

In non-central heavy-ion collisions, the measured correlations in momentum space among detected particles are dominated by correlations originating from collective anisotropic flow~\cite{Ollitrault:1992bk}.
Since anisotropic flow links the initial-state anisotropies in the collision geometry to the final-state anisotropies in momentum space, its measurements can be used to map out the whole history of the heavy-ion evolution. In particular, anisotropic flow is a sensitive probe of shear viscosity of the QGP, whose collective expansion is successfully described by its hydrodynamic response to the anisotropies in the initial-state geometry.
By using a Fourier series decomposition of the azimuthal distribution $f(\varphi)$ of reconstructed particles, anisotropic flow is quantified with flow magnitudes $v_n$ and symmetry planes $\Psi_n$~\cite{Voloshin:1994mz}
%
\begin{equation}
f(\varphi)=\frac{1}{2\pi}\left[1+2\sum_{n=1}^{\infty}v_n\cos[n(\varphi-\Psi_n)]\right]\,.
\label{eq:FourierSeries}
\end{equation}
%
Due to random fluctuations of the impact parameter vector (the vector connecting the centers of the two colliding ions) from one collision to another, this series cannot be obtained directly from the data using the simple relation $v_n e^{in\Psi_n} = \left<e^{in\varphi}\right>$.
Instead, multiparticle azimuthal correlations, which have additional advantages and are not influenced by these fluctuations for suitably chosen harmonics $n_i$, are utilized as experimental estimators according to the following analytic expression~\cite{Bhalerao:2011yg}
%
\begin{equation}
\left<e^{i(n_1\varphi_1+\cdots+n_k\varphi_k)}\right> = v_{n_1}\cdots v_{n_k}\,e^{i(n_1\Psi_{n_1}+\cdots+n_k\Psi_{n_k})} \,.
\label{eq:generalResult}
\end{equation}
%
In the above equation, angular brackets indicate an average over all distinct sets of $k$ azimuthal angles of the tracks obtained from the same event. Further technical details about multiparticle azimuthal correlations can be found in Refs.~\cite{Borghini:2000sa, Borghini:2001vi, Bilandzic:2010jr, Bilandzic:2013kga}.

With the multiparticle correlation techniques, the correlations between different flow amplitudes $v_n$ or symmetry planes $\Psi_n$ can be used to extract independent information about all stages in the heavy-ion evolution, when compared to the traditional measurements of individual flow amplitudes $v_n$~\cite{Parkkila:2021yha}.
Among them, the \textit{symmetric cumulants} (SC) were designed to study the genuine correlations (or cumulants) between the same lowest-order moments of two different flow amplitudes $v_m^2$ and $v_n^2$. 
The simplest realization of SC observables in terms of flow amplitudes and the corresponding experimental estimators built from multiparticle azimuthal correlations is given by~\cite{Bilandzic:2013kga, ALICE:2016kpq, ALICE:2017kwu}
%
\begin{equation}
\begin{split}
    \SC(m,n) & \equiv \langle v_m^2 v_n^2 \rangle - \langle v_m^2 \rangle\langle v_n^2 \rangle \\
            & = \langle\langle e^{i(m\varphi_1 + n\varphi_2 - m\varphi_3 - n\varphi_4)} \rangle\rangle - \langle\langle e^{im(\varphi_1 - \varphi_2)} \rangle\rangle\langle\langle e^{in(\varphi_1 - \varphi_2)} \rangle\rangle.
\label{eq:SC2h}
\end{split}
\end{equation}
%
In the first line above, $\langle\cdot\rangle$ indicates an average over all events. Meanwhile, $\langle\langle\cdot\rangle\rangle$ in the second line indicates that the averaging is performed in two steps --- first, over all distinct particle pairs or quadruplets in an event, and then in the second step, the all-event averages are obtained from these single-event averages by weighting them with a `number of combinations’ multiplicity weight $M(M-1)$ and $M(M-1)(M-2)(M-3)$ for 2- and 4-particle correlations, respectively~\cite{Bilandzic:2010jr}.
%
Furthermore, defined this way, SC observables are multivariate cumulants of flow amplitudes squared, and therefore they obey all non-trivial mathematical properties of multivariate cumulants for stochastic variables $v_n^2$ by definition~\cite{doi:10.1143/JPSJ.17.1100}. 
The azimuthal correlators are then used only via Eq.~\eqref{eq:generalResult} to estimate each individual term in the cumulant expansion~\cite{Bilandzic:2021rgb}, i.e., the cumulant expansion is not performed directly on the azimuthal angles.

The particular importance of SC in flow studies is that they bring new information on QGP properties, as it was shown in Ref.~\cite{ALICE:2016kpq}.
From experimental measurements of various combinations of $\SC(m,n)$ and their comparison to theoretical predictions, it was concluded that SC are sensitive to the temperature dependence of $\eta/s$. On the other hand, the analyses of single flow amplitudes can only probe the average value $\langle\eta/s(T)\rangle$~\cite{ALICE:2016kpq, ALICE:2017kwu, ALICE:2021adw}.
%
Motivated by this access to new information on the system properties, studies were conducted to generalize the SC from two to any number of harmonics using the cumulant formalism~\cite{Mordasini:2019hut}. The resulting three-harmonic SC were recently measured by ALICE in \PbPb~collisions recorded in 2010~\cite{ALICE:2021klf} and 2015~\cite{ALICE:2021adw}.

In parallel to the measurements of new flow observables, techniques like Bayesian optimization, where the model parameters are systematically varied to find the best fit to all selected experimental measurements, were applied.
In recent years, multiple Bayesian studies were conducted to constrain the QGP transport properties and details of the initial conditions in \PbPb~collisions~\cite{Bernhard:2015hxa, Bernhard:2016tnd, Bernhard:2019bmu, Auvinen:2020mpc, Nijs:2020ors, Nijs:2020roc, JETSCAPE:2020mzn, Parkkila:2021tqq, Parkkila:2021yha}. While much information on \PbPb~collisions was obtained through Bayesian analyses, the uncertainties on the extracted parameters are still quite large.
%
As demonstrated in recent studies~\cite{Parkkila:2021tqq, Parkkila:2021yha}, this problem can partially be solved by increasing the scope of the input experimental data, either from various beam-energy ranges or from additional independent observables sensitive to the initial conditions and system properties. 
In recent publications~\cite{Parkkila:2021tqq, Parkkila:2021yha}, dedicated sensitivity studies were also performed on sophisticated higher-harmonic observables for the first time, and it was concluded that the higher-order flow observables exhibit the highest sensitivity to the QGP transport properties.

The success of the SC observables in providing crucial input to Bayesian analyses has prompted a natural quest towards further generalizing the SC to a new and independent set of multiharmonic observables involving different moments of the flow amplitudes without any influence from the symmetry planes.
One example of such estimators are the \textit{asymmetric cumulants} (AC)~\cite{Bilandzic:2021rgb}. 
As they probe the genuine correlations between different moments of the flow amplitudes, they can, by definition, extract new information (about event-by-event flow fluctuations, non-linear response, etc.) which is inaccessible to the SC themselves.

The structure of this article is as follows. Section~\ref{sec:theory} reviews the theoretical definitions of AC and the corresponding experimental estimators. The technical details of the data analysis are introduced in Sec.~\ref{sec:data}. In Sec.~\ref{sec:results}, the first experimental measurements of the AC and their comparisons with state-of-the-art model predictions are presented. The article is summarized in Sec.~\ref{sec:summary}, while additional support material can be found in the Appendices.

%------------------------------%
\section{Definitions and scope}
\label{sec:theory}
This article presents the experimental measurements of the AC observables in their simplest realization which involves only two flow amplitudes, evaluated for three different pairs of harmonics chosen to be the same as the ones studied in Ref.~\cite{Parkkila:2021yha}.
For a given pair of different harmonics $(m,n)$, measuring the genuine correlations between $v_m^{2a}$ and $v_n^{2b}$ with AC observables (where $a$ and $b$ are non-equal positive integers) leads to independent information, due to the uniqueness of each multivariate cumulant. Therefore, this analysis focuses on the following new flow observables: $\AC_{2,1}(m,n)$, $\AC_{3,1}(m,n)$, and $\AC_{4,1}(m,n)$, to study correlations of the 2nd, 3rd, and 4th moment of $v_m^2$ with the 1st moment of $v_n^2$.
%
Two remarks have to be made: the lowest-order AC, $\AC_{1,1}(m,n)$, corresponds to the well-known $\SC(m,n)$ (Eq.~\eqref{eq:SC2h}), showing that AC are a natural generalization of SC.
Furthermore, AC observables are invariant under a simultaneous permutation of the indices and the harmonics, e.g., ${\rm AC}_{a,b}(m,n) = {\rm AC}_{b,a}(n,m)$.
These properties will be exploited in Sec.~\ref{sec:results} to discuss the comparisons between $\AC_{a,b}(m,n)$ and $\AC_{a,b}(n,m)$.

The expressions for the AC in terms of flow amplitudes are derived in Ref.~\cite{Bilandzic:2021rgb} and are recalled as
%
\begin{equation} \label{eq:AC21-mn}
    \AC_{2,1}(m,n) \equiv \langle v_m^4 v_n^2 \rangle - \langle v_m^4 \rangle\langle v_n^2 \rangle - 2 \langle v_m^2 v_n^2 \rangle\langle v_m^2 \rangle + 2 \langle v_m^2 \rangle^2 \langle v_n^2 \rangle,
\end{equation}
%
\begin{equation} \label{eq:AC31-mn}
\begin{split}
    \AC_{3,1}(m,n) & \equiv \langle v_m^6 v_n^2 \rangle - \langle v_m^6 \rangle\langle v_n^2 \rangle - 3 \langle v_m^2 v_n^2 \rangle\langle v_m^4 \rangle - 3 \langle v_m^4 v_n^2 \rangle\langle v_m^2 \rangle\\
    & \quad + 6 \langle v_m^4 \rangle\langle v_m^2 \rangle\langle v_n^2 \rangle + 6 \langle v_m^2 v_n^2 \rangle\langle v_m^2 \rangle^2 - 6 \langle v_m^2 \rangle^3 \langle v_n^2 \rangle,
\end{split}
\end{equation}
%
\begin{equation} \label{eq:AC41-mn}
\begin{split}
    \AC_{4,1}(m,n) & \equiv \langle v_m^8 v_n^2 \rangle - \langle v_m^8 \rangle\langle v_n^2 \rangle - 4 \langle v_m^2 v_n^2 \rangle\langle v_m^6 \rangle - 6 \langle v_m^4 v_n^2 \rangle\langle v_m^4 \rangle + 6 \langle v_m^4 \rangle^2 \langle v_n^2 \rangle - 4 \langle v_m^6 v_n^2 \rangle\langle v_m^2 \rangle\\
    & \quad + 8 \langle v_m^6 \rangle\langle v_m^2 \rangle\langle v_n^2 \rangle + 24 \langle v_m^2 v_n^2 \rangle\langle v_m^4 \rangle\langle v_m^2 \rangle + 12 \langle v_m^4 v_n^2 \rangle\langle v_m^2 \rangle^2 - 36 \langle v_m^4 \rangle\langle v_m^2 \rangle^2 \langle v_n^2 \rangle\\
    & \quad - 24 \langle v_m^2 v_n^2 \rangle\langle v_m^2 \rangle^3 + 24 \langle v_m^2 \rangle^4 \langle v_n^2 \rangle.
\end{split}
\end{equation}
%
Due to their cumulant nature, much information can already be extracted from the SC and AC themselves.
One example is the sign of the measured observables. In Ref.~\cite{ALICE:2016kpq}, it was shown how the event-by-event flow fluctuation patterns and the signatures of SC observables are related. This idea was also applied in the interpretation of the three-harmonic SC presented in Ref.~\cite{ALICE:2021klf}.
For AC, however, the defining equations are more involved and the direct physical interpretation of their signature, positive or negative, is not straightforward, even in the simplest case of ${\rm AC}_{2,1}$. Another example is the observation of zero magnitude, indicating either an absence of genuine correlations between all involved variables or a trivial consequence of underlying symmetries.
In the case of cumulants, however, this has to be interpreted carefully because in the absence of underlying symmetries either all cumulants at given order are identically zero, or none of them~\cite{doi:10.1143/JPSJ.17.1100}. Therefore, the nonvanishing results for $\SC(2,4)$, obtained previously in Refs.~\cite{ALICE:2016kpq,ALICE:2017kwu}, imply that all of $\AC_{a,b}(2,4)$ must be nonvanishing as well, and the fact that some of them are consistent with zero would indicate only the limited sensitivity in the available datasets.
In addition, a multivariate cumulant is nonvanishing if and only if all variables in its definition are correlated to each other~\cite{doi:10.1143/JPSJ.17.1100}.

As done for the SC, the AC can also be normalized to remove the sensitivity of the observable to the strength of the individual harmonics and extract only the contribution of the genuine correlations between the various moments of $v_m^2$ and $v_n^2$. This is done according to the standard procedure from Refs.~\cite{Taghavi:2020gcy, Bilandzic:2021rgb}, using
%
\begin{equation}
    \NAC_{a,1}(m,n) \equiv \frac{\AC_{a,1}(m,n)}{\langle v_m^2 \rangle^a \langle v_n^2 \rangle}, \quad a = 2,\ldots,4.
\label{eq:NAC_mn}
\end{equation}

Equations~\eqref{eq:AC21-mn} to~\eqref{eq:AC41-mn} cannot be used experimentally, as only the azimuthal angles of the final-state particles can be measured. Reliable experimental estimators for the various AC were presented in Appendix~C.2. of Ref.~\cite{Bilandzic:2021rgb}. As an example, the expression corresponding to Eq.~\eqref{eq:AC21-mn} is
%
\begin{equation}
\begin{split}
   \AC_{2,1}(m,n) & = \langle\langle {\rm e}^{i(m\varphi_1 + m\varphi_2 + n\varphi_3 - m\varphi_4 - m\varphi_5 - n\varphi_6)} \rangle\rangle\\
    & \quad - \langle\langle {\rm e}^{im(\varphi_1 + \varphi_2 - \varphi_3 -\varphi_4)} \rangle\rangle\langle\langle {\rm e}^{in(\varphi_1 - \varphi_2)} \rangle\rangle \\
    & \quad - 2\langle\langle {\rm e}^{i(m\varphi_1 + n\varphi_2 - m\varphi_3 - n\varphi_4)} \rangle\rangle \langle\langle {\rm e}^{im(\varphi_1 - \varphi_2)} \rangle\rangle \\
    & \quad + 2 \langle\langle {\rm e}^{im(\varphi_1 - \varphi_2)} \rangle\rangle^2 \langle\langle {\rm e}^{in(\varphi_1 - \varphi_2)} \rangle\rangle.
\end{split}
\end{equation}
%
Each of the azimuthal correlators in the above expression is isotropic (i.e., invariant under the transformation $\varphi\rightarrow\varphi+\alpha$ where $\alpha$ is an arbitrary angle), which ensures that AC are not affected by random event-by-event fluctuations of the impact parameter vector. In addition, in each azimuthal correlator, each harmonic appears an equal number of times with positive and negative signs, which ensures that any dependence on symmetry planes is canceled out exactly (see Eq.~\eqref{eq:generalResult}).

%------------------------------%
\section{Data analysis}
\label{sec:data}
The analysis is performed with data from \PbPb collisions at the center-of-mass energy per nucleon pair $\snn = 5.02$~TeV recorded by the ALICE detector in 2015 during the LHC Run 2 period. 

A detailed description of the ALICE detector and its subdetectors can be found in Refs.~\cite{ALICE:2008ngc, ALICE:2014sbx}.
The Time Projection Chamber (TPC)~\cite{Alme:2010ke} was used to reconstruct the charged-particle tracks and measure their momenta.
Two scintillator arrays, V0A and V0C~\cite{ALICE:2008ngc, ALICE:2013axi}, were used for triggering and the determination of the centrality of the collisions~\cite{ALICE:2015juo}. The TPC, V0A, and V0C cover the full azimuthal range within the pseudorapidity ranges $|\eta|<0.9$,\space \space $2.8 < \eta < 5.1$, and $-3.7 < \eta < -1.7$, respectively. %ALICE:2010mlf}.

The detector closest to the beam pipe is the Inner Tracking System (ITS)~\cite{CERN-LHCC-99-012, ALICE:2010tia}, which, besides the TPC, is also used for track reconstruction. It covers the full range in azimuthal angle and consists of six silicon layers, with three different detector technologies.
The two innermost layers, the Silicon Pixel Detectors (SPD), have a high spatial granularity making them ideal for reconstructing both primary and secondary vertices.

Minimum bias (MB) events are triggered by a coincident signal in both the V0A and V0C. The selected MB \PbPb~events all have a reconstructed primary vertex within $\pm 8.0$ cm from the nominal interaction point along the beam direction. 
Background events like beam--gas collisions and pileup were removed using the correlation between the information from the V0 detector and the SPD, as well as the correlation between the number of reconstructed tracks and the information from the ITS and TPC.
A further reduction of the pileup events was obtained using the correlation between the information provided by the SPD and by the Time-Of-Flight detector~\cite{Carnesecchi:2018oss}.
%
Furthermore, the data taken with each of the two possible polarizations of the magnetic field in the central barrel are both considered for the analysis.
Finally, an additional event criterion is applied after the track selection to reject all remaining events with less than 10 reconstructed tracks. This requirement is motivated by the calculation of the multiparticle azimuthal correlators themselves, and their corresponding event-by-event multiplicity weights. For a $k$-particle correlator, at least $k$ tracks are needed to have a valid expression for the correlator and corresponding multiplicity weight.
This criterion is set to 10 tracks, as this is the minimum number of tracks needed to calculate the largest correlator in this analysis, $\AC_{4,1}(m,n)$ (see Appendix~C in Ref.~\cite{Bilandzic:2021rgb} for the full estimator).
After all event selections are applied, approximately 50 million events in the centrality range of $5\%$ to $60\%$ are used in the analysis.
%
Furthermore, this range is divided into the following centrality classes: the two central classes 5--10\%, 10--20\% and the four semicentral classes 20--30\%, 30--40\%, 40--50\% and 50--60\%.

Tracks are reconstructed using the combined information given by the ITS and the TPC.
These tracks are selected to be within $|\eta|<0.8$ and $0.2<\pt<5$ GeV/$c$, where \pt~is the transverse momentum. The low $\pt$ cutoff is implemented for reducing biases due to a smaller reconstruction efficiency at very low transverse momentum, while the high $\pt$ cutoff is chosen for suppressing the contribution to the anisotropies in the measured azimuthal correlations from jets.
%
In order to avoid contributions from secondary particles, each track is required to have a distance of closest approach (DCA) to the primary vertex of less than 2 cm in the longitudinal direction and less than $0.0105+0.0350/\pt^{1.1}$ cm in the transverse direction, meaning a value ranging from 0.016~cm at \pt $=$ 5.0~GeV/$c$ to 0.22 cm at \pt~$=$~0.2~GeV/$c$.
Additionally, each track is required to have a minimum of 70 TPC space points out of the maximum of 159, and a $\chi^2$ value per space point from the track fit to be within the range of 0.1 to 2.5. The tracks are required to have a minimum of 2 hits in the ITS.
Requirements were also added to reject tracks with a kink topology (i.e., tracks that appear to change direction due to multiple scatterings and/or weak decays).

Corrections for both non-uniform reconstruction efficiency (NUE) as a function of transverse momentum and non-uniform acceptance (NUA) as a function of azimuthal angle are applied in the same way as in Refs.~\cite{Bilandzic:2013kga, ALICE:2020sup, ALICE:2021klf}. 
For the NUE corrections, a HIJING (Heavy-Ion Jet INteraction Generator)~\cite{Gyulassy:1994ew} simulation with a GEANT3~\cite{Brun:1994aa} detector model is used to correct for the effects of track reconstruction efficiency and contamination from secondary particles by constructing a \pt-dependent track weighting correction. 
The NUA corrections are obtained in a data-driven manner, based on the method presented in Ref.~\cite{Bilandzic:2013kga}.
%
Finally, the nonflow contribution from the 2-particle correlations appearing in the denominator of the NAC is suppressed with the use of a pseudorapidity gap $\vert \Delta\eta \vert > 1$. As the AC themselves were shown to be robust against nonflow~\cite{Bilandzic:2021rgb}, no gap is used in their computation.

Obtaining the statistical uncertainty through standard error propagation is impractical due to both non-trivial functional dependence and many covariance terms. Therefore, in this analysis, the statistical uncertainties were determined using the standard subsampling procedure (see Appendix~F in Ref.~\cite{Bilandzic:2021rgb}).

The systematic uncertainties are estimated by independent variations of each selection criterion. The significance of each variation is determined according to the Barlow test~\cite{Barlow:2002yb}.
For each variation, the results for $\AC_{a,b}(m,n)$ are compared with the ones obtained with the default selection, taking into account the correlations between their statistical uncertainties.
If the deviation is larger than one $\sigma$, where $\sigma$ is the uncertainty on the difference, the variation is deemed significant according to the Barlow test and is included into the quadratic sum to get the total systematic uncertainties.
Each selection criterion variation is described below and the trend of their relative deviations with respect to the default values observed in semicentral collisions (collisions in the centrality range 20--60\%) is reported in parentheses. Due to the small size of the signal, the variations measured in central events (centrality up to $20$\%) and for higher-order observables can fluctuate more significantly than the ones indicated here.

The effects of the centrality determination are estimated by changing the default V0 estimator to the SPD ($20$\%). The quality of the event selection is further tested by varying the longitudinal position of the primary vertex from $\pm$~8~cm to $\pm$~6~cm ($10$\%), and by applying a tighter pileup rejection factor ($4$\%).
Furthermore, the impact of the two configurations of the magnetic field polarity in the solenoid magnet of ALICE is investigated by performing the same analysis on the data sets taken for each orientation separately ($5$\%).
%
The minimum number of space points in the TPC required for the track reconstruction is increased from 70 to 80 ($2$\%), and the quality of the reconstruction fit per TPC cluster is reduced from $0.1 < \chi^2 < 2.5$ to $0.1 < \chi^2 < 2.3$ ($4$\%).
The longitudinal DCA is also tightened from 2 cm to 1 cm. It has to be noticed that this systematic variation mainly affects the higher-order combinations (3--60\%).
Finally, a change of the default tracking requirements to the so-called hybrid tracks, which are TPC--ITS tracks with looser DCA requirements in the transverse plane and along the beam axis, leads to an effect on all studied observables (0--30\%).

%------------------------------%
\section{Results and discussion}
\label{sec:results}
The results obtained in this analysis are first presented together, in order to gain insights on the strength of the different multiharmonic correlations and their dependence on centrality and cumulant order. Then, specific comparisons between the data and the model predictions are presented for a few selected AC and NAC.

The centrality dependence of the thirty-four different $\AC_{a,b}(m,n)$ and $\NAC_{a,b}(m,n)$ is shown in Fig.~\ref{fig:ALICEonly}. However, due to their large uncertainties, $\NAC_{4,1}(m,n)$ and $\NAC_{1,4}(m,n)$ are not presented here.
%
\begin{figure}[t]
    \centering
	\includegraphics[width=0.75\linewidth]{Figures/allAC_dataOnly_v20230127.pdf}
    \caption{Centrality dependence of the different orders of $\AC_{a,b}(m,n)$ (left) and $\NAC_{a,b}(m,n)$ (right) for four pairs of indices ($a,b$) and three pairs of harmonics ($m,n$) in \PbPb~collisions at \snn~=~5.02~TeV. The mirror combinations of $\AC_{a,b}(m,n)$ (closed markers), i.e., $\AC_{b,a}(m,n)$, are indicated with open markers and similar color scheme. The statistical (systematic) uncertainties are shown with lines (boxes). The data points are shifted horizontally for visibility.}
	\label{fig:ALICEonly}
\end{figure}
%
As the results for $\AC_{a,b}(2,3)$ and $\AC_{a,b}(2,4)$ (Fig.~\ref{fig:ALICEonly}~(a) and~(c), respectively) present similar behaviors, albeit with different signs, they are here discussed together.
%
A dependence of the magnitude of the genuine correlations on the cumulant order is observed. The results for $\SC(m,n) \equiv \AC_{1,1}(m,n)$ present the largest signals, followed by $\AC_{2,1}(m,n)$, while $\AC_{3,1}(m,n)$ and $\AC_{4,1}(m,n)$ are several orders of magnitude smaller.
This decreasing effect is more pronounced for the mirror combinations $\AC_{b,a}(m,n)$ when compared to their respective counterparts.
%
The splitting between the results for $\AC_{a,b}(m,n)$ and its mirror $\AC_{b,a}(m,n)$ is also present for $\NAC_{a,b}(2,3)$ and $\NAC_{a,b}(2,4)$ (Fig.~\ref{fig:ALICEonly}~(b) and~(d), respectively).
While the results for $\NAC_{2,1}(2,3)$ and $\NAC_{3,1}(2,3)$ demonstrate a trend similar as $\NSC(2,3)$, albeit with smaller magnitudes, the ones for $\NAC_{2,1}(2,4)$ and $\NAC_{3,1}(2,4)$ are in good agreement with $\NSC(2,4)$  and deviate only in more peripheral collisions.
In both panels, the mirror combinations (open markers) show an increase of the magnitude with the cumulant order.
%
Finally, Fig.~\ref{fig:ALICEonly}~(e) and~(f) present the centrality dependence of the different orders of $\AC_{a,b}(3,4)$ and $\NAC_{a,b}(3,4)$, respectively. The results for $\AC_{a,b}(3,4)$ present the same ordering of the magnitudes with the order of the AC itself, as observed in the two other pairs of harmonics.
However, no splitting between $\AC_{a,1}(3,4)$ and $\AC_{1,a}(3,4)$ can be seen at any order $a = 1,\ldots,4$ within uncertainties.

The measurements are compared with results from simulated events generated with the state-of-the-art chain of models \trento+VISH(2+1)+UrQMD~\cite{Shen:2014vra, Song:2007ux, Bass:1998ca, Bleicher:1999xi}, which will be referred to here as \trento+iEBE-VISHNU.
This model chain consists of the \trento~model~\cite{Moreland:2014oya} for the initial conditions, which are connected by free streaming to a 2+1-dimensional hydrodynamical model VISH(2+1)~\cite{Shen:2014vra}. The description of the heavy-ion evolution after the collision is continued after particlization with a hadronic cascade (UrQMD) model~\cite{Bass:1998ca,Bleicher:1999xi}.
The initial conditions, the specific shear viscosity $\eta/s(T)$, the specific bulk viscosity $\zeta/s(T)$, and other free parameters of this hybrid model, like, for instance, the Gaussian nucleon width and the minimum inter-nucleon distance, are obtained from a global Bayesian analysis. The data-to-model comparisons are performed with two sets of best-fit parameterizations given by maximum a posteriori (MAP) from two independent Bayesian analysis groups.
As the values for these two sets of best-fit parameters agree within their respective uncertainties from the Bayesian analysis, this work allows the observation of the deviations between the two parameterizations in the predictions of observables that were not used in the Bayesian optimization.
The extraction of the Duke 2019 parameterization~\cite{Bernhard:2019bmu} used a series of ALICE measurements (two-particle correlations, charged-particle multiplicities, etc.) as inputs into the Bayesian analysis. However, most of the used data are from \PbPb~collisions at $\snn = 2.76$~TeV and only a subset of available flow observables were used. 
The Jyväskylä 2022 parameterization~\cite{Parkkila:2021yha} utilized for the first time higher-order flow observables and data from \PbPb~collisions at both $\snn = 2.76$~TeV and $\snn = 5.02$~TeV.
New observables, such as SC~\cite{ALICE:2016kpq, ALICE:2017kwu, ALICE:2021adw, ALICE:2021klf} and flow harmonic mode couplings~\cite{ALICE:2020sup}, recently measured by the ALICE Collaboration, are more sensitive to the hydrodynamic transport coefficients $\eta/s$ and $\zeta/s$. This resulted in a reduction of the uncertainties of the extracted transport coefficients~\cite{Parkkila:2021yha}.
In these hydrodynamical model calculations, all the kinematic selections, such as \pt~and $\eta$ ranges, are matched with the ones reported in this article. In addition, the observables are extracted from the simulations using the same experimental analysis techniques.

Figure~\ref{fig:NAC23_withModel} shows the comparisons of the experimental data for $\NAC_{a,b}(2,3)$ with the predictions from hydrodynamic simulation using the Duke 2019 and Jyväskylä 2022 parameterizations.
%
\begin{figure}[b]
	\centering
	\includegraphics[width=0.75\linewidth]{Figures/NAC23_withModel_v20230127.pdf}
	\caption{Centrality dependence of the different orders $a = 1,\ldots,3$ of $\NAC_{a,1}(2,3)$ (cyan closed circles) and $\NAC_{1,a}(2,3)$ (pink open squares) compared with the theoretical predictions from \trento+iEBE-VISHNU for the Duke 2019~\cite{Bernhard:2019bmu} and Jyväskylä 2022~\cite{Parkkila:2021yha} parameterizations. Panel~(d) shows a close-up of the results of $\NAC_{3,1}(2,3)$. The statistical uncertainties in the calculations are indicated by the thickness of the colored bands. The data points for $\NSC(2,3)$ are taken from Ref.~\cite{ALICE:2021adw}.}
	\label{fig:NAC23_withModel}
\end{figure}
%
As discussed above for the data, the same splitting between $\NAC_{a,1}(2,3)$ and $\NAC_{1,a}(2,3)$ ($a = 2,3)$ can be observed in both model calculations.
The negative sign is visible both for $\NSC(2,3)$ and for some AC observables in the same pair of flow harmonics. The notable exception here is $\NAC_{3,1}(2,3)$, which is close to zero for the data and predicted to be positive in semicentral collisions by both model parameterizations.
%
Furthermore, no major difference can be seen between the predictions from Duke 2019 and Jyväskylä 2022 for any of the observables in Fig.~\ref{fig:NAC23_withModel}.
As the linear response $v_n = k_n\varepsilon_n\ (n = 2,3)$, where $k_n$ are the linear response coefficients and $\epsilon_n$ the eccentricities in the initial state, dominates in central collisions, the genuine correlations between $v_2$ and $v_3$ are expected to be sensitive only to the initial conditions in this regime. Therefore, a possible explanation of the similar predictions is the use of \trento~for both parameterizations.
It also has to be noted that both calculations reproduce the experimental data only qualitatively, and thus, the use of such new measurements as input in future Bayesian studies would allow further constraints on the descriptions of the early stages of heavy-ion collisions.

The comparison of the results for $\AC_{a,b}(2,4)$ with the calculations from both model parameterizations is shown in Fig.~\ref{fig:AC24_withModel}.
%
\begin{figure}[b]
	\centering
	\includegraphics[width=0.75\linewidth]{Figures/AC24_withModel_v20230127.pdf}
	\caption{Centrality dependence of the different orders $a = 1,\ldots,4$ of $\AC_{a,1}(2,4)$ (cyan closed circles) and $\AC_{1,a}(2,4)$ (pink open squares) compared with the theoretical predictions from \trento+iEBE-VISHNU for the Duke 2019~\cite{Bernhard:2019bmu} and Jyväskylä 2022~\cite{Parkkila:2021yha} parameterizations. Panels (e) and (f) show a close-up of the results of $\AC_{1,3}(2,4)$ and $\AC_{1,4}(2,4)$, respectively. The statistical uncertainties in the calculations are indicated by the thickness of the colored bands.}
	\label{fig:AC24_withModel}
\end{figure}
%
Overall, the behavior of the experimental data is better reproduced by the Jyväskylä 2022 parameterization, quantitatively for SC and qualitatively for the higher orders. The model predictions are also closer to the experimental values for $\AC_{1,a}(2,4)$ than for $\AC_{a,1}(2,4)$, for $a = 2,3,4$.
%
The observable $\AC_{a,1}(2,4)$ presents the same positive sign as $\SC(2,4)$. This is the case as well for $\AC_{1,2}(2,4)$ and $\AC_{1,3}(2,4)$ in semicentral collisions (see Fig.~\ref{fig:AC24_withModel}~(e) for a close-up of the latter observable).
However, both experimental and theoretical results for $\AC_{1,4}(2,4)$ are in agreement with zero within uncertainties up to centralities of 30\% for the data and 45\% for the model (Fig.~\ref{fig:AC24_withModel}~(f)).
Because of previous nonvanishing results for $\SC(2,4)$, this indicates that the used sample of Run 2 data is not sufficient to extract nonvanishing values for these particular AC observables.
%
The current study is expected to provide additional input on the higher-order terms of the non-linear response between the initial-state ellipticity term $\epsilon_2^2$ and the fourth-order flow amplitude, $v_4$, already studied in Ref.~\cite{Gardim:2011xv}.
As it was shown in recent Bayesian estimations~\cite{Parkkila:2021yha}, higher-order cumulants of flow amplitudes exhibit better sensitivity to QGP properties when compared to the traditional lower-order flow observables. The remaining tensions between data and predictions observed in $\AC_{1,a}(2,4)$ are thus a sign that the measurements of the genuine correlations between $v_2^{2}$ and $v_4^{2a}$ can bring further constraints on the non-linear response of $v_4$ and the hydrodynamical descriptions in the model.

The experimental results for $\NAC_{a,b}(3,4)$ are compared with the theoretical calculations from the two model parameterizations, as shown in Fig.~\ref{fig:NAC34_withModel}.
%
\begin{figure}[b]
	\centering
	\includegraphics[width=0.75\linewidth]{Figures/NAC34_withModel_v20230127.pdf}
	\caption{Centrality dependence of the different orders $a = 1,\ldots,3$ of $\NAC_{a,1}(3,4)$ (cyan closed circles) and $\NAC_{1,a}(3,4)$ (pink open squares) compared with the theoretical predictions from \trento+iEBE-VISHNU for the Duke 2019~\cite{Bernhard:2019bmu} and Jyväskylä 2022~\cite{Parkkila:2021yha} parameterizations. The statistical uncertainties in the calculations are indicated by the thickness of the colored bands. The data points for $\NSC(3,4)$ are taken from Ref.~\cite{ALICE:2021adw}.}
	\label{fig:NAC34_withModel}
\end{figure}
%
Similarly, as for the data, the predictions for $\NAC_{2,1}(3,4)$ and $\NAC_{1,2}(3,4)$ are in good agreement with each other within uncertainties (Fig.~\ref{fig:NAC34_withModel}~(b)), for both the Duke 2019 and Jyväskylä 2022 parameterizations.
%
Additionally, previous studies~\cite{Gardim:2011xv} have shown that the measurements of $v_4$ can be explained by a contribution from $\epsilon_2^2$ and none from $\epsilon_3$. The observed agreement may then originate from the interplay between the different stages of the evolution of the heavy-ion collisions.
For instance, an agreement between the initial-state predictions for $\NAC_{2,1}(3,4)$ and $\NAC_{1,2}(3,4)$ could indicate an absence of non-linear response between the higher orders of $\epsilon_3$ and $v_4$.
Similarly, different initial-state predictions could mean that the hydrodynamic evolution differently impacts the correlations between the higher moments of $\epsilon_3$ with $v_4$ and the ones between $\epsilon_3$ and the higher moments of $v_4$. However, these non-linear hydrodynamic effects would be such that the two observables converge onto each other in the final state.
%
Future studies of both the initial and final state, for instance, comparisons of the AC results from this study with initial-state predictions, would help in understanding the origin of the different behaviors --- splitting or no splitting --- observed in the various combinations of NAC.
%
Furthermore, it is noted that there is good agreement between the predictions obtained from the Duke 2019 and Jyväskylä 2022 parameterizations, and that they both capture quantitatively the experimental data for centralities higher than 15\%. However, none of the parameterizations capture the sign change between central and semicentral events observed in the experimental data.
In Ref.~\cite{ALICE:2021adw}, this effect was already observed for $\NSC(3,4)$, where the sign change could be reproduced by models using AMPT initial conditions but not by the ones using \trento.
It is therefore expected that future precision measurements of $\NAC_{a,b}(3,4)$ and their use as input data in Bayesian studies would help in tuning the different details of system evolution as a function of the collision centrality in the models.

Figure~\ref{fig:NAC_selected} summarizes the trends and ordering obtained for $\NAC_{a,b}(2,3)$ and $\NAC_{a,b}(2,4)$.
%
\begin{figure}[b]
	\centering
	\includegraphics[width=0.75\linewidth]{Figures/NSC_withModel_v20230301.pdf}
	\caption{Centrality dependence of the different orders $a = 1,\ldots,3$ of $\NAC_{a,1}(m,n)$ and $\NAC_{1,a}(m,n)$ compared with the theoretical predictions from \trento+iEBE-VISHNU for the Duke 2019~\cite{Bernhard:2019bmu} and Jyväskylä 2022~\cite{Parkkila:2021yha} parameterizations. The last point of $\NAC_{1,2}(2,4)$ and the results for $\NAC_{1,3}(2,3)$ are not shown for visibility purposes. The statistical uncertainties in the calculations are indicated by the thickness of the colored bands. The data points for $a = 1$ are taken from Ref.~\cite{ALICE:2021adw}.}
	\label{fig:NAC_selected}
\end{figure}

The main findings of this study on AC and NAC are summarized here as follows:
\begin{enumerate}[(i)]
	\item The magnitude of the NAC observables varies with different moments as well as with the collision centrality, indicating that non-linear responses are also reflected in higher moments.
	\item While the $\NAC_{2,1}(2,3)$ observable shows a negative decreasing magnitude toward peripheral collisions, the higher-moment correlations, $\NAC_{3,1}(2,3)$, are close to zero except in the most central collisions. The model calculations show a similar trend, but underestimate the data.
	\item Interestingly, the $\AC_{3,1}(2,4)$ observable is much stronger than $\AC_{1,3}(2,4)$ in peripheral collisions. The $\AC_{1,3}(2,4)$ observable is close to zero, while the $\AC_{1,4}(2,4)$ is compatible with zero within uncertainties. As non-zero measurements for SC$(2,4)$~\cite{ALICE:2016kpq} indicate that $v_2^2$ and $v_4^2$ are positively correlated, this does not mean that the higher-order fluctuations of $v_2^2$ and $v_4^2$ are uncorrelated, but that the used data sample is not enough to extract their nonvanishing values.  
	\item $\NAC_{a,1}(2,n)$ shows an inverted dependence on the moments when compared to $\NAC_{1,a}(2,n)$ for $n = 3,4$. No splitting is observed for $\NAC_{a,1}(3,4)$ and $\NAC_{1,a}(3,4)$.
\end{enumerate}

The agreement between model and data for each cumulant order is now quantified with the $\chi^2$-test ($\chi^2/N_{\rm dof}$), performed as in Ref~\cite{ALICE:2017kwu}
%
\begin{equation}
   \chi^2/N_{\rm dof}=\frac{1}{N_{\rm dof}}\sum\limits_{i=1}^{N_{\rm dof}} \frac{(y_{i}-f_{i})^2}{\sigma_{i}^2},
\end{equation}
%
where $y_{i}$ ($f_{i}$) is a measurement (model) value in a centrality bin ${i}$. The systematic and statistical uncertainties from the data are combined in quadrature $\sigma_{i}^2 = \sigma_{i, \rm stat}^2 + \sigma_{i, \rm syst}^2 + \sigma_{f_{i}, \rm stat}^2$, together with the statistical errors of the model calculations. The quantity $N_{\rm dof}$ corresponds to the number of bins in the centrality range used in $\chi^2$ calculations.
%
The results are shown in Fig.~\ref{fig:chi_squared} for both model parameterizations indicated on the upper $x$-axis. A grey box indicates that the experimental data have large uncertainties, which may influence the interpretation of the obtained $\chi^2$-value. 
For instance, an agreement between these large uncertainties of the data and the predictions could bias the $\chi^2$-test, leading to an artificially small value. Such bias would then be resolved with a larger data sample.
As they were not measured in the current analysis, $\NAC_{1,3}(2,4)$ and $\NAC_{1,3}(3,4)$ are shown with a white filling. 
%
\begin{figure}[b]
	\centering
	\includegraphics[width=0.75\linewidth]{Figures/chi2test_v20230209.pdf}
	\caption{Values for $\chi^2$-test calculated between the data and two different model calculations for all AC and NAC and all pairs of harmonics presented in this paper. The observables with large uncertainties are indicated with a grey filling, while an absence of observable (see Fig.~\ref{fig:ALICEonly}) is shown with a white filling.}
	\label{fig:chi_squared}
\end{figure}
%
A better agreement between the data and a model parameterization is represented by lower $\chi^2$-values.
%
% Start of the discussion part.
In Fig.~\ref{fig:chi_squared}, it is observed that the model description is better for higher-order harmonics such as all $\AC_{a,b}(3,4)$, and higher-order moments such as $\AC_{4,1}(m,n)$, as well as most of $\NAC_{2,1}(m,n)$ and $\NAC_{3,1}(m,n)$.
Generally, the Jyväskylä 2022 parameterization describes the data better than the Duke 2019 parameterization, except for $\SC(2,3)$, $\AC_{3,1}(2,3)$ and $\NSC(m,n)$ for all three pairs of harmonics.
It is also known that Jyväskylä 2022 has a deviation in its prediction of $v_2$ at \snn = 5.02 TeV~\cite{Parkkila:2021yha}, which may explain the differences observed in the descriptions of \AC involving $v_2$ but not of the NAC.
Ultimately, the $\chi^2$-test indicates that the independent AC observables are needed as input for future Bayesian analyses in order to reduce the uncertainty on the initial conditions and/or on the hydrodynamic transport coefficients such as $\eta/s(T)$ and $\zeta/s(T)$.

%------------------------------%
\section{Summary}
\label{sec:summary}
In conclusion, the first measurements of the correlations between different moments of two flow amplitudes in \PbPb~collisions at \snn = 5.02~TeV, using asymmetric cumulants, are shown. This work is a generalization of the previous measurements obtained with symmetric cumulants, proven to have an increased sensitivity to the initial conditions and medium properties.
Due to the cumulant uniqueness, each AC has access to new and independent information unreachable by measuring the SC only. Therefore, this article provides new measurements and resulting additional constraints for 34 new observables.
A dependence of the magnitude of the correlations on the moments and the centrality of the collision is observed, showing an influence of the non-linear response between the higher-moments as well. Furthermore, it has been seen that an increasing (decreasing) dependence with the moments for a given pair $(m,n)$ in $\NAC_{a,1}(m,n)$ generally leads to the decreasing (increasing) dependence for the mirror combination $\NAC_{1,a}(m,n)$.
The comparison of the experimental results with the predictions from two different parameterizations of the \trento+iEBE-VISHNU model chain obtained with Bayesian analyses shows a need for further improvements of the initial conditions and non-linear response. Future Bayesian analyses will therefore benefit by using these results as additional input data.

%%%%%%%%%%%%%%%%%%%%%%%%%%%%%%%%
% End main text 
%%%%%%%%%%%%%%%%%%%%%%%%%%%%%%%%

%%%%% acknowledgements - handled by EB chairs 
\newenvironment{acknowledgement}{\relax}{\relax}
\begin{acknowledgement}
\section*{Acknowledgements}
% add specific acknowledgements here 
% ...but please don't remove the line below: funding agencies
% will be acknowledged with a custom tex file handled by EB chairs after Collab Round 2
\input{fa_2023-03-06_Opt_C.tex}
\end{acknowledgement}

%%%%%%%% Bibliography 
\bibliographystyle{utphys}   % Remember we use title in the biblio
\bibliography{bibliography}
%\setcounter{footnote}{0} 

\cleardoublepage
\phantomsection
\addcontentsline{toc}{section}{\refname}

% REMOVE in  % \bibitem{}:
	% accented characters (acute or grave accent) [´`], 
	% comma [,], 
	% cedilla: a̧, b̧, ç, etc.
	% German Eszett: ß
	% letters with an umlaut mark or diaeresis [¨]
	% letters with breve: ğ
	% letters with breve diacritic mark [˘]
	% letters with circumflex diacritic (chevron-shaped) or caron [ˆˇ], 
	% letters with ogonek ą
	% letters with stroke: ł 
	% letters with the addition of a dot above [◌̇]
	% Scandinavian vowel ø Ø
	% these quotation mark: “ ”

% Regola sui cognomi composti. Se la prima parola è scritta in minuscolo (da, de, 't, van, von, etc.), si segue l'ordine alfabetico della seconda parola. Se la prima parola è scritta in maiuscolo oppure se c'è l'apostrofo, anche con singola lettera minuscola (d', De, Dell', Di, Van, etc.), si segue l'ordine alfabetico della prima parola/lettera.

\begin{thebibliography}{0}\small 
\thispagestyle{empty}
\markboth{Bibliography}{Bibliography}

\bibitem{Besov Il'In Nikol'skii "Integral Representations of Functions and Imbedding Theorems I"} O.V. Besov, V.P. Il'In, S.M. Nikol'skii, \textit{Integral Representations of Functions and Imbedding Theorems, Vol. I}, Ed. by M.H. Taibleson, J. Wiley \& Sons, Washington, 1978.
\bibitem{Besov Il'In Nikol'skii "Integral Representations of Functions and Imbedding Theorems II"} ------, \textit{Integral Representations of Functions and Imbedding Theorems, Vol. II}, Ed. by M.H. Taibleson, J. Wiley \& Sons, Washington, 1979.

\bibitem{Chen Cruzeiro "Stochastic geodesics and forward-backward stochastic differential equations on Lie groups"} X. Chen, A.B. Cruzeiro, \textit{Stochastic geodesics and forward-backward stochastic differential equations on Lie groups}, Discrete Contin. Dyn. Syst., Vol. 2013 (Supplement), pp. 115-121.

\bibitem{Cruzeiro "Hydrodynamics Probability and the Geometry of the Diffeomorphisms Group"} A.B. Cruzeiro, \textit{Hydrodynamics, Probability and the Geometry of the Diffeomorphisms Group}, in R.C. Dalang, M. Dozzi, F. Russo (Eds.), \textit{Seminar on Stochastic Analysis, Random Fields and Applications VI}, Centro Stefano Franscini, Ascona, May 2008, Birkhäuser · Springer Basel \textsc{ag}, Basel, 2011, pp. 83-93.

\bibitem{Cruzeiro and Zambrini "Feynman's Functional Calculus and Stochastic Calculus of Variations"} A.B. Cruzeiro and J.-C. Zambrini, \textit{Feynman's Functional Calculus and Stochastic Calculus of Variations}, in A.B. Cruzeiro, J.-C. Zambrini (Eds.), \textit{Stochastic Analysis and Applications}, Proceedings of the 1989 Lisbon Conference, Springer Science+Business Media, New York, 1991, pp. 82-95.

\bibitem{Dankel Jr. "Mechanics on Manifolds and the Incorporation of Spin into Nelson's Stochastic Mechanics"} T.G. Dankel Jr., \textit{Mechanics on Manifolds and the Incorporation of Spin into Nelson's Stochastic Mechanics}, Arch. Rational Mech. Anal., Vol. 37, № 3, 1970, pp. 192-221.

\bibitem{Dieudonne Schwartz "La dualite dans les espaces (F) et (LF)"} J. Dieudonné, L. Schwartz, \textit{La dualité dans les espaces $(\mathscr{F})$ et $(\mathscr{LF})$}, Ann. Inst. Fourier, Vol. 1, 1949, pp. 61-101.

\bibitem{Dohrn and Guerra "Nelson's stochastic mechanics on Riemannian manifolds"} D. Dohrn and F. Guerra, \textit{Nelson's stochastic mechanics on Riemannian manifolds}, Lett. Nuovo Cimento, Vol. 22, № 4, 1978, pp. 121-127.

\bibitem{Dohrn and Guerra "Geodesic correction to stochastic parallel displacement of tensors"} ------, \textit{Geodesic correction to stochastic parallel displacement of tensors}, in G. Casati, J. Ford (Eds.), \textit{Stochastic Behavior in Classical and Quantum Hamiltonian Systems}, Volta Memorial Conference, Como 1977, Springer-Verlag, Berlin, Heidelberg, 1979, pp. 241-249.

\bibitem{Dohrn Guerra Ruggiero "Spinning Particles and Relativistic Particles in the Framework of Nelson's Stochastic Mechanics"} D. Dohrn, F. Guerra, P. Ruggiero, \textit{Spinning Particles and Relativistic Particles in the Framework of Nelson's Stochastic Mechanics}, in S. Albeverio, Ph. Combe, R. Høegh-Krohn, G. Rideau, M. Sirugue-Collin, M. Sirugue and R. Stora (Eds.), \textit{Feynman Path Integrals}, Proceedings of the International Colloquium, Held in Marseille, May 1978, Springer-Verlag, Berlin, Heidelberg, 1979, pp. 165-181. 

\bibitem{Gliklikh and Vinokurova "The Newton-Nelson Equation on Fiber Bundles with Connections"} \textcyrillic{Ю.Е. Гликлих, Н.В. Винокурова}, \textcyrillic{\textit{Уравнение Ньютона–Нельсона на расслоениях со связностями}}, \textcyrillic{Фундаментальная и прикладная математика, том 20, № 3, 2015, pp. 61-81}.\footnote{ 
	{} An En. version, made by the authors themselves, Y.E. Gliklikh and N.V. Vinokurova, is in J. Math. Sci., Vol. 225, № 4, 2017, pp. 575-589, under the title \textit{The Newton–Nelson Equation on Fiber Bundles with Connections}.
	}

\bibitem{Grothendieck "Sur les espaces (F) et (DF)"} A. Grothendieck, \textit{Sur les espaces $(\mathscr{F})$ et $(\mathscr{DF})$}, Summa Bras. Math., Vol. 3, № 6, 1954, pp. 57-123. Fr. version is not found; version consulted: Ru. transl. by D.A. Raikov, Matematika, Vol. 2, № 3, 1958, pp. 81-127.
\bibitem{Grothendieck "Recoltes et Semailles. Reflexions et temoignage sur un passe de mathematicien"} ------, \textit{Récoltes et Semailles. Réflexions et témoignage sur un passé de mathématicien} [Juin 1983-Avril 1986], text in a free, online version: the number of pages refers to the page numbering of the manuscript (otm = of the manuscript).

\bibitem{Guerra and Ruggiero "A Note on Relativistic Markov Processes"} F. Guerra and P. Ruggiero, \textit{A Note on Relativistic Markov Processes}, Lett. Nuovo Cimento, Vol. 23, № 15, 9 Dic. 1978, pp. 529-534.

\bibitem{Ito "The Brownian motion and tensor fields on Riemannian manifold"} K. Itô, \textit{The Brownian motion and tensor fields on Riemannian manifold}, in  \textit{Proceedings of the International Congress of Mathematicians}, 15-22 August 1962, Stockholm, Institut Mittag-Leffler, Djursholm, Almqvist \& Wiksell, Uppsala, 1963, pp. 536-539.
\bibitem{Ito "Stochastic parallel displacement"} ------, \textit{Stochastic parallel displacement}, in M.A. Pinsky (Ed.), \textit{Probabilistic Methods in Differential Equations}, Proceedings of the Conference held at the University of Victoria, August 19-20, 1974, pp. 1-7.

\bibitem{"On the logarithmic normal distribution of particle sizes under grinding"} A.N. Kolmogorov, \textit{On the logarithmic normal distribution of particle sizes under grinding} (1941), in A.N. Shiryayev, \textit{Selected Works of A.N. Kolmogorov, Vol. II. Probability Theory and Mathematical Statistics}, Springer Science+Business Media, Dordrecht, 1992, pp. 281-284.

\bibitem{Littlewood Paley "Theorems on Fourier Series and Power Series"} J.E. Littlewood, R.E.A.C. Paley, \textit{Theorems on Fourier Series and Power Series}, J. Lond. Math. Soc. (1), Vol. 6, № 3, 1931, pp. 230-233.
\bibitem{Littlewood Paley "Theorems on Fourier Series and Power Series II"} ------, \textit{Theorems on Fourier Series and Power Series (II)}, J. Lond. Math. Soc. (2), Vol. 42, № 1, 1937, pp. 52-89.
\bibitem{Littlewood Paley "Theorems on Fourier Series and Power Series III"} ------, \textit{Theorems on Fourier Series and Power Series (III)}, J. Lond. Math. Soc. (2), Vol. 43, № 1, 1938, pp. 105-126.

\bibitem{Mandelbrot "Les objets fractals. Forme hasard et dimension"} B. Mandelbrot, \textit{Les objets fractals. Forme, hasard et dimension}, Flammarion, 1975, Paris (quatrième édition, 1995).
\bibitem{Mandelbrot "Measures of fractal lacunarity: Minkowski content and alternatives"} ------, \textit{Measures of fractal lacunarity: Minkowski content and alternatives}, in C. Bandt, S. Graf, M. Zähle (Eds.), \textit{Fractal Geometry and Stochastics}, Birkhäuser · Springer Basel \textsc{ag}, Basel, 1995, pp. 15-42.

\bibitem{Nelson "Derivation of the Schrodinger Equation from Newtonian Mechanics"} E. Nelson, \textit{Derivation of the Schrödinger Equation from Newtonian Mechanics}, Phys. Rev., Vol. 150, № 4, 1966, pp. 1079-1085. 
\bibitem{Nelson "Construction of Quantum Fields from Markoff Fields"} ------, \textit{Construction of Quantum Fields from Markoff Fields}, J. Funct. Anal., Vol. 12, № 1, 1973, pp. 97-112.
\bibitem{Nelson "Quantum fluctuations"} ------, \textit{Quantum fluctuations}, Princeton Univ. Press, Princeton (\textsc{nj}), 1985. 
\bibitem{Nelson "Dynamical Theories of Brownian Motion"} ------, \textit{Dynamical Theories of Brownian Motion}, re-typesetted second edition as \TeX{} file by J. Suzuki and revised by Nelson in 2001 (originally published by Princeton Univ. Press, 1967).

\bibitem{Niccolai "Notes in Pure Mathematics and Mathematical Structures in Physics"} E. Niccolai, \textit{Notes in Pure Mathematics \& Mathematical Structures in Physics}, \href{https://arxiv.org/abs/2105.14863}{arXiv:2105.14863} [math-ph], 2023 [v9]; the latest revision: \emph{Download} page at \href{https://www.edoardoniccolai.com}{https://www.edoardoniccolai.com}.
\bibitem{Niccolai "Spin and Torsion Tensors on Gauge Gravity: a Re-examination of the Einstein-Cartan Spatio-Temporal Theory"} ------, \textit{Spin \& Torsion Tensors on Gauge Gravity: a Re-examination of the Einstein–Cartan Spatio-Temporal Theory}, doi:\href{https://doi.org/10.5281/zenodo.7775360}{10.5281.7775360}, 2023 [v5], or \href{https://hal.science/hal-03948127}{hal-03948127} (HAL Id), 2023 [v3].

\bibitem{Nottale "Scale Relativity Fractal Space-Time and Quantum Mechanics"} L. Nottale, \textit{Scale Relativity, Fractal Space-Time and Quantum Mechanics}, Chaos Solitons Fractals, Vol. 4, № 3, 1994, pp. 361-388.
\bibitem{Nottale "Scale Relativity and Fractal Space-Time: Applications to Quantum Physics Cosmology and Chaotic Systems"} ------, \textit{Scale Relativity and Fractal Space-Time: Applications to Quantum Physics, Cosmology and Chaotic Systems}, Chaos Solitons Fractals, Vol. 7, № 6, 1996, pp. 877-938.
\bibitem{Nottale "Scale Relativity and Fractal Space-Time: Theory and Applications"} ------, \textit{Scale Relativity and Fractal Space-Time: Theory and Applications}, Found. Sci., Vol. 15, № 2, 2010, pp. 101-152; \href{https://arxiv.org/abs/0812.3857}{arXiv:0812.3857} [physics.gen-ph], 2008 [v1].
\bibitem{Nottale "Scale Relativity and Fractal Space-Time. A New Approach to Unifying Relativity and Quantum Mechanics"} ------, \textit{Scale Relativity and Fractal Space-Time. A New Approach to Unifying Relativity and Quantum Mechanics}, World Scientific, Singapore, 2011.

\bibitem{Nottale Celerier and T. Lehner Non-Abelian gauge field theory in scale relativity"} L. Nottale, M.-N. Célérier, and T. Lehner, \textit{Non-Abelian gauge field theory in scale relativity}, J. Math. Phys. 47, № 3, 2006, pp. 032303-1-19; \href{https://arxiv.org/abs/hep-th/0605280}{arXiv:hep-th/0605280}, 2006 [v1].

\bibitem{Nottale and Lehner "Turbulence and Scale Relativity"} L. Nottale and T. Lehner, \textit{Turbulence and Scale Relativity}, Phys. Fluids, Vol. 31, № 10, 2019, pp. 105109-1-22; \href{https://arxiv.org/abs/1807.11902}{arXiv:1807.11902} [physics.gen-ph], 2018 [v1].\bibitem{Ruggiero and Tartaglia "Einstein-Cartan theory as a theory of defects in space-time"} M.L. Ruggiero and A. Tartaglia, \textit{Einstein–Cartan theory as a theory of defects in space–time}, Amer. J. Phys., Vol. 71, № 12, 2003, pp. 1303-1313.

\bibitem{Paley and Wiener "Fourier Transforms in the Complex Domain"} R.E.A.C. Paley and N. Wiener, \textit{Fourier Transforms in the Complex Domain}, American Mathematical Society, New York, 1934.

\bibitem{Paley Wiener and Zygmund "Notes on random functions"} R.E.A.C. Paley, N. Wiener and A. Zygmund, \textit{Notes on random functions}, Math. Z., Vol. 37, № 1, 1933, pp. 647-668.

\bibitem{Sawano "Theory of Besov Spaces"} Y. Sawano, \textit{Theory of Besov Spaces}, Springer Nature, Singapore, 2018.

\bibitem{Schwartz "Produits tensoriels topologiques d'espaces vectoriels topologiques. Espaces vectoriels topologiques nucleaires"} L. Schwartz, \textit{Produits tensoriels topologiques d'espaces vectoriels topologiques. Espaces vectoriels topologiques nucléaires}, Séminaire Schwartz, Tome 1, 1953-1954, exp. № 1-24.

\bibitem{Sobolev "Sur un theoreme d'analyse fonctionnelle"} S.[L.] Sobolev (Soboleff), \textcyrillic{\textit{Об одной теореме функционального анализа}} (\textit{Sur un théorème d'analyse fonctionnelle}), Mat. Sb., Vol. 4(46), № 3, 1938, pp. 471-497.
\bibitem{Sobolev "Some Applications of Functional Analysis in Mathematical Physics"} ------, \textit{Some Applications of Functional Analysis in Mathematical Physics}, transl. from the Ru. by H.H. McFaden, American Mathematical Society (\textsc{ams}), Providence (\textsc{ri}), 2008\textsuperscript{re}.

\bibitem{Uhlenbeck and Ornstein "On the Theory of the Brownian Motion"} G.E. Uhlenbeck and L.S. Ornstein, \textit{On the Theory of the Brownian Motion}, Phys. Rev., Vol. 36, № 5, 1930, pp. 823-841.

\bibitem{Wiener "Nonlinear Problems in Random Theory"} N. Wiener, \textit{Nonlinear Problems in Random Theory}, The Technology Press of The Massachusetts Institute of Technology and J. Wiley \& Sons, New York, Chapman \& Hall, London, 1958.

\bibitem{Zastawniak "A Relativistic Version of Nelson's Stochastic Mechanics"} T. Zastawniak, \textit{A Relativistic Version of Nelson's Stochastic Mechanics}, Europhys. Lett., Vol. 13, № 1, 1990, pp. 13-17.
\end{thebibliography}
  

%%%%%%%%%%%%%%%%%%%%%%%%%%%%%%%%
% Appendices: yours (if any) + authorlist
%%%%%%%%%%%%%%%%%%%%%%%%%%%%%%%%
\newpage
\appendix
\section{Additional observables}
\label{app:OtherObservables}
The comparisons of the model predictions to the experimental data for $\NAC_{a,b}(2,3)$, $\AC_{a,b}(2,4)$, and $\NAC_{a,b}(3,4)$ are presented in the main part of this article.
In this Appendix, the comparisons for the other observables of this analysis are discussed. The complete list of combinations can be found in Figs.~\ref{fig:ALICEonly} and~\ref{fig:chi_squared}.

The predictions from Duke 2019 and Jyväskylä 2022 parameterizations and the experimental data for $\AC_{a,b}(2,3)$ and $\NAC_{a,b}(2,4)$ are presented in Figs.~\ref{fig:AC23_withModel} and~\ref{fig:NAC24_withModel}, respectively.
The same splitting with the cumulant order as observed in Fig.~\ref{fig:AC24_withModel} for $\NAC_{a,b}(2,3)$ and $\AC_{a,b}(2,4)$ can be seen here.
In both cases, $\NAC_{a,1}(2,4)$ show increasingly larger magnitudes than $\NAC_{a,1}(4,2)$, while the opposite stands true for the unnormalized AC.
%
\begin{figure}[b]
	\centering
	\includegraphics[width=0.75\linewidth]{Figures/AC23_withModel_v20230127.pdf}
	\caption{Centrality dependence of the different orders $a = 1,\ldots,4$ of $\AC_{a,1}(2,3)$ (cyan closed circles) and $\AC_{1,a}(2,3)$ (pink open squares) compared with the theoretical predictions from \trento+iEBE-VISHNU for the Duke 2019~\cite{Bernhard:2019bmu} and Jyväskylä 2022~\cite{Parkkila:2021yha} parameterizations. The statistical uncertainties in the calculations are indicated by the thickness of the colored bands.}
	\label{fig:AC23_withModel}
\end{figure}
%
The observables $\AC_{a,b}(2,3)$ and $\AC_{a,b}(2,4)$ (Figs.~\ref{fig:AC23_withModel} and~\ref{fig:AC24_withModel}) show similar agreements between predictions and data. The Jyväskylä 2022 parameterization has overall a better agreement with the data than the Duke 2019 one, and the higher-order AC are overall better reproduced than $\SC(m,n)$ and $\AC_{2,1}(m,n)$.
%
It can, however, be noted that while the Jyväskylä 2022 parameterization reproduces quantitatively $\SC(2,4)$, whereas the Duke 2019 one does not; the opposite behavior is observed for $\SC(2,3)$.
On the other hand, no big difference in the predictions of $\NAC_{a,1}(2,4)$ and $\NAC_{a,1}(4,2)$ between the Duke 2019 and Jyväskylä 2022 parameterizations can be observed. Possible effects, such as the ones at the origin of the 10\% discrepancy in the description of $v_2$ at 5.02~TeV by the Jyväskylä 2022 parameterization~\cite{Parkkila:2021yha}, could be at play in $\SC(2,3)$ and cancel out in $\NSC(2,3)$.

In the case of $\NAC_{a,b}(2,4)$ (Fig.~\ref{fig:NAC24_withModel}), a slightly better agreement with the data from Jyväskylä 2022 over Duke 2019 parameterizations can be observed. As stated above, the non-linear response between $v_4$ and $\epsilon _2^2$ makes $\AC_{a,b}(2,4)$, and thus $\NAC_{a,b}(2,4)$, especially sensitive to the hydrodynamical description of the collision.
%
\begin{figure}[t]
	\centering
	\includegraphics[width=0.75\linewidth]{Figures/NAC24_withModel_v20230127.pdf}
	\caption{Centrality dependence of the different orders $a = 1,\ldots,3$ of $\NAC_{a,1}(2,4)$ (cyan closed circles) and $\NAC_{1,a}(2,4)$ (pink open squares) compared with the theoretical predictions from \trento+iEBE-VISHNU for the Duke 2019~\cite{Bernhard:2019bmu} and Jyväskylä 2022~\cite{Parkkila:2021yha} parameterizations. The statistical uncertainties in the calculations are indicated by the thickness of the colored bands. The data points for $\NSC(2,4)$ are taken from Ref.~\cite{ALICE:2021adw}.}
	\label{fig:NAC24_withModel}
\end{figure}

The predictions for $\AC_{a,b}(3,4)$ are shown in Fig.~\ref{fig:AC34_withModel}.
As was already discussed for $\NAC_{a,b}(3,4)$ (Fig.~\ref{fig:NAC34_withModel}), the measurements for $\AC_{a,1}(3,4)$ and $\AC_{a,1}(4,3)$ are in agreement within uncertainties for both the data and the calculations.
Furthermore, the predictions from the Jyväskylä 2022 and Duke 2019 parameterizations are both in overall good agreement with the experimental data. 
%
\begin{figure}[t]
	\centering
	\includegraphics[width=0.75\linewidth]{Figures/AC34_withModel_v20230127.pdf}
	\caption{Centrality dependence of the different orders $a = 1,\ldots,4$ of $\AC_{a,1}(3,4)$ (cyan closed circles) and $\AC_{1,a}(3,4)$ (pink open squares) compared with the theoretical predictions from \trento+iEBE-VISHNU for the Duke 2019~\cite{Bernhard:2019bmu} and Jyväskylä 2022~\cite{Parkkila:2021yha} parameterizations. The statistical uncertainties in the calculations are indicated by the thickness of the colored bands.}
	\label{fig:AC34_withModel}
\end{figure}
\clearpage
\newpage

%%%%% Authorlist - please do not touch: handled by EB chairs 
\section{The ALICE Collaboration}
\label{app:collab}
\input{2023-03-06-Alice_Authorlist_2023-03-06_Opt_C.tex}  
\end{document}