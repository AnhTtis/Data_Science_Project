%\documentclass[aps,prl,floatfix,twocolumn,10pt]{revtex4}
\documentclass[aps,twocolumn,nofootinbib,superscriptaddress,floatfix]{revtex4}

\usepackage{color,amsmath,amssymb,graphicx,latexsym,subfigure}
\usepackage{threeparttable}
\usepackage[colorlinks]{hyperref}
\usepackage{ulem,soul}
\normalem % use italic rather than underline for the \emph


\def\nat{Nature\ }
\def\aap{Astron.\ Astrophys.\ }
\def\apj{Astrophys.\ J.\ }
\def\apjl{Astrophys.\ J.\ Lett.\ }
\def\apjs{Astrophys.\ J.\ Supp.\ }
\def\aj{Astron.\ J.\ }
\def\mnras{Mon.\ Not.\ Roy.\ Astron.\ Soc.\ }
\def\physrep{Phys.\ Rept.\ }
\def\prd{Phys.\ Rev.\ D\ }
\def\prl{Phys.\ Rev.\ Lett.\ }
\def\apss{Astrophys.\ Space\ Sci.\ }
\def\araa{Annu.\ Rev.\ Astron.\ Astrophys.\ }
\def\jcap{J.\ Cosmol.\ Astropart.\ Phys.\ }

%\usepackage{titlesec}
%\titleformat{\section}[runin]{\normalfont \bfseries}{\thesection}{1em}{}
\renewcommand{\dblfloatpagefraction}{0.99}

\newcommand{\red}[1]{\textcolor{red}{#1}}
\newcommand{\magenta}[1]{\textcolor{magenta}{#1}}
\newcommand{\blue}[1]{\textcolor{blue}{#1}}
\newcommand{\green}[1]{\textcolor{green}{#1}}
\newcommand{\Tr}{\text{Tr}}
\newcommand{\im}{\text{Im}}
\newcommand{\re}{\text{Re}}
\newcommand{\nn}{{\nonumber}}
\newcommand{\gev}{\,{\rm GeV}}
\newcommand{\tev}{\,{\rm TeV}}
\newcommand{\mev}{\,{\rm MeV}}
\newcommand{\pev}{\,{\rm PeV}}
\newcommand{\kev}{\,{\rm keV}}
\newcommand{\beq}{\begin{equation}}
\newcommand{\eeq}{\end{equation}}
\newcommand{\bea}{\begin{eqnarray}}
\newcommand{\eea}{\end{eqnarray}}
\newcommand{\seff}{\sin^2\theta_W^{\rm eff}}
\newcommand{\gsim}{\lower.7ex\hbox{$\;\stackrel{\textstyle>}{\sim}\;$}}
\newcommand{\lsim}{\lower.7ex\hbox{$\;\stackrel{\textstyle<}{\sim}\;$}}
\newcommand{\nnmb}{\nonumber}
\newcommand{\xpb}{\,{\rm pb}}
\newcommand{\mrm}{\mathrm}
\newcommand{\Realint}{\mathbb R}
\newcommand{\Zint}{\mathbb{Z}}
\newcommand{\Nint}{\mathdd{N}}
\newcommand{\be}{\begin{equation}}
\newcommand{\ee}{\end{equation}}
\newcommand{\ba}{\begin{eqnarray}}
\newcommand{\ea}{\end{eqnarray}}
\newcommand{\ra}{\rightarrow}
\newcommand{\td}{\tilde}
\newcommand{\ov}{\overline}
\newcommand{\norsl}{\normalsize\sl}
\newcommand{\ns}{\normalsize}
\newcommand{\refs}[1]{(\ref{#1})}
\newcommand{\V}{\mathcal{V}}
\newcommand{\K}{\mathcal{K}}
\newcommand{\LN}{\textrm{ln}}
\newcommand{\X}{\mathcal{X}}

\newcommand{\SgrA}{Sgr~A$^*$}

\newcommand*{\mkred}[1]{{\color{red}{#1}}}
\newcommand*{\mkblue}[1]{{\color{blue}{#1}}}
\newcommand*{\mkgreen}[1]{{\color{green}{#1}}}
\newcommand*{\mkcyan}[1]{{\color{cyan}{#1}}}
\newcommand*{\mkmag}[1]{{\color{magenta}{#1}}}

\newcommand*{\sming}[1]{ \mkblue{\textbf{[}} \hl{#1} \mkblue{\textbf{]$_{-\rm YLST}$}} }



%\voffset 1.25cm

\begin{document}

\title{Rapidly growing primordial black holes as seeds of the massive high-redshift JWST Galaxies} 

\author{Guan-Wen Yuan} 
\affiliation{Key Laboratory of Dark Matter and Space Astronomy, Purple Mountain Observatory, Chinese Academy of Sciences, Nanjing 210023, China}
\affiliation{School of Astronomy and Space Science, University of Science and Technology of China, Hefei 230026, China}

\author{Lei Lei} 
\affiliation{Key Laboratory of Dark Matter and Space Astronomy, Purple Mountain Observatory, Chinese Academy of Sciences, Nanjing 210023, China}
\affiliation{School of Astronomy and Space Science, University of Science and Technology of China, Hefei 230026, China}

\author{Yuan-Zhu Wang} 
\affiliation{Key Laboratory of Dark Matter and Space Astronomy, Purple Mountain Observatory, Chinese Academy of Sciences, Nanjing 210023, China}

\author{Bo Wang} 
\affiliation{School of Astronomy and Space Science, University of Science and Technology of China, Hefei 230026, China}
\affiliation{Department of Astronomy, School of Physical Sciences, University of Science and Technology of China, Hefei, Anhui 230026, China}
\affiliation{CAS Key Laboratory for Researches in Galaxies and Cosmology, University of Science and Technology of China, Hefei, Anhui 230026, China}

\author{Yi-Ying Wang} 
\affiliation{Key Laboratory of Dark Matter and Space Astronomy, Purple Mountain Observatory, Chinese Academy of Sciences, Nanjing 210023, China}
\affiliation{School of Astronomy and Space Science, University of Science and Technology of China, Hefei 230026, China}

\author{Chao Chen} 
\affiliation{Jockey Club Institute for Advanced Study, The Hong Kong University of Science and Technology, Hong Kong, China}

\author{Zhao-Qiang Shen} \email[]{zqshen@pmo.ac.cn}
\affiliation{Key Laboratory of Dark Matter and Space Astronomy, Purple Mountain Observatory, Chinese Academy of Sciences, Nanjing 210023, China}

\author{Yi-Fu Cai} \email[]{yifucai@ustc.edu.cn}
\affiliation{School of Astronomy and Space Science, University of Science and Technology of China, Hefei 230026, China}
\affiliation{Department of Astronomy, School of Physical Sciences, University of Science and Technology of China, Hefei, Anhui 230026, China}
\affiliation{CAS Key Laboratory for Researches in Galaxies and Cosmology, University of Science and Technology of China, Hefei, Anhui 230026, China}

\author{Yi-Zhong Fan} \email[]{yzfan@pmo.ac.cn}
\affiliation{Key Laboratory of Dark Matter and Space Astronomy, Purple Mountain Observatory, Chinese Academy of Sciences, Nanjing 210023, China}
\affiliation{School of Astronomy and Space Science, University of Science and Technology of China, Hefei 230026, China}


\begin{abstract}

A group of massive galaxies at redshifts of $z\geq 6.5$ have been recently detected by James Webb Space Telescope (JWST), which were unexpected to form at such early times within the standard Big Bang cosmology. 
In this work we propose that the formation of some $\sim 50~M_\odot$ primordial black holes (PBHs) formed in the early Universe via super-Eddington accretion within the dark matter halo can explain these observations. These PBHs may act as seeds for early galaxies formation with masses of $\sim 10^{9}-10^{10}~M_\odot$ at $z\sim 8$, hence accounting for the JWST observations. 
We use a hierarchical Bayesian inference framework to constrain the PBH mass distribution models, and find that the Lognormal model with the $M_{\rm c}\sim 35M_\odot$ is strongly preferred over other hypotheses. These rapidly growing BHs are expected to have strong radiation and may appear as the high-redshift compact objects, similar to those recently discovered by JWST. 

\end{abstract}

\date{\today}
\maketitle
 
% \section{Introduction}
{\bf Introduction.} The standard Big Bang cosmology, which has been supported by numerous precise cosmological measurements, has achieved remarkable success. However, in the past decade, observations and measurements have challenged some of its predictions~\cite{Perivolaropoulos:2021jda}. In particular, the values of $H_0$ and $\sigma_8$  appear to be in tension with the $\Lambda$CDM model, leading to renewed interest in alternative models, including the possibility that primordial black holes (PBHs) could be responsible for some of the observed phenomena.

%%% the background of PBH 
PBHs are formed from large density fluctuations that can detach from the cosmic expansion and collapse into black holes~\cite{1966AZh, 1971MNRAS}. There are numerous distinct mechanisms that could produce such over-densities, including quantum fluctuations, phase transitions, sound speed resonance~\cite{Cai:2018tuh}, and particular designer models of inflation that result in a narrow peak in the fluctuation power spectrum. 
The cosmological implications of PBHs have garnered great attention recently, in part because they may serve as both dark matter candidates and the seeds of supermassive black holes (SMBH) in high-redshift galaxies~\cite{Green:2020jor,Inayoshi:2019fun}. Therefore, investigating PBHs presents an exciting opportunity to explore the mechanics of the early Universe~\cite{Escriva:2022duf}. 

%%% the possibility of PBH seeding SMBH
The upcoming JWST observations are expected to shed further light on the nature of high-redshift galaxies and the role of PBHs in their formation, whose formation mechanisms remain unclear~\cite{Gardner:2006ky, Volonteri:2021sfo}.
Recently, its observations have identified several bright galaxy candidates at $z \gtrsim 7$~\cite{2022arXiv220712446L, 2023MNRAS.519.1201A, 2023ApJ...942L...9Y}, which are difficult to reconcile with the Big Bang cosmology predictions~\cite{Volonteri:2021sfo}. As a result, various works have suggested that PBHs could explain these observations by altering the matter power spectrum~\cite{Liu:2022bvr, 2023ApJ...944..113B, 2023MNRAS.519.4753T,2023arXiv230109403C}.
In this study, we show that PBHs with masses of $\sim 100 M_{\odot}$ could potentially grow up into SMBHs with masses of $10^{6}-10^{8} M_{\odot}$ observed in the centers of galaxies, through super-Eddington accretion within the dark matter halo.
Then, we adopt a hierarchical Bayesian inference framework to constrain the PBH mass distribution models, and find that the Lognormal model with the $M_{\rm c}\sim 35M_\odot$ is strongly preferred over other hypotheses. These growing PBHs with strong radiation may appear as high-redshift compact objects, and accounted for those observations of JWST. 

{\bf PBH Accretion in the Early Universe.}
The study of PBHs has gained much attention in recent years due to their potential shed light on a variety of cosmological phenomena, from the formation of high-redshift galaxies to the nature of DM and the origins of SMBH. Once created, PBHs may evolve through accretion throughout cosmic history, and this accretion of baryonic matter onto PBHs can significantly impact their masses~\cite{DeLuca:2020fpg, Serpico:2020ehh}. However, the physics of accretion is complex, as the accretion rate and the geometry of the accretion flow are intertwined and both play a crucial role in the evolution of the PBH masses. An analytic accretion model, such as the one developed by Mack, Ricotti and Ostriker~\cite{Mack:2006gz,Ricotti:2007jk,Ricotti:2007au}, can be used to study the accretion of baryonic matter onto PBHs and provide useful insights into the accretion behavior. 
In the intergalactic medium, a PBH with mass $M$ can accrete baryonic matter at the Bondi-Hoyle rate $\dot{M}_{\mathrm{B}}$, which is affected by factors such as gas number density, gas viscosity, PBH effective velocity, Hubble expansion, and Compton scattering between the gas and the CMB radiation. 

\begin{figure}%[htbp] 
\centering 
\includegraphics[width=0.96\linewidth]{mz_contourf.pdf} 
\caption{Initial PBH mass ($M_{\rm BH, i}$, represented by the colorbar) as a function of the galaxy mass $M_{\star}$ and the redshift $z$. 
The candidates observed by JWST are represented by the orange and magenta points, adopted from Labbe et al.\cite{2022arXiv220712446L} and Atek et al.\cite{2023MNRAS.519.1201A}, respectively (see also Table~\ref{data_table} in the {Supplementary Material}).} 
\label{contourf}
\end{figure}

It is important to note that PBHs only make up a tiny portion of DM in the Universe, and that there is a dominant dark matter halo surrounds a PBH and expands over time as long as the PBHs do not interact with one another. This dark matter halo, or "dark matter clothing", increases the gas accretion rate, acting as a catalyst for PBH accretion despite the fact that the direct DM accretion is negligible for the PBH evolution. 
Assuming a black hole with a power law density of $\rho \sim r^{-2.25}$, we have the typical halo radius $r_h$, and
the gas accreting onto an extended dark halo exhibits the same accretion rate behavior as a point mass when the Bondi radius $r_B$ is greater than twice the radius of the dark halo. However, in the outer area, the accretion parameter must be modified to introduce corrections to the naked case~\cite{Ricotti:2007au, Ricotti:2007jk}. 
After considering the effect of the dark halo, the efficiency and shape of the accretion process are entirely encoded in the dimensionless baryonic accretion rate $\dot{m}$, which is defined as  $\dot{m} = \dot{M}_{\rm B} / \dot{M}_{\rm Edd}$ with the Eddington accretion $\dot{M}_{\rm Edd} \equiv 1.44\times 10^{17} (M/M_{\odot})\,{\rm g\,s^{-1}}$. For instance, a thin accretion disk arises around the PBH if the angular momentum carried by the baryonic infalling material is sufficiently high. 

Usually, the PBHs with initial masses $M_{\rm BH, i}$ less than a few solar masses can not  grow up efficiently due to the low accretion rate.
However, for higher initial masses, accretion can be very efficient. 
Indeed, some PBHs formed in the very early Universe could grow up to $\sim 10^{6}~M_\odot$ at $z\sim 8$ through Bondi accretion~\cite{Ali-Haimoud:2016mbv,DeLuca:2020fpg}; see also the Supplementary Material for the details).
An empirical relationship between $M_{\rm BH}$ and the galaxy mass $M_{\star}$, which reads ${\rm log} M_{\rm BH}=\alpha+\beta {\rm log}(M_{\star}/{M_0})+\epsilon {\rm log}(1+z)$ with $M_0=3\times 10^{10} M_{\odot}$, has been proposed in Greene et al.~\cite{Greene:2019vlv}.
In this work we take the best-fit intercept for early-type galaxies, i.e., $\alpha = 7.89\pm 0.09$,  $\beta=1.33\pm 0.12$ and $\epsilon = 0.2$. In Figure~\ref{contourf} we show the $M_{\rm BH,i}$ needed to account for the current JWST observations of the high redshift massive galaxies within the PBH accretion scenario.  



%%%%%%%%%%%%%
{\bf Exploring PBH Mass Distribution with JWST Observations.}
Furthermore, Bondi accretion affects the mass distribution of PBHs with redshift. The fraction of PBHs with mass in the interval $(M, M+dM)$ at redshift $z$ is what we refer to as the mass function $\psi(M, z)$. The evolution of an initial $\psi(M, z_i)$ at formation redshift $z_i$ is governed by
\begin{equation}
\psi(M_f (M, z), z) dM_f = \psi(M, z_i)dM.
\end{equation}
where $M_f(M, z)$ is the final mass of a PBH that accretes mass $M$ at redshift $z$.
To distinguish various forms of theoretically predicted PBH mass functions, we consider the following typical PBH mass functions that arise in PBH formation models \cite{Carr:2020gox, Carr:2020xqk}, 

\begin{equation}\label{distribution}
\begin{aligned}
\psi_M=\left\{\begin{array}{ll}\vspace{0.3cm}
\frac{1}{\sqrt{2\pi} \sigma M} \exp (-\frac{\rm log^2 (M/M_c)}{2\sigma^2})  & \quad \text{Lognormal},\\ \vspace{0.3cm}
\sum_{n=1} A_n \delta (M-M_{cn}) & \quad \text{Multipeak} , \\ \vspace{0.3cm}
\frac{1}{2}\frac{M_c^{1/2}}{M^{3/2}} \Theta(M-M_c) & \quad \text{Powerlaw}, \\ \vspace{0.3cm}
\frac{1}{\sqrt{2\pi} \sigma } \exp (-\frac{\rm (M-M_c)^2}{2\sigma^2})  & \quad \text{Gaussian},\\ \vspace{0.3cm}
\frac{3.2}{M}\left(\frac{M}{M_{c}}\right)^{3.85} \exp^{-\left(\frac{M}{M_{c}}\right)^{2.85}} & \quad \text{Critical}.
\end{array}\right.
\end{aligned}
\end{equation}
where $\Theta(M-M_c)$ is the step function, and the $M_c, M_{cn}$ and $\sigma$ are parameters in these distributions. For the Multipeak model, we use two normalized Gaussian distributions with same width in our analysis, as the evolution of $\delta (M-M_{cn})$. 

\begin{figure*}%[htbp] 
\centering
\includegraphics[width=0.95\textwidth]{model_pdf.pdf}
\caption{The posterior population distribution of the PBH models within a 90\% credible interval. The distribution for Lognormal (red), Multipeak (orchid), Powerlaw (deepskyblue), Gaussian (orange) and Critical (grey) are shown, with each line plotted using its optimal parameters.
The Bayer factors for each model are displayed in the upper right corner, where the vertical dashed lines indicate very strong evidence (blue) and decisive Bayesian evidence (black). }
\label{model_pdf}
\end{figure*}


\begin{table*}
\centering
\setlength{\tabcolsep}{.5em}
\begin{tabular}{cccccccc}
\hline
 Models & $M_c$ &  $\sigma$ & $M_{c2}$ & $f_{M_c}$ & ${dof}$  &$\Delta \ln\mathcal{B}$ & $\Delta {\rm AIC}$ \\
\hline
Lognormal &  $34.73^{+7.18}_{-6.32}$ &  $0.87^{+0.19}_{-0.15}$ & - & - & 2 & 26.49 & 32.97 \\
\hline
Multipeak &  $40.15^{+4.55}_{-4.58}$ &  $19.99^{+3.85}_{-3.07}$ & $163.29^{+16.30}_{-16.29}$ & $0.89^{+0.05}_{-0.07}$ & 4 & 19.06 & 54.52 \\
\hline
Powerlaw & $20.45^{+2.43}_{-5.17}$ &  - & - & - & 1 & 4.35 & 74.10 \\
\hline
Gaussian &  $50.82^{+8.77}_{-8.57}$ &  $40.31^{+7.48}_{-5.82}$ & - & - & 2 & 3.45 & 76.62 \\
\hline
Critical & $64.57^{+5.02}_{-4.27}$ &  - & - & - & 1 &  0.00 & 88.42\\
\hline
\end{tabular}
\caption{Summary of the results for various initial mass functions of PBH. The first column lists the names of the models, followed by the 68\% credible intervals of their parameters. The last two columns list the Bayes factor $\Delta \ln\mathcal{B}$ and the Akaike information criterion $\Delta \mathrm{AIC}$. }
\label{evidence_table}
\end{table*}


We use the masses of high-redshift galaxies recently observed by JWST to constrain the hyper-parameters $\boldsymbol{\lambda}$ of each initial PBH mass function through hierarchical Bayesian inference~\cite{Thrane:2018qnx}. For a series of $N$ independent observations, the posterior distribution for $\boldsymbol{\lambda}$ is given by

\begin{equation}\label{fun1_hyperparameter}
p(\boldsymbol{\lambda} \mid \boldsymbol{d}) = \pi(\boldsymbol{\lambda}) \prod_{i}^{N}  \int \mathcal{L}({d_i} \mid {\theta}_i) p_{\mathrm{pop}}({\theta_i} \mid \boldsymbol{\lambda}) \mathrm{d} {\theta_i},
\end{equation}
where $\mathcal{L}({d_i} \mid {\theta}_i)$ denote the likelihood function of the JWST data given a galaxy's properties $\theta_i$ (the mass and the redshift). 
The distribution of $\theta_i$ as predicted by the PBH population models is denoted by $p_{\mathrm{pop}}({\theta_i} \mid \boldsymbol{\lambda})$, which satisfies $p_{\mathrm{pop}}({\theta_i} \mid \boldsymbol{\lambda}) = p(m)p(z)$, where $p(m)$ is the mass distribution of galaxies calculated from the initial PBH mass functions in Equation~(\ref{distribution}), and the galaxies are assumed distributing uniformly in the co-moving frame of the Universe. 
We assign Uniform priors $\pi(\boldsymbol{\lambda})$ for all of the hyper-parameters in this work. 
To approximate the reported result for the mass and redshift of each galaxy (as shown by a central value plus/minus the uncertainties in Table~\ref{data_table}, we use a two-dimensional skew normal distribution $\mathcal{N}(\theta_i)$, and assume that $\mathcal{L}({d_i} \mid {\theta}_i) \propto \mathcal{N}(\theta_i)$. Then the above equation can be calculated via Monte Carlo integration with sample points drawn from the skew normal distributions.


To investigate the statistical significance of our results, we calculate the Bayesian evidence for each model, which is the likelihood integrated over the prior parameter space, and compare them using Jeffreys' scale for evidence strength~\cite{jeffreys1998theory}. We also calculate the Akaike information criterion (AIC)~\cite{Akaike1974ANL} to compare models with different numbers of parameters. The AIC penalizes models with more parameters, and a difference of $\Delta \text{AIC}$ of 2 or more indicates strong evidence against the model with the higher AIC value. 
Our results, summarized in Table~\ref{evidence_table}, show that the Lognormal model is strongly preferred over the other models, with a Bayesian evidence of $\ln(\mathcal{B}) = 7.43$ compared to the next best model. 
In addition, the Critical model has the highest AIC value and is therefore strongly disfavored. The Gaussian and Powerlaw models are also  disfavored, with differences of $\Delta \text{AIC}$ of 43.65 and 41.13, respectively, compared to the Lognormal model. 


Figure~\ref{model_pdf} displays the constraints on the five different PBH initial mass functions and the Bayes factors of each model compared to the Critical model. The colored shaded zones represent the 90\% credible regions for the inferred population distribution. For all models, the majority of the PBH masses lie below $\sim 100 M_\odot$. According to Jeffreys' scale criterion~\cite{jeffreys1998theory}, the logarithmic Bayes factors for the Lognormal model, the Multipeak model, the Powerlaw model and the Gasssian model are 26.49, 19.06, 4.35 and 3.45, respectively, indicating they are decisively or strongly preferred by the data compared to the Critical model. 
Most strikingly, the Lognormal mass function stands out from the five models with a Bayes factor of 8.83 when compared to the second preferred model, so we conclude that the results from JWST observations are very informative in distinguishing the initial PBH mass function, and the current data has already put relatively tight constraints on the exact shape of the distribution. 
Moreover, the above analysis also highlights the importance of statistical analysis in making conclusions about the PBHs population and their implications for early cosmology. 


%%%%%%%%%%%%%
{\bf Summary.} We have shown in this study that PBHs with masses of $\sim 50~M_\odot$ formed in the early Universe can grow rapidly through accretion to masses of $\sim 10^{6}-10^{8}M_\odot$ by $z \sim 8$, which could potentially account for the massive high-redshift galaxies detected by JWST. 
Future observations with JWST of the Universe during the cosmic re-ionization era are expected to reveal these rapidly growing PBHs, as their accretion can yield  energetic radiation across a wide range of wavelengths. This, in turn, could seed a group of bright active galactic nuclei, which may appear as high-redshift compact objects, like that observed recently by JWST\cite{Lukas:2022agn}.

To constrain the formation models of PBHs, we used a hierarchical Bayesian inference framework and found that the Lognormal Model with $M_{\rm c}\sim 35M_\odot$ is strongly preferred over other scenarios. 
It is noteworthy that the initial mass distribution of the PBHs we found is markedly different from that expected in the pair-instability supernova scenario~\cite{Abel:2001pr, Heger:2002by}, which is characterized by an abrupt cutoff in the mass function at $\sim 40M_\odot$~\cite{2023arXiv230302973L}.
However, we caution that the current high-redshift galaxy sample is still small and the current model constraint result may be modified with future data. Nevertheless, our current result has demonstrated that the promising prospect of distinguishing between different PBH distribution models with future observation. 




{\bf Acknowledgements} We are grateful to Lei Feng, Qiang Yuan and Lei Zu for helpful discussion. This work is supported in part by the National Key R\&D Program of China (2021YFC2203100), by the NSFC (11921003, 11961131007, 11653002, 12003029, 12261131497), by the Fundamental Research Funds for Central Universities, by the CSC Innovation Talent Funds, by the CAS project for young scientists in basic research (YSBR-006), and by the USTC Research Funds of the Double First-Class Initiative.

\bibliographystyle{apsrev}
% \bibliography{reference}
% \begin{thebibliography}{92}
% \end{thebibliography}
\begin{thebibliography}{33}
\expandafter\ifx\csname natexlab\endcsname\relax\def\natexlab#1{#1}\fi
\expandafter\ifx\csname bibnamefont\endcsname\relax
  \def\bibnamefont#1{#1}\fi
\expandafter\ifx\csname bibfnamefont\endcsname\relax
  \def\bibfnamefont#1{#1}\fi
\expandafter\ifx\csname citenamefont\endcsname\relax
  \def\citenamefont#1{#1}\fi
\expandafter\ifx\csname url\endcsname\relax
  \def\url#1{\texttt{#1}}\fi
\expandafter\ifx\csname urlprefix\endcsname\relax\def\urlprefix{URL }\fi
\providecommand{\bibinfo}[2]{#2}
\providecommand{\eprint}[2][]{\url{#2}}

\bibitem[{\citenamefont{Perivolaropoulos and
  Skara}(2022)}]{Perivolaropoulos:2021jda}
\bibinfo{author}{\bibfnamefont{L.}~\bibnamefont{Perivolaropoulos}}
  \bibnamefont{and} \bibinfo{author}{\bibfnamefont{F.}~\bibnamefont{Skara}},
  \bibinfo{journal}{New Astron. Rev.} \textbf{\bibinfo{volume}{95}},
  \bibinfo{pages}{101659} (\bibinfo{year}{2022}), \eprint{2105.05208}.

\bibitem[{\citenamefont{{Zel'dovich} and {Novikov}}(1966)}]{1966AZh}
\bibinfo{author}{\bibfnamefont{Y.~B.} \bibnamefont{{Zel'dovich}}}
  \bibnamefont{and} \bibinfo{author}{\bibfnamefont{I.~D.}
  \bibnamefont{{Novikov}}}, \bibinfo{journal}{Astron. Z.}
  \textbf{\bibinfo{volume}{43}}, \bibinfo{pages}{758} (\bibinfo{year}{1966}).

\bibitem[{\citenamefont{{Hawking}}(1971)}]{1971MNRAS}
\bibinfo{author}{\bibfnamefont{S.}~\bibnamefont{{Hawking}}},
  \bibinfo{journal}{\mnras} \textbf{\bibinfo{volume}{152}}, \bibinfo{pages}{75}
  (\bibinfo{year}{1971}).

\bibitem[{\citenamefont{Cai et~al.}(2018)\citenamefont{Cai, Tong, Wang, and
  Yan}}]{Cai:2018tuh}
\bibinfo{author}{\bibfnamefont{Y.-F.} \bibnamefont{Cai}},
  \bibinfo{author}{\bibfnamefont{X.}~\bibnamefont{Tong}},
  \bibinfo{author}{\bibfnamefont{D.-G.} \bibnamefont{Wang}}, \bibnamefont{and}
  \bibinfo{author}{\bibfnamefont{S.-F.} \bibnamefont{Yan}},
  \bibinfo{journal}{Phys. Rev. Lett.} \textbf{\bibinfo{volume}{121}},
  \bibinfo{pages}{081306} (\bibinfo{year}{2018}), \eprint{1805.03639}.

\bibitem[{\citenamefont{Green and Kavanagh}(2021)}]{Green:2020jor}
\bibinfo{author}{\bibfnamefont{A.~M.} \bibnamefont{Green}} \bibnamefont{and}
  \bibinfo{author}{\bibfnamefont{B.~J.} \bibnamefont{Kavanagh}},
  \bibinfo{journal}{J. Phys. G} \textbf{\bibinfo{volume}{48}},
  \bibinfo{pages}{043001} (\bibinfo{year}{2021}), \eprint{2007.10722}.

\bibitem[{\citenamefont{Inayoshi et~al.}(2020)\citenamefont{Inayoshi, Visbal,
  and Haiman}}]{Inayoshi:2019fun}
\bibinfo{author}{\bibfnamefont{K.}~\bibnamefont{Inayoshi}},
  \bibinfo{author}{\bibfnamefont{E.}~\bibnamefont{Visbal}}, \bibnamefont{and}
  \bibinfo{author}{\bibfnamefont{Z.}~\bibnamefont{Haiman}},
  \bibinfo{journal}{Ann. Rev. Astron. Astrophys.}
  \textbf{\bibinfo{volume}{58}}, \bibinfo{pages}{27} (\bibinfo{year}{2020}),
  \eprint{1911.05791}.

\bibitem[{\citenamefont{Escriv\`a et~al.}(2022)\citenamefont{Escriv\`a, Kuhnel,
  and Tada}}]{Escriva:2022duf}
\bibinfo{author}{\bibfnamefont{A.}~\bibnamefont{Escriv\`a}},
  \bibinfo{author}{\bibfnamefont{F.}~\bibnamefont{Kuhnel}}, \bibnamefont{and}
  \bibinfo{author}{\bibfnamefont{Y.}~\bibnamefont{Tada}}
  (\bibinfo{year}{2022}), \eprint{2211.05767}.

\bibitem[{\citenamefont{Gardner et~al.}(2006)}]{Gardner:2006ky}
\bibinfo{author}{\bibfnamefont{J.~P.} \bibnamefont{Gardner}}
  \bibnamefont{et~al.}, \bibinfo{journal}{Space Sci. Rev.}
  \textbf{\bibinfo{volume}{123}}, \bibinfo{pages}{485} (\bibinfo{year}{2006}),
  \eprint{astro-ph/0606175}.

\bibitem[{\citenamefont{Volonteri et~al.}(2021)\citenamefont{Volonteri,
  Habouzit, and Colpi}}]{Volonteri:2021sfo}
\bibinfo{author}{\bibfnamefont{M.}~\bibnamefont{Volonteri}},
  \bibinfo{author}{\bibfnamefont{M.}~\bibnamefont{Habouzit}}, \bibnamefont{and}
  \bibinfo{author}{\bibfnamefont{M.}~\bibnamefont{Colpi}},
  \bibinfo{journal}{Nature Rev. Phys.} \textbf{\bibinfo{volume}{3}},
  \bibinfo{pages}{732} (\bibinfo{year}{2021}), \eprint{2110.10175}.

\bibitem[{\citenamefont{{Labbe} et~al.}(2022)\citenamefont{{Labbe}, {van
  Dokkum}, {Nelson}, {Bezanson}, {Suess}, {Leja}, {Brammer}, {Whitaker},
  {Mathews}, {Stefanon} et~al.}}]{2022arXiv220712446L}
\bibinfo{author}{\bibfnamefont{I.}~\bibnamefont{{Labbe}}},
  \bibinfo{author}{\bibfnamefont{P.}~\bibnamefont{{van Dokkum}}},
  \bibinfo{author}{\bibfnamefont{E.}~\bibnamefont{{Nelson}}},
  \bibinfo{author}{\bibfnamefont{R.}~\bibnamefont{{Bezanson}}},
  \bibinfo{author}{\bibfnamefont{K.}~\bibnamefont{{Suess}}},
  \bibinfo{author}{\bibfnamefont{J.}~\bibnamefont{{Leja}}},
  \bibinfo{author}{\bibfnamefont{G.}~\bibnamefont{{Brammer}}},
  \bibinfo{author}{\bibfnamefont{K.}~\bibnamefont{{Whitaker}}},
  \bibinfo{author}{\bibfnamefont{E.}~\bibnamefont{{Mathews}}},
  \bibinfo{author}{\bibfnamefont{M.}~\bibnamefont{{Stefanon}}},
  \bibnamefont{et~al.}, \bibinfo{journal}{Nature,}
  \bibinfo{eid}{arXiv:2207.12446} (\bibinfo{year}{2022}), \eprint{2207.12446}.

\bibitem[{\citenamefont{{Atek} et~al.}(2023)\citenamefont{{Atek}, {Shuntov},
  {Furtak}, {Richard}, {Kneib}, {Mahler}, {Zitrin}, {McCracken}, {Charlot},
  {Chevallard} et~al.}}]{2023MNRAS.519.1201A}
\bibinfo{author}{\bibfnamefont{H.}~\bibnamefont{{Atek}}},
  \bibinfo{author}{\bibfnamefont{M.}~\bibnamefont{{Shuntov}}},
  \bibinfo{author}{\bibfnamefont{L.~J.} \bibnamefont{{Furtak}}},
  \bibinfo{author}{\bibfnamefont{J.}~\bibnamefont{{Richard}}},
  \bibinfo{author}{\bibfnamefont{J.-P.} \bibnamefont{{Kneib}}},
  \bibinfo{author}{\bibfnamefont{G.}~\bibnamefont{{Mahler}}},
  \bibinfo{author}{\bibfnamefont{A.}~\bibnamefont{{Zitrin}}},
  \bibinfo{author}{\bibfnamefont{H.~J.} \bibnamefont{{McCracken}}},
  \bibinfo{author}{\bibfnamefont{S.}~\bibnamefont{{Charlot}}},
  \bibinfo{author}{\bibfnamefont{J.}~\bibnamefont{{Chevallard}}},
  \bibnamefont{et~al.}, \bibinfo{journal}{\mnras}
  \textbf{\bibinfo{volume}{519}}, \bibinfo{pages}{1201} (\bibinfo{year}{2023}),
  \eprint{2207.12338}.

\bibitem[{\citenamefont{{Yan} et~al.}(2023)\citenamefont{{Yan}, {Ma}, {Ling},
  {Cheng}, and {Huang}}}]{2023ApJ...942L...9Y}
\bibinfo{author}{\bibfnamefont{H.}~\bibnamefont{{Yan}}},
  \bibinfo{author}{\bibfnamefont{Z.}~\bibnamefont{{Ma}}},
  \bibinfo{author}{\bibfnamefont{C.}~\bibnamefont{{Ling}}},
  \bibinfo{author}{\bibfnamefont{C.}~\bibnamefont{{Cheng}}}, \bibnamefont{and}
  \bibinfo{author}{\bibfnamefont{J.-S.} \bibnamefont{{Huang}}},
  \bibinfo{journal}{\apjl} \textbf{\bibinfo{volume}{942}}, \bibinfo{eid}{L9}
  (\bibinfo{year}{2023}), \eprint{2207.11558}.

\bibitem[{\citenamefont{Liu and Bromm}(2022)}]{Liu:2022bvr}
\bibinfo{author}{\bibfnamefont{B.}~\bibnamefont{Liu}} \bibnamefont{and}
  \bibinfo{author}{\bibfnamefont{V.}~\bibnamefont{Bromm}},
  \bibinfo{journal}{Astrophys. J. Lett.} \textbf{\bibinfo{volume}{937}},
  \bibinfo{pages}{L30} (\bibinfo{year}{2022}), \eprint{2208.13178}.

\bibitem[{\citenamefont{{Biagetti} et~al.}(2023)\citenamefont{{Biagetti},
  {Franciolini}, and {Riotto}}}]{2023ApJ...944..113B}
\bibinfo{author}{\bibfnamefont{M.}~\bibnamefont{{Biagetti}}},
  \bibinfo{author}{\bibfnamefont{G.}~\bibnamefont{{Franciolini}}},
  \bibnamefont{and} \bibinfo{author}{\bibfnamefont{A.}~\bibnamefont{{Riotto}}},
  \bibinfo{journal}{\apj} \textbf{\bibinfo{volume}{944}}, \bibinfo{eid}{113}
  (\bibinfo{year}{2023}), \eprint{2210.04812}.

\bibitem[{\citenamefont{{Trinca} et~al.}(2023)\citenamefont{{Trinca},
  {Schneider}, {Maiolino}, {Valiante}, {Graziani}, and
  {Volonteri}}}]{2023MNRAS.519.4753T}
\bibinfo{author}{\bibfnamefont{A.}~\bibnamefont{{Trinca}}},
  \bibinfo{author}{\bibfnamefont{R.}~\bibnamefont{{Schneider}}},
  \bibinfo{author}{\bibfnamefont{R.}~\bibnamefont{{Maiolino}}},
  \bibinfo{author}{\bibfnamefont{R.}~\bibnamefont{{Valiante}}},
  \bibinfo{author}{\bibfnamefont{L.}~\bibnamefont{{Graziani}}},
  \bibnamefont{and}
  \bibinfo{author}{\bibfnamefont{M.}~\bibnamefont{{Volonteri}}},
  \bibinfo{journal}{\mnras} \textbf{\bibinfo{volume}{519}},
  \bibinfo{pages}{4753} (\bibinfo{year}{2023}), \eprint{2211.01389}.

\bibitem[{\citenamefont{{Cai} et~al.}(2023)\citenamefont{{Cai}, {Tang}, {Mo},
  {Yan}, {Chen}, {Ma}, {Wang}, {Luo}, {Easson}, and
  {Marciano}}}]{2023arXiv230109403C}
\bibinfo{author}{\bibfnamefont{Y.-F.} \bibnamefont{{Cai}}},
  \bibinfo{author}{\bibfnamefont{C.}~\bibnamefont{{Tang}}},
  \bibinfo{author}{\bibfnamefont{G.}~\bibnamefont{{Mo}}},
  \bibinfo{author}{\bibfnamefont{S.}~\bibnamefont{{Yan}}},
  \bibinfo{author}{\bibfnamefont{C.}~\bibnamefont{{Chen}}},
  \bibinfo{author}{\bibfnamefont{X.}~\bibnamefont{{Ma}}},
  \bibinfo{author}{\bibfnamefont{B.}~\bibnamefont{{Wang}}},
  \bibinfo{author}{\bibfnamefont{W.}~\bibnamefont{{Luo}}},
  \bibinfo{author}{\bibfnamefont{D.}~\bibnamefont{{Easson}}}, \bibnamefont{and}
  \bibinfo{author}{\bibfnamefont{A.}~\bibnamefont{{Marciano}}},
  \bibinfo{journal}{arXiv e-prints} \bibinfo{eid}{arXiv:2301.09403}
  (\bibinfo{year}{2023}), \eprint{2301.09403}.

\bibitem[{\citenamefont{De~Luca
  et~al.}(2020{\natexlab{a}})\citenamefont{De~Luca, Franciolini, Pani, and
  Riotto}}]{DeLuca:2020fpg}
\bibinfo{author}{\bibfnamefont{V.}~\bibnamefont{De~Luca}},
  \bibinfo{author}{\bibfnamefont{G.}~\bibnamefont{Franciolini}},
  \bibinfo{author}{\bibfnamefont{P.}~\bibnamefont{Pani}}, \bibnamefont{and}
  \bibinfo{author}{\bibfnamefont{A.}~\bibnamefont{Riotto}},
  \bibinfo{journal}{Phys. Rev. D} \textbf{\bibinfo{volume}{102}},
  \bibinfo{pages}{043505} (\bibinfo{year}{2020}{\natexlab{a}}),
  \eprint{2003.12589}.

\bibitem[{\citenamefont{Serpico et~al.}(2020)\citenamefont{Serpico, Poulin,
  Inman, and Kohri}}]{Serpico:2020ehh}
\bibinfo{author}{\bibfnamefont{P.~D.} \bibnamefont{Serpico}},
  \bibinfo{author}{\bibfnamefont{V.}~\bibnamefont{Poulin}},
  \bibinfo{author}{\bibfnamefont{D.}~\bibnamefont{Inman}}, \bibnamefont{and}
  \bibinfo{author}{\bibfnamefont{K.}~\bibnamefont{Kohri}},
  \bibinfo{journal}{Phys. Rev. Res.} \textbf{\bibinfo{volume}{2}},
  \bibinfo{pages}{023204} (\bibinfo{year}{2020}), \eprint{2002.10771}.

\bibitem[{\citenamefont{Mack et~al.}(2007)\citenamefont{Mack, Ostriker, and
  Ricotti}}]{Mack:2006gz}
\bibinfo{author}{\bibfnamefont{K.~J.} \bibnamefont{Mack}},
  \bibinfo{author}{\bibfnamefont{J.~P.} \bibnamefont{Ostriker}},
  \bibnamefont{and} \bibinfo{author}{\bibfnamefont{M.}~\bibnamefont{Ricotti}},
  \bibinfo{journal}{Astrophys. J.} \textbf{\bibinfo{volume}{665}},
  \bibinfo{pages}{1277} (\bibinfo{year}{2007}), \eprint{astro-ph/0608642}.

\bibitem[{\citenamefont{Ricotti}(2007)}]{Ricotti:2007jk}
\bibinfo{author}{\bibfnamefont{M.}~\bibnamefont{Ricotti}},
  \bibinfo{journal}{Astrophys. J.} \textbf{\bibinfo{volume}{662}},
  \bibinfo{pages}{53} (\bibinfo{year}{2007}), \eprint{0706.0864}.

\bibitem[{\citenamefont{Ricotti et~al.}(2008)\citenamefont{Ricotti, Ostriker,
  and Mack}}]{Ricotti:2007au}
\bibinfo{author}{\bibfnamefont{M.}~\bibnamefont{Ricotti}},
  \bibinfo{author}{\bibfnamefont{J.~P.} \bibnamefont{Ostriker}},
  \bibnamefont{and} \bibinfo{author}{\bibfnamefont{K.~J.} \bibnamefont{Mack}},
  \bibinfo{journal}{Astrophys. J.} \textbf{\bibinfo{volume}{680}},
  \bibinfo{pages}{829} (\bibinfo{year}{2008}), \eprint{0709.0524}.

\bibitem[{\citenamefont{Ali-Ha\"\i{}moud and
  Kamionkowski}(2017)}]{Ali-Haimoud:2016mbv}
\bibinfo{author}{\bibfnamefont{Y.}~\bibnamefont{Ali-Ha\"\i{}moud}}
  \bibnamefont{and}
  \bibinfo{author}{\bibfnamefont{M.}~\bibnamefont{Kamionkowski}},
  \bibinfo{journal}{Phys. Rev. D} \textbf{\bibinfo{volume}{95}},
  \bibinfo{pages}{043534} (\bibinfo{year}{2017}), \eprint{1612.05644}.

\bibitem[{\citenamefont{Greene et~al.}(2020)\citenamefont{Greene, Strader, and
  Ho}}]{Greene:2019vlv}
\bibinfo{author}{\bibfnamefont{J.~E.} \bibnamefont{Greene}},
  \bibinfo{author}{\bibfnamefont{J.}~\bibnamefont{Strader}}, \bibnamefont{and}
  \bibinfo{author}{\bibfnamefont{L.~C.} \bibnamefont{Ho}},
  \bibinfo{journal}{Ann. Rev. Astron. Astrophys.}
  \textbf{\bibinfo{volume}{58}}, \bibinfo{pages}{257} (\bibinfo{year}{2020}),
  \eprint{1911.09678}.

\bibitem[{\citenamefont{Carr et~al.}(2021)\citenamefont{Carr, Kohri, Sendouda,
  and Yokoyama}}]{Carr:2020gox}
\bibinfo{author}{\bibfnamefont{B.}~\bibnamefont{Carr}},
  \bibinfo{author}{\bibfnamefont{K.}~\bibnamefont{Kohri}},
  \bibinfo{author}{\bibfnamefont{Y.}~\bibnamefont{Sendouda}}, \bibnamefont{and}
  \bibinfo{author}{\bibfnamefont{J.}~\bibnamefont{Yokoyama}},
  \bibinfo{journal}{Rept. Prog. Phys.} \textbf{\bibinfo{volume}{84}},
  \bibinfo{pages}{116902} (\bibinfo{year}{2021}), \eprint{2002.12778}.

\bibitem[{\citenamefont{Carr and Kuhnel}(2020)}]{Carr:2020xqk}
\bibinfo{author}{\bibfnamefont{B.}~\bibnamefont{Carr}} \bibnamefont{and}
  \bibinfo{author}{\bibfnamefont{F.}~\bibnamefont{Kuhnel}},
  \bibinfo{journal}{Ann. Rev. Nucl. Part. Sci.} \textbf{\bibinfo{volume}{70}},
  \bibinfo{pages}{355} (\bibinfo{year}{2020}), \eprint{2006.02838}.

\bibitem[{\citenamefont{Thrane and Talbot}(2019)}]{Thrane:2018qnx}
\bibinfo{author}{\bibfnamefont{E.}~\bibnamefont{Thrane}} \bibnamefont{and}
  \bibinfo{author}{\bibfnamefont{C.}~\bibnamefont{Talbot}},
  \bibinfo{journal}{Publ. Astron. Soc. Austral.} \textbf{\bibinfo{volume}{36}},
  \bibinfo{pages}{e010} (\bibinfo{year}{2019}), \bibinfo{note}{[Erratum:
  Publ.Astron.Soc.Austral. 37, e036 (2020)]}, \eprint{1809.02293}.

\bibitem[{\citenamefont{Jeffreys}(1998)}]{jeffreys1998theory}
\bibinfo{author}{\bibfnamefont{H.}~\bibnamefont{Jeffreys}},
  \emph{\bibinfo{title}{The theory of probability}} (\bibinfo{publisher}{OUP
  Oxford}, \bibinfo{year}{1998}).

\bibitem[{\citenamefont{Akaike}(1974)}]{Akaike1974ANL}
\bibinfo{author}{\bibfnamefont{H.}~\bibnamefont{Akaike}},
  \bibinfo{journal}{IEEE Transactions on Automatic Control}
  \textbf{\bibinfo{volume}{19}}, \bibinfo{pages}{716} (\bibinfo{year}{1974}).

\bibitem[{\citenamefont{{Furtak} et~al.}(2022)\citenamefont{{Furtak}, {Zitrin},
  {Plat}, {Fujimoto}, {Wang}, {Nelson}, {Labb{\'e}}, {Bezanson}, {Brammer},
  {van Dokkum} et~al.}}]{Lukas:2022agn}
\bibinfo{author}{\bibfnamefont{L.~J.} \bibnamefont{{Furtak}}},
  \bibinfo{author}{\bibfnamefont{A.}~\bibnamefont{{Zitrin}}},
  \bibinfo{author}{\bibfnamefont{A.}~\bibnamefont{{Plat}}},
  \bibinfo{author}{\bibfnamefont{S.}~\bibnamefont{{Fujimoto}}},
  \bibinfo{author}{\bibfnamefont{B.}~\bibnamefont{{Wang}}},
  \bibinfo{author}{\bibfnamefont{E.~J.} \bibnamefont{{Nelson}}},
  \bibinfo{author}{\bibfnamefont{I.}~\bibnamefont{{Labb{\'e}}}},
  \bibinfo{author}{\bibfnamefont{R.}~\bibnamefont{{Bezanson}}},
  \bibinfo{author}{\bibfnamefont{G.~B.} \bibnamefont{{Brammer}}},
  \bibinfo{author}{\bibfnamefont{P.}~\bibnamefont{{van Dokkum}}},
  \bibnamefont{et~al.}, \bibinfo{journal}{arXiv e-prints}
  \bibinfo{eid}{arXiv:2212.10531} (\bibinfo{year}{2022}), \eprint{2212.10531}.

\bibitem[{\citenamefont{Abel et~al.}(2002)\citenamefont{Abel, Bryan, and
  Norman}}]{Abel:2001pr}
\bibinfo{author}{\bibfnamefont{T.}~\bibnamefont{Abel}},
  \bibinfo{author}{\bibfnamefont{G.~L.} \bibnamefont{Bryan}}, \bibnamefont{and}
  \bibinfo{author}{\bibfnamefont{M.~L.} \bibnamefont{Norman}},
  \bibinfo{journal}{Science} \textbf{\bibinfo{volume}{295}},
  \bibinfo{pages}{93} (\bibinfo{year}{2002}), \eprint{astro-ph/0112088}.

\bibitem[{\citenamefont{Heger et~al.}(2003)\citenamefont{Heger, Fryer, Woosley,
  Langer, and Hartmann}}]{Heger:2002by}
\bibinfo{author}{\bibfnamefont{A.}~\bibnamefont{Heger}},
  \bibinfo{author}{\bibfnamefont{C.~L.} \bibnamefont{Fryer}},
  \bibinfo{author}{\bibfnamefont{S.~E.} \bibnamefont{Woosley}},
  \bibinfo{author}{\bibfnamefont{N.}~\bibnamefont{Langer}}, \bibnamefont{and}
  \bibinfo{author}{\bibfnamefont{D.~H.} \bibnamefont{Hartmann}},
  \bibinfo{journal}{Astrophys. J.} \textbf{\bibinfo{volume}{591}},
  \bibinfo{pages}{288} (\bibinfo{year}{2003}), \eprint{astro-ph/0212469}.

\bibitem[{\citenamefont{{Li} et~al.}(2023)\citenamefont{{Li}, {Wang}, {Tang},
  and {Fan}}}]{2023arXiv230302973L}
\bibinfo{author}{\bibfnamefont{Y.-J.} \bibnamefont{{Li}}},
  \bibinfo{author}{\bibfnamefont{Y.-Z.} \bibnamefont{{Wang}}},
  \bibinfo{author}{\bibfnamefont{S.-P.} \bibnamefont{{Tang}}},
  \bibnamefont{and} \bibinfo{author}{\bibfnamefont{Y.-Z.} \bibnamefont{{Fan}}},
  \bibinfo{journal}{arXiv e-prints} \bibinfo{eid}{arXiv:2303.02973}
  (\bibinfo{year}{2023}), \eprint{2303.02973}.

\bibitem[{\citenamefont{De~Luca
  et~al.}(2020{\natexlab{b}})\citenamefont{De~Luca, Franciolini, Pani, and
  Riotto}}]{DeLuca:2020bjf}
\bibinfo{author}{\bibfnamefont{V.}~\bibnamefont{De~Luca}},
  \bibinfo{author}{\bibfnamefont{G.}~\bibnamefont{Franciolini}},
  \bibinfo{author}{\bibfnamefont{P.}~\bibnamefont{Pani}}, \bibnamefont{and}
  \bibinfo{author}{\bibfnamefont{A.}~\bibnamefont{Riotto}},
  \bibinfo{journal}{JCAP} \textbf{\bibinfo{volume}{04}}, \bibinfo{pages}{052}
  (\bibinfo{year}{2020}{\natexlab{b}}), \eprint{2003.02778}.

\end{thebibliography}




\section*{Supplementary Material}

{\bf PBH Accretion} The accretion of baryonic matter onto PBHs is believed to have played an important role in the early Universe, leading to the formation of seed black holes in the centers of early galaxies.
Additionally, the presence of PBHs during the epoch of reionization may have had significant implications for the thermal and ionization history of the Universe, with accretion potentially producing high energy radiation and particles that affect gas interactions.

\begin{figure}[htbp] 
\centering 
\includegraphics[width=0.96\linewidth]{mass_accretion.pdf} 
\caption{The accretion rate parameter $\dot{m}$ as a function of the final mass of PBH $M_{\rm BH, f}$ and redshift $z$. The orange and magenta points in the figure represent galaxy candidates identified by Labbe Ivo, et al~\cite{2022arXiv220712446L} and Atek.H et al~\cite{2023MNRAS.519.1201A}, respectively, with a relation of $M_{\star} - M_{\rm BH, f}$. The figure also shows trajectories of individual PBHs in the $(M, z)$ plane, represented by black dotted lines. The cutoff redshift $z_{\rm cut-off}$ is taken as 8 in this work.  } 
\label{pbh_mass_evolution}
\end{figure}


To model PBH accretion, we use a widely discussed analytic accretion model developed by Ricotti and Ostriker.
The Bondi-Hoyle rate of accretion can be expressed as $\dot{M}_{\mathrm{BH}}=4 \pi \lambda m_{H} n_{\mathrm{gas}} v_{\mathrm{eff}} r_{\mathrm{B}}^{2}$, where $r_B = GM/v_{\rm eff}^2$ is the Bondi-Hoyle radius, $n_{\rm gas}$ is the hydrogen gas number density, and $v_{\rm eff}=\sqrt{v_{\rm rel}^2 + c_s^2}$ is the PBH effective velocity, which is defined in terms of the PBH relative velocity $v_{\rm rel}$ with regard to the gas with sound speed $c_s$. The accretion parameter $\lambda$ accounts for the gas viscosity, the Hubble expansion, and the Compton scattering between the gas and the CMB.

In Figure~\ref{pbh_mass_evolution}, we present the accretion rate $\dot{m}$ as a function of PBH mass and redshift, with orange and magenta points representing sources identified by Labbe Ivo, et al~\cite{2022arXiv220712446L} and Atek.H et al~\cite{2023MNRAS.519.1201A}, respectively, via the relationship of $M_{\star} - M_{\rm BH, f}$.
The black dotted lines show the trajectories of PBH evolution with  $4M_{\odot}$, $10M_{\odot}$, $20M_{\odot}$, $40M_{\odot}$, $60M_{\odot}$, $100M_{\odot}$, $150M_{\odot}$, $200M_{\odot}$, $250M_{\odot}$ and $300M_{\odot}$ in the $(M, z)$ plane. We set $z_{\rm cut-off}$ to 8, and similar accretion rate can also be found in Figure 4 of Valerio De Luca et al~\cite{DeLuca:2020bjf}. 

%{\bf Mass Relationship Between SMBH and Galaxy.}
%In the local Universe, a historical trend of common growth between SMBHs and galaxies has been observed~\cite{Greene:2019vlv}. This co-evolution scenario is thought to be driven by feedback from active galactic nuclei (AGNs) and the initial mass density of SMBHs and galaxies. However, due to a lack of high-redshift measurements, we use the value reported by Greene et al~\cite{Greene:2019vlv} as a theoretical extrapolation. 


{\bf Observation Data} The JWST, launched on December 25, 2021, is the most powerful space telescope currently in operation, equipped with four instruments: Mid-Infrared Instrument (MIRI), Near-Infrared Spectrograph (NIRSpec), Near-Infrared Camera (NIRCam) and Fine Guidance Sensor/Near Infrared Imager and Slitless Spectrograph (FGS-NIRISS).
The Early Release Observations (EROs) of JWST included the observation of massive strong lensing cluster SMACSJ0723, and the spectra of its galaxies revealed both Ly$\alpha$ and Balmer break, providing information on their masses and redshifts through Spectral Energy Distribution (SED) fitting. Atek et al~\cite{2023MNRAS.519.1201A} identified 10 massive galaxy candidates at $z>9$ in the JWST-SMACS0723 deep field using this data.
In addition, the Cosmic Evolution Early Release Science (CEERS) program obtained multi-band images with NIRCam in a deep field covering an area ~$40$ $\rm arcmin^{2}$. Labbe et al~\cite{2022arXiv220712446L} utilized these data to identify 13 massive high-redshift galaxy candidates through SED fitting method. The details of these galaxy candidates are listed in Table~\ref{data_table}.  


\begin{table*}
\begin{center}
      {\small
      \begin{tabular}{lcccc}
            \hline
            \noalign{\smallskip}
             ID & RA & Dec  & $z_{\rm phot}$  & ${\rm log}(M_\star/M_\odot)$ \\
            \noalign{\smallskip}
            \hline
            \noalign{\smallskip}
            \multicolumn{5}{c}{$CEERS$ candidates~\cite{2022arXiv220712446L}}\\ \hline 
            \noalign{\smallskip}
            $2859$ & 14:19:21.728 & +52:49:04.592 & $8.11^{+0.90}_{-2.74}$ & $10.03^{+0.52}_{-0.80}$ \\
            $7274$ & 14:19:13.601 & +52:50:16.086 & $7.77^{+0.27}_{-2.15}$ & $9.87^{+0.31}_{-1.36}$ \\
            $11184$ & 14:19:34.194 & +52:51:24.811 & $7.72^{+0.47}_{-0.58}$ & $10.18^{+0.43}_{-0.44}$ \\
            $13050$ & 14:19:14.197 & +52:52:06.533 & $8.14^{+2.49}_{-2.89}$ & $10.14^{+0.54}_{-0.62}$ \\
            $14924$ & 14:19:30.276 & +52:52:50.997 & $8.83^{+0.69}_{-3.22}$ & $10.02^{+0.91}_{-1.64}$ \\
            $16624$ & 14:19:22.745 & +52:53:31.589 & $8.52^{+0.50}_{-0.83}$ & $9.30^{+0.77}_{-0.90}$ \\
            $21834$ & 14:19:36.534 & +52:56:21.731 & $8.54^{+1.55}_{-2.96}$ & $9.61^{+0.55}_{-1.53}$ \\
            $25666$ & 14:19:49.641 & +52:58:23.350 & $7.93^{+0.25}_{-2.33}$ & $9.52^{+0.57}_{-1.17}$ \\
            $28984$ & 14:20:00.682 & +53:00:27.339 & $7.54^{+1.25}_{-1.98}$ & $9.57^{+0.49}_{-0.45}$ \\
            $35300$ & 14:19:19.359 & +52:53:15.996 & $9.08^{+0.51}_{-3.52}$ & $10.40^{+0.63}_{-2.12}$ \\
            $37888$ & 14:19:39.002 & +52:56:57.968 & $6.51^{+2.12}_{-0.94}$ & $9.23^{+0.95}_{-1.17}$ \\
            $38094$ & 14:19:55.925 & +52:57:21.596 & $7.48^{+0.74}_{-0.56}$ & $10.89^{+0.23}_{-1.99}$ \\
            $39575$ & 14:20:01.296 & +52:59:48.142 & $8.62^{+0.56}_{-2.57}$ & $9.33^{+0.81}_{-1.18}$ \\ \hline
            \noalign{\smallskip}   
            \multicolumn{5}{c}{$SMACS$ candidates~\cite{2023MNRAS.519.1201A}}\\ \hline 
            \noalign{\smallskip}
            $SMACS_z10a$ & 7:23:26.252 & -73:26:56.940 & $9.78^{+0.02}_{-0.02}$ & $9.11^{+0.07}_{-0.07}$ \\
            $SMACS_z10b$ & 7:23:22.709 & -73:26:06.183 & $8.88^{+0.02}_{-0.02}$ & $10.20^{+0.03}_{-0.03}$ \\
            $SMACS_z10c$ & 7:23:20.169 & -73:26:04.233 & $9.77^{+0.02}_{-0.02}$ & $9.53^{+0.02}_{-0.02}$ \\
            $SMACS_z10d$ & 7:22:46.696 & -73:28:40.898 & $9.31^{+0.06}_{-0.08}$ & $7.77^{+0.14}_{-0.11}$ \\
            $SMACS_z10e$ & 7:22:45.304 & -73:29:30.557 & $10.89^{+0.16}_{-0.14}$ & $8.51^{+0.22}_{-0.16}$\\
            $SMACS_z11a$ & 7:22:39.505 & -73:29:40.224 & $11.05^{+0.09}_{-0.08}$ & $8.77^{+0.17}_{-0.24}$\\
            $SMACS_z12a$ & 7:22:47.380 & -73:30:01.785 & $12.20^{+0.21}_{-0.12}$ & $8.14^{+0.21}_{-0.17}$\\
            $SMACS_z12b$ & 7:22:52.261 & -73:27:55.497 & $12.26^{+0.17}_{-0.16}$ & $7.91^{+0.26}_{-0.17}$\\
            $SMACS_z16a$ & 7:23:26.393 & -73:28:04.561 & $15.92^{+0.17}_{-0.15}$ & $8.79^{+0.32}_{-0.33}$\\
            $SMACS_z16b$ & 7:22:39.439 & -73:30:08.185 & $15.32^{+0.16}_{-0.13}$ & $8.80^{+0.44}_{-0.25}$\\ 
           \noalign{\smallskip}
           \hline
           \hline
           \noalign{\smallskip} \noalign{\smallskip}
         \end{tabular} 
         }
\caption[]{\label{data_table} The data set of high-redshift galaxy candidates observed by JWST in Figure~\ref{contourf}. The CEERS candidates are from Labbe et al.~\cite{2022arXiv220712446L} and the SMACS candidates behind SMACS 0723-73 are identified by Atek et al~\cite{2023MNRAS.519.1201A}.}
\end{center}
\end{table*}



{\bf The Posterior Distribution} The posteriors of the hyper-parameters describing the initial mass distributions of PBHs (Logormal/Gaussian/Multipeak/Critical), as introduced in the main text, are displayed in Figure~\ref{four_corner}, respectively.

\begin{figure*}[htbp]
\centering
\includegraphics[width=0.49\textwidth]{Lognormal_corner.png}
\includegraphics[width=0.49\textwidth]{Gaussian_corner.png}
\includegraphics[width=0.49\textwidth]{Multipeak_corner.png}
\includegraphics[width=0.49\textwidth]{Critical_corner.png}
\caption{The posterior distributions of individual parameters for four different population models: Lognormal, Gaussian, Multipeak, and Critical. The upper left and right panels show the results for the Lognormal and Gaussian models, respectively. While the lower left and right panels refer to the Multipeak and Critical models. All models include the ionization fraction of the cosmic gas $x_e$, and we use the parameters listed in Table~\ref{evidence_table} with Uniform priors. The 68\% confidence intervals for each parameter are reported above each column, corresponding to the models shown in the inset.}
\label{four_corner}
\end{figure*}

\end{document}
