%\documentclass[aps,prl,preprint,groupedaddress]{revtex4-1}
\documentclass[prb,twocolumn,showpacs,preprintnumbers,superscriptaddress,amsmath,amssymb]{revtex4}
\usepackage{graphicx}
\usepackage{float}
\usepackage{amsmath}
\usepackage{subfigure}
\usepackage{braket}
\usepackage{color}
\usepackage{afterpage}
\usepackage[dvipsnames]{xcolor}
\usepackage{epstopdf}
\usepackage[normalem]{ulem}
\usepackage{ulem}
\definecolor{ao(english)}{rgb}{0.0, 0.42, 0.24}
\newcommand{\COMMENT}[1]{\textcolor{red}{[Comment: {#1}]}}
\newcommand{\REPLY}[1]{\textcolor{green}{[Renxi: {#1 }]}}
\newcommand{\cc}[1]{$\mathrm{g/cm^3}$}
\newcommand{\don}[1]{$\mathrm{n_{don}}$}
\newcommand{\acc}[1]{$\mathrm{n_{acc}}$}
\newcommand{\MC}[1]{\textcolor{black}{{#1}}}
\newcommand{\MCC}[1]{\textcolor{blue}{{#1}}}
\newcommand{\RX}[1]{\textcolor{orange}{{#1}}}
\newcommand{\MODIFY}[1]{\textcolor{brown}{{#1}}}
\begin{document}

\title{Characterization of the Hydrogen-Bond Network in High-Pressure Water by Deep Potential Molecular Dynamics}
\author{Renxi Liu}
\affiliation{HEDPS, CAPT, College of Engineering, Peking University, Beijing, 100871, P. R. China}
\affiliation{Academy for Advanced Interdisciplinary Studies, Peking University, Beijing, 90871, P. R. China}
\author{Mohan Chen}
\thanks{$^\ast$Corresponding author. Email:mohanchen@pku.edu.cn}
\affiliation{HEDPS, CAPT, College of Engineering, Peking University, Beijing, 100871, P. R. China}
\affiliation{Academy for Advanced Interdisciplinary Studies, Peking University, Beijing, 90871, P. R. China}
\affiliation{AI for Science Institute, Beijing 100080, P. R. China}
\date{\today}

\begin{abstract}
%The liquid water under pressure up to 1 GPa is believed to preserve tetrahedral hydrogen-bond structure, 
%with outer-shell water molecules pushed into the left voids.
%
%The orderly hydrogen-bonded first shell and the the water molecules pushed in disobeying tetrahedral structure forms a self-contradictory dilemma.
%
%Our molecular simulation shows that 
%the two-coordination hydrogen bond structure at the accepting end is partly corrupted by the high pressure and the angular distribution is largely influenced.
%
%Besides, high pressure does not produce denser coordination structure in the first shell.
%
%The compression is mainly achieved by increasing the frequency of presence of non-hydrogen-bonded water molecules in the inner shell.
%In addition, we adopt the deep potential molecular dynamics to yield more converged results by simulating a large system. We investigate the structural distortions of water at high pressures.
%
%By decomposing the simulated static structure factor into atomic contributions,
%we show the deformation of static structure factor under pressure is due to the inward compression of water molecules in the interstitial region.
%
%We also find that pressure damages the hydrogen-bond (HB) structure by decreasing the directionality of HB and deforming the 2-coordination structure on the accepting end.
%
%Finally, we show that the decrease in the tetrahedral order under pressure is mainly due to the more frequent presence of one or two non-hydrogen-bonded water molecules in the inner shell.
The hydrogen-bond (H-bond) network of high-pressure water is investigated by neural-network-based molecular dynamics (MD) simulations with the first-principles accuracy. The static structure factors (SSFs) of water at three densities, i.e., 1, 1.115 and 1.24 \cc{} are directly evaluated from 512-water MD trajectories, which are in quantitative agreement with the experiments.
%
We propose a new method to decompose the computed SSF and identify the changes in SSF with respect to the changes in H-bond structures.
%
We find a larger water density results in a higher probability for one or two non-H-bonded water molecules to be inserted into the inner shell, explaining the changes in the tetrahedrality of water under pressure.
%
We predict that the structure of the accepting end of water molecules is more easily influenced by the pressure than the donating end. Our work sheds new light on explaining the SSF and H-bond properties in related fields. 
\end{abstract}
\pacs{61.25.Em, 61.20.Ja, 71.15.Pd, 82.30.Rs, 31.15.E-, 62.50.-p}
\maketitle

%---------------
% Paragraph 1
%---------------
Liquid water under high pressure has been the subject of intensive studies due to its importance in physics, chemistry, and life sciences.~\cite{06CSR-Daniel, 09B-French, 11JCP-Kang, 00L-Galli}
%
Among them, one particular focus is the distortion of the hydrogen bond (HB) network with respect to external pressure.
%
In the last two decades, 
a number of X-ray~\cite{94JCP-Okhulkov, 96MP-Radnai, 10B-Katayama, 12JML-Yamaguchi, 02JPCM-Eggert, 09B-Weck, 13PNAS-Sahle, 16JCP-Skinner} and neutron diffraction~\cite{00CP-Soper, 00L-Soper, 06L-Strassle} experiments have been conducted to determine the static structure factor (SSF) and radial distribution function (RDF)
of water under high pressures.
%
%RDF, which describes the relative abundance of atoms at a certain distance from the central atom, can be effectively calculated from SSF through Fourier transform.
%
Through many efforts,
the influence of pressure on the SSF of liquid water is generally converged as follows.
%
Below 1 GPa, as pressure increases, the first peak of SSF rises significantly. In contrast, the second peak decreases monotonically in height, resulting in the fact that the highest second peak of SSF at ambient pressure shrinks into the shoulder of the first peak at 1 GPa.~\cite{16JCP-Skinner, 10B-Katayama, 02JPCM-Eggert}
%
The trend continues to higher pressures,
resulting in the fusion of the first two peaks into a single first peak at 4 GPa.~\cite{09B-Weck, 10B-Katayama, 12JML-Yamaguchi}
%
Recent works attribute the changes in the SSF to the damage of the tetrahedral structure of the inner shell~\cite{12JML-Yamaguchi, 16JCP-Skinner}.

Meanwhile, the RDF of oxygen atoms $\mathrm{g_{OO}(r)}$ also undergoes substantial changes as the pressure increases.
%
While the position of the first peak, which stands for the hydrogen-bonded water molecules, remains almost unchanged as pressure rises up to 1 GPa,~\cite{10B-Katayama, 00L-Soper, 00CP-Soper} 
%
the right shoulder of the first peak rises monotonically, and the second peak decreases to a local minimum.
%
Despite the converged experimental results,~\cite{16JCP-Skinner, 13JCP-Skinner} the influence of the hydrogen-bond network on the high-pressure water is still inconclusive, for instance, the following questions arise: (i) How do the changes in the HB network affect the first two peaks of SSF in the high-pressure water?
%
%since the SSF/RDF contains only the radial distribution information, the anisotropy of 
%intermolecular structure of water caused by HBs is lacking from experiments. 
(ii) In terms of the number and the directionality of HBs, how does the pressure influence the accepting and donating ends of water molecules? 
%
(iii) How are the changes in the tetrahedral structure of water related to the HB network?
%


%Past researches employing empirical potential structural refinement (EPSR) method~\cite{96CP-Soper} for fitting the experimental data
%recognized such features as sign of second-shell neighbors pressed into the void space left by the first-shell neighbors while leaving the tetrahedral structure of the first shell basically unchanged.~\cite{00CP-Soper, 00L-Soper, 12JML-Yamaguchi}

\afterpage{\begin{figure*}
  \includegraphics[width=15cm]{Figure1.pdf}
  \caption{\MC{Workflow for calculations of static structure factors of liquid water at three densities (1, 1.115, 1.24 \cc{}). Training data are obtained from AIMD simulations of 64 water molecules.}
  (a-c) show the workflow of our calculation of static structure factor (SSF) under the three different pressures. 
  (d) shows the formula used to calculate the SSF, in which $f_i$ stands for the shape factor of atom $i$ in X-ray diffraction. For $\mathrm{S_{OO}(Q)}$ calculation, all $f_i$ are set to 1.
  (e-g) show the calculated SSFs compared with experiment results~\cite{13JCP-Skinner, 16JCP-Skinner, 12JML-Yamaguchi}, which in 1 and 1.115 \cc{} are $\mathrm{S_{OO}(Q)}$ and in 1.24 \cc{} is the total $\mathrm{S(Q)}$, since only total SSF in 1.24 \cc{} is provided in Ref.~\onlinecite{12JML-Yamaguchi}.} \label{workflow}
 \end{figure*}}


%---------------
% Paragraph 2
%---------------
From a theoretical perspective,  
molecular dynamics (MD) serves as an important tool to investigate liquid water, bridging the gap between theory and experimentally measured SSF/RDF.
%{\it ab initio} molecular dynamics (AIMD)~\cite{85L-CPMD} with first-principles accuracy serves as an important tool to investigate liquid water, bridging the gap between theory and experimentally measured SSF/RDF.
%
Past MD works~\cite{97JCP-Bagchi, 99E-Starr, 00L-Galli, 03E-Saitta, 06L-Strassle, 07E-Yan, 10JCP-Ikeda, 14JPCL-Fanetti} generally yielded that the typical tetrahedral HB structure is barely influenced by pressures under 1 GPa,
while the HB structure outside the first shell is changed and
occasionally one or two non-hydrogen-bonded water are pushed into the first shell.
%
In recent years, 
{\it ab initio} molecular dynamics (AIMD)~\cite{85L-CPMD} with first-principles accuracy is vastly applied in place of classical MD,
lifting the accuracy of simulation to a quantitative agreement with experiments.~\cite{16JCP-Skinner, 15PCCP-Imoto, 19JCP-Imoto, 19JPCB-Vondracek}
%
%AIMD also confirmed the physical picture of liquid water compression depicted by classical MD.
%However, this is a somewhat self-contradictory picture,
%since the pushed-in water molecule is also hydrogen-bonded to the central water by two to four layers of HB.~\cite{15PCCP-Imoto}
%
%This indicate that at least the second and the third layers of HBs have to be skewed to some extent to ensure these waters to be pushed in,
%but the directionality and the resulting tetrahedrality is well preserved under pressure.
%
Although AIMD exhibits prediction power to explain the pressure effects on liquid water to a large extent,~\cite{15PCCP-Imoto, 16JCP-Skinner}
the small system size and short trajectory adopted by AIMD could only yield SSF with insufficient accuracy.
%
Besides, within the framework of density functional theory (DFT),~\cite{65PR-Kohn, 64PR-Hohenberg} as the GGA-level XC functionals systematically overestimate the strength of the H-bonds and underestimate the density of water,
adding dispersion correction based on empirical parameters~\cite{10JCP-Grimme} has become a common practice in the modeling of water with GGA-level functionals,
thereby limiting the first-principles nature of AIMD.~\cite{09JPCB-Schmidt, 17PNAS-Chen, 11JCP-Wang, 14JCP-Distasio, 12JCP-Ma, 16PNAS-Morawietz}
%
In this regard, using AIMD to directly compute the SSF of liquid water remains a challenging issue.

%In recent years, both RPBE~\cite{96L-Perdew, 98L-Zhang} and revPBE~\cite{99B-Hammer} XC functionals with D3 dispersion correction~\cite{10JCP-Grimme} have been used in the AIMD modelling of water under pressure respectively up to 1 GPa and 360 MPa.~\cite{15PCCP-Imoto, 16JCP-Skinner}
%
%Both studies showed good agreement with experiment results,
%as well as predicting a nearly tetrahedral first shell structure densified by interstitial water inserted into the void space.
%
%Particularly, by decomposing the RDF, Imoto {\it et al.} showed that the water molecules inserted into the interstitial region are mainly consisted of the second to the fourth-shell neighbors.~\cite{15PCCP-Imoto}
%
%Despite the apparent merits of the above two studies,
%the high computational cost of AIMD has limited the possibility of producing accurate SSF that is directly comparable to experiment from it.
%
%Meanwhile, 
%how the HB network maintain tetrahedral structure while compressing interstitial water molecules into the tetrahedral void areas at the same time remains unclear.


%---------------
% Paragraph 3
%---------------
\MC{
Our work aims to directly compute the static structure factors of liquid water at high pressures with first-principles accuracy and elucidate the related H-bond structures in detail.
%and ambient temperature conditions from state-of-the-art computational tools with a high accuracy.
%
%By doing so, we hope to elucidate how pressure crushes the long-range order of HB network while preserving the first shell RDF at the same time.
%
We perform AIMD simulations with the meta-GGA exchange-correlation functional, SCAN,~\cite{15L-Sun}
which is known to satisfy all 17 restrictions on semi-local exchange-correlation functional and excellent for water properties.~\cite{16NC-Sun, 17PNAS-Chen, 18JCP-Zheng, 19B-Xu, 20PNAS-Sharkas, 21B-Xu, 22JCP-Liu}
%
For instance, the SCAN functional substantially improves the strong H-bonds description provided by GGA functionals such as PBE.
%
Meanwhile, the added intermediate-ranged van der Waals interactions in SCAN pull the second-shell water molecules closer to the interstitial area,
causing a denser and more disordered interstitial area around water molecules closer to the experimental data.~\cite{17PNAS-Chen, 18JCP-Zheng}
%
Consequently, the SCAN functional accurately predicts the H-bonded structure of liquid water.~\cite{17PNAS-Chen, 19B-Xu}
%
In this work, we perform three AIMD simulations of bulk water (64 water molecules) at ambient temperature with fixed densities of 1.0, 1.115, and 1.24 \cc{}, which correspond to water densities under ambient pressure, 360 MPa, and 1 GPa, respectively.~\cite{02JPCRD-Wagner}
}


%---------------
% Paragraph 4
%---------------
\MC{A converged SSF needs MD simulations with a large system and a long trajectory, which is typically limited due to the high computational costs of AIMD.
In this regard, we adopt the Deep Potential Molecular Dynamics (DPMD)~\cite{18CPC-Wang, 18L-Zhang, 18CCP-Han} to train neural-network-based potentials for water with different densities; the workflow is illustrated in Figs.~\ref{workflow}(a)-(c).
%
The resulting neural networks are several orders of magnitude faster than AIMD and scale linearly with the number of atoms.
%
In previous works, DPMD has been validated to learn AIMD accuracy from both SCAN~\cite{21B-Xu, 20PNAS-Gartner, 21L-Zhang} and PBE0-TS functionals.~\cite{18L-Zhang}
%
In this work, the training sets, including the atomic positions, forces, and energy, are obtained from AIMD simulations with the SCAN functional in a 64-water cell. 
%
The energy predicted by DPMD is in quantitative agreement with the AIMD result (1 meV/atom).~\cite{18L-Zhang}
%
%We use the periodic boundary conditions.
%
As a result, DPMD simulations are performed for 512 water molecules for 300 ps.
}


%---------------
% Paragraph 5
%---------------
\MC{
Fig.~\ref{workflow}(d) lists the formula to compute SSF (S(Q)), and the results for the three densities are shown in Figs.~\ref{workflow}(e-g).
%
We find the DP model quantitatively reproduces the experimental data for all three densities.~\cite{13JCP-Skinner, 16JCP-Skinner, 12JML-Yamaguchi}
}
%
As the pressure increases from ambient to 1 GPa,
the first peak of $\mathrm{S_{OO}(Q)}$ significantly rises while the second peak decreases into a shoulder.
%
Correspondingly, 
as the position and shape of the first peak of $\mathrm{g_{OO}(r)}$ generally remain the same,
the second peak under ambient pressure decreases substantially as pressure rises.
%
The first minimum in between rises to form a bump beside the first peak.
%
The third peak also moves inward significantly as pressure rises.
Furthermore, we find that the height change of the first two peaks of SSF has nothing to do with the hydrogen-bonded first shell 
but purely results from the inward movement of non-hydrogen-bonded interstitial water molecules.

\begin{figure*}
  \includegraphics[width=15cm]{Figure2.pdf}
  \caption{
  (a-c) Static structure factor (SSF) $\mathrm{S_{OO}(Q)}$, SSF contributed by hydrogen-bonded water molecule pairs, and SSF contributed by non-hydrogen-bonded water molecule pairs. (d-f) Radial distribution function (RDF) $\mathrm{g_{OO}(r)}$, RDF contributed by hydrogen-bonded water molecule pairs, and RDF contributed by non-hydrogen-bonded water molecule pairs. Water systems with a density of 1 \cc{}, 1.115 \cc{}, and 1.24 \cc{} are respectively colored in red, blue, and green.}\label{ssf}
 \end{figure*}

%---------------
% Paragraph 6
%---------------
\MC{To further reveal the relations between the HBs in water and the detailed changes in SSF under pressure,
we decompose $\mathrm{S_{OO}(Q)}$ into contributions of different water molecule pairs with the formula as follows
%~\cite{03E-Saitta, 15PCCP-Imoto, 16JCP-Skinner}
\begin{align}
\begin{split}
    S_{\mathrm{OO}}(\mathrm{Q}) &= \frac{1}{N}\sum_{j = 1}^{N}\sum_{k = 1}^{N}e^{i(\mathbf{r_j} - \mathbf{r_k})\cdot\mathbf{Q}} \\
    &= \frac{1}{N}\left\{N + \sum_{j = 1}^{N-1}\sum_{k = j+1}^{N}\Big[e^{i(\mathbf{r_j} - \mathbf{r_k})\cdot\mathbf{Q}} + e^{i(\mathbf{r_k} - \mathbf{r_j})\cdot\mathbf{Q}}\Big]\right\} \\
    &= \frac{1}{N}\left\{N + 2\sum_{j = 1}^{N-1}\sum_{k = j+1}^{N}\Big[\cos{(\mathbf{r_j} - \mathbf{r_k})\cdot\mathbf{Q}}\Big]\right\},
\end{split}
\end{align}
where $N$ stands for the number of atoms, $j$ and $k$ stand for the indices of atoms, and $\mathbf{r}$ stands for the position of atom.
%
Specifically, $\mathrm{S_{OO}(Q)}$ of water is decomposed into two components by classifying the atom pairs into hydrogen-bonded pair and non-hydrogen-bonded pairs, as displayed in Figs.~\ref{ssf}(b) and (c), respectively.
%
Notably, the HB criteria are chosen with the O-O distance less than 3.5~\AA~and the O-O-H angle smaller than $\mathrm{30^\circ}$;~\cite{96N-Luzar}
%
although new criteria for defining HBs have been proposed~\cite{15PCCP-Imoto, 10JPCB-Henchman, 14JCP-Gasparotto, 16JCTC-Gasparotto} for both ambient and high-pressure water systems, our conclusions are insensitive to the hydrogen-bond criteria, more details of which are shown in Supplemental Material Section III.}


%---------------
% Paragraph 7
%---------------
%
\MC{
We find the changes in $\mathrm{S_{OO}(Q)}$ across the three densities mainly come from the non-hydrogen-bonded water molecules.
%
On the one hand, Fig.~\ref{ssf}(b) shows that the three $\mathrm{S_{OO}(Q)}$ contributed by the hydrogen-bonded water molecules in the first solvation shell at different pressures are almost identical, 
indicating that $\mathrm{S_{OO}(Q)}$ is insensitive to the changes in the hydrogen-bonded water molecules when the density rises from 1 to 1.24 \cc{}.
%
On the other hand, the three $\mathrm{S_{OO}(Q)}$ contributed by the non-hydrogen-bonded water molecules, as illustrated in Fig.~\ref{ssf}(c), exhibit substantially different features especially when the wave vector $\mathrm{Q<4}$~\AA$^{-1}$.
%
In detail, the first peak in Fig.~\ref{ssf}(c) decreases slightly when the water density changes from 1.0 to 1.115 \cc{},
but increases from 1.115 to 1.24 \cc{}.
%
Meanwhile, the peak moves towards a larger wave vector $\mathrm{Q}$ when the density increases.
%
During such a rightward move, 
the peak gradually merges into the first peak of hydrogen-bonded $\mathrm{S_{OO}(Q)}$ at 2.7 $\mathrm{\AA^{-1}}$, as illustrated in Fig.~\ref{ssf}(b).
%
In addition, we observe that the decrease of the second peak at 2.9 $\mathrm{\AA^{-1}}$ in Fig.~\ref{ssf}(a) 
corresponds to the decrease of the small peak at the same wave vector in Fig.~\ref{ssf}(c), suggesting this experimental feature of $\mathrm{S_{OO}(Q)}$ is also related to the non-hydrogen-bonded water molecules.
%
In summary, the above changes in $\mathrm{S_{OO}(Q)}$ from both hydrogen-bonded and non-hydrogen-bonded water molecules fully explain the increase of the first peak and decrease of the second peak in $\mathrm{S_{OO}(Q)}$ (Fig.~\ref{ssf}(a)) when the density increases from 1.0 to 1.24 \cc{}.
%
}

%---------------
% Paragraph 8
%---------------
%
\MC{The features of $\mathrm{S_{OO}(Q)}$ in Fig.~\ref{ssf}(a) can be further explained by its Fourier transform counterpart, which is the radial distribution function $\mathrm{g_{OO}(r)}$ shown in Fig.~\ref{ssf}(d).
%
By adopting the same criterion for decomposing $\mathrm{S_{OO}(Q)}$, we decompose $\mathrm{g_{OO}(r)}$ into hydrogen-bonded and non-hydrogen-bonded terms and the results are shown in Figs.~\ref{ssf}(e) and (f).
%
The main peaks in Fig.~\ref{ssf}(e) are contributed by hydrogen-bonded water molecules at different densities. We observe the unaffected positions of the three peaks as the density changes from 1.0 to 1.24 \cc{}, which is due to the relative preservation of local HB order.~\cite{10JCP-Ikeda}
We also notice that the amplitude of the peak decreases slightly at a larger density, which is caused solely by the increase in water density but not affected by the hydrogen-bonded tetrahedral structure.
%\sout{
%However, the amplitude of the peak decreases slightly when the density increases, which is caused by the distortion of the hydrogen-bonded tetrahedral structure under pressure ({\it vide infra}).~\cite{12JML-Yamaguchi} 
%}
%\REPLY{The distortion of tetrahedral structure does not affect the height of the peak. The decrease of the first peak of $g_{OO}(r)$ is because RDF is the ratio between density of O atom in the spherical shell at certain distance r and density in general, which could be denoted as $\frac{\rho(r)}{\rho}$. Therefore, as general density $\rho$ increases but density at distance r $\rho(r)$ remains unchanged, the RDF decreases. If we do not divide the density $\rho$ from the $\rho(r)$, the first peaks of the RDF at the three densities would have the same height. In other words, the distortion of the tetrahedrality leaves no clue in the RDF $g_{OO}(r)$.}
%
Furthermore, as illustrated in Fig.~\ref{ssf}(f), we find the water molecules in the second and third shells move towards the inner shell under pressure, which is evidenced by the disappearance of the second peak at 4.6 $\mathrm{\AA}$, and the leftward movement of the third peak from 6.7 to 6.1 $\mathrm{\AA}$.
%
In particular, when the density increases from 1.0 to 1.24 \cc{}, the non-hydrogen-bonded water molecules in the second shell move towards the region of the first shell, filling the O-O void space, as evidenced by the emergence of the peak at 3.4 $\mathrm{\AA}$. As a result, a shoulder of the first peak emerges in $\mathrm{g_{OO}(r)}$ when the density is 1.24 \cc{}, as shown in Fig.~\ref{ssf}(d).
%
In conclusion, the features of $\mathrm{S_{OO}(Q)}$ under pressure are mainly affected by the movements of the non-hydrogen-bonded water molecules from the interstitial region and beyond towards the inner hydrogen-bonded tetrahedral structure.
}
%
%From the distance perspective,
%these changes could be attributed to the compression of %the interstitial water by pressure.

%---------------
% Paragraph 9
%---------------
%{\bf Tetrahedral Order Parameter} 
\MC{
The $\mathrm{S_{OO}(Q)}$ and $\mathrm{g_{OO}(r)}$ yield HB information only in the radial direction. Since more experimental information is limited, first-principles simulations without empirical parameters play an important role in obtaining more HB network information for water under pressure.
%
Earlier literature reported that the H-bonded first shell of water molecules is almost unaffected, and the average number of HBs is not significantly influenced.~\cite{15PCCP-Imoto} However, when the water density changes from 1 to 1.24 \cc{}, the tetrahedral structure has been substantially altered according to our simulations.
%
Fig.~\ref{tetra}(a) illustrates the distributions of the tetrahedral parameter $q$ at the three densities from DPMD trajectories.~\cite{98MP-Chau, 01N-Errington}
%
The parameter $q$ measures the local structural order of liquid water in the sense of tetrahedrality.~\cite{98MP-Chau, 01N-Errington}
%
%The result is consistent with $q$ distributions from earlier simulations~\cite{10JCP-Ikeda, 16JCP-Skinner},
%which attributed the new rising peak at around $q$=0.5 to the distortion of tetrahedral structure caused by the insertion of a `fifth water molecule' into the inner shell.~\cite{16JCP-Skinner}
%
In our results,
$q$ distribution possesses a skewed peak at $q$=0.8 and a shoulder at $q$=0.5 under ambient pressure.
%
The height of the shoulder keeps rising while the peak decreases as the density changes from 1 to 1.24 \cc{},
forming a platform in the distribution between $q$=0.5 and $q$=0.8 at 1.24 \cc{},
which is in consistent with earlier studies.~\cite{10JCP-Ikeda, 16JCP-Skinner}
}


%---------------
% Paragraph 10
%---------------
%
%Correspondingly, the average q also decreases from 0.67 at 1 \cc{} to 0.61 at 1.24 \cc{}.
%
\MC{
To further investigate the changes in the distribution of the tetrahedral parameter $q$, we decompose the distribution by the number of H-bonded water molecules (denoted as $\mathrm{n_{HB}}$) within the four nearest neighbors of a water molecule.
%
The distributions contributed by $\mathrm{n_{HB}=}$4, 3, and 2 are shown in Figs.~\ref{tetra}(b), (c), and (d), respectively.
%
We find that the distribution involving four H-bonds exhibits a single peak at $q$=0.8 that represents a typical skewed tetrahedral HB structure in liquid water,
while the $\mathrm{n_{HB}=3}$ distribution exhibits a platform between $q$=0.5 and $q$=0.8, and $\mathrm{n_{HB}=2}$ forms a single-peak distribution centered at $q$=0.5.
%
As pressure increases, 
the percentage of $\mathrm{n_{HB}=4}$ drops substantially from 52.2\% (1 \cc{}) to 37.1\% (1.24 \cc{}) while percentages of $\mathrm{n_{HB}=2}$ and 3 increase,
%from 11.7\% and 34.5\% to 17.6\% and 42.2\%,
resulting in the significant deformation of $q$.
}
%
%The more frequent presence of one or two non-hydrogen-bonded water molecules in the inner shell
%constitutes the major reason for the decrease in the local order of water under pressure.
%


%---------------------
% Figure 3
%---------------------
\begin{figure}
  \includegraphics[width=8.5cm]{Figure3.pdf}
  \caption{
  (a) Distribution of the tetrahedral parameter $q$ at 1 \cc{} (red), 1.115 \cc{} (blue) and 1.24 \cc{} (green). 
  %
  (b-d) Distributions of $q$ contributed by water molecules whose 4, 3 and 2 neighbors out of the four closest neighbors are H-bonded, denoted by $\mathrm{n_{HB}=4, 3, 2}$, respectively.
  %
  Inset in each subplot depicts a typical water with its four closest neighbors of the corresponding $\mathrm{n_{HB}}$.
  }\label{tetra}
 \end{figure}

%---------------
% Paragraph 11
%---------------
%
\MC{
Past studies attribute the decrease in the tetrahedrality to the insertion of one or two additional non-H-bonded water into the inner shell.~\cite{16JCP-Skinner}
%
However, our analysis shows that these structures already exist under ambient pressure.
%
Applying higher pressure does not create denser geometries at higher density but increases the probability of observing these structures.
%
Although a water molecule could accept or donate 1 to 3 hydrogen bonds in the liquid state,~\cite{11ACR-Agmon}
we only observe five main kinds of stable H-bond structures from the MD trajectories.
%could stably constitute more than 1\% in the total population,
Fig.~\ref{hb_stru}(a) shows the classification of the five main H-bond structures, where a hydrogen-bond structure that accepts $m$ hydrogen bonds and donates $n$ hydrogen bonds is denoted as `$m_{\mathrm{A}}n_{\mathrm{D}}$'~\cite{16JCTC-Gasparotto}. 
%
We notice that the pressure mainly affects the accepting end rather than the donating end of water molecules.
For example, when the density rises from 1 to 1.24 \cc{}, the percentage of water molecules accepting 2 HBs substantially decreases from 64.7\% to 58.8\%,
while the percentage of donating 2 HBs only slightly decreases from 78.8\% to 78.3\%.
%
As a result, Fig.~\ref{hb_stru}(a) shows that the percentage of the $2_{\mathrm{A}}2_{\mathrm{D}}$ configurations decreases from 54.2\% at 1 \cc{} to 48.7\% at 1.24 \cc{}; around 5.5\% of water molecules deviate from the tetrahedral HB structure by gaining or losing an accepting HB. Consequently, the percentage of the $1_{\mathrm{A}}2_{\mathrm{D}}$ and $3_{\mathrm{A}}2_{\mathrm{D}}$ configurations increases monotonically as pressure increases.}
%
%In the light of the finding that some non-tetrahedral HB structures are highly correlated spatially,~\cite{16JCTC-Gasparotto}
%the dynamics of the HB network could be characterized by the interconversion between different HB structures,~\cite{10JPCB-Henchman, 06S-Laage, 08JPCB-Laage}
%but that has gone beyond the scope of the current work.



%---------------------
% Figure 4
%---------------------

%\afterpage{
\begin{figure}[htbp]
  \includegraphics[width=7.5cm]{Figure4.pdf}
  \caption{
  (a) Percentages of five major stable HB structures observed in liquid water with densities being 1, 1.115, and 1.24 \cc{}.
  (b-d) Distributions of the $\mathrm{O_A}$-$\mathrm{O}$-$\mathrm{O_A}$ angle (dashed line) and $\mathrm{O_D}$-$\mathrm{O}$-$\mathrm{O_D}$ angle (solid line) of the $\mathrm{2_A2_D}$ structure at the three different densities.
  Vertical dashed and solid lines in corresponding colors mark the peak positions of the $\mathrm{O_A}$-$\mathrm{O}$-$\mathrm{O_A}$ and $\mathrm{O_D}$-$\mathrm{O}$-$\mathrm{O_D}$ angle distribution, while the grey dashed line mark 109$^{\circ}$28$^{\prime}$.
  The three insets are the spatial distribution function (SDF) of the $\mathrm{2_A2_D}$ structure.
  }\label{hb_stru}
 \end{figure}
%}

%---------------
% Paragraph 12
%---------------
%Fig.~\ref{sdf1} displays the first-shell oxygen-oxygen SDF of water molecules at three different densities in terms of the 5 major HB configurations. 
%

\MC{Pressure mainly affects the HB distribution of the accepting end of a water molecule rather than the donating end. Figs.~\ref{hb_stru}(b), (c), and (d) provide more details about the distributions of O-O-O angles for the tetrahdral $\mathrm{2_A2_D}$ water at a density of 1, 1.115, and 1.24 \cc{}, respectively. Here, the O-O-O angles are divided into two components as contributed by the accepting and donating ends. The statistics are based on the $\mathrm{2_A2_D}$ structure, which dominates HB structures at all three densities.
%
We further add in the O-O spatial distribution function (SDF), which provides a 3-dimensional view of the distribution of H-bonded O atoms close to the O of a water molecule.
%
At the donating end, we find the $\mathrm{O_D}$-$\mathrm{O}$-$\mathrm{O_D}$ angle distribution barely changes with pressure, as the peak position (vertical solid line in corresponding color) stays almost unchanged, close to the typical $\mathrm{109^{\circ}28^{'}}$ of tetrahedral structure (vertical grey line). As shown in the insets of Figs.~\ref{hb_stru}(b-d), the SPD forms two separate disk-like distributions and remains unchanged as pressure increases.
%
On the contrary, the $\mathrm{O_A}$-$\mathrm{O}$-$\mathrm{O_A}$ angle distribution moves leftward as pressure increases, the peak of which (dashed line in corresponding color) decreases from $\mathrm{93.6^{\circ}}$ to $\mathrm{83.0^{\circ}}$. Correspondingly, the inset of Fig.~\ref{hb_stru}(b) shows that the SPD at the accepting end exhibits two separate distributions at 1 \cc{}, but the two distributions spread more loosely and begin to merge at 1.115 \cc{}, and the trend becomes more clear at 1.24 \cc{}.
}


%---------------
% Paragraph 13
%---------------
\MC{
In conclusion, we systematically investigated the changes in H-bond structures of liquid water under pressure by performing neural-network-based DPMD simulations with the first-principles accuracy. The use of machine-learning-assisted DPMD models largely boosted the efficiency of AIMD, enabling us to simulate a large cell with 512 water molecules and a trajectory length of 300 ps. Besides, the use of the SCAN functional provided first-principles accuracy for liquid water without empirical parameters. As a result, we 
directly computed the SSF of water at different densities and proposed a new method that links the H-bond structure information to the decomposed SSF. 
%Based on the 300-ps long trajectories for 512 water molecules, we directly computed the SSF of water at three densities, i.e., 1, 1.115, and 1.24 \cc{}. 
%The resulting SSFs are in quantitative agreement with the experiments.
%
%We proposed a new method to decompose the computed SSF and identify how the changes in SSF relate to the changes in H-bond structures.
%
%By analyzing the representative H-bonded structures, we found a larger density resulted in a higher probability of observing one or two non-H-bonded water molecules inserted into the inner shell, which explained 
%the changes in the tetrahedrality of water under pressure.
%
%We predicted that the accepting end of water molecules was more easily influenced by the pressure than the donating end. In detail, the accepting end of the water molecule loses its directionality and the 2-coordination structure to a large extent.
%
The results and analyses provided in this work may help better understand the influences of pressures on the H-bond network of water. Furthermore, it can be readily applied to study similar scientific problems that involve H-bond information in the X-ray or neutron diffraction experiments.
}

$\\$
{\bf Acknowledgements}
$\\$
The work of R.L. and M.C. is supported by the National Science Foundation of China under Grant No. 12122401 and No. 12074007. All of the numerical simulations were performed on the High-Performance Computing Platform of CAPT and Bohrium platform supported by DP Technology. 
%and recognized that the changes in the low-frequency part of the SSF as pressure increases are due to the inward movement of interstitial water,
%while the relatively unchanged high-frequency part comes from the unchanged RDF of the first solvation shell.
%
%We investigated the spatial distribution of the first shell by the classification of the five major HB structures and their spatial distribution representations. 
%
%The analysis reveals that the pressure disturbance significantly reduces the directionality of HB on the O end,
%while the 2-coordination structure of the accepting end is partly destructed by pressure at 1 GPa.
%
%The increased pressure is also found to push one or two non-hydrogen-bonded water molecules into the inner shell more frequently than ambient state,
%leading to the decrease in the local tetrahedral order.
%
%We hope the results and analyses implemented herein could help better understand the influence of pressure on HB network and HB structural dependence phenomenon in related field.

%The combining trend of the two accepting HB distribution is particularly significant in the most abundant tetrahedral structure (\acc{}=2, \don{}=2),
%indicating that the pressure is gradually demolishing the HB structure by removing the directionality and 2-coordination structure on the accepting end,
%although the RDF and coordination number still remain largely unchanged.

%the accepting end of a water loses its directionality and the 2-coordination structure to a large extent.


%
%We further investigate the SDF of each HB structure %different densities.
%
%
%Comparatively, 
%
%As density increases,
%the HB distribution at accepting end with 2 accepted HB has a strong tendency to join in the middle, such as $2_{\mathrm{A}}2_{\mathrm{D}}$ and $2_{\mathrm{A}}1_{\mathrm{D}}$, as shown in Fig.~\ref{sdf1}(c)(h)(m) and (d)(i)(n)).
%
%A Similar trend can also be seen in $3_{\mathrm{A}}2_{\mathrm{D}}$, 
%where the distribution in the middle get apparently wider at higher density (Fig.~\ref{sdf1}(e)(j)(o)).
%
%In this sense, pressure up to 1 GPa substantially degrades the directionality of HB on the accepting end.
%


%{\bf Hydrogen Bond} 
%
%We find that the distribution of the accepting (donating) HB becomes more localized as water donates (accepts) more HB.
%
%The phenomenon indicates that the strength of donated (accepted) HB are dependent on the number of accepted (donated) HB,
%in consistency with the dependence of the number of donated (accepted) HB on number of accepted (donated) HB earlierly discovered.~\cite{08MP-Markovitch, 11ACR-Agmon}
%
%The distribution of the donating HB is also more localized when water donates two rather than one HB.
%
%Since the electronic structure of the water molecule could be effectively represented by maximally localized Wannier function (MLWF)~\cite{97B-Marzari, 12RMP-Marzari}, 
%we could compare the electronic structure of the major HB structures by examining their MLWF distributions.
%
%Fig.~\ref{MLWF} displays the distribution of distance of center of MLWF from the oxygen atom in each HB structure at 1 \cc{},
%where the left peak represents the lone pair and the right peak represents the bonding pair.
%
%We could see that the number of accepted (donated) HB could directly influence the distance from bonding (lone) pair to O atom.
%
%The bonding (lone) pair gets closer to the O atom as the water molecule accepts more (donated less) HB. 
%(for the bonding pair, compare the right peak position of $\mathrm{n_{don}=2}$ cases; for the lone pair, compare the left peak position of $\mathrm{n_{acc}=2}$ cases).
%
%The bonding pair is also closer to the O atom when the molecule donates two rather than one HB.
%
%This explains the dependence of the strength, directionality and number of accepted (donated) HB on the number of donated (accepted/donated) HB.
%
%Furthermore, this could also be the electronic structural reason for a stronger tendency to give away a proton when hydronium accepts an additional HB as well.~\cite{09L-Berkelbach, 15JCP-Tse}


\bibliography{reference}
\end{document}
