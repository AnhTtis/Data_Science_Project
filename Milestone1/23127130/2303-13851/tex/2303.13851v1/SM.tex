\documentclass[prb,showpacs,preprintnumbers,superscriptaddress,amsmath,amssymb,manuscript=article]{revtex4}
%\documentclass[prb,twocolumn,showpacs,preprintnumbers,superscriptaddress,amsmath,amssymb]{revtex4}
\usepackage{graphicx}
\usepackage{float}
\usepackage{amsmath}
\usepackage{subfigure}
\usepackage{braket}
\usepackage{color}
\usepackage[dvipsnames]{xcolor}
\usepackage{epstopdf}
\usepackage[normalem]{ulem}
\usepackage{ulem}
\definecolor{ao(english)}{rgb}{0.0, 0.42, 0.24}

\renewcommand{\thetable}{S\arabic{table}}
\renewcommand{\theequation}{S\arabic{equation}}
\renewcommand{\thefigure}{S\arabic{figure}}

\newcommand{\COMMENT}[1]{\textcolor{red}{[Comment: {#1}]}}
\newcommand{\REPLY}[1]{\textcolor{green}{[Renxi: {#1 }]}}
\newcommand{\cc}[1]{$\mathrm{g/cm^3}$}
\newcommand{\don}[1]{$\mathrm{n_{don}}$}
\newcommand{\acc}[1]{$\mathrm{n_{acc}}$}
\newcommand{\MC}[1]{\textcolor{blue}{{#1}}}
\newcommand{\RX}[1]{\textcolor{orange}{{#1}}}
\newcommand{\MODIFY}[1]{\textcolor{brown}{{#1}}}
\begin{document}

\title{Supplemental Material for ``Characterization of the Hydrogen-Bond Network in High-Pressure Water by Deep Potential Molecular Dynamics"}
\author{Renxi Liu}
\affiliation{HEDPS, CAPT, College of Engineering, Peking University, Beijing, 100871, P. R. China}
\affiliation{Academy for Advanced Interdisciplinary Studies, Peking University, Beijing, 90871, P. R. China}
\author{Mohan Chen}
\thanks{$^\ast$Corresponding author. Email:mohanchen@pku.edu.cn}
\affiliation{HEDPS, CAPT, College of Engineering, Peking University, Beijing, 100871, P. R. China}
\affiliation{Academy for Advanced Interdisciplinary Studies, Peking University, Beijing, 90871, P. R. China}
\affiliation{AI for Science Institute, Beijing 100080, P. R. China}
\date{\today}
\pacs{61.25.Em, 61.20.Ja, 71.15.Pd, 82.30.Rs, 31.15.E-, 62.50.-p}
\maketitle
\section{Accuracy of DPMD Models}

\begin{figure}[htp]
  \includegraphics[width=12cm]{FigureS1.pdf}
  \caption{
  Comparison of RDFs $\mathrm{g_{OO}(r)}$, $\mathrm{g_{OH}(r)}$ and $\mathrm{g_{HH}(r)}$ between AIMD (red) and DPMD (blue) at 1, 1.115, and 1.24 \cc{}.
  The three rows from top to bottom correspond to $\mathrm{g_{OO}(r)}$, $\mathrm{g_{OH}(r)}$ and $\mathrm{g_{HH}(r)}$, respectively.
  The three columns from left to right correspond to 1, 1.115, and 1.24 \cc{}, respectively.
  }\label{pdf_compare}
 \end{figure}


In order to verify the accuracy of the trained DPMD models for liquid water, we compare the radial distribution functions (RDFs) and angle distribution functions (ADFs) as obtained from the AIMD and DPMD trajectories.
%
Fig.~\ref{pdf_compare} shows the RDFs including $\mathrm{g_{OO}(r)}$, $\mathrm{g_{OH}(r)}$ and $\mathrm{g_{HH}(r)}$ at density of 1.0, 1.115 and 1.24 \cc{}. The results are obtained from AIMD and DPMD trajectories.
%
We observe that the RDFs evaluated from the two trajectories match well.
%
Fig.~\ref{adf_compare} illustrates the O-O-O ADF of liquid water at the three densities; 
%
here the O-O-O angle refers to the angle formed by an O atom in a water molecule in the center and two of its neighbors.
%
In order to identify the first peak in the RDF, we set the neighboring cutoff radius to 3.215, 3.115, and 3.035 $\mathrm{\AA}$ for the 1, 1.115, and 1.24 \cc{} systems, respectively.
%
In fact, we find the resulting O-O-O distribution is relatively insensitive to the choice of the cutoff.
%
The above tests suggest that the AIMD and DPMD trajectories yield quantitatively comparable results,
suggesting that the structural features from AIMD are well preserved through the training of the DPMD models.


\begin{figure}[htp]
  \includegraphics[width=14cm]{FigureS2.pdf}
  \caption{
  (a) Illustration of the O-O-O angle, where $\mathrm{O_1}$-$\mathrm{O}$ and $\mathrm{O_2}$-$\mathrm{O}$ distances are less than a certain $\mathrm{r_{cut}}$.
  %
  (b-d) Comparison of the angular distribution function of O-O-O angle between AIMD (red) and DPMD (blue) at 1, 1.115, and 1.24 \cc{}.
  }\label{adf_compare}
 \end{figure}

\section{Radial Distribution Functions and Static Structure Factors}
The manuscript has compared the computed RDFs and SSFs with the experiments in Fig.~1(e-g).~\cite{13JCP-Skinner, 16JCP-Skinner, 12JML-Yamaguchi}
The RDFs and SSFs data are obtained at three water densities (1, 1.115, and 1.24 \cc{}) from AIMD and DPMD trajectories.
%
At the densities of 1 and 1.115 \cc{}, the SSFs of $\mathrm{S_{OO}}$ are shown. However,
at the density of 1.24 \cc{}, the total SSF is shown and compared since $\mathrm{S_{OO}}$ is not provided in Ref.~\onlinecite{12JML-Yamaguchi}.

%
For all three densities, both RDF and SSF show reasonable agreement with the experiment.
We also notice that the first peak of $\mathrm{g_{OO}}(r)$ is slightly higher than the experimental results,
indicating a slightly over-binding hydrogen bond (HB) described by the SCAN functional.
%
To generate the full SSF of water,
the form factors of the O and H atom are set to be fractional to the electron number in accordance with Ref.~\onlinecite{12JML-Yamaguchi}, which is $\mathrm{f_O: f_H=8:1}$.


%\begin{figure}
%  \includegraphics[width=17cm]{FigureS3.pdf}
%  \caption{
%  (a-c) Comparison of PDF between our calculation (in red) and experiment at density of 1, 1.115, 1.24 $\mathrm{g/cm^3}$.
%  (b-d) Comparison of SSF between our calculation (in red) and experiment at density of 1, 1.115, 1.24 $\mathrm{g/cm^3}$.
%  The experiment results for 1, 1.115 and 1.24 $\mathrm{g/cm^3}$ come from Ref.~\onlinecite{13JCP-Skinner, 16JCP-Skinner, 12JML-Yamaguchi} respectively.
%  The SSF for 1.24 $\mathrm{g/cm^3}$ is the total SSF, rather than $\mathrm{S_OO}$, since only total SSF is presented in Ref.~\onlinecite{12JML-Yamaguchi}.
%  }\label{pdf_exp}
% \end{figure}

\section{Hydrogen Bond Criteria}

\begin{figure}[htp]
  \includegraphics[width=14cm]{FigureS3.pdf}
  \caption{
 (a) A sketch illustrating the two values, $\mathrm{r_{OO}}$ and $\mathrm{\theta_{HOO}}$ involved in the definition of HB.
 %
 (b) Distribution of $\mathrm{\theta_{HOO}}$ at densities of 1, 1.115, 1.24 $\mathrm{g/cm^3}$.
 %
 (c-e) Joint distribution of $\mathrm{r_{OO}}$ and $\mathrm{\theta_{HOO}}$ at densities of 1, 1.115, 1.24 $\mathrm{g/cm^3}$.
 The adopted HB definition in the paper and a looser definition are respectively marked with red and yellow lines.
  }\label{HB_def}
 \end{figure}

Different definitions of HB exist for both ambient water and water under pressure.
%
As elucidated in the manuscript and previous literature, ranging from ambient pressure to 1 GPa,
the influence of high pressure on hydrogen bonds remains limited. In this regard, we keep the classical HB classification standard to facilitate comparison with earlier literature.
%
This is particularly true considering that the discussion in the manuscript only involves structural changes rather than dynamic changes.
%
However, it is still necessary to examine the sensitivity of our conclusions against changes in the HB criterion.

%
Figs.~\ref{HB_def}(c-e) show the joint distribution of $\mathrm{\theta_{HOO}}$ angle and $\mathrm{r_{OO}}$ between two water molecules (Fig.~\ref{HB_def}(a)).
%
At all three densities,
the H-bonded water pairs clearly form a relatively independent single-peak distribution separate from non-H-bonded water at the lower left part,
indicating that the HB distribution is barely changed by pressure under 1 GPa.
%
The vertical line and horizontal line in red represent $\mathrm{r_{OO}=3.5~\AA}$ and $\mathrm{\theta_{HOO}=30^{\circ}}$, respectively. The two red lines encircle the classical H-bonded distribution area defined by Luzar et al.~\cite{96N-Luzar}
%
As shown by the joint distribution and $\mathrm{g_{OO}(r)}$ in Figs. 2(d-e) of the manuscript,
the radial cutoff of 3.5 $\mathrm{\AA}$ is a safe choice.
%
Nevertheless, the angular cutoff of 30$\mathrm{^{\circ}}$ on $\mathrm{\theta_{HOO}}$ could be a strict criterion and might oversee a few marginal HB cases, as shown in Fig.~\ref{HB_def}(b).


How to classify hydrogen bonds more accurately is another issue that requires different perspectives to justify. In this work, we choose to draw a line on the joint distribution as a loose upper bound of HB to include as many (marginal) HB cases as possible.
%
The line chosen here is $\mathrm{\theta_{HOO} (^{\circ})=-32 \times r_{OO} (\AA) + 134}$, marked as yellow broken line in Figs.~\ref{HB_def}(c-e).
%
In the following part, we repeat the analyses related to the HB standard in the manuscript with this looser HB standard to justify the robustness of the conclusions in the manuscript.
%
The looser HB standard used here would be referred to as the new HB standard in the following parts.




\begin{figure}[htp]
  \includegraphics[width=14cm]{FigureS4.pdf}
  \caption{
  (a-c) Static structure factor (SSF) $\mathrm{S_{OO}(q)}$, SSF contributed by hydrogen-bonded water molecule pairs and SSF contributed by non-hydrogen-bonded water molecule pairs. (d-f) Radial distribution function (RDF) $\mathrm{g_{OO}(r)}$, RDF contributed by hydrogen-bonded water molecule pairs, and RDF contributed by non-hydrogen-bonded water molecule pairs. Water systems with a density of 1 \cc{}, 1.115 \cc{}, and 1.24 \cc{} are respectively colored in red, blue, and green.
  The new HB standard (yellow broken line in Fig.~\ref{HB_def}) is used herein.
  }\label{pdf_ssf}
 \end{figure}


\begin{figure}[htp]
  \includegraphics[width=10cm]{FigureS5.pdf}
  \caption{
 (a) Distribution of tetrahedral parameter q of water under three different pressures.
 (b-d) Distributions of $q$ contributed by water molecules whose 4, 3, and 2 neighbors out of the four closest neighbors are H-bonded, denoted by $\mathrm{n_{HB}=4, 3, 2}$, respectively. 
The HBs are classified using the new standard here (yellow broken line in Fig.~\ref{HB_def}).
  %
  Inset in each subplot depicts a typical water with its four closest neighbors of the corresponding $\mathrm{n_{HB}}$.
  }\label{tetra_q}
 \end{figure}

We replotted Figs.~2-4 in the manuscript with the new HB standard,
which are shown as Figs.~\ref{pdf_ssf}-\ref{HB_stru_stat}.
%
First, for the one-to-one decomposition of RDF and SSF, the new HB standard yields decomposed results (Fig.~\ref{pdf_ssf}) that are basically the same as Fig.~2 in the manuscript,
so the SSF changes with respect to increasing pressure are undoubtedly related to changes in the structure of non-H-bonded water molecules.

%
Second, the tetrahedral order parameter $q$ distribution and its decomposition with respect to $\mathrm{n_{HB}}$ are shown in Fig.~\ref{tetra_q}.
%
With the new HB definition,
the change in the order parameter distribution could still be attributed to the decrease in the $\mathrm{n_{HB}=4}$ proportion and increase in the $\mathrm{n_{HB}=2, 3}$ proportion.
%
However, the distribution of $q$ of $\mathrm{n_{HB}=3, 4}$ also slightly changes as the HB standard changes.
%
The $q$ distribution of $\mathrm{n_{HB}=4}$ between 0.4 and 0.6 slightly increases, forming a bump therein at the densities of 1.115 and 1 $\mathrm{g/cm^3}$.
%
Meanwhile, distribution of $\mathrm{n_{HB}=3}$ between 0.4 and 0.6 decreases,
resulting in a single-peak shape in closed resemblance to $\mathrm{n_{HB}=2}$ distribution.
%
This indicates that the loose HB standard classifies a number of $\mathrm{n_{HB}=3}$ cases under the classical standard into the $\mathrm{n_{HB}=4}$ class.
%
The order parameter $q$ of newly added $\mathrm{n_{HB}=4}$ cases are mostly between 0.4 and 0.6,
indicating a poor tetrahedrality in such coordination.
%
The change is not surprising since the new standard classifies more marginal HB into HB, 
which are mostly on the boundary between H-bonded and non-H-bonded water pairs.
%
These marginal HB are very likely to belong to not-so-tetrahedral HB structures.


 \begin{figure}[htp]
  \includegraphics[width=8.5cm]{FigureS6.pdf}
  \caption{
 (a) Percentages of five major HB structures observed in liquid water with densities being 1, 1.115, and 1.24 \cc{} under the new HB standard.
  (b-d) Distributions of the $\mathrm{O_A}$-$\mathrm{O}$-$\mathrm{O_A}$ angle (broken line) and $\mathrm{O_D}$-$\mathrm{O}$-$\mathrm{O_D}$ angle (solid line) of the $\mathrm{2_A2_D}$ structure at the three different densities under the new HB standard.
  Vertical broken and solid lines in corresponding colors mark the peak positions of the $\mathrm{O_A}$-$\mathrm{O}$-$\mathrm{O_A}$ and $\mathrm{O_D}$-$\mathrm{O}$-$\mathrm{O_D}$ angle distribution, while the grey broken line mark 109$^{\circ}$28$^{\prime}$.}\label{HB_stru_stat}
 \end{figure}

%
Third, Fig.~\ref{HB_stru_stat} shows the content in Fig.~4 of the manuscript with the new HB standard.
%
For the HB structure distribution,
the percentage of $\mathrm{2_A2_D}$ drops monotonically as percentage of $\mathrm{3_A2_D}$ increases as pressure increases, 
same as the results from classical standards.
%
But the percentage of $\mathrm{1_A2_D}$ does not vary much as pressure increases under the new standard.
%
The new HB standard increases the percentage of $\mathrm{2_A2_D}$ from 54.2\%, 49.4\%, 48.7\% under the classical standard to 61.0\%, 57.2\%, and 55.7\%.
%
This dramatic change, 
together with the change in the shape of $\mathrm{n_{HB}=4}$ q distribution, 
indicates that a number of marginal HBs belong to $\mathrm{2_A2_D}$ structures with poor tetrahedrality.
%
Furthermore, the percentage of HB structure with 3 donated HB (not shown in Fig.~\ref{HB_stru_stat}) also increases from 0.5\%, 0.8\%, and 1\% under the classical standard to 1.6\%, 3.1\%, and 5.6\%.
%
Since most 3-donated HB cases are bifurcated HB from `HB jump'\cite{06S-Laage},
the increase shows that HB jump is another important source of marginal HB,
and the frequency of HB jump grows as pressure increases.
%
The increase in frequency might be a sign of the growing instability of donated HB as pressure increases.
%
As for the $\mathrm{O}$-$\mathrm{O}$-$\mathrm{O}$ angle distribution,
the new HB standard yields similar results to the classical standard:
$\mathrm{O_A}$-$\mathrm{O}$-$\mathrm{O_A}$ angle decreases as pressure increases while $\mathrm{O_D}$-$\mathrm{O}$-$\mathrm{O_D}$ angle distribution basically remains unchanged by the pressure.


In conclusion,
switching the HB standard to a looser one generally does not change the conclusions related to HB classification.
%
A looser HB bound also reveals a few new understandings concerning HB structure.
%
For example, the marginal HBs mostly come from highly distorted $\mathrm{2_A2_D}$ structure or the process of HB jump, i.e. $\mathrm{3_D}$ structures.
%
The increase in the $\mathrm{3_D}$ structures at a higher pressure indicates a growing frequency of HB jump as pressure increases,
which requires further study in the future.

 
  \begin{figure}[htp]
  \includegraphics[width=12cm]{FigureS7.pdf}
  \caption{
 (a-e) Spatial distribution function (SPD) of O atom in the first-shell water of the five major HB structures under ambient pressure.
 (f) Distribution of $\mathrm{O_D}$-$\mathrm{O}$-$\mathrm{O_D}$ angle of $\mathrm{1_A2_D}$, $\mathrm{2_A2_D}$ and $\mathrm{3_A2_D}$ HB structures, and $\mathrm{O_A}$-$\mathrm{O}$-$\mathrm{O_A}$ angle of $\mathrm{2_A1_D}$ and $\mathrm{2_A2_D}$ HB structures. 
 (g) Distribution of distance between the Wannier center and the O atom of a water molecule.}\label{wannier}
 \end{figure}

\section{Definition of Tetrahedral Order Parameter}
The parameter is defined in consistency with Ref.~\onlinecite{01N-Errington} as $q=1-\frac{3}{8}\sum_{j=1}^{3}\sum_{k=j+1}^{4}(\cos\psi_{jk}+\frac{1}{3})^2$, 
where $\psi_{jk}$ denotes the angle formed by the O atom in a water molecule and two O atoms from its four closest neighbors, 
in the same way, as shown in Fig.~\ref{adf_compare}(a).
%
For standard tetrahedron, $\psi_{jk}=-1/3$ for all $j\neq k$, thereby rendering $q=1$.
%
As is stated in Ref.~\onlinecite{01N-Errington}, for purely random cases like an ideal gas, the ensemble average of $q$ is an integration over $\psi_{jk}$ from $0$ to $\pi$, 
rendering $\langle q \rangle = 0$.
%
However, in our simulated system, we find that a structure with $q=0.5$ could already be far from a standard tetrahedral structure,
as is shown in the inset of Fig. 3(b).


\section{Electronic Structure}
In the last section, we present our new findings about the relation between the HB structure and the HB strength.
%
The result is almost unaffected by the density range studied,
so only results from ambient water are provided here.
%
As shown in Figs.~\ref{wannier}(a-e), 
not only the shape but also the spread range of spatial distribution of HB depend on the HB structure.
%
Comparing the donating ends in Figs.~\ref{wannier}(b), (d) and (e), 
the donating HBs of a water molecule exhibit more localized distribution as the water accepts more HBs.
%
Similarly, comparing the accepting ends in Fig.~\ref{wannier}(a) and (b), (c) and (d),
the more HBs a water molecule donates,
the more localized its accepting HB distribution is observed.
%
The phenomenon is also revealed by the O-O-O angle distribution of different HB structures:
the spread of $\mathrm{O_D}$-$\mathrm{O}$-$\mathrm{O_D}$ angle distribution grows larger monotonically from $\mathrm{3_A2_D}$, $\mathrm{2_A2_D}$ to $\mathrm{1_A2_D}$.
%
Meanwhile, the $\mathrm{O_A}$-$\mathrm{O}$-$\mathrm{O_A}$ angle distribution of $\mathrm{2_A1_D}$ moves slightly leftward compared to that of $\mathrm{2_A2_D}$.
%
The phenomenon reminds us of the dependence of accepted (donated) HB number distribution on donated (accepted) HB number,~\cite{08MP-Markovitch}
which suggests that the water tends to accept (donate) more HBs as it donates (accepts) more HBs.
%
All of the trends indicate that the strength of HB is relevant to the HB number on the other side of the water molecule,
i.e., the strength of the accepted (donated) HB of a water molecule grows as the water donates (accepts) more HB. 
%
%Since the hydrogen bond is a combination of a covalent bond and electrostatic force to a large extent,
%there could be an electronic origin behind this strength trend we observe.


In view of this, we examine the dependence of electronic structure on the HB structure.
%
Since the electronic structure of the water molecule can be effectively represented by maximally localized Wannier function (MLWF)~\cite{97B-Marzari, 12RMP-Marzari}, 
we compare the electronic structure of the major HB structures by examining their MLWF distributions.
%
Fig.~\ref{wannier}(g) displays the distribution of the distance of the center of MLWF from the oxygen atom (short as $\mathrm{r_{O-MLWF}}$ in the following) in each HB structure at 1 \cc{},
where the left peak represents the lone pair, and the right peak represents the bonding pair.
%
We also check the MLWF distributions at different pressures,
and find out that the influence of pressure on Wannier function distribution is quite small.
%
Notably, we find that the $\mathrm{r_{O-MLWF}}$ distributions are rather different in different HB structures (Figs.~\ref{wannier}(a-e)).
%
For instance, the lone pairs of $\mathrm{1_A}$ structures form two peaks, the right one of which locates at the same position as $\mathrm{2_A}$ structures.
%
The separation of peaks of $\mathrm{1_A}$ structure suggests that the acceptance of one HB could directly pull the lone pair away from the O atom while leaving the other lone pair relatively unchanged.
%
On the donating end, the separation of peaks is not that strong; however,
we observe a similar situation:
donating an HB could push the corresponding bonding pair towards the O atom while leaving the other pair unmoved.
%
Furthermore, 
we focus on the dependence of the lone pair (bonding pair) on the number of donated (accepted) HB.
%
We find that
the distance between the lone (bonding) pairs and the O (H) atom becomes larger as the water molecule donates (accepts) more HBs.
%
A larger distance from the lone pair to the O atom stands for a more negative environment at the O end,
thereby enabling stronger HB to form at the accepting end.
%
On the contrary,
a larger distance from the bonding pair to the H atom stands for a more positive environment at the H end,
enabling stronger HB to form at the donating end.
%
The above analysis explains the HB number, the directionality, and the strength of one end of a water molecule depending on the HB number on the other end.

%
%Furthermore, we suggest that this could also be the electronic structural reason for a stronger tendency to give away a proton when hydronium accepts an additional HB.~\cite{09L-Berkelbach, 15JCP-Tse}
%
%Past researches suggest that the acceptance of an additional HB turns the hydronium ion into a more water-like HB structure, 
%thereby making the proton transfer easier to occur.
%
%We find that the increase in the accepted HB could result in an increase in the strength of donated HB by changing the bonding pair position.
%
%This is more likely to be the reason for the stronger tendency of PT of hydronium when accepting a HB.


\bibliography{SM_ref}
\end{document}