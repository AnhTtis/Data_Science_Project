\documentclass[letterpaper, 10 pt, conference]{ieeeconf}
\IEEEoverridecommandlockouts
\overrideIEEEmargins

\usepackage{url}

% Tables
\usepackage{tabu}
\usepackage{booktabs} % for professional tables
\usepackage{multirow}
\usepackage{multicol}
\usepackage{adjustbox}

% Comments/macros
\usepackage{xcolor}
\usepackage{xspace}
\usepackage{ifthen}

% Figures
\usepackage{graphicx}
\usepackage[font=small]{caption}
\usepackage{tikz}
\usetikzlibrary{positioning}

% Lists
\let\labelindent\relax %needed for IEEE conflict with enumitem
% \usepackage{enumitem} % for \begin{description}[align=left]

% Comments/useful macros
\usepackage{autolabtools}
\usepackage{amsmath,amssymb,amsfonts}
\usepackage{dsfont}
\usepackage{color}
\usepackage{verbatim}
\usepackage{subcaption}
\usepackage{balance}
\usepackage{gensymb}
\usepackage{siunitx} % Daniel: added based on Jeff's prior papers. (needs to be paired with next line :) -Jeff)
\sisetup{detect-all} % <-- fixes fonts in siunitx
\usepackage{algorithm}
% \usepackage{algorithmic}
\usepackage{algorithmicx}
% \usepackage[linesnumbered,ruled,vlined]{algorithm2e} 
\usepackage{algpseudocode}
\usepackage{alphalph}
\usepackage{comment}
\usepackage{soul}
\usepackage[shortlabels]{enumitem}

\graphicspath{{figures/}}
\newcommand{\ba}{\mathbf{a}}
\newcommand{\bd}{\mathbf{d}}
\newcommand{\bm}{\mathbf{m}}
\newcommand{\bo}{\mathbf{o}}
\newcommand{\br}{\mathbf{r}}
\newcommand{\bs}{\mathbf{s}}
\newcommand{\bx}{\mathbf{x}}
\newcommand{\bu}{\mathbf{u}}
\newcommand{\state}{\mathbf{s}}
\newcommand{\ie}{i.e., }
\newcommand{\eg}{e.g., }

\newcommand{\E}{\mathbb{E}}
\newcommand{\R}{\mathbb{R}}
\newcommand{\N}{\mathbb{N}}
\newcommand{\Z}{\mathbb{Z}}
\newcommand{\mO}{\mathcal{O}}
\newcommand{\mP}{\mathbb{P}}

\newtheorem{theorem}{Theorem}
\newtheorem{corollary}{Corollary}[theorem]
\newtheorem{lemma}[theorem]{Lemma}

% \algrenewcommand\algorithmicrequire{\textbf{Input:}}
\algrenewcommand\algorithmicensure{\textbf{Output:}}

% When the amsmath package is in use page breaks between equation lines are normally disallowed; the philosophy is that page breaks in such material should receive individual attention from the author. To get an individual page break inside a particular displayed equation, a \displaybreak command is provided. \displaybreak is best placed immediately before the \\ where it is to take effect. Like LATEX’s \pagebreak, \displaybreak takes an optional argument between 0 and 4 denoting the desirability of the pagebreak. \displaybreak[0] means "it is permissible to break here" without encouraging a break; \displaybreak with no optional argument is the same as \displaybreak[4] and forces a break.
%If you prefer a strategy of letting page breaks fall where they may, even in the middle of a multi-line equation, then you might put \allowdisplaybreaks[1] in the preamble of your document. An optional argument 1–4 can be used for finer control: [1] means allow page breaks, but avoid them as much as possible; values of 2,3,4 mean increasing permissiveness. When display breaks are enabled with \allowdisplaybreaks, the \\* command can be used to prohibit a pagebreak after a given line, as usual.
\allowdisplaybreaks[4]

% Daniel: for better argmax and argmin.
\DeclareMathOperator*{\argmax}{arg\,max}
\DeclareMathOperator*{\argmin}{arg\,min}
\DeclareMathOperator*{\minimize}{minimize}

% Daniel: feel free to add your names with your favorite color!
% Wrap around the shared "\remark" command for consistency, and to make it easy to turn all remarks off by defining an empty command. -Jeff
\newcommand{\remark}[3]{\hidable{{\color{#2}[#1: #3]}}}
% \renewcommand{\remark}[3]{} % To disable all remarks, uncomment
\newcommand{\lawrence}[1]{\remark{Lawrence}{orange}{#1}}
\definecolor{britishracinggreen}{rgb}{0.23, 0.53, 0.19}
\newcommand{\baiyu}[1]{\remark{Baiyu}{britishracinggreen}{#1}}
\newcommand{\dave}[1]{\remark{Dave}{red}{#1}}
\newcommand{\daniel}[1]{\remark{Daniel}{blue}{#1}}
\newcommand{\KG}[1]{\remark{KG}{red}{#1}}

\definecolor{navy}{rgb}{0,0,0.5}
\newcommand{\NEW}[1]{{\color{navy} #1}}


% Daniel: putting this here for the algorithm...
\newcommand{\bagging}{SLIP-Bagging\xspace}
\newcommand{\slip}{SLIP\xspace}

% Daniel: AH! importing this package makes it much better to make tables, in my opinion, since we can put a long caption at the table title. It does create a warning but I am not sure how to fix this.
\usepackage[font={small}]{caption}
\def\tablename{Table}


% Daniel: this way of doing it must be tied with \addbibresource and \printbibliography. Also keep hyperref imported LAST. If we are ever submitting to a venue which doesn't all for infinite reference space, adjust maxbibnames to be a little smaller so it will read as author et al.
\usepackage[backend=biber,
            hyperref=true,
            url=false,
            isbn=false,
            doi=false,
            backref=false,
            style=ieee,
            natbib=true,%compatibility aliases
            mincitenames=1,
            maxcitenames=1,
            citestyle=numeric-comp,
            sorting=none,%none vs nyt
            block=none,
            maxbibnames=99]{biblatex}
\usepackage{hyperref}


% Daniel: this needs to be paired up with \printbibliography at the end.
\addbibresource{references.bib}  


% \hide % use this to hide comments

\title{\LARGE \bf
Bagging by Learning to Singulate Layers Using Interactive Perception
}

\author{Lawrence Yunliang Chen$^1$, Baiyu Shi$^1$, Roy Lin$^1$, Daniel Seita$^2$, Ayah Ahmad$^1$, \\ Richard Cheng$^3$, Thomas Kollar$^3$, David Held$^2$, Ken Goldberg$^1$
\thanks{$^{1}$The AUTOLab at UC Berkeley (autolab.berkeley.edu).}%
\thanks{$^{2}$The Robotics Institute at Carnegie Mellon University.}
\thanks{$^{3}$Toyota Research Institute, Los Altos, USA.}
\thanks{Correspondence to: {\tt\scriptsize yunliang.chen@berkeley.edu}}
}

\begin{document}

\maketitle

\thispagestyle{empty}
\pagestyle{empty}

% \begin{abstract}
As models continue to grow in size, the development of memory optimization methods (MOMs) has emerged as a solution to address the memory bottleneck encountered when training large models. To comprehensively examine the practical value of various MOMs, we have conducted a thorough analysis of existing literature from a systems perspective. 
% Furthermore, we have evaluated the most widely adopted MOMs employed in mainstream frameworks for both vision and language models.
Our analysis has revealed a notable challenge within the research community: the absence of standardized metrics for effectively evaluating the efficacy of MOMs. The scarcity of informative evaluation metrics hinders the ability of researchers and practitioners to compare and benchmark different approaches reliably. Consequently, drawing definitive conclusions and making informed decisions regarding the selection and application of MOMs becomes a challenging endeavor.
To address the challenge, this paper summarizes the scenarios in which MOMs prove advantageous for model training. We propose the use of distinct evaluation metrics under different scenarios. By employing these metrics, we evaluate the prevailing MOMs and find that their benefits are not universal. We present insights derived from experiments and discuss the circumstances in which they can be advantageous.

\end{abstract}
% \section{Introduction}


Recent years have witnessed the rise of human digitization~\cite{habermannDeepCapMonocularHuman2020,alexanderCREATINGPHOTOREALDIGITAL,pengNeuralBodyImplicit2021,alldieckDetailedHumanAvatars2018, rajANRArticulatedNeural2020}. This technology greatly impacts the entertainment, education, design, and engineering industry.
There is a well-developed industry solution for this task.
High-fidelity reconstruction of humans can be achieved either with full-body laser scans~\cite{saitoSCANimateWeaklySupervised2021}, dense synchronized multi-view cameras~\cite{xiangModelingClothingSeparate2021a,xiangDressingAvatarsDeep2022a}, or light stages~\cite{alexanderCREATINGPHOTOREALDIGITAL}.
However, these settings are expensive and tedious to deploy and consist of a complex processing pipeline, preventing the technology's democratization.

Another solution is to view the problem as inverse rendering and learn digital humans directly from custom-collected data.
Traditional approaches directly optimize explicit mesh representation~\cite{loperSMPLSkinnedMultiperson2015, fangRMPERegionalMultiperson2018, pavlakosExpressiveBodyCapture2019} which suffers from the problems of smooth geometry and coarse textures~\cite{prokudinSMPLpixNeuralAvatars2020,alldieckVideoBasedReconstruction2018}. Besides, they require professional artists to design human templates, rigging, and unwrapped UV coordinates.
Recently, with the help of volumetric-based implicit representations~\cite{mildenhallNeRFRepresentingScenes2020, parkDeepSDFLearningContinuous2019, meschederOccupancyNetworksLearning2019} and neural rendering~\cite{laineModularPrimitivesHighPerformance2020, liuSoftRasterizerDifferentiable2019, thiesDeferredNeuralRendering2019}, 
one can easily digitize a quality-plausible human avatar from video footage~\cite{jiangNeuManNeuralHuman2022,wengHumanNeRFFreeviewpointRendering}.
Particularly, volumetric-based implicit representations~\cite{mildenhallNeRFRepresentingScenes2020, pengNeuralBodyImplicit2021} can reconstruct scenes or objects with much higher fidelity against previous neural renderer~\cite{thiesDeferredNeuralRendering2019,prokudinSMPLpixNeuralAvatars2020}, and is more user-friendly as it does not need any human templates, pre-set rigging, or UV coordinates.
Captured visual footage and corresponding skeleton tracking are enough for training.
However, better reconstructions and more friendly usability are at the expense of the following factors.
1) \textbf{Inefficiency:}
They require longer optimization times (typically tens of hours or days) and inference slowly.
Volume rendering~\cite{mildenhallNeRFRepresentingScenes2020,lombardiNeuralVolumesLearning2019} formulates images by querying the densities and colors of millions of spatial coordinates. 
In the training stage, due to memory constraints, only a small fraction of points are sampled which leads to slow convergence speed.
2) \textbf{Entangled representations}:
The geometry, materials, and motion dynamics are entangled in the neural networks. 
Due to the implicit nature of neural nets, one can hardly edit one property without touching the others~\cite{yuanNeRFEditingGeometryEditing2022a,liuEditingConditionalRadiance2021}.
3) \textbf{Graphics incompatibility}:
Volume rendering is incompatible with the current popular graphic pipeline,
which renders triangular/quadrilateral meshes efficiently with the rasterization technique.
Many downstream applications require mesh rasterization in their workflow (\eg, editing~\cite{foundationBlenderOrgHome}, simulation~\cite{benderPositionBasedSimulationMethods2015}, real-time rendering~\cite{akenine2019real}, ray-tracing~\cite{waldRTXRayTracing}).
Although there are approaches~\cite{lorensenMarchingCubesHigh,labelleIsosurfaceStuffingFast2007} can convert volumetric fields into meshes, the gaps from discrete sampling degrade the output quality in terms of both meshes and textures.


To address these issues, we present \textbf{EMA}, a method based on \textbf{E}fficient \textbf{M}eshy neural fields to reconstruct animatable human \textbf{A}vatars.
Our method enjoys flexibility from implicit representations and efficiency from explicit meshes, yet still maintains high-fidelity reconstruction quality.
Given video sequences and the corresponding pose tracking, our method digitizes humans in terms of canonical triangular meshes, physically-based rendering (PBR) materials, and skinning weights \textit{w.r.t.} skeletons.
We jointly learn the above components via inverse rendering~\cite{laineModularPrimitivesHighPerformance2020,chenDIBRLearningPredict2021,chenLearningPredict3D2019} in an end-to-end manner.
Each of them is derived from a separate neural field, which relaxes the requirements of a preset human template, rigging, or UV coordinates.
Specifically, we predict a canonical mesh out of a signed distance field (SDF) by differentiable marching tetrahedra~\cite{shenDeepMarchingTetrahedra2021,gaoGET3DGenerativeModel,gaoLearningDeformableTetrahedral2020,munkbergExtractingTriangular3D2022}, then we extend the marching tetrahedra~\cite{shenDeepMarchingTetrahedra2021} for spatial-varying materials by utilizing a neural field to predict PBR materials \textit{on the mesh surfaces} after rasterization~\cite{munkbergExtractingTriangular3D2022,hasselgrenShapeLightMaterial2022,laineModularPrimitivesHighPerformance2020}.
To make the canonical mesh animatable, we take another neural field to model the forward linear blend skinning for the meshes. 
Given a posed skeleton, the canonical mesh is then transformed into the corresponding poses.
Finally, we shade the mesh with a rasterization-based differentiable renderer~\cite{laineModularPrimitivesHighPerformance2020} and train our models with a photo-metric loss.
After training, we export the mesh with materials and discard the neural fields.

\looseness=-1
There are several merits of our method design.
1) \textbf{Efficiency}:
Powered by efficient mesh rendering, our method can render in real-time.
Besides, the training speed is boosted as well, 
since we compute loss holistically on the whole image and the gradients only flow on the mesh surface. In contrast, volume rendering takes limited pixels for loss computation and back-propagates the gradients in the whole space.
Our method only needs about an hour of training and minutes of optimization are enough for plausible avatar reconstruction.
2) \textbf{Disentangled representations}:
Our shape, materials, and motion modules are disentangled naturally by design, which facilitates editing. 
Besides, Canonical meshes with forward skinning modeling handle the out-of-distribution poses better.
3) \textbf{Graphics compatibility}:
Our derived mesh representation is compatible with 
the prominent graphic pipeline, which leads to instant downstream applications (\eg, the shape and materials can be edited directly in design software~\cite{foundationBlenderOrgHome}).
To further improve reconstruction quality, we additionally optimize image-based environment lights and non-rigid motions.


We conduct extensive experiments on standards benchmarks H36M~\cite{ionescuHuman36MLarge2014b} and ZJU-MoCap~\cite{pengNeuralBodyImplicit2021}.
Our method achieves very competitive performance for novel view synthesis, generalizes better for novel poses, 
and significantly improves both training time and inference speed against previous arts.
Our research-oriented code reaches real-time inference speed (100+ FPS for rendering $512\times512$ images).
We in addition showcase applications including novel pose synthesis, material editing, and relighting.
% \section{Related Work} \label{sec:related work}
\vspace{-0.2cm}
{\noindent \bf Vision-Language Pre-training.} In the early literature, \cite{Mori99,Frome13,Weston11} explore jointly training image-text embeddings using paired text documents. Recently, some studies have further scaled up the training with large-scale web data to form ``the \textbf{foundation} models'', {\em e.g.}, CLIP~\cite{Radford21}, ALIGN~\cite{Jia21}, Florence~\cite{yuan2021florence}, FILIP~\cite{yao2021filip}, VideoCLIP~\cite{xu2021videoclip}, and LiT~\cite{zhai2022lit}. These foundation models usually contain one visual encoder and one textual encoder, which are trained using simple noise contrastive learning for powerful cross-modal representations. They have shown promising potential in many tasks, such as image classification and detection, action recognition, and retrieval. In this paper, we use CLIP for low-shot temporal action localization, but the same technique should be applicable to other foundation models as well.



\vspace{0.1cm}
{\noindent \bf Prompting} refers to leveraging input instructions to steer foundation models for desired outputs. In the NLP domain, early papers~\cite{Gao21,Jiang20,Timo21,Shin20} focus on handcrafted prompt templates. To avoid labor and increase flexibility, some studies~\cite{Lester21,li21-prefixtuning,li2021prefix} propose learnable prompt tuning at the textual stream, showing strong low-shot generalization. In the CV domain, some recent papers~\cite{zhou2019learn,zhou2022conditional,ju2022prompting} introduce such randomly initialized prompt tuning to handle visual tasks, {\em e.g.}, image understanding~\cite{zhu2022prompt,lu2022prompt,yang2022learning,ma2023diffusionseg} and video understanding~\cite{jia2022visual,nag2022zero,ni2022expanding}. However, these studies ignore lexical ambiguity of category names, and cases that are not easy to describe in text. This paper designs novel conditional prompt tuning and language descriptions from LLMs, to solve these issues. 



\vspace{0.1cm}
{\noindent \bf Closed-set Temporal Action Localization} considers to detect and classify action instances from one pre-defined category list. Specifically, existing methods can be divided into two popular supervisions, {\em i.e.}, strong~\cite{zeng2019graph,lin2021learning,qing2021temporal} and weak~\cite{wang2017untrimmednets,ju2023constraint,ju2020point,yudistira2022weakly}. Strong supervision gives precise boundary labels and category labels for training. There are two detailed pipelines: the top-down framework~\cite{shou2016temporal,shou2017cdc,gao2017turn,chao2018rethinking,lin2017single,xu2017r,tan2021relaxed,zhu2021enriching,wang2022rcl,xu2020g} pre-defines extensive anchors, adopts fixed-length sliding windows to produce initial proposals, then regresses to refine boundaries; the bottom-up framework~\cite{zhao2017temporal,lin2018bsn,lin2019bmn,vo2023aoe,zhao2020bottom,bai2020boundary} learns frame-wise boundary detectors for the boundary frames, then groups extreme frames or estimates action lengths for proposal generation. In addition, several works~\cite{gao2018ctap,liu2019multi,yang2020revisiting} used various fusion strategies to complement these frameworks. On the other hand, weak supervision trains without boundary labels to alleviate annotation costs. The video-level setting learns from category labels~\cite{paul2018w,ju2022distilling}, the CAS-based framework~\cite{liu2019completeness,ju2021adaptive,min2020adversarial,narayan2021d2,lee2019background,lee2021weakly,zhao2021soda} and attention-based framework~\cite{nguyen2018weakly,luo2021action,nguyen2019weakly,shi2020weakly,gao2022fine,he2022asm,huang2021foreground,luo2020weakly,ma2022weakly} have been well studied. To generate better results from CAS or attention, some studies~\cite{shou2018autoloc,liu2019weakly} improved post-processing. To balance cost and performance, some papers introduced single-frame annotations~\cite{ju2021divide,ma2020sf,lee2021learning,yang2021background,mettes2019pointly} or instance-number annotations~\cite{narayan20193c,xu2019segregated}. 

Nevertheless, all the above methods assume that action categories remain identical for training and testing, which is an over-simplification of real application scenarios, limiting practical uses of the vision system.



\vspace{0.1cm} 
{\noindent \bf Low-Shot Temporal Action Localization} considers more realistic scenarios: generalize TAL towards action categories that are unseen (zero-shot) or with several support samples (few-shot). Existing methods~\cite{ju2022prompting,nag2022zero,zhang2022ow,bao2022opental} most rely on foundational models pre-trained on large-scale image-caption pairs for help. Typically, E-Prompt~\cite{ju2022prompting} is the first to construct wide baselines with popular prompt tuning~\cite{Lester21,li21-prefixtuning} and vanilla temporal modeling. STALE~\cite{nag2022zero} explores the one-stage framework to further simplify usage. Although promising, all above methods meet two main challenges: (1) For category semantics, the definition may be vague, inaccurate, or incomplete. (2) For visual motions, temporal modeling may be insufficient. In this paper, for detailed category understanding, we design novel language descriptions from LLMs and vision-conditional prompt tuning; for clearer motion understanding, we introduce optical flows to provide explicit motion inputs. 




% \section{Problem Statement}
\label{sec:ps}

The objective is to bring a long (3 m) cable containing semi-planar knots into an untangled configuration, where no knots remain (knots defined in Section \ref{sec:knot_def}).

% The objective for the robot is to untangle a long (3 m) cable consisting of any semi-planar knot from overhead RGB image observations. We use a bimanual robot to execute manipulation primitives until the cable reaches a fully untangled state with no knots.

The workspace is defined by an $(x, y, z)$ coordinate system and consists of a bilateral robot and a foam-padded manipulation surface, which lies in the $(x, y)$ plane. The workspace also contains a fixed overhead RGB-D camera that faces the manipulation surface and outputs grayscale images and depth data. However, depth data is not used in \peralgabbr{}. Rather, it is only used during manipulation. We work with a 300 cm cable. We assume the cable is visually distinguishable from the manipulation surface, its initial configuration has at least one endpoint visible, and is semi-planar as assumed in \citet{grannen2020untangling}, meaning each crossing in the knot has at most 2 intersecting cable segments. For perception experiments, we work with knots as tight as 5\,cm in diameter. For physical experiments, due to robot graspability constraints, we work with knots of varying density, or approximate diameter, upwards of 10 cm in diameter. We define cable state to be $\theta(s) = \{(x(s), y(s), z(s))\}$ where $s$ is an arc-length parameter that ranges $[0, 1]$, representing the normalized length of the cable. Here, $(x(s), y(s), z(s))$ is the location of a cable point at a normalized arc length of $s$ from the cable's first endpoint. We also define the range of $\theta(s)$---that is, the set of all points on the cable at time $t$---to be $\mathcal{C}_t$.  

% \todo{I don't think we should define things like this. We should be looking at the cable configuration as a whole} Note that although the knot itself is semi-planar, crossings in the cable can consist of more than 2 cable segments as slack can fall on top of the knot or form loops elsewhere in the cable such that crossings with more than 2 cable segments appear. \todo{Check note above + definitions below. Also, highlight in methods? That is, the fact that the tracer can handle more than 2 crossings}.

% \subsection{Cable Segment Definition (For Semi-Planar Knots)}
% Suppose there exists a crossing on the cable path at time $t$. Owing to the semi-planar nature of the knot, there will be two sightings of this crossing (once as an undercrossing, once as an overcrossing) on the cable path $P_t(s)$. Assume that the two indices at which this crossing is encountered on $P_t(s)$ are $a$ and $b$ ($a < b$). 

% We define the cable segments intersecting at this crossing to be $s_1$ and $s_2$. Here, $s_1$ and $s_2 \subset C_t$ such that $s_1 = \{P_t(s) \mid s \in [0, a]\}$ and $s_2 = \{P_t(s) \mid s \in [b, 1]\}$.

\subsection{Knot Definition}
\label{sec:knot_def}
Consider a pair of points $p_1$ and $p_2$ on the cable path at time $t$ with ($p_1, p_2 \in \mathcal{C}_t)$. Knot theory strictly operates with closed loops, so to form a loop with the current setup, we construct an imaginary cable segment with no crossings joining $p_1$ to $p_2$ \cite{reidemeister1983knot}. This imaginary cable segment passes above the manipulation surface to complete the loop between $p_1$ and $p_2$ (``$p_1\rightarrow p_2$ loop").
A knot exists between $p_1$ and $p_2$ at time $t$ if no combination of Reidemeister moves I, II (both shown in Figure \ref{fig:reid_cc}), and III can simplify the $p_1 \rightarrow p_2$ loop to an unknot, i.e. a crossing-free loop. In this paper, we aim to untangle semi-planar knots. For convenience, we define an indicator function $k(s):[0,1]\rightarrow\{0,1\}$ which is 1 if the point $\theta(s)$ lies between any such points $p_1$ and $p_2$, and 0 otherwise.

\begin{figure}[!t]
    \centering
    \includegraphics[width=1.0\linewidth]{figures/crossing_cancellation.pdf}
    \caption{\textbf{Reidemeister Moves and Crossing Cancellation}: Top left depicts Reidemeister Move II. Top right depicts Reidemeister Move I. The bottom row shows that by algorithmically applying Reidemeister Moves II and I, we can cancel trivial loops, even if they visually appear as knots.}
    \label{fig:reid_cc}
    % \vspace*{-0.25in}
\end{figure}

 
Based on the above knot definition, this objective is to remove all knots, such that $\int k(s)_0^1=0$. In other words, the cable, if treated as a closed loop from the endpoints, can be deformed into an unknot. We measure the success rate of the system at removing knots, as well as the time taken to remove these knots. 
% \section{Conclusion and Future Work}

In this paper, we propose an approach for bagging by singulating layers using interactive perception. Experiments show that \bagging achieves significantly higher success rates over baselines for opening a bag, inserting items, and lifting the bag. In future work, we plan to apply this approach to related tasks such as packing and wrapping. 

%We hope that this work will lead to an exciting era in robotic manipulation of bags and 3D deformable objects more generally.



\begin{abstract}
Many fabric handling and 2D deformable material tasks in homes and industry require singulating layers of material such as opening a bag or arranging garments for sewing. In contrast to methods requiring specialized sensing or end effectors, we use only visual observations with ordinary parallel jaw grippers. We propose \slip: Singulating Layers using Interactive Perception, and apply \slip to the task of autonomous bagging. We develop \bagging, a bagging algorithm that manipulates a plastic or fabric bag from an unstructured state, and uses \slip to grasp the top layer of the bag to open it for object insertion. In physical experiments, a YuMi robot achieves a success rate of 67\% to 81\% across bags of a variety of materials, shapes, and sizes, significantly improving in success rate and generality over prior work. Experiments also suggest that \slip can be applied to tasks such as singulating layers of folded cloth and garments.
Supplementary material is available at \url{https://sites.google.com/view/slip-bagging/}.
\end{abstract}


\section{Introduction}\label{sec:intro}
Many tasks in homes and factories require grasping a single layer of 2D deformable objects. Examples include taking one napkin from a stack of napkins, grasping the top layer of a folded towel to unfold it, grasping a single layer of a T-shirt to insert into a hanger, and grasping a single layer of a bag to hold it open while placing items inside. Humans manipulate such deformable objects with great dexterity using touch and vision. 
Such tasks are very challenging for robots. Enabling touch sensing may require equipping the robot end effector with compliant grippers or special tactile sensors such as the mini-Delta gripper~\cite{mannam2021low} and the ReSkin sensor~\cite{bhirangi2021reskin} used in Tirumala~et~al.~\cite{tirumala2022} and GelSight~\cite{yuan2017gelsight} used in Sunil et al.~\cite{sunil2023visuotactile} for cloth manipulation. % The challenge when using visual sensors is occlusion and uncertainty. In industries such as packaging, suction cup arrays and electrostatic mechanisms are a common solution to grasping deformables, but they only work on certain materials (e.g., flat, light-weight, smooth, and without holes). 

%In this work, we do not use any special hardware. 
In this work, we achieve single-layer grasping with a high success rate using a bimanual robot with an ordinary parallel-jaw gripper.
We use self-supervised learning to identify where to grasp, and we use interactive perception to determine the number of layers grasped.  The robot iteratively adjusts its grasp until it successfully grasps a single layer.

\begin{figure}[t]
\center
\includegraphics[width=0.49\textwidth]{figures/teaser_v5-min.png}
\caption{
\bagging. \textbf{Top 2 rows:} (1) Initial unstructured and deformed bag. (2) The robot flattens the bag, and then (3) uses \slip to grasp the top layer of the bag. The robot rotates the bag by $90^\circ$ and inserts objects into the bag. (4) The robot lifts the bag filled with the inserted items, so it is ready for transport. \textbf{Bottom 2 rows:} (A) Initial configuration of a piece of folded cloth. (B) The robot uses \slip to grasp the top layer of a folded square cloth. (C) After grasping the top layer, the robot lifts the cloth up. (D) After shaking, the cloth is successfully expanded.
}
\vspace{-15pt}
\label{fig:teaser}
\end{figure}




We focus on the task of autonomous bagging --- opening a deformable bag from an unstructured initial state and putting objects into it. This task has wide applications in retail, food handling, home cleaning, and packing. However, it involves perception and manipulation challenges for robots. %Compared to ropes and fabrics, there are more ways for 3D bags to deform and create occlusion. 
There are many ways for bags to deform and create self-occlusions.
Many plastic bags are also reflective and translucent, as well as elastoplastic, meaning they have a tendency to restore their shape under small forces. However, an ideal state for object placement---a bag standing upward with its opening wide open---is also not a naturally stable pose for soft deformable bags, which tend to lie on their side. An alternative approach is to open the bag and insert objects horizontally, but this requires grasping only the top layer to create the bag opening. This is nontrivial because: (1) The depth values from a typical RGBD camera are often severely inaccurate on bags due to reflection and transparency; (2) even if the depth of the top layer is known, moving the gripper to that height to perform the grasp will push the surface downward and cause a missed grasp due to lack of friction on the surface; and (3) grasping from the side is not always kinematically feasible since the two layers can be stuck together with no space in between.
%; and (4) the only reliable way to grasp the top layer from the top is to push it down to the table or other support surfaces and use the friction to pinch the bag while closing the gripper, but this often also accidentally grasps the bottom layer. 
As we demonstrate in experiments, a 1 mm change in the gripper height can lead to the difference between a missed (0-layer) grasp, a 1-layer grasp, and a 2-layer grasp of plastic bags, and the heights of successful 1-layer grasps are different each time depending on the shape of the bag (wrinkles and flatness of both the top and bottom layers).



We use interactive perception~\cite{bohg2017} to recognize how many layers a robot grasps, and to adjust its grasp if needed. While the robot cannot a priori know what grasp height it should go to grasp only one layer of the bag, after a grasp is performed it can tell how many layers it has grasped by perturbing the bag and observing how the bag moves with the gripper. Intuitively, if the bag does not move, it indicates a 0-layer grasp. If the top layer of the bag moves with the gripper while the bottom layer only moves a little and mostly stays on the table, it indicates a 1-layer grasp. If the entire bag moves with the gripper, it suggests a 2-layer grasp. 

We propose \slip: Singulating Layers using Interactive Perception. Using \slip, we present an algorithm for opening deformable bags, which we call \bagging, that is effective across a variety of bag materials, including non-reusable plastic bags such as thin and soft bags made of low-density polyethylene (LDPE) and thicker and stiffer grocery bags made of high-density polyethylene (HDPE), as well as reusable fabric bags such as drawstring backpacks, mesh drawstring bags, and large fabric handbags. Physical experiments suggest that \bagging achieves a 5x success rate compared to a prior state-of-the-art method for autonomous bagging~\cite{chen2022autobag}. Moreover, we conduct physical experiments to evaluate the applicability of \slip to singulating layers for a variety of fabrics and garments.

This paper makes the following contributions:
\begin{enumerate}[leftmargin=*]
    \item \slip, an algorithm to singluate layers of bags using interactive perception, with visual feedback to enable the robot to adapt its grasp height without tactile sensors;
    \item \bagging, an algorithm that uses \slip for opening and inserting objects into a deformable bag in an unstructured initial state;
    \item A \bagging system and physical experiments that achieve a success rate of 67\% to 81\% across bags of different materials, shapes, and sizes (unseen in training). On thin plastic bags, \bagging's success rate is 5x that of the state-of-the-art method designed specifically for thin plastic bags. 
    \item Physical experiments suggesting the applicability of \slip to other tasks such as singulating layers of fabrics and dresses.
\end{enumerate} 




\section{Related Work} \label{sec:related work}
\vspace{-0.2cm}
{\noindent \bf Vision-Language Pre-training.} In the early literature, \cite{Mori99,Frome13,Weston11} explore jointly training image-text embeddings using paired text documents. Recently, some studies have further scaled up the training with large-scale web data to form ``the \textbf{foundation} models'', {\em e.g.}, CLIP~\cite{Radford21}, ALIGN~\cite{Jia21}, Florence~\cite{yuan2021florence}, FILIP~\cite{yao2021filip}, VideoCLIP~\cite{xu2021videoclip}, and LiT~\cite{zhai2022lit}. These foundation models usually contain one visual encoder and one textual encoder, which are trained using simple noise contrastive learning for powerful cross-modal representations. They have shown promising potential in many tasks, such as image classification and detection, action recognition, and retrieval. In this paper, we use CLIP for low-shot temporal action localization, but the same technique should be applicable to other foundation models as well.



\vspace{0.1cm}
{\noindent \bf Prompting} refers to leveraging input instructions to steer foundation models for desired outputs. In the NLP domain, early papers~\cite{Gao21,Jiang20,Timo21,Shin20} focus on handcrafted prompt templates. To avoid labor and increase flexibility, some studies~\cite{Lester21,li21-prefixtuning,li2021prefix} propose learnable prompt tuning at the textual stream, showing strong low-shot generalization. In the CV domain, some recent papers~\cite{zhou2019learn,zhou2022conditional,ju2022prompting} introduce such randomly initialized prompt tuning to handle visual tasks, {\em e.g.}, image understanding~\cite{zhu2022prompt,lu2022prompt,yang2022learning,ma2023diffusionseg} and video understanding~\cite{jia2022visual,nag2022zero,ni2022expanding}. However, these studies ignore lexical ambiguity of category names, and cases that are not easy to describe in text. This paper designs novel conditional prompt tuning and language descriptions from LLMs, to solve these issues. 



\vspace{0.1cm}
{\noindent \bf Closed-set Temporal Action Localization} considers to detect and classify action instances from one pre-defined category list. Specifically, existing methods can be divided into two popular supervisions, {\em i.e.}, strong~\cite{zeng2019graph,lin2021learning,qing2021temporal} and weak~\cite{wang2017untrimmednets,ju2023constraint,ju2020point,yudistira2022weakly}. Strong supervision gives precise boundary labels and category labels for training. There are two detailed pipelines: the top-down framework~\cite{shou2016temporal,shou2017cdc,gao2017turn,chao2018rethinking,lin2017single,xu2017r,tan2021relaxed,zhu2021enriching,wang2022rcl,xu2020g} pre-defines extensive anchors, adopts fixed-length sliding windows to produce initial proposals, then regresses to refine boundaries; the bottom-up framework~\cite{zhao2017temporal,lin2018bsn,lin2019bmn,vo2023aoe,zhao2020bottom,bai2020boundary} learns frame-wise boundary detectors for the boundary frames, then groups extreme frames or estimates action lengths for proposal generation. In addition, several works~\cite{gao2018ctap,liu2019multi,yang2020revisiting} used various fusion strategies to complement these frameworks. On the other hand, weak supervision trains without boundary labels to alleviate annotation costs. The video-level setting learns from category labels~\cite{paul2018w,ju2022distilling}, the CAS-based framework~\cite{liu2019completeness,ju2021adaptive,min2020adversarial,narayan2021d2,lee2019background,lee2021weakly,zhao2021soda} and attention-based framework~\cite{nguyen2018weakly,luo2021action,nguyen2019weakly,shi2020weakly,gao2022fine,he2022asm,huang2021foreground,luo2020weakly,ma2022weakly} have been well studied. To generate better results from CAS or attention, some studies~\cite{shou2018autoloc,liu2019weakly} improved post-processing. To balance cost and performance, some papers introduced single-frame annotations~\cite{ju2021divide,ma2020sf,lee2021learning,yang2021background,mettes2019pointly} or instance-number annotations~\cite{narayan20193c,xu2019segregated}. 

Nevertheless, all the above methods assume that action categories remain identical for training and testing, which is an over-simplification of real application scenarios, limiting practical uses of the vision system.



\vspace{0.1cm} 
{\noindent \bf Low-Shot Temporal Action Localization} considers more realistic scenarios: generalize TAL towards action categories that are unseen (zero-shot) or with several support samples (few-shot). Existing methods~\cite{ju2022prompting,nag2022zero,zhang2022ow,bao2022opental} most rely on foundational models pre-trained on large-scale image-caption pairs for help. Typically, E-Prompt~\cite{ju2022prompting} is the first to construct wide baselines with popular prompt tuning~\cite{Lester21,li21-prefixtuning} and vanilla temporal modeling. STALE~\cite{nag2022zero} explores the one-stage framework to further simplify usage. Although promising, all above methods meet two main challenges: (1) For category semantics, the definition may be vague, inaccurate, or incomplete. (2) For visual motions, temporal modeling may be insufficient. In this paper, for detailed category understanding, we design novel language descriptions from LLMs and vision-conditional prompt tuning; for clearer motion understanding, we introduce optical flows to provide explicit motion inputs. 








\section{Problem Statement}\label{sec:PS}

% Daniel: rearranged here as we introduce the bags here.
\begin{figure}[t]
\center
\includegraphics[width=0.42\textwidth]{figures/bags_v3.pdf}
\caption{
Top: Training bags. Bottom: Test bags. % See Section~\ref{sec:exp} for more details.  and Table~\ref{tab:bag_result} for results.
% Categories from left to right: thin LDPE plastic bags, thick HDPE plastic bags, drawstring bags, and reusable handbags.
}
\vspace{-15pt}
\label{fig:bags}
\end{figure}

We study the autonomous bagging task. As defined in prior work~\cite{chen2022autobag}, the task consists of manipulating and opening a deformable bag from an unstructured state, inserting $n$ items, and then lifting it for transport. Unlike prior work~\cite{chen2022autobag} which only considers thin plastic bags made of low-density polyethylene (LDPE), we additionally test heavy-duty grocery plastic bags made of high-density polyethylene (HDPE), drawstring backpacks and mesh bags, and reusable fabric handbags (as shown in Figure~\ref{fig:bags}).

We consider a bimanual robot with two standard parallel jaw grippers, operating in an $(x, y, z)$ Cartesian coordinate frame with a flat manipulation surface parallel to the $xy-$plane and a calibrated overhead RGBD camera. At each time step, the robot uses the RGBD input $I \in \R ^ {W \times H \times 4}$ of the bag to select and execute a parameterized open-loop action primitive $\mathbf{a}$ according to a policy $\pi: I \mapsto \ba$. We assume the initial bag state is unstructured and resting stably on the workspace, that is, it may be deformed and compressed, but not tied or zipped. We assume a set of rigid objects $\mO$, placed in known poses for grasping. We use the following metrics for the task: (1) the success rate of grasping a single layer, (2) the percentage of objects successfully inserted into the bag, and (3) the percentage of objects contained in the bag after the robot lifts the bag.

We make the following assumptions: (1) the two sides of the bag do not stick to each other tightly (which occurs in brand-new plastic bags), (2) the bag can be segmented from the workspace via color thresholding, and (3) the size of the bag when fully flattened, denoted $A_{max}$, is known a priori.





\section{\slip: Singulating Layers using Interactive Perception}\label{sec:slip}

% Daniel: I edited Lawrence's slides for slip_v2. Also v3.
% Also moved this figure here in the section that describes SLIP.
\begin{figure*}[t]
\center
%\includegraphics[width=1.00\textwidth]{figures/slip_v2-min.png}
\includegraphics[width=1.00\textwidth]{figures/slip_v3-min.png}
\caption{
Examples of \slip in action. Each row shows an example of one cyclic, triangular trajectory $T$ in an iteration (``iter'' in Algorithm~\ref{alg:slip}) where one gripper moves the bag while the other one pins the bag. \textbf{Top row} a third-person view of the robot \textbf{Next three rows}: top-down RGB camera views of one cyclic trajectory for different trials. They show, respectively, a 0-layer, 1-layer, and 2-layer grasp on a plastic bag. We provide zoomed-in versions of images in the third column to see the layers in more detail. See Section~\ref{sec:slip} for details.
}
\vspace{-10pt}
\label{fig:slip}
\end{figure*}

In this section, we describe \slip in the context of grasping a single layer of a deformable bag, but the algorithm may apply to other fabric materials (see Section~\ref{sec:exp}). As discussed in Section~\ref{sec:intro}, \slip is motivated by how, after the robot has performed a grasp, it cannot easily determine how many layers it grasped from visual inputs of a static scene, as the top layer is occluding the layers underneath. However, by moving the gripper and observing how the bag is moved with the top layer, the robot can infer how many layers it has grasped. Formally, \slip requires 3 components: a cyclic trajectory $T$ of the robot gripper, a video classification model $M$, and an iterative height adjustment algorithm. %We describe each component in detail next.


\begin{algorithm}[t]
\caption{\slip: Singulating Layers using Interactive Perception}\label{alg:slip}
% \footnotesize
\begin{algorithmic}[1]
\Require RGBD camera, closed trajectory $T$ of the gripper, video classification model $M$, initial grasp height $h_0$, iterative height adjustment $\Delta h_+, \Delta h_-$, maximum number of iterations TrialMax, IterMax
\Ensure Single-layer grasp success
% \Statex
\State Grasp height $h = h_0$, trial = 0
\While{trial $<$ TrialMax}
    \State Sample a grasp location $(x, y)$
    \State iter = 0
    \While{iter $<$ IterMax}
        \State Grasp at location $(x, y, h)$
        \State Execute trajectory $T$ and record a video stream $V$ from the camera during the motion
        \State Number of layers grasped $n = M.predict(V)$
        \If{$n = 0$}
            $h \gets h - \Delta h_-$
        \ElsIf{$n = 1$}
            return True
        \ElsIf{$n \geq 2$}
            $h \gets h + \Delta h_+$
        % \ELSE
        %     \State success = True
        \EndIf
        \State Open gripper
        \State iter $\gets$ iter $+1$ 
    \EndWhile
        \State trial $\gets$ trial $+1$
\EndWhile
\State return False
\end{algorithmic}
\end{algorithm}



The trajectory $T$ needs to satisfy two properties: (1) The movement should reveal enough information for the robot camera to infer how many layers are grasped, and (2) the trajectory should be cyclic, so the bag roughly recovers its original state after executing $T$, allowing the robot to retry the grasp at the same location but with a different height. For (1), the main consideration is occlusion, as the robot gripper and wrist occlude the grasp point and its nearby region if we use a top-down grasp with an overhead camera. Thus, we tilt the gripper at an angle $\theta = 50^{\circ}$ so the grasp point is visible in the camera. For (2), we use a triangular trajectory, where the robot gripper first moves backward, then upward, and finally forward and downward back to the original position. To prevent the deformable object from translating as a whole, we use the robot's second gripper to pin the other side. See Figure~\ref{fig:slip} for a visualization. In our implementation, the trajectory takes about 5 secs.

While the robot executes the trajectory $T$, the camera takes an RGB video stream of the bag. A video classification model $M$ takes the video and classifies the grasp into 3 categories: 0 layer, 1 layer, and 2 layers. We use a SlowFast network with a ResNet-50 backbone~\cite{feichtenhofer2019slowfast}, which takes in 32 images of size 224 $\times$ 224 sampled with a uniform interval from the video stream. Fig.~\ref{fig:slip} illustrates the visual differences among different layers grasped on a plastic bag.

Given the model classification, \slip adjusts the gripper height and retries the grasp if it does not successfully grasp a single layer. We choose to use a fixed height adjustment each time which is similar to the strategy in~\cite{tirumala2022}, with height deltas $\Delta h_{-}$ and $\Delta h_{+}$. One could also choose to let the adjustment height decay over time or use the bisection method, but we empirically find the numbers of trials these approaches take are similar while a fixed height adjustment is more robust to a long sequence of iterations than a decaying step size, and more robust to minor bag state changes between grasps and model classification errors than bisection, since once bisection enters into a wrong interval, it may not succeed. The pseudocode of \slip is provided in Algorithm~\ref{alg:slip}.




\section{\bagging}

% Daniel: modified Lawrence's keynote to get this figure.
% Also put this by the section which actually describes SLIP-Bagging.
\begin{figure*}[t]
\center
%\includegraphics[width=1.00\textwidth]{figures/slip_bagging_v2.png}
\includegraphics[width=1.00\textwidth]{figures/slip_bagging_v3.png}
\caption{
\bagging Algorithm. (1) The robot starts with an unstructured bag with objects on the side. (2) \bagging then flattens the bag, and (3) uses \slip to grasp the top layer of the bag, followed by (4) insertion and (5) bag lifting. A trial is a full success if the robot lifts the bag with all items in it.
}
\vspace{-10pt}
\label{fig:slip-bag}
\end{figure*}

\subsection{Learned Perception Module}\label{subsec:perception_module}
Similar to Chen et al.~\cite{chen2022autobag}, we train a perception module to recognize the bag rim and handles. We represent bags through semantic segmentation, where each RGB image is classified per pixel into: bag handle, bag rim, remaining area of the bag, and the background. To collect training data, we use self-supervision with UV-fluorescent markings~\cite{LUV_2022}. We place 6 UV LED lights around the workspace and paint the handles and rim of the bags with UV-fluorescent markers. During data collection, we take image pairs of the bag under regular lighting and under UV lighting. When the UV lights are turned on, the painted regions glow unique colors that can be extracted in the image through color thresholding, allowing us to get ground truth segmentation labels corresponding to the bag image under regular lighting. The robot then performs a random action to disturb the bag into another state, and repeats the process. This allows us to obtain a large dataset with diverse bag configurations efficiently without the need of human annotations. See~\cite{chen2022autobag} and~\cite{LUV_2022} for details. 
%The main difference in this work compared to AutoBag~\cite{chen2022autobag} is that we show how this procedure applies to bags of other material and shapes such as a fabric bag (see Figure~\ref{fig:bags}).

\subsection{Action Primitives}
\bagging uses 4 manipulation primitives to flatten a bag from an unstructured state:
\begin{enumerate}[leftmargin=*]
\item \textbf{Shake}: Grasp a corner of the bag, lift it up, shake it a predefined number of times, followed by a swing action to lay the bag on the table. This action uses gravity and inertia to loosen the bag.
\item \textbf{Rotate}: For small plastic bags, the robot uses one hand to rotate it. It grasps the center of the bag and rotate a desired angle. For large fabric bags, the robot uses two hands since one hand is not effective due to underactuation. It grasps the left and right sides of the bag, and pushes one hand forward while pulling the other hand backward to rotate the bag.
\item \textbf{Dilate}: First, the left gripper pins the left side of the bag, while the right gripper presses the bag and moves from the center to the right. Then, the right gripper pins the right side of the bag, while the left gripper presses the bag and moves from the center to the left. This action flattens the bag.
\item \textbf{Fling}: Inspired by prior work that uses a fling action to smooth garments~\cite{ha2021flingbot,flinging_2022,speedfolding_2022}, we design a fling primitive to smooth fabric bags. Given two pick points, the robot lifts the bag above the surface with two hands, flings the bag forward and then backward while putting it down. The fling velocity is set to the robot's maximum speed.
\end{enumerate}
In addition, the robot recenters the bag as needed. For \textbf{Dilate}, the robot always first rotates the bag so that the bag's shorter axis aligns with the robot's horizontal axis, as \textbf{Dilate} can help expand that axis using friction.

\subsection{\bagging}
The \bagging algorithm consists of 5 steps: (1) flatten the bag, (2) grasp the top layer of the bag near the bag opening using \slip, (3) rotate the bag sideways, (4) use the other gripper to insert objects, and (5) lift the bag. See Figure~\ref{fig:slip-bag} for an overview.






\begin{figure*}[t]
\center
\includegraphics[width=1.00\textwidth]{figures/grasp_height_v2.pdf}
\caption{
Distribution of number of layers grasped for different grasp heights for 4 different bags.
}
\vspace{-5pt}
\label{fig:grasp_height}
\end{figure*}


\begin{table*}
\centering
\begin{tabular}{cc|cc| ccc| cccc| cccc | c}
    \toprule
\multirow{2}{*}{Category} 
 & \multirow{2}{*}{Bag}   & \multicolumn{2}{c}{Open/Flatten} & \multicolumn{3}{c}{Single-layer Grasp}  & \multicolumn{4}{c}{Full Success} & \multicolumn{4}{c}{\% Objects Inserted} & \multirow{2}{*}{\parbox{1.3cm}{\centering SB Failure Modes}} \\
    \cmidrule(lr){3-4} \cmidrule(lr){5-7} \cmidrule(lr){8-11} \cmidrule(lr){12-15} 
        &  &  AB  & SB  & PD  & HG   & SB  & PD  & HG  & AB  & SB & PD  & HG  & AB  & SB        \\
    \midrule
    \midrule
\multirow{2}{*}{Thin Plastic}  
        & Train    & 3/6    & \textbf{6/6}    & 1/6    & -    & \textbf{6/6}   & 1/6    & 0/6    & 1/6    & \textbf{5/6}   & 17\%   & 0\%    & 39\%   & \textbf{94\%}   &  (D)     \\
         & Test    & 3/6    & \textbf{6/6}    & 1/6    & -    & \textbf{6/6}   & 1/6    & 0/6    & 1/6  & \textbf{4/6}   & 17\%   & 0\%    & 36\%   & \textbf{75\%}   &  (D) $\times$ 2    \\
    \addlinespace
    \midrule
\multirow{2}{*}{Thick Plastic}
        & Train    & 3/6    & \textbf{5/6}    & 1/6     & -    & \textbf{5/5}$^*$   & 1/6    & 0/6    & 1/6 & \textbf{3/6}   & 17\%    & 0\%    & 36\%   & \textbf{56\%}   & (A) (C) (D)      \\
         & Test    & 2/6    & \textbf{5/6}    & 0/6    & -    & \textbf{4/5}$^*$  & 0/6    & 0/6    & 2/6 & \textbf{4/6}   & 0\%    & 0\%    & 33\%   & \textbf{67\%}   & (A) (B)     \\
    \addlinespace
    \midrule
\multirow{2}{*}{Drawswtring}
         & Train    & 0/6    & \textbf{6/6}    & 0/6    & -   & \textbf{5/6}   & 0/6   & -    & 0/6 & \textbf{5/6}   & 0\%   & -    & 0\%   & \textbf{83\%}   &  (B)     \\
         & Test    & 0/6    & \textbf{6/6}    & 1/6    & -   & \textbf{5/6}   & 0/6    & -    & 0/6 & \textbf{4/6}   & 0\%    & -    & 0\%   & \textbf{67\%}   &  (B) (C)    \\
    \addlinespace
    \midrule
% \multirow{2}{*}{Reusable}
Reusable        & Train    & 0/6    & \textbf{5/6}    & 0/6    & 5/6    & \textbf{3/5}$^*$   & 0/6    & 2/6   & 0/6 & \textbf{3/6}   & 0\%    & 36\%    & 0\%   &  \textbf{50\%}   &   (A) (B) $\times$ 2    \\
Handbag         & Test    & 0/6    & \textbf{6/6}    & 1/6   & 4/6    & \textbf{6/6}  & 1/6    & 3/6    & 0/6 & \textbf{4/6}   & 17\%    & 50\%    & 0\%   & \textbf{81\%}   &  (C) (D)      \\
    \bottomrule \\
\end{tabular}
$^*$Denominator is the number of successful flattened trials that proceed to the \slip stage. See Section~\ref{subsec:bag_results} for failure mode categories.
\caption{Physical experiment results of \bagging compared with baselines. 6 trials were run on each of the 8 bags (Fig.~\ref{fig:bags}) for each method. Each trial attempts to insert 6 rubber ducks, and ``\% Objects Inserted'' is the average percentage of ducks inserted and remaining in the bag after bag lifting. \textbf{PD}: Perceived-Depth baseline. \textbf{HG}: Handle Grasping baseline. \textbf{AB}: AutoBag. \textbf{SB}: SLIP-Bagging. The ``Single-layer Grasp'' column for HG refers to grasping the handle associated with the top layer for handbags whose handles overlap (and thus is not applicable to plastic bags and drawstring bags).
}
\label{tab:bag_result}
\vspace*{-10pt}
\end{table*}




In the first step, the robot iteratively checks the bag area and orientation at each time step to determine whether the bag reaches a ``flattened'' state ready for single-layer grasping. \bagging requires 3 threshold hyperparameters, $p_\text{small}$,$p_\text{large}$, and $\alpha$. For a bag with an area $A_{max}$ when fully flattened, if the current bag area is below $p_\text{small}A_{max}$, the robot performs the \textbf{Shake} primitive. If the current bag area is between $p_\text{small}A_{max}$ and $p_\text{large}A_{max}$, the robot uses \textbf{Dilate} to expand a plastic bag and \textbf{Fling} to expand a fabric bag. If the current bag area is greater than $p_\text{large}A_{max}$, the robot checks whether the bag opening is pointing forward. If the angle between the bag's opening and the robot's forward axis is greater than $\alpha$, the robot \textbf{Rotates} the bag; otherwise, the bag is considered to be successfully ``flattened.''

Next, the robot uses \slip to grasp the top layer of the bag. One hand pins the bottom of the bag while the other hand grasps a point near the center of the rim. We set the initial height of the grasp $h_0 = \max(h_{PD}, h_{min})$, where $h_{PD}$ is the perceived depth of the grasp point on the bag surface measured by the RGBD camera, and $h_{min}$ is a threshold to prevent the robot to grasp too deep in the case of erroneous depth measurements such as for mesh bags that have holes. We set $\Delta h_{-} = 1$ mm and $\Delta h_{+} = 3$ mm.

After successfully grasping the top layer of the bag, the robot rotates the bag by $90^\circ$ while the other arm grasps the inserted items and performs sideways insertion. Finally, the other arm goes inside the bag, grasps it, and lifts the bag together with the arm that has been holding the bag.




\section{Physical Experiments}\label{sec:exp}

For experiments, we use a bimanual ABB YuMi robot with an overhead RealSense D435 camera. The workspace has dimensions 60$\times$90 cm$^2$, and the bags we use range from 32$\times$32 cm$^2$ to 55$\times$55 cm$^2$.

\subsection{Implementation Details}
To train the perception module to recognize the rim and handles of bags, we collect a total of 7,500 images across 15 bags in 4 categories, with about 500 images for each bag, using the self-supervised process described in Sec.~\ref{subsec:perception_module}. The 4 categories are: non-reusable thin plastic bags made of LDPE, non-reusable thick plastic bags made of HDPE, drawstring bags including backpacks and mesh bags, and reusable fabric handbags. For the segmentation network, we use a U-Net architecture~\cite{ronneberger2015u} trained with soft DICE loss~\cite{milletari2016v}. We use one NVIDIA A100 GPU, with a batch size of 32, an initial learning rate of 5e-4, and a weight decay factor of 1e-5. The trained model achieves a 70\% intersection over union (IOU) on the validation set. 

To train the video classification model for \slip, we collect 800 video examples on 4 bags (one in each category), with 200 each. During data collection, for each sample, we manually set the bag into a roughly flat state, and then specify a grasp point as well as a grasp height. We randomize the grasp height so that the number of 0-layer, 1-layer, and 2-layer examples are balanced in the dataset. 
As described in Sec.~\ref{sec:slip}, we use a SlowFast network architecture~\cite{feichtenhofer2019slowfast}. The 32 image frames the model takes in are sampled from uniform intervals. We train the model on an A100 GPU with a batch size of 32, learning rate of 5e-4, and an Adam optimizer. The trained model achieves a 90\% accuracy on the validation set.

% Daniel: this reference was, bafflingly, cited when we wrote the 'ducks'!
% ~\cite{Burgert2022triton}
\subsection{Bagging Experiments Setup}
We evaluate \bagging on 8 bags, shown in Figure~\ref{fig:bags}. We use 6 rubber ducks of dimension $6 \times 7 \times 5$ cm$^3$ as the objects for insertion. For each trial, we randomly initialize the bag state by taking the bag, compressing and deforming it with our hands, dropping it onto the workspace, and letting the bag settle into a stable state. We allow for up to 30 actions (excluding recentering) for the robot to flatten the bag and up to 15 grasp iterations during \slip. If the robot encounters motion planning or kinematic errors during the trial, we reset the robot to the home position and continue the trial.

We compare \bagging to 3 baselines:
\begin{enumerate}[leftmargin=*]
    \item Perceived-Depth (PD): This method ablates the \slip algorithm in \bagging. Instead, the robot directly grasps at $h_{PD}$, the perceived depth of the grasp point on the bag surface measured by a depth camera. %Unlike \bagging, PD requires a depth camera (in addition to RGB information).
    \item Handle Grasping (HG): This method selects a handle to grasp instead of the bag rim. After grasping the handle and lifting the bag up, the robot performs sideways insertion underneath the grasping hand, similar to \bagging.
    \item AutoBag (AB)~\cite{chen2022autobag}: AutoBag uses ``Compress'' and ``Flip'' primitives to orient the bag upward and open the bag directly. It places the objects from the top onto the bag opening and then lifts the bag up instead of inserting the objects sideways.
\end{enumerate} 
For Perceived-Depth and Handle Grasping, we evaluate from an already-flattened bag state since the procedure of flattening the bag is the same as that in \bagging. For \bagging, we record the success rate of flattening the bag and grasping a single layer (which opens the bag from the side). For AutoBag, we measure the success rate of opening the bag upward. For each bag, we conduct 6 trials. 

\begin{figure}[t]
\center
\includegraphics[width=0.3\textwidth]{figures/garments_v0.png}
\caption{
Non-bag experiments. Left to right: Folded cloth, dress, hat. The robot pins at the region indicated by the green point and attempts to grasp a single layer at the yellow star region.
}
\vspace{-10pt}
\label{fig:garments}
\end{figure}


\begin{table}[t]
  \setlength\tabcolsep{5.0pt}
  \centering
  \vspace{-0em}
  \vspace{1pt}
  %\begin{adjustbox}{width=\columnwidth,center}
  \centering
  \footnotesize
  \begin{tabular}{c |ccc| c}
  \toprule
  \multirow{2}{*}{Objects}  &  \multicolumn{3}{c}{0-Shot Recall} &   \multicolumn{1}{|c}{\multirow{2}{*}{SLIP Success Rate}}  \\
  \cmidrule(lr){2-4}
   & 0-layer & 1-layer & 2-layer & \\
  \midrule
  Folded Cloth & 100\% & 100\%  & 62\% & 4/6 \\
  Dress & 100\% & 75\%  & 75\% & 4/6\\
  Hat & 100\% & 83\%  & 25\% & 5/6\\
  \bottomrule
  \end{tabular}
  \caption{Non-bag experiments. 
  The middle 3 columns show the (multi-class) recall of the video classification model trained on bags and tested on garments without finetuning. The last column shows the success rate of grasping a single layer using \slip with the classification model.
  }
  \label{tab:garment_result}
  \vspace*{-10pt}
\end{table}


\subsection{Bagging Experiments Results}\label{subsec:bag_results}


Figure~\ref{fig:grasp_height} shows the distribution of the number of layers grasped at various grasp heights (measured from the surface height) for each of the 4 training bags in the training data. As expected, as the grasp height decreases, it is less likely to grasp 0 layers and more likely to grasp 2 layers. However, while some grasp heights are more likely than others to grasp a single layer for each bag, there is no single grasp height that always works, as the success of single-layer grasping depends highly on the specific configuration of the bags. 

Results in Table~\ref{tab:bag_result} demonstrate that \bagging achieves a higher success rate than all baselines for all bags. Among the baselines, Perceived-Depth has a low success rate of grasping a single layer. This is because, for the mesh bag, the perceived depth is often too deep due to holes, resulting in 2-layer grasps, while for other bags, the perceived depth is often not deep enough and leads to 0-layer grasps. For the Handle Grasping baseline, it is not applicable to drawstring bags since they do not have handles, and while it achieves a high success rate on handbags, it is not effective for plastic bags. The handles of plastic bags are on the two sides, so grasping and lifting them does not help create an opening like with a handbag. Its failures on handbags are mainly due to mistakenly grasping the handle associated with the bottom layer or accidentally grasping both handles. For AutoBag, which is designed for thin plastic bags, is not effective on drawstring bags and fabric handbags. This is because its key action for opening the bag, ``Compress,'' uses air to inflate the bag during a downward motion, and so only works for bags with lightweight material and without holes. 


We observe 4 failure modes of \bagging:
\begin{enumerate}[(A)]
    \item Failure to flatten the bag with the correct orientation.
    \item Failure to successfully grasp a single layer of the bag.
    \item Bag slips out of the gripper after grasping a single layer.
    \item Robot hand hits the bag handles during insertion and does not put the objects inside the bag.
\end{enumerate}

Failure (A) occurs when the perception module fails to recognize the rim and handle regions, and thus rotates the bag in the wrong orientation. Failure (B) 
%occurs sometimes due to wrong predictions by the video classification model in \slip, 
usually occurs when the robot accidentally grasps both layers and manipulates the bag into a configuration 
%, causing the layers to roll into the bottom during the movement due to friction with the workspace surface. This can 
which prevents the robot from grasping a single layer in future trajectories. For bags with stiff plastic and fabric materials, failure (C) can occur when the single-layer grasp is not firm enough and the bag slips out. % gripper during further actions such as rotation and insertion (when the other hand accidentally hits the bag, it may knock it off). 
Failure (D) is often due to the bag not being rotated completely sideways, which means the insertion hand can fail to enter the bag. Additionally, for thin and thick plastic bags, their opening size is relatively small compared to their handles. Thus, during insertion, the robot hand may sometimes hit the handles or other parts of the bag, pushing the bag away or causing it to rotate, leading to insertion failure. 


\subsection{Single-Layer Grasping on Fabrics}

We test \slip on other materials to evaluate its applicability to general single-layer grasping tasks. We consider 3 deformable objects: a blue piece of cloth folded twice into a square (Fig.~\ref{fig:teaser}), a white dress, and a red hat (Fig.~\ref{fig:garments}). The task goal is to grasp their top layer only. We apply our video classification model to these objects without any finetuning. 

Table~\ref{tab:garment_result} shows the 0-shot multi-class recall metrics for the classification model as well as the success rate of achieving a single-layer grasp. In each case, the model predicts accurately on a 0-layer grasp and 1-layer grasp, but less accurately on a 2-layer grasp, for which there are greater visual differences across objects.  
%We hypothesize that this is because the visual appearances of a 0-layer and 1-layer are similar across objects, while they appear different under 2-layer grasps. 
A failure mode associated with grasping a folded cloth is that the cloth has 4 layers. Grasping 1, 2, and 3 layers look visually similar, so the model would mistaken those 2 and 3 layer grasps as a 1-layer grasp. 
While the model accuracy is lower than that of bags the model is trained on, the SLIP success rate is fairly high. The robot starts from a grasp height higher than the surface height and gradually decreases its height, so it 
suffices for the model to accurately recognize a 1-layer grasp.

Out of the 5 failed trials, 3 are due to model prediction errors, such as  
%incorrectly classifying a 0-layer or 2-layer grasp as a 1-layer grasp when it is not, or 
misclassifying a 1-layer (or 2-layer) grasp and adjusting the grasp height in the wrong direction. The other 2 failures occur from the object slipping out of the robot's 1-layer grasp during its gripper movement. % When a 1-layer grasp is weak, it may slip as the robot pulls it backward and forward. Unlike elastic plastic bags that have the tendency to recover to their original state, fabrics remain in the middle of the trajectory where it slips out of the gripper. Thus, when the robot tries to regrasp at the original position, the top layer is no longer there to be grasped.



\section{Conclusion and Future Work}

In this paper, we propose an approach for bagging by singulating layers using interactive perception. Experiments show that \bagging achieves significantly higher success rates over baselines for opening a bag, inserting items, and lifting the bag. In future work, we plan to apply this approach to related tasks such as packing and wrapping. 

%We hope that this work will lead to an exciting era in robotic manipulation of bags and 3D deformable objects more generally.




% Daniel: can remove if needed.
\section*{Acknowledgments}
{\footnotesize
This research was performed at the AUTOLAB at UC Berkeley in affiliation with the Berkeley AI Research (BAIR) Lab, and the CITRIS ``People and Robots'' (CPAR) Initiative. The authors are supported in part by donations from Toyota Research Institute and equipment grants from NVIDIA.  L.Y. Chen is supported by the National Science Foundation (NSF) Graduate Research Fellowship Program under Grant No. 2146752. D. Seita and D. Held are supported by NSF CAREER grant IIS-2046491. We thank Ryan Burgert for providing rubber ducks for our experiments and Kaushik Shivakumar for valuable feedback.
}


% Daniel: this is the old way of doing it. The alternative with \printbibliography I think will be better since we can customize it a little more and we can add hyperref (which didn't work when I tried this variant).
% \clearpage
% \balance  % We need this line for the correct ordering of reference. ( this line is in "balance" Package
%{\footnotesize
%\bibliographystyle{IEEEtran}
%\bibliography{references}
%}

% Daniel: these two commands need to be paired up with the \addbibresource{references.bib} command at the top where `references.bib` is the name of our file.
\renewcommand*{\bibfont}{\footnotesize}
\printbibliography

% \clearpage
% \section{Appendix for Proofs}

\paragraph{Proof of Theorem \ref{thm:main}.}

\begin{proof}
\label{proof:main}
Our proof has two steps. In Step 1, we will show that SimCLR is equivalent to minimizing the cross entropy loss defined in Eqn.~(\ref{eqn:cross-entropy}). 
In Step 2, we will show  that minimizing the cross-entropy loss 
is equivalent to spectral clustering on $\bfpi$. 
Combining the two steps together, we have proved our theorem. 

\textbf{Step 1: } SimCLR is equivalent to minimizing the cross entropy loss.

The cross-entropy loss takes expectation over 
$\bfW_\bfX\sim \mathbb{P}(\cdot ; \bfpi)$, 
which means $\bfW_\bfX$ has exactly one non-zero entry in each row $i$. By Lemma~\ref{lem:multinomial}, we know every row $i$ of $\bfW_\bfX$ is independent of other rows. Moreover, 
$\bfW_{\bfX,i}\sim \mathcal{M}(1, \bfpi_i/\sum_j \bfpi_{i,j})=\mathcal{M}(1, \bfpi_i)$, because $\bfpi_i$ itself is a probability distribution.
Similarly, we know $\bfW_\bfZ$ also has the row-independent property by sampling over $\mathbb{P}(\cdot;\bfK_\bfZ)$.
Therefore, by Lemma~\ref{lem:cross_split}, we know Eqn.~(\ref{eqn:cross-entropy}) is equivalent to:
\[
 -\sum_{i=1}^n \mathbb{E}_{\bfW_{\bfX,i}}[\log \mathbb{P}(\bfW_{\bfZ,i}=\bfW_{\bfX,i};\bfK_\bfZ)],
\]

This expression takes expectation over $\bfW_{\bfX,i}$ for the given row $i$. Notice that 
$\bfW_{\bfX,i}$ has exactly one non-zero entry, which equals $1$ (same for $\bfW_{\bfZ,i}$). 
As a result
we expand the above expression to be:
\begin{equation}
 -\sum_{i=1}^n \sum_{j\neq i} \Pr(\bfW_{\bfX,i,j}=1)\log \Pr(\bfW_{\bfZ,i,j}=1).
\label{eqn:detailed-expansion}    
\end{equation}


By Lemma~\ref{lem:multinomial}, $\Pr(\bfW_{\bfZ,i,j}=1)=\bfK_{\bfZ,i,j}/\|\bfK_{\bfZ,i}\|_1$ for $j\neq i$. Recall that $\bfK_\bfZ=(k(\bfZ_i-\bfZ_j))_{(i,j)\in[n]^2}$, which means 
$\bfK_{\bfZ,i,j}/\|\bfK_{\bfZ,i}\|_1=\frac{\exp(-\|\bfZ_i-\bfZ_j\|^2/{2\tau})}{\sum_{k\neq i}
\exp(-\|\bfZ_i-\bfZ_k\|^2/{2\tau})
}$ for $j\neq i$, when $k$ is the Gaussian kernel with variance $\tau$. 

Notice that $\bfZ_i=f(\bfX_i)$, so we know
\begin{equation}
-\log \Pr(\bfW_{\bfZ,i,j}=1)=
-\log \frac{\exp(-\|f(\bfX_i)-f(\bfX_j)\|^2/{2\tau})}{\sum_{k\neq i}
\exp(-\|f(\bfX_i)-f(\bfX_k)\|^2/{2\tau}),
}
\label{eqn:infonce-equivalence}    
\end{equation}


The right hand side is exactly the InfoNCE loss defined in Eqn.~(\ref{eqn:infonce}).
Inserting Eqn.~(\ref{eqn:infonce-equivalence}) into Eqn.~(\ref{eqn:detailed-expansion}), we get the SimCLR algorithm, which first samples augmentation pairs $(i,j)$ with $\Pr(\bfW_{\bfX,i,j}=1)$ for each row $i$, and then optimize the InfoNCE loss. 

\textbf{Step 2: } minimizing the cross entropy loss 
is equivalent to spectral clustering on $\bfpi$.


By Lemma~\ref{lem:convert_to_spectral}, we may further convert the loss to 
\begin{equation}
\label{eqn:main-theorem-repul-attr}
\min_{\bfZ}
-\sum_{(i,j)\in [n]^2} \mathbf{P}_{i,j}
\log k (\bfZ_i-\bfZ_j)+\log \mathbf{R}(\bfZ).
\end{equation}
Since $k$ is the Gaussian kernel, this reduces to \[
\min_\bfZ \mathrm{tr}(\bfZ^\top \mathbf{L}(\bfpi) \bfZ)
+\log \mathbf{R}(\bfZ),
\]

where we use the fact that $\mathbb{E}_{\bfW_\bfX\sim \mathbb{P}(\cdot; \bfpi)}[\mathbf{L}(\bfW_\bfX)]
=\mathbf{L}(\bfpi)
$, because the Laplacian operator is linear and $
\mathbb{E}_{\bfW_\bfX\sim \mathbb{P}(\cdot; \bfpi)}(\bfW_\bfX)=\bfpi
$.
\end{proof}

\paragraph{Proof of Theorem \ref{thm:clip}.}
\begin{proof}
Since $\bfW_\bfX\sim \mathbb{P}(\cdot;\bfpi_{\mathbf{A}, \mathbf{B}})$, we know 
$\bfW_\bfX$ has exactly one non-zero entry in each row, denoting the pair that got sampled. 
A notable difference compared to the previous proof is we now have $n_\mathcal{A}+n_\mathcal{B}$ objects in our graph. CLIP deals with this by taking a mini-batch of size $2N$, 
such that $n_\mathcal{A}=n_\mathcal{B}=N$, and adding the $2N$ InfoNCE losses together. We label the objects in $\mathcal{A}$ as $[n_\mathcal{A}]$, and the objects in $\mathcal{B}$ as $\{n_\mathcal{A}+1, \cdots, n_\mathcal{A}+n_\mathcal{B}\}$. 

Notice that $\bfpi_{\mathbf{A}, \mathbf{B}}$ is a bipartite graph, so the edges of objects in $\mathcal{A}$ will only connect to object in $\mathcal{B}$ and vice versa. We can define the similarity matrix in $\cZ$ as $\bfK_\bfZ$, 
where $\bfK_\bfZ(i, j+n_\mathcal{A})=\bfK_\bfZ(j+n_\mathcal{A},i)= k(\bfZ_i-\bfZ_j)$ for $i\in [n_\mathcal{A}], j\in [n_\mathcal{B}]$, and otherwise we set $\bfK_\bfZ(i,j)=0$. 
The rest is same as the previous proof. 
\end{proof}

\paragraph{Proof of Theorem \ref{thm:exponential}.}

\begin{proof}
\label{proof:exponential}
Since the objective function consists of a linear term combined with an entropy regularization, which is a strongly concave function, the maximization problem is a convex optimization problem. Owing to the implicit constraints provided by the entropy function, the problem is equivalent to having only the equality constraint. We then introduce the Lagrangian multiplier $\lambda$ and obtain the following relaxed problem:

$$
\widetilde{E}(\boldsymbol{\alpha})=\psi_{1}-\sum_{i=1}^n \alpha_{i} \psi_{i}+\tau \sum_{i=1}^n \alpha_{i}\log \alpha_{i}+\lambda\left(\boldsymbol{\alpha}^{\top} \mathbf{1}_n-1\right).
$$

As the relaxed problem is unconstrained, taking the derivative with respect to $\alpha_{i}$ yields

$$
\frac{\partial \widetilde{E}(\boldsymbol{\alpha})}{\partial \alpha_{i}}=-\psi_{i}+\tau\left(\log \alpha_{i}+\alpha_{i} \frac{1}{\alpha_{i}}\right)+\lambda=0.
$$

Solving the above equation implies that $\alpha_{i}$ takes the form
$
\alpha_{i}=\exp \left(\frac{1}{\tau} \psi_{i}\right) \exp \left(\frac{-\lambda}{\tau}-1\right).
$ Since $\alpha_{i}$ lies on the probability simplex, the optimal $\alpha_{i}$ is explicitly given by
$
\alpha^{*}_{i}=\frac{\exp \left(\frac{1}{\tau} \psi_{i}\right)}{\sum_{i^{\prime}=1}^n \exp \left(\frac{1}{\tau} \psi_{i^{\prime}}\right)} .
$ Substituting the optimal point into the objective function, we obtain
$$
\begin{aligned}
E\left(\boldsymbol{\alpha}^*\right)  &=\psi_1-\sum_{i=1}^n \frac{\exp \left(\frac{1}{\tau} \psi_{i}\right)}{\sum_{i^{\prime}=1}^n \exp \left(\frac{1}{\tau} \psi_{i^{\prime}}\right)} \psi_{i}+\tau \sum_{i=1}^n \frac{\exp \left(\frac{1}{\tau} \psi_{i}\right)}{\sum_{i^{\prime}=1}^n \exp \left(\frac{1}{\tau} \psi_{i^{\prime}}\right)}\log \frac{\exp \left(\frac{1}{\tau} \psi_{i}\right)}{\sum_{i^{\prime}=1}^n \exp \left(\frac{1}{\tau} \psi_{i^{\prime}}\right)} \\
& =\psi_1 - \tau \log \left(\sum_{i=1}^n \exp \left(\frac{1}{\tau} \psi_{i}\right)\right).
\end{aligned}
$$
Thus, the Lagrangian dual function is given by
\begin{equation*}
-E\left(\boldsymbol{\alpha}^*\right)= -\tau \log \frac{\exp \left(\frac{1}{\tau} \psi_{1}\right)}{\sum_{i=1}^n \exp \left(\frac{1}{\tau} \psi_{i}\right)}.\qedhere
\end{equation*}
\end{proof}



\section{More on Experiments} \label{section: experiment_details}

\paragraph{CIFAR-10 and CIFAR-100} CIFAR-10 ~\citep{krizhevsky2009learning} and CIFAR-100 ~\citep{krizhevsky2009learning} are well-known classic image classification datasets. Both CIFAR-10 and CIFAR-100 contain a total of 60k $32 \times 32$ labeled images of different classes, with 50k for training and 10k for testing. CIFAR-10 is similar to CIFAR-100, except there are 10 different classes in CIFAR-10 and 100 classes in CIFAR-100.

\paragraph{TinyImageNet} TinyImageNet ~\citep{le2015tiny} is a subset of ImageNet ~\citep{deng2009imagenet}. There are 200 different object classes in TinyImageNet, with 500 training images, 50 validation images, and 50 test images for each class. All the images in TinyImageNet are colored and labeled with a size of $64 \times 64$.

\textbf{Pseudo-code.} Algorithm \ref{alg:Training Procedure} presents the pseudo-code for our empirical training procedure.

\begin{algorithm}[!htbp]
\caption{Training Procedure}
\label{alg:Training Procedure}
\begin{algorithmic}[1]
\REQUIRE trainable encoder network $f$, batch size $N$, augmentation strategy \textit{aug}, loss function $L$ with hyperparameters \textit{args}
\FOR {sampled minibatch ${x_i}_{i=1}^N$}
\FORALL{$i \in { 1, ..., N }$}
\STATE draw two augmentations $t_i = \textit{aug}\left(x_i\right) $, $t_i' = \textit{aug}\left(x_i\right) $
\STATE $z_i = f\left(t_i\right)$, $z_i' = f\left(t_i'\right)$
\ENDFOR
\STATE compute loss $\mathcal{L} = L(N, z, z', \textit{args})$
\STATE update encoder network $f$ to minimize $\mathcal{L}$
\ENDFOR
\STATE \textbf{Return} encoder network $f$
\end{algorithmic}
\end{algorithm}

We also provide the pseudo-code for our core loss function used in the training procedure in Algorithm \ref{alg:Core loss}. The pseudo-code is almost identical to SimCLR's loss function, with the exception of an extra parameter $\gamma$.

\begin{algorithm}[!htbp]
\caption{Core loss function $\mathcal{C}$}
\label{alg:Core loss}
\begin{algorithmic}[1]
\REQUIRE batch size $N$, two encoded minibatches $z_1, z_2$, $\gamma$, temperature $\tau$
\STATE $z = \textit{concat}\left(z_1, z_2\right)$
\FOR {$i \in {1, ..., 2N }, j \in {1, ..., 2N}$ }
\STATE $s_{i,j} = \Vert z_i - z_j \Vert_2^{\gamma}$
\ENDFOR
\STATE \textbf{define} $l(i, j)$ \textbf{as} $l(i, j) = - \log \frac{exp\left(s_{i,j}/\tau \right)}{\sum_{k=1}^{2N} \mathbf{1}{[k \ne i]} exp\left(s{i, j} / \tau \right)} $
\STATE \textbf{Return} $\frac{1}{2N} \sum_{k=1}^N\left[l(i, i+N) + l(i+N, i)\right]$
\end{algorithmic}
\end{algorithm}

Utilizing the core loss function $\mathcal{C}$, we can define all kernel loss functions used in our experiments in Table \ref{table: loss definition}. For all $z_i \in z$ with even dimensions $n$, we define $z_{L_i} = z_i\left[0:n/2\right]$ and $z_{R_i} = z_i\left[n/2:n\right]$.

\begin{table}[ht]
\centering
\begin{tabular}{{@{}l|l@{}}}
Kernel  &  Loss function \\ \midrule
Laplacian & $\mathcal{C}\left(N, z, z', \gamma=1, \tau\right)$\\ \midrule
Sum       & $\lambda * \mathcal{C}\left(N, z, z', \gamma=1, \tau_1\right) + (1-\lambda) * \mathcal{C}\left(N, z, z', \gamma=2, \tau_2\right)$  \\ \midrule
Concatenation Sum&$\lambda * \mathcal{C}\left(N, z_L, z'_L, \gamma=1, \tau_1\right) + (1-\lambda) * \mathcal{C}\left(N, z_R, z'_R, \gamma=2, \tau_2\right)$\\ \midrule
$\gamma = 0.5$ & $\mathcal{C}\left(N, z, z', \gamma=0.5, \tau\right)$          \\ 

\end{tabular}

\caption{Definition of kernel loss functions in our experiments}
\label {table: loss definition}
\end{table}

\textbf{Baselines.} We reproduce the SimCLR algorithm using PyTorch Lightning~\citep{PytorchLightning}.

\textbf{Encoder details.}
The encoder $f$ consists of a backbone network and a projection network. We employ ResNet50~\citep{ResNet} as the backbone and a 2-layer MLP (connected by a batch normalization~\citep{ioffe2015batch} layer and a ReLU \cite{nair2010rectified} layer) with hidden dimensions 2048 and output dimensions 128 (or 256 in the concatenation kernel case).

\textbf{Encoder hyperparameter tuning.}
For each encoder training case, we randomly sample 500 hyperparameter groups (sample details are shown in Table \ref{table: Hyperparameter sample}) and train these samples simultaneously using Ray Tune ~\citep{RayTune}, with the ASHA scheduler~\citep{li2018massively}. Ultimately, the hyperparameter group that maximizes the online validation accuracy (integrated in PyTorch Lightning) within 5000 validation steps is chosen for the given encoder training case.

\begin{table}[ht]
\centering

\begin{tabular}{@{}l|l|l@{}}
\midrule
Hyperparameter  & Sample Range & Sample Strategy \\ \midrule
start learning rate & $\left[10^{-2}, 10\right]$ & log uniform \\ \midrule
$\lambda$       & $\left[0, 1\right]$ & uniform \\ \midrule
$\tau$, $\tau_1$, $\tau_2$ & $\left[0, 1\right]$ & log uniform \\ \midrule
\end{tabular}

\caption{Hyperparameters sample strategy}
\label {table: Hyperparameter sample}
\end{table}

\textbf{Encoder training.} 
We train each encoder using the LARS optimizer~\citep{LARSOptimizer}, LambdaLR Scheduler in PyTorch, momentum 0.9, weight decay $10^{-6}$, batch size 256, and the aforementioned hyperparameters for 400 epochs on a single A-100 GPU.

\textbf{Image transformation.} The image transformation strategy, including augmentation, is identical to the default transformation strategy provided by PyTorch Lightning.

\textbf{Linear evaluation.}
The linear head is trained using the SGD optimizer with a cosine learning rate scheduler, batch size 64, and weight decay $10^{-6}$ for 100 epochs. The learning rate starts at $0.3$ and ends at $0$.

\textbf{Moco Experiments.} We also tested our method based on MoCo~\citep{he2019moco}. The results are summarized in Table \ref{tab:results-moco}. Here we choose ResNet18~\citep{ResNet} as the backbone and set a temperature of $0.1$ as default. For our simple sum kernel, we set $\lambda=0.8$. The results show that our method outperforms the original MoCo method.

\begin{table}[thb]
\centering
\caption{MoCo Experiment Results on CIFAR-10 and CIFAR-100.}
\label{tab:results-moco}
\resizebox{\textwidth}{!}{%
\begin{tabular}{@{}c|ccc|ccc@{}}
\toprule
\multirow{3}{*}{Method} & \multicolumn{3}{c|}{CIFAR-10} & \multicolumn{3}{c}{CIFAR-100} \\ \cmidrule(lr){2-4} \cmidrule(lr){5-7} 
                        & 200 epochs & 400 epochs    & 1000 epochs   & 200 epochs & 400 epochs & 1000 epochs         \\ \midrule
MoCo (repro.)         & $76.41 \pm 0.12$    & $80.01 \pm 0.15$          & $84.45 \pm 0.08$    & $\mathbf{47.02 \pm 0.11}$ & $52.50 \pm 0.07$ & $57.62 \pm 0.15$            \\
\midrule
Laplacian Kernel        & ${78.09 \pm 0.10}$    & $\mathbf{83.85 \pm 0.09}$          & $\mathbf{88.34 \pm 0.16}$    & $46.12 \pm 0.22$   & $53.44 \pm 0.17$ & $59.10 \pm 0.14$        \\
Simple Sum Kernel & $\mathbf{78.12 \pm 0.15}$   & $83.23 \pm 0.18$ & $87.50 \pm 0.20$ & $46.65 \pm 0.06$ & $\mathbf{53.62 \pm 0.19}$ & $\mathbf{59.83 \pm 0.12}$\\
\bottomrule
\end{tabular}
}
\end{table}



\section{More Experiments on Synthetic Data}


Consider a scenario with $n$ clusters, each containing $k$ vertices. Let the probability of vertices $u$ and $v$ from the same cluster belonging to $\bfpi$ be $p$. Conversely, for vertices $u$ and $v$ from different clusters, let the probability of belonging to $\pi$ be $q$. We generate the graph $\bfpi$ randomly, based on $p$ and $q$. We experiment with values of $k=100$ and $n=6$ for ease of visualization, embedding all points in a two-dimensional space. Each vertex's initial position originates from a normal distribution. In each iteration, we sample a subgraph of $\bfpi$ uniformly, ensuring each vertex has an out-degree of $1$. We then optimize the corresponding vectors using InfoNCE loss with an SGD optimizer and iterate until convergence. Our experimental setup consists of an SGD learning rate of $1$, an InfoNCE loss temperature of $0.5$, and a batch size of $50$. We evaluate two scenarios with different $p$ and $q$ values: $p=1$, $q=0$, and $p=0.75$, $q=0.2$. The results of these experiments are visualized in Figure \ref{fig:vis-spectral-cluster}. The obtained embeddings exhibit the hallmark pattern of spectral clustering of graph $\bfpi$.

\begin{figure}[!tb]
\centering
\subfigure{
\includegraphics[width=1\textwidth]{Figures/cluster_pi.png}
\label{fig:vis-cluster}
}
\subfigure{
\includegraphics[width=1\textwidth]{Figures/noised_cluster_pi.png}
\label{fig:vis-noised-cluster}
}
\caption{Visualizations of the optimization process using InfoNCE Loss on the vectors corresponding to $\bfpi$. Points of identical color belong to the same cluster within $\bfpi$. To showcase the internal structure of $\bfpi$, we randomly select 10 vertices from each cluster to display the edge distribution of $\bfpi$.}
\label{fig:vis-spectral-cluster}
\end{figure}




\end{document}
