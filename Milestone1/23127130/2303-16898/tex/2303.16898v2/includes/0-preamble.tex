\usepackage{url}

% Tables
\usepackage{tabu}
\usepackage{booktabs} % for professional tables
\usepackage{multirow}
\usepackage{multicol}
\usepackage{adjustbox}

% Comments/macros
\usepackage{xcolor}
\usepackage{xspace}
\usepackage{ifthen}

% Figures
\usepackage{graphicx}
\usepackage[font=small]{caption}
\usepackage{tikz}
\usetikzlibrary{positioning}

% Lists
\let\labelindent\relax %needed for IEEE conflict with enumitem
% \usepackage{enumitem} % for \begin{description}[align=left]

% Comments/useful macros
\usepackage{autolabtools}
\usepackage{amsmath,amssymb,amsfonts}
\usepackage{dsfont}
\usepackage{color}
\usepackage{verbatim}
\usepackage{subcaption}
\usepackage{balance}
\usepackage{gensymb}
\usepackage{siunitx} % Daniel: added based on Jeff's prior papers. (needs to be paired with next line :) -Jeff)
\sisetup{detect-all} % <-- fixes fonts in siunitx
\usepackage{algorithm}
% \usepackage{algorithmic}
\usepackage{algorithmicx}
% \usepackage[linesnumbered,ruled,vlined]{algorithm2e} 
\usepackage{algpseudocode}
\usepackage{alphalph}
\usepackage{comment}
\usepackage{soul}
\usepackage[shortlabels]{enumitem}

\graphicspath{{figures/}}
\newcommand{\ba}{\mathbf{a}}
\newcommand{\bd}{\mathbf{d}}
\newcommand{\bm}{\mathbf{m}}
\newcommand{\bo}{\mathbf{o}}
\newcommand{\br}{\mathbf{r}}
\newcommand{\bs}{\mathbf{s}}
\newcommand{\bx}{\mathbf{x}}
\newcommand{\bu}{\mathbf{u}}
\newcommand{\state}{\mathbf{s}}
\newcommand{\ie}{i.e., }
\newcommand{\eg}{e.g., }

\newcommand{\E}{\mathbb{E}}
\newcommand{\R}{\mathbb{R}}
\newcommand{\N}{\mathbb{N}}
\newcommand{\Z}{\mathbb{Z}}
\newcommand{\mO}{\mathcal{O}}
\newcommand{\mP}{\mathbb{P}}

\newtheorem{theorem}{Theorem}
\newtheorem{corollary}{Corollary}[theorem]
\newtheorem{lemma}[theorem]{Lemma}

% \algrenewcommand\algorithmicrequire{\textbf{Input:}}
\algrenewcommand\algorithmicensure{\textbf{Output:}}

% When the amsmath package is in use page breaks between equation lines are normally disallowed; the philosophy is that page breaks in such material should receive individual attention from the author. To get an individual page break inside a particular displayed equation, a \displaybreak command is provided. \displaybreak is best placed immediately before the \\ where it is to take effect. Like LATEX’s \pagebreak, \displaybreak takes an optional argument between 0 and 4 denoting the desirability of the pagebreak. \displaybreak[0] means "it is permissible to break here" without encouraging a break; \displaybreak with no optional argument is the same as \displaybreak[4] and forces a break.
%If you prefer a strategy of letting page breaks fall where they may, even in the middle of a multi-line equation, then you might put \allowdisplaybreaks[1] in the preamble of your document. An optional argument 1–4 can be used for finer control: [1] means allow page breaks, but avoid them as much as possible; values of 2,3,4 mean increasing permissiveness. When display breaks are enabled with \allowdisplaybreaks, the \\* command can be used to prohibit a pagebreak after a given line, as usual.
\allowdisplaybreaks[4]

% Daniel: for better argmax and argmin.
\DeclareMathOperator*{\argmax}{arg\,max}
\DeclareMathOperator*{\argmin}{arg\,min}
\DeclareMathOperator*{\minimize}{minimize}

% Daniel: feel free to add your names with your favorite color!
% Wrap around the shared "\remark" command for consistency, and to make it easy to turn all remarks off by defining an empty command. -Jeff
\newcommand{\remark}[3]{\hidable{{\color{#2}[#1: #3]}}}
% \renewcommand{\remark}[3]{} % To disable all remarks, uncomment
\newcommand{\lawrence}[1]{\remark{Lawrence}{orange}{#1}}
\definecolor{britishracinggreen}{rgb}{0.23, 0.53, 0.19}
\newcommand{\baiyu}[1]{\remark{Baiyu}{britishracinggreen}{#1}}
\newcommand{\dave}[1]{\remark{Dave}{red}{#1}}
\newcommand{\daniel}[1]{\remark{Daniel}{blue}{#1}}
\newcommand{\KG}[1]{\remark{KG}{red}{#1}}

\definecolor{navy}{rgb}{0,0,0.5}
\newcommand{\NEW}[1]{{\color{navy} #1}}


% Daniel: putting this here for the algorithm...
\newcommand{\bagging}{SLIP-Bagging\xspace}
\newcommand{\slip}{SLIP\xspace}

% Daniel: AH! importing this package makes it much better to make tables, in my opinion, since we can put a long caption at the table title. It does create a warning but I am not sure how to fix this.
\usepackage[font={small}]{caption}
\def\tablename{Table}


% Daniel: this way of doing it must be tied with \addbibresource and \printbibliography. Also keep hyperref imported LAST. If we are ever submitting to a venue which doesn't all for infinite reference space, adjust maxbibnames to be a little smaller so it will read as author et al.
\usepackage[backend=biber,
            hyperref=true,
            url=false,
            isbn=false,
            doi=false,
            backref=false,
            style=ieee,
            natbib=true,%compatibility aliases
            mincitenames=1,
            maxcitenames=1,
            citestyle=numeric-comp,
            sorting=none,%none vs nyt
            block=none,
            maxbibnames=99]{biblatex}
\usepackage{hyperref}
