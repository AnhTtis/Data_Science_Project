% Template for ICASSP-2021 paper; to be used with:
%          spconf.sty  - ICASSP/ICIP LaTeX style file, and
%          IEEEbib.bst - IEEE bibliography style file.
% --------------------------------------------------------------------------
\documentclass{article}
\usepackage{spconf,amsmath,graphicx}
\usepackage{authblk}

% Example definitions.
% --------------------
\def\x{{\mathbf x}}
\def\L{{\cal L}}
\renewcommand{\nu}{\ensuremath{\mathbf{n}(\mathbf{u})}\xspace}  % the normal vector at pixel location \V{u}
\newcommand{\pu}{\ensuremath{\mathbf{p}(\mathbf{u})}\xspace}   % the 3d point correspoinding the pixel \V{u}
\newcommand{\du}{\ensuremath{d(\mathbf{u})}\xspace}  
\newcommand{\zu}{\ensuremath{z(\mathbf{u})}\xspace}
\newcommand{\eu}{\ensuremath{\mathbf{e}(\mathbf{u})}\xspace}
\newcommand{\up}{\ensuremath{\V{u}_{\V{p}}}\xspace}
\newcommand{\tup}{\ensuremath{\tilde{\V{u}}_{\V{p}}}\xspace}

\newcommand{\oz}{\ensuremath{\Omega_z}\xspace}  
\newcommand{\on}{\ensuremath{\Omega_n}\xspace}
\newcommand{\Nu}{\ensuremath{\mathcal{N}(\V{u})}\xspace}

\renewcommand{\ni}{normal integration\xspace}
\newcommand{\NI}{Normal Integration\xspace}
\newcommand{\dpe}{discrete Poisson's equation\xspace}
\newcommand{\Dpe}{Discrete Poisson's equation\xspace}


\newcommand{\z}{\ensuremath{\V{z}}\xspace}
\newcommand{\zs}{\ensuremath{\V{z}^*}\xspace}
\newcommand{\rz}{\ensuremath{\red{\V{z}}}\xspace}
\newcommand{\zt}{\ensuremath{\V{z}_{t}}\xspace}
\newcommand{\zto}{\ensuremath{\V{z}_{t+1}}\xspace}
\newcommand{\R}{\ensuremath{\mathbb{R}}\xspace}
\newcommand{\fz}{\ensuremath{f(\V{z})}\xspace}

\newcommand{\rt}{\ensuremath{\V{r}_{t}}\xspace}
\newcommand{\rto}{\ensuremath{\V{r}_{t+1}}\xspace}


\newcommand{\dup}{\ensuremath{\V{D}_u^{+}}\xspace}
\newcommand{\dun}{\ensuremath{\V{D}_u^{-}}\xspace}
\newcommand{\dvp}{\ensuremath{\V{D}_v^{+}}\xspace}
\newcommand{\dvn}{\ensuremath{\V{D}_v^{-}}\xspace}
\newcommand{\nx}{\ensuremath{\V{n}_x}\xspace}
\newcommand{\ny}{\ensuremath{\V{n}_y}\xspace}
\newcommand{\nz}{\ensuremath{\V{n}_z}\xspace}
\newcommand{\Nz}{\ensuremath{\V{N}_z}\xspace}

\newcommand{\ft}{\ensuremath{F(\red{\V{z}};\V{z}_t)}\xspace}
\newcommand{\ftt}{\ensuremath{F(\V{z}_t;\V{z}_t)}\xspace}
\newcommand{\fto}{\ensuremath{F(\V{z}_{t+1};\V{z}_t)}\xspace}

\newcommand{\dpu}{\ensuremath{\partial_u \V{p}}\xspace}
\newcommand{\dpv}{\ensuremath{\partial_v \V{p}}\xspace}

\renewcommand{\u}{\ensuremath{\V{u}}\xspace}
\newcommand{\dzdu}{\ensuremath{\partial_u z}\xspace}
\newcommand{\dzdv}{\ensuremath{\partial_v z}\xspace}
\newcommand{\dztdu}{\ensuremath{\partial_u \tilde{z}}\xspace}
\newcommand{\dztdv}{\ensuremath{\partial_v \tilde{z}}\xspace}
\newcommand{\dzpdu}{\ensuremath{\partial_{u}^{+} z}\xspace}
\newcommand{\dzpdv}{\ensuremath{\partial_{v}^{+} z}\xspace}
\newcommand{\dzndu}{\ensuremath{\partial_{u}^{-} z}\xspace}
\newcommand{\dzndv}{\ensuremath{\partial_{v}^{-} z}\xspace}

\newcommand{\dzpduv}{\ensuremath{\partial_{\{u,v\}}^{+} z}\xspace}
\newcommand{\dznduv}{\ensuremath{\partial_{\{u,v\}}^{-} z}\xspace}
\newcommand{\dzduv}{\ensuremath{\partial_{\{u,v\}} z}\xspace}

\newcommand{\dupz}{\ensuremath{\Delta_{u}^{+} z}\xspace}
\newcommand{\dunz}{\ensuremath{\Delta_{u}^{-} z}\xspace}
\newcommand{\dvpz}{\ensuremath{\Delta_{v}^{+} z}\xspace}
\newcommand{\dvnz}{\ensuremath{\Delta_{v}^{-} z}\xspace}

\newcommand{\nuv}{\ensuremath{\V{n}(u,v)}\xspace}
\newcommand{\zuv}{\ensuremath{z(u,v)}\xspace}
\newcommand{\puv}{\ensuremath{\V{p}(u,v)}\xspace}

\newcommand{\halfpi}{\ensuremath{\pm {\pi \over 2}}\xspace}


\newcommand{\curve}{\ensuremath{\mathbb{S}}\xspace}
\newcommand{\zenith}{zenith\xspace}
\newcommand{\surface}{\ensuremath{\mathcal{M}}\xspace}
\newcommand{\visibility}{\ensuremath{\Phi_{i}}\xspace}
\newcommand{\point}{\ensuremath{\V{x}}\xspace}
\newcommand{\normal}{\ensuremath{\V{n}}\xspace}
\newcommand{\tangent}{\ensuremath{\V{t}}\xspace}
\newcommand{\cameraNum}{\ensuremath{C}\xspace}
\newcommand{\cameraCenter}{\ensuremath{\V{o}_{i}}\xspace}
\newcommand{\viewDirection}{\ensuremath{\V{v}}\xspace}
\newcommand{\batchsize}{\ensuremath{P}\xspace}
\newcommand{\mask}{\ensuremath{O}\xspace}
\newcommand{\projectedTangentVector}{projected tangent vector\xspace}
\newcommand{\projectedTangentVectors}{projected tangent vectors\xspace}
\newcommand{\stackedTangentVectors}{\ensuremath{\V{T}(\point)}\xspace}
\newcommand{\diligentmv}{\mbox{DiLiGenT-MV}\xspace}
\newcommand{\diligent}{DiLiGenT}
\newcommand{\loss}{\mathcal{L}\xspace}
\newcommand{\opticalAxis}{\ensuremath{\V{e}_{z}\xspace}}
\newcommand{\opticalAxisViewI}{\ensuremath{\V{e}_{z_{i}}}\xspace}
\newcommand{\opticalAxisMatrix}{\ensuremath{\V{C}}\xspace}
\newcommand{\ms}{Mumford-Shah integrator\xspace}
\newcommand{\made}{MADE\xspace}

\newcommand{\pandora}{\mbox{PANDORA}\xspace}
\newcommand{\psnerf}{\mbox{PS-NeRF}\xspace}
\newcommand{\sdps}{\mbox{SDPS}\xspace}
\newcommand{\uanet}{\mbox{UA-MVPS}\xspace}
\newcommand{\rmvps}{\mbox{R-MVPS}\xspace}
\newcommand{\bmvps}{\mbox{B-MVPS}\xspace}
\newcommand{\volsdf}{\mbox{VolSDF}\xspace}
\newcommand{\unisurf}{\mbox{UNISURF}\xspace}


\newcommand{\mvas}{MVAS\xspace}

\newcommand{\tsc}{\mbox{TSC}\xspace}

\newcommand{\pointOne}{\ensuremath{\point_1}\xspace}
\newcommand{\pointTwo}{\ensuremath{\point_2}\xspace}
\newcommand{\pointsetOne}{\ensuremath{\chi_{1}}\xspace}
\newcommand{\pointsetTwo}{\ensuremath{\chi_{2}}\xspace}
\newcommand{\fscoreThreshold}{\ensuremath{\tau}\xspace}
\newcommand{\chamferDist}{\ensuremath{d(\pointsetOne, \pointsetTwo)}\xspace}
\newcommand{\precision}{\ensuremath{\mathcal{P}}\xspace}
\newcommand{\recall}{\ensuremath{\mathcal{R}}\xspace}
\newcommand{\fscore}{\ensuremath{\mathcal{F}}\xspace}

\newcommand{\phaseangle}{\ensuremath{\hat{\phi}}\xspace}
\newcommand{\azimuthangle}{\ensuremath{\phi}\xspace}

\newcommand{\colorbar}[3]{
\begin{tabular}[t]{@{}l@{}l@{}}
	\includegraphics[height=#1\linewidth,width=0.5em]{colorbar.pdf} & 
	\begin{tabular}[b]{@{}l}
		#2 \\ [#3pt]
		$0$
	\end{tabular}
\end{tabular}
}



% Title.
% ------
\title{Image Reconstruction Without Explicit Priors}
%
% Single address.
% ---------------
\name{Angela F. Gao$^{1*}$, Oscar Leong$^{1*}$, He Sun$^2$, Katherine L. Bouman$^1$}
\address{$^1$Computing and Mathematical Sciences, California Institute of Technology,  $^2$Peking University
\newline $^*$ denotes equal contribution}
% \author[1]{Angela F. Gao}
% \affil[1]{Computing and Mathematical Sciences, California Institute of Technology}
% %
% For example:
% ------------
%\address{School\\
%	Department\\
%	Address}
%
% Two addresses (uncomment and modify for two-address case).
% ----------------------------------------------------------
%\twoauthors
%  {A. Author-one, B. Author-two\sthanks{Thanks to XYZ %agency for funding.}}%
%	{School A-B\\
%	Department A-B\\
%	Address A-B}
%  {C. Author-three, D. Author-four\sthanks{The fourth author performed the work	while at ...}}
%	{School C-D\\
%	Department C-D\\
%	Address C-D}
%
\begin{document}
%\ninept
%
\setlength{\abovedisplayskip}{6pt}
\setlength{\belowdisplayskip}{6pt}
 \setlength{\textfloatsep}{10pt }
 \setlength{\abovecaptionskip}{1pt} 
 % \setlength{\belowcaptionskip}{1pt} 
\maketitle
%
\begin{abstract}
%We consider solving ill-posed imaging inverse problems under a generic forward model. Due to the ill-posedness of such problems, prior models that encourage certain image-based structure are required to reduce the space of possible images when solving for a solution. In many instances, obtaining access to a prior may be difficult or impossible. In contrast, we propose to solve inverse problems by directly learning an image prior from a collection of corrupted observations, without incorporating prior constraints on image structure. The key assumption of our work is that the ground-truth images we aim to reconstruct share common, low-dimensional structure. We show that one can capitalize on this structure by learning an image generator with a low-dimensional latent space such that each image posterior is modeled by the push-forward of the generator and a latent variational distribution. The parameters of the generator and variational distributions are learned by maximizing the Evidence Lower Bound (ELBO). We show that the ELBO is a suitable criterion to learn an image prior with experiments demonstrating that the ELBO can help select generative models directly from corrupted measurements. The framework we propose can handle general corruptions, and we show that only a small collection of measurements ($O(10)$ examples) are sufficient to learn effective priors.

We consider solving ill-posed imaging inverse problems without access to an explicit image prior or ground-truth examples. An overarching challenge in inverse problems is that there are many undesired images that fit to the observed measurements, thus requiring image priors to constrain the space of possible solutions to more plausible reconstructions. However, in many applications it is difficult or potentially impossible to obtain ground-truth images to learn an image prior. Thus, inaccurate priors are often used, which inevitably result in biased solutions. Rather than solving an inverse problem using priors that encode the explicit structure of any one image, we propose to solve a set of inverse problems jointly by incorporating prior constraints on the \textit{collective structure} of the underlying images.The key assumption of our work is that the ground-truth images we aim to reconstruct share common, low-dimensional structure. We show that such a set of inverse problems can be solved simultaneously by learning a shared image generator with a low-dimensional latent space.
%In essence, the core message of our work is that common structure across independent inverse problems is sufficient regularization alone.
%that can efficiently perform posterior sampling for each observation.
%We consider solving ill-posed imaging inverse problems without access to an explicit image prior. Ill-posed inverse problems are common in imaging applications ranging from denoising to recovery from compressed sensing measurements. In these problems, the measurements alone are not enough to constrain recovery of the clean underlying image, and an image prior must be employed to reduce the solution space when solving for the image. However, in many instances obtaining access to an accurate image prior may be difficult or impossible; using an inaccurate prior will result in biased solutions. In contrast, we propose to solve a collection of inverse problems jointly by incorporating prior constraints on the collection of measurements, rather than the image structure itself. The key assumption of our work is that the ground-truth images we aim to reconstruct share common, low-dimensional structure. We show that one can capitalize on this structure by learning an image generator with a low-dimensional latent space such that each image posterior is modeled by the push-forward of the generator and a latent variational distribution. 
The parameters of the generator and latent embedding are learned by maximizing a proxy for the Evidence Lower Bound (ELBO). Once learned, the generator and latent embeddings can be combined to provide reconstructions for each inverse problem.
%We show that the ELBO is a suitable criterion to learn an image ``prior" with experiments demonstrating that the ELBO can help select generative models directly from corrupted measurements. 
The framework we propose can handle general forward model corruptions, and we show that measurements derived from only a few ground-truth images ($O(10)$) are sufficient for image reconstruction without explicit priors. 
%In essence, the core message of our work is that common structure across multiple ill-posed inverse problems is sufficient regularization alone.

%We consider solving ill-posed imaging inverse problems without access to an explicit image prior or ground-truth examples. Ill-posed inverse problems are common in imaging applications ranging from denoising to recovery from compressed sensing measurements. An overarching challenge in these problems is that there are many undesired images that fit to the observed measurements, thus requiring image priors to constrain the space of possible solutions to more plausible reconstructions. However, in many instances either 1) innacurate priors are available, resulting in biased solutions or 2) it is either difficult or impossible to obtain ground-truth images to learn from. In contrast, we propose to solve a collection of inverse problems jointly by incorporating prior constraints on the collection of measurements, rather than the image structure itself. The key assumption of our work is that the ground-truth images we aim to reconstruct share common, low-dimensional structure. 
\end{abstract}
%
\begin{keywords}
Inverse problems, computational imaging, prior models, generative networks, Bayesian inference%, low-level vision, image reconstruction
\end{keywords}
%


\section{Introduction}  \label{sec:intro}

\ifdefined\NEURIPS
\ifdefined\CAMREADY
\def\thefootnote{*}\footnotetext{Equal contribution}\def\thefootnote{\arabic{footnote}}
\fi\fi

Deep learning is delivering unprecedented performance when applied to data modalities involving images, text and audio.
On the other hand, it is known both theoretically and empirically~\citep{shalev2017failures,abbe2018provable} that there exist data distributions over which deep learning utterly fails.
The question of \emph{what makes a data distribution suitable for deep learning} is a fundamental open problem in the field.

A prevalent family of deep learning architectures is that of \emph{locally connected neural networks}.
It includes, among others:
\emph{(i)}~convolutional neural networks, which dominate the area of computer vision; 
\emph{(ii)}~recurrent neural networks, which were the most common architecture for sequence (\eg~text and audio) processing, and are experiencing a resurgence by virtue of S4 models~\citep{gu2022efficiently}; 
and 
\emph{(iii)}~local variants of self-attention neural networks~\citep{tay2021long}.
Conventional wisdom postulates that data distributions suitable for locally connected neural networks are those exhibiting a ``local nature,'' and there have been attempts to formalize this intuition~\citep{zhang2017entanglement,jia2020entanglement,convy2022mutual}.
However, to the best of our knowledge, there are no characterizations providing necessary and sufficient conditions for a data distribution to be suitable to a locally connected neural network.

A seemingly distinct scientific discipline tying distributions and computational models is \emph{quantum physics}.
There, distributions of interest are described by \emph{tensors}, and the associated computational models are \emph{tensor networks}.
While there is shortage in formal tools for assessing the suitability of data distributions to deep learning architectures, there exists a widely accepted theory that allows for assessing the suitability of tensors to tensor networks.
The theory is based on the notion of \emph{quantum entanglement}, which quantifies dependencies that a tensor admits under partitions of its axes (for a given tensor~$\A$ and a partition of its axes to sets $\K$ and~$\K^c$, the entanglement is a non-negative number quantifying the dependence that~$\A$ induces between $\K$ and~$\K^c$).

In this paper, we apply the foregoing theory to a tensor network equivalent to a certain locally connected neural network, and derive theorems by which fitting a tensor is possible if and only if the tensor admits low entanglement under certain \emph{canonical partitions} of its axes.
We then consider the tensor network in a machine learning context, and find that its ability to attain low approximation error, \ie~to express a solution with low population loss, is determined by its ability to fit a particular tensor defined by the data distribution, whose axes correspond to features.
Combining the latter finding with the former theorems, we conclude that a \emph{locally connected neural network is capable of accurate prediction over a data distribution if and only if the data distribution admits low entanglement under canonical partitions of features}.
Experiments with different datasets corroborate this conclusion, showing that the accuracy of common locally connected neural networks (including modern convolutional, recurrent, and local self-attention neural networks) is inversely correlated to the entanglement under canonical partitions of features in the data (the lower the entanglement, the higher the accuracy, and vice versa).

The above results bring forth a recipe for enhancing the suitability of a data distribution to locally connected neural networks: given a dataset, search for an arrangement of features which leads to low entanglement under canonical partitions, and then arrange the features accordingly.
Unfortunately, the above search is computationally prohibitive.
However, if we employ a certain correlation-based measure as a surrogate for entanglement, \ie~as a gauge for dependence between sides of a partition of features, then the search converts into a succession of \emph{minimum balanced cut} problems, thereby admitting use of well-established graph theoretical tools, including ones designed for large scale~\citep{karypis1998fast,spielman2011spectral}.
We empirically evaluate this approach on various datasets, demonstrating that it substantially improves prediction accuracy of common locally connected neural networks (including modern convolutional, recurrent, and local self-attention neural networks).

The data modalities to which deep learning is most commonly applied~---~namely ones involving images, text and audio~---~are often regarded as natural (as opposed to, for example, tabular data fusing heterogeneous information).
We believe the difficulty in explaining the suitability of such modalities to deep learning may be due to a shortage in tools for formally reasoning about natural data.
Concepts and tools from physics~---~a branch of science concerned with formally reasoning about natural phenomena~---~may be key to overcoming said difficulty.
We hope that our use of quantum entanglement will encourage further research along this line.

\medskip

The remainder of the paper is organized as follows.
Section~\ref{sec:relate} reviews related work.
Section~\ref{sec:prelim} establishes preliminaries, introducing tensors, tensor networks and quantum entanglement.
Section~\ref{sec:fit_tensor} presents the theorems by which a tensor network equivalent to a locally connected neural network can fit a tensor if and only if this tensor admits low entanglement under canonical partitions of its axes. 
Section~\ref{sec:accurate_predict} employs the preceding theorems to show that in a classification setting, accurate prediction is possible if and only if the data admits low entanglement under canonical partitions of features.
Section~\ref{sec:enhancing} translates this result into a practical method for enhancing the suitability of data to common locally connected neural networks.
Finally, Section~\ref{sec:conc} concludes.
For simplicity, we treat locally connected neural networks whose input data is one-dimensional (\eg~text and audio), and defer to~\cref{app:extension_dims} an extension of the analysis and experiments to models intaking data of arbitrary dimension (\eg~two-dimensional images).

\vspace{-4mm}
\section{Approach}
\vspace{-2mm}

\begin{figure}
    \centering
    %\includesvg[width=0.4\textwidth, height=5cm]{figures/model-sel-MNIST-denoising-samples.svg}
    \includegraphics[width=0.48\textwidth]{figures/diagram_bh.pdf}
    \caption{
    We use a set of $N$ measurements $\{y^{(i)}\}_{i=1}^N$ from $N$ different ground-truth images to infer  $\{\phi^{(i)}\}_{i=1}^N$, the parameters of the latent distributions, and $\theta$, the parameters of the generative network $G$. All inferred parameters are colored as blue and the loss is given by Equation \eqref{eqref:learning-objective}. Here, $G\sharp P$ denotes the push-forward of distribution $P$ induced by $G$.}
    \label{fig:arch}
\end{figure}
\vspace{-2mm}


In this work, we propose to solve a set of inverse problems without access to an IGM by assuming that the set of ground-truth images have common, low-dimensional structure. Other works have considered solving linear inverse problems without an IGM \cite{Lehtinenetal18, liu2020rare}, but assume that one has access to multiple independent measurements of a single, static ground-truth image, limiting their applicability to many real-world problems. In contrast, in our work we do not require multiple measurements of the same ground-truth image.
% The measurements are not limited to the assumption that one has access to multiple independent measurements of a single, static target, unlike \cite{Lehtinenetal18, liu2020rare}, but operat, but operat.
% To do so, we learn an IGM. 
%In order to accurately learn an IGM, we motivate the use of the Evidence Lower Bound (ELBO) as a loss by showing that it provides a good criterion for selecting a prior model.

%Building upon this, we show that one can directly learn a prior from corrupted measurements alone by parameterizing the prior model as a deep neural network with a low-dimensional latent distribution. 
%and a latent posterior distribution with deep neural networks and fit the network weights to optimize a proxy for the ELBO. 
%The IGM network weights are shared across all images, capitalizing on the common structure present in the data, while the parameters of each latent distribution are learned jointly with the generator to model each complex image posterior. We will discuss this in more detail in Section \ref{sec:learning}
\vspace{-2mm}

\subsection{Motivation for ELBO as a model selection criterion} \label{sec:ELBO-intro}
\vspace{-1mm}

In order to accurately learn an IGM, we motivate the use of the Evidence Lower Bound (ELBO) as a loss by showing that it provides a good criterion for selecting a prior model. Suppose we are given noisy measurements from a single image: $y = f(x) + \eta$. In order to reconstruct the image $x$, we traditionally first require an IGM $G$ that captures the distribution $x$ was sampled from. A natural approach would be to find or select the model $G$ that maximizes the model posterior distribution $ p(G |y) \propto p(y | G)p(G).$
That is, conditioned on the noisy measurements, find the model of highest likelihood. Unfortunately computing $p(y | G)$ is intractable, but we show that it can be well approximated using the ELBO.
%Due to the intractability of computing $p(y | m)$, we instead consider the Evidence Lower Bound (ELBO). 

To motivate our discussion, consider estimating the conditional posterior $p(x|y,G)$ by learning the parameters $\phi$ of a variational distribution $q_{\phi}(x)$. Observe that the definition of the KL-divergence followed by using Bayes' theorem gives \begin{align*}
    & \infdiv{q_{\phi}(x)}{p(x|y,G)} 
     = \log p(y|G) \\
     & - \mathbb{E}_{x \sim q_{\phi}(x)}[\log p(y|x,G) + \log p(x|G) - \log q_{\phi}(x)].
\end{align*} The ELBO of an IGM $G$ given measurements $y$ is defined by \begin{align}
    \ELBO(G,q_{\phi};y) & := \mathbb{E}_{x \sim q_{\phi}(x)}[\log p(y|x,G) + \log p(x|G) \nonumber \\
    & - \log q_{\phi}(x)].
    \label{eq:ELBO} 
\end{align} Rearranging the above equation and using the non-negativity of the KL-divergence, we see that we can lower bound the model posterior as \begin{align}
    \log p(G|y) \geqslant \ELBO(G, q_{\phi};y) + \log p(G) - \log p(y).
\end{align} Note that $-\log p(y)$ is independent of the parameters of interest, $\phi$. If the variational distribution $q_{\phi}(x)$ is a good approximation to the posterior $p(x|y,G)$, $D_{\mathrm{KL}} \approx 0$ so maximizing $\log p(G|y)$ with respect to $G$ is approximately equivalent to maximizing $\ELBO(G, q_{\phi};y) + \log p(G)$.

 Each term in the ELBO objective encourages certain properties of the IGM $G$. In particular, the first term, $\E_{x \sim q_{\phi}(x)}[\log p(y|x,G)]$, requires that $G$ should lead to an estimate that is consistent with our measurements $y$. 
 %For example, if we have additive Gaussian noise in our measurements, maximizing this term corresponds to minimizing an $\ell_2$ reconstruction loss. 
 The second term, $\E_{x \sim q_{\phi}(x)}[\log p(x|G)]$, encourages images sampled from $q_{\phi}(x)$ to have high likelihood under our model $G$. 
 The final term is the entropy term, $\E_{x \sim q_{\phi}(x)} [-\log q_{\phi}(x)]$, which encourages a $G$ that leads to ``fatter'' minima that are less sensitive to small changes in likely images $x$ under $G$.
% to avoid collapsing to a degenerate solution and introduces a notion of confidence or uncertainty.

%Several works \cite{Abdellatif18, AbdellatifAlquier18, sun2022alpha} have proposed using the ELBO as a model selection criterion, but have not been demonstrated on high-dimensional imaging inverse problems.
\vspace{-4mm}

\subsection{ELBOProxy}
\vspace{-1mm}
Some IGM are explicit (e.g., Gaussian image prior), which allows for direct computation of $\log p(x|G)$. In this case, we can optimize the ELBO defined in Equation~\eqref{eq:ELBO} directly and then perform model selection. However, an important class of IGMs that we are interested in are those given by deep generative networks. Such IGMs are not probabilistic in the usual Bayesian interpretation of a prior, but instead implicitly enforce structure in the data. Moreover, terms such as $\log p(x|G)$ can only be computed directly if we have an injective map \cite{kothari2021trumpets}. This architectural requirement limits the expressivity of the network. Hence, we instead consider a proxy of the ELBO that is especially helpful for deep generative networks. Suppose our image generation model is of the form $x = G(z)$ where $G$ is a generative network and $z$ is a latent vector. Introducing a variational family for our latent representations $z \sim q_{\phi}(z)$ and using $\log p(z| G)$ in place of $\log p(x|G)$, we arrive at the following proxy of the ELBO:
\begin{align}
\ELBOProxy(G, q_{\phi};y) & := \E_{z \sim q_{\phi}(z)}[\log p(y |G(z)) \nonumber\\
& + \log p(z | G) - \log q_{\phi}(z)]. \label{eqref:ELBOProxy}  
\end{align} When $G$ is injective and $q_{\phi}(x)$ is the push-forward of $G$ through $q_{\phi}(z)$, then this proxy is exactly the ELBO in Eq. \eqref{eq:ELBO}. While $G$ may not necessarily be injective, we show empirically that the ELBOProxy is a useful criterion for selecting such models. %Nevertheless, we show empirically that the ELBOProxy is a useful criterion for selecting  generative networks. %While this proxy may not be exact for non-injective $G$, quality generators should not lack injectivity over high likelihood image samples. 


%\subsubsection{Variational family} In practice, there are a number of choices to parameterize the variational families $q_{\theta}(x)$ and $q_{\phi}(z)$. One could, for example, utilize a Gaussian parameterization which would entail learning a mean and covariance $(\mu, \Lambda)$. Another particularly flexible family of functions are Normalizing Flows \cite{Rezendeetal2014, RezendeMohamed2015, dinh2016density}, which are generative models capable of learning an invertible mapping between a simple latent distribution to a more complicated distribution of interest \cite{SunBouman}. %We explore both options in this work, and discuss these ideas in subsequent sections.

\noindent \textbf{Toy example:} To illustrate the use of the $\ELBOProxy$ as a model selection criterion, we conduct the following experiment that asks whether the $\ELBOProxy$ can identify the best model from a given set of plausible IGMs. For this experiment, we use the MNIST dataset \cite{MNIST} and consider two inverse problems: denoising and phase retrieval. We train a generative model $G_{c}$ on each class $c \in \{0,1,2,\dots,9\}$. Hence, $G_{c}$ is learned to generate images from class $c$ via $G_{c}(z)$ where $z \sim \mathcal{N}(0,I)$. Then, given noisy measurements $y_{c}$ of a single image from class $c$, we ask whether the generative model $G_{c}$ from the appropriate class would achieve the best $\ELBOProxy$. For denoising, our measurements are $y_c = x_c + \eta_c$ where $ \eta_c\sim \mathcal{N}(0, \sigma^2 I)$ and $\sigma = \sqrt{0.5}$. For phase retrieval, $y_c = |\mathcal{F}(x_{c})| + \eta_c$ where $\mathcal{F}$ is the Fourier transform and $\eta_{c} \sim \mathcal{N}(0, \sigma^2 I)$ with $\sigma = \sqrt{0.05}$.

We construct $5 \times 10$ arrays for each problem, where in the $i$-th row and $j$-th column, we compute the $-\ELBOProxy$ obtained by using model $G_{{j-1}}$ to reconstruct images from class $i-1$. We calculate $\ELBOProxy(G_{c}, q_{\phi_c};y_c)$ by parameterizing $q_{\phi_c}$ with a Normalizing Flow and optimizing network weights $\phi_c$ to maximize \eqref{eqref:ELBOProxy}. Results are shown in Fig.~\ref{fig:MNIST_ELBO_exp}. We note that all of the correct models are chosen in both denoising and phase retrieval. We also note some interesting cases where the $\ELBOProxy$ values are similar for certain cases, such as when recovering the $3$ or $4$ image when denoising. For example, when denoising the $4$ image, both $G_{4}$ and $G_{9}$ achieve comparable $\ELBOProxy$ values. By carefully inspecting the noisy image of the $4$, one can see that both models are reasonable given the structure of the noise. %It would be natural for a model selection criterion to assign similar values to semantically similar distributions of images.% In Section \ref{sec:results-model-selection}, we provide further experiments on the benefits of using the $\ELBOProxy$ as a model selection criterion.

\begin{figure}
    \centering
    \includegraphics[width=0.48\textwidth]{figures/model_selection.pdf}
    \caption{We consider two inverse problems: denoising and phase retrieval. Left: the two leftmost columns correspond to the ground truth image $x_c$ and the noisy measurements $y_c$. Center: in each row, we show the means of the distribution induced by the push-forward of $G_j$ and each latent distribution $z \sim q_{\phi_j}$ for $j \in \{0,\dots,9\}$. 
    %This mean reconstruction is done by sampling $T=100$ latent vectors $z^{(c)}_t \sim q_{\phi^{(c)}}$ and computing $\frac{1}{T}\sum_{t=1}^T m_c(z^{(c)}_t).$ 
    Right: each row of the array corresponds to the $-\ELBOProxy$ achieved by each model in reconstructing the images. Here, lower is better. Boxes highlighted in green correspond to the best $-\ELBOProxy$ values in each row. In all cases, the correct model was chosen.  
    }
    \label{fig:MNIST_ELBO_exp}
\end{figure}



% \begin{figure}[ht]
%     \centering
%     %\includegraphics[width=0.95\textwidth]{figures/mnistdenoising64.pdf}
%     \caption{\textbf{Denoising improves with more noisy MNIST observations.} We demonstrate our method of learning the IGM to perform denoising for increasing number of noisy images (4, 35, and 75 images from left to right). In each panel, we include the ground truth, noisy measurements, mean of the posterior, and standard deviation of the posterior. We also include the reconstructions using an IGM trained on the full clean MNIST 8's class. We observe that the mean reconstructions and standard deviations from our low-data IGMs become more similar to the full-data IGM with increasing data.}
%     \label{fig:MNIST_denoising}
% \end{figure}
%\vspace{-3mm}
\vspace{-3mm}
\subsection{Learning the IGM to solve inverse problems}\label{sec:learning}

\vspace{-1mm}

As the previous section illustrates, the $\ELBOProxy$ provides a good criterion for choosing an appropriate IGM from noisy measurements. Here, we consider the task of learning the IGM $G$ directly from corrupted data. 
%Learning the IGM helps solve inverse problems by parameterizing the image model as a generative deep neural network and optimizing the $\ELBOProxy$. 
We consider the setting where we have access to a collection of $N$ measurements $y^{(i)} = f(x^{(i)}) + \eta^{(i)}$ for $i \in [N]$. The key assumption we make is that common, low-dimensional structure is shared across the underlying images $\{x^{(i)}\}_{i=1}^N$. %\edit{Note that the forward models and noise distributions need not be the same across a whole set of measurements.} 
%Relative to typical generative modelling-based approaches or supervised learning approaches, we assume we have very few examples $N$, on the order of only 10's of examples. %Note that the goal here is to solve for $m$ while also solving for the variational distribution for each $x^{(i)}$. Inspired by generative neural network-based methods, we propose to parameterize the image model $m$ as a deep neural network and to learn latent variational distributions to capture the measurement posteriors.

%\subsubsection{Learning approach.} 
%As each ground-truth image is drawn from the same distribution of interest, 
We propose to find a \textit{shared} generator $G_{\theta}$ with weights $\theta$ along with latent distributions $q_{\phi^{(i)}}$ that can be used to reconstruct the full posterior of each image $x^{(i)}$ from its corresponding noisy measurements $y^{(i)}$. This approach is illustrated in Fig.~\ref{fig:arch}. Having the generator be shared across all images helps capture their common collective structure. Each corruption, however, could induce its own complicated image posteriors. Hence, we assign each measurement $y^{(i)}$ its own latent distribution to capture the differences in their posteriors. While the learned distribution may not necessarily be the true image posterior (as we are optimizing a proxy of the ELBO), it still captures a distribution of images that fit to the observed measurements.


%We would like our IGM $G$ to capture shared properties of the $N$ ground-truth images underlying the set of measurements. Each corruption, however, could induce its own complicated image posteriors. Thus, we propose to find a \textit{shared} generator $G_{\theta}$ with weights $\theta$ along with latent distributions $q_{\phi^{(i)}}$ that can be used to reconstruct the full posterior of each image $x^{(i)}$ from its corresponding noisy observation $y^{(i)}$. 

 More explicitly, given a set of measurements $\{y^{(i)}\}_{i = 1}^N$, we optimize the $\ELBOProxy$ from Equation \eqref{eqref:ELBOProxy} to jointly infer a  generator $G_{\theta}$ and variational distributions $\{q_{\phi^{(i)}}\}_{i = 1}^N$:
 \vspace{-2mm}
\begin{align}
 \max_{\theta, \phi^{(i)}} & \frac{1}{N}\sum_{i = 1}^N \ELBOProxy(G_{\theta},q_{\phi^{(i)}}; y^{(i)}) + \log p(G_{\theta}). \label{eqref:learning-objective} %\E_{z \sim q_{\phi^{(i)}}(z)}\left[\log p(y^{(i)}|G_m(z)) + \log p(z|G_m) - \log q_{\phi^{(i)}}(z)\right]
\end{align}
The expectation in this objective is approximated via Monte Carlo sampling. In terms of choices for $\log p(G_{\theta})$, we can add additional regularization to promote particular structures, such as smoothness. Here, we consider having sparse neural network weights as a form of implicit regularization and use dropout during training to represent $\log p(G_{\theta})$ \cite{srivastava2014dropout}.
 
 Once a generator $G_{\theta}$ and variational parameters $\{\phi^{(i)}\}_{i=1}^N$ have been learned, we solve the $i$-th inverse problem by simply sampling $\hat{x}^{(i)} = G_{\theta}(\hat{z}^{(i)})$ where $\hat{z}^{(i)} \sim q_{\phi^{(i)}}$. Producing multiple samples for each inverse problem can help visualize the range of uncertainty under the learned IGM $G_{\theta}$, while taking the average of these samples empirically provides clearer estimates with better metrics in terms of PSNR or MSE. We report PSNR outputs in our subsequent experiments. %and also visualize the standard deviation of our reconstructions.%, which helps provide a notion of uncertainty.



\input{sections/sec_results_final}
\vspace{-4mm}
\section{Conclusion}
\vspace{-2mm}
%In this work, we demonstrated how one can solve ill-posed inverse problems without explicit image priors by jointly learning a shared low-dimensional structure of the ground-truth images from corrupted measurements alone. {\color{red} fix beginning of this to be more similar to the intro} 

% We first demonstrated that, given a number of candidate IGMs, a proxy of the ELBO can be used to select the best model.

In this work, we showcased how one can solve a set of inverse problems without an IGM by leveraging common structure present across the underlying ground-truth images. We demonstrated that even with a small set of corrupted measurements, one can jointly solve these inverse problems by directly learning
% , from a set of corrupted measurements, 
an IGM that maximizes a proxy of the ELBO.
Overall, our work showcases the possibilities of solving inverse problems in a ``prior-free'' fashion, free from human bias typical of ill-posed image reconstruction. We believe our approach can aid automatic discovery of novel structure from scientific measurements without access to clean data, leading to potentially new avenues for scientific discovery.
\vspace{1mm}

\noindent \textbf{Acknowledgements}
This work was sponsored by NSF Award 2048237 and 1935980, an Amazon AI4Science Partnership Discovery Grant, and the Caltech/JPL President’s and Director’s Research and Development Fund  (PDRDF).  This research was carried out at the Jet Propulsion Laboratory and Caltech under a contract with the National Aeronautics and Space Administration and funded through the PDRDF.

% References should be produced using the bibtex program from suitable
% BiBTeX files (here: strings, refs, manuals). The IEEEbib.bst bibliography
% style file from IEEE produces unsorted bibliography list.
% -------------------------------------------------------------------------
\bibliographystyle{IEEEbib}
%\bibliography{strings,refs}
\bibliography{references.bib}
\end{document}