\vspace{-4mm}
\section{Conclusion}
\vspace{-2mm}
%In this work, we demonstrated how one can solve ill-posed inverse problems without explicit image priors by jointly learning a shared low-dimensional structure of the ground-truth images from corrupted measurements alone. {\color{red} fix beginning of this to be more similar to the intro} 

% We first demonstrated that, given a number of candidate IGMs, a proxy of the ELBO can be used to select the best model.

In this work, we showcased how one can solve a set of inverse problems without an IGM by leveraging common structure present across the underlying ground-truth images. We demonstrated that even with a small set of corrupted measurements, one can jointly solve these inverse problems by directly learning
% , from a set of corrupted measurements, 
an IGM that maximizes a proxy of the ELBO.
Overall, our work showcases the possibilities of solving inverse problems in a ``prior-free'' fashion, free from human bias typical of ill-posed image reconstruction. We believe our approach can aid automatic discovery of novel structure from scientific measurements without access to clean data, leading to potentially new avenues for scientific discovery.
\vspace{1mm}

\noindent \textbf{Acknowledgements}
This work was sponsored by NSF Award 2048237 and 1935980, an Amazon AI4Science Partnership Discovery Grant, and the Caltech/JPL President’s and Director’s Research and Development Fund  (PDRDF).  This research was carried out at the Jet Propulsion Laboratory and Caltech under a contract with the National Aeronautics and Space Administration and funded through the PDRDF.