\section{DISTRO: Diffusion denoised Smoothing for Robust OOD detection}\label{sec:distro}

In this section, we present our method.
Essentially, it combines three techniques: (i) a diffusion denoiser, (ii) a standard OOD detector, and (iii) a certified binary discriminator.
Each component of this method is designed to overcome a specific problem of ordinary classifiers, as they are not robust to adversarial attacks, either ID or OOD, and do not detect OOD inputs well.

\tikzstyle{denoises} = [draw, fill=lightgray!20, text width=10em, text centered, minimum height=10em, rounded corners]
\tikzstyle{classifier}=[draw, fill=lightgray!20, text width=6em, text centered, minimum height=2em, rounded corners]
\tikzstyle{discriminator}=[draw, fill=lightgray!20, text width=6em, text centered, minimum height=2em, rounded corners]
\tikzstyle{ann} = [right]
\def\blockdist{2.3}
\def\edgedist{2.5}
\pgfdeclarelayer{background}
\pgfdeclarelayer{foreground}
\pgfsetlayers{background,main,foreground}

\begin{figure}[ht]
    \begin{adjustbox}{width=0.45\textwidth,center}
    \begin{tikzpicture}
        \node (denoise) [denoises] {\begin{center}
        Denoiser\\ \vspace{0.5em}
        $\begin{aligned}
            \texttt{denoise}_{\text{once}}(x + \delta; t) \\
            \text{with}\;\delta \sim \set{N}(0, \sigma^2 I)
        \end{aligned}$ \end{center}};

        \path (denoise.east)+(5, 0) node (output) [ann] {$\prob(y \lvert x)$};
        \path (denoise.east)+(\blockdist, 1) node (classifier) [classifier] {
        \begin{center} Classifier\\ $h(\tilde{x})$\end{center}};
        \path (denoise.east)+(\blockdist, -1) node (discriminator) [discriminator] {
        \begin{center} Discriminator \\ $g(\tilde{x})$ \end{center}};

        \path [draw, ->] (denoise.west)+(-1, 0) -- node (input) [above] {$x$} (denoise.west);
        \draw[->] (denoise.east)+(0.5, 0) -- node (input_classifier) [above] {$\tilde{x}$} (denoise.east) -|  ([yshift=2mm] input_classifier.east) |- (classifier.west); % right feedback loop
        \draw[->] (denoise.east)+(0.5, 0) -- (input_classifier) (denoise.east) -|  ([yshift=2mm] input_classifier.east) node[above,pos=0.1] {} |- (discriminator.west); % right feedback loop
        
        \path [draw, ->] (classifier.east) -| ([xshift=5mm] classifier.east) node[above right] {$\prob(y \lvert x, i)$} |- (output.west);
        \path [draw, ->] (discriminator.east) -| ([xshift=5mm] discriminator.east) node[below right] {$\prob(i \lvert x)$} |- (output.west);

        \begin{pgfonlayer}{background}
            % Compute a few helper coordinates
            \path (denoise.west |- classifier.north)+(-0.3,0.5) node (a) {};
            \path (denoise.south -| discriminator.east)+(+0.3,-0.2) node (b) {};
            \path[rounded corners, draw=black!50, dashed] %fill=yellow!20
                (a) rectangle (b);
            \path (classifier.north west)+(-0.2,0.2) node (a) {};
            \path (classifier.south -| classifier.east)+(+0.2,-0.2) node (b) {};
            % \path[fill=blue!10,rounded corners, draw=black!50, dashed]
            %     (a) rectangle (b);
        \end{pgfonlayer}
    \end{tikzpicture}
    \end{adjustbox}
    \caption{Overview of DISTRO.}
    \label{fig:overview}
\end{figure}

In \autoref{fig:overview}, we show an overview of DISTRO. 
First, a diffusion denoiser is employed before the classifier itself to provide robustness against ID attacks. 
As a result, adversarial noise introduced by the attack is mitigated by the denoiser.
This technique has already been proven to be very efficient and does not affect clean accuracy \cite{dds}. 

Secondly, numerous post-hoc OOD detection methods exist. 
The most straightforward being MSP~\cite{msp}, which can be added to the image classifier without retraining or fine-tuning. 
Alternatively, standard OOD detection methods, such as OE~\cite{oe}, VOS~\cite{vos} or LogitNorm~\cite{logitnorm}, could also replace the classifier.

Thirdly, to make the model more robust to OOD adversarial attacks, we add a binary discriminator to the model that is trained to be certifiably robust against OOD attacks. 
Additionally, this discriminator is combined with the OOD detection method from (ii) which is necessary to have the property of asymptotic underconfidence for far-OOD inputs.

%The idea is to apply diffusion denoising models in combination with robust OOD detection methods to improve adversarial and certified robustness for ID accuracy and OOD detection, and without sacrificing clean accuracy.

\textbf{Configuration.}
This method does not require any new technical knowledge. 
We begin by making the assumption that OOD samples are unrelated and thus maximally un-informative to the ID data.
Thus, for every class $y \in \set{Y}$, the conditional distribution on the input $x$ is given as:
\begin{equation}\label{eq:joint_prob}
    \prob(y|x) = \prob(y|x,i)\prob(i|x) + \frac{1}{K}(1-\prob(i|x)),
\end{equation}
where $\prob(i|x)$ is the conditional distribution representing the probability that $x$ is part of the ID, while $\prob(y|x,i)$ is the conditional distribution representing the ID.
Similarly to \citet{prood}, we assign independent models to each distribution:
\begin{itemize}
    \item $\prob(y|x,i) = h(\mathtt{denoise}_{\text{once}}(x+\delta; t))$, where $h:\R^d\to [0, 1]$ is the confidence of the main classifier $F(x)$, and $\tilde{x} = \mathtt{denoise}_{\text{once}}(x+\delta; t)$ represents one single step of denoising operation with $\delta \sim \set{N}(0, \sigma^2 I)$.
    \item $\prob(i|x) = \frac{1}{1+e^{-g(x)}}$, where $g:\R^d\to \R$ refers to a binary discriminator trained in a certified robust manner based on an $\ell_\infty$-threat model as in \citet{good, prood}.
\end{itemize}

As can be seen, the denoiser is the main addition. 
The one-step denoiser $\mathtt{denoise}_{\text{once}}$ estimates the fully denoised image $x$ from the current timestep $t$. 
Then it computes the average between the denoised image and the noisy image from the previous timestep.
As discussed in ~\citet{dds}, multiple applications of the denoiser will only destroy information about $x$.
Denoising with iterative steps essentially transfers the classification task to the denoiser, which can determine how the image should be filled.
For these reason, we apply only a single step of denoising.

\textbf{Asymptotic Underconfidence.}
Here, we show that by coupling a classifier trained to be OOD aware with a diffusion denoiser and running a certified discriminator in parallel, we can guarantee asymptotic underconfidence for data \textit{far enough} from the training distribution.

To obtain asymptotic underconfidence of the joint classifier, we consider $\prob(y|x,i) \leq 1$ and rewrite ~\autoref{eq:joint_prob} as follows:
\begin{equation}
    \prob(y|x) \leq \frac{K-1}{K} \prob(i|x)+\frac{1}{K}.
\end{equation}
Since the right term only depends on $\prob(i|x)$, we just need to assure that $\lim_{\beta \to \infty}\prob(i|\beta x) \to 0$.
If we employ a certified binary discriminator, trained with IBP on OOD data, as descibed in \citet{prood}, to compute $\prob(i|x)$, we achieve asymptotic underconfidence independently of the main classifier.
Readers are referred to \citet{prood} for a more detailed explanation.

\begin{figure*}
\vspace{-0.5em}
    \begin{subfigure}{.45\textwidth}
        \centering
        \includegraphics[width=\textwidth]{graphics/asymptotic_underconfidence.pdf}
        \caption{MSP}
        \label{fig:sub_confidence_msp}
    \end{subfigure}
    \hfill
    \begin{subfigure}{.46\textwidth}
        \centering
        \includegraphics[width=\textwidth]{graphics/asymptotic_energy_2.pdf}
        \caption{Energy}
        \label{fig:sub_confidence_energy}
    \end{subfigure}
    \caption{Asymptotic confidence as: (a) MSP~\cite{msp} and (b) Energy~\cite{energy}, for several OOD detection models divided into two categories: \textit{standard} (continuous line) and \textit{guaranteed} (dashed line).}
    \label{fig:asymptotic_confidence}
\vspace{-1em}
\end{figure*}

\textbf{Empirical Evaluation.}
In \autoref{fig:asymptotic_confidence}, we show an empirical evaluation of the asymptotic confidence for standard and robust OOD detection methods\footnote{the models are described in \autoref{sec:experiments}.}.
In this test, we consider a single ID sample $x$ and multiply by a scalar $\beta$.
In \autoref{fig:sub_confidence_msp} we plot the MSP~\cite{msp} as confidence, while in \autoref{fig:sub_confidence_energy} we plot the Energy~\cite{energy} for increasing values of $\beta > 0$.
In the context of MSP, we observe that standard OOD detection methods are asymptotically overconfident, after a small drop, whereas certified methods such as GOOD~\cite{good}, ProoD~\cite{prood} and DISTRO converge to $1/K$.
On the other hand, for Energy as $\beta$ increases, VOS~\cite{vos}, LogitNorm~\cite{logitnorm}, and Plain models asymptotically decrease, whereas GOOD~\cite{good}, ProoD (Meinke at al., 2022), and DISTRO remain stable. 

As a result, underconfidence can be easily obtained when using an energy score instead of MSP, regardless of whether it is on a plain or OOD aware model. 
\textbf{d}However, asymptotic underconfidence does not necessarily imply that the model will perform better in detecting OOD samples since all inputs are usually normalized to some range (e.g. [0, 1] or [-1, 1]). 
Thus the choice of MSP over the energy function is directly related to the possibility of certified robustness for OOD samples.