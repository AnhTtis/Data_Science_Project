\documentclass[nohyperref]{article}


% hyperref makes hyperlinks in the resulting PDF.
% If your build breaks (sometimes temporarily if a hyperlink spans a page)
% please comment out the following usepackage line and replace
% \usepackage{icml2022} with \usepackage[nohyperref]{icml2022} above.
\usepackage{hyperref}

% Attempt to make hyperref and algorithmic work together better:
\newcommand{\theHalgorithm}{\arabic{algorithm}}

% Use the following line for the initial blind version submitted for review:
% \usepackage{icml2023}

% If accepted, instead use the following line for the camera-ready submission:
\usepackage[accepted]{icml2023}

% For theorems and such
\usepackage{amsmath, amssymb, amsthm}
\usepackage{mathtools, bm, bbm, thm-restate}
% Recommended, but optional, packages for figures and better typesetting:
\usepackage[skip=2pt]{caption}
\usepackage[skip=2pt]{subcaption}
\usepackage{microtype, nicefrac}
\usepackage{graphicx, wrapfig}
\usepackage{booktabs, adjustbox, multirow}
\usepackage{xcolor, colortbl}
\usepackage{tikz, float}
\usetikzlibrary{shapes,arrows}
\usepackage{threeparttable}
% if you use cleveref..
\usepackage[capitalize,noabbrev]{cleveref}

% Optional math commands from https://github.com/goodfeli/dlbook_notation.
\newcommand{\bbox}{\text{bbox}}
\newcommand{\alphapck}{\alpha_\bbox}
\newcommand{\kcycle}{\text{k-CyPCK}}
\newcommand{\cycle}{\text{-CyPCK}}

\newcommand{\I}{\mathbf{I}}
\newcommand{\Ia}{\I^\text{a}}
\newcommand{\Ib}{\I^\text{b}}
\newcommand{\Iatob}{\I^\text{a $\rightarrow$ b}}
\newcommand{\F}{\mathbf{F}}
\newcommand{\Fa}{\F^\text{a}}
\newcommand{\Fb}{\F^\text{b}}
\newcommand{\f}{\mathbf{f}}
\newcommand{\fa}{\f^\text{a}}
\newcommand{\fb}{\f^\text{b}}
\newcommand{\p}{\mathbf{p}}
\newcommand{\pa}{\p^\text{a}}
\newcommand{\pb}{\p^\text{b}}
\newcommand{\A}{\boldsymbol{\Phi}_\text{align}}
\newcommand{\G}{\mathbf{G}}
\newcommand{\C}{\mathbf{C}}
\newcommand{\Ca}{\C^\text{a}}
\newcommand{\Cb}{\C^\text{b}}
\newcommand{\cc}{\mathbf{c}}
\newcommand{\cca}{\cc^\text{a}}
\newcommand{\ccb}{\cc^\text{b}}
\newcommand{\Irec}{\I_\text{Recon}}
\newcommand{\M}{\mathbf{M}}
\newcommand{\Mrec}{\M_\text{Recon}}
\newcommand{\loss}{\mathcal{L}}
\newcommand{\T}{\mathcal{T}}
\newcommand{\W}{\mathcal{W}}
\newcommand{\Id}{\mathcal{I}}


%%%%%%%%%%%%%%%%%%%%%%%%%%%%%%%%
% THEOREMS
%%%%%%%%%%%%%%%%%%%%%%%%%%%%%%%%
\theoremstyle{plain}
\newtheorem{theorem}{Theorem}[section]
\newtheorem{proposition}[theorem]{Proposition}
\newtheorem{lemma}[theorem]{Lemma}
\newtheorem{corollary}[theorem]{Corollary}
\theoremstyle{definition}
\newtheorem{definition}[theorem]{Definition}
\newtheorem{assumption}[theorem]{Assumption}
\theoremstyle{remark}
\newtheorem{remark}[theorem]{Remark}

\newcommand{\norm}[1]{\left\lVert#1\right\rVert}
\DeclareMathOperator{\argmaxinline}{arg\,max}
\DeclareMathOperator{\argmininline}{arg\,min}
\newcommand{\set}[1]{\mathcal{#1}}
\newcommand{\prob}{\mathbb{P}}
\renewcommand{\vec}[1]{\mathbf{#1}}
\newcommand{\symvec}[1]{\bm{#1}}

\newcommand\eqdef{\stackrel{\mathclap{\scriptsize\mbox{def}}}{=}}
\newcommand{\gray}[1]{\textcolor{gray}{#1}}
\def\rowgray{\rowcolor{lightgray!20}}
% Todonotes is useful during development; simply uncomment the next line
%    and comment out the line below the next line to turn off comments
%\usepackage[disable,textsize=tiny]{todonotes}
\usepackage[textsize=tiny]{todonotes}


% The \icmltitle you define below is probably too long as a header.
% Therefore, a short form for the running title is supplied here:
\icmltitlerunning{Diffusion Denoised Smoothing for Certified and Adversarial Robust Out-Of-Distribution Detection}

\begin{document}

\twocolumn[
\icmltitle{Diffusion Denoised Smoothing for Certified and Adversarial \\ Robust Out-Of-Distribution Detection}

% It is OKAY to include author information, even for blind
% submissions: the style file will automatically remove it for you
% unless you've provided the [accepted] option to the icml2022
% package.

% List of affiliations: The first argument should be a (short)
% identifier you will use later to specify author affiliations
% Academic affiliations should list Department, University, City, Region, Country
% Industry affiliations should list Company, City, Region, Country

% You can specify symbols, otherwise they are numbered in order.
% Ideally, you should not use this facility. Affiliations will be numbered
% in order of appearance and this is the preferred way.
\icmlsetsymbol{equal}{*}

\begin{icmlauthorlist}
\icmlauthor{Nicola Franco}{iks}
\icmlauthor{Daniel Korth}{iks}
\icmlauthor{Jeanette Miriam Lorenz}{iks}
\icmlauthor{Karsten Roscher}{iks}
\icmlauthor{Stephan G{\"u}nnemann}{tum}
\end{icmlauthorlist}

\icmlaffiliation{iks}{Fraunhofer Institute for Cognitive Systems IKS, Munich, Germany}
\icmlaffiliation{tum}{Dept. of Informatics \& Munich Data Science Institute, Technical Univ. of Munich, Germany}

\icmlcorrespondingauthor{Nicola Franco}{nicola.franco@iks.fraunhofer.de}
% \icmlcorrespondingauthor{Firstname2 Lastname2}{first2.last2@www.uk}

% You may provide any keywords that you
% find helpful for describing your paper; these are used to populate
% the "keywords" metadata in the PDF but will not be shown in the document
\icmlkeywords{Robust Machine Learning, Robustness Certificates, Out-Of-Distribution, Randomized Smoothing}

\vskip 0.3in
]

% this must go after the closing bracket ] following \twocolumn[ ...

% This command actually creates the footnote in the first column
% listing the affiliations and the copyright notice.
% The command takes one argument, which is text to display at the start of the footnote.
% The \icmlEqualContribution command is standard text for equal contribution.
% Remove it (just {}) if you do not need this facility.

\printAffiliationsAndNotice{}  % leave blank if no need to mention equal contribution
% \printAffiliationsAndNotice{\icmlEqualContribution} % otherwise use the standard text.

\begin{abstract}
As the use of machine learning continues to expand, the importance of ensuring its safety cannot be overstated. 
A key concern in this regard is the ability to identify whether a given sample is from the training distribution, or is an "Out-Of-Distribution" (OOD) sample.
In addition, adversaries can manipulate OOD samples in ways that lead a classifier to make a confident prediction.
In this study, we present a novel approach for certifying the robustness of OOD detection within a $\ell_2$-norm around the input, regardless of network architecture and without the need for specific components or additional training.
Further, we improve current techniques for detecting adversarial attacks on OOD samples, while providing high levels of certified and adversarial robustness on in-distribution samples.
The average of all OOD detection metrics on CIFAR10/100 shows an increase of $\sim 13 \% / 5\%$ relative to previous approaches.
\end{abstract}

\section{Introduction}
\label{sec:introduction}
% \begin{itemize}
%     % Diffusion of FL
%     \item {\st{Diffusion of FL}}
%     % Security threats to FL
%     \item {\st{Security threats to FL with particular focus on model poisoning}}
%     % Limitations of existing countermeasures
%     \item {\st{Current countermeasures (e.g., KRUM) and their limitations}}
%     % Proposed method and its advantages
%     \item {\st{Intuitive description of the proposed method and its difference (i.e., advantages) w.r.t. state of the art}}
%     % Main contributions
%     \item {\st{Summary of the main contributions of this work}}
%     % Paper's structure and organization
%     \item {\st{Paper's structure and organization}}
% \end{itemize}

% Diffusion of FL
Recently, {\em federated learning} (FL) has emerged as the leading paradigm for training distributed, large-scale, and privacy-preserving machine learning (ML) systems~\cite{mcmahan2017googleai,mcmahan2017aistats}. 
The core idea of FL is to allow multiple edge clients to collaboratively train a shared, global model without disclosing their local private training data.
%Specifically, an FL system consists of a central server and many edge clients; 
A typical FL round involves the following steps: {\em(i)} the server randomly picks some clients and sends them the current, global model; {\em(ii)} each selected client locally trains its model with its own private data; then, it sends the resulting local model to the server;\footnote{Whenever we refer to global/local model, we mean global/local model {\em parameters}.} {\em(iii)} the server updates the global model by computing an \emph{aggregation function}, usually the average (FedAvg), on the local models received from clients.
% \begin{enumerate}
%     \item[{\em(i)}] the server sends the current, global model to the clients and appoints some of them for training;
%     \item[{\em(ii)}] each selected client locally trains its copy of the global model with its own private data; then, it sends the resulting local model back to the server;\footnote{Whenever we refer to global/local model, we mean global/local model {\em parameters}.}
%     \item[{\em(iii)}] the server updates the global model by computing an \emph{aggregation function} on the local models received from clients (by default, the average, also referred to as FedAvg~\cite{mcmahan2017aistats}).
% \end{enumerate}
This process goes on until the global model converges. %(e.g., after a certain number of rounds or other similar stopping criteria).
%\\
% The advantages of FL over the traditional, centralized learning paradigm are undoubtedly clear in terms of flexibility/scalability (clients can join/disconnect from the FL network dynamically), network communications (only model weights\footnote{We will use \textit{parameters} and \textit{weights} interchangeably.} are exchanged between clients and server), and privacy (each client's private training data is kept local at the client's end and not uploaded to the server).
\\
% Security threats to FL
%However, the growing adoption of FL also raises security concerns~\cite{costa2022covert}, particularly about its confidentiality, integrity, and availability.
Although its advantages over standard ML, FL also raises security concerns~\cite{costa2022covert}. %, particularly about its confidentiality, integrity, and availability~\cite{costa2022covert}.
% OLD, LONG VERSION
% Indeed, some work deals with privacy leakage that may expose the local data of some clients~\cite{melis2019sp}. 
% A large body of work, instead, investigates attacks that usually aim to detriment the predictive accuracy of the learned global model. For instance, \emph{data poisoning} attacks achieve this goal by letting an adversary pollute the training set of some corrupt FL clients with maliciously crafted examples~\cite{jagielski2018sp}.
% Similarly, in \emph{model poisoning} the attacker attempts to tweak the global model weights~\cite{bhagoji2019pmlr} by directly perturbing the local model's weights of some infected FL clients before these are sent to the central server for aggregation, usually via so-called Byzantine attacks. 
% It turns out that Byzantine model poisoning attacks severely impact standard FedAvg; therefore, more robust aggregation functions must be designed to make FL systems secure.
Here, we focus on \emph{untargeted model poisoning} attacks~\cite{bhagoji2019pmlr}, where an adversary attempts to tweak the global model weights %\footnote{We will use the terms \textit{parameters} and \textit{weights} interchangeably.} 
by directly perturbing the local model's parameters of some infected clients before these are sent to the central server for aggregation.
In doing so, the adversary aims to jeopardize the global model \textit{indiscriminately} at inference time.
Such model poisoning attacks severely impact standard FedAvg; therefore, more robust aggregation functions must be designed to secure FL systems.
\\
% In this paper, we focus on designing a novel robust aggregation scheme at the server's end to contrast the effect of Byzantine model poisoning attacks.
%
% Current countermeasures and their limitations
%Several countermeasures have been proposed in the literature to combat model poisoning attacks on FL systems.
% Some methods use simple statistics more robust than plain average to smooth the impact of malicious updates (e.g., Trimmed Mean and FedMedian~\cite{yin2018icml}). 
% Other defenses implement outlier detection techniques to discard malicious updates from the aggregation performed at the server's end. Those are either based on heuristics (e.g., Krum/Multi-Krum~\cite{blanchard2017nips} and Bulyan~\cite{mhamdi2018pmlr}) or data-driven approaches (e.g., K-means clustering~\cite{shen2016acm} or DnC via spectral analysis~\cite{shejwalkar2021ndss}). 
% Finally, some strategies rely on a centralized ``source of trust'' to spot potential malicious updates (e.g., FLTrust~\cite{cao2020fltrust}).
% Several countermeasures have been proposed in the literature to combat model poisoning attacks on FL systems, i.e., to discard possible malicious local updates from the aggregation performed at the server's end. 
% These techniques range from simple statistics more robust than plain average (e.g., Trimmed Mean and FedMedian~\cite{yin2018icml}) to outlier detection heuristics (e.g., Krum/Multi-Krum~\cite{blanchard2017nips} and Bulyan~\cite{mhamdi2018pmlr}) or data-driven approaches (e.g., spectral analysis via K-means clustering~\cite{shen2016acm} or spectral analysis), or methods based on ``source of trust'' (e.g., FLTrust~\cite{cao2020fltrust}).
% OLD, LONG VERSION
%Several countermeasures have been proposed in the literature to combat Byzantine model poisoning attacks on FL systems.
% Descriptive statistics
% For example, Trimmed Mean and FedMedian aggregate local model updates using more robust statistics than standard average~\cite{yin2018icml}.
%
% % Heuristics for outlier detection
% Many existing Byzantine-resilient strategies implement some outlier detection heuristics to discard the model updates sent by potentially malicious clients from the input of the aggregation function.
% One of the most popular heuristics is Krum~\cite{blanchard2017nips}.
% This strategy tries to mitigate the impact of Byzantine attacks by selecting as a global model the local model with the smallest sum of Euclidean distances to {\em all} the other local models.
% Although powerful, Krum requires the server to know (or, at least, estimate) the number of malicious FL clients upfront, which is generally impossible in a realistic attack scenario. %
% Moreover, Krum may become ineffective for complex, high-dimensional model parameter spaces due to the curse of dimensionality.
% Bulyan~\cite{mhamdi2018pmlr} tries to overcome this issue by combining Krum with a variant of Trimmed Mean.
% % Data-driven outlier detection
% Other strategies use data-driven outlier detection techniques -- e.g., via K-means clustering~\cite{shen2016acm} -- to spot potential malicious local model updates. 
% %For instance, Shen et al. propose to cluster local model updates with K-means and thus identify outliers.
%
% % Other techniques
% As far as the server is concerned, any local model received can be from a potential malicious client. 
% FLTrust~\cite{cao2020fltrust} assumes the server acts as a client, i.e., trains a local model on an additional {\em trustworthy} dataset at the server's end and compares it against all the local models from other clients. 
% This way, the server can rely on some ``source of trust'' when discarding potentially malicious clients.
%\\
% Limitations of existing Byzantine-resilient strategies
Unfortunately, existing defense mechanisms either rely on simple heuristics (e.g., Trimmed Mean and FedMedian by~\cite{yin2018icml}) or need strong and unrealistic assumptions to work effectively (e.g., foreknowledge or estimation of the number of malicious clients in the FL system, as for Krum/Multi-Krum~\cite{blanchard2017nips} and Bulyan~\cite{mhamdi2018pmlr}, which, however, cannot exceed a fixed threshold).
Furthermore, outlier detection methods using K-means clustering~\cite{shen2016acm} or spectral analysis like DnC~\cite{shejwalkar2021ndss} do not directly consider the temporal evolution of local model updates received.
Finally, strategies like FLTrust~\cite{cao2020fltrust} require the server to collect its own dataset and act as a proper client, thereby altering the standard FL protocol.
\\
% OLD, LONG VERSION
% Overall, existing Byzantine-resilient strategies are either simple heuristics (e.g., FedMedian) or, if they are more complex, they rely on strong and unrealistic assumptions to work effectively (e.g., knowing the number of malicious clients in the FL system in advance, as for Krum and alike).
% Furthermore, data-driven outlier detection methods do not consider the temporary evolution of local model updates received (e.g., K-means clustering). 
% Finally, strategies like FLTrust requires the server to collect its own dataset and act as a proper client, thereby altering the standard FL protocol.
%
% Description of the proposed method
This work introduces a novel pre-aggregation \textit{filter} robust to untargeted model poisoning attacks. Notably, this filter $(i)$ operates without requiring prior knowledge or constraints on the number of malicious clients and $(ii)$ inherently integrates temporal dependencies. 
The FL server can employ this filter as a preprocessing step before applying \textit{any} aggregation function, be it standard like FedAvg or robust like Krum or Bulyan.
Specifically, we formulate the problem of identifying corrupted updates as a multidimensional (i.e., matrix-valued) time series anomaly detection task. 
The key idea is that legitimate local updates, resulting from well-calibrated iterative procedures like stochastic gradient descent (SGD) with an appropriate learning rate, show \textit{higher predictability} compared to malicious updates. This hypothesis stems from the fact that the sequence of gradients (thus, model parameters) observed during legitimate training exhibit regular patterns, as validated in Section~\ref{subsec:intuition}. %until convergence. 
%This regularity may be more pronounced for smooth convex loss functions, but it can still be captured within an appropriate time window, even for more complex and convoluted loss surfaces. 
%We provide evidence of this claim in Appendix~B, where we show that the average mutual information (i.e., ``predictability''), calculated over pairs of legitimate model updates sent at different FL rounds, is significantly higher than the corresponding computation for a malicious client.
\\
Inspired by the matrix autoregressive (MAR) framework for multidimensional time series forecasting~\cite{chen2021je}, we propose the FLANDERS ({\em \textbf{F}ederated \textbf{L}earning meets \textbf{AN}omaly \textbf{DE}tection for a \textbf{R}obust and \textbf{S}ecure}) filter.
The main advantages of FLANDERS over existing strategies like FLDetector~\cite{zhao2020multivariate} are its resilience to large-scale attacks, where $50\%$ or more FL participants are hostile, and the capability of working under realistic non-iid scenarios.
We attribute such a capability to two key factors: $(i)$ FLANDERS works without knowing a priori the ratio of corrupted clients, and $(ii)$ it embodies temporal dependencies between intra- and inter-client updates, quickly recognizing local model drifts caused by evil players. Below, we summarize our main contributions:

\begin{itemize}
\item[{\em(i)}]
We provide empirical evidence that the sequence of models sent by legitimate clients is more predictable than those of malicious participants performing untargeted model poisoning attacks.
\\
\item[{\em(ii)}] 
We introduce FLANDERS, the first pre-aggregation filter for FL robust to untargeted model poisoning based on multidimensional time series anomaly detection.
\\
\item[{\em(iii)}] 
We integrate FLANDERS into Flower,\footnote{\scriptsize{\url{https://flower.dev/}}} a popular FL simulation framework for reproducibility.
\\
\item[{\em(iv)}] 
We show that FLANDERS improves the robustness of the existing aggregation methods under multiple settings: different datasets, client's data distribution (non-iid), models, and attack scenarios.
\\
\item[{\em(v)}] 
We publicly release all the implementation code of FLANDERS along with our experiments.\footnote{\scriptsize{\url{https://anonymous.4open.science/r/flanders_exp-7EEB}}}
\end{itemize}

% Paper's structure and organization
The remainder of the paper is structured as follows. %some related work and the current state-of-the-art solutions to security issues that FL entails. 
Section~\ref{sec:background} covers background and preliminaries. 
In Section~\ref{sec:related}, we discuss related work.
Section~\ref{sec:problem} and Section~\ref{sec:method} describe the problem formulation and the method proposed. % to tackle it. 
Section~\ref{sec:experiments} gathers experimental results. %, and Section~\ref{sec:limitations} discusses some limitations of this work.
Finally, we conclude in Section~\ref{sec:conclusion}.
 %discusses the limitations of this work and draws future research directions.
%reports conclusions and draws perspectives for future research directions.

%%%%%%% OLD %%%%%%%
%to overcome the resilience of Byzantine failures in distributed Stochastic Gradient Descent computations. 
% The strength of Krum is its time complexity, which is linear in the gradient dimension. 
% However, the robustness of the approach is guaranteed for gradient-based learning applications only when the majority of the clients are not compromised. 
% Besides, the aggregation mechanism of Krum, as well as that of similar methods, is robust from a coarse-grained perspective and does not provide solutions to errors and perturbations that may occur at inference time.
%A related approach to~\cite{blanchard2017nips} is the work of Su et al.~\cite{su2016dc}. Here, the authors propose an iterated approximate agreement to tackle a multi-layer scenario attacked by Byzantine agents. 
%However, the method works efficiently on the sole discrete context and it is inapplicable to continuous state environments.
%\gabri{Maybe, we should just talk about the main limitations of existing countermeasures without digging into their details (or, we can just mention Krum as this is the most popular one). I will move the description of all these methods to the Related Work section.}
\setlength{\tabcolsep}{1.6mm}{
\renewcommand\arraystretch{1.1}
\begin{table}[ht]
  \centering
  \scalebox{0.9}{
  \begin{tabular}{llcccc}
    \toprule
    &\multirow{2}*{Methods} & \multirow{2}*{Sal.} &   \multicolumn{2}{c}{VOC} & MS~COCO \\
    \cmidrule(r){4-5}\cmidrule(r){6-6}
    &&&\texttt{val}&\texttt{test}&\texttt{val}\\
    \hline
    \multirow{13}*{\rotatebox{90}{ResNet-50}}
    &IRN~\cite{irn}          \tiny{CVPR'19}     &              & 63.5       & 64.8          & 42.0  \\
    &LayerCAM~\cite{layercam}\tiny{TIP'21}      &              & 63.0       & 64.5          & -     \\
    &AdvCAM~\cite{advcam}    \tiny{CVPR'21}     &              & 68.1       & 68.0          & 44.2  \\
    &RIB~\cite{rib}          \tiny{NeurIPS'21}  &              & 68.3       & 68.6          & 44.2  \\
    &ReCAM~\cite{recam}      \tiny{CVPR'22}     &              & 68.5       & 68.4          & 42.9  \\
    % \rowcolor{Gray}
    &\cellcolor{Gray}IRN+\texttt{LPCAM}    &\cellcolor{Gray} & \cellcolor{Gray}68.6    & \cellcolor{Gray}68.7      & \cellcolor{Gray}44.5  \\
    &SIPE~\cite{sipe}        \tiny{CVPR'22}     &              & 68.8       & 69.7          & 40.6  \\
    &OOD~\cite{ood}+Adv      \tiny{CVPR'22}     &              & 69.8       & 69.9          & -     \\
    &AMN~\cite{amn}          \tiny{CVPR'22}     &              & 69.5       & 69.6          & 44.7  \\
    &\cellcolor{Gray}AMN+\texttt{LPCAM}    &\cellcolor{Gray} & \cellcolor{Gray}70.1    &\cellcolor{Gray} 70.4      & \cellcolor{Gray}45.5  \\ 
    &ESOL~\cite{esol}        \tiny{NeurIPS'22}  &              & 69.9$^*$   & 69.3$^*$      & 42.6  \\
    &CLIMS~\cite{clims}      \tiny{CVPR'22}     &              & 70.4$^*$   & 70.0$^*$      & -     \\
    &EDAM~\cite{edam}        \tiny{CVPR'21}     &\checkmark    & 70.9$^*$   & 71.8$^*$      & -     \\
    &\cellcolor{Gray}EDAM+\texttt{LPCAM}  &\cellcolor{Gray}\checkmark & \cellcolor{Gray}71.8$^*$ &\cellcolor{Gray} 72.1$^*$& \cellcolor{Gray}42.1\\
    \hline
    \multirow{9}*{\rotatebox{90}{WideResNet-38}}
    &Spatial-BCE~\cite{sbce} \tiny{ECCV'22}     &              & 70.0       & 71.3      & 35.2  \\
    &BDM~\cite{bdm}          \tiny{ACMMM'22}    &\checkmark    & 71.0       & 71.0      & 36.7  \\ 
    &RCA~\cite{rca}+OOA      \tiny{CVPR'22}     &\checkmark    & 71.1       & 71.6      & 35.7  \\
    &RCA~\cite{rca}+EPS      \tiny{CVPR'22}     &\checkmark    & 72.2       & 72.8      & 36.8  \\
    &HGNN~\cite{hgnn}        \tiny{ACMMM'22}    &\checkmark         & 70.5$^*$   & 71.0$^*$  & 34.5  \\ 
    &EPS~\cite{eps}          \tiny{CVPR'21}     &\checkmark         & 70.9$^*$   & 70.8$^*$  & -     \\
    &RPIM~\cite{rpim}        \tiny{ACMMM'22}    &\checkmark         & 71.4$^*$   & 71.4$^*$  & -     \\ 
    &L2G~\cite{l2g}          \tiny{CVPR'22}     &\checkmark         & 72.1$^*$   & 71.7$^*$  & 44.2  \\
    \hline
    \multirow{2}*{\rotatebox{90}{\small{DeiT-S}}}
    &MCTformer~\cite{mctformer}    \tiny{CVPR'22}     &                 & 71.9$^{\dag}$  & 71.6$^{\dag}$   & 42.0  \\
    &\cellcolor{Gray}MCTformer+\texttt{LPCAM}      &\cellcolor{Gray} & \cellcolor{Gray}72.6$^{\dag}$  & \cellcolor{Gray}72.4$^{\dag}$  &\cellcolor{Gray} 42.8 \\
    \bottomrule
  \end{tabular}}
  \vspace{-2mm}
  \caption{The mIoU results (\%) based on DeepLabV2 on VOC and MS~COCO. The side column shows three backbones of multi-label classification model. ``Sal.'' denotes using saliency maps. * denotes the segmentation model is pre-trained on MS~COCO. $^\dag$ denotes the segmentation model is pre-trained on VOC.
  }
  \vspace{-6mm}
  \label{table_related}
\end{table}
}


\section{Background on Network Calculus}
\label{sec: background}


\begin{figure*}[tbh]
\centering
\begin{subfigure}[b]{0.3\textwidth}
    \centering
    \includegraphics[width=\linewidth]{images/in-out.png}
    \caption{Arrival and departure data and their relation with delay $d(t)$ and backlog $b(t)$. For a FIFO system, the delay is the horizontal distance between $R(t)$ and $R^*(t)$ but some other multiplexing techniques may shift the data to a later priority, causing a longer delay.}
    \label{fig: data in-out}
\end{subfigure}
\hfill
\begin{subfigure}[b]{0.35\textwidth}
    \centering
    \includegraphics[width=\linewidth]{images/arrival-service.png}
    \caption{Characteristics of an arrival curve and a service curve. From any point of observation, the arriving data never exceeds its arrival curve; the departure data is also never less than the service curve with respect to the data arrival.}
    \label{fig: arrival-service curves}
\end{subfigure}
\hfill
\begin{subfigure}[b]{0.33\textwidth}
    \centering
    \includegraphics[width=\linewidth]{images/bound.png}
    \caption{Delay and backlog bounds of a system. Backlog is the maximum vertical distance between $\alpha(t)$ and $\beta(t)$; FIFO delay is their maximum horizontal distance; but for arbitrary multiplexing, the delay guarantee is when the system clears its buffer, thus it's the intersection of $\alpha(t)$ and $\beta(t)$.}
    \label{fig: system bounds}
\end{subfigure}
\caption{Network calculus framework. We let $R(t)$ and $R^*(t)$ be the arrival and departure data flow of a system; $\alpha(t)$ be the piecewise linear concave arrival curve and $\beta(t)$ be the piecewise linear convex service curve of a system.}
% \hossein{Better to show piece-wise linear concave arrival curve and piece-wise linear convex service curve instead of token-bucket and rate-latency.}}
\end{figure*}

We recall some of the network calculus essentials for a better understanding of the framework used in Saihu. In the following context, we use the following notation: $\mbb{R}^+$ is the set of non-negative real numbers; $[x]_+$ denotes $\max(0, x)$

The data flow is by convention modeled as a left-continuous wide-sense increasing function $R(t): \mbb{R}^+ \mapsto \mbb{R}^+$ with respect to time $t$~\cite{ncbook2001leboudec}. 

A system $\mcal{S}$ receives arrival data described as a cumulative function $R(t)$ and delivers departure data as another cumulative function $R^*(t)$. Figure~\ref{fig: data in-out} illustrates such a system $\mcal{S}$. The benefit of representing a system like this is that we can observe system backlog and delay with such a model. 

\begin{definition}[Backlog and Delay~\cite{ncbook2001leboudec}]
    The backlog of a system at time~$t$ is
    \begin{equation}
        b(t) = R(t) - R^*(t)
    \end{equation}
    
    The virtual delay of a FIFO system at time $t$ is
    \begin{equation}
        d_{FIFO}(t) = \inf \lbp \tau \geq 0 : R(t) \leq R^*(t+\tau) \rbp
    \end{equation}
\end{definition}



The backlog of a system can be viewed as the vertical distance between $R$ and $R^*$. The FIFO (\textit{First-in First-out}) delay is the horizontal distance between $R$ and $R^*$. One may obtain other delay values if the multiplexing technique is not FIFO.

% \begin{figure}
%     \centering
%     \includegraphics[width=0.9\linewidth]{images/in-out.png}
%     \caption{In/out data flow; delay and backlog}
%     \label{fig: data in-out}
% \end{figure}

Since we are interested in the system guarantee instead of a single instance of data flow, we would like to have general bounds to the arrival and departure data flows. Therefore, we define \textit{arrival curve} and \textit{service curve} as the bounds of arrival and departure data flows.

\begin{definition}[Arrival Curve~\cite{ncbook2001leboudec}]
    Given a wide-sense increasing function $\alpha: \mbb{R}^+ \mapsto \mbb{R}^+$, we say that a flow $R(t)$ is $\alpha$-constrained if and only if for all $s \leq t$:
    \begin{equation}
        R(t) - R(s) \leq \alpha(t-s)
    \end{equation}
    We say $R(t)$ has $\alpha$ as an arrival curve.
\end{definition}

\begin{definition}[Service Curve~\cite{ncbook2001leboudec}]
    Given a wide-sense increasing function $\beta: \mbb{R}^+ \mapsto \mbb{R}^+$ and $\beta(0) = 0$. A system $\mcal{S}$ having $R(t)$ and $R^*(t)$ as its arrival and departure flows. We say $\mcal{S}$ offers a service curve $\beta$ if and only if
    \begin{equation}
        R^*(t) \geq (R \otimes \beta)(t) =: \inf_{s \leq t} \lbp R(s) + \beta(t-s) \rbp
    \end{equation}
    where $\otimes$ denotes the min-plus convolution
\end{definition}

Figure~\ref{fig: arrival-service curves} illustrates the arrival and service curves. Any segment of arrival flow $R(t)$ is constrained by arrival curve $\alpha$ and the output curve $R^*(t)$ is always no less than the curve $R\otimes\beta$. As a result, an arrival curve upper bounds the incoming traffic, and a service curve lower bounds the outgoing traffic.

% \begin{figure}
%     \centering
%     \includegraphics[width=\linewidth]{images/arrival-service.png}
%     \caption{Arrival/Service curve}
%     \label{fig: arrival-service curves}
% \end{figure}

We consider 2 special types of curves throughout this paper, \textit{token-bucket} (or sometimes called \textit{leaky-bucket}) curve and \textit{rate-Latency} curve.

\begin{definition}[Token-bucket and Rate-latency~\cite{ncbook2001leboudec}]
    A token-bucket curve $\gamma_{r,b}$ with arrival rate $r$ and burst $b$ is defined as
    \begin{equation}
        \gamma_{r,b}(t) = b + rt
    \end{equation}

    A rate-latency curve $\beta_{R,T}$ with service rate $R$ and latency $T$ is defined as
    \begin{equation}
        \beta_{R,T}(t) = R \lb t - T \rb_+
    \end{equation}
\end{definition}

A token-bucket curve is determined by a burst $b$ and an arrival rate~$r$. Burst represents the maximum possible data volume that can arrive simultaneously, and arrival rate represents the maximum long-term data rate~\cite{bouillard2022tradeoff}.
A rate-latency curve is determined by a latency~$T$ and a service rate~$R$. Latency represents the time a server needs before starting to process the incoming data, and service rate represents the minimum rate to process data after the initial latency.

With the help of arrival and service curves, we can derive delay and backlog bounds for a system $\mcal{S}$ illustrated in Figure~\ref{fig: system bounds}. Suppose a system $\mcal{S}$ has arrival curve $\alpha$ and service curve~$\beta$, its worst-case backlog $b^*$ is the maximum vertical distance between~$\alpha$ and~$\beta$. Similarly, depending on the multiplexing technique applied to the system, its worst-case delay bound $d^*$ is the maximum horizontal distance between $\alpha$ and $\beta$ if $\mcal{S}$ is a FIFO system. If we don't have any information about its multiplexing technique, referred to as arbitrary multiplexing, the best we can say is that when $\alpha$ and $\beta$ intersect each other, where all data has been delivered out of the system. Consequently, the worst-case delay bound for arbitrary multiplexing is the time required for $\mcal{S}$ to clear its buffer.

% \begin{figure}
%     \centering
%     \includegraphics[width=\linewidth]{images/bound.png}
%     \caption{System delay/backlog bounds}
%     \label{fig: system bounds}
% \end{figure}

While a service curve captures the slowest possible output speed of a system, a link's transmission capacity limits the speed as well. Hence, we model this phenomenon using a \textit{greedy shaper} with a sub-additive function $\sigma: \mbb{R}^+ \mapsto \mbb{R}^+$ concatenated with a server. We consider a concatenation as shown in Figure \ref{fig: system}. By convention we assume $\sigma(0) = 0$ and $\beta(t) \leq \sigma(t), \forall t \in \mbb{R}^+$, meaning that the buffer is cleared at the beginning and the service never exceed its physical limitation. With the above definition, such greedy shaper conserves the service provided by the system due to theorem \ref{thm: shaping}.

\begin{figure}[thb]
    \centering
    \includegraphics[width=0.7\linewidth]{images/system.png}
    \caption{Shaping of departure data. A flow that has an arrival curve $\alpha$ feeds into a server with an arrival data flow $R(t)$. The server having service curve $\beta$ takes $R(t)$ and gives a departure data flow $R^*(t)$ to a shaper with shaping function $\sigma$. The shaper takes $R^*(t)$ and shape the data flow as another departure $D(t)$.}
    \label{fig: system}
\end{figure}


\begin{theorem}[Shaping conserves service \cite{ncbook2001leboudec}]
\label{thm: shaping}
Following the system shown in Figure \ref{fig: system}, we have
\begin{equation}
     D = R^* \otimes \sigma \geq \lp R \otimes \beta \rp \otimes \sigma = R \otimes \lp \beta \otimes \sigma \rp = R \otimes \beta
\end{equation}
\end{theorem}

In the following context, we model the shaping function $\sigma$ as a token-bucket curve $\gamma_{C,L}$ with transmission capacity $C$ and the packet size $L$ to capture the link capacity and packetization~\cite{bouillard2022tradeoff}.

\section{Method}
\label{sec:method}

% \ml{``Inconsistent'' to ``large variation''}

% In this section, we propose our methods based on the observations in Section \ref{sec:motivation}.
In this section, we propose two techniques to further enhance the strong baseline to capture the variation of activation distributions better.
We first introduce spatial re-scaling to adapt the network to pixel-to-pixel variation.
We then propose channel-wise shifting and re-scaling to better capture the channel-to-channel variation.
Meanwhile, as both of the two methods are image-dependent, the image-to-image variation can be captured naturally.
By combining the two methods with our strong baseline, we build our enhanced BNN for SR, named EBSR.

% Because the activation distributions among pixels, channels and images have large variations \red{**are highly inconsistent} in SR networks, we introduce spatial re-scaling to adapt to pixel-wise variations and channel shift and re-scaling to adapt to channel-wise variations. And both of them are image-dependent to adapt to image-wise variations, which means during inference our network re-scales and shifts the distributions of activations flexibly for different input images. Based on these methods, we build an enhanced binary neural network for image super-resolution (EBSR).

% According to [3], the difference of activation magnitudes indicates different scaling factors are needed for each pixel.

\subsection{Spatial Re-scaling}
% It is better to use different scaling factors for different pixels to reduce the quantization error and retain more detailed information for image super-resolution. 

% \ml{In the main method, we do not need to introduce the previous works but can focus on introducing our own method. Channel rescaling in Real-to-binary Net is not relevant in this context.}

% Re-scaling the output of binary convolutions was proposed at the birth of BNN in XNOR-Net \cite{rastegari2016xnor} to reduce quantization error and improve accuracy for image classification tasks.
% It is computed as below:
% \begin{equation}
% \mathcal{A} * \mathcal{W} \approx(\operatorname{sign}(\mathcal{A}) \circledast \operatorname{sign}(\mathcal{W})) \odot \mathcal{K} \alpha
% \label{eq:xnor-net rescale}
% \end{equation}
% where $\circledast$ denotes the binary convolution and $\odot$ denotes the element-wise multiplication.
% $\mathcal{A}$, $\mathcal{W}$, $\alpha$, and $\mathcal{K}$ denote the activation, weight, weight scaling factor, and activation scaling factor, respectively.
%  Later in XNOR-Net++ \cite{bulat2019xnor}, Bulat et al. fuse the activation and weight scaling factors into a single one that is learned end-to-end based on gradients and this improves the classification accuracy on ImageNet dataset.

% % It is computed as Eq.~\ref{eq:xnor-net rescale}, where $\circledast$ denotes 
% %  the binary convolution and $\odot$ denotes the element-wise multiplication. The binary convolution of $\mathcal{A}$ and $\mathcal{W}$ is rescaled by the weight scaling factor $\alpha$ and the activation scaling factor $\mathcal{K}$, both of which are calculated analytically.


% \zc{Similarly, you should explain the meaning of A, W and the operators $\circledast$ in the formula}
% Then in Real-to-binary Net \cite{martinez2020training}, Martinez et al. used a data-driven channel re-scaling module that takes the pre-convolution activations as input to predict the activation scaling factor. Unlike that in XNOR-Net++ \cite{bulat2019xnor}, these scaling factors are not fixed during inference but rather inferred from data. By doing this, they further improved the classification accuracy on ImageNet over XNOR-Net++. 
As is shown in Figure \ref{fig:pixel}, activation distributions have large pixel-to-pixel variation in SR networks
and the difference of activation magnitudes indicates different scaling factors are preferred for different pixels.
Inspired by \cite{martinez2020training}, we propose spatial re-scaling to better adapt the network to the spatial variation
of activation distributions in SR networks.
% fit the various pixel-wise distributions in SR networks.
We take the real-valued activations $A$ before convolution as input and predict pixel-wise scaling factors $S(A)$, which re-scale the binary convolution output. Spatial re-scaling process can be formulated as follows:
\begin{equation}
A * W \approx(\operatorname{sign}(A) \circledast \operatorname{sign}(W)) \odot \alpha \odot S(A)
\label{eq:spatial rescale}
\end{equation}
where $\circledast$ denotes 
the binary convolution and $\odot$ denotes the element-wise multiplication. $A$, $W$, $\alpha$, and $S\left(A\right)$ denote real-valued activations, weights, the scaling factor of weights, and the spatial-wise scaling factor of activations respectively. $S\left(A\right) \in \mathbb{R}^{1\times H\times W}$ can be calculated with a convolution and a sigmoid function.
% as $\sigma\left( CONV\left(A\right)\right)$. 
As shown in Figure \ref{fig:method}(a), real-valued activations first go through a convolution layer,
which has an input channel of $C$ and an output channel of 1, 
and then pass through a sigmoid function to produce the scaling factors $S(A)$ along the spatial dimension.
During inference, the scaling factor will change dynamically according to different input feature maps.
By re-scaling binary convolution output using $S(A)$, we can reduce the quantization error and the original pixel-wise information in FP activation
will be preserved much better.
Spatial re-scaling leads to a large PSNR improvement of 0.24 dB (from 30.30 dB to 31.54 dB) on Set5 and 0.22 dB (from 25.09 dB to 25.31 dB)
on Urban100 compared with our strong baseline. 

\subsection{Channel-wise Shifting and Re-scaling}

\begin{table}[!tb]
\centering
\caption{Comparison between whether to fuse channel-wise shifting and re-scaling or not based on our baseline with spatial re-scaling. }
\label{tab:fusing}

\scalebox{0.65}{
\begin{tabular}{c|cc|cc|cc}
\hline
\multirow{2}{*}{Method}     & \multirow{2}{*}{OPs} & \multirow{2}{*}{Params} & \multicolumn{2}{c|}{Set5} & \multicolumn{2}{c}{Urban100} \\ \cline{4-7} 
                            &                      &                         & PSNR        & SSIM        & PSNR          & SSIM         \\ \hline
Baseline + spatial re-scale & 2.16G                & 0.05M                   & 31.54       & 0.883       & 25.31         & 0.759        \\
+ channel-wise shift and re-scale             & 2.34G                & 0.09M                   & 31.61       & 0.885       & 25.35         & 0.761        \\
+ w/ fusing                   & 2.27G                & 0.08M                   & \textbf{31.64}       & \textbf{0.885}       & \textbf{25.36}         & \textbf{0.761}        \\ \hline
\end{tabular}
}
\end{table}

In SR networks, activation distributions exhibit larger channel-to-channel variation (Figure \ref{fig:chl}).
Both the mean and magnitude of the activation distributions vary significantly across channels.
% Thus we use channel-wise shifting and re-scaling to adapt to various channel-wise distributions. 
\cite{martinez2020training} has proposed the data-driven channel re-scaling, 
but our method differs from them in further introducing data-driven thresholds to handle the channel-wise variation of both mean and magnitude.
Since the blocks to generate the scaling factors and thresholds are very similar, we further propose to fuse them into one module.
% and fusing channel-wise shifting and re-scaling into one module.
We evaluate the effect of fusing the two blocks in Table \ref{tab:fusing}.
With channel-wise shifting and re-scaling fused, our models have fewer operations and parameters overhead and slightly higher performance.

For the specific process, we take the real-valued activations as input and predict different thresholds and scaling factors for each channel. They are also image dependent, e.g., $\beta_{i}$ in Eq.\ref{eq:act_binarize} is no longer fixed during inference but generated according to different input feature maps. Channel-wise shifting and re-scaling can be formulated as follows:
\begin{equation}
A * W \approx(\operatorname{sign}(A-C_s(A)) \circledast \operatorname{sign}(W)) \odot \alpha \odot C_r(A)
\label{eq:channel-wise_shift_and_rescale}
\end{equation}
where $\circledast$ denotes 
the binary convolution and $\odot$ denotes the element-wise multiplication. $C_s(A), C_r(A) \in \mathbb{R}^{C\times1\times1}$ denote the channel-wise threshold and scaling factor, respectively. 
We show the block diagram in Figure \ref{fig:method}(b).
The real-valued input feature map is first squeezed to a ${C\times1\times1}$ vector by a global average pooling (GAP) layer.
The subsequent fully connected layers and ReLU learn the channel-wise information and output a ${2C\times1\times1}$ vector.
Then the ${2C\times1\times1}$ vector is split into two ${C\times1\times1}$ vectors.
We use the first $C$ channels as the channel-wise bias and pass the last $C$ channels through a sigmoid layer 
as the channel-wise scaling factor, which are used to shift the real-valued activations and re-scale the binary convolution output, respectively. 


% \ml{We can mention previously, channel-wise re-scale has been proposed. We propose to fuse them. Add the comparison between fuse v.s. no fuse.}

\begin{figure}[!tbp]%
  \centering
    \includegraphics[width=0.4\textwidth]{fig/methods.png}
  
% \subfloat[channel-wise shifting\&re-scale]{
%     \label{subfig:channel-wise shifting and re-scale}
%     \includegraphics[width=0.2\textwidth]{fig/chl shift and rescale.png}
%   }

  \caption{Block diagram for spatial re-scaling, and channel-wise shifting and re-scaling.} 
  % Input A is the real-valued activation tensor and C, H, and W denote its dimension. GAP stands for global average pooling. The reduction ratio r is set to 16 for a better trade-off between the performance and the number of operations and parameters.}
  \label{fig:method}
\end{figure}


\subsection{Network Structure}

Combining the spatial re-scaling and the channel-wise shifting and re-scaling methods, we construct the enhanced convolution layer (E-Conv).
Then we build our EBSR model based on E-Conv.
In Figure \ref{fig:E-conv}, we compare the binary convolution layer used in the baseline network and our proposed E-Conv.
We use spatial and channel-wise scaling factors to re-scale the binary convolution output,
and use channel-wise shifting to learn appropriate thresholds for each channel before binarization.
The scaling factors and threshold used in E-Conv are learnable and depend on the real-valued input activations.
In this way, our proposed EBSR can adapt to pixel-to-pixel, channel-to-channel, and image-to-image variations
to reduce the large binarization error and preserve more details.
% In this way, our proposed E-Conv reduces the large quantization error caused by binarization and keeps the original information of input feature maps to a large extent.


\begin{figure}[!tb]%
  \centering

    \includegraphics[width=0.5\textwidth]{fig/E-conv.png}

  \caption{Comparison of (a) the binary convolution layer with a skip connection used in our baseline network and (b) the proposed E-Conv.}
  \label{fig:E-conv}
\end{figure}


Figure \ref{fig:network} shows the basic block based on the E-Conv and our EBSR composed of the basic blocks. Following existing works, the convolution layers in the head and tail modules are not binarized. We choose the lightweight EDSR which has 16 basic blocks and 64 channels, and EDSR which has 32 basic blocks and 256 channels as our backbones, which correspond to EBSR-light and EBSR, respectively.

\begin{figure}[!tb]%
  \centering
  {
    \includegraphics[width=0.35\textwidth]{fig/network.png}
  }
  
  \caption{The structure of our proposed EBSR.  Convolution layers in purple are real-valued vanilla 3x3 convolutions.}
  \label{fig:network}
\end{figure}
\section{DISTRO: Diffusion denoised Smoothing for Robust OOD detection}\label{sec:distro}

In this section, we present our method.
Essentially, it combines three techniques: (i) a diffusion denoiser, (ii) a standard OOD detector, and (iii) a certified binary discriminator.
Each component of this method is designed to overcome a specific problem of ordinary classifiers, as they are not robust to adversarial attacks, either ID or OOD, and do not detect OOD inputs well.

\tikzstyle{denoises} = [draw, fill=lightgray!20, text width=10em, text centered, minimum height=10em, rounded corners]
\tikzstyle{classifier}=[draw, fill=lightgray!20, text width=6em, text centered, minimum height=2em, rounded corners]
\tikzstyle{discriminator}=[draw, fill=lightgray!20, text width=6em, text centered, minimum height=2em, rounded corners]
\tikzstyle{ann} = [right]
\def\blockdist{2.3}
\def\edgedist{2.5}
\pgfdeclarelayer{background}
\pgfdeclarelayer{foreground}
\pgfsetlayers{background,main,foreground}

\begin{figure}[ht]
    \begin{adjustbox}{width=0.45\textwidth,center}
    \begin{tikzpicture}
        \node (denoise) [denoises] {\begin{center}
        Denoiser\\ \vspace{0.5em}
        $\begin{aligned}
            \texttt{denoise}_{\text{once}}(x + \delta; t) \\
            \text{with}\;\delta \sim \set{N}(0, \sigma^2 I)
        \end{aligned}$ \end{center}};

        \path (denoise.east)+(5, 0) node (output) [ann] {$\prob(y \lvert x)$};
        \path (denoise.east)+(\blockdist, 1) node (classifier) [classifier] {
        \begin{center} Classifier\\ $h(\tilde{x})$\end{center}};
        \path (denoise.east)+(\blockdist, -1) node (discriminator) [discriminator] {
        \begin{center} Discriminator \\ $g(\tilde{x})$ \end{center}};

        \path [draw, ->] (denoise.west)+(-1, 0) -- node (input) [above] {$x$} (denoise.west);
        \draw[->] (denoise.east)+(0.5, 0) -- node (input_classifier) [above] {$\tilde{x}$} (denoise.east) -|  ([yshift=2mm] input_classifier.east) |- (classifier.west); % right feedback loop
        \draw[->] (denoise.east)+(0.5, 0) -- (input_classifier) (denoise.east) -|  ([yshift=2mm] input_classifier.east) node[above,pos=0.1] {} |- (discriminator.west); % right feedback loop
        
        \path [draw, ->] (classifier.east) -| ([xshift=5mm] classifier.east) node[above right] {$\prob(y \lvert x, i)$} |- (output.west);
        \path [draw, ->] (discriminator.east) -| ([xshift=5mm] discriminator.east) node[below right] {$\prob(i \lvert x)$} |- (output.west);

        \begin{pgfonlayer}{background}
            % Compute a few helper coordinates
            \path (denoise.west |- classifier.north)+(-0.3,0.5) node (a) {};
            \path (denoise.south -| discriminator.east)+(+0.3,-0.2) node (b) {};
            \path[rounded corners, draw=black!50, dashed] %fill=yellow!20
                (a) rectangle (b);
            \path (classifier.north west)+(-0.2,0.2) node (a) {};
            \path (classifier.south -| classifier.east)+(+0.2,-0.2) node (b) {};
            % \path[fill=blue!10,rounded corners, draw=black!50, dashed]
            %     (a) rectangle (b);
        \end{pgfonlayer}
    \end{tikzpicture}
    \end{adjustbox}
    \caption{Overview of DISTRO.}
    \label{fig:overview}
\end{figure}

In \autoref{fig:overview}, we show an overview of DISTRO. 
First, a diffusion denoiser is employed before the classifier itself to provide robustness against ID attacks. 
As a result, adversarial noise introduced by the attack is mitigated by the denoiser.
This technique has already been proven to be very efficient and does not affect clean accuracy \cite{dds}. 

Secondly, numerous post-hoc OOD detection methods exist. 
The most straightforward being MSP~\cite{msp}, which can be added to the image classifier without retraining or fine-tuning. 
Alternatively, standard OOD detection methods, such as OE~\cite{oe}, VOS~\cite{vos} or LogitNorm~\cite{logitnorm}, could also replace the classifier.

Thirdly, to make the model more robust to OOD adversarial attacks, we add a binary discriminator to the model that is trained to be certifiably robust against OOD attacks. 
Additionally, this discriminator is combined with the OOD detection method from (ii) which is necessary to have the property of asymptotic underconfidence for far-OOD inputs.

%The idea is to apply diffusion denoising models in combination with robust OOD detection methods to improve adversarial and certified robustness for ID accuracy and OOD detection, and without sacrificing clean accuracy.

\textbf{Configuration.}
This method does not require any new technical knowledge. 
We begin by making the assumption that OOD samples are unrelated and thus maximally un-informative to the ID data.
Thus, for every class $y \in \set{Y}$, the conditional distribution on the input $x$ is given as:
\begin{equation}\label{eq:joint_prob}
    \prob(y|x) = \prob(y|x,i)\prob(i|x) + \frac{1}{K}(1-\prob(i|x)),
\end{equation}
where $\prob(i|x)$ is the conditional distribution representing the probability that $x$ is part of the ID, while $\prob(y|x,i)$ is the conditional distribution representing the ID.
Similarly to \citet{prood}, we assign independent models to each distribution:
\begin{itemize}
    \item $\prob(y|x,i) = h(\mathtt{denoise}_{\text{once}}(x+\delta; t))$, where $h:\R^d\to [0, 1]$ is the confidence of the main classifier $F(x)$, and $\tilde{x} = \mathtt{denoise}_{\text{once}}(x+\delta; t)$ represents one single step of denoising operation with $\delta \sim \set{N}(0, \sigma^2 I)$.
    \item $\prob(i|x) = \frac{1}{1+e^{-g(x)}}$, where $g:\R^d\to \R$ refers to a binary discriminator trained in a certified robust manner based on an $\ell_\infty$-threat model as in \citet{good, prood}.
\end{itemize}

As can be seen, the denoiser is the main addition. 
The one-step denoiser $\mathtt{denoise}_{\text{once}}$ estimates the fully denoised image $x$ from the current timestep $t$. 
Then it computes the average between the denoised image and the noisy image from the previous timestep.
As discussed in ~\citet{dds}, multiple applications of the denoiser will only destroy information about $x$.
Denoising with iterative steps essentially transfers the classification task to the denoiser, which can determine how the image should be filled.
For these reason, we apply only a single step of denoising.

\textbf{Asymptotic Underconfidence.}
Here, we show that by coupling a classifier trained to be OOD aware with a diffusion denoiser and running a certified discriminator in parallel, we can guarantee asymptotic underconfidence for data \textit{far enough} from the training distribution.

To obtain asymptotic underconfidence of the joint classifier, we consider $\prob(y|x,i) \leq 1$ and rewrite ~\autoref{eq:joint_prob} as follows:
\begin{equation}
    \prob(y|x) \leq \frac{K-1}{K} \prob(i|x)+\frac{1}{K}.
\end{equation}
Since the right term only depends on $\prob(i|x)$, we just need to assure that $\lim_{\beta \to \infty}\prob(i|\beta x) \to 0$.
If we employ a certified binary discriminator, trained with IBP on OOD data, as descibed in \citet{prood}, to compute $\prob(i|x)$, we achieve asymptotic underconfidence independently of the main classifier.
Readers are referred to \citet{prood} for a more detailed explanation.

\begin{figure*}
\vspace{-0.5em}
    \begin{subfigure}{.45\textwidth}
        \centering
        \includegraphics[width=\textwidth]{graphics/asymptotic_underconfidence.pdf}
        \caption{MSP}
        \label{fig:sub_confidence_msp}
    \end{subfigure}
    \hfill
    \begin{subfigure}{.46\textwidth}
        \centering
        \includegraphics[width=\textwidth]{graphics/asymptotic_energy_2.pdf}
        \caption{Energy}
        \label{fig:sub_confidence_energy}
    \end{subfigure}
    \caption{Asymptotic confidence as: (a) MSP~\cite{msp} and (b) Energy~\cite{energy}, for several OOD detection models divided into two categories: \textit{standard} (continuous line) and \textit{guaranteed} (dashed line).}
    \label{fig:asymptotic_confidence}
\vspace{-1em}
\end{figure*}

\textbf{Empirical Evaluation.}
In \autoref{fig:asymptotic_confidence}, we show an empirical evaluation of the asymptotic confidence for standard and robust OOD detection methods\footnote{the models are described in \autoref{sec:experiments}.}.
In this test, we consider a single ID sample $x$ and multiply by a scalar $\beta$.
In \autoref{fig:sub_confidence_msp} we plot the MSP~\cite{msp} as confidence, while in \autoref{fig:sub_confidence_energy} we plot the Energy~\cite{energy} for increasing values of $\beta > 0$.
In the context of MSP, we observe that standard OOD detection methods are asymptotically overconfident, after a small drop, whereas certified methods such as GOOD~\cite{good}, ProoD~\cite{prood} and DISTRO converge to $1/K$.
On the other hand, for Energy as $\beta$ increases, VOS~\cite{vos}, LogitNorm~\cite{logitnorm}, and Plain models asymptotically decrease, whereas GOOD~\cite{good}, ProoD (Meinke at al., 2022), and DISTRO remain stable. 

As a result, underconfidence can be easily obtained when using an energy score instead of MSP, regardless of whether it is on a plain or OOD aware model. 
\textbf{d}However, asymptotic underconfidence does not necessarily imply that the model will perform better in detecting OOD samples since all inputs are usually normalized to some range (e.g. [0, 1] or [-1, 1]). 
Thus the choice of MSP over the energy function is directly related to the possibility of certified robustness for OOD samples.
\section{Experiments}\label{sec:experiments}

In this section, DISTRO is evaluated for a variety of robust ID and OOD tests and is compared to previous approaches.
As baseline, we consider the pre-trained models\footnote{\href{https://github.com/AlexMeinke/Provable-OOD-Detection}{https://github.com/AlexMeinke/Provable-OOD-Detection}} from \citet{prood}.
The normal trained (\textbf{Plain}) and outlier exposure (\textbf{OE})~\cite{oe} models share the same ResNet18~\cite{resnet} architecture and hyperparameters as \textbf{ProoD}~\cite{prood}.
\textbf{GOOD}~\cite{good} uses a 'XL' convolutional neural network.
Additionally, we evaluate the pretrained DenseNet101~\cite{densenet} models for \textbf{ATOM}~\cite{atom} and \textbf{ACET}~\cite{acet}; and the standard OOD detection methods: \textbf{VOS}\footnote{\href{https://github.com/deeplearning-wisc/vos}{https://github.com/deeplearning-wisc/vos}}~\cite{vos} and \textbf{LogitNorm}\footnote{\href{https://github.com/hongxin001/logitnorm_ood}{https://github.com/hongxin001/logitnorm\_ood}}~\cite{logitnorm} with the pretrained WideResNet40~\cite{wideresnet} models provided in the respective works.
We consider \textbf{DDS}~\cite{dds} with a pre-trained diffusion model\footnote{\href{https://github.com/openai/improved-diffusion}{https://github.com/openai/improved-diffusion}} from \citet{nichol2021improved} in front of the OE classifier.
With \textbf{DISTRO}, we incorporate the same pre-trained diffusion model of DDS before the main classifier of ProoD, and maintain its discriminator.
The diffusion models have been used with the settings described in \citet{dds}.
In the context of $\ell_\infty$, we set $\sigma = \sqrt{d} \cdot \epsilon$.

We evaluate all methods on the standard datasets \texttt{CIFAR10/100}~\cite{cifar} as ID.
For the OOD detection evaluation we consider the following set of datasets: 
\texttt{CIFAR100/10}, \texttt{SVHN}~\cite{svhn}, LSUN~\cite{lsun} cropped (\texttt{LSUN\_CR}) and resized (\texttt{LSUN\_RS}),  TinyImageNet~\cite{tiny} cropped (\texttt{TinyImageNet\_CR}), \texttt{Textures}~\citep{textures} and synthetic (\texttt{Gaussian} and \texttt{Uniform}) noise distributions.
We use a random but fixed subset of 1000 images for all datasets considered as a test for OOD.
For ID, we consider the entire dataset.
We run all our experiments on a single NVIDIA A100. 

\subsection{In-Distribution Results}\label{sec:id-results}

Here, we compare clean, adversarial, and certified accuracy for ID samples.
Adversarial accuracy is evaluated with AutoAttack~\citep{apgd} for $\ell_\infty$-norm attacks of budget $\epsilon \in \{\nicefrac{2}{255}, \nicefrac{8}{255}\}$.
We ran the standard version of AutoAttack without additional hyper-parameters. 
Certified accuracy is evaluated for $\ell_2$-norm robustness of deviation $\sigma \in \{0.12, 0.25\}$.
To this end, random smoothing is performed on 10'000 Gaussian distributed samples around the input with a failure probability of $0.001$.
All $R>0$ are considered for the certified accuracy.
In the context of DISTRO and DDS we run 100 evaluation of the entire test set of \texttt{CIFAR10} to estimate the clean accuracy and report the average.
Further, we ran AutoAttack in both \textit{rand} and \textit{standard} modes, and considered the lowest results for DISTRO and DDS.


\begin{table}[htb]
\vspace{-0.5em}
    \centering
    \caption{\textbf{ID Accuracy}: Results of clean, adversarial and certified accuracy (\%) on the \texttt{CIFAR10} test set.
    The grayed-out models have an accuracy drop greater than $3\%$ relative to the model with the highest accuracy.}
    \label{tab:in-distribution}
    \begin{adjustbox}{width=0.5\textwidth,center}
        \begin{tabular}{llccccc}
            \toprule
            \multirow{2}{*}{Method} &\multirow{2}{*}{Clean} &\multicolumn{2}{c}{Adversarial ($\ell_\infty$)} &\multicolumn{2}{c}{Certified ($\ell_2$)} \\
            & &$\epsilon = \nicefrac{2}{255}$ &$\epsilon = \nicefrac{8}{255}$ &$\sigma=0.12$ &$\sigma = 0.25$ \\
            \midrule
            Plain$^*$       &95.01  &2.16   &0.00   &28.14  &14.17 \\
            OE$^*$          &95.53 &1.97   &0.00   &31.48  &10.88 \\
            VOS$^\dag$      &94.62  &2.24   &0.00   &13.13   &10.02       \\
            LogitNorm$^\ddag$  &94.48  &2.65   &0.00   &12.53  &10.25 \\
            \gray{ATOM$^*$}    &\gray{92.33}  &\gray{0.00}   &\gray{0.00}   &\gray{0.00}   &\gray{0.00}  \\
            \gray{ACET$^*$}    &\gray{91.49}  &\gray{69.01}  &\gray{6.04}   &\gray{57.13}  &\gray{12.48} \\
            \gray{GOOD$^*_{80}$} &\gray{90.13}  &\gray{11.65}  &\gray{0.23}   &\gray{17.33}  &\gray{10.31} \\
            ProoD$^*$ $\Delta=3$  &95.46  &2.69   &0.00   &33.92  &13.50 \\
            DDS                   &\textbf{95.55} &72.97 &24.09 &82.26 &64.58 \\
            DISTRO (our)          &95.47  &\textbf{73.34} &\textbf{27.14}  &\textbf{82.77}   &\textbf{65.63} \\
            \bottomrule
        \end{tabular}
    \end{adjustbox}
    \scriptsize{$*$ Pre-trained models from \citet{prood}, $\dagger$ Pre-trained from \citet{vos}, \\ $\ddag$ Pre-trained from \citet{logitnorm}.
    }
\vspace{-2em}
\end{table}

In \autoref{tab:in-distribution}, we show the results.
As expected, Plain and OE are not robust to adversarial attacks.
This applies to ProoD as well, since OE is its primary classifier.
Similarly, standard OOD detection methods, as LogitNorm and VOS, show poor robustness for ID data.
GOOD demonstrates better results than ProoD for adversarial attacks and worse in terms of certified accuracy.
Suprisingly, ACET reveals strong adversarial and certified accuracy despite of its reduced clean accuracy.
Meanwhile, ATOM results in zero for all tests since any slight perturbation of the input triggers the last neuron used for OOD detection.

\subsubsection*{Discussion}

It is clear that diffusion models can enhance adversarial and certified robustness while maintaining high clean accuracy.
As diffusion introduces variance into gradient estimators, standard attacks become much less effective.
Nevertheless, robustness accuracy of diffusion models varies over different runs for the same input, so it should be defined differently from deterministic accuracy, e.g. as expectation.
Luckily, one-shot diffusion introduces such a tiny variance that throughout a few of runs, our results were similar.
We present in section~\ref{ssec:faces} an application of PnP-HVAE on face images, using a pretrained state-of-the-art hierarchical VAE. 
Next, we study the application of our framework to natural images. To that end, we introduce  in section~\ref{ssec:patchVDVAE}  a patch hierachical VAE architecture, that is able to model natural images of different resolutions. In section~\ref{ssec:app_nat}, we provide deblurring, super-resolution and inpainting experiments to demonstrate the relevance of the proposed method.

Additional results are presented in Appendix~\ref{app:add}. All experiments can be reproduced using the code available at \url{https://github.com/jprost76/PnP-HVAE}.



\subsection{Face Image restoration (FFHQ)}\label{ssec:faces}
We first demonstrate the effectiveness of PnP-HVAE on highly structured data, by performing face image restoration.
Latent variable generative models can accurately model structured images such as face images \cite{karras2019style,vahdat2020nvae,child2021very,kingma2018glow}, and then be used to produce high quality restoration of such data. 
In our experiments, we use the VDVAE model of~\cite{child2021very}, pre-trained on the FFHQ dataset~\cite{karras2019style}, as our hierarchical VAE prior.
VDVAE has $L=66$ latent variable groups in its hierarchy and generates images at resolution $256\times256$.

We compare PnP-HVAE with the intermediate layer optimization algorithm (ILO)~\cite{daras2021intermediate} that is based on a different class of generative models than HVAE. ILO is a GAN inversion method which optimizes the image latent code along with the intermediate layer representation of a StyleGAN to generate an image consistent with a degraded observation.
We use the official implementation of ILO, along with a StyleGAN2 model~\cite{karras2020analyzing, stylegan2pytorch}, that was trained for 550k iterations on images of resolution $256\times256$ from FFHQ.  
As VDVAE and StyleGAN models are not trained on the same train-test split of FFHQ, we chose to evaluate the methods on a subset of 100 images from the CelebA dataset~\cite{liu2018large}. 
For super-resolution, the degradation model corresponds to the application of a gaussian low-pass filter followed by a $\times 4$ sub-sampling, and the addition of a gaussian white noise with $\sigma=3$.
For the deblurring, we considered motion blur and  gaussian kernels, both with a noise level $\sigma=8$. %

We provide quantitative comparisons in table~\ref{table:comp_ILO}, along with a visual comparison of the results in figure~\ref{fig:face_restoration}.
PnP-HVAE has the best  PSNR and SSIM results for all the considered restoration tasks, while ILO provides better results  for the perceptual distance.
By jointly optimizing the image and its latent variable, PnP-HVAE provides  results that are both realistic and consistent with the degraded observation.
On the other hand,  ILO  only optimizes on an extended latent space. This method generates  sharp and realistic images with better LPIPS scores,   
but the results lack  of consistency with respect to the observation, which explains the overall lower PSNR performance. 






\subsection{PatchVDVAE: a HVAE for natural images}\label{ssec:patchVDVAE}
Available generative models in the literature operate on images of  fixed resolutions and
are either restrained to datasets of limited diversity, or even to registered face images~\cite{kingma2018glow,child2021very, vahdat2020nvae, karras2019style}, or requiring additional class information~\cite{brock2018large, dhariwal2021diffusion, song2020score, luhman2022optimizing}.
Fitting an unconditional model on natural images appears to be a more difficult task, as their resolution can change, and their content is highly diverse.
The complexity of the problem can be reduced by learning a prior model on patches of reduced dimension. 
For image restoration problems, the patch model can be reused on images of higher dimensions~\cite{zoran2011learning,prost2021learning,altekruger2022patchnr}. When the model is a full CNN, the prior on the set of the  patches can  be computed efficiently by applying the network on the full image~\cite{prost2021learning}.

We thus introduce  patchVDVAE, a fully convolutional hierarchical VAE.
Contrary to existing HVAE models whose resolution is constrained by the constant tensor at the input of the top-down block, patchVDVAE can generate images of different resolutions by controlling the dimension of the input latent. 
This amounts to defining a prior on patches whose dimension corresponds to the receptive field of the VAE. A similar model is used for image denoising in~\cite{prakash2021interpretable}.

 
For PatchVDVAE architecture, we use the same bottom-up and top-down blocks as VDVAE~\cite{child2021very}, and replace the constant trainable input in the first top-down block by a latent variable, to make the model fully convolutional (details on the  architecture are given in Appendix~\ref{app:details}). 
The training dataset is composed of $128\times 128$ patches extracted from a combination of DIV2K~\cite{agustsson2017ntire} and Flickr2K~\cite{Lim_2017_CVPR_workshops} datasets.
We perform data augmentation by extracting  patches at $3$ resolutions: HR-images and $\times 2$ and $\times 4$ downscaled images. 
The model is trained for $7.10^5$ iterations with a batch size of $64$. Following the recommendation of~\cite{hazami2022efficient}, we use Adamax optimizer with an exponential moving average and gradient smoothing of the variance.
We set the decoder model to be a gaussian with diagonal covariance, as in~\cite{luhman2022optimizing}.
PatchVDVAE is fully convolutional and can generate images of dimension that are multiples of $64$ as illustrated by
figure~\ref{fig:vdvae}.

\newlength{\patchwidth}
\setlength{\patchwidth}{0.135\columnwidth}
\begin{figure}[!ht]
    \centering
    \begin{subfigure}[t]{.34\columnwidth}\hspace{0.1cm}
        \setlength{\tabcolsep}{0.02pt}
\renewcommand{\arraystretch}{0}
        \begin{tabular}{*{2}{p{1.03\patchwidth}}}
            \includegraphics[width=\patchwidth]{figures_arxiv/patchVDVAE/samples/generated/64x64/setup-5-image-0018.png} &
            \includegraphics[width=\patchwidth]{figures_arxiv/patchVDVAE/samples/generated/64x64/setup-5-image-0016.png} \\
            \includegraphics[width=\patchwidth]{figures_arxiv/patchVDVAE/samples/generated/64x64/setup-5-image-0008.png} &
            \includegraphics[width=\patchwidth]{figures_arxiv/patchVDVAE/samples/generated/64x64/setup-5-image-0019.png}   
        \end{tabular}
    \end{subfigure}\hspace{-0.15cm}
    \begin{subfigure}[t]{.64\columnwidth}
\begin{tabular}{cc}\vspace{-0.1cm}
\includegraphics[width=2\patchwidth]{figures_arxiv/patchVDVAE/samples/generated/256x256/setup-2-image-0009.png}&
        \includegraphics[width=2\patchwidth]{figures_arxiv/patchVDVAE/samples/generated/256x256/setup-2-image-0002.png}\end{tabular}

    \end{subfigure}
    \caption{\label{fig:vdvae} Left: $64\times64$ patches samples from our patchVDVAE model trained on patches from natural images.
    Right: PatchVDVAE is fully convolutional and it can generate images of higher resolution (here: $128\times128$).\vspace{-0.2cm}}
\end{figure}

\subsection{Natural images restoration}\label{ssec:app_nat}
We  evaluate PnP-HVAE on natural image restoration.
For each task, we report the average value of the PSNR, the SSIM, and the LPIPS metrics on $20$ images from the test set of the BSD dataset~\cite{MartinFTM01}.\\


\noindent
{\bf Image deblurring.}
In the experiments, we consider $2$ gaussian kernels and $2$ motion blur kernels from~\cite{levin2009understanding}, with $3$ different noise levels 
$\sigma \in \{2.55, 7.65, 12.75\}$.
As a baseline we consider  EPLL~\cite{zoran2011learning}, which learns a prior on image patches with a gaussian mixture model.
We also compare PnP-HVAE  with PnP-MMO and GS-PnP, $2$ competing convergent Plug-and-Play methods based on CNN denoisers.
PnP-MMO~\cite{pesquet2021learning} restricts the denoiser to be contraction in order to guarantee the convergence of the PnP forward-backard algorithm. GS-PnP~\cite{hurault2022gradient} considers a gradient step denoiser and reaches state-of-the-art performances of non converging methods~\cite{zhang2021plug}.
We set the temperature $\tau$  in our method as $0.95$, $0.8$ and $0.6$ for noise levels $2.55$, $7.65$ and $12.75$ respectively, and we let it run for a maximum of $50$ iterations. 
For the three compared methods we use the official implementations and pre-trained models provided by the respective authors. 
Details on the choice of hyperparameters for the concurrent methods are provided in the Appendix~\ref{app:details}
Figure~\ref{fig:deblurring_bsd} illustrates that our method provides correct deblurring results. 

According to table~\ref{tab:deb}, the performance of PnP-HVAE is between those of EPLL and GS-PnP and it outperforms PnP-MMO for large noise levels.\\

\begin{table}
\begin{center}\footnotesize
    \begin{tabular}{>{\centering}m{.3cm}*{5}{c}}
    $\sigma$ &Method & PSNR$\uparrow$ & SSIM$\uparrow$ & LPIPS$\downarrow$  \\ 
    \hline
    \multirow{4}{*}{\vcell{$2.55$}}
    & PnP-HVAE & $27.75$ & $0.79$ & $0.31$\\
    & GS-PNP \cite{hurault2022gradient} & $\mathbf{29.59}$ & $\mathbf{0.84}$ & $\mathbf{0.22}$\\
    & EPLL \cite{zoran2011learning} & $26.49$ & $0.71$ & $0.36$\\ 
    & PnP-MMO \cite{pesquet2021learning} & $\underbar{29.50}$ & $\underbar{0.83}$ & $\underbar{0.20}$ \\ \hline
    \multirow{4}{*}{\vcell{$7.65$}}
    & PnP-HVAE & $\underbar{26.36}$ & $\underbar{0.72}$ & $\underbar{0.40}$\\
    & GS-PNP \cite{hurault2022gradient} & $\mathbf{27.33}$ & $\mathbf{0.77}$ & $\mathbf{0.31}$\\
    & EPLL \cite{zoran2011learning} & $24.04$ & $0.66$ & $0.45$ \\ 
    & PnP-MMO \cite{pesquet2021learning} & $25.34$ & $0.69$ & $0.34$\\
    \hline
    \multirow{4}{*}{\vcell{$12.75$}}
    & PnP-HVAE & $\underbar{25.12}$ & $\mathbf{0.73}$ & $\underbar{0.47}$\\
    & GS-PNP \cite{hurault2022gradient} & $\mathbf{26.32}$ & $\mathbf{0.73}$ & $\mathbf{0.37}$\\
    & EPLL \cite{zoran2011learning} & $23.28$ & $0.61$ & $0.51$ \\ 
    & PnP-MMO \cite{pesquet2021learning} & $22.42$ & $0.53$& $0.54$ \\
    \hline
    &\vspace*{-.3cm}\\
            \multicolumn{2}{c}{Blur and motion kernels}& \multicolumn{3}{c}{
        \includegraphics*[scale=1]{figures_arxiv/kernels/4.png}\;\includegraphics*[scale=1]{figures_arxiv/kernels/7.png}\;\includegraphics*[scale=1]{figures_arxiv/kernels/9.png}\;\includegraphics*[scale=1]{figures_arxiv/kernels/11.png}} 
    \end{tabular}
        \caption{\label{tab:deb}Comparison  of PnP-HVAE  and other restoration methods on deblurring. Results are averaged on $4$ kernels.\vspace{-0.2cm}}% on image deblurring.}
    \end{center}
\end{table}

\begin{figure}
    
    \begin{subfigure}[h]{\linewidth}
        \centering
        \includegraphics*[width=\columnwidth]{figures_arxiv/deb_s255_k7.pdf}\vspace{-0.1cm}
        \caption{Gaussian blur, $\sigma=2.55$}
    \end{subfigure}
    \begin{subfigure}[h]{\linewidth}
        \centering
        \includegraphics*[width=\columnwidth]{figures_arxiv/deb_s765_k11.pdf}\vspace{-0.1cm}
        \caption{Motion blur, $\sigma=7.65$}
    \end{subfigure}\vspace*{-0.1cm}
    \caption{\label{fig:deblurring_bsd} Natural image deblurring\vspace{-0.1cm}}
\end{figure}

\noindent {\bf Effect of the temperature.}
PnP-HVAE gives control on the temperature of the prior over the latent space.
In figure~\ref{fig:temp_effect}, we illustrate that reducing the temperature increases the strength of the regularization prior. In this example the tuning $\tau=0.7$ produces the best performance.\\
\begin{figure}[!ht]
   
    \includegraphics[width=\columnwidth]{figures_arxiv/demo_temp.pdf}\vspace{-0.15cm}
    \caption{ \label{fig:temp_effect} Effect of the temperature in PnP-VAE on a deblurring problem, with $\sigma=7.65$.\vspace{-0.15cm}}
\end{figure}


\noindent
{\bf Image inpainting.}
Next we consider the task of noisy image inpainting. 
We compose a test-set of 10 images from the validation set of BSD~\cite{MartinFTM01} and we create masks
  by occluding diverse objects of small size in the images. 
A gaussian white noise with $\sigma=3$ is added to the images.
As a comparaison, we still consider GS-PnP and EPLL.
For PnP-HVAE, the temperature is set to $\tau=0.6$, and the algorithm is run for a maximum of $200$ iterations, unless the residual $||\x_{k+1}-\x_k||$ is on a plateau.
We provide on Table~\ref{tab:inpainting_bsd} the distortion metrics with the ground truth, as well as a visual
\begin{table}



\begin{center}
    \begin{tabular}{cccc}
        & PSNR$\uparrow$ & SSIM$\uparrow$ &LPIPS$\downarrow$ \\\hline
        PnP-HVAE  & $\mathbf{29.54}$ & $\mathbf{0.93}$ & $\mathbf{0.06}$\\
        GS-PNP & $28.52$ & $\mathbf{0.93}$ & $0.09$\\
        EPLL & $\underline{29.16}$ & $\mathbf{0.93}$ & $\mathbf{0.06}$\\
    \end{tabular}
    \caption{\label{tab:inpainting_bsd}Quantitative evaluation for inpainting on BSD.}
    \end{center}
\end{table}
comparison on figure~\ref{fig:inpainting_bsd}. 
With its hierarchical structure,  PnP-HVAE outperforms the compared methods. \vspace{0.05cm}



\begin{figure}[!h]
    \includegraphics[width=\columnwidth]{figures_arxiv/demo_inp_bsd2.pdf}\vspace{-0.1cm}
    \caption{\label{fig:inpainting_bsd}Natural image inpainting\vspace{-0.3cm}}
\end{figure}











\section{Conclusion}\label{sec:conclusion}

Current OOD robustness certification relies on external discriminators or loose certification mechanisms~\cite{prood}.
We propose an alternative using randomized smoothing~\cite{randomized_smoothing} for $\ell_2$-norm certificates, applicable to any classifier without specific requirements or training.
In comparison with previously proposed $\ell_\infty$-norm GAUC, standard approaches for OOD detection show non-zero results for guaranteed $\ell_2$-norm AUC and AUPR.
Unfortunately, a large number of samples derived around the input must be propagated through the network, increasing computational costs.
Additionally, we propose a method combining three techniques: diffusion denoising for noise removal, an OOD detection method, and a certified binary discriminator. 
This combination improves OOD robustness detection by around 13\%/5\% on CIFAR10/100 datasets compared to earlier approaches.

\bibliography{references/certificates, references/plain_ood, references/robust_ood, references/miscellaneous, references/datasets}
\bibliographystyle{icml2023}


\newpage
\appendix
\onecolumn
% To hide proofs : \newcommand{\maybehide}[1]{}
% To show proofs : \newcommand{\maybehide}[1]{#1}
\newcommand{\maybehide}[1]{#1}

\section{Proofs}
\label{sec:proofs}

\subsection{Weak Open CBV}

\subsubsection{General Lemmas}

\begin{proposition}[{\bf Diamond}]
    \label{prop:diamond}
The relation $\redcbv$ enjoys the diamond property: if $t \redcbv t_i\ (i=1,2)$ and $t_1 \neq t_2$, then there exists $t_3$ such that $t_i \redcbv t_3\ i=1,2$.
\end{proposition}

\propcharnfs*

\maybehide{\begin{proof}
    We are going to show this proposition by splitting the original statement into the two following ones:
    \begin{enumerate}
        \item \label{prop:char-nfs:1} $t \not\dred$ and $\neg\isvalue{t}$ iff $t \in \neutral$.
        \item \label{prop:char-nfs:2} $t \not\dred$ iff $t \in \normal$.
    \end{enumerate}
    The proof now follows by simultaneous induction over both these statements:
    \begin{itemize}
        \item[$\Ra$)] By induction over $t$: 
        \begin{enumerate}
            \item Let $t \not\dred$ and $\neg\isvalue{t}$. We want to show that $t \in \neutral$:
            \begin{itemize}
                \item Case $t = x$ or $t = \lam x.u$. Then $\neg\isvalue{t}$ does not hold. Therefore, the statement holds vacuously.
                \item Case $t = u p$. Since $u p \not\dred$, then, in particular, it must be the case that either $\neg\isabs{u}$ or $\neg\isvalue{p}$ must hold, according to rule (\ruleBeta):
                \begin{itemize}
                    \item Assume $\neg\isabs{u}$ holds. It must be the case that $u \not\dred$, according to rule (\ruleAppL). And it also must be the case that $p \not\dred$, according to rule (\ruleAppR). Therefore, $p \in \normal$, by the \ih (\cref{prop:char-nfs}.\ref{prop:char-nfs:2}). Now, we have to consider $u$, which can be a variable, or not:
                    \begin{itemize}
                        \item Case $u = x$. Then $u p \in x \ \normal \in \neutral$.
                        \item Case $u$ is not a variable. Then $\neg\isvalue{u}$ holds. Therefore, we have $u \in \neutral$, by the \ih (\cref{prop:char-nfs}.\ref{prop:char-nfs:1}). Thus, $u p \in \neutral \ \normal \in \neutral$.
                    \end{itemize}
                    \item Assume $\neg\isvalue{p}$ holds. Then it must be the case that $u \not\dred$, according to rule (\ruleAppL). And that $p \not\dred$, according to rule (\ruleAppR). Therefore, $u \in \normal$, by the \ih (\ref{prop:char-nfs}.\ref{prop:char-nfs:2}), and $p \in \neutral$, by the \ih (\cref{prop:char-nfs}.\ref{prop:char-nfs:1}). Thus, $u p \in \normal \ \neutral \in \neutral$.
                \end{itemize}
            \end{itemize}
            \item Let $t \not\dred$. We want to show that $t \in \normal$:
            \begin{itemize}
                \item Case $t \in \val$. Then, clearly $t \in \normal$.
                \item Case $t \not\in \val$. Then, $\neg\isvalue{t}$ holds. Therefore, $t \in \neutral$, by \cref{prop:char-nfs}.\ref{prop:char-nfs:1}. Thus, in particular, $t \in \normal$.
            \end{itemize}
        \end{enumerate}
        \item[$\La$)] By induction over $t \in \normal$:
        \begin{enumerate}
            \item Let $t \in \neutral$. We want to show that $t \not\dred$ and $\neg\isvalue{t}$:
            \begin{itemize}
                \item Case $t = u p \in x \ \normal$. Then $u = x$ and $p \in \normal$. Since $u = x$, then both rules (\ruleBeta) and (\ruleAppL) cannot be applied. Since $p \in \normal$, then $p \not\dred$, by the \ih (\cref{prop:char-nfs}.\ref{prop:char-nfs:2}). Therefore, rule (\ruleAppR) also cannot be applied. Thus, $u p \not\dred$. And we can conclude, since $\neg\isvalue{u p}$ clearly holds.
                \item Case $t = u p \in \normal \ \neutral$. Then $u \in \normal$ and $p \in \neutral$. Since $u \in \normal$, then $u \not\dred$, by the \ih (\cref{prop:char-nfs}.\ref{prop:char-nfs:2}). Since $p \in \neutral$, then $p \not\dred$ and $\neg\isvalue{p}$ holds, by the \ih (\cref{prop:char-nfs}.\ref{prop:char-nfs:1}). Since $\neg\isvalue{p}$, then rule (\ruleBeta) cannot be applied. Since $u \not\dred$ and $p \not\dred$, then rules (\ruleAppL) and (\ruleAppR) cannot be applied. Therefore, $u p \not\dred$. And we can conclude since $\neg\isvalue{u p}$ clearly holds.
                \item Case $t = u p \in \neutral \ \normal$. Then $u \in \neutral$ and $p \in \neutral$. Since $u \in \neutral$, then $u \not\dred$ and $\neg\isvalue{u}$ holds, by the \ih (\cref{prop:char-nfs}.\ref{prop:char-nfs:1}). Since $p \in \normal$, then $p \not\dred$, by the \ih (\cref{prop:char-nfs}.\ref{prop:char-nfs:2}). Since $\neg\isvalue{u}$, then rule (\ruleBeta) cannot be applied. Since $u \not\dred$ and $p \not\dred$, then rules (\ruleAppL) and (\ruleAppR) cannot be applied. Therefore $u p \not\dred$. And we can conclude since $\neg\isvalue{u p}$ clearly holds.
            \end{itemize}
            \item Let $t \in \normal$. We want to show that $t \not\dred$:
            \begin{itemize}
                \item Case $t \in \val$. Then, clearly $t \not\dred$.
                \item Case $t \not\in \val$. Then, $t \in \neutral$, by definition. Thus, $t \not\dred$ holds, by~\cref{prop:char-nfs}.\ref{prop:char-nfs:1}.
            \end{itemize}
        \end{enumerate}
    \end{itemize}
\end{proof}
}
  
\begin{lemma}[Relevance]
    Let $\Phi \tr \seqi{\Gam}{t}{\tau}{(b,s)}$. Then $\dom{\Gam} \subseteq \fv{t}$.
\end{lemma}

\maybehide{\begin{proof}
    The proof following by induction over $\Phi$. Case $\Phi$ ends with rule (\ruleAx) or (\ruleLamP), then $\Phi$ is clearly relevant. The other cases following easily from the \ih.
\end{proof}}

\subsubsection{Soundness (Auxiliary Lemmas)}

\begin{lemma}
    \label{lem:values-not-neutral}
    Let $\Phi \tr \seqi{\Gam}{t}{\tau}{(b,s)}$. If $t \in \val$, then $\tau \not= \tneutral$.
\end{lemma}

\maybehide{\begin{proof}
    By case analysis on the form of $t \in \val$:
    \begin{itemize}
        \item Case $t = x$. Then we have to consider two additional cases according to the last rule used in $\Phi$:
        \begin{itemize}
            \item Case $\Phi$ ends with rule (\ruleAx), then $\tau$ is of the form $\sig \not= \tneutral$.
            \item Case $\Phi$ ends with rule (\ruleMany), then $\tau$ is of the form $\M \not= \tneutral$.
        \end{itemize}
        \item Case $t = \lam x.t$. Then we have to consider three additional cases according to the last rule used in $\Phi$:
        \begin{itemize}
            \item Case $\Phi$ ends with rule (\ruleLam), then $\tau$ is of the form $\M \ta \del \not= \tneutral$.
            \item Case $\Phi$ ends with rule (\ruleMany), then $\tau$ is of the form $\M \not= \tneutral$.
            \item Case $\Phi$ ends with rule (\ruleLamP), then $\tau = \tabs \not= \tneutral$.
        \end{itemize}
    \end{itemize}
\end{proof}}

\begin{lemma}
    \label{lem:notabs-implies-negabs}
    If $\Phi \tr \seqi{\Gam}{t}{\tau}{(b,s)}$, such that $\Gam$ is tight. If $\tau \in \nott{\tabs}$, then $\neg\isabs{t}$.
\end{lemma}

\maybehide{\begin{proof}
    By induction over $\Phi$:
    \begin{itemize}
        \item Case $\Phi$ ends with rule (\ruleAx), (\ruleApp), (\ruleAppPOne), or (\ruleAppPTwo), then $\neg\isabs{t}$ holds by definition.
        \item Case $\Phi$ ends with rule (\ruleLam), (\ruleMany), or (\ruleLamP),  then $\tau \not\in \nott{\tabs}$. Therefore, these cases do not apply.
    \end{itemize}
\end{proof}}

\begin{lemma}[{\bf Zero Steps and Normal Forms}]
    \label{lem:zero-steps-nfs}
    Let $\Phi \tr \seqi{\Gam}{t}{\tau}{(b,s)}$ be tight. $b = 0$ iff $t \in \normal$.
\end{lemma}

\maybehide{\begin{proof} \mbox{}
    \begin{itemize}
        \item[$\Ra$)] We want to show that, if $b = 0$, then $t \in \normal$. For this, we are going to split the original statement into the two following ones:
        \begin{enumerate}
            \item \label{lem:zero-steps-nfs:1} Let $\Phi \tr \seqi{\Gam}{t}{\tau}{(0,s)}$ be tight and $\neg\isvalue{t}$, then $t \in \neutral$.
            \item \label{lem:zero-steps-nfs:2} Let $\Phi \tr \seqi{\Gam}{t}{\tau}{(0,s)}$ be tight, then $t \in \normal$.
        \end{enumerate}
        The proof now follows by simultaneous induction over both these statements:
        \begin{enumerate}
            \item Let $\Phi \tr \seqi{\Gam}{t}{\tau}{(0,s)}$ be tight and $\neg\isvalue{t}$:
            \begin{itemize}
                \item Case $\Phi$ ends with rule (\ruleAx), (\ruleLam), (\ruleMany), or (\ruleLamP), then $\isvalue{t}$ holds. Therefore, these cases do not apply.
                \item Case $\Phi$ ends with rule (\ruleApp), then $b > 0$. Therefore, this case does not apply.
                \item Case $\Phi$ ends with rule (\ruleAppPOne), then $t$ is of the form $up$ and $\Phi$ is of the following form:
                \[ \begin{prooftree}
                    \hypo{\Phi_u \tr \seqi{\Gam_u}{u}{\nott{\tabs}}{(0,s_u)}}
                    \hypo{\Phi_p \tr \seqi{\Gam_p}{p}{\tightt}{(0,s_p)}}
                    \infer2[(\ruleAppPOne)]{\seqi{\Gam_u + \Gam_p}{up}{\tneutral}{(0,1+s_u+s_p)}}
                \end{prooftree} \]
                where $\tau = \tneutral$, $\Gam = \Gam_u + \Gam_p$ is tight, and $s = 1 + s_u + s_p$. Moreover, $\Gam_u$ and $\Gam_p$ are tight. By the \ih (\cref{lem:zero-steps-nfs}.\ref{lem:zero-steps-nfs:2}) over $\Phi_u$ and $\Phi_p$, we have that $u, p \in \normal$. By~\cref{lem:notabs-implies-negabs}, we have that $\neg\isabs{u}$. Therefore, either $u$ is a variable or $u \in \neutral$ by definition. So, in both cases, we can conclude that $u p \in \neutral$.
                \item Case $\Phi$ ends with rule (\ruleAppPTwo), then $t$ is of the form $up$ and $\Phi$ is of the following form:
                \[ \begin{prooftree}
                    \hypo{\Phi_u \tr \seqi{\Gam_u}{u}{\tightt}{(0,s_u)}}
                    \hypo{\Phi_p \tr \seqi{\Gam_p}{p}{\tneutral}{(0,s_p)}}
                    \infer2[(\ruleAppPTwo)]{\seqi{\Gam_u + \Gam_p}{up}{\tneutral}{(0,1+s_u+s_p)}}
                \end{prooftree} \]
                where $\tau = \tneutral$, $\Gam = \Gam_u + \Gam_p$, and $s = 1 + s_u + s_p$. Moreover, $\Gam_u$ and $\Gam_p$ are tight. By the \ih (\cref{lem:zero-steps-nfs}.\ref{lem:zero-steps-nfs:2}) over $\Phi_u$, we have that $u \in \normal$. By applying~\cref{lem:values-not-neutral} to $\Phi_p$, we have that $\neg\isvalue{p}$. By the \ih (\cref{lem:zero-steps-nfs}.\ref{lem:zero-steps-nfs:1}) over $\Phi_p$, we have that $p \in \neutral$. So, in both cases, we can conclude that $up \in \neutral$.
            \end{itemize}
            \item Let $\Phi \tr \seqi{\Gam}{t}{\tau}{(0,s)}$ be tight:
            \begin{itemize}
                \item Case $\Phi$ ends with rule (\ruleAx), (\ruleLam), or (\ruleLamP). Then, clearly $t \in \val$, so we can conclude immediately.
                \item Case $\Phi$ ends with rule (\ruleMany), then $\tau$ is of the form $\M \not\in \tightt$. Therefore, this case does not apply.
                \item In all the remaining cases $\neg\isvalue{t}$ holds. Then $t \in \neutral$, by \cref{lem:zero-steps-nfs}.\ref{lem:zero-steps-nfs:1}, so $t \in \normal$.
            \end{itemize}
        \end{enumerate}
        \item[$\La)$] We want to show that, if $t \in \normal$, then $b = 0$. The proof follows by induction over $t \in \normal$:
        \begin{enumerate}
            \item Case $t \in \neutral$. Then we have to consider the following additional cases:
            \begin{itemize}
                \item Case $t = xp$, such that $p \in \normal$. Then there are three additional cases to consider:
                \begin{itemize}
                    \item Case $\Phi$ ends with (\ruleApp), then it must be of the following form:
                    \[ \begin{prooftree}
                        \hypo{\seqi{x : \mul{\M \ta \tau}}{x}{\M \ta \tau}{(0,0)}}
                        \hypo{\Phi_p \tr \seqi{\Gam_p}{p}{\M}{(b_p,s_p)}}
                        \infer2[(\ruleApp)]{\seqi{(x : \mul{\M \ta \tau}) + \Gam_p}{xp}{\tau}{(1+b_p,s_p)}}
                    \end{prooftree} \]
                    where $\Gam = (x : \mul{\M \ta \tau}) + \Gam_p$ is tight, $b = 1+b_p$, and $s = s_p$. But, $\mul{\M \ta \tau}$ is not tight, since $\M \ta \tau \not\in \tightt$. Therefore, this case does apply.
                    \item Case $\Phi$ ends with (\ruleAppPOne), then $\Phi$ must be of the following form:
                    \[ \begin{prooftree}
                        \hypo{\seqi{(x : \mul{\tvar})}{x}{\tvar}{(0,0)}}
                        \hypo{\Phi_p \tr \seqi{\Gam_p}{p}{\tightt}{(b_p,b_p)}}
                        \infer2[(\ruleAppPOne)]{\seqi{\Gam_u + \Gam_p}{up}{\tneutral}{(b_p,1+s_u+s_p)}}
                    \end{prooftree} \]
                    where $\tau = \tneutral$, $\Gam = (x : \mul{\tvar}) + \Gam_p$ is tight, $b = b_p$, and $s = 1+ s_u + s_p$. Moreover, $\Gam_p$ is tight. By the \ih over $\Phi_p$, we have that $b_p = 0$. So we can conclude with $b = b_u + b_p = 0$.
                    \item Case $\Phi$ ends with (\ruleAppPTwo). This case is very similar to the case where $\Phi$ ends with rule (\ruleAppPOne).
                \end{itemize}
                \item Case $t = up$, such that $u \in \normal$ and $p \in \neutral$. Then there are three additional cases to consider:
                \begin{itemize}
                    \item Case $\Phi$ ends with (\ruleApp), then it must be of the following form:
                    \[ \begin{prooftree}
                        \hypo{\seqi{\Gam_u}{u}{\M \ta \tau}{(b_u,s_u)}}
                        \hypo{\Phi_p \tr \seqi{\Gam_p}{p}{\M}{(b_p,s_p)}}
                        \infer2[(\ruleApp)]{\seqi{\Gam_u + \Gam_p}{up}{\tau}{(1+b_u+b_p,s_u+s_p)}}
                    \end{prooftree} \]
                    where $\tau = \tau$, $\Gam = \Gam_u + \Gam_p$ is tight, $b = 1 + b_u + b_p$, and $s = s_u + s_p$. By~\cref{lem:tight-spreading}.\ref{lem:tight-spreading:2}, we have that $\M \in \tightt$, which is a contradiction. Therefore, this case does not apply.
                    \item Case $\Phi$ ends with (\ruleAppPOne) or (\ruleAppPTwo). These cases are very similar to the corresponding cases when $t = x p$, such that $p \in \normal$.
                \end{itemize}
                \item Case $t = up$, such that $u \in \neutral$ and $p \in \normal$. Then there are three cases to consider:
                \begin{itemize}
                    \item Case $\Phi$ ends with (\ruleApp), then it must be of the following form:
                    \[ \begin{prooftree}
                        \hypo{\seqi{\Gam_u}{u}{\M \ta \tau}{(b_u,s_u)}}
                        \hypo{\Phi_p \tr \seqi{\Gam_p}{p}{\M}{(b_p,s_p)}}
                        \infer2[(\ruleApp)]{\seqi{\Gam_u + \Gam_p}{up}{\tau}{(1+b_u+b_p,s_u+s_p)}}
                    \end{prooftree} \]
                    where $\tau = \tau$, $\Gam = \Gam_u + \Gam_p$ is tight, $b = 1 + b_u + b_p$, and $s = s_u + s_p$. By~\cref{lem:tight-spreading}.\ref{lem:tight-spreading:2} over $u \in \neutral$, we have that $\M \ta \tau \in \tightt$, which is a contradiction. Therefore, this case does not apply.
                    \item Case $\Phi$ ends with (\ruleAppPOne) or (\ruleAppPTwo). These cases are very similar to corresponding cases when $t = x p$, such that $p \in \normal$, or $t = up$, such that $u \in \normal$ and $p \in \neutral$.
                \end{itemize}
            \end{itemize}
            \item Case $t \in \normal$. Then we can consider the two following additional cases:
            \begin{itemize}
                \item Case $t \in \val$. Then $\Phi$ must end with (\ruleAx), (\ruleLam), (\ruleMany), or (\ruleLamP). With the exception of the case where $\Phi$ ends with rule (\ruleMany), we can conclude $b = 0$ immediately for every other case, by definition. Case $\Phi$ ends with rule (\ruleMany), then $\tau$ is of the form $\M \not\in \tightt$. Therefore, this case does not apply.
                \item Case $t \not\in \val$. Then, $t \in \neutral$, by definition. Therefore, $b = 0$, by \cref{lem:zero-steps-nfs}.\ref{lem:zero-steps-nfs:1}.
            \end{itemize}
        \end{enumerate}
    \end{itemize}
\end{proof}}

\begin{lemma}
    \label{lem:corr-size-counter}
    Let $\Phi \tr \seqi{\Gam}{t}{\tau}{(b,s)}$ be tight. If $b = 0$ then $s = \size{t}$.
\end{lemma}

\maybehide{\begin{proof}
    The proof follows by induction over $\Phi$:
    \begin{itemize}
        \item Case $\Phi$ ends with rule (\ruleAx) or (\ruleLamP). Then $t \in \val$ and $s = 0$. So we can conclude with $\size{t} = 0 = s$.
        \item Case $\Phi$ ends with rule (\ruleLam). Then $\tau$ is of the form $\Gam_u(x) \ta \del \not\in \tightt$, so this case does not apply.
        \item Case $\Phi$ ends with rule (\ruleApp). Then $b > 0$, so this case does not apply.
        \item Case $\Phi$ ends with rule (\ruleMany). Then $\tau$ is of the form $\M \not\in \tightt$, so this case does not apply.
        \item Case $\Phi$ ends with rule (\ruleAppPOne). Then $t = up$ and $\Phi$ must be of the following form:
        \[ \begin{prooftree}
            \hypo{\Phi_u \tr \seqi{\Gam_u}{u}{\nott {\tabs}}{(0,s_u)}}
            \hypo{\Phi_p \tr \seqi{\Gam_p}{p}{\tightt}{(0,s_p)}}
            \infer2[(\ruleAppPOne)]{\seqi{\Gam_u + \Gam_p}{up}{\tneutral}{(0,1+s_u+s_p)}}
        \end{prooftree} \]
        where $\tau = \tneutral$, $\Gam = \Gam_u + \Gam_p$, and $s = 1 + s_u + s_p$. Moreover, $\Gam_u$ and $\Gam_p$ are tight. By the \ih over $\Phi_u$ and $\Phi_p$, we have $s_u = \size{u}$ and $s_p = \size{p}$. So we can conclude with $s = 1 + \size{u} + \size{p} = \size{up}$.
        \item Case $\Phi$ ends with rule (\ruleAppPTwo). This case is very similar to the case where $\Phi$ ends with rule (\ruleAppPOne).
    \end{itemize}
\end{proof}}

\begin{lemma}[{\bf Split for Values}]
    \label{lem:split-values}
    Let $\Phi_v \tr \seqi{\Gam}{v}{\M}{(b,s)}$, such that $\M = \sqcup_{\iI} \M_i$. Then, there exist ($\Phi^i_v \tr \seqi{\Gam_i}{v}{\M_i}{(b_i,s_i)})_{\iI}$, such that $\Gam = +_{\iI} \Gam_i$, $b = +_{\iI} b_i$, and $s = +_{\iI} s_i$.
\end{lemma}

\maybehide{\begin{proof}
    We start by noting that $\Phi_v$ must end with the rule ($\ruleMany$). Therefore, we have $\Gam = +_{\jJ} \Gam_j$, $\M = \mul{\sig_j}_{\jJ}$, $b = +_{\jJ} b_j$, $s = +_{\jJ} s_j$, and $(\Phi^j_v \tr \seqi{\Gam_j}{v}{\sig_j}{(b_j,s_j)})_{\jJ}$, for some $J$. Let $\M_i = \mul{\sig_k}_{\kK_i}$, for each $\iI$, such that $J = +_{\iI} K_i$. Then, by using rule ($\ruleMany$), we can build $\Phi^i_v \tr \seqi{\Gam_i}{v}{\M_i}{(b_i, s_i)}$, for each $\iI$, such that $\Gam_i = +_{\kK_i} \Gam_k$, $b_i = +_{\kK_i} b_k$, and $s_i = +_{\kK_i} s_k$. So we can conclude with $\Gam = +_{\jJ} \Gam_j = +_{\iI} (+_{\kK_i} \Gam_k) = +_{\iI} \Gam_i$, $b = +_{\jJ} b_j = +_{\iI} (+_{\kK_i} b_k) = +_{\iI} b_i$, and $s = +_{\jJ} s_j = +_{\iI} (+_{\kK_i} s_k) = +_{\iI} s_i$.
\end{proof}}

\subsubsection{Completeness (Auxiliary Lemmas)}

\begin{lemma}[{\bf Tight Spreading}]
    \label{lem:tight-spreading}
    Let $\Phi \tr \seqi{\Gam}{t}{\tau}{(b,s)}$, such that $\Gam$ is tight:
    \begin{enumerate}
        \item \label{lem:tight-spreading:1} If $b = 0$ and $\tau$ is not an arrow type or a multi-type, then $\tau \in \tightt$.
        \item \label{lem:tight-spreading:2} If $t \in \neutral$, then $\tau \in \tightt$.
    \end{enumerate}
\end{lemma}

\maybehide{\begin{proof} \mbox{}
    \begin{enumerate}
        \item We want to show that, if $b = 0$ and $\tau$ is not an arrow type or a multiset type, then $\tau \in \tightt$. The proof follows by induction over $\Phi$:
        \begin{itemize}
            \item Case $\Phi$ ends with rule ($\ruleAx$), then it is of the following form:
            \[ \begin{prooftree}
                \infer0[(\ruleAx)]{\seqi{x : \mul{\sig}}{x}{\sig}{(0,0)}}
            \end{prooftree} \]
            such that $\tau = \sig$, $\Gam = x : \mul{\sig}$, and $s = 0$. If $x : \mul{\sig}$ is tight, then $\sig \in \{\tabs, \tvar\}$. Therefore, we can conclude with $\sig \in \{\tabs, \tvar\} \subset \tightt$.
            \item Case $\Phi$ ends with rule (\ruleLam), then $\tau$ is an arrow type. Therefore, this case does not apply.
            \item Case $\Phi$ ends with rule (\ruleApp), then $b > 0$. Therefore, this case does not apply.
            \item Case $\Phi$ ends with rule (\ruleMany), then $\tau$ is a multiset type. Therefore, this case does not apply.
            \item Case $\Phi$ ends with rule (\ruleLamP), then $\tau = \tabs \in \tightt$. 
            \item Case $\Phi$ ends with rules (\ruleAppPOne) or (\ruleAppPTwo), then $\tau = \tneutral \in \tightt$.
        \end{itemize}
        \item We want to show that, if $t \in \neutral$, then $\tau \in \tightt$. By induction over $t \in \neutral$:
        \begin{itemize}
            \item Case $t = xp$, such that $p \in \normal$. Then we have to consider the following three cases depending on the last rule in $\Phi$:
            \begin{itemize}
                \item Case $\Phi$ ends with rule (\ruleApp), then it must be of the following form:
                \[ \begin{prooftree}
                    \hypo{\seqi{x : \mul{\M \ta \tau}}{x}{\M \ta \del}{(0,0)}}
                    \hypo{\Phi_p \tr \seqi{\Gam_p}{p}{\M}{(b_p,s_p)}}
                    \infer2[(\ruleApp)]{\seqi{(x : \mul{\M \ta \tau}) + \Gam_p}{xp}{\del}{(1+b_p,s_p)}}
                \end{prooftree} \]
                where $\Gam = (x : \mul{\M \ta \del}) + \Gam_p$ is tight, $b = 1+b_p$, and $s = s_p$. But, $\mul{\M \ta \del}$ is not tight, since $\M \ta \del \not\in \tightt$. Therefore, this case does apply.
                \item Case $\Phi$ ends with rule (\ruleAppPOne) or (\ruleAppPTwo). Then $\tau = \tneutral \in \tightt$, so we can conclude immediately.
            \end{itemize}
            \item Case $t = up$, such that $u \in \normal$ and $p \in \neutral$. Then we have to consider the following three cases depending on the last rule in $\Phi$:
            \begin{itemize}
                \item Case $\Phi$ ends with rule (\ruleApp), then it must be of the following form:
                \[ \begin{prooftree}
                    \hypo{\Phi_u \tr \seqi{\Gam_u}{u}{\M \ta \tau}{(b_u, s_u)}}
                    \hypo{\Phi_p \tr \seqi{\Gam_p}{p}{\M}{(b_p, s_p)}}
                    \infer2[(\ruleApp)]{\seqi{\Gam_u + \Gam_p}{up}{\tau}{(1+b_u+b_p, s_u+s_p)}}
                \end{prooftree} \]
                where $\Gam = \Gam_u + \Gam_p$ is tight, $b = 1 + b_u + b_p$, and $s = s_u + s_p$. Moreover, $\Gam_p$ is tight. By the \ih over $\Phi_p$, we have that $\M \in \tightt$, which is a contradiction. Therefore, this case does not apply.
                \item Case $\Phi$ ends with rule (\ruleAppPOne) or (\ruleAppPTwo). Then $\tau = \tneutral \in \tightt$, so we can conclude immediately.
            \end{itemize}
            \item Case $t = up$, such that $u \in \neutral$ and $p \in \normal$. Then we have to consider the following three cases depending on the last rule in $\Phi$:
            \begin{itemize}
                \item Case $\Phi$ ends with rule (\ruleApp), then it must be of the following form:
                \[ \begin{prooftree}
                    \hypo{\Phi_u \tr \seqi{\Gam_u}{u}{\M \ta \tau}{(b_u, s_u)}}n
                    \hypo{\Phi_p \tr \seqi{\Gam_p}{p}{\M}{(b_p, s_p)}}
                    \infer2[(\ruleApp)]{\seqi{\Gam_u + \Gam_p}{up}{\tau}{(1+b_u+b_p, s_u+s_p)}}
                \end{prooftree} \]
                where $\Gam = \Gam_u + \Gam_p$ is tight, $b = 1 + b_u + b_p$, and $s = s_u + s_p$. Moreover, $\Gam_p$ is tight. By the \ih over $\Phi_p$, we have that $\M \in \tightt$, which is a contradiction. Therefore, this case does not apply.
                \item Case $\Phi$ ends with rule (\ruleAppPOne) or (\ruleAppPTwo). Then $\tau = \tneutral \in \tightt$, so we can conclude immediately.
            \end{itemize}
        \end{itemize}
    \end{enumerate}
\end{proof}}

\begin{lemma}[{\bf Typability of Normal Forms}]
    \label{lem:typ-nfs}
    If $t \in \normal$, then there exists a tight derivation $\Phi \tr \seqi{\Gam}{t}{\tau}{(b,s)}$, such that $s = \size{t}$.
\end{lemma}

\maybehide{To show this proposition we are going to need to split the original statement into the two following ones:
\begin{enumerate}
    \item \label{prop:typ-nfs:1} If $t \in \neutral$, then there exists a tight derivation $\Phi \tr \seqi{\Gam}{t}{\tneutral}{(b,s)}$, such that $s = \size{t}$.
    \item \label{prop:typ-nfs:2} If $t \in \normal$, then there exists a tight derivation $\Phi \tr \seqi{\Gam}{t}{\tightt}{(b,s)}$, such that $s = \size{t}$.
\end{enumerate}
The proof follows by simultaneous induction over both these statements:
\begin{enumerate}
    \item Let $t \in \neutral$. We want to show that there exists a tight derivation $\Phi \tr \seqi{\Gam}{t}{\tneutral}{(b,s)}$:
    \begin{itemize}
        \item Case $t = up \in x \ \normal$. Then $u = x$ and $p \in \normal$. Therefore, there exists a tight derivation $\Phi_p \tr \seqi{\Gam_p}{p}{\tightt}{(b_p,s_p)}$, by the \ih (\cref{lem:typ-nfs}.\ref{prop:typ-nfs:2}), such that $\size{p} = s_p$. Thus, we can build $\Phi$ as follows:
        \[ \begin{prooftree}
            \infer0[(\ruleAx)]{\seqi{x : \mul{\tvar}}{x}{\tvar}{(0,0)}}
            \hypo{\Phi_p \tr \seqi{\Gam_p}{p}{\tightt}{(b_p,s_p)}}
            \infer2[(\ruleAppPOne)]{\seqi{ x : \mul{\tvar} + \Gam_p}{x p}{\tneutral}{(b_p,1+s_p)}}
        \end{prooftree} \]
        And we can conclude with $\Gam = x : \mul{\tvar} + \Gam_p$, $b = b_p$, and $s = 1+s_p = 1 + \size{x} + \size{p} = \size{xp}$.
        \item Case $t = up \in \normal \ \neutral$. Then $u \in \normal$ and $p \in \neutral$. Therefore, there exists a tight derivation $\Phi_u \tr \seqi{\Gam_u}{u}{\tightt}{(b_u,s_u)}$, such that $\size{u} = s_u$, by the \ih (\cref{lem:typ-nfs}.\ref{prop:typ-nfs:2}), and there exists a tight derivation $\Phi_p \tr \seqi{\Gam_p}{p}{\tneutral}{(b_p,s_p)}$, such that $\size{p} = s_p$ by the \ih (\cref{lem:typ-nfs}.\ref{prop:typ-nfs:1}). Thus, we can build $\Phi$ as follows:
        \[ \begin{prooftree}
            \hypo{\Phi_u \tr \seqi{\Gam_u}{u}{\tightt}{(b_u,s_u)}}
            \hypo{\Phi_p \tr \seqi{\Gam_p}{p}{\tneutral}{(b_p,s_p)}}
            \infer2[(\ruleAppPTwo)]{\seqi{\Gam_u + \Gam_p}{up}{\tneutral}{(b_u+b_p,1+s_u+s_p)}}
        \end{prooftree} \]
        And we can conclude with $\Gam = \Gam_u + \Gam_p$, $b = b_u+b_p$, and $s = 1+s_u+s_p = 1 + \size{u} + \size{p} = \size{up}$.
        \item Case $t = up \in \neutral \ \normal$. Then $u \in \neutral$ and $p \in \normal$. Therefore, there exists a tight derivation $\Phi_u \tr \seqi{\Gam_u}{u}{\tneutral}{(b_u,s_u)}$, such that $\size{u} = s_u$, by the \ih (\cref{lem:typ-nfs}.\ref{prop:typ-nfs:1}), and there exists a tight derivation $\Phi_p \tr \seqi{\Gam_p}{p}{\tightt}{(b_p,s_p)}$, such that $\size{p} = s_p$, by the \ih (\cref{lem:typ-nfs}.\ref{prop:typ-nfs:2}). Thus, we can build $\Phi$ as follows:
        \[ \begin{prooftree}
            \hypo{\Phi_u \tr \seqi{\Gam_u}{u}{\tneutral}{(b_u,s_u)}}
            \hypo{\Phi_p \tr \seqi{\Gam_p}{p}{\tightt}{(b_p,s_p)}}
            \infer2[(\ruleAppPOne)]{\seqi{\Gam_u + \Gam_p}{up}{\tneutral}{(b_u+b_p,1+s_u+s_p)}}
        \end{prooftree} \]
        And we can conclude with $\Gam = \Gam_u + \Gam_p$, $b = b_u+b_p$, and $s = 1 + s_u + s_p = 1 + \size{u} + \size{p} = \size{up}$.
    \end{itemize}
    \item Case $t \in \normal$. We want to show that there exists a tight derivation $\Phi \tr \seqi{\Gam}{t}{\tightt}{(b,s)}$:
    \begin{itemize}
        \item Case $t = x$. Then we can build $\Phi$ as follows:
        \[ \begin{prooftree}
            \infer0[(\ruleAx)]{\seqi{x : \mul{\sig}}{x}{\sig}{(0,0)}}
        \end{prooftree} \]
        by picking $\sig \in \{\tabs, \tvar\}$. And we can conclude with $\Gam = \eset$, $b = 0$, and $s = 0 = \size{x}$.
        \item Case $t = \lam x.u$. Then we can build $\Phi$ as follows:
        \[ \begin{prooftree}
            \infer0[(\ruleLamP)]{\seqi{}{\lam x.u}{\tabs}{(0,0)}}
        \end{prooftree} \]
        And we can conclude with $\Gam = \eset$, $b = 0$, and $s = 0 = \size{\lam x.u}$.
        \item The remaining cases are for when $t \in \neutral$, so they are subsumed by previous cases.
    \end{itemize}
\end{enumerate} 
}

\begin{lemma}[{\bf Merge for Values}]
    \label{lem:merge-values}
    Let $(\Phi^i_v \tr \seqi{\Gam_i}{v}{\M_i}{(b_i,s_i)})_{\iI}$. Then, there exists $\Phi_v \tr \seqi{\Gam}{v}{\M}{(b,s)}$, such that $\Gam = +_{\iI} \Gam_i$, $\M = +_{\iI} \M_i$, $b = +_{\iI} b_i$, and $s = +_{\iI}$.
\end{lemma}

\maybehide{\begin{proof}
    We start by noting that each $\Phi^i_v$ must end with the rule ($\ruleMany$). Therefore, for each $\iI$, we have $\Gam_i = +_{\kK_i} \Gam_k$, $\M_i = \mul{\sig_k}_{\kK_i}$, such that $b_i = +_{\kK_i} b_k$ and $s_i = +_{\kK_i} s_k$, and the following derivations $(\Phi^k_v \tr \seqi{\Gam_k}{v}{\sig_k}{(b_k,s_k)})_{\kK_i}$. Let $J = +_{\iI} K_i$ and $\M = \mul{\sig_j}_{\jJ} = \mul{\sig_k}_{\kK_i, \iI}$. We can use rule ($\ruleMany$) to build $\Phi_v \tr \seqi{\Gam}{v}{\M}{(+_{\jJ} b_j, +_{\jJ} s_j)}$. So we can conclude with $\Gam = +_{\jJ} \Gam_j = +_{\iI} (+_{\kK_i} \Gam_k) = +_{\iI} \Gam_i$, $b = +_{\jJ} b_j = +_{\iI} (+_{\kK_i} b_k) = +_{\iI} b_i$, and $s = +_{\jJ} s_j = +_{\iI} (+_{\kK_i} s_k) = +_{\iI} s_i$.
\end{proof}}

\subsubsection{Soundness and Completeness (Main Results)}

\begin{lemma}[{\bf Substitution and Anti-Substitution}]
    \label{lem:subsantisubs}
    \begin{enumerate} \mbox{}
        \item \label{lem:subs} Let $\Phi_t \tr \seqi{\Gam_t; x : \M}{t}{\tau}{(b_t,s_t)}$ and $\Phi_v \tr \seqi{\Gam_v}{v}{\M}{(b_v,s_v)}$, then there exists $\Phi_{t \subs{x}{v}} \tr \seqi{\Gam_t + \Gam_v}{t \subs{x}{v}}{\tau}{(b_t+b_v,s_t+s_v)}$.
        \item \label{lem:antisubs} Let $\Phi_{t \subs{x}{v}} \tr \seqi{\Gam_{t \subs{x}{v}}}{t \subs{x}{v}}{\tau}{(b,s)}$. Then, there exist $\Phi_t \tr \seqi{\Gam_t; x : \M}{t}{\tau}{(b_t,s_t)}$ and $\Phi_v \tr \seqi{\Gam_v}{v}{\M}{(b_v,s_v)}$, such that $\Gam_{t \subs{x}{v}} = \Gam_t + \Gam_v$, $b = b_t + b_v$, and $s = s_t + s_v$.
    \end{enumerate}
\end{lemma}

\maybehide{\begin{proof} \mbox{}
    \begin{enumerate}
        \item %\begin{proof}
    The proof follows by induction over $\Phi_t$:
    \begin{itemize}
        \item Case $\Phi_t$ ends with rule (\ruleAx). Then $t$ must be a variable and we need to consider two cases:
        \begin{itemize}
            \item Assume $t = y = x$. Then $\Gam_t = \eset$, $\tau = \M$, $t \subs{x}{v} = v$, $b_t = 0$, and $s_t = 0$. So we can take $\Phi_{t \subs{x}{v}} = \Phi_v$ and conclude with $\Gam_t + \Gam_v = \Gam_v$, $b_t + b_v = b_v$, and $s_t + s_v = s_v$.
            \item Assume $t = y \not= x$. Then $\M = \emul$, $\Gam_v = \eset$, $t \subs{x}{v} = t$, $b_v = 0$, and $s_v = 0$. So we can take $\Phi_{t \subs{x}{v}} = \Phi_t$ and conclude with $\Gam_t + \Gam_v = \Gam_t$, $b_t + b_v = b_t$, and $s_t + s_v = s_t$.
        \end{itemize}
        \item Case $\Phi_t$ ends with rule (\ruleLam). Then $t$ must be of the form $\lam y.u$ and $\Phi_t$ must be of the following form (by $\alpha$-conversion):
        \[ \begin{prooftree}
            \hypo{\Phi_u \tr \seqi{\Gam; x : \M}{u}{\tau'}{(b_t,s_t)}}
            \infer1[(\ruleLam)]{\seqi{(\Gam \sm y); x : \M}{\lam y.u}{\Gam(y) \ta \tau'}{(b_t, s_t)}}
        \end{prooftree} \]
        where $\tau = \Gam(y) \ta \tau'$ and $\Gam_t = (\Gam \sm y)$. By the \ih, we have the following derivation $\Phi_{u \subs{x}{v}} \tr \seqi{\Gam + \Gam_v}{u \subs{x}{v}}{\tau}{(b_t + b_v, s_t + s_v)}$. Therefore, we can construct $\Phi_{t \subs{x}{v}}$ as follows:
        \[ \begin{prooftree}
            \hypo{\Phi_{u \subs{x}{v}} \tr \seqi{\Gam + \Gam_v}{u \subs{x}{v}}{\tau'}{(b_t + b_v, s_t + s_v)}}
            \infer1[(\ruleLam)]{\seqi{(\Gam + \Gam_v) \sm y}{(\lam y.u) \subs{x}{v}}{\Gam(y) \ta \tau'}{(b_t + b_v, s_t + s_v)}}
        \end{prooftree} \]
        And we can conclude with $(\Gam + \Gam_v) \sm y = (\Gam \sm y) + \Gam_v = \Gam_t + \Gam_v$, by $\alpha$-conversion.
        \item Case $\Phi_t$ ends with rule ($\ruleApp$). Then $t$ must be of the form $up$ and $\Phi_t$ must be of the following form:
        \[ \begin{prooftree}
            \hypo{\Phi_u \tr \seqi{\Gam; x : \M_1}{u}{\M' \ta \tau}{(b_u, s_u)}}
            \hypo{\Phi_p \tr \seqi{\Del; x : \M_2}{p}{\M'}{(b_p,s_p)}}
            \infer2[(\ruleApp)]{\seqi{(\Gam + \Del); x : \M_1 \sqcup \M_2}{up}{\tau}{(1+b_u+b_p, s_u+s_p)}}
        \end{prooftree} \]
        where $\Gam_t = (\Gam + \Del)$, $\M = \M_1 \sqcup \M_2$, $b_t = 1 + b_u + b_p$, and $s_t = s_u + s_p$. By~\cref{lem:split-values}, we know there exist the following derivations $(\Phi^i_v \tr \seqi{\Gam^i_v}{v}{\M_i}{(b_i,s_i)})_{i \in \{1,2\}}$, such that $\Gam_v = \Gam^1_v + \Gam^2_v$, $b_v = b_1 + b_2$, and $s_v = s_1 + s_2$. By the \ih, we know there exist $\Phi_{u \subs{x}{v}} \tr \seqi{\Gam + \Gam^1_v}{u \subs{x}{v}}{\M' \ta \tau}{(b_u+b_1, s_u+s_1)}$ and $\Phi_{p \subs{x}{v}} \tr \seqi{\Del + \Gam^2_v}{p \subs{x}{v}}{\M'}{(b_p + b_2, s_p + s_2)}$. So we can construct $\Phi_{t \subs{x}{v}}$ as follows:
        \[ \begin{prooftree}
            \hypo{\Phi_{u \subs{x}{v}} \tr \seqi{\Gam + \Gam^1_v}{u \subs{x}{v}}{\M' \ta \tau}{(b_u+b_1, s_u+s_1)}}
            \hypo{\Phi_{p \subs{x}{v}} \tr \seqi{\Del + \Gam^2_v}{p \subs{x}{v}}{\M'}{(b_p+b_2,s_p+s_2)}}
            \infer2[(\ruleApp)]{\seqi{(\Gam + \Del) + (\Gam^1_v + \Gam^2_v)}{(u p) \subs{x}{v}}{\tau}{(1+b_u+b_p+b_1+b_2, s_u + s_p + s_1 + s_2)}}
        \end{prooftree} \]
        And we can conclude with $\Gam_t + \Gam_v = (\Gam + \Del) + (\Gam^1_v + \Gam^2_v)$, $b_t + b_v = 1 + b_u + b_p + b_1 + b_2$, and $s_t + s_v = s_u + s_p + s_1 + s_2$.
        \item Case $\Phi_t$ ends with rule ($\ruleMany$). Then $t$ must be of the form $w$ and $\Phi$ must be of the following form:
        \[ \begin{prooftree}
            \hypo{(\Phi^i_w \tr \seqi{\Gam_i; x : \M_i}{w}{\sig_i}{(b_i,s_i)})_{\iI}}
            \infer1[(\ruleMany)]{\seqi{+_{\iI} \Gam_i; x : \sqcup_{\iI} \M_i}{w}{\mul{\sig_i}_{\iI}}{(+_{\iI} b_i, +_{\iI} s_i)}}
        \end{prooftree} \]
        where $\tau = \mul{\sig_i}_{\iI}$, $\Gam_t = +_{\iI} \Gam_i$, $b_t = +_{\iI} b_i$, and $s_t = +_{\iI} s_i$. By~\cref{lem:split-values}, we have the following derivations $(\Phi^i_v \tr \seqi{\Gam^i_v}{v}{\M_i}{(b^i_v, s^i_v)})_{\iI}$, such that $\Gam_v = +_{\iI} \Gam^i_v$, $b_v = +_{\iI} b^i_v$, and $s_v = +_{\iI} s^i_v$. By the \ih over each $\Phi^i_w$, we have $(\Phi^i_{w \subs{x}{v}} \tr \seqi{\Gam_i + \Gam^i_v}{w \subs{x}{v}}{\sig_i}{(b_i + b^i_v, s_i + s^i_v)})_{\iI}$. Therefore, we can construct $\Phi_{t \subs{x}{v}}$ as follows:
        \[ \begin{prooftree}
            \hypo{(\Phi^i_{w \subs{x}{v}} \tr \seqi{\Gam_i + \Gam^i_v}{w \subs{x}{v}}{\sig_i}{(b_i + b^i_v, s_i + s^i_v)})_{\iI}}
            \infer1[(\ruleMany)]{\seqi{+_{\iI} (\Gam_i + \Gam^i_v)}{w \subs{x}{v}}{\mul{\sig_i}_{\iI}}{(+_{\iI} (b_i + b^i_v), +_{\iI} (s_i + s^i_v))}}
        \end{prooftree} \]
        And we can conclude with $\Gam_t + \Gam_v = +_{\iI} \Gam_i +_{\iI} \Gam^i_v = +_{\iI} (\Gam_i + \Gam^i_v)$, $b_t + b_v = +_{\iI} b_i +_{\iI} b^i_v = +_{\iI} (b_i + b^i_v)$, and $s_t + s_v = +_{\iI} s_i +_{\iI} s^i_v = +_{\iI} (s_i + s^i_v)$.
        \item Case $\Phi_t$ ends with rule (\ruleLamP). Then $t$ must be of the form $\lam y.u$, $\Gam_t = \eset$, $\tau = \tabs$, $\M = \emul$, $\Gam_v = \eset$, $t \subs{x}{v} = \lam y.(u \subs{x}{v}) = (\lam y.u) \subs{x}{v}$, $b_t = b_v = 0$, and $s_t = s_v = 0$. So we can construct $\Phi_{t \subs{x}{v}}$ as follows:
        \[ \begin{prooftree}
            \infer0[(\ruleLamP)]{\seqi{}{(\lam y.u) \subs{x}{v}}{\tabs}{(0,0)}}
        \end{prooftree} \]
        And conclude with $\Gam_t + \Gam_v = \eset$, $b_t + b_v = 0$, and $s_t + s_v = 0$.
        \item Case $\Phi_t$ ends with rule (\ruleAppPOne) or (\ruleAppPTwo), the proof is very similar to when $\Phi_t$ ends with rule (\ruleApp).
    \end{itemize}
%\end{proof}

        \item %\begin{proof}
    The proof follows by induction over $t$:
    \begin{itemize}
        \item Case $t = y$. Then we have to consider two cases:
        \begin{itemize}
            \item Case $t = y \not= x$. Then, $t \subs{x}{v} = y$. Let $\Gam_v = \eset$, $\M = \emul$, $b_v = 0$, and $s_v = 0$. Then, $\Phi_v$ is derivable using rule ($\ruleMany$). We also take $\Phi_t = \Phi_{t \subs{x}{v}}$, so that, in particular $\Gam_t = \Gam_{t \subs{x}{v}}$. Then, we conclude with $\Gam_{t \subs{x}{v}} = \Gam_t + \Gam_v = \Gam_t$, $b = b_t + b_v = b_t$, and $s = s_t + s_v = s_t$.
            \item Case $t = y = x$. Then, $t \subs{x}{v} = v$. Let $\Gam_t = \eset$, $b_t = 0$, and $s_t = 0$. Now, we have to consider two cases depending on the last rule used in $\Phi_{t \subs{x}{v}}$: 
            \begin{itemize}
                \item Case $\Phi_{t \subs{x}{v}}$ ends with rule ($\ruleAx$), then $\tau = \sig$. Let $\Gam_v = \Gam_{t \subs{x}{v}}$, $\M = \mul{\sig}$, $b_v = b$, and $s_v = s$. Then, we can build derivation $\Phi_v$ as follows:
                \[ \begin{prooftree}
                    \hypo{\Phi_{t \subs{x}{v}} \tr \seqi{\Gam_{t \subs{x}{v}}}{v}{\sig}{(b,s)}}
                    \infer1[(\ruleMany)]{\seqi{\Gam_{t \subs{x}{v}}}{v}{\mul{\sig}}{(b,s)}}
                \end{prooftree} \]
                Let $\Gam_t = \eset$, $b_t = 0$, and $s_t = 0$. Then, $\Phi_t \tr \seqi{x : \mul{\sig}}{x}{\sig}{(0,0)}$ is given by rule ($\ruleAx$). So we can conclude with $\Gam_{t \subs{x}{v}} = \Gam_v = \Gam_t + \Gam_v$, $b = b_v = b_t + b_v$, and $s = s_v = s_t + s_v$.
                \item Case $\Phi_{t \subs{x}{v}}$ ends with rule ($\ruleMany$), then $\tau = \mul{\sig_i}_{\iI}$, for some $I$. Let $\Gam_t = \eset$, and $\M = \mul{\sig_i}_{\iI}$. Then, we can build $\Phi_t$ as follows:
                \[ \begin{prooftree}
                    \infer0[(\ruleAx)]{(\seqi{x : \mul{\sig_i}}{x}{\sig_i}{(0,0)})_{\iI}}
                    \infer1[(\ruleMany)]{\seqi{x : \mul{\sig_i}_{\iI}}{x}{\mul{\sig_i}_{\iI}}{(0,0)}}
                \end{prooftree} \] 
                Then, we can take $\Phi_v = \Phi_{t \subs{x}{v}}$, so that $\Gam_v = \Gam_{t \subs{x}{v}}$, $b_v = b$, and $s_v = s$. And we can conclude $\Gam_{t \subs{x}{v}} = \Gam_v = \Gam_t + \Gam_v$, $b = b_v = b_t + b_v$, and $s = s_v = s_t + s_v$.
            \end{itemize}
        \end{itemize}
        \item Case $t = \lam y.u$. Then $t \subs{x}{v} = (\lam y.u) \subs{x}{v} = \lam y.(u \subs{x}{v})$ and we have to consider three cases:
        \begin{itemize}
            \item Case $\Phi_{t \subs{x}{v}}$ ends with rule (\ruleLam), then it must be of the following form:
            \[ \begin{prooftree}
                \hypo{\Phi_{u \subs{x}{v}} \tr \seqi{\Gam_{u \subs{x}{v}}; y : \M'}{u \subs{x}{v}}{\tau'}{(b, s)}}
                \infer1[(\ruleLam)]{\seqi{\Gam_{u \subs{x}{v}}}{\lam y.(u \subs{x}{v})}{\M' \ta \tau'}{(b, s)}}
            \end{prooftree} \]
            where $\tau = \M' \ta \tau'$, and $\Gam_{t \subs{x}{v}} = \Gam_{u \subs{x}{v}}$. By the \ih, we have the following derivations $\Phi_u \tr \seqi{\Gam_u; y: \M'; x : \M}{u}{\del}{(b_u, s_u)}$ and $\Phi_v \tr \seqi{\Gam_v}{v}{\M}{(b_v, s_v)}$, such that $\Gam_{u \subs{x}{v}} = \Gam_u + \Gam_v$, $b = b_u + b_v$, and $s = s_u + s_v$. And we can build $\Phi_{\lam y.u}$ as follows:
            \[ \begin{prooftree}
                \hypo{\Phi_u \tr \seqi{\Gam_u; y : \M'; x : \M}{u}{\tau'}{(b_u, s_u)}}
                \infer1[(\ruleLam)]{\seqi{\Gam_u; x : \M}{\lam y.u}{\M' \ta \tau'}{(b_u, s_u)}}
            \end{prooftree} \]
            So we can pick $\Phi_t = \Phi_{\lam y.u}$, and conclude with $\Gam_{t \subs{x}{v}} = \Gam_{u \subs{x}{v}} = \Gam_u + \Gam_v$, $b = b_u + b_v$, and $s = s_u + s_v$.
            \item Case $\Phi_{t \subs{x}{v}}$ ends with rule (\ruleLamP), then is must be of the following form:
            \[ \begin{prooftree}
                \infer0[(\ruleLamP)]{\seqi{}{\lam y.(u \subs{x}{v})}{\tabs}{(0,0)}}
            \end{prooftree} \]
            where $\tau = \tabs$, $\Gam_{t \subs{x}{v}} = \eset$, $b = 0$, and $s = 0$. Let $\Gam_t = \eset$, $\M = \emul$, $b_t = 0$, and $s_t = 0$. Then, we can build $\Phi_t$ as follows:
            \[ \begin{prooftree}
                \infer0[(\ruleLamP)]{\seqi{}{\lam y.u}{\tabs}{(0,0)}}
            \end{prooftree} \]
            Let $\Gam_v = \eset$, $b_v = 0$, and $s_v = 0$. Then $\Phi_v$ can be constructed by using rule (\ruleMany) with no premises. So we can conclude with $\Gam_{t \subs{x}{v}} = \eset = \Gam_t + \Gam_v$, and $b = 0 = b_t + b_v$, and $s = 0 = s_t + s_v$.
            \item Case $\Phi_{t \subs{x}{v}}$ ends with rule ($\ruleMany$). Then $t \subs{x}{v}$ and $t$ are values, and $\Phi_{t \subs{x}{v}}$ must be of the following form:
            \[ \begin{prooftree}
                \hypo{(\Phi_i \tr \seqi{\Gam_i}{t \subs{x}{v}}{\sig_i}{(b_i,s_i)})_{\iI}}
                \infer1[(\ruleMany)]{\seqi{+_{\iI} \Gam_i}{t \subs{x}{v}}{\mul{\sig_i}_{\iI}}{(+_{\iI} b_i, +_{\iI} s_i)}}
            \end{prooftree} \]
            where $\tau = \mul{\sig_i}_{\iI}$, $\Gam_{t \subs{x}{v}} = +_{\iI} \Gam_i$, $b = +_{\iI} b_i$, and $s = +_{\iI} s_i$. By the \ih over each $\Phi_i$, we have the following derivations $\Phi^i_t \tr \seqi{\Gam^i_t; x : \M_i}{t}{\sig_i}{(b^i_t, s^i_t)}$ and $\Phi^i_v \tr \seqi{\Gam^i_v}{v}{\M_i}{(b^i_v, s^i_v)}$, such that $\Gam_i = \Gam^i_t + \Gam^i_v$, $b_i = b^i_t + b^i_v$, and $s_i = s^i_t + s^i_v$,for each $\iI$. So we can build $\Phi_t$ as follows:
            \[ \begin{prooftree}
                \hypo{(\Phi^i_t \tr \seqi{\Gam^i_t; x : \M_i}{t}{\sig_i}{(b^i_t, s^i_t)})_{\iI}}
                \infer1[(\ruleMany)]{\seqi{+_{\iI} \Gam^i_t; x : \sqcup_{\iI} \M_i}{t}{\mul{\sig_i}_{\iI}}{(+_{\iI} b^i_t, +_{\iI} s^i_t)}}
            \end{prooftree} \]
            such that $\Gam_t = +_{\iI} \Gam^i_t$, $\M = \sqcup_{\iI} \M_i$, $b_t = +_{\iI} b^i_t$, and $s_t = +_{\iI} s^i_t$. By~\cref{lem:merge-values}, we can take the following derivation $\Phi_v \tr \seqi{+_{\iI} \Gam^i_v}{v}{\M}{(+_{\iI} b^i_v, +_{\iI} s^i_v)}$. And we can conclude with $\Gam_{t \subs{x}{v}} = +_{\iI} \Gam_i = +_{\iI} (\Gam^i_t + \Gam^i_v) = +_{\iI} \Gam^i_t +_{\iI} \Gam^i_v = \Gam_t + \Gam_v$, $b = +_{\iI} b_i = +_{\iI} (b^i_t + b^i_v) = +_{\iI} b^i_t +_{\iI} b^i_v = b_t + b_v$, and $s = +_{\iI} s_i = +_{\iI} (s^i_t + s^i_v) = +_{\iI} s^i_t +_{\iI} s^i_v = s_t + s_v$.
        \end{itemize}
        \item Case $t = up$. Then $t \subs{x}{v} = (u \subs{x}{v}) (p \subs{x}{v})$ and we have to consider three cases:
        \begin{itemize}
            \item Case $\Phi_{t \subs{x}{v}}$ ends with ($\ruleApp$), then it must be of the following form:
            \[ \begin{prooftree}
                \hypo{\Phi_{u \subs{x}{v}} \tr \seqi{\Gam_{u \subs{x}{v}}}{u \subs{x}{v}}{\M' \ta \tau}{(b', s')}}
                \hypo{\Phi_{p \subs{x}{v}} \tr \seqi{\Gam_{p \subs{x}{v}}}{p \subs{x}{v}}{\M'}{(b'', s'')}}
                \infer2[(\ruleApp)]{\seqi{\Gam_{u \subs{x}{v}} + \Gam_{p \subs{x}{v}}}{(u \subs{x}{v})(p \subs{x}{v})}{\tau}{(1+b'+b'', s'+s'')}}
            \end{prooftree} \]
            where $\Gam_{t \subs{x}{v}} = \Gam_{u \subs{x}{v}} + \Gam_{p \subs{x}{v}}$, $b = 1+b'+b''$, and $s = s' + s''$. By the \ih over $\Phi_{u \subs{x}{v}}$, we have the following derivations $\Phi_u \tr \seqi{\Gam_u; x : \M_1}{u}{\M' \ta \tau}{(b_u,s_u)}$ and $\Phi^1_v \tr \seqi{\Gam^1_v}{v}{\M_1}{(b^1_v,s^1_v)}$, such that $\Gam_{u \subs{x}{v}} = \Gam_u + \Gam^1_v$, $b' = b_u + b^1_v$, and $s' = s_u + s^1_v$. And by the \ih over $\Phi_{p \subs{x}{v}}$, we have the following derivation $\Phi_{p} \tr \seqi{\Gam_p; x : \M_2}{p}{\M'}{(b_p,s_p)}$ and $\Phi^2_v \tr \seqi{\Gam^2_v}{v}{\M_2}{(b^2_v,s^2_v)}$, such that $\Gam_{p \subs{x}{v}} = \Gam_p + \Gam^2_v$, $b'' = b_p + b^2_v$, and $s'' = s_p + s^2_v$. By~\cref{lem:merge-values}, we can take the following derivation $\Phi_v \tr \seqi{\Gam^1_v + \Gam^2_v}{v}{\M_1 \sqcup \M_2}{(b^1_v+b^2_v, s^1_v + s^2_v)}$, such that $\Gam_v = \Gam^1_v + \Gam^2_v$, $b_v = b^1_v + b^2_v$, and $s_v = s^1_v + s^2_v$. And we can build $\Phi_{up}$ as follows:
            \[ \begin{prooftree}
                \hypo{\Phi_u \tr \seqi{\Gam_u; x : \mul{\sig_i}_{\iI_1}}{u}{\M' \ta \tau}{(b_u,s_u)}}
                \hypo{\Phi_{p} \tr \seqi{\Gam_p; x : \mul{\sig_i}_{\iI_2}}{p}{\M'}{(b_p,s_p)}}
                \infer2[(\ruleApp)]{\seqi{(\Gam_u + \Gam_p); x : \mul{\sig_i}_{\iI}}{up}{\tau}{(1+b_u+b_p, s_u+s_p)}}
            \end{prooftree} \]
            such that $\Gam_t = \Gam_u + \Gam_p$, $b_t = 1+b_u + b_p$, and $s_t = s_u + s_p$. So we can pick $\Phi_t = \Phi_{up}$, and conclude with $\Gam_{t \subs{x}{v}} = \Gam_{u \subs{x}{v}} + \Gam_{p \subs{x}{v}} = \Gam_u + \Gam^1_v + \Gam_p + \Gam^2_v = (\Gam_u + \Gam_p) + (\Gam^1_v + \Gam^2_v) = \Gam_t + \Gam_v$, $b = 1+b'+b'' = 1+b_u + b^1_v + b_p + b^2_v = 1 + (b_u + b_p) + (b^1_v + b^2_v) = b_t + b_v$, and $s = s_u + s^1_v + s_p + s^2_v = (s_u + s_p) + (s^1_v + s^2_v) = s_t + s_v$.
            \item Case $\Phi_{t \subs{x}{v}}$ ends with (\ruleAppPOne) and (\ruleAppPTwo). These cases are very similar to the case where $\Phi_{t \subs{x}{v}}$ ends with rule (\ruleApp).
        \end{itemize}
    \end{itemize}
%\end{proof}
    \end{enumerate}
\end{proof}}

\begin{lemma}[{\bf Split Exact Subject Reduction and Expansion}]
    \label{lem:subjred-subjexp} \mbox{}
    \begin{enumerate} 
        \item \label{lem:subj-red} Let $\Phi_t \tr \seqi{\Gam}{t}{\tau}{(b,s)}$ be tight. If $t \dred t'$, then there exists $\Phi_{t'} \tr \seqi{\Gam}{t'}{\tau}{(b-1,s)}$.
        \item \label{lem:subj-exp} Let $\Phi_{t'} \tr \seqi{\Gam}{t'}{\tau}{(b,s)}$ be tight. If $t \dred t'$, then there exists $\Phi_t \tr \seqi{\Gam}{t}{\tau}{(b+1, s)}$.
    \end{enumerate}
\end{lemma}

\maybehide{\begin{proof} \mbox{}
    \begin{enumerate}
        \item %\begin{proof}
    We will actually prove the following stronger version of the statement, which allows us to reason inductively:

    Let $\Phi_t \tr \seqi{\Gam}{t}{\tau}{(b,s)}$, such that $\Gam$ is tight, and either $\tau$ is tight or $\neg\isvalue{t}$. If $t \dred t'$, then there exists $\Phi_{t'} \tr \seqi{\Gam}{t'}{\tau}{(b-1,s)}$.

    The proof now follows by induction over $\dred$:
    \begin{itemize}
        \item Case $t = (\lam x.u) v \dred u \subs{x}{v} = t'$.Assume that $\Phi_t$ ends with rule (\ruleAppPOne). Then $\lam x.u$ must be assigned type $\nott{\tabs}$, which is not possible by~\cref{lem:notabs-implies-negabs}. Now, assume that $\Phi_t$ ends with rule (\ruleAppPTwo). Then $v$ must be assigned typed $\tneutral$, which is not possible by~\cref{lem:values-not-neutral}. Therefore, $\Phi_t$ must be of the following form:
        \[ \begin{prooftree}
            \hypo{\Phi_u \tr \seqi{\Gam_u; x : \M}{u}{\tau}{(b_u,s_u)}}
            \infer1[(\ruleLam)]{\seqi{\Gam_u}{(\lam x.u)}{\M \ta \tau}{(b_u, s_u)}}
            \hypo{\Phi_{v} \tr \seqi{\Gam_v}{v}{\M}{(b_v,s_v)}}
            \infer2[(\ruleApp)]{\seqi{\Gam_u + \Gam_{v}}{(\lam x.u) v}{\tau}{(1+b_u+b_v, s_u+s_v)}}
        \end{prooftree} \]
        where $\tau \in \tightt$, $\Gam = \Gam_u + \Gam_v$ is tight, $b = 1 + b_u + b_v$, and $s = s_u + s_v$. By~\cref{lem:subsantisubs}.\ref{lem:subs}, we know there exists the following derivation $\Phi_{u \subs{x}{v}} \tr \seqi{\Gam_u + \Gam_v}{u \subs{x}{v}}{\tau}{(b_u+b_v,s_u+s_v)}$. So we can take $\Phi_{t'} = \Phi_{u \subs{x}{v}}$ and conclude with $b - 1 = b_u + b_v$.
        \item Case $t = up \dred u'p = t'$, such that $u \dred u'$. Then $\Phi_t$ must either end with (\ruleApp), (\ruleAppPOne), or (\ruleAppPTwo):
        \begin{itemize}
            \item Case $\Phi_t$ ends with rule (\ruleApp), then it must be of the following form:
            \[ \begin{prooftree}
                \hypo{\Phi_u \tr \seqi{\Gam_u}{u}{\M \ta \tau}{(b_u,s_u)}}
                \hypo{\Phi_p \tr \seqi{\Gam_p}{p}{\M}{(b_p,s_p)}}
                \infer2[(\ruleApp)]{\seqi{\Gam_u + \Gam_p}{up}{\tau}{(1 +b_u+b_p,s_u+s_p)}}
            \end{prooftree} \]
            where $\tau = \tau \in \tightt$, $\Gam = \Gam_u + \Gam_p$ is tight, $b = 1+b_u + b_p$, and $s = s_u + s_p$. Since $u \dred u'$, it is clear that $\neg\isvalue{u}$ holds. Moreover, $\Gam_u$ is necessarily tight. Therefore, by the \ih, there exists $\Phi_{u'} \tr \seqi{\Gam_u}{u'}{\M \ta \tau}{(b_u-1, s_u)}$. Thus, we can build $\Phi_{t'}$ as follows:
            \[ \begin{prooftree}
                \hypo{\Phi_{u'} \tr \seqi{\Gam_u}{u'}{\M \ta \tau}{(b_u-1, s_u)}}
                \hypo{\Phi_p \tr \seqi{\Gam_p}{p}{\M}{(b_p,s_p)}}
                \infer2[(\ruleApp)]{\seqi{\Gam_u + \Gam_p}{u'p}{\tau}{(b_u+b_p,s_u+s_p)}}
            \end{prooftree} \]
            And we can conclude with $b - 1= b_u + b_p$.
            \item Case $\Phi_t$ ends with rule (\ruleAppPOne) or (\ruleAppPTwo), the proof are similar to the one where $\Phi_t$ ends with rule (\ruleApp).
        \end{itemize}
        \item Case $t = up \dred up' = t'$, such that $u \not\dred$ and $p \dred p'$. Then $\Phi_t$ must either end with (\ruleApp), (\ruleAppPOne), or (\ruleAppPTwo):
        \begin{itemize}
            \item Case $\Phi_t$ ends with rule (\ruleApp), then it must be of the following form:
            \[ \begin{prooftree}
                \hypo{\Phi_u \tr \seqi{\Gam_u}{u}{\M \ta \tau}{(b_u,s_u)}}
                \hypo{\Phi_p \tr \seqi{\Gam_p}{p}{\M}{(b_p,s_p)}}
                \infer2[(\ruleApp)]{\seqi{\Gam_u + \Gam_p}{up}{\tau}{(1+b_u+b_p,s_u+s_p)}}
            \end{prooftree} \]
            where $\tau \in \tightt$, $\Gam = \Gam_u + \Gam_p$ is tight, $b = 1 + b_u + b_p$, and $s = s_u + s_p$. Since $p \dred p'$, it is clear that $\neg\isvalue{p}$. Moreover, $\Gam_p$ is necessarily tight. Therefore, by the \ih, we know there exists the following derivation $\Phi_{p'} \tr \seqi{\Gam_p}{p'}{\M}{(b_p-1, s_p)}$. Thus, we can build $\Phi_{t'}$ as follows:
            \[ \begin{prooftree}
                \hypo{\Phi_u \tr \seqi{\Gam_u}{u}{\M \ta \tau}{(b_u, s_u)}}
                \hypo{\Phi_{p'} \tr \seqi{\Gam_p}{p'}{\M}{(b_p-1,s_p)}}
                \infer2[(\ruleApp)]{\seqi{\Gam_u + \Gam_p}{up'}{\tau}{(b_u+b_p,s_u+s_p)}}
            \end{prooftree} \]
            And we can conclude with $b - 1 = b_u + b_p$.
            \item Case $\Phi_t$ ends with rule (\ruleAppPOne) or (\ruleAppPTwo), the proofs are similar to the ones where $\Phi_t$ ends with rule (\ruleApp).
        \end{itemize}
    \end{itemize}
%\end{proof}

        \item %\begin{proof}
    Just like for~\cref{lem:subjred-subjexp}.\ref{lem:subj-red}, we will actually prove the following stronger version of the statement, which allows us to reason inductively:

    Let $\Phi_{t'} \tr \seqi{\Gam}{t'}{\tau}{(b,s)}$, such that $\Gam$ is tight, and either ($\tau \in \tightt$ or $\neg\isvalue{t}$). If $t \dred t'$, then there exists $\Phi_t \tr \seqi{\Gam}{t}{\tau}{(b+1,s)}$.
    
    The proof now follows by induction over $\dred$:
    \begin{itemize}
        \item Case $t = (\lam x.u) v \dred u \subs{x}{v} = t'$. Then $\Phi_{t'} \tr \seqi{\Gam}{u \subs{x}{v}}{\tau}{(b,s)}$ and, by~\cref{lem:subsantisubs}.\ref{lem:antisubs}, there exist the following derivations $\Phi_u \tr \seqi{\Gam_u; x : \M}{u}{\tau}{(b_u, s_u)}$ and $\Phi_v \tr \seqi{\Gam_v}{v}{\M}{(b_v,s_v)}$, such that $\tau \in \tightt$, $\Gam = \Gam_u + \Gam_v$ is tight, $b = b_u + b_v$, and $s = s_u + s_v$. So we can build $\Phi_t$ as follows:
        \[ \begin{prooftree}
            \hypo{\Phi_u \tr \seqi{\Gam_u; x : \M}{u}{\tau}{(b_u, s_u)}}
            \infer1[(\ruleLam)]{\seqi{\Gam_u}{\lam x.u}{\M \ta \tau}{(b_u,s_u)}}
            \hypo{\Phi_v \tr \seqi{\Gam_v}{v}{\M}{(b_v,s_v)}}
            \infer2[(\ruleApp)]{\seqi{\Gam_u + \Gam_v}{(\lam x.u)v}{\tau}{(1+b_u+b_v, s_u+s_v)}}
        \end{prooftree} \]
        And we can conclude with $b + 1 = 1 + b_u + b_v$.
        \item Case $t = up \dred u'p = t'$, such that $u \dred u'$. Then $\Phi_{t'}$ must either end with (\ruleApp), (\ruleAppPOne), or (\ruleAppPTwo):
        \begin{itemize}
            \item Case $\Phi_{t'}$ ends with rule (\ruleApp), then it must be of the following form:
            \[ \begin{prooftree}
                \hypo{\Phi_{u'} \tr \seqi{\Gam_u}{u'}{\M' \ta \tau}{(b_u, s_u)}}
                \hypo{\Phi_p \tr \seqi{\Gam_p}{p}{\M'}{(b_p, s_p)}}
                \infer2[(\ruleApp)]{\seqi{\Gam_u + \Gam_p}{u'p}{\tau}{(1 + b_u + b_p, s_u + s_p)}}
            \end{prooftree} \]
            where $\tau \in \tightt$, $\Gam = \Gam_u + \Gam_p$ it tight, $b = 1 + b_u + b_p$, and $s = s_u + s_p$. Since $u \dred u'$, it is clear that $\neg\isvalue{u}$. Moreover, $\Gam_p$ is tight. Therefore, by the \ih, there exists the following derivation $\Phi_u \tr \seqi{\Gam_u}{u}{\M' \ta \tau}{(b_u + 1, s_u)}$. Thus, we can build $\Phi_{t'}$ as follows:
            \[ \begin{prooftree}
                \hypo{\Phi_u \tr \seqi{\Gam_u}{u}{\M' \ta \tau}{(b_u + 1, s_u)}}
                \hypo{\Phi_p \tr \seqi{\Gam_p}{p}{\M'}{(b_p, s_p)}}
                \infer2[(\ruleApp)]{\seqi{\Gam_u + \Gam_p}{up}{\tau}{(1 + b_u + 1 + b_p, s_u + s_p)}}
            \end{prooftree} \]
            And we can conclude with $b + 1 = (1 + b_u + b_p) + 1 = 1 + b_u + 1 + b_p$.
            \item Case $\Phi_{t'}$ ends with rule (\ruleAppPOne) or (\ruleAppPTwo), the proofs are similar to the one where $\Phi_{t'}$ ends with rule (\ruleApp).
        \end{itemize}
        \item Case $t = up \dred up' = t'$, such that $p \dred p'$. Then $\Phi_{t'}$ must either ends with (\ruleApp), (\ruleAppPOne), or (\ruleAppPTwo):
        \begin{itemize}
            \item Case $\Phi_{t'}$ ends with rule ($\ruleApp$), then it must be of the following form:
            \[ \begin{prooftree}
                \hypo{\Phi_u \tr \seqi{\Gam_u}{u}{\M' \ta \tau}{(b_u, s_u)}}
                \hypo{\Phi_{p'} \tr \seqi{\Gam_p}{p'}{\M'}{(b_p, s_p)}}
                \infer2[(\ruleApp)]{\seqi{\Gam_u + \Gam_p}{u p'}{\tau}{(1 + b_u + b_p, s_u + s_p)}}
            \end{prooftree} \]
            where $\tau \in \tightt$, $\Gam = \Gam_u + \Gam_{p'}$ is tight, $b = 1 + b_u + b_p$, $s_t = s_u + s_p$. Since $p \dred p'$, it is clear that $\neg\isvalue{p}$ holds. Moreover, $\Gam_p$ is tight. Therefore, by the \ih, we have the following derivation $\Phi_p \tr \seqi{\Gam_p}{p}{\M' \ta \tau}{(b_p + 1, s_p)}$. Thus, we can build $\Phi_{t'}$ as follows:
            \[ \begin{prooftree}
                \hypo{\Phi_u \tr \seqi{\Gam}{u}{\M' \ta \tau}{(b_u, s_u)}}
                \hypo{\Phi_p \tr \seqi{\Gam_p}{p}{\M'}{(b_p + 1, s_p)}}
                \infer2[(\ruleApp)]{\seqi{\Gam_u + \Gam_p}{up}{\tau}{(1 + b_u + b_p + 1, s_u + s_p)}}
            \end{prooftree} \]
            And we can conclude with $b + 1 = (1 + b_u + b_p) + 1 = 1 + b_u + b_p + 1$.
            \item Case $\Phi_{t'}$ ends with rule (\ruleAppPOne) or (\ruleAppPTwo), the proofs are similar to the one where $\Phi_{t'}$ ends with rule (\ruleApp).
        \end{itemize}   
    \end{itemize}
%\end{proof}
    \end{enumerate}
\end{proof}}

\begin{theorem}[{\bf Quantitative Soundness and Completeness}]
    \label{thm:soundnesscompleteness}
  \item \label{thm:soundness} If $\Phi \tr \seqi{\Gam}{t}{\tau}{(b,s)}$ is tight, then there exists $u \in \normal$ such that  $t \drred^b u$ with $\size{u} = s$.
  \item \label{thm:completeness} If $t \drred^b u$ with  $u \in \normal$, then there exists a tight type derivation $\Phi_t \tr \seqi{\Gam}{t}{\tau}{(b, \size{u})}$.
\end{theorem}

\maybehide{\begin{proof} \mbox{}
    \begin{enumerate} 
        \item %\begin{proof}
    The proof follows by induction over $b$:
    \begin{itemize}
        \item Case $b = 0$. Then $t \in \normal$, by~\cref{lem:zero-steps-nfs}. And $d = \size{t}$, by~\cref{lem:corr-size-counter}. So we can conclude with $u = t$.
        \item Case $b > 0$. Then $t \not\in \normal$, by~\cref{lem:zero-steps-nfs}. Therefore, there exists $t'$ such that $t \dred t'$, by~\cref{prop:char-nfs}. By\cref{lem:subjred-subjexp}.\ref{lem:subj-red}, there exists $\Phi_{t'} \tr \seqi{\Gam}{t'}{\tau}{(b-1, s)}$. By the \ih, there exists $u \in \normal$, such that $t' \drred^{b-1} u$, such that $d = \size{u}$. So we can conclude with $t \dred t' \drred^{b-1} u$, which means that $t \drred^b u$, as expected.
    \end{itemize}
%\end{proof}
        \item %\begin{proof}
    The proof follows by induction over $b$:
    \begin{itemize}
        \item Case $b = 0$. Then $t = u$, which means that $t \in \normal$. Therefore, we can conclude by~\cref{lem:typ-nfs}.
        \item Case $b > 0$. Then there exists $t'$, such that $t \dred t' \drred^{b-1} u$. By the \ih, there exists a tight derivation $\Phi_{t'} \tr \seqi{\Gam}{t'}{\tau}{(b-1, \size{u})}$. By\cref{lem:subjred-subjexp}.\ref{lem:subj-exp}, there exists a tight derivation $\Phi \tr \seqi{\Gam}{t}{\tau}{(b, \size{u})}$. So, we can conclude.
    \end{itemize}
%\end{proof}
    \end{enumerate}
\end{proof}}
  

  




\subsection{A \texorpdfstring{$\lambda$}{Lambda}-Calculus with Global State}

\subsubsection{General Lemmas}

\propnormalifffinal*

\maybehide{\begin{proof}
    \begin{itemize}
    \item[$\Ra$)] Let $(t, s)$ be \final. We consider two cases:
      \begin{itemize}
            \item Case $(t,s)$ is blocked. We reason by induction on blocked configurations. \begin{itemize}
                \item Case $(t,s) = (\get{l}{x}{u}, s)$, such that $l \not\in \dom{s}$. Then $(t, s) \not\ra$ is straightforward.
                \item Case $(t,s) = (v u, s)$ and $(u,s)$ is blocked.
                  Then by the \ih, we have that $(u,s) \not\ra$. Therefore, $(v u, s) \not\ra$ holds.
            \end{itemize}
          \item Case $t \in \normal$. We reason by induction on
            $\normal$. \begin{itemize}
                \item Case $t=v \in \val$. Then $(v,s) \not\ra$  is straightforward.
                \item Case $t \in \neutral$. Then $t = v u$ and we have to consider two different  cases: \begin{itemize}
                  \item Case  $v= x$ and $u \in \normal$. Then by the \ih, we have $(u,s) \not\ra$. Therefore, $(v u, s) \not\ra$ holds.
                    \item Case $v = (\lam x.p)$ and $u \in \neutral$. Then $u \in \normal$, and by the \ih, we have that $(u,s) \not\ra$. Therefore $(v u,s) \not\ra$ holds.
                \end{itemize}
            \end{itemize}
        \end{itemize}
      \item[$\La$)] Let $t \not \ra$. We reason by
        induction on $t$: \begin{itemize}
            \item Case $t = v$. Then $t \in \normal$. Therefore $(t,s)$ is \final.
            \item Case $t = v u$. Since $(v u, s) \not\ra$, then $(u,s) \not\ra$. By the \ih, we have $(u,s)$ \final. Now, we reason
              by cases: \begin{itemize}
                \item Case $(u, s)$ is blocked. Then, $(v u, s)$ is blocked by definition. 
                \item Case $u \in \normal$. Then we have two cases: \begin{itemize}
                    \item Case $u \in \neutral$. Then $vu \in \normal$. Therefore,  $(t,s)$ is \final.
                    \item Case $u \in \val$ and $v = \lam x.p$. Then $((\lam x.p) u, s) \ra (p \subs{x}{u}, s)$, which yields a contradiction with the hypothesis $t=vu\not\ra$. Thus, this case does not apply.
                \end{itemize}
            \end{itemize}
          \item Case $t = \get{l}{x}{u}$. Since $(\get{l}{x}{u},s) \not\ra$, then $l \not\in \dom{s}$. Therefore, $(\get{l}{x}{u},s)$ is blocked, which implies
$(t,s)$ is \final. 
            \item Case $t = \set{l}{v}{u}$. Then $(\set{l}{v}{u}, s) \ra (u, \upd{l}{v}{s})$, which yields to a contraction with the hypothesis  $t\not\ra$. 
              Therefore, this case does not apply.
        \end{itemize}
    \end{itemize}
\end{proof}
} 

\proptypedunblock*

\maybehide{\begin{proof}
    By induction on $t$: \begin{itemize}
        \item Case $t \in \val $ or $t = \set{l}{v}{t}$. Then the conclusion trivially holds, since clearly $(t,s)$ is not a blocked configuration.
        \item Case $t = \get{l}{x}{t}$. We have two  cases: \begin{itemize}
            \item Case $l \in \dom{s}$. Then $(t,s)$ is clearly unblocked.
            \item Case $l \not\in \dom{s}$. Let $\stype_0 = \conj{(l : \Gam(x))} \splus \stype$. Since $t = \get{l}{x}{u}$, then $\Phi$ must be of the following form:
            \[ \begin{prooftree}
                \hypo{\Phi \tr \seqi{\Gam_u \sm x}{\get{l}{x}{u}}{\comptype{
                    \stype_0}{\ctype}}{(b_u,m_u,d_u)}}
                \hypo{\Phi_s \tr \seqi{\Del}{s}{\stype_0}{(b_s,m_s,d_s)}}
                \infer2[(\ruleConf)]{\seqi{(\Gam_u \sm x) + \Del}{(\get{l}{x}{t}, s)}{\ctype}{(b_u+b_s,1+m_u+m_s,d_u+d_s)}}
              \end{prooftree} \] 
              where $\Gam = \Gam_u \sm x$, $b = b_u+b_s$, $m = 1+m_u+m_s$, and $d = d_u + d_s$. Thus, $l \in \dom{\conj{(l : \Gam_u(x))} \splus \stype}$, and so by
            \cref{lem:states-and-state-types} we have 
          $l \in \dom{s}$, which gives a contradiction with the hypothesis $l \not\in \dom{s}$. Therefore, this case does not apply,
        \end{itemize}
        \item Case $t = v u$. Assume $\Phi_v \tr \seqi{\Gam_v}{v}{\M \ta (\comptype{\stype'}{\ctype})}{(b_v,m_v,d_v)}$ and $\Phi_u \tr \seqi{\Gam_u}{u}{\tcomptype{\stype}{\M}{\stype'}}{(b_u,m_u,d_u)}$. Then $\Phi$ must be of the following form:
        \[ \begin{prooftree}
            \hypo{\Phi_v}
            \hypo{\Phi_u}
            \infer2[(\ruleApp)]{\seqi{\Gam_v + \Gam_u}{v u}{\comptype{\stype}{\ctype}}{(1+b_v+b_u,m_v+m_u,d_v+d_u)}}
            \hypo{\Phi_s \tr \seqi{\Del}{s}{\stype}{(b_s,m_s,d_s)}}
            \infer2[(\ruleConf)]{\seqi{(\Gam_v + \Gam_u) + \Del}{(v u, s)}{\ctype}{(1+b_v + b_u + b_s, m_v + m_u + m_s, d_v + d_u + d_s)}}
        \end{prooftree} \]
        where $\Gam = (\Gam_v + \Gam_u) + \Del$, $b = 1+b_v + b_u + b_s$, $m = m_v + m_u + m_s$, and $d = d_v + d_u + d_s$. Thus, we can build the following derivation for $(u,s)$:
        \[ \begin{prooftree}
            \hypo{\Phi_u \tr \seqi{\Gam_u}{u}{\tcomptype{\stype}{\M}{\stype'}}{(b_u,m_u,d_u)}}
            \hypo{\Phi_s \tr \seqi{\Del}{s}{\stype}{(b_s,m_s,d_s)}}
            \infer2[(\ruleConf)]{\seqi{\Gam + \Del}{(u,s)}{\conftype{\M}{\stype'}}{(b_u+b_s,m_u+m_s,d_u+d_s)}}
        \end{prooftree} \]
        By the \ih, we have that $(u,s)$ is unblocked. Therefore, $(v u, s)$ also unblocked.
    \end{itemize}
\end{proof}} 

\begin{lemma}[Relevance]
    Let $\Phi \tr \seqi{\Gam}{t}{\gtype}{(b,m,d)}$ (resp. $\Phi' \tr \seqi{\Gam}{s}{\stype}{(b',m',d')}$). Then $\dom{\Gam} \subseteq \fv{t}$ (resp. $\dom{\Gam} \subseteq \fv{s}$).
\end{lemma}

\maybehide{\begin{proof}
    The proof following by induction over $\Phi$ (resp. $\Phi'$). Case $\Phi$ (resp. $\Phi'$) ends with rule (\ruleAx), (\ruleAxP), or (\ruleLamP) (resp. rule (\ruleEmp)), then $\Phi$ (resp. $\Phi'$) is clearly relevant. The other cases follow easily from the \ih.
\end{proof}}

\subsubsection{Soundness Lemmas (Auxiliary Lemmas)}

\lemzerocounters*

\maybehide{\begin{proof} \mbox{}
    \begin{enumerate}
        \item \input{proofs/lem-zero-counters}
        \item \input{proofs/lem-zero-size-store}
    \end{enumerate}
\end{proof}} 

\begin{lemma}
    \label{lem:zero-counters-normal}
    Let $\Phi \tr \seqi{\Gam}{t}{\del}{(0,0,d)}$ be tight. If $t \in \normal$, then $\del = \stype \ra \tightt \tim \stype'$ and $\stype =\stype'$.
\end{lemma}

\maybehide{\begin{proof} 
  By induction on $t \in \normal$. We consider two cases:
  \begin{itemize}
    \item Case $t \in \val$. Then such a typing derivation can only end with rule (\ruleAx) followed by rule (\ruleLift) or (\ruleLamP)followed by rule (\ruleLift), in which cases the statement is obvious.
    \item Case $t = vu \in \neutral$. Since the first counter of the derivation is $0$, $\Phi$ can only end with a persistent rule (\ruleAppPOne) or (\ruleAppPTwo). In both cases, we can conclude by applying the \ih to $u \in \normal$ or $u \in \neutral$ and their type derivations, which gives  $\stype = \stype'$.
  \end{itemize}
\end{proof}} 

\lemzeronfs*

\maybehide{\begin{proof}\
  \begin{itemize}
    \item[$\Ra$)] By point (1) of~\cref{lem:zero-counters}.
    \item[$\La$)] By induction on $t$: \begin{itemize}
    \item Case $t \in \val$. There are six cases to consider for $\Phi$:
    \begin{itemize}    
      \item $\Phi$ ends with (\ruleAx). This case does not apply since the resulting type is not a monadic type. %Then $\Phi \tr \seqi{x:\mul{\rdel}}{x}{\rdel}{(0,0,0)}$ and the conclusion holds trivially.
      \item $\Phi$ ends with (\ruleLam). This case does not apply since the resulting type is not a monadic type.
      \item $\Phi$ ends with (\ruleMany). This case does not apply since the resulting type is not a monadic type.
      \item $\Phi$ ends with (\ruleLift). This case does not apply, since $\del = \tcomptype{\stype}{\M}{\stype'}$, but $\M \not\in \tightt$.
      \item $\Phi$ ends with (\ruleAxP). Then $\Phi \tr \seqi{x:\mul{\nott{\tneutral}}}{x}{\tcomptype{\stype}{\nott{\tneutral}}{\stype}}{(0,0,0)}$, with $\stype$ tight, and the conclusion holds trivially.
      \item $\Phi$ ends with (\ruleLamP). Then $\Phi \tr \seqi{}{\lambda x.t}{\tcomptype{\stype}{\vl}{\stype}}{(0,0,0)}$, with $\stype$ tight, and the conclusion holds trivially. 
    \end{itemize}
    \item Case $t = xu$. Then $u \in \normal$, by definition and there are two cases to consider for $\Phi$:
    \begin{itemize}
      \item If $\Phi$ ends with (\ruleApp). Then $\Phi_u \tr \seqi{\Gam_u}{u}{\tcomptype{\stype}{\M}{\stype'}}{(b_u,m_u,d_u)}$, $\Phi_x \tr \seqi{x : \M \ta (\comptype{\stype'}{\ctype})}{x}{\M \ta (\comptype{\stype'}{\ctype})}{(b_x,m_x,d_x)}$, such that $\Gam =  (x:\mul{\M \ta (\comptype{\stype'}{\ctype})}) + \Gam_u$ is tight. Absurd, since $\M \ta (\comptype{\stype'}{\ctype})$ is not tight, therefore this case does not apply.
      \item If $\Phi$ ends with (\ruleAppPOne). Then $\Phi_u \tr \seqi{\Gam_u}{u}{\tcomptype{\stype}{\tightt}{\stype}}{(b_u,m_u,d_u)}$, such that $\Gam = (x: \mul{\tvar})+\Gam_u$ is tight, $b = b_u$, $m =m_u$, $d = d_u+ 1$, and $\stype$ is tight. By the \ih\ on $u$, we have $b_u=m_u=0$, therefore $b = m = 0$.
      \end{itemize}
      \item Case $t = (\lam x.p) u$. Then $u \in \neutral$, by definition and there are two cases to consider for $\Phi$:
      \begin{itemize}
        \item If $\Phi$ ends with (\ruleApp). Then $\Phi_u \tr \seqi{\Gam_u}{u}{\tcomptype{\stype}{\M}{\stype'}}{(b_u,m_u,d_u)}$, $\Phi_{\lam x.p} \tr \seqi{\Gam_{\lam x.p}}{\lam x.p}{\M \ta (\comptype{\stype'}{\ctype})}{(b_p,m_p,d_p)}$, such that $\Gam = \Gam_u + \Gam_{\lambda x.p}$ is tight, $b = 1+b_l+b_u$, $m = m_l+m_u$, $d = d_l+ d_m$. Since $\Gam_u$ is tight and $u\in\neutral$, by~\cref{lem:comp-tight-spreading}, $\M \in \tightt$, which is absurd. Therefore, this case does not apply.
        \item If $\Phi$ ends with (\ruleAppPTwo). Then $\Phi_u \tr \seqi{\Gam_u}{u}{\tcomptype{\stype}{\tneutral}{\stype}}{(b_u,m_u,d_u)}$, such that $\Gam = \Gam_u$ is tight, $b = b_u$, $m=m_u$, $d = d_u+ 1$ and $\stype_f$ is tight. By the \ih\ on $u$, we have $b_u=m_u=0$. Therefore $b = m = 0$.
      \end{itemize}
    \end{itemize}
  \end{itemize}
\end{proof}
}

\begin{lemma}
    \label{lem:states-and-state-types}
    Let $\Phi \tr \seqi{\Del}{s}{\stype}{(b,m,d)}$. If $l \in \dom{\stype}$, then $l \in \dom{s}$.
\end{lemma}
  
\maybehide{\begin{proof}
    We proceed by proving the following stronger version of the statement: 
    
    Let $\Phi_s \tr \seqi{\Del_s}{s}{\stype_s}{(b_s,m_s,d_s)}$. If $l \in \dom{\stype_s}$, then $s \equivstate \upd{l}{v}{q}$, for some value $v$ and store $q$.
    
    The proof follows by induction on $\Phi_s$: 
    \begin{itemize}
        \item Case $\Phi_s$ ends with ($\ruleEmp$). Then the conclusion is vacuously true.
        \item Case $\Phi_s$ ends with ($\ruleUpd$). Then $\Phi_s$ is of the following form: 
        \[ \begin{prooftree}
            \hypo{\Phi_v \tr \seqi{\Gam_v}{v}{\M}{(b_v,m_v,d_v)}}
            \hypo{\Phi_q \tr \seqi{\Del_q}{q}{\stype_q}{(b_q,m_q,d_q)}}
            \infer2[(\ruleUpd)]{\seqi{\Gam_v + \Del_q}{\upd{l'}{v}{q}}{\conj{l' : \M}; \stype_q}{(b_v+b_q,m_v+m_q,d_v+v_q)}}
        \end{prooftree} \]
        where $\Del_s = \Gam_v + \Del_q$, $s = \upd{l'}{v}{q}$, $\stype_s = \conj{l' : \M}; \stype_q$, $b_s = b_v + b_q$, $m_s = m_v + m_q$, and $d_s = d_v + d_q$. Now we consider two  cases: 
        \begin{itemize}
            \item Case $l = l'$. Then we are done.
            \item Case $l \not= l'$. Since we are assuming that $l \in \dom{\stype_s}$, then it must be case that $l \in \dom{\stype_q}$. But, then by the \ih, we have $q \equivstate \upd{l}{w}{q'}$, for some value $w$ and store $q'$. Therefore, $s \equivstate \upd{l'}{v}{\upd{l}{w}{q'}} \equivstate \upd{l}{w}{\upd{l'}{v}{q'}}$.
        \end{itemize}
    \end{itemize}
    The correctness of the original statement now follows easily from the fact that, clearly, if $s \equivstate \upd{l}{v}{q}$, then $l \in \dom{s}$, by Definition~\ref{def:domainS}.
\end{proof}
} 

\begin{lemma}[{\bf Split Lemma}] \mbox{} 
    \label{lem:split-values-stores}
    \begin{enumerate}
        \item {(\bf Values)} \label{lem:com-split-values}  Let $\Phi_v \tr \seqi{\Gam}{v}{\M}{(b,m,d)}$, such that $\M = \sqcup_{\iI} \M_i$. Then, there exist ($\Phi^i_v \tr \seqi{\Gam_i}{v}{\M_i}{(b_i,m_i,d_i)})_{\iI}$, such that $\Gam = +_{\iI} \Gam_i$, $b = +_{\iI} b_i$, $m = +_{\iI} m_i$, and $d = +_{\iI} d_i$.
        \item {\bf (States)} \label{lem:split-state} Let $\Phi_s \tr \seqi{\Gam}{s}{\stype}{(b,m,d)}$, such that $l \in \dom{\stype}$. Then, $s \equivstate \upd{l}{v}{q}$, $\Phi_v \tr \seqi{\Gam_v}{v}{\stype(l)}{(b_v,m_v,d_v)}$ and $\Phi_q \tr \seqi{\Gam_q}{q}{\stype'}{(b_q,m_q,d_q)}$, such that $\Gam = \Gam_v + \Gam_q$, $\stype = \conj{(l : \stype(l))}; \stype'$, $b = b_v+b_q$, $m = m_v+m_q$, and $d = d_v + d_q$.
    \end{enumerate}
\end{lemma}

\maybehide{\begin{proof}
    The proof for values is very similar to the corresponding proof for $\lam_s$, so we are only going to show the split lemma for states.
    The proof follows by induction on the structure of $s$: \begin{itemize}
        \item Case $s = \estate$. Then the statement is vacuously true.
        \item Case $s = \upd{l'}{w}{q'}$. Then $\Phi_s$ is of the form: 
        \[ \begin{prooftree}
            \hypo{\Phi_{w} \tr \seqi{\Gam_{w}}{w}{\M}{(b_w,m_w,d_w)}}
            \hypo{\Phi_{q'} \tr \seqi{\Gam_{q'}}{q'}{\stype_{q'}}{(b_{q'}, m_{q'},d_{q'})}}
            \infer2[(\ruleUpd)]{\seqi{\Gam_{w} + \Gam_{q'}}{\upd{l'}{w}{q'}}{\conj{(l' : \M)}; \stype_{q'}}{(b_w+b_{q'},m_w+m_{q'},d_w+d_{q'})}}
        \end{prooftree} \] where $\Gam = \Gam_{w} + \Gam_{q'}$, $\stype = \conj{(l' : \M)}; \stype_{q'}$, $b = b_w + b_{q'}$, $m = m_w + m_{q'}$, and $d = d_w + d_{q'}$. 
        We  consider two cases: \begin{itemize}
            \item Case $l' = l$. Then we simply take $v = w$ and $q = q'$ and we are done.
            \item Case $l' \not= l$.  Since $l \in \dom{\conj{(l' : \M)}; \stype_{q'}}$ and $l' \not= l$, then $l \in \dom{\stype_{q'}}$. By applying the \ih to $q'$, we have that  $q' \equivstate \upd{l}{w'}{q''}$, $\Phi_{w'} \tr \seqi{\Gam_{w'}}{w'}{\stype_{q'}(l)}{(b_{w'},m_{w'},d_{w'})}$ and $\Phi_{q''} \tr \seqi{\Gam_{q''}}{q''}{\stype_{q''}}{(b_{q''},m_{q''},d_{q''})}$, such that $\Gam_{q'} = \Gam_{w'} + \Gam_{q''}$, $\stype_{q'} = \conj{(l : \stype_{q'}(l))}; \stype_{q''}$, $b_{q'} = b_{w'} + b_{q''}$, $m_{q'} = m_{w'} + m_{q''}$, and $d_{q'} = d_{w'} + d_{q''}$. But $s = \upd{l'}{w}{\upd{l}{w'}{q''}} \equivstate \upd{l}{w'}{\upd{l'}{w}{q''}}$, so we can take $v = w'$, $q = \upd{l'}{w}{q''}$, and consider $\Phi_q$ to be the following derivation:
            \[ \begin{prooftree}
                \hypo{\Phi_{w} \tr \seqi{\Gam_{w}}{w}{\M}{(b_w,m_w,d_w)}}
                \hypo{\Phi_{q''} \tr \seqi{\Gam_{q''}}{q''}{\stype_{q''}}{(b_{q''}, m_{q''}, d_{q''})}}
                \infer2[(\ruleUpd)]{\seqi{\Gam_{w} + \Gam_{q''}}{\upd{l'}{w}{q''}}{\conj{(l' : \M)}; \stype_{q''}}{(b_w+b_{q''}, m_w + m_{q''}, d_w+d_{q''})}}
            \end{prooftree} \] where  $\Gam_q = \Gam_{w} + \Gam_{q''}$ and $\stype_q=\conj{(l' : \M)}; \stype_{q''}$. We can then conclude with the following observations:
            \begin{itemize}
            \item $\Gam_v + \Gam_q = \Gam_{w'} +\Gam_{w} + \Gam_{q''} =
              \Gam_{w} + \Gam_{q'} = \Gam$,
                \item Since $\stype = \conj{(l' : \M)}; \stype_{q'}$ and $l' \not= l$, then $\stype(l) = \stype_{q'}(l)$ and
                \begin{align*}
                    \stype = \conj{(l' : \M)}; \stype_{q'} & = \conj{(l': \M)}; \conj{(l : \stype_{q'}(l))}; \stype_{q''} \\
                    & = \conj{(l : \stype_{q'}(l))}; \stype_{q} \\
                    & = \conj{(l : \stype(l))}; \stype_q
                \end{align*}
              \item $b_v + b_q= b_{w'} + b_{w} + b_{q''}=
                 b_w + b_{q'} = b$, $m_v + m_q= m_{w'} + m_{w} + m_{q''}=
                 m_w + m_{q'} = b$ and
                 $d_v + d_q= d_{w'} + d_{w} + d_{q''}=
                 d_w + d_{q'} = d$.
            \end{itemize} 
        \end{itemize}
    \end{itemize}
\end{proof}
} 

\begin{lemma}
    \label{lem:comp-values-not-neutral}
    Let $\Phi \tr \seqi{\Gam}{t}{\tcomptype{\stype}{\tau}{\stype'}}{(b,m,d)}$. If $t \in \val$, then $\tau \neq \tneutral$.
\end{lemma}

\maybehide{\begin{proof}
    By case analysis on the form of $t \in \val$:
    \begin{itemize}
        \item Case $t = x$. Then we have to consider three cases according to the last rule used in $\Phi$:
        \begin{itemize}
            \item Case $\Phi$ ends with rule (\ruleAx), then $t$ can only be assigned $\sig$. Therefore, this case does not apply.
            \item Case $\Phi$ ends with rule (\ruleMany), then $\tau = \M \neq \tneutral$.
        
            \item Case $\Phi$ ends with rule (\ruleLift). Then $\tau \in \{\tvar, \tabs, \M\}$, which means that $\tau \not= \tneutral$.
        \end{itemize}
        \item Case $t = \lam x.t$. Then we have to consider three cases according to the last rule used in $\Phi$:
        \begin{itemize}
            \item Case $\Phi$ ends with rule (\ruleLam), then $t$ can only be assigned $\sig$. Therefore, this case does not apply.
            \item Case $\Phi$ ends with rule (\ruleMany), then $\tau = \M  \neq \tneutral$.
            \item Case $\Phi$ ends with rule (\ruleLamP), then $\tau = \vl$. Therefore, this case does not apply.
            \item Case $\Phi$ ends with rule (\ruleLift). $\tau \in \{\tabs, \M\}$, which means that $\tau \not= \tneutral$.
        \end{itemize}
    \end{itemize}
\end{proof}} 

\begin{lemma}
    \label{lem:comp-notabs-implies-negabs}
    Let $\Phi \tr \seqi{\Gam}{t}{\tcomptype{\stype}{\tau}{\stype'}}{(b,m,d)}$, such that $\Gam$ is tight. If $\tau \in \nott{\vl}$, then $\neg\isabs{t}$.
\end{lemma}

\maybehide{\begin{proof}
    By induction over $\Phi$:
    \begin{itemize}
        \item Case $\Phi$ ends with rule (\ruleAx), (\ruleApp), (\ruleGet), (\ruleSet), (\ruleAxP) (\ruleAppPOne), or (\ruleAppPTwo), then $\neg\isabs{t}$ holds by definition.
        \item Case $\Phi$ ends with rule (\ruleLam), (\ruleMany), or (\ruleLamP), then $\tau \in \nott{\tabs}$ does not hold. Therefore, these cases do not apply.
    \end{itemize}
\end{proof}} 

\subsubsection{Completeness (Auxiliary Lemmas)}

\begin{lemma}[{\bf Merge for Values}]
    \label{lem:comp-merge-values}
    Let $(\Phi^i_v \tr \seqi{\Gam_i}{v}{\M_i}{(b_i,m_i,d_i)})_{\iI}$. Then, there exists $\Phi_v \tr \seqi{\Gam}{v}{\M}{(b,m,d)}$, such that $\Gam = +_{\iI} \Gam_i$, $\M = +_{\iI} \M_i$, $b = +_{\iI} b_i$, $m = +_{\iI} m_i$, and $d = +_{\iI}$.
\end{lemma}
We omit this proof given its similarity with the proof for system $\syscbv$.

\lemcomtightspreading*

\maybehide{\begin{proof}
  We want to show that, if $t \in \neutral$, then $\tau \in \tightt$, for some $\stype'$. We proceed by induction on the predicate  $t \in \neutral$:
    \begin{itemize}
        \item Case $t = xu$, such that $u \in \normal$. Then we have to consider the following two cases depending on the last rule in $\Phi$:
        \begin{itemize}
            \item Case $\Phi$ ends with rule ($\ruleApp$), then it must be of the following form:
            \[ \begin{prooftree}
                \infer0[(\ruleAx)]{\seqi{x : \mul{\M \ta (\comptype{\stype'}{\ctype})}}{x}{\M \ta (\comptype{\stype'}{\ctype})}{(0,0,0)}}
                \hypo{\Phi_u \tr \seqi{\Gam_u}{u}{\tcomptype{\stype}{\M}{\stype'}}{(b_u,m_u,d_u)}}
                \infer2[(\ruleApp)]{\seqi{(x : \mul{\M \ta (\comptype{\stype'}{\ctype})}) + \Gam_u}{xu}{\comptype{\stype}{\ctype}}{(1+b_u,m_u,d_u)}}
            \end{prooftree} \]
            where $\Gam = (x : \mul{\M \ta (\comptype{\stype'}{\ctype})}) + \Gam_p$ is tight, $b = 1+b_u$, $m = m_u$, and $d = d_u$. But $\M \ta (\comptype{\stype'}{\ctype}) \not\in \tightt$, therefore $\Gam$ is not tight and we have a contraction. Thus, this case does not apply.
            \item Case $\Phi$ ends with rule (\ruleAppPOne), then $\tau = \tneutral \in \tightt$, so we can conclude immediately.
        \end{itemize}
        \item Case $t = (\lambda x.p)u$, such that $u \in \neutral$. Then we have to consider the following two cases depending on the last rule in $\Phi$:
        \begin{itemize}
            \item Case $\Phi$ ends with rule ($\ruleApp$), then it must be of the following form:
            \[ \begin{prooftree}
                \hypo{\seqi{\Gam_p}{\lam x.p}{\M \ta (\comptype{\stype'}{\ctype})}{(b_p,m_p,d_p)}}
                \hypo{\Phi_u \tr \seqi{\Gam_u}{u}{\tcomptype{\stype}{\M}{\stype'}}{(b_u,m_u,d_u)}}
                \infer2[(\ruleApp)]{\seqi{\Gam_p + \Gam_u}{(\lam x.p)u}{\comptype{\stype}{\ctype}}{(1+b_p+b_u,m_p+m_u,d_p+d_u)}}
            \end{prooftree} \]
            where $\Gam = \Gam_u + \Gam_p$ is tight, $b = 1 + b_p + b_u$, $m = m_p + m_u$, and $d = d_p + d_u$. By the \ih on $u$, we have that $\M \in \tightt$, which is a contradiction. Therefore, this case does not apply.
            \item Case $\Phi$ ends with rule (\ruleAppPTwo). Then $\tau = \tneutral \in \tightt$, so we can conclude immediately.
        \end{itemize}
    \end{itemize}
\end{proof}
} 

\typstates*

\maybehide{\begin{proof} \mbox{}
    \begin{enumerate}
        \item \input{proofs/lem-typ-states}
        \item \input{proofs/lem-comp-typ-nfs}
    \end{enumerate}
\end{proof}


} 

\subsubsection{Soundness and Completeness (Main Lemmas)}

\lemcompsubsantisubs*

\maybehide{\begin{proof} \mbox{}
    \begin{enumerate}
        \item %\begin{proof}
    We are going to generalize the original statement by replacing $\del$ with $\gtype$.
    \\ \\
    The proof now follows by induction over the structure of $\Phi_t$:
        \begin{itemize}
            \item Case $\Phi_t$ ends with rule ($\ruleAx$). Then $t$ must be a variable and we must consider two cases:
            \begin{itemize}
                \item Assume $t = y = x$. Then $\Gam_t = \eset$, $\gtype = \M$, $t \subs{x}{v} = v$, $b_t = m_t = d_t = 0$. So we can take $\Phi_{t \subs{x}{v}} = \Phi_v$ and conclude with $\Gam_t + \Gam_v = \Gam_v$, $b_t + b_v = b_v$, $m_t + m_v = m_v$, and $d_t + d_v = d_v$.
                \item Assume $t = y \not= x$. Then $\M = \emul$, $\Gam_v = \eset$, $t \subs{x}{v} = t$, $b_v = 0$, $m_v = 0$, and $d_v = 0$. So we can take $\Phi_{t \subs{x}{v}} = \Phi_t$ and conclude with $\Gam_t + \Gam_v = \Gam_t$, $b_t + b_v = b_t$, $m_t + m_v = m_t$, and $d_t + d_v = d_t$.
            \end{itemize}
            \item Case $\Phi_t$ ends with (\ruleLam). Then $t$ must be of the form $\lam y.u$ and $\Phi_t$ must be of the following form (by $\alpha$-conversion):
            \[ \begin{prooftree}
                \hypo{\Phi_u \tr \seqi{\Gam; x : \M}{u}{\comptype{\stype}{\ctype}}{(b_t,m_t,d_t)}}
                \infer1[(\ruleLam)]{\seqi{(\Gam \sm y); x : \M}{\lam y.u}{\Gam(y) \ta (\comptype{\stype}{\ctype})}{(b_t,m_t,d_t)}}
            \end{prooftree} \]
            where $\Gam_t = (\Gam \sm y)$, and $\gtype = \Gam(y) \ta (\comptype{\stype}{\ctype})$. By the \ih, we have the following derivation $\Phi_{u \subs{x}{v}} \tr \seqi{\Gam + \Gam_v}{u \subs{x}{v}}{\comptype{\stype}{\ctype}}{(b_t+b_v,m_t+m_v,d_t+d_v)}$. Therefore, we can build $\Phi_{t \subs{x}{v}}$ as follows:
            \[ \begin{prooftree}
                \hypo{\Phi_{u \subs{x}{v}} \tr \seqi{\Gam + \Gam_v}{u \subs{x}{v}}{\comptype{\stype}{\ctype}}{(b_t+b_v,m_t+m_v,d_t+d_v)}}
                \infer1[(\ruleLam)]{\seqi{(\Gam + \Gam_v) \sm y}{\lambda y.u \subs{x}{v}}{\Gam(y) \ta (\comptype{\stype}{\ctype})}{(b_t+b_v,m_t+m_v,d_t+d_v)}}
            \end{prooftree} \]
            And we conclude with $(\Gam + \Gam_v) \sm y = (\Gam \sm y) + \Gam_v = \Gam_t + \Gam_v$, by $\alpha$-conversion.
            \item Case $\Phi_t$ ends with ($\ruleApp$). Then $t$ must be of the form $wu$ and $\Phi_t$ must be of following form:
            \[ \begin{prooftree}
                \hypo{\Phi_w \tr \seqi{\Gam; x : \M_1}{w}{\M' \ta (\comptype{\stype'}{\ctype})}{(b_w,m_w,d_w)}}
                \hypo{\Phi_u \tr \seqi{\Del; x : \M_2}{u}{\tcomptype{\stype}{\M'}{\stype'}}{(b_u,m_u,d_u)}}
                \infer2[(\ruleApp)]{\seqi{\Gam + \Del; x : \M_1 \sqcup \M_2}{wu}{\comptype{\stype}{\ctype}}{(1+b_w+b_u,m_w+m_u,d_w+d_u)}}
            \end{prooftree} \]
            such that $\Gam_t = \Gam + \Del$, $\M = \M_1 \sqcup \M_2$, $\gtype = \comptype{\stype}{\ctype}$, $b_t = 1+b_w+b_u$, $m_t = m_w+m_u$, and $d_t = d_w + d_u$. By~\cref{lem:split-values-stores}.\ref{lem:com-split-values}, we know there exist the following derivations $(\Phi^i_v \tr \seqi{\Gam^i_v}{v}{\M_i}{(b_i,m_i,d_i)})_{i \in \{1,2\}}$, such that $\Gam_v = \Gam^1_v + \Gam^2_v$, $b_v = b_1 + b_2$, $m_v = m_1 + m_2$, and $d_v = d_1 + d_2$. By the \ih, we know there exist $\Phi_{w \subs{x}{v}} \tr \seqi{\Gam + \Gam^1_v}{w \subs{x}{v}}{\M' \ta (\comptype{\stype'}{\ctype})}{(b_w+b_1,m_w+m_1,d_w+d_1)}$ and $\Phi_{u \subs{x}{v}} \tr \seqi{\Del + \Gam^2_v}{u \subs{x}{v}}{\tcomptype{\stype}{\M'}{\stype'}}{(b_u+b_2,m_u+m_2,d_u+d_2)}$. %Assume $\Phi_{w \subs{x}{v}} \tr \seqi{\Gam + \Gam^1_v}{w \subs{x}{v}}{\tcomptype{\M'}{\stype'}{\ctype}}{(b_w+b_1,m_w+m_1,d_w+d_1)}$ and $\Phi_{u \subs{x}{v}} \tr \seqi{\Del + \Gam^2_v}{u \subs{x}{v}}{\stype \ta (\comptype{\M'}{\stype'})}{(b_u+b_2,m_u+m_2,d_u+d_2)}$. 
            We can build $\Phi_{t \subs{x}{v}}$ as follows:
            \[ \begin{prooftree}
                \hypo{\Phi_{w \subs{x}{v}}}
                \hypo{\Phi_{u \subs{x}{v}}}
                \infer2[(\ruleApp)]{\seqi{(\Gam + \Del) + (\Gam^1_v + \Gam^2_v)}{(wu) \subs{x}{v}}{\comptype{\stype}{\ctype}}{(1+b_w+b_u+b_1+b_2,m_w+m_u+m_1+m_2,d_w+d_u+d_1+d_2)}}
            \end{prooftree} \]
            And we can conclude with $\Gam_t + \Gam_v = (\Gam + \Del) + (\Gam^1_v + \Gam^2_v)$, $b_t + b_v = 1 + b_w+b_u+b_1+b_2$, $m_t + m_v = m_w+m_u+m_1+m_2$, and $d_t + d_v = d_w+d_u+d_1+d_2$.
            \item Case $\Phi_w$ ends with ($\ruleMany$). Then $t$ must be of the form $w$ and $\Phi_t$ must be of the following form:
            \[ \begin{prooftree}
                \hypo{(\Phi^i_w \tr \seqi{\Gam_i; x : \M_i}{w}{\rdel_i}{(b_i,m_i,d_i)})_{\iI}}
                \infer1[(\ruleMany)]{\seqi{+_{\iI} \Gam_i; x : \sqcup_{\iI} \M_i}{w}{\mul{\rdel_i}_{\iI}}{(+_{\iI}b_i, +_{\iI}m_i, +_{\iI}d_i)}}
            \end{prooftree} \]
            such that $\Gam_t = +_{\iI} \Gam_i$, $\gtype = \mul{\rdel_i}_{\iI}$, $b_t = +_{\iI} b_i$, $m_t = +_{\iI} m_i$, and $d_t = +_{\iI} d_i$. By~\cref{lem:split-values-stores}.\ref{lem:com-split-values}, $(\Phi^i_v \tr \seqi{\Gam^i_v}{v}{\M_i}{(b^i_v,m^i_v,d^i_v)})_{\iI}$, such that $\Gam_v = +_{\iI} \Gam^i_v$, $b_v = +_{\iI} b^i_v$, $m_v = +_{\iI} m^i_v$, and $d_v = +_{\iI} d^i_v$. By the \ih over each $\Phi^i_v$, we have $(\Phi^i_{w \subs{x}{v}} \tr \seqi{\Gam_i + \Gam^i_v}{w \subs{x}{v}}{\rdel_i}{(b_i+b^i_v,m_i+m^i_v,d_i+d^i_v)})_{\iI}$. Therefore, we can build $\Phi_{t \subs{x}{v}}$ as follows:
            \[ \begin{prooftree}
                \hypo{(\Phi^i_{w \subs{x}{v}} \tr \seqi{\Gam_i + \Gam^i_v}{w \subs{x}{v}}{\rdel_i}{(b_i+b^i_v,m_i+m^i_v,d_i+d^i_v)})_{\iI}}
                \infer1[(\ruleMany)]{\seqi{+_{\iI} (\Gam^i_v + \Gam^i_w)}{w \subs{x}{v}}{\mul{\tau_i}_{\iI}}{(+_{\iI}(b_i+b^i_v),+_{\iI}(m_i+m^i_v),+_{\iI}(d_i+d^i_v))}}
            \end{prooftree} \]
            And we can conclude with $\Gam_t + \Gam_v = +_{\iI} \Gam_i +_{\iI} \Gam^i_v = +_{\iI} (\Gam_i + \Gam^i_v)$, $b_t + b_v = +_{\iI} b_i +_{\iI} b^i_v = +_{\iI} (b_i + b^i_v)$, $m_t + m_v = +_{\iI} m_i +_{\iI} m^i_v = +_{\iI} (m_i + m^i_v)$, and $d_t + d_v = +_{\iI} d_i +_{\iI} d^i_v = +_{\iI} (d_i + d^i_v)$.
            \item Case $\Phi_t$ ends with (\ruleLift). Then $t$ is a variable and $\Phi_t$ must be of the following form:
            \[ \begin{prooftree}
                \hypo{\Phi_w \tr \seqi{\Gam; x : \M}{w}{\M'}{(b_t,m_t,d_t)}}
                \infer1[(\ruleLift)]{\seqi{\Gam; x : \M}{w}{\tcomptype{\stype}{\M'}{\stype}}{(b_t,m_t,d_t)}}
            \end{prooftree} \]
            where $\gtype = \tcomptype{\stype}{\M'}{\stype}$. By the \ih, we have $\Phi_{w \subs{x}{v}} \tr \seqi{\Gam + \Gam_v}{w \subs{x}{v}}{\M'}{(b_t+b_v, m_t+m_v, d_t +d_v)}$. Therefore, we can build $\Phi_{t \subs{x}{v}}$ as follows:
            \[ \begin{prooftree}
                \hypo{\Phi_{w \subs{x}{v}} \tr \seqi{\Gam + \Gam_v}{w \subs{x}{v}}{\M'}{(b_t+b_v, m_t+m_v, d_t +d_v)}}
                \infer1[(\ruleLift)]{\seqi{\Gam + \Gam_v}{w \subs{x}{v}}{\tcomptype{\stype}{\M'}{\stype}}{(b_t+b_v, m_t+m_v, d_t +d_v)}}
            \end{prooftree} \]
            And we can conclude.
            \item Case $\Phi_t$ ends with ($\ruleGet$). Then $t$ must be of the form $\get{l}{y}{u}$ and $\Phi_t$ must be of the following form:
            \[ \begin{prooftree}
                \hypo{\Phi_u \tr \seqi{\Gam_u; x : \M}{u}{\comptype{\stype}{\ctype}}{(b_u,m_u,d_u)}}
                \infer1[(\ruleGet)]{\seqi{(\Gam_u \sm y); x : \M}{\get{l}{y}{u}}{\comptype{\conj{(l : \Gam_{u}(y))} \splus \stype}{\ctype}}{(b_u,1+m_u,d_u)}}
            \end{prooftree} \]
          where $\gtype = \comptype{\conj{(l : \Gam_{u}(y))} \splus \stype}{\ctype}$, $\Gam_t = \Gam_u \sm y$, $b_t = b_u$, $m_t = 1+m_u$, and $d_t = d_u$. By the \ih, we have $\Phi_{u \subs{x}{v}} \tr \seqi{\Gam_u + \Gam_v}{u \subs{x}{v}}{\stype \ra \ctype}{(b_u+b_v,m_u+m_v,d_u+d_v)}$. Therefore, we can build $\Phi_{t \subs{x}{v}}$ as follows:
            \[ \begin{prooftree}
                \hypo{\Phi_{u \subs{x}{v}} \tr \seqi{\Gam_u + \Gam_v}{u \subs{x}{v}}{\comptype{\stype}{\ctype}}{(b_u+b_v,m_u+m_v,d_u+d_v)}}
                \infer1[(\ruleGet)]{\seqi{(\Gam_u  + \Gam_v) \sm y}{\get{l}{y}{u} \subs{x}{v}}{\comptype{\conj{(l : \Gam_u(y))} \splus \stype}{\ctype}}{(b_u+b_v,1+m_u+m_v,d_u+d_v)}}
            \end{prooftree} \]
            And we can conclude with $(\Gam_u + \Gam_v) \sm y = (\Gam \sm y) + \Gam_v = \Gam_t + \Gam_v$ by $\alpha$-conversion, $b_t + b_v = b_u+b_v$, $m_t + m_v = 1+m_u+m_v$, and $d_t + d_v = d_u +d_v$.
            \item Case $\Phi_t$ ends with ($\ruleSet$). Then $t$ must be of the form $\set{l}{w}{u}$ and $\Phi_t$ must be of the following form:
            \[ \begin{prooftree}
                \hypo{\Phi_w \tr \seqi{\Gam_w; x : \M_1}{w}{\M'}{(b_w,m_w,d_w)}}
                \hypo{\Phi_u \tr \seqi{\Gam_u; x : \M_2}{u}{\comptype{\conj{(l : \M')}; \stype}{\ctype}}{(b_u,m_u,d_u)}}
                \infer2[(\ruleSet)]{\seqi{\Gam_w + \Gam_u; x : \M_1 \sqcup \M_2}{\set{l}{w}{u}}{\comptype{\stype}{\ctype}}{(b_w+b_u,1+m_w+m_u,d_w+d_u)}}
            \end{prooftree} \]
            where $\gtype = \comptype{\stype}{\ctype}$, $\Gam_t = \Gam_w + \Gam_u$, $\del = \comptype{\stype}{\ctype}$, $b_t = b_w+b_u$, $m_t = 1+m_w + m_u$, and $d_t = d_w + d_u$. By~\cref{lem:split-values-stores}.\ref{lem:com-split-values}, we have $\Phi^1_v \tr \seqi{\Gam^1_v}{v}{\M_1}{(b^1_v,m^1_v,d^1_v)}$ and $\Phi^2_v \tr \seqi{\Gam^2_v}{v}{\M_2}{(b^2_v,m^2_v,d^2_v)}$, such that $\Gam_v = \Gam^1_v + \Gam^2_v$, $b_v = b^1_v + b^2_v$, $m_v = m^1_v + m^2_v$, and $d_v = d^1_v + d^2_v$. By the \ih, we have $\Phi_{w \subs{x}{v}} \tr \seqi{\Gam_w + \Gam^1_v}{w \subs{x}{v}}{\M'}{(b_w+b^1_v,m_w+m^1_v,d_w+d^1_v)}$ and $\Phi_{u \subs{x}{v}} \tr \seqi{\Gam_u + \Gam^2_v}{u \subs{x}{v}}{\comptype{\conj{(l : \M')}; \stype}{\ctype}}{(b_u+b^2_v,m_u+m^2_u,d_u+d^2_v)}$. Assume $\Phi_{w \subs{x}{v}} \tr \seqi{\Gam_w + \Gam^1_v}{w \subs{x}{v}}{\M'}{(b_w+b^1_v,m_w+m^1_v,d_w+d^1_v)}$ and $\Phi_{u \subs{x}{v}} \tr \seqi{\Gam_u + \Gam^2_v}{u \subs{x}{v}}{\comptype{\conj{(l : \M')}; \stype}{\ctype}}{(b_u+b^2_v,m_u+m^2_u,d_u+d^2_v)}$. We can build $\Phi_{t \subs{x}{v}}$ as follows:
            \[ \begin{prooftree}
                \hypo{\Phi_{w \subs{x}{v}}}
                \hypo{\Phi_{u \subs{x}{v}}}
                \infer2[(\ruleSet)]{\seqi{(\Gam_w + \Gam_u) + (\Gam^1_v + \Gam^2_v)}{(wu) \subs{x}{v}}{\comptype{\stype}{\ctype}}{(b_w+b_u+b^1_v+b^2_v,1+m_w+m_u+m^1_v+m^2_v,d_w+d_u+d^1_v+d^2_v)}}
            \end{prooftree} \]
            And we can conclude with $\Gam_t + \Gam_v = (\Gam_w + \Gam_u) + (\Gam^1_v + \Gam^2_v)$, $b_t + b_v = b_w+b_u+b^1_v+b^2_v$, $m_t + m_v = 1+m_w+m_u+m^1_v+m^2_v$, $d_t + d_v = d_w+d_u+d^1_v+d^2_v$.
            \item Case $\Phi_t$ ends with (\ruleAxP). Then $t$ must be a variable and we must consider two cases:
            \begin{itemize}
                \item Assume $t = y = x$. Then $\Gam_t = \eset$, $\gtype = \stype \ta (\comptype{\nott{\tneutral}}{\stype})$, $t \subs{x}{v} = v$, $b_t = m_t = d_t = 0$. Moreover, $\M = \mul{\nott{\tneutral}}$. We have to consider two cases:
                \begin{itemize}
                    \item Case $v = z$. Then $\Phi_v \tr \seqi{z : \mul{\nott{\tneutral}}}{z}{\mul{\nott{\tneutral}}}{(0,0,0)}$. So we can take $\Phi_{t \subs{x}{v}}$ as the following derivation:
                    \[ \begin{prooftree}
                        \infer0[(\ruleAxP)]{\seqi{z : \mul{\nott{\tneutral}}}{z}{\tcomptype{\stype}{\nott{\tneutral}}{\stype}}{(0,0)}}
                    \end{prooftree} \]
                    and conclude with $\Gam_t + \Gam_v = \Gam_v = (z : \mul{\nott{\tneutral}})$, $b_t + b_v = b_v = 0$, $m_t + m_v = m_v = 0$, and $d_t + d_v = d_v$.
                    \item Case $v = \lam z.p$. This case does not apply, by~\cref{lem:comp-notabs-implies-negabs}.
                \end{itemize}
                \item Assume $t = y \neq x$. Then $\M = \emul$, $\Gam_v = \eset$, $t \subs{x}{v} = t$, $b_v = 0$, $m_v = 0$, and $d_v = 0$. So we can take $\Phi_{t \subs{x}{v}} = \Phi_t$ and conclude with $\Gam_t + \Gam_v = \Gam_t$, $b_t + b_v = b_t$, $m_t + m_v = m_t$, and $d_t + d_v = d_t$.
            \end{itemize}
            \item Case $\Phi_t$ ends with (\ruleLamP). Then $t$ is of the form $\lam y.u$, $\Gam_t = \eset$, $\gtype = \tcomptype{\stype}{\vl}{\stype}$, $\M = \emul$, $\Gam_v = \eset$, $t \subs{x}{v} = \lam y.(u \subs{x}{v}) = (\lam y.u) \subs{x}{v}$, $b_t = b_v = 0$, $m_t = m_v = 0$, and $d_t = d_v = 0$. So we can build $\Phi_{t \subs{x}{v}}$ as follows:
            \[ \begin{prooftree}
                \infer0[(\ruleLamP)]{\seqi{}{(\lam y.u) \subs{x}{v}}{\tcomptype{\stype}{\vl}{\stype}}{(0,0,0)}}
            \end{prooftree} \]
            And conclude with $\Gam_t + \Gam_v = \eset$, $b_t = b_v = 0$, $m_t = m_v = 0$, and $d_t = d_v = 0$.
            \item Case $\Phi_t$ ends with (\ruleAppPOne). Then $t$ is of the form $yu$ and we have to consider two cases:
            \begin{itemize}
                \item Case $y = x$. Then $\Phi_t$ must be of the following form:
                \[ \begin{prooftree}
                    \hypo{\seqi{\Gam_u}{u}{\tcomptype{\stype}{\tightt}{\stype}}{(b_u,m_u,d_u)}}
                    \infer1[(\ruleAppPOne)]{\seqi{(x : \mul{\tvar} \sqcup \Gam_u(x)); (\Gam_u \sm x)}{x u}{\tcomptype{\stype}{\tneutral}{\stype}}{(b_u,m_u,1+d_u)}}
                \end{prooftree} \]
                such that $\Gam_t = (\Gam_u \sm x)$, $b = b_u$, $m = m_u$, and $d = 1+d_u$. Then $\M = \mul{\tvar} \sqcup \Gam_u(x)$ and, by~\cref{lem:split-values-stores}.\ref{lem:com-split-values}, we have $\Phi^1_v \tr \seqi{\Gam^1_v}{v}{\mul{\tvar}}{(b^1_v,m^1_v,d^1_v)}$ and $\Phi^2_v \tr \seqi{\Gam^2_v}{v}{\Gam_u(x)}{(b^2_v,m^2_v,d^2_v)}$, such that $\Gam_v = \Gam^1_v + \Gam^2_v$, $b_v = b^1_v + b^2_v$, $m_v = m^1_v + m^2_v$, and $d_v = d^1_v + d^2_v$. By the \ih, we know there exists $\Phi_{u \subs{x}{v}} \tr \seqi{(\Gam_u \sm x) + \Gam^2_u}{u \subs{x}{v}}{\tcomptype{\stype}{\tightt}{\stype}}{(b_u+b^2_v, m_u+m^2_v, d_u+d^2_v)}$. Now, we need to consider two cases:
                \begin{itemize}
                    \item Case $v = z$. Then $\Phi^1_v \tr \seqi{z : \mul{\tvar}}{z}{\mul{\tvar}}{(0,0,0)}$ and $\Phi^2_v \tr \seqi{z : \Gam_u(x)}{z}{\Gam_u(x)}{(0,0)}$. Therefore, we can build $\Phi_{t \subs{x}{v}} = \Phi_{v \subs{x}{v}}$ as follows:
                    \[ \begin{prooftree}
                        \hypo{\Phi_{u \subs{x}{v}} \tr \seqi{(\Gam_u \sm x) + \Gam_u(x)}{u \subs{x}{v}}{\tcomptype{\stype}{\tightt}{\stype}}{(b_u+b^2_v, m_u+m^2_v, d_u+d^2_v)}}
                        \infer1[(\ruleAppPOne)]{\seqi{(z : \mul{\tvar}) + (\Gam_u \sm x + (z : \Gam_u(x)))}{z (u \subs{x}{v})}{\tcomptype{\stype}{\tneutral}{\stype}}{(b_u+b^2_v,m_u+m^2_v,1+d_u+d^2_v)}}
                    \end{prooftree} \]
                    where $(z : \mul{\tvar}) + (\Gam_u \sm x + (z : \Gam_u(x))) = (\Gam_u \sm x) + (z : \mul{\tvar} \cup \Gam_u(x)) = \Gam_u + \Gam_v$, $b_u + b^2_v = b + b^1_v + b^2_v = b + b_v$, $m_u + m^2_v = m + m^1_v + m^2_v = m + m_v$, and $d_u + d^2_v = d + d^1_v + d^2_v = d + d_v$.
                    \item Case $v = \lam z.p$. This case does not apply, since it is not possible to assign $\tvar$ to $\lam z.p$, by~\cref{lem:comp-notabs-implies-negabs}.
                \end{itemize}
                \item Case $y \neq x$. Then, the proof is very similar to when $\Phi_t$ ends with rule ($\ruleApp$).
            \end{itemize}
            \item Case $\Phi_t$ ends with (\ruleAppPTwo), the proof is very similar to when $\Phi_t$ ends with rule (\ruleAppPOne).
        \end{itemize}
%\end{proof}

        \item %\begin{proof}
    We are going to generalize the original statement by replacing $\del$ with $\gtype$.
    \\ \\
    The proof follows by induction over $t$:
    \begin{itemize}
        \item Case $t = y$. Then we have to consider two cases:
        \begin{itemize}
            \item Let $t = y \not= x$. Then $t \subs{x}{v} = y$. Let $\Gam_v = \eset$, $\M = \emul$, $b_v = m_v = d_v = 0$. Then, $\Phi_v$ is derivable using rule ($\ruleMany$) with no premise. We also take $\Phi_t = \Phi_{t \subs{x}{v}}$, so that, in particular $\Gam_t = \Gam_{t \subs{x}{v}}$. Then, we can conclude with $\Gam_{t \subs{x}{v}} = \Gam_t + \Gam_v = \Gam_t$, $b = b_t + b_v = b_t$, $m = m_t + m_v = m_t$, and $d = d_t + d_v = d_t$.
            \item Let $t = y = x$. Then $t \subs{x}{v} = v$. Let $\Gam_t = \eset$, and $b_t = m_t = s_t = 0$. Now we will consider two cases depending on the form of $v$:
            \begin{itemize}
                \item Case $v = z$. Then $t \subs{x}{v} = z$ and we can proceed by case analysis of the last rule in $\Phi_{t\subs{x}{v}}$. In all of them, we can build $\Phi_t$ from $\Phi_{t \subs{x}{v}}$, by simply replacing $x$ with $z$, and $\Phi_v$ as follows:
                \[ \begin{prooftree}
                    \infer0[(\ruleAx)]{\seqi{z : \mul{\sig}}{z}{\sig}{(0,0,0)}}
                    \infer1[(\ruleMany)]{\seqi{z : \mul{\sig}}{z}{\mul{\sig}}{(0,0,0)}}
                \end{prooftree} \]
                And we can conclude since all the counters are zero.
                \item Case $v = \lam z.p$. Then $t \subs{x}{v} = \lam z.p$ and we can proceed by case analysis of the last rule in $\Phi_{t \subs{x}{v}}$. In all of them, we can always build $\Phi_t$ using either (\ruleAx) (case (\ruleApp)), (\ruleAxP) (case (\ruleLamP)),  (\ruleAx) plus (\ruleMany) (case (\ruleMany)), or (\ruleAx) plus (\ruleMany) plus (\ruleLift) (case (\ruleLift)). $\Phi_v$  is either $\Phi_{t \subs{x}{v}}$ (case (\ruleMany)), or it can be built from $\Phi_{t \subs{x}{v}}$ plus rule (\ruleMany) (all other cases).
            \end{itemize}
        \end{itemize}
        \item Case $t = \lam y.u$. Then $t \subs{x}{v} = (\lam y.u)\subs{x}{v} = \lam y.(u \subs{x}{v})$ and we must consider three cases:
        \begin{itemize}
            \item Case $\Phi_{t \subs{x}{v}}$ ends with rule (\ruleLam), then it must be of the following form: 
            \[ \begin{prooftree}
                \hypo{\Phi_{u \subs{x}{v}} \tr \seqi{\Gam_{u \subs{x}{v}}; y : \M'}{u \subs{x}{v}}{\comptype{\stype}{\ctype}}{(b,m,d)}}
                \infer1[(\ruleLam)]{\seqi{\Gam_{u \subs{x}{v}}}{\lam y.(u \subs{x}{v})}{\M' \ta (\comptype{\stype}{\ctype})}{(b,m,d)}}
            \end{prooftree} \]
            where $\gtype = \M' \ta (\comptype{\stype}{\ctype})$ and $\Gam_{t \subs{x}{v}} = \Gam_{u \subs{x}{v}}$. By the \ih, we have $\Phi_u \tr \seqi{\Gam_u; y : \M'; x : \M}{u}{\comptype{\stype}{\ctype}}{(b_u,m_u,d_u)}$ and $\Phi_v \tr \seqi{\Gam_v}{v}{\M}{(b_v,m_v,d_v)}$, such that $\Gam_{u \subs{x}{v}} = \Gam_u + \Gam_v$, $b = b_u + b_v$, $m = m_u + m_v$, and $d = d_u + d_v$. So we can build $\Phi_{\lam y.u}$ as follows:
            \[ \begin{prooftree}
                \hypo{\Phi_u \tr \seqi{\Gam_u; y: \M'; x : \M}{u}{\comptype{\stype}{\ctype}}{(b_u,m_u,d_u)}}
                \infer1[(\ruleLam)]{\seqi{\Gam_u; x : \M}{\lam y.u}{\M' \ta (\comptype{\stype}{\ctype})}{(b_u,m_u,d_u)}}
            \end{prooftree} \]
            And we can pick $\Phi_t = \Phi_{\lam y.u}$, and conclude with $\Gam_{t \subs{x}{v}} = \Gam_{u \subs{x}{v}} = \Gam_u + \Gam_v$, $b = b_u + b_v$, $m = m_u + m_v$, and $d = d_u + d_v$.
            \item Case $\Phi_{t \subs{x}{v}}$ ends with rule (\ruleLamP). Then it must be of the following form:
            \[ \begin{prooftree}
                \infer0[(\ruleLamP)]{\seqi{}{\lam y.(u \subs{x}{y})}{\tcomptype{\stype}{\vl}{\stype}}{(0,0,0)}}
            \end{prooftree} \]
            where $\Gam_{t \subs{x}{v}} = \eset$, $\gtype = \tcomptype{\stype}{\vl}{\stype}$, and $b = m = d = 0$. Let $\Gam_t = \eset$, $\M = \emul$, and $b_t = m_t = d_t = 0$. Then, we can construct $\Phi_t$ as follows:
            \[ \begin{prooftree}
                \infer0[(\ruleLamP)]{\seqi{}{\lam y.u}{\tcomptype{\stype}{\vl}{\stype}}{(0,0,0)}}
            \end{prooftree} \]
            Let $\Gam_v = \eset$, and $b_v = m_v = d_v = 0$. Then $\Phi_v$ can be constructed by using rule ($\ruleMany$) with no premises. So we can conclude with $\Gam_{t \subs{x}{v}} = \eset = \Gam_t + \Gam_v$, and $b = 0 = b_t + b_v$, $m = 0 = m_t + m_v$, and $d = 0 = d_t + d_v$.
            \item Case $\Phi_{t \subs{x}{v}}$ ends with rule ($\ruleMany$). Then $t \subs{x}{v}$ is a value, and $\Phi_{t \subs{x}{v}}$ must be of the following form:
            \[ \begin{prooftree}
                \hypo{(\Phi_i \tr \seqi{\Gam_i}{t \subs{x}{v}}{\rdel_i}{(b_i,m_i,d_i)})_{\iI}}
                \infer1[(\ruleMany)]{\seqi{+_{\iI} \Gam_i}{t \subs{x}{v}}{\mul{\rdel_i}_{\iI}}{(+_{\iI} b_i, +_{\iI} m_i, +_{\iI} d_i)}}
            \end{prooftree} \]
            where $\gtype = \mul{\rdel_i}_{\iI}$, $\Gam_{t \subs{x}{v}} = +_{\iI} \Gam_i$, $b = +_{\iI} b_i$, $m = +_{\iI} m_i$, and $d = +_{\iI} d_i$. By the \ih over each $\Phi_i$, we have the following derivations $\Phi^i_t \tr \seqi{\Gam^i_t; x : \M_i}{t}{\rdel_i}{(b^i_t,m^i_t,d^i_t)}$ and $\Phi^i_v \tr \seqi{\Gam^i_v}{v}{\M_i}{(b^i_v, m^i_v, d^i_v)}$, such that $\Gam_i = \Gam^i_t + \Gam^i_v$, $b = b^i_t + b^i_v$, $m = m^i_t + m^i_v$, and $d = d^i_t + d^i_v$, for each $\iI$. So we can construct $\Phi_t$ as follows:
            \[ \begin{prooftree}
                \hypo{(\Phi^i_t \tr \seqi{\Gam^i_t; x : \M_i}{t}{\rdel_i}{(b^i_t,m^i_t,d^i_t)})_{\iI}}
                \infer1[(\ruleMany)]{\seqi{+_{\iI} \Gam^i_t; x : \sqcup_{\iI} \M_i}{t}{\mul{\rdel_i}_{\iI}}{(+_{\iI} b^i_t, +_{\iI} m^i_t, +_{\iI} d^i_t)}}
            \end{prooftree} \]
            such that $\Gam_t = +_{\iI} \Gam^i_t$, $\M = \sqcup_{\iI} \M_i$, $b_t = +_{\iI} b^i_t$, $m_t = +_{\iI} m^i_t$, and $d_t = +_{\iI} d^i_t$. By~\cref{lem:comp-merge-values}, we can take the following derivation $\Phi_v \tr \seqi{+_{\iI} \Gam^i_v}{v}{\M}{(+_{\iI} b^i_v, +_{\iI} m^i_v, +_{\iI} d^i_v)}$. And we can conclude with $\Gam_{t \subs{x}{v}} = +_{\iI} \Gam_i = +_{\iI} (\Gam^i_t + \Gam^i_v) = +_{\iI} \Gam^i_t +_{\iI} \Gam^i_v = \Gam_t + \Gam_v$, $b = +_{\iI} b_i = +_{\iI} (b^i_t + b^i_v) = +_{\iI} b^i_t +_{\iI} b^i_v = b_t + b_v$, $m = +_{\iI} m_i = +_{\iI} (m^i_t + m^i_v) = +_{\iI} m^i_t +_{\iI} m^i_v = m_t + m_v$, and $d = +_{\iI} d_i = +_{\iI} (d^i_t + d^i_v) = +_{\iI} d^i_t +_{\iI} d^i_v = d_t + d_v$.
        \end{itemize}
        \item Let $t = wu$. Then $t \subs{x}{v} = (wu) \subs{x}{v} = (w \subs{x}{v})(u \subs{x}{v})$, and we have to consider three cases:
        \begin{itemize}
            \item Case $\Phi_{t \subs{x}{v}}$ ends with ($\ruleApp$). Assume $\Phi_{w \subs{x}{v}} \tr \seqi{\Gam_{w \subs{x}{v}}}{w \subs{x}{v}}{\M' \ta (\comptype{\stype'}{\ctype})}{(b',m',d')}$ and $\Phi_{u \subs{x}{v}} \tr \seqi{\Gam_{u \subs{x}{v}}}{u \subs{x}{v}}{\tcomptype{\stype}{\M'}{\stype'}}{(b'',m'',d'')}$. $\Phi_{t \subs{x}{v}}$ must be of the following form:
            \[ \begin{prooftree}
                \hypo{\Phi_{w \subs{x}{v}}}
                \hypo{\Phi_{u \subs{x}{v}}}
                \infer2[(\ruleApp)]{\seqi{\Gam_{w \subs{x}{v}} + \Gam_{u \subs{x}{v}}}{(w \subs{x}{v})(u \subs{x}{v})}{\comptype{\stype}{\ctype}}{(1+b'+b'',m'+m'',d'+d'')}}
            \end{prooftree} \]
            where $\gtype = \comptype{\stype}{\ctype}$, $\Gam_{t \subs{x}{v}} = \Gam_{w \subs{x}{v}} + \Gam_{u \subs{x}{v}}$, $b = 1+b'+b''$, $m = m'+m''$, and $d = d'+d''$. By the \ih over $\Phi_{w \subs{x}{v}}$, we have $\Phi_w \tr \seqi{\Gam_w; x : \M_1}{w}{\M' \ta (\comptype{\stype'}{\ctype})}{(b_w,m_w,d_w)}$ and $\Phi^1_v \tr \seqi{\Gam^1_v}{v}{\M_1}{(b^1_v,m^1_v,d^1_v)}$, such that $\Gam_{w \subs{x}{v}} = \Gam_w + \Gam^1_v$, $b' = b_w + b^1_v$, $m' = m_w + m^1_v$, and $d' = d_w + d^1_v$. And by the \ih over $\Phi_{u \subs{x}{v}}$, we have $\Phi_u \tr \seqi{\Gam_u; x : \M_2}{u}{\tcomptype{\stype}{\M'}{\stype'}}{(b_u, m_u,d_u)}$ and $\Phi^2_v \tr \seqi{\Gam^2_v}{v}{\M_2}{(b^2_v,m^2_v,d^2_v)}$, such that $\Gam_{u \subs{x}{v}} = \Gam_u + \Gam^2_v$, $b'' = b_u + b^2_v$, $m'' = m_u + m^2_v$, and $d'' = d_u + d^2_v$. By~\cref{lem:comp-merge-values}, we can take $\Phi_v \tr \seqi{\Gam^1_v + \Gam^2_v}{v}{\M_1 \sqcup \M_2}{(b^1_v+b^2_v, m^1_v+m^2_v, d^1_v+d^2_v)}$, such that $\Gam_v = \Gam^1_v + \Gam^2_v$, $b_v = b^1_v + b^2_v$, $m_v = m^1_v + m^2_v$, and $d_v = d^1_v + d^2_v$. And we can build $\Phi_{wu}$ as follows:
            \[ \begin{prooftree}
                \hypo{\Phi_w}
                \hypo{\Phi_u}
                \infer2[(\ruleApp)]{\seqi{(\Gam_w + \Gam_u); x : \M_1 \sqcup \M_2}{wu}{\comptype{\stype}{\kappa}}{(1+b_w+b_u,m_w+m_u,d_w+d_u)}}
            \end{prooftree} \]
            such that $\Gam_t = \Gam_w + \Gam_u$, $b_t = 1 + b_w + b_u$, $m_t = b_w + b_u$, and $d_t = d_w + d_u$. So we can pick $\Phi_t = \Phi_{wu}$, and conclude with $\Gam_{t \subs{x}{v}} = \Gam_{w \subs{x}{v}} + \Gam_{u \subs{x}{v}} = (\Gam_w + \Gam^1_v) + (\Gam_u + \Gam^2_v) = (\Gam_w + \Gam_u) + (\Gam^1_v + \Gam^2_v) = \Gam_t + \Gam_v$, $b = 1 + b' + b'' = 1 + b_w + b^1_v + b_u + b^2_v = (1 + b_w + b_u) + (b^1_v + b^2_v) = b_t + b_v$, $m = m' + m'' = m_w + m^1_v + m_u + m^2_v = (m_w + m_u) + (m^1_v + m^2_v) = m_t + m_v$, and $d = d' + d'' = d_w + d^1_v + d_u + d^2_v = (d_w + d_u) + (d^1_v + d^2_v) = d_t + d_v$.
            \item Case $\Phi_{t \subs{x}{v}}$ ends with (\ruleAppPOne) or (\ruleAppPTwo). These cases are very similar to the case where $\Phi_{t \subs{x}{v}}$ ends with (\ruleApp).
        \end{itemize}
        \item Let $t = \get{l}{y}{u}$. Then $t \subs{x}{v} = \get{l}{y}{u \subs{x}{v}}$ and $\Phi_{t \subs{x}{v}}$ must be of the following form:
        \[ \begin{prooftree}
            \hypo{\Phi_{u \subs{x}{v}} \tr \seqi{\Gam_{u \subs{x}{v}}; y : \M'}{u \subs{x}{v}}{\comptype{\stype}{\ctype}}{(b,m',d)}}
            \infer1[(\ruleGet)]{\seqi{\Gam_{u \subs{x}{v}}}{\get{l}{y}{u \subs{x}{v}}}{\comptype{\conj{(l : \M')} \splus \stype}{\ctype}}{(b,1+m',d)}}
        \end{prooftree} \]
        where $\Gam_{t \subs{x}{v}} = \Gam_{u \subs{x}{v}}$ and $m = 1+m'$. By the \ih, we have $\Phi_u \tr \seqi{\Gam_u; y : \M'; x : \M}{u}{\comptype{\stype}{\ctype}}{(b_u,m_u,d_u)}$ and $\Phi_v \tr \seqi{\Gam_v}{v}{\M}{(b_v,m_v,d_v)}$, such that $\Gam_{u \subs{x}{v}} = \Gam_u + \Gam_v$, $b = b_u + b_v$, $m' = m_u + m_v$, and $d = d_u + d_v$. So we can build $\Phi_{\get{l}{y}{u}}$ as follows:
        \[ \begin{prooftree}
            \hypo{\Phi_u \tr \seqi{\Gam_u; y : \M'; x : \M}{u}{\comptype{\stype}{\ctype}}{(b_u,m_u,d_u)}}
            \infer1[(\ruleGet)]{\seqi{\Gam_{u}; x : \M}{\get{l}{y}{u}}{\comptype{\conj{(l : \M')} \splus \stype}{\ctype}}{(b_u,1+m_u,d_u)}}
        \end{prooftree} \]
        And we can pick $\Phi_t = \Phi_{\get{l}{y}{u}}$, and conclude with $\Gam_{t \subs{x}{v}} = \Gam_{u \subs{x}{v}} = \Gam_u + \Gam_v$, $b = b_u + b_v$, $m = 1 + m' = 1 + m_u + m_v = (1 + m_u) + m_v$, and $d = d_u + d_v$.
        \item Let $t = \set{l}{w}{u}$. Then $t \subs{x}{v} = (\set{l}{w}{u}) \subs{x}{v} = \set{l}{w \subs{x}{v}}{u \subs{x}{v}}$. Assume $\Phi_{w \subs{x}{v}} \tr \seqi{\Gam_{w \subs{x}{v}}}{w \subs{x}{v}}{\M}{(b',m',d')}$ and $\Phi_{u \subs{x}{v}} \tr \seqi{\Gam_{u \subs{x}{v}}}{u \subs{x}{v}}{\comptype{\conj{(l : \M)}; \stype}{\ctype}}{(b'',m'',d'')}$. $\Phi_{t \subs{x}{v}}$ must be of the following form:
        \[ \begin{prooftree}
            \hypo{\Phi_{w \subs{x}{v}}}
            \hypo{\Phi_{t \subs{x}{v}}}
            \infer2[(\ruleSet)]{\seqi{\Gam_{w \subs{x}{v}} + \Gam_{u \subs{x}{v}}}{\set{l}{w \subs{x}{v}}{u \subs{x}{v}}}{\comptype{\stype}{\ctype}}{(b'+b'',1+m'+m'',d'+d'')}}
        \end{prooftree} \]
        where $\Gam_{t \subs{x}{v}} = \Gam_{w \subs{x}{v}} + \Gam_{u \subs{x}{v}}$, $b = b'+ b''$, $m = 1 + m'+m''$, and $d = d' + d''$. By the \ih over $\Phi_{w \subs{x}{v}}$, we have $\Phi_w \tr \seqi{\Gam_w; x : \M_1}{w}{\M}{(b_w,m_w,d_w)}$ and $\Phi^1_v \tr \seqi{\Gam^1_v}{v}{\M_1}{(b^1_v,m^1_v,d^1_v)}$, such that $\Gam_{w \subs{x}{v}} = \Gam_w + \Gam^1_v$, $b' = b_w + b^1_v$, $m'= m_w + m^1_v$, and $d' = d_w+d^1_v$. And by the \ih over $\Phi_{u \subs{x}{v}}$, we have $\Phi_u \tr \seqi{\Gam_u; x : \M_2}{u}{\comptype{\conj{(l : \M)}; \stype}{\ctype}}{(b_u,m_u,d_u)}$ and $\Phi^2_v \tr \seqi{\Gam^2_v}{v}{\M_2}{(b^2_v,m^2_v,d^2_v)}$, such that $\Gam_{u \subs{x}{v}} = \Gam_u + \Gam^2_v$, $b'' = b_u + b^2_v$, $m'' = m_u + m^2_v$, and $d'' = d_u + d^2_v$. By~\cref{lem:comp-merge-values}, we can take $\Phi_v \tr \seqi{\Gam^1_v + \Gam^2_v}{v}{\M_1 \sqcup \M_2}{(b^1_v + b^2_v, m^1_v+m^2_v, d^1_v + d^2_v)}$, such that $\Gam_v = \Gam^1_v + \Gam^2_v$, $b_v = b^1_v + b^2_v$, $m_v = m^1_v + m^2_v$, and $d_v = d^1_v + d^2_v$. And we can build $\Phi_{\set{l}{w}{u}}$ as follows:
        \[ \begin{prooftree}
            \hypo{\Phi_w \tr \seqi{\Gam_w; x : \M_1}{w}{\M}{(b_w,m_w,d_w)}}
            \hypo{\Phi_u \tr \seqi{\Gam_u; x : \M_2}{u}{\comptype{\conj{(l : \M)}; \stype}{\ctype}}{(b_u,m_u,d_u)}}
            \infer2[(\ruleSet)]{\seqi{(\Gam_w + \Gam_u); x : \M_1 \sqcup \M_2}{\set{l}{w}{u}}{\comptype{\stype}{\ctype}}{(b_w+b_u, 1+m_w+m_u,d_w+d_u)}}
        \end{prooftree} \]
        such that $\Gam_t = \Gam_w + \Gam_u$, $b_t = b_w + b_u$, $m_t = 1 + m_w + m_u$, and $d_t = d_w + d_u$. So we can pick $\Phi_t = \Gam_{\set{l}{w}{u}}$, and conclude with $\Gam_{t \subs{x}{v}} = \Gam_{w \subs{x}{v}} + \Gam_{u \subs{x}{v}} = (\Gam_w + \Gam^1_u) + (\Gam_u + \Gam^2_v) = (\Gam_w + \Gam_u) + (\Gam^1_v + \Gam^2_v) = \Gam_t + \Gam_v$, $b = b' + b'' = (b_w + b^1_v) + (b_u + b^2_v) = (b_w + b_u) + (b^1_v + b^2_v) = b_t + b_v$, $m = 1+ m' + m'' = 1+ (m_w + m^1_v) + (m_u + m^2_v) = (1 + m_w + m_u) + (m^1_v + m^2_v) = m_t + m_v$, and $d = d' + d'' = (d_w + d^1_v) + (d_u + d^2_v) = (d_w + d_u) + (d^1_v + d^2_v) = d_t + d_v$.
    \end{itemize}
%\end{proof}

    \end{enumerate}
\end{proof}}

\lemexactredexp*

\maybehide{\begin{proof} \mbox{}
    \begin{enumerate} 
        \item %\begin{proof}
  We show a stronger statement of the form:

  Let $(t,s) \red[\gname] (u,q)$. If $\Phi \tr \seqi{\Gam}{(t,s)}{\ctype}{(b,m,d)}$, $\Gam$ is tight,  and ($\ctype$ is tight or $\neg \isvalue{t}$), then $\Phi' \tr \seqi{\Gam}{(u,q)}{\ctype}{(b',m',d)}$, where $\gname =\beta$ implies $b' = b - 1$ and $m' = m$, while  $\gname \in \{\getname, \setname\}$ implies $b'=b$ and $m' = m - 1$.

  We proceed by induction on $(t,s) \ra (u,q)$: 
  \begin{itemize}
    \item Case $(t,s) = ((\lam x.p) v,s) \redbeta (p \subs{x}{v}, s) = (u,q)$. Let $\Phi_{(\lam x.p) v}$ be the sub-derivation for $(\lam x.p) v$ in $\Phi$. Assume that $\Phi_{(\lam x.p)v}$ ends with rule (\ruleAppPTwo). Then $v$ must be assigned type $\comptype{\stype}{\tneutral \tim \stype}$, which is not possible by~\cref{lem:comp-values-not-neutral}. Let $\Phi_0$ be the following derivation:
    \[ \begin{prooftree}
      \hypo{\Phi_p \tr \seqi{\Gam_{\lam x.p}; x : \M}{p}{\comptype{\stype}{\ctype}}{(b_p,m_p,d_p)}}
        \infer1[(\ruleLam)]{\seqi{\Gam_{\lam x.p}}{\lam x.p}{\M \ta (\comptype{\stype}{\ctype})}{(b_p,m_p,d_p)}}
        \hypo{\Phi_v \tr \seqi{\Gam_v}{v}{\M}{(b_v,m_v,d_v)}}
        \infer1[(\ruleLift)]{\seqi{\Gam_v}{v}{\tcomptype{\stype}{\M}{\stype}}{(b_v,m_v,d_v)}}
        \infer2[(\ruleApp)]{\seqi{\Gam_{\lam x.p}+\Gam_v}{(\lam x.p)v}{\comptype{\stype}{\ctype}}{(1+b_v+b_p,m_v+m_p,d_v+d_p)}}
    \end{prooftree} \]
    $\Phi_{(\lam x.p) v}$ must end with rule ($\ruleApp$) and $\Phi$ must be of the following form:
    \[ \begin{prooftree}
        \hypo{\Phi_0}
        \hypo{\Phi_s \tr \seqi{\Del}{s}{\stype}{(b_s,m_s,d_s)}}
        \infer2[(\ruleConf)]{\seqi{\Gam_{\lam x.p}+ \Gam_v + \Del}{((\lam x.p)v, s)}{\ctype}{(1+b_v+b_p+b_s,m_v+m_p+m_s,d_v+d_p+d_s)}}
    \end{prooftree} \]
    where  $\Gam = \Gam_{\lam x.p}+ \Gam_v + \Del$,  $b = 1+ b_v + b_p + b_s$, $m = m_v + m_p + m_s$, and $d = d_v + d_p + d_s$. By~\cref{lem:comp-subs-antisubs}.\ref{lem:comp-subs}, there exists $\Phi_{p \subs{x}{v}} \tr \seqi{\Gam_{\lam x.p} +\Gam_v}{p \subs{x}{v}}{\comptype{\stype}{\ctype}}{(b_v+b_p,m_v+m_p,d_v+d_p)}$, therefore we can build $\Phi_{(p\subs{x}{v},s)}$ as follows:
    \[ \begin{prooftree}
        \hypo{\Phi_{p \subs{x}{v}} \tr \seqi{\Gam_{\lam x.p}+\Gam_v}{p \subs{x}{v}}{\comptype{\stype}{\ctype}}{(b_v+b_p,m_v+m_p,d_v+d_p)}}
        \hypo{\Phi_s \tr \seqi{\Del}{s}{\stype}{(b_s,m_s,d_s)}}
        \infer2[(\ruleConf)]{\seqi{\Gam_{\lam x.p} + \Gam_v + \Del}{(p \subs{x}{v}, s)}{\ctype}{(b_v+b_p+b_s,m_v+m_p+m_s,d_v+d_p+d_s)}}
    \end{prooftree} \]
    We can finally conclude since the first counter is equal to $b-1$, while the second and third remain the same.
  \item Case $(t,s) = (vp,s) \ra (vp',q) = (u,q)$, such that $(p,s) \ra (p',q)$. Then we have three cases for the type derivation $\Phi_p$ of $p$ inside $\Phi$: 
    \begin{itemize}
      \item Case $\Phi_p$ ends with ($\ruleApp$). Let $\Phi_0$ be the following derivation:
      \[ \begin{prooftree}
        \hypo{\Phi_v \tr \seqi{\Gam_v}{v}{\M \ta (\comptype{\stype'}{\ctype})}{(b_v,m_v,d_v)}}
            \hypo{\Phi_p \tr \seqi{\Gam_p}{p}{\tcomptype{\stype}{\M}{\stype'}}{(b_p,m_p,d_p)}}
            \infer2[(\ruleApp)]{\seqi{\Gam_v + \Gam_p}{vp}{\comptype{\stype}{\ctype}}{(1+b_v+b_p,m_v+m_p,d_v+d_p)}}
      \end{prooftree} \]
      $\Phi$ must be of the following form:
        \[ \begin{prooftree}
            \hypo{\Phi_0}
            \hypo{\Phi_s \tr \seqi{\Del}{s}{\stype}{(b_s,m_s,d_s)}}
            \infer2[(\ruleConf)]{\seqi{\Gam_v + \Gam_p + \Del}{(vp, s)}{\kappa}{(1+b_v+b_p+b_s,m_v+m_p+m_s,d_v+d_p+d_s)}}
        \end{prooftree} \]
        where $\Gam = \Gam_v + \Gam_p + \Del$ is tight, $b = 1+b_v+b_p+b_s$, $m = m_v+m_p+m_s$, and $s = d_v+d_p+d_s$. Therefore, we can build the following derivation for $(p,s)$:
        \[ \begin{prooftree}
            \hypo{\Phi_p \tr \seqi{\Gam_p}{p}{\tcomptype{\stype}{\M}{\stype'}}{(b_p,m_p,d_p)}}
            \hypo{\Phi_s \tr \seqi{\Del}{s}{\stype}{(b_s,m_s,d_s)}}
            \infer2[(\ruleConf)]{\seqi{\Gam_p + \Del}{(p,s)}{\conftype{\M}{\stype'}}{(b_p+b_s,m_p+m_s,d_p+d_s)}}
        \end{prooftree} \]
        Since $\Gam$ is tight, then $\Gam_p + \Del$ is tight. Moreover, $(p,s) \ra (p',q)$ implies that $\neg \isvalue{p}$. Then we can apply the \ih, and thus there exists a derivation for $(p',q)$ that must be of the following form:
        \[ \begin{prooftree}
            \hypo{\Phi_{p'} \tr \seqi{\Gam_{p'}}{p'}{\tcomptype{\stype''}{\M}{\stype'}}{(b_{p'},m_{p'},d_{p'})}}
            \hypo{\Phi_q \tr \seqi{\Del_q}{q}{\stype''}{(b_q,m_q,d_q)}}
            \infer2[(\ruleConf)]{\seqi{\Gam_{p'} + \Del_q}{(p',q)}{\conftype{\M}{\stype'}}{(b_{p'}+b_q,m_{p'}+m_q,d_{p'}+d_q)}}
        \end{prooftree} \]
        where $\Gam_{p'} + \Del_q = \Gam_p + \Del$ is tight,  and the counters are related properly. Let $\Phi_0$ be the following derivation:
        \[ \begin{prooftree}
          \hypo{\Phi_v \tr \seqi{\Gam_v}{v}{\M \ta \comptype{\stype'}{\ctype'}}{(b_v,m_v,d_v)}}
            \hypo{\Phi_{p'} \tr \seqi{\Gam_{p'}}{p'}{\tcomptype{\stype''}{\M}{\stype'}}{(b_{p'},m_{p'},d_{p'})}}
            \infer2[(\ruleApp)]{\seqi{\Gam_v+\Gam_{p'}}{vp'}{\comptype{\stype''}{\ctype'}}{(1+b_v+b_{p'},m_v+m_{p'},d_v+d_{p'})}}
        \end{prooftree} \]
        We can build $\Phi_{(u,q)}$ as follows:
        \[ \begin{prooftree}
            \hypo{\Phi_0}
            \hypo{\Phi_q \tr \seqi{\Del_q}{q}{\stype''}{(b_q,m_q,d_q)}}
            \infer2[(\ruleConf)]{\seqi{\Gam_v + \Gam_{p'} + \Del_q}{(vp', q)}{\ctype'}{(1+b_v+b_{p'}+b_q,m_v+m_{p'}+m_q,d_v+d_{p'}+d_q)}}
        \end{prooftree} \]
        where $\Gam_{p'} + \Gam_v + \Del_q = \Gam_v + \Gam_p + \Del = \Gam$, $b' = 1+b_v+b_{p'}+b_q$, $m' = m_v+m_{p'}+m_q$, and $d' = d_v+d_{p'}+d_q$. We can conclude since the counters are related properly according to the \ih.  
        \item Case $\Phi_p$ ends with (\ruleAppPOne) or (\ruleAppPTwo). These two cases are very similar to the previous case.
      \end{itemize}
      \item Case $(t,s) = (\get{l}{x}{p},s) \ra (p \subs{x}{v},s) = (u,q)$, where $s \equivstate \upd{l}{v}{s'}$. Let $\Phi_0$ be the following derivation:
      \[ \begin{prooftree}
        \hypo{\Phi_{p} \tr \seqi{\Gam_{p}}{p}{\comptype{\stype}{\ctype}}{(b_p,m_p,d_p)}}
        \infer1[(\ruleGet)]{\seqi{\Gam_{p} \sm x}{\get{l}{x}{p}}{\comptype{\conj{(l : \Gam_{p}(x))} \splus \stype}{\ctype}}{(b_p,1+m_p,d_p)}}
      \end{prooftree} \]
      $\Phi$ must be of the following form:
        \[ \begin{prooftree}
        \hypo{\Phi_0}
          \hypo{\Phi_{s} \tr \seqi{\Del}{s}{\conj{(l : \Gam_{p}(x))} \splus  \stype}{(b_s,m_s,d_s)}}
          \infer2[(\ruleConf)]{\seqi{(\Gam_{p} \sm x) + \Del}{(\get{l}{x}{p}, s)}{\kap}{(b_p+b_s,1+m_p+m_s,d_p+d_s)}}
        \end{prooftree} \] 
        where $\Gam = (\Gam_{p} \sm x) + \Del$ is tight, $b = b_p + b_s$, $m = 1+ m_p + m_s$, and  $d = d_p + d_s$. Since $\Phi_{s} \tr \seqi{\Del}{s}{\conj{(l : \Gam_{p}(x))} \splus \stype}{(b_s,m_s,d_s)}$, then~\cref{lem:split-values-stores}.\ref{lem:split-state} gives $s \equivstate \upd{l}{v_0}{s'_0}$, but we necessarily have $v_0 = v$ and $s'_0 = s'$. Moreover, the lemma also gives $\Phi_v \tr \seqi{\Del_v}{v}{\Gam_{p}(x) \sqcup \stype(l)}{(b_v,m_v,d_v)}$ and $\Phi_{s'} \tr \seqi{\Del_{s'}}{s'}{\stype'}{(b_{s'},m_{s'},d_{s'})}$, where $\conj{(l : \Gam_{p}(x))} \splus \stype = \conj{(l : \Gam_{p}(x) \sqcup \stype(l))};\stype'$, $\Del = \Del_v + \Del_{s'}$, $b_s=b_v + b_{s'}$, $m_s=m_v + m_{s'}$, and $d_s=d_v + d_{s'}$. Thus, by~\cref{lem:split-values-stores}.\ref{lem:com-split-values} there exist $\Phi^1_v \tr \seqi{\Del^1_v}{v}{\Gam_{p}(x)}{(b^1_v,m^1_v,d^1_v)}$ and $\Phi^2_v \tr \seqi{\Del^2_v}{v}{\stype(l)}{(b^2_v,m^2_v,d^2_v)}$, such that $\Del_v = \Del^1_v + \Del^2_v$, $b_v = b^1_v+b^2_v$, $m_v = m^1_v+m^2_v$, and $d_v = d^1_v+d^2_v$. From $\Phi_{p} \tr \seqi{\Gam_{p}}{p}{\comptype{\stype}{\ctype}}{(b_p,m_p,d_p)}$ and $\Phi^1_v \tr \seqi{\Del^1_v}{v}{\Gam_{p}(x)}{(b^1_v,m^1_v,d^1_v)}$, we obtain $\Phi_{p \subs{x}{v}} \tr \seqi{(\Gam_{p} \sm x) +\Del^1_v}{p\subs{x}{v}}{\comptype{\stype}{\ctype}}{(b_p+b^1_v,m_p+m^1_v,d_p+d^1_v)}$, by~\cref{lem:comp-subs-antisubs}.\ref{lem:comp-subs}. We now construct an alternative type derivation for $s$ of the form:
        \[ \begin{prooftree}
            \hypo{\Phi^2_v \tr  \seqi{\Del^2_v}{v}{\stype(l)}{(b^2_v,m^2_v,d^2_v)}}
            \hypo{\Phi_{s'} \tr \seqi{\Del_{s'}}{s'}{\stype'}{(b_{s'},m_{s'},d_{s'})}}
            \infer2[(\ruleUpd)]{\seqi{\Del^2_v+ \Del_{s'}}{\upd{l}{v}{s'}}{\conj{(l:\stype(l))};\stype'}{(b^2_v+b_{s'},m^2_v+m_{s'},d^2_v+d_{s'})}}
        \end{prooftree} \]
        Let $q = s= \upd{l}{v}{s'}$ and let $\Phi_q$ be this new derivation above. Notice also that $\stype = \conj{(l:\stype(l))}; \stype'$. Then we can construct $\Phi'$ as follows:
        \[ \begin{prooftree}
            \hypo{\Phi_{p\subs{x}{v}}}
            \hypo{\Phi_q}
            \infer2[(\ruleConf)]{\seqi{(\Gam_{p} \sm x) + \Del^1_v + \Del^2_v + \Del_{s'}}{(p \subs{x}{v}, s)}{\kap}{(b,m,d)}}
        \end{prooftree} \]
        Notice that the type environment of the conclusion is $(\Gam_{p} \sm x) + \Del^1_v + \Del^2_v + \Del_{s'} = (\Gam_{p} \sm x) + \Del_v + \Del_{s'} = (\Gam_{p} \sm x) + \Del = \Gam $, and the counters are as expected.
        \item Case $(t,s) = (\set{l}{v}{p},s) \ra (p, \upd{l}{v}{s}) = (u,q)$. Let $\Phi_0$ be the following derivation:
        \[ \begin{prooftree}
          \hypo{\Phi_{v} \tr \seqi{\Gam_v}{v}{\M}{(b_v,m_v,d_v)}}
            \hypo{\Phi_{p} \tr \seqi{\Gam_{p}}{p}{\comptype{\conj{(l : \M)}; \stype}{\ctype}}{(b_p,m_p,d_p)}}
            \infer2[(\ruleSet)]{\seqi{\Gam_v + \Gam_{p}}{\set{l}{v}{p}}{\comptype{\stype}{\ctype}}{(b_v+b_p,1+m_v+m_p,s_v+s_p)}}
        \end{prooftree} \]
        $\Phi$ must be of the following form:
        \[ \begin{prooftree}
          \hypo{\Phi_0}    
            \hypo{\Phi_{s} \tr \seqi{\Gam_{s}}{s}{\stype}{(b_s,m_s,d_s)}}
            \infer2[(\ruleConf)]{\seqi{(\Gam_v + \Gam_{p}) + \Gam_{s}}{(\set{l}{v}{p}, s)}{\kap}{(b_v+b_p+b_s,1+m_v+m_p+m_s,d_v+d_p+d_s)}}
        \end{prooftree} \]
        where  $\Gam = (\Gam_v + \Gam_{p}) + \Gam_{s}$ is tight, $b = b_v+b_p+b_s$, $m=1+m_v+m_p+m_s$ and $d=d_v+d_p+d_s$. Therefore, we can build $\Phi_{\upd{l}{v}{s}}$ as follows:
        \[ \begin{prooftree}
          \hypo{\Phi_{v} \tr \seqi{\Gam_v}{v}{\M}{(b_v,m_v,d_v)}}
          \hypo{\Phi_{s} \tr \seqi{\Gam_{s}}{s}{\stype}{(b_s,m_s,d_s)}}
          \infer2[(\ruleUpd)]{\seqi{\Gam_v + \Gam_{s}}{\upd{l}{v}{s}}{\conj{(l : \M)}; \stype}{(b_v+b_s,m_v+m_s,d_v+d_s)}}
        \end{prooftree} \]
        Assume And we can build $\Phi'$ as follows:
        \[ \begin{prooftree}
            \hypo{\Phi_{p} \tr \seqi{\Gam_{p}}{p}{\comptype{\conj{(l : \M)}; \stype}{\ctype}}{(b_p,m_p,d_p)}}
            \hypo{\Phi_{\upd{l}{v}{s}}}
            \infer2[(\ruleConf)]{\seqi{\Gam_{p} + (\Gam_v + \Gam_{s})}{(p, \upd{l}{v}{s})}{\kap}{(b_v+b_v+b_s,m_v+m_v+m_s,d_v+d_v+d_s)}}
        \end{prooftree} \]
        Notice that the type environment of the conclusion is $\Gam_{p} + (\Gam_v + \Gam_{s}) = \Gam$, and the counters are as expected.
    \end{itemize}
%\end{proof}

        \item %\begin{proof}
    We show a stronger statement of the form:

    Let $(t,s) \red[\gname] (u,q)$. If  $\Phi' \tr \seqi{\Gam}{(u,q)}{\ctype}{(b',m',d)}$, $\Gam$ is tight, and ($\ctype$ is tight or $\neg \isvalue{t}$), then $\Phi \tr \seqi{\Gam}{(t,s)}{\ctype}{(b,m,d)}$, where $\gname =\beta$ implies $b' = b - 1$ and $m' = m$, while $\gname \in \{\getname, \setname\}$ implies $b'=b$ and $m' = m - 1$.

    We proceed by induction on $(t, s) \red (u,q)$: 
    \begin{itemize}
        \item Case $(t,s) = ((\lam x.p) v,s) \redbeta (p \subs{x}{v}, s) = (u,q)$. Then $\Phi'$ must be of the following form:
        \[ \begin{prooftree}
            \hypo{\Phi_{p \subs{x}{v}} \tr \seqi{\Gam_{p \subs{x}{v}}}{p \subs{x}{v}}{\comptype{\stype}{\ctype}}{(b'',m'',d'')}}
            \hypo{\Phi_s \tr \seqi{\Gam_s}{s}{\stype}{(b_s,m_s,d_s)}}
            \infer2[(\ruleConf)]{\seqi{\Gam_{p \subs{x}{v}} + \Gam_s}{(p \subs{x}{v}, s)}{\ctype}{(b''+b_s,m''+m_s,d''+d_s)}}
        \end{prooftree} \]
        such that $\Gam = \Gam_{p \subs{x}{v}} + \Gam_s$, $b' = b''+b_s$, $m' = m''+m_s$, and $d' = d''+d_s$. By~\cref{lem:comp-subs-antisubs}.\ref{lem:comp-antisubs}, there exist $\Phi_p \tr \seqi{\Gam_p; x : \M}{p}{\comptype{\stype}{\ctype}}{(b_p,m_p,d_p)}$ and $\Phi_{v} \tr \seqi{\Gam_v}{v}{\M}{(b_v,m_v,d_v)}$, such that $\Gam_{p \subs{x}{v}} = \Gam_p + \Gam_v$, $b'' = b_p+b_v$, $m'' = m_p+m_v$, and $d'' = d_p + d_v$. We can build $\Phi$ as follows:
        \[ \begin{prooftree}
            \hypo{\Phi_p \tr \seqi{\Gam_p; x : \M}{p}{\comptype{\stype}{\ctype}}{(b_p,m_p,d_p)}}
            \infer1[(\ruleLam)]{\seqi{\Gam_p}{\lam x.p}{\M \ta (\comptype{\stype}{\ctype})}{(b_p,m_p,d_p)}}
            \hypo{\Phi_{v} \tr \seqi{\Gam_v}{v}{\M}{(b_v,m_v,d_v)}}
            \infer1[(\ruleLift)]{\seqi{\Gam_v}{v}{\tcomptype{\stype}{\M}{\stype}}{(b_v,m_v,d_v)}}
            \infer2[(\ruleApp)]{\seqi{\Gam_p + \Gam_v}{(\lam x.p)v}{\comptype{\stype}{\ctype}}{(1+b_p+b_v,m_p+m_v,d_p+d_v)}}
            \hypo{\Phi_s}
            \infer2[(\ruleConf)]{\seqi{(\Gam_p + \Gam_v) + \Gam_s}{((\lam x.t')v, s)}{\ctype}{(1+b_p+b_v+b_s,m_p+m_v+m_s,d_p+d_v+d_s)}}
        \end{prooftree} \]
        such that $b = 1+b_p+b_v+b_s$, $m = m_p+m_v+m_s$, and $d = d_p+d_v+d_s$. And we can conclude with $\Gam = \Gam_{p \subs{x}{v}} + \Gam_s = (\Gam_p + \Gam_v) + \Gam_s$, $b' = b'' + b_s = b_p + b_v + b_s = (1 + b_p + b_v + b_s) - 1 = b - 1$, $m' = m'' + m_s = (m_p + m_v) + m_s = m$, and $d' = d'' + d_s = (d_p + d_v) + d_s = d$.
        \item Case $(t,s) = (vp,s) \ra (vp',q) = (u,q)$, such that $(p,s) \ra (p',q)$. Then we have three cases for the type derivation $\Phi_{p'}$ of $p'$ inside $\Phi'$:
        \begin{itemize}
            \item Case $\Phi_{vp'}$ ends with ($\ruleApp$). Let $\Phi_0$ be the following derivation:
            \[ \begin{prooftree}
                \hypo{\Phi_{v} \tr \seqi{\Gam_v}{v}{\M \ta \comptype{\stype'}{\ctype}}{(b_v,m_v,d_v)}}
                    \hypo{\Phi_{p'} \tr \seqi{\Gam_{p'}}{p'}{\tcomptype{\stype}{\M}{\stype'}}{(b'',m'',d'')}}
                    \infer2[(\ruleApp)]{\seqi{\Gam_v + \Gam_{p'}}{v p'}{\comptype{\stype}{\ctype}}{(1+b_v+b'', m_v+m'', d_v+d'')}}
            \end{prooftree} \]
            $\Phi'$ must be of the following form: 
            \[ \begin{prooftree}
                \hypo{\Phi_0}
                \hypo{\Phi_q \tr \seqi{\Gam_q}{q}{\stype}{(b_q,m_q,d_q)}}
                \infer2[(\ruleConf)]{\seqi{(\Gam_v + \Gam_{p'}) + \Gam_q}{(v p', q)}{\ctype}{(1+b_v+b''+b_q,m_v+m''+m_q,d_v+d''+d_q)}}
            \end{prooftree} \]
            such that $\Gam = (\Gam_v + \Gam_{p'}) + \Gam_q$ tight, $b' = 1+b_v+b''+b_q$, $m' = m_v+m''+m_q$, and $d' = d_v+d''+d_q$. So we can build $\Phi_{(p',q)}$ as follows:
            \[ \begin{prooftree}
                \hypo{\Phi_{p'} \tr \seqi{\Gam_{p'}}{p'}{\tcomptype{\stype}{\M}{\stype'}}{(b'',m'',d'')}}
                \hypo{\Phi_q \tr \seqi{\Gam_q}{q}{\stype}{(b_q,m_q,d_q)}}
                \infer2[(\ruleConf)]{\seqi{\Gam_{p'} + \Gam_q}{(p', q)}{\conftype{\M}{\stype'}}{(b''+b_q,m''+m_q,d''+d_q)}}
            \end{prooftree} \]
            Since $\Gam$ is tight, then $\Gam_{p'} + \Gam_q$ is tight. Moreover, $(p, s) \red (p',q)$ implies $\neg\isvalue{p}$. Then we can apply the \ih, and thus there exists a derivation for $(p,s)$ that must be of the following form:
            \[ \begin{prooftree}
                \hypo{\Phi_p \tr \seqi{\Gam_p}{p}{\tcomptype{\stype''}{\M}{\stype'}}{(b_p,m_p,d_p)}}
                \hypo{\Phi_s \tr \seqi{\Gam_s}{s}{\stype''}{(b_s,m_s,d_s)}}
                \infer2[(\ruleConf)]{\seqi{\Gam_p + \Gam_s}{(p, s)}{\conftype{\M}{\stype'}}{(b_p+b_s,m_p+m_s,d_p+d_s)}}
            \end{prooftree} \]
            where $\Gam_p + \Gam_s = \Gam_{p'} + \Gam_q$ is tight, and either (1) $b''+b_q = b_p+b_s-1$, $m''+m_q=m_p+m_s$, and $d''+d_q = d_p+d_s$, or (2) $b''+b_q = b_p+b_s$, $m''+m_q=m_p+m_s-1$, and $d''+d_q = d_p+d_s$. So, we can build $\Phi$ as follows:
            \[ \begin{prooftree}
                \hypo{\Phi_{v} \tr \seqi{\Gam_v}{v}{\M \ta (\comptype{\stype'}{\ctype})}{(b_v,m_v,d_v)}}
                \hypo{\Phi_p \tr \seqi{\Gam_p}{p}{\tcomptype{\stype''}{\M}{\stype'}}{(b_p,m_p,d_p)}}
                \infer2[(\ruleApp)]{\seqi{\Gam_v + \Gam_p}{vp}{\comptype{\stype''}{\ctype}}{(1+b_v+b_p,m_v+m_p,d_v+d_p)}}
                \hypo{\Phi_s}
                \infer2[(\ruleConf)]{\seqi{(\Gam_v + \Gam_p) + \Gam_s}{(vp, s)}{\kappa}{(1 + b_v + b_p+b_s,m_v+m_p+m_s,d_v+d_p+d_s)}}
            \end{prooftree} \]
            where $\Gam_v + \Gam_p + \Gam_s = \Gam_v + \Gam_{p'} + \Gam_q = \Gam$, $b = 1+b_v+b_p+b_s$, $m = m_v+m_p+m_s$, and $d = d_v+d_p+d_s$. We can conclude since:
            \begin{itemize}
                \item Case (1): $b' = 1 + b_v + b'' + b_q = 1 + b_v + b_p + b_s - 1 = b -1$, and the other counters are easy to check;
                \item Case (2): $m' = m_v + m'' + m_q = m_v + m_p + m_s - 1 = m - 1$, and the other counters are easy to check.
            \end{itemize}
            \item Case $\Phi_{vp'}$ ends with (\ruleAppPOne) or (\ruleAppPTwo). These two cases are very similar to the previous case.
        \end{itemize}
        \item Case $(t,s) = (\get{l}{x}{p},s) \ra (p \subs{x}{v},s) = (u,q)$, such that $s \equivstate \upd{l}{v}{s'}$. Let $\Phi_0$ be the following derivation:
        \[ \begin{prooftree}
            \hypo{\Phi^2_v \tr \seqi{\Gam^2_v}{v}{\M_2}{(b^2_v,m^2_v,d^2_v)}}
            \hypo{\Phi_{s'} \tr \seqi{\Gam_{s'}}{s'}{\stype}{(b_{s'},m_{s'},d_{s'})}}
            \infer2[(\ruleUpd)]{\seqi{\Gam^2_v + \Gam_{s'}}{\upd{l}{v}{s'}}{\conj{(l : \M_2)}; \stype}{(b^2_v+b_{s'},m^2_v+m_{s'},d^2_v+d_{s'})}}
        \end{prooftree} \]
        Then $\Phi'$ must be of the following form:
        \[ \begin{prooftree}
            \hypo{\Phi_{p \subs{x}{v}} \tr \seqi{\Gam_{p \subs{x}{v}}}{p \subs{x}{v}}{\comptype{\conj{(l : \M)}; \stype}{\ctype}}{(b'',m'',d'')}}
            \hypo{\Phi_0}
            \infer2[(\ruleConf)]{\seqi{\Gam_{p \subs{x}{v}} + (\Gam^2_v + \Gam_{s'})}{(p \subs{x}{v}, \upd{l}{v}{s'})}{\ctype}{(b''+b^2_v+b_{s'},m''+m^2_v+m_{s'},d''+d^2_v+d_{s'})}}
        \end{prooftree} \]
        such that $\Gam = \Gam_{p \subs{x}{v}} + (\Gam^2_v + \Gam_{s'})$, $b' = b'' + b^2_v + b_{s'}$, $m' = m'' + b^2_v + b_{s'}$, and $d' = d'' +d^2_v+d_{s'}$. By~\cref{lem:comp-subs-antisubs}.\ref{lem:comp-antisubs}, there exist $\Phi_p \tr \seqi{\Gam_p; x : \M_1}{p}{\comptype{\conj{(l : \M_2)}; \stype}{\ctype}}{(b_p,m_p,d_p)}$ and $\Phi^1_v \tr \seqi{\Gam^1_v}{v}{\M_1}{(b^1_v,m^1_v,d^1_v)}$, such that $\Gam_{p \subs{x}{v}} = \Gam_p + \Gam^1_v$, $b'' = b_p + b^1_v$, $m'' = m_p + m^1_v$, and $d'' = d_p + d^1_v$. Therefore, we can build $\Phi_{\get{l}{x}{p}}$ as follows:
        \[ \begin{prooftree}
            \hypo{\Phi_p \tr \seqi{\Gam_p; x : \M_1}{p}{\comptype{\conj{(l : \M_2)}; \stype}{\ctype}}{(b_p,m_p,d_p)}}
            \infer1[(\ruleGet)]{\seqi{\Gam_p}{\get{l}{x}{p}}{\comptype{\conj{(l : \M_1 \sqcup \M_2)}; \stype}{\ctype}}{(b_p,1+m_p,d_p)}}
        \end{prooftree} \]
        By~\cref{lem:comp-merge-values}, we have $\Phi_v \tr \seqi{\Gam^1_v + \Gam^2_v}{v}{\M_1 \sqcup \M_2}{(b^1_v+b^2_v,m^1_v+m^2_v,d^1_v+d^2_v)}$. Thus, we can build $\Phi_{\upd{l}{v}{s'}}$ as follows:
        \[ \begin{prooftree}
            \hypo{\Phi_v \tr \seqi{\Gam^1_v + \Gam^2_v}{v}{\M_1 \sqcup \M_2}{(b^1_v+b^2_v,m^1_v+m^2_v,d^1_v+d^2_v)}}
            \hypo{\Phi_{s'} \tr \seqi{\Gam_{s'}}{s'}{\stype}{(b_{s'},m_{s'},d_{s'})}}
            \infer2[(\ruleUpd)]{\seqi{(\Gam^1_v + \Gam^2_v) + \Gam_{s'}}{\upd{l}{v}{s'}}{\conj{(l : \M_1 \sqcup \M_2)}; \stype}{(b^1_v+b^2_v+b_{s'},m^1_v+m^2_v+m_{s'},d^1_v+d^2_v+d_{s'})}}
        \end{prooftree} \]
        Finally, we can build $\Phi$ as follows:
        \[ \begin{prooftree}
            \hypo{\Phi_{\get{l}{x}{p}}}
            \hypo{\Phi_{\upd{l}{v}{s'}}}
            \infer2[(\ruleConf)]{\seqi{\Gam_p + (\Gam^1_v + \Gam^2_v) + \Gam_{s'}}{(\get{l}{x}{p}, \upd{l}{v}{s'})}{\ctype}{(b_p+b^1_v+b^2_v+b_{s'},1+m_p+m^1_v+m^2_v+m_{s'},d_p+d^1_v+d^2_v+d_{s'})}}
        \end{prooftree} \]
        such that $b = b_p+b^1_v+b^2_v+b_{s'}$, $m = 1+m_p+m^1_v+m^2_v+m_{s'}$, and $d = d_p+d^1_v+d^2_v+d_{s'}$. And we can conclude with $\Gam = \Gam_{p \subs{x}{v}} + (\Gam^2_v + \Gam_{s'}) = \Gam_p + \Gam^1_v + \Gam^2_v + \Gam_{s'}$, $b' = b'' + b^2_v + b_{s'} = b_p + b^1_v + b^2_v + b_{s'} = b$, and $m' = m'' + m^2_v + m_{s'} = m_p + m^1_v + m^2_v + m_{s'} = (1 + m_p + m^1_v + m^2_v + m_{s'}) - 1 = m - 1$, $d' = d'' + d^2_v + d_{s'} = d_p + d^1_v + d^2_v + d_{s'} = d$.
        \item Case $(t,s) = (\set{l}{v}{p},s) \ra (p, \upd{l}{v}{s}) = (u,q)$. Let $\Phi_0$ be the following derivation:
        \[ \begin{prooftree}
            \hypo{\Phi_{v} \tr \seqi{\Gam_v}{v}{\M}{(b_v,m_v,d_v)}}
            \hypo{\Phi_{s} \tr \seqi{\Gam_s}{s}{\stype}{(b_s,m_s,d_s)}}
            \infer2[(\ruleUpd)]{\seqi{\Gam_v + \Gam_s}{\upd{l}{v}{s}}{\conj{(l : \M)}; \stype}{(b_v+b_s,m_v+m_s,d_v+d_s)}}
        \end{prooftree} \]
        $\Phi'$ must be of the following form:
        \[ \begin{prooftree}
            \hypo{\Phi_p \tr \seqi{\Gam_p}{p}{\comptype{\conj{(l : \M)}; \stype}{\ctype}}{(b_p,m_p,d_p)}}
            \hypo{\Phi_0}
            \infer2[(\ruleConf)]{\seqi{\Gam_p + (\Gam_v + \Gam_s)}{(p, \upd{l}{v}{s})}{\kap}{(b_p+b_v+b_s,m_p+m_v+m_s,d_p+d_v+d_s)}}
        \end{prooftree} \]
        such that $\Gam = \Gam_p + (\Gam_v + \Gam_s)$, $b' = b_p + b_v + b_s$, $m' = m_p + m_v + m_s$, and $d' = d_p + d_v + d_s$. Therefore, we can build $\Phi$ as follows:
        \[ \begin{prooftree}
            \hypo{\Phi_{v} \tr \seqi{\Gam_v}{v}{\M}{(b_v,m_v,d_v,)}}
            \hypo{\Phi_p \tr \seqi{\Gam_p}{p}{\comptype{\conj{(l : \M)}; \stype}{\ctype}}{(b_p,m_p,d_p)}}
            \infer2[(\ruleSet)]{\seqi{\Gam_v + \Gam_p}{\set{l}{v}{p}}{\comptype{\stype}{\ctype}}{(b_v+b_p,1+m_v+m_p,d_v+d_p)}}
            \hypo{\Phi_s}
            \infer2[(\ruleConf)]{\seqi{(\Gam_v + \Gam_p) + \Gam_s}{(\set{l}{v}{p}, s)}{\ctype}{(b_v+b_p+b_s,1+m_v+m_p+m_s,d_v+d_p+d_s)}}
        \end{prooftree} \]
        Notice that the type environment of the conclusion is $(\Gam_v + \Gam_p) + \Gam_s = \Gam$, and the counters are as expected.
    \end{itemize}
%\end{proof}
    \end{enumerate}
\end{proof}}

\compsoundness*

\maybehide{\begin{proof} \mbox{}
    \begin{enumerate}
        \item %\begin{proof}
    The proof follows by induction over $b+m$:
    \begin{itemize}
        \item Case $b+m = 0$. Then $b=m=0$, therefore $t \in \normal$, by point (1) of~\cref{lem:zero-counters}, and  $d = \size{t}$,  by point (2) of~\cref{lem:zero-counters}. Let $u = t$ and $q=s$, then  we can conclude since $\size{(u,q)} = \size{u} =\size{t} = d$.
        \item Case $b+m > 0$. Then $b>0$ or $m>0$, and in either case $t \not\in \normal$,  by~\cref{lem:zero-nfs}. Note that $(t,s)$ is not final because $t$ is unblocked by~\cref{prop:typed-unblock}. Therefore, by~\cref{prop:normal-iff-final} there exists $(t',s')$ such that $(t,s) \gsred (t',s')$. By~\cref{lem-exact-red-exp}.\ref{lem:subj-comp-red}, there exists $\Phi' \tr \seqi{\Gam}{(t',s')}{\ctype}{(b',m',d)}$, such that $b'+m'=b+m-1$. By the \ih, there exists $(u,q)$, such that $u\in \normal$, $(t',s') \gsrred^{(b',m')} (u,q)$ and $d = \size{(u,q)}$. So we can conclude with $(t,s) \gsred (t',s') \gsrred^{(b',m')} (u,q)$, which means that $(t,s) \drred^{(b,m)} (u,q)$, as expected.
    \end{itemize}
%\end{proof}

        \item %\begin{proof}
    By induction over $b + m$: \begin{itemize}
        \item Case $b + m = 0$. Then $b = m = 0$ and $(t,s) = (u,q)$. We can conclude by~\cref{lem:typestatesnfs}.\ref{lem:typ-states} and~\cref{lem:typestatesnfs}.\ref{lem:comp-typ-nfs}.
        \item Case $b + m > 0$. Then there exists $(t',s')$, such that $(t,s) \ra^{(1,0)} (t',s') \rra^{(b-1,m)} (u,q)$ or $(t,s) \ra^{(0,1)} (t',s') \rra^{(b,m-1)} (u,q)$. By the \ih, there exists $\Phi' \tr \seqi{\Gam}{(t',s')}{\kap}{(b',m',\size{(u,q)})}$ tight, such that $b' + m' = b + m - 1$. By~\cref{lem-exact-red-exp}.\ref{lem:comp-subj-exp}, we have $\Phi \tr \seqi{\Gam}{(t,s)}{\kap}{(b'',m'',\size{(u,q)})}$ tight, such that $b'' + m'' = 1+ b' + m'$. Therefore, $b'' + m'' = b + m$, since the fact that $b'' = b$, and $m'' = m$ can be easily checked by a simple case analysis.
    \end{itemize}
%\end{proof}
    \end{enumerate}
\end{proof}}


\section{Training, Datasets \& Resources}\label{app:settings}

In the context of DISTRO, we use the pre-trained diffusion model with 50M parameters, trained on \texttt{CIFAR10} with $1000$ steps and the cosine noise schedule~\citep{nichol2021improved}. 
Meanwhile, we trained the same model from scratch on \texttt{CIFAR100} for $1000$ steps, with a batch size of 128, a learning rate of 3$e$-4 and the cosine noise schedule.
The pretrained models OE, ATOM, ACET, ProoD and GOOD were trained with 80M Tiny Images~\cite{torralba200880} as OOD dataset.
The 80M Tiny Images dataset has been retracted because of concerns about offensive class labels. 
However, since previous studies have been conducted using this dataset, we compare our results to theirs.

We evaluate all methods on the standard datasets \texttt{CIFAR10/100}~\cite{cifar} as ID.
For the OOD detection evaluation we consider the following set of datasets: 
\texttt{CIFAR100/10}, \texttt{SVHN}~\cite{svhn}, LSUN~\cite{lsun} cropped (\texttt{LSUN\_CR}) and resized (\texttt{LSUN\_RS}) to $32\times32$,  TinyImageNet~\cite{tiny} cropped (\texttt{TinyImageNet\_CR}) to $32\times32$, \texttt{Textures}~\citep{textures} and synthetic (\texttt{Gaussian} and \texttt{Uniform}) noise distributions.
We use a random but fixed subset of 1000 images for all datasets considered as a test for OOD.
For ID, we consider the entire dataset.
We run all our experiments on a single NVIDIA A100. 



\section{Adversarial AUC, AUPR and FPR}\label{app:adversarial}

We use the settings in \citet{prood} to ensure a fair comparison.
Our goal is to maximize the confidence within the $\ell_\infty$-norm of adversarial attacks on OOD data.
We use an ensemble of projected gradient descent (PGD) \cite{madry2018towards} and 5000 queries with the black-box Square Attack \cite{squareattack}.
APGD \cite{apgd} is used with 500 iterations and 5 random restarts. The attack also includes a 200-step PGD with momentum of 0.9 and backtracking that starts with a step size of 0.1, which is halved if the gradient step does not increase confidence, and is multiplied by 1.1 otherwise.

Since robust OOD models are trained to be \textit{flat} on the out-distribution, disappearing gradients~\cite{pgd} pose a significant challenge for evaluating adversarial metrics~\cite{good, prood}.
As a result, a variety of starting points is necessary.
In following \citet{prood}, we start PGD from: 
i) a decontrasted version of the image, i.e. the point that minimizes the $\ell_\infty$-distance to the grey image $\{0.5\}^d$ within the threat model, ii) 3 uniform samples drawn from the threat model, and iii) 3 versions of the original image perturbed by Gaussian noise with $\sigma = 10^{-4}$ and then clipped to the threat model.
All steps of the attack are clipped to $[0,1]^d$, and the final score for OOD detection is directly optimized.
We present in section~\ref{ssec:faces} an application of PnP-HVAE on face images, using a pretrained state-of-the-art hierarchical VAE. 
Next, we study the application of our framework to natural images. To that end, we introduce  in section~\ref{ssec:patchVDVAE}  a patch hierachical VAE architecture, that is able to model natural images of different resolutions. In section~\ref{ssec:app_nat}, we provide deblurring, super-resolution and inpainting experiments to demonstrate the relevance of the proposed method.

Additional results are presented in Appendix~\ref{app:add}. All experiments can be reproduced using the code available at \url{https://github.com/jprost76/PnP-HVAE}.



\subsection{Face Image restoration (FFHQ)}\label{ssec:faces}
We first demonstrate the effectiveness of PnP-HVAE on highly structured data, by performing face image restoration.
Latent variable generative models can accurately model structured images such as face images \cite{karras2019style,vahdat2020nvae,child2021very,kingma2018glow}, and then be used to produce high quality restoration of such data. 
In our experiments, we use the VDVAE model of~\cite{child2021very}, pre-trained on the FFHQ dataset~\cite{karras2019style}, as our hierarchical VAE prior.
VDVAE has $L=66$ latent variable groups in its hierarchy and generates images at resolution $256\times256$.

We compare PnP-HVAE with the intermediate layer optimization algorithm (ILO)~\cite{daras2021intermediate} that is based on a different class of generative models than HVAE. ILO is a GAN inversion method which optimizes the image latent code along with the intermediate layer representation of a StyleGAN to generate an image consistent with a degraded observation.
We use the official implementation of ILO, along with a StyleGAN2 model~\cite{karras2020analyzing, stylegan2pytorch}, that was trained for 550k iterations on images of resolution $256\times256$ from FFHQ.  
As VDVAE and StyleGAN models are not trained on the same train-test split of FFHQ, we chose to evaluate the methods on a subset of 100 images from the CelebA dataset~\cite{liu2018large}. 
For super-resolution, the degradation model corresponds to the application of a gaussian low-pass filter followed by a $\times 4$ sub-sampling, and the addition of a gaussian white noise with $\sigma=3$.
For the deblurring, we considered motion blur and  gaussian kernels, both with a noise level $\sigma=8$. %

We provide quantitative comparisons in table~\ref{table:comp_ILO}, along with a visual comparison of the results in figure~\ref{fig:face_restoration}.
PnP-HVAE has the best  PSNR and SSIM results for all the considered restoration tasks, while ILO provides better results  for the perceptual distance.
By jointly optimizing the image and its latent variable, PnP-HVAE provides  results that are both realistic and consistent with the degraded observation.
On the other hand,  ILO  only optimizes on an extended latent space. This method generates  sharp and realistic images with better LPIPS scores,   
but the results lack  of consistency with respect to the observation, which explains the overall lower PSNR performance. 






\subsection{PatchVDVAE: a HVAE for natural images}\label{ssec:patchVDVAE}
Available generative models in the literature operate on images of  fixed resolutions and
are either restrained to datasets of limited diversity, or even to registered face images~\cite{kingma2018glow,child2021very, vahdat2020nvae, karras2019style}, or requiring additional class information~\cite{brock2018large, dhariwal2021diffusion, song2020score, luhman2022optimizing}.
Fitting an unconditional model on natural images appears to be a more difficult task, as their resolution can change, and their content is highly diverse.
The complexity of the problem can be reduced by learning a prior model on patches of reduced dimension. 
For image restoration problems, the patch model can be reused on images of higher dimensions~\cite{zoran2011learning,prost2021learning,altekruger2022patchnr}. When the model is a full CNN, the prior on the set of the  patches can  be computed efficiently by applying the network on the full image~\cite{prost2021learning}.

We thus introduce  patchVDVAE, a fully convolutional hierarchical VAE.
Contrary to existing HVAE models whose resolution is constrained by the constant tensor at the input of the top-down block, patchVDVAE can generate images of different resolutions by controlling the dimension of the input latent. 
This amounts to defining a prior on patches whose dimension corresponds to the receptive field of the VAE. A similar model is used for image denoising in~\cite{prakash2021interpretable}.

 
For PatchVDVAE architecture, we use the same bottom-up and top-down blocks as VDVAE~\cite{child2021very}, and replace the constant trainable input in the first top-down block by a latent variable, to make the model fully convolutional (details on the  architecture are given in Appendix~\ref{app:details}). 
The training dataset is composed of $128\times 128$ patches extracted from a combination of DIV2K~\cite{agustsson2017ntire} and Flickr2K~\cite{Lim_2017_CVPR_workshops} datasets.
We perform data augmentation by extracting  patches at $3$ resolutions: HR-images and $\times 2$ and $\times 4$ downscaled images. 
The model is trained for $7.10^5$ iterations with a batch size of $64$. Following the recommendation of~\cite{hazami2022efficient}, we use Adamax optimizer with an exponential moving average and gradient smoothing of the variance.
We set the decoder model to be a gaussian with diagonal covariance, as in~\cite{luhman2022optimizing}.
PatchVDVAE is fully convolutional and can generate images of dimension that are multiples of $64$ as illustrated by
figure~\ref{fig:vdvae}.

\newlength{\patchwidth}
\setlength{\patchwidth}{0.135\columnwidth}
\begin{figure}[!ht]
    \centering
    \begin{subfigure}[t]{.34\columnwidth}\hspace{0.1cm}
        \setlength{\tabcolsep}{0.02pt}
\renewcommand{\arraystretch}{0}
        \begin{tabular}{*{2}{p{1.03\patchwidth}}}
            \includegraphics[width=\patchwidth]{figures_arxiv/patchVDVAE/samples/generated/64x64/setup-5-image-0018.png} &
            \includegraphics[width=\patchwidth]{figures_arxiv/patchVDVAE/samples/generated/64x64/setup-5-image-0016.png} \\
            \includegraphics[width=\patchwidth]{figures_arxiv/patchVDVAE/samples/generated/64x64/setup-5-image-0008.png} &
            \includegraphics[width=\patchwidth]{figures_arxiv/patchVDVAE/samples/generated/64x64/setup-5-image-0019.png}   
        \end{tabular}
    \end{subfigure}\hspace{-0.15cm}
    \begin{subfigure}[t]{.64\columnwidth}
\begin{tabular}{cc}\vspace{-0.1cm}
\includegraphics[width=2\patchwidth]{figures_arxiv/patchVDVAE/samples/generated/256x256/setup-2-image-0009.png}&
        \includegraphics[width=2\patchwidth]{figures_arxiv/patchVDVAE/samples/generated/256x256/setup-2-image-0002.png}\end{tabular}

    \end{subfigure}
    \caption{\label{fig:vdvae} Left: $64\times64$ patches samples from our patchVDVAE model trained on patches from natural images.
    Right: PatchVDVAE is fully convolutional and it can generate images of higher resolution (here: $128\times128$).\vspace{-0.2cm}}
\end{figure}

\subsection{Natural images restoration}\label{ssec:app_nat}
We  evaluate PnP-HVAE on natural image restoration.
For each task, we report the average value of the PSNR, the SSIM, and the LPIPS metrics on $20$ images from the test set of the BSD dataset~\cite{MartinFTM01}.\\


\noindent
{\bf Image deblurring.}
In the experiments, we consider $2$ gaussian kernels and $2$ motion blur kernels from~\cite{levin2009understanding}, with $3$ different noise levels 
$\sigma \in \{2.55, 7.65, 12.75\}$.
As a baseline we consider  EPLL~\cite{zoran2011learning}, which learns a prior on image patches with a gaussian mixture model.
We also compare PnP-HVAE  with PnP-MMO and GS-PnP, $2$ competing convergent Plug-and-Play methods based on CNN denoisers.
PnP-MMO~\cite{pesquet2021learning} restricts the denoiser to be contraction in order to guarantee the convergence of the PnP forward-backard algorithm. GS-PnP~\cite{hurault2022gradient} considers a gradient step denoiser and reaches state-of-the-art performances of non converging methods~\cite{zhang2021plug}.
We set the temperature $\tau$  in our method as $0.95$, $0.8$ and $0.6$ for noise levels $2.55$, $7.65$ and $12.75$ respectively, and we let it run for a maximum of $50$ iterations. 
For the three compared methods we use the official implementations and pre-trained models provided by the respective authors. 
Details on the choice of hyperparameters for the concurrent methods are provided in the Appendix~\ref{app:details}
Figure~\ref{fig:deblurring_bsd} illustrates that our method provides correct deblurring results. 

According to table~\ref{tab:deb}, the performance of PnP-HVAE is between those of EPLL and GS-PnP and it outperforms PnP-MMO for large noise levels.\\

\begin{table}
\begin{center}\footnotesize
    \begin{tabular}{>{\centering}m{.3cm}*{5}{c}}
    $\sigma$ &Method & PSNR$\uparrow$ & SSIM$\uparrow$ & LPIPS$\downarrow$  \\ 
    \hline
    \multirow{4}{*}{\vcell{$2.55$}}
    & PnP-HVAE & $27.75$ & $0.79$ & $0.31$\\
    & GS-PNP \cite{hurault2022gradient} & $\mathbf{29.59}$ & $\mathbf{0.84}$ & $\mathbf{0.22}$\\
    & EPLL \cite{zoran2011learning} & $26.49$ & $0.71$ & $0.36$\\ 
    & PnP-MMO \cite{pesquet2021learning} & $\underbar{29.50}$ & $\underbar{0.83}$ & $\underbar{0.20}$ \\ \hline
    \multirow{4}{*}{\vcell{$7.65$}}
    & PnP-HVAE & $\underbar{26.36}$ & $\underbar{0.72}$ & $\underbar{0.40}$\\
    & GS-PNP \cite{hurault2022gradient} & $\mathbf{27.33}$ & $\mathbf{0.77}$ & $\mathbf{0.31}$\\
    & EPLL \cite{zoran2011learning} & $24.04$ & $0.66$ & $0.45$ \\ 
    & PnP-MMO \cite{pesquet2021learning} & $25.34$ & $0.69$ & $0.34$\\
    \hline
    \multirow{4}{*}{\vcell{$12.75$}}
    & PnP-HVAE & $\underbar{25.12}$ & $\mathbf{0.73}$ & $\underbar{0.47}$\\
    & GS-PNP \cite{hurault2022gradient} & $\mathbf{26.32}$ & $\mathbf{0.73}$ & $\mathbf{0.37}$\\
    & EPLL \cite{zoran2011learning} & $23.28$ & $0.61$ & $0.51$ \\ 
    & PnP-MMO \cite{pesquet2021learning} & $22.42$ & $0.53$& $0.54$ \\
    \hline
    &\vspace*{-.3cm}\\
            \multicolumn{2}{c}{Blur and motion kernels}& \multicolumn{3}{c}{
        \includegraphics*[scale=1]{figures_arxiv/kernels/4.png}\;\includegraphics*[scale=1]{figures_arxiv/kernels/7.png}\;\includegraphics*[scale=1]{figures_arxiv/kernels/9.png}\;\includegraphics*[scale=1]{figures_arxiv/kernels/11.png}} 
    \end{tabular}
        \caption{\label{tab:deb}Comparison  of PnP-HVAE  and other restoration methods on deblurring. Results are averaged on $4$ kernels.\vspace{-0.2cm}}% on image deblurring.}
    \end{center}
\end{table}

\begin{figure}
    
    \begin{subfigure}[h]{\linewidth}
        \centering
        \includegraphics*[width=\columnwidth]{figures_arxiv/deb_s255_k7.pdf}\vspace{-0.1cm}
        \caption{Gaussian blur, $\sigma=2.55$}
    \end{subfigure}
    \begin{subfigure}[h]{\linewidth}
        \centering
        \includegraphics*[width=\columnwidth]{figures_arxiv/deb_s765_k11.pdf}\vspace{-0.1cm}
        \caption{Motion blur, $\sigma=7.65$}
    \end{subfigure}\vspace*{-0.1cm}
    \caption{\label{fig:deblurring_bsd} Natural image deblurring\vspace{-0.1cm}}
\end{figure}

\noindent {\bf Effect of the temperature.}
PnP-HVAE gives control on the temperature of the prior over the latent space.
In figure~\ref{fig:temp_effect}, we illustrate that reducing the temperature increases the strength of the regularization prior. In this example the tuning $\tau=0.7$ produces the best performance.\\
\begin{figure}[!ht]
   
    \includegraphics[width=\columnwidth]{figures_arxiv/demo_temp.pdf}\vspace{-0.15cm}
    \caption{ \label{fig:temp_effect} Effect of the temperature in PnP-VAE on a deblurring problem, with $\sigma=7.65$.\vspace{-0.15cm}}
\end{figure}


\noindent
{\bf Image inpainting.}
Next we consider the task of noisy image inpainting. 
We compose a test-set of 10 images from the validation set of BSD~\cite{MartinFTM01} and we create masks
  by occluding diverse objects of small size in the images. 
A gaussian white noise with $\sigma=3$ is added to the images.
As a comparaison, we still consider GS-PnP and EPLL.
For PnP-HVAE, the temperature is set to $\tau=0.6$, and the algorithm is run for a maximum of $200$ iterations, unless the residual $||\x_{k+1}-\x_k||$ is on a plateau.
We provide on Table~\ref{tab:inpainting_bsd} the distortion metrics with the ground truth, as well as a visual
\begin{table}



\begin{center}
    \begin{tabular}{cccc}
        & PSNR$\uparrow$ & SSIM$\uparrow$ &LPIPS$\downarrow$ \\\hline
        PnP-HVAE  & $\mathbf{29.54}$ & $\mathbf{0.93}$ & $\mathbf{0.06}$\\
        GS-PNP & $28.52$ & $\mathbf{0.93}$ & $0.09$\\
        EPLL & $\underline{29.16}$ & $\mathbf{0.93}$ & $\mathbf{0.06}$\\
    \end{tabular}
    \caption{\label{tab:inpainting_bsd}Quantitative evaluation for inpainting on BSD.}
    \end{center}
\end{table}
comparison on figure~\ref{fig:inpainting_bsd}. 
With its hierarchical structure,  PnP-HVAE outperforms the compared methods. \vspace{0.05cm}



\begin{figure}[!h]
    \includegraphics[width=\columnwidth]{figures_arxiv/demo_inp_bsd2.pdf}\vspace{-0.1cm}
    \caption{\label{fig:inpainting_bsd}Natural image inpainting\vspace{-0.3cm}}
\end{figure}











\section{Experimental Results on Similar Model Capacity}\label{app:standardized}

In this section we compare each method on a similar model architecture, since the results in \autoref{tab:contribution} are not only dependent on the method performance, but also on the capacity of the model and OOD dataset used.
Therefore we retrain all presented methods using a ResNet18 architecture for \texttt{CIFAR10} and \texttt{CIFAR100} respectively. 
For methods that require an additional OOD dataset for training, such as OE \cite{oe}, ACET \cite{acet}, ATOM \cite{atom}, ProoD \cite{prood} and DISTRO, we use the same subset of \texttt{OpenImages}~\cite{openimages} containing 50'000 images.
Furthermore, we consider an input normalization of $0.5$ across all dimensions for both mean and standard deviation.
In addition, we attempt to be as minimally intrusive as possible when it comes to the default training procedure.


For Plain, OE and LogitNorm we run the implementation\footnote{\href{https://github.com/Jingkang50/OpenOOD}{https://github.com/Jingkang50/OpenOOD}} from \citet{yang2022openood} and leave the hyperparameters unchanged.
Similarly for ACET and ATOM, we only change the model architecture and normalization and run both  implementations from ATOM\footnote{\href{https://github.com/jfc43/informative-outlier-mining}{https://github.com/jfc43/informative-outlier-mining}}.
Lastly, we train ProoD\footnote{\href{https://github.com/AlexMeinke/Provable-OOD-Detection}{https://github.com/AlexMeinke/Provable-OOD-Detection}} from \citet{prood} using their training configuration files, where the discriminator is trained for 1000 epochs and the bias shift ($\Delta$) is 3/1 for \texttt{CIFAR10/100}, respectively.


% All images are normalized with mean and standard deviation set to $0.5$ across all channels. 
% For Plain, Outlier Exposure and LogitNorm we use RandomHorizontalFlip and RandomCrop as augmentation techniques. 
% We optimize these models using Stochastic Gradient Descent for 100 epochs, with an initial learning rate of $0.1$, annealing to $10^{-6}$ over the course of training, momentum of $0.9$ and weight\_decay of $0.0005$. 
% The training batch size is $128$ for all methods.
% In general for ATOM and ACET, we try to follow the hyperparameters from the Github implementation. 
% For ACET and ATOM the training is mostly similar, but we use SGD with weight decay of $0.0001$ and a the initial learning rate of $0.1$ is multiplied by $0.1$ at epoch 50, 75 and 90. 
% The PGD attack in ACET is conducted using $\eps=2/255$ for 5 epochs using a step size of 2. 
% To select the outliers in ATOM, we use a quantile size of $1/8$.
% For ProoD, we follow their original implementation and just change add the normalization of $0.5$ for both mean and standard deviation.

\begin{table*}[htb]
\vspace{-0.5em}
    \centering
    \caption{\textbf{Robust OOD detection with ResNet18.} The guaranteed $\ell_2$-norm is computed for $\sigma = 0.12$, while the adversarial and guaranteed $\ell_\infty$-norm are computed for $\epsilon = 0.01$. The grayed-out models have an accuracy drop of $>3\%$ relative to the model with the highest accuracy. \textbf{Bold} numbers are superior results.
    } 
    \label{tab:ood_average_standard}
    \begin{adjustbox}{width=\textwidth,center}
    \begin{tabular}{lr|rrrr|rrrr|rrrr}
    \toprule
        ID: CIFAR10 &Acc. &AUC$\uparrow$ &\multicolumn{2}{c}{GAUC$\uparrow$} &AAUC$\uparrow$ &AUPR$\uparrow$ &\multicolumn{2}{c}{GAUPR$\uparrow$} &AAUPR$\uparrow$ &FPR$\downarrow$ &\multicolumn{2}{c}{GFPR$\downarrow$} &AFPR$\downarrow$ \\
        &   &   &\multicolumn{1}{c}{$\ell_2$} &\multicolumn{1}{c}{$\ell_\infty$} &\multicolumn{1}{c|}{$\ell_\infty$} &   &\multicolumn{1}{c}{$\ell_2$} &\multicolumn{1}{c}{$\ell_\infty$} &\multicolumn{1}{c|}{$\ell_\infty$} &   &\multicolumn{1}{c}{$\ell_2$} &\multicolumn{1}{c}{$\ell_\infty$} &\multicolumn{1}{c}{$\ell_\infty$} \\
        \midrule
 		Plain &94.32 &92.28 &35.81 &0.00 &23.71 &99.00 &46.83 &0.00 &82.00 &40.21 &93.56 &100.0 &98.88 \\
		LogitNorm &94.71 &95.58 &34.19 &0.00 &35.00 &99.54 &49.63 &0.00 &85.14 &33.06 &95.12 &100.0 &92.20\\
		OE &92.41 &97.35 &50.56 &0.00 &37.95 &99.71 &62.25 &0.00 &85.51 &13.44 &100.0 &100.0 &74.91\\
		ACET &93.66 &\textbf{97.86} &37.45 &0.00 &65.21 &\textbf{99.75} &50.26 &0.00 &91.99 &8.94 &100.0 &100.0 &\textbf{50.29} \\
		\gray{ATOM} &\gray{91.90} &\gray{98.12} &\gray{97.98} &\gray{97.63} &\gray{62.79} &\gray{99.78} &\gray{98.16} &\gray{99.78} &\gray{91.49} &\gray{8.7} &\gray{9.42} &\gray{0.00} &\gray{51.56} \\
		ProoD &\textbf{95.20} &96.91 &44.95 &\textbf{63.44} &64.61 &99.63 &60.27 &\textbf{94.37} &94.42 &\textbf{16.03} &100.0 &\textbf{91.90} &78.22 \\
		DISTRO (our) &\textbf{95.20} &96.80 &\textbf{86.63} &59.86 &\textbf{71.70} &99.62 &\textbf{90.80} &93.78 &\textbf{95.72} &16.55 &\textbf{66.88} &99.96 &67.59\\
        \midrule
        ID: CIFAR100 &Acc. &AUC$\uparrow$ &\multicolumn{2}{c}{GAUC$\uparrow$} &AAUC$\uparrow$ &AUPR$\uparrow$ &\multicolumn{2}{c}{GAUPR$\uparrow$} &AAUPR$\uparrow$ &FPR$\downarrow$ &\multicolumn{2}{c}{GFPR$\downarrow$} &AFPR$\downarrow$ \\
        &   &   &\multicolumn{1}{c}{$\ell_2$} &\multicolumn{1}{c}{$\ell_\infty$} &\multicolumn{1}{c|}{$\ell_\infty$} &   &\multicolumn{1}{c}{$\ell_2$} &\multicolumn{1}{c}{$\ell_\infty$} &\multicolumn{1}{c|}{$\ell_\infty$} &   &\multicolumn{1}{c}{$\ell_2$} &\multicolumn{1}{c}{$\ell_\infty$} &\multicolumn{1}{c}{$\ell_\infty$} \\
        \midrule
		Plain &77.54 &84.50 &38.11 &0.00 &24.17 &98.16 &44.96 &0.00 &82.32 &67.61 &100.0 &100.0 &98.04 \\
		LogitNorm &76.25 &84.06 &40.93 &0.00 &47.64 &98.04 &46.80 &0.00 &87.25 &73.70 &100.0 &100.0 &87.98\\
		OE &75.84 &88.96 &38.90 &0.00 &17.90 &\textbf{98.72} &48.82 &0.00 &81.43 &49.61 &100.0 &100.0 &99.41\\
		\gray{ACET} &\gray{73.71} &\gray{95.65} &\gray{42.03} &\gray{0.00} &\gray{52.49} &\gray{99.44} &\gray{48.54} &\gray{0.00} &\gray{89.23} &\gray{13.96} &\gray{100.0} &\gray{100.0} &\gray{60.39} \\
		ProoD &\textbf{77.77} &\textbf{89.47} &40.72 &\textbf{37.68} &49.16 &98.66 &49.97 &\textbf{89.66} &91.08 &\textbf{40.44} &100.0 &100.0 &84.15 \\
		DISTRO (our) &77.73 &88.90 &\textbf{55.57} &29.71 &\textbf{51.89} &98.60 &\textbf{67.62} &87.44 &\textbf{91.71} &43.24 &100.0 &100.0 &\textbf{79.34} \\
        \bottomrule
    \end{tabular}
    \end{adjustbox}
  \vspace{-1em}
\end{table*}


\end{document}