\section{Conclusion}\label{sec:conclusion}

As of now, certifying the robustness of OOD detection requires external binary discriminators or loose certification mechanisms~\cite{prood}.
This work proposes an alternative approach to compute $\ell_2$-norm robustness certificates for OOD data using randomized smoothing~\cite{randomized_smoothing}.
This technique can be used to certify the confidence of any classifier without requiring certified binary discriminators or specific training methods.
In comparison with previously proposed $\ell_\infty$-norm GAUC, standard approaches for OOD detection show non-zero results for guaranteed $\ell_2$-norm AUC and AUPR.
Unfortunately, a large number of samples derived around the input must be propagated through the network, increasing computational costs.

Additionally, we propose a new method as combination of three techniques.
First, we use a diffusion denoiser~\cite{dds} to remove noise from adversarial attacks before it reaches the classifier.
As a result, we were more successful in defending from adversarial attacks on ID and OOD data.
Second, we add an OOD detection method like OE to better distinguish between ID and OOD.
Lastly, we include a certified binary discriminator~\citep{good, prood} to ensure low confidence for samples \textit{far enough} from the training distribution.
Combining these methods improved the performances on several OOD robustness detection metrics by an average of $\sim 13\%/5\%$ relative to previous approaches on \texttt{CIFAR10/100} datasets.