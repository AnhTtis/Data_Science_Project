
In this section, we describe a simple approach to proving equivalence between
counting and detection problems, by combining Fredman's trick with Equality Product.  To illustrate the basic idea, we focus on
the equivalence of \AEExactTriCount{} and \AEExactTri{}.  With more work and additional ideas, the approach can also establish the equivalence of \ThreeSUMCount{} and \ThreeSUM{}, as we will later explain in Section~\ref{sec:counting}.



We use $G$ to denote the input graph of an \AEExactTri{}  or \AEExactTriCount{} instance, we use $w$ to denote the weight function, and we use $W_{ij}$ to denote the set of $k$ where $(i, j, k)$ forms a triangle whose edge weights sum up to the target value $t$.

\begin{lemma}
\label{lem:exact-tri}
Given an $n$ node graph $G$, a target value $t$, and a subset $S \subseteq V(G)$, we can compute a matrix $D$ in $\OO(|S| \cdot n^{(3+\omega)/2})$ time such that $D_{ij} = |W_{ij}|$ whenever $S \cap W_{ij} \ne \emptyset$. 
\end{lemma}
\begin{proof}
For every $s \in S$, we do the following. Let $A^{(s)}$ be a matrix where $A^{(s)}_{ik} = w(i, k) - w(i, s)$ and $B^{(s)}_{kj} = w(s, j) - w(k, j)$. Then we use compute the equality product $C^{(s)}$ of $A^{(s)}$ and $B^{(s)}$ in $\OO(n^{(3+\omega)/2})$ time for each $s$ \cite{MatIPL}. Finally, if there exists $s \in S$ such that $A_{is} + B_{sj} + w(i, j) = t$, we let $D_{ij}$ be $C^{(s)}_{ij}$ for an arbitrary $s$ with the property; otherwise, we let $D_{ij}$ be $0$ (we don't care about its value in this case). The running time for computing $D$ is clearly $\OO(|S| \cdot n^{(3+\omega)/2})$. 

Suppose $S \cap W_{ij} \ne \emptyset$ for some $(i, j)$. Then $D_{ij}$ equals $C^{(s)}_{ij}$ for some $s$ where $A_{is} + B_{sj} + w(i, j) = t$. By Fredman's trick, $A^{(s)}_{ik} = B^{(s)}_{kj}$ if and only if $w(i, k) + w(k, j) + w(i, j) = w(i, s) + w(s, j) + w(i, j) = t$. Therefore, $D_{ij} = C^{(s)}_{ij} = |W_{ij}|$.
\end{proof}

\begin{theorem}
\label{thm:exact-tri-count}
If \AEExactTri{} for $n$-node graphs has an $O(n^{3-\eps})$ time algorithm for some $\eps > 0$, then \AEExactTriCount{} for $n$-node graphs  has an $O(n^{3-\eps'})$ time algorithm for some $\eps' > 0$
\end{theorem}

\begin{proof}
Given a \AEExactTriCount{} instance on an $n$-node graph $G$, we first list up to $n^{0.99}$ elements in $W_{ij}$ for every $i, j$. 
By well-known techniques (e.g. \cite{focsyj}), an $O(n^{3-\eps})$ time \AEExactTri{} algorithm implies an $O(n^{3-\eps''})$ for $\eps'' > 0$ time algorithm for listing up to $n^{0.99}$ witnesses for each $(i, j)$ in an \AEExactTri{} instance. 

If we list less than $n^{0.99}$ elements for some $(i, j)$, we can output the number of elements we list as the exact witness count for $(i, j)$. By the standard greedy algorithm for hitting set, in $\OO(n^{2.99})$ time, we can find a set $S$ of size $\OO(n^{0.01})$ that intersect with $W_{ij}$ for the remaining pairs $(i, j)$. Therefore, we can apply Lemma~\ref{lem:exact-tri} to compute the number of witnesses for these remaining $(i, j)$ pairs in $\OO(|S| \cdot n^{(3+\omega)/2}) \le O(n^{2.70})$ time. 

The total running time for the \AEExactTriCount{} instance is thus $\OO(n^{3-\eps''} + n^{2.99}+n^{2.70})$, which is truly subcubic. 
\end{proof}

The reduction from \AEExactTri{} to \AEExactTriCount{} is trivial. 

\begin{remark}
\label{rem:exact-tri-count} \rm
Given Theorem~\ref{thm:exact-tri-count}, it is simple to derive a subcubic equivalence between \ExactTri{} and \ExactTriCount{}. First, the reduction from \ExactTri{} to \ExactTriCount{} is trivial. To reduce \ExactTriCount{} to \ExactTri{}, we first reduce \ExactTriCount{} to \AEExactTriCount{} in the trivial way, then use Theorem~\ref{thm:exact-tri-count} to further reduce it to \AEExactTri{}, and finally reduce it to \ExactTri{} by known reductions~\cite{focsyj}. 
\end{remark}

In Section~\ref{sec:counting}, we will use a similar approach to obtain other equivalence
results between counting and detection problems.  In particular, the proof of subquadratic equivalence
between \ThreeSUMCount{} and \ThreeSUM{} will require further technical ideas: we will need to 
exploit or modify known reductions from \AllThreeSUMConv{} to \AEExactTri{},
and \ThreeSUM{} to \ThreeSUMConv{}.
