
In this section, we apply the idea of combining Fredman's trick with Equality Product to prove conditional lower bounds
for problems with intermediate complexity.
To illustrate the idea,
we focus on lower bounds for the \uAPSP{} problem
under the \StrongAPSP{}, but the approach also leads to lower bounds for \MinWitness{} and other problems, under the \StrongAPSP{} as well as the
\uAPSPH{}, as we will explain later in Sections~\ref{sec:intapsp-lower-bound:more}--\ref{sec:uapsp-lower-bound}.
As noted earlier, the \StrongAPSP{} is equivalent to the hypothesis
that $M^*(n,n,n\mid n^{3-\omega})$ is not truly subcubic.
Thus, it suffices to describe fine-grained reductions from the \MinPlus{} problem for two $n\times n$ matrices 
with bounded integer entries from $[n^{3-\omega}]$, to the \uAPSP{}  problem.

When devising a reduction from one 
problem to another (as in typical NP-hardness proofs), we often concentrate on understanding the power of the latter problem.
In our new reductions, we will focus mostly on the former problem instead, interestingly:
we will attempt to design an algorithm to solve the \MinPlus{} problem for bounded integers in subcubic time, and 
only at the end, reveal how an oracle for \uAPSP{} (or \MinWitness{} and other problems) could help.

\subsection{Preliminaries: Generalized Equality Products}\label{sec:prelim:eq:prod}

By Matou\v sek's technique for dominance product~\cite{MatIPL, YusterDom},
the equality product of an $n_1\times n_2$
and an $n_2\times n_3$ matrix can be computed in
time $\OO\big(\min_r (n_1n_2n_3/r \,+\, M(n_1,rn_2,n_3)) \big)$.
For example, in the case $n_1=n_2=n_3=n$, the bound is at most $\OO(\min_r (n^3/r + rn^\omega))=\OO(n^{(3+\omega)/2})$, as we have stated before, if we don't use rectangular matrix multiplication exponents.
We begin with the following lemma describing a straightforward generalization, which will be useful later:

\begin{lemma}\label{lem:geneqprod}
Given $n_1\times n_2$ matrices $A$ and $A'$, and
$n_2\times n_3$ matrices $B$ and $B'$,
define the \emph{generalized equality product} of $(A,A')$ and $(B,B')$ to be the following $n_1\times n_3$ matrix $E$:
\[ E_{ij}\ := \min_{k:\,A_{ik}=B_{kj}} (A'_{ik} + B'_{kj}).
\]
Suppose that all matrix entries of $A'$ and $B'$ are in $[\pm \ell] \cup \{\infty\}$.  For any $r$, we can compute $E$ in time
\[ \OO\big(n_1n_2n_3/r \,+\, M^*(n_1,rn_2,n_3\mid \ell) \big).
\]
\end{lemma}
\begin{proof}
Sort $B_k=(B_{kj})_{j\in[n_3]}$, i.e., the $k$-th column of $B$.
Let $F_k$ be the set of elements that have frequency more than $n_3/r$ in $B_k$.
Note that $|F_k|\le r$.
We divide into two cases, computing two $n_1\times n_3$ matrices
$E^L$ and $E^H$;
the answers will be $E_{ij}=\min\{E_{ij}^L,E_{ij}^H\}$.

\begin{itemize}
\item[\bf--]
{\bf Low-frequency case: computing $\displaystyle E_{ij}^L=\min_{k:\, A_{ik}=B_{kj}\not\in F_k} (A'_{ik} + B'_{kj})$.}
Initially set $E_{ij}^L=\infty$.
For each $i\in [n_1]$ and $k\in [n_2]$,
if $A_{ik}\not\in F_k$, we examine each of the at most $n_3/r$ indices $j$ with $A_{ik}=B_{kj}$, and
reset $E_{ij}^L=\min\{E_{ij}^L,\, A'_{ik}+B'_{kj}\}$.
All this takes $O(n_1n_2\cdot n_3/r)$ time.

\item[\bf--]
{\bf High-frequency case: computing $\displaystyle E_{ij}^H=\min_{k:\, A_{ik}=B_{kj}\in F_k} (A'_{ik} + B'_{kj})$.}
For each $i\in [n_1]$, $k\in[n_3]$, and $p\in F_k$,
let $\hat{A}_{i,(k,p)}=A'_{ik}$ if $A_{ik}=p$, and $\hat{A}_{i,(k,p)}=\infty$ otherwise.
For each $k\in[n_3]$, $j\in [n_2]$, and $p\in F_k$,
let $\hat{B}_{(k,p),j}=B'_{kj}$ if $B_{kj}=p$, and $\hat{B}_{(k,p),j}=\infty$ otherwise.
We let $E_{ij}^H = \min_{k\in [n_2],p\in F_k} (\hat{A}_{i,(k,p)} + \hat{B}_{(k,p),j})$.
This can be computed by a Min-Plus product in 
$O(M^*(n_1,rn_2,n_3\mid \ell))$ time.
\end{itemize}

\negbigskip
\end{proof}

\subsection{The Key Reduction}

We now present an approach to solve the \MinPlus{} problem for two matrices with
integer entries in $[\ell]$, by reducing it to 
instances that have simultaneously
a smaller inner dimension~$s$ and smaller integer entries in $[t]$ with $t\le \ell$: 

\begin{theorem}\label{thm:main} 
For any $r,s,t$ with $s\le n_2$ and $t\le \ell$,
\[ M^*(n_1,n_2,n_3\mid\ell)\ =\ 
\OO\big( (n_2/s) M^*(n_1,s,n_3\mid t)
\,+\, sn_1n_2n_3/r \,+\,  sM^*(n_1,rn_2,n_3\mid \ell/t) \big).
\]
\end{theorem}
\begin{proof}
Let $g=\lceil \ell/t\rceil$.
Let $A$ be an $n_1\times n_2$ matrix and $B$ be an $n_2\times n_3$ matrix, where all matrix entries are in $[\ell]\cup\{\infty\}$.
We describe an algorithm to compute the Min-Plus product of $A$ and $B$.
Without loss of generality, we may assume that $(A_{ik}\bmod g) < g/2$ for all $i,k$ with $A_{ik}$ finite, since we can separate the problem into two instances $(A^<,B)$ and $(A^\ge,B)$ for
two matrices $A^<$ and $A^\ge$, where
$A^<_{ik} = A_{ik}$ and $A^{\ge}_{ik}=\infty$ if $(A_{ik}\bmod g) < g/2$,
and $A^{<}_{ik} = \infty$ and $A^{\ge}_{ik}=A_{ik}-g/2$ if $(A_{ik}\bmod g) \ge g/2$.
Similarly, we may assume that $(B_{kj}\bmod g) < g/2$ for all $k,j$ with $B_{kj}$ finite.

For each $i,k$, write $A_{ik}$ as $A'_{ik}g + A''_{ik}$
with $0 \le A'_{ik} \le t$ and $0 \le A''_{ik} < g/2$.  
Similarly, for each $k,j$, write $B_{kj}$ as $B'_{kj}g + B''_{kj}$
with $0 \le B'_{kj} \le t$ and $0 \le B''_{kj} < g/2$.
(Set $A'_{ik}=A''_{ik}=\infty$ if $A_{ik}=\infty$, and $B'_{kj}=b''_{kj}=\infty$ if $B_{kj}=\infty$.)

We first compute the Min-Plus product $C'$ of $A'$ and $B'$ (i.e., $C'_{ij} = \min_{k} (A'_{ik}+B'_{kj})$), in time
$O(M^*(n_1,n_2,n_3\mid t)) \le O( (n_2/s)\cdot M^*(n_1,s,n_3\mid t))$.

Let $W_{ij}=\{k\in [n_2]: A'_{ik}+B'_{kj}=C'_{ij}\}$; the elements in $W_{ij}$
are the \emph{witnesses} for $C'_{ij}$.
The Min-Plus product $C$ of $A$ and $B$ is given by $C_{ij} = C'_{ij}g + C''_{ij}$, where
\[ C''_{ij}\: :=\: \min_{k\in W_{ij}} (A''_{ik} + B''_{kj}).
\]
It suffices to describe how to compute $C''$.  We divide into two cases:

\begin{itemize}
\item
{\bf Few-witnesses case: computing $C''_{ij}$ for all $i,j$ with $|W_{ij}|\le n_2/s$.}
For each such $(i,j)$, we will explicitly enumerate all witnesses in $W_{ij}$.
This can be done by standard techniques for witness finding~\cite{AlonGMN92, seidel1995}:
first, observe that if the witness is unique (i.e., $|W_{ij}|= 1$), it can be found
by performing $O(\log n_2)$ Min-Plus products (namely,
for each $\ell\in [\log n_2]$, the $\ell$-th bit of
the witness for $C'_{ij}$ is 1 iff
$\min_{k\in K_\ell} (A'_{ik}+B'_{kj})=C_{ij}$, where $K_{\ell}:=\{k\in [n_2]:
\mbox{the $\ell$-th bit of $k$ is 1}\}$).
We take a random subset $R\subseteq [n_2]$ of $s$ indices,
and find witnesses for the Min-Plus product of
$(A'_{ik})_{i\in [n_1],k\in R}$ and 
$(B'_{kj})_{k\in R,j\in [n_3]}$ if the witnesses are unique.
This takes $\OO(M^*(n_1,s,n_3\mid t))$ time.
Fix $i,j$ with $|W_{ij}|\le n_2/s$.
For a fixed element $w\in W_{ij}$, the probability that $w$ is found,
i.e., $w$ is in $R$ but no
other element of $W_{ij}$ is in $R$, is $\Omega((s/n_2)\cdot (1-s/n_2)^{n_2/s})
=\Omega(s/n_2)$.  By repeating $O((n_2/s)\log(n_1n_2n_3))$ times (with
different choices of $R$),
all witnesses in $W_{ij}$ would be found w.h.p.
Once the entire witness set $W_{ij}$ is found, we can compute each $C''_{ij}$ naively in 
$O(|W_{ij}|)$ time.
The total running time is $\OO((n_2/s)\cdot M^*(n_1,s,n_3\mid t))$.


\item
{\bf Many-witnesses case: computing $C''_{ij}$ for all $i,j$ with $|W_{ij}| > n_2/s$.}
Pick a random subset $H$ of size $c_0s\log(n_1n_2n_3)$ for a sufficiently large constant~$c_0$.  Then $H$ hits (i.e., intersects) every witness set $W_{ij}$ with $|W_{ij}|>n_2/s$ w.h.p.

We do the following: for each $k_0\in H$ and
for each $i\in[n_1],j\in [n_3]$, if 
$A'_{ik_0}+B'_{k_0j}=C'_{ij}$ (i.e., $k_0\in W_{ij}$), set
\begin{equation}\label{eqn:eqprod}
C''_{ij}\ = \min_{k:\, A'_{ik}-A'_{ik_0}=B'_{k_0j}-B'_{kj}} (A''_{ik} + B''_{kj}).
\end{equation}
Correctness of (\ref{eqn:eqprod}) follows immediately from ``Fredman's trick'': $A'_{ik}-A'_{ik_0}=B'_{k_0j}-B'_{kj}$ is equivalent to $A'_{ik}+B'_{kj}=A'_{ik_0}+B'_{k_0j}$, which is equivalent to $k\in W_{ij}$, assuming
$A'_{ik_0}+B'_{k_0j}=C'_{ij}$.
Thus, the above correctly computes $C''_{ij}$ for every $i,j$ with $|W_{ij}|>n_2/s$, since
$H$ hits $W_{ij}$.

Finally, we observe that for a fixed $k_0$,
the right-hand side in (\ref{eqn:eqprod}) corresponds
precisely to a generalized equality product!  By
Lemma~\ref{lem:geneqprod}, they can be computed in
$\OO(n_1n_2n_3/r + M^*(n_1,rn_2,n_3\mid g))$ time, for each of
the $\OO(s)$ choices of $k_0$.
\end{itemize}

\negbigskip
\end{proof}

Note that Fredman's trick was originally introduced to solve APSP or compute Min-Plus products for arbitrary \emph{real}-valued inputs.  It is interesting that the trick is useful even when input values are in a restricted integer range
(in $[t]$).

Note also that a more naive attempt to prove the above theorem is to just bound $M^*(n_1,n_2,n_3\mid \ell)$
by $(n_2/s)M^*(n_1,s,n_3\mid \ell)$,
and once the inner dimension $n_2$ is reduced to $s$ in the subproblems, 
try to use hashing to reduce the range of the integers to, say, $[\OO(s^2)]$.  However, while such a hashing approach
might work for equality-type problems (e.g., \AEExactTri{}), it does not work at all
for \MinPlus{}.


\subsection{Consequences}

In Theorem~\ref{thm:main}, we can directly bound the third term by using existing 
matrix multiplication results~\cite{AlonGMN92}, leading to the following corollary:

\begin{corollary}\label{cor:intapsp0}
For any constant $0<\beta\le (3-\omega)/2$,
if $M^*(n,n^\beta,n\mid n^{2\beta})=O(n^{2+\beta-\eps})$,
then $M^*(n,n,n\mid n^{3-\omega})=O(n^{3-\Omega(\eps)})$.
\end{corollary}
\begin{proof}
By Theorem~\ref{thm:main},
\begin{eqnarray*}
 M^*(n,n,n\mid n^{3-\omega}) &=& \OO\left((n/s)M^*(n,s,n\mid t) + sn^3/r + (sn^{3-\omega}/t) M(n,rn,n)\right)\\
&\le&  \OO\left((n/s)M^*(n,s,n\mid t) + sn^3/r + rsn^3/t\right).
\end{eqnarray*}
Setting $r=n^\beta$, $s=n^{\beta-\eps'}$, and $t=n^{2\beta}$ with $\eps'=\eps/2$ yields $M^*(n,n,n\mid n^{3-\omega})=O(n^{3-\Omega(\eps)})$.
\end{proof}


The above corollary
establishes
a conditional lower bound of $n^{2+\beta-\eps}$ for
 the subproblem
of Min-Plus product for rectangular matrices of dimension
$n\times n^\beta$ and $n^\beta\times n$ for integers bounded by $n^{2\beta}$, under the \StrongAPSP{}.
This lower bound is tight in the sense that $O(n^{2+\beta})$ is
an obvious upper bound (though the range of allowed integer values $[n^{2\beta}]$ may not
be tight).  We will now use this corollary to derive conditional
lower bounds for \uAPSP{}.


Let $\TunwtdirAPSP(n)$ be the time complexity of \uAPSP{} on $n$-node graphs.
More generally, let $\TunwtdirAPSP(n,m)$ be the time complexity of APSP on an unweighted directed graph with $n$ nodes and $m$ edges.
Chan, Vassilevska W., and Xu~\cite{CVXicalp21} have given
a simple reduction of Min-Plus product for rectangular matrices of certain inner dimensions and
integer ranges to \uAPSP{}, as summarized by the following lemma
(it is easy to check that the graph in their reduction has $O(nx)$ edges).

\begin{lemma}\label{lem:reduce:from:minplus}
For any $x,y$, we have
$M^*(n,x,n\mid y)= O(\TunwtdirAPSP(n,nx))$ if $xy\le n$.
\end{lemma}

Combining Corollary~\ref{cor:intapsp0} and Lemma~\ref{lem:reduce:from:minplus} immediately gives the following:

\begin{corollary}\label{cor:strong-intapsp-imply}
For any constant $\beta\le \min\{1/3, (3-\omega)/2\}$, if $\TunwtdirAPSP(n,n^{1+\beta})=O(n^{2+\beta-\eps})$, 
then $M^*(n,n,n\mid n^{3-\omega}) = O(n^{3-\Omega(\eps)})$.
\end{corollary}

By setting $\beta=1/3$, we have thus proved that \uAPSP{} cannot be solved
in $O(n^{7/3-\eps})$ time under the \StrongAPSP{}, assuming $1/3\le (3-\omega)/2$ (this assumption can be removed, as we observe later in Section~\ref{sec:uapsp-lower-bound}).  In particular, if $\omega=2$, this implies that
 \uAPSP{} is strictly harder than unweighted undirected APSP,
as the latter problem can be solved in $\OO(n^\omega)$ time~\cite{seidel1995}.

Furthermore, \uAPSP{} for a graph with $n^{1+\beta}$ edges cannot be solved in $O(n^{2+\beta-\eps})$ time for any $\beta\le 1/3$ under the same hypothesis, assuming $1/3\le (3-\omega)/2$ (again this assumption can be removed).  In other words, the naive algorithm by repeated BFSs is essentially \emph{optimal} for sufficiently sparse graphs.

If we assume a weaker hypothesis that APSP does not
have truly subcubic algorithms for edge weights in $[n]$ 
instead of $[n^{3-\omega}]$, it can be checked that we still
get a lower bound near $n^{2+(3-\omega)/3}$ for \uAPSP{}.
In fact, assuming that APSP does not
have truly subcubic algorithms for edge weights in $[n^\lambda]$,
we can still obtain a super-quadratic lower bound for \uAPSP{} for
$\lambda$ as large as 1.99, if $\omega=2$.
For simplicity, we will concentrate only on the version of the \StrongAPSP{} with $\lambda=3-\omega$ throughout the paper.

In Sections~\ref{sec:intapsp-lower-bound:more}--\ref{sec:uapsp-lower-bound}, we will use the same approach to derive further conditional lower bounds for \MinWitness{}, \APLSP{}, and \BatchMode{},
from both the \StrongAPSP{} and the \uAPSPH{}.
