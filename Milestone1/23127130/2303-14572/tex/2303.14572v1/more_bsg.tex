\section{Bounded-Difference or Monotone Min-Plus Product from the Triangle Decomposition Theorem}
\label{app:bd:diff}

In this appendix, we note that our Triangle Decomposition Theorem (Theorem~\ref{thm:tri:decompose})
implies a truly subcubic algorithm for bounded-difference Min-Plus product. 
Existing algorithms \cite{BringmannGSW16, WilliamsX20,GuPWX21, ChiDXsoda22, ChiDXZstoc22} are faster, but our presentation is simpler, and so is interesting from the pedagogical perspective in our opinion.
In a way, the Triangle Decomposition Theorem clarifies conceptually why a subcubic algorithm is possible (if we don't care too much about optimizing the exponent in the running time).



\begin{theorem}
There is a truly subcubic algorithm for computing the Min-Plus product of two integer $n\times n$
matrices $A$ and $B$ satisfying the \emph{bounded difference property}, i.e., $|A_{i,k+1}-A_{ik}|\le c_0$ for all $i,k$
and $|B_{k+1,j}-B_{kj}|\le c_0$ for all $k,j$ for some constant $c_0$.
\end{theorem}
\begin{proof}
\newcommand{\CCC}{\tilde{C}}
Let $C=A\star B$ denote the matrix that we want to compute.
Let $\ell$ be a parameter, and let $D$ be the set of all indices in $[n]$ that are divisible by $\ell$.
Let $\ell'=2c_0\ell+1$.
We first compute $\CCC_{ij}=\min_{k\in D} (\up{A_{ik}/\ell'} + \up{B_{kj}/\ell'})$ for every $i,j\in [n]$.
By brute force, this takes $O(n^3/\ell)$ time.
Note that $\CCC_{ij}\ge C_{ij}/\ell'$.
For each $r\in [\ell]$, $i,j\in[n]$ and $k \in D$,
let $A_{ik}^{(r)} = A_{i,k+r} - \up{A_{ik}/\ell'}\ell'$ and
$B_{kj}^{(r)} = B_{k+r,j} - \up{B_{kj}/\ell'}\ell'$.  Note that $A_{ik}^{(r)},B_{kj}^{(r)}\in [\pm O(c_0\ell)]$.

To compute $C=A\star B$, we use the following formula:
\begin{eqnarray*}
 C_{ij} &=& \min_{k\in D,\, r\in[\ell]} (A_{i,k+r} + B_{k+r,j})\\
        &=& \min_{\Delta\in\{0,1,2\}} \left((\CCC_{ij} + \Delta)\ell'+ \min_{r\in[\ell],\, k\in D:\, \up{A_{ik}/\ell'} + \up{B_{kj}/\ell'} = \CCC_{ij}+\Delta}
(A_{ik}^{(r)} + B_{kj}^{(r)})  \right).
\end{eqnarray*}
The reason for restricting the range of $\Delta$ to $\{0,1,2\}$ is this:
if $\up{A_{ik}/\ell'} + \up{B_{kj}/\ell'} \ge \CCC_{ij}+3$,
then $A_{ik}/\ell' + B_{kj}/\ell' \ge \CCC_{ij}+1$, and so
$A_{i,k+r}+B_{k+r,j} \ge A_{ik}+B_{kj} - 2c_0\ell > \CCC_{ij}\ell'$; but
we know that the true value of $C_{ij}$ is at most $\CCC_{ij}\ell'$.

To evaluate the above formula, let's fix $\Delta\in \{0,1,2\}$.
We want to compute 
\[ C'_{ij}\ := \min_{r\in [\ell],\, k\in D:\, \up{A_{ik}/\ell'} + \up{B_{kj}/\ell'} = \CCC_{ij}+\Delta}
(A_{ik}^{(r)} + B_{kj}^{(r)}).
\]
Initialize $C'_{ij}$ to $\infty$.
Consider the tripartite graph with nodes $\{u_i:i\in[n]\}$, $\{x_k:k\in D\}$,
and $\{v_j:j\in[n]\}$, where $u_ix_k$ has weight $\up{A_{ik}/\ell'}$, and 
$x_kv_j$ has weight $\up{B_{kj}/\ell'}$, and $u_iv_j$ has weight $-\CCC_{ij}-\Delta$.
Apply Theorem~\ref{thm:tri:decompose} to get subgraphs $G^{(\lam)}$ and a set $R$
in $\OO(n^3/s+s^2n^2)$ time.

We first examine each triangle $u_ix_kv_j\in R$ and each $r\in[\ell]$ and reset $C'_{ij}$
to $A_{ik}^{(r)} + B_{kj}^{(r)}$ if it is smaller than the current value.
This takes $\OO(\ell \cdot (n^3/\ell)/s) = \OO(n^3/s)$ time.

Next, for each $\lam$ and each $r\in[\ell]$, we compute $\min_{k\in D:\, u_ix_k,x_kv_j\in G^{(\lam)}} (A'_{ik} +B'_{kj})$ and 
reset $C'_{ij}$ to this value if it is smaller, for every $u_iv_j\in G^{(\lam)}$.  
This reduces to a Min-Plus product instance on integers in $[\pm O(c_0\ell)]$
and takes $\OO(c_0\ell n^\omega)$ time for each $\lam$ and $r$.  The total time
is $\OO(s^3\cdot \ell \cdot c_0\ell n^\omega) = \OO(c_0\ell^2 s^3 n^\omega)$.

The overall running time is $\OO(n^3/\ell + n^3/s + c_0\ell^2 s^3 n^\omega)$.
Setting $\ell=s=n^{(3-\omega)/6}$ gives a bound of $\OO(n^{(15+\omega)/6})$ for constant $c_0$
(which is improvable by using rectangular matrix multiplication).
\end{proof}

The above algorithm easily extends to the case where we allow $O(n^{2-\delta})$ exceptional pairs $(i,k)$ to
violate the bounded difference property $|A_{i,k+1}-A_{ik}|\le c_0$, and similarly  $O(n^{2-\delta})$ exceptional pairs $(k,j)$ to
violate the bounded difference property $|B_{k+1,j}-B_{kj}|\le c_0$.  We just need to add an extra cost of
$O(\ell n^{2-\delta}\cdot n)$.  The total time bound $\OO(\ell n^{3-\delta} + n^3/\ell + n^3/s + 
c_0\ell^2 s^3 n^\omega)$ remains truly subcubic for an appropriate choice of $\ell$ and $s$.

The case of matrices with monotone rows/columns and integer entries in $[n]$ easily reduce to the bounded difference case with a nonconstant $c_0=n^\delta$ and $O(n^{2-\delta})$ exceptional pairs.  So, we also get a truly subcubic
algorithm (with an appropriate choice of $\delta$ and a slightly worse final exponent) for the monotone case.  





\section{More on BSG}\label{app:bsg}

In this appendix, we show that the proof of
Theorem~\ref{thm:bsg:simple} can be modified to hold not just for
indexed sets, but also for arbitrary subsets $A$ and $C$ of an abelian group, if we ignore the construction time.
This requires a more clever argument in the ``low-frequency'' case.
(The proof of Theorem~\ref{thm:bsg:gower} can also be modified in a similar way.)

\begin{theorem} {\bf (Simpler BSG Covering)}\ \ \label{thm:bsg:cover:simple:app}
Given subsets $A$ and $C$ of size $n$ of an abelian group and a parameter $s$,
there exist a collection of $\ell=\OO(s^3)$ subsets $A^{(1)},\ldots,A^{(\ell)}\subseteq A$, and a set $R$ of $\OO(n^2/s)$ pairs in $A\times A$, such that
\begin{enumerate}
\item[\rm(i)] $\{(a,b)\in A\times A: a-b\in C\}\ \subseteq\
R\,\cup\, \bigcup_\lam (A^{(\lam)}\times A^{(\lam)})$, and
\item[\rm(ii)] $\sum_\lam |A^{(\lam)} - A^{(\lam)}| = \OO(s^2 n^{3/2})$.
\end{enumerate}
\end{theorem}
\begin{proof}\

\begin{itemize}
\item {\bf Few-witnesses case.}
First add $\{(a,b)\in A\times A:\ a-b\in C,\ \pop_A(a-b)\le n/s\}$ to $R$.
The number of pairs added to $R$ is $\OO(n\cdot n/s)$.

\item {\bf Many-witnesses case.}
Let $F =\{h: \pop_A(h) > n/r\}$.
Then $|F|=O(rn)$, since the total popularity is $n^2$.
By adding extra elements to $F$, we may assume that $|F|=\Theta(rn)$.
Pick a random subset $H\subseteq F$ of size $c_0sr\log n$ for a sufficiently large constant $c_0$.

\begin{itemize}
\item {\bf Low-frequency case.}
Add the following to $R$:
\[ \{(a,b)\in A\times A:\ |\{(a',b')\in A\times A: a-a'=b-b'\not\in F\}| > n/(2s)\}.
\]
Since for each $(a,a')$ with $a-a'\not\in F$, there are at most $n/r$ choices of $(b,b')$ satisfying $a-a'=b-b'$,
the number of pairs added to $R$ is $\OO(\frac{n^2\cdot n/r}{n/(2s)})=\OO(n^2/s)$
by choosing $r:=s^2$.
\item {\bf High-frequency case.}
For each $h\in H$,
add the following subset to the collection:
\[ A^{(h)} = \{a\in A:\ a-h\in A\}.
\]
The number of subsets is $\OO(sr)=\OO(s^3)$.
\end{itemize}
\end{itemize}

\emph{Correctness.}
To verify (i), consider a fixed pair $(a,b)\in A\times A$ with $a-b\in C$.
If $\pop_A(a-b)\le n/s$, then $(a,b)\in R$ due to
the ``few-witnesses'' case.
So assume $\pop_A(a-b)>n/s$.
Furthermore, assume that $|\{(a',b')\in A\times A: a-a'=b-b'\not\in F\}| \le n/(2s)$,
for otherwise $(a,b)\in R$ due to the ``low-frequency'' case.
There are at least $n/s$ pairs $(a',b')\in A\times A$ with $a'-b'=a-b$,
which also satisfy $a-a'=b-b'$ by Fredman's trick.  Among them,
there are at least $n/(2s)$ pairs $(a',b')\in A\times A$ with $a-a'=b-b'\in F$.
For $h=a-a'=b-b'$, we have $(a,b)\in A^{(h)}\times A^{(h)}$.
So, the probability that $(a,b)\in A^{(h)}\times A^{(h)}$ for a random $h\in F$
is $\Omega(\frac{n/(2s)}{rn})=\Omega(1/(sr))$.
Thus, $(a,b)\in \bigcup_{h\in H} (A^{(h)}\times A^{(h)})$ w.h.p.\ for
a random subset $H\subset F$ of size $c_0sr\log n$.

To verify (ii), consider a fixed $h$.
The set $A^{(h)}-A^{(h)}$ is equal to
$\{c:\ \exists a,b,a',b'\in A,\ c=a-b,\ a-h=a',\ b-h=b'\}$.
Let $Y_c = \{a\in A:\ a-c\in A\}$.
Then $A^{(h)}-A^{(h)}$ is contained in
$\{c: \exists a\in Y_c\ \mbox{and}\ a-h\in Y_c\}$.
The expected size of $A^{(h)}-A^{(h)}$ for a random $h\in F$ is thus bounded by
\[ \sum_c \min\{ |Y_c|\cdot |Y_c|/|F|,\ 1 \}\ \le\ O(n^2/\sqrt{|F|})
\ =\ O(n^{3/2}/\sqrt{r}),
\]
since $\sum_c |Y_c| = O(n^2)$.

The expected total sum $\sum_{h\in H}  |A^{(h)}-A^{(h)}|$ is bounded by $\OO(sr\cdot n^{3/2}/\sqrt{r}) = \OO(s^2n^{3/2})$.
\end{proof}

Chan and Lewenstein~\cite{ChanLewenstein} posed the following interesting combinatorial question:
\begin{quote}
    Given sets $A,B,C$ in an abelian group of size $n$,
    we want to cover $\{(a,b)\in A\times B: a+b\in C\}$ by bicliques $A^{(\lam)}\times B^{(\lam)}$, so as to minimize the cost function $\sum_\lam |A^{(\lam)}+B^{(\lam)}|$.
    Prove worst-case bounds on the minimum cost as a function of $n$.
\end{quote}
They observed that Theorem~\ref{thm:BSG0:cover} implies an $O(n^{13/7})$ upper bound for this problem.  Theorem~\ref{thm:bsg:cover:simple:app} implies an improved $\OO(n^{11/6})$ upper bound
(as we can use ``singleton'' bicliques to cover $R$
and choose $s$ to minimize $\OO(n^2/s+s^2n^{3/2})$).



If we just want an analog of the BSG Theorem that extracts a single subset rather than constructs a cover  (in the style of Theorem~\ref{thm:BSG0}), the proof of the above theorem  becomes even simpler (the ``few-witnesses'' and ``low-frequency'' cases may be skipped):

\begin{theorem}\label{thm:bsg:simpler} {\bf (Simpler BSG)}\ \ 
Given subsets $A$ and $C$ of size $n$ of an abelian group and a parameter $s$,
if $|\{(a,b)\in A\times A: a-b\in C\}|\ge n^2/s$,
then there exists a subset $A'\subseteq A$ of size $\Omega(n/s)$,
such that 
\[ |A'-A'| \:=\: O(s^{1/2}n^{3/2}).\]
\end{theorem}
\begin{proof}
Let $F =\{x: \pop_A(x) > n/(2s)\}$.  
Then $|F|\ge n/(2s)$, because otherwise,
$\sum_{c\in C}\pop_A(c) < |F|n + n^2/(2s) < n^2/s$, contradicting the
stated assumption.

Pick a random $h\in F$ and let $A^{(h)} = \{a\in A:\ a-h\in A\}$ as before.
Then $|A^{(h)}|=\pop_A(h) > n/(2s)$.

By the same argument as before,
the expected size of $A^{(h)}-A^{(h)}$ for a random $h\in F$ is 
bounded by $O(n^2/\sqrt{|F|})$, which is now $O(s^{1/2}n^{3/2})$.
\end{proof}

The above $O(s^{1/2}n^{3/2})$ bound is better than the
known $O(s^5n)$ bound (Theorem~\ref{thm:BSG0}) when
$s\gg n^{1/9}$, though the latter holds only for the bichromatic sum set setting.\footnote{
Bounds of the form $O(s^{O(1)}n)$ were also known in the setting of monochromatic difference sets, but direct comparisons require care: many previous work such as \cite{Gowers01,Schoen} started from a related but different assumption, namely, that
the \emph{energy} $|\{(a,b,a',b')\in A\times A\times A\times A: a+b=a'+b'\}|$
is at least $n^3/s'$ for some parameter $s'$.  This parameter $s'$ is not identical to the parameter $s$ in Theorem~\ref{thm:bsg:simpler}, though they are related polynomially (or quadratically).
}

Along the same lines, we can obtain the following combinatorial result from the Triangle Decomposition Theorem, which might be of independent interest:

\begin{theorem}
Given a real-weighted tripartite graph $G$ with $n$ nodes, and given a parameter $s$,
if $G$ contains $\Omega(n^3/s)$ zero-weight triangles,
then there exists a subgraph $G'$ with $\OOmega(n^3/s^4)$ triangles (and thus
$\OOmega(n^2/s^4)$ edges)
such that all triangles in $G'$ are zero-weight triangles.
\end{theorem}
\begin{proof}
Apply Theorem~\ref{thm:tri:decompose} with $s$ replaced by $s'$.
The size of $R$ is $\OO(n^3/s')$, which can be made less than $n^3/(2s)$ for some choice of  $s'=\widetilde{\Theta}(s)$.
Then there exists $G^{(\lam)}$
with $|\Triangles(G^{(\lam)})|=\OOmega((n^3/s)\cdot (1/s^3))$.
\hspace*{\fill}
\end{proof}
