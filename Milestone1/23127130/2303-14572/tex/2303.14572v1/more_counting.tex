
\subsection{Exact \texorpdfstring{$k$}{k}-Clique and Minimum \texorpdfstring{$k$}{k}-Clique}

Let $G$ be the input graph of an \ExactKClique{} or \ExactKCliqueCount{} instance, and let $w$ be the weight function of the graph. For every set $I \subseteq V(G)$ of size $k-1$, we use $W_I$ to denote the set of $j$ where $I \cup \{j\}$ forms a $k$-clique whose edge weights sum up to the required value $t$. 

\begin{theorem}
\label{thm:exact-k-clique-counting}
If \ExactKClique{} for $n$-node graphs has an $O(n^{k-\eps})$ time algorithm for some $\eps > 0$, then \ExactKCliqueCount{} for $n$-node graphs  has an $O(n^{k-\eps'})$ time algorithm for some $\eps' > 0$
\end{theorem}
\begin{proof}
Similar as before, by well-known techniques \cite{focsyj}, given a \ExactKCliqueCount{} instance on a graph $G$ with $n$ nodes and a target $t$, we can use the $O(n^{k-\eps})$ time algorithm for \ExactKClique{} to list up to $n^{0.99}$ witnesses for every set $I$ of $k-1$ nodes, in $O(n^{k-\eps''})$ time for some $\eps'' > 0$. 

Then we enumerate all possible subsets $J \subseteq V(G)$ of size $k-3$. For each $J$, we can reduce the problem of counting witnesses for sets $I \supseteq J$ of size $k-1$ to a \AEExactTriCount{} instance $(G', t')$ in a standard way \cite{Nesetril1985}. The set of nodes of $G'$ corresponds to $V(G) \setminus J$. Let the edge weight between $i_1$ and $i_2$ be $w'(i_1, i_2) = 2w(i_1, i_2) + \sum_{j \in J} (w(j, i_1) + w(j, i_2))$. Also, let $t'$ be $2t-2\sum_{\substack{j_1, j_2 \in J \\ j_1 < j_2}} w(j_1, j_2)$. It is not difficult to verify that $(i_1, i_2, i_3)$ forms an exact triangle in $G'$ if and only if $J \cup \{i_1, i_2, i_3\}$ forms an exact $k$-clique in $G$. Thus, the number of witnesses of $(i_1, i_2)$ in $G'$ equals the number of witnesses of $J \cup \{i_1, i_2\}$ in $G$. We then proceed similar to the proof of Theorem~\ref{thm:exact-tri-count}. If the number of witnesses of $(i_1, i_2)$ in $G'$ is less than $n^{0.99}$, then we have already listed all of their witnesses in the $O(n^{k-\eps''})$ time step. For $(i_1, i_2)$ that has at least $n^{0.99}$ witnesses, we can find a set $S$ of size $\OO(n^{0.01})$ that intersects with each of $W_{i_1, i_2}$ in $\OO(n^{2.99})$ time, and then apply Lemma~\ref{lem:exact-tri} to compute the witness count for these $(i_1, i_2)$ pairs in $\OO(|S| \cdot n^{(3+\omega)/2}) \le O(n^{2.70})$ time. 

Finally, summing up $W_I$ for every distinct set $I$ of $k-1$ nodes gives the total exact triangle count of $G$. The overall running time for the \ExactKCliqueCount{} instance is thus $\OO(n^{k-\eps''} + n^{k-3} \cdot (n^{2.99} + n^{2.70})) = \OO(n^{k-\min\{\eps'', 0.01\}})$. 
\end{proof}

We can similarly show that \MinKCliqueCount{} reduces to \MinKClique{}. Since the proof is essentially the same, we omit the proof of the following theorem for conciseness. 
\begin{theorem}
\label{thm:min-k-clique-counting}
If \MinKClique{} for $n$-node graphs has an $O(n^{k-\eps})$ time algorithm for some $\eps > 0$, then \MinKCliqueCount{} for $n$-node graphs  has an $O(n^{k-\eps'})$ time algorithm for some $\eps' > 0$
\end{theorem}

We then reduce \MinKClique{} to \MinKCliqueCount{}.
\begin{theorem}
\label{thm:min-k-clique-counting-rev}
If \MinKCliqueCount{} for $n$-node graphs has an $O(n^{k-\eps})$ time algorithm for some $\eps > 0$, then \MinKClique{} for $n$-node graphs  has an $O(n^{k-\eps'})$ time algorithm for some $\eps' > 0$
\end{theorem}
\begin{proof}
Let $G$ be the input graph for a \MinKClique{} instance. Without loss of generality, we can assume $G$ is a $k$-partite graph on node parts $V_1 \cup \cdots \cup V_k$. 

We first multiply all the edge weights of $G$ by a large enough number $M \ge 10 k \cdot n^k$. Then for each $i \in [k]$, $v_i \in V_i$, we add $v_i \cdot n^{i-1}$ to all edges adjacent to $v_i$. After these transformations, there will be a unique minimum weight $k$-clique in the graph. Furthermore, this $k$-clique must also be a minimum weight $k$-clique in the original graph. We denote this minimum weight $k$-clique by $(u_1, \ldots, u_k) \in V_1 \times \cdots \times V_k$. 

Then for each $i \in [k]$ and each $p \in \left[\lceil \log(n) \rceil \right]$, we do the following. Let $G^{(i, p)}$ be a copy of the graph (after the weight changes), and we duplicate all nodes $v \in V_i$ whose $p$-th bit in its binary representation is $1$. 
We use the assumed \MinKCliqueCount{} algorithm the count the number of minimum weight $k$-cliques in in $G^{(i, p)}$. 
If the number of minimum weight $k$-clique in $G^{(i, p)}$ is $2$, then we know the $p$-th bit of $u_i$ is $1$; otherwise, the $p$-th bit of $u_i$ is $0$. 

After all $k \lceil \log(n) \rceil$ rounds, we can recover $(u_1, \ldots, u_k)$, and thus compute the weight of the minimum weight $k$-clique in the original graph. 
\end{proof}

\subsection{Monochromatic Convolution}

\begin{theorem}
\label{thm:mono-conv-count}
If \MonoConv{} for  length $n$ arrays has an $O(n^{1.5-\eps})$ time algorithm for some $\eps > 0$, then \MonoConvCount{} for length $n$ arrays  has an $O(n^{1.5-\eps'})$ time randomized algorithm for some $\eps' > 0$
\end{theorem}
\begin{proof}
If \MonoConv{} has an $O(n^{1.5-\eps})$ time algorithm, then \ThreeSUM{} has a truly subquadratic time algorithm~\cite{lincoln2020monochromatic}, and consequently \AllThreeSUM{} has a truly subquadratic time algorithm~\cite{focsyj}. Furthermore, by Theorem~\ref{thm:all-3sum-count}, \AllThreeSUMCount{} has an $O(n^{2-\eps''})$ time algorithm for some $\eps'' > 0$. 

Thus, it suffices to reduce \MonoConvCount{} to \AllThreeSUMCount{}. Lincoln, Polak, and Vassilevska W.~\cite{lincoln2020monochromatic} showed that a truly subquadratic time algorithm for \AllThreeSUM{} implies a truly sub-$n^{1.5}$ time algorithm for \MonoConv{}. It is not difficult to check that their reduction preserves the number of solutions, and thus works for the counting versions as well. 
\end{proof}


\subsection{All-Pairs Shortest Paths}
\label{sec:apsp-count-mod}

\begin{theorem}
\label{thm:apsp-count-mod}
If \APSP{} for $n$-node  graphs with positive edge weights has an $O(n^{3-\eps})$ time algorithm for some $\eps > 0$, then \APSPCountMod{U} for $n$-node  graphs with positive edge weights has an $O(n^{3-\eps'})$ time algorithm for some $\eps' > 0$, for any $\OO(1)$-bit integer $U \ge 2$.
\end{theorem}
\begin{proof}

Let $(A, A')$ and $(B, B')$ be two pairs of $n \times n$ matrices where the entries of $A'$ and $B'$ are $\OO(1)$-bit integers. We define two ``funny'' matrix products (one of them was defined in the proof of Theorem~\ref{thm:apsp-count}). If $(C, C') = (A, A') \oplus (B, B')$, then $C_{ij} = \min(A_{ij}, B_{ij})$, and $C'_{ij} = [A_{ij} = C_{ij}]A'_{ij}+[B_{ij} = C_{ij}]B'_{ij}$, where $[\cdot]$ denotes the indicator function. Clearly, we can compute $(A, A') \oplus (B, B')$ in $\OO(n^2)$ time.

\begin{claim}
\label{cl:apsp-count-mod}
Suppose \APSP{} on $n$-node  graphs with positive edge weights has an $O(n^{3-\eps})$ time algorithm for some $\eps > 0$, then we can compute $(A, A') \otimes (B, B')$ in $O(n^{3-\eps''})$ time for some $\eps'' > 0$ where the entries of $A'$ and $B'$ are $\OO(1)$-bit integers. 
\end{claim}
\begin{proof}
If \APSP{} on $n$-node  graphs with positive edge weights has an $O(n^{3-\eps})$ time algorithm, then so does \MinPlus{} \cite{focsyj}. By Theorem~\ref{thm:minplus-count}, \MinPlusCount{} also has a truly subcubic time algorithm. It remains to reduce computing $(A, A') \otimes (B, B')$ to \MinPlusCount{}. 

For $p \in [O(\log n)]$, let $A^{(p)}$ be the matrix where $A^{(p)}_{ij} = A_{ij}$ if the $p$-th bit of $A'_{ij}$ is $1$, and $A^{(p)}_{ij} = \infty$ otherwise. We can similarly define $B^{(p)}$. Also, let $J$ be the $n \times n$ matrix whose  entries are all $1$. It is then not difficult to verify that
$$(A, A') \otimes (B, B') = \bigoplus_{p, q} (A^{(p)}, 2^p J) \otimes (B^{(q)}, 2^q J).$$
To compute each term in the above ``sum'', say $(C^{(p, q)}, C'^{(p, q)}) = (A^{(p)}, 2^p J) \otimes (B^{(q)}, 2^q J)$, we first use the assumed \MinPlus{} algorithm to compute $C^{(p, q)} = A^{(p)} \star B^{(q)}$ in $O(n^{3-\eps})$ time. Then $C'^{(p, q)}_{ij}$ is exactly the number of witnesses for $C^{(p, q)}$ in the previous Min-Plus product, multiplied by $2^{p+q}$, so we can use the truly subcubic time algorithm for \MinPlusCount{} to compute $C'^{(p, q)}$. 
\end{proof}


As showed in the proof of Theorem~\ref{thm:apsp-count}, \APSPCount{} reduces to $O(\log n)$ instances of the funny matrix product; also, now we can mod all entries in $A', B'$ by $U$ after each funny matrix product to keep them $\OO(1)$-bit integers. 
Thus, by Claim~\ref{cl:apsp-count-mod}, \APSPCountMod{U} has an $\OO(n^{3-\eps''})$ time algorithm assuming \APSP{} can be solved in truly subcubic time. 
\end{proof}


\begin{theorem}
\label{thm:apsp-count-mod-rev}
If \APSPCountMod{c} for $n$-node  graphs with positive edge weights has an $O(n^{3-\eps})$ time algorithm for some $\eps > 0$, where $c \ge 2$ is some $\OO(1)$-bit integer, 
then \APSP{} for $n$-node  graphs with positive edge weights has an $O(n^{3-\eps'})$ time algorithm for some $\eps' > 0$, 
\end{theorem}
\begin{proof}
Suppose there is an $O(n^{3-\eps})$ time algorithm for \APSPCountMod{c} for $n$-node graphs positive edge weights, then there is also an $O(n^{3-\eps})$ time algorithm for counting the number of witnesses modulo $c$ for \MinPlus{} for $n \times n$ matrices, by following the standard reduction from \MinPlus{} to \APSP{}~\cite{focsyj}. 

Then by essentially the same proof to the proof of Theorem~\ref{thm:minplus-count-rev}, there exists an $\OO(n^{3-\eps})$ time algorithm for \MinPlus{}. Finally, there is a truly subcubic time algorithm for \APSP{} since \APSP{} and \MinPlus{}  are subcubically equivalent \cite{focsyj}. 
\end{proof}