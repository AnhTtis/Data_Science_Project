


Continuing the approach in Section~\ref{sec:intapsp-lower-bound}, we now derive
conditional lower bounds for more problems with intermediate complexity from the \StrongAPSP{} (which as noted before 
is equivalent to the hypothesis
that $M^*(n,n,n\mid n^{3-\omega})$ is not truly subcubic).
We begin with a useful lemma:

\begin{lemma}\label{lem:down}
For any positive constants $\beta,\gamma,c$ with $0<\gamma<\beta$,
if $M^*(n,n^{\beta},n\mid n^{c\beta})=O(n^{2+\beta-\eps})$,
then $M^*(n,n^{\gamma},n\mid n^{c\gamma})=O(n^{2+\gamma-\Omega(\eps)})$.
\end{lemma}
\begin{proof}\

\negbigskip\negbigskip
\begin{eqnarray*}
M^*(n,n^\gamma,n\mid n^{c\gamma}) &\le & 
O(n^{2(1-\gamma/\beta)} M^*(n^{\gamma/\beta},n^\gamma,n^{\gamma/\beta}\mid n^{c\gamma}))\\
&\le &  \OO(n^{2(1-\gamma/\beta)}\cdot (n^{\gamma/\beta})^{2+\beta-\eps})
\ =\ O(n^{2+\gamma-\Omega(\eps)}).
\end{eqnarray*}

\negbigskip
\end{proof}



Lemma~\ref{lem:down} allows us to remove the assumption $\beta\le (3-\omega)/2$ in Corollary~\ref{cor:intapsp0} (since we can replace $\beta$ with any sufficiently small positive constant $\gamma$).  Consequently, Corollary~\ref{cor:strong-intapsp-imply}
holds for all $\beta\le 1/3$, regardless of the value of $\omega$.  For convenience, we repeat the statement of Corollary~\ref{cor:intapsp0} below, with the assumption removed:

\begin{corollary}\label{cor:intapsp}
For any constant $\beta>0$,
if $M^*(n,n^\beta,n\mid n^{2\beta})=O(n^{2+\beta-\eps})$,
then $M^*(n,n,n\mid n^{3-\omega})=O(n^{3-\Omega(\eps)})$.
\end{corollary}


\begin{remark}\label{rmk:uapsp}\rm
By applying Lemma~\ref{lem:down} with $\beta=1$, $\gamma=1/2$, and $c=1$, we see that
$M^*(n,n,n\mid n)=O(n^{3-\eps})$ implies
$M^*(n,\sqrt{n},n\mid\sqrt{n})=O(n^{5/2-\Omega(\eps)})$.  If $\omega=2$, Chan, Vassilevska W. and Xu~\cite{CVXicalp21} showed that the \uAPSPH{} is equivalent to the claim that $M^*(n,\sqrt{n},n\mid\sqrt{n})$ is not in $O(n^{5/2-\eps})$ for any $\eps>0$.  Consequently, the \uAPSPH{} implies the \StrongAPSP{} when $\omega=2$.
\end{remark}

\begin{remark}\label{rmk:strongapsp:dir}\rm
In the definition of the \StrongAPSP{}, it does not matter whether the input graph is undirected or directed---the directed version is also equivalent to the statement that
$M^*(n,n,n\mid n^{3-\omega})$ is not truly subcubic: 
Directed APSP for integer weights in $[n^{3-\omega}]$ can be 
solved by Zwick's algorithm~\cite{zwickbridge,CVXicalp21} in
time $\OO(\max_\ell M^*(n,n/\ell,n\mid \ell n^{3-\omega}))$.
If $M^*(n,n,n\mid n^{3-\omega})=O(n^{3-\eps})$, then for $\ell\le n^\delta$,
we have $M^*(n,n/\ell,n\mid \ell n^{3-\omega})\le \OO(M^*(n,n,n\mid n^{3-\omega+\delta}) \le O(n^{(3-\eps)(3-\omega+\delta)/(3-\omega)})=O(n^{3-\Omega(\eps)})$ for a sufficiently small $\delta$.  On the other hand,
for $\ell>n^\delta$, we trivially have  $M^*(n,n/\ell,n\mid \ell n^{3-\omega})\le O(n^3/\ell) = O(n^{3-\delta})$.  
\end{remark}




\subsection{Min-Witness Product}

Let $\Tminwit(n)$ be the time complexity of computing min-witness product
for two  $n\times n$ Boolean matrices.
More generally,
let $\Tminwit(n_1,n_2,n_3)$  be the time complexity of \MinWitness{}
for an $n_1\times n_2$ Boolean matrix and an $n_2\times n_3$ Boolean matrix.  We observe that 
Min-Plus product for rectangular matrices of sufficiently small inner dimensions and
integer ranges can be reduced to \MinWitness{}:

\begin{lemma}\label{lem:reduce:from:minplus:minwit}
For any $x,y$, we have
$M^*(n,x,n\mid y)= O(\Tminwit(n,xy^2,n))$.
\end{lemma}
\begin{proof}
Suppose that we are given an $n\times x$ matrix $A$
and an $x\times n$ matrix $B$ where all matrix entries are in $[y]$.
For each $i\in[n]$, $k\in [x]$, and $u,v\in [y]$, let 
$A'_{i,(k,u,v)}=1$ if $A_{ik}=u$, and $A'_{i,(k,u,v)}=0$ otherwise.
For each $j\in[n]$, $k\in [x]$, $u,v\in [y]$, let 
$B'_{(k,u,v),j}=1$ if $B_{kj}=v$, and $B'_{(k,u,v),j}=0$ otherwise.
The number of triples $\tau=(k,u,v)$ is $xy^2\le n$.  
By sorting the triples in increasing order of $u+v$,
 the Min-Plus product of $A$
and $B$ can be computed from the Min-Witness product of $(A'_{i,\tau})_{i\in[n],\tau\in [x]\times[y]^2}$ and $(B'_{\tau, j})_{\tau\in [x]\times[y]^2, j\in[n]}$. 
\hspace*{\fill}
\end{proof}

Combining Corollary~\ref{cor:intapsp} and Lemma~\ref{lem:reduce:from:minplus:minwit} immediately gives the following:

\begin{corollary}\label{cor:strong-intapsp-imply:minwit}
If $\Tminwit(n,n^{5\beta},n)=O(n^{2+\beta-\eps})$, then
$M^*(n,n,n\mid n^{3-\omega})= O(n^{3-\Omega(\eps)})$.
\end{corollary}

By setting $\beta=1/5$, we have thus proved that 
\MinWitness{} of two $n \times n$ Boolean matrices
cannot be computed in $O(n^{11/5-\eps})$ time under the \StrongAPSP{}.
In particular, if $\omega=2$,
this implies that \MinWitness{} is strictly harder than  Boolean matrix multiplication.

Furthermore, by setting $\beta=\gamma/5$, 
\MinWitness{} of an $n\times n^{\gamma}$ and an $n^{\gamma}\times n$ Boolean matrix cannot be computed in $O(n^{2+\gamma/5-\eps})$ time for any $\gamma$ under the same hypothesis.  This implies that for any $\gamma\le 0.3138$, \MinWitness{} is strictly harder than  Boolean matrix multiplication, as $\omega(1,0.3138,1)=2$~\cite{legallurr}.  This result interestingly rules out
the possibility that the polynomial method~\cite{Williams18,AbboudWY15,ChanW21} could be used to transform
the Min-Witness product of an $n\times d$ and a $d\times n$ matrix
into a standard product of an $n\times d^{O(1)}$ and a $d^{O(1)}\times n$ matrix.  It also contrasts \MinWitness{} with, for example, \Dominance{} or \Equality{},
which does have near-quadratic time complexity when the inner dimension $d$ is smaller than $n^{0.1569}$
(since as mentioned in Section~\ref{sec:prelim:eq:prod}, Matou\v sek's technique~\cite{MatIPL,YusterDom} yields a time bound 
of $\OO(\min_r (dn^2/r + M(n,dr,n))\le \OO(n^2 + M(n,d^2,n))$). 



\subsection{All-Pairs Shortest Lightest Paths}

Let $\TundirAPSLP(n,m)$ be the time complexity of \APSLP{} on graphs with $n$ nodes and $m$ edges.  We observe the following:

\begin{lemma}\label{lem:reduce:from:minplus:apslp}
For any $x,y$, we have
$M^*(n,x,n\mid y)=O(\TundirAPSLP(n,nx))$ if $xy^2\le n$.
\end{lemma}
\begin{proof}
Suppose that we are given an $n\times x$ matrix $A$
and an $x\times n$ matrix $B$ where all matrix entries are in $[y]$.
We construct an undirected graph as follows:
\begin{itemize}
\item For each $i\in [n]$, create a node $s[i]$.
\item For each $k\in [x]$ and $u\in [y]$, create a node $w_1[k,u]$.
\item For each $k\in [x]$, create a node $w_2[k]$.
\item For each $k\in [x]$ and $v\in [y]$, create a node $w_3[k,v]$.
\item For each $j\in [n]$, create a node $t[j]$.
\item For each $i\in [n]$ and $k\in [x]$, create an edge $s[i]w_1[k,u]$ of weight~1 where $u=a_{ik}$.
\item For each $k\in [x]$ and $u\in [y]$, create a
 path between $w_1[k,u]$ and $w_2[k]$
that has $u$ edges of weight 2 and $y-u$ edges of weight 1 (so that the path has $y$ edges and weight $y+u$).
\item For each $k\in [x]$ and $v\in [y]$, create a
 path between $w_2[k]$ and $w_3[k,v]$ 
that has $v$ edges of weight 2 and $y-v$ edges of weight 1 (so that the path has $y$ edges and weight $y+v$).
\item For each $j\in [n]$ and $k\in [x]$, create an edge $w_3[k,v]t[j]$ of weight~1 where $v=b_{kj}$.
\end{itemize}
The number of nodes in the graph is $O(n+xy^2)=O(n)$,
and the number of edges is $O(nx+xy^2)=O(nx)$.
For each $i,j\in [n]$, all lightest paths from $s[i]$ to $t[j]$
have $2y+2$ edges, and among them, the shortest has weight $2y+2+\min_k (a_{ik}+b_{kj})$.
\end{proof}

Combining Corollary~\ref{cor:intapsp} and Lemma~\ref{lem:reduce:from:minplus:apslp}
immediately gives the following:

\begin{corollary}\label{cor:strong-intapsp-imply:apslp}
For any constant $\beta\le 1/5$,
if $\TundirAPSLP(n,n^{1+\beta})=O(n^{2+\beta-\eps})$, then
$M^*(n,n,n\mid n^{3-\omega})= O(n^{3-\Omega(\eps)})$.
\end{corollary}

By setting $\beta=1/5$, we have thus proved that \APSLP{} cannot be solved
in $O(n^{11/5-\eps})$ time under the \StrongAPSP{}.  If $\omega=2$, this implies that
\APSLP{} is strictly harder than undirected APSP for such weighted graphs.  (The current best algorithm for \APSLP{} has running time $\OO(n^{2.58})$, or if $\omega=2$, $\OO(n^{5/2})$~\cite{CVXicalp21}.)  The same result holds for \APLSP{}.

Furthermore, \APSLP{} 
with $n^{1+\beta}$ edges cannot be solved in $O(n^{2+\beta-\eps})$ time for any $\beta\le 1/5$ under the same hypothesis.  Thus, the naive algorithm is essentially \emph{optimal} for sufficiently sparse graphs.

The same results hold for the similar problem of \APLSP{} (in the proof of Lemma~\ref{lem:reduce:from:minplus:apslp}, we just modify the path from $w_1[k,u]$ and $w_2[k]$
to use $y-u$ edges of weight 2 and $2u$ edges of weight 1, so that the
path has $y+u$ edges and weight $2y$).


\subsection{Batched Range Mode}


For a different application about data structures, 
we next consider  the range mode problem, which has received considerable attention and has been extensively studied in the literature \cite{KrizancMS05,CDLMW,WilliamsX20,GaoHe22,JinX22}.
Let $\Trangemode(N,Q\mid \sigma)$ be the total time complexity of \BatchMode{} for $Q$ range mode queries in an array of $N$ elements in $[\sigma]$.

\begin{lemma}\label{lem:reduce:from:minplus:mode}
For any $x,y$, we have
$M^*(n,x,n\mid y)= O(\Trangemode(nxy,n^2\mid x))$.
\end{lemma}
\begin{proof}
\newcommand{\ssigma}{\sigma'}
\newcommand{\ttau}{\tau'}
Suppose that we are given an $n\times x$ matrix $A$
and an $x\times n$ matrix $B$ where all matrix entries are in $[y]$.
We create an array holding a string $S$ over the alphabet $[x]$, defined as follows: 
\begin{itemize}
\item Let $\sigma_i = 1^{A_{i1}}2^{A_{i2}}\cdots x^{A_{ix}}$
and $\ssigma_i = 1^{y-A_{i1}}2^{y-A_{i2}}\cdots x^{y-A_{ix}}$,
which have length $O(xy)$.
\item Let $\tau_j = 1^{B_{1j}}2^{B_{2j}}\cdots x^{B_{xj}}$
and $\ttau_i = 1^{y-B_{1j}}2^{y-B_{2j}}\cdots x^{y-B_{xj}}$,
which have length $O(xy)$.
\item Let $S=\sigma_n\ssigma_n\cdots \sigma_2\ssigma_2\sigma_1\ssigma_1
\ttau_1\tau_1\ttau_2\tau_2\cdots\ttau_n\tau_n,$
which has length $O(nxy)$.
\end{itemize}

For each $i,j\in [n]$, 
consider the substring $S_{ij}=\ssigma_i\sigma_{i-1}\ssigma_{i-1}\cdots
\sigma_1\ssigma_1\ttau_1\tau_1\cdots\ttau_{j-1}\tau_{j-1}\ttau_j$.
For each $k\in[x]$, the frequency of $k$ in $S_{ij}$
is precisely $iy+jy - A_{ik}-B_{kj}$.
Thus, the mode of $S_{ij}$ is an index $k$ minimizing $A_{ik}+B_{kj}$.
So, the Min-Plus product can be computed by answering $O(n^2)$ range mode queries on $S$.  
\end{proof}

Combining Corollary~\ref{cor:intapsp} and Lemma~\ref{lem:reduce:from:minplus:mode}
immediately gives the following:

\begin{corollary}\label{cor:strong-intapsp-imply:mode}
For any constant $\beta$, 
if $\Trangemode(n^{1+3\beta},n^2\mid n^\beta)=O(n^{2+\beta-\eps})$, then
$M^*(n,n,n\mid n^{3-\omega})= O(n^{3-\Omega(\eps)})$.
\end{corollary}

By setting $\beta=1/3$ and $n=\sqrt{N}$, we have thus proved that 
\BatchMode{} for $N$ queries on $N$ elements cannot be
solved in $O(N^{7/6-\eps})$ time under the \StrongAPSP{}.
Previously, Chan et al.~\cite{CDLMW} gave a reduction
from  Boolean matrix multiplication implying a better, near-$N^{3/2}$ conditional lower bound for 
\emph{combinatorial} algorithms (under the Combinatorial BMM Hypothesis); this matched upper bounds of known
combinatorial algorithms~\cite{KrizancMS05,CDLMW}.
However, for noncombinatorial
algorithms, their lower bound was near $N^{\omega/2}$,
which is trivial if $\omega=2$.
The distinction between combinatorial vs.\ noncombinatorial algorithms is especially important for the range mode problem, as it is actually possible to beat $N^{3/2}$ using fast matrix multiplication, as first shown by Vassilevska W. and Xu~\cite{WilliamsX20}.  The current fastest algorithm by Gao and He~\cite{GaoHe22} runs in  $O(N^{1.4797})$ time.  Our new 
lower bound reveals that there is a limit on how much fast matrix multiplication can help.

(For still more recent work on range mode, see \cite{JinX22} for
a conditional lower bound for the dynamic version of the range mode problem, but again this is only for combinatorial algorithms.)


Furthermore, by  using the fact that $\Trangemode(n^{1+3\beta},n^2\mid n^\beta)\le O(n^{1-3\beta})\cdot $\\ $
\Trangemode(n^{1+3\beta},n^{1+3\beta}\mid n^\beta)$ and setting $N=n^{1+3\beta}$ and $\gamma=\beta/(1+3\beta)$, we see that
\BatchMode{} for $N$ queries on $N$ elements in a universe 
of size $\sigma=N^\gamma$
cannot be answered in $O(N^{1+\gamma-\eps})$ time for
any $\gamma\le 1/6$, under the same hypothesis.  This lower bound is \emph{tight}, since an $O(N\sigma)$
upper bound is known~\cite{CDLMW}.

Furthermore, by setting $n=\sqrt{Q}$ and $n^\beta=(N/\sqrt{Q})^{1/3}$, \BatchMode{} for $Q$ queries on $N$ elements cannot be solved in $O(Q^{5/6}N^{1/3-\eps})$ time for any $Q\le N^2$ under the same hypothesis.
For example, for $Q=N^{1.6}$, the lower bound is near $N^{1.666}$
(in other words, we need at least $N^{0.066}$ time per query).
In contrast, the previous reduction by Chan et al.~\cite{CDLMW} from
Boolean matrix multiplication
gives a lower bound of $M(\sqrt{Q},N/\sqrt{Q},\sqrt{Q})$,
which is only near linear in $Q$ when $Q=N^{1.6}$, as $\omega(0.8,0.2,0.8)=1.6$.
(Known combinatorial algorithms have running time near $O(\sqrt{Q}N)$ as a function of $N$ and $Q$~\cite{KrizancMS05,CDLMW}.)

The same results hold for the similar problem 
of \emph{range minority}~\cite{ChanDSW15} (finding a least frequent element in a range).


\subsection{Dynamic Shortest Paths in Unweighted Planar Graphs}

\newcommand{\Tdynplanar}{\mbox{\sc planar-dyn-sp}}

As another example of an application to data structure problems, we now consider the dynamic shortest path problem for unweighted planar graphs.
Let $\Tdynplanar(N,Q,U)$ be the time complexity of 
performing an offline sequence of $Q$ shortest path distance queries and
$U$ edge updates on an unweighted, undirected planar graph 
with $N$ nodes.  

\begin{lemma}\label{lem:dynplanar}
For any $\alpha,\beta\le 1$, 
\[ M^*(n,n^\beta,n\mid X)=
O(n^{1-\alpha}\cdot \Tdynplanar((n^{2\alpha+\beta} + n^{\alpha+2\beta})X, n^{1+\alpha}, n^{1+\beta})).\]
\end{lemma}
\begin{proof}
Abboud and Dahlgaard~\cite[Proof of Theorem~1]{AbbDah} reduced the
computation of the Min-Plus product of an $n\times n^\beta$ and
an $n^\beta\times n^\alpha$ matrix with entries from $[X]$, to
the problem of performing an offline sequence of $O(n^{1+\alpha})$ 
shortest path queries
and $O(n^{1+\beta})$ edge-weight changes on a \emph{weighted} planar graph with $O(n^{\alpha+\beta})$
nodes.  
In their graph construction, all edges have integer weights bounded by $O(n^\alpha X)$, except
for $O(n^\beta)$ edges having integer weights bounded by $O(n^{\alpha+\beta} X)$.
(The $X$ factor was stated as $X^2$ in their paper, but as they remarked at the end of their Section~2,
$X^2$ can be lowered to $X+1$.)
The weighted graph can be turned into an unweighted graph, simply by
subdividing each edge.  More precisely, we create a path $\pi_e$ of length $\ell$ 
for an edge $e$ with weight upper-bounded by $\ell$.  Whenever the weight 
of $e$ changes,
we can redirect an endpoint of $e$ to an appropriate node in the path $\pi_e$,
using $O(1)$ updates in this unweighted graph.
The resulting unweighted planar graph has  $O((n^{2\alpha+\beta} + n^{\alpha+2\beta})X)$ nodes.  Hence,
Abboud and Dahlgaard's reduction implies that 
$M^*(n,n^\beta,n^\alpha\mid X)=
O(\Tdynplanar((n^{2\alpha+\beta} + n^{\alpha+2\beta})X, n^{1+\alpha}, n^{1+\beta}))$.

The lemma then follows, as $M^*(n,n^\beta,n\mid X) = O(n^{1-\alpha}\cdot 
M^*(n,n^\beta,n^\alpha\mid X))$.
\end{proof}

Combining Corollary~\ref{cor:intapsp} and Lemma~\ref{lem:dynplanar} (with $\alpha=\beta$)
immediately gives the following:

\begin{corollary}\label{cor:strong-intapsp-imply:planar}
For any constant $\beta$,
if $\Tdynplanar(n^{5\beta}, %
n^{1+\beta},n^{1+\beta})=O(n^{1+2\beta-\eps})$, then
$M^*(n,n,n\mid n^{3-\omega})= O(n^{3-\Omega(\eps)})$.
\end{corollary}

By setting $\beta=1/4$ and $N=n^{5/4}$, %
we have thus proved that 
an offline sequence
of $N$ shortest path queries and $N$ updates on an unweighted, undirected $N$-node planar
graph cannot be processed in $O(N^{6/5-\eps})$ time, under the
\StrongAPSP{}.  This rules out the existence of data structures with $N^{o(1)}$ time per operation.
Abboud and Dahlgaard~\cite{AbbDah} proved
a better lower bound near $N^{3/2}$ in the weighted case under the APSP Hypothesis, 
or near $N^{4/3}$ in the unweighted case under the OMv Hypothesis~\cite{HenzingerKNS15}, but the latter
bound under the OMv Hypothesis holds only for \emph{online} queries and updates, when considering general noncombinatorial algorithms.
We have obtained the first conditional lower bounds for the unweighted case that hold in the offline setting.

Gawrychowski and Janczewski~\cite{GawJan} have adapted Abboud and Dahlgaard's technique to prove conditional lower bounds for certain dynamic data structure versions of the \emph{longest increasing subsequence (LIS)} problem.  In the unweighted case, their reduction was again based on the OMv Hypothesis and applicable only for the online setting.  Our approach should similarly yield new conditional lower bounds in the offline setting for their problem.

The preceding applications are not meant to be exhaustive, but
the applications to \BatchMode{} and dynamic planar shortest paths should suffice to
illustrate the potential usefulness of our technique to (unweighted) data structure problems in general.
A common way to obtain conditional lower bounds for such data structure problems is via reduction from Boolean matrix multiplication, which is useful only for combinatorial algorithms, or
via the OMv Hypothesis, which is only for online settings.
Our technique provides a new avenue, allowing us to obtain (weaker, but still nontrivial) lower bounds for general noncombinatorial algorithms in offline or batched settings: namely, it
suffices to reduce from rectangular Min-Plus products when the inner dimension 
and the integer range are both small.


\subsection{Min-Witness Equality Product}

Lastly, we revisit the \MinWitnessEq{} problem.
Let $\Tminwiteq(n)$ be the time complexity of \MinWitnessEq{}
for $n\times n$ matrices.  Chan, Vassilevska W., and Xu~\cite{CVXicalp21} showed
that $\TunwtdirAPSP(n)=\OO(\Tminwiteq(n))$, so we immediately obtain a near-$n^{7/3}$ lower bound
for \MinWitnessEq{} under the \StrongAPSP{}.  However, the following corollary gives an alternative lower bound, which is worse if $\omega=2$, but is better if the current bound of
$\omega$ turns out to be close to tight.  Note that $(2\omega+5)/4$ is strictly larger
than $\omega$ for all $\omega\in [2,2.373)$.

\begin{corollary}
If $\Tminwiteq(n) = O(n^{(2\omega+5)/4-\eps})$,
then $M^*(n,n,n\mid n^{3-\omega}) = O(n^{3-\Omega(\eps)})$.
\end{corollary}
\begin{proof}
Let $\Tgeneq(n_1,n_2,n_3\mid\ell)$ be the time complexity of
the generalized equality product problem in Lemma~\ref{lem:geneqprod}.
Let $\Teq(n_1,n_2,n_3)$ be the time complexity of \ExistEquality{}, which is a variant of \Equality{} where we only need to determine if each of the outputs of the standard \Equality{} is nonzero or not.
It is easy to see that $\Tgeneq(n_1,n_2,n_3\mid\ell) \le O(\ell^2\cdot \Teq(n_1,n_2,n_3))$.

The proof of Theorem~\ref{thm:main} shows that for any $s\le n_2$
and $t\le\ell$,
\[ M^*(n_1,n_2,n_3\mid\ell)\ =\ 
\OO\big( (n_2/s) M^*(n_1,s,n_3\mid t)
\,+\, s\cdot\Tgeneq(n_1,n_2,n_3\mid \ell/t) \big). \]
Consequently, for any $t\le n^{3-\omega}$,
\[ M^*(n,n,n\mid n^{3-\omega})\ =\ 
\OO\big( (n/s) M^*(n,s,n\mid t)
\,+\, s(n^{3-\omega}/t)^2\cdot \Teq(n,n,n) \big). \]
Set $t=n/s$.
Chan, Vassilevska W. and Xu~\cite{CVXicalp21} gave a reduction showing
that $M^*(n,s,n\mid n/s)=O(\Tminwiteq(n))$.
Trivially, $\Teq(n,n,n)=O(\Tminwiteq(n))$.
It follows that $M^*(n,n,n\mid n^{3-\omega})\ =\ O((n/s + s^3/n^{2\omega-4})\cdot \Tminwiteq(n))$.  The result follows by setting $s=n^{(2\omega-3)/4}$.
\end{proof}




\section{Lower Bounds under the \uAPSPH{}}\label{sec:uapsp-lower-bound}


We can similarly apply our key reduction in Theorem~\ref{thm:main} to
obtain (better) lower bounds under the \uAPSPH{},
using the following corollary:

\begin{corollary}\label{cor:unwtdirapsp}
Let $\rho$ be such that $\omega(1,\rho,1)=1+2\rho$.
Fix any constant $\sigma>\rho$, and
let $\kappa=\frac{\omega(1,\sigma,1)-1-2\rho}{\sigma-\rho}$.
For any constant $\beta$,
if $M^*(n,n^\beta,n\mid n^{(1+\kappa)\beta})=O(n^{2+\beta-\eps})$,
then $\TunwtdirAPSP(n)=O(n^{2+\rho-\Omega(\eps)})$.
\end{corollary}
\begin{proof}
Chan, Vassilevska W. and Xu~\cite{CVXicalp21} have shown that $\TunwtdirAPSP(n)=O(n^{2+\rho-\Omega(\eps)})$ is equivalent to
$M^*(n,n^\rho,n\mid n^{1-\rho})=O(n^{2+\rho-\Omega(\eps)})$.

By Theorem~\ref{thm:main},
\[ M^*(n,n^\rho,n\mid n^{1-\rho})\ =\ \OO\left((n^{\rho}/s)M^*(n,s,n\mid t) + sn^{2+\rho}/r + (sn^{1-\rho}/t) M(n,rn^{\rho},n)\right).
\]
Setting $s=n^{\beta-\eps'}$, $t=n^{(1+\kappa)\beta}$, and $r=n^{\beta}$ with $\eps'=\eps/2$ yields
\[ M^*(n,n^\rho,n\mid n^{1-\rho})\ =\ \OO\left(n^{\rho-\beta+\eps'}M^*(n,n^\beta,n\mid n^{(1+\kappa)\beta}) + n^{2+\rho-\eps'} + n^{1-\rho-\kappa\beta+\omega(1,\rho+\beta,1)-\eps'}\right),
\]
which is $\OO(n^{2+\rho-\Omega(\eps)})$,
since $\omega(1,\rho+\beta,1) \le \omega(1,\rho,1) + 
\frac{\omega(1,\sigma,1) - \omega(1,\rho,1)}{\sigma-\rho}\beta
= 1+2\rho + \kappa\beta$, by convexity of $\omega(1,\cdot,1)$.

The above assumes $(1+\kappa)\beta\le 1-\rho$ and $\beta\le\sigma-\rho$, but 
this assumption may be removed, since Lemma~\ref{lem:down} allows us to replace $\beta$ with any sufficiently small positive constant $\gamma$.
\end{proof}

We pick $\sigma=0.85$.  By known bounds~\cite{legallurr}, $\omega(1,0.85,1)<2.258317$.
Since $\rho\ge 0.5$, we have $\kappa\le \frac{1.258317-2\rho}{0.85-\rho}<0.7381$.
If $\omega=2$, then $\kappa=0$.


\subsection{Min-Witness Product}

Combining Corollary~\ref{cor:unwtdirapsp} and Lemma~\ref{lem:reduce:from:minplus:minwit} immediately gives the following:

\begin{corollary}\label{cor:lower-bounds-under-uAPSP:minwit}
Let $\rho$ and $\kappa$ be as in Corollary~\ref{cor:unwtdirapsp}.
For any constant $\beta$,
if $\Tminwit(n,n^{(3+2\kappa)\beta},n)=O(n^{2+\beta-\eps})$, then
$\TunwtdirAPSP(n)=O(n^{2+\rho-\Omega(\eps)})$.
\end{corollary}

By setting $\beta=1/(3+2\kappa)$ (which is $1/3$ if $\omega=2$, or $<0.223$ regardless), we have thus proved that \MinWitness{} of two $n\times n$ Boolean matrices
cannot be computed in $O(n^{7/3-\eps})$ time if $\omega=2$, or
$O(n^{2.223})$ time regardless of the value of $\omega$, under the \uAPSPH{}. 
(This is better than the near-$n^{11/5}$ lower bound we obtained from the \StrongAPSP{}.)
The question of proving lower bounds for \MinWitness{} from the \uAPSPH{} was left open in the paper by
Chan, Vassilevska W. and Xu~\cite{CVXicalp21} (they were only able to do so for \MinWitnessEq{}).

Furthermore, by setting $\beta=\gamma/(3+2\kappa)$, 
\MinWitness{} of an $n\times n^{\gamma}$ and an $n^{\gamma}\times n$ Boolean matrix cannot be computed in $O(n^{2+0.223\gamma-\eps})$ time for any $\gamma\le 1$ under the same hypothesis. 


\subsection{All-Pairs Shortest Lightest Paths}

Combining Corollary~\ref{cor:unwtdirapsp} and Lemma~\ref{lem:reduce:from:minplus:apslp} immediately gives the following:

\begin{corollary}\label{cor:lower-bounds-under-uAPSP:apslp}
Let $\rho$ and $\kappa$ be as in Corollary~\ref{cor:unwtdirapsp}.
For any constant $\beta\le \frac1{3+2\kappa}$
and $\TundirAPSLP(n,n^{1+\beta})=O(n^{2+\beta-\eps})$, then
$\TunwtdirAPSP(n)=O(n^{2+\rho-\Omega(\eps)})$.
\end{corollary}

By setting $\beta=1/(3+2\kappa)$, we have thus proved that \APSLP{} cannot be solved
in $O(n^{7/3-\eps})$ time if $\omega=2$, or
$O(n^{2.223})$ time regardless of the exact value of $\omega$, under the \uAPSPH{}. 
(This is better than the near-$n^{11/5}$ lower bound we obtained from the \StrongAPSP{}.)
The same result holds for \APLSP{}.  Previously, Chan, Vassilevska W. and Xu~\cite{CVXicalp21} proved a still better near-$n^{\rho}$ lower bound for $\{0,1\}$-weighted  \APLSPIntro{}  from the same hypothesis, but their proof crucially relied on zero-weight edges and also
did not work for \APSLPIntro{} (leaving open the question of finding nontrivial conditional lower bounds for both \APLSP{} and \APSLP{}, which we answer here). 


\subsection{Batched Range Mode}

By combining  Chan, Vassilevska W. and Xu's observation that 
$\TunwtdirAPSP(n)=\OO(\max_\ell M^*(n,n/\ell,n\mid \ell))$~\cite{CVXicalp21,zwickbridge} with
Lemma~\ref{lem:reduce:from:minplus:mode}, and
setting $n=\sqrt{N}$, we see that
\BatchMode{} for $N$ queries on $N$ elements cannot be
solved in $O(N^{5/4-\eps})$ time if $\omega=2$, or
in $O(N^{1+\rho/2-\eps})$ time regardless, under the \uAPSPH{}.

To obtain a lower bound for general $Q$, we
combine Corollary~\ref{cor:unwtdirapsp} and Lemma~\ref{lem:reduce:from:minplus:mode}:

\begin{corollary}\label{cor:lower-bounds-under-uAPSP:mode}
Let $\rho$ and $\kappa$ be as in Corollary~\ref{cor:unwtdirapsp}.
For any constant $\beta$,
if $\Trangemode(n^{1+(2+\kappa)\beta},n^2\mid n^\beta)=O(n^{2+\beta-\eps})$, 
then $\TunwtdirAPSP(n)=O(n^{2+\rho-\Omega(\eps)})$.
\end{corollary}

Thus, by setting $n=\sqrt{Q}$ and $n^\beta=(N/\sqrt{Q})^{1/(2+\kappa)}$, \BatchMode{} of $Q$ queries on $N$ elements cannot be answered in
 $O(Q^{3/4}N^{1/2-\eps})$ time if $\omega=2$,
or $O(Q^{1-0.365/2}N^{0.365-\eps})$ time 
regardless, for any $Q\le N^2$, under the \uAPSPH{}.  
For example, for $Q=N^{1.6}$, the lower bound is near $N^{1.7}$
if $\omega=2$, or near $N^{1.673}$ regardless.
(This is slightly better than the lower bound we obtained in previous section from the \StrongAPSP{}.)

\subsection{Dynamic Shortest Paths in Planar Graphs}


Combining Corollary~\ref{cor:unwtdirapsp} and Lemma~\ref{lem:dynplanar} with $\alpha=\beta$
immediately gives the following:

\begin{corollary}\label{cor:lower-bounds-under-uAPSP:planar}
Let $\rho$ and $\kappa$ be as in Corollary~\ref{cor:unwtdirapsp}.
For any constant $\beta$,
if
$\Tdynplanar(n^{(4+\kappa)\beta},n^{1+\beta},n^{1+\beta})=O(n^{1+2\beta-\eps})$,
then
$\TunwtdirAPSP(n)=O(n^{2+\rho-\Omega(\eps)})$.
\end{corollary}

By setting $\beta=1/(3+\kappa)$
and $N=n^{1+\beta}$, we have thus proved that 
an offline sequence
of $N$ shortest path queries and $N$ updates on an unweighted, undirected $N$-node planar
graph cannot be processed in $O(N^{5/4-\eps})$ time if $\omega=2$, 
of $O(N^{1.211})$
time regardless, under the
\uAPSPH{}.  
(This is slightly better than the lower bound we obtained in previous section from the \StrongAPSP{}.)

\subsection{An Equivalence Result}

We also obtain an interesting equivalence result:

\begin{corollary}
Let $\alpha$ be such that $\omega(1,\alpha,1)=2$.
For any constants $\beta,\gamma\in (0,\alpha)$,
there exists $\eps>0$ such that $M^*(n,n^\beta,n\mid n^\beta)=O(n^{2+\beta-\eps})$ if and only if there exists $\eps'>0$ such that
$M^*(n,n^\gamma,n\mid n^\gamma) = O(n^{2+\gamma-\eps'})$.
\end{corollary}
\begin{proof}
W.l.o.g., assume $\gamma<\beta$.
The ``only if'' direction is shown in Lemma~\ref{lem:down}.
For the ``if'' direction,
suppose $M^*(n,n^\gamma,n\mid n^\gamma) = O(n^{2+\gamma-\eps'})$.
By Theorem~\ref{thm:main},
\[ M^*(n,n^\beta,n\mid n^\beta)\ =\ \OO\left((n^{\beta}/s)M^*(n,s,n\mid t) + sn^{2+\beta}/r + (sn^\beta/t) M(n,rn^\beta,n)\right).
\]
Setting $s=n^{\gamma-\eps}$, $t=n^{\gamma}$, and
$r=n^\gamma$, with $\eps=\eps'/2$
yields $M^*(n,n^\beta,n\mid n^\beta)=O(n^{2+\beta-\Omega(\eps')})$,
assuming that $\gamma\le \alpha-\beta$.

This assumption may be removed, since Lemma~\ref{lem:down} allows
us to replace $\gamma$ with a sufficiently small positive constant $\gamma'$.
\end{proof}

\begin{corollary}
If $\omega=2$,
then for any constant $\beta\in (0,1)$,
there exists $\eps>0$ such that
$\TunwtdirAPSP(n)=O(n^{2.5-\eps})$ iff
there exists $\eps'>0$ such that $M^*(n,n^\beta,n\mid n^\beta)=\OO(n^{2+\beta-\eps'})$.
\end{corollary}
\begin{proof}
If $\omega=2$, then $\rho=1/2$ and then
Chan, Vassilevska W. and Xu's result~\cite{CVXicalp21} showed that $\TunwtdirAPSP(n)=O(n^{2.5-\eps})$ for some $\eps>0$
is equivalent to $M^*(n,\sqrt{n},n\mid \sqrt{n})=O(n^{2.5-\eps'})$
for some $\eps'>0$.
Since $\omega=2$ implies $\alpha=1$, we can apply
the preceding corollary for any $\beta\in(0,1)$ and $\gamma=1/2$.
\hspace*{\fill}
\end{proof}

Let $\Phi(\beta)$ be the claim that $M^*(n,n^\beta,n\mid n^\beta)$
is not in $O(n^{2+\beta-\eps})$ for any $\eps>0$.
If $\omega=2$, $\Phi(1)$ is just
the \StrongAPSP{}, but intriguingly, by the above corollary, $\Phi(0.99)$
is equivalent to the \uAPSPH{}, which has given us strictly better conditional lower bound results.




