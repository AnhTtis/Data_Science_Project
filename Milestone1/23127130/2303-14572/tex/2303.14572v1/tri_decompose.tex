
\newcommand{\Triangles}{\mbox{\rm Triangles}}
\newcommand{\ZeroTriangles}{\mbox{\rm Zero-Triangles}}

In this section, we introduce a decomposition theorem
for zero-weight triangles, which encapsulates some of the key ideas
we have used, and which may be viewed as an alternative to the BSG Theorem.
We will describe applications of this decomposition theorem to some algorithmic problems that
were previously solved via the BSG Theorem by Chan and Lewenstein~\cite{ChanLewenstein},
as well as a new application to the \#APSP problem for arbitrary weighted graphs.

In a weighted tripartite graph $G$ with node sets $U$, $X$, and $V$, let $\Triangles(G)$ denote
the set of all triangles in $U\times X\times V$ in $G$, and 
let $\ZeroTriangles(G)$ denote
the set of all zero-weight triangles in $U\times X\times V$ in~$G$.


\begin{theorem}\label{thm:tri:decompose}
{\bf (Triangle Decomposition)}
Given a real-weighted tripartite graph $G$ with $n_1$, $n_2$, and $n_3$ nodes in its three parts $U$, $X$, and $V$, and given a parameter $s$,
there exist a collection of $\ell=\OO(s^3)$ subgraphs $G^{(1)},\ldots,G^{(\ell)}$ of $G$, and a set $R$ of $\OO(n_1n_2n_3/s)$ triangles,
such that 
\[ \ZeroTriangles(G)\: =\: R \,\cup\, \bigcup_{\lam=1}^\ell \Triangles(G^{(\lam)}).
\]
The subsets in the union above are disjoint.
The $G^{(\lam)}$'s and $R$ can be constructed in 
$\OO(n_1n_2n_3+sn_1n_2+sn_2n_3+s^2n_1n_3)$ deterministic time.
And if the edge weights between $U$ and $V$  later change, then the $G^{(\lam)}$'s and $R$ can be updated in $\OO(n_1n_2n_3/s + s^2n_1n_3)$ time.

Furthermore, the subgraphs can be grouped into $\OO(s)$ categories of
$\OO(s^2)$ subgraphs each, such that if an edge $uv\in U\times V$ is
present in one subgraph, it is present in all subgraphs of the same category.
\end{theorem}
\begin{proof}
Let $\{u_i:i\in [n_1]\}$, $\{x_k:k\in [n_2]\}$, and $\{v_j:j\in[n_3]\}$ be the nodes of the three parts.
Let the weight of $u_ix_k$ be $a_{ik}$, the weight of $x_kv_j$ be $b_{kj}$, and the weight of $u_iv_j$ be $-c_{ij}$.

As a preprocessing step, we sort the multiset
$\{a_{ik}+b_{kj}: k\in [n_2]\}$
for each $i\in [n_1]$ and $j\in [n_3]$, in $\OO(n_1n_2n_3)$ total time.
Let
$W_{ij}=\{k\in [n_3]: a_{ik}+b_{kj}=c_{ij}\}$.  Note that we can generate
the elements in $W_{ij}$ by searching in the sorted lists in $\OO(1)$ time per element.

\begin{itemize}
\item {\bf Few-witnesses case.}
For each $i,j$ with $|W_{ij}|\le n_2/s$,
add the triangle $u_ix_kv_j$ to $R$ for every $k\in W_{ij}$.
The number of triangles added to $R$ is $O(n_1n_3\cdot n_2/s)$. The
running time of this step is also $\OO(n_1n_3 \cdot n_2/s)$.

\item {\bf Many-witnesses case.}
Find a set $H\subseteq [n_2]$ of $O(s\log(n_1n_2n_3))$ nodes that
hit all $W_{ij}$ with $|W_{ij}| > n_2/s$.
Here, we can use the standard greedy algorithm for hitting sets, which is deterministic and 
takes time linear in the total size of the sets $W_{ij}$; as we can
reduce each set's size to $n_2/s$ before running the hitting set algorithm, 
this takes $\OO(n_1n_2n_3/s)$ time.
For each $i,j$ with $|W_{ij}|>n_2/s$, let $k_0[i,j]$ be some $k_0\in W_{ij}\cap H$.

For each $k_0\in H$ and $k\in[n_2]$,
let $L_{k_0k}$ be the multiset $\{b_{k_0j}-b_{kj}: j\in[n_3]\}$,
and let $F_{k_0k}$ be the elements that have frequency
more than $n_3/r$ in $L_{k_0k}$.
Note that $|F_{k_0k}|\le r$.

\begin{itemize}
\item {\bf Low-frequency case.}
For each $k_0\in H$ and $i\in[n_1]$ and $k\in[n_2]$,
if $a_{ik}-a_{ik_0}\not\in F_{k_0k}$,
we examine each of the at most $n_3/r$ indices $j$
with $a_{ik}-a_{ik_0}=b_{k_0j}-b_{kj}$,
and add $u_ix_kv_j$ to $R$ if it is a zero-weight triangle and $|W_{ij}|>n_2/s$.
The number of triangles added to $R$ is
$\OO(sn_1n_2\cdot n_3/r)=\OO(n_1n_2n_3/s)$, by choosing $r:=s^2$.
The running time of this step is also bounded by $\OO(n_1n_2n_3/s)$.

\item {\bf High-frequency case.}
For each $k_0\in H$ and $p\in [r]$, create a subgraph $G^{(k_0,p)}$
of $G$:
\begin{itemize}
\item For each $i\in[n_1]$ and $j\in[n_3]$, keep edge $u_iv_j$ iff
$k_0[i,j]=k_0$ (in particular, $a_{ik_0}+b_{k_0j}=c_{ij}$).
\item For each $i\in[n_1]$ and $k\in[n_2]$, keep edge $u_ix_k$ iff $a_{ik}-a_{ik_0}$ is the $p$-th element of
$F_{k_0k}$.
\item For each $j\in[n_3]$ and $k\in[n_2]$, keep edge $x_kv_j$ iff $b_{k_0j}-b_{kj}$ is the $p$-th element of $F_{k_0k}$.
\end{itemize}
Note that if $u_ix_kv_j$ is a triangle in $G^{(k_0,p)}$, then
$a_{ik}-a_{ik_0}=b_{k_0j}-b_{kj}$ and $a_{ik_0}+b_{k_0j}=c_{ij}$,
and by Fredman's trick, $a_{ik}+b_{kj}=a_{ik_0}+b_{k_0j}=c_{ij}$,
implying that $u_ix_kv_j$ is a zero-weight triangle.
The running time of this step is  bounded by $\OO(sn_1n_2+sn_2n_3+rn_1n_3)$.
\end{itemize}
\end{itemize}
The number of subgraphs created is $\OO(sr)=\OO(s^3)$.

\smallskip
\emph{Correctness.} Consider a zero-weight triangle $u_ix_kv_j$ in $G$.
If $|W_{ij}|\le n_2/s$, the triangle is in $R$ due to the ``few-witnesses'' case.
So assume $|W_{ij}|>n_2/s$.  Let $k_0=k_0[i,j]$.
We know that $a_{ik}+b_{kj}=c_{ij}=a_{ik_0}+b_{k_0j}$ and by Fredman's trick,
$a_{ik}-a_{ik_0}=b_{k_0j}-b_{kj}$.
If $a_{ik}-a_{ik_0}\not\in F_{k_0k}$, then
the triangle is in $R$ due to the ``low-frequency'' case.
Otherwise, it is a triangle in $G^{(k_0,p)}$ for some $p\in [r]$
due to the ``high-frequency'' case.

\emph{Update edge weights.} If the edge weights between $U$ and $V$ change, we only need to rerun the parts of our algorithms that use the weights $c$. In particular, we need to rerun the low-frequency case, which has running time $\OO(n_1 n_2 n_3 / s)$, and the first subcase of the high-frequency case, which has running time $\OO(r n_1 n_3) = \OO(s^2 n_1 n_3)$. Overall, the running time for updating the edge weights between $U$ and $V$ is $\OO(n_1n_2n_3/s + s^2n_1n_3)$.
\end{proof}



\subsection{Application 1: Exact Triangle in Preprocessed Universes}

As one simple application of the decomposition theorem, we can solve \AEExactTri{}
in truly subcubic time in  a ``preprocessed universe'' setting, where
the input is a subgraph of a preprocessed graph. 
It is convenient to define a variant of the problem,
\AEExactTri{}', which is \AEExactTri{} where the input graph is tripartite with three parts $U$, $X$, and $V$, but we only need to report if each edge between $U$ and $V$ is in an exact triangle. If $|U|=|V|=|X|$, \AEExactTri{}' is equivalent to \AEExactTri{}.

\begin{corollary}\label{cor:zerotri:prep}
Given a real-weighted tripartite graph $G$ with $n_1$, $n_2$, and $n_3$ nodes in its three parts $U$, $X$, and $V$, and given $s$, we can preprocess $G$
in $\OO(n_1n_2n_3+sn_1n_2+sn_2n_3+s^2n_1n_3)$ time, so that for any given subgraph $G'$ of $G$, 
we can 
solve \AEExactTri{}' 
on $G'$ in
$\OO(n_1n_2n_3/s + s M(n_1,s^2n_2,n_3))$ time.

For example, for $n_1=n_2=n_3=n$, after preprocessing in $\OO(n^3)$ time,
we can solve \AEExactTri{} on $G'$ in time $O(n^{2.83})$, or if $\omega=2$, $\OO(n^{11/4})$.
\end{corollary}
\begin{proof}
During preprocessing, we apply Theorem~\ref{thm:tri:decompose} 
 to compute the subgraphs $G^{(\lam)}$ and $R$ in $\OO(n_1n_2n_3/s + sn_1n_2+sn_2n_3+s^2n_1n_3)$ time.

During a query for a given subgraph $G'$ and a target value $t$, if $t$ has changed, we first subtract $t$ from all the edge weights between $U$ and $V$ %
to transform the problem to detecting zero-weight triangles.
We can update the $G^{(\lam)}$'s and $R$ in $\OO(n_1n_2n_3/s + s^2 n_1n_3)$ time.

Next, for each $\lam$, we check whether after removing edges not present in $G'$, the subgraph $G^{(\lam)}$
has a triangle (which would automatically be a zero-weight triangle)
through each edge in $U\times V$.
Since triangle finding (without weights) reduces to matrix multiplication,
the running time is $\OO(s^3 M(n_1,n_2,n_3))$.

We can do slightly better using the grouping of the subgraphs: for each category $\Lambda$, define
a tripartite graph $G^{(\Lambda)}$ with parts $U$, $X\times \Lambda$, and $V$,
and for each $\lam\in\Lambda$, include an edge between $u$ and $(x,\lam)$ if $ux\in G^{(\lam)}\cap G'$,
and between $(x,\lam)$ and $v$ if $xv\in G^{(\lam)}\cap G'$.  For each category $\Lambda$,
we check whether the subgraph $G^{(\Lambda)}$ has a triangle through
each edge.  The running time becomes $\OO(s M(n_1,s^2n_2,n_3))$.
Also, the term $O(s^2n_1n_3)$ for
the update cost can be lowered to $O(n_1n_3)$ now when working with the $G^{(\Lambda)}$'s instead of the $G^{(\lambda)}$'s, as can be checked from our construction (since each edge $u_iv_j$ occurs in one category).

For $n_1=n_2=n_3=n$, we choose $s=n^{0.17+\eps}$ (using the fact that
$\omega(1,1.34,1)<2.657$~\cite{legallurr}), or if $\omega=2$, $s=n^{1/4}$.
\end{proof}


\subsection{Application 2: 3SUM in Preprocessed Universes}

The above implies also a new algorithm for 
\ThreeSUM{} with a preprocessed universe, previously studied
by Chan and Lewenstein~\cite{ChanLewenstein}, who obtained
$\OO(n^{13/7})$ query time, after preprocessing in $\OO(n^2)$
randomized time or $\OO(n^\omega)$ deterministic time.  Our query time is strictly better if $\omega$ is 2 (or is sufficiently close to 2), and we also obtain \emph{deterministic} $\OO(n^2)$ preprocessing time regardless of
$\omega$.

\begin{corollary}\label{cor:3sum:prep0}
We can preprocess sets $A,B,C$ of $n$ integers in $[n^{O(1)}]$
in $\OO(n^2)$ deterministic time, so that given any subsets $A'\subseteq A$,
$B'\subseteq B$, and $C'\subseteq C$, we can solve
\AllThreeSUM{} on $(A',B',C')$ in time $O(n^{1.891})$, or if $\omega=2$, $\OO(n^{11/6})$.
\end{corollary}
\begin{proof}
We use a known reduction from (All-Nums-) \ThreeSUM{} to (All-Nums-) \ThreeSUMConv{} (one of the reductions by Chan and He~\cite{ChanHe}
is deterministic and increases running time only by polylogarithmic
factors when the input numbers are polynomially bounded),
in combination with a known reduction 
from \AllThreeSUMConv{} to  \AEExactTri{}~\cite{VassilevskaW09}.
The problem is reduced to $\OO(1)$ instances of 
the problem in Corollary~\ref{cor:zerotri:prep}
with $n_1=n/q$, $n_2=q$ and $n_3=n$ for a parameter $q$
(the original reduction~\cite{VassilevskaW09} has $q=\sqrt{n}$, but we will
do better with a different choice of $q$).
It is straightforward to check that these
reductions carry over to the preprocessed universe setting. 
We then obtain preprocessing time $\OO(n^2+snq+s^2n^{2}/q)=\OO(n^2+s^3n^{1.5})$ and
query time $\OO(n^2/s + s M(n/q,s^2q,n))
=\OO(n^2/s + s M(s\sqrt{n},s\sqrt{n},n))$ by setting $q=\sqrt{n}/s$.
We choose $s=n^{0.109+\eps}$  (using the fact that
$\omega(0.609,0.609,1)<1.781$~\cite{legallurr}), or if $\omega=2$, $s=n^{1/6}$.
\end{proof}

\begin{remark}\rm
Corollary~\ref{cor:zerotri:prep} and Corollary~\ref{cor:3sum:prep0} can also be used to solve \AEExactTriCount{} and \AllThreeSUMCount{} in the preprocessed universe. This is because Theorem~\ref{thm:tri:decompose} provides a decomposition, so we can sum up the counts in all the cases (In Corollary~\ref{cor:3sum:prep0}, we also have to be careful when applying the reduction in \cite{ChanHe}, to make sure we do not over count, by using inclusion-exclusion). Prior method for \ThreeSUM{} in the preprocessed universe \cite{ChanLewenstein} cannot compute counts, because it relies on the BSG theorem, which only provides a covering. 
\end{remark}

In Section~\ref{sec:3sum:prep:rand}, we will describe a still better solution to 3SUM in preprocessed universes, with randomization, using FFT instead of fast matrix multiplication.


\subsection{Application 3: A Deterministic 3SUM Algorithm for Bounded Monotone Sets}

Another interesting application is the following:

\begin{corollary}
Given monotone sets $A,B,C\subseteq [n]^d$ for any constant $d\ge 2$,
we can solve \AllThreeSUM{} on $A, B, C$ in
$O(n^{2-1/O(d)})$ deterministic time. 
\end{corollary}
\begin{proof}
Chan and Lewenstein~\cite{ChanLewenstein} observed that
\ThreeSUM{} for bounded monotone sets in any constant dimension $d$
reduces to \ThreeSUM{} for ``clustered'' sets, which in turn
reduces to \ThreeSUM{} with preprocessed universes on $O(n/\ell)$ elements
and $O(\ell^{3d})$ queries.
Using Corollary~\ref{cor:3sum:prep0} gives
total deterministic time $\OO((n/\ell)^2 + \ell^{3d} (n/\ell)^{1.891})$,
which is $\OO(n^{2-0.218/(3d+0.109)})$ by setting $\ell=n^{0.109/(3d+0.109)}$.
(The exponent here is certainly improvable, by solving the problem
using our techniques more directly, instead of applying a black-box reduction to \ThreeSUM{} with preprocessed universes.)
\end{proof}

The above result provides the first truly subquadratic
\emph{deterministic} algorithm for bounded monotone 3SUM in arbitrary constant dimensions---Chan and Lewenstein~\cite{ChanLewenstein} gave subquadratic randomized algorithms 
with $O(n^{2-1/(d+O(1))})$ running time, but they had
nontrivial deterministic algorithms only for $d\le 7$ under
the current matrix multiplication bounds.

We can also apply the triangle decomposition theorem to obtain 
subquadratic algorithms for monotone or bounded-difference Min-Plus convolution
(which were first obtained by Chan and Lewenstein~\cite{ChanLewenstein}, and followed by \cite{ChiDXZstoc22}),
and subcubic algorithms for monotone or bounded-difference  Min-Plus products
(which were first obtained by Bringmann et al.~\cite{BringmannGSW16}, and followed by \cite{WilliamsX20,GuPWX21, ChiDXsoda22,ChiDXZstoc22}).
Since previous algorithms have been found for these problems, we will omit the details here and refer to Appendix~\ref{app:bd:diff}.  

The main message is that many of the results in Chan and Lewenstein's
paper can be obtained alternatively using our decomposition theorem, which is simpler
and more elementary than the BSG Theorem, if we are interested in
subquadratic algorithms but don't care about the
precise values in the exponents.  The advantage is simplicity---additive combinatorics is not needed after all!  (However, the BSG Theorem is still potentially useful in optimizing those exponents.)



\subsection{Application 4: A Truly Subquartic \#APSP Algorithm}

As another simple, interesting application of the triangle decomposition theorem,
we obtain the first truly subquartic algorithm for \APSPCount{}
for arbitrary weighted graphs:

\begin{theorem}
\label{thm:apsp-count}
\APSPCount{} for $n$-node graphs with positive edge weights has an algorithm running in
$O(n^{3.83})$ time, or if $\omega=2$, $\OO(n^{15/4})$ time.
\end{theorem}
\begin{proof}
We define the following ``funny'' matrix product $\otimes$: if $(C, C') = (A, A') \otimes (B, B')$, then $C = A \star B$ and $C'_{ij} = \sum_{k \in [n]: C_{ij} = A_{ik} + B_{kj}} A'_{ik}B'_{kj}$. 

\begin{claim}
\label{cl:apsp-count}

Let $(A, A')$ and $(B, B')$ be two pairs of $n \times n$ matrices where the entries of $A'$ and $B'$ are (large) $\ell$-bit integers. Then we can compute $(A, A') \otimes (B, B')$ in time $\OO(n^3 + \ell n^{2.83})$, or if $\omega=2$, $\OO(n^3 + \ell n^{11/4})$.
\end{claim}
\begin{proof}
First compute $C=A\star B$ naively in $O(n^3)$ time.  Initialize the entries of $C'$ to 0.
Consider the tripartite graph
with nodes $\{u_i:i\in[n]\}$, $\{x_k:k\in[n]\}$, and $\{v_j:j\in[n]\}$, where $u_ix_k$ has weight $A_{ik}$, and $x_kv_j$ has weight $B_{kj}$, and $u_iv_j$ has weight $-C_{ij}$.
Apply Theorem~\ref{thm:tri:decompose} to obtain subgraphs $G^{(\lam)}$ and a set $R$ in $\OO(n^3 + s^2n^2)$ time.

We first examine each triangle $u_ix_kv_j\in R$ and add
$A'_{ik}B'_{kj}$ to $C'_{ij}$.  This takes $\OO(\ell n^3/s)$ time.

Next, for each $\lam$, we compute $\sum_{k}  [u_ix_k\in G^{(\lam)}] A'_{ik} \cdot  [x_kv_j\in G^{(\lam)}] B'_{kj}$ and add it to $C'_{ij}$ for every $u_iv_j\in G^{(\lam)}$.  
This reduces to a standard matrix product on $\ell$-bit integers
and takes $\OO(\ell n^\omega)$ time for each $\lam$.  The total time
is $\OO(\ell\cdot s^3 n^\omega)$.
As before, we can do slightly better using the grouping of the subgraphs, which improve the running time to
$\OO(\ell\cdot s M(n,s^2n,n))$. 

As in Corollary~\ref{cor:zerotri:prep}, we choose $s=n^{0.17+\eps}$, or if $\omega=2$, $s=n^{1/4}$.
\end{proof}


Given an input graph $G$ with positive edge weights, let $D^{(=2^i)}$ be the distance matrix for paths of (unweighted) length exactly $2^i$, and $D'^{(=2^i)}$ be the number of paths of (unweighted) length exactly $2^i$ that match the distance in $D^{(=2^i)}$. Similarly, we define $D^{(<2^i)}$ as the distance matrix for paths of (unweighted) length less $2^i$, and define $D'^{(<2^i)}$ similarly. 

For $i=0$, it is easy to see that $D^{(=1)}$ is exactly the weight matrix of $G$, and $D'^{(=1)}$ is the adjacency matrix of $G$. Also, $D^{(<1)}$ is the matrix whose diagonal entries are all $0$, and other entries are all $\infty$. Finally $D'^{(<1)}$ equals the $n \times n$ identity matrix. 

For $i>0$, we can use the following recurrences:
\begin{equation*}
    \begin{split}
        (D^{(=2^i)}, D'^{(=2^i)}) &= (D^{(=2^{i-1})}, D'^{(=2^{i-1})}) \otimes  (D^{(=2^{i-1})}, D'^{(=2^{i-1})}),\\
        (D^{(<2^i)}, D'^{(<2^i)}) &= (D^{(<2^{i-1})}, D'^{(<2^{i-1})}) \otimes  (D^{(=2^{i-1})}, D'^{(=2^{i-1})}).
    \end{split}
\end{equation*}
It is not difficult, though a bit tedious, to verify the correctness of these recurrences. 

The matrix $D'^{(<2^i)}$ gives the result for \APSPCount{} when $2^i > n$. 
Therefore, 
\APSPCount{}
reduces to $O(\log n)$ instances of the funny product $\otimes$, when the matrices $A'$ and $B'$ are  $\OO(n)$-bit numbers. Then applying Claim~\ref{cl:apsp-count} with $\ell=\OO(n)$ yields the theorem.
\end{proof}

In Section~\ref{sec:bsg}, we will return to the BSG Theorem and describe
more variants and applications.