

In Section~\ref{sec:decomposition-zero-tri}, we have seen how our techniques may
replace some of the previous uses of the BSG Theorem in algorithmic applications.
In this section, we show how our techniques can actually prove
variants of the BSG Theorem itself.

We begin with a quick review of the BSG Theorem.  Many different 
versions of the theorem can be found in the literature, and
the following is one version that is easy to state:

\begin{theorem}\label{thm:BSG0}
{\bf (BSG Theorem)}\ \ 
Given subsets $A$, $B$, and $C$ of size $n$ of an abelian group,
and a parameter $s$,
if $|\{(a,b)\in A\times B:\ a+b\in C\}|\ge n^2/s$, 
then there exist subsets $A'\subseteq A$ and $B'\subseteq B$
both of size $\Omega(n/s)$, such that
\[ |A'+B'|\:=\: O(s^5n).
\]
\end{theorem}

The earliest version of the theorem, with super-exponential
factors in $s$, was obtained by
Balog and Szemer\'edi~\cite{BalogSze}, via the regularity lemma.
Gowers~\cite{Gowers01} was the first to obtain a version with polynomial
dependency on $s$.  The version stated above was proved by Balog~\cite{Balog07} and Sudakov, Szemer\'edi and Vu~\cite{SudakovSV94}.  Although the proof is not long and does not need advanced tools, it is clever and
not easy to think of; see~\cite{TaoVu06,Lovett17,Viola11} for various
different expositions.


Chan and Lewenstein~\cite{ChanLewenstein} gave algorithmic applications
using  the following variant which we will call the ``BSG Covering
Theorem'' (it was called the ``BSG Corollary'' in their paper).  Instead of extracting a single
pair of large subsets $(A',B')$, the goal is to construct a cover
by multiple pairs of subsets $(A^{(i)},B^{(i)})$: 

\begin{theorem}\label{thm:BSG0:cover}
{\bf (BSG Covering)}\ \ 
Given subsets $A$, $B$, and $C$ of size $n$ of an abelian group,
and a parameter $s$,
there exist a collection of $\ell=O(s)$ subsets $A^{(1)},\ldots,A^{(\ell)}\subseteq A$ and $B^{(1)},\ldots,B^{(\ell)}\subseteq B$, and a set $R$ of $O(n^2/s)$ pairs in $A\times B$, such that
\begin{enumerate}
\item[\rm(i)] $\{(a,b)\in A\times B: a+b\in C\}\ \subseteq\
R\,\cup\, \bigcup_\lam (A^{(\lam)}\times B^{(\lam)})$, and
\item[\rm(ii)] $|A^{(\lam)} + B^{(\lam)}| = O(s^5 n)$ for each $\lam$
(and so $\sum_\lam (|A^{(\lam)} + B^{(\lam)}|) = O(s^6 n)$).
\end{enumerate}
\end{theorem}

The BSG Covering Theorem is not implied by the BSG Theorem as stated,
but the known proofs by Balog~\cite{Balog07} and Sudakov et al.~\cite{SudakovSV94}
established an extension of the BSG Theorem  that involves an input graph, 
and repeated applications of this theorem indeed provide multiple pairs of subsets 
satisfying the stated properties.


\newcommand{\pop}{\mbox{\rm pop}}

\subsection{A New Simpler Version}

We will now show how our techniques, combined with some new extra ideas, can
derive a version of the BSG Covering Theorem where  the
$O(s^6n)$ bound is weakened to $\OO(s^2n^{3/2})$.  Although the new bound is superlinear in $n$, the lower polynomial dependency on $s$ actually compensates to yield
improved results in some algorithmic applications of the Covering Theorem.
In particular, $\OO(s^2n^{3/2})$ is better when $s\gg n^{1/8}$.
A key advantage of the new proof is its simplicity:  it constructs a cover directly, instead of repeatedly extracting subsets one at a time (thus, it avoids
the need to extend the BSG Theorem with an input graph, and thereby simplifies
the algorithm considerably).

We focus on the setting where each input set $A$ is of the
form $\{(i,a_i): i\in [n]\}$ for a sequence of integers or reals $a_1,\ldots,a_n$.
We call such a set an \emph{indexed set}.
This case turns out to be sufficient for applications involving
integer input (because known hashing-based techniques used in reductions
from \ThreeSUM{} to \ThreeSUMConv{}~(e.g. \cite{kopelowitz2016higher}) can map integer sets to indexed sets---\ThreeSUMConv{} is just \ThreeSUM{} for indexed sets).
Focusing on indexed sets makes the proof more intuitive, and also makes
the construction more efficient.  (In Appendix~\ref{app:bsg}, we present a variant of
the proof for general sets in an abelian group.)

Note that in the theorem stated below, we work with ``monochromatic'' difference
sets of the form $A-A$, instead of bichromatic sum sets $A+B$.  This
form is actually
more general, since given $A$ and $B$, we can reset $A$ to $A\cup (-B)$.
The reduction does not go the other way, since knowing that $|A^{(\lam)}+B^{(\lam)}|$
is small does not mean $|(A^{(\lam)}\cup (-B^{(\lam)})) - (A^{(\lam)}\cup (-B^{(\lam)}))|$ is small.  (The proofs for the known $O(s^6n)$ bound work only for the
bichromatic sum sets but not for monochromatic difference sets, whereas Gower's earlier
proof works for monochromatic difference sets.)

\begin{theorem}\label{thm:bsg:simple}
{\bf (Simpler BSG Covering)}\ \ 
Given indexed sets $A$ and $C$ of size $n$ and a parameter $s$,
there exist a collection of $\ell=\OO(s^3)$ subsets $A^{(1)},\ldots,A^{(\ell)}\subseteq A$, and a set $R$ of $\OO(n^2/s)$ pairs in $A\times A$, such that
\begin{enumerate}
\item[\rm(i)] $\{(a,b)\in A\times A: a-b\in C\}\ \subseteq\
R\,\cup\, \bigcup_\lam (A^{(\lam)}\times A^{(\lam)})$, and
\item[\rm(ii)] $\sum_\lam |A^{(\lam)} - A^{(\lam)}| = \OO(s^2 n^{3/2})$.
\end{enumerate}
The $A^{(\lam)}$'s and $R$ can be constructed in $\OO(n^2)$
Las Vegas randomized time.
\end{theorem}
\begin{proof}
Let $A=\{(i,a_i): i\in [n]\}$ and $C=\{(k,c_k): k\in [n]\}$.
As a preprocessing step, we
sort the multiset $\{a_{i+k}-a_i: i\in [n]\}$ for each $i$, 
in $\OO(n^2)$ total time.
Let $W_k=\{i\in [n]: a_{i+k} - a_i = c_k\}$.
\begin{itemize}
\item {\bf Few-witnesses case.}
For each $k$ with $|W_k|\le n/s$, add $\{((i+k,a_{i+k}),(i,a_i)): i\in W_k\}$ to $R$.
The number of pairs added to $R$ is $O(n\cdot n/s)$.
The running time of this step is also $\OO(n\cdot n/s)$.

\item {\bf Many-witnesses case.}
Pick a random subset $H\subseteq [\pm n]$ of size $c_0s\log n$ for a sufficiently large constant $c_0$.  
Let $L^{(h)}$ be the multiset $\{a_{i+h}-a_i: i\in [n]\}$.
Let $F^{(h)}$ be the elements of frequency more than $n/r$ in $L^{(h)}$.
Note that $|F^{(h)}|\le r$.
These can be computed in $\OO(sn)$ time.

\begin{itemize}
\item {\bf Low-frequency case.}
For each $h\in H$ and $i\in [n]$,
if $a_{i+h}-a_i\not\in F^{(h)}$,
we examine each of the at most $n/r$ indices $j$ with
$a_{i+h}-a_i=a_{j+h}-a_j$ and add $((j,a_j),(i,a_i))$ to $R$.
The number of pairs added to $R$ is $\OO(sn\cdot n/r)=\OO(n^2/s)$
by choosing $r:=s^2$.
The running time of this step is also bounded by $\OO(n^2/s)$.
\item {\bf High-frequency case.}
For each $h\in H$ and $f\in F^{(h)}$,
add the following subset to the collection:
\[ A^{(h,f)} = \{(i,a_i)\in A: a_{i+h}-a_i = f\}.
\]
The number of subsets is $\OO(s\cdot r)=\OO(s^3)$.
The running time of this step is \[O\left(\sum_{h\in H,\ f\in F^{(h)}}|A^{(h,f)}|\right)=\OO(sn).\]
\end{itemize}
\end{itemize}

\emph{Correctness.}
To verify (i), consider a pair $((i+k,a_{i+k}),(i,a_i))\in A\times A$ with 
$a_{i+k}-a_i=c_k$.
If $|W_k|\le n/s$, then $((i+k,a_{i+k}),(i,a_i))\in R$ due to
the ``few-witnesses'' case.
So assume $|W_k|>n/s$.
Then $H$ hits $W_k-i$ w.h.p., so
there exists $h\in H$ with $i+h\in W_k$, i.e.,
$a_{i+h+k}-a_{i+h}=c_k$.  Since $a_{i+k}-a_i=c_k$,
we have $a_{i+h+k}-a_{i+k}=a_{i+h}-a_i$ (by Fredman's trick).
Let $f=a_{i+h}-a_i$.
If $f\not\in F^{(h)}$, then $((i+k,a_{i+k}),(i,a_i))\in R$  due to
the ``low-frequency'' case.
If $f\in F^{(h)}$, then $((i+k,a_{i+k}),(i,a_i))\in A^{(h,f)}\times A^{(h,f)}$ due to the ``high-frequency'' case.

Thus far, the proof ideas are similar to what we have seen before.  However, to verify (ii), we will propose a new  probabilistic argument.
Consider a fixed $h$.
We want to bound the sum $S^{(h)}:=\sum_{f\in F^{(h)}} |A^{(h,f)}-A^{(h,f)}|$.
This is equivalent to bounding the number of triples $(k,c,f)$ such that
$f\in F^{(h)}$ and $\exists i$ with $a_{i+k}-a_i=c$ and
$a_{i+h}-a_i=a_{i+k+h}-a_{i+k}=f$.  Note that we also have
$a_{i+k+h}-a_{i+h}=c$ (by Fredman's trick).
Let $Y_{k,c} = \{i: a_{i+k}-a_i=c\}$
and $q$ be a parameter.
Then $S^{(h)}$ is upper-bounded by the number of triples $(k,c,f)$ satisfying
\begin{enumerate}
\item $|Y_{k,c}|>q$ and $f\in F^{(h)}$, or
\item $|Y_{k,c}|\le q$ and $\exists i$ with $i\in Y_{k,c}$ and $i+h\in Y_{k,c}$ and $f=a_{i+h}-a_i$.
\end{enumerate}
Since $\sum_{k,c} |Y_{k,c}|=O(n^2)$,
the number of triples of type~1 is $O((n^2/q)\cdot r)=\OO(s^2n^2/q)$.
On the other hand, the expected number of triples of type~2 for a random
$h\in[\pm n]$ is 
\[
O\left(\sum_{k,c:\: |Y_{k,c}|\le q} |Y_{k,c}|\cdot |Y_{k,c}|/n\right)
\ =\ O(n^2\cdot q/n)\ =\ O(nq).
\]
Hence, $\Ex[S^{(h)}] = \OO(s^2n^2/q + nq) = \OO(sn^{3/2})$
by choosing $q=s\sqrt{n}$.

The expected total sum $\sum_{h\in H}\sum_{f\in F^{(h)}} |A^{(h,f)}-A^{(h,f)}|$ is $\OO(s^2n^{3/2})$.
Thus, the probability that the desired bounds are not met is less than an arbitrarily small constant (by Markov's inequality and a union bound).

Note that the algorithm can be converted to Las Vegas, because we can
verify correctness of the construction by examining all $O(n^2)$ pairs
and computing all difference sets $A^{(h,f)}-A^{(h,f)}$ using
known output-sensitive algorithms~\cite{ColeHariharanSTOC02,ChanLewenstein,BringmannFN} in total time $\OO(n^2 + s^2n^{3/2})=\OO(n^2)$ (we may assume $s<n^{1/4}$, for otherwise the theorem is trivial).
\end{proof}

\subsection{Application: Improved 3SUM in Preprocessed Universes}
\label{sec:3sum:prep:rand}

As one immediate application, we can solve  \ThreeSUM{} with preprocessed
universe, improving Chan and Lewenstein's previous solution which required
$\OO(n^{13/7})$ query time~\cite{ChanLewenstein}, and also improving Corollary~\ref{cor:3sum:prep0} regardless of the value of $\omega$: the query algorithm does not use
fast matrix multiplication but uses FFT instead, though randomization is
now needed in the  preprocessing algorithm.

\begin{corollary}
We can preprocess sets $A$, $B$, and $C$ of $n$ integers
in $\OO(n^2)$ Las Vegas randomized time, so that given any subsets $A'\subseteq A$,
$B'\subseteq B$, and $C'\subseteq C$, we can solve \AllThreeSUM{} on $(A',B',C')$ in $\OO(n^{11/6})$ time.
\end{corollary}
\begin{proof}
Kopelowitz, Pettie and Porat~\cite{kopelowitz2016higher} gave a simple randomized reduction from \ThreeSUM{} to $O(\log n)$ instances of \ThreeSUMConv{} via hashing.
The same approach works in the preprocessed universe setting, and
transform the input into $O(\log n)$ instances where $A$, $B$, and $C$
are indexed sets.

During preprocessing, we apply Theorem~\ref{thm:bsg:simple}
to $A\cup (-B)$, producing subsets $A^{(\lam)}$ and a set $R$ of pairs.

During a query with given subsets $A'\subseteq A$,
$B'\subseteq B$, and $C'\subseteq C$,
we first examine each pair $(a,-b)\in R$ and check whether $a\in A'$,
$b\in B'$, and $a+b\in C'$.  This takes $\OO(n^2/s)$ time.

Next, for each $\lam$, we compute $(A^{(\lam)}\cap A') + ((-A^{(\lam)})\cap B')$
by known FFT-based algorithms~\cite{ColeHariharanSTOC02,ChanLewenstein,BringmannFN}; the running time is
near-linear in the output size, which is bounded by $|A^{(\lam)}-A^{(\lam)}|$.
For each output value $c$, we check whether $c\in C'$.

The total query time is $\OO(n^2/s + s^2n^{3/2})$.
Choosing $s=n^{1/6}$ yields the theorem.
\end{proof}

We remark that the same $\OO(n^{11/6})$ bound holds for a slightly more general case when
$C'$ is an arbitrary set of $n$ integers, i.e., the preprocessing does not need the set $C$.
This is because in the proof of Theorem~\ref{thm:bsg:simple},
the $\OO(n^2)$-time preprocessing step is independent of $C$, and
the rest of the construction takes $\OO(n^2/s +sn)$ time.
In contrast, Chan and Lewenstein's paper obtained a weaker 
$\OO(n^{19/10})$ bound for the same case without $C$.


\subsection{Reinterpreting Gower's Version}

In this section, we show that Gower's proof~\cite{Gowers01} 
(see also a related recent proof by Schoen~\cite{Schoen}) can also be modified to
construct a cover directly.\footnote{
There are multiple different exposition of Gower's and subsequent proofs of the BSG Theorem in the literature. For example, 
some presentations \cite{Balog07,SudakovSV94,TaoVu06,Viola11} cleanly separate the algebraic from the combinatorial components, by
reducing the problem to some combinatorial lemma about graphs (counting paths of length 2 or 3 or 4).
But these versions of the proof do not achieve our goal of computing a cover directly and efficiently.  Our reinterpretation is nontrivial and requires examining Gower's proof from the right perspective.
}  
Gower's proof requires more clever arguments; our new presentation highlights the similarities
and differences with the proof of Theorem~\ref{thm:bsg:simple}.
This variant of the proof will be needed in a later
application  in Section~\ref{sec:min-equal-conv} to conditional lower bounds---so ideas from additive combinatorics
will be useful after all!

In the following, we let $\pop_A(x)$ (the \emph{popularity of $x$}) denote the number of pairs $(a,b)\in A\times A$ with
$x=a-b$; in other words, $\pop_A(x)=|\{a\in A: a-x\in A\}|$.

\newcommand{\AAA}{\widetilde{A}}

\begin{theorem}
\label{thm:bsg:gower}
Given indexed sets $A$ and $C$ of size $n$ and a parameter $s$,
there exist a collection of $\ell=\OO(s^3)$ subsets $A^{(1)},\ldots,A^{(\ell)}\subseteq A$, and a set $R$ of $\OO(n^2/s)$ pairs in $A\times A$, such that
\begin{enumerate}
\item[\rm(i)] $\{(a,b)\in A\times A: a-b\in C\}\ \subseteq\
R\,\cup\, \bigcup_\lam (A^{(\lam)}\times A^{(\lam)})$, and
\item[\rm(ii)] $|A^{(\lam)} - A^{(\lam)}| = \OO(s^6 n)$ for each $\lam$
(and so $\sum_\lam |A^{(\lam)} - A^{(\lam)}| = \OO(s^9 n)$).
\end{enumerate}
The $A^{(\lam)}$'s and $R$ can be constructed in $\OO(n^2)$
Las Vegas randomized time.
\end{theorem}
\begin{proof}
We follow the proof in Theorem~\ref{thm:bsg:simple} but
modify the handling of the ``high-frequency'' case.
Let $t$ be a parameter.  Define 
\begin{eqnarray*}
G^{(h,f)}&=&\{ (a,b)\in A^{(h,f)}\times A^{(h,f)}: \pop_A(a-b)\le n/t\}\\
Z^{(h,f)}&=&\{a\in A^{(h,f)}: \deg_{G^{(h,f)}}(a) > |A^{(h,f)}|/4\}.
\end{eqnarray*}
Here, $\deg_{G^{(h,f)}}(a)$ refers to the degree of $a$ in $G^{(h,f)}$ when
viewed as a graph.
For each $h\in H$ and $f\in F^{(h)}$,
add $Z^{(h,f)}\times A^{(h,f)}$ and $A^{(h,f)}\times Z^{(h,f)}$ to $R$.
For each $h\in H$ and $f\in F^{(h)}$,
instead of adding the subset $A^{(h,f)}$, we add the subset
$\AAA^{(h,f)}:=A^{(h,f)}\setminus Z^{(h,f)}$ to the collection.

\smallskip
\emph{Correctness.}
To analyze this modified construction, first observe that every pair
previously covered by $A^{(h,f)}\times A^{(h,f)}$ is now covered by
$\AAA^{(h,f)}\times \AAA^{(h,f)}$ or by the extra pairs added to $R$ (i.e., $Z^{(h,f)}\times A^{(h,f)}$ or $A^{(h,f)}\times Z^{(h,f)}$).

We bound
the expected number of extra pairs added to $R$.
Consider a fixed $h$.
Note that $|Z^{(h,f)}|\le O\left(\frac{|G^{(h,f)}|}{|A^{(h,f)}|/4}\right)$, and so $\sum_f |Z^{(h,f)}||A^{(h,f)}| = O\left(\sum_f |G^{(h,f)}|\right)$.
The sum $\sum_f |G^{(h,f)}|$ is bounded by the number of triples $(i,j,f)$ such that
$a_{i+h}-a_i=a_{j+h}-a_j=f$ and $\pop_A((i-j,a_i-a_j))\le n/t$.
This is bounded by the number of pairs $(i,j)$ such that 
$\pop_A((i-j,a_i-a_j))\le n/t$ and
$a_i-a_j=a_{i+h}-a_{j+h}$ (by Fredman's trick).
For a fixed $(i,j)$ with $\pop_A((i-j,a_i-a_j))\le n/t$,
the number of $h$'s with $a_i-a_j=a_{i+h}-a_{j+h}$ is at most $n/t$,
and so the probability that $a_i-a_j=a_{i+h}-a_{j+h}$ for a random $h\in [\pm n]$ is $O(1/t)$.
It follows that 
\[ \Ex_h\left[ \sum_f |Z^{(h,f)}||A^{(h,f)}| \right] = O(n^2\cdot 1/t).
\]
Consequently, the expected number of extra pairs added to $R$
is $\OO(s\cdot n^2/t)=\OO(n^2/s)$ by setting $t := s^2$.

Finally, we consider a fixed $h$ and fixed $f\in F^{(h)}$ and provide an upper bound on
$|\AAA^{(h,f)}-\AAA^{(h,f)}|$.  For each $c\in \AAA^{(h,f)}-\AAA^{(h,f)}$,
pick a lexicographically smallest $(a,b)\in \AAA^{(h,f)}\times\AAA^{(h,f)}$
with $c=a-b$.
Consider all $y\in A^{(h,f)}$ with $(a,y),(b,y)\not\in G^{(h,f)}$;
the number of such $y$'s is at least $|A^{(h,f)}|-|A^{(h,f)}|/4-|A^{(h,f)}|/4=\Omega(|A^{(h,f)}|)=\Omega(n/r)$ (since $|A^{(h,f)}|>n/r$ for $f\in F^{(h)}$).  For each such $y$, examine
each $(a',a'')\in A\times A$ with $a-y=a'-a''$
and each $(b',b'')\in A\times A$ with $b-y=b'-b''$,
and mark the quadruple $(a',a'',b',b'')$.
Since $(a,y),(b,y)\not\in G^{(h,f)}$,
there are at least $n/t$ choices of $(a',a'')$ and
at least $n/t$ choices of $(b',b'')$ for each such $y$.
Letting $Q$ be the number of quadruples marked, we obtain 
\[ Q\ =\ \Omega\left(|\AAA^{(h,f)}-\AAA^{(h,f)}|\cdot (n/r)\cdot (n/t)^2\right).
\]
On the other hand, each quadruple $(a',a'',b',b'')$, is marked once,
since it uniquely determines the element $c=(a'-a'')-(b'-b'')$, from
which $(a,b)$ is uniquely determined and $y=a-(a'-a'')$ is uniquely
determined.  Thus, $Q=O(n^4)$.  We conclude that
\[ |\AAA^{(h,f)}-\AAA^{(h,f)}|\ =\ O\left(\frac{n^4}{(n/r)\cdot (n/t)^2}\right)\ =\ O(rt^2n)\ =\ \OO(s^6n).
\]

\smallskip
\emph{Construction time.}
One could naively construct the sets $Z^{(h,f)}$ and
$\AAA^{(h,f)}$ in $O(\sum_{h,f} |A^{(h,f)}|^2)$ time,
but a faster way is to use random sampling.
Given any value $c$, we can approximate $\pop_A(c)$
with additive error $\delta n/t$ w.h.p.\ by
taking a random subset $A'\subseteq A$
of $O((1/\delta^2)t\log n)=\OO(t)$ elements and computing
$|\{a\in A': a-c\in A\}|\cdot |A|/|A'|$ (by a standard Chernoff bound).
We will not construct $G^{(h,f)}$ explicitly.
Instead, given any $(a,b)$, we can test for membership in $G^{(h,f)}$
in $\OO(t)$ time.
Furthermore, given $a$, we can approximate $\deg_{G^{(h,f)}}(a)$
with additive error $\delta |A^{(h,f)}|$ w.h.p.\ by
taking a random subset $A''\subseteq A^{(h,f)}$
of $O((1/\delta^2)\log n)=\OO(1)$ elements and computing
$|\{y\in A'': (a,y)\in G^{(h,f)}\}|\cdot |A^{(h,f)}|/|A''|$.
This way, $Z^{(h,f)}$ (and thus $\AAA^{(h,f)}$) can be generated in $\OO(t|A^{(h,f)}|)$ time for each $h$ and $f$.  The total time bound is
$\OO(t\sum_{h,f}|A^{(h,f)}|)=\OO(tsn)=\OO(s^3n)$, which is dominated by
other costs (we may assume that $s<n^{1/6}$, for otherwise the theorem is trivial).
Due to these approximations, our earlier analysis needs small adjustments
in the constant factors, but is otherwise the same.

As before, the algorithm can be converted to Las Vegas in $\OO(n^2)$ additional time.
\end{proof}

\newcommand{\sss}{\hat{s}}

One important advantage of the above proof is that
the running time is actually \emph{subquadratic}, excluding the $\OO(n^2)$-time
preprocessing step, which is needed only in the ``few-witnesses'' case (ignoring the conversion to Las Vegas).
In particular, we immediately obtain subquadratic running time for the following variant of the theorem, which requires only the ``many-witnesses''
case (where we reset $r$ and $t$ to $s\sss$ instead of $s^2$).
This variant will be useful later.


\begin{theorem}
\label{thm:bsg:gower:fast}
Given an indexed set $A$ of size $n$ and parameters $s$ and $\sss$,
there exist a collection of $\ell=\OO(s^2\sss)$ subsets $A^{(1)},\ldots,A^{(\ell)}\subseteq A$, and a set $R$ of $\OO(n^2/\sss)$ pairs in $A\times A$, such that
\begin{enumerate}
\item[\rm(i)] $\{(a,b)\in A\times A: \pop_A(a-b) > n/s\}\ \subseteq\ 
R\,\cup\, \bigcup_\lam (A^{(\lam)}\times A^{(\lam)})$, and
\item[\rm(ii)] $|A^{(\lam)} - A^{(\lam)}| = \OO(s^3\sss^3 n)$ for each $\lam$
(and so $\sum_\lam |A^{(\lam)} - A^{(\lam)}| = \OO(s^5\sss^4 n)$).
\end{enumerate}
The $A^{(\lam)}$'s and $R$ can be constructed in $\OO(n^2/\sss + s^2\sss n)$
Monte Carlo randomized time.
\end{theorem}

Although we are able to reinterpret Gower's proof, we are unable to modify the proof by
Balog~\cite{Balog07} or Sudakov et al.~\cite{SudakovSV94} to achieve similar subquadratic
construction time.
