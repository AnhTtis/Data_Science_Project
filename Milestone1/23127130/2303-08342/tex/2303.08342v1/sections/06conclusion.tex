\section{Conclusion}\label{sec:Conclusion}

In conclusion, we proposed an architecture for a contextual PPAP that allows it to utilize multimodal features from the acoustic, visual, and participant domains in predicting perceptual ratings of augmented soundscapes while being compatible with its audio-only version via the zeroing of information from other domains. We established the efficacy of the modified architecture as \textit{ISO Pleasantness} models trained using the ARAUS dataset, and demonstrated how the contextual PPAP could also be used as a model to observe the impact of demographic factors on soundscape perception. Future work could involve the deployment of the contextual PPAP as a pre-trained model in an automatic masker selection system, followed by in-situ verification experiments to assess the ecological validity of the results obtained in this study in a real-life deployment context. Alternatively, the impact of other contextual factors, such as physiological measurements and ambient weather conditions, can be explored as additional input modalities for the contextual PPAP.