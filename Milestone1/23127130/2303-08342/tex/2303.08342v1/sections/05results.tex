\section{Results and Discussion}\label{sec:Results and Discussion}

\Cref{tab:ablation-results} displays the results of the validation experiments and abalation study described in \Cref{sec:Validation Experiments}. All models performed better than the baseline \textsc{aPPAP} except for the \textsc{ip}+\textsc{iv} variants with mid-level and late fusion. To quantify the significance of any performance differences, we performed Kruskal-Wallis tests with Bonferroni correction between the \textsc{aPPAP} and the variants of the \textsc{cPPAP} investigated for this study. All \textsc{ip}+\textsc{ev} variants had significantly improved performance over the \textsc{aPPAP}, indicating that the fusion of additional information from the participant modality allowed the trained models to better predict the \textsc{isoPl} values derived from the ARAUS dataset responses. With late fusion, the \textsc{ip}+\textsc{ev} variant also performed the best among all investigated models with a cross-validation MSE of 0.1183$\pm$0.0011, a 2.8\% improvement over the \textsc{aPPAP}.

\section{Results}
\label{results}

\begin{figure*}[ht]
    \centering
    \includegraphics[scale=0.15,trim={0 2.5cm 0 5cm},clip]{images/aoi-single_burst}
    \caption{The time average peak Age of Information with burst and \gls{soa} loss values against the dynamic reliability logic for different network topologies.}
    \label{fig:aoi_burst}\vspace{-0.4cm}
\end{figure*}


This paper focuses on both transport layer and application layer metrics to determine the feasibility of dynamic reliability. For this, we have selected the session packet volume, as transmitted, retransmitted, lost and backlogged packets as \glspl{kpi} for the transport layer; while focusing on the \gls{aoi} for the application layer. The \gls{aoi} was chosen as a crucial indicator for the freshness of packets in real-time applications. More specifically, this work adopts the time average peak \gls{aoi} equation \cite{aoi_equation} depicted in Eq. \ref{aoi}, where $\Delta(r_{i+1})$ is the $i$th update at the time it was received at the server, for a session time period of $\tau$.

\begin{equation}
    \label{aoi}
    \gls{aoi}_\tau = \frac{1}{n-1}\sum_{i=1}^{n-1} \Delta(r_{i+1})
\end{equation}

We include a comparison between the vanilla QUIC implementation which does not enjoy the dynamic reliability extension, with a number of dynamic reliability policies. The tests were run a number of times for statistical significance, with the mean value of vanilla implementation used as a baseline for comparison. The topology utilised both random loss and bursty loss to explore the bounds of dynamic reliability. The \gls{soa} loss in the figures correspond to the loss values presented in Table. \ref{tab:path_char}, for ease of comparison between bursty and random loss scenarios.

\subsection{Transport-Layer KPIs}

To analyse the performance gain at the transport layer due to dynamic reliability, the volume of transmitted and backlogged packets is examined. The figures are in the form of boxplots, which take the vanilla implementation as a benchmark, depicted as the red dashed line.

As seen in Fig. \ref{fig:sent_burst}, the loss plays a crucial role in the performance of the reliability policies. The policies under random loss did incredibly well for the networks with a larger capacity, namely \gls{mmwave} and Sub-6~GHz, whereas for burst loss, the lower network capacities had a larger packet reduction. With the increase in burst loss, the behaviour of the set split reliable policies became unpredictable, if a reliable assignment happened to coincide with a burst loss, the number of transmitted packets increases, and vice versa. On the other hand, in smarter policies, such as Loss-Aware, the performance lightly matched the vanilla baseline, as the reliable assignment dominated the session to compensate for a higher burst loss. Not only that but, the burst loss also impacted the variance of the transmitted packets for the policies.

Unsurprisingly, the unreliable focused policy, 80-20 split, outperformed other policies for all topologies in random and bursty loss scenarios, with an approximate reduction of 80\%. That being said, the majority of the policies reduced the transmitted packets on the link by approximately 70\% for random loss, while the reduction started at $\approx 15\%$ and decreased as the loss increased for the burst loss scenario.

The retransmitted and lost packets, not shown due to space limitations, followed the same trend as the transmitted packets for the random loss scenarios. However, for the burst loss scenarios, the larger capacity networks had a lower reduction in the retransmitted and lost packets. This can be seen as a favorable outcome since the lower capacity networks are scarce on resources. It is important to note that the Loss-Aware policy mimicked the vanilla approach as the burst loss increased, signifying the overwhelming appointment of reliable packets in adapting to the harsh burst loss conditions.
 
Alternatively, Fig. \ref{fig:backlog_burst} clearly shows a stark comparison between the policies and loss scenario in the reduction of the backlogged packets. The Loss-Aware policy for random loss scenario reduced the backlogged packets by up to 50\%, beating all other policies by approximately 30\%. Furthermore, it is clear that the unreliability focused policies resulted in the lowest backlog for the session. In comparison, we notice that the burst loss and the backlogged frequency have a positive correlation, where the maximum reduction of the backlogged packets for the policies is at most 20\%. Much like the transmitted packets, the probability of a burst loss occurrence plays a vital role in the number of retransmissions sent and by extension the number of backlogged packets. Thus, we can conclude that the stress placed on the buffer is a result of the reliable packets which is tightly coupled with the congestion on the session. Whereas, unreliable focused policies did not encounter such a phenomenon regardless if it was experiencing a burst loss.


\subsection{Application-Layer KPIs}

The feasibility of dynamic reliability for real-time applications can be determined by the \gls{aoi}, with comparison across different topologies and policies. If we take a strict approach and consider anything below $10$~ms is real-time \cite{real-time}, then all the reliability policies passed that requirement, which is attractive for real-time applications, as shown in Fig. \ref{fig:aoi_burst}. Utilising the median as an estimate of the runs, the policies in the WLAN and Sub-6~GHz topology with random loss floated around $4-5$~ms with negligible difference, while the \gls{aoi} for \gls{mmwave} was $\approx 2-3$~ms. It is clear that the \gls{aoi} and the network capacity have a negative correlation, as the network capacity decreases, the \gls{aoi} increases. The same correlation is extended to the bursty loss scenarios, where \gls{mmwave} dominated the other topologies. That being said, it is crucial to note that the \gls{aoi} for the reliability policies is often slightly better than or equal to the \gls{aoi} of the vanilla implementation, proving that dynamic reliability reduces the congestion of the session at no cost to the \gls{aoi}.


Moreover, the \textsc{ep}+\textsc{iv} variants, which used additional information from the visual modality, also performed better than the \textsc{aPPAP} but improvements were insignificant. This could be because the images used to derive the visual embeddings $\boldsymbol{r}$ were an \textit{objective} characteristic of the environment, whereas the participant embeddings $\boldsymbol{h}$ captured the participants' \textit{subjective} perception of the environment, thus making the \textsc{ep}+\textsc{iv} variants perform worse than the \textsc{ip}+\textsc{ev} variants. Alternative inputs from the visual modality better representing subjective perception could involve the subjectively-rated visual amenity and visual pleasantness \cite{Ricciardi2015SoundData}, if further data collection beyond the present iteration of the ARAUS dataset can be performed.

In addition, the \textsc{ip}+\textsc{iv} variants, which used information from both the participant and visual modalities, only had significant improvement over the \textsc{aPPAP} with early fusion at the feature augmentation block $f_g$. With mid-level and late fusion, the \textsc{ip}+\textsc{iv} variant actually performed worse than the \textsc{aPPAP}, which possibly hints at overfitting for the \textsc{mf} and \textsc{lf} scenarios due to the combined increase in number of non-acoustic predictor variables used at those stages. Therefore, early fusion via \Cref{eq:f_g_EF} is likely to be more suitable for the combination of multiple modalities.

Finally, at inference time, models using information from the participant modality can be used to simulate the ratings of hypothetical participants experiencing the same soundscape by varying $\boldsymbol{p}$ while keeping $\boldsymbol{s}$, $\boldsymbol{m}$, $\gamma$, and $\boldsymbol{b}$ constant. Averaging the ratings over a large variety of soundscapes, such as that in the ARAUS dataset, thus alllows us to isolate changes in perception due purely to participant-linked information while heightening the generalizability of both the models and the dataset.

As an illustration, using a trained \textsc{ip}+\textsc{iv} model undergoing early fusion, we stimulated hypothetical participants experiencing all the audio-visual stimuli in the ARAUS dataset, and present the mean \textsc{isoPl} ratings in \Cref{fig:demo-exploration}. These participants were each represented by a different value of $\boldsymbol{p}$, where individual dimensions were varied while maintaining all other dimensions at their mean values in the training set (a ``ceteris paribus'' assumption). We can see, for instance, that mean \textsc{isoPl} ratings decrease nonlinearly with increasing noise sensitivity, and that there is a fairly linear relationship between the satisfaction that a hypothetical participant has with the overall acoustic environment in Singapore with the same \textsc{isoPl} ratings.

\begin{figure}[!t]
    \centering
    \includegraphics[width=0.95\columnwidth]{assets/demo-exploration-means.pdf}
    \vspace{-15pt}
    \caption{Mean \textsc{isoPl} predictions by the \textsc{cPPAP} (\textsc{ip}+\textsc{iv}+\textsc{ef} variant, seed 2) across all ARAUS dataset samples as a function of $[0,1]$-normalized PIQ items used in \Cref{sec:Validation Experiments}. Faded vertical lines denote the mean values of the same PIQ items within the ARAUS dataset.}
    \label{fig:demo-exploration}
    \vspace{-10pt}
\end{figure}