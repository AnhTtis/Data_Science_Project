\section{Problem Statement}\label{sec:problem}

Given a set of jobs $\mathcal{J}=\{0,...,J\}$ that represent a mission and a set of time units $\mathcal{T}=\{0,...,T\}$ that represents the orbit period, the objective function \eqref{objetivo} represents the goal of maximizing the mission quality of service (QoS) metric, which is represented as the sum of the priority values $u_{j,t}$ for all the jobs $j$ over all the periods $t$.
\begin{align} \label{objetivo}
QoS :~~
	\underset{x_{j,t}}{\max} ~~
	\underbrace{\sum_{j=1}^{J} \sum_{t=1}^{T} {u}_{j,t} x_{j,t} }_{ \text{Quality of Service}}
% x_{j,t}\in\{0,1\}, ~~\quad \forall j\in\mathcal{J}, t\in\mathcal{T} 
\end{align}
Variable $x_{jt}$ represents the binary decision of scheduling job $j$ at time $t$, which takes on value $1$ if job $j$ is scheduled to run at time $t$ and $0$ otherwise.

In constraints \eqref{phiA} to \eqref{phiE}, the variable $\phi_{j,t}$ is used to describe the period between task executions. In essence, these equations enforce the relationship between $\phi_{j,t}$ and $x_{j,t}$ such that $\phi_{j,t}$ assumes value 1 only in the time step the task $j$ started running and is later used to reflect the desired period between task executions in the scheduling problem.
%
\begin{subequations} \label{qos_constraints}
    \begin{align}
        &\phi_{j,t} \geq x_{j,t}, &~\forall j\in\mathcal{J},\, t = 1 \label{phiA} \\
        %%%%%%%%%%%%%%%%%%%%%%%%%%%%%%%%%%%%%%%%%%%%%%%%%%%%%%%%%%%%%%%%%%%%%%%%%%%%%%%%%%%%%%%%%%%%%%%%%%%
        &\phi_{j,t} \geq x_{j,t} - x_{j,(t-1)}, &~\forall j\in\mathcal{J}, \,\forall t\in\mathcal{T}: t > 1  \label{phiB} \\
        %%%%%%%%%%%%%%%%%%%%%%%%%%%%%%%%%%%%%%%%%%%%%%%%%%%%%%%%%%%%%%%%%%%%%%%%%%%%%%%%%%%%%%%%%%%%%%%%%%%
        &\phi_{j,t} \leq x_{j,t}, &~\forall j\in\mathcal{J}, \,\forall t\in\mathcal{T}  \label{phiC} \\
        %%%%%%%%%%%%%%%%%%%%%%%%%%%%%%%%%%%%%%%%%%%%%%%%%%%%%%%%%%%%%%%%%%%%%%%%%%%%%%%%%%%%%%%%%%%%%%%%%%%
        &\phi_{j,t} \leq 2 - x_{j,t} - x_{j,(t-1)}, &~\forall j\in\mathcal{J}, \,\forall t\in\mathcal{T}: t > 1 \label{phiE} \\
        %%%%%%%%%%%%%%%%%%%%%%%%%%%%%%%%%%%%%%%%%%%%%%%%%%%%%%%%%%%%%%%%%%%%%%%%%%%%%%%%%%%%%%%%%%%%%%%%%%%        %%%%%%%%%%%%%%%%%%%%%%%%%%%%%%%%%%%%%%%%%%%%%%%%%%%%%%%%%%%%%%%%%%%%%%%%%%%%%%%%%%%%%%%%%%%%%%%%%%%
        &\sum_{t=1}^{\textcolor{black}{w^{\min}_{j}}} x_{j,t} = 0,  & \forall j\in\mathcal{J} \, \label{window1} \\
        %%%%%%%%%%%%%%%%%%%%%%%%%%%%%%%%%%%%%%%%%%%%%%%%%%%%%%%%%%%%%%%%%%%%%%%%%%%%%%%%%%%%%%%%%%%%%%%%%%%
        &\sum_{t=w^{\max}_{j}+1}^{T} x_{j,t} = 0, & \forall j\in\mathcal{J}  \label{window2}\\
        %%%%%%%%%%%%%%%%%%%%%%%%%%%%%%%%%%%%%%%%%%%%%%%%%%%%%%%%%%%%%%%%%%%%%%%%%%%%%%%%%%%%%%%%%%%%%%%%%%%
        &\sum_{l=t}^{t+{t}^{\min}_{j}-1} x_{j,l} \geq {t}^{\min}_{j} \phi_{j,t}, \forall t \in \{1,...,T-{t}^{\min}_{j} + 1\}, &\forall j\in\mathcal{J} \label{c} \\
        %%%%%%%%%%%%%%%%%%%%%%%%%%%%%%%%%%%%%%%%%%%%%%%%%%%%%%%%%%%%%%%%%%%%%%%%%%%%%%%%%%%%%%%%%%%%%%%%%%%
        &\sum_{l=t}^{t+{t}^{\max}_{j}} x_{j,l} \leq {t}^{\max}_{j}, ~\forall t \in \{1,...,T-{t}^{\max}_{j}\}, &\forall j\in\mathcal{J} \label{d} \\
        %%%%%%%%%%%%%%%%%%%%%%%%%%%%%%%%%%%%%%%%%%%%%%%%%%%%%%%%%%%%%%%%%%%%%%%%%%%%%%%%%%%%%%%%%%%%%%%%%%%
        &\sum_{l=t}^{T} x_{j,l} \geq (T - t + 1) \phi_{j,t}, \forall t \in \{T-{t}^{\min}_{j} + 2,...,T\}, & \forall j\in\mathcal{J}\, \label{e} \\
        %%%%%%%%%%%%%%%%%%%%%%%%%%%%%%%%%%%%%%%%%%%%%%%%%%%%%%%%%%%%%%%%%%%%%%%%%%%%%%%%%%%%%%%%%%%%%%%%%%%
        &\sum_{l=t}^{t+{p}^{\min}_{j}-1} \phi_{j,l} \leq 1,  \forall t \in \{1,...,T-{p}^{\min}_{j}+1\}, & \forall j\in\mathcal{J} \label{f} \\
        %%%%%%%%%%%%%%%%%%%%%%%%%%%%%%%%%%%%%%%%%%%%%%%%%%%%%%%%%%%%%%%%%%%%%%%%%%%%%%%%%%%%%%%%%%%%%%%%%%%
        & \sum_{l=t}^{t+{p}^{\max}_{j}-1} \phi_{j,l} \geq 1,  \forall t \in \{1,...,T-{p}^{\max}_{j}+1\}, & \forall j\in\mathcal{J} \label{g} \\
        %%%%%%%%%%%%%%%%%%%%%%%%%%%%%%%%%%%%%%%%%%%%%%%%%%%%%%%%%%%%%%%%%%%%%%%%%%%%%%%%%%%%%%%%%%%%%%%%%%%
        &\sum_{t=1}^{T} \phi_{j,t} \geq {y}^{\min}_{j}, &\forall j\in\mathcal{J}  \label{a} \\
        %%%%%%%%%%%%%%%%%%%%%%%%%%%%%%%%%%%%%%%%%%%%%%%%%%%%%%%%%%%%%%%%%%%%%%%%%%%%%%%%%%%%%%%%%%%%%%%%%%%
        &\sum_{t=1}^{T} \phi_{j,t} \leq {y}^{\max}_{j}, &\forall j\in\mathcal{J}  \label{b} \\
        %%
        & \phi_{j,t}\in\{0,1\}, &~\forall j\in\mathcal{J}, t\in\mathcal{T}  \label{phi_bin} \\
        %%%
        & x_{j,t}\in\{0,1\}, &~\forall j\in\mathcal{J}, t\in\mathcal{T}  \label{x_bin}  
    \end{align}
\end{subequations}


The constraints \eqref{window1} and \eqref{window2} are related to the execution of tasks in a given time window. The first, \eqref{window1}, states that the sum of the binary variables $x_{j,t}$ over the time interval $[1, w^{\min}_{j}]$ must be equal to zero, for all $j \in \mathcal{J}$. Here, $w^{\min}_{j}$ is the time when task $j$ can start execution, meaning that it cannot run before this time. 
The second type of constraints, \eqref{window2}, states that the sum of the binary variables $x_{j,t}$ over the time interval $[w^{\max}_{j} + 1, T]$ must also be equal to zero, for all $j \in \mathcal{J}$. 
Here, $w^{\max}_{j}$ is the maximum allowed time window for task $j$, and $T$ is the total number of time steps in the scheduling horizon. This means that if task $j$ starts, it must finish before the time point $w^{\max}_{j}$; otherwise the task cannot be executed. These constraints enforce that the tasks are executed only within the specified time windows, which can be used to ensure that a payload, for instance, runs only when passing above a certain territory. 


Now, constraints \eqref{c} ensure that if $\phi_{j,t}$ is 1, meaning that task $j$ started running at time $t$, then at least ${t}^{\min}_{j}$ units of $x_{j,l}$ in the corresponding time window must also be $1$. Similarly,  \eqref{d} ensures that the number of $x_{j,l}$ values equal to $1$ in the corresponding time window is limited by ${t}^{\max}_{j}$. 
Complementary, \eqref{e} ensures that if $\phi_{j,t}$ is 1, then all $x_{j,l}$ values from $t$ to the end of the time horizon must also be 1 so that, if a task starts at the end of the orbit, then it executes until the final time step. Therefore, \eqref{c} to \eqref{e} ensure the task running time requirements are met. Constraints \eqref{f} and \eqref{g} states that the sum $\phi_{j,t}$ over a window of size ${p}^{\min}_{j}$ or ${p}^{\min}_{j}$ must be equal to 1, ensuring that the period of execution of this task is respected.

Constraint \eqref{a} specifies that the sum of all values of $\phi_{j,t}$ for job $j$ must be greater than or equal to a lower limit ${y}^{\min}_{j}$. This means that the job must be performed a minimum number of times within the given time period. Constraint \eqref{b} specifies that the sum of all values of $\phi_{j,t}$ for job $j$ must be less than or equal to an upper limit ${y}^{\max}_{j}$. This means that the job must be performed at most the specified maximum number of times within the given time period.


Regarding the energy management formulations, equation \eqref{EN_b} calculates the balance energy at time step $t$, $b_t$, by subtracting the total energy generated from the solar panels ($q_j x_{j,t}$) from the total energy demand ($r_t$). The second equation, \eqref{EN_i}, calculates the energy required from or delivered to the battery ($i_t$) at time step $t$. 
%%
\begin{subequations} \label{energy_constraints}
\begin{align}
    &\sum_{j=1}^{J} q_{j} x_{j,t} \leq r_t + \gamma~V_{b}, & \forall t\in\mathcal{T} \label{EN_r} \\
    %%
    & b_{t} = r_{t} - \sum_{j \in \mathcal{J}} q_{j} x_{j,t}, &  \forall t \in \mathcal{T} \label{EN_b} \\
    %%%%%%%%%%%%%%%%%%%%%%%%%%%%%%%%%%%%%%%%%%%%%%%%%%%%%%%%%%%%%%%%%%%%%%%%%%%%%%%%%%%%%%%%%%%%%%%%%%%
    & i_{t} = {b_{t} \over V_{b}}, &  \forall t \in \mathcal{T} \label{EN_i}\\
    %%%%%%%%%%%%%%%%%%%%%%%%%%%%%%%%%%%%%%%%%%%%%%%%%%%%%%%%%%%%%%%%%%%%%%%%%%%%%%%%%%%%%%%%%%%%%%%%%%%
    &\text{SoC}_{t+1} = \text{SoC}_{t} + {{i_{t}~e} \over {60~Q}}, & \forall t \in \mathcal{T}  \label{EN_SOC1}\\
    %%%%%%%%%%%%%%%%%%%%%%%%%%%%%%%%%%%%%%%%%%%%%%%%%%%%%%%%%%%%%%%%%%%%%%%%%%%%%%%%%%%%%%%%%%%%%%%%%%%
    &\text{SoC}_{t} \leq 1, & \forall t\in\mathcal{T} \label{EN_SOC2}   \\
    %%%%%%%%%%%%%%%%%%%%%%%%%%%%%%%%%%%%%%%%%%%%%%%%%%%%%%%%%%%%%%%%%%%%%%%%%%%%%%%%%%%%%%%%%%%%%%%%%%%
    &\text{SoC}_{t} \geq \rho, & \forall t\in\mathcal{T} \label{EN_SOC3} 
    %%%%%%%%%%%%%%%%%%%%%%%%%%%%%%%%%%%%%%%%%%%%%%%%%%%%%%%%%%%%%%%%%%%%%%%%%%%%%%%%%%%%%%%%%%%%%%%%%%%
\end{align}
\end{subequations}


Equation \eqref{EN_SOC1} establishes the state of charge (SoC) of the battery at every unit of time in the orbit. It is given as the sum of the state of charge at time $t$ and the energy balance in this time step resulting from the current flowing in or out of the battery, expressed in terms of the battery capacity ($Q$). 
It also considers the battery charge and discharge efficiency ($e$). Constraints \eqref{EN_SOC2} state that the State of Charge (SoC) at any time must be less than or equal to $1$. This means that the battery can never be overcharged,
Complementary constraints \eqref{EN_SOC3} state that the SoC at any time must be greater than or equal to $\rho$ --- it is a typical practice in such sensitive applications to impose large margins of safety. Finally, constraints \eqref{EN_r} ensure that the power demand does not exceed power availability. The battery can provide up to $\gamma \cdot V_b$ Watts of power.  


%%%%%%%%%%%%%%%%%%%%%%%%
