\section{Conclusion}

This work has proposed a novel approach to tackle the ONTS problem using graph neural networks.
Our experiments showed that our proposed architecture, SatGNN, is successful in classifying the feasibility and the optimality of candidate solutions to varied instances of the ONTS problem.
Not only the model was able to generalize to unseen instances, but it also showed promising results on out-of-distribution instances, which were larger than the ones seen during training.
This shows how the inherent symmetries of graph neural networks make them suitable for dealing with the structures of optimization problems.

By leveraging on the optimality classification results, we used the SatGNN to generate candidate solutions to the binary variables of the problem.
Through these candidate solutions, we were able to fix the variables, reducing the size of the problem and, consequently, the time the solver takes to converge.
This approach for early fixing outperformed using the Gurobi solver alone, even when no tolerance is permitted for the optimal.
Furthermore, generalization to larger instances is still a challenge for early fixing, even though promising results were observed.

In summary, this work has shown how graph neural networks can be used to improve nanosatellite task scheduling.
As we propose a supervised learning method, our approach is still limited to the availability of labeled data, which can be costly to obtain.
Nonetheless, our results suggest that using graph neural networks for combinatorial optimization problems holds excellent promise and opens up new avenues for future research.
 