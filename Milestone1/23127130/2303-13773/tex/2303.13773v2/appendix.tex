\section{Random instance generation}\label{appx:random-instance}

For any particular mission size and orbital length, the main objective is to generate a realistic ONTS case using random data.
We have taken the FloripaSat-I mission as a reference for instance generation, which has an altitude of 628 kilometers and an orbital period of 97.2 minutes~\cite{marcelino_critical_2020}.
The battery-related parameters are fixed for all instances as
\begin{align*}
    e &\gets 0.9 \\
    Q &\gets 5 \\
    \gamma &\gets 5 \\
    V_b &\gets 3.6 \\
    \rho &\gets 0.0
\end{align*}
The attitude considered here is the Nadir, in which the satellite turns at the same rate around the Earth, so one side (or axis) always faces the Earth's surface.
This analytical model then utilizes a rotation matrix to simulate the satellite's dynamics and can be adapted for larger or different geometries by adjusting the normal vectors representing the body.

To generate realistic power input vectors $\boldsymbol{r}$, we use 2 years of historical data of solar irradiance in orbit.
More precisely, a window is randomly drawn from the historical data, and an analytical model is used to determine the power input vector.
Once orbits are stable and solar flux constant -- $1360 W/m^2$  -- one can calculate this vector by knowing the spacecraft orbit, attitude -- its kinematics -- and size~\cite{filho_comprehensive_2020}.

The remaining parameters are drawn uniformly, given handcrafted limits to increase the feasibility rate.
Specifically, given the number of tasks $J$ and the number of time units $T$, the parameters are drawn as
\begin{align*}
    u_j &\gets \mathcal{U}(1, J) \\
    q_j &\gets \mathcal{U}(0.3, 2.5) \\
    y_j^{\min} &\gets \mathcal{U}(1, \lceil T/45 \rceil) \\
    y_j^{\max} &\gets \mathcal{U}(y_j^{\min}, \lceil T/15 \rceil) \\
    t_j^{\min} &\gets \mathcal{U}(1, \lceil T/10 \rceil) \\
    t_j^{\max} &\gets \mathcal{U}(t_j^{\min}, \lceil T/4 \rceil) \\
    p_j^{\min} &\gets \mathcal{U}(t_j^{\min}, \lceil T/4 \rceil) \\
    p_j^{\max} &\gets \mathcal{U}(p_j^{\min}, T) \\
    w_j^{\min} &\gets \mathcal{U}(0, \lceil T/5 \rceil) \\
    w_j^{\max} &\gets \mathcal{U}(\lfloor T-\lceil T/5 \rceil \rfloor, T)
.\end{align*}
