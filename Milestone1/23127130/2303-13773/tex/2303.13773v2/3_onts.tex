\section{Offline Nanosatellite Task Scheduling (ONTS)}\label{sec:onts}

Nanosatellite scheduling problems concern the decisions on each task's start and finish time.
The tasks usually require periodic execution and during restricted moments along the orbit.
Besides time, energy availability through the orbit is a crucial resource to be considered.
Figure \ref{fig:example-scheduling} shows an example of optimal scheduling, in which each job is represented by a different color and the activation and deactivations are shown through the steps of the signals.
Proper scheduling must account for energy management such that the tasks do not draw more energy than the system can provide and that the battery is not depleted before the end of the mission.
Energy management is a difficult task since the nanosatellite draws power from its solar panels, with the energy availability depending on the attitude of the nanosatellite (which affects the orientation of the solar panels) and the trajectory with respect to Earth's shadow, as illustrated in Figure \ref{fig:onts-orbit}.

\begin{figure}
    \centering
    \includegraphics{schedule_example.pdf}
    \caption{Illustration of an optimum scheduling for 9 tasks on a horizon of 125 time steps. Each color represents the execution of the different tasks.}
    \label{fig:example-scheduling}
\end{figure}

\begin{figure}
    \centering
    \includegraphics[width=0.5\textwidth]{onts_orbit.png}
    \caption{Illustration of a nanosatellite's orbit around Earth. Image from Rigo et al.~\cite{rigo_branch-and-price_2022}.}
    \label{fig:onts-orbit}
\end{figure}

\subsection{MILP Formulation}\label{sec:problem}

A formulation that considers the realistic constraints and objectives of ONTS was proposed by Rigo et al.~\cite{rigo_task_2021} and is presented here.
Given a set of jobs $\mathcal{J}=\{0,...,J\}$ that represents the tasks, and a set of time units $\mathcal{T}=\{0,...,T\}$ that represents the scheduling horizon, variables \[
    \bm{x}=(x_{1,1},\ldots,x_{1,T},\ldots,x_{J,1},\ldots,x_{J,T})
    % \bm{x}=[x_{j,t}]_{\substack{j=1,\ldots,J \\ t=1,\ldots,T}}
\]
represent the allocation of jobs $j\in \mathcal{J}$ at times $t\in\mathcal{T}$, i.e., $x_{j,t}=1$ indicates that job $j$ is scheduled to execute at time $t$.
Naturally, $x_{j,t}$ are binary variables, i.e., $\bm{x} \in \{0,1\}^{JT}$.

It is assumed that there exists a priority for every job, which is defined by $\bm{u} = (u_1,\ldots,u_{J})$.
The goal is to maximize the mission's Quality of Service (QoS) \eqref{eq:qos}, which is the sum of the allocations weighted by the priorities,
\begin{align}\label{eq:qos}
{\rm QoS}(\bm{x};\bm{u}) =
	% \underset{x_{j,t}}{\max} ~~
	\sum_{j=1}^{J} \sum_{t=1}^{T} {u}_{j} x_{j,t}
% x_{j,t}\in\{0,1\}, ~~\quad \forall j\in\mathcal{J}, t\in\mathcal{T} 
\end{align}

To formalize the ONTS problem of maximizing QoS, we follow the formulation proposed by Rigo et al.~\cite{rigo_task_2021}.
The following constraints are added to ensure that the scheduling respects the requirements and specificities of each job:
\begin{subequations}\label{eq:qos-constraints}
    \begin{align}
        &\phi_{j,t} \geq x_{j,t}, &~\forall j\in\mathcal{J},\, t = 1 \label{phiA} \\
        %%%%%%%%%%%%%%%%%%%%%%%%%%%%%%%%%%%%%%%%%%%%%%%%%%%%%%%%%%%%%%%%%%%%%%%%%%%%%%%%%%%%%%%%%%%%%%%%%%%
        &\phi_{j,t} \geq x_{j,t} - x_{j,(t-1)}, &~\forall j\in\mathcal{J}, \,\forall t\in\mathcal{T}: t > 1  \label{phiB} \\
        %%%%%%%%%%%%%%%%%%%%%%%%%%%%%%%%%%%%%%%%%%%%%%%%%%%%%%%%%%%%%%%%%%%%%%%%%%%%%%%%%%%%%%%%%%%%%%%%%%%
        &\phi_{j,t} \leq x_{j,t}, &~\forall j\in\mathcal{J}, \,\forall t\in\mathcal{T}  \label{phiC} \\
        %%%%%%%%%%%%%%%%%%%%%%%%%%%%%%%%%%%%%%%%%%%%%%%%%%%%%%%%%%%%%%%%%%%%%%%%%%%%%%%%%%%%%%%%%%%%%%%%%%%
        &\phi_{j,t} \leq 2 - x_{j,t} - x_{j,(t-1)}, &~\forall j\in\mathcal{J}, \,\forall t\in\mathcal{T}: t > 1 \label{phiE} \\
        %%%%%%%%%%%%%%%%%%%%%%%%%%%%%%%%%%%%%%%%%%%%%%%%%%%%%%%%%%%%%%%%%%%%%%%%%%%%%%%%%%%%%%%%%%%%%%%%%%%        %%%%%%%%%%%%%%%%%%%%%%%%%%%%%%%%%%%%%%%%%%%%%%%%%%%%%%%%%%%%%%%%%%%%%%%%%%%%%%%%%%%%%%%%%%%%%%%%%%%
        &\sum_{t=1}^{\textcolor{black}{w^{\min}_{j}}} x_{j,t} = 0,  & \forall j\in\mathcal{J} \, \label{window1} \\
        %%%%%%%%%%%%%%%%%%%%%%%%%%%%%%%%%%%%%%%%%%%%%%%%%%%%%%%%%%%%%%%%%%%%%%%%%%%%%%%%%%%%%%%%%%%%%%%%%%%
        &\sum_{t=w^{\max}_{j}+1}^{T} x_{j,t} = 0, & \forall j\in\mathcal{J}  \label{window2}\\
        %%%%%%%%%%%%%%%%%%%%%%%%%%%%%%%%%%%%%%%%%%%%%%%%%%%%%%%%%%%%%%%%%%%%%%%%%%%%%%%%%%%%%%%%%%%%%%%%%%%
        &\sum_{l=t}^{t+{t}^{\min}_{j}-1} x_{j,l} \geq {t}^{\min}_{j} \phi_{j,t},  &\forall t \in \{1,...,T-{t}^{\min}_{j} + 1\}, \forall j\in\mathcal{J} \label{c} \\
        %%%%%%%%%%%%%%%%%%%%%%%%%%%%%%%%%%%%%%%%%%%%%%%%%%%%%%%%%%%%%%%%%%%%%%%%%%%%%%%%%%%%%%%%%%%%%%%%%%%
        &\sum_{l=t}^{t+{t}^{\max}_{j}} x_{j,l} \leq {t}^{\max}_{j},  &\forall t \in \{1,...,T-{t}^{\max}_{j}\}, \forall j\in\mathcal{J} \label{d} \\
        %%%%%%%%%%%%%%%%%%%%%%%%%%%%%%%%%%%%%%%%%%%%%%%%%%%%%%%%%%%%%%%%%%%%%%%%%%%%%%%%%%%%%%%%%%%%%%%%%%%
        &\sum_{l=t}^{T} x_{j,l} \geq (T - t + 1) \phi_{j,t},  & \forall t \in \{T-{t}^{\min}_{j} + 2,...,T\}, \forall j\in\mathcal{J}\, \label{e} \\
        %%%%%%%%%%%%%%%%%%%%%%%%%%%%%%%%%%%%%%%%%%%%%%%%%%%%%%%%%%%%%%%%%%%%%%%%%%%%%%%%%%%%%%%%%%%%%%%%%%%
        &\sum_{l=t}^{t+{p}^{\min}_{j}-1} \phi_{j,l} \leq 1,   & \forall t \in \{1,...,T-{p}^{\min}_{j}+1\}, \forall j\in\mathcal{J} \label{f} \\
        %%%%%%%%%%%%%%%%%%%%%%%%%%%%%%%%%%%%%%%%%%%%%%%%%%%%%%%%%%%%%%%%%%%%%%%%%%%%%%%%%%%%%%%%%%%%%%%%%%%
        & \sum_{l=t}^{t+{p}^{\max}_{j}-1} \phi_{j,l} \geq 1,  & \forall t \in \{1,...,T-{p}^{\max}_{j}+1\},  \forall j\in\mathcal{J} \label{g} \\
        %%%%%%%%%%%%%%%%%%%%%%%%%%%%%%%%%%%%%%%%%%%%%%%%%%%%%%%%%%%%%%%%%%%%%%%%%%%%%%%%%%%%%%%%%%%%%%%%%%%
        &\sum_{t=1}^{T} \phi_{j,t} \geq {y}^{\min}_{j}, &\forall j\in\mathcal{J}  \label{a} \\
        %%%%%%%%%%%%%%%%%%%%%%%%%%%%%%%%%%%%%%%%%%%%%%%%%%%%%%%%%%%%%%%%%%%%%%%%%%%%%%%%%%%%%%%%%%%%%%%%%%%
        &\sum_{t=1}^{T} \phi_{j,t} \leq {y}^{\max}_{j}, &\forall j\in\mathcal{J}  \label{b} \\
        %%
        & \phi_{j,t}\in\{0,1\}, &~\forall j\in\mathcal{J}, t\in\mathcal{T}  \label{phi_bin} \\
        %%%
        & x_{j,t}\in\{0,1\}, &~\forall j\in\mathcal{J}, t\in\mathcal{T}.  \label{x_bin}  
    \end{align}
\end{subequations}
Note that auxiliary binary variables \[
    \bm{\phi}=(\phi_{1,1},\ldots,\phi_{1,T},\ldots,\phi_{J,1},\ldots,\phi_{J,T})
\] are used, which take a positive value if, and only if, job $j$ was not running at time $t-1$, but started running at time $t$.
Constraints \eqref{phiA} to \eqref{phiE} are used to ensure this behavior.
We also consider that jobs may run only during a time window defined by parameters $w^{\min}_{j}$ and $w^{\max}_{j}$, which is ensured by constraints \eqref{window1} and \eqref{window2}.
Such behavior is necessary to ensure that a payload, for instance, runs only when passing above a certain territory.

Jobs may also have limits on continuous execution.
If a job $j$ starts running at time $t$, then it must run for at least $t^{\min}_{j}$ time steps, and at most $t^{\max}_{j}$ time steps.
This is ensured by constraints \eqref{c} and \eqref{d}.
The formulation, through constraint \eqref{e}, also allows a job to start at the last time steps and keep running until the end, assuming it will keep running at the start of the following schedule.

A job may require to be executed periodically, at least every $p^{\min}_{j}$ time steps, and at most every $p^{\max}_{j}$ time steps.
This is ensured through constraints \eqref{f} and \eqref{g}, over the $\phi_{j,t}$ variables.
A job may also require multiple executions through the planning horizon, starting at least $y^{\min}_{j}$ times, and at most $y^{\max}_{j}$ times, which is ensured through constraints \eqref{a} and \eqref{b}.

Beyond job-specific restrictions in equation \eqref{eq:qos-constraints}, the formulation also covers energy management,
\begin{subequations} \label{eq:energy-constraints}
\begin{align}
    &\sum_{j=1}^{J} q_{j} x_{j,t} \leq r_t + \gamma~V_{b}, & \forall t\in\mathcal{T} \label{EN_r} \\
    & b_{t} = r_{t} - \sum_{j \in \mathcal{J}} q_{j} x_{j,t}, &  \forall t \in \mathcal{T} \label{EN_b} \\
    & i_{t} = \frac{b_{t}}{V_{b}}, &  \forall t \in \mathcal{T} \label{EN_i}\\
    &\text{SoC}_{t+1} = \text{SoC}_{t} + \frac{i_{t}~e}{60~Q}, & \forall t \in \mathcal{T}  \label{EN_SOC1}\\
    &\text{SoC}_{t} \leq 1, & \forall t\in\mathcal{T} \label{EN_SOC2}   \\
    &\text{SoC}_{t} \geq \rho, & \forall t\in\mathcal{T} \label{EN_SOC3} 
\end{align}
\end{subequations}
Parameter $r_t$ provides the power available at time $t$ from the solar panels and $q_j$ the power required for executing job $j$.
Thus, constraint \eqref{EN_r} limits the power consumption, with $\gamma \cdot V_b$ being the maximum power the battery can provide.
Constraints \eqref{EN_b} to \eqref{EN_SOC1} update the auxiliary variables $b_t$ and ${\rm SoC}_t$, which represent the exceeding power and State of Charge (SoC) at time $t$, based on the battery capacity $Q$ and the discharge efficiency $e$.
The SoC of the battery must stay within the limits given in constraints \eqref{EN_SOC2} and \eqref{EN_SOC3}, with $\rho$ being a lower limit usually greater than zero as a safety measure.

Thus, the MILP formulation is the maximization of \eqref{eq:qos} while subject to constraints \eqref{eq:qos-constraints} and \eqref{eq:energy-constraints}.
In other words, we can write any instance $I\in\mathcal{I}$, where $\mathcal{I}$ is the set of all possible instances of the ONTS problem, as
\begin{equation}\label{eq:formulation}
\begin{split}
    I : \max_{\bm{x},\bm{\phi},{\rm SoC}} ~& \underbrace{\sum_{j=1}^{J} \sum_{t=1}^{T} {u}_{j} x_{j,t} }_{ \text{QoS}}  \\
    \text{s.t.}  ~& \eqref{eq:qos-constraints},\eqref{eq:energy-constraints}  \\
    & \bm{x},\bm{\phi} \in \{0,1\}^{JT} , {\rm SoC}\in [0,1]^T
.\end{split}
\end{equation}
Note that the continuous variables ${\rm SoC}$ can be completely determined by the binary variables $\bm{x}$ and $\bm{\phi}$, so our problem can be reduced to finding an assignment $\bm{z}=(\bm{x},\bm{\phi}) \in \{0,1\}^{2JT}$.

Any instance $I\in\mathcal{I}$ is parameterized by the number of tasks $J$, the number of time units $T$ and the parameters $\bm{u}$, $\bm{q}$, $\bm{y}^{\min}$, $\bm{y}^{\max}$, $\bm{t}^{\min}$, $\bm{t}^{\max}$, $\bm{p}^{\min}$, $\bm{p}^{\max}$, $\bm{w}^{\min}$, $\bm{w}^{\max}$, $\bm{r}$, $\rho$, $e$, $Q$, $\gamma$, and $V_b$.
We will denote $\Pi_{J,T}$ the parameter space through which any instance $I\in\mathcal{I}$ can be uniquely determined by some parameter vector $\pi\in\Pi_{J,T}$ (given adequate $J$ and $T$).



%%%%%%%%%%%%%%%%%%%%%%%%%%%
