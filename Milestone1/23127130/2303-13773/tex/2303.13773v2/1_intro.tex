\section{Introduction} \label{sec:intro}

Nanosatellites have attracted attention from the industry and academia for over a decade, primarily because of their applications at reduced development and launch costs~\cite{shiroma_cubesats_2011,lucia_computational_2021,Nagel2020,saeed_cubesat_2020}.
Due to limited computational and energy resources, this spacecraft standard is associated with difficulties in mission planning.
Optimizing task scheduling is key for guaranteeing a return on investment, as it maximizes resource usage, increases data quality, and brings about cost savings and mission success.
Therefore, the Offline Nanosatellite Task Scheduling (ONTS) problem is crucial in developing, deploying, and operating nanosatellites in orbit.
The problem aims at finding a schedule for task execution in orbit that maximizes Quality-of-Service (QoS), taking into account factors such as priority, minimum and maximum activation events, execution time-frames, periods, and execution windows, as well as the limitations of the satellite's power resources and the complexities of energy harvesting and management systems.

During the lifetime of a mission (from launch to disposal), the ONTS problem must be solved recurrently and iteratively.
New schedules must be generated (and deployed) over time, and the optimal schedule is found by iteratively figuring out how many tasks and which tasks are possible to fit in the scheduled timespan.
Traditional mathematical formulations and exact algorithms have been proposed to solve the ONTS problem, starting from Integer Programming (IP)~\cite{rigo_task_2021} to Mixed Integer Linear Programming (MILP)~\cite{rigo_nanosatellite_2021,seman_energy-aware_2022} and Continuous-Time techniques~\cite{camponogara_continuous-time_2022}.
However, due to the NP-hard nature of the problem, efficiently solving large problems, i.e., fitting many tasks and planning for a long time horizon, is still an open area for research.

Meanwhile, several recent investigations~\cite{bengio_machine_2021,karimi-mamaghan_machine_2022,pacheco2023deeplearningbased,han_gnn-guided_2023} have considered machine learning tools to address combinatorial optimization problems, such as the single machine problem~\cite{parmentier_structured_2023}, resource-constrained project scheduling~\cite{guo_prediction_2023}, and knapsack problems~\cite{yang_learning_2022}.
Graph Neural Networks (GNNs), in particular, have gained popularity in recent years to solve combinatorial optimization problems when the underlying structure can be represented as a graph~\cite{zhang_survey_2023}, which is the case for MILP~\cite{khalil_mip-gnn_2022}.

This paper investigates the novel application of GNNs for the ONTS problem.
More specifically, we consider two research questions:
\begin{itemize}
    \item Can a graph neural network learn the structure of the ONTS problem?
    \item Is a GNN-based heuristic effective (fast and accurate) for the ONTS problem?
\end{itemize}
To address these two questions, we propose two sets of experiments.
% In the first, we train GNNs to predict the feasibility and optimality of candidate solutions given an instance of the problem.
% In our results, the deep learning models successfully learned both characteristics and generalized to unseen instances.
% In the second set of experiments, we train GNNs to generate candidate solutions given problem instances and use the resulting models to build matheuristics.
% Our results show the GNN-based matheuristics being particularly successful for quickly providing high-quality solutions, overcoming the MILP solver (SCIP).
First, we train GNNs to predict the feasibility and optimality of candidate solutions given an instance of the problem.
In the second set of experiments, we train GNNs to generate candidate solutions given problem instances and use the resulting models to build matheuristics~\cite{boschetti_matheuristics_2022}.
Our results indicate that: the proposed GNN can learn the feasibility and optimality of candidate solutions, even when given instances larger (and harder) than those seen during training.
Furthermore, GNN-based matheuristics effectively provide high-quality solutions for large problems, overcoming the MILP solver (SCIP) both in the time to generate feasible solutions and the quality of the solutions found during limited time.
In summary, the main contributions of this paper are as follows:
\begin{itemize}
\item The introduction of SatGNN, a pioneering GNN architecture suitable to the ONTS problem;
\item A showcase of SatGNN's impressive performance in accurately classifying feasibility, predicting optimality, and generating effective heuristic solutions;
\item A demonstration of GNN's robust generalization abilities for optimization problems, particularly when dealing with larger ONTS instances;
\item An implication of the potential for a transformative impact on complex space mission scheduling.
\end{itemize}

% Our experimental results show promising results, with the resulting heuristic overcoming the SCIP solver in the hardest problems.
% This study investigates applications of GNNs for the ONTS problem
% Taking advantage of all this recent progress in GNN research and its successful application to optimization problems, this study proposes a novel solution methodology to the ONTS problem.
% By representing the problem as a bipartite graph, we leverage the robust representation learning capabilities of GNNs.
% The parameters of the MILP problem generate feature vectors fed into the model, allowing us to encode both the structure and parameters of each instance of the problem.
% GNNs can handle graphs of arbitrary size to handle optimization problems with varying numbers of variables and constraints.
% Our method employs two fully-connected, single-layer multilayer perceptron (MLP) networks with ReLU activations to encode the features of variables and constraints into hidden features that are updated using a two-step message-passing mechanism.
% The parameters can then be optimized similarly to conventional deep learning models.
% The proposed GNN model is then shown here to accelerate task scheduling by efficiently learning the relationships between tasks and resources and optimizing mission planning.

The remainder of this paper is organized as follows:
Section \ref{sec:rel-work} provides an overview of the related literature.
Section \ref{sec:onts} describes the problem in detail, providing context and background information and formulating the optimization problem.
Section \ref{sec:gnns} provides the necessary theoretical background on graph neural networks and their application to linear optimization problems.
The data acquisition, the proposed GNN architecture, and the heuristics for optimization problems are presented in Section \ref{sec:meth}.
The experiments that tackle the two research questions posed above along with the results are presented in Section \ref{sec:exps}.
Finally, Sections \ref{sec:discussion} and \ref{sec:conclusion} discuss the key findings and provide concluding remarks and future research directions.


