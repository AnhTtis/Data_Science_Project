\documentclass[aps,reprint,twocolumn,english,pra,notitlepage,nofootinbib,floatfix
%,longbibliography
%,superscriptaddress
%linenumbersskdjbsvbisvbisubvss
]{revtex4-2}
%%%%%%%%%%%%%%%%%%%%%%%%%%%%%%%%%%%%%%
%% ADD OR REMOVE "final" TO INCLUDE/REMOVE TODOS AND SECTION HEADINGS. Add/remove linenumbers 
%%%%%%%%%%%%%%%%%%%%%%%%%%%%%%%%%%%%%%
\usepackage[T1]{fontenc}
\usepackage[utf8]{inputenc}
\setcounter{secnumdepth}{3}
\usepackage{refstyle}
\usepackage{amsmath}
\usepackage{bm} % for bold symbols in math mode
\usepackage{amssymb}
\usepackage[pdftex, hidelinks]{hyperref}

\usepackage[usenames,dvipsnames]{xcolor}
\usepackage{bbm}
\usepackage{tikz}
\usepackage{xspace}
\usepackage{bm}
\usepackage{booktabs}
\usepackage{babel}
\usepackage[separate-uncertainty=true]{siunitx}
\usepackage[makeroom]{cancel}
\usepackage{placeins}
\usepackage{physics}
%\usepackage[]{placeins} % HACK: Force floats to be in their original section. Used for appendix.
\usepackage[obeyFinal,colorinlistoftodos,color=green]{todonotes} %% CHANGE FINAL/DRAFT IN DOCUMENTCLASS
%\usepackage{ifdraft} %% CHANGE FINAL/DRAFT IN DOCUMENTCLASS
%\usepackage[Symbolsmallscale]{upgreek}
\usepackage[normalem]{ulem} % for crossing out text (ie text strikethrough)
\usepackage{wasysym}
\usepackage{graphicx}
\usetikzlibrary{matrix}
\newcommand\myshade{85}
\colorlet{mylinkcolor}{violet}
\colorlet{mycitecolor}{YellowOrange}
\colorlet{myurlcolor}{Aquamarine}
\hypersetup{
  linkcolor  = mylinkcolor!\myshade!black,
  citecolor  = mycitecolor!\myshade!black,
  urlcolor   = myurlcolor!\myshade!black,
  colorlinks = true,
}
\usepackage{soul}

\renewcommand{\t}[1]{\mathrm{#1}}
\newcommand{\blue}[1]{\textcolor{blue}{#1}}
\newcommand{\green}[1]{\textcolor{green}{#1}}
\newcommand{\red}[1]{\textcolor{red}{#1}}
\newcommand{\orange}[1]{\textcolor{orange}{#1}}
\newcommand{\purple}[1]{\textcolor{purple}{#1}}
\newcommand\eref[1]{(\ref{#1})}

\DeclareMathAlphabet\mathbfcal{OMS}{cmsy}{b}{n}

\makeatletter

\def\@fnsymbol#1{\ensuremath{\ifcase#1\or *\or *,\dagger\or \ddagger\or
   \mathsection\or \mathparagraph\or \|\or *\or \dagger\dagger
   \or \ddagger\ddagger \else\@ctrerr\fi}}

\AtBeginDocument{\providecommand\figref[1]{\ref{fig:#1}}}
\RS@ifundefined{subsecref}
  {\newref{subsec}{name = \RSsectxt}}
  {}
\RS@ifundefined{thmref}
  {\def\RSthmtxt{theorem~}\newref{thm}{name = \RSthmtxt}}
  {}
\RS@ifundefined{lemref}
  {\def\RSlemtxt{lemma~}\newref{lem}{name = \RSlemtxt}}
  {}

\makeatother

\DeclareMathOperator{\Var}{Var}


\begin{document}
\title{Acoustic frequency atomic spin oscillator in the quantum regime}


\author{Jun Jia}
\author{Valeriy Novikov}
\author{Tulio Brito Brasil}
\author{Emil Zeuthen}
\author{J\"{o}rg Helge M\"{u}ller}
\author{Eugene S. Polzik}
\affiliation{Niels Bohr Institute, University of Copenhagen, Copenhagen, Denmark}

\begin{abstract}
We experimentally demonstrate quantum behavior of a macroscopic atomic spin oscillator in the acoustic frequency range. Quantum back-action of the spin measurement, ponderomotive squeezing of light, and oscillator spring softening are observed at spin oscillation frequencies down to 6 kHz. Quantum noise sources characteristic of spin oscillators operating in the near-DC frequency range are identified and means for their mitigation are presented. These results constitute an important step towards quantum noise reduction and entanglement-enhanced sensing in the acoustic range using a negative-mass reference frame. In particular, the results are relevant for broadband noise reduction in gravitational wave detectors.
\end{abstract}
\maketitle
\section{Introduction}

Quantum mechanics implies that the measurement of a specific observable, e.g., position or a spin projection, is accompanied by the injection of noise in the canonically conjugate variable, e.g., momentum or another spin projection. 
This noise, called quantum back-action (QBA) \cite{Caves1980}, together with the imprecision shot noise, determines the precision bounds in quantum metrology tasks. The performance achieved with balanced (and uncorrelated) QBA and imprecision noise sources is referred to as the standard quantum limit (SQL). The microscopic mechanism behind the QBA depends on the physical platform. In the case of interferometric displacement measurements (such as gravitational wave detectors), it is due to the shot noise of light, and manifests itself as fluctuations in the laser radiation-pressure force. In spin-polarized systems the QBA mechanism is attributable to the light shift caused by quantum fluctuations of the Faraday probe polarization \cite{Julsgaard2001}. 
Recently, QBA has been observed in various quantum systems \cite{Cripe2019,Collective_Mechanical_Modes,variation_measurement,Moller2017}.

Atomic spin ensembles have become a rich resource for quantum sensing and engineering macroscopic quantum states \cite{MitchellRevModPhys,BudkerRevModPhys, Vasilakis2015,Bao2020,Hosten2016,Hosten2016_2}. A remarkable feature of spin ensembles 
is the ability to implement an effective negative-mass oscillator, demonstrated in several protocols, such as entanglement-assisted magnetometry \cite{Krauter2011, w_w} and quantum memory for a set of two-mode-squeezed states \cite{Jensen2011}. A central application of such an oscillator is the broadband QBA evasion in hybrid systems proposed in \cite{Tsang2012,HammererEugene2015}.  As proposed in Refs.~\cite{Khalili2018,Zeuthen2019}, combining a GWD with a negative-mass spin oscillator using a recently demonstrated two-color source of entangled light \cite{Brasil2022} allows for cancellation of both shot noise and QBA noise, enabling broadband sensitivity beyond the SQL. QBA-free sensing in the acoustic frequency range would enable new sensing applications beyond SQL. It has become increasingly important in current and future gravitational wave detectors (GWDs) \cite{DanilishinKhalili2012} as they approach SQL-limited performance in the acoustic frequency band \cite{QNlimitedLIGOreviewSchnabel,SNlimitedLIGO,Acernese2020-BAevidenceLIGO}.

Here we demonstrate the QBA-limited performance of a spin oscillator in the audio-frequency band.  Analogous to optomechanics, the atomic spin ensemble produces ponderomotive squeezing of the probe light \cite{Brooks2012}. Combining our investigation of the QBA with the observation of squeezing, we demonstrate another crucial element of the proposal \cite{Zeuthen2019}: the virtual oscillator-frequency downshift, which is necessary for matching the spin response to that of the (quasi-free-mass) GWD. We also observe and model the residual noise sources limiting the present performance and outline ways to overcome them. 
A spin-polarized atomic ensemble precessing at frequency $\Omega_{S}\sim |\vb*{B}|$ in a magnetic field $\vb*{B}$  
acts as an oscillator with an effective positive or negative mass depending on the orientation of the collective spin $\hat{\vb*{J}}$ with respect to  $\vb*{B}$ \cite{HammererEugene2015}. The ensemble is driven and probed by light (Fig.~\ref{fig:levelschemesimple}) with the interaction defined by the quantum nondemolition (QND) Hamiltonian  $\hat{H}_{\text{int}}\sim a_{1}\hat{S}_{z}\hat{J}_{z}$ \cite{ReviewHammerer2010}, where $a_{1}$ is the vector polarizability and $\hat{S}_{z}$ is a component of the Stokes vector operator $\vb*{\hat{S}}$ \cite{Moller2017}. The collective spin state of the atomic ensemble is probed by the light field, which we describe by homodyne quadratures  $\hat{Q}_{L}(\phi)$=$\hat{P}_{L}\cos(\phi)$+$\hat{X}_{L}\sin(\phi)$ with a quadrature angle $\phi$, where $\hat{X}_{L}$ and $\hat{P}_{L}$ are the normalized Stokes operators
representing amplitude and phase quadrature respectively (See Methods, Sec.~G for details). The power spectral density (PSD) $S_{S}$ for the detected optical field normalized to the shot noise is
\begin{align}
\label{eq:PSDspins}
S_{S}(\Omega)|_{\hat{Q}^{}_{L}(\phi)}={}& 
1+  
4\eta S_\t{QBA} \cos^2(\phi )  
+2\eta S_\t{corr} \sin(2\phi)
 \nonumber \\
&+4\eta S_\t{th} \cos ^2(\phi )
+ \eta S_\t{bb}\cos ^2(\phi ). 
\end{align} 
The terms in Eq.~\eref{eq:PSDspins} are the contributions from shot noise (SN), QBA noise, cross-correlations between the QBA and SN, atomic thermal fluctuations, and broadband spin-response noise. % respectively to the total noise budget. 
The nominal imprecision noise level is represented by unity, the strength of QBA noise term $S_\t{QBA}=\Gamma^{2}_{S}\left| \chi_{S}( \Omega ) \right|^{2}$
is defined by the atomic readout rate $\Gamma_{S} \propto g_{cs}^2 S_{x}J_{x} \propto d$, where $g_{cs}$ is the photon to atom coupling rate \cite{HammererEugene2015} and $d$ is the optical depth of the spin ensemble \cite{ThomasEntanglement2021}. The spectral response of the oscillator is established by the susceptibility function $\chi_{S}( \Omega )=\Omega_{S}/[( \gamma_{S}/2 -i\Omega ) ^2+\Omega_{S}^2]$, where 
the spin damping rate $\gamma_{S}=\gamma_{S,0}+\gamma_{S,\t{pb}}$ is decomposed into a probe power-broadening part
$\gamma_{S,\t{pb}}\propto\Gamma_{S}$ and an intrinsic linewidth $\gamma_{S,0}$. The cross-correlation term $S_\t{corr}=\Gamma^{}_{S} \mathrm{Re}\left[\chi_{S}( \Omega ) \right]$ is non-zero at detection quadratures $\hat{Q}_{L,\t{out}}(\phi)$ different from the pure phase or amplitude quadratures and constitutes a crucial element in the implementation of the quantum noise reduction proposed in Ref.~\cite{Zeuthen2019} because these correlations can induce an effective frequency downshift (virtual spring softening) \cite{Khalili2012}.

The term $S_\t{th}\approx 2\gamma_{S}\Gamma_{S} \left| \chi_{S}( \Omega ) \right|^2 S_{\hat{f}_\t{th}}$ in the second line of Eq.~\eref{eq:PSDspins}
is the response of the spin oscillator to the stochastic force 
$\hat{f}_\t{th}$ that has the spectrum
$S_{\hat{f}_\t{th}}=\left(n_{S}+1/2\right)$, where $n_{S}$ is the thermal occupancy of the spin oscillator. The reduction of the effective $n_{S}$ is one of the challenges we address in the present experiment. Finally, the contribution of $S_\t{bb}$ arises from extraneous, fast-decaying atomic modes coupling to the probe light \cite{Thomas2020thesis}. In the present work, it is minimized by employing a top-hat probe beam with a high cell filling factor (Methods, Sec.~B). The measurement precision of the indicated noise contributions except the nominal shot noise can be improved with a better overall detection efficiency $\eta$.

Note that a proper choice of $\phi$ allows for destructive interference between SN and QBA. 
 As a result, the output light noise drops below the shot-noise level in a certain frequency range, provided that the thermal contribution $\propto S_\t{th}$ is sufficiently small. 
Such ponderomotive squeezing \cite{Brooks2012}
relies on the interference between imprecision shot noise and QBA noise, and thus allows us to calibrate the QBA as we discuss below. 
Analogous to the ponderomotive squeezing in optomechanics \cite{Nielsen2016}, the maximal degree of squeezing in the limit of $\gamma_{S}\ll \Gamma_{S}, \Omega_{S}$ is
\begin{equation}
S_{S}(\Omega_\t{opt})|_{\hat{Q}_{L,\t{out}}(\phi_\t{opt})}\approx 1-\eta\frac{C_q}{C_q+1},
\label{eq:PonderSqFromCoop}
\end{equation} and is achieved in a narrow frequency range around  $\Omega\approx\Omega_\t{opt}$ when the optimal phase $\phi_\t{opt}$ of the detection quadrature is selected and the broadband noise is ignored. The quantum cooperativity 
\begin{equation}
C_{q}=\frac{\int_{\Omega}S_\t{QBA} }{\int_{\Omega}S_\t{th}}=\frac{\Gamma_{S}}{\gamma_{S}\left( 1+2n_{S} \right)},
\label{eq:Coop}
\end{equation}
is the ratio between the QBA and the thermal noise.

\section{Experimental results}

\begin{figure}[t]
\includegraphics[width=0.48\textwidth]{plots/Fig1.pdf}   
\caption{\textbf{Schematics of the experimental set-up.} The spin ensemble is probed by linearly polarized off-resonant light 
with a top-hat spatial mode shape. The probe polarization angle $\alpha$ with respect to the $x$-axis is adjusted for the QND measurement (Methods, Sec.~G) of the collective atomic spin.  A quarter- and a half-wave plate define the quadrature and thus the virtual rigidity shift detected by the polarization homodyning.}
\label{fig:levelschemesimple}
\end{figure}

The ensemble of $N_S \approx  10^{10} - 10^{11}$ Cesium-133 atoms is contained in an antirelaxation-coated vapor cell ($2\times 2\times 80\, \t{mm}^3$) heated by a low-noise heater to 40 $^{\circ}\t{C}$ providing a large optical depth and cooperativity \cite{Thomas2020thesis} (Fig.~\ref{fig:levelschemesimple}). 
To minimize the optical losses,  both input and output surfaces are anti-reflection coated with an overall transmission of $96 \%$. The PSD of the output probe light ($\sim 1\,\t{mW}$) is measured by polarization homodyning detection with an overall detection efficiency of $\eta\approx92 \%$ and more than 14 dB shot noise clearance above the electronic noise from the analysis frequency of 4 kHz. The homodyne phase $\phi$ is controlled by wave plates. 

The spin oscillator is prepared by optically pumping the atomic ensemble either to the lowest ($\ket{F=4, m_{F}=-4}$) or to the highest ($\ket{F=4, m_{F}=4}$) Zeeman sublevel with a degree of spin polarization of $\lesssim 98\%$ (Methods, Sec.~C). Low electro-magnetic noise, as required to reach quantum-limited performance, is achieved by a combination of magnetic coils and shielding (Methods, Sec.~A). The sign of the effective oscillator frequency $\Omega_S$, equivalent to the sign of the effective mass, can be controlled by the direction of the magnetic field or, alternatively, by the direction of circular polarization of the pump fields. The probe beam is linearly polarized at an QND coupling angle $\alpha$ relative to the magnetization axis $x$. 

From the analysis of the spin noise spectrum, we extract the parameters of the collective spin oscillator system appearing in Eq.~(\ref{eq:PSDspins}), with the cross-validations of the readout rate $\Gamma_{S}$ and the decay rate $\gamma_{S}$, found from the coherent induced Faraday rotation technique (CIFAR, see Ref.~\cite{ThomasCifar2021} and Methods, Sec.~D). The thermal occupancy is found from the atomic spin polarization using the magneto-optical resonance method (MORS, see Ref.~\cite{JulsgaardMORS2004} and Methods, Sec.~C).

\begin{figure}[h!]
\includegraphics[width=0.48\textwidth]{plots/Fig2.pdf}
\caption{The left [a, c, e] and right [b, d, f] columns display the configurations of the spin system with positive and negative mass respectively. \textbf{Top panel}: Atomic ensemble is described as a harmonic oscillator within the 2-level-system approximation (either of $m_{F}=\pm4$ and adjacent Zeeman sublevels on hyperfine level $F=4$).
If a single excitation lowers the energy of the system [b], then the oscillator has an effective negative frequency and mass. \textbf{Middle panel}: the spectra of the optical field after probing the atomic spin oscillator ($|\Omega_{S}|/(2\pi)=0.96$ MHz), of which the phase quadrature is detected. We distinguish the positive- [c] and negative-mass [d] configurations, comparing the frequency of the strongest transition $\Omega_{S1a}$ ($\Omega_{S1d}$) to the other transitions from the $F=4$ multiplet [only $\Omega_{S1b}$ ($\Omega_{S1c}$) can be identified]. In addition, we also observe the $F=3$ unresolved Zeeman manifold with a characteristic spectral frequency $\Omega_{S2}$ that is always higher than $\Omega_{S1i}$, regardless of the sign of the mass of the oscillator at $F=4$. Insets: the sign of the resonance frequency defines the orientation of rotation in phase space.
\textbf{Bottom panel}: Adjustment of homodyne detection phase $\phi=\phi_{\text{opt}}$ allows for the observation of ponderomotive squeezing [e, f].
}
\label{f:oscillator1MHz}
\end{figure}

\begin{figure}[b]
\includegraphics[width=0.49\textwidth]{plots/Fig3.pdf}
\caption{ [a] Total force-normalized quantum noise of light (SN and QBAN) exhibiting the tuning of effective resonance frequency $\Omega_{S}$, corresponding to the position of the minimum for each curve. The shift depends on the homodyne detection phase $\phi$, see Eq.~\eref{eq:Virtual}. In addition, the effective readout rate $\tilde{\Gamma}_{S}=\Gamma_{S}\cos^{2}\phi$ is reduced, leading to a reduction of the parabola's vertical offset and an increase of its steepness (see SI for details). [b] The corresponding unnormalized spectra showing ponderomotive squeezing. 
}
\label{fig:VirtRigByAtoms}
\end{figure}

\begin{figure*}[t]
\includegraphics[width=0.75\textwidth]{plots/Fig4.pdf}
\caption{Quantum backaction and ponderomotive squeezing for different $\Omega_{L}$, at the probe detuning of $1.6\,\t{GHz}$ 
[b, c] and at the detuning $3.2\,\t{GHz}$ [a, d, e, f]. The quantum back-action noise and thermal noise are colored with purple and green, respectively.}
\label{fig:mainresultsJUN}
\end{figure*}

Utilizing the broad frequency tunability of the spin oscillator, we begin with quantum noise measurements in the MHz frequency range, presenting the QBA-dominated spin response in both the positive- and negative-mass configuration (Fig.~\ref{f:oscillator1MHz}, [a, b]). Due to imperfect optical pumping, levels other than $\ket{F=4, m_{F}=-4}$ (or $\ket{F=4, m_{F}=4}$) are significantly populated and thus several peaks appear in the Zeeman-splitting-resolved spectrum. Specifically, we observe three peaks around $\Omega/(2\pi)\approx 960$ kHz (Fig.~\ref{f:oscillator1MHz}, [c, d])  and identify those as the transitions $\ket{m_{F}=\pm 4}\leftrightarrow\ket{m_{F}=\pm 3}$ and $\ket{m_{F}=\pm 3}\leftrightarrow\ket{m_{F}=\pm 2}$ within the $F=4$ hyperfine multiplet, as well as unresolved lines corresponding to the $F=3$ multiplet. The strength of the transitions between Zeeman sublevels in $F=4$ depends on their populations. Thus, the prevailing  $\ket{F=4,m_{F}=4}\leftrightarrow\ket{F=4,m_{F}=3}$ transition corresponds to the inverted spin population and the negative-mass oscillator (Fig.~\ref{f:oscillator1MHz}, [b, d, f]) \cite{Moller2017}, whereas the strong $\ket{F=4,m_{F}=-4}\leftrightarrow\ket{F=4,m_{F}=-3}$ transition corresponds to the positive-mass oscillator (Fig.~\ref{f:oscillator1MHz}, [a, c, e]).
The significant contributions from the $\ket{F=4,m_{F}=\pm 3}\leftrightarrow\ket{F=4,m_{F}=\pm 2}$ transitions and the $F=3$ manifold at $\Omega_{S}/(2\pi)\approx 1\,\t{MHz}$ require an extension of the model in Eq.~\eref{eq:PSDspins} to multiple oscillators whose QBA noise contributions are driven by the same vacuum fluctuations of light and can thus interfere constructively or destructively. Applying a multi-oscillator model (Methods, Sec.~E) we estimate the overall integrated areas of the QBA noise and the thermal noise, respectively, and find the cooperativity $C_{q}= 2.7\pm0.3$ for the positive-mass oscillator. The ponderomotive squeezing $S_{SS}={-4.8}$ dB estimated for these parameters using Eq.~(\ref{eq:PonderSqFromCoop}) agrees well with the observed squeezing (Fig.~\ref{f:oscillator1MHz},~[e,f]). The validity of the reconstructed QBA and thermal noise from the multi-oscillator model is corroborated by exploring their dependence on optical detuning $\Delta$ (Fig.~\ref{fig:DCnoiseJUN} and Methods, Sec.~E). The negative-mass oscillator displays slightly reduced ponderomotive squeezing, presumably due to a sub-optimal current ratio between the magnetic coils which causes extra inhomogeneous broadening leading to an increased spin damping rate.

When the Larmor frequency is reduced to 130 kHz, the quadratic Zeeman splitting becomes small and the spin ensemble behaves as a single oscillator described by Eq.~\eref{eq:PSDspins}. From the fits of the pure phase quadrature $\hat{P}_{L,\t{out}}$ and the quadrature $\hat{Q}_{L,\t{out}}(\phi_\t{opt})$ yielding the strongest ponderomotive squeezing, we extract the essential parameters of the atomic spin ensemble. The readout rate $\Gamma_{S}/(2\pi)=3.8\,\t{kHz}$ and the linewidth $\gamma_{S}/(2\pi)=0.22\,\t{kHz}$ are in reasonable agreement with the results of the CIFAR calibration, whereas the amount of thermal noise, encoded in the thermal occupation $n_{S}=3.7$, is bigger than the value $n_{S} \approx 0.6$ measured at $\Omega_{S}/(2\pi)\approx1 \,\t{MHz}$. Finally, we find the cooperativity $C_{q}=1.9$ and the level of ponderomotive squeezing $S_{SS}=-4.0\,\t{dB}$, which matches well to  the experimentally observed value and thus supports the validity of the model.

\textbf{Frequency downshift of the spin oscillator by virtual rigidity.}
We now demonstrate the frequency shift of the spin oscillator due to virtual rigidity whenever detected at a rotated quadrature $\phi\neq 0$, a crucial element of the broadband quantum-noise reduction scheme for gravitational wave detection beyond the SQL presented in Ref.~\cite{Zeuthen2019}. 
The central challenge of any such scheme is that the light quadrature responsible for the quantum noise in the gravitational-wave interferometer output varies as a function of the observation (Fourier) frequency, as dictated by the free-mass susceptibility $\chi_I\propto -1/\Omega^2$. 
The idea in Ref.~\cite{Zeuthen2019} is to engineer an effective spin oscillator with the same susceptibility except for an overall sign flip; when this oscillator is subjected to quantum noise suitably correlated with that impinging on the gravitation-wave interferometer, the corresponding optical output will be perfectly anti-correlated with the quantum-noise contribution in the interferometer output (in the limit of perfect light correlations and vanishing intrinsic spin noise $\gamma_S\rightarrow 0$). 
The spectrum from a spin oscillator is altered when $\phi\neq 0$ by the resulting cross-correlations between SN and QBA, as captured by Eq.~\eref{eq:PSDspins}. 
A clear correspondence to the interferometer quantum noise is achieved by changing to a new basis of uncorrelated SN and QBA quadratures, yielding the effective 
susceptibility of the spin oscillator shifted by the virtual rigidity due to quantum-noise correlations \cite{Zeuthen2019} (assuming $\gamma_S \ll \Omega_S$)
\begin{equation}
\tilde{\chi}^{-1}_{S}(\Omega) =\frac{\Omega^{2}_{S}-\Omega^{2}-i\gamma_{S}\Omega}{\Omega_{S}}+\Gamma_{S}\sin(2\phi).
\label{eq:Virtual}
\end{equation} 
Combining the two terms in this equation leads to the effective oscillator frequency  $\tilde{\Omega}_{S}=\Omega_{S}\sqrt{1+\Gamma_{S}\sin(2\phi)/\Omega_{S}}$ with $-\pi/2<\phi\,\t{sign}(\Omega_S)<0$ leading to virtual spring softening. The effective frequency shift can be seen from the effective oscillator response (shown in Fig.~\ref{fig:VirtRigByAtoms}) obtained by normalizing the squeezing spectrum by the transfer function that maps a force acting on the oscillator into the output light.
We observe the frequency shift in the range $\lesssim0.8$ kHz with its sign depending on the sign of the effective mass of the oscillator. We note that setting the homodyne detection phase to $\phi=\pm\t{sign}(\Omega_S)\pi/4$ would move $\tilde{\Omega}_{S}$ approximately 2 kHz down (up) from the initial $\Omega_{S}$ for the given experimental parameters of the atomic ensemble. 
The ideal regime for application to GWD noise evasion is when the readout rate $\Gamma_S$ exceeds $\Omega_{S}$, as it opens up the possibility to reduce the effective resonance frequency down to zero, $\tilde{\Omega}_S=0$, which occurs at $\Gamma_S \sin(2\phi)=-\Omega_S$. Based on the present demonstration, we can envision a realistic spin oscillator with bare frequency $\Omega_{S}$ in the kHz range whose susceptibility is modified by the virtual rigidity so as to match the susceptibility of a free mass, characteristic of the GWDs. 

\textbf{Spin oscillator noise in the acoustic frequency range.}
After those calibrations of the quantum spin oscillator, we turn to the QBA measurements in the acoustic frequency range (see Fig.~\ref{fig:mainresultsJUN}). 
As the Larmor frequency is reduced further into the audio band, we observe  a dramatic enhancement of $n_{S}$, while $\Gamma_{S}$, $\gamma_{S}$ as well as the broadband noise $S_\t{bb}$ remain unchanged. The increase in the spin thermal noise is the critical factor, leading to the reduction of the cooperativity. As a result, the ponderomotive squeezing almost entirely disappears at $\Omega_{S}/(2\pi)=10$ kHz (Fig.~\ref{fig:mainresultsJUN}[b]).  

\begin{figure}[t]
   \includegraphics[width = 0.48\textwidth]{plots/Fig5.pdf}
   \caption{
   \textbf{Top left} [a]: The spectrum of intensity noise of the probe laser.
   \textbf{Bottom left} [c]: Reduction of the observed ponderomotive squeezing with decreasing Larmor frequency $\Omega_{S}$ is shown with red dots and red dashed line. Blue curve (with the shaded blue area as a confidence interval 1$\sigma$) predicts the decline of quantum noise suppression caused by the power fluctuations of the probe laser within the Faraday interaction model, Eq.~(\ref{eq:coopwithclassnoise}).
   \textbf{Top right} [b]:
   Spectra of the light probing the spin ensemble reveal the strong DC component ($\Omega/(2\pi)\lesssim20$ kHz, being clearly separated from the response at $\Omega_{S}$), presumably leading to the reduction of ponderomotive squeezing in the acoustic frequency range faster than  Eq.~\ref{eq:coopwithclassnoise} predicts. The DC noise contribution decreases as the optical detuning $\Delta$ is increased.
   \textbf{Bottom right} [d]: Comparison of QBA, thermal, and DC noise areas as a function of $\Delta$.}
   \label{fig:DCnoiseJUN}
\end{figure}

\begin{figure}[t]
\includegraphics[width=0.45\textwidth]{plots/Fig6.pdf}
\caption{Influence of probe detuning $\Delta$ on the degree of ponderomotive squeezing measured at $\Omega_{S}/(2\pi)=11$ kHz [a], where `DC' noise has a significant contribution to the noise budget. At the optimal detuning ($\Delta_\t{opt}/(2\pi) \in 3.0-3.5$ GHz in the considered case) the ratio between QBA noise and uncorrelated noise sources (including `DC' noise) is maximized
and the best  squeezing is observed [b].}
\label{fig:PonderSq11kHzDeltaScannedJUNmain}
\end{figure}

The degradation of the ponderomotive squeezing in the low radio-frequency range is caused by the ubiquitous growth of technical noise, mainly due to power fluctuations of the probe laser (Fig.~\ref{fig:DCnoiseJUN}, [a]). The increase of the colored intensity noise towards the acoustic band is exhibited in Fig.~\ref{fig:mainresultsJUN} [b, c] and is modeled as a larger effective thermal occupation $n_{S}$ affecting $S_\t{th}$ [Eq.~\eref{eq:PSDspins}]. 
We include the power fluctuations as an extra noise source perceived by atoms within the Faraday interaction model 
(see also Methods, Sec.~F). It readily explains the reduction of ponderomotive squeezing as the Larmor frequency is reduced from $\Omega_{S}/(2\pi)\approx1$ MHz down to $100$ kHz. However, in the audio and near-DC frequency ranges the discrepancy between the expected and the observed quantum noise becomes more significant than the model of Eq.~\eref{eq:coopwithclassnoise} predicts (Fig.~\ref{fig:DCnoiseJUN}, [c]). Below we discuss the reasons for this and demonstrate the way to mitigate the extra noise.

\textbf{Suppression of the near-dc quantum oscillator noise.}
We have found that a fundamental reason for the enhanced quantum noise atomic oscillators in the near-DC frequency range is the deviation of the light-spin interaction from the QND Hamiltonian $\sim a_{1}\hat{S}_{z}\hat{J}_{z}$ discussed above. The description of the ground-state multiplet $F=4$ of Cesium atoms requires extension beyond the two-level (spin-$1/2$) model \cite{Colangelo_2013} implied by the QND Hamiltonian. Such extension involves alignment operators $\hat{j}^{2}_{x}-\hat{j}^{2}_{y}$, $\{\hat{j}_{x},\hat{j}_{y}\}\equiv$  $\hat{j}_{x}\hat{j}_{y}+\hat{j}_{y}\hat{j}_{x}$ that couple to a probe field through the  atomic tensor component proportional to the tensor polarizability  $a_{2}$ \cite{Vasilyev2012}. Accordingly, the following amendment to the QND interaction Hamiltonian must be included 
\begin{equation}
 \hat{H}^{(2)}_\t{int}\sim  a_{2}\left[\hat{S}_{y}\left\{\hat{j}_{x},\hat{j}_{y}\right\}  + \hat{S}_{x}\left(\hat{j}^{2}_{x}-\hat{j}^{2}_{y}\right)\right].
\label{eq:SpinLightHamiltonianPart2-highorder0}
\end{equation} 
The effect of the first term in the square brackets is centered around the Larmor frequency $\Omega_{S}$ and can be adjusted by the input polarization of light as presented in Fig.~\ref{f:tensorPPinteraction} (Methods, Sec.~G1). The second term affects the spin noise at $\Omega=0$ and $\Omega=2\Omega_{S}$ since the matrix element $\bra{F,m_{F,f}}\hat{j}_{x}^2 -\hat{j}_{y}^2\ket{F,m_{F,i}}$ is non-zero for $|m_{F,f}-m_{F,i}|=0,2$
, respectively \cite{Kozlov2021}. We observe both the $\Omega=2\Omega_{S}$ and $\Omega=0$ spectral components (Fig.~\ref{fig:alignmentnoise} in Methods, Sec.~G2) \cite{Fomin2020}, but mainly focus on the latter, which we will refer to it as ``DC noise''. The zero-frequency component amplified by the intensity noise of the probe laser spans up to $\Omega/(2\pi)\lesssim 20$ kHz, as shown in Fig.~\ref{fig:DCnoiseJUN}[b]. Consequently, the contribution of the DC noise to the noise budget leads to further deterioration of the ponderomotive squeezing in the audio-frequency band (Eq.~\eref{eq:coopwithclassAndDCnoise} in Methods, Sec.~G2).

Crucially, we find that this near-dc spin noise can be strongly suppressed by minimizing the alignment term in the Hamiltonian~\eref{eq:SpinLightHamiltonianPart2-highorder0}. Indeed, the DC alignment noise which is $\sim  a_{2}$ is reduced by increasing the optical detuning $\Delta$. This is evidenced by the observed enhancement of the ponderomotive squeezing at $\Omega/(2\pi)\lesssim 20$ kHz, as shown in the right column [b, d] of Fig.~\ref{fig:DCnoiseJUN} (see Methods, Sec.~G2 for details). By increasing $\Delta/(2\pi)$ from its initial value of $1.6$ GHz to $3.1$ GHz, squeezing has been enhanced within the range of $\Omega_{S}/(2\pi)\in 10-20$ kHz and has been observed down to 6 kHz (Fig.~\ref{fig:mainresultsJUN}, [d]). The improvement of ponderomotive squeezing follows the increase of the cooperativity. Based on the analysis of the DC noise as a function of $\Delta$, we predict the existence of an optimal detuning $\Delta_\t{opt}$ yielding the best quantum cooperativity $C_{q}$ and ponderomotive squeezing (Methods, Sec.~G2), and confirm it experimentally (Fig.~\ref{fig:PonderSq11kHzDeltaScannedJUNmain}, $\Delta_\t{opt}/(2\pi) \in 3.0-3.5$ GHz at $\Omega_{S}/(2\pi)\approx11$ kHz). 

\section{Conclusions and outlook}

We have experimentally demonstrated a macroscopic quantum spin oscillator in the acoustic frequency range. Quantum-backaction-dominated performance has been achieved for both positive and negative effective masses of the oscillator. We have shown effective spring softening due to virtual rigidity, an effect critical for the implementation of broadband quantum noise reduction in the acoustic and near-DC frequency bands relevant for various applications including gravitational wave detection beyond the SQL.  
We have identified the deleterious effect of the tensor spin polarizability on the low-frequency spin quantum noise and have found a way to minimize it by an optimal choice of detuning $\Delta$ of the probe light.

\begin{figure}[t]
    \includegraphics[width=0.47\textwidth]{plots/Fig7.pdf}
    \caption{The strain-referenced quantum noise of GWD with characteristic coupling rate $\Omega_{qI}/(2\pi)=100$ Hz \cite{Zeuthen2019,Cahillane_2022} for a standard quantum-noise-limited interferometer (magenta curve) and the configuration with an injected frequency-independent 10 dB phase-squeezed vacuum state of light (blue dashed curve) is compared with the \textit{projected} sensitivity of a joint measurement in the reference frame of a negative-mass spin oscillator linked to the GWDs by utilizing an entangled state of light (10 dB two-mode-squeezed vacuum state). See comments in the text.}
    \label{f:jointlcoop}
\end{figure}
The reported results constitute an important milestone towards the implementation of the proposal \cite{Khalili2018} to suppress quantum noise in interferometer-type GWDs using a negative-mass atomic oscillator as a reference. 
Fig.~\ref{f:jointlcoop} illustrates the expected broadband noise reduction in the GWD signal below the SQL obtained by combining the spin ensemble and the entangled light source demonstrated in Ref.~\cite{Brasil2022}. 
The dark red curve represents the case of quantum cooperativity $C_{q}=40$, corresponding approximately to the ratio between readout rate $\Gamma_{S}$ and power broadening contribution to the decoherence rate $\gamma_{S,\t{pb}}$ in the present experiment, while assuming the absence of thermal noise $n_{S}=0$ and negligible intrinsic linewidth $\gamma_{S,0}\ll \gamma_{S,\t{pb}}$. Inclusion of a moderate thermal-noise level $n_{S}=2$ reduces $C_{q}$ and adds extra uncorrelated noise limiting the performance of the scheme as shown by the light red curve.
The orange dashed vertical line indicates the initial resonance frequency of the spin oscillator $|\Omega_{S,\t{GWD}}|/(2\pi)\approx76$ Hz which is optimal for the implementation of the virtual rigidity scheme in the presented frequency range. The choice of initial Larmor frequency different from $\Omega_{S,\t{GWD}}$ will result in an increase of the power-broadened decay rate $\gamma_{S,\t{pb}}$ that has a detrimental impact on the quantum noise cancellation. 
The systems are considered to be lossless; classical noise and tensor noise of the spin system are neglected, implying the ideal case of both a quantum-noise-limited atomic oscillator and an entangled state of light persisting down to the DC frequency range. See Ref.~\cite{Zeuthen2019} for details of the model.
 Notably, the QBA-dominated spin system demonstrated in the present work, but operated in the power-broadening-limited regime, allows for a substantial sensitivity improvement of GWDs.
 
In a broader perspective, the reported results are relevant for quantum sensing of electro-magnetic fields in the acoustic range of sideband frequencies. 
Detection of weak e.-m.\ fields in the acoustic frequency range is a powerful tool in biomedical applications \cite{Jensen2018}. In the field of hybrid optomechanics, coupling of the atomic spin oscillator to a trapped dielectric nanoparticle would allow to suppress quantum noise of motion in the $\sim10-200\,\t{kHz}$ frequency range \cite{Tebbenjohanns2021}.
%==========================================================================================================

\bibliography{references}
\newpage

\section*{Methods}
\subsection{Atomic vapor cell and PCB coils}
\begin{figure}[h!]
\centering
\includegraphics[width = 0.35\textwidth]{plots/Fig8.pdf}
\caption{Atomic experimental setup [a]. The cesium vapor cell is surrounded by a PCBs coil system [b] inside a 5-layer magnetic shield. The different cell cross sections ($1\times 1 - 5\times5\,\t{mm}^2$) allow to choose the dark decoherence rate from 100 to 6 Hz [c].}
\label{fig:cell and PCB coils}
\end{figure}


The spin ensemble of $N_S\approx$  $10^{10} \sim 10^{11}$ Cesium-133 atoms is contained in an antirelaxation-coated (C30+) rectangular channel ($2\times 2 \times 80\, \t{mm}^3$) as shown in Fig.~\ref{fig:cell and PCB coils}[a] providing a good balance between the large quantum cooperativity $C_q$ ($\propto \rho \sigma_{Cs} l_{z}$) \cite{Thomas2020thesis} and the low-frequency quantum-noise-dominated performance for our experiment. The spin-preserving coating grants a room temperature dark decoherence rate of $\sim 50$ Hz during the experiment and the connection to a Cesium atom reservoir allows adjusting the vapor density $\rho$ based on the operational temperature.  
The vapor cell is placed in magnetic fields provided by PCB coils as in Fig.~\ref{fig:cell and PCB coils}[a]. The inner bias magnetic field $B$ is generated by a PCB coils system in Fig.~\ref{fig:cell and PCB coils}[b] (combination of a linear-gradient and a parabolic magnetic field) driven by an ultra-low-AC-noise current source which provides the  
inhomogeneity  $ < 0.1 \permil$ within the cell volume \cite{Ryan2020thesis}. The setup is positioned in a 5-layer magnetic shield protecting the spins from perturbations from the external DC and RF magnetic fields. 
The setup with freely adjustable PCBs coils system allows tuning the Larmor frequency from a few Hz up to 1MHz without obviously affecting the intrinsic line-width $\gamma_{S,0}/(2\pi)$.

\subsection{Broadband noise reduction (BNR)}

\begin{figure}[h!]
\centering
\includegraphics[width=0.45\textwidth]{plots/Fig9.pdf}
\caption{Spectrum of spin noise [a] probed by a $1.65\times 1.65\,\t{mm}^2$ square top-hat beam [d], including broadband [b] and narrowband atomic responses [c], $\Omega_{S}/(2\pi)=1$ MHz.}
\label{fig:BroadBandNoiseValera}
\end{figure}
When a linearly polarized probe light interacts with a spin ensemble and records the dynamic of the collective spin system, the measured spin noise spectrum (SNS) in Fig.~\ref{fig:BroadBandNoiseValera} [a] would be affected by various dephasing mechanisms, such as wall collision, the probe beam size, and atomic motion diffusion characteristics \cite{Thomas2020thesis,Shaham2020}. Therefore, the SNS in an atomic vapor cell is a combination of Lorentzians ($S_\t{total}(\Omega )\rightarrow \sum{\varGamma _i\chi _i(\gamma _{Si},\Omega )}$ -- different from multiple oscillators model described in the Sec.~E) with the relative weights which correspond to the overlap of the probe beam spatial profile (e.c. Gaussian distribution) with each given spin diffusion mode. The pronounced narrowband noise spectrum in Fig.~\ref{fig:BroadBandNoiseValera} [c] (orange area) originated from the sum of the slowly decaying modes, and the broadband spin response floor [b] (green area) is due to the modes which decay rapidly due to the motion of atoms in and out of the probe beam. With the help of a diffractive beam shaper and a telescope system, we could produce a $1.65\times 1.65\,\t{mm}^2$ square top-hat beam (as shown in Fig.~\ref{fig:BroadBandNoiseValera} [d]) collimated along 8 cm (corresponds to a cell filling factor of $72\%$), The increased filling factor for the rectangular cell channel helps to reduce the broadband noise down to less than 0.3 in shot noise units and to improve the relative amplitude ratio between the narrowband and broadband response up to $\sim$ 600, making the contribution of the $S_\t{bb}$ term in Eq.~\eref{eq:PSDspins} negligible.

\subsection{Preparation and characterization of atomic state} 

The Hamiltonian for an ensemble of atomic spins with a collective angular momentum $\hat{\vb*{J}}=\sum_{k=1}^{N}\hat{\vb*{j}}^{k}$ in the external magnetic field $\vb*{B}$ is  $\hat{H}_{B}\sim-\vb*{J}\cdot\vb*{B}$. 
All $N$ atoms are initially prepared in the state $6S_{1/2}$, $\ket{F=4,m_{F}=- 4}$ or $\ket{F=4,m_{F}=4}$, where $m_{F}$ denotes the Zeeman sublevel within the hyperfine manifold $F$.
The ensemble is then polarized along $x$-axis (see Fig. \ref{fig:levelschemesimple}), so that the component $\hat{J}_{x}$  becomes a macroscopic variable $\hat{J}_{x} \rightarrow J_{x}=\hbar F N/2$. Within the Holstein-Primakov approximation, the spin precesses in the $yz$-plane $\sim\Omega_{S}\left(\hat{J}^{2}_{z}+\hat{J}^{2}_{y}\right)$ at the Larmor frequency $\Omega_{S}\sim |\vb*{B}|$. 
The collective spin can be co-oriented ($J_{x}>0$) or counter-oriented ($J_{x}<0$) with respect to $\vb*{B}$. This leads to opposite directions of rotation of the $\hat{J}_{y(z)}$-components, or equivalently, to the opposite signs of $\Omega_{S}$ (see Fig.~\ref{f:oscillator1MHz}). This situation is commonly referred to a spin oscillator with a negative or positive effective mass \cite{HammererEugene2015}. 

\begin{figure}[h]
\includegraphics[width=0.49\textwidth]{plots/Fig10.pdf}
\caption{Level scheme of optical pumping [a]. The spin polarization used for the calibration of thermal noise is prepared at 98$\%$ [b] in the pulsed MORS measurement  and goes down to 89$\%$ [c] with a ($\approx1$mW) probe light.}
\label{fig:levelschemecomplete}
\end{figure}

The detailed configuration of atomic levels without Zeeman splitting is depicted in Fig.~\ref{fig:levelschemecomplete}[a] which outlines the pumping scheme. Circularly polarized pump and repump lasers are tuned to the  $\ket{6S_{1/2},F=4} \leftrightarrow \ket{6P_{1/2},F'=4} $ and $\ket{6S_{1/2},F=3} \leftrightarrow \ket{6P_{3/2},F'=4} $ transitions respectively as in Fig.~\ref{fig:levelschemecomplete}, which corresponds to  in the D1 and D2 lines. The applied method of atomic polarization characterization is based on MORS \cite{JulsgaardMORS2004}. We extract the spin polarization $\approx 98 \%$ in Fig.~\ref{fig:levelschemecomplete}[b] (equivalent to the thermal occupation of $n_S \sim 0.15$) by the pulsed MORS measurement with a ($\approx1$mW) probe light in this experiment. The orientation goes down to 89 $\%$ (Fig.~\ref{fig:levelschemecomplete}[c], yielding $n_S \sim 0.6$) in the regime of continuous probing under the same optical power. The repump power $P_{re}\approx5$mW was conditioned upon the maximum available laser power, whereas the pump power $P_{p}\approx50\mu$W was chosen after the optimization of ponderomotive squeezing at $\Omega_{S}/(2\pi)\sim1$ MHz. The power broadening from the pump and repump lasers contribute $<100$Hz decoherence to the spin linewidth. From the pulsed MORS we estimate the intrinsic linewidth $\gamma_{S0}/(2\pi)\approx150\,\t{Hz}$ which contains all decay contributions except for the power broadening induced by the probe field. 

\subsection{Calibration of readout rate}

\begin{figure}[h!]
\centering
\includegraphics[width=0.49\textwidth]{plots/Fig11.pdf}
\caption{Implementation of CIFAR technique for calibration of the readout rate $\Gamma_{S}$. Top panel [a]: the experimental layout. Bottom panel: results of the CIFAR amp [b] and phase signal [c] at different modulation phases $\theta$.}
\label{f:methodsCIFAR}
\end{figure}

To calibrate the spin measurement rate $\Gamma_{S}$ and damping rate $\gamma_{S}$, we investigate the atomic response to strong modulation of the probe light polarization. The outlined technique is referred to as Coherently induced Faraday rotation (CIFAR) \cite{ThomasCifar2021}. The experimental setup is shown in Fig.~\ref{f:methodsCIFAR}[a]. A weak linearly polarized optical field denoted as `drive' is phase-modulated at frequency $\Omega_\t{mod}$ using piezo-electric transducer PZT$_{1}$ and subsequently overlapped with the orthogonal polarized Local Oscillator (LO) on a polarizing beam-splitter PBS$_{2}$. One of the output modes of PBS$_{2}$ thus
contains the state $\hat{Q}^\t{mod}_{L,in}(\theta)\sim(\hat{X}_{L,in}\sin{\theta}+\hat{P}_{L,in}\cos{\theta})\sin(\Omega_\t{mod}t)$ with an arbitrary modulated polarization quadrature. The phase angle $\theta$ is set by the phase lock loop between LO and drive fields with a feedback signal applied to the piezo element PZT$_{2}$ in one of the interferometer arms. The optical mode $\hat{Q}^\t{mod}_{L,in}(\theta)$ probes the atomic oscillator and the output field is detected on the 'self-homodyning' detection setup. Scanning the modulation frequency $\Omega_\t{mod}$ around Larmor frequency $\Omega_{S}$, one obtains the characteristic shape of the measured spectrum signal $S_{\text{CIFAR}}(\Omega_\t{mod})$ that provides information about $\Gamma_{S}$ and $\gamma_{S}$. However, the correctness of extracted parameters strongly depends on the precise knowledge of the modulation phase $\theta$. To account for that, we perform the fit of $S_{\text{CIFAR}}(\Omega_\t{mod})$ at several points of locked $\theta$ and obtain results for $\Gamma_{S}$ and $\gamma_{S}$ as shown on the Fig.~\ref{f:methodsCIFAR}[b,c]. The uncertainty $\sim15\%$ on both parameters is mainly attributed to imperfect calibration of $\theta$, which is limited by the software of a FPGA board in the phase lock loop. This circumstance might also address the discrepancy between the values of the readout rate from CIFAR technique and from the fit of the full spin noise model Eq.~\eref{eq:PSDspins}. Therefore, we consider the CIFAR calibration as a rough estimation of measurement and damping rates of a spin system.

\subsection{Faraday (QND) interaction model for multiple oscillators}

To fit the data set $\Omega_{S}/(2\pi)=960$ kHz with resolved Zeeman transitions at $F=4$ and $F=3$ manifold, we have extended the spin noise model with multiple oscillators: in Eq.~\eref{eq:PSDspins} the replacement $\chi (\Omega_{S},\gamma_{S},\Omega)\rightarrow \sum_i^n\chi_{i}(\Omega _{S,i},\gamma_{S,i},\Omega )$ is implemented ($n=3$ since three peaks are distinguished). We assign specific readout rate $\Gamma_{S,i}$, damping rate $\gamma_{S,i}$ and thermal noise constants $n_{S,i}$ for each oscillator. Note that such approach assumes the interferences of QBA among three oscillators with different phases together with the summation of their individual thermal noise. When the fit is performed, relative phases between oscillators within $\ket{F=4}$ can be adjusted. However, the signs of gyromagnetic ratios $g_F$ for hyperfine levels $F=3$ and $F=4$ are opposite. So the respective oscillators are approximately out of phase since the minor difference ($\sim0.3\%$) in absolute values of $g_{F=4}$ and $g_{F=3}$ can be neglected. 

To confirm the reliability of extracted QBA noise and thermal noise, we calibrate them as a function of the optical detuning $\Delta$. From the model Eq.~\eref{eq:PSDspins}, the expressions for the integrated areas are $\int_{\Omega}S_{QBA}\sim\Gamma^2_{S}\int_{\Omega}\left| \chi_{S}\left( \Omega \right) \right|^{2}=$ $\Gamma^{2}_{S}/\gamma_{S}\sim A^{2}/\left[\Delta^{2} \left(\gamma_{0}\Delta^{2}+C\right)\right]$ and  $\int_{\Omega}S_{Th.}\sim\gamma_{S}\Gamma_{S}\int_{\Omega}\left| \chi_{S}\left( \Omega \right) \right|^{2}=\Gamma_{S}\sim A/\Delta^{2}$ respectively. Here $\Gamma_{S}= A/\Delta^{2}$, $\gamma_{S}=\gamma_{0}+C/\Delta^{2}$, where $A$, $C$ and $\gamma_{0}$ are constant parameters independent of $\Delta$. We validate the expected behavior both for $\int_{\Omega}S_{QBA}$ and $\int_{\Omega}S_{Th.}$ while varying $\Delta$, as shown in Fig.~\ref{fig:DCnoiseJUN}. Only two oscillators from $F=4$ are taken into account when fitting $\int_{\Omega}S_{QBA}(\Delta)$ and $\int_{\Omega}S_{Th.}(\Delta)$, whereas the third oscillator response from $F=3$ is discarded. 

\subsection{Including the intensity noise in the Faraday model}

The classical light intensity noise,  uncorrelated with the QBA, appears in the spin noise spectrum as classical back-action noise. 
Using the simplified spin-1/2 model, we give here an intuitive explanation of the intensity noise coupling to the spin dynamics with the unbalanced $\sigma_{+}$, $\sigma_{-}$ diagonal transitions. As a result, the classical noise of the probe laser is mapped onto the output Stokes quadrature. This amplitude noise is coupled through the same readout rate as for quantum noise limited light during the QND interaction. But in contrast to QBA, such intensity noise cannot set the correlations between phase and amplitude quadratures. Such approach allows a quantitative estimate of the reduction of cooperativity as in Fig.~\ref{fig:DCnoiseJUN}[c]. To do so, we define the classical back-action noise $S_\t{cl.BA}$ as a response of the atomic system to the laser intensity noise. The expression ~\eref{eq:Coop} for cooperativity is now converted to
\begin{equation}
  C_{q}=\frac{\int_{\Omega}S_{QBA} }{\int_{\Omega}S_{Th.}+\int_{\Omega}S_{cl.BA}}=\frac{\int_{\Omega}S_{QBA} }{\int_{\Omega}S_{Th.}+Z\int_{\Omega}S_{QBA}},
 \label{eq:coopwithclassnoise}
\end{equation} where the ratio $Z$ is the amount of extra intensity noise in shot noise units.

\subsection{Spin alignment noise}

An atomic spin ensemble driven by Hamiltonian \eref{eq:SpinLightHamiltonianPart2-highorder0} demonstrates the distinctive features of linear birefringence. At the quantum level, the composite dynamics of the spin alignment interaction causes several phenomena, such as a tensor-induced Stark shift of the oscillator's Larmor frequency, cooling or amplification of the spin state, and even spin dynamics beyond the oscillation frequency. In this section, we will give an overview of the influence of each alignment operator on the atomic spin dynamics.

\subsubsection{Impact on spectral frequencies $\Omega\approx\Omega_{S}$}

\begin{figure}[t]
   \includegraphics[width=0.48\textwidth]{plots/Fig12.pdf}
   \caption{ Effect of the varying tensor alignment term ($\mathcal{E}_{S}$ controlled by the input polarization angle $\alpha$) on the spin noise spectrum [a] and ponderomotive squeezing [b]. The QND interaction is chosen whereby the tensor contribution is minimized in the amplitude quadrature of light [c].}
   \label{f:tensorPPinteraction}
\end{figure}

We start with the term $\{\hat{j}_{x},\hat{j}_{y}\}$. After applying the approximation $\{\hat{j}_{x},\hat{j}_{y(z)}\}\approx 7\hat{j}_{y(z)}$ valid in a two-level model, the total interaction is described by \cite{Thomas2020thesis}
\begin{equation}
\begin{split}
\hat{H}_{int.}&\sim a_{1}\left( \hat{S}_{z}\hat{J}_{z} +\mathcal{E}_{S}\hat{S}_{\bot}\hat{J}_{y}    \right), \\ 
\mathcal{E}_{S}&=\quad -14\left(\frac{a_{2}}{a_{1}}\right) \cos{(2\alpha)},
\end{split}
\label{eq:tensorPPinteraction}
\end{equation} 
where Stokes operators were redefined as $[\hat{S}_{||},\hat{S}_{\bot}]^{T}=\vb*{R}(2\alpha)[\hat{S}_{x},\hat{S}_{y}]^{T}$, where $\vb*{R}(2\alpha)$ is the rotation matrix. The presence of the $\hat{S}_{\bot}\hat{J}_{y}$ term added to the Faraday rotation $\hat{S}_{z}\hat{J}_{z}$ means that the interaction has deviated from the QND-interaction. It affects the response of the atomic system recorded onto the phase light quadrature $\hat{P}_{L, out}$ (see Fig.~\ref{f:tensorPPinteraction}, [a]). Such impact might be seen as an effective change of the QND readout rate $\Gamma_{S}$ and inducing a dynamic contribution to the damping rate $\gamma_{S}/2\sim \gamma_{QND}/2 + \mathcal{E}_{S}\Gamma_{S}$. Consequently, the maximal level of ponderomotive squeezing is altered (Fig.~\ref{f:tensorPPinteraction}, [b]), when $\hat{Q}^{}_{L}(\phi_\t{opt})$ is selected. Finally, the amplitude output light quadrature $\hat{X}_{L, out}$, being a QND-variable otherwise, is now also disturbed. This is manifested in a characteristic dip/peak as demonstrated in Fig.~\ref{f:tensorPPinteraction}, [c]. At the same time, we notice that the strength of the $\hat{S}_{\bot}\hat{J}_{y}$ term is controlled by the angle $\alpha$ of the probe input polarization. In the present experiment we wish to work at the  QND configuration with the largest QBA noise, which is set by rotating a halfwave plate in front of the cell and making the spectrum of $\hat{X}_{L,out}$ flat around $\Omega_{S}$ corresponding to $\mathcal{E}_{S} \approx 0$  (also depicted in Fig.~\ref{f:tensorPPinteraction}, [c]). We note that the alignment operator studied here is also responsible for the tensor Stark shift effect moving the resonance frequency $\Omega_{S}$ (clearly seen in Fig.~\ref{f:tensorPPinteraction}, [c]). It has to be taken into account when estimating the size of the frequency shift by virtual rigidity.

\subsubsection{DC alignment noise}

\begin{figure}[h!]
   \includegraphics[width=0.48\textwidth]{plots/Fig13.pdf}
   \caption{  The spectrum of $S_{S}|_{\hat{Q}_{L,\t{out}}(\phi_\t{opt})}$ ($\phi_\t{opt}$ yields the strongest ponderomotive squeezing)  when $\Omega_{S}/(2\pi)$ is set to 43 kHz. Apart from the dispersive signal centered at $\Omega_{S}/(2\pi)$, the peak at twice Larmor frequency is visible together with noise enhancing towards $\Omega_{S}/(2\pi)=0$. The last two effects originate from the tensor interaction term given by Eq.~\eref{eq:SpinLightHamiltonianPart2-highorder0}.}
   \label{fig:alignmentnoise}
\end{figure}
 
It is conceivable that the spin ensemble can sense fluctuations of the probe laser
via a mechanism responsible for the tensor interaction ($\sim a_{2}$). In particular, coupling through the alignment operator $\hat{j}^{2}_{x}-\hat{j}^{2}_{y}$ explains the abrupt rise of noise centered at zero frequency, being clearly separated from Larmor peak, as shown in Fig.~\ref{fig:alignmentnoise} ($\Omega_{S}/(2\pi)=43$ kHz). However, the `DC'-noise component has a tangible overlap with the Larmor peak shifted down to the acoustic range. We study the detrimental influence of `DC'-noise on the ponderomotive squeezing and introduce the term $S_{DC}$ added to the uncorrelated noise $S_{cl.BA}$ in Eq.~\eref{eq:coopwithclassnoise},
\begin{equation}
  C_{q}=\frac{\int_{\Omega}S_{QBA} }{\int_{\Omega}S_{Th.}+\int_{\Omega}S_{cl.BA}+\int_{\Omega}S_{DC}}.
 \label{eq:coopwithclassAndDCnoise}
\end{equation}
The inclusion of $S_{DC}$ explains the mismatch between experimentally observed ponderomotive squeezing in the acoustic band and the value predicted by Eq.~\eref{eq:coopwithclassnoise}. 
In the analysis of the DC noise, we make the following assumption: $S_{DC}\sim \left(a_{2}/a_{1}\right)^{2} \left(\Gamma_{S}\right)^{2}|\chi_{DC}(\Omega)|^{2}$, where $\chi_{DC}(\Omega)\sim1/(\Omega_{DC}-\Omega+i\gamma_{S,DC})=1/(i\gamma_{S,DC}-\Omega)$ and the bandwidth of the DC noise is the same order as the spin damping rate $\gamma_{S,DC}\sim\gamma_{S}$. Using the approximation $a_{2}\sim 1/\Delta$, one may  expect $\int_{\Omega} S_{DC}\sim\left(a_{2}/a_{1}\right)^{2}\int_{\Omega}S_{QBA}\sim 1/\left[\Delta^{4} \left(\gamma_{0}\Delta^{2}+C\right)\right]$ for the integral area of DC-noise. Such functional dependence on $\Delta$ is then confirmed by the experimental data as shown in  Fig.~\ref{fig:DCnoiseJUN}. Comparing with the $\int_{\Omega} S_{QBA}$ and $\int_{\Omega} S_{Th.}$ noise,  the faster down-scaling of $\int_{\Omega} S_{DC}$ with the probe detuning $\Delta$ opens up the possibility to optimize the ponderomotive squeezing in audio frequency by increasing $\Delta$. Indeed, exploiting the 'worst case scenario' approximation for the 'DC'-noise $\int_{\Omega}S_{DC}\sim1/\Delta^{4}$ (in reality it can scale down only faster) and inserting  $\int_{\Omega}S_{QBA}(\Delta)$, $\int_{\Omega}S_{Th.}(\Delta)$, $\int_{\Omega}S_{cl. BA}(\Delta)$, we rewrite the right-hand side of Eq.~\eref{eq:coopwithclassAndDCnoise} in the form
\begin{equation}
\frac{ A^{2}}{\left(ZA^{2}+A\cdot C+D\gamma_{0}\right) + A\gamma_{0}\Delta^{2}+\left(C\cdot  D/\Delta^{2}\right)  },
 \label{eq:coopwithclassAndDCnoise1}
\end{equation}  

The presence of the term $C\cdot D/\Delta^{2}$ allows for increasing the cooperativity by enlarging $\Delta$ until the denominator in Eq.~\eref{eq:coopwithclassAndDCnoise1} reaches the minimum. The optimal detuning is defined by the actual values of all coefficients in Eq.~\eref{eq:coopwithclassAndDCnoise1} and appears to be $\Delta_\t{opt}/(2\pi)\in 3-3.5$ GHz for a chosen set of parameters (shown in Fig.~\ref{fig:PonderSq11kHzDeltaScannedJUNmain}). Exceeding this level brings us to the regime where reduction of the DC-noise term cannot compensate for the decline of  $S_{QBA}/S_{Th.}$ due to the significance of the intrinsic spin linewidth.
In the future, we will be able to increase the intensity of the probe field without adding technical noise. with the help of an intensity stabilization system. The atomic damping rate will then become limited by power broadening $\gamma_{S}\approx\gamma_{S,pb}\gg\gamma_{S,0}$, and the impact of the term $A\gamma_{0}\Delta^{2}$ in Eq.~\eref{eq:coopwithclassAndDCnoise1} will be negligible. Ultimately, even further detuning (before approaching the other hyperfine level) will maximally diminish the contribution of the DC alignment noise and push quantum noise dominated spin oscillator down to an even lower oscillation frequency. 

As a final remark, we note that the amount of 'DC'-noise depends on the phase of the detection quadrature. In particular, $S_{DC}$ is maximized in the amplitude Stokes quadrature, thus having a direct impact on the ponderomotive squeezing spectrum. In contrast, 'DC' alignment is not present when the phase Stokes quadrature is observed. Also, it seems to be independent of the input light polarization (angle $\alpha$). Those effects require further investigation.

\section*{Acknowledgments}
We gratefully acknowledge conversations with J.\ Appel, M.\ Zugenmaier, R.\ Thomas and M.\ Parniak. Mikhail Balabas fabricated the alkene coated vapour cell used for this experiement. Contributions of Ryan Yde to the initial stages of the experiment are gratefully acknowledged. This work was funded by the European Research Council (ERC) under the Horizon 2020 (grant agreement No 787520) and by VILLUM FONDEN under a Villum Investigator Grant no.\ 25880. J.J. is partially supported by the China Scholarship Council. 

\section*{Author Contributions}
JJ, VN and TBB performed the experiments. EZ contributed the theory of virtual rigidity, JHM contributed to the experiment, ESP led the project. All authors contributed to writing the manuscript. JJ and VN contributed equally to this work.
\section*{Author Information}
The authors declare no competing financial 
interests. Correspondence and 
requests for materials should be addressed to E.S.P. (polzik@nbi.ku.dk).

\section*{Data Availability Statement}
The data that support the findings of this study are available from the corresponding author upon reasonable request.

%=====================================================

\appendix
\clearpage
\setcounter{page}{1}
\renewcommand{\thepage}{SI~\arabic{page}}

\setcounter{figure}{0}
\renewcommand{\thefigure}{SI\arabic{figure}}

\setcounter{table}{0}
\renewcommand{\thetable}{SI\arabic{table}}

\clearpage
\onecolumngrid
\section*{Supplementary Information}

\subsection{Virtual rigidity}

Input-output relations for light quadrature probing atomic ensemble in QND-regime and in approximation of $\Omega\sim|\Omega_{S}|\gg\gamma^{}_{S}$ \cite{Thomas2020thesis}:
\begin{equation}
\begin{bmatrix} \hat{X}_{L,out} \\ \hat{P}_{L,out} \end{bmatrix}= \left\{
\begin{bmatrix}1 & 0  \\ 0 & 1  \end{bmatrix} +
2\Gamma^{}_{S}\begin{bmatrix}0 & 0  \\ \chi^{}_{S} & 0  \end{bmatrix}
\right\}
\begin{bmatrix}\hat{X}_{L,in} \\ \hat{P}_{L,in}\end{bmatrix}+
\sqrt{\Gamma^{}_{S}\gamma^{}_{S}} 
\begin{bmatrix}0 & 0  \\ -i\chi^{}_{S} & \chi^{}_{S}   \end{bmatrix}
\begin{bmatrix} \hat{f}^{}_{Th,X} \\ \hat{f}^{}_{Th,P}, \end{bmatrix}
\label{eq:SIinputoutquad}
\end{equation} resulting in the following expression for arbitrary detection quadrature $\hat{Q}_{L,out}(\phi)$=$\hat{P}_{L,out}\cos{(\phi)}$+$\hat{X}_{L,out}\sin{(\phi)}$:

\begin{equation}
\label{eq:virtrig1}
\hat{Q}^{}_{L,out}(\phi)=\hat{Q}^{}_{L,in}(\phi)+2\Gamma^{}_{S}\chi^{}_{S}\cos{(\phi)} \hat{X}^{}_{L,in}+\sqrt{2\Gamma^{}_{S}\gamma^{}_{S}}\chi^{}_{S}\cos{(\phi)}\hat{f}^{}_{th}.
\end{equation} where $\hat{f}_{th}= (-i\hat{f}_{th,X} + \hat{f}_{th,P})/\sqrt{2}$ with uncorrelated $\hat{f}_{th,X}$ and $\hat{f}_{th,P}$. Eq.\eref{eq:virtrig1}
is then re-written in the new basis
$[\hat{Q}_{L}(\phi),\hat{Q}_{L}(\phi_{\bot})]^{T}=\vb*{R}(\phi)[\hat{X}_{L},\hat{P}_{L}]^{T}$ (where $\vb*{R}(\phi)$ is the rotation matrix):

\begin{equation}
\label{eq:virtrig2}
 \hat{Q}^{}_{L,out}(\phi)=\left[1+\chi^{}_{S}\Gamma^{}_{S}\sin{(2\phi)}\right]\hat{Q}^{}_{L,in}(\phi)+
 2\chi^{}_{S}\Gamma^{}_{S}\cos^{2}{(\phi)} \hat{Q}_{L,in}(\phi_{\bot})+\sqrt{2\Gamma^{}_{S}\gamma^{}_{S}}\chi^{}_{S}\cos{(\phi)}\hat{f}^{}_{th}.
\end{equation} The spectrum of $\hat{Q}^{}_{L,out}(\phi)$ is given by Eq.\eref{eq:PSDspins}, where $\langle \hat{Q}^{\dagger}_{L,in}(\phi)  \hat{Q}^{}_{L,in}(\phi) \rangle =\langle \hat{Q}^{\dagger}_{L,in}(\phi_{\bot})  \hat{Q}^{}_{L,in}(\phi_{\bot}) \rangle=1/4$ and $\langle \hat{Q}^{\dagger}_{L,in}(\phi)  \hat{Q}^{}_{L,in}(\phi_{\bot}) \rangle=0$. We then take the response of oscillator to the thermal force
$N_{th}=\sqrt{2\Gamma^{}_{S}\gamma^{}_{S}}\chi^{}_{S}\cos{(\phi)}$
out of brackets in the right part of Eq.\eref{eq:virtrig2} and present it in the form
$\hat{Q}^{}_{L,out}=N_{th}\left[ \hat{f}^{}_{th} + \hat{f}^{}_{LN}  \right]$, where we introduced the renormalized light force driving the spin oscillator:
\begin{equation}
\label{eq:virtrig3}
\hat{f}^{}_{LN}=\frac{\chi^{-1}_{S}+\Gamma^{}_{S}\sin{(2\phi)}}{\sqrt{2\Gamma_{S}\gamma_{S}}\cos{(\phi)}}\hat{Q}^{}_{L,in}(\phi)+
\sqrt{\frac{2\Gamma_{S}}{\gamma_{S}}}\cos{(\phi)}\hat{Q}^{}_{L,in}(\phi_{\bot}).
\end{equation} 
The tuning of the detection phase $\phi$ leads to the transformation of the susceptibility function as presented in Eq.\eref{eq:Virtual} in the main text. The shift of effective frequency $\tilde{\Omega}_{S}=\Omega_{S}\sqrt{1+\Gamma_{S}\sin(2\phi)/\Omega_{S}}$ is accompanied by a reduction of the effective readout rate $\tilde{\Gamma}_{S}=\Gamma_{S}\cos^{2}(\phi)$. We retrieve the spectra $\langle \hat{f}^{\dagger}_{LN}\hat{f}^{}_{LN} \rangle$ of the light force from experimental data and plot them on the Fig.~\ref{fig:VirtRigByAtoms}, confirming virtual rigidity principle.

\end{document}