\usepackage{adjustbox,rotating,booktabs,tcolorbox} % For formal tables
% \usepackage[xcolor,notion]{knowledge}

\clubpenalty = 10000
\widowpenalty = 10000
\displaywidowpenalty = 10000

\usepackage{quiver}
\usepackage{etoolbox}
\usepackage{fullshort}
\setboolean{fullpaper}{true}

\newcommand\ALinline[1]{{\textcolor{blue}{AL: #1}}}
\newcommand\AL[1]{\marginpar{\textcolor{blue}{AL: #1}}}

% \newcommand\tom[1]{\marginpar{\textcolor{vert}{TH: #1}}}
% \newcommand\tominline[1]{\textcolor{vert}{TH: #1}}
\newcommand\tom[1]{}
\newcommand\tominline[1]{}
% \newcommand\ah[1]{\marginpar{\textcolor{red}{AH: #1}}}
% \newcommand\marco[1]{\marginpar{\textcolor{blue}{MM: #1}}}
% \newcommand\marcoinline[1]{{\textcolor{blue}{MM: #1}}}

% \newcommand\ALinline[1]{}
% \newcommand\AL[1]{}
% \newcommand\tom[1]{}
% \newcommand\ah[1]{}
% \newcommand\marco[1]{} \newcommand\marcoinline[1]{}


% \renewcommand{\marginparsep}{10pt}
% \renewcommand{\marginparwidth}{2.7cm}

\usepackage[toc,page]{appendix}
\usepackage{savesym}
\savesymbol{widering}
% \usepackage{yhmath}
% \usepackage{stackrel}
\usepackage{xspace,mathpartir,xparse,url,multirow,wrapfig}
% \usepackage{cmll,etoolbox}
\usepackage{thmtools,thm-restate}
\usepackage{ebmath,ebproof,ebutf8}
\usepackage{listings}

\newcommand{\isempty}[1]%
{
  \ifthenelse{\equal{#1}{}}%
    {EMPTY}% if #1 is empty
    {FULL, it contains the string '#1'}% if #1 is not empty
}
\newcommand{\coqident}[2]{\texttt{\lstinline{#1.#2}}}
\newcommand{\coqidentwoprefix}[2]{\texttt{\lstinline{#2}}}

\usepackage{environ}

%\def\thmt@innercounters{equation,lem,thm,cor,prop,theorem,lemma}

\usepackage{tikz-cats}
\usetikzlibrary{matrix,cd}
\tikzcdset{arrow style=tikz}
\tikzcdset{every arrow/.append style = -{Computer Modern Rightarrow[]}}
\tikzset{every arrow/.append style = -{Computer Modern Rightarrow[]}}
\tikzset{
  labelslt/.style n args={2}{%
    labelat={[left]{$\scriptstyle #1$}}{.45},%
    labelat={[right]{$\scriptstyle #2$}}{.55}%
    } %
  }
\tikzset{
  labelsltat/.style n args={4}{%
    labelat={[left]{$\scriptstyle #1$}}{#3},%
    labelat={[right]{$\scriptstyle #2$}}{#4}%
    } %
  }
\tikzset{twoof/.style={twocenter={#1}}}
\newcommand{\labelise}[2]{{{}^{#1}}•#2}

\newcounter{trou}[figure]
\setcounter{trou}{1}
\renewcommand{\thetrou}{(\arabic{figure}.\arabic{trou})}
\newcommand{\newtrou}{\thetrou \stepcounter{trou}}


% \usepackage{tocloft}
% \newcommand{\relate}[3]{\path (#1) -- node[anchor=mid] {$#2$} (#3) ;}
% \newcommand{\justify}[4][0.5]{\path (#2) -- node[anchor=mid,pos=#1] {\tiny (#3)} (#4) ;}
\renewcommand{\diaggrandhauteur}{.6}
\renewcommand{\diaggrandlargeur}{1}

\renewcommand{\TH}[1]{\marginpar{TH: #1}}
\newcommand{\THinline}[1]{\textcolor{red}{TH: #1}}

\newcommand{\psh}[1][ℂ]{\widehat{#1}}
\newcommand{\xto}[1]{\xrightarrow{#1}}
\newcommand{\xinto}[1]{\xarrow[into]{#1}}
\newcommand{\xonto}[1]{\xarrow[onto]{#1}}
\newcommand{\xotni}[1]{\xarrow[otni]{#1}}
\newcommand{\xTo}[1]{\xarrow[cell=0]{#1}}
\newcommand{\xmapsto}[1]{\xarrow[mapsto]{#1}}
\newcommand{\xleadsto}[1]{\xarrow[leadsto={1}]{#1}}
\newcommand{\xLeadsto}[1]{\xarrow[cell=0,leadsto={1}]{#1}}
\newcommand{\xoldto}[1]{\xarrow[pro,->]{#1}}
\newcommand{\xOldto}[1]{\xarrow[cell=0,pro]{#1}}
\newcommand{\xmodto}[1]{\xarrow[->,mod]{#1}}
\newcommand{\xotdom}[1]{\xarrow[<-,mod]{#1}}
\newcommand{\xot}[1]{\xleftarrow{#1}}
\newcommand{\shiftbar}[1]{⇑^{#1}}
\newcommand{\restrover}[2]{{#1}_{{/}#2}}
\newcommand{\restr}[2]{{#1}_{{|}#2}}
\newcommand{\mrestr}[2]{{#1}_{{↾}#2}}
\newcommand{\ens}[1]{\{ #1 \}}
\newcommand{\wbotright}[1]{{#1}^⋔}
\newcommand{\wbotleft}[1]{{{}^⋔{#1}}}
\newcommand{\wbotrightleft}[1]{\wbotleft{(\wbotright{#1})}}
\newcommand{\source}{𝐬}
\newcommand{\labels}{𝐥}
\newcommand{\but}{𝐭}
\newcommand{\op}[1]{{#1}^{\mathit{op}}}
\newcommand{\myop}{\mathit{op}}
\newcommand{\abar}{\overline{a}}
\newcommand{\bbar}{\overline{b}}
\newcommand{\fn}{\mathit{fn}}
\newcommand{\nfn}{¬\mathit{fn}}
\newcommand{\leftsub}[1]{{}_{#1}}
\newcommand{\contcons}[2]{\mathbin{\leftsub{#1}._{#2}}}
\newcommand{\dotjl}{\contcons{j}l}
\newcommand{\lv}[1]{\overleftarrow{#1}}
\newcommand{\DB}[1][S]{\mathrm{DB}_{#1}}

\DeclareMathOperator{\alg}{-\,\mathbf{alg}}
\DeclareMathOperator{\Alg}{-\,\mathbf{Alg}}
\DeclareMathOperator{\bptalg}{-\,\mathbf{alg}^•}
\DeclareMathOperator{\ptalg}{-\,\mathbf{alg}^{\pt}}
\newcommand{\pt}{∘}
\DeclareMathOperator{\Mod}{-\,\mathbf{Mod}}
\DeclareMathOperator{\Act}{-\,\mathbf{Act}}
\DeclareMathOperator{\Mnd}{-\,\mathbf{Mnd}}
\DeclareMathOperator{\mods}{-\,\mathbf{mod}}
\DeclareMathOperator{\Mon}{-\,\mathbf{Mon}}
\DeclareMathOperator{\Trans}{-\mathbf{Trans}}
\DeclareMathOperator{\DTrans}{-\mathbf{DTrans}}
\DeclareMathOperator{\DBAlg}{-\,\mathbf{DBAlg}}
\DeclareMathOperator{\RelMnd}{-\,\mathbf{RelMnd}}
\DeclareMathOperator{\AMon}{-\,\mathbf{AMon}}
\DeclareMathOperator{\Gph}{-\,\mathbf{Gph}}
\DeclareMathOperator{\DGph}{-\,\mathbf{DGph}}
\DeclareMathOperator{\mon}{-\,\mathbf{mon}}
\DeclareMathOperator{\Lifted}{-\,\mathbf{Liftable}}
\DeclareMathOperator{\LiftedAff}{-\,\mathbf{LiftableAff}}
\DeclareMathOperator{\LiftedAffPt}{-\,\mathbf{LiftableAffPt}}
\DeclareMathOperator{\id}{id}
\DeclareMathOperator{\el}{el}
\DeclareMathOperator{\colim}{colim}
\DeclareMathOperator{\bool}{bool}
\DeclareMathOperator{\num}{num}
\DeclareMathOperator{\obs}{obs}
\DeclareMathOperator{\sil}{sil}
\DeclareMathOperator{\scc}{succ}
\DeclareMathOperator{\prd}{pred}
\DeclareMathOperator{\nil}{nil}
\DeclareMathOperator{\si}{if}
\DeclareMathOperator{\siz}{ifz}
\DeclareMathOperator{\scase}{scase}
\DeclareMathOperator{\suc}{succ}
\DeclareMathOperator{\pred}{pred}
\DeclareMathOperator{\cons}{cons}
\DeclareMathOperator{\rec}{rec}
\DeclareMathOperator{\true}{true}
\DeclareMathOperator{\false}{false}
\DeclareMathOperator{\tl}{tl}
\DeclareMathOperator{\hd}{hd}
\DeclareMathOperator{\close}{close}
\DeclareMathOperator{\val}{val}
\DeclareMathOperator{\var}{var}
\DeclareMathOperator{\ev}{ev}
\DeclareMathOperator{\Lan}{Lan}
\DeclareMathOperator{\Ran}{Ran}
\DeclareMathOperator{\upswap}{σ}

\DeclareMathOperator{\im}{im}
\DeclareMathOperator{\add}{add}
\DeclareMathOperator{\mul}{mul}
\DeclareMathOperator{\app}{app}
\DeclareMathOperator{\lapp}{lapp}
\DeclareMathOperator{\abs}{abs}
\DeclareMathOperator{\diff}{diff}
\DeclareMathOperator{\plus}{plus}

\DeclareTextMath\tmlambda[lambda]{\lambda}

\newcommand{\dom}{\mathrm{dom}}


\newcommand\Tstrut{\rule{0pt}{3ex}}
\newcommand\Bstrut{\rule[-0.9ex]{0pt}{0pt}}
\newcommand\Bstrutbas{\rule[-1.2ex]{0pt}{0pt}}
\newcommand\Tstruthaut{\rule{0pt}{2em}}
\newcommand\Tstrutpetit{\rule{0pt}{1em}}

\newcommand{\subst}{\mathit{subst}}
\newcommand{\ST}{\mathrm{ST}}
\newcommand{\std}{\mathit{std}}
\newcommand{\wb}{\mathit{wb}}
\newcommand{\lax}{\mathit{lax}}
\newcommand{\pack}{\mathit{pack}}
\newcommand{\dipl}{\mathit{dipl}}
\newcommand{\struct}{\mathit{struct}}
\newcommand{\cocont}{{\mathit{cocont}}}
\newcommand{\algebraic}{{\mathit{alg}}}
\newcommand{\pointed}{{\mathit{pt}}}
\newcommand{\mnd}{{\mathit{mnd}}}
\newcommand{\ar}{{\mathit{ar}}}

\newcommand{\bind}{{\mathrm{bind}}}
\newcommand{\ret}{{\mathrm{ret}}}
\newcommand{\Kl}{{\mathrm{Kl}}}
\newcommand{\RelMod}{{\mathbf{RelMod}}}
\newcommand{\DbMod}{{\mathbf{DBMod}}}
\newcommand{\DbMon}{{\mathbf{DBMnd}}}

\newcommand{\lhs}{\mathsf{lhs}}
\newcommand{\rhs}{\mathsf{rhs}}
\usepackage{bbding}
% \newcommand{\fleur}{{\text{{\tiny \SixFlowerPetalDotted}}}}
\newcommand{\fleur}{*}
\newcommand{\iso}{≅}
\newcommand{\red}[1]{#1}

\newcommand{\liftleft}[1]{{{}^{#1}{↑}}}

\newcommand{\ajustedroit}[2][1]{\adjustbox{max width=#1\columnwidth,max height=.95\textheight}{#2}}%
\newcommand{\ajustetourne}[1]{\adjustbox{max width=\columnwidth,max height=.95\textheight}{
    \begin{turn}{90}
      #1
    \end{turn}
  }}%



\newcommand{\leaves}[1]{
  \path[draw] (#1.-140) -- node[coordinate,pos=.8] (left) {} +(0,-.2) ;
  \path[draw] (#1.-40) -- node[coordinate,pos=.8] (right) {} +(0,-.2) ;
  \path (left) -- node {$…$} (right) ;
}

\newcommand{\mkroot}[1]{
  \path[draw] (#1.90) -- +(0,.2) ;
}


% \declaretheorem[name=Theorem,numberwithin=section]{theorem}
% \declaretheorem[name=Lemma,sibling=theorem]{lemma}
% \declaretheorem[name=Proposition,numberlike=theorem]{proposition}
% \declaretheorem[name=Corollary,numberlike=theorem]{corollary}
% \declaretheorem[name=Terminology,sibling=theorem]{terminology}
% \declaretheorem[name=Notation,sibling=theorem]{notation}

\hyphenation{mo-noi-dal}

%
% If you use the hyperref package, please uncomment the following line
% to display URLs in blue roman font according to Springer's eBook style:
\renewcommand\UrlFont{\color{blue}\rmfamily}

\newcommand{\alert}[1]{\textbf{#1}}


% Astuce pour écrire ℸ dans le code et avoir un Γ inversé dans le pdf
\NewDocumentCommand{\GammaInv}{}{\mathpalette\doGammaInv\relax}
\makeatletter
\NewDocumentCommand{\doGammaInv}{mm}{%
  \reflectbox{$\m@th#1\Gamma$}%
}
\makeatother
\renewcommand{\daleth}{\GammaInv}

\newcommand{\montitre}{A more general categorical framework for
congruence of applicative bisimilarity in higher-order languages}
\newcommand{\monsoustitre}{} \newcommand{\montitrecourt}{A more general categorical framework for applicative bisimilarity}

\newcommand{\monabstract}{   }

\newcommand{\shift}{\mathrm{shift}}

\newtheorem{terminology}{Terminology}
\newtheorem{construction}{Construction}

\definecolor{bois}{rgb}{.5,0,0}
\definecolor{vert}{rgb}{0,0.6,0}
\definecolor{rouge}{rgb}{0.8,0,0}
\definecolor{violet}{rgb}{0.8,0,0.4}
\definecolor{marron}{rgb}{0.4,0.4,0}
\definecolor{bleu}{rgb}{0,0,0.6}
\AtBeginDocument
{
\newcommand{\colorie}[2]{{\color{#1} #2}}
\renewcommand{\vert}[1]{\colorie{vert}{#1}}
\newcommand{\rouge}[1]{\colorie{red}{#1}}
\newcommand{\orange}[1]{\colorie{orange}{#1}}
\newcommand{\violet}[1]{\colorie{violet}{#1}}
\newcommand{\marron}[1]{\colorie{marron}{#1}}
\newcommand{\bleu}[1]{\colorie{bleu}{#1}}
}


\newcommand{\enhanced}{enhanced\xspace}
\newcommand{\Enhanced}{Enhanced\xspace}
\newcommand{\enhancement}{enhancement\xspace}
\newcommand{\Enhancement}{Enhancement\xspace}
\newcommand{\anenhanced}{an enhanced\xspace}

\newcommand{\Fwow}[1][X]{Σ₀^{{+};{∼_{#1}}}}


%% 2012 ACM Computing Classification System (CSS) concepts
%% Generate at 'http://dl.acm.org/ccs/ccs.cfm'.
 \begin{CCSXML}
<ccs2012>
<concept>
<concept_id>10003752.10010124.10010131.10010133</concept_id>
<concept_desc>Theory of computation~Denotational semantics</concept_desc>
<concept_significance>500</concept_significance>
</concept>
<concept>
<concept_id>10003752.10010124.10010131.10010134</concept_id>
<concept_desc>Theory of computation~Operational semantics</concept_desc>
<concept_significance>500</concept_significance>
</concept>
<concept>
<concept_id>10003752.10003753.10003761.10003764</concept_id>
<concept_desc>Theory of computation~Process calculi</concept_desc>
<concept_significance>300</concept_significance>
</concept>
<concept>
<concept_id>10003752.10010124.10010131.10010137</concept_id>
<concept_desc>Theory of computation~Categorical semantics</concept_desc>
<concept_significance>300</concept_significance>
</concept>
</ccs2012>
\end{CCSXML}

\ccsdesc[500]{Theory of computation~Denotational semantics}
\ccsdesc[500]{Theory of computation~Operational semantics}
\ccsdesc[300]{Theory of computation~Process calculi}
\ccsdesc[300]{Theory of computation~Categorical semantics}%% End of generated code

\keywords{syntax ;  variable binding ; substitution ; category theory}


\DeclareUnicodeMathCharacter{21A4}{\mathrel{\xarrow[otspam]{}}}
