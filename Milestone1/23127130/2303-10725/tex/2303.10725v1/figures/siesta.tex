\begin{figure*}[t]
  \centering
   \includegraphics[width=0.65\textwidth]{images/siesta_iccv.png}
   \caption{A high-level overview of SIESTA. During the \emph{Wake Phase}, it transforms raw inputs into intermediate feature representations using network $\mathcal{H}$. The inputs are then compressed with tensor quantization and cached. Then, weights belonging to recently seen classes in network $\mathcal{F}$ are updated with a running class mean using the output vectors from $\mathcal{G}$. 
   Finally, inference is performed on the current sample.
   During the \emph{Sleep Phase}, a sampler uses a rehearsal policy to choose which examples should be reconstructed from the cached data for each mini-batch. Then, networks $\mathcal{G}$ and $\mathcal{F}$ are updated with backpropagation in a supervised manner. The wake/sleep cycles alternate.
    }
   \label{fig:siesta-overview}
\end{figure*}