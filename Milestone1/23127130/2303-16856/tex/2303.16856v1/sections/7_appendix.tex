\clearpage
\appendix

\begin{figure*}[t]
    \centering
    \includegraphics[width=160mm]{figures/styles.pdf} 
    \caption{\textbf{Pose Distributions of Generated Dance Motions on Different Styles of Music.} Different styles of dance are clearly distinguished, and the movements of each style are diverse.}
    \label{fig:styles}
\end{figure*}

\begin{figure*}[t]
    \centering
    \includegraphics[width=160mm]{figures/style_corr.pdf} 
    \caption{\textbf{A Ballet Jazz Style Dance Synthesized by Our Method.} Poses in the blue box is the seed motion. The generated dance is realistic and style-consistent.}
    \label{fig:style_corr}
\end{figure*}

\begin{figure*}[t]
    \centering
    \includegraphics[width=160mm]{figures/beat_corr.pdf} 
    \caption{\textbf{Visualization of Beat Matching.} A lock dance generated by our method. Red lines denote the musical beats, while kinetic beats are marked by boxes with different colors. The generated dance and music are well-matched in rhythm.}
    \label{fig:beat_corr}
\end{figure*}

\begin{figure*}[t]
    \centering
    \includegraphics[width=160mm]{figures/struct.pdf} 
    \caption{\textbf{Perception of Choreographic Structure.} A house dance synthesized by our method, where the same color box frames the similar motion segment. The input music contains some cyclic phrases (e.g., chorus), and repeated musical phrases correspond to repeated generated movements, indicating that our long-history attention mechanism successfully perceives the structure of the music.}
    \label{fig:struct}
\end{figure*}