\section{Related Work}
\label{sec:related_work}

\subsection{Human Motion Prediction}
The goal of the human motion prediction task is to predict future motions based on historical motions, which requires the model to learn a robust prior of human movements, including the skeleton structure and the underlying kinematic information. Conventional methods \cite{arikan2002interactive,kim2003rhythmic} usually construct a motion graph \cite{kovar2008motiongraph} and find an optimal path based on the transition probabilities between nodes. Motion graph has good interpretability, but the motion transitions are usually not smooth and the diversity of generated motions is limited by the graph scale. With the development of deep learning, deep-based methods have dominated this area. Compared with motion graphs, neural networks are able to generate smooth and diverse motions while the model can effectively compress large-scale dataset into the network parameters. Most existing deep-based models opt an auto-regressive manner to model the prediction of human motions, like recurrent neural networks (RNNs) \cite{fragkiadaki2015recurrent,mahmood2019amass,ghosh2017learning,martinez2017human,li2017auto,ao2022rhythmicgesticulator}, convolutional neural networks (CNNs) \cite{li2018convolutional,liu2020trajectorycnn}, and recent Transformer-based networks \cite{aksan2021spatio,mao2022weakly,ao2023gesturediffuclip}.

\subsection{Music-driven Dance Synthesis}
Music-driven dance synthesis system aims to model the complex temporal harmony between music and dance. Existing methods are divided into two lines, which are the motion graph-based method and the end-to-end method.

\subsubsection{Motion Graph-based Method}
The core of the motion graph-based method is to model the matches between two motion sequences and between music and motion. Early methods use some statistical models like the hidden markov model (HMM) \cite{kim2006making,lee2013music} to compute the transition probability between motion sequences and the emission probability between music and motion. Recently some works \cite{chen2021choreomaster,ye2020choreonet} explore the utilization of deep neural networks to model the matches above and achieve convincing results on the carefully collected dataset. Although motion graph-based methods can generate realistic motions, they must trade off the model capacity and the complexity of inference because of fussy training procedures.

\subsubsection{End-to-end Method}
Due to the strict requirement of the dataset and complicated training procedures, the popularity of motion graph-based methods is fading and end-to-end methods dominate the dance synthesis. In the context of the end-to-end model, dance generation is defined as a sequence-to-sequence translation task from music to motions without carefully labeled datasets. One line of the end-to-end model is the deterministic model. Different model structures, such as restricted boltzmann machines \cite{alemi2017groovenet}, LSTM \cite{huang2020dance,aristidou2021rhythm}, and Transformer \cite{li2021choreographer,li2022danceformer,siyao2022bailando}, are tried to improve the synthesis performance. Another line belongs to the probabilistic model. Because of the inherent one-to-many mapping from music to dance, some works utilize probabilistic models like generative adversarial networks (GANs) \cite{lee2019dancing2music,sun2020deepdance,ferreira2021learning} and normalizing flow \cite{guillermo2021transflower} to avoid falling into a mean pose. Most end-to-end approaches follow the auto-regressive manner, which suffers from severe error accumulation problem in long motion sequence generation. Both Huang \etal \cite{huang2020dance} and Aristidou \etal \cite{aristidou2021rhythm} draw on the idea of acRNN \cite{li2017auto} to alleviate this problem. Because we use Transformer instead of RNN, we design a new attention mechanism to avoid generated motions to freeze during long-term dance synthesis.

Most of the existing methods, whether motion graph-based or end-to-end, require paired data for training, which suffers from the data-limited problem. In this paper, we propose an unpaired data training scheme to loose the requirement on the dataset.