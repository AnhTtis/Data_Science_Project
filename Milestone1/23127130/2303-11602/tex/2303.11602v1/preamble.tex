


\usepackage[utf8]{inputenc}
%\usepackage{amsmath, amssymb, amscd, amsthm, amsfonts}
\usepackage{amsmath, amssymb, amsthm, amsfonts}
\usepackage{graphicx}
\usepackage[hidelinks]{hyperref}
\usepackage{braket}
\usepackage[dvipsnames]{xcolor}
\newcommand{\innerp}[2]{\left\langle #1 \vert #2 \right\rangle}
\usepackage{bbm}


\makeatletter
\newenvironment{breakablealgorithm}
  {% \begin{breakablealgorithm}
   \begin{center}
     \refstepcounter{algorithm}% New algorithm
     \hrule height.8pt depth0pt \kern2pt% \@fs@pre for \@fs@ruled
     \renewcommand{\caption}[2][\relax]{% Make a new \caption
       {\raggedright\textbf{\fname@algorithm~\thealgorithm} ##2\par}%
       \ifx\relax##1\relax % #1 is \relax
         \addcontentsline{loa}{algorithm}{\protect\numberline{\thealgorithm}##2}%
       \else % #1 is not \relax
         \addcontentsline{loa}{algorithm}{\protect\numberline{\thealgorithm}##1}%
       \fi
       \kern2pt\hrule\kern2pt
     }
  }{% \end{breakablealgorithm}
     \kern2pt\hrule\relax% \@fs@post for \@fs@ruled
   \end{center}
  }
\makeatother

\newcommand\g{\boldsymbol g}
\newcommand\projspace{\mathbf P}
\newcommand\Lt{\mathcal H}
\renewcommand\H{\mathcal H}
\newcommand\CC{\mathbb C}
\newcommand\RR{\mathbb R}
\newcommand\EE{\mathbb E}
\newcommand\C{\mathcal C}
\newcommand\ec[1]{[#1]}
\newcommand\Span{\operatorname{span}}
\newcommand\ol[2]{\langle#1,#2\rangle}
\newcommand\Lq{\ell}
\newcommand\Lpsi{\ell^{\psi}}
\newcommand\localE{\mathcal{E}}
\newcommand\lE\localE
\newcommand\psiv{\Psi}
\newcommand\ifrac[2]{#1/#2}
\newcommand\eemph[1]{\textbf{#1}}
\newcommand\F{\mathcal F}
\renewcommand\L{\mathcal L}
\newcommand\FG{\mathcal G}
\newcommand\functional[2]{\braket{#1,#2}}
\newcommand\argmin{\operatorname{argmin}}
\newcommand\argmax{\operatorname{argmax}}
\renewcommand\v{\mathbf v}
\newcommand\w{\mathbf w}
\renewcommand\H{\mathcal H}
\newcommand\algsym{\mathfrak A}
\newcommand\NN{\mathbb{N}}
% Todonotes is useful during development; simply uncomment the next line
%    and comment out the line below the next line to turn off comments
%\usepackage[disable,textsize=tiny]{todonotes}
\usepackage[textsize=tiny]{todonotes}

%%%%% comments
\usepackage{soul}

%\usepackage{color}
\newenvironment{temp}{\begingroup\color{gray}\itshape}{\endgroup}
\newcommand\shorttemp[1]{\begin{temp}#1\end{temp}}


\iffinal
\newcommand\cmnt[3]{}
\else
\newcommand\cmnt[3]{\textcolor{#3}{\hl{[#1: #2]}}}
\fi

\newcommand\hide[2]{{\textcolor{gray}{}}} % the first argument is a comment about why it's hidden 
\newcommand\replace[2]{\begingroup\color{red}#2\endgroup}

%individual
\usetikzlibrary{positioning, calc, fit}


\newenvironment{longcomment}{\begingroup\color{red}\itshape}{\endgroup}


\newcommand{\ZD}[1]{\cmnt{ZD}{#1}{blue}}
\newcommand{\NA}[1]{\cmnt{NA}{#1}{red}}
\newcommand{\GG}[1]{\cmnt{GG}{#1}{magenta}}
\newcommand{\LL}[1]{\cmnt{LL}{#1}{brown}}



%%%%%
\newcommand\wf{{f}}