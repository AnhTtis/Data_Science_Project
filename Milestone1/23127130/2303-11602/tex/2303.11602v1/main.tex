
%%%%%%%% ICML 2022 EXAMPLE LATEX SUBMISSION FILE %%%%%%%%%%%%%%%%%

\documentclass[nohyperref]{article}


% Recommended, but optional, packages for figures and better typesetting:
\usepackage{microtype}
\usepackage{graphicx}
\usepackage{subcaption}
\usepackage{booktabs} % for professional tables

\newif\iffinal
\finaltrue
% \usepackage[latin1]{inputenc}
\usepackage[british]{babel}
\usepackage[all]{xy}
\usepackage{amscd}
\usepackage{amssymb}
\usepackage{amsthm}
\usepackage{enumitem}
\usepackage{mathrsfs,bbm}
\usepackage{xcolor,graphicx}
\usepackage{graphics}
\usepackage{soul}
\usepackage{comment}
\usepackage[all]{xy}
\usepackage{amscd}
\usepackage{amssymb,amsmath,latexsym}
\usepackage{amsthm}
\usepackage{enumitem}
\usepackage{mathrsfs,bbm}
\usepackage{dsfont}
\usepackage{tikz-cd}
\usepackage[T1]{fontenc}
\usepackage[utf8]{inputenc}  
 %
%%%%%%%%%%%%%%%%%%%%%%%%%%%%%%%%%%
%pagestyle
%%%%%%%%%%%%%%%%%%%%%%%%%%%%%%%%%%
%\pagestyle{plain}
\textwidth=430pt
\headsep=.7cm
\evensidemargin=15pt
\oddsidemargin=15pt
\leftmargin=0cm
\rightmargin=0cm
%%
%%%%%%%%%%%%%%%%%%%%%%%
\newcommand*\fixitem {\item[]%
  \refstepcounter{enumi}\hskip-\leftmargin\labelenumi\hskip\labelsep}
\newtheorem*{mainthm}{Main Theorem}
\newtheorem*{mainthm1}{Theorem}
\newtheorem*{maincor}{Corollary}
\usepackage[colorlinks=true]{hyperref}
\DeclareMathOperator{\Forall}{\forall}
\DeclareMathOperator{\Exists}{\exists}
\DeclareMathOperator{\ord}{ord}
\newcommand{\phiD}{\varphi_D}
\newcommand{\phiDI}{\varphi_{\mathbf{D}_I}}
\newcommand{\phiDIj}{\varphi_{\mathbf{D}_I (j)}}
\newcommand{\phiH}{\varphi_H}
\newcommand{\phiTimes}{\phiD \otimes \phiH}
\newcommand{\phiTimesDI}{\varphi_{\mathbf{D}_I} \otimes \phiH}
\newcommand{\R}{\mathscr{A}}
\newcommand{\X}{\mathscr{X}}
\newcommand{\Xf}{\mathscr{X}_{(k_0 ,i)}[r_0]}
\newcommand{\Xfr}{\mathscr{X}_{(k_0,i)}[r]}
\newcommand{\hotimes}{\widehat{\otimes}}
\newcommand{\C}{\mathbb{C}_p}
\newcommand{\V}{\mathscr{V}}
\newcommand{\B}{\mathscr{B}}
\newcommand{\dualD}{\mathfrak{D}}
\newcommand{\Dg}{\mathbf{D}}
\newcommand{\DD}{\mathcal{D}^0}
\newcommand{\DDg}{\mathcal{D}}
\newcommand{\DV}{\mathcal{D}}
\newcommand{\W}{\mathscr{W}_N}
\newcommand{\Ao}{\mathbf{A}^\circ}
\newcommand{\AoK}{\mathbf{A}^\circ_{\K}}
\newcommand{\AK}{\mathbf{A}_{/\K}}
\newcommand{\OOO}{\mathscr{A}^\circ}
\newcommand{\K}{\mathcal{K}} 
\newcommand{\OK}{\mathcal{O}_{\K}}
\newcommand{\varprojlog}[1]{\underleftarrow{\log\!^{#1}}}
\newcommand{\T}{\mathscr{T}}
\newcommand{\TT}{\mathbf{T}}
\newcommand{\VV}{\mathbf{V}}
\newcommand{\HH}{\mathcal{H}}
\newcommand{\hh}{\mathcal{H}^+}
\newcommand{\HG}[2]{\mathcal{H}_{#1}(#2)}
\newcommand{\hhl}{\mathcal{H}^{+,[l]}}
\newcommand{\hhj}{\mathcal{H}^{+,[j]}}
\newcommand{\hhjj}{\mathcal{H}^{+,[l,l']}}
\newcommand{\GS}{G_{\mathbb{Q},S}}
\newcommand{\Rf}{R_{(k_0 ,i)}[r_0]}
\newcommand{\Rfr}{R_{(k_0 ,i)}[r]}
\newcommand{\parT}{\langle T\rangle}
\newcommand{\Zf}{Z_{(k_0 ,i)}[r_0]}
\newcommand{\Zfr}{\mathscr{Z}_{(k_0 ,i)}[r]}
\newcommand{\ZFf}{\mathscr{Z}_{(k_0 ,i)}[r_0]}
\newcommand{\ZFfr}{\mathscr{Z}_{(k_0 ,i)}[r]}
\newcommand{\ZF}{\mathscr{Z}}
% hyperref makes hyperlinks in the resulting PDF.
% If your build breaks (sometimes temporarily if a hyperlink spans a page)
% please comment out the following usepackage line and replace
% \usepackage{icml2022} with \usepackage[nohyperref]{icml2022} above.
\usepackage{hyperref}


% Attempt to make hyperref and algorithmic work together better:
\newcommand{\theHalgorithm}{\arabic{algorithm}}


% Use the following line for the initial blind version submitted for review:
%\usepackage{icml2023}
%\usepackage{icmlcustompreprint}

% If accepted, instead use the following line for the camera-ready submission:
\usepackage[accepted]{icml2023}

% For theorems and such
\usepackage{amsmath}
\usepackage{amssymb}
\usepackage{mathtools}
\usepackage{amsthm}

% if you use cleveref..
\usepackage[capitalize,noabbrev]{cleveref}

%%%%%%%%%%%%%%%%%%%%%%%%%%%%%%%%
% THEOREMS
%%%%%%%%%%%%%%%%%%%%%%%%%%%%%%%%
\theoremstyle{plain}
\newtheorem{theorem}{Theorem}[section]
\newtheorem{proposition}[theorem]{Proposition}
\newtheorem{lemma}[theorem]{Lemma}
\newtheorem{corollary}[theorem]{Corollary}
\newtheorem{itdef}[theorem]{Convention}
\newtheorem{problem}{Problem}[section]
\theoremstyle{definition}
\newtheorem{definition}[theorem]{Definition}
\newtheorem{assumption}[theorem]{Assumption}
\newtheorem{remark}[theorem]{Remark}
%\theoremstyle{remark}
\newtheorem{example}[theorem]{Example}
\newtheorem{conjecture}[theorem]{Conjecture}


% The \icmltitle you define below is probably too long as a header.
% Therefore, a short form for the running title is supplied here:
%\icmltitlerunning{Submission and Formatting Instructions for ICML 2022}



%
%%%%%%%%% ICML 2023 EXAMPLE LATEX SUBMISSION FILE %%%%%%%%%%%%%%%%%
%
%\documentclass{article}
%
%% Recommended, but optional, packages for figures and better typesetting:
%\usepackage{microtype}
%\usepackage{graphicx}
%\usepackage{subfigure}
%\usepackage{booktabs} % for professional tables
%
%% hyperref makes hyperlinks in the resulting PDF.
%% If your build breaks (sometimes temporarily if a hyperlink spans a page)
%% please comment out the following usepackage line and replace
%% \usepackage{icml2023} with \usepackage[nohyperref]{icml2023} above.
%\usepackage{hyperref}
%
%
%% Attempt to make hyperref and algorithmic work together better:
%\newcommand{\theHalgorithm}{\arabic{algorithm}}
%
%% Use the following line for the initial blind version submitted for review:
%\usepackage{icmlcustompreprint}
%
%% If accepted, instead use the following line for the camera-ready submission:
%% \usepackage[accepted]{icml2023}
%
%% For theorems and such
%\usepackage{amsmath}
%\usepackage{amssymb}
%\usepackage{mathtools}
%\usepackage{amsthm}
%
%% if you use cleveref..
%\usepackage[capitalize,noabbrev]{cleveref}
%
%%%%%%%%%%%%%%%%%%%%%%%%%%%%%%%%%
%% THEOREMS
%%%%%%%%%%%%%%%%%%%%%%%%%%%%%%%%%
%\theoremstyle{plain}
%\newtheorem{theorem}{Theorem}[section]
%\newtheorem{proposition}[theorem]{Proposition}
%\newtheorem{lemma}[theorem]{Lemma}
%\newtheorem{corollary}[theorem]{Corollary}
%\theoremstyle{definition}
%\newtheorem{definition}[theorem]{Definition}
%\newtheorem{assumption}[theorem]{Assumption}
%\theoremstyle{remark}
%\newtheorem{remark}[theorem]{Remark}
%
%% Todonotes is useful during development; simply uncomment the next line
%%    and comment out the line below the next line to turn off comments
%%\usepackage[disable,textsize=tiny]{todonotes}
%\usepackage[textsize=tiny]{todonotes}
%
%
%% The \icmltitle you define below is probably too long as a header.
%% Therefore, a short form for the running title is supplied here:
%
%


\begin{document}

\twocolumn[
\icmltitle{Convergence of stochastic gradient descent  on parameterized sphere with applications to variational Monte Carlo simulation}
%\icmltitle{Convergence of variational Monte Carlo simulation and SGD on implicit subsets of the sphere}
%\icmltitle{Convergence of stochastic gradient descent on projective space with applications to variational Monte Carlo simulation}

% It is OKAY to include author information, even for blind
% submissions: the style file will automatically remove it for you
% unless you've provided the [accepted] option to the icml2022
% package.

% List of affiliations: The first argument should be a (short)
% identifier you will use later to specify author affiliations
% Academic affiliations should list Department, University, City, Region, Country
% Industry affiliations should list Company, City, Region, Country

% You can specify symbols, otherwise they are numbered in order.
% Ideally, you should not use this facility. Affiliations will be numbered
% in order of appearance and this is the preferred way.
\icmlsetsymbol{equal}{*}

\begin{icmlauthorlist}
\icmlauthor{Nilin Abrahamsen}{bm}
\icmlauthor{Zhiyan Ding}{bm}
\icmlauthor{Gil Goldshlager}{bm}
\icmlauthor{Lin Lin}{bm,lbl}
%\icmlauthor{Firstname5 Lastname5}{yyy}
%\icmlauthor{Firstname6 Lastname6}{sch,yyy,comp}
%\icmlauthor{Firstname7 Lastname7}{comp}
%\icmlauthor{}{sch}
%\icmlauthor{Firstname8 Lastname8}{sch}
%\icmlauthor{Firstname8 Lastname8}{yyy,comp}
%\icmlauthor{}{sch}
%\icmlauthor{}{sch}
\end{icmlauthorlist}

\icmlaffiliation{bm}{Department of Mathematics, University of California, Berkeley, CA 94720, USA}
\icmlaffiliation{lbl}{Applied Mathematics and Computational Research Division, Lawrence Berkeley National Laboratory, Berkeley, CA 94720, USA}

%\icmlaffiliation{bm}{Department of XXX, University of YYY, Location, Country}
%\icmlaffiliation{comp}{Company Name, Location, Country}
%\icmlaffiliation{sch}{School of ZZZ, Institute of WWW, Location, Country}

%\icmlcorrespondingauthor{Nilin Abrahamsen}{nilin@berkeley.edu}

% for arxiv purpose if needed.
\icmlcorrespondingauthor{Lin Lin}{linlin@math.berkeley.edu}
% add yourself here

%\icmlcorrespondingauthor{Firstname2 Lastname2}{first2.last2@www.uk}

% You may provide any keywords that you
% find helpful for describing your paper; these are used to populate
% the "keywords" metadata in the PDF but will not be shown in the document
\icmlkeywords{Machine Learning, ICML}

\vskip 0.3in
]

% this must go after the closing bracket ] following \twocolumn[ ...

% This command actually creates the footnote in the first column
% listing the affiliations and the copyright notice.
% The command takes one argument, which is text to display at the start of the footnote.
% The \icmlEqualContribution command is standard text for equal contribution.
% Remove it (just {}) if you do not need this facility.

%\printAffiliationsAndNotice{}  % leave blank if no need to mention equal contribution

\printAffiliationsAndNotice{}

\begin{abstract}
%We analyze the stochastic gradient descent algorithm for scale-invariant loss functions in a parameterized function space. This problem is equivalent to a many-to-one parameterization of the sphere of normalized functions. We prove that the expected loss converges in a scale-invariant supervised learning setting as well as in the scale-invariant variational Monte Carlo setting employed in quantum chemistry.
We analyze stochastic gradient descent (SGD) type algorithms on a high-dimensional sphere which is parameterized by a neural network up to a normalization constant. We provide a new algorithm for the setting of supervised learning and show its convergence both theoretically and numerically. We also provide the first proof of convergence for the unsupervised setting, which corresponds to the widely used variational Monte Carlo (VMC) method in quantum physics. 
\end{abstract}


% \ifshowchanges
% \iffinal\else
% %DIF 1c1
%DIF LATEXDIFF DIFFERENCE FILE
%DIF DEL mainPRE-ACCEPT.tex   Tue Mar 14 16:01:46 2023
%DIF ADD main.tex             Tue Mar 14 16:14:09 2023
%DIF < \documentclass[conference]{IEEEtran}
%DIF -------
\documentclass[sigconf]{acmart} %DIF > 
%DIF -------
%\documentclass[letterpaper, 10 pt, conference]{ieeeconf} 
%DIF 3c3
%DIF < \IEEEoverridecommandlockouts
%DIF -------
%\IEEEoverridecommandlockouts %DIF > 
%DIF -------

% \documentclass[runningheads]{llncs}

% The preceding line is only needed to identify funding in the first footnote. If that is unneeded, please comment it out.
\usepackage{url}
%DIF 9-10c9-10
%DIF < \usepackage{cite}
%DIF < \usepackage{amsmath,amssymb,amsfonts}
%DIF -------
% \usepackage{cite} %DIF > 
\usepackage{amsmath,amsfonts} %DIF > 
%DIF -------
\usepackage{accents}
% \usepackage{algorithmic}
\usepackage[linesnumbered,ruled]{algorithm2e}
\usepackage[noend]{algpseudocode}
% \usepackage[nottoc]{tocbibind}
% \usepackage{diagbox}
\usepackage{graphicx}
\usepackage{textcomp}
%DIF 19-20c19-20
%DIF < \usepackage[table,xcdraw]{xcolor}
%DIF < \usepackage{xcolor}
%DIF -------
%\usepackage[table,xcdraw]{xcolor} %DIF > 
%\usepackage{xcolor} %DIF > 
%DIF -------
\usepackage{comment}
\usepackage{multicol,multirow}
\usepackage[T1]{fontenc}
\def\BibTeX{{\rm B\kern-.05em{\sc i\kern-.025em b}\kern-.08em
    T\kern-.1667em\lower.7ex\hbox{E}\kern-.125emX}}
\usepackage{todonotes}
\usepackage{caption}
\usepackage{wrapfig}
\usepackage{subfig}
\newsavebox{\measurebox}
%\usepackage[disable]{todonotes}
\usepackage{threeparttable}
\usepackage[font=small]{caption} 
\usepackage{enumitem}
% \setlist[itemize]{align=left,left=-10pt..-5em}
% \usepackage[hang,flushmargin]{footmisc}
% \usepackage{lipsum}

%One must be careful when importing hyperref. Usually, it has to be the last package to be imported, but there might be some exceptions to this rule.
\usepackage{hyperref}
\hypersetup{
    colorlinks=false
}

\def\UrlBreaks{\do\/\do-}

\makeatletter
\newcommand{\algorithmfootnote}[2][\footnotesize]{%
  \let\old@algocf@finish\@algocf@finish% Store algorithm finish macro
  \def\@algocf@finish{\old@algocf@finish% Update finish macro to insert "footnote"
    \leavevmode\rlap{\begin{minipage}{\linewidth}
    #1#2
    \end{minipage}}%
  }%
}
\makeatother

\newcommand*{\Scale}[2][4]{\scalebox{#1}{$#2$}}%
 %DIF > 
%% %DIF > 
%% \BibTeX command to typeset BibTeX logo in the docs %DIF > 
\AtBeginDocument{% %DIF > 
  \providecommand\BibTeX{{% %DIF > 
    Bib\TeX}}} %DIF > 
 %DIF > 
%% Rights management information.  This information is sent to you %DIF > 
%% when you complete the rights form.  These commands have SAMPLE %DIF > 
%% values in them; it is your responsibility as an author to replace %DIF > 
%% the commands and values with those provided to you when you %DIF > 
%% complete the rights form. %DIF > 
\setcopyright{acmcopyright} %DIF > 
\copyrightyear{2023} %DIF > 
\acmYear{2023} %DIF > 
\acmDOI{XX.XXXX/XXXXXXX.XXXXXXX} %DIF > 
 %DIF > 
%% These commands are for a PROCEEDINGS abstract or paper. %DIF > 
\acmConference[CHASE '23]{EEE/ACM international conference on Connected Health: Applications, Systems and Engineering Technologies}{June 21-23, %DIF > 
  2023}{Orlando, FL} %DIF > 
%% %DIF > 
%%  Uncomment \acmBooktitle if the title of the proceedings is different %DIF > 
%%  from ``Proceedings of ...''! %DIF > 
%% %DIF > 
%%\acmBooktitle{Woodstock '18: ACM Symposium on Neural Gaze Detection, %DIF > 
%%  June 03--05, 2018, Woodstock, NY} %DIF > 
\acmPrice{15.00} %DIF > 
\acmISBN{978-1-4503-XXXX-X/18/06} %DIF > 
%DIF PREAMBLE EXTENSION ADDED BY LATEXDIFF
%DIF UNDERLINE PREAMBLE %DIF PREAMBLE
\RequirePackage[normalem]{ulem} %DIF PREAMBLE
\RequirePackage{color}\definecolor{RED}{rgb}{1,0,0}\definecolor{BLUE}{rgb}{0,0,1} %DIF PREAMBLE
\providecommand{\DIFaddtex}[1]{{\protect\color{blue}\uwave{#1}}} %DIF PREAMBLE
\providecommand{\DIFdeltex}[1]{{\protect\color{red}\sout{#1}}}                      %DIF PREAMBLE
%DIF SAFE PREAMBLE %DIF PREAMBLE
\providecommand{\DIFaddbegin}{} %DIF PREAMBLE
\providecommand{\DIFaddend}{} %DIF PREAMBLE
\providecommand{\DIFdelbegin}{} %DIF PREAMBLE
\providecommand{\DIFdelend}{} %DIF PREAMBLE
\providecommand{\DIFmodbegin}{} %DIF PREAMBLE
\providecommand{\DIFmodend}{} %DIF PREAMBLE
%DIF FLOATSAFE PREAMBLE %DIF PREAMBLE
\providecommand{\DIFaddFL}[1]{\DIFadd{#1}} %DIF PREAMBLE
\providecommand{\DIFdelFL}[1]{\DIFdel{#1}} %DIF PREAMBLE
\providecommand{\DIFaddbeginFL}{} %DIF PREAMBLE
\providecommand{\DIFaddendFL}{} %DIF PREAMBLE
\providecommand{\DIFdelbeginFL}{} %DIF PREAMBLE
\providecommand{\DIFdelendFL}{} %DIF PREAMBLE
%DIF HYPERREF PREAMBLE %DIF PREAMBLE
\providecommand{\DIFadd}[1]{\texorpdfstring{\DIFaddtex{#1}}{#1}} %DIF PREAMBLE
\providecommand{\DIFdel}[1]{\texorpdfstring{\DIFdeltex{#1}}{}} %DIF PREAMBLE
\newcommand{\DIFscaledelfig}{0.5}
%DIF HIGHLIGHTGRAPHICS PREAMBLE %DIF PREAMBLE
\RequirePackage{settobox} %DIF PREAMBLE
\RequirePackage{letltxmacro} %DIF PREAMBLE
\newsavebox{\DIFdelgraphicsbox} %DIF PREAMBLE
\newlength{\DIFdelgraphicswidth} %DIF PREAMBLE
\newlength{\DIFdelgraphicsheight} %DIF PREAMBLE
% store original definition of \includegraphics %DIF PREAMBLE
\LetLtxMacro{\DIFOincludegraphics}{\includegraphics} %DIF PREAMBLE
\newcommand{\DIFaddincludegraphics}[2][]{{\color{blue}\fbox{\DIFOincludegraphics[#1]{#2}}}} %DIF PREAMBLE
\newcommand{\DIFdelincludegraphics}[2][]{% %DIF PREAMBLE
\sbox{\DIFdelgraphicsbox}{\DIFOincludegraphics[#1]{#2}}% %DIF PREAMBLE
\settoboxwidth{\DIFdelgraphicswidth}{\DIFdelgraphicsbox} %DIF PREAMBLE
\settoboxtotalheight{\DIFdelgraphicsheight}{\DIFdelgraphicsbox} %DIF PREAMBLE
\scalebox{\DIFscaledelfig}{% %DIF PREAMBLE
\parbox[b]{\DIFdelgraphicswidth}{\usebox{\DIFdelgraphicsbox}\\[-\baselineskip] \rule{\DIFdelgraphicswidth}{0em}}\llap{\resizebox{\DIFdelgraphicswidth}{\DIFdelgraphicsheight}{% %DIF PREAMBLE
\setlength{\unitlength}{\DIFdelgraphicswidth}% %DIF PREAMBLE
\begin{picture}(1,1)% %DIF PREAMBLE
\thicklines\linethickness{2pt} %DIF PREAMBLE
{\color[rgb]{1,0,0}\put(0,0){\framebox(1,1){}}}% %DIF PREAMBLE
{\color[rgb]{1,0,0}\put(0,0){\line( 1,1){1}}}% %DIF PREAMBLE
{\color[rgb]{1,0,0}\put(0,1){\line(1,-1){1}}}% %DIF PREAMBLE
\end{picture}% %DIF PREAMBLE
}\hspace*{3pt}}} %DIF PREAMBLE
} %DIF PREAMBLE
\LetLtxMacro{\DIFOaddbegin}{\DIFaddbegin} %DIF PREAMBLE
\LetLtxMacro{\DIFOaddend}{\DIFaddend} %DIF PREAMBLE
\LetLtxMacro{\DIFOdelbegin}{\DIFdelbegin} %DIF PREAMBLE
\LetLtxMacro{\DIFOdelend}{\DIFdelend} %DIF PREAMBLE
\DeclareRobustCommand{\DIFaddbegin}{\DIFOaddbegin \let\includegraphics\DIFaddincludegraphics} %DIF PREAMBLE
\DeclareRobustCommand{\DIFaddend}{\DIFOaddend \let\includegraphics\DIFOincludegraphics} %DIF PREAMBLE
\DeclareRobustCommand{\DIFdelbegin}{\DIFOdelbegin \let\includegraphics\DIFdelincludegraphics} %DIF PREAMBLE
\DeclareRobustCommand{\DIFdelend}{\DIFOaddend \let\includegraphics\DIFOincludegraphics} %DIF PREAMBLE
\LetLtxMacro{\DIFOaddbeginFL}{\DIFaddbeginFL} %DIF PREAMBLE
\LetLtxMacro{\DIFOaddendFL}{\DIFaddendFL} %DIF PREAMBLE
\LetLtxMacro{\DIFOdelbeginFL}{\DIFdelbeginFL} %DIF PREAMBLE
\LetLtxMacro{\DIFOdelendFL}{\DIFdelendFL} %DIF PREAMBLE
\DeclareRobustCommand{\DIFaddbeginFL}{\DIFOaddbeginFL \let\includegraphics\DIFaddincludegraphics} %DIF PREAMBLE
\DeclareRobustCommand{\DIFaddendFL}{\DIFOaddendFL \let\includegraphics\DIFOincludegraphics} %DIF PREAMBLE
\DeclareRobustCommand{\DIFdelbeginFL}{\DIFOdelbeginFL \let\includegraphics\DIFdelincludegraphics} %DIF PREAMBLE
\DeclareRobustCommand{\DIFdelendFL}{\DIFOaddendFL \let\includegraphics\DIFOincludegraphics} %DIF PREAMBLE
%DIF END PREAMBLE EXTENSION ADDED BY LATEXDIFF

\begin{document}
\DIFdelbegin %DIFDELCMD < \bstctlcite{IEEEexample:BSTcontrol}
%DIFDELCMD < %%%
\DIFdelend %DIF > \bstctlcite{IEEEexample:BSTcontrol}

\DIFdelbegin %DIFDELCMD < \title{%%%
\DIFdelend \DIFaddbegin \title[short title]{\DIFadd{Short: }\DIFaddend Basal-Adjust: Trend Prediction Alerts and Adjusted Basal Rates for Hyperglycemia Prevention}

\DIFdelbegin %DIFDELCMD < \author{Short Paper}
%DIFDELCMD < \begin{comment}%DIFDELCMD < 
%DIFDELCMD < \author{Chloe Smith\\
%DIFDELCMD < % }
%DIFDELCMD < % \authorrunning{F. Author et al.} 
%DIFDELCMD < % First names are abbreviated in the running head.
%DIFDELCMD < % If there are more than two authors, 'et al.' is used.
%DIFDELCMD < %
%DIFDELCMD < 
%DIFDELCMD < % \institute
%DIFDELCMD < { University of Virginia, Charlottesville, VA 22904, USA %\\
%DIFDELCMD < \email
%DIFDELCMD < %Department of Electrical and Computer Engineering,
%DIFDELCMD < {\{cas8ds\}@virginia.edu}\\
%DIFDELCMD < 
%DIFDELCMD < 
%DIFDELCMD < 
%DIFDELCMD < }
%DIFDELCMD < }
%DIFDELCMD < \end{comment}
%DIFDELCMD < %%%
\DIFdelend %DIF >  \author{Short Paper}
\DIFaddbegin \begin{comment}
\author{\\
% }
% \authorrunning{F. Author et al.} 
% First names are abbreviated in the running head.
% If there are more than two authors, 'et al.' is used.
%

% \institute
{ University of Virginia, Charlottesville, VA 22904, USA %\\
\email
%Department of Electrical and Computer Engineering,
{\{cas8ds\}@virginia.edu}\\



}
}
\end{comment}
\DIFaddend 

\DIFdelbegin %DIFDELCMD < \maketitle
%DIFDELCMD < \thispagestyle{plain}
%DIFDELCMD < \pagestyle{plain}
%DIFDELCMD < %%%
\DIFdelend \DIFaddbegin \author{\DIFadd{Chloe Smith}}
%DIF >  \authornotemark[1]
\email{cas8ds@virginia.edu}
\affiliation{%
  \institution{University of Virginia}
  \city{Charlottesville}
  \state{Virginia}
  \country{USA}
  \postcode{22904}
  }
\DIFaddend 


\DIFaddbegin \author{\DIFadd{Maxfield Kouzel}}
%DIF >  \authornotemark[1]
\email{mak3zaa@virginia.edu}
\affiliation{%
  \institution{University of Virginia}
  \city{Charlottesville}
  \state{Virginia}
  \country{USA}
  \postcode{22904}
}


\author{\DIFadd{Xugui Zhou}}
%DIF >  \authornotemark[1]
\email{xugui@virginia.edu}
\affiliation{%
  \institution{University of Virginia}
  \city{Charlottesville}
  \state{Virginia}
  \country{USA}
  \postcode{22904}
}

\author{\DIFadd{Homa Alemzadeh}}
%DIF >  \authornotemark[1]
\email{ha4d@virginia.edu}
\affiliation{%
  \institution{University of Virginia}
  \city{Charlottesville}
  \state{Virginia}
  \country{USA}
  \postcode{22904}
}


\DIFaddend \begin{abstract} 
Significant advancements in type 1 diabetes treatment have been made in the development of state-of-the-art Artificial Pancreas Systems (APS). However, lapses currently exist in the timely treatment of unsafe blood glucose (BG) levels, especially in the case of rebound hyperglycemia. We propose a machine learning (ML) method for predictive BG scenario categorization that outputs messages alerting the patient to upcoming BG trends \DIFdelbegin \DIFdel{in order }\DIFdelend to allow for earlier, educated treatment. 
In addition to standard notifications of predicted hypoglycemia and hyperglycemia, we introduce BG scenario-specific alert messages and the preliminary steps towards precise \DIFdelbegin \DIFdel{increased basal rate }\DIFdelend \DIFaddbegin \DIFadd{basal }\DIFaddend suggestions for prevention of rebound hyperglycemia.
%As part of this method, we introduce carbohydrate input in a state of hypoglycemia as a parameter to determine increase basal insulin rates to be precisely recommended in our messages when a safe state is recovered in order to prevent rebound hyperglycemia. 
Experimental evaluation on the DCLP3 clinical dataset achieves >98\% accuracy and >79\% precision for predicting rebound high events for patient alerts.
\end{abstract}
%on a realistic closed-loop APS testbed using simulated data, an actual clinical dataset, and one participant's data

\DIFdelbegin %DIFDELCMD < \begin{IEEEkeywords}
%DIFDELCMD < %%%
%\DIFdel{safety, cyber-physical systems, medical device, wearable systems, body sensor networks.
%%DIFDELCMD < \end{IEEEkeywords}
%DIFDELCMD < %%%
\DIFdelend \DIFaddbegin \ccsdesc[500]{Computer systems organization~Dependable and fault-tolerant system; Embedded and cyber-physical systems}
\ccsdesc{Applied computing~Life and medical sciences}
\ccsdesc[100]{Computing Methodologies~Machine learning; Modeling and simulation}
\DIFaddend 

\DIFaddbegin \keywords{
Safety, Cyber-Physical Systems, Medical Device, Wearable Systems, Body Sensor Networks.}

\maketitle
\thispagestyle{plain}
\pagestyle{plain}

\DIFaddend % \vspace{-2em}
\section{Introduction}
%para 1
% millions of people are suffering from Type I diabetes. APS is one of the effective treatments, however, often makes over- or under-adjustments of the BG and insulin amount, causing hypoglycemia to hyperglycemia rebound ...

%para 2
%DIF <  limitations of current work: (1) only use the nighttime data \cite{Taisa2020, Dutta2018,dsn2021zhou,zhou2023knowsafe} (2) without meals input
%DIF >  limitations of current work: (1) only use the nighttime data \cite{Taisa2020, Dutta2018,zhou2021data,zhou2023knowsafe} (2) without meals input

%para 3
% we propose a novel framework that encodes the carb info into an attention layer and considers its impact on the weights of each neuron at the different periods after taking a meal for better prediction of future BG sequences. The encoded carb info is stored as a short memory in the NN that indicated whether additional insulin adjustment is required using a binary output layer according to the BG sequence prediction. 
%experiments on the UVA-Padova simulator and an actual dataset from xx patients 

Patients with type 1 \DIFdelbegin \DIFdel{Diabetes are tasked with the monitoring and treatment of }\DIFdelend \DIFaddbegin \DIFadd{diabetes monitor and treat }\DIFaddend their BG levels, as their pancreases do not \DIFdelbegin \DIFdel{successfully }\DIFdelend produce the required levels of insulin. 
\DIFdelbegin \DIFdel{Great strides have been made in the management of type 1 Diabetes in the last 50 years, and in particular the development of wearable insulin pumps and Continuous Glucose Monitors (CGM) \mbox{%DIFAUXCMD
\cite{pumpHist}}\hskip0pt%DIFAUXCMD
,\mbox{%DIFAUXCMD
\cite{cgmHist}}\hskip0pt%DIFAUXCMD
. }\DIFdelend %DIF > Great strides have been made in the management of type 1 Diabetes in the last 50 years, and in particular, the development of wearable insulin pumps and Continuous Glucose Monitors (CGM) \cite{pumpHist},\cite{cgmHist}. 
Recent diabetes management research largely focuses on the development of \DIFdelbegin \DIFdel{closed-loop }\DIFdelend APS that combine insulin pump and CGM technology \cite{closedSupport}. One \DIFdelbegin \DIFdel{large task at hand }\DIFdelend \DIFaddbegin \DIFadd{present task }\DIFaddend is the improvement of APS controllers \DIFdelbegin \DIFdel{\mbox{%DIFAUXCMD
\cite{study1}}\hskip0pt%DIFAUXCMD
, \mbox{%DIFAUXCMD
\cite{study2}}\hskip0pt%DIFAUXCMD
,\mbox{%DIFAUXCMD
\cite{study3} }\hskip0pt%DIFAUXCMD
}\DIFdelend \DIFaddbegin \DIFadd{\mbox{%DIFAUXCMD
\cite{study1,study2,study3} }\hskip0pt%DIFAUXCMD
or monitors \mbox{%DIFAUXCMD
\cite{zhou2021data,zhou2023hybrid,zhou2022robustness} }\hskip0pt%DIFAUXCMD
}\DIFaddend to detect and prevent \DIFdelbegin \DIFdel{cases of }\DIFdelend hyperglycemia \cite{hyper} and hypoglycemia \DIFdelbegin \DIFdel{\mbox{%DIFAUXCMD
\cite{hypo} }\hskip0pt%DIFAUXCMD
}\DIFdelend \DIFaddbegin \DIFadd{\mbox{%DIFAUXCMD
\cite{mayolow}}\hskip0pt%DIFAUXCMD
, }\DIFaddend which in mild cases can cause blurred vision, \DIFdelbegin \DIFdel{difficulty concentrating, weakness or drowsiness}\DIFdelend \DIFaddbegin \DIFadd{weakness or confusion}\DIFaddend , and in extreme cases result in \DIFdelbegin \DIFdel{a }\DIFdelend coma or even death. \DIFdelbegin %DIFDELCMD < 

%DIFDELCMD < %%%
\DIFdelend \DIFaddbegin \DIFadd{An assurance case has been developed to verify the safety of ML-enabled APS \mbox{%DIFAUXCMD
\cite{bagheri2022towards}
}\hskip0pt%DIFAUXCMD
. }\DIFaddend %Several studies have focused on addressing various lapses in this process \cite{study1}, \cite{study2},\cite{study3}. 
%One such lapse is in rebound highs/lows \cite{reboundCGM}. 
 \DIFdelbegin \DIFdel{One of the limitations of the existing APS is in addressing the }\DIFdelend \DIFaddbegin \DIFadd{Existing APS doesn't fully address }\DIFaddend rebound highs/lows \cite{reboundCGM}.   
These are typical scenarios for \DIFdelbegin \DIFdel{patients with }\DIFdelend type 1 diabetes \DIFdelbegin \DIFdel{where after having treated }\DIFdelend \DIFaddbegin \DIFadd{patients where after treating }\DIFaddend an episode of \DIFdelbegin \DIFdel{hypoglycemia, their BG levelsincrease beyond }\DIFdelend \DIFaddbegin \DIFadd{out-of-range BG levels, BG shifts through }\DIFaddend the target range \DIFdelbegin \DIFdel{into a range of hyperglycemia ("rebound high"). Similarly, it is not uncommon for patients, after having treated hyperglycemia, to fall past the safe range into unsafe hypoglycemia ("rebound low "). 
Current BG prediction models and APS can }\DIFdelend \DIFaddbegin \DIFadd{to the opposite unsafe range (i.e., low to high or vice versa). 
%DIF > Similarly, it is not uncommon for patients, after having treated hyperglycemia, to fall past the safe range into unsafe hypoglycemia ("rebound low"). 
Current APS }\DIFaddend alert users about \DIFdelbegin \DIFdel{the rebound highs/lows}\DIFdelend \DIFaddbegin \DIFadd{upcoming trends}\DIFaddend , but they neglect to take or advise precise treatment\DIFdelbegin \DIFdel{actions}\DIFdelend .
%Hypoglycemia in a mild case can cause "blurred vision, difficulty concentrating, confused thinking, slurred speech, numbness, and drowsiness" among other symptoms. If hypoglycemia persists, it can cause "seizures, coma, and very rarely death" \cite{hypo}. Hyperglycemia commonly causes frequent urination, increase thirst, blurred vision, and feeling weak or unusually tired. In extreme cases, it can cause ketoacidosis, which can result in a coma or even death \cite{hyper}.

%In this paper, we discuss two causes of concern that, when addressed, can decrease the time out of the safe BG range.
In this paper, we propose methods for the prediction and prevention of rebound \DIFdelbegin \DIFdel{hyperglycemia }\DIFdelend \DIFaddbegin \DIFadd{highs }\DIFaddend which can be integrated with \DIFdelbegin \DIFdel{the existing APS applications }\DIFdelend \DIFaddbegin \DIFadd{existing APS }\DIFaddend to address the two following \DIFdelbegin \DIFdel{rebound }\DIFdelend concerns.
%\begin{itemize}
%    \item {
\DIFdelbegin \DIFdel{The first is the concern that patients with type 1 diabetes in the safe range }\DIFdelend \DIFaddbegin \DIFadd{First, patients }\DIFaddend may take action too late to prevent BG levels outside \DIFdelbegin \DIFdel{of }\DIFdelend the safe range. 
%For instance, if the CGM is currently reading a BG level in range (e.g., 140 mg/dL), the patient may not know when to consume carbs and/or glucose if at all, even if the BG level is indicated to be decreasing. By the time the patient is made aware of BG being low, the BG is already in the state of hypoglycemia or close to it. 
To address this case, we propose a predictive alert mechanism that uses an ML algorithm to estimate BG levels an hour \DIFdelbegin \DIFdel{in advance}\DIFdelend \DIFaddbegin \DIFadd{prior}\DIFaddend , categorizes BG trends into one of the several BG scenarios \DIFdelbegin \DIFdel{, and outputs messages alerting }\DIFdelend \DIFaddbegin \DIFadd{(e.g., }\textit{\DIFadd{rebound high}}\DIFadd{), and alerts }\DIFaddend the user of both general high/low BG levels and specific rebound scenarios.
%[decide if we want treatment suggestion as well. Most diabetics would know what to do with this existing message I would hope, but we could add "consider eating some carbs" if we want] 
Since carbohydrate \DIFaddbegin \DIFadd{(carb, CHO) }\DIFaddend ingestion takes time to affect BG levels, \DIFdelbegin \DIFdel{and insulin takes even longer to do so}\DIFdelend \DIFaddbegin \DIFadd{insulin even more}\DIFaddend , earlier action can mitigate unsafe BG levels \DIFdelbegin \DIFdel{. This will not only help prevent rebounds, but can also help }\DIFdelend \DIFaddbegin \DIFadd{and help users }\DIFaddend stay within the safe range. %those in the safe range to stay within it.
%}
%\item{

\DIFdelbegin \DIFdel{The second concern is of the over-correction of }\DIFdelend \DIFaddbegin \DIFadd{Second, }\DIFaddend low BG levels \DIFdelbegin \DIFdel{. In addition to the generation of predictive BG alerts mentioned above, we }\DIFdelend \DIFaddbegin \DIFadd{may be over-corrected. We }\DIFaddend introduce a method to recommend increases in basal \DIFdelbegin \DIFdel{rate administered by the insulin pump by a threshold }\DIFdelend \DIFaddbegin \DIFadd{injections given via insulin pump }\DIFaddend when BG has shifted to the safe range, based on \DIFdelbegin \DIFdel{carbohydrate }\DIFdelend \DIFaddbegin \DIFadd{carb }\DIFaddend input entered when BG was \DIFdelbegin \DIFdel{below the safe range}\DIFdelend \DIFaddbegin \DIFadd{low}\DIFaddend . This adjusted basal output \DIFdelbegin \DIFdel{acts as a "carb carry-over" for earlier
carbohydrate inputs to account }\DIFdelend \DIFaddbegin \DIFadd{accounts for earlier
carb inputs once }\DIFaddend in the safe range, preventing a \DIFdelbegin \DIFdel{reactionary hyperglycemia episode from occurringafter experiencing hypoglycemia}\DIFdelend \DIFaddbegin \DIFadd{rebound high from occurring}\DIFaddend .
%}
%\end{itemize}
Addressing \DIFdelbegin \DIFdel{the issues of not foreseeing future BG trends and not carrying-over insulin that accounts for carbs after hypoglycemia both }\DIFdelend \DIFaddbegin \DIFadd{these two concerns }\DIFaddend help to attain in-range BG values. %for patients with type 1 diabetes.
\DIFdelbegin %DIFDELCMD < 

%DIFDELCMD < %%%
\DIFdelend Our contributions are as follows: 
\begin{enumerate}
    \item Predictive \DIFdelbegin \DIFdel{blood sugar alert mechanism with scenario-specific language }\DIFdelend \DIFaddbegin \DIFadd{alert mechanism }\DIFaddend based on an ML prediction model for BG \DIFdelbegin \DIFdel{scenario categorization }\DIFdelend \DIFaddbegin \DIFadd{scenario-specific categorization and messaging
    }\DIFaddend \item Preliminary steps towards developing a \DIFdelbegin \DIFdel{carry-over carb input method for increased basal rate }\DIFdelend \DIFaddbegin \DIFadd{method for delayed increased basal }\DIFaddend recommendations to patients for preventing hyperglycemia \DIFaddbegin \DIFadd{based on prior carb intake
    }\DIFaddend \item Evaluation using a realistic testbed's simulated \DIFdelbegin \DIFdel{data }\DIFdelend and actual patient data
\end{enumerate}

%\section{Random notes -- ignore! Thanks}
%Suspension of basal insulin to avoid hypoglycemia in type 1 diabetes treated with insulin pump - https://www.ncbi.nlm.nih.gov/pmc/articles/PMC4285755/
%\\
%Evidence That Hyperglycemia After Recovery From Hypoglycemia Worsens Endothelial Function and Increases Oxidative Stress and Inflammation in Healthy Control Subjects and Subjects With type 1 Diabetes - https://www.ncbi.nlm.nih.gov/pmc/articles/PMC3478543/
%\\
%: Ask UVA Endocrinologists about rebound high -- done
%\\Our main contributions are as follows: 
%\\1) Carry-over carb input method for hypoglycemia
%\\2) ML blood sugar prediction method
%\\3) an app to alert user to Carry-over carb and critical blood sugar predictions
%\\4) Quantitative simulation and qualitative in-person evaluation

\section{Background}
\textbf{Artificial Pancreas Systems:}
\begin{figure}
        \centering
    \includegraphics[scale=.38]{Figures/APS picture-5.png} \\
        \caption{Artificial Pancreas System}
        \label{fig:APS}
        \vspace{-1.5em}
    \end{figure}
As shown in Figure \ref{fig:APS}, APS are closed-loop systems that \DIFdelbegin \DIFdel{mimic the work of the pancreas to control blood sugar }\DIFdelend \DIFaddbegin \DIFadd{control BG }\DIFaddend levels using three devices: (i) a CGM that detects BG levels, (ii) a controller that intakes \DIFdelbegin \DIFdel{BG levels from the CGM (and at times external }\DIFdelend \DIFaddbegin \DIFadd{these BG levels (and }\DIFaddend user input, such as \DIFdelbegin \DIFdel{carbohydrate input}\DIFdelend \DIFaddbegin \DIFadd{carbs}\DIFaddend ) and determines insulin pump instructions (\DIFdelbegin \DIFdel{insulin dose)based on the inputs}\DIFdelend \DIFaddbegin \DIFadd{dose)}\DIFaddend , and (iii) an insulin pump which \DIFdelbegin \DIFdel{delivers insulin into the body based on the instructions from the controller}\DIFdelend \DIFaddbegin \DIFadd{injects instructed insulin doses}\DIFaddend . %This link between the body, CGM, controller, and insulin pump injecting back into the body causes the APS to be referred to as a \emph{Closed Loop System}. 
\DIFdelbegin \DIFdel{In this paper, we focus on the APS that allows for standard carbohydrate input. Users can also interact with the system by changing BG levels through }\DIFdelend \DIFaddbegin \DIFadd{We focus on APS that allow for standard carb input. BG levels are also impacted by }\DIFaddend temporary basal rate adjustments (e.g., for exercise), extended boluses (e.g., for \DIFdelbegin \DIFdel{fat-heavy }\DIFdelend \DIFaddbegin \DIFadd{fat-rich }\DIFaddend meals), or consuming \DIFdelbegin \DIFdel{carbohydrates }\DIFdelend \DIFaddbegin \DIFadd{carbs }\DIFaddend (e.g., \DIFdelbegin \DIFdel{in treating }\DIFdelend \DIFaddbegin \DIFadd{for }\DIFaddend low BG).

\DIFaddbegin \textbf{\DIFadd{Basal: }}
\DIFadd{Basal is an insulin injection often taken at a small, hourly rate to mimic a pancreas' gradual release to prevent high BG.
%DIF > \textbf{Bolus: }
%DIF > A bolus is an injection of insulin taken to combat high BG, oftentimes taken shortly before a meal as a measure to prevent hyperglycemia .
}

\DIFaddend \textbf{BG Safe Ranges: }
The safe range\DIFdelbegin \DIFdel{for BG levels is defined to be between }\DIFdelend \DIFaddbegin \DIFadd{, in which no negative symptoms exist, is defined between BG levels }\DIFaddend 70 and 180 $mg/dL$. \DIFdelbegin \DIFdel{BG values within this range are considered safe since they do not cause unpleasant symptoms or health risks. }\DIFdelend %that occur outside of the specified range. 
\emph{Hypoglycemia} refers to \DIFdelbegin \DIFdel{the cases in which BG levels are }\DIFdelend \DIFaddbegin \DIFadd{BG levels }\DIFaddend below 70 $mg/dL$, and \emph{hyperglycemia} refers to \DIFdelbegin \DIFdel{the cases where BG levels are }\DIFdelend \DIFaddbegin \DIFadd{BG levels }\DIFaddend above 180 $mg/dL$.

\textbf{Over-correction of Low BG: }
Over-correction of low BG occurs when consumption of \DIFdelbegin \DIFdel{carbohydrates }\DIFdelend \DIFaddbegin \DIFadd{carbs }\DIFaddend is beyond what is required to draw BG levels up from hypoglycemia to the safe range. 
%For example, if a patient with type 1 Diabetes has hypoglycemia at the start of a meal containing more carbohydrates than is needed to treat the hypoglycemia and does not carefully track BG level changes and administer insulin appropriately, an over-correction of low BG is likely to occur, resulting in hyperglycemia.
%For example, it is not uncommon for a patient with type 1 Diabetes to have a dip in BG before a meal. Yet, as the patient may receive an alert that their BG is below range soon before a meal and BG takes roughly 15-20 minutes to rise to a safe level, it is a notable case to consider the consumption of carbohydrates above what is required to raise BG from hypoglycemia (such as would be found in a typical meal). In this case, the insulin pump would be reticent to inject insulin for the whole carbohydrate intake (rightly so) as a safety precaution, since insulin lowers BG and the patient is currently in a state of hypoglycemia in this scenario. Yet, with the excessive ingestion of carbohydrates, it is common for this overcorrection to cause a spike in BG following the recovery of the safe state.
\DIFdelbegin \DIFdel{When a patient with type 1 diabetes experiences hypoglycemia, also called "low blood sugar," proper treatment involves raising }\DIFdelend \DIFaddbegin \DIFadd{In cases of hypoglycemia, treatment raises }\DIFaddend BG levels by consuming \DIFaddbegin \DIFadd{a set number of }\DIFaddend fast-acting \DIFdelbegin \DIFdel{carbohydrates (around 15-20 grams of sugary substances, such as glucose tablets or juice) and then waiting for a short period of time (typically about 15 minutes) \mbox{%DIFAUXCMD
\cite{mayolow}}\hskip0pt%DIFAUXCMD
. Afterwards, if BG levels remain below range, then the aforementioned treatment  is repeated. Once the BG is }\DIFdelend \DIFaddbegin \DIFadd{carbs in set intervals until BG is }\DIFaddend back in range \DIFdelbegin \DIFdel{, the patient consumes a portion of protein and longer-acting carbohydrates 
in order to prevent the BG from dropping back down into the low range. This dosing of carbohydrates in the case of hypoglycemia }\DIFdelend \DIFaddbegin \DIFadd{\mbox{%DIFAUXCMD
\cite{mayolow}}\hskip0pt%DIFAUXCMD
. This }\DIFaddend is less precise than a correction insulin dose for \DIFdelbegin \DIFdel{high blood sugar}\DIFdelend \DIFaddbegin \DIFadd{hyperglycemia}\DIFaddend , which is calculated \DIFdelbegin \DIFdel{based on the exact BG level
%DIF < . This makes it easier to incorrectly dose 
(although incorrect insulin dosing is also a problem \mbox{%DIFAUXCMD
\cite{smartPump}}\hskip0pt%DIFAUXCMD
)}\DIFdelend \DIFaddbegin \DIFadd{rather than set}\DIFaddend .
In addition, \DIFdelbegin \DIFdel{hypoglycemia at the start of meals can lead to the consumption of more carbohydrates than is }\DIFdelend \DIFaddbegin \DIFadd{meals often possess more carbs than are }\DIFaddend needed to treat \DIFdelbegin \DIFdel{the hypoglycemia, but }\DIFdelend \DIFaddbegin \DIFadd{hypoglycemia, and }\DIFaddend insulin dosing to \DIFdelbegin \DIFdel{counter them when in a state of hypoglycemia }\DIFdelend \DIFaddbegin \DIFadd{account for excess carbs when in this state }\DIFaddend can be risky.
%DIF > When a patient with type 1 diabetes experiences hypoglycemia, "low blood sugar," proper treatment involves raising BG levels by consuming fast-acting carbs (around 15-20 grams of sugary substances, such as glucose tablets) and then waiting for a set period of time (typically about 15 minutes) \cite{mayolow}. Afterwards, if BG levels remain below range, then the aforementioned treatment is repeated. Once BG is in range, the patient eats a portion of protein and longer-acting carbs 
%DIF > in order to prevent BG from dropping back into the low range.
%DIF > This dosing of carbs in the case of hypoglycemia is less precise than a correction insulin dose for high blood sugar, which is calculated based on the exact BG level
%DIF > . This makes it easier to incorrectly dose 
%DIF > (although incorrect insulin dosing is also a problem \cite{smartPump}). In addition, hypoglycemia at the start of meals can lead to the consumption of more carbs than is needed to treat the hypoglycemia, but insulin dosing to counter them when in a state of hypoglycemia can be risky.

\textbf{Rebound High/Low: }
\DIFdelbegin \DIFdel{When an over-correction of carbohydrates lifts the BG above the safe range rather than settling within the safe range, it is referred to as a }\DIFdelend \DIFaddbegin \DIFadd{Rebound hyperglycemia, or }\DIFaddend "\DIFdelbegin \DIFdel{rebound high.}\DIFdelend \DIFaddbegin \DIFadd{a rebound high,}\DIFaddend " \DIFdelbegin \DIFdel{Rebound hyperglycemia }\DIFdelend is defined in previous works as a "series of sensor glucose values (SGVs) >180 $mg/dL$ starting within two hours of an antecedent SGV <70 $mg/dL$" \cite{reboundCGM}. \DIFaddbegin \DIFadd{We define a rebound low as the reverse, a series of sensor glucose values (SGVs) <70 $mg/dL$ starting within two hours of an antecedent SGV <180 $mg/dL$. }\DIFaddend %In our system, 2 hours is equivalent to 24 APS timesteps.
%The term "rebound high" can become messy in the literature, as the same term is at 
\DIFdelbegin \DIFdel{The term }\DIFdelend "\DIFdelbegin \DIFdel{rebound }\DIFdelend \DIFaddbegin \DIFadd{Rebound }\DIFaddend high" \DIFdelbegin \DIFdel{is at times also used in the literature to }\DIFdelend \DIFaddbegin \DIFadd{can also }\DIFaddend refer to the Somogyi effect \cite{Somogyi}\DIFdelbegin \DIFdel{.
}\DIFdelend \DIFaddbegin \DIFadd{, 
}\DIFaddend %, which is itself often confused with the Dawn Phenomenon. The Dawn Phenomenon refers to a natural hormone release that increases in blood sugar in the morning while asleep. The Somogyi effect also occurs in the morning, but it is an increase in blood sugar that comes as a rebound from low blood sugar overnight, typically caused by natural hormone releases as an attempt to raise BG levels \cite{defDawnSomogyi}.
\DIFdelbegin \DIFdel{The Somogyi effect is }\DIFdelend an increase in blood sugar \DIFdelbegin \DIFdel{that comes }\DIFdelend in the morning as a \DIFdelbegin \DIFdel{rebound }\DIFdelend \DIFaddbegin \DIFadd{"rebound" }\DIFaddend from low blood sugar overnight, typically caused by natural hormone releases \DIFdelbegin \DIFdel{as an attempt to raise BG levels }\DIFdelend \cite{defDawnSomogyi}.
Although the Somogyi effect \emph{is} a rebound high, %However, in this study, when we refer to "rebound highs," we are not talking specifically about the rebounds that occur without intervention. We are speaking of rebound highs that occur from over-correction of hypoglycemia which then causes a spike in blood sugar.
%However, 
in this paper, we refer to "rebound highs" that occur due to over-correction of low BG\DIFdelbegin \DIFdel{causing a spike in blood sugar}\DIFdelend , and not the rebounds that occur without intervention. 

\section{Related Work}
%Type 1 Diabetes has been a target area of research for years, and thus there is a large body of related works from which we highlight the main categories that this paper builds upon.

%\begin{itemize}
%\item{
\textbf{Predictive BG Alerts:} \DIFdelbegin \DIFdel{Great strides have been made in predictive BG monitoring and notification in recent years. Some of the most popular manufacturers of medical devices for the management of }\DIFdelend \DIFaddbegin \DIFadd{In recent years, popular }\DIFaddend type 1 \DIFdelbegin \DIFdel{Diabetes }\DIFdelend \DIFaddbegin \DIFadd{diabetes management device manufacturers }\DIFaddend have rolled out predictive BG messages. \DIFdelbegin \DIFdel{For instance, }\DIFdelend Medtronic's Guardian Connect System 
\DIFdelbegin \DIFdel{reports to be able to }\DIFdelend \DIFaddbegin \DIFadd{claims to }\DIFaddend predict BG 10-60 minutes in advance and \DIFdelbegin \DIFdel{to notify the patient in order to inform preemptive treatment decisions }\DIFdelend \DIFaddbegin \DIFadd{inform the patient of preemptive treatments }\DIFaddend \cite{fdaGuardian}. 
%ML-based prediction
\DIFdelbegin \DIFdel{Recent advancement in }\DIFdelend \DIFaddbegin \DIFadd{New }\DIFaddend data-driven technologies 
\DIFdelbegin \DIFdel{has also promoted research works that can }\DIFdelend %DIF > has 
\DIFaddbegin \DIFadd{can also }\DIFaddend predict BG accurately for a maximum \DIFdelbegin \DIFdel{period }\DIFdelend of 2 hours in \DIFdelbegin \DIFdel{the future }\DIFdelend \DIFaddbegin \DIFadd{advance }\DIFaddend \cite{Taisa2020,Dutta2018}. However, these \DIFdelbegin \DIFdel{works primarily focus on BG prediction at nighttime without considering }\DIFdelend \DIFaddbegin \DIFadd{do not consider }\DIFaddend more complex scenarios, such as multiple variate meal inputs \DIFaddbegin \DIFadd{outside of night hours}\DIFaddend . %zhou2023knowsafe
\DIFdelbegin \DIFdel{In addition, they do not recognize and refer to common scenarios (such as }\DIFdelend \DIFaddbegin \DIFadd{They also lack precise treatment advice and labeled scenarios (e.g., }\DIFaddend "rebound high")\DIFdelbegin \DIFdel{beyond general hyperglycemia and hypoglycemia 
alerts and lack precise treatment advice.
}\DIFdelend \DIFaddbegin \DIFadd{.
%DIF > do not recognize and refer to common scenarios (such as "rebound high") beyond general hyperglycemia and hypoglycemia alerts and lack precise treatment advice. 
}\DIFaddend %In addition, though they seem to be fairly proficient at predictive notification, another limitation of them that we sought to address in our study was sorting various scenarios with the common language of "rebound high" or "rebound low" when on the outside of the range on either end and predicting to swap to the other. Though this sort of adjustment is small, it is helpful to progress the notifications to recognize and use language for scenarios that are common and used in the language of both endocrinologists and lay patients with type 1 diabetes. A third limitation we address is the lack of precise treatment advice. We sought to provide insulin dosage suggestions based on carbohydrate input in the case of a predicted rebound high.
%}

%\item{
\textbf{Rebound Hyperglycemia:}
\DIFdelbegin \DIFdel{A large focus }\DIFdelend \DIFaddbegin \DIFadd{Much }\DIFaddend of our work \DIFdelbegin \DIFdel{is rebound hyperglycemiacases}\DIFdelend \DIFaddbegin \DIFadd{focuses on rebound hyperglycemia}\DIFaddend . A very recent study \DIFdelbegin \DIFdel{on this topic evaluates using predictive BG alerts to prevent rebound hyperglycemia events \mbox{%DIFAUXCMD
\cite{reboundCGM}}\hskip0pt%DIFAUXCMD
. The study utilized existing predictive alerts that notify patients of predicted hyperglycemia and hypoglycemia in general. With this system in place, they evaluated the impact of the predictive systems on preventing cases of specifically rebound hyperglycemia within participants}\DIFdelend \DIFaddbegin \DIFadd{evaluates the impact of existing nonspecific predictive alerts to notify patients and prevent cases of rebound hyperglycemia \mbox{%DIFAUXCMD
\cite{reboundCGM}}\hskip0pt%DIFAUXCMD
}\DIFaddend .
However, \DIFdelbegin \DIFdel{there are similar limitations }\DIFdelend \DIFaddbegin \DIFadd{similar limitations exist }\DIFaddend in this rebound-focused work\DIFdelbegin \DIFdel{that we address}\DIFdelend .
%\begin{enumerate}
    %\item{
    \DIFdelbegin \DIFdel{First, this study also }\DIFdelend \DIFaddbegin \DIFadd{This study }\DIFaddend did not use alerts that specifically recognize \DIFdelbegin \DIFdel{the case of rebound highsand/or use that language in the predictive alerts .%DIF < }
    %DIF < \item {
    Second, alerts that simply notify the user of predicted out-of-range BG values require the user to create treatment actions without }\DIFdelend \DIFaddbegin \DIFadd{rebound highs, and the alerts did not provide }\DIFaddend direct recommendations of insulin amounts.
    %DIF < and we use them in our own study, they act as calls to action for the user to give treatment based on the alert information. This creates a limitation of relying on user action when it is not needed.
    %DIF > \item {
    %DIF > Second, alerts that simply notify the user of predicted out-of-range BG values require the user to create treatment actions without direct recommendations of insulin amounts. %and we use them in our own study, they act as calls to action for the user to give treatment based on the alert information. This creates a limitation of relying on user action when it is not needed.
    %\\There is research on giving automatic corrective insulin doses for high blood sugar \cite{smartPump}, but an automated increase in basal insulin rate or automatic extended bolus insulin dose once back in safe range based on carbs entered while in a state of hypoglycemia is largely missing from this discussion.}
    %}

    \DIFdelbegin \DIFdel{In order to address the limitationsof existing works}\DIFdelend \DIFaddbegin \DIFadd{To address such limitations}\DIFaddend , we propose a BG prediction model that \DIFdelbegin \DIFdel{produces messages by acknowledging specific 
    scenarios by name and recommending a precise increased insulin in the case of }\DIFdelend \DIFaddbegin \DIFadd{provides specific 
    scenario alerts and gives precise insulin recommendations for }\DIFaddend a rebound high.
    %DIF < \end{enumerate}
%DIF < }
%DIF < \end{itemize}

\section{Methods}
\begin{figure}[t!]
    \centering
\includegraphics[scale=.45]{Figures/Fig2RememberBolus.png} \\
    \caption{Overall Design of Rebound High Evaluation Module}
    \label{fig:reboundHighArch}
    \vspace{-1.5em}
\end{figure}
We propose to supplement the existing APS controllers with mechanisms for preventing rebound hyperglycemia events. 
As shown in Figure \ref{fig:reboundHighArch}, our proposed Rebound High Evaluation module is designed using
%A preliminary survey, 
ML-based APS state prediction, a recommended increased basal insulin to \DIFdelbegin \DIFdel{carry-over prior carbohydrate }\DIFdelend \DIFaddbegin \DIFadd{address carb }\DIFaddend input, and response alert message generation to assist patients in the detection and mitigation of rebound highs. %which can be integrated with the existing APS controllers to assist patients in 

\begin{comment}
\subsection{Preliminary Survey}A preliminary survey was sent to ~10 participants to gain a general understanding of hypoglycemia treatment and rebound highs in the experience of actual
patients. The survey involved three demographic questions and eight questions about responding to and predicting out-of-range BG levels.
\end{comment}

\subsection{ML Model for Blood Sugar Prediction}\label{workflowdata}
\begin{figure}[b!]
    \vspace{-2em}
    \centering
\includegraphics[scale=.2]{Figures/FigEncodeDecode-6.png} \\
    \caption{ML Model for Blood Sugar Prediction}
    \label{fig:encodeDecode}
\end{figure}
\DIFaddbegin 

\DIFaddend To forecast future BG levels and detect \DIFdelbegin \DIFdel{future BG }\DIFdelend trends, we design an encoder-decoder ML model with an attention layer \DIFdelbegin \DIFdel{that takes in a sequence of APS states , where each }\DIFdelend \DIFaddbegin \DIFadd{given APS states as input that outputs expected BG values (see Figure \ref{fig:encodeDecode}). Each }\DIFaddend state contains BG, insulin dosage, insulin on board (IOB), and \DIFdelbegin \DIFdel{carbohydrates, and outputs expected BG values (see Figure \ref{fig:encodeDecode})}\DIFdelend \DIFaddbegin \DIFadd{carbs}\DIFaddend . The encoder generates a deep representation of the sequence of input states that the decoder uses to initialize its internal state as it predicts the BG value for the next timestep using the current APS state.
\DIFaddbegin 

\DIFadd{We chose this architecture due to its success in sequence generation tasks \mbox{%DIFAUXCMD
\cite{hochreiter1997, BahdanauAttn}}\hskip0pt%DIFAUXCMD
.
Unlike other blood glucose prediction models that give a single estimate of BG at a fixed prediction horizon (e.g. 30 or 60 minutes), the encoder-decoder model generates a full sequence of BG predictions up to and including the prediction horizon where each previous prediction is factored into the next prediction autoregressively. This allows for the detection of the earliest possible occurrence of an adverse event, such as a rebound high. 
}

\DIFadd{The attention layer has also been shown to help the decoder process the information learned by the encoder during prediction by amplifying relevant features, suppressing unimportant ones, and improving performance of encoder-decoder models in a variety of application scenarios \mbox{%DIFAUXCMD
\cite{BahdanauAttn, Zhu2022, Muralidhar2019}}\hskip0pt%DIFAUXCMD
. Finally, it followed to use a recurrent neural network as the underlying layer in the encoder and decoder because the APS data has temporal order, so we chose the LSTM to retain past information over longer input sequences.
}



\DIFaddend The following equations describe the flow of information through the model:
%DIF < \vspace{-1em}
%DIF < {%\scriptsize
\DIFdelbegin \begin{align*}
    \DIFdel{h_{enc} }&\DIFdel{= Encoder(X_{[t-n+1, t]}) }\\
    \DIFdel{s_{t'+1} }&\DIFdel{= (h_{t'+1}, z_{t'+1}) = LSTM(x_{t'}, s_{t'}, c_{t'+1}) }\\
    \DIFdel{c_{t'+1} }&\DIFdel{= A \cdot h_{enc} }\\
    \DIFdel{A }&\DIFdel{= \text{softmax}\left( w_v \cdot \text{tanh} ( W_q h_{t'} + W_k h_{enc} ) \right) }\\
    \DIFdel{\widehat{bg}_{t'+1} }&\DIFdel{= f(h_{t'+1})
}\end{align*}%DIFAUXCMD
\DIFdelend \DIFaddbegin \begin{align}
    \DIFadd{h_{enc} }&\DIFadd{= Encoder(X_{[t-n+1, t]}) }\\
    \DIFadd{s_{t'+1} }&\DIFadd{= (h_{t'+1}, z_{t'+1}) = LSTM(x_{t'}, s_{t'}, c_{t'+1}) }\\
    \DIFadd{c_{t'+1} }&\DIFadd{= A \cdot h_{enc} }\\
    \DIFadd{A }&\DIFadd{= \text{softmax}\left( w_v \cdot \text{tanh} ( W_q h_{t'} + W_k h_{enc} ) \right) }\\
    \DIFadd{\widehat{bg}_{t'+1} }&\DIFadd{= f(h_{t'+1})
}\end{align}\DIFaddend 
%}%
\noindent where $X_{[t-n+1, t]}$ is the past $n$ APS states, and $h_{enc}$ represents the input learned by the encoder for all timesteps. In the decoder, \DIFaddbegin \DIFadd{described by the final four equations (2)-(5), }\DIFaddend $x_{t'}$ is the projected APS state for future time $t'$, $s_{t'}$ is the decoder's internal state \DIFdelbegin \DIFdel{composed of }\DIFdelend \DIFaddbegin \DIFadd{with }\DIFaddend a hidden state $h_{t'}$ and a cell state $z_{t'}$, and $c_{t'}$ is the context vector generated by the attention layer using the attention weight matrix $A$. The decoder's hidden state $h_{t'}$ is passed through a feedforward neural network, $f$, to generate the predicted BG value, $\widehat{bg}$. The decoder calculates $s_{t'}$ for $t' \in [t+1, t+m]$, where $m$ is the length of BG predictions. $W_q$, $W_k$, and $w_v$ are weights tuned in the attention layer.

The attention layer within the decoder calculates how much the decoder should weigh each part of the input data (i.e.\DIFaddbegin \DIFadd{, }\DIFaddend how much attention it should give) and uses a feedforward neural network to generate attention weights \cite{BahdanauAttn}. \DIFdelbegin \DIFdel{In order to }\DIFdelend \DIFaddbegin \DIFadd{To }\DIFaddend help the model learn the delayed action of \DIFdelbegin \DIFdel{carbohydrates on BG}\DIFdelend \DIFaddbegin \DIFadd{carbs on BG, }\DIFaddend which needs 15-30 minutes to start impacting BG \DIFdelbegin \DIFdel{\mbox{%DIFAUXCMD
\cite{30min}}\hskip0pt%DIFAUXCMD
}\DIFdelend \DIFaddbegin \DIFadd{\mbox{%DIFAUXCMD
\cite{30min, freeman2009}}\hskip0pt%DIFAUXCMD
}\DIFaddend , we customize the attention layer to increase the attention weight for input states 30 minutes after a \DIFdelbegin \DIFdel{carbohydrate }\DIFdelend \DIFaddbegin \DIFadd{carb }\DIFaddend input by 10\%. \DIFaddbegin \DIFadd{Forcing the model to focus on these states should help it learn  to account for the physiological delay that the meal carbs take to affect BG. 
}\DIFaddend %to focus the model's attention on more physiologically relevant parts of the input.   
\subsection{Recommended \DIFaddbegin \DIFadd{Increased }\DIFaddend Basal \DIFdelbegin \DIFdel{Increase }\DIFdelend for Carb Carry-over}
We propose a method to find \DIFdelbegin \DIFdel{a basal rate percent increase threshold for when carbohydrates }\DIFdelend \DIFaddbegin \DIFadd{basal increase when carbs }\DIFaddend are entered into the APS \DIFdelbegin \DIFdel{controller by the patient while 
BG is below the safe range}\DIFdelend \DIFaddbegin \DIFadd{while 
hypoglycemic}\DIFaddend . We calculate an \DIFdelbegin \DIFdel{increase }\DIFdelend \DIFaddbegin \DIFadd{increased }\DIFaddend basal (IB) using the existing \DIFdelbegin \DIFdel{carbohydrate }\DIFdelend \DIFaddbegin \DIFadd{carb }\DIFaddend input data in the Basal-Bolus controller (typically used for meal bolus generation). We then recommend the IB to be applied once the BG level has returned to a safe range.

% There exists an increase bolus such that if the increase bolus is injected (given carbohydrates?), then from current time until 24 timesteps (2 hours) from now, BG will be in range
Specifically, our goal is to find a new, corrected basal \DIFdelbegin \DIFdel{rate }\DIFdelend $IB$ that is $\delta$ units of insulin greater than the current basal rate, $r_t$, for time $t$ such that all predicted BG values \DIFdelbegin \DIFdel{within the hour $BG_{[t,t+12]}$ }\DIFdelend \DIFaddbegin \DIFadd{once the target range is entered, $t_{tr}$, }\DIFaddend are in range \DIFdelbegin \DIFdel{: 
}\DIFdelend \DIFaddbegin \DIFadd{within the next hour $BG_{[t_{tr},t_{tr}+12]}$: 
}\DIFaddend \begin{equation}
    IB = r_t + \delta \Rightarrow 70 < BG\DIFdelbegin \DIFdel{_{[t,t+12]} }\DIFdelend \DIFaddbegin \DIFadd{_{[t_{tr},t+12-t_{tr}]} }\DIFaddend < 180
    % \exists {IB} \ni Inj(IB) \rightarrow (i = t, ..., t+24 | 70 < x_i < 180)
\end{equation}

%As a simple solution, we can use fixed $\delta$s (0.0, 0.2, 0.4, ... 3.8, 4.0). 
\DIFdelbegin \DIFdel{We set an optimization problem that minimizes timesteps outside of the BG safe range by running in the prediction model using a set of fixed  $\delta$s to find an optimal $\delta$ among them. We then add this $\delta$ value to the normal basal rate to find an $IB$ to be suggested in the alert message.
}\DIFdelend %DIF > We set an optimization problem that minimizes timesteps outside of the BG safe range by running in the prediction model using a set of fixed  $\delta$s to find an optimal among them. We then add this $\delta$ to the normal basal rate to find an $IB$ to be suggested in the alert.
%Where $Inj()$ is to inject, $i$ is an APS timestep, $t$ is current time, and $x_t$ is predicted BG at time $t$
\DIFdelbegin \DIFdel{Solving this optimization problem }\DIFdelend \DIFaddbegin \DIFadd{Optimizing }\DIFaddend for the most accurate IB is the subject of future work and beyond the scope of this paper. For this paper, we used domain knowledge and heuristics to add some fixed $\delta$ values for \DIFdelbegin \DIFdel{the ideal value }\DIFdelend \DIFaddbegin \DIFadd{an ideal }\DIFaddend to be found among them \DIFaddbegin \DIFadd{via optimization that minimizes timesteps outside of the BG safe range in predictions}\DIFaddend . 
%Future work will implement an optimization of IB.
%A solution to this problem involves... which is the subject of future work and beyond the scope of this paper.
\subsection{Rebound High Evaluation and Alert Generation}
\begin{table*}[]
    \centering
    \caption{Rebound High Alert System Performance Metrics}
    \vspace{-0.5em}
    \begin{tabular}{c|c|c}
        &\textbf{BG Scenario} & \textbf{Alert Message} \\
        \hline
        1 & $((x_{p} > 180)\wedge(t=p-24,...,p|x_t \not < 70)$ & "Your blood sugar is predicted to increase to 180 mg/dL in the next hour.” \\
        2 & $((x_{p} > 180)\wedge(t=p-24,...,p|x_t < 70)$ & “You are predicted to have a rebound high in the next hour.  \DIFdelbeginFL \DIFdelFL{Increase basal }\DIFdelendFL \\
        &&  \DIFdelbeginFL \DIFdelFL{rate to }\DIFdelendFL \DIFaddbeginFL \DIFaddFL{Give a temp basal of }\DIFaddendFL $IB$ \DIFdelbeginFL \DIFdelFL{for a period }\DIFdelendFL once you are back in range.”  \\
        3 & $((x_{p} < 70)\wedge(t=p-24,...,p|x_t \not > 180)$ & “Your blood sugar is predicted to decrease to 70 $mg/dL$ in the next hour.”  \\
        4 & $((x_{p} < 70)\wedge(t= p-24,...,p|x_t > 180)$ & "You are \DIFdelbeginFL \DIFdelFL{expected }\DIFdelendFL \DIFaddbeginFL \DIFaddFL{predicted }\DIFaddendFL to have a rebound low in the next hour. \DIFdelbeginFL \DIFdelFL{Consider suspending }%DIFDELCMD < \\
%DIFDELCMD <         && %%%
\DIFdelFL{insulin for a period}\DIFdelendFL \DIFaddbeginFL \DIFaddFL{Suspend insulin temporarily}\DIFaddendFL ." \\
        5 & else & - no message - \\
        \vspace{1em}
    \end{tabular}

    $x_t$ represents predicted BG level at a given timestep $t$, $p$ is the most up-to-date predicted timestep, $IB$ is the calculated Increased Basal value.
    \label{tab:alert_messages}
    \vspace{-1em}
\end{table*}
% how model is used for rebound high detection
The \DIFaddbegin \DIFadd{prediction model's }\DIFaddend forecasted BG sequence \DIFdelbegin \DIFdel{from the prediction model }\DIFdelend can be used to detect hyperglycemia, hypoglycemia, and rebound events in advance. Predicted \DIFdelbegin \DIFdel{BG }\DIFdelend and past BG states are categorized to output appropriate alert messages as shown in Table \ref{tab:alert_messages}.
If BG is predicted to be \DIFdelbegin \DIFdel{above range without having been below range in the past hour (row }\DIFdelend \DIFaddbegin \DIFadd{out of range without being a rebound (rows }\DIFaddend 1\DIFdelbegin \DIFdel{), or BG is predicted to be below range without having been above range in the past hour (row }\DIFdelend \DIFaddbegin \DIFadd{, }\DIFaddend 3), then general alerts \DIFdelbegin \DIFdel{such as those in the existing models \mbox{%DIFAUXCMD
\cite{fdaGuardian} }\hskip0pt%DIFAUXCMD
}\DIFdelend are issued.
%If the model is set to make BG predictions for the next hour, a rebound high can be anticipated up to 1 hour beforehand by checking whether any of the expected BG levels are hyperglycemic when after a recent hypoglycemic BG value.
%In the case that a BG is predicted to be out of range (70 mg/dL to 180 mg/dL), a message will be generated to notify the patient of the predicted BG value and the estimated time, as seen in \ref{tab:alert_messages}. A typical prediction horizon 12 timesteps from the current state, where each time step is 5 minutes, resulting in prediction up to 60 minutes before the occurrence of an adverse event. This means that we can generate alerts to warn users to take preventative action well before their BG levels become abnormal. Messages have the format of “Your blood sugar is predicted to (increase/decrease) to (70 $mg/dL$/180 $mg/dL$) in 60 minutes.” 
%Similar message of general high/low predictions, as output by our classifying ML method, exist in current models \cite{fdaGuardian}
%However, in addition to these notifications of general high and low BG alerts (as is commonplace), we have included messages more tailored to the situations of rebound highs and lows. For instance, the message "You are predicted to have a rebound high in the next hour. Please increase basal rate by XXx\% for additional carbs once you are back in range." followed by a message once back in range reminding the patient to increase basal by a suggested amount for excess carbohydrates.
If \DIFdelbegin \DIFdel{BG is predicted to be }\DIFdelend \DIFaddbegin \DIFadd{predicted BG is }\DIFaddend high within two hours of being \DIFdelbegin \DIFdel{below range }\DIFdelend \DIFaddbegin \DIFadd{low }\DIFaddend (row 2), then a rebound high alert with recommended IB is issued.
If \DIFdelbegin \DIFdel{BG is predicted to be }\DIFdelend \DIFaddbegin \DIFadd{predicted BG is }\DIFaddend low within two hours of being \DIFdelbegin \DIFdel{above range }\DIFdelend \DIFaddbegin \DIFadd{high }\DIFaddend (row 4), then a rebound low alert with recommended basal suspension is issued.
These messages include scenario-specific language for the rebound cases \DIFdelbegin \DIFdel{, and they provide specific action }\DIFdelend \DIFaddbegin \DIFadd{and provide specific }\DIFaddend suggestions.


% we may need to formally define the problem and ML structure as we did in the KnowsSafe paper
%\textbf{Problem Statement:}
We specifically define the following contextual conditional event to trigger an alert for a rebound high:
%over a combined of input BG trace $BG_{[t-n+1, t]}$ and output BG trace $BG_{[t+1, t+m]}$: 

\vspace{-1em}
% \begin{align}
{%\scriptsize
%Given that current BG is below 70, the probablility of a rebound high is the probablility that predicted BG will be above 180 within 24 steps of the low
%P(RH|x_{c} > 180)= P((x_{t} < 70 | (c - 24) < t < c))
%\begin{equation}    
%P(RH|x_c < 70) = P(x_t > 180 | t = c,...,c+24)
%\end{equation}
}%

\begin{equation}    
%y_t = P(\exists k \in [0. 24]: x_{t_{c}+24-k}>24 |x_{t_{c}-k}<70)
% \Scale[0.8]{y_t = P(\exists l, h \in [t-12,t+12]: l < h \hspace{0.5mm} \wedge \hspace{0.5mm} BG_{l}<70 \hspace{0.5mm} \wedge \hspace{0.5mm} BG_{h}>180)}
%\Scale[0.9]{y_t = P(\exists t \ni t-12<l<h<t+12: BG_{l}<70 \wedge BG_{h}>180)}
\DIFdelbegin %DIFDELCMD < \Scale[0.9]{y_t = P(\exists t-12<l<h<t+12: BG_{l}<70 \wedge BG_{h}>180)}
%DIFDELCMD < %%%
\DIFdelend \DIFaddbegin \Scale[0.99]{y_t = P(\exists t-12<l\leq t < h\leq t+12: BG_{l}<70 \wedge BG_{h}>180)}
\DIFaddend %
% {y_t = P(\exists l, h \in [0,12]: BG_{t-l}<70 \hspace{0.5mm} \wedge \hspace{0.5mm} BG_{t+h}>180)}
\end{equation}
%\begin{equation}    
% RH \Leftrightarrow \exists t_l, t_h \in [t-n+1, t+m] \text{s.t.} t_l < t_u \wedge BG_{t_u} < 70 \wedge BG_{t_h} > 180
%y_t = P(\exists l, h \in [0,24]: BG_{u}>180 |BG_{l}<70)
%\end{equation}
% \begin{equation}
% \label{eq:ml-model}
% \begin{split}
%     % x_{t} &= (x_{1}(t),x_{2}(t),...,x_{m}(t))^T \\ 
%     x_{t} &= (x_{1_{t}},x_{2_{t}},...,x_{n_{t}}) \\
%    % 
%     y_{t} &= p(\exists t'\in [t,t_{e}]: x_{t'} \in {\mathcal{X}}_{h}|x_{t},u_{t})
% \end{split}
% \end{equation}
\noindent where $t$ is the current time and $l$, $h$ are APS timesteps. An alert triggers \DIFdelbegin \DIFdel{whenever }\DIFdelend \DIFaddbegin \DIFadd{when }\DIFaddend $y_t \ge 0.5$. $BG$ is the sequence of \DIFdelbegin \DIFdel{blood glucose }\DIFdelend \DIFaddbegin \DIFadd{BG }\DIFaddend values over 2 hours: (1) observed \DIFdelbegin \DIFdel{BG }\DIFdelend values within the previous hour and (2) predicted \DIFdelbegin \DIFdel{BG }\DIFdelend values for the next hour from the prediction model. This allows for the detection of rebound highs up to 1 hour in advance.
%The prediction model takes in the first hour of BG values as input and predicts the next hour of BG values as output, allowing for detection of rebound highs up to 1 hour in advance.
%\noindent Where $k$ and $t$ are APS timesteps, $t_c$ is current time, $BG$ indicates measured BG, and $x$ represents predicted BG at a given time
%where $RH$ is a rebound high event, $c$ represents current time, $t$ indicates timesteps, $BG$ indicated measured BG, and $x$ represents predicted BG to detect rebound highs.



% we also need to define rebound high/low more apparently as well as its labelling method.
% how we label rebound highs
%A rebound hyperglycemia occurs in when the BG level exceeds 180 mg/dL after being below 70 mg/dL within 2 hours (24 APS timesteps) prior. 

%  I think we could have an overall struct figure of our design in the beginning and then present each section in the similar workflow in the graph.

%\subsection{Predictive Critical BG Alerts:} moved above
\begin{comment}
{\emph{Carb Entry Checker:} 
    %In the "Any clarification/notes about early questions/responses?" the only note left was "Not the best diabetic at remembering to dose."
    For a patient to manage their type 1 diabetes well, it is critical for them to give insulin doses for the amount of carbohydrates they consume. Yet, the failures in dosing appropriately have been widely reported \cite{doseprob1, doseprob3}, with the general conclusion that "Adherence to insulin therapy is generally poor" \cite{doseprob2}. Though patients can set reminders to dose on devices such as phones, with varying mealtimes this is not entirely effective. There are two options for carb reminders in this proposed application:
    \\\begin{enumerate}
    \item {A very general and simplistic method can exist to generate warning messages when blood sugar readings from the CGM increase have increased more than 10 mg/dL within 4 CGM time steps (20 minutes). Of course, this would be an unnecessary message in the case of raising blood sugar apart from a meal, such as routinely occurs during exercise.
    \item A more precise measurements can be found in state-of-the-art meal detection studies. For example, in an earlier work, a simulation-based explanation of meal detection is proposed which is able to detect a meal an average of 27 minutes into the meal within 1 g of carbohydrates consumed \cite{mealDetect}. In order to add this to an app it would have to be partnered with one of these algorithms, made similarly from scratch, or found as an open-source version.
    \\The message at a time when a meal has been detected would be structured like this: "Your blood sugar is increasing rapidly. Please check that you have entered all carbs." If using a more precise algorithm, it could even recommend a predicted carb count, or even inject an insulin dose automatically. This alert would act as a safety net for missing insulin doses.
    %\\It is very pertinent to the discussion surrounding the prevention of rebound hyperglycemia. As in the survey, all participants indicated that when they noticed low blood sugar at the start of the meal they would eat immediately after treating with glucose, and 66.7\% indicated that they typically waited until they were back in range to give insulin for their meal, combined with the facts that the main CGMs on the market today do not notify the patient when their blood sugar is back in range and there has been evidence of missing insulin doses within the treatment of type 1 diabetes among patients, a reminder alert to give insulin for carbs consumed once BG values are increasing could clearly decrease the number of times that a patient forgets to give insulin for an over-correction of low BG.
    }
    \end{enumerate}}
\emph{General Carb entry reminders:} This element of the app is the least complex, but in the same vein of it being critical for patients with type 1 diabetes to have carbs accounted for in their systems, we envision the app having regular reminds several times in a day at common meal (and potentially snack) times.

This feature existing within the same app as the carb entry checker makes sense since they are so closely related, and it can prevent hyperglycemia through patients having support to build carb-entering habits.
%\end{itemize}
\end{comment}


\section{Experimental Evaluation}
\subsection{Datasets}
\begin{figure}
        \centering
    \includegraphics[scale=.45]{Figures/FigEvalGenRemember.png} \\
        \caption{Experimental Evaluation with Realistic Testbeds and Datasets}
        \label{eval}
        \vspace{-2em}
    \end{figure}
As shown in Fig. \ref{eval}, %our closed-loop APS testbed uses the Basal-Bolus controller with UVA-Padova Type I Diabetes Simulation. 
we use three different datasets for developing and evaluating our models, including (i) simulation data generated from a closed-loop APS testbed, (ii) a dataset collected from diabetes patients in a clinical trial, and (iii) APS data collected from one patient with type 1 diabetes. 

% Simulation data description
\textbf{Simulation Data:} The first dataset we used was generated using the UVA/Padova Type I Diabetes Simulator integrated with an open-source APS controller (Basal-Bolus) \DIFdelbegin \DIFdel{\mbox{%DIFAUXCMD
\cite{zhou2022design}}\hskip0pt%DIFAUXCMD
}\DIFdelend \DIFaddbegin \DIFadd{\mbox{%DIFAUXCMD
\cite{zhou2022design,zhou2021data}}\hskip0pt%DIFAUXCMD
}\DIFaddend . 
The simulator uses an Ordinary Differential \DIFdelbegin \DIFdel{Equation (ODE)-based }\DIFdelend \DIFaddbegin \DIFadd{Equation-based }\DIFaddend patient model that updates using insulin received from the pump and pre-scheduled meals. The pump insulin is determined by the controller, which sets basal rates based on the virtual patient's profile and assigns boluses after a meal. CGM readings are taken from the true patient subcutaneous BG with simulated noise. This simulates a closed-loop APS system, and meals can be scheduled to occur at any simulation time\DIFdelbegin \DIFdel{\ref{eval}}\DIFdelend .

For 10 adult virtual patients, we chose 6 initial BG values each from 80 to 180 and ran 75 simulations at each initial value. In each simulation, the data interacts with the controller for 145 timesteps (12 hours + initial state)\DIFdelbegin \DIFdel{, with each timestep in the simulation representing }\DIFdelend \DIFaddbegin \DIFadd{. Each timestep represented }\DIFaddend 5 minutes in the actual APS. Each simulation had a meal of a random size between 30 and 100 grams of \DIFdelbegin \DIFdel{carbohydrates }\DIFdelend \DIFaddbegin \DIFadd{carbs }\DIFaddend after the first hour. This yielded 65,250 samples per patient \DIFdelbegin \DIFdel{, }\DIFdelend or roughly 7.5 months of data.


% Clinical trail data description
\textbf{Clinical Trial Data:} We also used DCLP3, a publicly-available clinical trial dataset with \DIFdelbegin \DIFdel{6 months of }\DIFdelend CGM and insulin pump data for each participant \cite{DCLP3-dataset}. We chose 5 patients with at least 6 months of data\DIFdelbegin \DIFdel{to use}\DIFdelend .

% Chloe's data
\textbf{Participant Data:} We also used 5 days of APS data (BG and insulin) retrieved from a patient with type 1 diabetes participating in our research. Because this is not enough data to train the model alone, we used transfer learning with testbed simulation data to supplement model learning.
%by training the model on data from 5 different virtual patients from the UVA/Padova simulator, then fine tuning the model using part of the participant data, and finally evaluating the model on the rest of the participant data.
\DIFdelbegin %DIFDELCMD < 

%DIFDELCMD < %%%
\DIFdelend \DIFaddbegin {\bfseries\small \Text\par}
\DIFaddend \subsection{Labeling and Model Training}
\DIFaddbegin \DIFadd{During evaluation, the encoder was composed of 1 LSTM layer and the decoder was composed of the attention layer, 1 LSTM layer, and 2 dense layers (both with linear activation). Each LSTM layer and the attention layer had hidden dimension of 64, the first dense layer had hidden dimension of 32, and the final had 1. Training was conducted used the Adam optimizer with learning rate 1e-3 for 10 epochs for Simulation data and 20 epochs for DCLP3 data.
}

\DIFaddend For the Simulation and DCLP3 Clinical Trial datasets, we split the data for each patient into train and test sets using an 80/20 split. For participant data, we used the same 80/20 split, but we first trained the model on data from 5 different virtual patients from the UVA/Padova simulator and then fine tuned the model with a low learning rate \DIFaddbegin \DIFadd{of 1e-5 }\DIFaddend to the participant train data. The model was trained using teacher forcing and mean squared error (MSE) loss. 

Because \DIFdelbegin \DIFdel{carbohydrate }\DIFdelend \DIFaddbegin \DIFadd{carb }\DIFaddend inputs from meals are recorded as one large input regardless of consumption rate, we assumed a casual eating rate of 5 CHO g/min and split the \DIFdelbegin \DIFdel{carbohydrates }\DIFdelend \DIFaddbegin \DIFadd{carbs }\DIFaddend from the meal across multiple timesteps. For example, if a meal of 60 g of CHO was recorded \DIFdelbegin \DIFdel{a }\DIFdelend \DIFaddbegin \DIFadd{at }\DIFaddend timestep $t$, we split this into 25 g at $t$, 25 g at $t+1$, and 10 g at $t+2$. This eating rate is an approximation across different patients and different food types and is meant to help the model learn by increasing the number of inputs where the CHO value is non-zero.

To determine frequency of rebound highs in each dataset, we traversed \DIFdelbegin \DIFdel{through the dataset sorted by time}\DIFdelend \DIFaddbegin \DIFadd{the dataset chronologically}\DIFaddend ,  keeping track of the most recent BG level below 70 $mg/dL$. For each CGM measurement above 180 $mg/dL$ \DIFdelbegin \DIFdel{, we checked whether it was }\DIFdelend \DIFaddbegin \DIFadd{and }\DIFaddend within 2 hours (24 timesteps) of the most recent BG sample below 70 $mg/dL$, \DIFdelbegin \DIFdel{and if it was, it was labeled as }\DIFdelend \DIFaddbegin \DIFadd{we labeled }\DIFaddend a rebound high and \DIFaddbegin \DIFadd{reset }\DIFaddend the <70 sample \DIFdelbegin \DIFdel{was reset }\DIFdelend to prevent double counting.

The model monitor generates a rebound high alert whenever a \DIFdelbegin \DIFdel{high BG level follows a low BG level }\DIFdelend \DIFaddbegin \DIFadd{hyperglycemia follows hypoglycemia }\DIFaddend in the combined sequence of input and predicted BG values. To test the \DIFdelbegin \DIFdel{ability of the modelto generate an alert when necessary}\DIFdelend \DIFaddbegin \DIFadd{model's ability to generate
fitting alerts}\DIFaddend , we used the test data from the regression task. If the combined sequence of input and true BG values contained a rebound high, then the combined sequence of input and predicted BG values should also contain a rebound high. Otherwise, the alert would not be triggered. Therefore, we could use the regression test data to evaluate the accuracy of rebound high alerts as well. 

\subsection{Results}
% results
The accuracy of \DIFaddbegin \DIFadd{the }\DIFaddend BG regression model on each patient is shown in Table \ref{reg_results}. In general, the root mean squared error (RMSE) of the model is higher on clinical trial patient data than simulated data \DIFdelbegin \DIFdel{because the clinical trial data suffers from }\DIFdelend \DIFaddbegin \DIFadd{due to }\DIFaddend missing records of 
\DIFdelbegin \DIFdel{carbohydrate meals}\DIFdelend \DIFaddbegin \DIFadd{carbs}\DIFaddend , time inconsistencies, and minor physiological effects \DIFdelbegin \DIFdel{not accounted for in the simulation's patient model. During the simulation of virtual patients, we could record both CGM measurements and the patient's actual subcutaneous BG levels that the CGM samples, allowing us to determine }\DIFdelend \DIFaddbegin \DIFadd{absent from the simulation. We could determine virtual patient }\DIFaddend simulation sensor error by calculating the RMSE between \DIFaddbegin \DIFadd{both the }\DIFaddend CGM measurements and \DIFaddbegin \DIFadd{true }\DIFaddend subcutaneous BG levels. \DIFdelbegin \DIFdel{This CGM RMSE provides a target for model performance, because }\DIFdelend \DIFaddbegin \DIFadd{Since }\DIFaddend the model learns from CGM data and cannot reduce \DIFdelbegin \DIFdel{this }\DIFdelend \DIFaddbegin \DIFadd{the }\DIFaddend random noise on its own\DIFaddbegin \DIFadd{, the CGM RMSE provides a target for model performance}\DIFaddend . We do not have a similar estimate for clinical trial data because we do not know the patients' true internal states. % Prediction accuracy for the BG Regression is shown in Table \ref{reg_results}.

\DIFaddbegin \DIFadd{In Table \ref{tab:comp_results}, we compare our approach with a carb focusing attention layer to the same model without carb focusing (i.e., regular attention) and the same without attention (i.e., encoder outputs directly to the decoder LSTM) for a subset of 5 virtual patients and DCLP3 patients each. Including the attention layer improved the model's RMSE on test data for all but one DCLP3 patient. Incorporating carb focusing improved model performance in all but two patients (virtual 2 and DCLP3 6). This may indicate their bodies process carbs quickly, meaning attention should not be delayed to states after 30 minutes from the meal start. We also evaluated the model performance on participant data without transfer learning and measured an RMSE of 35.609, higher than transfer learning's 31.213.
}

\DIFaddend % \subsection{Simulation Evaluation}
\DIFdelbegin \DIFdel{The number of rebound high events in eachdataset is shown in Figure \ref{fig:rebound_high_hist}. }\DIFdelend %DIF > The number of rebound high events in each dataset is shown in Figure \ref{fig:rebound_high_hist}. 
Rebound highs are much more frequent in real patient data, as all but one DCLP3 patient had over 90 rebound highs in 6 months and the participant had 3 in 5 days, while simulated patients only had 3-12 in 225 days\DIFdelbegin \DIFdel{of simulation}\DIFdelend . 
\begin{table}[t!]
    %\vspace{-2em}
    \centering
    \caption{Prediction Accuracy for BG Regression}
    \begin{tabular}{c c c c c c c}
         \multicolumn{2}{c}{\textbf{Patient ID}} & \textbf{RMSE} & \textbf{CGM RMSE} & \multicolumn{2}{c}{\textbf{Patient ID}} & \textbf{RMSE} \\
        \hline

        \parbox[t]{10mm}{\multirow{10}{*}{UVA Sim.}} & 1 & 11.808 & 11.011 & \parbox[t]{10mm}{\multirow{5}{*}{DCLP3}} & 3 & 27.750\\
         & 2 & 13.539 & 11.061 && 4 & 28.546\\
         & 3 & 14.166 & 11.148 && 5 & 17.988\\
         & 4 & 12.043 & 11.105 && 6 & 32.718\\
         & 5 & 14.010 & 11.114 && 7 & 38.050\\ \cline{5-7}
         & 6 & 12.342 & 11.027 & Participant & P & 31.213\\
         & 7 & 13.338 & 11.020 \\
         & 8 & 11.875 & 10.982 \\
         & 9 & 12.390 & 11.039 \\
    \end{tabular}
    \label{tab:reg_results}
    \label{reg_results}
    \vspace{-1.5em}
\end{table}
\begin{comment}
\begin{table}[t!]
    %\vspace{-2em}
    \centering
    \caption{Prediction Accuracy for BG Regression}
    \begin{tabular}{c c c c}
         \multicolumn{2}{c}{\textbf{Patient ID}} & \textbf{RMSE} & \textbf{CGM RMSE} \\
        \hline
        \parbox[t]{10mm}{\multirow{10}{*}{UVA Sim.}} & 1 & 11.808 & 11.011 \\
         & 2 & 13.539 & 11.061 \\
         & 3 & 14.166 & 11.148 \\
         & 4 & 12.043 & 11.105 \\
         & 5 & 14.010 & 11.114 \\ 
         & 6 & 12.342 & 11.027 \\
         & 7 & 13.338 & 11.020 \\
         & 8 & 11.875 & 10.982 \\
         & 9 & 12.390 & 11.039 \\
         & 10 & 14.748 & 11.129 \vspace{1mm}\\ \hline
         \parbox[t]{10mm}{\multirow{5}{*}{DCLP3}} & 3 & 27.750 & - \\
         & 4 & 28.546 & - \\
         & 5 & 17.988 & - \\
         & 6 & 32.718 & - \\
         & 7 & 38.050 & - \\ \hline
         Participant & P & 31.213 & - 
    \end{tabular}
    \label{tab:reg_results}
    \label{reg_results}
    \vspace{-1.5em}
\end{table}
\end{comment}
\DIFdelbegin \DIFdel{Because the DCLP3 patients had far more rebound high events}\DIFdelend \DIFaddbegin \DIFadd{For this reason}\DIFaddend , we use \DIFdelbegin \DIFdel{these }\DIFdelend \DIFaddbegin \DIFadd{the }\DIFaddend patients to evaluate the rebound high alert system, as shown in Table \ref{tab:alert_metrics}. One \DIFdelbegin \DIFdel{of the }\DIFdelend DCLP3 \DIFdelbegin \DIFdel{patients (number }\DIFdelend \DIFaddbegin \DIFadd{patient, }\DIFaddend 5\DIFdelbegin \DIFdel{) did not have the same high number of rebound highs as the others, and ended up having }\DIFdelend \DIFaddbegin \DIFadd{, had }\DIFaddend no rebound highs in the test data\DIFdelbegin \DIFdel{to trigger an alert, and was therefore }\DIFdelend \DIFaddbegin \DIFadd{, so it was }\DIFaddend omitted. The model performed with >98\% accuracy and >79\% precision for all DCLP3 patients, but \DIFdelbegin \DIFdel{suffered from }\DIFdelend \DIFaddbegin \DIFadd{had }\DIFaddend a low recall score, indicating that false negatives were too frequent \DIFdelbegin \DIFdel{and may be a consequence of the model's poorer performance on clinical trial data}\DIFdelend \DIFaddbegin \DIFadd{(likely due
to lack of volatile BG traces in training data)}\DIFaddend .

%DIF > Analysis showed the data traces on which the ML model performed poorly were characterized by meals with high CHO content, large insulin fluctuations, and hyperglycemia, which are conditions associated with rebound highs. While the model predicts BG trends well when in range, the model struggles with volatile traces with higher than normal change in BG, likely due to an lack of examples of these scenarios in training data because they are less frequent.
\DIFaddbegin 

\DIFaddend %Using simulation data as detailed previously \ref{workflowdata}, we achieved Xxx\% accuracy and Xxx\% timeliness in Xxx runs.
%get our data, (transfer learning on my data), train the model, evaluate on test

%\subsubsection{Participant Data}
%In the participant data (trained using UVA/Padova Type I Diabetes Simulator and tuned on some of the participant data \ref{workflowdata}), we found 3 rebound high events in ~5 days of data.
%This resulted in Xxx\% accuracy and Xxx\% timeliness.


\DIFaddbegin \begin{table}[t!]
    %DIF > \vspace{-2em}
    \centering
    \caption{\DIFaddFL{RMSE Comparison of Model Performance with and without Carb Focusing and Attention}}
    \begin{tabular}{c c c c c c c}
        \multicolumn{2}{c}{\textbf{Patient ID}} & \textbf{\DIFaddFL{Our Approach}} & \textbf{\DIFaddFL{No Carb Focus}} & \textbf{\DIFaddFL{No Attention}} \\
        \hline

        \parbox[t]{10mm}{\multirow{5}{*}{UVA Sim.}} & \DIFaddFL{1 }& \textbf{\DIFaddFL{11.808}} & \DIFaddFL{17.860 }& \DIFaddFL{13.438 }\\
         & \DIFaddFL{2 }& \DIFaddFL{13.539 }& \textbf{\DIFaddFL{12.229}} & \DIFaddFL{32.228 }\\
         & \DIFaddFL{3 }& \textbf{\DIFaddFL{14.166}} & \DIFaddFL{21.616 }& \DIFaddFL{14.329}\\
         & \DIFaddFL{4 }& \textbf{\DIFaddFL{12.043}} & \DIFaddFL{23.589 }& \DIFaddFL{70.644}\\
         & \DIFaddFL{5 }& \textbf{\DIFaddFL{14.010}} & \DIFaddFL{16.381 }& \DIFaddFL{14.350}\\
         \parbox[t]{10mm}{\multirow{5}{*}{DCLP3}} & \DIFaddFL{3 }& \textbf{\DIFaddFL{27.750}} & \DIFaddFL{31.105 }& \DIFaddFL{33.790}\\
         & \DIFaddFL{4 }& \DIFaddFL{28.546 }& \DIFaddFL{29.310 }& \textbf{\DIFaddFL{27.133}}\\
         & \DIFaddFL{5 }& \textbf{\DIFaddFL{17.988}} & \DIFaddFL{18.140 }& \DIFaddFL{18.119}\\
         & \DIFaddFL{6 }& \DIFaddFL{32.718 }& \textbf{\DIFaddFL{27.882}} & \DIFaddFL{28.580}\\
         & \DIFaddFL{7 }& \textbf{\DIFaddFL{38.050}} & \DIFaddFL{66.761 }& \DIFaddFL{60.748}\\

    \end{tabular}
    \label{tab:comp_results}
    \vspace{-1.5em}
\end{table}


\DIFaddend \begin{table}[h!]
    \caption{Rebound High Alert System Performance Metrics}
    \resizebox{\columnwidth}{!}{%
    \begin{tabular}{c|c|c|c|c|c}
        \textbf{DCLP3} & \textbf{Rebound High} & \textbf{Accuracy} & \textbf{Precision} & \textbf{Recall} & \textbf{F1 Score} \\
        \textbf{Patient}&\textbf{Alerts to Issue}&&&& \\
        \hline
        3 & 148 & 98.3 & 96.0 & 21.7 & 35.4 \\
        4 & 221 & 99.1 & 79.4 & 36.5 & 50.0 \\
        6 & 231 & 98.8 & 89.9 & 46.3 & 61.1 \\
        7 & 221 & 98.6 & 96.9 & 28.5 & 44.1 \\
    \end{tabular}
    }
    \label{tab:alert_metrics}
    \vspace{-1.5em}
\end{table}

\begin{comment}
\subsection{Initial Survey}

The commonplace of rebound hyperglycemia is supported by the survey responses of a survey we sent to ~10 patients with type 1 diabetes. When asked to select on a scale from "Never" to "Very Often" with the statement "After I have low blood sugar, I have a rebound high blood sugar" all of them answered in a mid range but one who selected "Very often." When asked the same in regards to the statement "After I have a high blood sugar, I have a rebound low blood sugar" all of them answered in a mid range.

All participants indicated that "If [they] realize [they] have low blood sugar at the start of a meal," they will still eat the meal before their blood sugar returns to safe range. Two-thirds stated that in the case of being low at a mealtime, they would wait until they are back in range to give insulin for the meal. Of the participants who indicated waiting until they were back in range, 74.96\% said they would give a full injection for the carbohydrates eaten, whereas the remaining 25.04\% give a partial insulin injection for the carbohydrates eaten. This section of the survey gives support for our carry-over carb method and application reminders for when the user is back in range, since all of the participants indicated over correcting for their hypoglycemia at mealtimes, and the majority wait until they are back in range to give any insulin. This leaves room for error, since current CGMs do not typically alert users when they are back in range.

Two-thirds of participants indicated that they don't feel low blood sugar symptoms until they are in the low range (they start feeling symptoms at 70 $mg/dL$ or below), and 83.3\% indicated that they don't begin to feel high BG symptoms until they are in high range (they start feeling symptoms at 180 $mg/dL$ or above). This gives support for rebound high predictive alerts and our carry-over carb method to prevent the rebound highs. 
\end{comment}

\section{Conclusion and Future Work}
This paper proposes \DIFdelbegin \DIFdel{extending the APS controllers with capabilities for the prediction and prevention of rebound hyperglycemia events. We present an }\DIFdelend \DIFaddbegin \DIFadd{expanding the capacities of APS  controllers to predict and prevent rebound highs. Our }\DIFaddend ML BG prediction method \DIFdelbegin \DIFdel{that outputs trending alert messages using scenario-specific language to refer to rebound highs along with suggested increased bolus }\DIFdelend \DIFaddbegin \DIFadd{provides scenario-specific alert messages and suggests increased basal }\DIFaddend thresholds based on \DIFdelbegin \DIFdel{carbohydrate }\DIFdelend \DIFaddbegin \DIFadd{carb }\DIFaddend input in cases\DIFdelbegin \DIFdel{of hypoglycemia when a rebound high is predicted}\DIFdelend . Experimental results show \DIFdelbegin \DIFdel{that the proposed method achieves }\DIFdelend >98\% accuracy and >79\% precision for all DCLP3 patients.

Future work will \DIFdelbegin \DIFdel{involve the optimization and automation of carbohydrate carry-over bolus increases since our process has the limitations of heuristicsand requiring unnecessary user action. 
In addition, future work will involve calculating more exact carbohydrate }\DIFdelend \DIFaddbegin \DIFadd{optimize away from heuristics, automatically apply $IB$ rates, and calculate more exact carb }\DIFaddend intake recommendations in the \DIFdelbegin \DIFdel{cases of hypoglycemia and the }\DIFdelend case of rebound lows \DIFaddbegin \DIFadd{beyond general insulin suspension.
}\DIFaddend %In addition, we conducted a small survey surrounding habits of out of range BG prediction and treatment, which supported the need for study. We also 
%future work includes the proposal, creation, and evaluation of an application or feature set to be added to an existing controller in order to synthesize our prediction and alert system with carbohydrate entry reminders (both reminders at set time intervals and detected reminders based on meal prediction). 
\DIFdelbegin \DIFdel{We will also focus on the integration of the proposed rebound evaluation module with carbohydrate entry reminders on an existing APS controller and evaluation using patient feedback. 
}\DIFdelend %DIF > %%%%%We will also focus on the integration of the proposed rebound evaluation module with carb entry reminders on an existing APS controller and evaluation using patient feedback. 
%In addition, future work will involve calculating more exact carbohydrate intake recommendations in the cases of hypoglycemia and the case of rebound lows, similarly to how we calculated IB in this paper.

\DIFdelbegin %DIFDELCMD < \begin{figure}[t!]
%DIFDELCMD <     \vspace{-0.5em}
%DIFDELCMD <     \centering
%DIFDELCMD <     \includegraphics[scale=0.45]{Figures/rebound_high_histogram.png}
%DIFDELCMD <     \vspace{-0.2em}
%DIFDELCMD <     %%%
%DIFDELCMD < \caption{%
{%DIFAUXCMD
\DIFdelFL{Rebound High Count Across Patients}}
    %DIFAUXCMD
%DIFDELCMD < \label{fig:rebound_high_hist}
%DIFDELCMD <     \vspace{-1em}
%DIFDELCMD < \end{figure}
%DIFDELCMD < %%%
\DIFdelend %DIF > \begin{figure}[t!]
 %DIF >    \vspace{-0.5em}
  %DIF >   \centering
   %DIF >  \includegraphics[scale=0.45]{Figures/rebound_high_histogram.png}
    %DIF > \vspace{-0.2em}
    %DIF > \caption{Rebound High Count Across Patients}
    %DIF > \label{fig:rebound_high_hist}
    %DIF > \vspace{-1em}
%DIF > \end{figure}

%As we have proposed an app that will synthesize several of our ideas, in the future a creation of the app and a study of its effects would be a good step forward. An extended research study on both the app once created as well as the carry-over carb method using real-time participants could prove very useful. Finally, an automation of the carry-over insulin increase rather than a reminder for patient action would eliminate unnecessary patient action in the instance of predicted rebound high.

\DIFdelbegin %DIFDELCMD < \begin{comment}%DIFDELCMD < 
%DIFDELCMD < 
%DIFDELCMD < % \vspace{-1em}
%DIFDELCMD < \section*{Acknowledgment}
%DIFDELCMD < \vspace{-0.75em}
%DIFDELCMD < This material is based upon work supported by the National Science Foundation (NSF) under Grant No. 1748737.
%DIFDELCMD < \end{comment}
%DIFDELCMD < %%%
\DIFdelend \DIFaddbegin \begin{comment}

% \vspace{-1em}
\section*{Acknowledgment}
% \vspace{-0.75em}
This material is based upon work supported by the National Science Foundation (NSF) under Grant No. 1748737.
\end{comment}
\DIFaddend 

%DIF < This work was supported by a grant from the U.S. National Science Foundation (No. 1748737) and a research innovation award from the University of Virginia.
%DIF <  %This work was partially supported by a grant from the U.S. National Science Foundation (Award No. 1748737) and a SEAS research innovation award from the University of Virginia, School of Engineering and Applied Science (SEAS).
%DIF <  \vspace{-1.5em}
\DIFdelbegin %DIFDELCMD < \bibliographystyle{IEEEtran}
%DIFDELCMD < \bibliography{main.bib}
%DIFDELCMD < %%%
\DIFdelend %DIF > % The next two lines define the bibliography style to be used, and
%DIF > % the bibliography file.
\DIFaddbegin \bibliographystyle{ACM-Reference-Format}
\bibliography{main}

\DIFaddend %%TC:endignore
% \section{Appendix for Proofs}

\paragraph{Proof of Theorem \ref{thm:main}.}

\begin{proof}
\label{proof:main}
Our proof has two steps. In Step 1, we will show that SimCLR is equivalent to minimizing the cross entropy loss defined in Eqn.~(\ref{eqn:cross-entropy}). 
In Step 2, we will show  that minimizing the cross-entropy loss 
is equivalent to spectral clustering on $\bfpi$. 
Combining the two steps together, we have proved our theorem. 

\textbf{Step 1: } SimCLR is equivalent to minimizing the cross entropy loss.

The cross-entropy loss takes expectation over 
$\bfW_\bfX\sim \mathbb{P}(\cdot ; \bfpi)$, 
which means $\bfW_\bfX$ has exactly one non-zero entry in each row $i$. By Lemma~\ref{lem:multinomial}, we know every row $i$ of $\bfW_\bfX$ is independent of other rows. Moreover, 
$\bfW_{\bfX,i}\sim \mathcal{M}(1, \bfpi_i/\sum_j \bfpi_{i,j})=\mathcal{M}(1, \bfpi_i)$, because $\bfpi_i$ itself is a probability distribution.
Similarly, we know $\bfW_\bfZ$ also has the row-independent property by sampling over $\mathbb{P}(\cdot;\bfK_\bfZ)$.
Therefore, by Lemma~\ref{lem:cross_split}, we know Eqn.~(\ref{eqn:cross-entropy}) is equivalent to:
\[
 -\sum_{i=1}^n \mathbb{E}_{\bfW_{\bfX,i}}[\log \mathbb{P}(\bfW_{\bfZ,i}=\bfW_{\bfX,i};\bfK_\bfZ)],
\]

This expression takes expectation over $\bfW_{\bfX,i}$ for the given row $i$. Notice that 
$\bfW_{\bfX,i}$ has exactly one non-zero entry, which equals $1$ (same for $\bfW_{\bfZ,i}$). 
As a result
we expand the above expression to be:
\begin{equation}
 -\sum_{i=1}^n \sum_{j\neq i} \Pr(\bfW_{\bfX,i,j}=1)\log \Pr(\bfW_{\bfZ,i,j}=1).
\label{eqn:detailed-expansion}    
\end{equation}


By Lemma~\ref{lem:multinomial}, $\Pr(\bfW_{\bfZ,i,j}=1)=\bfK_{\bfZ,i,j}/\|\bfK_{\bfZ,i}\|_1$ for $j\neq i$. Recall that $\bfK_\bfZ=(k(\bfZ_i-\bfZ_j))_{(i,j)\in[n]^2}$, which means 
$\bfK_{\bfZ,i,j}/\|\bfK_{\bfZ,i}\|_1=\frac{\exp(-\|\bfZ_i-\bfZ_j\|^2/{2\tau})}{\sum_{k\neq i}
\exp(-\|\bfZ_i-\bfZ_k\|^2/{2\tau})
}$ for $j\neq i$, when $k$ is the Gaussian kernel with variance $\tau$. 

Notice that $\bfZ_i=f(\bfX_i)$, so we know
\begin{equation}
-\log \Pr(\bfW_{\bfZ,i,j}=1)=
-\log \frac{\exp(-\|f(\bfX_i)-f(\bfX_j)\|^2/{2\tau})}{\sum_{k\neq i}
\exp(-\|f(\bfX_i)-f(\bfX_k)\|^2/{2\tau}),
}
\label{eqn:infonce-equivalence}    
\end{equation}


The right hand side is exactly the InfoNCE loss defined in Eqn.~(\ref{eqn:infonce}).
Inserting Eqn.~(\ref{eqn:infonce-equivalence}) into Eqn.~(\ref{eqn:detailed-expansion}), we get the SimCLR algorithm, which first samples augmentation pairs $(i,j)$ with $\Pr(\bfW_{\bfX,i,j}=1)$ for each row $i$, and then optimize the InfoNCE loss. 

\textbf{Step 2: } minimizing the cross entropy loss 
is equivalent to spectral clustering on $\bfpi$.


By Lemma~\ref{lem:convert_to_spectral}, we may further convert the loss to 
\begin{equation}
\label{eqn:main-theorem-repul-attr}
\min_{\bfZ}
-\sum_{(i,j)\in [n]^2} \mathbf{P}_{i,j}
\log k (\bfZ_i-\bfZ_j)+\log \mathbf{R}(\bfZ).
\end{equation}
Since $k$ is the Gaussian kernel, this reduces to \[
\min_\bfZ \mathrm{tr}(\bfZ^\top \mathbf{L}(\bfpi) \bfZ)
+\log \mathbf{R}(\bfZ),
\]

where we use the fact that $\mathbb{E}_{\bfW_\bfX\sim \mathbb{P}(\cdot; \bfpi)}[\mathbf{L}(\bfW_\bfX)]
=\mathbf{L}(\bfpi)
$, because the Laplacian operator is linear and $
\mathbb{E}_{\bfW_\bfX\sim \mathbb{P}(\cdot; \bfpi)}(\bfW_\bfX)=\bfpi
$.
\end{proof}

\paragraph{Proof of Theorem \ref{thm:clip}.}
\begin{proof}
Since $\bfW_\bfX\sim \mathbb{P}(\cdot;\bfpi_{\mathbf{A}, \mathbf{B}})$, we know 
$\bfW_\bfX$ has exactly one non-zero entry in each row, denoting the pair that got sampled. 
A notable difference compared to the previous proof is we now have $n_\mathcal{A}+n_\mathcal{B}$ objects in our graph. CLIP deals with this by taking a mini-batch of size $2N$, 
such that $n_\mathcal{A}=n_\mathcal{B}=N$, and adding the $2N$ InfoNCE losses together. We label the objects in $\mathcal{A}$ as $[n_\mathcal{A}]$, and the objects in $\mathcal{B}$ as $\{n_\mathcal{A}+1, \cdots, n_\mathcal{A}+n_\mathcal{B}\}$. 

Notice that $\bfpi_{\mathbf{A}, \mathbf{B}}$ is a bipartite graph, so the edges of objects in $\mathcal{A}$ will only connect to object in $\mathcal{B}$ and vice versa. We can define the similarity matrix in $\cZ$ as $\bfK_\bfZ$, 
where $\bfK_\bfZ(i, j+n_\mathcal{A})=\bfK_\bfZ(j+n_\mathcal{A},i)= k(\bfZ_i-\bfZ_j)$ for $i\in [n_\mathcal{A}], j\in [n_\mathcal{B}]$, and otherwise we set $\bfK_\bfZ(i,j)=0$. 
The rest is same as the previous proof. 
\end{proof}

\paragraph{Proof of Theorem \ref{thm:exponential}.}

\begin{proof}
\label{proof:exponential}
Since the objective function consists of a linear term combined with an entropy regularization, which is a strongly concave function, the maximization problem is a convex optimization problem. Owing to the implicit constraints provided by the entropy function, the problem is equivalent to having only the equality constraint. We then introduce the Lagrangian multiplier $\lambda$ and obtain the following relaxed problem:

$$
\widetilde{E}(\boldsymbol{\alpha})=\psi_{1}-\sum_{i=1}^n \alpha_{i} \psi_{i}+\tau \sum_{i=1}^n \alpha_{i}\log \alpha_{i}+\lambda\left(\boldsymbol{\alpha}^{\top} \mathbf{1}_n-1\right).
$$

As the relaxed problem is unconstrained, taking the derivative with respect to $\alpha_{i}$ yields

$$
\frac{\partial \widetilde{E}(\boldsymbol{\alpha})}{\partial \alpha_{i}}=-\psi_{i}+\tau\left(\log \alpha_{i}+\alpha_{i} \frac{1}{\alpha_{i}}\right)+\lambda=0.
$$

Solving the above equation implies that $\alpha_{i}$ takes the form
$
\alpha_{i}=\exp \left(\frac{1}{\tau} \psi_{i}\right) \exp \left(\frac{-\lambda}{\tau}-1\right).
$ Since $\alpha_{i}$ lies on the probability simplex, the optimal $\alpha_{i}$ is explicitly given by
$
\alpha^{*}_{i}=\frac{\exp \left(\frac{1}{\tau} \psi_{i}\right)}{\sum_{i^{\prime}=1}^n \exp \left(\frac{1}{\tau} \psi_{i^{\prime}}\right)} .
$ Substituting the optimal point into the objective function, we obtain
$$
\begin{aligned}
E\left(\boldsymbol{\alpha}^*\right)  &=\psi_1-\sum_{i=1}^n \frac{\exp \left(\frac{1}{\tau} \psi_{i}\right)}{\sum_{i^{\prime}=1}^n \exp \left(\frac{1}{\tau} \psi_{i^{\prime}}\right)} \psi_{i}+\tau \sum_{i=1}^n \frac{\exp \left(\frac{1}{\tau} \psi_{i}\right)}{\sum_{i^{\prime}=1}^n \exp \left(\frac{1}{\tau} \psi_{i^{\prime}}\right)}\log \frac{\exp \left(\frac{1}{\tau} \psi_{i}\right)}{\sum_{i^{\prime}=1}^n \exp \left(\frac{1}{\tau} \psi_{i^{\prime}}\right)} \\
& =\psi_1 - \tau \log \left(\sum_{i=1}^n \exp \left(\frac{1}{\tau} \psi_{i}\right)\right).
\end{aligned}
$$
Thus, the Lagrangian dual function is given by
\begin{equation*}
-E\left(\boldsymbol{\alpha}^*\right)= -\tau \log \frac{\exp \left(\frac{1}{\tau} \psi_{1}\right)}{\sum_{i=1}^n \exp \left(\frac{1}{\tau} \psi_{i}\right)}.\qedhere
\end{equation*}
\end{proof}



\section{More on Experiments} \label{section: experiment_details}

\paragraph{CIFAR-10 and CIFAR-100} CIFAR-10 ~\citep{krizhevsky2009learning} and CIFAR-100 ~\citep{krizhevsky2009learning} are well-known classic image classification datasets. Both CIFAR-10 and CIFAR-100 contain a total of 60k $32 \times 32$ labeled images of different classes, with 50k for training and 10k for testing. CIFAR-10 is similar to CIFAR-100, except there are 10 different classes in CIFAR-10 and 100 classes in CIFAR-100.

\paragraph{TinyImageNet} TinyImageNet ~\citep{le2015tiny} is a subset of ImageNet ~\citep{deng2009imagenet}. There are 200 different object classes in TinyImageNet, with 500 training images, 50 validation images, and 50 test images for each class. All the images in TinyImageNet are colored and labeled with a size of $64 \times 64$.

\textbf{Pseudo-code.} Algorithm \ref{alg:Training Procedure} presents the pseudo-code for our empirical training procedure.

\begin{algorithm}[!htbp]
\caption{Training Procedure}
\label{alg:Training Procedure}
\begin{algorithmic}[1]
\REQUIRE trainable encoder network $f$, batch size $N$, augmentation strategy \textit{aug}, loss function $L$ with hyperparameters \textit{args}
\FOR {sampled minibatch ${x_i}_{i=1}^N$}
\FORALL{$i \in { 1, ..., N }$}
\STATE draw two augmentations $t_i = \textit{aug}\left(x_i\right) $, $t_i' = \textit{aug}\left(x_i\right) $
\STATE $z_i = f\left(t_i\right)$, $z_i' = f\left(t_i'\right)$
\ENDFOR
\STATE compute loss $\mathcal{L} = L(N, z, z', \textit{args})$
\STATE update encoder network $f$ to minimize $\mathcal{L}$
\ENDFOR
\STATE \textbf{Return} encoder network $f$
\end{algorithmic}
\end{algorithm}

We also provide the pseudo-code for our core loss function used in the training procedure in Algorithm \ref{alg:Core loss}. The pseudo-code is almost identical to SimCLR's loss function, with the exception of an extra parameter $\gamma$.

\begin{algorithm}[!htbp]
\caption{Core loss function $\mathcal{C}$}
\label{alg:Core loss}
\begin{algorithmic}[1]
\REQUIRE batch size $N$, two encoded minibatches $z_1, z_2$, $\gamma$, temperature $\tau$
\STATE $z = \textit{concat}\left(z_1, z_2\right)$
\FOR {$i \in {1, ..., 2N }, j \in {1, ..., 2N}$ }
\STATE $s_{i,j} = \Vert z_i - z_j \Vert_2^{\gamma}$
\ENDFOR
\STATE \textbf{define} $l(i, j)$ \textbf{as} $l(i, j) = - \log \frac{exp\left(s_{i,j}/\tau \right)}{\sum_{k=1}^{2N} \mathbf{1}{[k \ne i]} exp\left(s{i, j} / \tau \right)} $
\STATE \textbf{Return} $\frac{1}{2N} \sum_{k=1}^N\left[l(i, i+N) + l(i+N, i)\right]$
\end{algorithmic}
\end{algorithm}

Utilizing the core loss function $\mathcal{C}$, we can define all kernel loss functions used in our experiments in Table \ref{table: loss definition}. For all $z_i \in z$ with even dimensions $n$, we define $z_{L_i} = z_i\left[0:n/2\right]$ and $z_{R_i} = z_i\left[n/2:n\right]$.

\begin{table}[ht]
\centering
\begin{tabular}{{@{}l|l@{}}}
Kernel  &  Loss function \\ \midrule
Laplacian & $\mathcal{C}\left(N, z, z', \gamma=1, \tau\right)$\\ \midrule
Sum       & $\lambda * \mathcal{C}\left(N, z, z', \gamma=1, \tau_1\right) + (1-\lambda) * \mathcal{C}\left(N, z, z', \gamma=2, \tau_2\right)$  \\ \midrule
Concatenation Sum&$\lambda * \mathcal{C}\left(N, z_L, z'_L, \gamma=1, \tau_1\right) + (1-\lambda) * \mathcal{C}\left(N, z_R, z'_R, \gamma=2, \tau_2\right)$\\ \midrule
$\gamma = 0.5$ & $\mathcal{C}\left(N, z, z', \gamma=0.5, \tau\right)$          \\ 

\end{tabular}

\caption{Definition of kernel loss functions in our experiments}
\label {table: loss definition}
\end{table}

\textbf{Baselines.} We reproduce the SimCLR algorithm using PyTorch Lightning~\citep{PytorchLightning}.

\textbf{Encoder details.}
The encoder $f$ consists of a backbone network and a projection network. We employ ResNet50~\citep{ResNet} as the backbone and a 2-layer MLP (connected by a batch normalization~\citep{ioffe2015batch} layer and a ReLU \cite{nair2010rectified} layer) with hidden dimensions 2048 and output dimensions 128 (or 256 in the concatenation kernel case).

\textbf{Encoder hyperparameter tuning.}
For each encoder training case, we randomly sample 500 hyperparameter groups (sample details are shown in Table \ref{table: Hyperparameter sample}) and train these samples simultaneously using Ray Tune ~\citep{RayTune}, with the ASHA scheduler~\citep{li2018massively}. Ultimately, the hyperparameter group that maximizes the online validation accuracy (integrated in PyTorch Lightning) within 5000 validation steps is chosen for the given encoder training case.

\begin{table}[ht]
\centering

\begin{tabular}{@{}l|l|l@{}}
\midrule
Hyperparameter  & Sample Range & Sample Strategy \\ \midrule
start learning rate & $\left[10^{-2}, 10\right]$ & log uniform \\ \midrule
$\lambda$       & $\left[0, 1\right]$ & uniform \\ \midrule
$\tau$, $\tau_1$, $\tau_2$ & $\left[0, 1\right]$ & log uniform \\ \midrule
\end{tabular}

\caption{Hyperparameters sample strategy}
\label {table: Hyperparameter sample}
\end{table}

\textbf{Encoder training.} 
We train each encoder using the LARS optimizer~\citep{LARSOptimizer}, LambdaLR Scheduler in PyTorch, momentum 0.9, weight decay $10^{-6}$, batch size 256, and the aforementioned hyperparameters for 400 epochs on a single A-100 GPU.

\textbf{Image transformation.} The image transformation strategy, including augmentation, is identical to the default transformation strategy provided by PyTorch Lightning.

\textbf{Linear evaluation.}
The linear head is trained using the SGD optimizer with a cosine learning rate scheduler, batch size 64, and weight decay $10^{-6}$ for 100 epochs. The learning rate starts at $0.3$ and ends at $0$.

\textbf{Moco Experiments.} We also tested our method based on MoCo~\citep{he2019moco}. The results are summarized in Table \ref{tab:results-moco}. Here we choose ResNet18~\citep{ResNet} as the backbone and set a temperature of $0.1$ as default. For our simple sum kernel, we set $\lambda=0.8$. The results show that our method outperforms the original MoCo method.

\begin{table}[thb]
\centering
\caption{MoCo Experiment Results on CIFAR-10 and CIFAR-100.}
\label{tab:results-moco}
\resizebox{\textwidth}{!}{%
\begin{tabular}{@{}c|ccc|ccc@{}}
\toprule
\multirow{3}{*}{Method} & \multicolumn{3}{c|}{CIFAR-10} & \multicolumn{3}{c}{CIFAR-100} \\ \cmidrule(lr){2-4} \cmidrule(lr){5-7} 
                        & 200 epochs & 400 epochs    & 1000 epochs   & 200 epochs & 400 epochs & 1000 epochs         \\ \midrule
MoCo (repro.)         & $76.41 \pm 0.12$    & $80.01 \pm 0.15$          & $84.45 \pm 0.08$    & $\mathbf{47.02 \pm 0.11}$ & $52.50 \pm 0.07$ & $57.62 \pm 0.15$            \\
\midrule
Laplacian Kernel        & ${78.09 \pm 0.10}$    & $\mathbf{83.85 \pm 0.09}$          & $\mathbf{88.34 \pm 0.16}$    & $46.12 \pm 0.22$   & $53.44 \pm 0.17$ & $59.10 \pm 0.14$        \\
Simple Sum Kernel & $\mathbf{78.12 \pm 0.15}$   & $83.23 \pm 0.18$ & $87.50 \pm 0.20$ & $46.65 \pm 0.06$ & $\mathbf{53.62 \pm 0.19}$ & $\mathbf{59.83 \pm 0.12}$\\
\bottomrule
\end{tabular}
}
\end{table}



\section{More Experiments on Synthetic Data}


Consider a scenario with $n$ clusters, each containing $k$ vertices. Let the probability of vertices $u$ and $v$ from the same cluster belonging to $\bfpi$ be $p$. Conversely, for vertices $u$ and $v$ from different clusters, let the probability of belonging to $\pi$ be $q$. We generate the graph $\bfpi$ randomly, based on $p$ and $q$. We experiment with values of $k=100$ and $n=6$ for ease of visualization, embedding all points in a two-dimensional space. Each vertex's initial position originates from a normal distribution. In each iteration, we sample a subgraph of $\bfpi$ uniformly, ensuring each vertex has an out-degree of $1$. We then optimize the corresponding vectors using InfoNCE loss with an SGD optimizer and iterate until convergence. Our experimental setup consists of an SGD learning rate of $1$, an InfoNCE loss temperature of $0.5$, and a batch size of $50$. We evaluate two scenarios with different $p$ and $q$ values: $p=1$, $q=0$, and $p=0.75$, $q=0.2$. The results of these experiments are visualized in Figure \ref{fig:vis-spectral-cluster}. The obtained embeddings exhibit the hallmark pattern of spectral clustering of graph $\bfpi$.

\begin{figure}[!tb]
\centering
\subfigure{
\includegraphics[width=1\textwidth]{Figures/cluster_pi.png}
\label{fig:vis-cluster}
}
\subfigure{
\includegraphics[width=1\textwidth]{Figures/noised_cluster_pi.png}
\label{fig:vis-noised-cluster}
}
\caption{Visualizations of the optimization process using InfoNCE Loss on the vectors corresponding to $\bfpi$. Points of identical color belong to the same cluster within $\bfpi$. To showcase the internal structure of $\bfpi$, we randomly select 10 vertices from each cluster to display the edge distribution of $\bfpi$.}
\label{fig:vis-spectral-cluster}
\end{figure}


\end{document}


% \fi
% \fi

\section{Introduction}

Stochastic gradient descent (SGD) is one of the key techniques behind the success of machine learning based on deep neural networks. 
Although the SGD algorithm is commonly formulated in the Euclidean space, SGD type algorithms can be formulated in a much broader context, such as optimizing objective functions defined on Riemannian manifolds \cite{bonnabel_stochastic_2013,Zhang_2016,tripuraneni_averaging_2018,Sato_2019}. In this context the gradient step can be followed by a projection step, or it can be defined using an exponential map, such that the parameter remains on the manifold. 

We focus on the case when the manifold is a sphere. We can then apply the general Riemannian optimization techniques of projections or exponential mappings, or the problem  can be reformulated as the minimization of a scale-invariant loss \cite{Saliman_2016} over an unconstrained space. In either case, the element to be optimized over the sphere is required to be represented explicitly in memory and can be considered as a high-dimensional parameter. 
 
This paper, on the other hand, considers a setting where the element of the high-dimensional sphere is not stored explicitly in memory. The most significant motivation for this setting is the optimization of wavefunctions in quantum physics, both in the context of supervised learning and unsupervised learning (the latter is more commonly known in physics as the variational  Monte Carlo method, see e.g. \citet{Becca2017}). Usually in this context the parameter $\theta$ is stored explicitly in memory and is unconstrained, but the wavefunction $f_\theta$ belongs to the unit sphere of an infinite-dimensional function space, so we can neither enumerate all of its entries nor directly evaluate functionals of $f_\theta$ such as $\|f_\theta\|$. Instead, we assume only that the algorithm has access to samples $(X_i,F_\theta(X_i))$, where $X_i$ are random samples from the domain of $f$ and $F_\theta$ consists of $f$ and its low-order derivatives with respect to $\theta$ and $x$\footnote{For example in quantum chemistry we need first-order derivatives with respect to $\theta$ and second-order derivatives with respect to $x$.}.

% GG: included this in Problem Setup 1.1
% We consider the problem of optimizing a loss function $\L$ over the unit sphere in a parameterized function space. This is equivalent with an unconstrained optimization of a scale-invariant loss $\L(f/\|f\|)$ over functions $f_\theta$ parameterized by the unconstrained parameter $\theta$. }

% This application motivates us to consider the problem of optimizing a loss function $\L$ over the unit sphere in a parameterized function space. This is equivalent with an unconstrained optimization of a scale-invariant loss $\L(f/\|f\|)$ over functions $f_\theta$ parameterized by the unconstrained parameter $\theta$. 
%This paper considers the convergence of SGD type algorithms on a sphere of vectors $f/\|f\|$ when the dimension of $f$ is too large to be stored in memory. We represent a point on the sphere non-uniquely by a non-normalized parameterized vector $f_\theta$ which can be accessed through random samples of its coordinates $f_\theta(x)$. 

% To the best of our knowledge, previous works analyzing stochastic gradient descent on manifolds require that the stochastic gradient step is moved from the tangent space to the manifold using an exponentiation mapping or a projection step. This operation is natural when $f$ is stored in memory as a vector but cannot be implemented when the vector is parameterized as $f_\theta$ and given our random access model.
% (GG: the above is alluded to a couple paragraphs up, and also stated directly in section 1.3.3.


\ZD{Reorganize this section}

\subsection{Problem setup}

 

Let $\H$ be a function space with an inner product $\ol{f}{g}$ and norm $\|f\|^2=\ol{f}{f}$. Specifically, we consider the space $\H=L^2(\Omega,\rho)$ of real-valued square integrable functions (the complex-valued case can be treated analogously) on $\Omega$ with respect to a positive measure $\rho$.
The neural network maps a parameter $\theta\in\RR^d$ to a function $f_\theta\in\H$.

We consider the problem of optimizing a loss function $\L(f_\theta)$ over the unit sphere of normalized functions. In this setting we cannot directly project $f_\theta$ to the sphere of normalized functions since we do not know a parameter $\tilde\theta$ such that $f_{\tilde\theta}=f_\theta/\|f_\theta\|$. We can solve this problem by introducing another degree of freedom and parameterizing the function space as $(\lambda,\theta)\mapsto\lambda f_\theta$, but even so the accurate estimation of the normalization factor $\lambda$ is costly in general.

%An alternative perspective is to consider the quotient space $\H/\RR_+$, 
% take another view: Let $\H'$ be the nonzero elements of $\H$ and identify the sphere $\tilde\H$ with 
%whose elements are the rays
%\begin{equation}
%    [f]=\{\lambda f\:|\:\lambda>0\}.
%\end{equation}

We instead transform the optimization problem over the sphere of normalized functions into one where $f_\theta$ is unconstrained in $\H$. A loss function on the sphere then corresponds to a \emph{scale-invariant} loss $\L:\H\to\RR$, meaning that $\L(\lambda f)=\L(f)$ for any $\lambda>0$. This formulation is distinct from previous works~\cite{Saliman_2016,kodryan_training_2022}, which  studied loss functions that are scale-invariant in the \emph{parameter} (see \cref{fig:vf}).
%\vspace -.5cm
%
%
\begin{figure}[h]
\[
%\arraycolsep=1.4pt
\def\arraystretch{2.2}
%
\begin{array}{ccccc}
&&\v\in\RR^d&\xrightarrow[]{\text{scale-invariant }\L} &\L(\v)\\
\hline
\multicolumn{1}{|l}{\theta\in\RR^d}
&\xrightarrow[]{}&f_\theta\in\Lt&\xrightarrow[]{\text{scale-invariant }\L} &
\multicolumn{1}{l|}{L(\theta)=\L(f_\theta)}\\
\hline
\end{array}
\]
\caption{The typical setting for scale-invariant learning (top) vs our setting (bottom).}
\label{fig:vf}
\end{figure}


\subsubsection{Supervised learning}\label{sec:super_l}
We first consider the problem of minimizing the distance from the line $\{\lambda f|\lambda>0\}$ to a target function $g\in\H$. That is, 
\begin{equation}\label{supervisedC}
\mathcal{L}(f)=\min_{\lambda>0}\left\|\lambda f-g\right\|^2\,.
\end{equation}
This loss function is scale-invariant by construction and has a closed-form expression
\[
\mathcal{L}(f)=\|g\|^2-\left(\ol{g}{f}/\|f\|\right)^2\textbf{1}_{\ol{g}{f}>0}\,.
\]
Minimizing $\mathcal{L}(f_\theta)$ is therefore equivalent to minimizing $-\ol{g}{f_\theta}/\|f_\theta\|$. 
\begin{problem}[Supervised learning]\label{pro:slos}
Given a target function $g\in\H$, access to samples $X_i\sim\rho$, and the ability to evaluate $g$, $f_\theta$, and $\partial f_\theta / \partial \theta$ at a point $X_i$,
%and a parameterization $\Phi:\theta\in\RR^d\mapsto f_\theta\in\H$, 
minimize
\begin{equation}\label{eqn:Ltheta}
L(\theta)=-\frac{\ol{g}{f_\theta}}{\|f_\theta\|}\,
\end{equation}
over parameter vectors $\theta\in\mathbb{R}^d$.
\end{problem}
The gradient of this loss with respect to $\theta$ can be derived from the literature on scale-invariant losses using the chain rule, but simply plugging in an approximate value of $\|f_\theta\|$ to the resulting formula yields a biased estimator. In fact, the SGD algorithm with this biased estimator does not have theoretical convergence guarantees, and numerical results demonstrate that such an SGD algorithm fails to reach convergence in practice (see \cref{fig:balance}). 


\subsubsection{Unsupervised variational Monte Carlo simulation}\label{sec:VMC}

%\LL{ some backgrounds in VMC and recent neural network based literature. @Gil can add some from the previous paper} 

A typical goal in quantum physics and quantum chemistry is to find the function that minimizes the \emph{energy}, a scale-invariant functional of the form $\L:\wf\to\ol{\wf}{H\wf}/\braket{\wf,\wf}$. The function $f$ is known in this context as the wave function and is usually denoted by $\psi$. The Hermitian operator $H$ is known as the \emph{Hamiltonian} and the minimizer $\wf_0$ is known as the \emph{ground state}. We consider the most common case when $H$ is a real symmetric operator, in which case $\wf_0$ can be taken to be real so that it suffices to minimize over only real $\wf$.

One particular approach to solving this minimization problem is known as the \emph{variational Monte Carlo} (VMC) algorithm \cite{Foulkes2001, Toulouse2016, Becca2017}. In the context of VMC simulations
 we take $\rho$ as the Lebesgue measure and  $\Omega=\mathbb{R}^{n_p \cdot D}$, where $n_p$ is the number of particles and $D$ is the number of spatial dimensions. Let $p_\wf$ be the probability density given by
\begin{equation}p_\wf(x)=|\wf(x)|^2/\|\wf\|^2\label{pdef}\,.\end{equation}
%where $\|\cdot\|$ is the $L^2$-norm in the Lebesgue measure. 
In practice $\|\wf\|$ is unknown, and samples $X \sim p_\wf$ are generated using Markov chain Monte Carlo (MCMC); in this work we assume these samples are exact and independent.

The energy of $\wf$ can then be expressed as an expectation with respect to $p_\wf$:
$
\L(\wf)=\EE_{X\sim p_{\wf}}\localE_\wf(X),
$
where
\begin{equation}\label{eqn:deflocalE}
\localE_\wf(x)=\frac{(H\wf)(x)}{\wf(x)}.
\end{equation}
is called the \emph{local energy}.
\begin{problem}[VMC]\label{vmcprob}
Minimize
    \begin{equation}\label{eqn:VMC_L}
        L(\theta)=\EE_{X\sim p_\theta}\localE_\theta(X),
        %L(\theta)=\int_{\Omega}\localE_\theta(x)p_\theta(x)dx
    \end{equation}
    over parameter vectors $\theta\in\mathbb{R}^d$, where $\localE_\theta=H\wf_\theta/\wf_\theta$, and $p_{\theta}(x)=|\wf_{\theta}(x)|^2/\|\wf_\theta\|^2$. We assume query access to samples from $X_i\sim p_\theta$ and the ability to efficiently evaluate $f_\theta,Hf_\theta$, and $\partial f_\theta / \partial \theta$ at a point $X_i$.
\end{problem}

% In the VMC setting, $\Omega$ is the classical Eulidean space, $\rho$ is the Lebesgue measure, $\mathcal{H}$ is the space of classical $L^2$ functions.

\subsection{Contribution}\label{sec:ct}


\begin{enumerate}

% \item  To obtain an unbiased gradient estimator for the supervised learning problem (Problem \ref{pro:slos}) would require computing the exact norm of $\|f_\theta\|$, and we do not have access to this quantity. To overcome this difficulty 
% we provide an estimator $G$ which is \emph{directionally unbiased}, meaning that $\EE[G]$ is a positive multiple of $\nabla_\theta L(\theta)$.
% We prove that the convergence rate of the SGD algorithm with this gradient estimator (Algorithm \ref{alg:GE}) matches that of classical SGD (see Corollary \ref{cor:super}).

\item  We provide a new algorithm for \cref{pro:slos} which can effectively optimize $f$ in the absence of the access to the exact norm $\|f_\theta\|$. The algorithm is based on a new estimator $G$ for the loss gradient, which is \emph{directionally unbiased} in that $\EE[G]$ is a positive multiple of $\nabla_\theta L(\theta)$. We prove that the convergence rate of the SGD algorithm with this gradient estimator (Algorithm \ref{alg:GE}) matches that of classical SGD (see Corollary \ref{cor:super}).

\item We prove for the first time that the convergence rate of SGD (Algorithm \ref{alg:VMC}) for VMC simulations (Problem \ref{vmcprob}) matches that of classical SGD (see Corollary \ref{cor:VMC}) under proper assumptions. 

% Despite the wide usage of VMC, to our knowledge, this is the first work 
% showing that the convergence rate of the SGD algorithm can match that of classical SGD 
% the convergence of such algorithms has not been presented in the literature. Under proper assumptions, we show that Algorithm \ref{alg:VMC} has the same convergence rate as classical SGD .


%introduces additional difficulties in this step. 
%After diving into the calculation, we successfully demonstrate the boundedness under proper assumptions. 
%We also numerically verify the validity of our assumption in the real training process (see Section \ref{sec:Exp}).\LL{ check} 

\item In both cases we provide new bounds on the Lipschitz constant of $\nabla L$ and the variance of the gradient estimator, which is a key step in proving the convergence.

\end{enumerate}


%The lack of access to the overall normalization factor introduces difficulties in the analysis of both cases. To resolve these difficulties, we introduce a technique to bound the ratio of normalization factors  under proper assumptions. This allows us to bound the Lipschitz constant of $\nabla L$ and the variance of the gradient estimator, which is a key step in proving the convergence of SGD type algorithms.
% 
% \begin{itemize}
%     \item We and prove the convergence analysis of our algorithm. These losses can be seen as a general framework for different optimization problems in the real application. 
%     \item We give the complexity proof of the VMC algorithm. VMC algorithm is a popular and useful algorithm in the quantum community while, to the best of our knowledge, the convergence of it has never been studied before. 
% \end{itemize}


\subsection{Background and related works}
% 
% The literature related to our results fall into three main categories: Scale-invariant loss functions, SGD on manifolds, and variational Monte Carlo with neural networks for quantum chemistry.

%\LL{ the information in this section needs to be consolidated.} 
%In this section, we give a review for the previous work. We emphasize that although the setting in our paper seems to be similar to some previous works, there are essential differences that matter a lot in the design of algorithm and theoretical result.



\subsubsection{Minimizing scale-invariant objectives}\label{re:msfd}

%A function $L(\v)$ of parameters $\v\in\RR^n$ is scale-invariant if $$
%Comparing our setting with the previous work, the closest one is the training of scale-invariance loss in parameter, meaning that $L(C\theta)=L(\theta)$ for any $C>0$. 

Using scale-invariance in the parameter to stabilize the training of SGD is a well-studied technique. Many structures and methods have been proposed along these lines, such as normalized NN~\cite{Ioffe_2015,Wu_2018,Ba_2016}, $\mathcal{G}$-SGD~\cite{meng_mathcalg-sgd_2021}, 
SGD+WD~\cite{van_2017,arora2018theoretical,li_robust_2022,Wan_2021}, and projected/normalized (S)GD~\cite{Saliman_2016,kodryan_training_2022}. 
In these previous works, the parameter $\v$ is stored explicitly and $\L$ is scale-invariant in the parameter $\v$, meaning $\L(\lambda\v)=\v$ for any $\lambda>0$. 

% and the training of $\v$ can be seen as a training process on the sphere.
%In this paper, we consider the infinite-dimensional analogue $\L(f)$ of $L(\v)$, where $f_\theta$ is parameterized by finite-dimensional weight vector $\theta$.
%This difference makes the gradient calculation and analysis of the algorithm much harder. 
% For $\gamma=1$ the gradient formula of \cite{Saliman_2016} evaluates to
% \begin{equation}\label{findim}
% \nabla_\v \ell(\v)=\frac{1}{\|\v\|}\nabla\tilde\ell(\bar\v)-\frac{\nabla\tilde\ell(\bar\v)\cdot \v}{\|\v\|^3}\v.
% \end{equation}
% \cref{rem:connect} shows how we can relate the gradient of \cref{pro:slos} to \cref{findim}.

Our setting differs from these works in that we re-parameterize $L(\theta)=\L(\v_\theta)$ (in our setting we refer to this $\v_\theta$ as $f_\theta$). %This small change introduces many extra difficulties. 
Using the chain rule, the gradient of $L$ can then be written as
\begin{align}\label{chainrule}
\nabla_\theta L(\theta)
&=[\nabla_\v \L](\v_\theta)\cdot \nabla_\theta\v_\theta,
\end{align}
where the dot product is along the dimension of $\v$
(see \cite{Saliman_2016} for an expression for $\nabla_\v\L$). Expanding \cref{chainrule} yields an expression with factors of $\|\v_\theta\|$, and the analogous quantity $\|f_\theta\|$ is not available in our setting.

%As already mentioned in Section \ref{sec:ct}, to numerically approximate $\nabla L(\theta)$, we need to exactly know the norm of $\v_\theta$. In the previous work, this norm is always available because it is the norm of the parameter $\v$. However, in our case, $\|\v_\theta\|$ is the norm of our neural network function and it is impossible to obtain an accurate estimate of it without going through the entire data set. Theoretically, this lack of information and the extra product term $\nabla_\theta\v$ also complicatesclever the analysis of the algorithm. 
%For example, if $l$ is the square loss, then $\nabla_\v \L$ is linear and Lipschitz. This property is no longer straightforward after replacing $\v$ by  $\v_\theta$, and showing the Lipschitz property of $L$ becomes a difficult task. 


\subsubsection{SGD on Riemannian manifolds}\label{re:mnf}
%Finally, it is worth mentioning that, after assuming the loss function is scale-invariant in the parameter $\v$, 
The training of a scale-invariant objective can be seen as a training process on the sphere, which is a special case of learning on Riemannian manifolds. The convergence of SGD on Riemannian manifolds has already been well-studied. \cite{bonnabel_stochastic_2013} shows the asymptotic  convergence of SGD on Riemannian manifolds such as the sphere. Further variations of the SGD algorithm on Riemannian manifolds have been proposed and studied, such as SVRG on Riemannian manifolds~ \cite{Zhang_2016}, \cite{Sato_2019} and ASGD on Riemannian manifolds~\cite{tripuraneni_averaging_2018}. To the best of our knowledge, all previous works ensure the parameter stays on the manifold by either taking stochastic gradient steps directly on the manifold or projecting them back onto it. This is in contrast to our setting in which we have a parameterization of $f_\theta\in\H$ and take stochastic gradient steps directly in the parameter space. 



\subsubsection{Supervised learning of quantum states} 
%\begin{itemize}\item 

Supervised learning has been used to study the power of neural networks for representing quantum states \cite{Cai2018, Carleo2019}. 
%This supervised learning of a quantum state corresponds directly to the problem of supervised learning on the implicitly parameterized sphere. 
Supervised learning is also used in the pretraining step of the FermiNet \cite{Pfau_2020} and related works \cite{Spencer_2020, Gerard2022, Glehn2022} in order to stabilize and accelerate subsequent VMC simulations.
% Although the pretraining procedure used in these works does not directly match our formulation of supervised learning on the implicitly parameterized sphere, but it is similar in spirit and our procedure could provide an alternative, more general form of pretraining for the FermiNet or other neural quantum state ansatzes.


%\item 

%\end{itemize}

\subsubsection{Optimization methods in variational Monte Carlo simulation}

% The success of VMC with neural quantum states has inspired several recent works that aim to study the theoretical and practical capacity of neural networks to represent quantum states and related function classes \cite{HanLiLinEtAl2019, SannaiTakaiCordonnier2019, KerivenPeyre2019, Hutter2020, Jeff_2021, Nilin2022}. 

Historically, the number of parameters used in VMC simulations was relatively small, and methods such as stochastic reconfiguration  \cite{Sorella98, Sorella2001} and the linear method \cite{Nightingale2001, toulouse2007} were preferred for parameter optimization. Recently, neural networks have been used to parameterize the wave function in an approach known as \emph{neural quantum states} \cite{Carleo2017}. Subsequent works have shown that VMC with neural quantum states can match or exceed state-of-the-art high accuracy quantum simulation methods on a wide variety of problems in quantum many-body physics and chemistry \cite{Nomura2017, Choo2018, Nagy2019, Luo2019, Han2019, Yang2020, Hermann2020, Pfau_2020, StokesRobledo2020, Choo2020, Glehn2022, Gerard2022, Hermann_review}. Due to the complexity and the large number of parameters in neural quantum states, first order methods are usually the only practical option for their optimization. In this context, our work explores the performance of the simplest first order stochastic optimization method for VMC simulations. 


%\begin{longcomment}
%\NA{brainstorming}
%\begin{itemize}
%\item making optimization on a manifold into an unconstrained one
%\item framing VMC as as SGD problem
%\end{itemize}
%\end{longcomment}
%\begin{temp}
%The main contribution of our work can be summarized in the following two points:
%\begin{itemize}
%    \item We propose a new algorithm (Algorithm \ref{alg:GE}) to train the scale-invariant loss \eqref{eqn:Ltheta} in Problem \ref{pro:slos} and prove the convergence analysis of our algorithm. These losses can be seen as a general framework for different optimization problems in the real application. 
%    \item We give the complexity proof of the VMC algorithm. VMC algorithm is a popular and useful algorithm in the quantum community while, to the best of our knowledge, the convergence of it has never been studied before. 
%\end{itemize}
%
%\NA{moved from related work:}
%
%\begin{temp}
%In contrast to the established literature which considers scale-invariance in the parameter space to stabilize training, we are instead interested in scale-invariance in an infinite-dimensional intermediate space which represents the wave function. In particular, we do not have direct access to $\|f\|$ as is required to compute the pullback $\tilde\L$ but will instead need to estimate it by sampling. 
%\end{temp}
%
%\NA{moved from related work:}
%We emphasize that, different from finite dimensional scale-invariant training, it is difficult to directly apply SGD in our setting to obtain an unbiased gradient estimator for $\nabla_\theta L(\theta)$. There are two reasons: 1. The denominator of $\delta\L_f$ depends on the norm of $\|f\|$. Generally speaking, we need to use a lot of data points to obtain an accurate approximation to it; 2. The inner production of $\delta\L_{f_\theta}$ and $\nabla_\theta f_\theta$ produces some nonlinear term, which might make the design of SGD difficult. To overcome these two difficulties, in Section \ref{sec:Duge}, we need to carefully design the algorithms and the gradient estimators for Problems \ref{pro:slos}, \ref{vmcprob}.
%
%
%
%We propose an algorithm for Problem \ref{pro:slos} in Section \ref{sec:pbgea}. This minimization problem can be seen as a general framework for different optimization problems and accelerate the training process:
%\begin{itemize}
%    \item Consider the following optimization problem:
%    \begin{equation}\label{super_l_C}
%    \min_{C\in\mathbb{R},\theta\in\mathbb{R}^d}L(C,\theta)=\|Cf_\theta-g\|^2\,.
%    \end{equation}
%    For fixed $\theta$, we have $\mathrm{argmin}_{C\in\mathbb{R}}L(C,\theta)=\|Cf_\theta-g\|^2=\ol{g}{f_\theta}/\|f_\theta\|^2$. Plugging this into $L(C,\theta)$, minimizing \eqref{super_l_C} is equivalent to minimizing \eqref{eqn:Ltheta}.
%
%    As will be shown in Sections \ref{sec:pbgea} and \ref{sec:alg_2_ana}. One advantage of directly minimizing \eqref{eqn:Ltheta} is that the whole process is scale-invariant in $f_\theta$. In particular, as shown in Section \ref{sec:alg_2_ana}, using Algorithm \ref{alg:GE}, we can use the same learning rate to obtain the same convergence rate if $f_\theta$ is rescaled to $cf_\theta$ with $c>0$. In real application, this is very useful. In some cases (see the second point below), $f_\theta$ might become very small during the training process, meaning that $|f_\theta|\ll |g|\sim O(1)$. This implies, to achieve a first-order stationary point, the number $C$ should be pretty large. On the other hand, because of the noise in SGD, we often choose smaller and smaller learning rates in the iteration to stabilize the training process. Taking these two facts into consideration, we need to run a large number of iterations to achieve the stationary point, which greatly increases the cost of the training.
%
%\end{itemize}
%
%\end{temp}

% \section{Sketch of algorithm and main results}

% %\NA{I think it would be good to let $\eta_m=1/\sqrt m$ and use Cauchy-Schwartz on the LHS of} \eqref{eqn:gradient_F_bound} 
% %\NA{to get $\EE[\operatorname{average}(|\nabla_\theta L|^4)]\le \log M/M$. I think the statement is cleaner if we don't specify a stopping time}\ZD{I agree.}



% We give variants of stochastic gradient descent (SGD) for Problems \ref{pro:slos} and \ref{vmcprob}. The SGD algorithm $\algsym$ can be stated as follows and depends on a problem-dependent distribution $P$. In the VMC setting we allow $P$ to depend on the state of the algorithm at step $m$:
% \begin{itemize}
%     \item (Initialization): Pick an initial parameter $\theta_0\in\RR^d$.
%     \item (Iterative updates): 
%     For each $m\in\NN_0$ $\algsym$ receives a sample $S_m\sim P$ and computes $G_m=\algsym(S_1,\ldots,S_m;\theta_0,\ldots,\theta_m)$ which represents an estimate of $\nabla L(\theta_m)$. $\algsym$ then defines
%     \[\theta_{m+1}=\theta_m-\eta_m G_m.\]
%     %Given $M>0$. For each $0\leq m\leq M-1$,
%     %$\theta_{m+1}=\theta_m-\eta_m G_m$,
%     %where $G_m$ is a gradient estimator for $\nabla L(\theta)$ and $\{\eta_m\}_{m=0}$ is the learning rate.
% \end{itemize}
% To complete the description of the SGD algorithm we need to specify the distribution $\rho$ and the gradient estimator for each problem. In each case we let $P$ be of the form $P=p^{\otimes r}$ for a for a small constant $r=O(1)$. In particular, $r=2$ works, i.e., $S_m=(X^m_1,X^m_2)$ where $X^m_1,X^m_2\overset{\text{iid}}{\sim}p$.
% \begin{itemize}
% \item 
% In the supervised learning setting $p=\rho$. That is, we sample $X^m_i$ from the probability measure defining the inner product on $\H$.
% \item
% In the VMC setting $p=p_{\theta_m}$. That is, we get samples $X^m_r$ from the $\theta_m$-dependent distribution $p_{\theta_m}=|\wf_\theta|^2/\|\wf_\theta\|^2$.
% \end{itemize}


% %
% %
% %We give variants of stochastic gradient descent (SGD) for Problems \ref{pro:slos} and \ref{vmcprob}. In both cases, we focus on the loss function $L(\theta)$ for $\theta\in\mathbb{R}^d$, which is induced by the scale invariant loss $\L(f)$ for $f\in\H$. Note that $L(\theta)$ is not scale invariant with respect to $\theta$.
% %
% %The SGD algorithm $\algsym$ can be stated as follows and depends on a problem-dependent distribution $P$. In the VMC setting we allow $P$ to depend on the state of the algorithm at step $m$:
% %\begin{itemize}
% %    \item (Initialization): Pick an initial parameter $\theta_0\in\RR^d$.
% %    \item (Iterative updates): 
% %    For each $m\in\NN_0$ $\algsym$ receives a sample $S_m\sim P$ and computes $G_m=\algsym(S_1,\ldots,S_m;\theta_0,\ldots,\theta_m)$ which represents an estimate of $\nabla_\theta L(\theta_m)$. $\algsym$ then defines
% %    \[\theta_{m+1}=\theta_m-\eta_m G_m.\]
% %    %Given $M>0$. For each $0\leq m\leq M-1$,
% %    %$\theta_{m+1}=\theta_m-\eta_m G_m$,
% %    %where $G_m$ is a gradient estimator for $\nabla_\theta L(\theta)$ and $\{\eta_m\}_{m=0}$ is the learning rate.
% %\end{itemize}
% %To complete the description of the SGD algorithm we need to specify the distribution $\rho$ and the gradient estimator for each problem. 
% %
% %% In each case we let be of the form $P=\rho^{\otimes r}$ for a for a small constant $r=O(1)$. In particular, $r=2$ works, i.e., $S_m=(X^m_1,X^m_2)$ where $X^m_1,X^m_2\overset{\text{iid}}{\sim}\rho$.
% %\begin{itemize}
% %\item 
% %In the supervised learning setting we sample $X^m_i$ from the probability measure $\rho$ defining the inner product on $\H$.%, i.e., $\ol{f}{g}=\int fgd\rho$
% %\item
% %In the VMC setting we get samples $X^m_r$ from the $\theta_m$-dependent distribution $\rho=p_{\theta_m}=|\wf_\theta|^2/\|\wf_\theta\|^2$.\ZD{Need to revise this part. In the VMC case, this $\rho$ is different from the $\rho$ in the definition of $\mathcal{H}$.}
% %\end{itemize}

% \subsection{Directionally unbiased gradient estimators}\label{sec:Duge}
% Finding a good SGD algorithm depends on constructing unbiased gradient estimators $G$.

% For \cref{vmcprob} (VMC) we show that having access to samples from $p_\theta(x)=|\wf_{\theta}(x)|^2/\|\wf_\theta\|^2$ allows us to directly construct an unbiased gradient estimator $G_m$ for $\nabla_\theta L(\theta_m)$, and the complexity result is consistent with the result of classical SGD. 

% In contrast, the supervised learning setting of Problem \ref{pro:slos} exhibits a subtlety which separates our setting from that of finite-dimensional scale-invariance: The lack of an unbiased estimator for $1/\|f\|$ and $1/\|f\|^3$ prevents us from obtaining an unbiased estimate for $\nabla_\theta f$. We therefore adjust our aim and show that it suffices to have an estimator whose \emph{direction} is unbiased, in the sense that $\EE[G]\in[\nabla_\theta f]$, while its length is only required to be of the correct order of magnitude.

% %In the supervised learning setting we will observe that we have some leeway in the scaling of $G$. In particular, we observe a subtlety in passing from the setting of scale-invariance on the parameter space to the parameterized infinite-dimensional setting: To estimate the gradient \cref{deltheta} we depend on an approximate estimate of $\|f\|$ which leads to a biased estimate of $1/\|f\|$. it then becomes important to balance the two terms in the stochastic gradient estimate so that this bias becomes a single multiplicative factor. \NA{add numerics for the unbalanced vs balanced}
% %
% %%For the VMC setting we will show that it is possible to find a cheap way to construct an unbiased estimator for $\nabla_\theta L(\theta)$ using the special structure of \eqref{eqn:VMC_L}.
% %%
% %%
% %%The detailed discussion of these algorithms is put in Section \ref{sec:Duge}.\hide{We emphasize that} The constructions of the gradient estimator $G_m$ for Problems \ref{pro:slos}, \ref{vmcprob} are very different. 
% %
% %The proper construction of $G_m$ for Problem \ref{pro:slos} is much more difficult. After some calculation (see in Section \ref{sec:pbgea}), we can show that, to obtain an unbiased estimator for $\nabla_\theta L(\theta)$ in Problem \ref{vmcprob}, we must estimate $\|f_\theta\|$ accurately, which is already an expensive task. To overcome this difficulty, we search for a partially biased gradient estimator for Problem \ref{pro:slos}. In particular, in Section \ref{sec:pbgea}, we construct a gradient estimator that is unbiased in the direction, meaning that $\mathbb{E}(G_m)$ is in the same direction as $\nabla_\theta L(\theta)$. Using this new gradient estimator and under proper assumption, we can also show similar convergence result as Problem \ref{pro:slos}.

% \subsection{Formal results}
% We prove that SGD converges in the scale-invariant supervised learning problem and in the VMC setting: 
% %Our direction-unbiased gradient estimators allow us to prove convergence of 

% \begin{theorem}\label{shortmainthm1}
% Suppose that
% \[\|f_\theta\|_4/\|f_\theta\|,\|\nabla f_\theta\|_4/\|gf_\theta\|,\|\mathsf{H}_\theta f_\theta\|_4/\|g_\theta\|\]
% are uniformly bounded by a constant $C_g$ and let $\epsilon>0$. Suppose that we have an estimator for $\|f_\theta\|$ such that $Z_i=\Theta(\|f_{\theta_i}\|)$ for each step $i$. Then there exists a way to construct $G_m$ so that when $M=\Theta(1/\epsilon^4)$, the average of $\EE[|\nabla_\theta L|]$ over the first $M$ steps is at most $\epsilon$.
% \end{theorem}

% \begin{theorem}\label{shortmainthm2}
% Assume that $\|H\wf_\theta/\wf_\theta\|_{4,p_\theta}$, $\|\nabla H\wf_\theta/\wf_\theta\|_{4,p_\theta}$, $\|\nabla\wf_\theta/\wf_\theta\|_{4,p_\theta}$,  $\|\mathsf{H}_\theta\wf_\theta/\wf_\theta\|_{2,p_\theta}$
% are all uniformly bounded by a constant $C_\wf$, and let $\epsilon>0$. Then there is $M=\Theta\left(1/{\epsilon^4}\right)$
% such that the average of $\EE[|\nabla_\theta L|]$ over the first $M$ steps of SGD is at most $\epsilon$.
% \end{theorem}

% The full statements are given in \cref{thm:main_result_alg1,thm:complexity_VMC} and their corollaries.
%\LL{ Combine sec 3 and 4?}
%done


%\LL{ Combine sec 3 and 4?}
%done

%\NA{revise}


%  \subsection{Unbiased gradient estimator for scale-invariant supervised learning}
%  \ZD{Can we comment out this section? I am not sure why we put it here and why it is useful?}
 
%  \NA{it gives the direction-unbiased gradient estimator with smaller variance than}\cref{eqn:G_r_theta}

% \begin{temp}
%  In this section, we first give a simple formulation for unbiased gradient estimator.

%  Given $r\ge2$ i.i.d. samples $X_i\sim\rho$. We write $f_i=f(X_i)$, $g_i=g(X_i)$, and $\nabla_\theta g_i=\nabla_\theta g_\theta(X_i)$.
%  %, and use the counting measure to define
%  \begin{equation}
%  \begin{aligned}
%  \ol{f}{g}_r=\frac1r\sum_{i=1}^rf_i g_i,
%  \qquad
%  \|f\|_r^2=\frac1r\sum_{i=1}^rf_i^2.
%  \end{aligned}
%  \end{equation}
%  \begin{lemma}\label{unbiasedsupervised}
%  Given samples $X_1,\ldots,X_n$, let
%  \begin{align}
%  G_n&=\frac1{r-1}\sum_{j=1}^r a_j\nabla f_j,
%  \\
%  %a_j&=(\tfrac1r\sum_{i=1}^r f_i^2)g_j-(\frac1r\sum_{i=1}^rg_if_i)f_j.
%  a_j&=-\|f\|_r^2\:g_j+\ol{g}{f}_r\:f_j.
%  \end{align}
%  %\[
%  %\begin{aligned}
%  %%G_n&=-\frac{n}{n-1}(\|g\|_n^2\ol{f}{\nabla g}_n-\ol{f}{g}_n\ol{g}{\nabla g}_n)
%  %&=-\frac1{n(n-1)}\sum_{i=1}^n\sum_{j=1}^n(g_i^2f_j\nabla g_j-f_ig_ig_j\nabla g_j).
%  %\end{aligned}
%  %\]
%  Then $G_r$ is an unbiased estimator for $G=\frac12\|f_\theta\|^3\nabla f_\theta$.
%  \end{lemma}
%  \begin{proof}
%  \cref{coordinatewise} states that
%  \[G=-\|f\|^2\ol{g}{\nabla f}+\ol{g}{f}\ol{f}{\nabla f}.\]
%  For each $i\neq j$ we have an unbiased estimator for $G$ given by
%  \[-f_i^2g_j\nabla f_j+g_if_if_j\nabla f_j.\]
%  Taking the average over all pairs $i\neq j$ gives another unbiased estimator for $G$:
%  \[-\frac1{n(n-1)}\sum_{i\neq j}(f_i^2g_j\nabla f_j-g_if_if_j\nabla f_j).\]
%  We may add the terms $i=j$ without changing the value because all such terms evaluate to $0$.
%  \end{proof}
% \end{temp}

\section{Supervised learning on sphere}\label{sec:slos}

%\ZD{I change the $Z$ to $\widetilde{Z}$ in this section to show it's an approximation. Please check.}

%In this section, we consider solving the Problem \ref{pro:slos}. 
%The gradient of the loss 
In order to adapt the SGD algorithm to the supervised learning problem \ref{pro:slos} we begin by writing down the gradient of the loss function \eqref{eqn:Ltheta}:
\begin{equation}\label{gradient_L_supervised}
\begin{aligned}
&\nabla_\theta L(\theta)=-\frac{\functional{g}{\nabla_\theta f_\theta}}{\|f_\theta\|}+\frac{\functional{g}{f_\theta}\functional{f_\theta}{\nabla_\theta f_\theta}}{\|f_\theta\|^3}\,.
\end{aligned}
\end{equation}


\begin{remark}
Given an estimate $\tilde{Z}$ of $\|f_\theta\|$ and samples $X_1,X_2\sim\rho$, let $f_i=f_\theta(X_i)$, $\nabla_\theta f_i=\nabla_\theta f_\theta(X_i)$, and $g_i=g(X_i)$. Then \cref{gradient_L_supervised} suggests a plug-in estimate of $\nabla f_\theta$ given by
\begin{equation}\label{eq:plugin}
-\frac{g_1\nabla_\theta f_1}{\tilde{Z}}+\frac{g_2f_2f_1\nabla_\theta f_1}{\tilde{Z}^3}\,,
\end{equation}
where the product in the second term uses two samples so that its expectation factors into a product.
 
However, this turns out to be a poor gradient estimator. In fact, \cref{fig:balance} demonstrates that when using this gradient estimate in the SGD algorithm, $f_\theta$ fails to converge to the target. The reason for this failure appears to be the fact it is \emph{unbalanced} in the sense that each term has a different power of $1/\tilde{Z}$. This biases the direction of \cref{eq:plugin}--its expectation is not parallel with $\nabla_\theta L$ unless the two terms of \cref{gradient_L_supervised} are parallel or $\tilde{Z}=\|f_\theta\|$ exactly.
\end{remark}



\subsection{Directionally unbiased gradient estimator}
By balancing $\tilde{Z}$ in each term we obtain a replacement for \cref{eq:plugin} which is  \emph{directionally unbiased} meaning that $\EE[G]$ is in the positive span of $\nabla_\theta L$. In the case $n=2$ our directionally unbiased estimator takes the form
\begin{equation}\label{eq:unbiased}
G=\frac{-f_2^2g_1\nabla_\theta f_1+g_2f_2f_1\nabla_\theta f_1}{\tilde{Z}^3}\,.
\end{equation}

Define $Z_\theta=\|f_\theta\|$. The general form for minibatch sizes $n>2$ follows from the lemma below:
\begin{lemma}\label{lem:unbiased_super}
Given $n\ge2$ and i.i.d. samples $X_i\sim\rho$, let $\ol{f}{g}_n=\frac1n\sum_{i=1}^nf(X_i)g(X_i)$ and $\|{f}\|_n^2=\frac1n\sum_{i=1}^nf(X_i)^2$, and define
\begin{equation}\label{eqn:G_unbiased}
G=\frac{1}{Z^3_\theta}\frac1{n-1}\sum_{j=1}^n a_j\nabla f_{\theta}(X_j)\,,
\end{equation}
where $a_j=-\|f_{\theta}\|_n^2\:g(X_j)+\ol{g}{f_{\theta}}_n\:f_{\theta}(X_j)$. Then $\mathbb{E}_X(G)=\nabla_\theta L(\theta)$.
\end{lemma}

The proof can be found in Appendix \ref{sec:minibath_sup}. Given an estimator $\tilde{Z}$ for $Z_\theta$  we then choose the gradient estimator: %This lemma implies, if $Z_\theta$ is known, for each fixed $\theta$, it is possible to construct a cheap unbiased estimation of $\nabla_\theta L(\theta)$.

\begin{equation}\label{eqn:G_r_theta}
G=\frac{1}{\tilde{Z}^3}\frac1{n-1}\sum_{i=1}^n a_i\nabla f_{\theta}(X_i)\,,
\end{equation}
where %$Z$ is a constant and 
$\{a_i\}^n_{i=1}$ are defined in Lemma \ref{lem:unbiased_super}. Then
\begin{equation}\label{eqn:G_r_theta_biased}
\mathbb{E}_{r_1,r_2}\left(G(\theta)\right)=\frac{Z^3_\theta}{\tilde{Z}^3}\nabla_\theta L(\theta)\,.
\end{equation}


%To show that \cref{eqn:G_r_theta} in directionally unbiased, define $Z_\theta=\|f_\theta\|$. Suppose we already know $Z_\theta$, then we can use the following lemma to construct a unbiased gradient estimation:


%\subsection{Directionally unbiased gradient estimation and algorithm}\label{sec:pbgea}
%
%As discussed before, to use \eqref{eqn:G_unbiased}, we first need to estimate the denominator $Z_\theta$, which seems a difficult and expensive task in general.
%
%What if we use a partially biased gradient estimation? In the classical gradient descent, one reason to use the gradient to move the parameter is that it is the fastest decaying direction of the loss function and it guarantees the decaying of the loss function after movement. In our case, if we can find a gradient estimator that is unbiased in the direction, this estimator still guarantees the similar decaying property of classical gradient descent in the expectation sense. In particular, for any given $\theta$, we define
%This means that, after taking the expectation, the approximated gradient $G$ is in the same direction as $\nabla_\theta L$.
Using \eqref{eqn:G_r_theta_biased} and Taylor's expansion, we obtain that if
\[
\theta_{m+1}=\theta_m-\eta_mG(\theta_m)\,,
\]
with $\frac{\eta_mZ^3_\theta}{\tilde{Z}^3}$ small enough,
\[
\mathbb{E}_{r_1,r_2}\left(G(\theta_{m+1})\right)\lesssim G(\theta_{m})-\eta_m \frac{Z^3_\theta}{\tilde{Z}^3}|\nabla_\theta L(\theta_m)|^2\,,
\]
which implies the decaying of the loss function in the expectation sense. 

In the above derivation, the choice of $\eta_m$ and the decaying rate of the loss function depends on the ratio $Z_{\theta_m}/\widetilde{Z}$. In practice, if we do not have a proper choice of preconditioner $\widetilde{Z}$, it is hard to give a reasonable prior choice of $\eta_m$ that guarantees fast convergence to the first-order stationary point. Inspired by the idea of variance reduction techniques~\cite{SAGA-2013,SAGA-2014,Johnson_Zhang}, to give a reasonable preconditioner for $Z_{\theta_m}$, we calculate an estimate for $Z_{\theta_m}$ only once for every $N$ steps, where $N>0$ is the length of each epoch. Between the epochs, per step, we set the preconditioner $\widetilde{Z}\approx Z_{\lfloor m/N\rfloor*N}$ so that the ratio between $\widetilde{Z}$ and $Z_{\theta_m}$ remains bounded. 

The detailed algorithm is summarized in Algorithm \ref{alg:GE}.
\begin{breakablealgorithm}
      \caption{Stochastic gradient descent algorithm for supervised learning}
  \label{alg:GE}
  \begin{algorithmic}[1]
  \STATE \textbf{Preparation:} $g_n$; $f_\theta(x)$; number of iterations: $M$; learning rate $\eta_m$; initial parameter: $\theta_0$; $N$; $n$
  \STATE \textbf{Running:}
  \STATE $m\gets 0$;
  \WHILE{$m\leq M$}
  \IF{$m$ is a multiple of $N$}
  \STATE $\widetilde{Z}_{m}\gets \text{an estimation of}\ \|f_{\theta_m}\|$; 
  \ENDIF
  \STATE Sample $\{X^m_i\}^{n}_{i=1}$ independently from $\rho$;
  \STATE $a_i\gets-\|f_{\theta_m}\|_n^2\:g(X_i)+\ol{g}{f_{\theta_m}}_n\:f_{\theta_m}(X_i)$
  \STATE $G_{m}\gets \frac{1}{\widetilde{Z}_{\lfloor m/N\rfloor*N}^3}\frac1{n-1}\sum_{i=1}^n a_i\nabla f_{\theta_m}(X_i)$;
  \STATE $\theta_{m+1}\gets \theta_m-\eta_m G_m$;
  \ENDWHILE
    \STATE \textbf{Output:} $\theta_m$;
    \end{algorithmic}
\end{breakablealgorithm}

\begin{remark} To obtain an estimation of $\|f_{\theta_m}\|$, we can sample $J$ data points $\{X_j\}^J_{j=1}$ independently from the distribution $\rho$ and use $\sqrt{\frac{1}{J}\sum^J_{j=1}|f_\theta(X_j)|^2}$ to approximate $\|f_\theta\|$ with large enough $J$. Because $J$ needs to be large enough to ensure a good approximation, it is very expensive to use this approximation in every step. In the above algorithm, since we only use this estimation once for every $N$ steps, the average number of evaluations of $f$ is $J/N$, which is acceptable if $N$ is large enough.
\end{remark}



\subsection{Analysis of Algorithm \ref{alg:GE}}\label{sec:alg_2_ana}

We now study the complexity of Algorithm \ref{alg:GE}. We first propose some additional assumptions:
\begin{assumption}\label{assum_g_scale_inva} There exists a constant $C_f$ such that for any $\theta\in\mathbb{R}^d$
\begin{itemize}
    \item (Value bound): 
    \begin{equation}\label{eqn_value_gnbound}
    \left\|f_\theta^2\right\|^{1/2}/\|f_\theta\|\leq C_f\,.
    \end{equation}
    \item (Gradient bound): 
    \begin{equation}\label{eqn_Gradient_gnbound}
    \left\|\left|\nabla_\theta f_\theta\right|^2\right\|^{1/2}/\|f_\theta\|\leq C_f\,.
    \end{equation}
    \item (Hessian bound): 
    \begin{equation}\label{eqn_Hessian_gnbound}
    \left\|\left\|\mathsf{H}_\theta f_\theta\right
    \|^2_2\right\|^{1/2}/\|f_\theta\|\leq C_f\,,
    \end{equation}
    where $\mathsf{H}_\theta f$ is the hessian of $f$.
\end{itemize}

\end{assumption}
Under the Assumption \ref{assum_g_scale_inva}, we establish the Lipschitz property of $\nabla L$ and provide bounds for the variance of the gradient estimator (see \cref{lem:svsl_2}). These results play a crucial role in demonstrating the convergence of our method.~\ZD{I add a sentence here.}
We note that Assumption \ref{assum_g_scale_inva} is scale-invariant in $f_\theta$, meaning the same assumption holds with the same constant for any $\lambda f_\theta$, where $\lambda>0$. 

Now, we are ready to introduce our complexity results for Algorithm \ref{alg:GE}:
\begin{theorem}\label{thm:main_result_alg1}
Assume the Assumption \ref{assum_g_scale_inva} holds true,  $|g_n|\leq C_g$ for all $1\leq n\leq N$, and 
\begin{equation}\label{eqn:distance_bound}
\frac{1}{C_r}\leq \frac{\|f_{\theta_m}\|}{\left\|f_{\theta_{\lfloor m/N\rfloor*N}}\right\|},\frac{\widetilde{Z}_{\lfloor m/N\rfloor*N}}{\left\|f_{\theta_{\lfloor m/N\rfloor*N}}\right\|}\leq C_r\,,\quad \mathrm{a.s.}
\end{equation}
for all $m\in\mathbb{N}$, where $\theta_m$ comes from Algorithm \ref{alg:GE}. There exists a constant $C$ that only depends on $C_r,C_f,C_g$ such that, when $\eta_m<\frac{1}{C}$ for any $m\geq 0$, 
\begin{equation}\label{eqn:gradient_F_bound}
\sum^M_{m=0}\eta_m\mathbb{E}\left(\left|\nabla_\theta L(\theta_m)\right|^2\right)\leq 2C^2_rL(\theta_0)+C\sum^M_{m=0}\frac{\eta^2_m}{n}\,.
\end{equation}
for all $M>0$.
\end{theorem}
We put the proof of the above theorem in Appendix \ref{sec:pf_of_thm:main_result_alg1}. In the proof, we first bound the $\mathbb{E}|G_m|^2$ and show that  $\nabla L$ has a uniform Lipschitz constant. Then, the decay rate of the loss function in each step can be lower bounded by $|\nabla L|^2$, which finally gives us an upper bound of $|\nabla L|^2$ as shown in \eqref{eqn:gradient_F_bound}.

In the above theorem, \eqref{eqn:distance_bound} is an important and natural assumption to show the convergence of Algorithm \ref{alg:GE}. As discussed before, to avoid calculating $\|f_\theta\|$ in each iteration, we use $\widetilde{Z}_{\lfloor m/N\rfloor*N}$ as a preconditioner to replace $Z_m$ in the denominator. This approximation is reasonable only when the preconditioner is not too far away from $Z_m$. Thus, it is natural to expect that we need \eqref{eqn:distance_bound} to prove a fast convergence rate of Algorithm \ref{alg:GE}. Besides, we note that, according to Appendix \ref{sec:svsl_1} Lemma \ref{lem:prior_VMC}, in each epoch of training process, if $\left|\theta_m-\theta_{\lfloor m/N\rfloor*N}\right|$ is bounded ($\eta_m$ is chosen to be small enough) and the estimation of $Z_{\lfloor m/N\rfloor*N}$ is enough accurate, then the assumption \eqref{eqn:distance_bound} is automatically satisfied.

Using Theorem \ref{thm:main_result_alg1}, it is straightforward to show the following corollary: 
\begin{corollary}\label{cor:super} Assume the Assumption \ref{assum_g_scale_inva} and \eqref{eqn:distance_bound} hold true. 
\begin{itemize}
\item Choosing $\eta_m=\Theta\left(n\epsilon^2\right)$ and $M=\Theta\left(1/(n\epsilon^4)\right)$,
we have $\min_{0\leq m\leq M}\mathbb{E}\left(\left|\nabla_\theta L(\theta_m)\right|\right)\leq \epsilon$.

\item Choosing $\eta_m=\Theta\left(\sqrt{\frac{n}{m+1}}\right)$,
we have $\min_{0\leq m\leq M}\mathbb{E}\left(\left|\nabla_\theta L(\theta_m)\right|\right)=O\left(\left(nM\right)^{-1/4}\right)$.

\end{itemize}
\end{corollary}

We note that this bound matches the standard $O(M^{-1/4})$ convergence rate of SGD to first-order stationary point for non-convex functions.

\section{Algorithm and convergence result for VMC}
%\subsection{Unbiased gradient estimator for VMC and algorithm}
In the VMC setting, we assume that the MCMC subroutine is able to sample exactly from the distribution $p_\theta=q_\theta/Q_\theta$ where $Q_\theta=\int q_\theta(x) dx$. This sampling oracle makes it possible to construct a cheap and unbiased gradient estimation for the loss function $L$ \eqref{eqn:VMC_L}.

We first calculate the exact gradient of the loss function  $L$. Different from the supervised-learning case, when computing the gradient, we need to take into account both the change in energy due to the changing local energy function \emph{and} the change in energy due to the changing probability distribution.

Let  $\Lpsi_\theta=\log|\wf_\theta|$. After some calculation (see \cref{gradproofs}), we obtain 
\begin{equation}
\label{eqn:gradient_L_VMC}
\nabla_{\theta} L(\theta)=2 \mathbb{E}_{X\sim p_\theta}\left[\left(\mathcal{E}_{\theta}(X)-L(\theta)\right) \nabla_{\theta}\Lpsi_{\theta}(X)\right].
\end{equation}

Using $p_\theta(x)$ and \eqref{eqn:gradient_L_VMC}, we can obtain the following unbiased gradient estimators for $\nabla_\theta L(\theta)$ (proof in \cref{gradproofs}):

\begin{lemma}\label{lem:unbiased_gradient} Given $n \geq 2$ i.i.d. samples $X_{i} \sim p_{\theta}$, let $\hat{L}(\theta)=\frac{1}{n} \sum_{i=1} \mathcal{E}_{\theta}\left(X_{i}\right)$ and define the gradient estimate
\begin{equation}\label{eqn:VMCgradient_estimator}
G(\theta)=\frac{2}{n-1} \sum_{i=1}^{n}\left(\mathcal{E}_{\theta}\left(X_{i}\right)-\hat{L}(\theta)\right) \nabla_{\theta} \log \wf_{\theta}\left(X_{i}\right) .
\end{equation}
Then $G(\theta)$ is an unbiased estimator of the gradient of the population energy.
%~\ZD{We might move the proof (or even the lemma) to the appendix.}
\end{lemma}

Using the unbiased gradient estimator \eqref{eqn:VMCgradient_estimator} from \cref{lem:unbiased_gradient}, we obtain the following SGD type algorithm to minimize $L(\theta)$:
\begin{breakablealgorithm}
      \caption{Stochastic gradient descent algorithm for variational Monte Carlo}
  \label{alg:VMC}
  \begin{algorithmic}[1]
  \STATE \textbf{Preparation:} number of iterations: $M$; learning rate $\eta_m$; initial parameter: $\theta_0$; number of samples each iteration: $n$;
  \STATE \textbf{Running:}
  \STATE $m\gets 0$;
  \WHILE{$m\leq M$}
  \STATE Sample $\{X^m_i\}^n_{i=1}$ independently according to the density $p_{\theta^m}(x)=\frac{\left|\wf_{\theta_m}(x)\right|^{2}}{\int_{\Omega}\left|\wf_{\theta_m}(x)\right|^{2} d x} $;
  \STATE Compute $G_m$ using Eq. \eqref{eqn:VMCgradient_estimator};
  \STATE $\theta_{m+1}\gets \theta_m-\eta_m G_m$;
  \ENDWHILE
    \STATE \textbf{Output:} $\theta_m$;
    \end{algorithmic}
\end{breakablealgorithm}

\subsection{Analysis of Algorithm \ref{alg:VMC}}

In this section, we study the complexity of Algorithm \ref{alg:VMC}. First, given any $\theta$, we define $\|\cdot\|_{q,p_\theta}$ as the $q$-th norm under the measure $ p_\theta(x)dx$. We need the following assumptions for the theoretical result:
\begin{assumption}\label{assum_g_VMC} There exists a constant $C_{\wf}$ such that for any $\theta\in\mathbb{R}^d$, 
\begin{itemize}
    \item (Value bound): 
    \begin{equation}\label{eqn_value_gnbound_vmc}
    \|H\wf_\theta/\wf_\theta\|_{4,p_\theta}\leq C_\wf\,.
    \end{equation}
    
    \item (Gradient bound): 
    \begin{equation}\label{eqn_Gradient_gnbound_vmc}
    \begin{aligned}
&\|\nabla_\theta H\wf_\theta/\wf_\theta\|_{2,p_\theta}\leq C_\wf\,,\\
&\|\nabla_\theta \wf_\theta/\wf_\theta\|_{4,p_\theta}\leq C_\wf
\,.
\end{aligned}
    \end{equation}
    \item (Hessian bound): 
    \begin{equation}\label{eqn_Hessian_gnbound_vmc}
    \|\mathsf{H}_\theta\wf_\theta/\wf_\theta\|_{2,p_\theta}\leq C_\wf\,,
    \end{equation}
    where $\mathsf{H}_\theta \wf_\theta$ is the hessian of $\wf_\theta$ in $\theta$.
\end{itemize}

\end{assumption}
We note that Assumption \ref{assum_g_VMC} is scale-invariant in $\wf_\theta$. If $\wf_\theta(x)$ satisfies Assumption \ref{assum_g_VMC}, then $\lambda\wf_\theta(x)$ also satisfies the assumption with the same constant $\lambda$ for any $\lambda>0$. We establish new bounds on the Lipschitz constant of $\nabla L$ and the variance of the gradient estimator (see \cref{lem:prior_VMC}), subject to the constraints specified in Assumption \ref{assum_g_VMC}. These bounds are crucial in demonstrating the convergence of our method.~\ZD{I add a sentence here.}

Now, we are ready to give the convergence result for Algorithm \ref{alg:VMC}:
\begin{theorem}\label{thm:complexity_VMC}
Assume the Assumption \ref{assum_g_VMC} holds true. There exists a constant $C>0$ that only depends on $C_\wf$ such that, when $\eta_m<\frac{1}{C}$ for any $m\geq 0$, 
\begin{equation}\label{eqn:gradient_F_bound_vmc}
\sum^M_{m=0}\eta_m\mathbb{E}\left(\left|\nabla_\theta L(\theta_m)\right|^2\right)\leq 2F(\theta_0)+C\sum^M_{m=0}\frac{\eta^2_m}{n}\,.
\end{equation}
for any $M>0$. 
\end{theorem}
We put the proof of the above theorem in Appendix \ref{sec:pf_of_thm:complexity_VMC}. Similar to the supervised-learning case, the upper bound \eqref{eqn:gradient_F_bound_vmc} is important for us to determine the $\eta_m$ so that the parameter $\theta_m$ converges to a first-order stationary point in the expectation sense when $m$ is moderately large. Using Theorem \ref{thm:complexity_VMC}, we can show a similar result as supervised-learning:
\begin{corollary}\label{cor:VMC} Assume the Assumption \ref{assum_g_VMC} holds true.  
\begin{itemize}
\item Choosing $\eta_m=\Theta\left(n\epsilon^2\right)$ and $M=\Theta\left(1/(n\epsilon^4)\right)$,
we have $\min_{0\leq m\leq M}\mathbb{E}\left(\left|\nabla_\theta L(\theta_m)\right|\right)\leq \epsilon$.

\item Choosing $\eta_m=\Theta\left(\sqrt{\frac{n}{m+1}}\right)$, we have $\min_{0\leq m\leq M}\mathbb{E}\left(\left|\nabla_\theta L(\theta_m)\right|\right)=O\left(\left(nM\right)^{-1/4}\right)$.
\end{itemize}
\end{corollary}

This bound shows that the convergence of Algorithm \ref{alg:VMC} is same as the convergence rate of SGD to first-order stationary point for non-convex functions.
%\section{Scale-invariant supervised learning}\label{sec:sisl}
%
%\ZD{We might find some places to say the reason we want to do this scale-invariant supervised learning. Section \ref{sec:sisl} is not a too late place.}
%
%In this section, we consider solving Problem \ref{genvmcprob} with trivial sampling oracle, meaning that $p$ is a uniform distribution on the data set. Without loss of generality, we assume the underlining distribution of data $\rho(x)=\frac{1}{N}\sum^N_{n=1}\delta_{x=n}$, the sampling distribution $p(x)=\frac{1}{N}\sum^N_{n=1}\delta_{x=n}$, and 
%\[
%\localE_\theta(g)(x)=\frac{f(x)g_\theta(x)}{\|g_\theta\|}=-\frac{f(x)g_\theta(x)}{\frac{1}{N}\sum g^2_\theta(n)}\,.
%\]
%
%For simplicity, we write $f(n)=f_n$ and $g_\theta(n)=g_n(\theta)$. Then, minimizing \eqref{genvmc} is equivalent to minimize the loss function
%\begin{equation}\label{eqn:L_theta}
%L(\theta)=\mathcal{L}\left(g_\theta\right)=-\frac{\frac{1}{N}\sum^N_{n=1}f_ng_n(\theta)}{\sqrt{\frac{1}{N}\sum^N_{n=1}g^2_n(\theta)}}
%\end{equation}
%with gradient
%\begin{equation}\label{gradient_L_supervised}
%\begin{aligned}
%&\nabla_\theta L(\theta)=-\frac{\frac{1}{N}\sum^N_{n=1}f_n\nabla_\theta g_n(\theta)}{\sqrt{\frac{1}{N}\sum^N_{n=1}g^2_n(\theta)}}\\
%&+\frac{\left(\frac{1}{N}\sum^N_{n=1}f_n g_n(\theta)\right)\left(\frac{1}{N}\sum^N_{n=1}g_n(\theta) \nabla_\theta g_n(\theta)\right)}{(\frac{1}{N}\sum^N_{n=1}g^2_n(\theta))^{3/2}}\,.
%\end{aligned}
%\end{equation}
%Define 
%\begin{equation}\label{eqn:Z_theta}
%Z_\theta=\sqrt{\frac{1}{N}\sum^N_{n=1}g^2_n(\theta)}\,.
%\end{equation}
%Suppose we already know the denominator of \eqref{gradient_L_supervised}, then
%\[
%\begin{aligned}
%&\nabla_\theta L(\theta)=-\mathbb{E}_{r_1,r_2}\left(\frac{f_{r^m_1}\nabla_\theta g_{r^m_1}(\theta)g^2_{r^m_2}(\theta)}{Z^3_\theta}\right)\\
%&+\mathbb{E}_{r_1,r_2}\left(\frac{f_{r^m_2} g_{r^m_2}(\theta)\left(g_{r^m_1}(\theta) \nabla_\theta g_{r^m_1}(\theta)\right)}{Z^{3}_\theta}\right)\,,
%\end{aligned}
%\]
%where $r_1,r_2$ are uniformly sampled from $1,2,\cdots, N$. This implies, if $Z_\theta$ is known, for each fixed $\theta$, to get an unbiased estimation of $\nabla_\theta L(\theta)$, we only need to calculate 
%$g$, $\nabla_\theta g$ at two different points. 
%
%However, different from VMC setting, in this case, the sampling oracle $p$ only samples uniform distribution on the data set. This implies that, for every parameter $\theta$, we don't have an efficient way to calculate the denominator $Z_\theta$ in the gradient form \eqref{gradient_L_supervised}. Then, if we naively apply the idea of Algorithm \ref{alg:VMC} from VMC into this setting and calculate the denominator in each iteration to guarantee an unbiased gradient estimation, we need to calculate $g_n(\theta)$ for every data point $n$, which is as expensive as classical gradient descent. 



\section{Empirical convergence of supervised learning}\label{sec:Exp}
%\GG{Section title here is inconsistent with title of VMC section below}
\cref{losscomp} shows the convergence of SGD for supervised learning using the loss function \eqref{eqn:Ltheta} compared to using the standard $L^2$-loss $f\mapsto\|f-g\|^2$.
In order to have the same initial loss under both training procedures and avoid giving a disadvantage to the non-invariant $L^2$-based training, our Ansatz  includes a multiplicative pre-factor $\lambda$ in its parameters, and  $\lambda$ is initialized to normalize $\lambda f_\theta$ from the beginning of training. This comparison therefore isolates the difference between the two training losses due to the different training dynamics. We observe %\cref{losscomp} shows 
that training with the scale-invariant loss leads to a smaller value of the loss function by two orders of magnitude.
%\LL{ Though obvious, it is common to  add a sentence summarizing the finding in the figure.} 

\begin{figure}[h]
\includegraphics[width=.45\textwidth]{SI_vs_l2.pdf}
\caption{Comparison of the convergence when trained using the loss \eqref{eqn:Ltheta} versus training with the standard $L^2$ loss. In both cases the quantity plotted is the scale-invariant loss.}
\label{losscomp}
\end{figure}



\cref{fig:balance} shows the convergence of the SGD algorithm using the directionally unbiased gradient estimator of \cref{eq:unbiased} compared to that of the unbalanced and therefore directionally biased plug-in estimator of \cref{eq:plugin}. We find that training with the directionally unbiased estimator leads to convergence of the scale-invariant loss whereas training with the directionally biased plug-in estimate does not lead to convergence.   %\LL{ Same comment as above} 

\begin{figure}[h]
\includegraphics[width=.45\textwidth]{balancedSGD.pdf}
\caption{Comparison of gradient estimators for supervised learning. $f_\theta$ converges to the target using the directionally unbiased balanced gradient estimator \cref{eq:unbiased}, but it fails when using the plug-in variant \cref{eq:plugin}. The algorithm is a variant of \cref{alg:GE} where $Z^2$ is a running average of previous $f(X^m_{1,2})^2$.}
\label{fig:balance}
\end{figure}

\cref{thm:main_result_alg1} holds under the condition of a Hessian bound, \cref{eqn_Hessian_gnbound_vmc}. We use $\|\nabla L(\theta_{i+1})-\nabla L(\theta_i)\|/\|\theta_{i+1}-\theta_i\|$ as a proxy for the magnitude of the Hessian, noting that this is also the quantity which we need for the proof of the SGD algorithm (see \cref{eqn:gradient_Lip_vmc}).


\begin{figure}[h]
\includegraphics[width=.45\textwidth]{H_est.pdf}
\caption{Lipschitz estimate for the gradient in the supervised learning setting. $\|\nabla L(\theta_{i+1})-\nabla L(\theta_i)\|/\|\theta_{i+1}-\theta_i\|$ stays bounded by a small constant as the loss is minimized (red, dashed). }%\GG{Mention "supervised" in the caption so it is clear which algorithm this pertains to?}  }
\label{Hessianfig}
\end{figure}



\cref{Hessianfig} shows $\|G(\theta_{m+1};X^m)-G(\theta_{i};X^m)\|/\|\theta_{i+1}-\theta_i\|$
, which due to the bias-variance decomposition is an upper bound on $\frac{\|\nabla L(\theta_{i+1})-\nabla L(\theta_i)\|}{\|\theta_{i+1}-\theta_i\|}$ in the mean-square sense. %\GG{The preceding is only true if we take expectations of both sides over multiple runs, right?}  
We observe that this quantity is bounded by a small constant, justifying condition \eqref{eqn_Hessian_gnbound}.


\section{Empirical convergence of VMC algorithm}

We provide numerical evidence for the convergence of \cref{alg:VMC} on the Hydrogen square (H$_4$) model depicted in \cref{fig:h4conf}. This system involves only four particles but is strongly correlated and known to be difficult to simulate accurately. We define $\wf_\theta$ using the FermiNet ansatz \cite{Pfau_2020, Spencer_2020} and run 200,000 training steps using stochastic gradient descent with a learning rate schedule $\eta(t) = \frac{0.05}{\sqrt{1 + t/10000}}$. 

With this learning rate schedule, \cref{cor:VMC} implies that the running minimum of $\mathbb{E}(|\nabla_\theta L(\theta_m)|)$ should converge as $O(M^{-1/4})$ as the optimization progresses. To verify this bound, we plot in \cref{fig:H4_G} the running minimum of $|G(\theta)|$, which is our numerical proxy for $|\nabla_\theta L(\theta)|$ as per \cref{lem:unbiased_gradient}. We compare two distinct VMC runs using 10 and 1000 walkers, respectively, to explore how the convergence varies with the accuracy of the gradient estimate. We estimate the convergence rate by measuring the overall slope of the log-log plot, starting from step 200 to avoid the initial pre-asymptotic period. We find that the gradient of the loss converges roughly as $M^{-0.27}$ when using 10 walkers, closely matching the expected bound. With 1000 walkers the convergence goes as $M^{-0.38}$, which is somewhat faster than the theoretical bound, as might be expected given that the estimate of the gradient is very accurate with so many walkers.

Since the bounding of the Lipschitz constant of the gradient of the loss function is critical to our convergence proof, we supply in \cref{fig:H4_L} a numerical estimate of this quantity during the VMC run. Here we show data only for the case of 1000 walkers, as the Lipschitz constant estimate is extremely noisy with 10 walkers. The data suggest that the value of the Lipschitz constant is well below $1000$ for this simulation.
\begin{figure}
\centering
\begin{tikzpicture}

\coordinate (1) at (0,0);
\coordinate (2) at (2,0);
\coordinate (3) at (2,2);
\coordinate (4) at (0,2);
\coordinate (5) at ($(1)!.5!(2)$);

\fill (1) circle (2pt) node [below left] {H};
\fill (2) circle (2pt) node [below right] {H};
\fill (3) circle (2pt) node [above right] {H};
\fill (4) circle (2pt) node [above left] {H};
\node at (5) [below] {$1.0$ Bohr};

\draw[dashed] (1)--(2)--(3)--(4)-- cycle;

\end{tikzpicture}
\caption{Atomic configuration for the square H$_4$ model.}
\label{fig:h4conf}
\end{figure}

\begin{figure}
\includegraphics[width=0.45\textwidth]{H4_G.pdf}
\caption{Convergence of VMC run on the H$_4$ square. The running minimum is taken to smooth out the data and match the form of \cref{cor:VMC}.}
\label{fig:H4_G} 
\end{figure}

\begin{figure}
\includegraphics[width=0.45 \textwidth]{H4_L.pdf}
\caption{Lipschitz constant for VMC run on the H$_4$ square with 1000 walkers. The constant is numerically approximated using the formula $|G(\theta_{m+1})- G(\theta_m)|/|\theta_{m+1} - \theta_m|$.}
\label{fig:H4_L}
\end{figure}



% \subsection{Convergence of VMC algorithm}

% We provide numerical results for the convergence of Algorithm \ref{alg:VMC} on two systems. We use the VMCNet code \cite{Jeff_2021} with a simple stochastic gradient descent optimizer to run all experiments. For each system we present results using both 10 MCMC walkers and 1000 MCMC walkers. 

% \subsubsection{Quantum Harmonic Oscillator}
% \label{sec:osc}
% We first study a toy system consisting of five non-interacting fermions, three up-spin and two down-spin, under the quantum harmonic oscillator Hamiltonian. We define $\Psi_\theta$ to be the ground-state wavefunction for this quantum Harmonic oscillator with a spring constant of $\theta$, but we initialize $\theta$ to not match the true spring constant $\theta^*$ of the Hamiltonian. The VMC optimization thus amounts to tuning $\theta$ to match $\theta^*$. For this system we run 10,000 training epochs using a learning rate schedule $\eta(t) = \frac{0.01}{\sqrt{1 + t/10}}$. 

% \LL{Perhaps this scaling statement should be clarified / mentioned earlier} 
% With this learning rate schedule, Theorem \ref{thm:complexity_VMC} implies a bound on the convergence rate of the gradient of the loss:
% \begin{equation}
% \label{eq:vmc_exp_conv}
% \min_{0 \leq m \leq M} \mathbb{E}(|\nabla_\theta L(\theta_m)|^2) = \tilde{O}(M^{-1/2}).
% \end{equation}
% To verify this bound, we plot in panel (a) of Figure \ref{fig:qho} the running minimum of $|G(\theta)|^2$ over the training process, which is our numerical proxy for $|\nabla_\theta L(\theta)|^2$ as per Lemma \ref{lem:unbiased_gradient}. On this toy system the observed convergence is much more rapid than the bound of $M^{-1/2}$. 

% In panels (b) and (c) of Figure \ref{fig:qho} we plot the quantities that appear in the value bound and the second gradient bound of Assumption \ref{assum_g_VMC}. From these plots we see that $C_\Psi$ could potentially take on a value of about 10.0 for this system. Finally, in panel (d) we plot an estimate of the Lipschitz constant of the gradient of the loss function using parameter vectors from subsequent epochs, namely $
% \frac{|G(\theta_{m+1})- G(\theta_m)|}{|\theta_{m+1} - \theta_m|}$. This quantity is used as a proxy for the ratio $ \frac{|\nabla_\theta L(\theta_{m+1}) - \nabla_\theta L(\theta_m)|}{|\theta_{m+1} - \theta_m|}$ which appears in Lemma \ref{lem:prior_VMC} and is used to bound the variance of the gradient estimator $G(\theta)$. The data suggest that the relevant constant for that bound, $C' (C_\Psi^4 +1)$, can be set to $1e4$ for this system. \LL{ may need to discuss} 

% \subsubsection{H$_4$ square}
% We study the Hydrogen square (H$_4$) with a bond distance of 1.0 Bohr under the standard electronic structure Hamiltonian. This system, pictured in Figure \ref{fig:h4conf}, involves only four fermions but is strongly correlated and is physically much more complex than the quantum harmonic oscillator example presented previously. We define $\Psi_\theta$ using the FermiNet ansatz \cite{Pfau_2020, Spencer_2020} and run 200,000 training epochs using a learning rate schedule $\eta(t) = \frac{0.05}{\sqrt{1 + t/10000}}$. The resulting statistics are shown in Figure \ref{fig:H4} and mirror exactly the statistics shown in Figure \ref{fig:qho}. On this system, the convergence rate of the gradient of the energy approximately matches the lower bound of $M^{-1/2}$. Furthermore, the value bound and the gradient bound suggest that the value of $C_\Psi$ for this system could be chosen to be $1e3$, while the Lipschitz estimate indicates that the value of $C'(C_\Psi^4+1)$ could be set at $1e4$.

% \begin{figure}
% \centering
% \begin{tikzpicture}

% \coordinate (1) at (0,0);
% \coordinate (2) at (2,0);
% \coordinate (3) at (2,2);
% \coordinate (4) at (0,2);
% \coordinate (5) at ($(1)!.5!(2)$);

% \fill (1) circle (2pt) node [below left] {H};
% \fill (2) circle (2pt) node [below right] {H};
% \fill (3) circle (2pt) node [above right] {H};
% \fill (4) circle (2pt) node [above left] {H};
% \node at (5) [below] {$1.0$ Bohr};

% \draw[dashed] (1)--(2)--(3)--(4)-- cycle;

% \end{tikzpicture}
% \caption{Atomic configuration for the square H$_4$ model.}
% \label{fig:h4conf}
% \end{figure}

% \begin{figure*}
% \centering
%     \begin{subfigure}[b]{0.4\textwidth}
%         \includegraphics[width=\textwidth]{plots_vmc/QHO_G.pdf}
%         \caption{}
%         \label{subfig:QHO_G}
%     \end{subfigure}
%     \begin{subfigure}[b]{0.4\textwidth}
%         \includegraphics[width=\textwidth]{plots_vmc/QHO_EL.pdf}
%         \caption{}   
%         \label{subfig:QHO_EL}
%     \end{subfigure} 
%     \begin{subfigure}[b]{0.4\textwidth}
%         \includegraphics[width=\textwidth]{plots_vmc/QHO_GLP.pdf}
%         \caption{}
%         \label{subfig:QHO_GLP}
%     \end{subfigure}
%     \begin{subfigure}[b]{0.4\textwidth}
%         \includegraphics[width=\textwidth]{plots_vmc/QHO_L.pdf}
%         \caption{}
%         \label{subfig:QHO_L}
%     \end{subfigure}
% \caption{Selected training statistics from VMC run on quantum harmonic oscillator. (a) Convergence of SGD as tracked by the running minimum of $|G(\theta)|^2$. (b) Value bound quantity $\mathbb{E}(E_L^4)^{1/4}$ of Equation \ref{eqn_value_gnbound_vmc}. (c) Gradient bound quantity $\mathbb{E}(\nabla_\theta \log^4 \Psi)^{1/4}$ of Equation \ref{eqn_Gradient_gnbound_vmc}. (d) Lipschitz constant estimate $\frac{|G(\theta_{m+1})- G(\theta_m)|}{|\theta_{m+1} - \theta_m|}$. Data for (d) is partial because $\theta$ reached its exact value within 1000 epochs in both optimization runs, making $\theta_{m+1} = \theta_m $ so that no estimate of the Lipschitz constant could be made for later epochs.}
% \label{fig:qho}
% \end{figure*}

% \begin{figure*}
% \centering
%     \begin{subfigure}[b]{0.4\textwidth}
%         \includegraphics[width=\textwidth]{plots_vmc/H4_G.pdf}
%         \caption{}
%         \label{subfig:H4_G}
%     \end{subfigure}
%     \begin{subfigure}[b]{0.4\textwidth}
%         \includegraphics[width=\textwidth]{plots_vmc/H4_EL.pdf}
%         \caption{}   
%         \label{subfig:H4_EL}
%     \end{subfigure} 
%     \begin{subfigure}[b]{0.4\textwidth}
%         \includegraphics[width=\textwidth]{plots_vmc/H4_GLP.pdf}
%         \caption{}
%         \label{subfig:H4_GLP}
%     \end{subfigure}
%      \begin{subfigure}[b]{0.4\textwidth}
%         \includegraphics[width=\textwidth]{plots_vmc/H4_L.pdf}
%         \caption{}
%         \label{subfig:H4_L}
%     \end{subfigure}
% \caption{Selected training statistics from VMC run on H$_4$ square. (a) Convergence of SGD as tracked by the running minimum of $|G(\theta)|^2$.  (b) Value bound quantity $\mathbb{E}(E_L^4)^{1/4}$ of Equation \ref{eqn_value_gnbound_vmc}. (c) Gradient bound quantity $\mathbb{E}(\nabla_\theta \log^4 \Psi)^{1/4}$ of Equation \ref{eqn_Gradient_gnbound_vmc}. (d) Lipschitz constant estimate $\frac{|G(\theta_{m+1})- G(\theta_m)|}{|\theta_{m+1} - \theta_m|}$.}
% \label{fig:H4}
% \end{figure*}

\section{Conclusion}

% \LL{ Instead of emphasizing the infinite dimensional space (which does not play an explicit role), can we say something along the line of technical contributions, and alluding the future work beyond projective spaces? (i.e., relevant for excited state VMC simulations)} 
% In scientific computing it is often necessary to parameterize an infinite-dimensional function space with a finite set of parameters. It is non-trivial to constrain the parameterized function to lie on a sub-manifold of the function space. In this paper we consider optimization problems over the unit sphere of a function space and transform them into an unconstrained optimization problem with a scale-invariant loss. We prove that the SGD algorithm in the full parameterized function converges in a supervised learning setting, and we show that the energy converges in the variational Monte Carlo. We believe that the pullback technique can be extended to more general quotient spaces, to transform optimization problems over submanifolds of function spaces into unconstrained optimization problems. 

We consider the problem of optimizing a vector in a high dimensional sphere that is parameterized by a neural network up to a normalization constant. Existing manifold optimization algorithms are not directly applicable due to the inability to efficiently compute the normalization constant. We address this problem in the settings of both supervised and unsupervised learning. To our knowledge, this leads to the first proof of the convergence of SGD type algorithms for VMC simulations of ground states of quantum systems.

SGD is only the simplest stochastic optimization method. Over the last two decades, there has been increasing interest in developing more efficient and scalable optimization methods for VMC simulations~\cite{Sandvik2007, Neuscamman2012, Otis2019}, and our analysis may be a starting point for analyzing such methods.  It may also be possible to generalize this work from a high dimensional sphere to a Grassmann manifold, parameterized by a neural network up to a gauge matrix. Such a setting could be applicable to VMC simulations of \textit{excited states} of quantum systems.

\vspace{1em}
\textbf{Acknowledgment}

This work was supported by the Simons Foundation under Award No. 825053 (N. A.), and by the NSF Quantum Leap Challenge Institute (QLCI) program under Grant number OMA-2016245 (Z. D.). This work is supported by the U.S. Department of Energy, Office of Science, Office of Advanced Scientific Computing Research, Department of Energy Computational Science Graduate Fellowship under Award Number DE-SC0023112 (G. G.). This material is also based upon work supported by the U.S. Department of Energy, Office of Science, Office of Advanced Scientific Computing Research and Office of Basic Energy Sciences, Scientific Discovery through Advanced Computing (SciDAC) program (L.L.).  L.L. is a Simons Investigator. 


\vspace{1em}
\textbf{Disclaimer}

This report was prepared as an account of work sponsored by an agency of the United States Government. Neither the United States Government nor any agency thereof, nor any of their employees, makes any warranty, express or implied, or assumes any legal liability or responsibility for the accuracy, completeness, or usefulness of any information, apparatus, product, or process disclosed, or represents that its use would not infringe privately owned rights. Reference herein to any specific commercial product, process, or service by trade name, trademark, manufacturer, or otherwise does not necessarily constitute or imply its endorsement, recommendation, or favoring by the United States Government or any agency thereof. The views and opinions of authors expressed herein do not necessarily state or reflect those of the United States Government or any agency thereof.



% As discussed in the above remark, Algorithm \ref{alg:GE} is cheaper than classical GD only when $N>\frac{1}{\epsilon^2}$. This phenomenon is similar to the case when people compare SGD with GD. The $O(M^{-1/4})$ convergence rate mainly comes from the large gradient estimation variance in each iteration, which introduces extra error and slows the convergence speed. To overcome this bottleneck, Algorithm \ref{alg:VGD} combines variance reduction skills with the original $G_m$ to introduce a new gradient approximation that has smaller variance. Theoretically, the variance reduction skills ensure the cost of Algorithm \ref{alg:VGD} is always smaller than that of classcial GD, which is stated in the following theorem:
% \begin{theorem}\label{thm:algorithm_VGD}
% Assume the Assumption \ref{assum_g_scale_inva} holds true,  $|f_n|\leq C_f$ for all $1\leq n\leq N$, and 
% \begin{equation}\label{eqn:distance_bound_2}
% \frac{1}{C_r}\leq \frac{\frac{1}{N}\sum^N_{n=1}g^2_n(\theta_m)}{\frac{1}{N}\sum^N_{n=1}g^2_n(\theta^{\lfloor m/N\rfloor*N})}\leq C_r\,,\quad \mathrm{a.s.}
% \end{equation}
% for all $m\in\mathbb{N}$, where $\theta_m$ comes from Algorithm \ref{alg:GE}.  Set $\eta_m=\eta$, then there exists a constant $C$ that only depends on $C_r,C_f,C_g$ such that if $\eta\leq C\log^{1/2}(N)/N$, then for any $M>0$
% \begin{equation}\label{eqn:gradient_F_bound_2}
% \frac{1}{M+1}\sum^M_{m=0}\mathbb{E}\left(\left|\nabla F(\theta_m)\right|^2\right)\leq \frac{2C_rF(\theta_0)}{\eta (M+1) }\,.
% \end{equation}
% In particular, when $M=\Theta\left(\frac{N}{\log^{1/2}(N)\epsilon^2}\right)$, we have $\min_{0\leq m\leq M}\mathbb{E}\left(\left|\nabla F(\theta_m)\right|\right)\leq \epsilon$.
% \end{theorem}
% We put the proof of the above theorem in Appendix \ref{sec:pf_thm:algorithm_VGD}. To ensure $\min_{0\leq m\leq M}\mathbb{E}\left(\left|\nabla_\theta F(\theta_m)\right|\right)\leq \epsilon$ using Algorithm \ref{alg:VGD}, the number of gradient ($\nabla g_n$)  and value ($g_n$) calculations is $\Theta(M)=\Theta(N/(\log^{1/2}(N)\epsilon^2))$, which is always smaller than the cost of GD by a log factor in $N$.
\bibliography{references}
\bibliographystyle{icml2022}


%%%%%%%%%%%%%%%%%%%%%%%%%%%%%%%%%%%%%%%%%%%%%%%%%%%%%%%%%%%%%%%%%%%%%%%%%%%%%%%
%%%%%%%%%%%%%%%%%%%%%%%%%%%%%%%%%%%%%%%%%%%%%%%%%%%%%%%%%%%%%%%%%%%%%%%%%%%%%%%
% APPENDIX
%%%%%%%%%%%%%%%%%%%%%%%%%%%%%%%%%%%%%%%%%%%%%%%%%%%%%%%%%%%%%%%%%%%%%%%%%%%%%%%
%%%%%%%%%%%%%%%%%%%%%%%%%%%%%%%%%%%%%%%%%%%%%%%%%%%%%%%%%%%%%%%%%%%%%%%%%%%%%%%
\newpage
\appendix
\onecolumn

%\section{Derivation of VMC gradient for real \texorpdfstring{$\wf$}{Lg}}\label{sec:deri_gradient}
%Here we derive \eqref{eqn:gradient_L_VMC}. See \citep[Appendix E]{Jeff_2021} for a derivation that includes the complex case.
%\begin{align}
%\nabla_{\theta} L(\theta) & =\int_{\mathbb{R}^d} \nabla_{\theta} \mathcal{E}_{\theta}(x) p_{\theta}(x) d x+\int_{\mathbb{R}^d} \mathcal{E}_{\theta}(x) \nabla_{\theta} p_{\theta}(x) d x=A+B, \label{eqn:gradient_derive}\\
%A & =\mathbb{E}_{X\sim p_\theta}\left[\nabla_{\theta} \mathcal{E}_{\theta}(X)\right], \label{eqn:gradient_A}\\
%B & =\mathbb{E}_{X\sim p_\theta}\left[\mathcal{E}_{\theta}(X) \frac{\nabla_{\theta} p_{\theta}(X)}{p_{\theta}(X)}\right], \label{eqn:gradient_B}
%\end{align}
%We observe that $A=0$ for real $\wf_{\theta}$. Indeed,
%\begin{equation}\label{eqn:E}
%\nabla_{\theta} \mathcal{E}_{\theta}=\nabla_{\theta} \frac{H \wf_{\theta}}{\wf_{\theta}}=\frac{\wf_{\theta} \nabla_{\theta} H \wf_{\theta}-H \wf_{\theta} \nabla_{\theta} \wf_{\theta}}{\wf_{\theta}^{2}}\,.
%\end{equation}
%So, because $H$ is symmetric:
%\[
%A_{i}=\mathbb{E}_{X\sim p_\theta}\left[\partial_{\theta_{i}} \mathcal{E}_{\theta}\right]=\frac{1}{\left\|\wf_{\theta}\right\|^{2}}\left(\int_{\mathbb{R}^d} \wf_{\theta} H\left(\partial_{\theta_{i}} \wf_{\theta}\right)-\int_{\mathbb{R}^d} H \wf_{\theta} \partial_{\theta_{i}} \wf_{\theta}\right)=0,
%\]
%where $\theta=\left(\theta_{i}\right)_{i}$. So $A=0$ and we go on to compute $B$. Since $p_{\theta}$ is only known up to a $\theta$-dependent constant we need to rewrite \eqref{eqn:gradient_B} in terms of $\wf_{\theta}$:
%\begin{equation}\label{eqn:gp_over_p}
%\begin{aligned}
%\frac{\nabla_{\theta} p_{\theta}(X)}{p_{\theta}(X)} & =\nabla_{\theta} \log p_{\theta}(X) \\
%& =2 \nabla_{\theta} \log \wf_{\theta}(X)-2 \nabla_{\theta} \log \left\|\wf_{\theta}\right\| \\
%& =2 \frac{\nabla_{\theta} \wf_{\theta}(X)}{\wf_{\theta}(X)}-v_{\theta},
%\end{aligned}
%\end{equation}
%where $v_{\theta}$ does not depend on $X$. Now note that $\nabla_{\theta} p_{\theta} / p_{\theta}$ is centered:
%\[
%\mathbb{E}_{X\sim p_\theta}\left[\frac{\nabla_{\theta} p_{\theta}(X)}{p_{\theta}(X)}\right]=\nabla_{\theta} \int_{\mathbb{R}^d} p_{\theta}(x)dx=0 .
%\]
%This determines $v_{\theta}$ in \eqref{eqn:gp_over_p}, which we substitute into \eqref{eqn:gradient_B} to get
%\[
%\begin{aligned}
%B & =2 \mathbb{E}_{X\sim p_\theta}\left[\mathcal{E}_{\theta}(X) \frac{\nabla_{\theta} \wf_{\theta}(X)}{\wf_{\theta}(X)}\right]-2 \mathbb{E}_{X\sim p_\theta}\left[\mathcal{E}_{\theta}(X)\right] \mathbb{E}_{X\sim p_\theta}\left[\frac{\nabla_{\theta} \wf_{\theta}(X)}{\wf_{\theta}(X)}\right] \\
%& =2 \mathbb{E}_{X\sim p_\theta}\left[\left(\mathcal{E}_{\theta}(X)-L(\theta)\right) \frac{\nabla_{\theta} \wf_{\theta}(X)}{\wf_{\theta}(X)}\right] .
%\end{aligned}
%\]

\section{Proofs of gradient estimators}
\label{gradproofs}
\begin{proof}[Proof of \eqref{eqn:gradient_L_VMC}]

Let $q_\theta=|\wf_\theta|^2$, $\Lq_\theta(x)=\log q_\theta(x)$, and $Q_\theta=\|\wf_\theta\|^2$. Write \eqref{eqn:VMC_L} as
\[L(\theta)=\ifrac{\ol{q_\theta}{\localE}}{Q_\theta},\]
Then by the product formula,
\begin{equation}
\begin{aligned}
\nabla_\theta L(\theta)
&=
\frac{\nabla\ol{q_\theta}{\lE_\theta}}{Q_\theta}-\frac{\ol{q_\theta}{\lE_\theta}\nabla Q_\theta}{Q^2_\theta}=:A-B.
\end{aligned}
\label{AB}
\end{equation}
Continuing, we have
\begin{equation}
\label{eqA}
\begin{aligned}
A
%= \frac{\nabla\ol{q}{\lE}}{\|q\|_1}
&=\ol{\lE}{\nabla_\theta q_\theta}/Q_\theta+\ol{q_\theta}{\nabla\lE}/Q_\theta
\\&= \ol{\lE_\theta}{\frac{\nabla_\theta q_\theta}{q_\theta}\frac{q_\theta}{Q_\theta}}+\ol{\frac{q_\theta}{Q_\theta}}{\nabla_\theta\lE_\theta}
\\&= \EE_{p_\theta}[\lE_\theta\nabla_\theta\log q_\theta]+\EE_{p_\theta}[\nabla\lE_\theta].
\end{aligned}
\end{equation}
and
\begin{equation}
\label{eqB}
\begin{aligned}
B
&= \ol{\frac{q_\theta}{Q_\theta}}{\lE_\theta}\frac{\nabla_\theta Q_\theta}{Q_\theta}
\\&= L(\theta)\ol{\frac{q_\theta}{Q_\theta}}{\frac{\nabla_\theta q_\theta}{q_\theta}}
%\\&= \ol{p}{\lE}\ol{p}{\frac{\nabla q}{q}}
= L(\theta)\EE_{p_\theta}[\nabla_\theta\log q_\theta].
\end{aligned}
\end{equation}
Substitute \eqref{eqA}, \eqref{eqB} into \eqref{AB} to obtain 
\begin{equation}\label{eqn:gradient_L_VMC_true}
\nabla_\theta L(\theta)
=\EE_{X\sim p_\theta}\left[(\lE_\theta(X)-L(\theta))\nabla_\theta\Lq_\theta(X)+\nabla_\theta\lE_\theta(X)\right]\,.
\end{equation}



Since $q_\theta=\wf_\theta^2$, we have $\Lq_\theta=2\Lpsi_\theta$, so by \eqref{eqn:gradient_L_VMC_true} we have
\[\nabla_\theta L(\theta)
=\EE_{X\sim p_\theta}[2\big(\lE_\theta(X)-L(\theta))\nabla_\theta\Lpsi_\theta(X)+\nabla_\theta\lE_\theta(X)\Big].\]
It remains to show that $\EE_{X\sim p_\theta}[\partial_i\lE_\theta(X)]=0$ where $\partial_i$ is differentiation with respect to $\theta_i$. Write
\begin{equation}
%\label{eqn: use \cref{genvmclem} andE}
\partial_i\mathcal{E}_{\theta}=\partial_i\frac{H \wf_{\theta}}{\wf_{\theta}}=\frac{\partial_i H \wf_{\theta}\cdot\wf_{\theta}-H \wf_{\theta}\cdot \partial_i\wf_{\theta}}{\wf_{\theta}^{2}}\,.
\end{equation}
So, because $H$ is symmetric:
\[
\mathbb{E}_{X\sim p_\theta}\left[\partial_i\mathcal{E}_{\theta}\right]=\frac{\ol{H\partial_i\wf_\theta}{\wf_\theta}-\ol{H\wf_\theta}{\partial_i\wf_\theta}}{\|\wf_\theta\|^2}=0,
\]
\end{proof}


\begin{proof}[Proof of Lemma \ref{lem:unbiased_gradient}]
\[
\begin{aligned}
G(\theta)  =&\frac{2}{n-1} \sum_{i=1}^{n} \mathcal{E}_{\theta}\left(X_{i}\right) \nabla_{\theta} \log \wf_{\theta}\left(X_{i}\right)\\
&-\frac{2}{n(n-1)} \sum_{j=1}^{n} \sum_{i=1}^{n} \mathcal{E}_{\theta}\left(X_{j}\right) \nabla_{\theta} \log \wf_{\theta}\left(X_{i}\right) \\
 =&\frac{2}{n} \sum_{i=1}^{n} \mathcal{E}_{\theta}\left(X_{i}\right) \nabla_{\theta} \log \wf_{\theta}\left(X_{i}\right)\\
 &-\frac{2}{n^{2}-n} \sum_{i \neq j} \mathcal{E}_{\theta}\left(X_{i}\right) \nabla_{\theta} \log \wf_{\theta}\left(X_{j}\right) .
\end{aligned}
\]
Here, the terms $i=j$ in the expansion of the rightmost term cancel with the first sum. But now each sum is the average of terms with the correct expectation $\left(2 \mathbb{E}_{X\sim p_\theta}\left[\mathcal{E}_{\theta}(X) \nabla_{\theta} \log \wf_{\theta}(X)\right]\right.$ and $2 \mathcal{L}_{\theta} \mathbb{E}_{X\sim p_\theta} \nabla_{\theta} \log \wf_{\theta}(X)$ respectively $)$.
\end{proof}

\section{The directionally unbiased gradient estimator for supervised learning}\label{sec:minibath_sup}
 Given $n\ge2$ i.i.d. samples $X_i\sim\rho$. We write $f_i=f_\theta(X_i)$, $g_i=g(X_i)$, and $\nabla f_i=\nabla_\theta f_\theta(X_i)$.
 %, and use the counting measure to define
 \begin{equation}
 \begin{aligned}
 \ol{f}{g}_n=\frac1n\sum_{i=1}^nf_i g_i,
 \qquad
 \|f\|_n^2=\frac1n\sum_{i=1}^nf_i^2.
 \end{aligned}
 \end{equation}
 \begin{lemma}\label{unbiasedsupervised}
 Given samples $X_1,\ldots,X_n$, let
 \begin{align}
 G_n&=\frac1{n-1}\sum_{j=1}^n a_j\nabla f_j,\label{eqn:G_n}
 \\
 %a_j&=(\tfrac1r\sum_{i=1}^r f_i^2)g_j-(\frac1r\sum_{i=1}^rg_if_i)f_j.
 a_j&=-\|f\|_n^2\:g_j+\ol{g}{f}_n\:f_j.
 \end{align}
 Then $G_r$ is an unbiased estimator for $G=\|f_\theta\|^3\nabla_\theta L(\theta)$.
 \end{lemma}
 \begin{proof}[Proof of Lemma \ref{unbiasedsupervised}]
 \cref{eqn:G_n} states that
 \[G=-\|f_\theta\|^2\ol{g}{\nabla_\theta f_\theta}+\ol{g}{f_\theta}\ol{f_\theta}{\nabla_\theta f_\theta}.\]
 For each $i\neq j$ we have an unbiased estimator for $G$ given by
 \[-|f_i|^2g_j\nabla f_j+g_if_if_j\nabla f_j.\]
 Taking the average over all pairs $i\neq j$ gives another unbiased estimator for $G$:
 \[-\frac1{n(n-1)}\sum_{i\neq j}(f_i^2g_j\nabla f_j-g_if_if_j\nabla f_j).\]
 We may add the terms $i=j$ without changing the value because all such terms evaluate to $0$.
 \end{proof}


\section{Proof of Theorem \ref{thm:complexity_VMC}}\label{sec:pf_of_thm:complexity_VMC}
To prove Theorem \ref{thm:complexity_VMC}, we first show the following lemma:
\begin{lemma}\label{lem:prior_VMC}
Under the Assumption \ref{assum_g_VMC}, there exists uniform constant $C'$ such that for any $\theta,\widetilde{\theta}\in\mathbb{R}^d$,
\begin{equation}\label{eqn:gradient_bound_vmc}
\left|\nabla_\theta L(\theta)\right|\leq 4C^2_\wf\,,
\end{equation}
\begin{equation}\label{eqn:gradient_Lip_vmc}
\left|\nabla_\theta L(\theta)-\nabla_\theta L\left(\widetilde{\theta}\right)\right|\leq C'(C^4_\wf+1)|\theta-\widetilde{\theta}|\,,
\end{equation}
and
\begin{equation}\label{eqn:variance_bound_vmc}
\mathbb{E}_{\{X_i\}^n_{i=1}}\left(\left|G(\theta)\right|^2\right)\leq |\nabla_\theta L(\theta)|^2+\frac{C'(C^4_\wf+1)}{n}\,,
\end{equation}
where $G(\theta)$ is defined in \eqref{eqn:VMCgradient_estimator}.
\end{lemma}
The above lemma is important for the proof of Theorem \ref{thm:complexity_VMC}. First, that inequality \eqref{eqn:variance_bound_vmc} gives an upper bound for the gradient estimator. Using this upper bound, we can show that each iteration of Algorithm \ref{alg:VMC} is close to the classical gradient descent when the learning rate $\eta$ is small enough. Second, \eqref{eqn:gradient_Lip_vmc} means the hessian of $L$ is bounded. This ensures the classical gradient descent has a faster convergence rate to the first-order stationary point, which further implies the fast convergence of Algorithm \ref{alg:VMC}.

We first prove Lemma \ref{lem:prior_VMC}:
\begin{proof}[Proof of Lemma \ref{lem:prior_VMC}] Define
\[
Z_\theta=\sqrt{\int_{\Omega}\left|\wf_{\theta}(x)\right|^{2} d x}\,.
\]

First, to prove \eqref{eqn:gradient_bound_vmc}, using \eqref{eqn:gradient_L_VMC}, we have
\[
\begin{aligned}
\left|\nabla_\theta L(\theta)\right|&=2\mathbb{E}_{X\sim p_\theta}\left|\left(\mathcal{E}_{\theta}(X)-L(\theta)\right) \frac{\nabla_{\theta} \wf_{\theta}(X)}{\wf_{\theta}(X)}\right|\\
&\leq 2\left(\mathbb{E}_{X\sim p_\theta}\left|\mathcal{E}_{\theta}(X)-L(\theta)\right|^2\right)^{1/2} \left(\mathbb{E}_{X\sim p_\theta}\left|\frac{\nabla_{\theta} \wf_{\theta}(X)}{\wf_{\theta}(X)}\right|^2\right)^{1/2}\leq 4C^{2}_\wf\,,
\end{aligned}
\]
where we use H\"older's inequality in the first inequality, \eqref{eqn_value_gnbound_vmc} and the second inequality of  \eqref{eqn_Gradient_gnbound_vmc} in the last inequality.

Next, to prove \eqref{eqn:gradient_Lip_vmc}, for fixed $\theta\in\Omega$, we define $R(w)=\frac{Z_w}{Z_\theta}$ and  $r(t)=\frac{Z_{\theta+t\left(\widetilde{\theta}-\theta\right)}}{Z_\theta}$. Using the second inequality of \eqref{eqn_Gradient_gnbound_vmc}, we obtain
\[
\begin{aligned}
\left|\nabla_w R(w)\right|=&\left|\frac{\int \nabla_w \wf_w(x)\wf_w(x)dx}{\sqrt{\int |\wf_w(x)|^2dx}\sqrt{\int |\wf_\theta(x)|^2dx}}\right|\\
\leq &\frac{\sqrt{\int |\nabla_w \wf_w(x)|^2dx}\sqrt{\int |\wf_w(x)|^2dx}}{\sqrt{\int |\wf_w(x)|^2dx}\sqrt{\int |\wf_\theta(x)|^2dx}}\\
\leq &\sqrt{\frac{\int_{\Omega}\left|\nabla_w \wf_w(x)\right|^2dx}{\int_{\Omega}\left| \wf_w(x)\right|^2dx}}\frac{Z_w}{Z_\theta}\\
= &\left(\mathbb{E}_{X\sim p_w}\left(\left|\frac{\nabla_w  \wf_w(X)}{\wf_w(X)}\right|^2\right)\right)^{1/2}\frac{Z_w}{Z_\theta}\leq &C_{\wf}R(w)\,.
\end{aligned}
\]
where we use H\"older's inequality in the first inequality and the second inequality of  \eqref{eqn_Gradient_gnbound_vmc} in the last inequality.

This implies
\[
\frac{d}{dt}r(t)\leq \left|\nabla_w R(w)|_{w=\theta+t\left(\widetilde{\theta}-\theta\right)}\right|\left|\widetilde{\theta}-\theta\right|\leq C_\wf \left|\widetilde{\theta}-\theta\right| r(t)\,.
\]
Combining this inequality with the fact that $r(0)=1$, we have
\[
\frac{Z_{\theta+t\left(\widetilde{\theta}-\theta\right)}}{Z_\theta}-1=r(t)-1\leq \exp\left(C_\wf \left|\widetilde{\theta}-\theta\right| t\right)-1
\]
for any $t\in[0,1]$, and
\begin{equation}\label{eqn:ratio_Lip_VMC}
\frac{Z_{\theta+\left(\widetilde{\theta}-\theta\right)}}{Z_\theta}-1\leq C_{\wf}\exp\left(C_{\wf}|\widetilde{\theta}-\theta|\right)|\widetilde{\theta}-\theta|
\end{equation}
by the Gr\"onwall's inequality.

Now, we are ready to show \eqref{eqn:gradient_Lip_vmc}. Using the formula of $\mathcal{E}(\theta)$ (equation \eqref{eqn:deflocalE}), we write 
\begin{equation}\label{eqn:gL_lip_seperation}
\begin{aligned}
&\left|\nabla_\theta L(\theta)-\nabla_\theta L\left(\widetilde{\theta}\right)\right|\\
\leq &2\left|\frac{1}{Z^2_\theta}\int_{\Omega} H\wf_\theta(x)\nabla_\theta \wf_\theta(x)dx-\frac{1}{Z^2_{\widetilde{\theta}}}\int_{\Omega} H\wf_{\widetilde{\theta}}(x)\nabla_\theta \wf_{\widetilde{\theta}}(x)dx\right|\\
&+2\left|\frac{1}{Z^2_\theta}\int_{\Omega} L(\theta)\wf_\theta(x)\nabla_\theta \wf_\theta(x)dx-\frac{1}{Z^2_{\widetilde{\theta}}}\int_{\Omega} L\left(\widetilde{\theta}\right)\wf_{\widetilde{\theta}}(x)\nabla_\theta \wf_{\widetilde{\theta}}(x)dx\right|\,.
\end{aligned}
\end{equation}
We first deal with the first term. For simplicity, we first assume $|\widetilde{\theta}-\theta|\leq \epsilon$ for an arbitrary $\epsilon>0$:
\begin{equation}\label{eqn:gL_lip_first} 
\begin{aligned}
&\left|\frac{1}{Z^2_\theta}\int_{\Omega} H\wf_\theta(x)\nabla_\theta \wf_\theta(x)dx-\frac{1}{Z^2_{\widetilde{\theta}}}\int_{\Omega} H\wf_{\widetilde{\theta}}(x)\nabla_\theta \wf_{\widetilde{\theta}}(x)dx\right|\\
\leq &\underbrace{\left|\frac{1}{Z^2_\theta}\int_{\Omega} \left(H\wf_\theta(x)\nabla_\theta \wf_\theta(x)-H\wf_{\widetilde{\theta}}(x)\nabla_{\theta} \wf_{\widetilde{\theta}}(x)\right)dx\right|}_{\mathrm{(I)}}\\
&+\underbrace{\left|\left(\frac{1}{Z^2_{\widetilde{\theta}}}-\frac{1}{Z^2_\theta}\right)\int_{\Omega} H\wf_{\widetilde{\theta}}(x)\nabla_\theta \wf_{\widetilde{\theta}}(x)dx\right|}_{\mathrm{(II)}}\,.
\end{aligned}
\end{equation}
\begin{itemize}
    \item For (I), we have
    \begin{align*}
    &\left|\frac{1}{Z^2_\theta}\int_{\Omega} \left(H\wf_\theta(x)\nabla_\theta \wf_\theta(x)-H\wf_{\widetilde{\theta}}(x)\nabla_{\theta} \wf_{\widetilde{\theta}}(x)\right)dx\right|\\
    \leq & \left|\widetilde{\theta}-\theta\right|\int^1_{0}\left|\frac{Z^2_{\theta+t\left(\widetilde{\theta}-\theta\right)}}{Z^2_\theta}\right|\left|\frac{\nabla H\wf_{\theta+t\left(\widetilde{\theta}-\theta\right)}(x)\nabla_{\theta} \wf_{\theta+t\left(\widetilde{\theta}-\theta\right)}(x)}{Z^2_{\theta+t\left(\widetilde{\theta}-\theta\right)}}\right|dt\\
    &+\left|\widetilde{\theta}-\theta\right|\int^1_{0}\left|\frac{Z^2_{\theta+t\left(\widetilde{\theta}-\theta\right)}}{Z^2_\theta}\right|\left|\frac{H\wf_{\theta+t\left(\widetilde{\theta}-\theta\right)}(x)\left\|\mathsf{H}_{\theta} \wf_{\theta+t\left(\widetilde{\theta}-\theta\right)}(x)\right\|_2}{Z^2_{\theta+t\left(\widetilde{\theta}-\theta\right)}}\right|dt\\
    \leq & (C_{\wf}\exp\left(C_{\wf}\epsilon\right)\epsilon+1)^2\left|\widetilde{\theta}-\theta\right|\int^1_{0}\mathbb{E}_{X\sim p_{\theta+t\left(\widetilde{\theta}-\theta\right)}}\left(\left|\frac{\nabla H\wf_{\theta+t\left(\widetilde{\theta}-\theta\right)}(X)\nabla_{\theta} \wf_{\theta+t\left(\widetilde{\theta}-\theta\right)}(X)}{\left|\wf_{\theta+t\left(\widetilde{\theta}-\theta\right)}(X)\right|^2}\right|\right)dt\\
    &+(C_{\wf}\exp\left(C_{\wf}\epsilon\right)\epsilon+1)^2\left|\widetilde{\theta}-\theta\right|\int^1_{0}\mathbb{E}_{X\sim p_{\theta+t\left(\widetilde{\theta}-\theta\right)}}\left(\left|\frac{H\wf_{\theta+t\left(\widetilde{\theta}-\theta\right)}(x)\left\|\mathsf{H}_{\theta} \wf_{\theta+t\left(\widetilde{\theta}-\theta\right)}(x)\right\|_2}{\left|\wf_{\theta+t\left(\widetilde{\theta}-\theta\right)}(X)\right|^2}\right|\right)dt\\
    \leq & 2(C_{\wf}\exp\left(C_{\wf}\epsilon\right)\epsilon+1)^2C^2_\wf\left|\widetilde{\theta}-\theta\right|\,,
    \end{align*}
    where we use fundamental theorem of calculus and triangle inequality for integral in the first inequality, \eqref{eqn:ratio_Lip_VMC} in the second inequality, H\"older's inequality, \eqref{eqn_value_gnbound_vmc}-\eqref{eqn_Hessian_gnbound_vmc} in the last inequality.
    \item For (II), we have
    \begin{equation}
    \begin{aligned}
    &\left|\left(\frac{1}{Z^2_{\widetilde{\theta}}}-\frac{1}{Z^2_\theta}\right)\int_{\Omega} H\wf_{\widetilde{\theta}}(x)\nabla_\theta \wf_{\widetilde{\theta}}(x)dx\right|\\
    \leq &\left|\frac{Z^2_{\widetilde{\theta}}}{Z^2_\theta}-1\right|\left|\int_{\Omega} \frac{H\wf_{\widetilde{\theta}}(x)\nabla_\theta \wf_{\widetilde{\theta}}(x)}{Z^2_{\widetilde{\theta}}}dx\right|\\
    \leq &(C_{\wf}\exp\left(C_{\wf}\epsilon\right)\epsilon+1)C^3_{\wf}\exp\left(C_{\wf}\epsilon\right)\left|\widetilde{\theta}-\theta\right|\,,
    \end{aligned}
    \end{equation}
    where we use $|\widetilde{\theta}-\theta|\leq \epsilon$, H\"older's inequality, \eqref{eqn_value_gnbound_vmc}, the second inequality of  \eqref{eqn_Gradient_gnbound_vmc}, and \eqref{eqn:ratio_Lip_VMC}  in the last inequality.
\end{itemize}
Plugging these two terms into \eqref{eqn:gL_lip_first}, we have
\[
\begin{aligned}
&\left|\frac{1}{Z^2_\theta}\int_{\Omega} H\wf_\theta(x)\nabla_\theta \wf_\theta(x)dx-\frac{1}{Z^2_{\widetilde{\theta}}}\int_{\Omega} H\wf_{\widetilde{\theta}}(x)\nabla_\theta \wf_{\widetilde{\theta}}(x)dx\right|\\
\leq &C^2_\wf(C_{\wf}\exp\left(C_{\wf}\epsilon\right)\epsilon+1)\left(C_{\wf}\exp\left(C_{\wf}\epsilon\right)+2C_{\wf}\exp\left(C_{\wf}\epsilon\right)\epsilon+2\right)\left|\widetilde{\theta}-\theta\right|\,.
\end{aligned}
\]
for any $|\theta-\widetilde{\theta}|\leq \epsilon$. Since $\epsilon$ is arbitrary, we let $\epsilon\rightarrow0$ and use triangle inequality to obtain
\begin{equation}\label{eqn:gL_lip_first_bound}
\begin{aligned}
&\left|\frac{1}{Z^2_\theta}\int_{\Omega} H\wf_\theta(x)\nabla_\theta \wf_\theta(x)dx-\frac{1}{Z^2_{\widetilde{\theta}}}\int_{\Omega} H\wf_{\widetilde{\theta}}(x)\nabla_\theta \wf_{\widetilde{\theta}}(x)dx\right|\\
\leq &C^2_\wf(C_\wf+2)\left|\widetilde{\theta}-\theta\right|\,.
\end{aligned}
\end{equation}
for any $\widetilde{\theta},\theta\in\mathbb{R}^d$.

To deal with the second term of \eqref{eqn:gL_lip_seperation}, we first use \eqref{eqn_value_gnbound_vmc} and \eqref{eqn:gradient_bound_vmc} to obtain
\[
L\left(\theta+t\left(\widetilde{\theta}-\theta\right)\right)\leq C_\wf,\quad \left|L(\theta)-L\left(\widetilde{\theta}\right)\right|\leq 4C^2_\wf\left|\widetilde{\theta}-\theta\right|,\quad\forall t\in[0,1]\,.
\]
Using similar arguments as before, we can show that
\begin{equation}\label{eqn:gL_lip_second_bound}
\left|\frac{1}{Z^2_\theta}\int_{} L(\theta)\wf_\theta(x)\nabla_\theta \wf_\theta(x)dx-\frac{1}{Z^2_{\widetilde{\theta}}}\int_{} L\left(\widetilde{\theta}\right)\wf_{\widetilde{\theta}}(x)\nabla_\theta \wf_{\widetilde{\theta}}(x)dx\right|\leq C(C^4_\wf+1)|\widetilde{\theta}-\theta|\,,
\end{equation}
where $C$ is a uniform constant. Plugging \eqref{eqn:gL_lip_first_bound} and \eqref{eqn:gL_lip_second_bound} into \eqref{eqn:gL_lip_seperation}, we prove \eqref{eqn:gradient_Lip_vmc}\,.

Finally, to prove \eqref{eqn:variance_bound_vmc}, noticing
\[
G(\theta)=\frac{2}{n}\sum_{i=1}^{n} \mathcal{E}_{\theta}\left(X_{i}\right) \nabla_{\theta} \log \wf_{\theta}\left(X_{i}\right)-\frac{2}{n^{2}-n} \sum_{i \neq j} \mathcal{E}_{\theta}\left(X_{i}\right) \nabla_{\theta} \log \wf_{\theta}\left(X_{j}\right)\,,
\]
and
\[
\mathbb{E}_{\{X_i\}^n_{i=1}}\left(G(\theta)\right)=\nabla_\theta L(\theta)\,,
\]
we have
\[
\begin{aligned}
  &\mathbb{E}_{\{X_i\}^n_{i=1}}\left(\left|G(\theta)\right|^2\right)\\
  \leq &|\nabla_\theta L(\theta)|^2+\frac{8}{n^2}\sum^n_{i=1}\mathbb{E}_{X_i\sim p_\theta}\left(\left|\mathcal{E}_{\theta}\left(X_{i}\right) \nabla_{\theta} \log \wf_{\theta}\left(X_{i}\right)-\mathbb{E}_{X\sim p_\theta}\left[\mathcal{E}_{\theta}(X) \nabla_{\theta} \log \wf_{\theta}(X)\right]\right|^2\right)\\
  &+\frac{8}{(n^2-n)^2}\mathbb{E}\left(\sum_{i\neq j}\mathcal{E}_{\theta}\left(X_{i_1}\right) \nabla_{\theta} \log \wf_{\theta}\left(X_{j_1}\right)-\mathcal{L}_{\theta} \mathbb{E}_{X\sim p_\theta} \nabla_{\theta} \log \wf_{\theta}(X)\right)^2\\
  \leq &|\nabla_\theta L(\theta)|^2+\frac{8}{n}\mathbb{E}_{X\sim p_\theta}\left|\mathcal{E}_{\theta}\left(X\right) \nabla_{\theta} \log \wf_{\theta}\left(X\right)-\mathbb{E}_{X\sim p_\theta}\left[\mathcal{E}_{\theta}(X) \nabla_{\theta} \log \wf_{\theta}(X)\right]\right|^2\\
  &+\frac{8}{(n^2-n)^2}\left(\sum_{i_1=i_2\,\text{or}\,j_1=j_2} 1\right)\mathbb{E}_{(X_1,X_2)\sim p^{\otimes 2}_\theta}\left|\mathcal{E}_{\theta}\left(X_{1}\right) \nabla_{\theta} \log \wf_{\theta}\left(X_{2}\right)- \mathcal{L}_{\theta} \mathbb{E}_{X\sim p_\theta} \nabla_{\theta} \log \wf_{\theta}(X)\right|^2\\
  \leq &|\nabla_\theta L(\theta)|^2+\frac{8}{n}\mathbb{E}_{X\sim p_\theta}\left|\mathcal{E}_{\theta}\left(X\right) \nabla_{\theta} \log \wf_{\theta}\left(X\right)\right|^2+\frac{C}{n}\mathbb{E}_{(X_1,X_2)\sim p^{\otimes 2}_\theta}\left|\mathcal{E}_{\theta}\left(X_{1}\right) \nabla_{\theta} \log \wf_{\theta}\left(X_{2}\right)\right|^2\\
\end{aligned}\,,
\]
where we use $X_{i}$ and $X_j$ are independent in the first two inequalities. 
Using \eqref{eqn_value_gnbound_vmc} and the second inequality of  \eqref{eqn_Gradient_gnbound_vmc}, it is straightforward to show:
\[
\mathbb{E}_{X\sim p_\theta}\left|\mathcal{E}_{\theta}\left(X\right) \nabla_{\theta} \log \wf_{\theta}\left(X\right)\right|^2\leq \left(\mathbb{E}_{X\sim p_\theta}\left(\left|\frac{H \wf_\theta(X)}{\wf_\theta(X)}\right|^4\right)\right)^{1/2}\left(\mathbb{E}_{X\sim p_\theta}\left(\left|\frac{\nabla_\theta \wf_\theta(X)}{\wf_\theta(X)}\right|^4\right)\right)^{1/2}\leq C^4_\wf\,,
\]
and
\[
\mathbb{E}_{(X_1,X_2)\sim p^{\otimes 2}_\theta}\left|\mathcal{E}_{\theta}\left(X_{1}\right) \nabla_{\theta} \log \wf_{\theta}\left(X_{2}\right)\right|^2\leq \mathbb{E}_{X\sim p_\theta}\left(\left|\frac{H \wf_\theta(X)}{\wf_\theta(X)}\right|^2\right)\mathbb{E}_{X\sim p_\theta}\left(\left|\frac{\nabla_\theta \wf_\theta(X)}{\wf_\theta(X)}\right|^2\right)\leq C^4_\wf\,.
\]
This concludes the proof of \eqref{eqn:variance_bound_vmc}.
%\LL{Again worth writing down the details of the calculation later.} ~\ZD{Done.}
\end{proof}

Now, we are ready to prove Theorem \ref{thm:complexity_VMC}:
\begin{proof}[Proof of Theorem \ref{thm:complexity_VMC}]
Denote the probability filtration 
\[
\mathcal{F}_m=\sigma(X^j_i,1\leq i\leq n, 1\leq j\leq m)\,.
\]
According to the algorithm, we have
\[
\theta_{m+1}=\theta_m-\eta_m G_m\,.
\]
Plugging $\theta_{m+1}$ into $L(\theta)$, we have
\[
L(\theta_{m+1})\leq L(\theta_m)-\eta_m\nabla_\theta L(\theta_m)\cdot G_m+\frac{C\eta_m^2}{2}|G_m|^2
\]
where we use \eqref{eqn:gradient_Lip_vmc} and $C$ is a constant that only depends on $C_\wf$. This implies
\[
\begin{aligned}
\mathbb{E}\left(L(\theta_{m+1})|\mathcal{F}_{m-1}\right)\leq &L(\theta_m)-\eta_m\left|\nabla_\theta L(\theta_m)\right|^2+\frac{C\eta_m^2}{2}\mathbb{E}(|G_m|^2|\mathcal{F}_{m-1})
\end{aligned}
\]
where we use $\mathbb{E}_{\{X_i\}^n_{i=1}}\left(G(\theta)\right)=\nabla_\theta L(\theta)$. Furthermore, using \eqref{eqn:variance_bound_vmc} and $\eta_m<\frac{1}{C}$, we have 
\[
\begin{aligned}
\mathbb{E}\left(L(\theta_{m+1})|\mathcal{F}_{m-1}\right)\leq &L(\theta_m)-\left(\eta_m-C\eta_m^2/2\right)\left|\nabla_\theta L(\theta_m)\right|^2+\frac{C\eta_m^2}{2n}\\
\leq &L(\theta_m)-\frac{\eta_m}{2}\left|\nabla_\theta L(\theta_m)\right|^2+\frac{C\eta_m^2}{2n}
\end{aligned}
\]
Taking the average in $m=1,2,3,\cdots,M-1$, we prove \eqref{eqn:gradient_F_bound_vmc}.
\end{proof}

\section{Proof of Theorem \ref{thm:main_result_alg1}}\label{sec:pf_of_thm:main_result_alg1}
In this section, we prove Theorem \ref{thm:main_result_alg1}. We first show the following lemma: 
\begin{lemma}\label{lem:svsl_2} Under the Assumption \ref{assum_g_scale_inva}, there exists a constant $C'$ that only depends on $C_g$ such that for any $\theta,\widetilde{\theta}\in\mathbb{R}^d$,
\begin{equation}\label{eqn:gradient_Lip_2}
\left|\nabla_\theta L(\theta)-\nabla_\theta L\left(\widetilde{\theta}\right)\right|\leq C'|\theta-\widetilde{\theta}|\,.
\end{equation}
and
\begin{equation}\label{eqn:variance_bound_sup}
\mathbb{E}_{\{X_i\}^n_{i=1}}\left(\left|G(\theta)\right|^2\right)\leq \frac{Z^6_\theta}{Z^6}\left(|\nabla_\theta L(\theta)|^2+\frac{C'}{n}\right)\,,
\end{equation}
where $G(\theta)$ is defined in \eqref{eqn:G_r_theta}.
\end{lemma}
\begin{proof}[Proof of Lemma \ref{lem:svsl_2}] The proof is very similar to the proof of Lemma \ref{lem:prior_VMC} after setting $H\wf_\theta(x)=g(x)\left(\functional{g(x)}{\wf_\theta(x)}\right)$.
\end{proof}

Now, we are ready to prove Theorem \ref{thm:main_result_alg1}.
\begin{proof}[Proof of Theorem \ref{thm:main_result_alg1}] Without loss of generality, let $0\leq m\leq N$. According to the algorithm, we have
\[
\theta_{m+1}=\theta_m-\eta_m G_m\,.
\]
Plugging $\theta_{m+1}$ into $L(\theta)$, we have
\[
L(\theta_{m+1})\leq L(\theta_m)-\eta_m\nabla_\theta L(\theta_m)\cdot G_m+\frac{C\eta_m^2}{2}|G_m|^2
\]
where we use \eqref{eqn:gradient_Lip_2}. Here $C$ is a constant that only depends on $C_g$. This implies
\begin{equation}\label{eqn:L_bound}
\begin{aligned}
\mathbb{E}\left(L(\theta_{m+1})|\mathcal{L}_{m-1}\right)\leq &L(\theta_m)-\eta_m\left|\nabla_\theta L(\theta_m)\right|^2\frac{Z^3_m}{\widetilde{Z}^3_{\lfloor m/N\rfloor*N}}\\
&+\frac{C\eta^2_m}{2}\mathbb{E}(|G_m|^2|\mathcal{L}_{m-1})
\end{aligned}
\end{equation}
Using \eqref{eqn:variance_bound_sup} and \eqref{eqn:distance_bound},
\begin{equation}\label{eqn:bound_fourth_term}
\mathbb{E}(|G_m|^2|\mathcal{L}_{m-1})\leq C\left(|\nabla_\theta L(\theta_m)|^2+\frac{C'}{n}\right)\,,
\end{equation}
where $C$ is a constant depends on $C_g$ and $C_f$.
 
Plugging \eqref{eqn:bound_fourth_term} into \eqref{eqn:L_bound}, using \eqref{eqn:distance_bound}, and choosing $\eta_m$ small enough, we have
\begin{equation}\label{eqn:bound_f_2}
\mathbb{E}\left(L(\theta_{m+1})\right)\leq L(\theta_m)-\frac{\eta_m}{2C^2_r}\left|\nabla_\theta L(\theta_m)\right|^2+\frac{C\eta^2_m}{n}\\
\end{equation}
Taking the average in $m=1,2,3,\cdots,M-1$, we prove \eqref{eqn:gradient_F_bound}.
\end{proof}

\section{Lipschitz of the $Z_\theta$}\label{sec:svsl_1}
\begin{lemma}\label{lem:svsl_1} Under the Assumption \ref{assum_g_scale_inva}, given $\theta,\widetilde{\theta}\in\mathbb{R}^d$ and assume 
\[
\left|\widetilde{\theta}-\theta\right|\leq C_r\,,
\]
then there exists a constant $C'$ that only depends on $C_g$ and $C_r$ such that
\begin{equation}\label{eqn:ratio_Lip_1}
\frac{\|f_{\widetilde{\theta}}\|}{\|f_{\theta}\|}\leq \exp(C'C_r)\,.
\end{equation}
\end{lemma}

\begin{proof}[Proof of Lemma \ref{lem:svsl_1}] Fixing $\theta,\widetilde{\theta}$, we define
\[
R(t)=\frac{\left\|f_{\theta+t(\widetilde{\theta}-\theta)}\right\|^2}{\|f_{\theta}\|^2}\,.
\]
Then
\[
\begin{aligned}
|R'(t)|\leq &\left|\frac{2\functional{\nabla_\theta f_{\theta+t(\widetilde{\theta}-\theta)}}{f_{\theta+t(\widetilde{\theta}-\theta)}}}{\|f_{\theta}\|^2}\right||\widetilde{\theta}-\theta|\\
\leq &\left|\frac{2\functional{\nabla_\theta f_{\theta+t(\widetilde{\theta}-\theta)}}{f_{\theta+t(\widetilde{\theta}-\theta)}}}{\left\|f_{\theta+t(\widetilde{\theta}-\theta)}\right\|^2}\right|\left|\frac{\left\|f_{\theta+t(\widetilde{\theta}-\theta)}\right\|^2}{\|f_{\theta}\|^2}\right||\widetilde{\theta}-\theta|\\
\leq &C|\widetilde{\theta}-\theta|R(t)\,,
\end{aligned}
\]
where we use \eqref{eqn_value_gnbound} and \eqref{eqn_Gradient_gnbound} in the last inequality. Because $R(0)$=1, using Gr\"onwall's inequality, we obtain
\[
\frac{\|f_{\widetilde{\theta}}\|}{\|f_{\theta}\|}=R^{1/2}(1)\leq \exp(C|\theta-\widetilde{\theta}|)\leq \exp(CC_r)\,.
\]
\end{proof}

% \section{Proof of Theorem \ref{thm:algorithm_VGD}}\label{sec:pf_thm:algorithm_VGD}
% \begin{proof}[Proof of Theorem \ref{thm:algorithm_VGD}] Without loss of generality, first let $0\leq m\leq N$. According to the algorithm, we have
% \[
% \theta_{m+1}=\theta_m-\eta G_m\,.
% \]
% Similar to previous proof, we have
% \begin{equation}\label{eqn:F_bound_2}
% \mathbb{E}\left(F(\theta_{m+1})|\mathcal{F}_{m-1}\right)\leq F(\theta_m)-\frac{\eta}{C_r}\left|\nabla F(\theta_m)\right|^2+\frac{C\eta^2}{2}\mathbb{E}(|G_m|^2|\mathcal{F}_{m-1})
% \end{equation}
% For the fourth term, we notice that
% \[
% \begin{aligned}
% \mathbb{E}(|G_m|^2|\mathcal{F}_{m-1})=&\mathbb{E}(|G_m-\mathbb{E}(G_m|\mathcal{F}_{m-1})|^2|\mathcal{F}_{m-1})+\left|\mathbb{E}(G_m|\mathcal{F}_{m-1})\right|^2\\
% \leq &\mathbb{E}(|G_m-\mathbb{E}(G_m|\mathcal{F}_{m-1})|^2|\mathcal{F}_{m-1})+C|\nabla F(\theta_m)|^2
% \end{aligned}
% \]
% where we use \eqref{eqn:expectation-difference} and \eqref{eqn:distance_bound_2} in the inequality. We bound the first term by 
% \[
% \begin{aligned}
% &\mathbb{E}(|G_m-\mathbb{E}(G_m|\mathcal{F}_{m-1})|^2|\mathcal{F}_{m-1})\\
% \leq &2\mathbb{E}(|G_{m,1}-\mathbb{E}(G_{m,1}|\mathcal{F}_{m-1})|^2|\mathcal{F}_{m-1})\\
% &+2\mathbb{E}(|G_{m,2}-\mathbb{E}(G_{m,2}|\mathcal{F}_{m-1})|^2|\mathcal{F}_{m-1})\,.
% \end{aligned}
% \]
% The first term can be bounded using 
% \[
% \begin{aligned}
% &\mathbb{E}(|G_{m,1}-\mathbb{E}(G_{m,1}|\mathcal{F}_{m-1})|^2|\mathcal{F}_{m-1})\\
% \leq &C\frac{\frac{1}{N}\sum^N_{n=1}|f_n|^2|\nabla g_{n}(\theta_0)-\nabla g_n(\theta_m)|^2}{Z^2_0}\frac{\frac{1}{N}\sum^N_{n=1}\left(g^4_n(\theta_m)+g^4_n(\theta_0)\right)}{Z^4_0}\\
% &+C\frac{\left|\frac{1}{N}\sum^N_{n=1}f_n\nabla g_n(\theta_m)\right|^2}{Z^2_0}\frac{\frac{1}{N}\sum^N_{n=1}\left|g^2_n(\theta_m)-g^2_n(\theta_0)\right|^2}{Z^4_0}\\
% \leq &C|\theta_m-\theta_0|^2\,,
% \end{aligned}
% \]
% where we use \eqref{eqn_value_gnbound}-\eqref{eqn_Hessian_gnbound}, and \eqref{eqn:distance_bound_2} in the last inequality, and $C$ is a constant that only depends on $C_r,C_g,C_f$. Similarly, 
% \[
% \begin{aligned}
% &\mathbb{E}(|G_{m,2}-\mathbb{E}(G_{m,2}|\mathcal{F}_{m-1})|^2|\mathcal{F}_{m-1})\\
% \leq & 4\frac{\frac{1}{N}\sum^N_{n=1}|f_n|^2| g_{n}(\theta_0)-g_n(\theta_m)|^2}{Z^2_0}\frac{\frac{1}{N}\sum^N_{n=1}\left(\left|g_n(\theta_m)\nabla g_n(\theta_m)\right|^2+\left|g_n(\theta_0)\nabla g_n(\theta_0)\right|^2\right)}{Z^4_0}\\
% &+4\frac{\frac{1}{N}\sum^N_{n=1}| g_n(\theta_0)\nabla g_n(\theta_0)-g_n(\theta_m)\nabla g_n(\theta_m)|^2}{Z^4_0}\frac{\left|\frac{1}{N}\sum^N_{n=1}f_n(\theta_m)g_n(\theta_m)\right|^2}{Z^2_0}\\
% \leq &C|\theta_m-\theta_0|^2
% \end{aligned}
% \]
% where we use \eqref{eqn_value_gnbound}- \eqref{eqn_Hessian_gnbound}, and \eqref{eqn:distance_bound_2} in the last inequality, and $C$ is a constant that only depends on $C_r,C_g,C_f$. Thus, we have
% \[
% \mathbb{E}(|G_m-\mathbb{E}(G_m|\mathcal{F}_{m-1})|^2|\mathcal{F}_{m-1})\leq C|\theta_m-\theta_0|^2
% \]
% and
% \[
% \mathbb{E}(|G_m|^2|\mathcal{F}_{m-1})\leq C\left(|\theta_m-\theta_0|^2+|\nabla F(\theta_m)|^2\right)\,.
% \]
% Since $\left|\theta_m-\theta_0\right|^2\leq  \eta^2m\sum^{m-1}_{i=0}|G_i|^2$ and $|G_0|^2= |\nabla F(\theta_0)|^2$, we have
% \begin{equation}\label{eqn_fourth_term_bound_2}
% \begin{aligned}
% \mathbb{E}(|G_m|^2|\mathcal{F}_{m-1})\leq & C\left(\eta^2m\sum^{m-1}_{i=0}|G_i|^2+|\nabla F(\theta_m)|^2\right)\\
% \leq &\frac{1}{2}\frac{\log(N)}{N}\sum^{m-1}_{i=0}|G_i|^2+C|\nabla F(\theta_m)|^2\\
% \leq & C|\nabla F(\theta_m)|^2+\frac{C\log(N)}{N}\sum^{m-1}_{i=0}\left(1+\frac{1}{2}\frac{\log(N)}{N}\right)^{m-1-i}|\nabla F(\theta_i)|^2\\
% \leq & C|\nabla F(\theta_m)|^2+\frac{C\log(N)}{\sqrt{N}}\sum^{m-1}_{i=0}|\nabla F(\theta_i)|^2
% \end{aligned}
% \end{equation}
% where we use $\eta<\frac{1}{\sqrt{2}}\frac{\log^{1/2}(N)}{\sqrt{C}N}$ and $m<N$ in the second inequality, and $\left(1+\frac{1}{2}\frac{\log(N)}{N}\right)^{m-1-i}<\sqrt{N}$ in the last inequality.

% Plugging \eqref{eqn_fourth_term_bound_2} into \eqref{eqn:F_bound_2} and taking expectation on both sides, we have
% \[
% \begin{aligned}
% \mathbb{E}\left(F(\theta_{m+1})\right)\leq &\mathbb{E}\left(F(\theta_m)\right)-\frac{\eta}{C_r}\mathbb{E}\left(\left|\nabla F(\theta_m)\right|^2\right)\\
% &+C\eta^2\left(\mathbb{E}\left(|\nabla F(\theta_m)|^2\right)+\frac{\log(N)}{\sqrt{N}}\sum^{m-1}_{i=0}|\nabla F(\theta_i)|^2\right)
% \end{aligned}
% \]

% Finally, taking the summation in $1\leq m\leq M-1$, we have
% \[
% \begin{aligned}
% \mathbb{E}\left(F(\theta_{M})\right)\leq & \mathbb{E}\left(F(\theta_{0})\right)-\frac{\eta }{C_r}\sum^{M-1}_{m=0}\mathbb{E}\left(\left|\nabla F(\theta_{m})\right|^2\right)+C\eta^2\sum^{M-1}_{m=0}\mathbb{E}\left(|\nabla F(\theta_i)|^2\right)\\
% &+C\eta^2\frac{\log(N)}{\sqrt{N}}\sum^{M-1}_{m=1}\sum^{m-1}_{i=\lfloor m/N\rfloor*N}\mathbb{E}\left(|\nabla F(\theta_i)|^2\right)\\
% \leq & \mathbb{E}\left(F(\theta_{0})\right)-\frac{\eta }{C_r}\sum^{M-1}_{m=0}\mathbb{E}\left(\left|\nabla F(\theta_{m})\right|^2\right)+C\eta^2\sum^{M-1}_{m=0}\mathbb{E}\left(|\nabla F(\theta_i)|^2\right)\\
% &+C\eta^2\sqrt{N}\log(N)\sum^{M-1}_{m=1}\mathbb{E}\left(|\nabla F(\theta_m)|^2\right)\\
% \leq &\mathbb{E}\left(F(\theta_{0})\right)-\frac{\eta }{2C_r}\sum^{M-1}_{m=0}\mathbb{E}\left(\left|\nabla F(\theta_{m})\right|^2\right)\,,
% \end{aligned}
% \]
% where we use $\eta(1+\sqrt{N}\log(N))<1/(2CC_r)$ in the last inequality. This proves \eqref{eqn:gradient_F_bound_2}.
% \end{proof}
% %%%%%%%%%%%%%%%%%%%%%%%%%%%%%%%%%%%%%%%%%%%%%%%%%%%%%%%%%%%%%%%%%%%%%%%%%%%%%%%
% %%%%%%%%%%%%%%%%%%%%%%%%%%%%%%%%%%%%%%%%%%%%%%%%%%%%%%%%%%%%%%%%%%%%%%%%%%%%%%%









\end{document}


% This document was modified from the file originally made available by
% Pat Langley and Andrea Danyluk for ICML-2K. This version was created
% by Iain Murray in 2018, and modified by Alexandre Bouchard in
% 2019 and 2021 and by Csaba Szepesvari, Gang Niu and Sivan Sabato in 2022. 
% Previous contributors include Dan Roy, Lise Getoor and Tobias
% Scheffer, which was slightly modified from the 2010 version by
% Thorsten Joachims & Johannes Fuernkranz, slightly modified from the
% 2009 version by Kiri Wagstaff and Sam Roweis's 2008 version, which is
% slightly modified from Prasad Tadepalli's 2007 version which is a
% lightly changed version of the previous year's version by Andrew
% Moore, which was in turn edited from those of Kristian Kersting and
% Codrina Lauth. Alex Smola contributed to the algorithmic style files.
