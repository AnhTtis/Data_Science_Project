\section{Summary and outlook}
\label{conclusions}

This Colloquium gives an overview of the exciting recent developments along a new avenue of experimental and theoretical studies of the gravitational structure of hadrons, especially the proton. 

The gravitational form factors of the proton rose to prominence after the
works of Xiangdong Ji \cite{Ji:1994av,Ji:1996ek} 
illustrated how they can be used to gain insight
into fundamental questions such as: how much do quarks and gluons
contribute to the mass and the spin of the proton? Soon afterwards, Maxim Polyakov~\cite{Polyakov:2002yz} showed that they also provide information about the spatial distribution of mass and spin, and allow one to study the forces at play in the bound system.
These works triggered many follow-up studies and
investigations which have deepened our understanding
of proton structure.

Through matrix elements of the energy-momentum operator, the gravitational form factors of the proton and other hadrons have been studied in theoretical approaches including a wide range of models and in numerical calculations in the framework of lattice QCD. 
In broad terms, the simplest aspects of the EMT structure of the proton and other hadrons (such as the pion) have been understood from theory for some number of years, and first-principles calculations providing complete and controlled decompositions of the proton's mass and spin, for example, are now available. On the other hand, more complicated aspects of proton and nuclear structure, such as gluon gravitational form factors, the $x$-dependence of generalized parton distributions, and energy-momentum tensor matrix elements in light nuclei, have been computed for the first time in the last several years, as yet without complete systematic control, and significant progress can yet be expected over the next decade. Theory insight into these fundamental aspects of proton and nuclear structure is thus currently in a phase of rapid progress, complementing the improvement of experimental constraints on these quantities and, importantly, providing predictions which inform the target kinematics for future experiments.

The first experimental results, discussed in this colloquium, are based on precise measurements of the deeply virtual Compton scattering process with polarized electron beam, that determined both, the beam-spin asymmetry and the absolute differential cross section of $\vec{e}p \to ep\gamma$. Measurements covered a limited range in the kinematic variables which made it necessary to employ information from high-energy collider data to constrain the global data fit in the region that was not covered in the CLAS experiment. 
Consequently, large systematic uncertainties were assigned to the results. 

New experimental results on DVCS measurements with polarized electron beams at higher energy have recently been published from  experiments with CLAS12~\cite{CLAS:2022syx} and from Hall A at Jefferson Laboratory~\cite{JeffersonLabHallA:2022pnx}. They extend the kinematic reach both to higher and to lower values in $\xi$, and increase the range covered in $Q^2$. The latter will allow for more sensitive measurements of the $Q^2$ evolution of the DVCS cross section. These new data may also support application of machine learning techniques and artificial neural networks in the higher level data analysis as have been developed by several groups~\cite{Kumericki:2019ddg,Berthou:2015oaw,Grigsby:2020auv}.  

Ongoing experiments and future planned measurements that employ proton and deuterium (neutron) targets, spin-polarized transversely to the beam direction, have strong sensitivity to CFF $\mathcal{E}$. Precise knowledge of the kinematic dependence of $\mathcal{E}(\xi,t)$ is needed to measure the quark angular momentum distribution encoded in the GFF $J_q(t)$ of the proton~\cite{Ji:1996ek}, as defined in Sect.~\ref{sec-II.A}.

The plan to extend the Jefferson Lab's electron accelerator energy reach to 22~GeV would more fully open access to employing $J/\Psi$ production near threshold in a wide $t$ range, and some $\xi$ range to access the gluon part $D_G(t)$ of the proton's $D$-term.  

DVCS data from the COMPASS experiment at CERN with 160~GeV of oppositely polarized $\mu^+$ and $\mu^-$ beams~\cite{COMPASS:2018pup} reach to smaller $\xi$ values and into the sea-quark region. The average of the measured $\mu^+$ and $\mu^-$ cross sections allows for the determination of Im$\cal{H}$. Results from high statistics runs that cover the lower $x_B$ domain are expected in the near future. With these new data, the difference of $\mu^+$ and $\mu^-$ cross sections can also be formed to  obtain the charge asymmetry, which provides direct access to Re$\cal{H}$.

A long term perspective is provided by the planned Electron-Ion Collider projects in the US~\cite{AbdulKhalek:2021gbh,Burkert:2022hjz} and in China~\cite{Anderle:2021wcy}. The US project will extend the kinematic reach in $x_B > 10^{-4}$ and thus will cover with high operational luminosity up to $10^{34}\, \text{cm}^{-2}\,\text{s}^{-1}$ the gluon dominated domain. It features polarized electron and polarized proton beams, the latter longitudinally or transversely polarized, and light ion beams. The EIcC in China focuses on the lower energy domain with $x_B > 10^{-3}$ that connects more closely to the kinematics of the fixed target experiments at Jlab that operate at very high luminosity in the valence quark and the $q\bar{q}$-sea domain. 

Currently available data allowed for a pioneering first step into this emerging new field of the proton internal structure, complementing what has been learned in many detailed experiments over the past 70 years of studies of the proton electromagnetic structure, with the first result on the proton's mechanical structure. 

\begin{figure}[t!]
\includegraphics[width=1.0\columnwidth]{Figures/PressureScale.png}
\caption{\footnotesize Comparison of peak pressures inside various objects on earth, in the solar system, and in the universe.}
\label{PressureScale}
\end{figure} 
 
This new avenue of research has been rapidly developing theoretically, and the first experimental results on the proton firmly established the study of mechanical properties of sub-atomic particle as an exciting new field of fundamental science. Many objects on earth, in the solar system and in the universe are described by their equation of state, where the internal pressure plays an essential role. Some of these objects are listed in Figure~\ref{PressureScale}. The study discussed in this Colloquium adds the smallest object with the highest internal pressure to this list of objects that have been studied so far. The peak pressure inside the proton is approximately $10^{35}$ Pascal. It tops by 30 orders of magnitude the atmospheric pressure on earth. It even exceeds the pressure in the core of the most densely packed known macroscopic objects in the universe, neutron stars, which is given as $1.6\times 10^{34}$ Pascal in \cite{Ozel:2016oaf}. Other subatomic objects such as pions, kaons, hyperons, and light and heavy nuclei may be subject of experimental investigation in the future. The scientific instruments needed to study them efficiently are in preparation. 

The gravitational form factors
provide the key to address fundamental questions 
about the mass, spin, and internal forces 
inside the proton and other hadrons. 
Theoretical, experimental and phenomenological
studies of gravitational form factors provide exciting insights. 
In this emerging new field, there are many
inspiring lessons to learn and there is much to
look forward to.
 


\begin{acknowledgments}

The authors cannot mention the names of all
colleagues and acknowledge all discussions
during and prior to the preparation of this
Colloquium. 
We wish to name only Maxim Polyakov 
who until his untimely death in August 2021
influenced or initiated many of the reviewed~topics. \\
%
Special thanks go to Joanna Griffin 
for professional assistance with 
preparing diagrams and figures.
%
This material is based upon work supported by the U.S. Department of Energy, Office of Science, Office of Nuclear Physics under contract DE-AC05-06OR23177.
P.~Schweitzer was supported by the NSF grant under the Contract No.~2111490. 
P.~Shanahan was supported in part by the U.S.\ Department of Energy, Office of Science, Office of Nuclear Physics, under grant Contract Number DE-SC0011090 and by Early Career Award DE-SC0021006, by a NEC research award, and by the Carl G and Shirley Sontheimer Research Fund. This work was supported in part by the Department of Energy within framework of the QGT Topical Collaboration.
\end{acknowledgments}



\section*{Acronyms}
{A heavy use of acronyms can make a text 
difficult to read for readers not familiar 
with the field, while no use of acronyms can 
make it unreadable for those who are familiar.  
The authors found it indispensable to
introduce a number of acronyms which are
explained at their first occurrence and
are collected here for convenience.

\ \\
\begin{tabular}{ll}
{\it AM}    & angular momentum\\
{\it BSA}   & beam spin asymmetry\\
{\it CFF}   & Compton form factor \\
{\it DIS}   & deep inelastic scattering \\
{\it DVCS}  & deeply virtual Compton scattering \\
%{\it EIC}   & Electron-Ion Collider\\
{\it EMT}   & energy momentum tensor \\
{\it FF}    & form factor \\
%{\it GDA}   & generalized distribution amplitude\\
{\it GFF}   & gravitational form factor \\
{\it GPD}   & generalized parton distribution \\
{\it JLab}  & Jefferson Lab \\
{\it PDF}   & parton distribution function \\
{\it QCD}   & quantum chromodynamics \\
{\it QED}   & quantum electrodynamics \\ 
{\it TCS}   & time-like Compton scattering \\
\end{tabular}

}