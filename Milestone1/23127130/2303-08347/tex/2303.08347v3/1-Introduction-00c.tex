\section{Introduction}
\label{Sec-1:intro} 

This Colloquium reviews the recent theoretical 
and experimental progress in studies of the 
gravitational form factors of the proton and other 
hadrons, which has shed fascinating new light on 
the proton's structure and its mechanical properties. 
To place this emerging area in context, the history 
of proton structure and its description in 
quantum chromodynamics are first reviewed.

\vspace{-4mm}

\subsection{Anomalous magnetic moment} \vspace{-2mm}
\label{Sec-1A:magnetic-moment}
 
Soon after the proton \cite{Rutherford:1919fnt} and neutron 
\cite{Chadwick:1932ma} were established as the constituents 
of atomic nuclei, experiments showed that these spin-$\frac12$
particles with nearly equal masses 
$M_N \simeq 940\,{\rm MeV}/c^2$ are not pointlike 
elementary fermions. If they were, the Dirac equation would 
predict the magnetic moment of the proton to be one nuclear 
magneton  $\mu_N\equiv e\hbar/(2M_N)$ and that of an 
electrically neutral particle like the neutron to be zero. 
Instead, the proton magnetic moment was measured to be about 
$\mu_p \simeq 2.5\,\mu_N$ \cite{Frisch-Stern}. Later, the 
neutron magnetic moment was found to be 
$\mu_n \simeq -1.5\,\mu_N$ \cite{Alvarez:1940zz};
for the modern values of the magnetic moments see \cite{ParticleDataGroup:2022pth}.
These experiments have shown that the 
nucleon is not a pointlike elementary particle,  
giving birth in 1933 to the field of proton structure. 

Protons and neutrons are hadrons, particles that feel the 
strong force, which is the strongest interaction known in 
nature. Based on approximate isospin symmetry, they are 
understood as partnered (isospin up/down) states, referred 
to collectively as the nucleon \cite{Heisenberg:1932dw}.
As the constituents of nuclei, nucleons are responsible 
for more than $99.9\,\%$ of the mass of matter in the 
visible universe, and have naturally become the most 
experimentally studied objects in hadronic physics.


\begin{figure}
\includegraphics[width=0.9\columnwidth]{Figures/Figure1_D8.jpg}

\vspace{-6mm}

\caption{\label{Fig1}
(a) The elastic electron-proton scattering process
in which the electromagnetic form factors (FFs)
are measured.
(b) Inclusive deep inelastic scattering (DIS) 
where the proton is dissociated into a 
final state consisting of unresolved hadrons. 
In the Bjorken limit $p\cdot q\to \infty$ 
and $Q^2 = -q^2 \to \infty$ with 
$x_B=Q^2/(2p\cdot q)$ fixed, DIS 
is interpreted  
in the so-called infinite-momentum frame 
as the scattering of electrons off pointlike
quarks carrying the fraction $x$ of the nucleon's
momentum, where $x=x_B$ up to corrections 
suppressed by $M_N^2/Q^2$.}
\end{figure}

\vspace{-4mm}

\subsection{The proton's finite size}\vspace{-2mm}
\label{sect-I.B.}

An important milestone in the field of nucleon structure 
was brought by studies of elastic electron-proton 
scattering, shown in Fig.~\ref{Fig1}a,
which revealed early insights into the proton's size.
The deviations in scattering data from expectations for pointlike particles are encoded in terms of  form factors (FFs) defined 
through matrix elements of the electromagnetic current operator,
$\la p',\vec{s}^{\,\prime} |J^\mu_{em}|p,\vec{s}\ra$,
where $|p,\vec{s}\ra$ is the initial state of the proton with momentum $p$ polarized along the $\vec{s}$ direction, and analogously for the final proton state. 

These FFs would be constants for pointlike particles,
but they were found to be pronounced functions of the 
Mandelstam variable $t=(p^\prime-p)^2$. A spin-$\frac12$ 
particle has two electromagnetic FFs, $F_1(t)$ and $F_2(t)$,
defined such that $F_1(0)$ is the electric charge in units of 
$e$, and $F_2(0)$ is the anomalous magnetic moment, i.e.,\ the
deviation from the value predicted by the Dirac equation, in 
units of $\mu_N$. 
Knowledge of the $t$-dependence of electromagnetic FFs
allowed information about the spatial distributions
of electric charge and magnetization to be inferred 
\cite{Sachs:1962zzc}
(more discussion of this interpretation can be found in~\cite{Lorce:2020onh,Chen:2022smg,Chen:2023dxp}). 
This led to the first determination of the proton charge 
radius of $(0.74\pm 0.24)\,$fm \cite{Mcallister:1956ng}. 
These experiments have continued to this day, and, using 
a variety of experimental techniques, they resulted in 
a much more precise knowledge of the proton's charge 
radius~\cite{ParticleDataGroup:2022pth}.



\subsection{Discovery of partons}
\label{Sect_IC}
The 1950s witnessed immense progress in accelerator and detection
techniques followed by a proliferation of discoveries of strongly 
interacting particles and resonances, including particles like the 
antiproton, % 1952
$\Delta$, % 1950s
and $\Xi$, % 1952 (charged); 1959 (neutral) 
% and $\Omega$, % 1964 (AFTER THE CITED REVIEW)
see the early review \cite{EarlyReview1961}.
On the theory side, this led to the development of the quark 
model \cite{Gell-Mann:1964ewy,Zweig:1964ruk} in which hadrons 
are classified according to quantum numbers which are understood 
to arise from various combinations of ``quarks''. The ``quarks'' 
in this model were group-theoretical objects, and their dynamics 
were unknown.

The next milestone was brought by high-energy experiments carried
out at the Stanford Linear Accelerator, where the Bjorken scaling 
predicted on the basis of current algebra and dispersion relation 
techniques \cite{Bjorken:1968dy} was observed in inclusive deep 
inelastic scattering (DIS) \cite{Bloom:1969kc}.
The response of the nucleon in DIS is described by structure
functions which, on general grounds, are functions of the 
Lorentz invariants $p\cdot q$ and $Q^2 = -q^2$, where $p^\mu$ is 
the nucleon four-momentum and $q^\mu$ the four-momentum transfer, 
see Fig.~\ref{Fig1}b. Bjorken scaling is the property that, 
in the high-energy limit $p\cdot q\to\infty$ and $Q^2\to\infty$
with their ratio fixed,
the structure functions are, to a first approximation, functions
of a single variable $x_B=Q^2/(2p\cdot q)$ which on kinematical 
grounds satisfies $0<x_{B}<1$. 

The physical significance of this non-trivial observation was
interpreted in the parton model \cite{Feynman:1969ej}, where the DIS 
process proceeds as shown in Fig.~\ref{Fig1}b, namely the electrons
scatter off nearly free electrically-charged pointlike particles 
called partons, with a cross-section that can be calculated in 
quantum electrodynamics (QED). The structure of the nucleon in 
DIS is described in terms of parton distribution functions (PDFs),
depicted by the green ellipse in Fig.~\ref{Fig1}b. 
(More precisely, PDFs are defined after squaring the amplitude in Fig.~\ref{Fig1} and summing over the complete set of states $X$.)
In modern
terminology, the PDFs in unpolarized DIS are denoted $f_1^a(x)$, 
with $a$ labelling the type of parton. More precisely,
$f_1^a(x)\,\ud x$ is the probability to find a parton of 
type $a$ in the initial state inside of a nucleon moving with 
nearly the speed of light 
(an appropriate picture in DIS where $x\approx x_B$) and carrying 
a fraction of the nucleon's momentum in the interval $[x,\;x+\ud x]$. 
It was soon realized that the electrically charged partons, 
identified with quarks and antiquarks, carry only half of the 
nucleon's momentum between them.



\subsection{Colored quarks and gluons, QCD, and confinement}
\label{sect-I.D}

The discovery of proton substructure and the development of the 
parton model were key to establishing quantum chromodynamics
(QCD) as the theory of the fundamental interaction between 
quarks carrying $N_c=3$ different color charges (and antiquarks
carrying the corresponding anticharges)
\cite{Gross:1973id,Politzer:1973fx,Fritzsch:1973pi}.
The color forces are mediated by the exchange of spin-1 gluons 
which also carry color charges (as opposed to electrically neutral 
photons which mediate interactions in QED). 
Evidence for the existence of gluons has been found in the study 
of $e^+e^-$ annihilation processes \cite{TASSO:1980lqw}. Being 
electrically neutral, the gluons are ``invisible'' in interactions 
with electrons, and account for the missing half of the 
proton momentum in DIS.

The QCD Lagrangian is given by
\be\label{Eq:Lagrangian}
      {\cal L} = \sum_q \overline{\psi}_q(i\slashed{D}+m_q)\psi_q
      - \tfrac{1}{4}\,F^2,
\ee 
where $\overline{\psi}_q$ and $\psi_q$ denote the quark and 
antiquark fields and $m_q$ denotes the current quark masses. 
The summation runs over the quark flavors 
$q\in\{u,\,d,\,s,\,c,\,b,\,t\}$. 
The covariant derivative is defined as 
$iD_\mu = i\partial_\mu+g A_\mu^cT^c$ and 
$F^2=F^c_{\mu\nu}F^{c\mu\nu}$ with 
$F^c_{\mu\nu}=\partial_\mu A^c_\nu-\partial_\nu A^c_\mu+g f^{cde}A^d_\mu A^e_\nu$.
%, where $T^c$ are the
Here $A_\mu^c$ are the gauge (gluon) fields and $T^c$ the generators in the fundamental representation 
of SU($N_c$) with $c\in\{1,\,\dots,\,N_c^2-1\}$ and $f^{cde}$ 
are the structure constants of the SU($N_c$) group. 
Non-abelian gauge theories like QCD are renormalizable
\cite{tHooft:1972tcz} with the coupling constant
$\alpha_s(\mu)=g(\mu)^2/(4\pi)$ depending on the 
renormalization scale $\mu$. 
% For example, one has at the typical hadronic scale $\alpha_s(1\rm{~GeV})\approx 0.4$, while at the mass of the $Z$ boson $\alpha_s(91\rm{~GeV})\approx 0.12$. 
When it comes to describing hadrons, the scale is $\mu \sim 1\,\rm{GeV}$ and $\alpha_s(\mu)$ is of order unity. The interaction is thus strong and the solution of (\ref{Eq:Lagrangian}) requires nonperturbative techniques.
However, in high-energy processes such as DIS, where the 
renormalization scale is identified with the hard scale of
the process, $\alpha_s(Q)$ decreases with increasing $Q$ reaching $\alpha_s(91\rm{~GeV})\approx 0.12$ at the scale of the $Z$-boson mass. 
 This property, known as asymptotic freedom, explains why 
quarks, antiquarks and gluons appear in such reactions as 
nearly free partons to a first approximation. The fact that 
free color charges are never observed in nature
gave rise to the confinement hypothesis, whose theoretical 
explanation is still an outstanding open question.


\subsection{Proton mass, spin and \boldmath $D$-term}

While the fundamental degrees of freedom and their interaction described in terms of the Lagrangian (\ref{Eq:Lagrangian}) are well-established, many questions remain open. For instance, 
% in a ``static'' quark model one would naively attribute the spin $\frac12$ of the nucleon to the spins of the quarks. In nature, due to the relatively light $u$- and $d$-quarks being confined within distances of ${\cal O}(1\,{\rm fm})$, Heisenberg's uncertainty principle implies an ultra-relativistic motion of the quarks. It must be expected that, e.g., the orbital motion of quarks has an important role in the spin budget of the nucleon. At the quantitative level, the nucleon spin decomposition is, however, still not known precisely \cite{Ji:2020ena}. The central question in QCD and it is the question that motivates this colloquium,is how 3 quarks with masses of 2-5 MeV$/c^2$ can bind to give a nucleon mass of about 940 MeV$/c^2$. In other words, the mass of the visible universe comes essentially from QCD and has little to do with the Higgs mechanism. 
% \peter{In particular, we want to address this issue by determining how mass and stress are distributed in the proton.}{I do not think it makes sense to mention stress here. I would remove this.}
the proton and neutron quantum numbers arise from combining 3 light quarks, $uud$ and $udd$, whose masses in the QCD Lagrangian (\ref{Eq:Lagrangian}) are explained by the Brout-Englert-Higgs mechanism \cite{Englert:2014zpa,Higgs:2014aqa}. The smallness of $m_u \sim 2\,{\rm MeV}/c^2$ and $m_d \sim 5\,{\rm MeV}/c^2$, however, gives rise to one of the central questions of QCD, namely how does the nucleon mass of 940 MeV$/c^2$ come about? 
(A wide-spread misconception is that $m_u+m_u+m_d \sim 9\,{\rm MeV}/c^2$ only explains about $1\,\%$ of the proton mass. This is incorrect, as in QCD the quark mass contribution is due to the operator $m_q\,\overline{\psi}_q\psi_q$ which 
includes virtual quark-antiquark pair contributions, leading to 
a much larger fraction (about 10-15\,\%) of the proton mass as will be discussed in Sec.~\ref{mass-spin}.)

Another central question concerns the proton spin. In a ``static'' quark model one would naively attribute the spin $\frac12$ of the nucleon to the spins of the quarks. In nature, due to the relatively light $u$- and $d$-quarks being confined within distances of ${\cal O}(1\,{\rm fm})$, Heisenberg's uncertainty principle implies an ultra-relativistic motion of the quarks. It must be expected that, e.g., the orbital motion of quarks has an important role in the spin budget of the nucleon. At the quantitative level, the nucleon spin decomposition is, however, still not known precisely \cite{Ji:2020ena}. 

The answers to these questions lie in the matrix elements of the energy-momentum tensor (EMT), 
an operator in quantum field theory of central importance 
that is associated with the invariance of the theory under 
spacetime translations. These matrix elements 
encode~key information including the mass 
and spin of a particle, the less well-known but equally 
fundamental $D$-term ($D$ stands for the German word {\it Druck} meaning pressure), as well as information about the 
distributions of energy, angular momentum, and various 
mechanical properties such as, e.g., internal forces inside the system. 
These properties are encoded in the gravitational form factors. In the standard model (plus gravity) the EMT couples to gravitons, so the direct way to measure its matrix elements would be graviton-proton scattering.  
Since the gravitational interaction between 
a proton and an electron is
(at currently achievable lab energies) $10^{-39}$
times weaker than their electromagnetic interaction, direct 
use of gravity to probe proton structure is impossible 
in electron-proton scattering, 
and in fact in any accelerator experiment in the foreseeable future.
However, we have learned how to apply indirect methods to acquire information about the EMT through studies of hard exclusive reactions.
The purpose of this Colloquium is to review the progress in theory, experiment, and interpretation of the EMT matrix elements. While the main focus is on the proton, also other hadrons will be discussed to provide a wider context and improve understanding.

% \peter{A simple explanation of why electron scattering can measure matrix elements of $T_{\mu\nu}$ is that we study the short distance structure of 
% $\langle h|j_\mu(0)j_\nu(x)|h\rangle$, and the quark $T_{\mu\nu}\sim \bar\psi\gamma_\nu\partial_\nu\psi$ is the lowest dimension operator that appears.} 
% {\color{orange}This text is taken from the suggestion by RMP editor Thomas Schaefer. The notations should be modified to conform with our choices. Maybe we  could add that this corresponds to the $J=2$ coupling of the graviton?}
% \peter{}{This is wrong. The story is more complex: in Bjorken-limit  one obtains from $\langle h'|j_\mu(0)j_\nu(x)|h\rangle$ the non-local operator $\langle h'|\Psi_q(0)\gamma^+\Psi(z)|h\rangle$ with light-like $z$ and not indicated gauge link. This expectation value of a nonlocal operator contains A LOT of information, including information on the EMT (via 2nd Mellin moment which reduces it to a local operator). I mention this here to make sure we agree among ourselves, but I do not think it makes sense to explain it to a colloquium reader. 
% The argument with $J=2$ in the paragraph above Eq.~(26) explains it more adequately. Maybe we simply tell the editor that it is explained there?}
