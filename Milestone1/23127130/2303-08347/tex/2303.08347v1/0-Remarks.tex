\newpage
%
%\peter{}{This column is intentionally left blank for better overview.}

\newpage \noindent
This is created to add and address remarks easily (notice: there are still many remarks left to address in the text)




\ \\
{\bf To be addressed}

\begin{itemize}

\item {\bf Cleaning of latex source}\\ \ \\
    \begin{tabular}{l|c|c}
    part        & \ status \ & \ who \ \\ \hline
    Sec.~I      & clean & PS \\
    Sec.~II     & clean & PS \\
    Sec.~III    & clean & CL \\
    Sec.~IV     & clean & CL\\
    Sec.~V      & clean & CL\\
    Sec.~VI     & clean & PS \\
    Sec.~VII    & clean & VB \\
    Acknow.     & clean & VB \\
    Acronyms    & clean & PS \\
    References  & clean & PS 
    \end{tabular} \ \\

\item {\bf NEWEST comments on figures}\\
    Fig.~1: 
    \cedric{}{["FF" should be replaced with "FFs"]} \\
    Fig.~2: 
    \cedric{}{["FF" should be replaced with "FFs"]} \\
    Fig.~3:
    \cedric{}{[Arrows are missing on all orange lines. The font size for "GPDs" is a bit too large compared to other figures. The quark loop on the right panel should be in orange and have arrows. The label $\gamma$ could be moved closer to the left tip of the wavy line]} \\
    Fig.~4:
    \cedric{}{[Arrows are missing on the orange quark lines. The purple blob should be green like the other non-perturbative parts and contain the "DA". "GPD" should be replaced with $GPDs$]}\\
    Fig.~15:
    \cedric{}{[Change "Core White Dwarf" into "Core of a White Dwarf". Also, nucleons and neutron stars have actually similar core pressure]} \volker{}{My (elementary) understanding of the physics of n-stars: Gravitation overcomes electromagnetic interaction of atoms, while it does not overcome the nuclear forces between nucleons. Then pressure in N-stars should  result from forces between neutrons (nucleons) and not from forces between quarks, which is what DVCS measures. Since binding energies between nucleons in nuclei is O(10MeV) and between quarks O(100MeV) I would expect higher pressure inside protons compared to pressure in n-stars. I'd be happy to be proven wrong.....}   
    
\end{itemize}

\newpage
{\bf Potentially resolved items}

\begin{itemize}

\item  Fig.~5: \peter{}{In the bottom panel, the $y$-axis shows only one single value, namely 1. Perhaps a second value could be added, e.g.\ at 0.5, to make the scale on that figure unambiguous.}\volker{}{I think the scale is unambiguous. It has at the bottom 0.1 in addition to the 1.0. In between there are dashed lines for every step 0.2, 0.3, ...}   
\peter{}{Sorry. I overlooked the second value $10^{-1}$ on the bottom of the $y$-axis. In principle, it is okay.}

\item {\bf Acronym JLab (?)}\\
    In the current draft, we use ``Jefferson Lab''
    11 times in the main text, and we use 2 times
    ``JLab'' {\it without} defining this Acronym.\\
    I would be comfortable using JLab throughout
    and adding it in the list of acronyms.
    But I am flexible.\\
    In any case, we should unify the notation!!\\
    Done (we use JLab, added in Acronym List)


\item {\bf Fig.~1} \volker{}{Completed. }

\item {\bf Fig.~4} \peter{}{Please, add the labels GPDs on the green blobs.} \volker{}{Completed.}

\item {\bf Fig.~10} \cedric{}{[Write $\mathcal C_{\mathcal H}(0)=0$ in the lower plot]} \peter{}{The large font of ``$\mathcal C_{\mathcal H}(0)=0$'' gives the impression this a ``title'' for the entire figure. To avoid this, maybe you can choose a smaller font for ``$\mathcal C_{\mathcal H}(0)=0$'' and include a line (analog to the lines of "Global fit" and other labels). }\volker{}{Done, I think. }

\item {\bf Comments on $A_a(0)$}
\peter{}{Cedric added the remark that the $A_a(0)$ are related to $\int dx x f_1^a(x)$ below Eq.~(24). I moved this remark to the paragraph below Eq.~(5) and rephrased below Eq.~(24). Please, check and edit as needed.}

\item {\bf Explanation of ``dilations''}
\peter{}{I believe the term was not explained. I made a minimal edit at the beginning of Sec.~\ref{Subsec:trace-anomaly}. Please, check.}

\item {\bf Acronyms}
\peter{}{It is possible to add a
"section" where the 
acronyms are explained. Could be 
beneficial for non-expert readers.
We can decide about it.
I pasted the instructions (see below)
and tried to identify our acronyms.
Many are inevitable, but a few could
be eliminated. \\
Done, though see above ``JLab''}


\item {\bf Acknowledgements}
\peter{}{I added some text. I do not
know what you think, but in my opinion
we wouldn't be writing this Colloquium 
without the works and ideas of
Maxim Polyakov. I tried to mention this. 
But we can rewrite or remove it, if you 
do not like it.}\volker{}{I think we absolutely need to mention Maxim)}

\item {\bf Figures}
\volker{}{All done. One new figure 16 added "PressureScale", and another Figure 15 on CFF H at EIC. The latter was later removed again. }




\item {\bf New comments on Figures (2nd iteration)}
\peter{}{

{\bf Fig.~1:} what if we show elastic scattering in Fig.~1, left panel? \\
Reasons: (i) The generic ``unfactorized'' DIS process is not really needed, because we have the factorized version in the right panel. (ii) Historically it was of importance (on same footing as DIS, also a Nobel Prize.) 
(iii) We now describe in the text in a cumbersome way what elastic scattering is, so we could streamline that and be more ``pedagogical''.\volker{}{Done!}

{\bf Fig.~2:} are the  arrow and $t$ needed? I would propose to remove them. \volker{}{Done!}

{\bf Fig.~3:} The right panel is the "unfactorized process". The process factorized in terms of GDAs looks differently. It is a ``handbag rotated by 90 degrees.'' I added a figure for illustrative purposes.}\volker{}{Done!} 

\item {\bf Style: ``we'' or ``not we''}\\
\phiala{}{First vs third person should be consistent throughout... looks to me like most sections do not use first-person.}
\peter{}{I wrote some of the ``we''. I will try to remove it.}



\item {\bf Merging Sec.~II and IV}\\
There is a number of benefits of doing this, see discussions in the chat. 
\peter{}{I made a first try. More edits might be necessary, and Cedric may be better qualified for that than I am. Some of the text added by Cedric in response to Phiala's comments (e.g. regarding insufficient connectedness with other parts) can now perhaps be streamlined, because it is "all in context".}
\peter{}{The merging seems successful to me.}

\item {\bf Title} \\
\peter{}{In the text we agreed to speak of GFFs. Shouldn't we change accordingly the title to ``Gravitational Form Factors of the Proton'' or so?} \volker{}{sounds good to me, except that we had agreed with editor on title .. but that may be ok }\cedric{}{It shouldn't be too hard a job for the editor to change a title ;) I agree it is better (and shorter and easier to read) like it is now.}\volker{}{It is done}

\item {\bf Outline}\\ 
\cedric{}{[Will there be an outline or is it simply replaced by the table of contents ?]}
\peter{}{An outline does not seem compulsory. 
% I found one recent example WITH an outline \href{https://journals.aps.org/rmp/pdf/10.1103/RevModPhys.94.041001}{[see link no.~1]} and another recent example WITHOUT an outline \href{https://journals.aps.org/rmp/pdf/10.1103/RevModPhys.94.031003}{[see link no.~2]}. In our case, we have a couple of ``inversions''. E.g. we mention ``mechanical properties'' in Sec. VI and explain them in the interpretation in Sec. VII. I would say that we either pay attention to such ``inversions'' to avoid confusing the reader, or we add an outline. For now, I created a specific subsection in the Introduction for an outline. I think it would be helpful for the reader. But we can contemplate about this point, and make up our mind later.
My concern was that in many places ``mechanical properties'' are mentioned before they are introduced which could be resolved with an outline.}
\cedric{}{When we mention ``mechanical properties'', we could just add a few words indicating that they'll be discussed a bit later. I think that would be smoother for the reader.}
\peter{}{I like this idea. I added a brief mention of ``mechanical properties'' in the Introduction, and removed the ``Outline Subsection''. That could be the solution, as far as I am concerned.}\volker{}{Sounds good to me too.}

\item {\bf Section numbering in Introduction}\\ \peter{}{I think we can have real subsections in the Introduction. I slightly changed the organization. }

\item {\bf Table of Contents}
\volker{}{I think we can save some space by eliminating the table of content, adding a short abstract, and adding an outline ($\approx 1$) paragraph?.  }
\peter{}{It seems a table of contents is mandatory, but does not count towards the page limit. Here are two examples:
\href{https://journals.aps.org/rmp/pdf/10.1103/RevModPhys.94.041001}{[see link no.~1]} and \href{https://journals.aps.org/rmp/pdf/10.1103/RevModPhys.94.031003}{[see link no.~2]}.
Both have table of contents. In one of them, the references start on page 23 (!) but the first 1.5 pages is front matter, abstract and table of contents. Perhaps going 1/2 page
over page limit is acceptable...}


\item {\bf Abstract}\\
\peter{}{An abstract is needed where we
must avoid acronyms. Length limit $<$ 
one paragraph (whatever this means) per
the instructions:
\href{https://journals.aps.org/rmp/authors}{https://journals.aps.org/rmp/authors}.
A minimal abstract is created as place holder.}



\item {\bf Bjorken scaling} 
This is a detail in Sec.~I. I tried to keep that part short and historically correct. The comments became bulky, so I put it here. The previous text and comments were: 

The next milestone was brought by high-energy experiments carried
out at the Stanford Linear Accelerator,
% (SLAC) WE DO NOT USE THIS ACRONYM
where the predicted Bjorken scaling \cite{Bjorken:1968dy} was observed in inclusive 
%deep-inelastic electron-nucleon scattering (DIS) \phiala{}{DIS first appears in caption 1 before this}
DIS \cite{Bloom:1969kc} 
%(Nobel Prize 1990, J. I. Friedman, H. W. Kendall, R. E. Taylor)
(Friedman, Kendall, Taylor, Nobel Prize 1990). 
The response of the nucleon in DIS is described by 
structure functions which, on general grounds, are functions of the 
Lorentz invariants $p\cdot q$ and $Q^2 = -q^2$, where $p^\mu$ is the
nucleon four-momentum and $q^\mu$ the four-momentum transfer, 
see Fig.~\ref{Fig1}a. Bjorken-scaling is the property that, 
in the high-energy limit $p\cdot q\to\infty$ and $Q^2\to\infty$
with their ratio fixed,
the structure functions are, to a first approximation, functions
of a single variable $x_B=Q^2/(2p\cdot q)$ which on kinematical grounds
satisfies $0<x_{B}<1$. 
\phiala{}{We say Bjorken scaling was observed but then it is much later in the paragraph where we actually explain what it is. Perhaps "The next milestone was brought by high-energy deep-inelastic electron-nucleon scattering (DIS) experiments \cite{Bloom:1969kc} (Friedman, Kendall, Taylor, Nobel Prize 1990) carried out at the Stanford Linear Accelerator. The response of the nucleon in DIS is described by structure functions which, on general grounds, are functions of the Lorentz invariants $p\cdot q$ and $Q^2 = -q^2$, where $p^\mu$ is the nucleon four-momentum and $q^\mu$ the four-momentum transfer, see Fig.~\ref{Fig1}a. Experimentally, so-called Bjorken-scaling \cite{Bjorken:1968dy} was observed: in the high-energy limit $p\cdot q\to\infty$ and $Q^2\to\infty$ with their ratio fixed, the structure functions are, to a first approximation, functions of a single variable $x_B=Q^2/(2p\cdot q)$ which on kinematical grounds satisfies $0<x_{B}<1$." }
\peter{}{The story is complex and requires more space that we can afford to use. It was really predicted (and explained based on current algebra) by Bjorken who had to force the experimentalists to carry out the measurements (who got Nobel Prizes, and he did not). What we mean by ``explanation'' nowadays was Feynman's parton model, i.e.\ it is perhaps more correct to speak of an ``interpretation.'' If we would like to stress the historical developments, then perhaps we can cite an early Rev.~Mod.~Phys.\ from the 1970s to give credit to the developments, and keep it reasonably short here....?} \volker{}{Let's not forget that this is part of an introduction to a review of a new development, it is not part of a review, which there are already many, so we should keep it short...}

\peter{}{In the main text, I made a few edits. I believe it is now  historically more precise, and still short. At this point, I am mainly driven by keeping this short. If somebody thinks that the subject should be presented in a different way or with more detail, then please feel free to rewrite.}

\item {\bf citation for gluon discovery at the end of Sec.~IC.}
\phiala{}{consider citation}
\peter{}{I added: \cite{TASSO:1980lqw}.
There are 4 earlier DESY papers on the gluon discovery in 3-jet events some of which were published in 1979. It would take 2 lines of text to cite them all.... I therefore choose here a somewhat later paper where they confirmed the spin-1 of the gluon. I hope you agree. A nice (personal) account of the early history by John Ellis can be found in: 1409.4232 [hep-ph].} \cedric{}{I concur :)}

\item {\bf\boldmath $x_B$ and $x$.}
      \peter{}{In the caption of Fig.~1, should it be $x$ or $x_B$? Do we always pay attention to the distinction between these two? I have to confess, that I did not pay attention so far and might have confused it in some places....}
\cedric{}{I think in the figures we mean $x$ from Feynman's parton model. This would also be more consistent with GPDs where $x_B$ is related to $\xi$ and not $x$.}
\peter{}{I tried to fix that. I believe it is okay now,}


\item {\bf Comments on Length of Sec.~III}\\
    \volker{} {This section now far exceeds the 1 page limit. What could we eliminate?}
    \peter{}{Yes, we planned 1 page for this section. This was unrealistic. We reshuffled some material which has shortened other parts. We {\it maybe} could try to reduce by 1 column.}
    \volker{}{let us wait until the paper is in final draft to see if the section needs to be shortened}

\item {\bf DVCS with positrons in Sec.~III} \volker{} {I added a very short paragraph on the use of positron beams on extracting $D_q(t)$, please comment.}
    
\item {\bf Factorization (!)}
    \peter{}{[$\dots$] I am not sure about $J/\psi$ threshold production.}
    \volker{}{Maybe the paper Kharzeev, Dmitri E (2021), “Mass radius of the proton,” Phys. Rev. D 104 (5), 054015, arXiv:2102.00110 [hep-ph] and references their has something relevant to say about this? }
    \peter{}{Yes, Kharzeev speaks of factorization. I am not sure there is a proof to all orders like for DVCS or DVMP
    [$\dots$] Regarding
    $\pi$-production and transversity GPDS: this is unrelated to GFFs, and we maybe can skip it? This would solve the problem of how to comment on the non-factorization of this process. \volker{}{Sounds good to me.}}\cedric{}{Probably we should also mention the paper by Feng et al. [Sun:2021gmi] where a pQCD calculation challenges the relation between gluon GFFs and near threshold quarkonium production.} 
    \peter{}{Thanks for bringing this up. I included a cautious remark about $J/\psi$ threshold production and added this and two more references.}


\item {\bf Volker's comment on $D$-term of atoms in Sec..~VII}
\volker{}{Do we really want to depart from our main focus on the "proton" as the title says? It may confuse the non-expert reader. While the pion that we also briefly discuss is at least a hadron, that is not the case for the hydrogen atom or in general atoms as systems (I think). Maybe we could even eliminate all discussion on non-nucleonic systems. What does everyone think?}
\peter{}{I think this is important. Ji et al oppose
the pressure interpretation with the argument that the
$D$-term is positive for atoms which are of course stable,
and our interpretation and allusions to stability make 
no sense. See \cite{Ji:2021mfb,Ji:2022exr}. 
Here I try to explain why they are right about atoms,
but we are right about hadrons. To best of my knowledge,
the above argument is ``original'' and has not been
published anywhere, and I would like to include it here. 
Also, if they ask Ji or one of his followers to referee
this colloquium, then this argument might help to get 
us through the review process. \volker{}{Ok then so be it.}}

\item {\bf Deconvolution and shadow GPDs}\\
    \cedric{}{A model-independent extraction of GPDs is likely out of question due to ambiguities associated with the deconvolution problem. Do we mention the recent results from Mezrag et al. about shadow GPDs? [Bertone:2021yyz]}
    \peter{}{Bertone:2021yyz talk only about DVCS and conclude that ``only multi-channel analysis of experimental data beyond leading order, over wide kinematic domains accessible in collider experiments, and within a complete framework such as PARTONS may provide the needed leverage to quantitatively constrain GPDs. Double DVCS offers a direct access to GPDs at $x\neq \xi$ and seems a natural candidate to make the deconvolution well-defined.'' I added the citation to that paper, but would hesitate to explain what a shadow GPD is. It may be too technical for a colloquium. There is certainly a lot more work needed. E.g. Bertone:2021yyz do not include the $D$-term!}
    \volker{}{I would caution against including blanket statements about collider experiments + PARTONS and DDVCS to do the job of quantitatively constraining GPDs. The EIC luminosity will not be high enough, by far, to make even significant contribution to DDVCS. While the EIC covers the range $0.0001 < x_B < 0.05$ well for DVCS, this is only 5\% of the kinematic space, while 95\% of the phase space it poorly covered, where much higher luminosity is needed that may only be achieved from fixed target experiments.} 

\item  {\bf more comments on atoms}\\
\cedric{}{[H-atom and proton are clearly different systems, but it does not allow us to evade the problem of positive $D$-term in my understanding]}
\peter{}{Totally agree. Should one add anything here to stress something?}\volker{}{Could one just add something like (better formulated): "For hadronic systems, like protons, baryons in general, and mesons, for which the D-term has been computed within models and in LQCD, it has always been found negative $D(0) < 0.$. This universal behavior may not be true for non-hadronic systems, e.g. the D-term was found $D_{^1H} (0) > 0$ for the hydrogen atom, reference.} \peter{}{Not here. But perhaps it would make sense to add such a statement below Eq.~(49), and maybe we can even have a equation dedicated to stating that $D(0)<0$. I will try to implement this.}

    
\item {\bf Rearrangement in Theory Results section.}
    \peter{}{ {\bf LARGER POINT}
I found one omission: the dispersion relation
study by Barbara Pasquini, Maxim Polyakov and Marc Vanderhaeghen. This is not a model. It is actually model-independent. I would propose to have a section 
``Results from limits in QCD and dispersion relations''
where we discuss (i) large $N_c$, small $t$, large $t$, 
and dispersion relations which is valid for intermediate $t$. The common theme of this subsection would be 
``model-independent results'', and it would appear here:
after the models but before lattice QCD. What do you think?}
\volker{}{The Dispersion analysis is actually referenced in the caption of Fig. 11 and the curve included with the data and the ChQSM curve. I thought this is where it belonged. Having a separate section as suggested would emphasize the model-independency aspect. }
\peter{}{I tried. Please, check out how it looks like.} \cedric{}{Sounds good to me!} \volker{}{I like it too!}
\peter{}{New text added and approved in chat by CL.}


\item {\bf Fig.~1}
\cedric{}{I would replace $N(p)$ simply by $p$.}
\volker{}{It is done. New mods have been requested 12/13/22}

\item {\bf Fig.~2}
\peter{}{I would suggest to remove the arrow
and ``$t$'' whose meaning is unclear to most
physicists outside our field.
The arrow and ``$t$'' are not really needed.
If you would like the variables to be
mentioned, we could say in the caption that 
``$\xi$ and $t$ are the observable 
kinematic variables defined in the text 
while $x$ is integrated out''
or something analog. Another suggestion:
add labels ``DVCS'' and ``DVMP'' analogous 
to Fig.~3.
Then we can skip ``(left/right)'' in the 
caption.}
\volker{}{Let's wait until we get the modified graphs back from the illustrator and then decide how to describe the figures in the caption}\cedric{}{I fully agree with Peter's suggestions.}\volker{}{I was reluctant in removing the $t$ and $\xi$ as both of these quantities are critical for the data analysis discussed in section VI. (in fact the only quantities that matter, except $Q^2$). 12/13/22 }

\item {\bf Fig.~3}
\peter{}{After the more explicit Fig.~2,
I would suggest to make the subsequent 
figures less technical. Here I would 
suggest to 
(i) remove the arrow and ``$t$'',
(ii) remove the labels $x\pm\xi$,
(iii) and make the quark lines going 
out of the blob a bit longer.
But the colors for the lines, vertices, 
ect should be the same as in Fig.~2 
and the labels for the electron, proton
and (real/virtual) photons should remain.}
\volker{}{The new Fig.3 is now a photoproduction graph of $J/\psi$ threshold production, so no e, e', x, and $\xi$ are needed for diagram on the left, right diagram will be redone as shown in bottom version. 12/14/22}\volker{}{it is now replaced.}

\item {\bf Fig.~4}
    \peter{}{The caption speaks of a
    ``dashed line'' but the figure shows none.
    I think I understand what you mean, but
    I would not recommend to add such a 
    ``dashed line.'' I like that this figure
    is less technical. But I agree with Cedric
    (mentioned in an email) that we should use the same coloring as in the previous diagrams for the proton and other lines.
    Maybe the proton lines can also be drawn
    at an angle as in the previous figures.
    (For us it's all the same, but for other
    readers it costs a lot of concentration
    to ``read'' and understand our diagrams.
    Also I would suggest to
    add the labels ``GPDs'' on the green
    blobs.}\volker{}{The dashed line was in the original graph and I had asked Joanna to remove it, but forgot to change the caption. The other comments are, in part, already in the upcoming new versions implemented. The additional comments (add GPD) will need to go in the next (final?) iteration. Still waiting for the updated version.12/14/22}
    
\item {\bf Fig. 8 and its caption.}
\volker{}{The current gluon data were cut and pasted in the graph. They will be replaced by the illustrator later this week. }
\peter{}{We need to refer to the original papers for the different calculations (I can add that). Also, not sure it is necessary to show the gluon lattice data. The models show the TOTAL $D(t)=\sum_a D_a(t)$. Perhaps we can also show the total $D(t)$ from lattice? Lastly, should we show here the experimental results? They 
are shown later, and we could describe in text which models come closer. In any case, the comparison (total $D(t)$ vs a result on the quark part) is subtle.}
\volker{}{Maybe we take the graph out all together?}
\peter{}{Yes, we can think about this. 
Maybe we could show only models and dispersion
relations? Lattice and experimental results 
can be seen elsewhere. This would be a much ``cleaner figure''.... (just an idea).}
\volker{}{I agree. But who will be doing the new graphs? I am away in Morocco tomorrow until 12/10.} \peter{}{I can do a similar figure to the above, but without lattice and experimental data. This will be okay for now. We can replace it later by a nicer figure prepared by artists, if my figure will not be good enough.}
\volker{}{I have the gluon lattice data from Phiala.}
\peter{}{I would include only models here, because lattice data will be shown separately in the lattice section.}
\volker{}{Ok lets proceed like that then.} 
\peter{}{By the way, I tried to say ``Hello'' in the chat. But in this way, we have now a very nice caption :-)}
\volker{}{Is there a smiling emoji?  }
\peter{}{\Smiley \ 
%(needs {\tt $\backslash$usepackage\{marvosym\}})
}

\item {\bf New idea regarding Fig.~8}\\
    \peter{}{What if we completely remove it?  This figure is subjecive and biased by my choice of models (available until 2018 and probably incomplete already at that time). We anyway show only a small fraction of models, and we cannot do justice to {\it all} the new model calculations. However, later in Fig.~11 the chiral quark soliton model and Barbara et al's dispersive calculations are shown. [There is {\it no} bias in that figure, because I did not draw it. \Smiley ]. Removing Fig.~8 will save a lot of space which can be used e.g. in Conclusions. Would you agree?}\volker{}{Yes, I agree. I did draw the figure 11, and I am glad you like it. I could ask our artistic illustrator Joanna to redraw it better. In principle we could add the quark LQCD data as unbiased, but errors are very large. I was hoping for Phiala's new quark data, but they may not become available very soon. }\phiala{}{Yes, I don't have a timeline, it may still be a number of weeks before the new quark data is out unfortunately.} \volker{}{I included a new Fig.8 showing the LQCD data only, needs proper description. 12/14/22}

\item {\bf Appendix}
\peter{}{Somewhere in text an appendix is mentioned. Do we need one? Would it count towards our 20 page-limit?}\cedric{}{Maybe it is better not to include any appendix, and simply refer to the appropriate references. \volker{}{I replaced the reference to an appendix with a couple sentences describing the approved program in words. Please check the text.}}

\item {\bf\boldmath Some (settled) discussion on $x\pm\xi$ in caption of Fig.~1.}
\peter{}{Interestingly, everybody draws DVCS diagrams like this with $x+\xi$ for the quark line coming out of the proton and $x-\xi$ for the recombining quark line. But then the momentum transfer along the lightcone is actually $(-2\xi)$, isn't it? That's immaterial, since GPDs are invariant under $\xi\to(-\xi)$ due to time reversal, so it is actually correct either way. But I just noticed it. It's quite odd. Or, am I overlooking anything? Anyway, perhaps we can leave the figure as is and be less explicit in this caption. Addition: it seems the convention is
$\xi=(p^+-p^{\prime+})/(p^++p^{\prime+})$ so everything is fine (and we do not quote the above definition, and do not talk about much about lightcone --- because it is all focused on GFFs.) But the ``momentum transfer to the quark along the lightcone should be $(-2\xi)$ in my opinion...}\cedric{}{Starting from a quark with momentum fraction $x+\xi$ and ending up with a quark with momentum fraction $x-\xi$ just means that there has been a momentum fraction transfer of $(x-\xi)-(x+\xi)=-2\xi$ to the quark (and hence the target), so this seems consistent to me.}\volker{}{It looks a bit odd to me too, because $\xi$ is positive number related to $x_B$,  which is also a positive number. A -2$\xi$ would reduce x by some value of $x_B$. I would expect that the virtual photon transfers some positive momentum to the quark, at least the way the diagram is drawn. }
\peter{}{I think I created some confusion. It works as follows: consider the process in the $\gamma^\ast$-proton center-of-mass frame. The quark is hit by the $\gamma^\ast$ and SLOWED down. Hence, the momentum transfer along the lightcone is negative. Kind of makes sense. But typically we do not draw the diagrams like that. I confused myself with this stupid remark. \Frowny \ But everything is correct and makes (now) sense \Smiley} \  \volker{}{Yes in cms it makes sense to me as well.}

\item {\bf Remarks on Sec.~VC}

\peter{}{
{\bf LARGER POINT:} Maybe the following ``future developments'' should come after the ``First successful extraction'' part (which is ``current status'') in a separate subsection called ``Future experimental developments''? \volker{}{I have tried to do this. } 
%Then  we could remove the bullets (saves space) and add citations(?) to proposals or papers (if they exist). Also, there could be a 1-2 paragraphs on EIC (very long term plans). Curently EIC is not mentioned in the text. What do you think?
}
%\volker{}{EIC is mentioned in the Conclusion/outlook part with reference to our PSQ@EIC preprint that just appeared on the arXiv. I think it is better than to mix it up with actual ongoing or approved experiments.  }\peter{}{Agreed.}
 \\ 
\peter{}{I think we can shorten here. 
Most of the text on TCS and DDVCS 
(until the sentence on the ongoing 
"feasibility study in DDVCS") is
already mentioned in Sec.~III and repeated
here. (Actually, these parts may be better 
than the current descriptions of TCS and 
DDVCS in Sec.~III, so it could go there 
and replace or augment the corresponding
paragraphs there.)
But it would be good to add concrete
information on TCS (GluEx is ongoing,
I think. Anything else planned/ongoing?
Maybe $J/\Psi$ electroproduction...?).
We maybe also can remove the subsubsections,
and discuss this in one continuous subsection
with several paragraphs.}\volker{}{I reduced the discussion here to short paragraphs on the different programs related to the experimental aspects, while leaving the discussion of sensitivity to GPDs/GFFs in section III. Please comment.... }
\peter{}{The new text is much improved.
(Perhaps I will have some minor suggestions after a second reading.)}

\

\peter{}{
The reference \cite{Voutier:2014kea} was cited in the part on the processes (Sec.~III) in the context of beam charge asymmetry. I replaced it by references to more original works \cite{Kivel:2000fg,Belitsky:2001ns}. But I think that \cite{Voutier:2014kea} might be relevant and could be cited somewhere in this section. (?)}\volker{}{I think it is not needed here. }



\end{itemize}

\newpage

%\

%\newpage