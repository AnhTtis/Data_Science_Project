\section{The energy-momentum tensor} \label{sec-II}

In this section, after reviewing the 
definition and properties of the EMT
in QCD, the gravitational form factors 
(GFFs) of the proton are
introduced. It is shown how GFFs can
be leveraged to elucidate the proton's 
mass and spin decompositions.



\subsection{Definition of the EMT operator}

In QCD, the EMT 
$T^{\mu\nu}=\sum_qT^{\mu\nu}_q+T^{\mu\nu}_G$ can 
be decomposed into gauge-invariant quark and gluon
parts as 
\begin{equation}\label{EMTop}
    \begin{aligned} 
    T^{\mu\nu}_q
    &=\overline\psi_q\gamma^\mu\,iD^\nu\psi_q,\\
    T^{\mu\nu}_G
    &=-F^{a\mu\lambda}F^{a\nu}_{\phantom{a\nu}\lambda}
    +\tfrac{1}{4}\,g^{\mu\nu} F^2
    \end{aligned}  
\end{equation}
with $g_{\mu\nu}=\text{diag}(+1,-1,-1,-1)$ the 
Minkowski metric. In quantum field theory, the
expressions for the matrix elements of bare 
operators contain divergences and must be 
renormalized~\cite{tHooft:1972tcz}. Therefore,
each term in \eqref{EMTop} is understood as a 
renormalized operator defined at some 
renormalization scale $\mu$. 
The components of the EMT are interpreted in the 
same way as in the classical theory, namely $T^{00}$ 
is the energy density, $T^{0i}$ is the momentum density,
$T^{i0}$ is the energy flux, and $T^{ij}$ is the momentum
flux or stress tensor. In the literature, one often
considers only the symmetric part of \eqref{EMTop}, 
known as the Belinfante EMT~\cite{Belinfante:1962zz}. 
The antisymmetric part is a total derivative whose
presence does not affect the conserved Poincar\'e charges
(i.e.~four-momentum and spin) but impacts the definition 
of the corresponding 
densities~\cite{Leader:2013jra,Lorce:2017wkb}.



\subsection{Trace anomaly}
\label{Subsec:trace-anomaly}

At the classical level, the QCD 
Lagrangian (\ref{Eq:Lagrangian}) is
approximately invariant under scale transformations
$x\mapsto x' = \lambda x$. More precisely, 
the associated classical (dilation) current is 
conserved up to quark mass effects, which are small 
for $u$- and $d$-flavors. 
Contrary to spacetime translations, this symmetry is
affected by quantum corrections. As a result, even 
though the renormalized EMT looks formally the same 
as in the classical theory, its trace (which measures 
the non-conservation of the dilation current) receives 
anomalous contributions~\cite{Collins:1976yq,Nielsen:1977sy}
\begin{equation}\label{EMTtrace}
    g_{\mu\nu}T^{\mu\nu}=\sum_q (1+\gamma_m)m_q\,\overline\psi_q\psi_q+\tfrac{\beta(g)}{2g}\,F^2,
\end{equation}
where $\gamma_m$ is the anomalous quark mass 
dimension and $\beta(g)=\partial g/\partial\ln\mu$ 
is the QCD beta function. As will be discussed later,
this trace anomaly plays an important role when
discussing the mass and mechanical properties of 
the proton. Note that $g_{\mu\nu}T^{\mu\nu}_q$ and 
$g_{\mu\nu}T^{\mu\nu}_G$ mix with each other under 
renormalization, and each one contains both quark 
and gluon scalar
operators~\cite{Hatta:2018sqd,Tanaka:2018nae,Ahmed:2022adh}. 



\subsection{Definition of the proton GFFs}
\label{Subsec:def-GFFs}

The gateway to the mass and spin decompositions, 
and mechanical properties of the proton, are
the matrix elements of the EMT operator.
In terms of proton momentum eigenstates 
$|p,\vec s\rangle$, polarized in the $\vec s$ 
direction and normalized as 
$\la p^\prime,\vec s|p,\vec s\rangle=2p^0(2\pi)^3\delta^{(3)}(\vec p^{\,\prime}-\vec p)$, 
the EMT matrix elements can conveniently be parametrized
as~\cite{Kobzarev:1962wt,Pagels:1966zza,Ji:1996ek,Bakker:2004ib,Lorce:2022cle}
\begin{equation}\label{EMTparam}
\begin{aligned}
&   \la p^\prime,\vec s^{\,\prime}|  
    T_a^{\mu\nu}(0) |p,\vec s\rangle
    =\overline u(p^\prime,\vec s^{\,\prime})\Bigg[A_a(t)\,\frac{P^\mu P^\nu}{M_N}\\
&   + D_a(t)\,
    \frac{\Delta^\mu\Delta^\nu-g^{\mu\nu}\Delta^2}{4M_N}+ \bar{C}_a(t)\,M_N\,g^{\mu\nu}\\
&   +
    J_a(t)\ \frac{P^{\{\mu}i\sigma^{\nu\}\lambda}
    \Delta_\lambda}{M_N}
    -S_a(t)\ \frac{P^{[\mu}i\sigma^{\nu]\lambda}
    \Delta_\lambda}{M_N}\Bigg]u(p,\vec s)
\end{aligned}    
\end{equation}
for $a\in\{q,G\}$. Here $u(p,\vec s)$ is the usual 
free Dirac spinor, $M_N$ denotes the nucleon mass, 
and the symmetric kinematical variables are defined as
\begin{equation}
    P=\tfrac{1}{2}(p'+p),       \quad
        \Delta=p'-p,            \quad
        t=\Delta^2.
\end{equation}
The first four GFFs are associated with the symmetric (Belinfante)
part of the EMT $T^{\{\mu\nu\}}_a\equiv\tfrac{1}{2}(T^{\mu\nu}_a+T^{\nu\mu}_a)$. As one can see from  \eqref{EMTop}, only the quark part receives an antisymmetric contribution $T^{[\mu\nu]}_q\equiv\tfrac{1}{2}(T^{\mu\nu}_q-T^{\nu\mu}_q)$ and $S_G(t)=0$. 
The GFFs of specific partons inherit a renormalization
scale dependence from the associated operators, which 
is omitted in the notation for convenience. The total 
GFFs (summed over $q$ and~$G$) of the symmetric EMT 
are renormalization scale independent. 

On top of restricting the number of GFFs, Poincar\'e 
symmetry imposes additional constraints, namely
\ba
    A(0)&=&\sum_qA_q(0)+A_G(0)=1,  \label{Tconstraint}\\
    J(0)&=&\sum_qJ_q(0)+J_G(0)=\tfrac{1}{2}, \label{Lconstraint}\\
    \tfrac{1}{2}\Delta\Sigma &=& \sum_q S_q(0), \label{Lconstraint-2} \\
    \bar C(t)&=&\sum_q\bar C_q(t)+\bar C_G(t)=0, \label{cbar-constraint}
\ea
where (\ref{Tconstraint}) follows from translation
symmetry \cite{Ji:1997pf}, while~\eqref{Lconstraint} 
and~\eqref{Lconstraint-2} result from Lorentz symmetry 
\cite{Ji:1996ek,Bakker:2004ib}, with
$\tfrac{1}{2}\Delta\Sigma$ denoting the quark spin 
contribution to the nucleon spin. The constraint 
(\ref{cbar-constraint}), valid for any $t$,
follows from EMT conservation 
$\partial_\mu T^{\mu\nu}=0$. Interestingly, the
renormalization-scale invariant
quantity~\cite{Polyakov:1999gs} 
\begin{equation}\label{Eq:D-define}
    D\equiv D(0)=\sum_qD_q(0)+D_G(0),
\end{equation}
known as the $D$-term ($D$ stands for the German 
word {\it Druck} meaning pressure), is a global 
property of the proton (and, in fact, any hadron), 
whose value is not fixed by spacetime symmetries 
\cite{Polyakov:1999gs}. Its physical interpretation 
will be discussed in Sec.~\ref{interpretation}. 

Until recently, the only information about GFFs 
known from phenomenology was
$A_a(0)=\int_{-1}^1\ud x\, x\,f^a_1(x)$, corresponding 
to the fraction of proton momentum carried by the 
partons $a$ as inferred from DIS experiments, and
$S_q(0)=\frac{1}{2}\int_{-1}^1\ud x\, g^q_1(x)$, 
where $g^q_1(x)$ is the quark helicity 
distribution~\cite{Aidala:2012mv}.

\subsection{Decomposition of proton mass}
\label{mass-spin}

As the GFFs encode matrix elements of the EMT, 
the values taken by the former at $t=0$ provide 
the necessary quantitative input for decomposing 
the nucleon mass and spin into various quark and 
gluon contributions.

The concept of mass is directly related to the EMT. 
The starting point is the total four-momentum operator
\begin{equation}
    \mathcal P^\mu=\int\ud^3r\,T^{0\mu},
\end{equation}
defined in terms of the total EMT of the system. The
corresponding expectation value in a four-momentum 
eigenstate can be expressed as
\begin{equation}\label{PEV}
    \langle\mathcal P^\mu\rangle\equiv\frac{\langle p,\vec s|\mathcal P^\mu|p,\vec s\rangle}{\langle p,\vec s|p,\vec s\rangle}=p^\mu,
\end{equation}
which is just the forward (i.e. $\Delta=0$) matrix
element with appropriate normalization. 
Using translation symmetry 
$T^{\mu\nu}(x)=e^{i\mathcal P\cdot x}T^{\mu\nu}(0) e^{-i\mathcal P\cdot x}$, 
one finds that \eqref{PEV} can be put into the form
\begin{equation}\label{PEV2}
   \frac{\langle p,\vec s|T^{0\mu}(0)|p,\vec s\rangle}{2p^0}=p^\mu.
\end{equation}

A mass decomposition arises naturally from \eqref{PEV2}
by considering the energy component $\mu=0$ in the rest
frame (or equivalently by contraction with the four-
velocity $p^\mu/M_N$ in an arbitrary frame). One then 
finds that the quark and gluon contributions to the 
proton mass $M_N=\sum_qU_q+U_G$~\cite{Lorce:2017xzd} 
are given by
\begin{equation}
    U_a=\frac{\langle p,\vec s|T^{00}_a(0)|p,\vec s\rangle}{2p^0}\bigg|_{\vec p=\vec 0}=[A_a(0)+\bar C_a(0)]\,M_N.
\end{equation}
Combining this with the definition of the quark mass 
contribution
\begin{equation}
    U_m=\sum_q\sigma_q\equiv\frac{\langle p,\vec s|\sum_q m_q\,\overline\psi_q\psi_q|p,\vec s\rangle}{2p^0}\bigg|_{\vec p=\vec 0},
\end{equation}
one obtains a three-term mass decomposition~\cite{Rodini:2020pis,Metz:2020vxd}
\begin{equation}\label{MassSR3term}
    M_N=\sum_qU^\text{kin+pot}_q+U_m+U_G,
\end{equation}
where $U^\text{kin+pot}_q\equiv U_q-\sigma_q$ is 
interpreted as the quark kinetic and potential energy.

More generally, one can consider the entire set of
components
\begin{equation}\label{EMTrest}
    \frac{\langle p,\vec s|T^{\mu\nu}_a(0)|p,\vec s\rangle}{2p^0}\bigg|_{\vec p=\vec 0}=\begin{pmatrix}U_a&0&0&0\\
    0&W_a&0&0\\ 0&0&W_a&0\\ 0&0&0&W_a\end{pmatrix},
\end{equation}
where $W_a=-\bar C_a(0)M_N$ is the pressure-volume 
work defined as the partial isotropic stress 
$\frac{1}{3}\sum_i\langle T^{ii}_a\rangle$ considered 
in the proton rest frame and integrated over all space.
For a massive bound system in mechanical equilibrium, 
the total pressure-volume work should 
vanish~\cite{Laue:1911lrk}
\begin{equation}\label{virialthm}
\sum_a W_a=0,
\end{equation}
a result known as the virial theorem for a closed
system~\cite{Lorce:2021xku}. Since $U_a$ and $W_a$ 
mix under Lorentz transformations, two distinct
constraints $A(0)=1$ and $\bar C(0)=0$ can then 
be derived from four-momentum conservation expressed 
by \eqref{PEV2}.

In addition to the mass decomposition~\eqref{MassSR3term} 
resulting directly from the definition of the proton 
mass as the rest-frame energy (usually expressed in 
the covariant way as $p^2=M_N^2$), two other mass decompositions have been proposed in the literature 
with the particularity of involving \textit{both} 
$U_a$ and $W_a$. The first follows from the trace of
\eqref{EMTrest} and the virial theorem~\eqref{virialthm}, 
leading to the expression~\cite{Hatta:2018sqd,Tanaka:2018nae}
\begin{equation}\label{Tracedec1}
    M_N=\sum_qI_q+I_G
\end{equation}
with $I_a=U_a-3W_a=[A_a(0)+4\bar C_a(0)]\,M_N$. 
An older variant of this trace 
decomposition~\cite{Shifman:1978zn,Donoghue:1992dd} is
given by the forward matrix element of \eqref{EMTtrace}
\begin{equation}\label{Tracedec2}
    M_N=M_m+M_A,
\end{equation}
where $M_m=U_m$ and the anomalous contribution is
\begin{equation}
  M_A=\frac{\langle p,\vec s|\sum_q\gamma_mm_q\,\overline\psi_q\psi_q+\frac{\beta(g)}{2g}\,F^2|p,\vec s\rangle}{2p^0}\bigg|_{\vec p=\vec 0}.  
\end{equation}
The difference between~\eqref{Tracedec1} 
and~\eqref{Tracedec2} is related to the operator mixing 
mentioned earlier. The quantities $\sum_qI_q$ and $I_G$ 
are scale-dependent and can be simply interpreted as the 
traces of the quark and gluon contributions to the EMT 
(each containing part of the trace anomaly), whereas $M_m$
and $M_A$ are scale-independent and are interpreted as 
the non-anomalous and anomalous contributions to the EMT 
trace. Current phenomenology~\cite{Hoferichter:2015hva} 
and Lattice QCD calculations~\cite{Alexandrou:2019brg} 
indicate that $M_m/M_N\approx 10\%$, which is at the 
origin of the claim that most of the proton mass comes 
from the trace anomaly (and hence from the gluons, 
since $\gamma_m$ is small). This picture based on 
\eqref{Tracedec2} is, however, misleading as one has 
in fact decomposed $g_{\mu\nu}T^{\mu\nu}=T^{00}-\sum_i T^{ii}$ 
and therefore combined the internal energies $U_a$ 
with the pressure-volume works $W_a$. Since 
$\sum_qU_q$ and $U_G$ turn out to be of the same 
order of magnitude, the smallness of $M_m$ 
(or $\sum_qI_q$) relative to $M_A$ (or $I_G$) 
actually indicates that the quark (gluon) 
pressure-volume work is large and positive 
(negative)~\cite{Lorce:2017xzd}.

The popular mass 
decomposition~\cite{Ji:1994av,Ji:1995sv,Ji:2021mtz} 
can be seen as a blend of \eqref{MassSR3term} 
and~\eqref{Tracedec2}. Instead of immediately 
splitting the EMT into quark and gluon contributions 
as in \eqref{EMTop}, one first separates it into 
traceless and pure trace tensors
\begin{equation}
    T^{\mu\nu}=(T^{\mu\nu}-\tfrac{1}{4}\,g^{\mu\nu}g_{\alpha\beta}T^{\alpha\beta})+\tfrac{1}{4}\,g^{\mu\nu}g_{\alpha\beta}T^{\alpha\beta}.
\end{equation}
The motivation is that these tensors belong to 
different representations of the Lorentz group, 
and hence do not mix under Lorentz transformations 
or renormalization. Considering the rest-frame
expectation value of the energy component $\mu=\nu=0$,
one concludes using the virial theorem~\eqref{virialthm}
that $3/4$ of the proton mass comes from the traceless
part, and $1/4$ comes from the pure trace part. As a
second step, the pure trace part is decomposed similarly
to~\eqref{Tracedec2}, while the traceless part is 
decomposed into quark and gluon contributions, similarly 
to \eqref{EMTop}. One then arrives at
\begin{equation}
    M_N=\sum_qM_q+M_G+\tfrac{1}{4}(M_m+M_A)
\end{equation}
with $M_a=\frac{3}{4}(U_a+W_a)=\frac{3}{4}A_a(0)\,M_N$ 
for $a=q,G$. As a final step, one rearranges the quark 
mass contribution
\begin{equation}\label{Jimassdec}
    M_N=\sum_qM'_q+M_m+M_G+\tfrac{1}{4}M_A
\end{equation}
with $M'_q\equiv M_q-\tfrac{3}{4}\sigma_q$, similarly 
to~\eqref{MassSR3term}. A convenient feature of this 
four-term mass decomposition is that there is no need 
to determine the value of $\bar C_a(0)$, simplifying 
the scale dependence of the individual terms. 
Recalling that the values of the $A_a(0)$ are known 
from DIS, cf.\ Sec.~\ref{Subsec:def-GFFs}, the $M_{q,G}$
terms are directly proportional to measurable quantities.
The drawback, however, is that the corresponding renormalized operators do not  in general have a 
clean interpretation in terms of parton 
energies~\cite{Metz:2020vxd,Lorce:2021xku}. Note also 
that only \textit{two} independent inputs, viz.\ 
$\sum_qA_q(0)$ and $\sum_q\sigma_q$, are needed to 
fix the values of the four terms in~\eqref{Jimassdec}. 
The latter are therefore not independent and satisfy 
the relation
\begin{equation}
    \sum_qM'_q+M_G=\tfrac{3}{4}M_A,
\end{equation}
which is a direct consequence of the virial theorem.



\subsection{Decomposition of proton spin}

A similar discussion elucidates the proton spin 
decomposition. The total angular momentum (AM) operator 
is defined, in terms of the Belinfante (symmetric) EMT 
$T^{\mu\nu}_\text{Bel}=T^{\{\mu\nu\}}$, as 
\begin{equation}
\label{Eq:total-OAM-Bel}
    \mathcal J^i=\int\ud^3r\,\epsilon^{ijk}r^jT^{0k}_\text{Bel}.
\end{equation}
Because of the explicit factor of $r^j$, the forward 
matrix elements of this operator turn out to be ill-defined. A proper treatment requires the use of wave 
packets and amounts to considering matrix elements 
with non-vanishing momentum 
transfer~\cite{Bakker:2004ib,Leader:2013jra}. 
The forward limit is then recovered at the end of 
the calculation as a consequence of the integration 
over all space.

For convenience, only the longitudinal AM 
(i.e.~the component along the proton average momentum 
$\vec P=\frac{1}{2}(\vec p^{\,\prime}+\vec p)$ defining 
the $z$-direction) is considered here. The discussion 
about the transverse AM turns out to be much more complex 
because of its dependence on both $|\vec P|$ and the 
choice of origin, see e.g.~\cite{Lorce:2018zpf,Lorce:2021gxs} and references therein. From 
the splitting of the EMT in~\eqref{EMTop}, one finds that 
the quark and gluon contributions to the proton spin 
$\langle\mathcal J^z\rangle=\sum_qJ^z_q+J^z_G$ are 
given by~\cite{Ji:1996ek}
\begin{equation}
    J^z_a=J_a(0),
\end{equation}
for a proton polarized in the $z$-direction. 

Working instead with an asymmetric EMT, the quark AM 
operator can be further decomposed into orbital and 
intrinsic AM terms
\begin{equation}\label{Ja}
    \mathcal J^i_q=\int\ud^3r\,\epsilon^{ijk}r^jT^{0k}_q+\int\ud^3r\,\tfrac{1}{2}\overline\psi_q\gamma^i\gamma_5\psi_q.  
\end{equation}
Calculating the corresponding matrix elements, 
one then finds that $J^z_q=L^z_q+S^z_q$ with
\begin{equation}\label{quarkspinOAM}
\begin{aligned}
    L^z_q&=J_q(0)-S_q(0),\\
    \sum_qS^z_q&=\tfrac{1}{2}\Delta\Sigma.
\end{aligned}
\end{equation}
Combining the results~\eqref{Ja} and~\eqref{quarkspinOAM} 
with the fact that the proton is a spin-$\frac{1}{2}$ 
particle, one arrives at the constraints given 
in~\eqref{Lconstraint} and~\eqref{Lconstraint-2}. 

Since gluons are spin-$1$ particles, one may wonder 
whether the gluon AM could also be decomposed into 
orbital and intrinsic contributions. This can be done, 
but it requires non-local operators to preserve gauge 
invariance~\cite{Chen:2008ag,Hatta:2011ku,Lorce:2012rr,Lorce:2012ce,Leader:2013jra,Wakamatsu:2014zza}. 
One is then led to the canonical (or Jaffe-Manohar) 
spin decomposition~\cite{Jaffe:1989jz}, to be
distinguished from the one derived here from the 
local EMT~\eqref{EMTop} and known as the kinetic 
(or Ji) spin decomposition~\cite{Ji:1996ek}.


