\newpage
\vfill\eject

\peter{}{BELOW WE CAN COLLECT SOME TEXT WHICH WE REMOVE SOMEWHERE, BUT WHICH MIGHT POTENTIALLY BE USEFUL IN ANOTHER PART OF THE DRAFT OR RESTORED IF NEEDED (OR, REMOVED FOREVER, IF NOT NEEDED)}

\

\subsection*{Text from previous Sec.~III B on ``Time-like Compton Scattering''}

\peter{}{Can the Eq.~(14, 15) and Fig.~5 be removed? They do not seem to be essential, and one can just describe everything in words. See below for an attempt.
\\ \volker{}{Yes, to both questions. But where is the shortened version?} } \peter{}{In progress (I just had a meeting) \\ \ }



Time-like Compton scattering (TCS)~\cite{Berger:2001xd,Pire:2011st,CLAS:2021lky}, a process that is time-reversed with respect to DVCS and contains the same GPDs, can be used to cross check the results from DVCS. In contrast to DVCS, TCS enables access to both Re$\mathcal{H}$ and Im$\mathcal{H}$ through, respectively, the polarized beam spin asymmetry $A_{LU}$ given by
\begin{equation}
    A_{LU} = \frac{1}{P_e}\frac{N^+ - N^-}{N^+ + N^-},
\end{equation}
where $P_e$ is electron beam polarization, and the $N^+$ and $N^-$ correspond to the count rates for beam helicity $+$ and beam helicity $-$, respectively. The forward-backward asymmetry $A_{FB}$ of the final-state $e^+e^-$ pair in the centre-of-mass frame, is given by
\begin{equation}
    A_{FB}(\theta,\phi) =  \frac{\ud\sigma(\theta,\phi)-\ud\sigma(\pi - \theta, \pi + \phi)}{\ud\sigma(\theta,\phi)+\ud\sigma(\pi - \theta, \pi + \phi)},
\end{equation}
and is illustrated in Fig.~\ref{fig:fb-asym}
}
%\cedric{The interference between the TCS and the BH amplitudes (see Fig.~\ref{fig:tcs-bh}), can be isolated experimentally, and is given as:
%\begin{multline}
%\frac{\ud^4 \sigma_{INT}}{\ud{Q^\prime}^2 \ud t\, \ud\Omega} = \\ A_X\,\frac{1 + \cos^2\theta}{\sin\theta}\left[\cos\phi\, \text{Re}\tilde{M}^{--} - \nu\sin\phi\, \text{Im}\tilde{M}^{--}\right]
%\end{multline}
%with $\nu$ is the beam spin polarization and
%\begin{equation}
%\tilde{M}^{--} = F_1\mathcal{H} - \xi(F_1+F_2)\tilde{\mathcal{H}} -\frac{t}{4m_p^2}\,F_2\mathcal{E},
%\end{equation}
%where $F_1(t)$ and $F_2(t)$ are the usual Dirac and Pauli electromagnetic FFs.
%The beam spin asymmetry $A_{LU}$ is given by: 
%\begin{equation}
%    A_{LU} = \frac{N^+ - N^-}{N^+ + N^-},
%\end{equation}
%and the forward-backward asymmetry $A_{FB}$ illustrated in Fig.~\ref{fig:fb-asym} is given as: 
%\begin{equation}
%    A_{FB}(\theta,\phi) =  \frac{\ud\sigma(\theta,\phi)-\ud\sigma(\pi - \theta, \pi + \phi)}{\ud\sigma(\theta,\phi)+\ud\sigma(\pi - \theta, \pi + \phi)}.
%\end{equation}

\begin{figure}[th!]
    \centering
  \includegraphics[width=1.0\columnwidth]{Figures/Figure4_D3-01.jpg}
  %    \includegraphics[width=1.0\columnwidth]{Figures/TCS-BH.png}
    \caption{The leading TCS diagram and related Bethe-Heitler contributions}
    \label{OLD-fig:tcs-bh}
\end{figure}

\begin{figure}[th!]
    \centering
    \includegraphics[width=0.8\columnwidth]{Figures/TCS-BH-FB.png}
    \caption{TCS-BH kinematics defining the azimuthal angle $\phi$ and the polar angle 
    $\theta$ of the $e^+e^-$ pair.  }
    \label{OLD-fig:fb-asym}
\end{figure}
%\cedric{Similarly to DVCS, the asymmetry $A_{LU}$ will isolate Im$\mathcal{H}_Q$. The forward-backward asymmetry, illustrated in Fig.~\ref{fig:tcs-bh}, is used to directly access Re$\mathcal{H}_Q$.}{} 
In this regard TCS provides an advantage over DVCS, with the caveat that the TCS event rate is significantly reduced compared to DVCS. 
In experiments employing large acceptance detector systems, both DVCS and TCS processes can be  measured simultaneously, in quasi-real photo-production at very small $Q^2 \to 0$, and in real photoproduction where the external production target acts as a radiator of real photons that undergo TCS further downstream in the same target cell. 

\subsection*{Old text from previous Sec.~III C on ``Other Processes''}


\subsection*{Text removed from Sec.~V on Theory Results}

Matrix elements of the EMT operator have been studied in theoretical approaches such as models and in numerical calculations in the framework of lattice QCD. Theory insight into these fundamental aspects of proton and nuclear structure is currently in a phase of rapid progress, complementing the improvement of experimental constraints on these quantities and, importantly, providing predictions which inform the target kinematics for future experiments. 

In broad terms, the simplest aspects of the EMT structure of the proton and other simple hadrons (such as the pion) have been understood from theory for some number of years, and first-principles calculations providing complete and controlled decompositions of the proton's mass and spin, for example, are now available. On the other hand, more complicated aspects of proton and nuclear structure, such as gluon GFFs, the $x$-dependence of GPDs, and EMT matrix elements in light nuclei, have been computed for the first time in the last several years, as yet without complete systematic control, and significant progress can yet be expected over the next decade.
\phiala{}{this is a bit too lattice-centric at the moment, please modify}
\peter{}{The text is nice, but very generic and relatively long. Perhaps we can use some of it e.g.\ in the Introduction/Conclusions?}

\subsection*{Too lengthy text on pion and chiral symmetry}

In general, $D$-terms of hadrons cannot be computed analytically, but the pion is a notable exception. 
The Lagrangian of QCD, Eq.~(\ref{Eq:Lagrangian}),
possesses a $U(3)_L\times U(3)_R$ global 
flavor symmetry if 
we neglect the relatively small masses of the 
light up, down, and strange quarks. One would
expect this symmetry to be approximately reflected 
in the hadronic spectrum implying, e.g., 
similar masses for the the nucleon and its 
negative-parity partner $N(1535)$.
However, the latter is almost 600 MeV heavier 
than the nucleon, an effect 
that cannot be be attributed to current quark mass
effects. Rather it is explained by the phenomenon 
of spontaneous symmetry breaking 
\cite{Nambu:1961tp,Nambu:1961fr}.
The $U(3)_L\otimes U(3)_R$ symmetry corresponds to
$U(1)_V\otimes U(1)_A\otimes SU(3)_V\otimes SU(3)_A$.
$U(1)_V$ underlies the conservation of
quark flavor in strong interactions, and also  
$SU(3)_V$ is approximately realized in nature (corresponding 
to Gell-Mann and Zweig's eightfold way). 
But $SU(3)_A$ is broken spontaneously, which 
is referred to as chiral symmetry breaking with
the emergence of Goldstone bosons of chiral
symmetry breaking: pions, kaons, and $\eta$-meson
(and $U(1)_A$ is broken by quantum corrections
due to the chiral anomaly).
\peter{}{(we can shorten the above by starting
from $SU(3)_L\otimes SU(3)_R$ and putting $U(1)_V$ and $U(1)_A$ under the carpet.)} \cedric{}{[I agree. Moreover talking about $SU(3)_A$ is an abuse of language used by some physicists that forget that the elements of $SU(3)_A$ do not form a group. The correct statement is that $SU(3)_L\otimes SU(3)_R$ is broken down to $SU(3)_V$.]}

Strictly speaking, Goldstone bosons should be
massless, but in nature the small masses of 
light quarks give pions, kaons and $\eta$ 
finite masses. The chiral symmetry also 
severely restricts the possible interactions
of Goldstone bosons allowing one to evaluate
exactly their EMT matrix elements 
in the chiral limit (and for $t\to0$).
In this way, one obtains for the pion $D$-term
\be
       \lim\limits_{m_\pi\to 0} D_\pi = -1.
\ee
The same result is obtained for kaons and
$\eta$. Deviations from the chiral limit 
can be systematically calculated in chiral
perturbation theory \cite{Donoghue:1991qv}
and may be expected to be small for pions
and more sizable for the heavier kaons and
$\eta$ \cite{Hudson:2017xug}.
The gravitational interactions of Goldstone
bosons have been studied in 
\cite{Voloshin:1982eb,Leutwyler:1989tn}.

\subsection*{Some other parts}

% pentaquarks and tetraquarks with hidden charm as hadroquarkonia \cite{Eides:2015dtr}.
\volker{}{Text removed from section VII that were marked by CL for deletion.}
\cedric{In this work we have made use of the same fit with the parameterization used in
%~\cite{Burkert:2018bqq}
. We believe that fits 
constrained through parameterization with few fit parameters are currently the best way of extracting the mechanical 
properties of the proton from the current status of DVCS data, given that data cover limited kinematic space that requires 
extrapolations to carry out the Fourier transformation involved.}{}\footnote{\cedric{Comments made in
%~\cite{Kumericki:2019ddg}
compare results of an artificial neural network (ANN) approach, which failed to extract a non-zero D-term, to the extraction of the D-term in
%~\cite{Burkert:2018bqq} 
(BEG) that shows a result with significance of 5.8$\sigma$ for $D_Q(t=0)$. The BEG analysis presumes smooth changes from bin to bin in $\xi$ enabling a  parameterized representation of the CFF $\mathcal{H}$ with a small number of fit parameter spanning the entire $\xi$ space. In contrast the ANN approach treats each data bin independently from the  neighboring bins leading to large fluctuation and error bars. In addition, and in distinction to ANN, the BEG parameterized fit employs a subtracted dispersion relation, which requires integrating Im${\cal{H}}(\xi,t)$ over the full $\xi$ range with the GFF $D(t)$ as a subtraction term of the DR.  A multipole representation of the form factor $\Delta(t)$ was included in the fit. The global fit is constrained by the locally extracted Im${\cal{H}}({\xi},t)$ and Re${\cal{H}}({\xi},t)$ from CLAS high precision BSA and differential cross sections.}{}

\cedric{In this work fit results used in
%~\cite{Burkert:2018bqq} 
were employed. We believe that fits 
constrained through parameterization with few fit parameters are currently the best way of extracting the mechanical properties of the proton from the current status of DVCS data, given that data cover limited kinematic space that requires extrapolations to carry out the dispersion integral and the Fourier transformation involved. 
%The price one pays is in the large systematic uncertainties of the final results.  

To distinguish our approach from other approaches in extracting the D-term and mechanical properties derived from it, we 
make some clarifying comments. In the present and in earlier work
%~\cite{Burkert:2018bqq} 
both 
CFF ${\cal{H}(\xi, \it{t})}$ and $\Delta(t)$ are parameterized and fitted directly to the DVCS data, both in cross section and polarization asymmetries, with the dispersion relation included in the fit. Therefore we only have a small number of parameters adjusted to the entire available 
kinematic space. The parameterization provides strong constraints and results in small fit errors, which puts more effort on the estimate of realistic systematic uncertainties. Given the current status of the limited amount of experimental data with often large statistical uncertainties, 
we believe this approach is most suitable to extract new physics from these data.
%\begin{figure}[th!] 
%\hspace{0.5cm}
%\includegraphics[width=0.45\columnwidth]{Figures/tangential_force.png}
%\caption{\footnotesize Spatial distribution of tangential forces. Tangential forces exhibit a node near a 
%distance $r \approx  0.45$fm from the center, where they also reverse direction as indicated by the direction and 
%lengths of the colored arrows. The arrows represent the forces acting along the orientation of the surface. If one 
%changes the direction of the normal to the surface one also changes the arrow direction, so that pressure 
%acts equally on both sides of a surface immersed in the system.}
%\label{shear_stress}
%\end{figure}  

Figure~\ref{pressure-shear-1D} shows results for the \cedric{normal}{isotropic} and shear forces. 
This represent the first effort in determining these
forces in the proton using the experimental DVCS process and its relation to the GFFs. While these results still have
significant systematic uncertainties due to the limited kinematic range covered in the DVCS data, 
they lead to important first conclusions.}{} } 

\volker{}{Removed from Section VII. 

\subsection{The proton's mechanical size} 
\label{r-mech}
The charge radius of the proton has been studied in elastic electron-proton scattering experiment since 1956 through measurements of the electric Sachs form factor~\cite{Sachs:1962zzc} $G_E(t)$, where the mean squared charge radius $\langle{r^2}\rangle_\text{ch}$ is defined as the slope of the electric $G_E(t)$ at $t=0$: 
\begin{eqnarray} 
\langle{r^2}\rangle_\text{ch} = -6\, \frac{\ud G_E(t)}{\ud t}\Big|_{t=0}. 
\label{charge}
\end{eqnarray}   
The experiment of Hofstadter~\cite{Mcallister:1956ng} had firmly established the protons finite size at $\sqrt{\langle{r^2}\rangle_\text{ch}}=(0.75\pm 0.25)~\rm{fm}$. Experiments have since much improved the precision of the experimental methods leading to the latest value of  $\sqrt{\langle{r^2}\rangle_\text{ch}} = (0.831 \pm 0.007 \pm 0.012)~\rm{fm}$~\cite{Xiong:2019umf} from electron scattering experiments.

Beside the charge radius of the proton, the knowledge of the GFF $D^q(t)$ enables the determination of the mechanical radius of the proton. The mean squared mechanical radius is defined through the total radial force~\cite{Polyakov:2018zvc}  
\begin{eqnarray}
\langle{r^2}\rangle_\text{mech} =\frac{\int \ud^3r~r^2 [\frac{2}{3} s(r) + p(r)] }{\int \ud^3r ~[\frac{2}{3} s(r) + p(r)]}=\frac{6D(0)}{\int_{-\infty}^0 \ud t\, D(t)}
\label{mech}
\end{eqnarray}
It can be seen from the equations (\ref{charge}) and (\ref{mech}) that the proton's charge form factor and the mechanical form factor are differently defined and their respective determinations require different structural information.    

Using the multipole form (\ref{multipole}) and the fit parameter (\ref{F2}) and (\ref{F3}), we can determine the mechanical radius of the proton as 
\begin{eqnarray}
\sqrt{\langle{r^2}\rangle_\text{mech}} =(0.63 \pm 0.06 \pm 0.13)\, {\rm fm}
\end{eqnarray}
\cedric{}{[I do not think this quantity makes much sense for it has been obtained from $D_{u+d}(t)$, whereas the formula~\eqref{mech} has been derived for the {\it total} GFF $D(t)$.]}
This value is significantly smaller than the proton's charge radius of 
\begin{eqnarray}
\sqrt{\langle{r^2}\rangle_\text{ch}} =(0.8408 \pm 0.0004)\, {\rm fm}
\end{eqnarray}

A qualitative explanation of the difference is, that the charge radius is measured at very small values of $Q^2$ to determine the slope of the electric form factor at $Q^2 \to 0$. It is strongly influenced by the proton's peripheral pion cloud, while the mechanical radius is measured at large $Q^2$, with the (gravitational) probe coupling to mass and pressure that are concentrated closer to the proton's center and therefore result in a smaller mechanical size of the proton. \cedric{}{[It has been observed in some models that gluon forces tend to have larger spatial distributions than quark forces. The small value for the mechanical radius may probably be explained by the fact that only the quark contribution has been considered.]}}


\subsection*{Parts from ``Interpretation''}

Seems no part is here.

\subsection*{Formula's added and immediately removed from the new Interpretation-section}



 \begin{eqnarray} 
 s(r)&=&-\frac{1}{2}r\frac{d}{dr}\frac{1}{r}\frac{d}{dr}\widetilde{D}(r)\,,
 \label{Eq:s-from-D}\\
 p(r)&=& 
 \frac{1}{3}\frac{1}{r^2}\frac{d}{dr}r^2\frac{d}{dr}\widetilde{D}(r)\,,
 \label{Eq:p-from-D}
 \end{eqnarray} 
 with the generating function $ \widetilde{D}(r)$ defined as
 \be
 \widetilde{D}(r) = 
 \int\frac{d^3\Delta}{(2\pi)^3}\exp^{-i\vec{\Delta}\vec{r}}
 D(-\vec{\Delta}^2) \,.
 \ee


%\cedric{}{[I think we can just remove the following paragraph]}
%However, we note that already with the beam-spin asymmetry employing polarized electrons and with unpolarized cross section measurements, significant information about the CFF $\mathcal{H}$ may be obtained because of its large contributions in these experiments. An example is the polarized beam spin dependent cross section\begin{eqnarray}
%\ud\sigma_{LU} \propto %\sin\phi\,\text{Im}[F_1\mathcal{H}+\xi(F_1+F_2)\widetilde{\mathcal{H}}-kF_2\mathcal{E}]\,\ud\phi, 
%d\sigma_{UL} \propto \sin\phi {Im}[F_1\widetilde{\mathcal{H}}+\xi(F_1 + F_2)(\mathcal{H} + \frac{x_B}{2}\mathcal{E}) - \xi kF_2\widetilde{\mathcal{E}}]d\phi, \nonumber \\ \nonumber
%d\sigma_{LL} \propto (A+B\cos\phi){Re}[F_1\widetilde{\mathcal{H}} + \xi(F_1+F_2) (\mathcal{H} + \frac{x_B}{2}\mathcal{E}]d\phi, \nonumber \\
%d\sigma_{UT} \propto cos\phi\sin(\phi_s-\phi)Im[k(F_2\mathcal{H}-F_1\mathcal{E})]d\phi\nonumber,  \\ 
%\label{asym}
%\end{eqnarray}
%\noindent{where $F_1$ and $F_2$ are the Dirac and Pauli form factors that parameterize elastic electron-proton scattering in the real Bethe-Heitler (BH) amplitude, $\phi$ is the angle between the electron scattering plane and the hadronic ($\gamma$-$p$) plane.} Other similar relations describe sensitivity of different beam/target polarization to the GPDs.


%\subsubsection{Fixed-$t$ dispersion relation to determine the GFF $D_Q(t)$} \cedric{}{[I think we can move this subsection and merge it with the corresponding part in section VI. In particular, we do not discuss the connection between the subtraction constant and the $D$-term here.]} An important tool to be employed in the analysis of the CFF $\mathcal{H}$ is a dispersion relation (DR) that connects  Im${\mathcal H_Q(\xi,t)}$ and Re${\mathcal H_Q(\xi,t)}$. It has been shown to contain a real subtraction term \cedric{$\Delta(t)$}{$\mathcal C_{\mathcal H_Q}(t)$}, which has a direct relation to the GFF $D_Q(t)$~\cite{Diehl:2007jb,Anikin:2007tx}. The DR is given as: \begin{multline}  {\rm Re}{\mathcal H_Q}(\xi,t) =  \cedric{\Delta(t)}{\mathcal C_{\mathcal H_Q}(t)} \\+\frac{ 1}{\pi} \,{\cal P} \int_0^1 \text{d}\xi' \left[\frac{1}{\xi-\xi'}  -\frac{1}{\xi+\xi'}\right] {\rm Im}{\mathcal H_Q}(\xi',t), \label{DR}
%\end{multline} where $\mathcal{P}$ is the principal value of the Cauchy integral.   The polarization asymmetries and differential cross sections may be used to extract \cedric{the local}{} Im$\mathcal{H}_Q(\xi,t)$ \cedric{fitted to the beam spin asymmetry data}{} at fixed kinematics in $\xi$ and $t$ in the ranges covered by the data.  Using this technique to determine the subtraction term \cedric{$\Delta(t)$}{$\mathcal C_{\mathcal H_Q}(t)$} requires the full integral over $0 \leq \xi \leq 1$ at fixed $t$ to be evaluated. As this process requires an extrapolation to both $\xi=0$ and to $\xi=1$ that are unreachable in experiments, a parameterization of the $\xi$-dependence of Im$\mathcal{H}_Q$ close to these limits must be incorporated when fitting the data. 
