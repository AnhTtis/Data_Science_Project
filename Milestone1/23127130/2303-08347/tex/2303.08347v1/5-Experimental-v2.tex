\section{Experimental results}
\label{experiments}

This section presents a brief overview of the current landscape 
of the experimental program targeting the probing of GPDs and 
the extraction of the GFFs of the nucleon and other hadrons. 
The first extraction of the proton $D$-term form factor $D_q(t)$ 
from experiment based on data from Jefferson Lab (JLab)
is described in detail. The extraction of $D_q(t)$ of the 
neutral pion from Belle data, and other phenomenological 
results, are also reviewed.



\subsection{DVCS in fixed-target and collider experiments } 

In this section, the past, current and planned DVCS experiments 
are briefly reviewed. Experimental measurements of DVCS are 
difficult as they require high luminosity to measure small 
cross sections with polarized electron, positron or muon
beams at different energies, and employing polarized
nucleon targets, both longitudinal and transverse, to
measure observables with 
sensitivity to different combinations of CFFs. The measurements 
also require advanced detection capabilities to ensure reaction 
exclusivity in a wide  range in kinematics.

The first observation of the $\sin(\phi)$-dependence for the 
$ {\vec e} p \rightarrow e'p'\gamma $ process as signature of 
the interference of the DVCS and Bethe-Heitler amplitudes came 
from the CLAS \cite{CLAS:2001wjj} and HERMES \cite{HERMES:2001bob}
detectors, while the H1 \cite{H1:2001nez} and ZEUS \cite{ZEUS:2003pwh}
collaborations measured, for the first time, the DVCS cross sections 
with both electron and positron beams on unpolarized protons in the 
kinematic domain of $x_B < 0.01$.

These initial results triggered the development of a worldwide dedicated experimental program to measure the DVCS process with   
high precision with HERMES at HERA, Hall A and CLAS at 
JLab,  and COMPASS at CERN. 
This experimental program required major construction projects, and new detector and accelerator upgrades employing state-of-the-art technologies.

HERMES provided a set of observables with different experimental conditions, electron and positron 
beams
with both longitudinal and transverse spin polarizations fo the lepton and proton \cite{HERMES:2011bou}. These data had important impact in the initial development and constraint of GPD models and   
the first global fits. Although HERMES was able to provide a complete set of DVCS measurements, the HERA machine operated at very low luminosity providing only low-statistics results until HERMES ended operation in 2007.

The COMPASS experiment took first DVCS data in a pilot run in 2012 with 160~GeV of oppositely polarised $\mu^+$ and $\mu^-$ beams~\cite{COMPASS:2018pup}. The average of the measured $\mu^+$ and $\mu^-$ cross sections allows for the determination of Im$\cal{H}$. Results from an order of magnitude higher statistics obtained in 2016 and 2017 will soon be available. With these new data, the difference of $\mu^+$ and $\mu^-$ cross sections can also be formed to provide access to Re$\cal{H}$.

In the 6 GeV era of JLab, dedicated DVCS experiments were designed and performed at low to moderate $Q^2$ in the
region $x_B\approx 1/3$ involving modest upgrades to the detectors and the operating luminosity. The first dedicated experiment was the DVCS cross section measurements in Hall A, resulting in a first $Q^2$-scaling test
\cite{JeffersonLabHallA:2006prd}. 

Special detector configurations were incorporated with the construction of new equipment that doubled the luminosity the CLAS detector is able to operate at, to $2\times 10^{34}$\,cm$^{-2}$\,s$^{-1}$. The CLAS collaboration has 
measured DVCS beam spin asymmetries and cross sections \cite{CLAS:2007clm,CLAS:2015uuo} and longitudinally polarized target-spin asymmetries \cite{CLAS:2006krx,CLAS:2014qtk,CLAS:2015bqi}. The high statistical accuracy of the data collected in the large kinematic range of CLAS was essential for the first extraction of information on the proton $D$-term. This goes beyond the tomographic imaging with GPDs, and opened up the experimental investigation of the gravitational structure of the proton. The details of the high-level analysis will be described in the second part of this section.

One of the primary motivations of the 12~GeV energy upgrade at JLab was the study of the 3D structure of the nucleon by measurement of deeply exclusive processes with DVCS as the flagship program. In addition to the accelerator upgrade, this required a major upgrade to the experimental equipment from CLAS to CLAS12~\cite{Burkert:2020akg} with an order of magnitude increase in luminosity and an upgrade of Hall C. A comprehensive DVCS program has been approved as a major part of the 12~GeV science program. It incorporates experiments with polarized electron beams and energies from 6 to 11 GeV, longitudinally polarized hydrogen and deuterium targets, as well as conditionally approved measurements with transversely polarized proton targets, and with positron beams.  

The first high-precision results of DVCS cross sections measured at 12~GeV in Hall A, focused on high values of the Bjorken variable $x_B$, have been published~\cite{JeffersonLabHallA:2022pnx}. The results of beam-spin asymmetry measurements at different energies with CLAS12 are in the public domain~\cite{CLAS:2022syx}. The comprehensive DVCS measurements expected from JLab with the 12~GeV capabilities will significantly improve our understanding of the mechanical structure of the proton. 

\subsection{First extraction of
the proton GFF \boldmath $D_q(t)$}
\label{Subsec:D-term-at-JLab}

In this section, the data and procedure used in~\cite{Burkert:2018bqq} to obtain the first determination of the quark contribution to the $D$-term of the proton are described. This work is based on two main pieces of experimental 
information from the CLAS detector at JLab~\cite{CLAS:2003umf}, namely the beam-spin 
asymmetry (BSA) measured with spin-polarized electron beams, and the unpolarized cross section for DVCS 
on the proton. 

The polarization asymmetries and differential cross sections have been used to extract the imaginary and real parts of the CFF ${\cal H}$ respectively. Using the 
dispersion relation technique to determine the subtraction term $\mathcal C_{\mathcal H}(t)$ requires the full integral over $0 \leq \xi \leq 1$ at fixed $t$ to be evaluated. As this process requires an extrapolation to both $\xi=0$ and to $\xi=1$ that are unreachable in experiments, a parameterization of the $\xi$-dependence of Im$\mathcal{H}$ close to these limits must be incorporated when fitting the data. 

In the first step, local fits of the BSA~\cite{CLAS:2007clm} and of the unpolarized differential cross-sections~\cite{CLAS:2015uuo} for DVCS were performed to estimate Im${\mathcal H(\xi,t)}$ and Re${\mathcal H(\xi,t)}$ at fixed kinematics in $\xi$ and $t$ in the ranges covered by the data. The BSA is defined as 
\begin{eqnarray}
A_{LU}(\xi, t) = \frac{N^+(\xi, t) - N^-(\xi, t)}{N^+(\xi, t) + N^-(\xi, t) } ,
\end{eqnarray}
where $N^+$ and $N^-$ refer to the  measured event rates at electron helicity $+1$ and $-1$, respectively. 


The experimentally-measured BSA in $\vec{e} p \to e p \gamma$ contains not only the DVCS term, with the photon generated at the proton vertex, but also the Bethe-Heitler term with the photon generated at the incoming or scattered electron, respectively (see Fig.~\ref{DVCS-BH}). Both have the same final state and thus interfere. They generate a $\sin\phi$-dependent interference contribution as seen in Fig.~\ref{BSA-DVCS-BH}. The DVCS term is dominated by the CFF Im$\mathcal{H}$ and the Bethe-Heitler term is real and is given by the elastic electromagnetic FFs.  

It is important to note that this analysis does not rely on extracted cross sections but on asymmetries of event rates in specific bins. This is an essential 
advantage 
as it avoids accounting for systematic uncertainties that must be included in the cross section extraction. The uncertainties in $A_{LU}(\xi,t)$ are dominated by statistics rather than systematic uncertainties, which determines the local values of Im$\mathcal{H}$ very precisely as can be seen in the top panel of Fig. \ref{BSA-DVCS-BH}, which shows the BSA and the differential cross sections for selected kinematic bins. 
\begin{figure}[t!] 
\includegraphics[width=0.9\linewidth]{Figures/DVCS-BH-BSA.png}
\includegraphics[width=0.9\linewidth]{Figures/CRS-DVCS-BH.png}

\vspace{-2mm}

\caption{Top: The expected $\sin{\phi}$ dependence   is fit to the data. The thick solid lines 
are the global fits using the parameterization  
according to~\eqref{km-para}. The thin solid lines represent 
local fits. The thin gray lines represent estimates of the systematic uncertainties. Bottom: The unpolarized cross section at fixed $\xi$ and $Q^2$ for different $t$ values. The azimuthal $\phi$ angle dependence of the cross section is fitted to the experimental data. The thin solid line is the global fit. The thin gray lines represent local fits with dashed lines showing the systematic uncertainties. The thick black lines 
show the Bethe-Heitler contributions. Note the logarithmic vertical scale.}
\label{BSA-DVCS-BH}
\end{figure}
  
In the second step, the 
Im$\mathcal{H}(\xi,t)$ are fit with the functional form used in global fits~\cite{Muller:2013jur,Kumericki:2016ehc} with the parameters fit to the local CLAS data. The imaginary part is written as:
\begin{equation}
    \begin{aligned}
{\textrm{Im}}\mathcal{H}(\xi,t) =  \frac{\mathcal N}{1+\xi}\frac{\left(\frac{2\xi}{1+\xi}\right)^{-\alpha(t)}\left(\frac{1-\xi}{1+\xi}\right)^b}{ \left(1-\frac{1-\xi}{1+\xi}\frac{t}{M^2}\right)^{p}}, \label{km-para} 
\end{aligned}
\end{equation}
where $\mathcal N$ is a free normalization constant, $\alpha(t)$ is fixed from small-$x$ Regge phenomenology as $\alpha(t)=0.43+0.85\,t~\text{GeV}^{-2}$,
$b$ is a free parameter controlling the large-$x$ behavior, $p$ is fixed to 1 for the valence quarks, and $M$ is a free parameter controlling the $t$-dependence.   


The real and imaginary part are fit together including the subtraction term in the dispersion relation~\eqref{DR}.  
Fig.~\ref{CFF} compares the local fit with the global fit 
for one of the $t$ values. The global and local fits show good agreement in $\xi$ and $t$ kinematics where they overlap.    

\begin{figure}[ht!]
\vspace{-0.4cm}

\includegraphics[width=0.92\linewidth]{Figures/Figure10_D4-02.jpg}
\vspace{-0.6cm}

\includegraphics[width=0.833\linewidth]{Figures/Figure10_D4-01.png}

\vspace{-3mm}

\caption{\footnotesize Top: The Im${\cal H}$ data points are plotted as function of $\xi$ from local fits to the $A_{LU}$ data~\cite{CLAS:2007clm} for 
$-t=0.13$-$0.15$ GeV$^2$. The red line is the global fit constrained by the local data points. The light-green error band
is due to the uncertainty of the other CFFs. The outer dark-green band shows the total systematic uncertainty to the imaginary part of the fit. The upper and lower solid lines 
are due to uncertainties from possibly steeper or less steep $\xi$-dependences of Im${\cal H}$.   Bottom: Re$\cal{H}$ data as extracted from unpolarized cross section data~\cite{CLAS:2015uuo}. 
 The central red curve shows the result of the global fit with the dispersion relation applied and 
the fit parameters of the multipolar form for $\mathcal C_{\mathcal H}(t)$. The other colored lines/bands describe the same contribution as for Im${\cal H}$ propagated with the dispersion relation.  
The black curve separated from the error bands shows the real part of the amplitude computed from the imaginary part using the dispersion relation and setting $\mathcal C_{\mathcal H}(0)$ to zero. The difference of solid red and solid black line shows the effect of the subtraction term. Note that all markers in Re${\cal H}$ contribute to the precision of a single $-t$ value in $\mathcal C_{\mathcal H}(t) $, resulting in a small fit uncertainty.}
\label{CFF}
\end{figure}


\begin{figure}[ht]
\includegraphics[width=0.9\columnwidth]{Figures/DR-sub-term.png}
\caption{\footnotesize The subtraction term $\mathcal C_{\mathcal H}(t)$ as determined from the dispersion relation in the global fit (markers). 
The uncertainties represent results of the fit errors. The hatched area at the bottom represents the estimated 
systematic uncertainties as described in Fig.~\ref{CFF} for one of the bins in $-t$.The dashed and solid-blue curves show the dispersive calculation~\cite{Pasquini:2014vua} and chiral quark-soliton model predictions~\cite{Goeke:2007fp}, respectively.}
\label{Dt}
\end{figure} 

In the  
fit, $\mathcal C_{\mathcal H}(t)$ is obtained at fixed  
$t$. The results for the subtraction term and the fit to the multipole form
\begin{eqnarray}
\mathcal C_{\mathcal H}(t) &=& \mathcal C_{\mathcal H}(0) \bigg[1 + \frac{(-t)}{M^2}\bigg]^{-\lambda} \label{multipole} \\  \nonumber
\end{eqnarray} 
are displayed in Fig.~\ref{Dt}, where $\mathcal C_{\mathcal H}(0)$, $\lambda$ and $M^2$ are the fit parameters, with their values found to be: 
\begin{equation}
    \begin{aligned}
\mathcal C_{\mathcal H}(0) &=-2.27\pm 0.16 \pm 0.36,  
\\ 
M^2 &=1.02\pm  0.13 \pm 0.21  {\rm ~GeV^2},  \\ 
 \lambda &=2.76 \pm  0.23 \pm 0.48. 
\end{aligned}
\label{parameters}
\end{equation} 
 \noindent The first error is the fit uncertainty, and the second error is due to the systematic uncertainties.
Adding the fit errors for $\mathcal C_{\mathcal H}(0)$ and the systematic errors in quadrature $\sigma_{\mathcal C_{\mathcal H}(0) } = \sqrt{0.16^2 + 0.36^2} \approx  0.39$, 
the significance $S$ of the knowledge of the subtraction term is: 
\begin{eqnarray}
S = \frac{\mathcal C_{\mathcal H}(0) }{\sigma_{\mathcal C_{\mathcal H}(0)}} \approx 5.8.
\end{eqnarray}
More flexible analyses based on unconstrained artificial neural network techniques~\cite{Kumericki:2019ddg,Dutrieux:2021nlz} find however that a more conservative extraction of the subtraction constant from the currently available experimental data remains compatible with zero within large uncertainties. 


\begin{figure}[th!] 
\includegraphics[width=0.92\linewidth]{Figures/TCS-BSA-FBA.png}
\caption{The TCS polarized BSA (top) and the TCS $A_{FB}$ (bottom) for an average  1.8~GeV mass of the time-like photon $M_{e^+e^-}$. A value for $A_{LU}$ of (20-25)\% is consistent with what is measured in DVCS and projects out Im${\cal H}$. The FBA projects out Re${\cal H}$ that relates directly to the protons $D_q(t)$-term. 
The data require the presence of the $D$-term as seen in the difference of the dashed magenta line and the solid red line. At the kinematics of the data in Fig.~\ref{Dt},
about half of the asymmetry may be due to the $D$-term 
when comparing calculations without and with the 
$D$-term~\cite{Vanderhaeghen:1999xj,Pasquini:2014vua}.}
\label{TCS-BSA-FBA}
\end{figure}
In the analysis of~\cite{Burkert:2018bqq}, the term $d_3^q(t)$ and other higher-order terms have been omitted in the expansion~\eqref{Gegenbauer} to extract the GFF $D_q(t)$. The estimated effect is included in the systematic error analysis. It is also assumed that $u$ and $d$ quarks have the same first moments $d_1^u \approx d_1^d \approx {d_1^{u+d} /2}$, an assumption justified in the large-$N_c$ limit~\cite{Goeke:2001tz}. Under these approximations, it follows from~\eqref{Dtermint} that
\begin{eqnarray}
\mathcal C_{\mathcal H}(t)\approx\frac{10}{9}\,d_1^{u+d}(t) = \frac{25}{18}\,D_{u+d}(t).    
\label{F7}
\end{eqnarray}
The truncation in~\eqref{Gegenbauer} leads to a systematic uncertainty of a priori unknown magnitude. For $Q^2\to\infty$, the higher order terms $d^q_3, d^q_5,\cdots$ vanish. But at the $Q^2$ that can be reached in the current experiments, they are not necessarily negligible.    
The results of the chiral quark-soliton model, which predicts values of $d^{u+d}_1$ close to findings
in the experimental analysis~\cite{Goeke:2007fp}, can been used to estimate the contribution of the $d_3^q$ term. At the kinematics relevant for this analysis a ratio $d^{u+d}_3/d^{u+d}_1 \approx 0.3$ was found~\cite{Kivel:2000fg}. A systematic uncertainty of $\delta(d^{u+d}_1)/d^{u+d}_1 =\pm 0.30$ has therefore been included into the results of~\cite{Burkert:2018bqq} for $d_1^{u+d}(t)$.

One may ask if the first two terms in the Gegenbauer polynomial expansion $d_1^q(t)$ and $d_3^q(t)$ could be separated in some way to reduce the systematics. This has been studied in~\cite{Dutrieux:2021nlz} by including the $Q^2$-evolution into the phenomenological analysis. It was found that, assuming the same $t$-dependence, the two terms cannot currently be separated given the limited range in $Q^2$ covered by the data. In the future one may expect Lattice QCD to be able to provide a model-independent evaluation of this higher-order contribution. 


To conclude this section, the determination of $\mathcal C_{\mathcal H}(t)$ suggests that the quark contribution $\sum_qD_q(t)$ to the proton's GFF $D(t)$ is non-zero and large. These results have been supported in a recent paper on the first measurement of TCS~\cite{CLAS:2021lky} as shown in Fig.~\ref{TCS-BSA-FBA}, where the contribution of the $D$-term to the forward-backward asymmetry is seen to be 
significant. Moreover, predictions in the 
chiral quark-soliton model \cite{Goeke:2007fp} and from dispersive analysis~\cite{Pasquini:2014vua} shown in Fig.~\ref{Dt} are consistent with the results discussed here within the systematic uncertainties.     


\subsection{Future experimental developments to access GFFs.}

As discussed in section III, measurements of DVCS have so far been most effective in obtaining information related to GPDs. However, there are different experimental processes that may be employed to provide additional, or independent, information on the GPDs and GFFs.  

Implementation of a high-duty-cycle positron source, both polarized and unpolarized~\cite{PEPPo:2016saj}, at JLab would significantly enhance its capabilities in the extraction of the CFF  Re$\mathcal{H}(\xi,t)$ and thus of the gravitational form factor $D_q(t)$ and of the mechanical properties of the proton.  

The time-like Compton scattering process will be measured in parallel to the DVCS process employing the large acceptance detector systems such as CLAS12~\cite{Burkert:2020akg}. The TCS event rate is much reduced compared to DVCS and requires higher luminosity for similar sensitivity to $\mathcal{H}$. 
In experiments employing large acceptance detector 
systems, both DVCS and TCS processes are measured 
simultaneously, in quasi-real photo-production at very small $Q^2 \to 0$, and in real photo-production, where
the external production target acts as a radiator of 
real photons that undergo TCS further downstream 
in the same target cell. The science program in Hall D at JLab has measured the energy-dependence of $J/\Psi$ production with real photon beams~\cite{GlueX:2019mkq}, where data close to threshold have been employed for the extraction of the gluon scalar radius~\cite{Kharzeev:2021qkd} of the proton.   

The DDVCS process enables access to GPDs in their full kinematic dependencies on $x, \xi, t$, see Sec.~\ref{sec:EMT-FF}. At the same time it is reduced in rate by orders of magnitude compared to 
DVCS~\cite{Kopeliovich:2010xm}  
requiring  
higher luminosity than is currently achievable. 
Nevertheless, special equipment that would 
comply with such requirements has been proposed~\cite{Chen:2014psa}. Such measurements are currently planned at JLab in Hall A and  
Hall B. 
 
   
Finally, an energy-doubling of the existing electron accelerator at JLab is currently under consideration~\cite{Arrington:2021alx}. This upgrade would   extend the DVCS program to higher $Q^2$ and lower $x_B$ and better link the DVCS measurements at the current 12 GeV operation to the kinematic reach that will be available at the Electron-Ion Collider, a flagship future facility in preparation at the Brookhaven National Laboratory (discussed further below). It will also more fully open the charm sector to access the gluon GFFs. The energy upgrade will be essential for a comprehensive measurement of deeply exclusive meson production, both in the vector meson and the pseudo-scalar meson sector.





\subsection{Other phenomenological studies}

The first extraction of the $\pi^0$ GFFs in the time-like 
region~\cite{Kumano:2017lhr} based on the
process $\gamma \gamma^* \to \pi^0 \pi^0$, 
depicted in
Fig.~\ref{Jpsi-gamma-gamma-pi-pi},
which was measured in the Belle experiment 
in $e^+e^-$ collisions \cite{Belle:2015oin}.
For the quark contribution to the $\pi^0$ $D$-term 
the value $\sum_qD_q(0)\approx - 0.75$ was reported, but systematic uncertainties have not been estimated.
It has recently been observed in~\cite{Lorce:2022tiq} that kinematical corrections may significantly impact 
the extraction of 
generalized distribution amplitudes
from experimental data and should be 
taken into account in future analyses.

Phenomenological studies of gluon GFFs have been
presented in \cite{Kou:2021qdc,Wang:2022ndz} 
where the gluon $D_G(t)$ form factor of the proton
was extracted from $J/\Psi$ threshold photoproduction
data. (For remarks on the
theoretical status of this process
see Sec.~\ref{Subsec:Jpsi-pi0-GDA}.) A similar study for the lighter
$\phi$-meson was presented in \cite{Hatta:2021can}.


