\section{Measuring gravitational form factors} 
\label{sec:EMT-FF}

There is no direct way to measure the proton GFFs, as it would require measurements of the graviton-proton  interaction~\cite{Kobzarev:1962wt,Pagels:1966zza}. More recent theoretical developments have shown, however, that the GFFs may be probed indirectly in various exclusive processes. This is the subject of
this section.

\subsection{Deeply virtual Compton scattering (DVCS)}
\label{DVCS}

In DVCS, the most explored process so far that accesses the GFFs,
high-energy charged leptons scatter off protons 
or nuclei by exchanging a deeply virtual photon,
producing a real photon in the final state
\cite{Muller:1994ses,Radyushkin:1996nd,Ji:1996nm}.
In  the limit $Q^2\to\infty$ and $P\cdot q\to\infty$ with 
$(-t)\ll Q^2$, the 
process is described in 
QCD with collinear factorization \cite{Collins:1998be}.
In this limit the DVCS amplitude is written in terms 
of a ``hard part'' which can be calculated order by order 
in perturbative QCD,  and a nonperturbative ``soft part'' 
described in terms of generalized parton distributions 
(GPDs) as shown in Fig.~\ref{DVCS-BH}a. 
GPDs are universal, i.e.,\ the same non-perturbative functions enter the description of different hard exclusive reactions. 

\begin{figure}[b]
\includegraphics[width=0.95\linewidth]{Figures/Figure2_D7.jpg}

\vspace{-10mm}\hspace{4mm}
{\footnotesize  (a) \hspace{30mm} (b)}
\vspace{-2mm}
\caption{\footnotesize\label{DVCS-BH} (a)  
QCD factorization of the DVCS amplitude.
The perturbatively calculable ``hard part'' 
is shown to lowest order in the strong coupling.
The nonperturbative ``soft part'' is described 
by the universal quark GPDs. 
(b) One of the QED diagrams for the 
amplitude of the Bethe-Heitler 
process, which has the same final state as DVCS 
and interferes with it. The  
Bethe-Heitler process is calculable requiring only the proton electromagnetic FFs as input.}
\end{figure}

GPDs are functions of $x$, $\xi$, $t$, where the skewness variable $\xi$, given in the high-energy limit by 
$\xi = x_B/(2-x_B)$, and $t$ are observable, while 
$x$ enters the description of the DVCS amplitude 
only as an integration variable 
and is integrated over in DVCS observables. GPDs 
encompass both PDFs and the electromagnetic FFs 
discussed in Sec.~\ref{Sec-1:intro}.  
Starting from GPDs, one recovers PDFs 
when $\xi\to 0$ and $t\to 0$; 
integrating particular GPDs over $x$ yields the
electromagnetic FFs.

GPDs parameterize the matrix elements of 
certain non-local operators which can be expanded in 
terms of an infinite tower of local operators with various
$J^{PC}$ quantum numbers.
This includes operators with the quantum numbers of 
the graviton ($J=2$), and so part of the information 
about how the proton would interact with a graviton is
encoded within this tower. As the electromagnetic coupling 
to quarks is many orders of magnitude stronger than gravity,
the DVCS process is an effective tool to probe the proton's
gravitational properties.  Gluon GPDs are accessible in DVCS
only at higher orders in $\alpha_s$. 

The leading contribution to DVCS is described in terms of four GPDs. Two of them, namely $H_q(x,\xi,t)$ and $E_q(x,\xi,t)$, give access to quark GFFs as follows
\begin{equation}\label{mellin-1}
\begin{aligned}
\int_{-1}^{1} \mathrm{d}x \, x H_q(x, \xi, t)  = A_q(t) + \xi^2 D_q(t),\\
\int_{-1}^{1} \mathrm{d}x \, x E_q(x, \xi, t) = B_q(t) -
\xi^2 D_q(t) ,
\end{aligned}
\end{equation}
where $B_q(t)=2J_q(t)-A_q(t)$. $B_q(0)$ is the quark
contribution to the proton's anomalous gravitomagnetic 
moment. Analogous relations hold for gluons, and $B(0)=\sum_aB_a(0)$ vanishes~\cite{Kobzarev:1962wt,Teryaev:1999su,Brodsky:2000ii,Lowdon:2017idv,Cotogno:2019xcl,Lorce:2019sbq} due to the constraints~\eqref{Tconstraint} and~\eqref{Lconstraint}.

The actual observables in DVCS are Compton form factors (CFFs), e.g.~${\mathcal H}(\xi,t)$, which are 
complex-valued convolution integrals defined at leading order in $\alpha_s(Q)$ 
\begin{multline} 
    \text{Re}{\mathcal H}(\xi,t) + i\, \text{Im}{\mathcal H}(\xi,t)  = \label{GPD-CFF} \\
    \sum_q e^2_q\int_{-1}^{1} dx \left[ \frac{1}{\xi-x-i\epsilon} -  \frac{1}{\xi+x-i\epsilon} \right] H_q(x,\xi,t).
\end{multline}
and similarly for other GPDs. The CFFs can be related to experimentally accessible quantities such as differential cross sections and beam and target polarization asymmetries.  

The DVCS cross section is typically very small.
Fortunately, DVCS interferes with the Bethe-Heitler 
process, which can be computed in QED given the proton's electromagnetic FFs, and has the 
same final state but with the final state photon emitted 
from the electron lines, see Fig.~\ref{DVCS-BH}b.
The interference term projects out Im${\mathcal H(\xi, t)}$
when a spin-polarized
electron beam is employed, while Re${\mathcal H(\xi, t)}$ 
contributes dominantly to the unpolarized DVCS cross section,
and may be constrained through precise unpolarized cross 
section measurements.

The convolution integrals like (\ref{GPD-CFF}) cannot be inverted in a model-independent way
to yield GPDs \cite{Bertone:2021yyz}. 
However, with experimental information from other exclusive processes becoming available (to be discussed below), the GPDs may be further constrained. Presently, a model-independent extraction of the GPDs
and, via 
(\ref{mellin-1}), of the GFFs $A_q(t)$ and $J_q(t)$ 
is not possible. In the case of the GFF $D_q(t)$, however, the situation is more fortunate. In particular, the real and imaginary parts of ${\mathcal H(\xi,t)}$
are related by the fixed-$t$ dispersion relation 
\cite{Diehl:2007jb,Anikin:2007tx} 
\begin{multline}
 {\rm Re}{\mathcal H}(\xi,t) = \mathcal C_{\mathcal H}(t) \\+\frac{ 1}{\pi} \,{\text{P.V.}} \int_0^1 \text{d}\xi' \left[\frac{1}{\xi-\xi'}  -\frac{1}{\xi+\xi'}\right] {\rm Im}{\mathcal H}(\xi',t), \label{DR}
\end{multline}
where $\text{P.V.}$ denotes the principal value of the Cauchy integral. This expression contains a real subtraction term $\mathcal C_{\mathcal H}(t)$ given by 
\begin{equation}
    \mathcal C_{\mathcal H}(t)=2\sum_qe^2_q\int_{-1}^1\ud z\,\frac{D^q_\text{term}(z,t)}{1-z},
\end{equation}
where $D^q_\text{term}(z,t)$, originally introduced in
\cite{Polyakov:1999gs}
and further elucidated in
\cite{Teryaev:2001qm},
has the expansion \cite{Goeke:2001tz}
\begin{equation}\label{Gegenbauer}
    D^q_\text{term}(z,t)=(1-z^2)\sum_{\text{odd}\,n}d^q_n(t)\,C^{3/2}_n(z)
\end{equation}
with $C^\alpha_n(z)$ the Gegenbauer polynomials
which diagon\-alize the leading-order evolution equations (the renormalization scale dependence is suppressed throughout this work).
In the limit of renormalization scale $\mu\to\infty$,
all $d^q_n(t)$ go to zero except $d^q_1(t)$, which is 
related to the GFF $D_q(t)$ as follows
\begin{equation}\label{Dtermint}
    D_q(t) = \frac{4}{5}\, d^q_1(t) =
    \int_{-1}^1\ud z\,z\,D^q_\text{term}(z,t) \,.
\end{equation}
Thus, extracting information on 
Im${\mathcal H(\xi,t)}$ and Re${\mathcal H(\xi,t)}$ and their scale dependence from experimental data provides access to the GFF $D_q(t)$.

\subsection{DVCS with positron and electron beams}

When DVCS data with positron 
{and electron beams are available,}
it is possible to measure the beam charge asymmetry 
$A_C = (\sigma_{\rm DVCS}^{e^-}-\sigma_{\rm DVCS}^{e^+})/(\sigma_{\rm DVCS}^{e^-}+\sigma_{\rm DVCS}^{e^+})$. 
At leading twist, the numerator of $A_C$ is given
by the real part of the 
DVCS Bethe-Heitler
interference term providing the cleanest access to
Re$\mathcal{H}$ \cite{Kivel:2000fg,Belitsky:2001ns}. 
In contrast to this, in DVCS with electrons (or positrons)
alone, additional theoretical assumptions in the CFF
extraction procedure are unavoidable~\cite{CLAS:2021gwi}. 

\begin{figure}[t]
\includegraphics[width=1.0\columnwidth]{Figures/Figure3_D7.jpg}
%\includegraphics[width=0.5\columnwidth]{Figures/Figure2_D4-01.jpg} 

\vspace{-3mm}
{\footnotesize  (a) \hspace{35mm} (b)}

\vspace{-2mm}
\caption{\footnotesize  
(a) The process $\gamma\gamma^\ast\to\pi^0\pi^0$ 
is described in terms of generalized distribution 
amplitudes (GDAs), which provide access to GFFs in 
the time-like region $t>0$.
(b) Threshold $J/\Psi$ photo-production on proton. 
This process is sensitive to the gluon GPDs.  
}
\label{Jpsi-gamma-gamma-pi-pi}
\end{figure}


\vspace{-4mm}

\subsection{\boldmath $\gamma\gamma^\ast\to\pi^0\pi^0$ and $J/\Psi$ threshold production}
\label{Subsec:Jpsi-pi0-GDA}

\vspace{-2mm}

The process $\gamma\gamma^\ast\to\pi^0\pi^0$ shown in 
Fig.~\ref{Jpsi-gamma-gamma-pi-pi}a
can be studied, e.g., at electron-positron
colliders, and is described in terms of generalized 
distribution amplitudes which correspond to GPDs 
continued analytically from the $t$- to the $s$-channel~\cite{Muller:1994ses,Diehl:1998dk}.
In this way, one can access information
on GFFs in the time-like region where $t>0$.
This process provides a unique opportunity to study 
the structure of unstable hadrons like pions that are not available
as targets.

Exclusive $J/\Psi$ photo-production at threshold
\cite{Kharzeev:1995ij,Kharzeev:2021qkd},
depicted in 
Fig.~\ref{Jpsi-gamma-gamma-pi-pi}b,
is expected to be sensitive to gluon GPDs, 
which in DVCS are accessible only at higher 
orders in $\alpha_s$.
The process of heavy vector quarkonium 
photoproduction was shown to factorize in the heavy 
quark limit at one-loop order in perturbative QCD
\cite{Ivanov:2004vd}. While it was argued that the
factorization in terms of gluon GPDs remains valid 
at threshold \cite{Guo:2021ibg}, a recent study has 
challenged this picture \cite{Sun:2021gmi} and the theoretical status of this process is currently 
uncertain. $J/\Psi$ photoproduction can also be studied with quasi-real photons of virtualities as low as $Q^2 \lesssim 0.1\,{\rm GeV}^2$ emitted by electrons, in conjunction with electroproduction and DVCS.
Both processes have been used to obtain information 
about GFFs, and will be discussed in more detail in
Sec.~\ref{experiments}.




\subsection{Time-like Compton scattering, double DVCS and deeply virtual meson production} 

Several other processes provide complementary 
information about the nucleon GFFs. One of them
is time-like Compton scattering (TCS), $\gamma p \to p'\gamma^\ast$,
where the final state virtual photon produces 
an $e^+e^-$ pair \cite{Berger:2001xd,Pire:2011st,CLAS:2021lky}.
This time-reversed version of DVCS provides an
important opportunity to test the universality of GPDs. 
In TCS,
Im$\mathcal{H}$ can be accessed through
the polarized beam spin asymmetry and
Re$\mathcal{H}$ through a forward-backward
asymmetry of the final-state $e^+e^-$ pair
in its centre-of-mass frame.


\begin{figure}[t]
    \centering
    \includegraphics[width=1.0\columnwidth]{Figures/Figure4_D7.jpg}
{\footnotesize  (a) \hspace{37mm} (b)}
\vspace{-2mm}
    \caption{\footnotesize The leading DDVCS diagram 
    (a) and one of the leading diagrams for Deeply
    Virtual Meson Production (b).
    The ellipse where the meson is produced
    is the nonperturbative distribution amplitude. 
    \label{fig:ddvcs-dvmp}}
\end{figure}

The double DVCS (DDVCS) process displayed in 
Fig.~\ref{fig:ddvcs-dvmp}a may play an important role at future facilities. 
It is a variant of DVCS 
with the final-state time-like photon decaying into a $e^+e^-$ or $\mu^+\mu^-$  pair.
While in DVCS the GPDs are sampled along the lines
$x=\pm\xi$ in the convolution integrals (\ref{GPD-CFF}), 
this constraint is relaxed in DDVCS due 
to the variable invariant mass of the  
lepton pair. This 
is an advantage of this process, and 
will be of importance for less model-dependent global extractions of GPDs. 

Deeply virtual meson production 
is another process sensitive to 
GPDs, providing another important
test of their universality,
see Fig.~\ref{fig:ddvcs-dvmp}b.
Production of different vector mesons provides 
sensitivity to GPDs of different quark flavors
which is an advantage over DVCS. At the same time,
however, this process is more difficult to analyze
than DVCS since gluons contribute on the same
footing as quarks (Fig.~\ref{fig:ddvcs-dvmp}b
only shows a quark diagram) and one 
in general expects larger power corrections.

Finally, a new class of hard scattering processes with a larger number of particles in the final state has recently emerged~\cite{Qiu:2022bpq}. The $\gamma N \to \gamma \gamma N'$ process
\cite{Pedrak:2020mfm,Grocholski:2022rqj} with a large invariant diphoton mass is particularly interesting since, contrary to DVCS and TCS, it gives access to the charge-conjugation odd part of the quark GPDs. Other processes involving a meson-meson 
\cite{Ivanov:2002jj} or photon-meson pair~\cite{Boussarie:2016qop,Duplancic:2018bum} produced with a large invariant mass have also been proposed.

Sec.~\ref{experiments} will focus on DVCS and TCS as the most suitable processes for near-future studies of proton CFFs.


