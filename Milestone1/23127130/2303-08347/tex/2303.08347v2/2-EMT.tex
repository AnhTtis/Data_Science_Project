\section{The energy-momentum tensor} \label{sec-II}

In this section, after reviewing the 
definition and properties of the EMT
in QCD, the gravitational form factors 
(GFFs) of the proton are
introduced. It is shown how GFFs can
be leveraged to elucidate the proton's 
mass and spin decompositions.



\subsection{Definition of the EMT operator}
\label{sec-II.A}
In QCD, the EMT 
$T^{\mu\nu}=\sum_qT^{\mu\nu}_q+T^{\mu\nu}_G$ can 
be decomposed into gauge-invariant quark and gluon
parts as 
\begin{equation}\label{EMTop}
    \begin{aligned} 
    T^{\mu\nu}_q
    &=\overline\psi_q\gamma^\mu\,iD^\nu\psi_q,\\
    T^{\mu\nu}_G
    &=-F^{c\mu\lambda}F^{c\nu}_{\phantom{c\nu}\lambda}
    +\tfrac{1}{4}\,g^{\mu\nu} F^2
    \end{aligned}  
\end{equation}
with $g_{\mu\nu}=\text{diag}(+1,-1,-1,-1)$ the 
Minkowski metric. In quantum field theory, the
expressions for the matrix elements of bare 
operators contain divergences and must be 
renormalized~\cite{tHooft:1972tcz}. Therefore,
each term in \eqref{EMTop} is understood as a 
renormalized operator defined at some 
renormalization scale $\mu$. 
The components of the EMT are interpreted in the 
same way as in the classical theory, namely $T^{00}$ 
is the energy density, $T^{0i}$ is the momentum density,
$T^{i0}$ is the energy flux, and $T^{ij}$ is the momentum
flux or stress tensor. 

Since the antisymmetric part $T^{[\mu\nu]}=\frac{1}{2}(T^{\mu\nu}-T^{\nu\mu})$ of \eqref{EMTop} can be written as a total divergence using the equations of motion, it does not contribute to the total four-momentum and angular momentum of the system. In the literature, one often
considers only the symmetric part $T^{\{\mu\nu\}}=\frac{1}{2}(T^{\mu\nu}+T^{\nu\mu})$, 
known as the Belinfante EMT~\cite{Belinfante:1962zz},
%. The price to pay is that one loses in that case the distinction between orbital angular momentum and spin \cite{Leader:2013jra,Lorce:2017wkb}.
where the distinction between orbital angular momentum and spin is lost \cite{Leader:2013jra,Lorce:2017wkb}.



\subsection{Trace anomaly}
\label{Subsec:trace-anomaly}

% The trace of the EMT $g_{\mu\nu}T^{\mu\nu}$ measures the breaking of dilation symmetry. At the classical level one finds that $g_{\mu\nu}T^{\mu\nu}_\text{class}=\sum_q m_q\overline\psi_q\psi_q$, indicating that dilation symmetry is explicitly broken by the presence of quark masses. Indeed, for $m_q=0$ the QCD Lagrangian (\ref{Eq:Lagrangian}) would be invariant at the classical level under scale transformations $x\mapsto x' = \lambda x$. It turns out that dilation symmetry is also broken by quantum corrections. As a result, even though the renormalized EMT looks formally the same as in the classical theory, its trace receives anomalous (i.e.~nonclassical) contributions~\cite{Collins:1976yq,Nielsen:1977sy}
%\begin{equation}\label{EMTtrace}
%    g_{\mu\nu}T^{\mu\nu}=\sum_q (1+\gamma_m)m_q\,\overline\psi_q\psi_q+\tfrac{\beta(g)}{2g}\,F^2,
%\end{equation}
% where $\gamma_m$ is the anomalous quark mass dimension and $\beta(g)=\partial g/\partial\ln\mu$ is the QCD beta function. As will be discussed later, this trace anomaly plays an important role when discussing the mass and mechanical properties of the proton. Note that $g_{\mu\nu}T^{\mu\nu}_q$ and $g_{\mu\nu}T^{\mu\nu}_G$ mix with each other under renormalization, and each one contains both quark and gluon scalar operators~\cite{Hatta:2018sqd,Tanaka:2018nae,Ahmed:2022adh}. {\color{blue}[Maybe we can remove this last, more technical sentence]} 


The invariance of the classical Lagrangian of a theory under a
certain symmetry implies the existence of a conserved, so-called 
Noether, current \cite{Noether:1918zz}. For instance, 
the EMT is the Noether current associated with the 
invariance of a theory under space-time translations. 
If the classical symmetry is obeyed in quantum field theory
(as is the case for space-time translations) one obtains a
conservation law.

If a classical symmetry is spoiled by quantum effects, 
then one speaks of a ``quantum anomaly'' and there is no
associated conservation law. One important example is the 
trace anomaly 
(for another example see Sec.~\ref{Sec-4A-chiral-symmetry}): 
the QCD Lagrangian 
(\ref{Eq:Lagrangian}) is ``approximately'' invariant under 
scale transformations $x\mapsto x' = \lambda x$ with 
arbitrary $\lambda>0$. It is not an exact symmetry 
since the divergence of the corresponding Noether current 
does not vanish but is equal at the classical level to
$g_{\mu\nu}T^{\mu\nu}_\text{class}=\sum_qm_q\,\overline\psi_q\psi_q$.
In the light quark sector, due to the smallness of the 
up- and down-quark masses, one would nevertheless expect this 
to be a good approximate symmetry similarly to the isospin
symmetry encountered in Sec.~\ref{Sec-1A:magnetic-moment}.  
However, quantum corrections alter the trace of the EMT as
\cite{Collins:1976yq,Nielsen:1977sy}
\begin{equation}\label{EMTtrace}
    g_{\mu\nu}T^{\mu\nu}=\sum_q (1+\gamma_m)m_q\,\overline\psi_q\psi_q+\tfrac{\beta(g)}{2g}\,F^2,
\end{equation}
where $\gamma_m$ is the anomalous quark mass 
dimension and $\beta(g)=\partial g/\partial\ln\mu$ 
is the QCD beta function which describes how the coupling 
changes with the renormalization scale. As will be discussed 
later, the trace anomaly plays an important role for
the mass and mechanical properties of the proton. 
For more details, see \cite{Braun:2003rp} 
and \cite{Hatta:2018sqd,Tanaka:2018nae,Ahmed:2022adh}.



\subsection{Definition of the proton gravitational form factors}
\label{Subsec:def-GFFs}

The electromagnetic structure of the proton is encoded in the matrix elements of the electromagnetic current $\langle p',\vec s^{\,\prime}|J^{\mu}_{em}|p,\vec s\rangle$. Similarly, the matrix elements of the EMT operator $\la p^\prime,\vec s^{\,\prime}|T_a^{\mu\nu} |p,\vec s\rangle$ for quarks ($a=q$) and gluons ($a=G$) allow one to study 
% not only the mechanical properties of the proton, but also the decomposition of the proton mass and spin into quark and gluon contributions. 
the mass and spin decompositions, as well as the mechanical properties.
% Thanks to Poincar\'e symmetry, these matrix elements can be 
%\cedric{written in terms of a finite number of tensor structures~\cite{Kobzarev:1962wt,Pagels:1966zza,Ji:1996ek,Bakker:2004ib,Lorce:2022cle}
%\begin{equation}\label{EMTparam}
%\begin{aligned}
%&   \la p^\prime,\vec s^{\,\prime}|  
%    T_a^{\mu\nu} |p,\vec s\rangle
%    =\overline u(p^\prime,\vec s^{\,\prime})\Bigg[A_a(t)\,\frac{P^\mu P^\nu}{M_N}\\
%&   + D_a(t)\,
%    \frac{\Delta^\mu\Delta^\nu-g^{\mu\nu}\Delta^2}{4M_N}+ \bar{C}_a(t)\,M_N\,g^{\mu\nu}\\
%&   +
%    J_a(t)\ \frac{P^{\{\mu}i\sigma^{\nu\}\lambda}
%    \Delta_\lambda}{M_N}
%    -S_a(t)\ \frac{P^{[\mu}i\sigma^{\nu]\lambda}
%    \Delta_\lambda}{M_N}\Bigg]u(p,\vec s),
%\end{aligned}    
%\end{equation}
%where the EMT operators of quarks ($a=q$) and gluons ($a=G$) are evaluated at the origin $x^\mu=0$. Here $u(p,\vec s)$ is the usual free Dirac spinor, $M_N$ denotes the nucleon mass, and the symmetric kinematical variables are defined as
%\begin{equation}
%    P=\tfrac{1}{2}(p'+p),       \quad
%        \Delta=p'-p.
%\end{equation}
%The coefficients $A_a(t)$, $D_a(t)$, $\bar C_a(t)$, $J_a(t)$ and $S_a(t)$ are Lorentz-invariant functions of the square four-momentum transfer $t=\Delta^2$ which fully characterize the EMT matrix elements.}{..}

Thanks to Poincar\'e symmetry, these matrix elements can be 
written as~\cite{Kobzarev:1962wt,Pagels:1966zza,Ji:1996ek,Bakker:2004ib,Lorce:2022cle}
\begin{equation}\label{EMTparam}
\begin{aligned}
&   \la p^\prime,\vec s^{\,\prime}|  
    T_a^{\mu\nu} |p,\vec s\rangle
    =\overline u(p^\prime,\vec s^{\,\prime})\Bigg[A_a(t)\,\frac{P^\mu P^\nu}{M_N}\\
&   + D_a(t)\,
    \frac{\Delta^\mu\Delta^\nu-g^{\mu\nu}\Delta^2}{4M_N}+ \bar{C}_a(t)\,M_N\,g^{\mu\nu}\\
&   +
    J_a(t)\ \frac{P^{\{\mu}i\sigma^{\nu\}\lambda}
    \Delta_\lambda}{M_N}
    -S_a(t)\ \frac{P^{[\mu}i\sigma^{\nu]\lambda}
    \Delta_\lambda}{M_N}\Bigg]u(p,\vec s)
\end{aligned}    
\end{equation}
    with $P=(p'+p)/2$ and $\Delta=p'-p$ the symmetric kinematical variables, $u(p,\vec s)$ the usual free Dirac spinor, and $M_N$ the nucleon mass. The Lorentz-invariant functions $A_a(t)$, $D_a(t)$, $\bar C_a(t)$, $J_a(t)$ and $S_a(t)$ depend on the square of the four-momentum transfer $t=\Delta^2$. They are the EMT analogues of the more familiar electromagnetic FFs, and are accordingly called gravitational form factors (GFFs). In contrast to the electromagnetic FFs, these GFFs inherit also a renormalization
scale dependence from the associated operators, which 
is omitted in the notation for convenience. The \textit{total} 
GFFs $\sum_a A_a(t)$, $\sum_a D_a(t)$, $\sum_a\bar C_a(t)$ and $\sum_a J_a(t)$
are, however, renormalization scale independent~\cite{Nielsen:1977sy}. 


On top of restricting the number of GFFs, Poincar\'e 
symmetry imposes additional constraints, namely
\ba
    A(0)&=&\sum_qA_q(0)+A_G(0)=1,  \label{Tconstraint}\\
    J(0)&=&\sum_qJ_q(0)+J_G(0)=\tfrac{1}{2}, \label{Lconstraint}\\
    \tfrac{1}{2}\Delta\Sigma &=& \sum_q S_q(0), \label{Lconstraint-2} \\
    \bar C(t)&=&\sum_q\bar C_q(t)+\bar C_G(t)=0, \label{cbar-constraint}
\ea
where (\ref{Tconstraint}) follows from translation
symmetry \cite{Ji:1997pf}, while~\eqref{Lconstraint} 
and~\eqref{Lconstraint-2} result from Lorentz symmetry 
\cite{Ji:1996ek,Bakker:2004ib}, with
$\tfrac{1}{2}\Delta\Sigma$ denoting the quark spin 
contribution to the nucleon spin. The constraint 
(\ref{cbar-constraint}), valid for any $t$,
follows from EMT conservation 
$\partial_\mu T^{\mu\nu}=0$. Interestingly, the
renormalization-scale invariant
quantity~\cite{Polyakov:1999gs} 
\begin{equation}\label{Eq:D-define}
    D\equiv D(0)=\sum_qD_q(0)+D_G(0),
\end{equation}
known as the $D$-term, is a global 
property of the proton (and, in fact, any hadron), 
whose value is not fixed by spacetime symmetries 
\cite{Polyakov:1999gs}. Its physical interpretation 
will be discussed in Sec.~\ref{interpretation}. 

Until recently, the only information about GFFs 
known from phenomenology was
$A_a(0)=\int_{-1}^1\ud x\, x\,f^a_1(x)$, corresponding 
to the fraction of proton momentum carried by the 
partons $a$ as inferred from DIS experiments, and
$S_q(0)=\frac{1}{2}\int_{-1}^1\ud x\, g^q_1(x)$, 
where $g^q_1(x)$ is the quark helicity 
distribution~\cite{Aidala:2012mv}.

\subsection{Decomposition of proton mass}
\label{mass-spin}

Just like the charge density is defined via a Fourier transform of the matrix elements of the electromagnetic current, the spatial distributions of energy and momentum read~\cite{Polyakov:2002yz,Polyakov:2018zvc,Lorce:2018egm}
\begin{equation}\label{Eq:static-EMT}
    \mathcal T^{\mu\nu}_a(\vec r)=\int\frac{\ud^3\Delta}{(2\pi)^32E}\,e^{-i\vec\Delta\cdot\vec r}\,\langle p'|T^{\mu\nu}_a|p\rangle
\end{equation}
in the so-called Breit frame defined by the conditions $\vec p^{\,\prime}=-\vec p=\vec\Delta/2$ and $p'^0=p^0=E=\sqrt{M_N^2+\vec\Delta^2/4}$. For ease of notation, the dependence on the nucleon polarization is omitted. Integrating over space, one obtains
\begin{equation}
    \int\ud^3r\,\mathcal T^{\mu\nu}_a(\vec r)=\frac{\langle p|T^{\mu\nu}_a|p\rangle}{2M_N}\bigg|_{\vec p=\vec 0}
\end{equation}
i.e.,~the matrix elements for the proton at rest. More explicitly, one finds 
\begin{equation}\label{EMTrest}
   \int\ud^3r\,\mathcal T^{\mu\nu}_a(\vec r)=\begin{pmatrix}U_a&0&0&0\\
    0&W_a&0&0\\ 0&0&W_a&0\\ 0&0&0&W_a\end{pmatrix}.
\end{equation}
The components $\mathcal T^{00}(\vec r)$ and $\frac{1}{3}\sum_i\mathcal T^{ii}(\vec r)$ represent the energy density and the isotropic pressure in the system, and so $U_a=\int\ud^3r\,\mathcal T^{00}_a(\vec r)=[A_a(0)+\bar C_a(0)]\,M_N$ and $W_a=\frac{1}{3}\sum_i\int\ud^3r\,\mathcal T^{ii}_a(\vec r)=-\bar C_a(0)M_N$ are respectively interpreted as the quark or gluon contributions to internal energy and pressure-volume work.

Since by definition $p^2=M_N^2$, the proton mass can be identified with the total energy in the rest frame
\begin{equation}\label{EnergySR}
    \sum_a U_a=M_N.
\end{equation}
Moreover, the proton being a bound state at mechanical equilibrium, the virial theorem says that the total pressure-volume work must vanish~\cite{Laue:1911lrk,Lorce:2017xzd,Lorce:2021xku}
\begin{equation}\label{sumrules}
    \sum_a W_a=0.
\end{equation}
These are two \textit{independent} sum rules underlying the various mass decompositions proposed in the literature, see~\cite{Lorce:2021xku} for a detailed review. To keep the following discussion as simple as possible, the standard $\overline{\text{MS}}$ scheme with the additional requirement that the trace anomaly arises purely from the gluonic sector is used in the following~\cite{Metz:2020vxd,Lorce:2021xku}. 


Defining the quark mass contribution to the nucleon mass via
\begin{equation}
    M_m=\sum_q\sigma_q\equiv\frac{\langle p|\sum_q m_q\,\overline\psi_q\psi_q|p\rangle}{2M_N}\bigg|_{\vec p=\vec 0},
\end{equation}
one obtains a three-term mass decomposition directly from the energy sum rule~\eqref{EnergySR}
\begin{equation}\label{MassSR3term}
    M_N=\sum_qM_q+M_m+M_G,
\end{equation}
where $M_q=U_q-\sigma_q$ and $M_G=U_G$ can, respectively, be interpreted as the kinetic+potential energies of quarks and gluons~\cite{Rodini:2020pis,Metz:2020vxd}. Motivated by the fact that the traceless part of the gluon EMT can directly be accessed in high-energy experiments, a further of decomposition of the gluon energy
\begin{equation}\label{JiSR}
    M_G=\bar M_G+\tfrac{1}{4}M_A
\end{equation}
into the traceless part $\bar M_G=\frac{3}{4}(U_G+W_G)=\tfrac{3}{4}A_G(0)M_N$ and pure trace part $\frac{1}{4}M_A=\frac{1}{4}(U_G-3W_G)$ has been proposed in~\cite{Ji:1994av,Ji:1995sv,Ji:2021mtz}. Since at the classical level the gluon EMT is traceless, $\bar M_G$ was interpreted as the ``classical'' gluon energy and $\frac{1}{4}M_A$ with
\begin{equation}
  M_A=\frac{\langle p|\sum_q\gamma_mm_q\,\overline\psi_q\psi_q+\frac{\beta(g)}{2g}\,F^2|p\rangle}{2M_N}\bigg|_{\vec p=\vec 0}  
\end{equation}
as the ``quantum anomalous energy''. This interpretation is, however, not supported by a careful analysis in the $\overline{\text{MS}}$ scheme. Indeed, at the level of renormalized operators, it is the \textit{total} gluon energy density (and not its traceless part) that has the familiar form $T^{00}_G=\frac{1}{2}(\vec E^2+\vec B^2)$, ensuring that time translation symmetry remains \textit{exact} under renormalization~\cite{Nielsen:1977sy,Suzuki:2013gza,Tanaka:2018nae,Metz:2020vxd,Lorce:2021xku,Ahmed:2022adh,Tanaka:2022wzy}. A recent explicit one-loop calculation within the scalar diquark model~\cite{Amor-Quiroz:2023rke} confirms that, unlike the EMT trace, the total energy does not receive any intrinsic anomalous contribution.

Since mass is a Lorentz-invariant quantity, one sometimes prefers to start from the trace of the EMT
\begin{equation}
    \langle p|g_{\mu\nu}T^{\mu\nu}|p\rangle=2p^2=2M_N^2
\end{equation}
and then decompose it into quark and gluon contributions~\cite{Shifman:1978zn,Donoghue:1992dd,Hatta:2018sqd,Tanaka:2018nae}, leading to the sum rule
\begin{equation}\label{TraceSR}
    M_N=M_m+M_A.
\end{equation}
Current phenomenology~\cite{Hoferichter:2015hva} 
and Lattice QCD calculations~\cite{Alexandrou:2019brg} 
indicate that $M_m/M_N\approx 10\%$, suggesting that most of the proton mass comes 
from the trace anomaly (and hence from the gluons, 
since $\gamma_m$ is small). To clarify the actual meaning of this result, it has been noted in~\cite{Lorce:2017xzd} that the sum rule~\eqref{TraceSR} is equivalent to writing
\begin{equation}
    M_N=\sum_a\int\ud^3r\,g_{\mu\nu}\mathcal T^{\mu\nu}_a(\vec r)=\sum_a\left(U_a-3W_a\right).
\end{equation}
While the total pressure-volume work vanishes owing to the virial theorem~\eqref{sumrules}, it does nevertheless contribute to the \textit{separate} quark and gluon contributions to the EMT trace. Since $\sum_qU_q$ and $U_G$ turn out to be of the same 
order of magnitude, the smallness of $M_m$ 
relative to $M_A$ indicates in reality that $\sum_qW_q=-W_G>0$. In other words, the net quark force is repulsive and is exactly balanced by the net attractive gluon force.

Since the four-momentum (and hence the mass) of a system is defined via the $T^{0\mu}$ components of the EMT, it has been argued in~\cite{Lorce:2017xzd,Lorce:2021xku} that a genuine mass decomposition should in principle \textit{not} entail the components $T^{ii}$. In particular, the quantities $\bar M_g$ and $M_A$ involve the gluon pressure-volume work $W_g$, and hence do not have a clean interpretation as mass contributions. From this point of view, both~\eqref{JiSR} and~\eqref{TraceSR} should rather be regarded as mere sum rules mixing the genuine mass decomposition~\eqref{MassSR3term} with the virial theorem~\eqref{sumrules}.


\subsection{Decomposition of proton spin}

A similar discussion elucidates the proton spin 
decomposition. The total angular momentum (AM) operator 
is defined, in terms of the Belinfante (symmetric) EMT 
$T^{\mu\nu}_\text{Bel}=T^{\{\mu\nu\}}$, as 
\begin{equation}
\label{Eq:total-OAM-Bel}
    \mathcal J^i=\int\ud^3r\,\epsilon^{ijk}r^jT^{0k}_\text{Bel}.
\end{equation}
Because of the explicit factor of $r^j$, the expectation value of this operator in a momentum eigenstate turns out to be ill-defined. A proper treatment requires the use of wave 
packets and amounts to considering matrix elements 
with non-vanishing momentum 
transfer~\cite{Bakker:2004ib,Leader:2013jra}. 

For convenience, only the longitudinal AM 
(i.e.,~the component along the proton average momentum 
$\vec P=\frac{1}{2}(\vec p^{\,\prime}+\vec p)$ defining 
the $z$-direction) is considered here. The discussion 
about the transverse AM turns out to be much more complex 
because of its dependence on both $|\vec P|$ and the 
choice of origin, see e.g.~\cite{Lorce:2018zpf,Lorce:2021gxs} and references therein. From 
the splitting of the EMT in~\eqref{EMTop}, one finds that 
the quark and gluon contributions to the proton spin 
$\langle\mathcal J^z\rangle=\sum_qJ^z_q+J^z_G$ are 
given by~\cite{Ji:1996ek}
\begin{equation}
    J^z_a=J_a(0),
\end{equation}
for a proton polarized in the $z$-direction. 

Working instead with an asymmetric EMT, the quark AM 
operator can be further decomposed into orbital and 
intrinsic AM terms
\begin{equation}\label{Ja}
    \mathcal J^i_q=\int\ud^3r\,\epsilon^{ijk}r^jT^{0k}_q+\int\ud^3r\,\tfrac{1}{2}\overline\psi_q\gamma^i\gamma_5\psi_q.  
\end{equation}
Calculating the corresponding matrix elements, 
one then finds that $J^z_q=L^z_q+S^z_q$ with
\begin{equation}\label{quarkspinOAM}
\begin{aligned}
    L^z_q&=J_q(0)-S_q(0),\\
    \sum_qS^z_q&=\tfrac{1}{2}\Delta\Sigma.
\end{aligned}
\end{equation}
Combining the results~\eqref{Ja} and~\eqref{quarkspinOAM} 
with the fact that the proton is a spin-$\frac{1}{2}$ 
particle, one arrives at the constraints given 
in~\eqref{Lconstraint} and~\eqref{Lconstraint-2}. 

Since gluons are spin-$1$ particles, one may wonder 
whether the gluon AM could also be decomposed into 
orbital and intrinsic contributions. This can be done, 
but it requires non-local operators to preserve gauge 
invariance~\cite{Chen:2008ag,Hatta:2011ku,Lorce:2012rr,Lorce:2012ce,Leader:2013jra,Wakamatsu:2014zza}. 
One is then led to the canonical (or Jaffe-Manohar) 
spin decomposition~\cite{Jaffe:1989jz}, to be
distinguished from the one derived here from the 
local EMT~\eqref{EMTop} and known as the kinetic 
(or Ji) spin decomposition~\cite{Ji:1996ek}. Finally, it is possible to push this analysis further and study the spatial distribution of angular momentum~\cite{Lorce:2017wkb}.


