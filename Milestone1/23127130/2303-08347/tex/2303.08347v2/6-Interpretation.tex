\section{Interpretation}
\label{interpretation}

In section~\ref{sec-II} various properties of the GFFs 
have been discussed at zero momentum transfer. Much of the
recent interest in GFFs comes from the fact that they 
contain information on the spatial 
distributions of energy, angular momentum, and internal forces 
that can be accessed at non-zero momentum transfer $t$, via 
an appealing interpretation which is reviewed here.


\subsection{\boldmath The static EMT}
\label{Sec:static-EMT}

The 3D interpretation 
{\cite{Polyakov:2002yz} in Eq.}~\eqref{Eq:static-EMT} 
of the information encoded by GFFs provides 
analogies to intuitive concepts such as pressure. 
%\cedric{The interpretation is carried out in the 
%Breit frame, where $\Delta^\mu =(0,\vec{\Delta})$ and 
%$P^\mu=(E,\vec 0)$ with 
%$E=(M^2_N+\frac14\vec{\Delta}^2)^{1/2}$, 
%by introducing the static EMT
%\begin{equation}
%    \mathcal T^{\mu\nu}(\vec{r}) = 
%    \int\frac{\ud^3\Delta}{(2\pi)^32E}\,e^{-i\vec{\Delta}\cdot\vec{r}}\,
%    \langle p'|T^{\mu\nu}(0)|p\rangle, 
    %\label{Eq:static-EMT}
%\end{equation}
%where for brevity the dependence of 
%$\mathcal T^{\mu\nu}(\vec{r})$ on the nucleon 
%polarization \cite{Polyakov:2002yz} is suppressed.}{}
A 2D interpretation can also be carried out in other 
frames~\cite{Lorce:2018egm,Freese:2021czn,Freese:2021mzg} 
with Abel transformations allowing one to relate 2D and 
3D interpretations~\cite{Panteleeva:2021iip}.

Considering 2D EMT distributions for a nucleon state 
boosted to the infinite-momentum frame has the advantage 
that in this case the {nucleon can be perfectly localized around the} transverse center of momentum \cite{Burkardt:2000za}. 
In other frames or in 3D, an exact probabilistic parton 
density interpretation does not hold in general.
The reservations are analogous to those in the case of, 
e.g., the interpretation of the electric FF in terms of 
a 3D electrostatic charge distribution (and the definition 
of electric mean square charge radius which, despite all
caveats, remains a popular concept, giving an idea of the
proton's size).
The 3D EMT description is nevertheless mathematically 
rigorous \cite{Polyakov:2018zvc} and can be interpreted 
in terms of quasi-probabilistic distributions from a 
phase-space point of view \cite{Lorce:2018egm,Lorce:2020onh}.  
A strict probabilistic interpretation is, however, justified 
for heavy nuclei and for the nucleon in the large-$N_c$ 
limit, where recoil effects can be safely neglected  
\cite{Polyakov:2002yz,Goeke:2007fp,Polyakov:2018zvc,Lorce:2022cle}.

%\cedric{In~(\ref{Eq:static-EMT}) the total static EMT is considered, but one can also define separate quark and gluon static EMTs \cite{Polyakov:2002yz,Lorce:2018egm}.}{}
The meaning of the different components of the static EMT is 
intuitively clear, with $\mathcal T^{00}(\vec{r})$ denoting 
the energy distribution and $\mathcal T^{0k}(\vec{r})$ 
representing the spatial distribution of  momentum.  
In the following sections the focus is on 
$\mathcal T^{ij}(\vec{r})$ which are perhaps the most 
interesting components of the static EMT, thanks to 
their relation to the stress tensor and the $D$-term.



\subsection{\boldmath The stress tensor and the $D$-term}
\label{Sec:stress-tensor}

The key to the mechanical properties of the proton is the 
symmetric stress tensor $\mathcal T^{ij}(\vec{r})$ given 
by \cite{Polyakov:2002yz} 
\begin{equation}
    \label{Eq:Tij-p-s}
    \mathcal T^{ij}(\vec{r}) = 
    \biggl(\frac{r^ir^j}{r^2}-\frac13\,\delta^{ij}\biggr)\,s(r)
    + \delta^{ij}\,p(r)\,
\end{equation}
with $s(r)$ known as the shear force (or anisotropic stress) 
and $p(r)$ as the pressure with $r=|\vec{r}|$. 
Both are connected by the differential equation 
$\frac23\,\frac{\ud}{\ud r}s(r)+\frac2r\,s(r)+\frac{\ud}{\ud r}p(r)=0$ 
and $p(r)$ obeys $\int_0^\infty \ud r\,r^2p(r)=0$ 
\cite{von-Laue:1911}, a necessary but not sufficient 
condition for stability. These relations originate from 
the EMT conservation expressed by 
$\nabla^i\mathcal T^{ij}(\vec{r}) = 0$ for the static EMT. 
The total $D$-term $D(0)$ can be expressed in terms of 
$p(r)$ and $s(r)$ in two equivalent ways,
\begin{eqnarray} 
    D(0) 
    = - \frac{4}{15}\,M_N \!\!\int{\ud^3r}\, r^2 s(r) 
    = M_N\!\!\int{\ud^3r}\,r^2 p(r) \label{Eq:D-term} 
    \,. \;\;\;\;\;
\end{eqnarray} 
The form of the stress tensor~\eqref{Eq:Tij-p-s} is valid 
for spin-0 and spin-$\frac12$ hadrons; for higher spins
see~\cite{Cosyn:2019aio,Polyakov:2019lbq,Cotogno:2019vjb,Kim:2020lrs,Ji:2021mfb}.

If the GFF $D(t)$ is known, then $s(r)$ and $p(r)$ 
are obtained as follows \cite{Polyakov:2018zvc}
%{\begin{eqnarray}
%D(t) &=&  4M_N \int \ud^3r\, {\frac{j_2(r\sqrt{-t})}{t}}\, s(r) \label{sr}\,, \\ \ 
% &=& 12M_N \int \ud^3r\, {\frac {j_0(r\sqrt{-t})} {2t}}\, p(r)\,, \label{pr}  
%\end{eqnarray} }
\begin{eqnarray} 
    s(r) &=& -\frac{1}{4M_N}\,r\,\frac{\ud}{\ud r}
    \frac{1}{r}\frac{\ud}{\ud r}\widetilde{D}(r),\label{sr-II} \\
    p(r) &=& \frac{1}{6M_N}\frac{1}{r^2}\frac{\ud}{\ud r}
    r^2\frac{\ud}{\ud r}\widetilde{D}(r),\label{pr-II}
\end{eqnarray} 
where $\widetilde{D}(r)= \int\frac{\ud^3\Delta}{(2\pi)^3}\,e^{-i{\vec\Delta\cdot\vec r}} D(-{\vec \Delta}^2)$. 
If the separate $D_q(t)$ and $D_G(t)$ GFFs are known,
``partial'' quark and gluon shear forces $s_q(r)$ and 
$s_G(r)$ can be defined in analogy to (\ref{sr-II}). 
In order to define ``partial'' quark and gluon pressures, 
in addition to $D_q(t)$ and $D_G(t)$ knowledge of 
$\bar{C}_q(t)=-\bar{C}_G(t)$ is required. 
The latter are responsible for ``reshuffling'' forces
between the gluon and quark subsystems inside the proton
\cite{Lorce:2017xzd,Polyakov:2018exb} and are difficult to access 
experimentally. 
$\bar{C}_q(t)$ was studied in the bag model \cite{Ji:1997gm}, chiral quark-soliton model \cite{Goeke:2007fp}, instanton vacuum model \cite{Polyakov:2018exb} and lattice QCD \cite{Liu:2021gco}. Estimates guided by renormalization group methods \cite{Hatta:2018sqd,Tanaka:2018nae,Ahmed:2022adh} yield $\bar{C}_q(0)=-0.163(3)$ at $\mu = 2\,{\rm GeV}$ in $\overline{\text{MS}}$ scheme \cite{Tanaka:2022wzy}.



\subsection{\boldmath Normal forces and the sign of the $D$-term}
\label{Sec:D-term-sign}

The stress tensor $\mathcal T^{ij}(\vec{r})$ can be 
diagonalized, with one eigenvalue given by the normal 
force per unit area $p_n(r)=\frac23\,s(r)+p(r)$ with 
the pertinent eigenvector $\vec{e}_r$. The other two
eigenvalues are degenerate (for spin-0 and spin-$\frac12$) 
and are known as tangential forces per unit area, 
$p_t(r)=-\,\frac13\,s(r)+p(r)$, with eigenvectors 
which can be chosen to be unit vectors in the 
$\vartheta$- and $\varphi$-directions in 
spherical coordinates \cite{Polyakov:2018zvc}.

The normal force appears when considering the force 
$F^i=\mathcal T^{ij}\ud S^j=p_n(r)\,\ud S\,e_r^i=[\frac23\,s(r)+p(r)]\,\ud S\,e_r^i$ 
acting on a radial area element $\ud S^j = \ud S\,e_r^j$,
where $e_r^j=r^j/r$.
General mechanical stability arguments require this force 
to be directed towards the outside, or else the system would 
implode. This implies that the normal force per unit area 
must be positive 
\begin{equation}
    p_n(r) = \frac23\,s(r)+p(r)>0 
    \label{Eq:normal-force-positivity}\,.
\end{equation}
As an immediate consequence of 
(\ref{Eq:normal-force-positivity}) 
one concludes by means of Eq.~\eqref{Eq:D-term} 
that \cite{Perevalova:2016dln}
\be\label{Eq:D-sign}
      D(0) < 0\,.
\ee
For hadronic systems like protons, hyperons, mesons or 
nuclei for which the $D$-term has been computed 
(in models, chiral perturbation theory, lattice QCD or 
by dispersive techniques, see Sec.~\ref{Sec-5:theory}) 
or inferred from experiment (in the case of the proton 
and $\pi^0$, see Sec.~\ref{experiments}) it has always 
been found to be negative in agreement with 
(\ref{Eq:D-sign}).

The above definitions and conclusions are more than 
just a fruitful analogy to mechanical systems. At this 
point it is instructive to recall how one calculates 
the radii of neutron stars, which are amenable to an
unambiguous 3D interpretation. 
In these macroscopic hadronic systems, general relativity
effects cannot be neglected and are incorporated in the
Tolman-Oppenheimer-Volkoff equation, which is solved by 
adopting a model for the nuclear matter equation of state.
The solution yields (in our notation) $p_n(r)$ inside the
neutron star as function of the distance $r$ from the center.
The obtained solution is positive in the center and decreases 
monotonically until it drops to zero at some $r=R_\ast$, 
and would be negative for $r>R_\ast$ corresponding to a
mechanical instability. 
This is avoided and a stable solution is obtained by 
defining $r=R_\ast$ to be the radius of the neutron star, 
see for instance \cite{Prakash:2000jr}. 
Thus, the point where the normal force per unit area drops 
to zero coincides with the ``edge'' of the system. 

The proton has of course no sharp ``edge'', being 
surrounded by a ``pion cloud'' due to which the normal 
force does not drop literally to zero but exhibits a 
Yukawa-type suppression at large $r$ proportional to 
$\frac{1}{r^6}\,e^{-2m_\pi r}$ \cite{Goeke:2007fp}.
In the less realistic but very instructive 
bag model, there is an ``edge'' at the bag boundary, 
where $p_n(r)$ drops to zero \cite{Neubelt:2019sou}. 
In contrast to the neutron star one does not determine the 
``edge'' of the bag model in this way. Rather the normal 
force drops ``automatically'' to zero at the bag radius 
which reflects the fact that from the very beginning 
the bag model was constructed as a simple but mechanically
stable model of hadrons \cite{Chodos:1974je}.



\subsection{\boldmath The mechanical radius of the proton and neutron}
\label{Sec:r-mech}

The ``size'' of the proton is commonly defined through 
the electric charge distribution which is indeed a useful 
concept, though only for charged hadrons. 
For an electrically neutral hadron like the neutron, 
the particle size cannot be inferred in this way.
In that case, one may still define an electric
mean square charge radius $r_{\rm ch}^2 = 6\,G'_E(0)$ 
in terms of the derivative of the electric FF $G_E(t)$
at $t=0$. But for the neutron $r_{\rm ch}^2<0$ which
gives insights about the distribution of electric 
charge inside neutron, but not about its size. 
This is ultimately due to the neutron's charge 
distribution not being positive definite.

The positive-definite normal force per unit area, 
~(\ref{Eq:normal-force-positivity}), is an ideal quantity to define
the size of the nucleon. One can define the {\it mechanical radius} 
as \cite{Polyakov:2018guq,Polyakov:2018zvc}
\begin{equation}
\label{Eq:mech-r}
r_{\rm mech}^2 
= \frac{\int \ud^3r\,r^2\,p_n(r)}
       {\int \ud^3r\,p_n(r)} 
= \frac{6 D(0)}{\int_{-\infty}^0\ud t\,D(t)}\,.
\end{equation}
Interestingly, this is an ``anti-derivative'' of a GFF 
as compared to the electric mean square charge radius defined 
in terms of the derivative of the electric FF at $t=0$.
With this definition, the proton and neutron have the same
radius (modulo isospin violating effects). Notice also
that the (isovector) electric mean square 
charge radius diverges in the chiral limit and is therefore inadequate 
to define the proton size in that case, 
while the mechanical radius in~(\ref{Eq:mech-r}) 
remains finite in the chiral limit \cite{Polyakov:2018zvc}. 
The mechanical radius of the proton is predicted to be somewhat smaller 
than its charge radius in soliton models \cite{Goeke:2007fp,Cebulla:2007ei}. 
The charge and mechanical radii become equal in 
the non-relativistic limit 
which was derived in the bag model \cite{Neubelt:2019sou,Lorce:2022cle}.


\subsection{First visualization of forces from experiment}

\begin{figure}[bhp!]
\includegraphics[width=1.0\columnwidth]{Figures/pressure.png} 
\includegraphics[width=1.0\columnwidth]{Figures/shear.png}
\caption{\footnotesize The distributions of pressure 
$r^2p_q(r)$ (top) and shear stress $r^2s_q(r)$ (bottom) 
on quarks in the proton 
based on JLab data \cite{Burkert:2018bqq,Burkert:2021ith}.
The central lines show the best 
fit.  The outer shaded areas mark the uncertainties when 
only data prior to the CLAS data are included. 
The inner shaded areas represent the uncertainties 
when the CLAS data are used. The widths of the bands 
are dominated by systematic uncertainties [which include extrapolation in unmeasured $\xi$-region when 
evaluating~(\ref{DR}) and neglect of higher-order terms 
in the Gegenbauer expansion described in~\eqref{F7}]. 
The dotted magenta curves represent the model predictions
of~\cite{Goeke:2007fp}.}
\label{pressure-shear}
\end{figure} 

\begin{figure*}[thbp!]
\includegraphics[width=2\columnwidth]{Figures/normal-tangential-force.png}
\caption{\footnotesize 2D display of the quark contribution to the distribution of forces in the proton as a function of the distance from the proton's center~\cite{Burkert:2021ith}. The light gray shading and longer arrows indicate areas of stronger forces, the dark shading and shorter arrows indicate areas of weaker forces. Left panel: Normal forces as a function of distance from the center. The arrows change magnitude and point always radially outwards. Right panel: Tangential forces as a function of distance from the center. The forces change direction and magnitude as indicated by the direction and lengths of the arrows. They change sign near 0.4~fm from the proton center. }
\label{normal-tangential-force}
\end{figure*}

The first visualization of the %pressure and shear
force distributions in the proton was presented in \cite{Burkert:2018bqq} which will be reviewed here.
%
As detailed in Sec.~\ref{Subsec:D-term-at-JLab}, 
the DVCS data from JLab experiments 
\cite{CLAS:2007clm,CLAS:2015uuo} provided  
information on the observable $\mathcal C_{\mathcal H}(t)$
in~\eqref{DR}, from which, under certain reasonable 
(at present necessary) assumptions, information about 
the quark contribution $D_{u+d}(t)$ of the proton 
was deduced. Based on this information, \eqref{pr-II}  
yields the results for the pressure $p_q(r)$ and the 
shear force $s_q(r)$ of quarks displayed in 
Fig.~\ref{pressure-shear}
(the index $q$ denotes here $u+d$ quark contributions, 
with heavier quarks neglected). In order to obtain 
$p_q(r)$, the additional assumption was made that
$\bar C_q(t)$ can be neglected.

The $r^2p_q(r)$ distribution is positive,  peaks near 
$0.25\,$fm, changes sign near $0.6\,$fm, and reaches 
its minimum value around 1.0$\,$fm. The peak value of
$r^2s_q(r)$ is around $20$~MeV\,fm$^{-1}$, and occurs 
near $0.6\,$fm from the proton's center, where the shear
force, given by $4\pi r^2s_q(r)$, reaches $240$~MeV
fm$^{-1}$ or $38$ kN, an appreciably strong force inside
the tiny proton. It is interesting to observe that these 
results are consistent with predictions from the chiral 
quark-soliton model \cite{Goeke:2007fp}
within the (large) systematic uncertainties in the data.

The quark contribution to the normal and tangential forces, 
$p_{n,q}$ and $p_{t,q}(r)$ as defined in
Sec.~\ref{Sec:D-term-sign},
are displayed in a two-dimensional plot in 
Fig.~\ref{normal-tangential-force}. This figure shows the 
3D distributions inside the proton in a slice going through 
the ``equatorial plane''.
The normal forces are strongest at mid-distances near $0.5\,$fm 
from the proton center and drop towards the center and towards 
the outer periphery. The tangential forces exhibit a node near
$0.40\,$fm from the center. 



\subsection{\boldmath The $D$-term and long-range forces}

Among the open questions in theory is the issue of how to define 
the $D$-term in the presence of long-range forces. % such as QED.
It was shown in a classical model of the proton
\cite{Bialynicki-Birula:1993shm} that $D(t)$ diverges like 
$1/\sqrt{-t}$ for $t\to0$ due to the $\frac1r$-behavior of 
the Coulomb potential  \cite{Varma:2020crx}. This result is 
model-independent and was found also for charged pions 
in chiral perturbation theory \cite{Kubis:1999db}, in 
calculations of quantum corrections to the Reissner-Nordstr\"om 
and Kerr-Newman metrics \cite{Donoghue:2001qc},
and for the electron in QED \cite{Metz:2021lqv}. 

The deeper reason why $D(t)$ diverges for $t\to0$ 
due to QED effects might be ultimately related 
to the presence of a massless physical state (the
photon) which has profound consequences in a theory.
Notice that $D(t)$ is the only GFF which exhibits 
this feature when QED effects are included. There are 
two reasons for this. First, the other proton GFFs 
are constrained at $t=0$, see~\eqref{Tconstraint}
and~\eqref{Lconstraint}, while $D(t)$ is not.
Second, $D(t)$ is the GFF most sensitive to 
forces  in a system~\cite{Hudson:2017xug}.
Notice that $D(t)$ is multiplied by the prefactor 
$(\Delta^\mu\Delta^\nu - g^{\mu\nu}\Delta^2)$ such
that despite the divergence of $D(t)$ due to QED 
effects the matrix element 
$\langle p'|T_a^{\mu\nu}|p\rangle$ is 
well-behaved in the forward limit.

There have been studies of the $D$-term for the H-atom
\cite{Ji:2021mfb,Ji:2022exr}, which defy the interpretation
presented here. This is perhaps not a surprise considering 
the differences between hadronic and atomic bound states.  
Atoms are comparatively large, low-density objects.
Pressure concepts from continuum mechanics might not 
apply to atoms whose stability is well-understood within 
non-relativistic quantum mechanics.
In contrast to this, the proton as a QCD bound state has 
nearly the same mass as an H-atom but a much smaller size
$\sim10^{-15}$m and constitutes a compact high-density system 
(15 orders of magnitude more dense than an atom) 
where continuum mechanics concepts may be applied 
and provide insightful interpretations. 
Another important aspect might be played
by the role of confinement absent for atoms which 
can be easily ionized. Hadrons constitute 
a much different type of bound state in this
respect. More theoretical work is needed to
clarify these issues.


