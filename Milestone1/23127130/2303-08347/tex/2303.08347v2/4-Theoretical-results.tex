\section{Theoretical Results}
\label{Sec-5:theory}

GFFs were introduced by \cite{Kobzarev:1962wt}
who considered spin-0 and spin-$\frac12$
particles and parity-violating weak effects 
(not discussed here), proved the
vanishing of proton's anomalous gravitomagnetic moment $B(0)=0$,
and showed that one would need energies around the 
Planck scale to measure GFFs in gravitational
interactions. This section presents an overview of GFFs from the theory perspective with 
particular focus on $D(t)$, the least known of the total GFFs.
Despite the focus on the proton, it will be insightful to mention other hadrons for comparison when appropriate.



\subsection{\boldmath Chiral symmetry and the $D$-term of the pion}
\label{Sec-4A-chiral-symmetry}

GFFs received little attention from the community
until it was realized that matrix elements such as 
$\langle\pi,\pi|T^{\mu\nu}|0\rangle$
enter the QCD description of hadronic decays of
charmonia \cite{Novikov:1980fa,Voloshin:1980zf} 
or the decay of a hypothetical light Higgs
boson, an idea entertained in the early 1990s
when the possibility of a light Higgs was not yet
experimentally excluded \cite{Donoghue:1990xh}.
These matrix elements are related to pion GFFs
in the timelike region $t>0$. 

In general, hadronic EMT matrix elements cannot 
be computed analytically in QCD, but the pion is a notable 
exception. The QCD Lagrangian~\eqref{Eq:Lagrangian} 
exhibits a classical symmetry under global left- 
and right-handed rotations in the flavor space of
up, down and strange quarks. This symmetry is approximate
due to the small but non-zero quark masses $m_q$. 
If this symmetry were realized in nature, then 
% for instance the nucleon state $N(940)\frac{1}{2}^+$ with a mass of 940~MeV$/c^2$ and $J^P = \frac{1}{2}^+$, should have the same mass as its negative-parity partner, the $N(1535)\frac{1}{2}^-$ state modulo small corrections due to the small $m_q$. 
{for example the nucleon state N(940) 
(here N stands for a state with nucleon isopin quantum number
and the number in the brackets is the rounded mass of
the state in ${\rm GeV}/c^2$) 
with the spin-parity quantum numbers $J^P={\frac12}^+$
should have the same mass as its negative-parity partner
$N(1535)$ with $J^P={\frac12}^-$ modulo small
corrections due to the small $m_q$.}
However, the latter is 
almost 600 MeV$/c^2$ heavier than the nucleon, an 
effect that cannot be attributed to current 
quark mass effects. The phenomenon that a 
symmetry of the Lagrangian is not realized in
the particle spectrum is known as spontaneous 
symmetry breaking \cite{Nambu:1961tp,Nambu:1961fr}. 
It is accompanied by the  emergence of massless Goldstone
bosons, corresponding in QCD to pions, kaons, and $\eta$-mesons,
which are not massless but are very light 
compared to other hadrons. 

In theoretical calculations, chiral symmetry is a powerful 
tool allowing one to evaluate the matrix elements of 
Goldstone bosons in the chiral limit (and for $t\to0$).
In this way, one obtains for the pion (and kaon and $\eta$) $D$-term \cite{Novikov:1980fa}
\be
       \lim\limits_{m_\pi\to 0} D_\pi = -1.
\ee
Deviations from the chiral limit 
are systematically calculable in chiral
perturbation theory \cite{Donoghue:1991qv}
and are expected to be small for pions and 
more sizable for kaons and the
$\eta$-meson \cite{Hudson:2017xug}.
The relation between the stability of the pion and spontaneous chiral symmetry breaking was discussed by~\cite{Son:2014sna}, and the
gravitational interactions of 
Goldstone bosons were studied by 
\cite{Voloshin:1982eb,Leutwyler:1989tn}.
{
For hadrons other than pions, the techniques based on the chiral limit of QCD cannot predict the $D$-term, but they can still be explored to provide insights on some properties of $D(t)$, as will be discussed in Sec.~\ref{Sec-4C-limits-in-QCD}. }

\subsection{GFFs in model studies}
\label{Sec-4B-in-models}

Interest in GFFs was once again renewed 
after it was shown that they can be
inferred from hard-exclusive reactions 
via GPDs and play a key role for the
understanding of the mass and spin 
structure of the proton, see 
Sec.~\ref{sec-II}, and further stimulated
by their interpretation in terms of forces
inside hadrons \cite{Polyakov:2002yz}.
The first model study of proton GFFs was
presented by \cite{Ji:1997gm} in the bag model, followed by works in
the chiral quark-soliton model  \cite{Petrov:1998kf,Schweitzer:2002nm,Ossmann:2004bp,Goeke:2007fp,Goeke:2007fq,Wakamatsu:2007uc,Kim:2021jjf}
and Skyrme models \cite{Cebulla:2007ei,Jung:2013bya,Perevalova:2016dln}. 

Extensive GFF model studies for the nucleon 
and other hadrons were presented in 
light-front constituent quark models
    \cite{Pasquini:2007xz,Sun:2020wfo},
diquark approaches \cite{Hwang:2007tb,Kumar:2017dbf,Chakrabarti:2020kdc,Choudhary:2022den,Fu:2022rkn,Amor-Quiroz:2023rke}, 
holographic AdS/QCD models 
\cite{Abidin:2008hn,Abidin:2009hr,Brodsky:2008pf,Chakrabarti:2015lba,Mondal:2015fok,Mondal:2016xsm,Mamo:2019mka,Mamo:2021krl,Mamo:2022eui,Fujita:2022jus},
a large-$N_c$ bag model 
    \cite{Neubelt:2019sou,Lorce:2022cle},
a cloudy bag model 
    \cite{Owa:2021hnj},
light-cone QCD sum rules
\cite{Anikin:2019kwi,Azizi:2019ytx,Aliev:2020aih,Azizi:2020jog,Ozdem:2020ieh},
the Nambu--Jona-Lasinio model
    \cite{Freese:2019bhb},
chiral quark-soliton model with strange
and heavier quarks
    \cite{Kim:2020nug,Won:2022cyy,Ghim:2022zob}, 
a dual model with complex Regge trajectories
    \cite{Fiore:2021wuj}
and in an instant-form relativistic 
impulse approximation approach
    \cite{Krutov:2020ewr,Krutov:2022zgg}.
Algebraic GPD Ans\"atze were used to shed
light on pion and kaon GFFs 
    \cite{Raya:2021zrz}
and toy models 
    \cite{Kim:2022wkc}
as well as light-cone convolution models 
    \cite{Freese:2022yur}
were used to study the deuteron GFFs.

The $D$-terms of nuclei were studied in
the liquid-drop model 
    \cite{Polyakov:2002yz},
revealing that for nuclei $D(0)\propto A^{7/3}$ 
grows strongly with mass number $A$. Studies 
in the Walecka model 
    \cite{Guzey:2005ba}
support this prediction which can 
be tested in DVCS experiments with
nuclear targets. Different results were
obtained in a non-relativistic nuclear 
spectral function approach 
    \cite{Liuti:2005qj}. 
Nuclear GFFs were also investigated
in Skyrme model frameworks
\cite{Kim:2012ts,Jung:2014jja,Kim:2022syw,GarciaMartin-Caro:2023klo}.

The GFFs for a constituent quark were studied
in a light-front Hamiltonian approach
    \cite{More:2021stk,More:2023pcy} 
which, after rescaling and regularization 
of infrared divergences, reproduces QED 
results for an electron
    \cite{Metz:2021lqv,Freese:2022jlu}. 
GFFs of the photon in QED were studied in
\cite{Friot:2006mm,Gabdrakhmanov:2012aa,Polyakov:2019lbq,Freese:2022ibw}.
An insightful model for composite 
particles is the $Q$-ball system where stable, metastable, unstable states 
were investigated, showing that, among all studied particle
properties, $D(0)$ is most sensitive to 
details of the dynamics
    \cite{Mai:2012yc,Mai:2012cx,Cantara:2015sna}.
Remarkably, the same conclusions were obtained in 
the bag model where, e.g., for the $N^{\rm th}$
highly excited nucleon state the mass increases as $M\propto N^3$ 
whereas $D(0)\propto N^8$ grows much more
strongly with $N$ \cite{Neubelt:2019sou}. 

\subsection{\boldmath Limits in QCD and 
dispersion relations}
\label{Sec-4C-limits-in-QCD}

Model-independent results for GFFs can be obtained 
in certain limiting situations in QCD, e.g., when the 
number of colors $N_c\to\infty$ or when $|t|$
becomes very small or very large, and through the use
of dispersion relation methods. These methods
are complementary to the nonperturbative lattice QCD methods which are reviewed in the next
section.

In the large-$N_c$ limit of QCD, baryons
are described as solitons of mesonic fields
\cite{Witten:1979kh}. Large-$N_c$ QCD has not been
solved (in 3+1 dimensions) and the soliton field
is not known (though it can be modelled).
Nontrivial results can, however, be derived 
based on the known symmetries of the
large-$N_c$ soliton field which are generally
well-supported in nature \cite{Dashen:1993jt}
despite $N_c=3$.
The relations of the GFFs of the nucleon and
$\Delta$ were studied in the large-$N_c$ limit 
of QCD in \cite{Panteleeva:2020ejw}.
The GFFs of the $\Delta$ are difficult to
measure, but such relations can be tested, e.g., in 
soliton models like the chiral quark-soliton model
or Skyrme model (mentioned in the previous subsection)
or in lattice QCD, discussed in the next section.

At small $|t|$, one can use chiral perturbation theory,
where one writes down an effective Lagrangian in terms of hadronic degrees of freedom with the most general interactions allowed by chiral symmetry, and free parameters which can be inferred from comparison of observable quantities with experiment.
A pioneering study to lowest order in chiral perturbation
theory was presented in \cite{Belitsky:2002jp} and 
studies at next-to-leading order \cite{Diehl:2006ya}
have been completed in
\cite{Alharazin:2020yjv}. In this way, one can obtain valuable model-independent information on the
$t$-dependence of GFFs for small $t$. For instance, for the nucleon 
the slope of $D(t)$ at $t=0$ diverges in the chiral
limit as
\be
      \frac{\ud}{\ud t}\,D(t)\bigg|_{t=0} = -
      \frac{g_A^2 M_N}{40\pi f_\pi^2 m_\pi} + \dots,
\ee
where $g_A=1.26$ is the isovector axial constant, 
$f_\pi = 93\,\rm MeV$ is the pion decay constant,
$m_\pi$ is the pion mass, and the dots indicate
(finite) higher-order chiral corrections. Such
results are reproduced in chiral soliton models
\cite{Goeke:2007fp,Cebulla:2007ei}.
The value of the $D$-term cannot be determined 
exactly in chiral perturbation theory for hadrons
other than Goldstone bosons. It is, however, 
possible to derive an upper bound, e.g., 
for the nucleon $D/M_N \le -(1.1\pm 0.1)~\rm GeV^{-1}$
in the chiral limit \cite{Gegelia:2021wnj}.
The GFFs of the $\rho$-meson
\cite{Epelbaum:2021ahi} and $\Delta$-resonance
\cite{Alharazin:2022wjj} have also been studied in chiral
perturbation theory.

Model-independent results for GFFs can also
be derived for asymptotically large momentum
transfers using power counting and perturbative
QCD methods
\cite{Tanaka:2018wea,Tong:2021ctu,Tong:2022zax}.
For instance, the proton GFFs $A_a(t)$ 
for quarks and gluons behave like $1/t^2$
at large $(-t)\to\infty$.
Since QCD factorization of hard exclusive processes
requires $(-t)\ll Q^2$ and $Q^2$ is in practice often
not large in current experimental settings, such results provide important theoretical guidelines 
to extrapolate to larger $|t|$. However,
based on experience
with analogous perturbative QCD predictions for 
the electromagnetic pion form factor, see 
e.g.\ \cite{Horn:2016rip} for a review, it is 
difficult to anticipate how large the momentum
transfer $t$ must be for a form factor to reach 
the asymptotic regime.

A theoretical study of the quark contribution to the nucleon GFF $D_q(t)$ in the range $0 < (-t) < 1\,{\rm GeV}^2$ was presented in \cite{Pasquini:2014vua} based on dispersion theory methods which rely on general principles like relativity, causality and unitarity. This approach does not require modelling other than making use of available information on pion-nucleon partial-wave helicity amplitudes and relying on mild assumptions like the saturation of the $t$-channel unitarity relation in terms of the two-pion intermediate states or input pion PDF parametrizations. 

\subsection{Lattice QCD}


Complementing the insights gained from models of proton and nuclear structure, numerical lattice QCD calculations give direct and controllable QCD predictions for matrix elements of the EMT operator. 
In particular, lattice QCD is the only known systematically improvable approach to computing observables in QCD in the low-energy (non-perturbative) regime. The approach proceeds via a discretisation of the QCD Lagrangian (\ref{Eq:Lagrangian}) onto a Euclidean space-time lattice, with a finite lattice spacing which is not physical but acts as a method of regularisation of the theory. Calculations then proceed via Monte-Carlo integration of the high-dimensional discretised path-integral; continuum QCD results are recovered in the limit of vanishing lattice discretisation scale, infinite lattice volume, and precise matching of the bare quark masses to reproduce simple physical observables. By this approach, matrix elements of local operators, such as the separated quark and gluon components of the EMT in proton or nuclear states, may be computed directly.

In the current era of precision lattice QCD calculations of proton structure, particular efforts have been made to determine the complete decomposition of the proton's spin and momentum into individual quark and gluon contributions with high precision and systematic control. For example, recent lattice QCD studies have isolated all angular momentum components in the kinetic (or Ji) decomposition~\cite{Alexandrou:2020sml,Wang:2021vqy}, with $\approx 10\%$ uncertainty in the total quark and gluon contributions; the results from one collaboration are shown in Fig.~\ref{fig:spindecomp}. This example illustrates the complementarity between theory and experiment in this area; flavour separation in lattice QCD calculations is in principle more straightforward, although some contributions, such as those from gluons or arising from ``disconnected'' contributions, e.g. strange and charm quarks in the proton, are difficult to compute because of signal-to-noise challenges. Computing the gluon spin and orbital angular momentum in the Jaffe-Manohar decomposition introduces additional challenges to the lattice QCD approach, but first results have been achieved based on constructions using both local and non-local operators~\cite{Yang:2016plb,Engelhardt:2020qtg}. 

\begin{figure}[ht!]
\includegraphics[width=0.98\columnwidth]{Figures/ProtonSpinDecomp.png}
\caption{\footnotesize Proton spin decomposition computed in lattice QCD in ~\cite{Alexandrou:2020sml}, given in the $\overline{\text{MS}}$ scheme at 2~GeV. Each component includes the contribution of both the quarks and antiquarks ($q^+=q+\overline{q}$); outer (inner) coloured bars denote the total (purely connected) contributions.}
\label{fig:spindecomp}
\end{figure}  

In the same vein, precise decompositions of the quark and gluon contributions to the proton's momentum, which are related to the mass decomposition, have been achieved with complete systematic control in the same computational frameworks that yielded the spin decomposition~\cite{Alexandrou:2020sml,Wang:2021vqy}. Contributions from the trace anomaly to the proton's mass decomposition are more difficult to compute directly with systematic control, but have been constrained using the trace sum rule~\eqref{TraceSR}; Fig.~\ref{fig:massdecomp} shows the first insight from lattice QCD into the pion mass (or quark mass) dependence of the proton's mass decomposition~\cite{Yang:2018nqn}. It is particularly notable that while the quark scalar condensate contribution varies rapidly with quark mass, the other contributions, including that of the trace anomaly, remain approximately constant. 

\begin{figure}[hbpt]

\includegraphics[width=0.9\columnwidth]{Figures/Fig6.pdf}


\caption{\footnotesize Ji's mass decomposition (i.e.~combination of~\eqref{MassSR3term} and~\eqref{JiSR}) for a proton computed in lattice QCD in~\cite{Yang:2018nqn} at a scale $\mu=2$ GeV, as a function of the pion mass.}
\label{fig:massdecomp}

\vspace{5mm}

\includegraphics[width=0.88\columnwidth]{Figures/AgGFFLattice.pdf}
\caption{\footnotesize $A_G(t)$ GFF for various hadrons from \cite{Pefkou:2021fni}, with quark masses corresponding to a larger-than-physical value of the pion mass of 450~MeV. }
\label{fig:AgGFFLattice}
\end{figure}


While local matrix elements in nuclear states can in principle be computed in lattice QCD in the same way as in the proton state, such calculations face significant practical and computational challenges, in particular compounding factorial and exponential growth in computational cost with the atomic number of the nuclear state. To date, a single first-principles calculation of 
isovector quark momentum fraction $A_{u-d}(0)$ in $^3\text{He}$~\cite{Detmold:2020snb} has been achieved; despite significant systematic uncertainties, including the result into global fits of experimental lepton-nucleus scattering data yields improved constraints on the nuclear parton distributions. Over the coming decade, it can be anticipated that the control and precision achieved in first-principles calculations of simple aspects of the gravitational structure of the proton will be extended to nuclear states.

Beyond forward-limit matrix elements, lattice QCD has also been used to compute the quark and gluon GFFs of the proton and other hadrons. Such calculations are computationally more demanding than those needed to constrain the forward-limit components, and statistical uncertainties increase with $|t|$. As a result, these studies have not yet achieved the same level of systematic control as the spin and mass decomposition. Nevertheless, the quark contributions to the proton's GFFs (and those of other hadrons such as the pion) have been computed with $|t|\lesssim 1~\text{GeV}^2$~\cite{Bali:2016wqg,Brommel:2005ee,Brommel:2007zz,Yang:2018bft,Alexandrou:2018xnp,Yang:2018nqn,Alexandrou:2017oeh,Alexandrou:2020sml,Alexandrou:2019ali,LHPC:2007blg}. The gluon contributions to the proton's GFFs are far less well-constrained, and almost all calculations to date have been performed with quark masses corresponding to larger-than-physical values of the pion mass~\cite{Shanahan:2018nnv,Shanahan:2018pib,Pefkou:2021fni,Detmold:2017oqb}. Nevertheless, the gluon GFFs with $|t|\lesssim 2~\text{GeV}^2$ were computed for a range of hadrons in \cite{Pefkou:2021fni}, allowing qualitative comparisons of their $t$-dependence as illustrated in Fig.~\ref{fig:AgGFFLattice}. Of particular recent interest has been the $D(t)$ GFF, which does not have a sum-rule constraint in the foward limit; a comparison between lattice QCD calculations of the quark and gluon contributions is illustrated in Fig.~\ref{fig:Dg}. 

\begin{figure}[tp!]
\includegraphics[width=1.0\columnwidth]{Figures/Du+d-Dg-LQCD.pdf}
\caption{\footnotesize $D_G(t)$ and $D_{u+d}(t)$ GFF for the proton from \cite{Shanahan:2018pib} and \cite{LHPC:2007blg} respectively, with quark masses corresponding to a pion mass of approximately 450 MeV. \label{fig:Dg}}
\end{figure}

In contrast to local matrix elements, matrix elements defined with light-cone separations, yielding e.g. the $x$-dependence of GPDs, can not be directly computed in Euclidean spacetime, but must be approached by indirect means. 
Significant developments over the last two decades have 
yielded a range of complementary approaches to direct calculations of GPDs themselves in the lattice QCD framework~\cite{Detmold:2005gg,PhysRevLett.110.262002,Chambers:2017dov,Ma:2017pxb,Radyushkin:2017cyf,Constantinou:2020hdm,Detmold:2021uru}. Given the significant technical and computational challenges of these approaches, the first lattice QCD studies of the $x$-dependence of the proton GPDs were achieved only recently in 2020~\cite{Alexandrou:2020zbe,Lin:2020rxa}. Calculations with complete systematic control will require continued efforts over the coming years. 

% Even more recently, the first QCD calculation of the chiral-odd transversity GPDs of the proton~\cite{Alexandrou:2021bbo} has been achieved. While systematic uncertainties remain to be controlled in this case also, these early results already showcase the complementarity between theory and experiment which will continue to grow in the coming years; while they are no more difficult to compute in the lattice QCD approach than the helicity-preserving distributions, the transversity GPDs cannot be measured in DVCS (but can in deeply virtual meson production), and are less constrained experimentally than their chiral-even counterparts. 

