\documentclass[preprint, aps,prl,footinbib]{revtex4-2}
%
\usepackage{graphicx}%
\usepackage{hyperref}
\usepackage{amsmath}
\usepackage{blindtext}
\usepackage{float}
\renewcommand{\figurename}{Figure}
\renewcommand{\tablename}{Table}
\usepackage{xcolor}
\usepackage{amssymb}
\DeclareMathOperator\erfc{erfc}

\begin{document}

\title{Biological rhythms generated by a single activator-repressor loop with heterogeneity and diffusion.}

\author{Pablo Rojas$^1$, Oreste Piro$^{2,3}$ and Martin E. Garcia$^1$}

\affiliation{$^1$Theoretical Physics and Center for Interdisciplinary Nanostructure Science and Technology (CINSaT), University of Kassel, Kassel, Germany}
 
\affiliation{$^2$Department of Ecology and Marine Resources, Institut Mediterrani d’Estudis Avançats, IMEDEA (CSIC–UIB), Balearic Islands, Spain}
\affiliation{$^3$Departament de F{\'i}sica, Universitat de les Illes Balears, Ctra. de Valldemossa, km 7.5, Palma de Mallorca E-07122, Spain}


\begin{abstract}
Common models of circadian rhythms are constructed as compartmental reactions of well mixed biochemicals involving a negative-feedback loop containing several intermediate reaction steps in order to enable oscillations. Spatial transport of reactants is mimicked as an extra compartmental reaction step. In this letter, we show that a single activation-repression biochemical reaction pair is enough to produce sustained oscillations, if the sites of both reactions are spatially separated and molecular transport is mediated by diffusion. Our proposed scenario is the simplest possible one in terms of the participating chemical reactions and provides a conceptual basis for understanding biological oscillations and triggering  \textit{in-vitro} assays aimed at constructing minimal clocks.
\end{abstract}

\maketitle
%

Biological rhythms such as circadian, infradian and ultradian ones often originate in oscillations at a cellular level induced by complex mechanisms of gene auto-regulation composed by a number of coupled molecular reactions \cite{forger2017biological,gonze2002,novak2008,bharath2014}. 
An activation-repression process alone leads to a single-mode dynamical system with negative feedback unable by itself to support oscillatory behavior. 
Therefore, cellular clocks must resort to some kind of delay mechanism that effectively complements the negative feedback naturally implicit in the repression component to achieve limit-cycle dynamics. 
Most common models of such cellular rhythms are based on compartmental descriptions (individual substances reacting in a homogeneous well-mixed batch) of the biochemistry, including several intermediate molecular reactions: transcription, translation, phosphorylation, degradation, etc. 
Although the spatial inhomogeneities inside of the cell are recognized to play a role and the transport of molecules between different regions of the cell should be taken into account, this is usually done by mimicking such transport as an extra compartmental reaction \cite{goldbeter1995model,forger2017biological}. 
For example, mRNA migration from the nucleus is represented as "nuclear mRNA" reacting to produce "cytoplasmic mRNA" in the same hypothetical well mixed compartment.  In order to oscillate, these type of models need to include at least three reactions steps in the loop, but often a much larger number is required to avoid unrealistic parametrization of the reaction constants\cite{griffith68}. The purpose of this letter is to show that a much simpler reaction scheme is possible if the inhomogeneities of the reactants distributions and the transport are properly assessed. 
In particular, we will show that only one step expression-repression negative-feedback reaction is enough to produce sustained circadian oscillations, provided that the locations of transcription and translation are spatially separated and molecular transport is mediated by intra-cellular diffusion. 

In a well mixed reaction model, a single expression-repression reaction step is commonly represented in an abstract manner by the following couple of ordinary differential equations\cite{goodwin65,griffith68}
%
\begin{eqnarray}
\label{one-step}
\dot{M}(t)   & =& \omega_M \frac{1}{1+P(t)^h} -\gamma_M \, M(t)  \\
\dot{P}(t)  & =& \omega_P M(t) - \gamma_P \, P(t),   \nonumber
\end{eqnarray}
%
where $M(t)$ and $P(t)$ represent respectively the time evolution of the concentrations of the mRNA produced by a given gene and that of a protein --synthesized by the same mRNA-- that inhibits the production of further mRNA. 
The inhibitory ability of $P$ is governed by the Hill type function in the first term in the right hand side of the first equation while the second term represents the spontaneous degradation process of mRNA molecules at a rate $\alpha_M$. 
In turn, the rate of production of $P$, is proportional to the concentration $M$ and decays with a coefficient $\alpha_P$. 
The so called \textit{cooperativity} $h$ in the Hill function is a parameter that  accounts for the number of molecules necessary to inhibit the expression of the corresponding gene. 
Notice that the divergence of the vector field defining this dynamics, is constant and negative for any values of the pair $(M,P)$. 
This  means that the dynamical system contracts areas of any element of the space state and therefore it cannot support limit cycles or self-sustained oscillations \cite{griffith68}. 
Instead, it can be proved that Eqs.(\ref{one-step}) only have stable fixed points as stationary solutions. 

In the realm of compartmental models, hence, in order to produce a cellular rhythm with a biochemical circuit of expression-repression feedback such a circuit must include at least an extra intermediate reaction to introduce a delay capable of destabilizing one of these resting states. 
One such example introduced by Goodwin (\cite{goodwin65} reads as:
%
\begin{eqnarray}
\label{goodwin_three_steps}
\dot{M} &=&\omega_M \frac{1}{1+R^{h}}-\gamma_M M   \nonumber\\ 
\dot{P} &=&\omega_P M-\gamma_P P \label{int_step}\\ 
\dot{R} &=&\omega_R P-\gamma_R R    \nonumber
\end{eqnarray}  
%
where $P$ is an extra metabolite that mediates the production of the blocking protein $R$. 
This model does produce oscillations provided that $h>8$, a value quite far from a realistic interpretation so that more sophisticated models resort to include a few extra steps of phosphorylation and de-phosphorylation before the actual repression takes place\cite{gonze2002}. An extension of Eqs. (2) to $n$ reaction steps is straightforward (See Supplemental Material). 
In Fig.\ref{figure_intro} a scheme for this extension is shown, along with a plot of the minimal $h$ that produce oscillations in a compartmental model of $n$ components (values that satisfy the secant condition, first derived in \cite{tyson1978dynamics}). 
\begin{figure}[b] 
    \includegraphics[width=3.3in]{goodwin_scheme.pdf}
    \caption{(a) Scheme of the model proposed by Goodwin. A biochemical negative feedback loop where each component is produced by a first order reaction from the previous component, with exception of the first one, whose zero-order reaction is repressed by the last component. (b) Secant condition. Minimal cooperativity that leads to oscillations for each length $n$ of the biochemical chain.}
    \label{figure_intro}
\end{figure}
The real world, however, is neither compartmental nor well-mixed. The cells are of small but finite size and inside them a great deal of inhomogeneity is present --localization of species occur within nanodomains \cite{fonkeu2019mrna,hafner2019localized,sun2021prevalence,wu2016translation,jackson2020camp}. 
While mRNA translation occurs in some place in the cytoplasm relatively distant to from the nucleus, it is in this nucleus where the gene generates the mRNA and where the resulting protein may exert its repressive influence. 
Therefore, one trip of a new born mRNA molecule to the cytoplasm sites populated by ribosomes and an another trip back to the nucleus for the resulting proteins are needed for the cycle of expression and repression to be completed. 
In the compartmental approach, these trips are introduced in the form of fictitious intermediate reaction steps: mRNA$_{Nucleus}$ $ \rightarrow$ mRNA$_{Cytoplasm}$, P$_{Cytoplasm}$ $ \rightarrow$ P$_{Nucleus}$, etc. each one providing an equation of a similar structure as Eq.(\ref{int_step}). 
A related and less realistic approach consists in introducing a prescribed delay in the equations\cite{novak2008,bordyugov2013}, with the purpose of mimicking molecular transport times as well as extra steps. 

In our work, in contrast, we investigate the impact of including a realistic model for the transport across the inhomogeneties of the cell based on the process of molecular diffusion and the result is surprising: a loop based on Eqs.(\ref{one-step}) does now oscillate. Not only does it, but it also requires a cooperativity $h$ much smaller than the simplest compartmental model capable or generating cellular rhythms.

In fact, the investigation of the role of heterogeneity in the emergence of oscillations in reaction diffusion systems whose homogeneous counterpart shows only intrinsic homeostatic equilibrium has a long history. For instance the case of pancreatic $\beta$-cells has been studied in \cite{Julyan_hetero}. More recently several cases of gene regulatory networks have been studied under the same light \cite{krause2018heterogeneity,naqib2012tunable,chaplain2015hopf,macnamara2016diffusion,macnamara2017spatio} 

In our model, as in real cells, the transcription and translation processes are located in spatially different regions. The gene (GEN) produces mRNA at the nucleus, well separated from the ribosomes (RIB), at which protein synthesis takes place. Now, both  mRNA and P migrate through diffusion, so that now the corresponding concentrations $m(\vec{r},t)$ and  $p(\vec{r},t)$  depend on space and time. Thus, Eq. (1) must be rewritten in order to account for diffusion, yielding 
%
\begin{eqnarray}
\label{activator-repressor-diffusion}
\dot{m}(\vec{r},t)   & = & \frac{\omega_m}{1+p(\vec{r},t)^h} \, f_{GEN}(\vec{r}) -\gamma_m \, m(\vec{r},t)  \nonumber \\
&  & + D_m \, \nabla^2 m(\vec{r},t)    \\
\dot{p}(\vec{r},t) & =& \omega_p \; f_{RIB}(\vec{r})  \, m(\vec{r},t) - \gamma_p \, p(\vec{r},t)  \nonumber \\
&  & + D_p \nabla^2 \, p(\vec{r},t) ,   \nonumber 
\end{eqnarray}
%
where $f_{GEN}(\vec{r})$ and $f_{RIB}(\vec{r})$ refer to the spatial distributions of the gene and the ribosomes, respectively.  $D_m$ and $D_p$ are the effective diffusion coefficients experienced by the mRNA and P molecules. The parameters $\omega_m$ and $\omega_P$ refer to the rates of production of mRNA and P. Eqs. (\ref{activator-repressor-diffusion}) reduce to the Goodwin model for two components if one neglects the dependence of the concentrations on the spatial coordinates (well-mixed case).

\begin{figure} 
    \includegraphics[width=3.3in]{lim_cyc_xp.pdf}
    \caption{A modified version of the Goodwin model that accounts for diffusion of the components over a heterogeneous spatial domain, show oscillations that are dependent on the distance between the reaction zones. (a) Scheme of the model. (b) Trajectories after initial transient of the quantities $M(t)$ --as \emph{mRNA}-- and $P(t)$ --as \emph{protein}-- (s. Eq. \ref{integral_values}) for different distances and cooperativities. The higher the cooperativity $h$, the wider the range of distances where oscillations occur. The case $h=3$ shows oscillations for a narrow range of distances.  $ D = 0.1$ , $ \gamma = 0.1 $, $ R_{m} = R_{p} = 0.5$, $ R_{cell} = 10$ , $\omega_m = 10 $ , $\omega_p = 20 $   }
    \label{1d-diffusion}
\end{figure}
%

Of course, Eqs. \ref{activator-repressor-diffusion} can be generalized to include also intermediate reactions, but for the moment let us concentrate in the simpler case equivalent to the non oscillatory $n=2$ case of the Goodwin family \cite{goodwin65}. Let us also consider the dynamical behavior of the spatially one-dimensional version of  Eqs.(\ref{activator-repressor-diffusion}) obtained after the replacements $\vec{r}\rightarrow x$ and $\nabla^2\rightarrow \partial_{xx}$. We choose also the corresponding distributions $f_{GEN}(x)$ and $f_{RIB}(x)$ to be boxcar functions centered around the positions $x_m = 0$ and $x_p \neq 0$, with corresponding extents $2R_{m}$ and $2R_{p}$, respectively (see model details in the Supplemental Material).


For the sake of comparison with the compartmental models we compute the values of the total amount of $m$ and $p$ within the cell to be, a quantity that should be somehow proportional to the homonyms in the Goodwin's family: 
\begin{eqnarray}
\label{integral_values}
	M(t) & = & \int_\Omega m(x,t) dx \nonumber \\
	P(t) & = & \int_\Omega p(x,t) dx, 
\end{eqnarray}
where $\Omega$ is the cell volume. 


In Fig.\ref{1d-diffusion} we plot the final fate of the evolution of $M(t)$ and $P(t)$ as $t\rightarrow\infty$ displayed for different values of $x_p$ and $h$. Remarkably, for intermediate values of distances between the sources, the system display sustained oscillations. However, for lower and higher distances, the oscillations disappear and the system converges to a steady state. Notice that the higher the cooperativity, the shorter the necessary distance between nucleus and ribosomes to produce oscillations, but even with values as small as $h=3$ a single two-component expression-repression reaction generates limit cycles in a relatively wide range of nucleus-ribosome distances. 
The outcome is already interesting because it underlines the fact that purely diffusive transport is enough to allow for oscillatory behavior in a negatively feedback control circuit of only two species. 
This fact is reminiscent of the instabilities found in reaction-diffusion spatial patterns, where homogeneous profiles are turned unstable under certain values of diffusion coefficients \cite{turing1952,maini2012turing}. 
But from the point of view of the application to cellular rhythms, the result unveils another surprise: the relatively small cooperativity parameter necessary to achieve oscillations. 


In Fig.\ref{limit_cicle}a the temporal evolution of $M(t)$ and $P(t)$ corresponding to two different initial configurations of $m(x,t)$ and $p(x,t)$ is shown. 
The existence of a limit cycle is evident, a fact that confirms, on the one hand, the existence of regular stable oscillations in the system and on the other hand, a certain type of correspondence with a compartmental model of a number of variables at least larger than 3. 
The initial configurations in the considered cases are chosen to be $m(x,0) = m_0 $ and $ p(x,0) = p_0$ such that one configuration falls inside and the other outside the limit cycle. 
Fig.\ref{limit_cicle}b shows, for the same set of parameters corresponding to Fig.\ref{limit_cicle}a, the onset of oscillations as the cooperativity is increased from very low values, where no limit cycles are observed, to higher values, passing through a bifurcation as in the compartmental model. 
Recalling Fig.\ref{figure_intro}, in the compartmental model we need to take into account six components ($n=6$) to support oscillations with cooperativity $h=3$.


\begin{figure}[b]
    \includegraphics[width=3.3in]{lim_cyc_h_bif.pdf}
    \caption{ (a) Parametric plot in terms of the total amounts of mRNA and protein ($M$ and $P$) for a single one dimensional transcription-translation feedback loop with two spatially separated reactions and diffusion, showing two example trajectories (red, green)converging to the limit cycle (black) for $h=4$. (b) Plot of minima and maxima (lines) as function of $h$, with shaded regions indicating oscillations. The onset of oscillations occur below $h=3$, a value achieved only in longer chains in the well-mixed case (Fig. \ref{figure_intro}). Parameters as in Fig. \ref{1d-diffusion}.} 
    \label{limit_cicle}
\end{figure}

The onset of the oscillations is aided by the delays induced by the diffusive transport of the species. We derived an estimation of the time-scale associated with the diffusion process, by solving analytically a simplified problem, in which a punctual source of a single species is introduced as a perturbation at finite time in an infinite domain (see Supplemental Material). We termed it as relaxation time $t_R = (1/4\gamma) (\sqrt{1+4(x/\lambda)^2} -1)$,  which converges asymptotically to a linear dependence on the distance (related derivations can be found in \cite{berezhkovskii2011formation} and  \cite{ellery2012critical}, which named their results as local accumulation time and critical time, respectively). It is simple to verify this linear dependence in the spatio-temporal distribution of $m(x,t)$ and $p(x,t)$(see Supplemental Material).


%
In the following we show that the central idea of our model -- the fact that self-sustained oscillations in cells can arise from the spatial separation between transcription and translation sites and without resorting to complex loops of intermediate reactions -- is not an artifact of the partial differential equations (PDE) representation of this scenario. An alternative simulation of this process can be achieved by modeling the individual trajectories of the reacting molecules by an stochastic Langevin equation\cite{erban2020}. This approach should include also a proper description  of the probability of reaction which we implement through an agent based algorithm. The deterministic results of the PDE should then be compared with the long term ensemble average of the stochastic model. 
%
%
We confirmed the existence of self-sustained oscillations by running simulations of the stochastic counterpart of the deterministic representation. In order to perform the simulations, we coupled reaction and diffusion of individual molecules through a spatial domain, by means of the Gillespie algorithm and the Langevin equation. Molecules are subjected to the set of reactions 
%

\begin{eqnarray}
\label{gillespie-react-diff}
mRNA     & \rightarrow   & \phi        \nonumber  \\
protein  & \rightarrow   & \phi         \nonumber \\
\phi     & \rightarrow  & mRNA           \\
\phi     & \rightarrow    & protein     \nonumber \\
x_i(t+\Delta t) & =&  x_i(t ) + \sqrt{2D \Delta t} \xi , \nonumber 
\end{eqnarray}
%
whereby $\phi$ denotes chemical species that are of no interest in order to represent degradation and creation of $mRNA$ and $protein$ molecules, $ x_i(t )$ is the position of the molecule $i$ at time $t$, $ \xi $ is a Gaussian white noise \cite{erban2020}. The reaction rates in each reaction are such that the system is equivalent to Eq.\ref{activator-repressor-diffusion} (see details in Supplemental Material). D is the diffusion coefficient.
In Fig.\ref{panel} we show the orbits resulting from both the deterministic and the stochastic approaches. Both stochastic and deterministic numerical results clearly show oscillations only for intermediate values of separation between sources, as it is evident from time series in both representations. In the stochastic case, the periodicity is sharply revealed by the peaks in the autocorrelation function of individual time series. Notice that the first peak is shifted to higher periods as the separation becomes greater. As in the case of the deterministic approach, this shift can be explained by the increased delay in the transport of molecules. 

\begin{figure*} 
    \includegraphics[width=6.6in]{sto_det_ph_diag.pdf}
    \caption{Stochastic (agent-based) and deterministic (partial differential equations) models show onset of oscillations of the number of $mRNA$ and $protein$ molecules in approximately the same range of separations. (a) Super-imposed trajectories from the deterministic and stochastic models. Limit cycles from the deterministic model and oscillatory noisy trajectories from the stochastic agent-based model appear in intermediate ranges of separation. (b) Time series of $mRNA$ and $ protein $ in a segment of the same trajectories. (c) The stochastic model displays noisy uncorrelated fluctuations for small and large distances between sources, as reflected in the autocorrelation function. For intermediate distances, the autocorrelation function clearly indicates oscillations, reflected in the peaks appearing at periodic lags . $h= 10 $, $ D = 0.1$ , $ \gamma = 0.1 $, $ R_{m} = R_{p} = 0.5$, $ R_{cell} = 10$ , $\omega_m = \omega_p = 10 $. (d) Example biochemical loops of three species show enhanced oscillations when sources of molecular species are separated. (upper panel) A fictitious feedback loop of three species constructed with a set of parameters compatible with experiments (s. Refs \cite{fonkeu2019mrna,sturrock2011spatio}) that would not oscillate in the well-mixed case, display oscillations if the distance between sources is appropriate. (lower panel) A feedback loop constructed with a set of parameters that yields oscillations in the compartmental models (well-mixed case) (s. Ref. \cite{gonze2013goodwin}), looses and regains oscillatory behavior when the distance between sources is increased, yielding a distinct diffusion-dominated oscillation parameter region. For low $h$, oscillations are only possible in the diffusion-dominated region .}
    \label{panel}
\end{figure*}

%
Finally, we consider a loop that includes three molecular species, as in Eqs. \ref{goodwin_three_steps}. We choose two sets of parameters. In the first set, both kinetic and transport parameters are compatible with experimentally obtained measurements on biological systems \cite{fonkeu2019mrna,sturrock2011spatio}. In the second set, kinetic parameters are taken from published models that display oscillations in the well-mixed case \cite{gonze2013goodwin}(see complete parameter list in Supplemental Material). 
The production site for the third species is chosen to be identical to that for the second species, i.e. $x_r=x_p$, and the distance is defined to the production site of the first species. We identified the combination of $h$ and $x_p$ that yield oscillatory solutions. 
The results for the first set (Fig.\ref{panel}d--upper panel) demonstrate that a simple negative feedback loop in a biological cell can transition from homeostatic to oscillatory behavior for distances compatible with the dimensions of a cell, provided the cooperativity is at least 5. 
The results for the second set (Fig.\ref{panel}d--lower panel) reveal that a system that is tuned to display oscillations if there is no separation, can loose this feature if the distance between sources is increased, and recover it again if this separation is further increased. Furthermore, the oscillations appear in an extended region of parameters, where far smaller values of $h$ are required and amplitudes can be higher (see Supplemental Material). 
The disconnection between the two oscillatory regions of parameters, as well as their shape, implies that diffusion not only arises as a distinct mechanism to enable oscillations, but also improves their robustness.



Summarizing, we showed that a simple biochemical feedback loop, that in well-mixed conditions would be doomed to display homeostatic behavior, can be turned into an oscillator by physically separating the domains where reactions occur. This study can reconcile the parameter choice in mathematical modeling with the biologically plausible values. It can provide the modeling framework to recent experimental observations that suggest a crucial role of the localization of synthesis within cells \cite{fonkeu2019mrna,hafner2019localized,sun2021prevalence,wu2016translation,weitz2014diversity}.
%


PR is grateful to Claudia R. Arbeitman for useful discussions. M.E.G acknowledges support from the DFG through the Grant RTG 2749/1 "Multiscale Clocks".  O.P. acknowledges COST Action CA21169. Part of the present research was carried out within the framework of the activities of the Spanish Government through the "Maria de Maeztu Centre of Excellence accreditation to IMEDEA (CSIC-UIB) (CEX2021-001198). MEG and PR acknowledge support from the  University of Kassel (ZFF-PROJEKT 2377) and PR from the Joachim Herz Stiftung. 
\bibliography{diff_clocks}

\end{document}











