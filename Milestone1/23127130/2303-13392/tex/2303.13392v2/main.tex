%  LaTeX support: latex@mdpi.com 
%  For support, please attach all files needed for compiling as well as the log file, and specify your operating system, LaTeX version, and LaTeX editor.

%=================================================================
\documentclass[preprints,article,accept,pdftex,moreauthors]{Definitions/mdpi} 

%--------------------
% Class Options:
%--------------------
%----------
% journal
%----------
% Choose between the following MDPI journals:
% acoustics, actuators, addictions, admsci, adolescents, aerobiology, aerospace, agriculture, agriengineering, agrochemicals, agronomy, ai, air, algorithms, allergies, alloys, analytica, analytics, anatomia, animals, antibiotics, antibodies, antioxidants, applbiosci, appliedchem, appliedmath, applmech, applmicrobiol, applnano, applsci, aquacj, architecture, arm, arthropoda, arts, asc, asi, astronomy, atmosphere, atoms, audiolres, automation, axioms, bacteria, batteries, bdcc, behavsci, beverages, biochem, bioengineering, biologics, biology, biomass, biomechanics, biomed, biomedicines, biomedinformatics, biomimetics, biomolecules, biophysica, biosensors, biotech, birds, bloods, blsf, brainsci, breath, buildings, businesses, cancers, carbon, cardiogenetics, catalysts, cells, ceramics, challenges, chemengineering, chemistry, chemosensors, chemproc, children, chips, cimb, civileng, cleantechnol, climate, clinpract, clockssleep, cmd, coasts, coatings, colloids, colorants, commodities, compounds, computation, computers, condensedmatter, conservation, constrmater, cosmetics, covid, crops, cryptography, crystals, csmf, ctn, curroncol, cyber, dairy, data, ddc, dentistry, dermato, dermatopathology, designs, devices, diabetology, diagnostics, dietetics, digital, disabilities, diseases, diversity, dna, drones, dynamics, earth, ebj, ecologies, econometrics, economies, education, ejihpe, electricity, electrochem, electronicmat, electronics, encyclopedia, endocrines, energies, eng, engproc, entomology, entropy, environments, environsciproc, epidemiologia, epigenomes, est, fermentation, fibers, fintech, fire, fishes, fluids, foods, forecasting, forensicsci, forests, foundations, fractalfract, fuels, future, futureinternet, futurepharmacol, futurephys, futuretransp, galaxies, games, gases, gastroent, gastrointestdisord, gels, genealogy, genes, geographies, geohazards, geomatics, geosciences, geotechnics, geriatrics, grasses, gucdd, hazardousmatters, healthcare, hearts, hemato, hematolrep, heritage, higheredu, highthroughput, histories, horticulturae, hospitals, humanities, humans, hydrobiology, hydrogen, hydrology, hygiene, idr, ijerph, ijfs, ijgi, ijms, ijns, ijpb, ijtm, ijtpp, ime, immuno, informatics, information, infrastructures, inorganics, insects, instruments, inventions, iot, j, jal, jcdd, jcm, jcp, jcs, jcto, jdb, jeta, jfb, jfmk, jimaging, jintelligence, jlpea, jmmp, jmp, jmse, jne, jnt, jof, joitmc, jor, journalmedia, jox, jpm, jrfm, jsan, jtaer, jvd, jzbg, kidneydial, kinasesphosphatases, knowledge, land, languages, laws, life, liquids, literature, livers, logics, logistics, lubricants, lymphatics, machines, macromol, magnetism, magnetochemistry, make, marinedrugs, materials, materproc, mathematics, mca, measurements, medicina, medicines, medsci, membranes, merits, metabolites, metals, meteorology, methane, metrology, micro, microarrays, microbiolres, micromachines, microorganisms, microplastics, minerals, mining, modelling, molbank, molecules, mps, msf, mti, muscles, nanoenergyadv, nanomanufacturing,\gdef\@continuouspages{yes}} nanomaterials, ncrna, ndt, network, neuroglia, neurolint, neurosci, nitrogen, notspecified, %%nri, nursrep, nutraceuticals, nutrients, obesities, oceans, ohbm, onco, %oncopathology, optics, oral, organics, organoids, osteology, oxygen, parasites, parasitologia, particles, pathogens, pathophysiology, pediatrrep, pharmaceuticals, pharmaceutics, pharmacoepidemiology,\gdef\@ISSN{2813-0618}\gdef\@continuous pharmacy, philosophies, photochem, photonics, phycology, physchem, physics, physiologia, plants, plasma, platforms, pollutants, polymers, polysaccharides, poultry, powders, preprints, proceedings, processes, prosthesis, proteomes, psf, psych, psychiatryint, psychoactives, publications, quantumrep, quaternary, qubs, radiation, reactions, receptors, recycling, regeneration, religions, remotesensing, reports, reprodmed, resources, rheumato, risks, robotics, ruminants, safety, sci, scipharm, sclerosis, seeds, sensors, separations, sexes, signals, sinusitis, skins, smartcities, sna, societies, socsci, software, soilsystems, solar, solids, spectroscj, sports, standards, stats, std, stresses, surfaces, surgeries, suschem, sustainability, symmetry, synbio, systems, targets, taxonomy, technologies, telecom, test, textiles, thalassrep, thermo, tomography, tourismhosp, toxics, toxins, transplantology, transportation, traumacare, traumas, tropicalmed, universe, urbansci, uro, vaccines, vehicles, venereology, vetsci, vibration, virtualworlds, viruses, vision, waste, water, wem, wevj, wind, women, world, youth, zoonoticdis 
% For posting an early version of this manuscript as a preprint, you may use "preprints" as the journal. Changing "submit" to "accept" before posting will remove line numbers.

%---------
% article
%---------
% The default type of manuscript is "article", but can be replaced by: 
% abstract, addendum, article, book, bookreview, briefreport, casereport, comment, commentary, communication, conferenceproceedings, correction, conferencereport, entry, expressionofconcern, extendedabstract, datadescriptor, editorial, essay, erratum, hypothesis, interestingimage, obituary, opinion, projectreport, reply, retraction, review, perspective, protocol, shortnote, studyprotocol, systematicreview, supfile, technicalnote, viewpoint, guidelines, registeredreport, tutorial
% supfile = supplementary materials

%----------
% submit
%----------
% The class option "submit" will be changed to "accept" by the Editorial Office when the paper is accepted. This will only make changes to the frontpage (e.g., the logo of the journal will get visible), the headings, and the copyright information. Also, line numbering will be removed. Journal info and pagination for accepted papers will also be assigned by the Editorial Office.

%------------------
% moreauthors
%------------------
% If there is only one author the class option oneauthor should be used. Otherwise use the class option moreauthors.

%---------
% pdftex
%---------
% The option pdftex is for use with pdfLaTeX. Remove "pdftex" for (1) compiling with LaTeX & dvi2pdf (if eps figures are used) or for (2) compiling with XeLaTeX.

%=================================================================
% MDPI internal commands - do not modify
\firstpage{1} 
\makeatletter 
\setcounter{page}{\@firstpage} 
\makeatother
\pubvolume{1}
\issuenum{1}
\articlenumber{0}
\pubyear{2023}
\copyrightyear{2023}
%\externaleditor{Academic Editor: Firstname Lastname}
\datereceived{ } 
\daterevised{ } % Comment out if no revised date
\dateaccepted{ } 
\datepublished{ } 
%\datecorrected{} % For corrected papers: "Corrected: XXX" date in the original paper.
%\dateretracted{} % For corrected papers: "Retracted: XXX" date in the original paper.
\hreflink{https://doi.org/} % If needed use \linebreak
%\doinum{}
%\pdfoutput=1 % Uncommented for upload to arXiv.org

%=================================================================
% Add packages and commands here. The following packages are loaded in our class file: fontenc, inputenc, calc, indentfirst, fancyhdr, graphicx, epstopdf, lastpage, ifthen, float, amsmath, amssymb, lineno, setspace, enumitem, mathpazo, booktabs, titlesec, etoolbox, tabto, xcolor, colortbl, soul, multirow, microtype, tikz, totcount, changepage, attrib, upgreek, array, tabularx, pbox, ragged2e, tocloft, marginnote, marginfix, enotez, amsthm, natbib, hyperref, cleveref, scrextend, url, geometry, newfloat, caption, draftwatermark, seqsplit
% cleveref: load \crefname definitions after \begin{document}
\usepackage[utf8]{inputenc}

\usepackage[normalem]{ulem}

\usepackage{notoccite}

\newcommand{\pic}[2]{\begin{center} \includegraphics[scale=#1]{#2}\end{center}}
\newcommand{\re}[1]{\mathrm{Re}\left(#1\right)}
\newcommand{\im}[1]{\mathrm{Im}\left(#1\right)}
\newcommand{\bdot}[1]{\dot{ \bb {#1}}}
\newcommand{\bddot}[1]{\ddot{ \bb {#1}}}
\newcommand{\bidot}[1]{\dot{ \bi{ #1}}}
\newcommand{\biddot}[1]{\ddot{ \bi {#1}}}
\newcommand{\ep}{\varepsilon}
\newcommand{\for}{\quad \quad \mathrm{for} \quad\quad}
\newcommand{\then}{\quad \quad \implies \quad\quad}
\newcommand{\an}{\quad \quad \mathrm{and} \quad\quad}
\newcommand{\ifff}{\quad \quad \mathrm{if} \quad\quad}
\newcommand{\where}{\quad \quad \mathrm{where} \quad\quad}
\newcommand{\tms}{\mathrm{x} }
\newcommand{\dg}{^\dagger}
\newcommand{\semi}{\quad \quad \mathrm{;} \quad\quad}
\newcommand{\paren}[1]{\left( #1 \right)}
\newcommand{\brac}[1]{\left[ #1 \right]}
\newcommand{\bra}[1]{\left\langle #1 \right|}
\newcommand{\exv}[1]{\left\langle #1 \right\rangle}
\newcommand{\pwisein}{\left\{ \begin{array}{ll}}
\newcommand{\pwiseout}{\end{array}\right.}
\newcommand{\ket}[1]{\left| #1 \right\rangle}
\newcommand{\bracket}[2]{\left\langle #1 | #2 \right\rangle}
\newcommand{\trace}[1]{\mathrm{Tr} \left( #1 \right)}
\renewcommand{\det}[1]{\mathrm{det}\left( #1 \right)}
\newcommand{\del}[1]{\frac{\partial}{\partial #1}}
\newcommand{\fulld}[1]{\frac{d}{d #1}}
\newcommand{\fulldd}[2]{\frac{d #1}{d #2}}
\newcommand{\dell}[2]{\frac{\partial #1}{\partial #2}}
\newcommand{\delltwo}[2]{\frac{\partial^2 #1}{\partial #2 ^2}}
\newcommand{\bb}{\mathbf}
\newcommand{\bi}{\boldsymbol}
\newcommand{\eq}[1]{\begin{equation} #1 \end{equation}}
\newcommand{\radhalf}{ \frac{ \sqrt{2}}{2}}
\newcommand{\sigx}{\left( \begin{array}{cc} 0 & 1\\ 1 & 0 \end{array}\right)}
\newcommand{\sigy}{\left( \begin{array}{cc} 0 & -i\\ i & 0 \end{array}\right)}
\newcommand{\sigz}{\left( \begin{array}{cc} 1 & 0\\ 0 & -1 \end{array}\right)}
\renewcommand{\matrix}[1]{\left( \begin{array} #1 \end{array}\right)}
\newcommand{\thermo}[3]{\left( \frac{\partial #1}{\partial #2} \right)_{#3}}
\newcommand{\coolfrac}[2]{\left( \frac{ #1}{ #2} \right)}
\newcommand{\highlight}[1]{\colorbox{yellow}{$\displaystyle #1$}}

%=================================================================
% Please use the following mathematics environments: Theorem, Lemma, Corollary, Proposition, Characterization, Property, Problem, Example, ExamplesandDefinitions, Hypothesis, Remark, Definition, Notation, Assumption
%% For proofs, please use the proof environment (the amsthm package is loaded by the MDPI class).

%=================================================================
% Full title of the paper (Capitalized)
%\Title{Water-methanol mixture confined in a graphene slit-pore}

\Title{Size-pore-dependent methanol sequestration from water-methanol mixtures by an embedded graphene slit}

% MDPI internal command: Title for citation in the left column
\TitleCitation{Size-pore-dependent methanol sequestration from water-methanol mixtures by an embedded graphene slit}

% Author Orchid ID: enter ID or remove command
\newcommand{\orcidauthorA}{0009-0000-5328-9117} % Add \orcidA{} behind the author's name
\newcommand{\orcidauthorB}{0000-0002-1595-7804} % Add \orcidB{} behind the author's name
\newcommand{\orcidauthorC}{0000-0002-1977-1724} % Add \orcidC{} behind the author's name
\newcommand{\orcidauthorD}{0000-0003-3006-2766} % Add \orcidD{} behind the author's name

% Authors, for the paper (add full first names)
\Author{Roger Bellido Peralta$^{1,\dagger}$\orcidA{}, 
Fabio Leoni$^{2}$\orcidB{},  
Carles Calero$^{1,3}$\orcidC{},
and
Giancarlo Franzese$^{1,3}\orcidD{}$ 
}

%\longauthorlist{yes}

% MDPI internal command: Authors, for metadata in PDF
\AuthorNames{Roger Bellido Peralta, Fabio Leoni and Carles Calero, Giancarlo Franzese}

% MDPI internal command: Authors, for citation in the left column
\AuthorCitation{Bellido Peralta, R.; Leoni, F; Calero, C.; Franzese, G.}
% If this is a Chicago style journal: Lastname, Firstname, Firstname Lastname, and Firstname Lastname.

% Affiliations / Addresses (Add [1] after \address if there is only one affiliation.)
\address{%
$^{1}$ \quad Secci\'o de F\'isica Estad\'istica i Interdisciplin\`aria - Departament de F\'isica de la Mat\`eria Condensada, Universitat de Barcelona, Mart\'{\i} i Franqu\`es 1, 08028 Barcelona, Spain\\
$^{2}$ \quad Department of Physics, Sapienza University of Rome, Italy; \\
$^{3}$ \quad Institut de Nanoci\`{e}ncia i Nanotecnologia, Universitat de Barcelona, 08028 Barcelona, Spain.
}

% Contact information of the corresponding author
\corres{Correspondence: gfranzese@ub.edu;
%Tel.: (optional; include country code; if there are multiple corresponding authors, add author initials) +xx-xxxx-xxx-xxxx (F.L.)
}

% Current address and/or shared authorship
\firstnote{Currently at National Graphene Institute, University of Manchester, UK; roger.bellido@postgrad.manchester.ac.uk} 
% \secondnote{These authors contributed equally to this work.}
% The commands \thirdnote{} till \eighthnote{} are available for further notes

%\simplesumm{} % Simple summary

%\conference{} % An extended version of a conference paper

% Abstract (Do not insert blank lines, i.e. \\) 
% \abstract{
% }

\abstract{
The separation of liquid mixture components is relevant in many applications--going from water purification to biofuel
production--and a growing concern related to the UN Sustainable Development Goals (SDGs), such as ``Clean water and Sanitation'' and ``Affordable and clean energy''.
One promising technique is using graphene slit-pores as filters, or sponges, because the confinement potentially affects the properties of the mixture components in different ways, favoring their separation. 
However, no systematic study shows how the size of a pore changes the thermodynamics of the surrounding mixture.
Here, we focus on water-methanol mixtures and explore, using Molecular Dynamics simulations, the effects of a graphene pore, with size ranging from 6.5 to 13 \AA,
for three compositions: pure water, 90\%-10\%, and 75\%-25\% water-methanol. 
We show that tuning the pore size can change the mixture pressure, density, and composition in bulk due to the size-dependent methanol sequestration within the pore.
Our results can help in optimizing the graphene pore size for filtering applications.  
}

% Keywords
\keyword{Molecular Dynamics, nanoconfinement, graphene, water, methanol, sequestration.} 

% The fields PACS, MSC, and JEL may be left empty or commented out if not applicable
%\PACS{J0101}
%\MSC{}
%\JEL{}

%%%%%%%%%%%%%%%%%%%%%%%%%%%%%%%%%%%%%%%%%%
% Only for the journal Diversity
%\LSID{\url{http://}}

%%%%%%%%%%%%%%%%%%%%%%%%%%%%%%%%%%%%%%%%%%
% Only for the journal Applied Sciences
%\featuredapplication{Authors are encouraged to provide a concise description of the specific application or a potential application of the work. This section is not mandatory.}
%%%%%%%%%%%%%%%%%%%%%%%%%%%%%%%%%%%%%%%%%%

%%%%%%%%%%%%%%%%%%%%%%%%%%%%%%%%%%%%%%%%%%
% Only for the journal Data
%\dataset{DOI number or link to the deposited data set if the data set is published separately. If the data set shall be published as a supplement to this paper, this field will be filled by the journal editors. In this case, please submit the data set as a supplement.}
%\datasetlicense{License under which the data set is made available (CC0, CC-BY, CC-BY-SA, CC-BY-NC, etc.)}

%%%%%%%%%%%%%%%%%%%%%%%%%%%%%%%%%%%%%%%%%%
% Only for the journal Toxins
%\keycontribution{The breakthroughs or highlights of the manuscript. Authors can write one or two sentences to describe the most important part of the paper.}

%%%%%%%%%%%%%%%%%%%%%%%%%%%%%%%%%%%%%%%%%%
% Only for the journal Encyclopedia
%\encyclopediadef{For entry manuscripts only: please provide a brief overview of the entry title instead of an abstract.}

%%%%%%%%%%%%%%%%%%%%%%%%%%%%%%%%%%%%%%%%%%
% Only for the journal Advances in Respiratory Medicine
%\addhighlights{yes}
%\renewcommand{\addhighlights}{%

%\noindent This is an obligatory section in “Advances in Respiratory Medicine”, whose goal is to increase the discoverability and readability of the article via search engines and other scholars. Highlights should not be a copy of the abstract, but a simple text allowing the reader to quickly and simplified find out what the article is about and what can be cited from it. Each of these parts should be devoted up to 2~bullet points.\vspace{3pt}\\
%\textbf{What are the main findings?}
% \begin{itemize}[labelsep=2.5mm,topsep=-3pt]
% \item First bullet.
% \item Second bullet.
% \end{itemize}\vspace{3pt}
%\textbf{What is the implication of the main finding?}
% \begin{itemize}[labelsep=2.5mm,topsep=-3pt]
% \item First bullet.
% \item Second bullet.
% \end{itemize}
%}

%%%%%%%%%%%%%%%%%%%%%%%%%%%%%%%%%%%%%%%%%%
\begin{document}

\section{Introduction}

According to the latest UN progress report on accomplishing the Sustainable Development Goals (SDGs), millions still need 'clean water and sanitation' (SDG 6) and 'affordable and clean energy' (SDG 7). As a result, meeting the targets by the 2030 deadline is currently unreachable. Therefore, there is an urgent need for more research and investment from governments and businesses to accelerate the implementation of these goals and ensure a sustainable future for everyone. In particular, SDG 6 and SDG 7 are particularly relevant for addressing some of the most pressing challenges of our time. SDG 7 seeks to ensure access to affordable, reliable, sustainable, and modern energy for all. This is essential for reducing greenhouse gas emissions \cite{Fetisov_2023}, such as the byproducts of the oil industry \cite{Fetisov2023}, improving health and well-being, enhancing economic productivity, and creating opportunities for innovation and social inclusion. SDG 6 aims to ensure the availability and sustainable management of water and sanitation for all. This is vital for preventing diseases, improving hygiene, reducing inequalities, protecting ecosystems, and supporting human dignity. Furthermore, they are directly related to SDG 3 (Good health and well-being), SDG 10 (Reduced inequalities), SDG 11 (Sustainable cities and communities), SDG 12 (Responsible consumption and production), SDG 13 (Climate action), SDG 14 (Life below water), SDG 15 (Life on land), and indirectly to all the others.

Finding new and efficient ways to separate water from methanol is relevant in this context. Indeed, water and methanol are often mixed in various industrial processes, such as biodiesel production, wastewater treatment, and solvent extraction. However, separating water and methanol is challenging and energy-intensive, as they form an azeotropic mixture that cannot be easily distilled. By developing more efficient and cost-effective separation methods, such as membrane technology, adsorption, or extraction, these processes' energy consumption and environmental impact can be reduced, contributing to the goal of affordable and clean energy. Furthermore, by recovering water and methanol from these mixtures, the quality and quantity of water resources can be improved, as well as the availability of methanol as a renewable fuel or chemical feedstock, contributing to the goal of clean water and sanitation.

Water and methanol are fully miscible liquids at ambient conditions due to their hydrogen bonds \cite{Soetens:2015aa}, and their mixtures are standard in food processing, preservation, pharmaceutical, and chemical industries. For example, water-methanol blends are used in power generation applications, including gas turbines, fuel cells, green alternative fuels, improved combustion engines, and solar plants \cite{Ren_2000, Boysen_2004, Liu_2007, Yanju_2008, Miganakallu:2020aa, Toledo-Camacho:2021aa}. Also, methanol is often added to water to lower its freezing point and improve its flow \cite{Miller:1964aa, Sun:2011aa}. Nevertheless, it is necessary to separate the two components in several applications. For example, separating water and methanol is essential in producing biofuels to ensure the quality and efficiency of the final product \cite{Cortright:2002aa, Masoumi:2021aa}, or in chemical manufacturing processes to maintain the desired concentration of the reactants and prevent unwanted side reactions \cite{Dalena:2018aa}. 

However, the separation of methanol from water is usually performed by inefficient and energetically-intensive distillation \cite{Liang_2014}. Therefore, to minimize energy waste and increase efficiency, researchers have investigated water purification via nanomembrane filtering using chemical functionalization \cite{Azamat:2019ua, Azamat:2021aa}, adsorption on graphite pores \cite{Prslja:2019aa}, or infinite graphite sheets \cite{Mosaddeghi:2019aa}. In general, nanomaterials and nanomembranes combined with advanced catalytic, photothermal, adsorption, and filtration processes provide fast, efficient, and tunable alternatives compared to conventional routes in water remediation. However, many challenges regarding scalability and sustainability are still open \cite{Esplandiu:2023aa}.

Recently, graphene-based membranes have been reported \cite{Nair2012} as a method that changes the dynamics of the confined fluids when compared to bulk. Furthermore, several works have been published exploring all types of materials, looking for the selectivity of water or methanol over the other component of the mixture \cite{Mahmood2012, Villegas2015, Hung2017, Kachhadiya2021}. Also, Molecular Dynamics (MD) simulations of atomistic models clarified the physical mechanisms of these changes \cite{calero2020} and the differences among various typologies of fluids \cite{Leoni:2021aa, LF2014, Leoni:2016aa}. Table \ref{tab:results} summarizes some of the recently obtained main experimental and theoretical results.

\begin{table}
\begin{tabular}{| m{2.2cm} | m{1cm} | m{1.8cm} | m{6cm} | m{0.5cm} |}
\hline
Material & Method & Selective to & Main Result & Ref. \\ \hline \hline
\begin{tabular}[c]{@{}l@{}}Porous BNNS \\ -H, -F \\ -OH \end{tabular}& \begin{tabular}[c]{@{}l@{}} Simu-\\ lations \end{tabular}& \begin{tabular}[c]{@{}l@{}} Methanol \\ Water \end{tabular}& Each molecule has higher free energy in correspondence to the pores it cannot permeate through. & \cite{Azamat:2019ua} \\ \hline
BNNT & \begin{tabular}[c]{@{}l@{}} Simu-\\ lations \end{tabular}& Alcohols & Alcohols can easily break their hydrogen bonds to enter and occupy the nanotubes,  having a strong interaction with them. & \cite{Azamat:2021aa} \\ \hline
Pristine graphene & \begin{tabular}[c]{@{}l@{}} Simu-\\ lations \end{tabular}& Methanol & Methanol gets preferentially absorbed into a graphene slit pore. When mixed with water, the two liquids couple and diffuse. & \cite{Prslja:2019aa} \\ \hline
Graphite plates & \begin{tabular}[c]{@{}l@{}} Simu-\\ lations \end{tabular}& Methanol & Preferential absorption of methanol on graphite sheets due to Van der Walls interactions between the methyl groups and the carbon.  & \cite{Mosaddeghi:2019aa} \\ \hline

GO & \begin{tabular}[c]{@{}l@{}} Experi-\\ ments \end{tabular}& Water & Low friction flow of a water monolayer through 2D channels between graphene sheets, while helium remains in feed. & \cite{Nair2012} \\ \hline
SA/PVA  & \begin{tabular}[c]{@{}l@{}} Experi-\\ ments \end{tabular}& Water        & $\uparrow$ T $\Rightarrow$ $\uparrow$ mobility of the polymer chain$\Rightarrow$ $\uparrow$ flux, little selectivity reduction. At 5\% PVA composition, the material has surface pores, and at 20\% has cracks. Optimum PVA composition at 10\%. &  \cite{Mahmood2012}    \\ \hline
PHB  & \begin{tabular}[c]{@{}l@{}} Experi-\\ ments \end{tabular}& Water        & Pure substance pervaporation shows good MeOH permeation. It has water selectivity in a mixture due to the hydrogen bond network. MeOH has reduced mobility when mixed with water. & \cite{Villegas2015}     \\ \hline
rGO/CS  & \begin{tabular}[c]{@{}l@{}} Experi-\\ ments \end{tabular}& Water        & The interlayer space due to the CS leads to molecularly sieve water, and the hydrophobicity of GO provides good flux. & \cite{Hung2017}     \\ \hline
\begin{tabular}[c]{@{}l@{}}ZIF-8/PVDF \\ ZIF-67/PVDF\end{tabular} & \begin{tabular}[c]{@{}l@{}} Experi-\\ ments \end{tabular}& Water        & ZIF-67/PVDF membrane enhances flux due to its hydrophilicity. However, $\uparrow$ water \% in the feed $\Rightarrow$ $\uparrow$ swelling $\Rightarrow$ $\downarrow$ selectivity as volume increases and MeOH molecules can also pass through. $\uparrow$ T $\Rightarrow$ $\uparrow$ polymer chain mobility $\Rightarrow$ $\uparrow$ flux, $\downarrow$ selectivity. & \cite{Kachhadiya2021}     \\ \hline

\end{tabular}
\caption{\textbf{Summary of experimental or simulation results with different membranes for water-methanol mixtures.} List of acronyms: SA (sodium alginate), PVA (polyvinyl alcohol), PHB (poly(3-hydroxybutyrate)), rGO (reduced Graphene Oxide), CS (chitosan), ZIF-n (Zeolitic Imidazolate Framework, where the number "n" is not related to the structure, just used as naming \cite{Park2006}), PVDF (polyvinylidene fluoride), BNNS (boron nitride nanosheets, functionalized with -H,-F groups and -OH group), BNNT (boron nitride nanotubes).
Symbols $\uparrow$, $\downarrow$, $\Rightarrow$ stands for {\it increasinging, decreasing, implying}, respectively.}
\label{tab:results}
\end{table}



In particular, water has peculiar properties that are anomalous compared to other fluids  \cite{life, fermi2012, Gallo:2021wx}, and its interaction with nanointerfaces dramatically modifies its structure \cite{MCF2017}, thermodynamics and dynamics \cite{calero2020}, leading to unusual transport properties for both water and solutes \cite{Corti:2021uy}.

On the other hand, methanol, the smallest alcohol, has an apolar methyl group (CH$_3$) and a polar hydroxyl group (OH). The polar moiety can form hydrogen bonds of strength and length similar to water which, together with the methanol's small size, allows it to fully integrate into the water's hydrogen bond network \cite{Hus:2014aa}. 

Previous works have shown that, under slit-pore confinement, water's thermal diffusion coefficient $D_{\|}$ parallel to the walls is non-monotonic when the pore width $\delta$ changes below 1.5 nm \cite{calero2020}. 
This property is a consequence of water's ability to form hydrogen bonds. However, recent atomistic simulations show that this behavior is also present in liquids without hydrogen bonds, including simple van der Waals liquids or not-network-forming anomalous liquids \cite{Leoni:2021aa}. Nevertheless, the study shows that the mechanism leading to the variation of $D_{\|}$ in confined water is unique and different from other liquids  \cite{Leoni:2021aa}. Therefore, using nano-confining graphene slit-pores to separate water from methanol based on physical processes is an appealing possibility.

However, no systematic study shows how the size of a graphene slit-pore changes the composition and thermodynamics of the surrounding mixture in which it is embedded as a solute and how to optimize it for filtering applications. 
This geometry reminds the recent application of nanoengineered graphene pores used as a {\it sponge} for overcoming the limitations of the existing water treatment systems \cite{Esplandiu:2023aa,
Baptista_Pires_2021}.

Here, we investigate the capacity of a graphene slit-pore to sequester methanol from a water-methanol mixture and its effects on the mixture. In particular, we study how different water-methanol mixtures' properties-such as density, pressure, and composition- are affected as the width $\delta$ of an embedded graphene slit-pore changes. 
To this end, we perform Molecular Dynamics (MD) simulations of water-methanol mixtures of different compositions with graphene slit-pores of different sizes as solutes. 
The water-methanol mixture is described using tested coarse-grained models (see Sec.~\ref{sec:Methodology}) based on Continuous Shouldered Well (CSW) and Lennard-Jones potentials. 

%%%%%%%%%%%%%%%%%%%%%%%%%%%%%%%%%%%%%%%%%%
\section{Results and Discussion}

\subsection{Number density}

\subsubsection{The pure CSW case.}

First, we check the behavior of the pure CSW in the subregion $V'$ outside the pore (Fig. \ref{figbulk}a) and find that the slit-pore width $\delta$ affects weakly its density (Fig.~\ref{figbulk}a, b). In particular, the CSW density in $V'$ has a minor decrease 
for increasing $\delta$, but remains close to the overall density of 0.036 $\textup{\r{A}}^{-3}$ within the error, which is the nominal number density in the entire simulation box, including the slit-pore (see Sec.~\ref{sec:Methodology}).
Therefore, for the considered range of $\delta$, the nano-pore does not adsorb much water-like liquid inside, consistent with its hydrophobic properties at the macroscopic scale.

Nevertheless, in Ref.~\cite{Leoni:2021aa} a similar confined liquid (CSW with $\Delta=30$) shows free energy and confined density extrema at 
$\sim7.5\ \textup{\r{A}}$,
$\sim9.5\ \textup{\r{A}}$,
$\sim10.5\ \textup{\r{A}}$,  
and 
$\sim12\  \textup{\r{A}}$ that seems to correlate with the small density variations we find here, although within the error bars. Hence, we explore, next, the behavior of the mixture in $V'$ to better investigate the pore-size effects.

\begin{figure}%[h!]
\centering
\includegraphics[width=1\columnwidth]{images/den_bulk.png}
\caption{{\bf  The slit-pore width affects the densities of the mixture outside the pore.}
For pure CSW (red) and CSW-methanol compositions 90\%-10\% (blue) and 75\%-25\% (green), the change in $\delta$ implies variations of density $\rho_{V'}$ of the mixture (panel a) and each component (CSW, panel b, and methanol, panel c) outside the pore. 
In each panel, horizontal dashed lines mark the values of $\rho_{V'}$ that would be expected without the embedded slit-pore for the mixtures with different compositions as indicated in the legend in panel c, for the mixture (a), the CSW (b), the methanol (c).
}
\label{figbulk}
\end{figure}

\subsubsection{The mixture case.}

At some fixed pore sizes, e.g., $\delta=9$ \AA, we find an overall decrease for the mixture density $\rho_{V'}$ when we increase the methanol concentration from 0\% to 10\% and 25\% (Fig.~\ref{figbulk} a). 
%
This behavior is consistent with what is expected for bulk. \cite{Sentenac:1998aa, perry1999perry}. At other values of $\delta$, e.g., $\delta=7$ \AA, the trend is nonmonotonic in methanol concentration, suggesting an unexpected behavior.

\begin{figure}%[h]
\centering
\includegraphics[width=1\columnwidth]{images/per_met.png}
\caption{{\bf  The sequestration of methanol inside the graphene slit-pore affects its concentration outside.}
The methanol concentrations inside (left panel) and outside the pore (right panel) are anti-correlated for 
both mixture compositions 90\%-10\% (blue) and 75\%-25\% (green) CSW-methanol. 
Inside the pore, the concentration can increase by $\simeq$ 320\%, in the first case, compared to its nominal bulk value and 250\%, in the second case, while outside can be $\simeq$ 75\% or 84\% lower, respectively.
}
\label{figmet}
\end{figure}

Indeed, surprisingly, we observe that changes in the slit pore width $\delta$ affect the mixture $\rho_{V'}$ outside the error bars. Moreover, the effect in the mixture is more evident than in pure CSW liquid. Therefore, the pore-size dependence of the density in $V'$ depends mainly on the methanol-pore interaction.

To better understand this dependence, we calculate the concentration of each component in $V'$ separately as a function of $\delta$.
The overall CSW number density in the subvolume $V'$, $\rho_{CSW}$, should decrease from 0.036 $\textup{\r{A}}^{-3}$ at 0\% methanol concentration, to 0.0324 $\textup{\r{A}}^{-3}$ at 10\% methanol, to 0.027 $\textup{\r{A}}^{-3}$ at 25\% methanol.
However, we find that at the highest methanol concentration, $\rho_{CSW}$ is $\approx 4\%$ above the expected value (Fig.~\ref{figbulk} b). 
%
Therefore, there is less methanol in the solution than expected due to the presence of the graphene pore.

Furthermore, $\rho_{CSW}$, at nominal 25\% methanol concentration, tends to increase for larger pore sizes $\delta$ (Fig.~\ref{figbulk} b). Hence, the amount of methanol in the mixture in $V'$ decreases for greater $\delta$, and the strength of the effect is proportional to the nominal methanol concentration.

This conclusion is confirmed when we explicitly calculate the methanol number density in $V'$, $\rho_{meth}$, (Fig.~\ref{figbulk} c). The expected values for $\rho_{meth}$ 
are 0.0036 $\textup{\r{A}}^{-3}$ at 10\% methanol concentration and 0.0090 $\textup{\r{A}}^{-3}$ at 25\% methanol. However, we find that $\rho_{meth}$ can be as low as $\approx 83\%$ of the expected values.

In particular, the decrease of $\rho_{meth}$ in both cases is non-monotonic, with maxima and minima that partially correlate with the confined CSW density extrema at 
$\sim7.5\ \textup{\r{A}}$,
$\sim9.5\ \textup{\r{A}}$,
$\sim10.5\ \textup{\r{A}}$,  
and 
$\sim12\  \textup{\r{A}}$
found in Ref.~\cite{Leoni:2021aa}, suggesting that the polar part of the methanol interaction could partially be responsible for this non-monotonicity.
Furthermore, the decrease is more evident when the nominal methanol concentration is higher, in agreement with the $\rho_{CSW}$ behavior.
Finally, the overall $\rho_{meth}$ decreases for increasing $\delta$, suggesting an increased absorption of methanol inside the graphene pore when its size increases.

To test this result directly, we calculate the methanol sequestered by the pore inside walls as a function of $\delta$ for the two mixture compositions (Fig.~\ref{figmet}, left panel).
We find that methanol in the pore exceeds what would be expected from simple osmotic equilibrium. 
In particular, at the overall 10\% and 25\% methanol compositions, we find up to 320\% and 250\% more methanol than expected inside the pore, respectively.
%
Furthermore, the greater $\delta$, the higher the methanol sequestered, above the value expected by simple osmotic diffusion. Hence, the pore inside adsorbs more methanol when its size increases.

Consequently, the
methanol amounts inside and outside the pore are anti-correlated. The effect in $V'$ is more substantial for higher methanol overall concentration (Fig.~\ref{figmet}). 
%
For increasing slit size, the trend for the methanol concentration is to increase inside and decrease outside the pore. However, the methanol concentration changes non-monotonically, correlating with the oscillatory properties of the confined CSW liquid \cite{Leoni:2021aa}.


\subsection{Pressure in $V'$}

Our results show a possible correlation between the observed variation of the mixture density in $V'$ and the properties of the confined CSW model adopted for the polar interactions of the mixture components. Although the CSW model has been tested in the literature for the mixture bulk-properties \cite{Marques:2020aa}, there is no study about its 
reliability when the bulk embeds a graphene pore. 
Therefore, we calculate the mixture's pressure $P_{V'}$ in the subvolume $V'$ to test if the coarse-grained model qualitatively reproduces the correct thermodynamics (Fig.\ref{figp_bulk}).


\begin{figure}%[h!]
\centering
\includegraphics[width=0.7\columnwidth]{images/p_bulk.png}
\caption{{\bf  The pressure in $V'$ at different mixture compositions and graphene slit-pore sizes.} $P_{V'}$ increases when the methanol concentration goes from 0\% to 10\%, to 25\%, and has a weak dependence on the pore's width $\delta$. The dependence is more evident for higher methanol concentrations.
}
\label{figp_bulk}
\end{figure}

First, we find that $P_{V'}$ increases as we increase the amount of methanol in the system. This is consistent with the expected thermodynamics  \cite{Sentenac:1998aa, perry1999perry}.

At 0\% and low, 10\%, methanol concentration, 
the overall $P_{V'}$ does not change significantly within the error bars when the slit-pore width varies. 
However,  for 25\% methanol concentration, the $P_{V'}$ tends to decrease weakly for increasing pore's width $\delta$. 

Because our calculations show that the density of the mixture in $V'$ weakly oscillates with $\delta$ (Fig.~\ref{figbulk} a), to test the mixture's equation of state, at least qualitatively, we make a parametric plot of $P_{V'}$ as a function of $\rho_{V'}$ (Fig.~\ref{figp_vs_d}).
Within the range of the observed $\delta$-dependent variations, 
we find a behavior that is qualitatively consistent with the existing experimental results, being $P_{V'}$ an increasing function of $\rho_{V'}$ that is approximately linear within the error bars
 \cite{Sentenac:1998aa, perry1999perry}.


\begin{figure}%[h]
\centering
\includegraphics[width=0.7\columnwidth]{images/p_vs_d.png}
\caption{{\bf The pressure and density of the mixture in $V'$ behave as expected.}
We calculate the density $\rho_{V'}$ and pressure $P_{V'}$ in $V'$ for pure CSW (red) and CSW-methanol mixtures with compositions 90\%-10\% (blue) and 75\%-25\% (green) for different slit-pore widths (Fig.~\ref{figbulk}a and Fig.~\ref{figp_bulk}, respectively). The parametric plot shows that 
the two quantities are proportional and, at fixed $\rho_{V'}$, $P_{V'}$ increases for increasing methanol concentration, consistent with experiments in bulk \cite{Sentenac:1998aa, perry1999perry}.
}
\label{figp_vs_d}
\end{figure}

Therefore, the coarse-grained model preserves the mixture equation of state qualitatively under the thermodynamic conditions we explore here. This occurs despite the lack of directionality in the polar interaction. As discussed in Ref.~\cite{Vilaseca2011}, a consequence of this approximation
 is the incorrect estimate of the polar entropic contribution to the free energy of the mixture, as demonstrated in Ref.~\cite{Leoni:2021aa} when comparing the free energy of the confined CSW potential with that of the atomistic TIP4P/2005 water model.

\section{Materials and Methods}
\label{sec:Methodology}

In this study, we use a coarse-grained model for the water-methanol mixture \cite{Marques:2020aa} in which both molecules are represented schematically as beads, one for water and two for methanol. Here, the OH is modeled using the Continuous Shouldered Well (CSW) potential \cite{Fr07a}, which has been used to reproduce several properties of systems as different as liquid metals, colloids, or, as in our case of interest, water (and hydroxyl functional groups) \cite{Hus:2014aa, Hus2014}.

Although it does not reproduce all the water properties due to its lack of directionality \cite{Vilaseca2011}, the CSW represents a simple approximation that significantly reduces the computation time in MD simulations compared to atomistic models. Furthermore, it is simple enough to be studied analytically, as shown, for example, in Ref.s \cite{Hus:2013aa, Hus2013, Hus2014, Munao:2015vo}.


\subsection{The coarse-grained models for the mixture}

\begin{figure}
\centering
\includegraphics[width=0.8\columnwidth]{images/potentials_ad.png}
\caption{{\bf Interactions potentials between the fluid beads.} 
The CSW potential (red line) has two characteristic length scales (the repulsive shoulder and the attractive well) as the hydrogen bond between polar groups in methanol and water.
The 24-6 Lennard-Jones (LJ, black line) and 24-6 LJ with Lorentz-Berthelot (LB) mixing rules (green line) are the interaction potentials for the methyl-methyl and methyl-hydroxyl interactions, respectively.}
\label{figpotentials}
\end{figure}

Following previous studies of water-methanol mixtures \cite{Marques:2020aa}, we represent a water molecule as a single (polar) CSW bead, while methanol as a dumbbell (two touching beads) made of an apolar 24-6 Lennard-Jones (LJ) bead for the CH$_3$ moiety and a polar CSW bead for the OH group. 
The CSW potential for the polar-polar (OH-[OH, H$_2$O], H$_2$O-H$_2$O) interactions is defined as (Fig. \ref{figpotentials}, red line)

\begin{equation} \label{eq:CSW}
    U_{\rm CSW}(r) \equiv \frac{U_R}{1+\exp \left( \frac{\Delta (r-R_R)}{a}\right)} - U_A \exp \left( - \frac{(r-R_A)^2}{2 \omega_A ^2 }\right) +U_A \left( \frac{a}{r} \right)^{24},
\end{equation}

with parameters

\begin{align*}
    \frac{U_R}{U_A}&=2, \, U_A=0.2 \, {\rm kcal/mol}, \\
    \frac{R_R}{a}&=1.6, \, a=1.77 \, \textup{\r{A}},\\
    \frac{R_A}{a}&=2, \, \left( \frac{\omega_A}{a} \right)^2=0.1, \\
    \Delta &= 15,
\end{align*}
where $a$ stands for the hard-core distance (the diameter of the particles); $R_R$ and $R_A$ are the repulsive radius and the distance of the attractive minimum, respectively; $U_R$ and $U_A$ are the energy of the repulsive shoulder and the attractive well, respectively; $\Delta$ controls the softness of the potential at $R_R$; and $\omega^2_A$ is the variance of the Gaussian centered in $R_A$ \cite{Fr07a, Vilaseca2010}. The values of the parameters for the CSW model are set in agreement with previous works \cite{Munao:2015vo, Hus:2014aa, Marques:2020aa, Leoni:2021aa}. Specifically, $U_A$ and $a$ are chosen to allow a comparison with atomistic water models \cite{Leoni:2021aa}, and $\Delta$ is as in Refs.~\cite{Munao:2015vo, Hus:2014aa} to benchmark our results against those for bulk.

The CH$_3$-CH$_3$ interaction (Fig. \ref{figpotentials}, black line) is described by the 24-6 LJ potential \cite{Munao:2015vo, Hus:2014aa, Marques:2020aa}
\begin{equation} \label{eq:LJ}
    U_{\rm LJ}(r)\equiv\frac{4}{3} 2^{2/3}\epsilon_{\rm LJ} \left[ \left(\frac{\sigma_{\rm LJ}}{r}\right)^{24} -\left( \frac{\sigma_{\rm LJ}}{r} \right)^6 \right],
\end{equation}
with
\begin{align*}
    \frac{\sigma_{\rm LJ}}{a}&=1.0, \hspace{0.2cm} \sigma_{\rm LJ}=1.77 \, \textup{\r{A}}, \\
    \frac{\epsilon_{\rm LJ}}{U_A}&=0.1, \hspace{0.2cm} \epsilon_{\rm LJ}=0.02 \, {\rm kcal/mol},
\end{align*}
where the values of $\sigma_{LJ}$ and $\epsilon_{LJ}$ are chosen such to compare with the CSW parameters.

The CH$_3$-[OH, H$_2$O] interaction is modeled with the 24-6 LJ potential employing the Lorentz-Berthelot mixing rules (Fig. \ref{figpotentials}, green line)
\begin{equation} \label{eq:LJ}
    U_{mix}(r)\equiv\frac{4}{3} 2^{2/3}\epsilon_{mix} \left[ \left(\frac{\sigma_{mix}}{r}\right)^{24} -\left( \frac{\sigma_{mix}}{r} \right)^6 \right],
\end{equation}
with
\begin{align*}
    \sigma_{mix}&\equiv\frac{1}{2}(\sigma_{\rm LJ}+a) = 1.77 \, \textup{\r{A}},\\
    \epsilon_{mix}&\equiv\sqrt{\epsilon_{\rm LJ}U_A} = 0.06 \, {\rm kcal/mol}.
\end{align*}

We consider three mixture compositions, 100\% CSW, 90\%-10\% CSW-methanol, and 75\%-25\% CSW-methanol. The first case allows us to establish a benchmark, while the other two can be compared against the bulk cases in Ref.~\cite{Marques:2020aa}. Regarding the relevance in actual cases, the 90\%-10\% composition can be compared to a mildly polluted water mixture. On the other hand, the more concentrated mixture with 75\%-25\% composition is possibly relevant in industry processes.


\subsection{The model for the graphene slit pore}

\begin{figure}%[h!]
\centering
\includegraphics[width=0.8\columnwidth]{images/gr_LB_ad.png}
\caption{{\bf Interactions potentials between the fluid beads and the graphene atoms.} 
The 12-6 LJ graphene-fluid interaction (violet line) is compared with the mixing potential in Fig.\ref{figpotentials} (green line). }
\label{figgrah_pot}
\end{figure}


Each graphene sheet is modeled as a honeycomb lattice in agreement with its atomic structure. 
Each graphene atom interacts with the fluid particles via the standard 12-6 LJ potential (Fig.\ref{figgrah_pot}, violet line)
\begin{equation}
    U_{\rm LJ}^{\rm graphene}(r)\equiv 4 \epsilon_g \left[ \left(\frac{\sigma_g}{r}\right)^{12}- \left( \frac{\sigma_g}{r} \right)^6 \right],
\end{equation}
with $\sigma_g=3.26\ \textup{\r{A}}$ and $\epsilon_g=0.1$ kcal/mol  \cite{Leoni:2021aa}, as established in literature \cite{https://doi.org/10.1002/jcc.20289}.
The positions of graphene atoms are kept fixed during the simulation.

\subsection{Graphene slit-pore geometry}

\begin{figure}
\centering
\includegraphics[width=0.6\columnwidth]{images/simulation_box.png}
\caption{{\bf Simulation snapshot for a 90\%-10\% CSW-methanol mixture}. The two navy blue boxes, connected by the periodic boundary conditions, correspond to the subregion with volume $V'$ in which we calculate the observables. 
Blue beads are CSW particles; methyl groups are purple; hydroxyl groups are green; pink lines represent the graphene lattice. The yellow box emphasizes the subvolume $V_s$ in which we compute the confined mixture composition.
}
\label{figsim}
\end{figure}

The graphene slit-pore is composed of two parallel sheets of sizes $l_{x, gr}=49\ \textup{\r{A}}$, $l_{y, gr}=51\ \textup{\r{A}}$ and width $l_{z, gr}=\delta$.
%
We vary $\delta$  from $6.5\,\textup{\r{A}}$ up to $13\,\textup{\r{A}}$.
%
The pore is included in a volume $V=L_x\, L_y\, L_z$,  with $L_x=L_y=84 $\AA\  and $L_z=98 $\AA\ and periodic boundary conditions (p.b.c.).

All the observables are calculated in a subvolume $V'\equiv L_{x}'\, L_{y}'\, L_{z}'$ sufficiently separated from the graphene walls to avoid direct interface effects,  with 
$L_{x}'=L_{y}'=84$ \AA, and $L_{z}'=44\ \textup{\r{A}}$ 
 (Fig. \ref{figsim}). With this choice of parameters, the distance of $V'$ from a graphene wall is always $>20$ \AA. 
 
To evaluate the changes of properties in $V'$, we compute the number density of the mixture and of both of its components as:
\begin{equation}
    \rho_\alpha\equiv  \frac{
    \exv{N_\alpha}
    }
    {V'},
\end{equation}
where $\exv{N_\alpha}$ is the ensemble average of the number of molecules $\alpha$, with $\alpha$ standing for CSW, methanol, or both (for which we use the symbol $\rho_{V'}$), inside the subregion $V'$. 
We also calculate the pressure of the system,
\begin{equation} \label{pressure}
    P_{V'}=\frac{\exv{N}k_BT}{V'} + \frac{1}{3V'}\exv{\sum_{i=1}^N\sum_{i>j}^N \mathbf{r}_{ij} \cdot \mathbf{f}_{ij}}.
\end{equation}
The first (kinetic) term comprises the ensemble average of all molecules $\exv{N}$, the Boltzmann constant $k_B$, the fixed temperature of the simulation $T$, and the volume of the subregion $V'$. The second term is the Virial, averaged over the ensemble of all the pairs of molecules $i$, $j$ in the volume $V'$, and $\mathbf{r}_{ij} \cdot \mathbf{f}_{ij}$ the scalar product between their distances and forces.
Finally, we compute the mixture composition in the subregion $V'$ outside the pore (Fig. \ref{figsim}).


For comparison, we also calculate the amount of methanol inside the pore as a function of the width $\delta$.
To minimize the edge effects of the walls, we consider only 
a reduced region of the slit-pore, i.e., a central subvolume $V_{s}=L_{x,s}\, L_{y,s}\,\delta$, where $L_{x,s}=L_{y,s}=30\ \textup{\r{A}}$.
 

\subsection{Molecular Dynamics}

We analyze the system by Molecular Dynamics (MD) simulations, where Newton's equations of motion of each degree of freedom are solved numerically in the canonical ensemble with a fixed total number $N=25,000$ of particles composing the mixture, in a total volume $V$, at fixed temperature $T=100$K controlled by the Nos\'e-Hoover thermostat, as implemented in the LAMMPS software \cite{LAMMPS}.
%
For the graphene slit-pore in the simulation box with p.b.c., we set the initial configuration by intertwining two hexagonal lattices (maximum packing) of CSW beads and methanol dimers and melting them during the initial equilibration procedure. Without the slit-pore, the number density would be $N/V=  0.036$ \AA$^{-3}$. 

We adopt the Leap-Frog integration algorithm \cite{Birdsall_2018}, a standard second-order and time-reversible integration method, with time-step $\delta t= 1$ fs.  We equilibrate the system for 0.1 ns to properly mix the two liquids and analyze the data collected every 100 fs for the next 0.1 ns.

\section{Summary and conclusions}

Water-methanol separation is relevant in several industrial applications, including methanol extraction for biofuels \cite{Ren_2000, Boysen_2004}. Still, traditional methods have limited efficiency and high economic costs  \cite{Liang_2014}. Therefore, exploring alternative approaches is technologically significant, scientifically challenging, and timely for the UN Sustainable Development Goals (SDGs)  ``Clean Water and Sanitation'' and ``Affordable and clean energy''. In particular, graphene sponges are novel nanomaterials that can effectively remove contaminants from water. They have several advantages over conventional methods, such as high surface area, tunable pore size, and multifunctional properties. Moreover, they can combine different water purification mechanisms, such as adsorption, catalytic, and electrocatalytic degradation of pollutants. This makes them promising candidates for various environmental engineering and water treatment applications \cite{Baptista_Pires_2021, Esplandiu:2023aa}.

Here, we investigate by Molecular Dynamics how an embedded graphene slit-pore modifies the properties of a water-methanol mixture. We find that the preferential interaction of the graphene with the hydrophobic moiety of the methanol induces an effective decrease of methanol concentration in the solution. 
On the other hand, 
the methanol accumulation inside the pore can be as high as 320\% of its nominal concentration under the condition we explore here, leading to a methanol depletion in the solution composition up to $\simeq 84$\% of its overall value.
Consequently, the methanol concentrations inside and outside the pore are anti-correlated with a more evident effect for higher methanol overall concentration (Fig.~\ref{figmet}). 


We observe that the slit pore width $\delta$ has an appreciable effect on all the solution thermodynamic quantities of the mixture and its components outside the pore. In particular, their densities (Fig.~\ref{figbulk} a-c) and the mixture pressure (Fig.~\ref{figp_bulk}) decrease for increasing $\delta$. 
Therefore, these quantities can be tuned by changing the size of the embedded pore due to the size-dependent adsorption of methanol within the graphene-confined region.

Because we coarse-grain the mixture based on the CSW water-like liquid \cite{Fr07a} and the CSW-based dumbbell methanol model \cite{Hus:2014aa} to make the simulation efficient, we test if these results could be due to the approximation of our approach. Indeed,
both models have been tested in the literature \cite{oliveira2008, Munao:2015vo}, reproducing properties of the mixture  \cite{Marques:2020aa, Prslja:2019aa,Sentenac:1998aa,perry1999perry}.
Nevertheless, the lack of a directional interaction, leading to the formation of a hydrogen bond network, makes these models able to reproduce only part of the properties of the two components of the mixture, as discussed for the case of water, for example, in Ref.s~\cite{Vilaseca2011, Leoni:2021aa}.
We, therefore, test if the changes we find for the mixture density and pressure outside the pore as a function of the pore size are thermodynamically consistent. We find that they follow qualitatively the expected equation of state of the mixture, validating our results.
These results encourage further investigation to find the optimal parameters for graphene slit-pore and sponge applications in nano-filtering and purification of water-alcohol mixtures by physical mechanisms. In addition, further investigation will be necessary to understand if, for example, the sequestration of methanol within the graphene layers increases for an increasing number of layers.


\vspace{6pt} 

%%%%%%%%%%%%%%%%%%%%%%%%%%%%%%%%%%%%%%%%%%
%% optional
%\supplementary{The following supporting information can be \colorbox{yellow}{downloaded} at:  \linksupplementary{s1}, Figure S1: title;
%Table S1: title; Video S1: title.
%}

%%%%%%%%%%%%%%%%%%%%%%%%%%%%%%%%%%%%%%%%%%
\authorcontributions{
%{\colorbox{yellow}{For research articles with several authors}, a short paragraph specifying their contributions must be provided. The following statements should be used 
Conceptualization, F.L, C.C.B., and G.F.; 
methodology, F.L, C.C.B., and G.F.; 
software, R.B.P.; 
validation, R.B.P., F.L, C.C.B., and G.F.; 
formal analysis, R.B.P.; 
investigation, R.B.P., F.L, C.C.B., and G.F.; 
resources, G.F.; 
data curation, R.B.P.; 
writing---original draft preparation, R.B.P., and G.F.; 
writing---review and editing, R.B.P., F.L, C.C.B., and G.F.; 
visualization, R.B.P., and G.F.; 
supervision, F.L, C.C.B., and G.F.; 
project administration, G.F.; 
funding acquisition, C.C.B., G.F. 
All authors have read and agreed to the published version of the manuscript.
%please turn to the  \href{http://img.mdpi.org/data/contributor-role-instruction.pdf}{CRediT taxonomy} for the term explanation. Authorship must be limited to those who have contributed substantially to the work reported.
}

\funding{
%Please add: ``This research received no external funding'' or ``
This research was funded by 
%NAME OF FUNDER
MCIN/AEI/ 10.13039/ 501100011033 and ``ERDF A way of making Europe" grant number PGC2018-099277-B-C22 and PID2021-124297NB-C31.
%'' and ``The APC was funded by XXX
%''. Check carefully that the details given are accurate and use the standard spelling of funding agency names at \url{https://search.crossref.org/funding}, any errors may affect your future funding.
}


\dataavailability{
%We encourage all authors of articles published in MDPI journals to share their research data. In this section, please provide details regarding where 
Data supporting reported results are available upon request.
%can be found at 
%, including links to publicly archived datasets analyzed or generated during the study. 
%Where no new data were created, or where data is unavailable due to privacy or ethical re-strictions, a statement is still required. Suggested Data Availability Statements are available in section “MDPI Research Data Policies” at \url{https://www.mdpi.com/ethics}.
} 

\acknowledgments{G.F. acknowledges the Visitor Program of the Max Planck Institute for The Physics of Complex Systems for supporting a six-month visit that started on November 2022. 
}

\conflictsofinterest{
%Declare conflicts of interest or state ``
The authors declare no conflict of interest.
%Authors must identify and declare any personal circumstances or interest that may be perceived as inappropriately influencing the representation or interpretation of reported research results. Any role of the funders in the design of the study; in the collection, analyses or interpretation of data; in the writing of the manuscript; or in the decision to publish the results must be declared in this section. If there is no role, please state ``
The funders had no role in the design of the study; in the collection, analyses, or interpretation of data; in the writing of the manuscript; or in the decision to publish the~results.
} 

%%%%%%%%%%%%%%%%%%%%%%%%%%%%%%%%%%%%%%%%%%
%% Only for journal Encyclopedia
%\entrylink{The Link to this entry published on the encyclopedia platform.}

\abbreviations{Abbreviations}{
The following abbreviations are used in this manuscript:\\

\noindent 
\begin{tabular}{@{}ll}
MDPI & Multidisciplinary Digital Publishing Institute\\
%DOAJ & Directory of open access journals\\
CSW & Continuous shouldered well\\
MSD & Mean square displacement\\
CM & Center of mass\\
LJ & Lennard-Jones\\
%LB & Lorentz-Berthelot\\
MD & Molecular dynamics
\end{tabular}
}

%%%%%%%%%%%%%%%%%%%%%%%%%%%%%%%%%%%%%%%%%%
%% Optional
%\appendixtitles{no} % Leave argument "no" if all appendix headings stay EMPTY (then no dot is printed after "Appendix A"). If the appendix sections contain a heading then change the argument to "yes".
%\appendixstart
%\appendix
%\section[\appendixname~\thesection]{}
%\subsection[\appendixname~\thesubsection]{}
%The appendix is an optional section that can contain details and data supplemental to the main text---for example, explanations of experimental details that would disrupt the flow of the main text but nonetheless remain crucial to understanding and reproducing the research shown; figures of replicates for experiments of which representative data are shown in the main text can be added here if brief, or as Supplementary Data. Mathematical proofs of results not central to the paper can be added as an appendix.

%\begin{table}[H] 
%\caption{This is a table caption.\label{tab5}}
%\newcolumntype{C}{>{\centering\arraybackslash}X}
%\begin{tabularx}{\textwidth}{CCC}
%\toprule
%\textbf{Title 1}	& \textbf{Title 2}	& \textbf{Title 3}\\
%\midrule
%Entry 1		& Data			& Data\\
%Entry 2		& Data			& Data\\
%\bottomrule
%\end{tabularx}
%\end{table}

%\section[\appendixname~\thesection]{}
%All appendix sections must be cited in the main text. In the appendices, Figures, Tables, etc. should be labeled, starting with ``A''---e.g., Figure A1, Figure A2, etc.

%%%%%%%%%%%%%%%%%%%%%%%%%%%%%%%%%%%%%%%%%%
\begin{adjustwidth}{-\extralength}{0cm}
%\printendnotes[custom] % Un-comment to print a list of endnotes

\reftitle{References}

% Please provide either the correct journal abbreviation (e.g. according to the “List of Title Word Abbreviations” http://www.issn.org/services/online-services/access-to-the-ltwa/) or the full name of the journal.
% Citations and References in Supplementary files are permitted provided that they also appear in the reference list here. 

%=====================================
% References, variant A: external bibliography
%=====================================
%\bibliographystyle{ieeetr}
\bibliography{water-methanol}
%=====================================
% References, variant B: internal bibliography
%=====================================
% \begin{thebibliography}{999}
% % Reference 1
% \bibitem[Author1(year)]{ref-journal}
% Author~1, T. The title of the cited article. {\em Journal Abbreviation} {\bf 2008}, {\em 10}, 142--149.
% % Reference 2
% \bibitem[Author2(year)]{ref-book1}
% Author~2, L. The title of the cited contribution. In {\em The Book Title}; Editor 1, F., Editor 2, A., Eds.; Publishing House: City, Country, 2007; pp. 32--58.
% % Reference 3
% \bibitem[Author3(year)]{ref-book2}
% Author 1, A.; Author 2, B. \textit{Book Title}, 3rd ed.; Publisher: Publisher Location, Country, 2008; pp. 154--196.
% % Reference 4
% \bibitem[Author4(year)]{ref-unpublish}
% Author 1, A.B.; Author 2, C. Title of Unpublished Work. \textit{Abbreviated Journal Name} year, \textit{phrase indicating stage of publication (submitted; accepted; in press)}.
% % Reference 5
% \bibitem[Author5(year)]{ref-communication}
% Author 1, A.B. (University, City, State, Country); Author 2, C. (Institute, City, State, Country). Personal communication, 2012.
% % Reference 6
% \bibitem[Author6(year)]{ref-proceeding}
% Author 1, A.B.; Author 2, C.D.; Author 3, E.F. Title of presentation. In Proceedings of the Name of the Conference, Location of Conference, Country, Date of Conference (Day Month Year); Abstract Number (optional), Pagination (optional).
% % Reference 7
% \bibitem[Author7(year)]{ref-thesis}
% Author 1, A.B. Title of Thesis. Level of Thesis, Degree-Granting University, Location of University, Date of Completion.
% % Reference 8
% \bibitem[Author8(year)]{ref-url}
% Title of Site. Available online: URL (accessed on Day Month Year).
% \end{thebibliography}

% If authors have biography, please use the format below
%\section*{Short Biography of Authors}
%\bio
%{\raisebox{-0.35cm}{\includegraphics[width=3.5cm,height=5.3cm,clip,keepaspectratio]{Definitions/author1.pdf}}}
%{\textbf{Firstname Lastname} Biography of first author}
%
%\bio
%{\raisebox{-0.35cm}{\includegraphics[width=3.5cm,height=5.3cm,clip,keepaspectratio]{Definitions/author2.jpg}}}
%{\textbf{Firstname Lastname} Biography of second author}

% For the MDPI journals use author-date citation, please follow the formatting guidelines on http://www.mdpi.com/authors/references
% To cite two works by the same author: \citeauthor{ref-journal-1a} (\citeyear{ref-journal-1a}, \citeyear{ref-journal-1b}). This produces: Whittaker (1967, 1975)
% To cite two works by the same author with specific pages: \citeauthor{ref-journal-3a} (\citeyear{ref-journal-3a}, p. 328; \citeyear{ref-journal-3b}, p.475). This produces: Wong (1999, p. 328; 2000, p. 475)

\PublishersNote{}

\clearpage
%\onecolumngrid

% \begin{center}
% {\bf\large{Supplementary Material}}
% \end{center}


% \setcounter{equation}{0}
% \setcounter{figure}{0}
% \setcounter{table}{0}
% \setcounter{section}{0}
% %\setcounter{page}{1}
% \makeatletter
% \renewcommand{\theequation}{S\arabic{equation}}
% \renewcommand{\thefigure}{S\arabic{figure}}
% \renewcommand{\thetable}{S\arabic{table}}
% \renewcommand{\thesection}{S\arabic{section}}
% %\renewcommand{\bibnumfmt}[1]{[S#1]}
% %\renewcommand{\citenumfont}[1]{S#1}






%%%%%%%%%%%%%%%%%%%%%%%%%%%%%%%%%%%%%%%%%%
%% for journal Sci
%\reviewreports{\\
%Reviewer 1 comments and authors’ response\\
%Reviewer 2 comments and authors’ response\\
%Reviewer 3 comments and authors’ response
%}
%%%%%%%%%%%%%%%%%%%%%%%%%%%%%%%%%%%%%%%%%%

\end{adjustwidth}
\end{document}
