\section{Problem statement}
\label{sec::problem_statement}
Let $\rvx_t \in \mathbb{R}^d$ be a state at time $t$ described by $d$-dimensional vectors and $\rvz_t \in \mathbb{R}^k$ be the $k$-dimensional measurements results. 
For example, the object location on the plane $x, y$ and its heading $\phi$ can be considered as $3D$ state, while the results of distance measurements to the $k$ nearest beacons can be considered as $kD$ measurement vector.

Visualization of the object localization problem is shown in Figure~\ref{fig::localization}. 
Here the state $\rvx_t$ consists of coordinates in the plane and heading, $\rvz_t$ are distances between the object and beacons. 
The dashed arrows denote object motion between the states at two consecutive moments of time.
Note that, both state vectors and measurement vectors are noisy.

\begin{figure}[!h]
    %\centering
    \hspace{-0cm}
    \hspace{-2cm}
    \includegraphics[width=15cm, trim={0cm 0cm 0cm 0cm}, clip]
    %{pics/localization.pdf}
    {pics/problem_statement.pdf}
    \vspace{-1cm}
    \caption{Localization problem visualization. The object moves along the dashed line. The distances from the object to the beacons are measured along the dotted lines. The beacons are marked as stars. The obstacles are shown as  blocks}
    \label{fig::localization}
\end{figure}


Assume that we know a motion equation, which relates states at two consecutive moments of time $\rvx_t$ and state $\rvx_{t-1}$:
\begin{equation}
\label{eq::motion}
\rvx_t = f(\rvx_{t - 1}, \rvu_{t}, \boldsymbol{\eta}),
\end{equation}
where $\rvu_{t}$ is an external control and $\boldsymbol{\eta}$ is a motion noise.
In the example from Figure~\ref{fig::localization}, the state is updated as:
 \begin{equation}
 \begin{split}
 &x_{t}=x_{t-1}+(u_t+\eta_r) \cos(\phi_{t-1}+ \Delta \phi_{t} + \eta_{\phi})\\ 
 &y_{t}=y_{t-1}+(u_t+\eta_r) \sin(\phi_{t-1}+\Delta \phi_{t} + \eta_{\phi})\\
 &\phi_{t}=\phi_{t-1} + \Delta \phi_{t} + \eta_{\phi},
 \end{split}
 \end{equation} 
where $\boldsymbol{\eta}=(\eta_r, \eta_{\phi})$ are radial and tangential components of the noise $\boldsymbol{\eta}$, which model uncertainty in object motion.
Also, external control vector $\rvu_t$ has the following form $\rvu_t = [u_t, \Delta \phi_t]$. 

Since there is a noise in the motion equation, we aim to correct the next state with additional measurements that fine-tunes the location of the object in the environment. 
Assume that the measurement equation is known:
\begin{equation}
\label{eq::measurement}
\rvz_t = g(\rvx_t, \boldsymbol{\zeta}),
\end{equation}
where $g: \mathbb{R}^d\rightarrow \mathbb{R}^k$ is a measurement function, which maps $d$-dimensional state $\rvx_t$ to the $k$-dimensional measurement~$\rvz_t$.
Also, denote by $\boldsymbol{\zeta}$ the measurement noise. 
In the aforementioned example (see~Figure~\ref{fig::localization}), the measurement function~$g$ computes the distances between the current object position and the beacons' positions, i.e. $z_k=\|\rvx^{(p)}_t - \rvs_k \| + \zeta_k$, where $\rvs_k$ is the position of the $k$-th beacon and $\rvx^{(p)}_t$ is the position of the object, which is a subvector of state vector $\rvx_t$.

% It should be mentioned, that the essential features existing in the environment Add possible environments are obstacles and beacons. Usually, the obstacles are encoded in the motion function $f(\cdot)$ and beacons in the measurement function $g(\cdot)$.

Despite the measurements aimed to correct object state, a filtering algorithm may show poor performance if the environment is highly symmetric. 
The environment symmetry may cause that different states $\rvx_t^{(i)}$ are mapped to similar measurements $\rvz_t^{(i)}$. 
If the difference between $\rvz_t^{(i)}$ is within measurement noise, the filter algorithm provides an incorrect state vector. 
Figure~\ref{fig::sym_env_example} shows examples of symmetric and non-symmetric environments.
A more detailed discussion of symmetric environments is presented in Section~\ref{sec::experiments}, where the evaluation of filtering methods in such environments is presented.

\begin{figure}[!h]
    \centering
    \begin{subfigure}{0.45\textwidth}
    \includegraphics[width=\textwidth]{pics/_w18.pdf}
    \end{subfigure}
    ~
    \begin{subfigure}{0.45\textwidth}
    \includegraphics[width=\textwidth]{pics/_w18_1.pdf}
    \end{subfigure}
    \caption{Example of symmetric (left) and non-symmetric (right) environments. Black crosses indicate beacons. Grey blocks indicate obstacles.}
    \label{fig::sym_env_example}
\end{figure}

The localization problem can be formally stated as the minimization of the mean squared error between the predicted states and the ground-truth ones in the time moments $t=1,\ldots,T$:
\begin{equation}
\begin{split}
&\min_h \frac{1}{T}\sum_{t=1}^{T}{\|h(\rvx_t, \rvz_t)-\rvx^{*}_t\|_2^2},\\
\text{s.t. } & \rvx_t = f(\rvx_{t - 1}, \rvu_{t}, \boldsymbol{\eta}_t)\\
& \rvz_t = g(\rvx_t, \boldsymbol{\zeta}),
\end{split}
\label{eq:problem_statement}
\end{equation}
where $\rvx^{*}_t$ denotes the ground-truth state at time $t$, and function $h$ depends on both state and measurement vectors and provides the estimate of the ground-truth state. 
Further, we use the mean squared error (MSE) loss function: 
\begin{equation}
MSE = \frac{1}{T}\sum_{t=1}^{T}{\|h(\rvx_t, \rvz_t)-\rvx^{*}_t\|_2^2}
\label{eq::mse_def}
\end{equation}
and the final state error (FSE) loss function:
\begin{equation}
FSE = \|h(\rvx_T, \rvz_T)-\rvx^{*}_T\|_2^2
\label{eq::fse_def}
\end{equation}
to evaluate the performance of the considered methods.
In problem~(\ref{eq:problem_statement}) target function $h$ encodes a particular filtering method that eliminates the noise from the measured state, e.g. Kalman filter or particle filter.
A brief description of these filters is given below for the readers' convenience.
% An opposite case is the function $h$ is parameterized with parameter $\vtheta$.
% In the latter case, we come to neural net-based methods, where $h$ is a neural network with some pre-defined architecture and $\vtheta$ is a vector of its parameters.
% Therefore, the optimization problem for this case is the following
% \begin{equation}
% \min_{\vtheta} \frac{1}{T}\sum_{t=1}^{T}{\|h_{\vtheta}(\rvx_{t-1})-\rvx^{*}_t\|_2^2}.
% \end{equation}

% The equations considered above describe generic approach. 
% More specifically, the particle filter algorithm is based on probability density function, which estimates position $\rvx$ and uncertainty $\rvx$ at time $t$ using previous states $\rvx_1,..,\rvx_{t-1}$, measurements results before current step $\rvz_1,..,\rvz_{t-1}$, and probabilities $p(\rvx_t)$, which determine pdf of particle states.

% Add further about the other criterion of solving the localization problem like robustness, variance reduction etc
% In the final paragraph make a link to the next section

