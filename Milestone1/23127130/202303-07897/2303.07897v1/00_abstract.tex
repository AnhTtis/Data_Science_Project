%\secion{abstract}
This study considers the object localization problem and proposes a novel multiparticle Kalman filter to solve it in complex and symmetric environments.
Two well-known classes of filtering algorithms to solve the localization problem are Kalman filter-based methods and particle filter-based methods.
We consider these classes, demonstrate their complementary properties, and propose a novel filtering algorithm that takes the best from two classes.
We evaluate the multiparticle Kalman filter in symmetric and noisy environments.
Such environments are especially challenging for both classes of classical methods.
We compare the proposed approach with the  particle filter since only this method is feasible if the initial state is unknown.
In the considered challenging environments, our method outperforms the particle filter in terms of both localization error and runtime.

% The two classes of filtering methods to solve this problem are studied.
% The first one consists of classical methods like the Kalman filter, particle filter, and their modifications.
% The second one utilizes deep learning techniques and their combinations with classical approaches.
% The main goal of the work is the systematic study of different methods, implement these methods if there are no open-source implementations, and propose a new approach that addresses the issue of complex and symmetric environments.
% The features of such environments are motion and measurement noises, multiple sensors, symmetric worlds, labyrinths, etc.
% Among considered and newly developed, two approaches need to be mentioned: the Kalman particle filter and the model-based particle filter network. The model-based method fits multiple parameters, such as motion and measurement functions, adjusts noise level, etc. using available train data. Both these methods are novel and outperform considered competitors.
