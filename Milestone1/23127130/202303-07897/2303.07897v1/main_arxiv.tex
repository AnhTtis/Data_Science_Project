\documentclass[12pt]{article}
\usepackage{arxiv}
\usepackage{algorithm,algorithmic}
\usepackage{amsmath,amssymb,amsfonts}
\newcommand{\bbox}{\text{bbox}}
\newcommand{\alphapck}{\alpha_\bbox}
\newcommand{\kcycle}{\text{k-CyPCK}}
\newcommand{\cycle}{\text{-CyPCK}}

\newcommand{\I}{\mathbf{I}}
\newcommand{\Ia}{\I^\text{a}}
\newcommand{\Ib}{\I^\text{b}}
\newcommand{\Iatob}{\I^\text{a $\rightarrow$ b}}
\newcommand{\F}{\mathbf{F}}
\newcommand{\Fa}{\F^\text{a}}
\newcommand{\Fb}{\F^\text{b}}
\newcommand{\f}{\mathbf{f}}
\newcommand{\fa}{\f^\text{a}}
\newcommand{\fb}{\f^\text{b}}
\newcommand{\p}{\mathbf{p}}
\newcommand{\pa}{\p^\text{a}}
\newcommand{\pb}{\p^\text{b}}
\newcommand{\A}{\boldsymbol{\Phi}_\text{align}}
\newcommand{\G}{\mathbf{G}}
\newcommand{\C}{\mathbf{C}}
\newcommand{\Ca}{\C^\text{a}}
\newcommand{\Cb}{\C^\text{b}}
\newcommand{\cc}{\mathbf{c}}
\newcommand{\cca}{\cc^\text{a}}
\newcommand{\ccb}{\cc^\text{b}}
\newcommand{\Irec}{\I_\text{Recon}}
\newcommand{\M}{\mathbf{M}}
\newcommand{\Mrec}{\M_\text{Recon}}
\newcommand{\loss}{\mathcal{L}}
\newcommand{\T}{\mathcal{T}}
\newcommand{\W}{\mathcal{W}}
\newcommand{\Id}{\mathcal{I}}

\usepackage{todonotes}
\usepackage{float}
\usepackage{multicol,multirow}
\usepackage{secdot}
% \usepackage[margin=2.5cm]{geometry}
%\usepackage{authblk}
% \renewcommand{\baselinestretch}{1.5}
%\renewcommand\Authands{and}
\usepackage{amsmath,array,ragged2e}
\newcolumntype{C}{>{\Centering\arraybackslash}m{0.14\linewidth}}
%\usepackage{natbib}
\usepackage{booktabs}
\usepackage{adjustbox}
\usepackage{subcaption}
\usepackage{graphicx}
\usepackage{hyperref}
%\frame{\titlepage}
\graphicspath{{./pics/}}
\usepackage{lineno}

\def\eqref#1{(\ref{#1})}

\begin{document}

\title{Multiparticle Kalman filter for object localization in symmetric environments}
% \title{Novels in Particle Filter Method for Objects Localization in Case of Noisy Measurements}


\author{Roman Korkin \\ 
Novosibirsk Technology Center \\
Schlumberger, Russia\\
\url{korkin.rv@phystech.edu}
\And 
Ivan Oseledets \\
Skoltech, Moscow, Russia \\
AIRI, Moscow, Russia \\
\url{i.oseledets@skoltech.ru}
\And Aleksandr Katrutsa\thanks{Corresponding author}\\
Skoltech, Moscow, Russia \\
AIRI, Moscow, Russia \\
\url{aleksandr.katrutsa@phystech.edu}
}

% \ead{korkin.rv@phystech.edu}
% \author[add2,add3]{Aleksandr Katrutsa\corref{cor1}}
% %\corref{cor2}]
% \ead{aleksandr.katrutsa@phystech.edu}
% \cortext[cor1]{Corresponding author}

% \address[add1]{Novosibirsk Technology Center Schlumberger}
% \address[add2]{Skolkovo Institute of Science and Technology, Moscow, Russia}
% \address[add3]{AIRI, Moscow, Russia}
\maketitle

\begin{abstract}


\begin{abstract}
% \vspace{-1em}
The diffusion-based generative models have achieved remarkable success in text-based image generation. However, since it contains enormous randomness in generation progress, it is still challenging to apply such models for real-world visual content editing, especially in videos. 
In this paper, we propose \texttt{FateZero}, a zero-shot text-based editing method on real-world videos without per-prompt training or use-specific mask. 
\RM{Specifically, different from a pipeline of two independent inversion and then generation stages, we find the intermediate attention maps during inversions store better structure and motion information. We thus reform them to temporally casual attention and replace them in the generation progress. To further reduce the unnecessary semantic leakage of source video and enhance the editing quality, we then remix the temporally casual attentions via the cross-attention features of the source prompt as the mask.}
To edit videos consistently, we propose several techniques based on the pre-trained models. Firstly, in contrast to the straightforward DDIM inversion technique, our approach captures intermediate attention maps during inversion, which effectively retain both structural and motion information. These maps are directly fused in the editing process rather than generated during denoising. To further minimize semantic leakage of the source video, we then fuse self-attentions with a blending mask obtained by cross-attention features from the source prompt. Furthermore, we have implemented a reform of the self-attention mechanism in denoising UNet by introducing spatial-temporal attention to ensure frame consistency.
Yet succinct, our method is the first one to show the ability of zero-shot text-driven video style and local attribute editing from the trained text-to-image model. We also have a better zero-shot shape-aware editing ability based on the text-to-video model~\cite{tuneavideo}. \RM{Besides video, our unified method also achieves state-of-the-art performance in zero-shot image editing.\chenyang{Need exp or remove the zero-shot image}} Extensive experiments demonstrate our superior temporal consistency and editing capability than previous works.
% The code will be released.
% \chenyang{emphasize: our observation at inversion time} \xiaodong{replacing the bold part to the actual pipeline: \textbf{Specifically, we work on replacing and mixing the attention maps between the inversion and generation since the self-attention map keeps the structure of the original natural image and the cross-attention is semantic-related, after remixing, we replace them in the corresponding generation steps for denoising.}}
% \footnote{Since there is no general video diffusion model is publicly available, we use one-shot video generation method~(Tune-A-Video~\cite{tuneavideo}) as the pretrained video diffusion model for zero-shot video editing\xiaodong{can be removed if we actually zero-shot on video}.}.
\end{abstract}
\end{abstract}

% \begin{keyword}
% localization problem \sep Kalman filter \sep particle filter \sep symmetric environment
% \end{keyword}

% \linenumbers
%\cortext[cor1]{Please address correspondence to Author2 or Author3}
%\cortext[cor1]{Please address correspondence to Author2 or Author3}


%\affil[1]{Novosibirsk Technology Center Schlumberger \LaTeX\ University}
%\affil[2]{Moscow Institute of Physics and Technology \LaTeX\ University}

\section{INTRODUCTION}

\looseness=-1
Recent advances in hardware and software make autonomous robots more and more important in various environments, from manufacturing and warehouses to healthcare and living spaces.
Learning has been one of the most promising tools for operating robots in such unstructured environments, enabling them to acquire complex perception and reasoning capabilities.
However, the current status quo of designing robots does not account for the impact of learning: rather, many robots are still designed based on human experts' intuition or hand-crafted heuristics.
Therefore, such designs can lead to a sub-optimal performance by causing unexpected visual occlusions.
This is where the idea of guiding robot design to improve the robot learning capability comes in, inspired by the evolutionary process.

\looseness=-1
The evolution of physical attributes through natural selection has played a significant role in the emergence of advanced cognitive abilities among living beings~\cite{manning1998introduction}.
By embracing the idea of evolution, robots also have the potential to evolve their designs for better real-world performance. 
However, it is extremely challenging to encapsulate all the perception, control, and hardware design into a single holistic evaluation pipeline due to the complexity of the components. 
For instance, it will be extremely expensive to formulate a typical two-loop design optimization process, which searches robot morphologies in the outer loop and trains a policy for each given design in the inner loop.
As such, traditional design optimization techniques focus on enhancing certain attributes in isolation or exploring based on hand-designed heuristics.

\begin{figure}
  \centering
  \vspace{3mm}
  \includegraphics[width=0.5\textwidth]{figs/teaser.png}
  \vspace{-6mm}
  \caption{
  A side-by-side comparison of existing vs proposed hardware design optimization approaches.
  }
  \vspace{-5mm}
  \label{fig:teaser}
\end{figure}


\looseness=-1
This work particularly aims to discuss the design optimization for vision-based manipulation.
In the general context of manipulation, visual sensors provide a rich stream of information that allow robots to perform tasks such as grasping, object manipulation, and assembly.
The use of visual sensors in manipulation, however, inevitably poses challenges associated with complete or partial field-of-view occlusion, which can degrade performance due to limited perception. 
Whether an occlusion is harmless or fatal often depends on the task at hand and the stage of task execution.
Such scenarios motivate us to explore hardware optimization without neglecting the interplay between the robot's morphology, onboard perception abilities, and their interaction in different tasks.

\looseness=-1
We present a learning-oriented morphology optimization framework to improve the initial robot design crafted by a human expert.
Our key idea is to develop a \fullpolicynamenobold (\shortname) that is capable of controlling a range of morphologies using onboard visual observations, greatly reducing the costs of traditional two-loop design optimization. 
The controller is trained using a novel \onestagefullname (\onestageshortname) formulation, which unifies the typical two-stage approach of teacher and student training \cite{chen2020learning}.
In our experience, this novel formulation is essential for the student policy to learn behaviors characteristic of its own observation capability and not of a privileged agent.
Once the morphology-agnostic controller is learned, we optimize robot design parameters using Vizier~\cite{google_vizier} optimizer using the controller's performance as a surrogate measure (Fig. \ref{fig:teaser}).
We leverage simulation throughout this work to accelerate the process of looking for an optimal configuration without actually building a physical robot, which is both costly and time-consuming.

\looseness=-1
We evaluate the proposed technique for optimizing the morphology of a mobile manipulator.
We find that our framework can find an improved morphology that shows better performance on overall tasks and facilitates a more sample efficient learning.
Specifically, an optimized design leads to a $15$-$20$\% success rate improvement on various manipulation tasks and is $25$x more data efficient when trained from scratch.
With this work, we would like to highlight the untapped potential of learning-based robot design optimization and show how robot designs can be tailored for better performance and learning with onboard sensing.
\paragraph{Related works.}

% Add implicit comparison with the proposed method

There are multiple combinations of the particle filter (PF) with other methods~\cite{Rao, Reconcillation, particle_swarm, particle_genetic, Tutorial, SHARIATI201932}.
% Rao-Blackwellised PF \cite{Rao}, Box PF method ~\cite{Interval}, PF with data reconciliation \cite{Reconcillation}, PF and particle swarm method~\cite{particle_swarm}, PF with genetic algorithm~\cite{particle_genetic}, and many others.
Rao-Blackwellised PF~\cite{Rao} splits state vectors into two parts. 
The first part is processed with the Kalman filter, and the second part is processed with the PF. 
This approach is applicable for high dimensional problems, where the standard particle filter may fail.
The Box PF method~\cite{Interval} describes the state vector distribution as a sum of uniform probability density functions. 
This method decreases the number of particles resulting in the same accuracy as PF.
In study~\cite{Reconcillation} the measurement test criterion and data reconciliation are proposed to derive reliable initial states under sufficient information about measurements.

Other studies are devoted to a combination of the particle filter with nature-inspired optimization methods like particle swarm method~\cite{particle_swarm} or genetic algorithm~\cite{particle_genetic}.
Such combinations incorporate elements of these optimization methods into the particle filter, e.g. stochastic resampling is replaced by the crossover and mutation operations. 
Although these combinations provide better localization accuracy, they are computationally expensive.

There are also various combinations of Kalman and particle filters. 
In~\cite{Tutorial} extended Kalman particle filter is presented.
This filtering algorithm reduces uncertainty in every particle motion due to the additional Kalman-type update for every particle. 
However, 
% inconsistency between the state covariance matrix and process covariance matrix may lead to slow convergence and 
it requires a large number of particles and converges slowly if the state noise model is incorrect.
Another combination is proposed in~\cite{SHARIATI201932} and it is used diagonal process covariance matrices fitted from experimental data.
Therefore, it requires more data and can not treat an arbitrary localization problem.
% However, large weights may be assigned to irrelevant particles due to inconsistency between the state covariance matrix and process covariance matrix. 
% Therefore, the convergence becomes slower or may require a large number of particles.

% Other approach of such combination was proposed in~\cite{SHARIATI201932}, where  proposes a similar approach to \textbf{PFKU}, however, there are some differences.
% In particular, it uses diagonal process covariance matrices fitted from experimental data.
% In contrast, \textbf{PFKU} explicitly estimates covariance matrices that are not diagonal.
% Thus, \textbf{PFKU} is not based on experimental data and is more universal than this competitor.


% The other version of the Kalman and particle filters combination is discussed in ~\cite{SHARIATI201932}, where multiple particles are initiated as in the particle filter, and then the Kalman filter prediction and update steps are used for every particle, followed by the resampling procedure. 
% Here the weights updates are rather straightforward as in the standard particle filter. 
% The approach used in the present paper is almost the same as in the ~\cite{SHARIATI201932}, however, there are some modifications. 
% First, in the paper ~\cite{SHARIATI201932} the covariance matrices are taken from fitting experimental data and they are diagonal. 
% In the present paper, we assume the motion model is known: the standard robot motion based on external speed and heading. 
% Thus there is no need for the covariance matrices to be taken from the experimental data, they are calculated from the first principles and are not supposed to be diagonal. 

% The only factors which are not known are the speed and heading noise level, thus two different values for these noises will be tried.

%It should also be noted Rao–Blackwellized %particle filter~\cite{Rao}. 
% In some specific cases, the probability density function allows the split of the state vector into two parts and treats them sequentially~\cite{Rao}. 
% Then, often the first part can be processed with the Kalman filter, while the second by the particle filter. 
% This approach speeds up computations and is very useful in high dimensions, where standard particle filter doesn't perform well. In the considered test environments it does not give any benefits as both initial location and heading are unknown. 

% The other combination of Kalman and particle filters is suggested in~\cite{KPF_occlusion} avoid the occlusion problem in image processing during object tracking. 
% The Kalman filter is replaced by the particle filter when occlusion occurs and the particle filter is used while the system is stable, and then the Kalman filter is used again. The approach looks promising for occlusion problem reduction purposes, though does not look suitable for the robots' in-door tracking. In~\cite{EPF_EKF, UPF} there are more discussions on other combinations of these two filters. 


% The study~\cite{particle_swarm} suggests a particle filter based on the particle swarm optimization method. 
% By using the PSO algorithm, the particles area around the current state is determined depending on the measurement results. 
% Then, in order to attain diversity and convergence, the particles are distributed through the area. 
% According to the simulation results, the PSO-PF algorithm is more accurate than the standard particle filter, though has higher complexity.
% Another combination of particle filter and nature-inspired optimization method, for instance, genetic algorithm, is considered in~\cite{particle_genetic}. 
% Here the pure stochastic resampling is replaced by adopting crossover and mutation operators, which enhances the particles' diversity and mitigates the sample impoverishment problem.

% Such approaches are very important as allowing particle filter applications in FPGA devices, smartphones, etc.

% There are also approaches based on neural network combination with particle filter~\cite{PFRNN}. Here, the time recurrent neural network is being developed with the $N$ particles: $N$ hidden states and their weights evolving with time. At each time step, modified GRU or LSTM cell is used to update hidden states and weights based on encoded motion and measurement. Then, the differentiable resampling is used to update weights and resample hidden states. This approach reduces the required number of particle by increasing the dimension of a hidden space. The drawbacks of this approach are a long time for training and necessity to have a large dataset. 

% These two filters and their multiple modifications cover the majority of problems, so other methods~\cite{ex_filters}, such as Wiener filter~\cite{Kalman_book}, distribution filters, conjugate analysis approach, differential geometrical approach, and interacting multiple models are rather considered exotic.

\section{Problem statement}
\label{sec::problem_statement}
Let $\rvx_t \in \mathbb{R}^d$ be a state at time $t$ described by $d$-dimensional vectors and $\rvz_t \in \mathbb{R}^k$ be the $k$-dimensional measurements results. 
For example, the object location on the plane $x, y$ and its heading $\phi$ can be considered as $3D$ state, while the results of distance measurements to the $k$ nearest beacons can be considered as $kD$ measurement vector.

Visualization of the object localization problem is shown in Figure~\ref{fig::localization}. 
Here the state $\rvx_t$ consists of coordinates in the plane and heading, $\rvz_t$ are distances between the object and beacons. 
The dashed arrows denote object motion between the states at two consecutive moments of time.
Note that, both state vectors and measurement vectors are noisy.

\begin{figure}[!h]
    %\centering
    \hspace{-0cm}
    \hspace{-2cm}
    \includegraphics[width=15cm, trim={0cm 0cm 0cm 0cm}, clip]
    %{pics/localization.pdf}
    {pics/problem_statement.pdf}
    \vspace{-1cm}
    \caption{Localization problem visualization. The object moves along the dashed line. The distances from the object to the beacons are measured along the dotted lines. The beacons are marked as stars. The obstacles are shown as  blocks}
    \label{fig::localization}
\end{figure}


Assume that we know a motion equation, which relates states at two consecutive moments of time $\rvx_t$ and state $\rvx_{t-1}$:
\begin{equation}
\label{eq::motion}
\rvx_t = f(\rvx_{t - 1}, \rvu_{t}, \boldsymbol{\eta}),
\end{equation}
where $\rvu_{t}$ is an external control and $\boldsymbol{\eta}$ is a motion noise.
In the example from Figure~\ref{fig::localization}, the state is updated as:
 \begin{equation}
 \begin{split}
 &x_{t}=x_{t-1}+(u_t+\eta_r) \cos(\phi_{t-1}+ \Delta \phi_{t} + \eta_{\phi})\\ 
 &y_{t}=y_{t-1}+(u_t+\eta_r) \sin(\phi_{t-1}+\Delta \phi_{t} + \eta_{\phi})\\
 &\phi_{t}=\phi_{t-1} + \Delta \phi_{t} + \eta_{\phi},
 \end{split}
 \end{equation} 
where $\boldsymbol{\eta}=(\eta_r, \eta_{\phi})$ are radial and tangential components of the noise $\boldsymbol{\eta}$, which model uncertainty in object motion.
Also, external control vector $\rvu_t$ has the following form $\rvu_t = [u_t, \Delta \phi_t]$. 

Since there is a noise in the motion equation, we aim to correct the next state with additional measurements that fine-tunes the location of the object in the environment. 
Assume that the measurement equation is known:
\begin{equation}
\label{eq::measurement}
\rvz_t = g(\rvx_t, \boldsymbol{\zeta}),
\end{equation}
where $g: \mathbb{R}^d\rightarrow \mathbb{R}^k$ is a measurement function, which maps $d$-dimensional state $\rvx_t$ to the $k$-dimensional measurement~$\rvz_t$.
Also, denote by $\boldsymbol{\zeta}$ the measurement noise. 
In the aforementioned example (see~Figure~\ref{fig::localization}), the measurement function~$g$ computes the distances between the current object position and the beacons' positions, i.e. $z_k=\|\rvx^{(p)}_t - \rvs_k \| + \zeta_k$, where $\rvs_k$ is the position of the $k$-th beacon and $\rvx^{(p)}_t$ is the position of the object, which is a subvector of state vector $\rvx_t$.

% It should be mentioned, that the essential features existing in the environment Add possible environments are obstacles and beacons. Usually, the obstacles are encoded in the motion function $f(\cdot)$ and beacons in the measurement function $g(\cdot)$.

Despite the measurements aimed to correct object state, a filtering algorithm may show poor performance if the environment is highly symmetric. 
The environment symmetry may cause that different states $\rvx_t^{(i)}$ are mapped to similar measurements $\rvz_t^{(i)}$. 
If the difference between $\rvz_t^{(i)}$ is within measurement noise, the filter algorithm provides an incorrect state vector. 
Figure~\ref{fig::sym_env_example} shows examples of symmetric and non-symmetric environments.
A more detailed discussion of symmetric environments is presented in Section~\ref{sec::experiments}, where the evaluation of filtering methods in such environments is presented.

\begin{figure}[!h]
    \centering
    \begin{subfigure}{0.45\textwidth}
    \includegraphics[width=\textwidth]{pics/_w18.pdf}
    \end{subfigure}
    ~
    \begin{subfigure}{0.45\textwidth}
    \includegraphics[width=\textwidth]{pics/_w18_1.pdf}
    \end{subfigure}
    \caption{Example of symmetric (left) and non-symmetric (right) environments. Black crosses indicate beacons. Grey blocks indicate obstacles.}
    \label{fig::sym_env_example}
\end{figure}

The localization problem can be formally stated as the minimization of the mean squared error between the predicted states and the ground-truth ones in the time moments $t=1,\ldots,T$:
\begin{equation}
\begin{split}
&\min_h \frac{1}{T}\sum_{t=1}^{T}{\|h(\rvx_t, \rvz_t)-\rvx^{*}_t\|_2^2},\\
\text{s.t. } & \rvx_t = f(\rvx_{t - 1}, \rvu_{t}, \boldsymbol{\eta}_t)\\
& \rvz_t = g(\rvx_t, \boldsymbol{\zeta}),
\end{split}
\label{eq:problem_statement}
\end{equation}
where $\rvx^{*}_t$ denotes the ground-truth state at time $t$, and function $h$ depends on both state and measurement vectors and provides the estimate of the ground-truth state. 
Further, we use the mean squared error (MSE) loss function: 
\begin{equation}
MSE = \frac{1}{T}\sum_{t=1}^{T}{\|h(\rvx_t, \rvz_t)-\rvx^{*}_t\|_2^2}
\label{eq::mse_def}
\end{equation}
and the final state error (FSE) loss function:
\begin{equation}
FSE = \|h(\rvx_T, \rvz_T)-\rvx^{*}_T\|_2^2
\label{eq::fse_def}
\end{equation}
to evaluate the performance of the considered methods.
In problem~(\ref{eq:problem_statement}) target function $h$ encodes a particular filtering method that eliminates the noise from the measured state, e.g. Kalman filter or particle filter.
A brief description of these filters is given below for the readers' convenience.
% An opposite case is the function $h$ is parameterized with parameter $\vtheta$.
% In the latter case, we come to neural net-based methods, where $h$ is a neural network with some pre-defined architecture and $\vtheta$ is a vector of its parameters.
% Therefore, the optimization problem for this case is the following
% \begin{equation}
% \min_{\vtheta} \frac{1}{T}\sum_{t=1}^{T}{\|h_{\vtheta}(\rvx_{t-1})-\rvx^{*}_t\|_2^2}.
% \end{equation}

% The equations considered above describe generic approach. 
% More specifically, the particle filter algorithm is based on probability density function, which estimates position $\rvx$ and uncertainty $\rvx$ at time $t$ using previous states $\rvx_1,..,\rvx_{t-1}$, measurements results before current step $\rvz_1,..,\rvz_{t-1}$, and probabilities $p(\rvx_t)$, which determine pdf of particle states.

% Add further about the other criterion of solving the localization problem like robustness, variance reduction etc
% In the final paragraph make a link to the next section



% Brief description
% Kalman
% Particle

% Discussion with pros and cons
% Proposed method in details
\section{Filtering algorithms based on the Kalman filter}

% The simplest НижеBayesian filter is Kalman filter~\cite{Kalman}.
The Kalman filter is based on the assumption of Gaussian distribution of a state, control, and measurements vectors, i.e. $\vx_t\sim\mathcal{N}(\rvx_t, \mP_t)$, $\vu_t\sim \mathcal{N}(\rvu_t, \mQ)$, and $\vz_t\sim\mathcal{N}(\rvz_t, \mR)$, where~$\mP_t$, $\mQ$, and $\mR$ are the state, process, and measurement covariance matrices.
At each step, the state $\rvx_t$ is updated with a linear transformation $\mF$ and the known control dynamics~$\rvu_t$: 
\begin{equation}
    \rvx^-_{t+1} = \mF \rvx_t + \rvu_t.
    \label{eq::Kalman_motion}
\end{equation} 
Besides the measurement vector $\rvz^*_t$, we compute the predicted measurement vector 
\begin{equation}
    \rvz_t=\mH\rvx_t,
    \label{eq::Kalman_pred_measurement}
\end{equation}
which plays the crucial role in the correction of $\rvx^-_{t+1}$.
To correct updated state $\rvx^-_{t+1}$, Kalman filter uses the following equation 
\begin{equation}
    \rvx_{t+1} = \rvx_{t+1}^- + \mK_{t+1}(\rvz^*_{t+1} - \rvz_{t+1}),
    \label{eq::Kalman_x_update}
\end{equation}
where the Kalman gain $\mK_{t+1} = \mP_{t+1}^- \mH^\top (\mH \mP_{t+1}^- \mH^\top + \mR)^{-1}$.
The state covariance matrix~$\mP_{t+1}$ firstly predicted from motion equation as 
\begin{equation}
    \mP_{t+1}^- = \mF \mP_{t} \mF^\top + \mQ,
    \label{eq::Kalman_P_update}
\end{equation}
where $\mQ$ is a pre-defined constant process covariance matrix related to the motion noise $\boldsymbol{\eta}$ from~\eqref{eq::motion}.
And then it is updated with the Kalman gain $\mK_{t+1}$ according to the following formula: $\mP_{t+1} = (\mI - \mK_{t+1} \mH)\mP_{t+1}^-$.
The equations of the state~\eqref{eq::Kalman_motion} and measurement~\eqref{eq::Kalman_pred_measurement} updates are the particular cases of~\eqref{eq::motion} and~\eqref{eq::measurement}, respectively.
% The measurement true result is $\rvz_t$, while its predicted value $\tilde{\rvz}$ is found from a measurement matrix $\mH$, such that  $\tilde{\rvz}_t=\mH\rvx_t$, 
% The process covariance matrix $\mQ_t$ is 
% \begin{align*}
% \mQ_t= \langle (\rvx_t(\boldsymbol{\eta}) - \rvx_t(\boldsymbol{0})) (\rvx_t(\boldsymbol{\eta}) - \rvx_t(\boldsymbol{0}))^{\top} \rangle_{\boldsymbol{\eta}}
% \end{align*}

% \begin{align*}
% \mR=\sigma d^2 {\bf 1},
% \end{align*}

% \begin{align*}
% \mQ_t = \begin{bmatrix}
%        \cos\phi^2\,\sigma u^2 + u^2\,\sin\phi^2\, \sigma \phi^2  & (\sigma u^2 + u^2 \sigma\phi^2) \sin2\phi/2 & -u\,\sin\phi\,\sigma\phi^2 \\
%        ( \sigma u^2 + u^2 \sigma\phi^2) \sin2\phi/2 & \sin\phi^2\,\sigma u^2  + u^2 \, \cos\phi^2\sigma \phi^2 & u \cos\phi\,\sigma\phi^2\\
%        -u\,\sin\phi\,\sigma\phi^2 & u\, \cos\phi\,\sigma\phi^2 & \sigma\phi^2
% \end{bmatrix},
% \end{align*}

% \begin{align*}
% \mP_0=
% \begin{bmatrix}
%            \sigma x_0^2 & 0 & 0\\
%            0 & \sigma y_0^2  & 0\\
%            0 & 0 & \sigma \phi_0^2.
%     \end{bmatrix}
% \end{align*}
% These results are also noisy and the noise is described by the measurement covariance matrix $\mR$.
%Here  noise $\zeta \sim \mathcal{N}(0,\mR)$, where $\mR$ is measurement covariance matrix.
% Note, that the state, process, i.e, $\rvx_{t-1}$ to $\rvx_t$ transition, and measurement covariance matrices $\mP$, $\mQ_t$, and $\mR$ must be input parameters in the algorithm. 
% The first one determines the uncertainty of initial state knowledge and thus initialization is task-dependent. 
% The matrix $\mQ_t$ is a process covariance matrix, which determines the correlation between two consecutive states $\rvx_{t+1}$ and $\rvx_t$. 
% The measurement covariance matrix is also task-dependent and in the simplest case contains square errors on the diagonal. 
% The example will be considered later for the object on the plane motion. 
% To apply the filter for the plane motion directly its extended version will be used. 
% The complete description of the Kalman filter algorithm is the following:
% \begin{itemize}
% \item Predict stage
%  \begin{equation}
%  \begin{split}
% &\rvx_{t+1}^- = \mF \rvx_t + \rvu_t\\
% &\mP_{t+1}^- = \mF \mP_{t} \mF^T + \mQ_{t+1},
% \end{split}
%  \label{eq::SKF}
% \end{equation}
% \item Update stage
%  \begin{equation}
%  \begin{split}
%  &\tilde{\rvz}_{t+1} = \mH \rvx_{t+1}^-\\
%  &\rvy_{t+1} = \rvz_{t+1} - \tilde{\rvz}_{t+1}\\
%  &\mK_{t+1} = \mP_{t+1}^- \mH^\top (\mH \mP_{t+1}^- \mH^\top + \mR)^{-1}\\
%  &\rvx_{t+1} = \rvx_{t+1}^- + \mK_{t+1} \rvy_{t+1}\\ &\mP_{t+1} = (\mI - \mK_{t+1} \mH)\mP_{t+1}^-.
%  \end{split}
%  \end{equation}
%  \end{itemize}

This method iteratively gives estimate of state $\rvx_{t+1}$ and covariance matrix $\mP_{t+1}$ from the previous state $\rvx_{t}$ and measurement results $\rvz_{t+1}$. 
The main advantages of Kalman filter are low computational costs and low memory consumption. 
At the same time, it has significant drawbacks, like the linearity of motion and measurement equations, and the assumptions on the normal noise.
 
To generalize the Kalman filter to non-linear motion and measurement equations, the Extended Kalman Filter was proposed~\cite{EKF}. 
It operates with non-linear equations on coordinates $\rvx$ and measurements $\rvz$. 
These non-linear equations are incorporated in the standard Kalman filter as: 
\begin{equation}
\begin{split}%\label{5a}
&\rvx_{t+1}^- = f(\rvx_t, \rvu_{t+1}, 0)\\
&\mF_{t+1} = \left.\frac{\partial f(\rvx, \rvu_{t+1})}{\partial \rvx}\right|_{\rvx = \rvx_t}
\end{split}
\qquad \begin{split}%\label{6a}
 &\rvz_{t+1} = g(\rvx_{t+1}^-, 0)\\ 
 &\mH_{t+1} = \left.\frac{\partial g(\rvx)}{\partial \rvx}\right|_{\rvx = \rvx^-_{t+1}}\\
 \end{split}
 \label{eq::EKF}
 \end{equation}
In addition, since motion noise $\boldsymbol{\eta}$ is not additive, the covariance process matrix $\mQ \equiv \mQ_t$ depends on time moment  and is recomputed in every time step as follows
\begin{equation}
    \mQ_t = \frac1s \sum_{i=1}^s (\rvx_{t+1}^- - f(\rvx_t, \rvu_{t+1}, \boldsymbol{\eta}_i)) 
 (\rvx_{t+1}^- - f(\rvx_t, \rvu_{t+1}, \boldsymbol{\eta}_i)^\top,
 \label{eq::Qt_EKF}
\end{equation}
where $\boldsymbol{\eta}_i, \; i=1,\ldots,s$ are sampled from some pre-defined distribution $\mathcal{N}(0, \mM)$ corresponding to our assumption on the motion noise.
Other equations in the Extended Kalman filter coincide with equations in the Kalman filter. 

% \begin{itemize}
% \item Predict stage
% \begin{equation}
% \begin{split}%\label{5a}
% &\rvx_{t+1}^- = f(\rvx_t, \rvu_{t+1})\\
% &\mP_{t+1}^- = \mF \mP_{t} \mF^T + \mQ_{t+1}\\
% % &\mF = \frac{\partial f(\rvx_{t, 1}, \rvx_{t, 2}, ...)}
% % {\partial(\rvx_{t,1}, \rvx_{t2}, ...)}
% &\mF = \left.\frac{\partial f(\rvx, \rvu_{t+1})}{\partial \rvx}\right|_{\rvx = \rvx_t}
% \end{split}
% \label{eq::EKF}
% \end{equation}
% \item Update stage 
%  \begin{equation}
%  \begin{split}%\label{6a}
%  &\tilde{\rvz}_{t+1} = g(\rvx_{t+1}^-)\\ 
%  &\mH_{t+1} = \left.\frac{\partial g(\rvx)}{\partial \rvx}\right|_{\rvx = \rvx^-_{t+1}}\\
%  &\mK_{t+1} = \mP_{t+1}^- \mH_{t+1}^T (\mH \mP_{t+1}^- \mH_{t+1}^T + \mR)^{-1}\\
%  &\rvx_{t+1} = \rvx_{t+1}^- + \mK_{t+1} (\rvz_{t+1} - \tilde{\rvz}_{t+1})\\
%  &\mP_{t+1} = (\mI - \mK_{t+1} \mH_{t+1})\mP_{t+1}^-
%  \end{split}
%  \end{equation}
% \end{itemize}

Since the exact computation of Jacobians in~\eqref{eq::EKF} can be computationally intensive,
the unscented Kalman filter~\cite{Julier} computes $f$ and $g$ at the specific sigma points and uses the computed values to approximate the corresponding Jacobians.
This approach reduces the runtime of every iteration but can lead to slow convergence if the approximations of Jacobians are not sufficiently accurate.

In this section, we have briefly described some filtering algorithms inspired by the classical Kalman filter.
They extend the classical Kalman filter to non-linear motion and measurement equations.
However, they still assume that the noise distribution is normal and require the initial object state and its covariance.
These assumptions and requirements limit the practical usage of the aforementioned filtering algorithms in a real-world scenario.



% Although the Extended Kalman filter addresses the non-linearity of motion and measurement equations, it still suffers from non-Gaussian noise.
% Moreover, this filtering algorithm requires the initial object state and its covariance, which may not be available in a real-world scenario. 
 
% This method iteratively gives new expectations for state $\rvx_{t+1}$ and covariance matrix $\mP_{t+1}$ from the previous moment of time, motion, and measurement results. 
% The main advantages are low complexity $O(d^{2.376})$ and low memory consumption $O(d^2)$, where $d$ is a state dimension.
% At the same time, it is still valid under assumptuion of the normal distribution of noise. Moreover, the iteration algorithm requires knowledge of the initial state and its covariance, which might not be true in the general case. 
 
% The only challenge in nonlinear equations is covariance matrix description. 
% This is reached via Taylor expansion and uses Jacobians for covariance matrices $\mP_t$ and $\mH$. 
% The following equation~(\ref{eq::EKF}) are used instead of~(\ref{eq::SKF}):

% The other modification of the Kalman filter -- unscented Kalman filter. It also deals with nonlinear functions $f$ and $g$ but instead of direct Jacobian computation, its estimation is used. The estimation is based on the states at the specific, so-called sigma points. It speeds up the Kalman steps computations. 
% However, all other Kalman filter drawbacks remain. 
% Though this might be important in high dimensional systems, in the considered test problems the Jacobian may be explicitly computed and the analytical expression is used, which is always better than the approximate methods.

% One more extension of Kalman filter, the so called ensemble Kalman filter is interesting, as uses idea similar to particle filtering. Here the ensemble of states is used to avoid using covariance matrix. This makes it faster for high-dimensional filtering. However, the method still relies on the approximation of the prior states by a Gaussian, which is a strong limitation in many cases.


\section{Particle filter algorithm}

% Description and motivation of the main steps
% Discussion of issues related to the environemnt properties
% Link to the next section where they re addressed by our approach

As it was previously mentioned, the Kalman filter requires Gaussian distribution of motion and measurement noise and known initial state. 
If these requirements do not hold, the particle filter method~\cite{PF1996} can help.
This method successfully operates with arbitrary  distributions of state and measurement noise and even with the unknown initial state.

The idea of the particle filter method is to generate a set of trial state vectors $\rvx_{t=0}^{i} \in \mathbb{R}^d, \; i=1,\ldots,N$, which are called particles and the  corresponding weights~$w^i$ which are initialized as $1 / N$.
These vectors are used to approximate unknown distribution $p(\vx_t)$ and weights
\begin{equation}
    w^i_t = \mathbb{P}(\vx_{t} = \rvx_t^{i}).
\label{eq::weights}
\end{equation}
Then, the particle states are updated according to the motion equation: $\rvx_{t+1}^i= f(\rvx_t^i, \rvu_t, \boldsymbol{\eta}^i)$, where external control $\rvu_t$ is the same for all particles. 
At this stage, the information is partially lost due to the uncertainty $\boldsymbol{\eta}^i$ in the external control.

After that, we perform the measurement and obtain  $\rvz_{t+1}$.
Then, to estimate the uncertainty in the measured $\rvz_{t+1}$, we compute its conditional likelihood given state vector $\rvx_{t}$: 
\begin{equation}
    \mathcal{L}(\rvz_t|\rvx_t^i) = \prod_{j=1}^K p(\rvz^j_t|\rvx^i_t),
    \label{eq::likelihood}
\end{equation} 
where $K$ is a number of beacons, see Section~\ref{sec::problem_statement}. 
Here $p(\rvz^j_t|\rvx_t^i)$ is the probability to get measurement value $\rvz^j_t$ for beacon with index $j$ at time $t$ given the state $\rvx^i_t$.
In the general case, $p(\rvz|\rvx)$ is calculated from the distribution of the  measurement noise $\boldsymbol{\zeta}$.
In this study, the distribution of measurement noise is Gaussian, i.e. $\boldsymbol{\zeta}\sim\mathcal{N}(0, \mR)$, where $\mR = \sigma^2 \mI$. 
The predicted values of distances to $K$ beacons from the $i$-th particle are $\rvz_t^i = g(\rvx_t^{i}, \boldsymbol{\zeta})$.
At the same time, we perform measurement from the ground-truth object position to the $K$ nearest beacons and get values $z_1^*,\ldots,z_K^*$ stacked in the vector $\rvz_t^* \in \mathbb{R}^K$.
Therefore, we can estimate the likelihood~\eqref{eq::likelihood} with the following formula: 
$\mathcal{L}(\rvz_{t+1}|\rvx_{t+1}^i) = \frac{1}{(2\pi)^{K/2} \sqrt{\det{\mR}}} \exp \left(
% -\sum_{j=1}^{K}\frac{(z_{j, t+1}-\tilde{z}_{j, t+1}^i)^2}{2\sigma^2}
-\frac12 (\rvz^*_t - \rvz_t^{i})^\top \mR^{-1} (\rvz^*_t - \rvz_t^{i})
\right)$.
%, where $\rvz^*_t$ is the ground-truth measurement vector. 

Then, to compute the updated particles' weights $w^i_{t+1}$ we use likelihood~\eqref{eq::likelihood} and current weights: 
\begin{equation}
\label{eq::particle_weights}
w_{t+1}^i \sim \mathcal{L}(\rvz_{t+1}|\rvx_{t+1}^i) w_t^i,
\end{equation}
where $\sim$ indicates equality up to the normalization factor, i.e. $\sum_{i=1}^N w_{t+1}^i = 1$.
% Now, the updated weights $w_{t+1}^i$ are not normalized since resampling procedure affects them further.
From the updated weights $w^i_{t+1}$ the object state $\rvx_{t+1}$ can be estimated as expectation over the generated particles:
\begin{equation}
\begin{split}
& \rvx_{t+1} = \E [\vx_{t+1}] = \sum_{i=1}^N w^i_{t+1} \rvx_{t+1}^i.\\
% & \Var [\vx_t]=\sum_{i=1}^N w_t^i \rvx^{i2} - \left(\sum_{i=1}^N w_t^i \rvx_t^i\right)^2
\end{split}
\end{equation}
According to~\cite{proof} at a large enough number of particles and sufficiently large time steps, values $\rvx_t$ converge to known values $\rvx_t^*$. 
% However, at the finite number of particles and finite time steps even at noise-free measurements, this randomized approach may suffer from the impoverishment and degeneracy problems drawback~\cite{Frederick}. 

% Degeneracy problem occurs when due to measurements most particles have too low weights. To address it resampling is being applied. The idea of  resampling is to reduce the particles with low weights and duplicate those with higher weights so that the new weights of particles are more uniform. That may be achieved in many ways and there will discuss in ~\ref{resampling}. But the simplest way is to perform the sampling with repetitions of indices proportional to the particle weights.
% However, this may cause another problem -- impoverishment or loss of particles diversity. In this case, the majority of particles are concentrated within a small area.
% This may cause an incorrect representation of posterior distribution and finally incorrect state estimation. The situation will not be fixed regardless of the number of further time steps and measurements performed. 
% The problem of particle filter impoverishment is very broad and is a topic for separate research (see, for example~\cite{degeneracy}). 
As it was mentioned above, despite the simplicity of implementation and universality of this method, it has drawbacks such as degeneracy and impoverishment~\cite{fight}.
The formal measure of degeneracy is the number of effective particles:
\begin{equation}
    N_{eff} = \left\lfloor\frac{1}{\sum_{i=1}^{N} (w^i)^2}\right\rfloor,
\end{equation}
which varies from $1$ to $N$. 
The worst case is $N_{eff} = 1$, i.e. the single particle has non-zero weight. 
The best case is $N_{eff} = N$, i.e. all particles have the same values of weights $w^i = 1/N$. 
If $N_{eff}$ drops below a pre-defined threshold, for example, $N_{eff} \le N / 4$ or $N_{eff} \le  N / 2$, it indicates a degeneracy problem.
To address this problem, a resampling procedure is used.

The widespread resampling procedure is known as stochastic resampling~\cite{resamp1, Tutorial}. 
Formally it samples $N$ indices of particles from the multinomial distribution with repetitions.
The parameters of multinomial distributions are weights $w_t^i$.
After that, the $j$-th particle state is updated with the $i_j$ particle state, where $i_j$ is sampled index.
The described stochastic resampling procedure is summarized as follows:
\begin{equation}
\begin{split}
    &i_1,\ldots,i_N \sim Multinomial(w^1_{t+1},\ldots,w^N_{t+1})\\
    &\rvx^1_{t+1},\ldots,\rvx^N_{t+1} \leftarrow  \rvx^{i_1}_{t+1},\ldots,\rvx^{i_N}_{t+1}\\
    &w^i_{t+1}=\frac{1}{N}.
\end{split}
\label{eq::stoch_resampling}
\end{equation}
% The stochasticity can lead to reduce variability in the set of trial states and generate irrelevant states, which prevent proper convergence. 
% Moreover, as it is shown in ~\cite{towards} it is a biased procedure that could affect method performance. 
%One of the way to improve its performance is the replacement of random resampling by one of the deterministic methods~\cite{resamp1, towards}:
%\begin{equation}
%\begin{split}
%    &\rvx^1_t,\ldots,\rvx^N_t \leftarrow  %R(\rvx^{i_1}_t,\ldots,\rvx^{i_N}_t, w^1_t, \ldots,w^N_t  \mid %\boldsymbol{\evomega})\\
%    &w^i_t=\frac{1}{N},
%\end{split}
%\label{eq::stoch_resampling}
%\end{equation}
%where $R$ is a deterministic resampling function, which depends on parameter vector $\boldsymbol{\evomega}$.
% The other approach~\cite{curse_dim} uses the block particle filter approach that avoids impoverishment problems in particle filter in the case of high dimensional space. 

After the resampling of particles is done, few particles have the same states. 
Therefore, they represent the target probability density function poorly and particle filter may converge to the wrong state.
This problem is known as impoverishment~\cite{fight}. 
To improve the diversity of particles, random noise with sufficiently large variance is added to particle states~\cite{towards}.
The described particle filter method is summarized in Algorithm~\ref{alg::PF}.

% Add complete description of particle filter

\begin{algorithm}[!h]
\caption{Particle filter method with stochastic resampling.}
\label{alg::PF}
\begin{algorithmic}[1]
\REQUIRE Number of particles $N \geq 1$, number of beacons $K \geq 1$, motion and measurement equations $f$, $g$, covariance matrices $\mM$ and $\mR$ of motion and measurement noise.
\ENSURE predicted object states $\rvx_{t}$ for $t=1,\ldots,T$
\FOR{$i = 1$ to $N$}
    \STATE Initialize weights $w^i \gets 1/N$
    \STATE Initialize particle state $\rvx_1^i \gets \mathcal{U}(\rvx_{\min},\rvx_{\max})$
 \ENDFOR
\FOR{$t=1$ to $T$}
    \FOR{$i=1$ to $N$}
        \STATE Generate $\boldsymbol{\eta}^i \sim \mathcal{N}(0, \mM)$, $\boldsymbol{\zeta}^i \sim \mathcal{N}(0, \mR)$
        \STATE Update state $\rvx_{t+1}^i = f(\rvx_t^i,\rvu_t, \boldsymbol{\eta}^i)$ and perform measurements $\rvz^i_{t+1} = g(\rvx_{t+1}^i, \boldsymbol{\zeta}^i)$
        \STATE Compute likelihood $\mathcal{L}(\rvz_{t+1}|\rvx_{t+1}^i) = \frac{1}{(2\pi)^{K/2} \sqrt{\det{\mR}}}\exp \left( -\frac12 (\rvz^*_t - \rvz_t^{i})^\top \mR^{-1} (\rvz^*_t - \rvz_t^{i}) \right)$
        \STATE Update particle's weight $w_{t+1}^i \sim \mathcal{L}(\rvz_{t+1}|\rvx_{t+1}^i) w_t^i$
        %\STATE $i_1,\ldots,i_N \sim Multinomial(w^1_{t+1},\ldots,w^N_{t+1})$
        %\STATE $\rvx^1_{t+1},\ldots,\rvx^N_{t+1} \gets  \rvx^{i_1}_{t+1},\ldots,\rvx^{i_N}_{t+1}$
        %\STATE $w^i_t \gets\frac{1}{N}$
    \ENDFOR
    \STATE Perform resampling of the updated states according to~\eqref{eq::stoch_resampling}
    \STATE $\rvx_{t+1} \gets \sum_{i=1}^N w_{t+1}^i \rvx_{t+1}^i$
\ENDFOR
\end{algorithmic}
\end{algorithm}


The impoverishment problem might be especially severe in highly symmetric environments. 
In this case, consistent state representation in several similar subparts of the environment requires the number of particles proportional to the number of subparts, see Figure~\ref{fig::sym_env_example}~(left). 
Moreover, if the environment is highly symmetric, the filtered trajectory may coincide with the ground truth but only up to the symmetric subpart, see Figure~\ref{fig::symmetric_env_traj}, where the predicted trajectory is computed by the particle filter method.
\begin{figure}[!h]
    \centering
\includegraphics[width=0.5\textwidth]
    {{pics/_27_27.pdf}}
    \caption{Visualization of the symmetric test environment with the ground truth and filtered trajectories marked with circles and crosses, respectively. The measurements are performed from the five nearest beacons. The final positions are shown with a big circle and cross. Note that the final filtered points may differ from the corresponding points in the ground-truth trajectory, though they are identical up to symmetry. Number of particles $N=10000$.}
    %\textbf{MKF} algorithm is used, noise level $4\Sigma$
    \label{fig::symmetric_env_traj}
\end{figure}
To reduce the number of particles necessary for convergence in a symmetric environment, we suggest equipping every particle with equations~\eqref{eq::EKF} and~\eqref{eq::Qt_EKF} from the Extended Kalman filter. 
A detailed description of the proposed Multiparticle Kalman filter (MKF) is given in the next section.

% From our practical experience, we provide a list of a few simple recipes to address the impoverishment problem of the particle filter method:
% \begin{itemize}
%     \item increase number of particles $N$: the larger number of particles is, the higher probability to get particles with near-optimal states and thus ensure faster convergence; 
%     \item increase motion noise $\eta$: the motion noise should be sufficiently large, otherwise, the exploration ability of resampled particles is limited.
%     \item reduce the precision of measurements, i.e. eliminate instability in the computation of likelihood~(\ref{eq::likelihood}) 
%     \item tracking likelihood values, i.e. if the likelihood of measurement for all particles is sufficiently small, then regenerate trial states and restart method. 
% \end{itemize}

%The pseudocode of the particle filter is shown in Algorithm~\ref{al::PFAlg}. 
%Here $g(\rvz^i, \rvx^i)$ is a measurement function, $\Delta$ is a state noise $\eta$ amplitude, $T$ is the total time.

% The example of particle filter performance is shown in the same world, which is used for Kalman filter~(\ref{fig::pf_world1}). 
% Here there is one beacon only and the particle initial state is known within $1$ cell.
% Therefore, particles are randomly generated in this cell. 
% The filter converges for all such conditions in a few time steps. 

% \begin{figure}[!ht]
%     \centering
%     \begin{subfigure}{0.45\textwidth}
%     \centering
%     \includegraphics[scale=0.33]
%     {pics/SPF_1.pdf}
%     \caption{PF step 1}
%     \end{subfigure}
%     ~
%     \begin{subfigure}{0.45\textwidth}
%     \centering
%     \includegraphics[scale=0.33]
%     {pics/SPF_2.pdf}
%     \caption{PF step 2}
%     \end{subfigure}
%     \\
%     \begin{subfigure}{0.45\textwidth}
%     \centering
%     \includegraphics[scale=0.33]
%     {pics/SPF_25.pdf}
%     \caption{PF step 25}
%     \end{subfigure}
%     ~
%     \begin{subfigure}{0.45\textwidth}
%     \centering
%     \includegraphics[scale=0.33]
%     {pics/SPF_40.pdf}
%     \caption{PF step 40}
%     \end{subfigure}
% \caption{Particle filter performance example}
% \label{fig::pf_world1}
% \end{figure}




\section{Multiparticle Kalman filter}

We propose a new natural combination of Kalman and particle filters. 
The goal of such a combination is to develop an algorithm, which deals with the unknown initial states like particle filter and accelerates convergence to an optimal state like the Kalman filter.
Since the symmetric environments are especially challenging for the particle filter, we will use them to illustrate the accelerated convergence of the proposed method.

The idea of the proposed method is to generate a set of trial state vectors (particles) $\rvx_{t=0}^{i} \in \mathbb{R}^d, \; i=1,\ldots,N$ with corresponding weights~$w_{t=0}^i = 1/N$, and also covariance matrices $\mP_{t=0}^i$.
Again, vectors $\rvx_t^{i}$ and weights $w_t^i$ play the same role as in the particle filter.

Then, we use the formulas from the Extended Kalman filter for every generated particle in parallel. 
In particular, equations~\eqref{eq::Kalman_x_update},~\eqref{eq::Kalman_P_update},~\eqref{eq::EKF},~\eqref{eq::Qt_EKF} are used to compute the updated state vector of every particle $\rvx^i_{t+1}$.
After that, to update weights $w_t^{i+1}$ we compute distances to the beacons from the updated states $\rvz_{t+1}^{i} = g(\rvx_{t+1}^{i+})$.
% \begin{itemize}
% \item Predict stage
% \begin{equation}
% \begin{split}
% &\rvx_{t+1}^{i-} = f(\rvx^i_t, \rvu_{t+1})\\
% &\mP_{t+1}^{i-} = \mF^i \mP^i_{t} \mF^{i\top} + \mQ^i_{t+1}\\
% &\mF^i = \left.\frac{\partial f(\rvx, \rvu_{t+1})}{\partial \rvx}\right|_{\rvx = \rvx^i_t}
% \end{split}
% \label{eq::kpf_predict}
% \end{equation}
% \item Update stage
%  \begin{equation}
%  \begin{split}
%  &\tilde{\rvz}^i_{t+1} = g(\rvx_{t+1}^{i-})\\ 
%  &\mH^i_{t+1} = \left.\frac{\partial g(\rvx)}{\partial \rvx}\right|_{\rvx = \rvx^{i-}_{t+1}}\\
%  &\mK^i_{t+1} = \mP_{t+1}^{i-} \mH_{t+1}^{i\top} (\mH_{t+1}^i \mP_{t+1}^{i-} \mH_{t+1}^{i\top} + \mR)^{-1}\\
%  &\rvx^{i+}_{t+1} = \rvx_{t+1}^{i-} + \mK^i_{t+1} (\rvz^*_{t+1} - \rvz^i_{t+1})\\
%  &\mP^{i+}_{t+1} = (\mI - \mK^i_{t+1} \mH^i_{t+1})\mP_{t+1}^{i-}
%  \end{split}
%  \label{eq::kpf_update}
%  \end{equation} 
% \end{itemize}
% The only difference with the Kalman filter algorithm are the outputs: new states and covariance matrices $\rvx^{i+}, \mP^{i+}_{t+1}$, which are not final for this time step yet.
% The particles are already moved, see~\eqref{eq::kpf_predict}, so that the particle filter predict stage is not needed. 
% In~(\ref{eq::kpf_update}), we perform update for every particle separately, and now we need update weights $w^i_t$. 
% For that we need an updated prediction of the distances to the beacons. 
% They can be found from new states $\rvz_{t+1}^{i} = g(\rvx_{t+1}^{i+})$.
Then the weights are updated in the same way as in the particle filter based on the likelihood 
\[
\mathcal{L}(\rvz_{t+1}|\rvx_{t+1}^{i+}) = \prod_{j=1}^K p(\rvz^j_{t+1}|\rvx^{i+}_{t+1}),
\]
where $K$ is the number of beacons, which are used to measure distances.
% In particular, if the measurement noise $\boldsymbol{\zeta} \sim \mathcal{N}(0, \mR)$, then likelihood has the following form: $\mathcal{L}(\rvz_{t+1}|\rvx_{t+1}^i)\sim\exp \left( -\frac12 (\rvz^*_t - \rvz_t^{i})^\top \mR^{-1} (\rvz^*_t - \rvz_t^{i}) \right)$.
Now, the updated weights are computed as follows:
\begin{equation}
w_{t+1}^i \sim \mathcal{L}^i(\rvz_{t+1}|\rvx_{t+1}^{i+}) w_t^i.
\end{equation}

To prevent the degeneracy phenomenon, we perform resampling of the obtained particle states.
We follow the resampling procedure used in the particle filter~\eqref{eq::stoch_resampling} and modify it to resample not only particle states $\rvx^{i}_t$ but also corresponding covariance matrices $\mP^i_{t+1}$. 
The modified resampling procedure used in the proposed method is summarized in~\eqref{eq::kpf_resampling}:
% This is performed in the same way as a stochastic resampling in the standard particle filter. Every particle is resampled with the probability proportional to its weight and then inherits from the original particle both state and covariance matrix:
\begin{equation}
\begin{split}
    &i_1,\ldots,i_N \sim Multinomial(w^1_{t+1},\ldots,w^N_{t+1})\\
    &\rvx^{1+}_{t+1},\ldots,\rvx^{N+}_{t+1} \leftarrow  \rvx^{i_1+}_{t+1},\ldots,\rvx^{i_N+}_{t+1}\\
    &\mP^{1}_{t+1},\ldots,\mP^{N}_{t+1} \leftarrow  \mP^{i_1+}_{t+1},\ldots,\mP^{i_N+}_{t+1}\\
    &w^i_{t+1}=\frac{1}{N}.
\end{split}
\label{eq::kpf_resampling}
\end{equation}
% These equations~(\ref{eq::kpf_resampling}) give the resulting values for the covariance matrices $\mP^i_{t+1}$ and weights~$w_{t+1}^i$. 
Now to address the impoverishment issue, we add noise to the resampled states: 
\begin{equation}
    %\rvx^{i}_{t+1} = \rvx^{i+}_{t+1} + \boldsymbol{\varepsilon}^i,
    \rvx^{i}_{t+1} = f(\rvx^{i+}_{t+1}, \boldsymbol{0}, \boldsymbol{\eta}^i),
\end{equation}
where $\boldsymbol{\eta}^i \sim \mathcal{N}(0, \mM)$, where $\mM$ is given covariance matrix.

% is computed according to~\eqref{eq::Qt_EKF} separately for every $i$-th particle.

Note that the per-iteration computational complexity of the presented algorithm is higher than the complexity of the standard particle filter. 
Indeed, in addition to the particle filter steps, the proposed method processes $d \times d$ matrices for every particle.
Therefore, to process every particle memory complexity is increased up to $O(d^2)$ caused by storing matrices, and computational complexity is increased up to $O(d^3)$ operations caused by matrix multiplications\footnote{This complexity can be slightly reduced to $O(d^{2.32})$ according to~\cite{duan2022faster}}. 
Despite this, we observe in the experiments (see Section~\ref{sec::experiments}) that it ensures faster convergence since requires fewer particles.




% ($O(Nd^3)$ and memory consumption $O(Nd^2)$ than the standard particle filter algorithm $O(N\, d)$, where $N$, and $d$ are particles number and problem dimension. 
% However despite of that as it will be shown later it may converge faster as usually requires significantly less particles number.

% Thus, we present the novel combination of extended Kalman filter and particle filter. In Section~\ref{sec::experiments} we demonstrate its superior performance compared to competitors.
% Below we discuss the proposed method and highlight its difference from the extended Kalman particle filter which relies on the same idea.  

% \paragraph{Discussion}
% The suggested method relies more on a particle filter part, while the Kalman step is used only for coordinates correction in an intermediate step. 

% The example of Kalman particle filter performance is shown in Figure~\ref{fig::kpf_world1}. 
% Here there is one beacon and the particle's initial location is known within a $1$ cell, the uncertainty is included in the covariance matrix. 
% The filter converges for all such conditions in a few time steps. 

% \begin{figure}[!ht]
%     \centering
%     \begin{subfigure}{0.45\textwidth}
%     \centering
%     \includegraphics[scale=0.33]
%     {pics/KPF_1.pdf}
%     \caption{KPF step 1}
%     \end{subfigure}
%     ~
%     \begin{subfigure}{0.45\textwidth}
%     \centering
%     \includegraphics[scale=0.33]
%     {pics/KPF_2.pdf}
%     \caption{KPF step 2}
%     \end{subfigure}
%     \\
%     \begin{subfigure}{0.45\textwidth}
%     \centering
%     \includegraphics[scale=0.33]
%     {pics/KPF_25.pdf}
%     \caption{KPF step 25}
%     \end{subfigure}
%     ~
%     \begin{subfigure}{0.45\textwidth}
%     \centering
%     \includegraphics[scale=0.33]
%     {pics/KPF_40.pdf}
%     \caption{KPF step 40}
%     \end{subfigure}
% \caption{Kalman particle filter performance example}
% \label{fig::kpf_world1}
% \end{figure}

% The other approach, combining particle and Kalman filter exist. 
% In~\cite{EPF_EKF} extended Kalman particle filter is described.
% This method is similar to the proposed one but differs in the weights update scheme.
% It updates weights according to the following equation
% \[
% w^i_{t+1} = \frac{p^i_{t+1} \mathcal{L}^i(\rvz_{t+1}|\rvx_{t+1}^{i+}) w^i_{t}}{d^i_{t+1}},
% \]
% where $d^i_{t+1} = \mathbb{P}(\boldsymbol{\xi} = \rvx^i_{t+1}-\rvx^{i+}_{t+1})$, where $\boldsymbol{\xi} \sim \mathcal{N}(0, \mP^i_{t+1})$ and $p^i_{t+1} = \mathbb{P}(\boldsymbol{\delta} = \rvx^i_{t+1}-\rvx^{i+}_{t+1})$, where $\boldsymbol{\delta} \sim \mathcal{N}(0, \mQ_{t+1}^i)$.
% The motivation of this scheme is to compensate small weights corresponding to the relevant particles with multipliers $d^i_{t+1}$ and $p^i_{t+1}$.
% The natural drawback of this scheme is that large weights are assigned to irrelevant particles due to possible inconsistency between covariance matrices $\mP_{t+1}^i$ and likelihood $\mathcal{L}^i$.
% Therefore, the convergence becomes slower.

% The next issue of this approach is in computing $p^i_{t+1}$ since for some environments corresponding covariance matrix can become singular.
% For example, if particles move on the plane, then states contain $3$ dimensions: two coordinates $x, y$ and angle $\phi$.
% Then, the process covariance matrix $\mQ_{t+1}^i$ is of the size $3 \times 3$.
% At the same time, noise $\boldsymbol{\eta}=(\eta_r, \eta_{\phi})$ independently affects angle and speed, and therefore has only two independent elements.
% Thus, the corresponding covariance matrix is a low rank, which leads to numerical instabilities.
% Although regularization can help, it requires more tuning of the hyperparameters.

% In~\cite{SHARIATI201932} the same method as we propose here has been applied for the identification of the nonlinear dynamic of an autonomous underwater vehicle. However, the covariance matrices are taken from experimental data fitting and they are diagonal. In the present paper we demonstrate applicability of the method for the classical object motion on the plane. It is shown that the classical filter performance depends significantly on the noise level chosen. At the same time, the proposed \textbf{PFKU} has higher immunity to the noise level and thus provides more robust results in case if the internal system noise is unknown. Moreover, the Kalman particle filter performs better at the same computational resources, and is much more accurate then particle filter at the same number of particles.




% Environments
% Assumptions on matrices
% Hyperparameters
% 
\section{Computational experiments}
\label{sec::experiments}

In this section, we present the description of the experiments for comparison of the proposed multiparticle Kalman filter (MKF) with a standard particle filter in symmetric and non-symmetric environments.
We exclude the Kalman-based filters from our comparison since they require the object's initial state, which is unknown according to our assumption.
Every experiment is conducted on a single NVIDIA Tesla V100 GPU.


\subsection{Test environments}

To evaluate the performance of the proposed method and compare it with the particle filter we use different types of environments.
In particular, we consider the symmetric environments with an increasing number of symmetrical subparts and call them \textit{world $10\times 10$}, \textit{World $18\times 18$}, and \textit{WORLD $27\times 27$}.
The number of beacons is also increased with the number of symmetrical subparts which makes filtering of object states more challenging.
The considered symmetric environments are shown in Figure~\ref{fig::symmetric_envs}.

\begin{figure}[!ht]
\centering
%\begin{adjustbox}{minipage=\linewidth,scale=0.7}
    \begin{subfigure}{0.3\textwidth}
    \centering
    \includegraphics[width=\textwidth]
    %{pics/world_fig_10.pdf}
    {pics/_w10.pdf}
    \caption{\textit{world $10\times 10$}}
    \end{subfigure}
    ~
    \begin{subfigure}{0.3\textwidth}
    \centering
    \includegraphics[width=\textwidth]
    {pics/_w18.pdf}
    \caption{\textit{World $18\times 18$}}
    \end{subfigure}
    ~
    \begin{subfigure}{0.3\textwidth}
    \centering
    \includegraphics[width=\textwidth]
    {pics/_w27.pdf}
    \caption{\textit{WORLD $27\times 27$}}
    \end{subfigure}
\caption{Visualization of symmetrical test environments. Beacons and obstacles are shown as black crosses and grey blocks, respectively.}
\label{fig::symmetric_envs}
%\end{adjustbox}
\end{figure}

To illustrate the effect of symmetry in an environment, we remove a subpart in every environment described above such that they become nonsymmetric.
The nonsymmetric analogs of the aforementioned symmetrical environments are shown in Figure~\ref{fig::nonsym_envs} and we call them \textit{n-world $10\times 10$}, \textit{n-World $18\times 18$}, and \textit{n-WORLD $27\times 27$}, respectively.
\begin{figure}[!ht]
\centering
%\begin{adjustbox}{minipage=\linewidth,scale=0.7}
    \begin{subfigure}{0.3\textwidth}
    \centering
    \includegraphics[width=\textwidth]
    %{pics/world_fig_10.pdf}
    {pics/_w10_1.pdf}
    \caption{\textit{n-world $10\times 10$}}
    \end{subfigure}
    ~
    \begin{subfigure}{0.3\textwidth}
    \centering
    \includegraphics[width=\textwidth]
    {pics/_w18_1.pdf}
    \caption{\textit{n-World $18\times 18$}}
    \end{subfigure}
    ~
    \begin{subfigure}{0.3\textwidth}
    \centering
    \includegraphics[width=\textwidth]
    {pics/_w27_1.pdf}
    \caption{\textit{n-WORLD $27\times 27$}}
    \end{subfigure}
\caption{Visualization of \emph{nonsymmetric} test environments. Beacons and obstacles are shown as black crosses and grey blocks, respectively.}
\label{fig::nonsym_envs}
\end{figure}

One more test environment \emph{Labyrinth} is considered in~\cite{towards} and is shown in Figure~\ref{fig::labyrinth}. 
Compared to the previous environments, \emph{Labyrinth} environment has a lower degree of symmetry and allows solving localization problems accurately and comparatively fast in terms of the number of time steps for the convergence.
Therefore, we also compare the considered methods in this relatively friendly environment. 
% As it was mentioned above some types of the environments may cause poor performance of filters:
% \begin{itemize}
%     \item Environments with few symmetrically located obstacles and beacons.
%     \item Environments with multiple beacons and high noise, so that multiple state may reproduce the same measurements results within noise level.
%     \item Environments with unknown object's initial state. 
% \end{itemize}
% The first part of the experiments is done with three environments with increasing complexity~\cite{PFRNN}: world $10\times 10$, World $18\times 18$, and WORLD $27\times 27$ (see \ref{fig::worlds}). 
% The second part contains same worlds with "slightly broken" symmetry, i.e. one section is removed and few less number of landmarks. It should demonstrate essential improvement of the accuracy of predictions (\ref{fig::worlds}). 
% The last test world is \emph{Labyrinth}, similar to what has been used in~\cite{towards}. In contrary to very symmetric worlds it allows precise coordinate determination and is interesting to find out how many particles are required to both algorithms (particle filter and \textbf{PFKU}) to converge.
% The Kalman filter will not be considered as it shows poor performance at unknown initial conditions. 
% The standard particle filter and \textbf{PFKU} will be compared at different parameters of noise values and number of particles.
\begin{figure}[!ht]
\centering
\includegraphics[width=10cm]{pics/_w_lab.pdf}
\caption{Visualization of the \emph{Labyrinth} environment. Beacons and obstacles are shown as black crosses and grey blocks, respectively.
}
\label{fig::labyrinth}
\end{figure}

% Here start the text from section Test environments

% \begin{figure}[!ht]
% \centering
%     \begin{subfigure}[t]{0.3\textwidth}
%     \centering
%     \includegraphics[width=\textwidth]
%     %{pics/world_fig_10.pdf}
%     {pics/_10_10.pdf}
%     \caption{\textit{world $10\times 10$}}
%     \end{subfigure}
%     ~
%     \begin{subfigure}[t]{0.3\textwidth}
%     \centering
%     \includegraphics[width=\textwidth]
%     {pics/_18_18.pdf}
%     \caption{\textit{World $18\times 18$}}
%     \end{subfigure}
%     ~
%     \begin{subfigure}[t]{0.3\textwidth}
%     \centering
%     \includegraphics[width=\textwidth]
%     {pics/_27_27.pdf}
%     \caption{\textit{WORLD $27\times 27$}}
%     \end{subfigure}
% \caption{Visualization of test environments with 5 nearest beacons. In every world, one of the true (circle) and predicted (cross) trajectories are shown. The last positions are shown by a big circle and cross. It is important to note that in very symmetric environments the last points in the predicted trajectory may differ from the corresponding points in a true trajectory, though they are identical up to symmetry.}
% \label{fig::worlds_tracks}
% \end{figure}


% \begin{figure}[!ht]
% \centering
% %\begin{adjustbox}{minipage=\linewidth,scale=0.7}
%     \begin{subfigure}[t]{0.31\textwidth}
%     \centering
%     \includegraphics[width=\textwidth]
%     {pics/_10_10_1.pdf}
%     \caption{\textit{n-world $10\times 10$}}
%     \end{subfigure}
%     ~
%     \begin{subfigure}[t]{0.31\textwidth}
%     \centering
%     \includegraphics[width=\textwidth]
%     {pics/_18_18_1.pdf}
%     \caption{\textit{n-World $18\times 18$}}
%     \end{subfigure}
%     ~
%     \begin{subfigure}[t]{0.32\textwidth}
%     \centering
%     \includegraphics[width=\textwidth]
%     {pics/_27_27_1.pdf}
%     \caption{\textit{n-WORLD $27\times 27$}}
%     \end{subfigure}
% \caption{Visualization of test environments with 5 nearest beacons. Note, that the predicted trajectories normally converge to true.}
% \label{fig::worlds_tracks1}
% %\end{adjustbox}
% \end{figure}


% \begin{figure}[!ht]
% \centering
% \includegraphics[width=0.5\textwidth]{pics/34_14.pdf}
% \caption{\emph{Labyrinth} environment. At a sufficiently large number of particles predicted trajectory converges to a true one normally in a few steps.}
% %\label{fig::Labyrinth1}
% \end{figure}



% \subsection{Assumptions on matrices}
% The simple calculations provide the following values for measurement covariance ($\mR$), process covariance ($\mQ$), and initial position matrix ($\mP_{t=0})$:

% \begin{align*}
% \mR=\sigma d^2 {\bf 1},
% \end{align*}

% \begin{align*}
% \mQ = \begin{bmatrix}
%        \cos\phi^2\,\sigma u^2 + u^2\,\sin\phi^2\, \sigma \phi^2  & (\sigma u^2 + u^2 \sigma\phi^2) \sin2\phi/2 & -u\,\sin\phi\,\sigma\phi^2 \\
%        ( \sigma u^2 + u^2 \sigma\phi^2) \sin2\phi/2 & \sin\phi^2\,\sigma u^2  + u^2 \, \cos\phi^2\sigma \phi^2 & u \cos\phi\,\sigma\phi^2\\
%        -u\,\sin\phi\,\sigma\phi^2 & u\, \cos\phi\,\sigma\phi^2 & \sigma\phi^2
% \end{bmatrix},
% \end{align*}

% \begin{align*}
% \mP_0=
% \begin{bmatrix}
%            \sigma x_0^2 & 0 & 0\\
%            0 & \sigma y_0^2  & 0\\
%            0 & 0 & \sigma \phi_0^2.
%     \end{bmatrix}
% \end{align*}

% The matrices are build on the simple assumption of known motion model and the only hyperparameters are $\sigma d^2, \sigma u^2, \sigma \phi^2$, $\sigma x_0^2, \sigma y_0^2, \sigma \phi_0^2$.


\subsection{Trajectory generation procedure}

To generate trajectories for an object in the considered environments, the following procedure is used.
The object starts motion from a random location without obstacles and has a randomly chosen direction.
In every step, it moves according to external and known velocity $u \in [0, 0.5]$, and the direction remains the same from the previous step within the noise.
If this movement leads to a collision with an obstacle, the object's direction is changed randomly to avoid collision with another obstacle.
To simulate engine noise, the velocity~$u$ is perturbed by $\eta_r \sim \mathcal{U}[-0.02, 0.02]$ and the direction $\phi$ is also perturbed by $\eta_{\phi} \sim 2\pi \alpha$, where $\alpha \sim \mathcal{U}[-0.01, 0.01]$.
The direction perturbation simulates the uncertainty in the object control system.
In the considered environments, we set the number of time steps in every trajectory $T=100$. 
To make a fair comparison of the considered methods, we generate $10000$ trajectories for testing.
% The externally controlled direction remains the same as before unless object hits the wall; in this case the direction is randomly chosen to ensure that the next location is free. 
% But the true direction is also noisy with noise level $\delta\phi \sim 2\pi\, \mathcal{U}[-0.01, 0.01]$.

\paragraph{Filter input data.}
The input data for every filter algorithm consists of the following parts: ground-truth measurement vector $\rvz^*_t$, external control vector $\rvu_t = [u_{r}, \Delta \phi_t]$.
An element $\Delta \phi_t \neq 0$ only if the collision with obstacle appears.
Note that, the object movement is affected by the additional noise $\eta_{u}$ and $\eta_{\phi}$.
Therefore, the filtering methods have to identify the ground-truth object state including its position and heading.

% The motion is represented by the intended distance to move ahead a new intended heading~\cite{Tutorial}. 
% The resulting motion differs from the input due to noise over distance and angle as described above. 
% The measurements are the distances to few (five in our examples) nearest beacons, which are also noisy with the noise distributed as $\delta\phi \sim U[-0.1, 0.1]$.

% The task is to predict coordinates and headings at any time moment. In all environments, $5$ nearest beacons data are used for the measurements.

%%%%%%%%%%%%%%%%%%%%%%%%%%%%%%%%%%%%%%%%%%%%%%%%%
\subsection{Hyperparameters}
\label{sec:hyperparams}
Before one runs the proposed filtering method, the following hyperparameters have to be set: covariance matrices of motion and measurement noise $\mM$ and $\mR$, and initial state covariance matrix $\mP_{t=0}$. 
These hyperparameters significantly affect the performance of the method and have to be tuned carefully.
According to~\cite{towards}, a filtering approach based on particles works better if the variances of motion and measurement noise exceed the ground-truth variances in measurement devices and the object control system.
However, the excessively large variance of motion and measurement noise may lead to slow convergence.
In the experiments, we use the following ground-truth variances in measurement devices and the object control system: $\mR_0 = \sigma_{d0}^2 \mI$, where $\sigma_{d0}^2 = 0.01$ and $\mM_0 = \mathrm{diag}(\sigma^2_{r0}, \sigma^2_{\phi 0})$, $\sigma_{r0} = 0.02$, $\sigma_{\phi 0} = 0.01 \cdot 2\pi$.
At the same time, to study the robustness of the proposed approach to different scales of motion and measurement variance, we consider the following settings.
The first setup is $\mM = \mM_0$ and $\mR = 2\mR_0$.
The second setup is $\mM = 4\mM_0$ and $\mR = 2\mR_0$.
The initial state covariance matrix $\mP_{t=0} = \mathrm{diag}(\sigma_{x}^2, \sigma_{y}^2, \sigma_{\phi}^2)$ and $\sigma_{x}^2=\sigma_{y}^2=\frac{w\,h}{12}$ and $\sigma_{\phi}^2=\frac{(2\pi)^2}{12}$, where $w, h$ are width and height of environment and factor $1/12$ is used to model the uniform distribution of particle states in the environment.

\subsection{Comparison of multiparticle Kalman filter with particle filter}

In this section, we provide a comparison of the considered filtering methods in the aforementioned environments.
However, before presenting the comparison results we introduce the upper bound of the MSE error that indicates the poor quality of the state estimate.
The na\"ive filtering method just generates uniformly random states of the object in the given environment.
Therefore, we can estimate MSE between randomly generated states and the ground-truth states for the considered environments as $MSE_{random}=\frac{w^2+h^2}{6}$, where~$w$ and~$h$ are the width and height of the environment, respectively.
If a filtering method generates states such that MSE between them and the ground-truth states is larger than $MSE_{random}$, we consider such filtering completely useless and show this threshold in the plots below. 

% An upper bound of MSE is such value, that if the model gives a larger value then it is useless. 
% For example, MSE is given by the model that generates random states.
% It is easy to guess that for rectangular environments with scales $w$ and $h$, the upper boundary for MSE is the mean of the square distance between two random points. 
% It is $MSE_{random}=\frac{w^2+h^2}{6}$.

In the experiments, we compare the proposed filtering method with the classical particle filter (see Algorithm~\ref{alg::PF}).
Kalman filter and its modifications are excluded from the comparison since they do not perform well without knowledge of the initial state. 
% Sometimes, large absolute value elements of a covariance matrix $\mP$ may help to converge, however, it is not always true.
Moreover, the Gaussian distribution of state vectors assumes an elliptical uncertainty region that is irrelevant to the considered environments.
We use both MSE~\eqref{eq::mse_def} and FSE~\eqref{eq::fse_def} loss functions.
Also, we compare the robustness of the considered methods to the scale of motion covariance matrix $\mM$.
In particular, the first setting is $\mM = \mM_0$, which is further referred to as $\Sigma$ in legends.
The second setting is $\mM = 4\mM_0$, which is further referred to as $4\Sigma$ in legends.
Here we denote by $\mM_0$ the ground-truth covariance matrix of the noise from the object control system.
The measurement noise covariance matrix in both settings is $\mR = 2\mR_0$, where $\mR_0$ is the covariance matrix of the noise from a measurement device.
The values for $\mM_0$ and $\mR_0$ used in our simulations are given in Section~\ref{sec:hyperparams}.

The comparison results of the proposed filtering method with the particle filter in terms of the MSE~\eqref{eq::mse_def} are shown in Figures~\ref{fig::MSE} and~\ref{fig::MSE1} for symmetric and non-symmetric environments, respectively.
Both plots show that the proposed filtering method (MKF) requires fewer particles to achieve smaller values of MSE in the considered environments.
Also, one can observe that the filtering process in non-symmetric environments is more accurate and robust than in symmetric environments.
The smaller value of MSE indicates higher accuracy and the robustness is illustrated by the number of particles necessary for the convergence of MSE.
Also, these plots show that the proposed method is less sensitive to the estimate of motion noise than the particle filter.

\begin{figure}[!ht]
    %\centering
    \begin{subfigure}{0.3\textwidth}
    %\centering
    \includegraphics[width=\textwidth]
    {pics/world10_performance.pdf}
    \caption{\textit{world $10 \times 10$}}
    \end{subfigure}
    ~
    \begin{subfigure}{0.3\textwidth}
    % \centering
    \includegraphics[width=\textwidth]
    {pics/world18_performance.pdf}
    \caption{\textit{World $18 \times 18$}}
    \end{subfigure}
    ~
    \begin{subfigure}{0.3\textwidth}
    \centering
    \includegraphics[width=\textwidth]
    {pics/world27_performance.pdf}
    \caption{\textit{WORLD $27 \times 27$}}
    \end{subfigure}
\caption{Dependence of MSE on the number of particles in three symmetric environments. Multiparticle Kalman filter (MKF) demonstrates more accurate filtering of states and requires fewer particles for MSE convergence compared to the particle filter (PF). Our method is also less sensitive to the estimate of the motion noise than the particle filter.}
\label{fig::MSE}
\end{figure}

\begin{figure}[!ht]
    %\centering
    \begin{subfigure}{0.3\textwidth}
    %\centering
    \includegraphics[width=\textwidth]
    {pics/world10_1_performance.pdf}
    \caption{\textit{n-world $10 \times 10$}}
    \end{subfigure}
    ~
    \begin{subfigure}{0.3\textwidth}
    %\centering
    \includegraphics[width=\textwidth]
    {pics/world18_1_performance.pdf}
    \caption{\textit{n-World $18 \times 18$}}
    \end{subfigure}
~
    \begin{subfigure}{0.3\textwidth}
    \centering
    \includegraphics[width=\textwidth]
    {pics/world27_1_performance.pdf}
    \caption{\textit{n-WORLD $27 \times 27$}}
    \end{subfigure}
\caption{Dependence of MSE on the number of particles in three non-symmetric environments. 
Multiparticle Kalman filter (MKF) demonstrates more accurate filtering of states and requires fewer particles for MSE convergence compared to the particle filter (PF). 
Our method is also less sensitive to the estimate of the motion noise than the particle filter.}
\label{fig::MSE1}
\end{figure}

Additional experiments are carried out to evaluate the considered filtering methods in terms of the final state error function~\eqref{eq::fse_def}.
The comparison results are shown in Figures~\ref{fig::Fin} and~\ref{fig::Fin1} for symmetric and non-symmetric environments, respectively.
The final states are computed after $100$ time steps in the considered environments.
These plots demonstrate the same trends that are observed in the analysis of MSE dependence on the number of particles presented in Figures~\ref{fig::MSE} and~\ref{fig::MSE1}.

\begin{figure}[!ht]
    %\centering
    \begin{subfigure}{0.3\textwidth}
    %\centering
    \includegraphics[width=\textwidth]
    {pics/world10fin_performance.pdf}
    \caption{\textit{world $10 \times 10$}}
    \end{subfigure}
    ~
    \begin{subfigure}{0.3\textwidth}
    %\centering
    \includegraphics[width=\textwidth]
    {pics/world18fin_performance.pdf}
    \caption{\textit{World $18 \times 18$}}
    \end{subfigure}
~
    \begin{subfigure}{0.3\textwidth}
    \centering
    \includegraphics[width=\textwidth]
    {pics/world27fin_performance.pdf}
    \caption{\textit{WORLD $27 \times 27$}}
    \end{subfigure}
\caption{Dependence of the final error loss function (FSE) on the number of particles used in the PF and MKF in the considered symmetric environments. MKF provides more accurate filtering of the states and requires fewer particles for convergence of FSE. Our filtering method is also less sensitive to the estimate of the motion noise than the particle filter.}
\label{fig::Fin}
\end{figure}


\begin{figure}[!ht]
    %\centering
    \begin{subfigure}{0.3\textwidth}
    %\centering
    \includegraphics[width=\textwidth]
    {pics/world10_1fin_performance.pdf}
    \caption{\textit{n-world $10 \times 10$}}
    \end{subfigure}
    ~
    \begin{subfigure}{0.3\textwidth}
    %\centering
    \includegraphics[width=\textwidth]
    {pics/world18_1fin_performance.pdf}
    \caption{\textit{n-World $18 \times 18$}}
    \end{subfigure}
~
    \begin{subfigure}{0.31\textwidth}
    \centering
    \includegraphics[width=\textwidth]
    {pics/world27_1fin_performance.pdf}
    \caption{\textit{n-WORLD $27 \times 27$}}
    \end{subfigure}
\caption{Dependence of the final error loss function (FSE) on the number of particles used in the PF and MKF in the considered non-symmetric environments. 
MKF provides more accurate filtering of the states and requires fewer particles for the convergence of FSE. Our method is also less sensitive to the estimate of the motion noise than the particle filter.}
\label{fig::Fin1}
\end{figure}

Last but not least comparison of the particle filter and the multiparticle Kalman filter is performed in the \empty{Labyrinth} environment (see Figure~\ref{fig::labyrinth}).
Figure~\ref{fig::lab_mse_fse} shows that the proposed filtering method outperforms the particle filter in terms of both MSE and FSE quality criteria.
Also, we again observe the smaller number of particles required for the convergence of both loss functions.
The proposed method is more robust with respect to the motion noise level than the particle filter, which is aligned with previous results.

\begin{figure}[!ht]
%\centering
\begin{subfigure}{0.45\textwidth}
\includegraphics[width=\textwidth]
    {pics/world_lab_performance.pdf}
\end{subfigure}
    ~
    \begin{subfigure}{0.45\textwidth}
\includegraphics[width=\textwidth]
    {pics/world_labfin_performance.pdf}
\end{subfigure}
    \caption{Dependence of MSE (left) and FSE (right) values on the number of particles in the \emph{Labyrinth} environment. Our method (MKF) is also less sensitive to the estimate of the motion noise than the particle filter.}
    \label{fig::lab_mse_fse}
\end{figure}

\paragraph{Variance analysis.}
To make the previous plots clear, we do not provide confidence intervals there.
To fill this gap in the reporting comparison results, we summarize the FSE values and the corresponding variance in Table~\ref{tab::variance_comparison}.
This table shows that the MKF provides a more accurate and less variable estimation of the final state for both symmetric and nonsymmetric environments.
This gain is observed uniformly with respect to the considered range of the number of particles.

\begin{table}[!h]
%\fontsize{7pt}{7pt}\selectfont
    \centering
    \caption{FSE values and variance comparison of particle filter (PF) and the proposed multiparticle Kalman filter (MKF). Standard deviation is given in braces near the corresponding mean FSE value. In these simulations, we use $\mM = 4\mM_0$, which is equal to $4\Sigma$ setting.}
    \begin{adjustbox}{width=\columnwidth,center}
    \begin{tabular}{ccccccccccc}
    \toprule
    Number of particles & \multicolumn{2}{c}{$N=100$} & \multicolumn{2}{c}{$N=500$} & \multicolumn{2}{c}{$N=1000$} & \multicolumn{2}{c}{$N=4000$} & \multicolumn{2}{c}{$N=10000$}\\
    \midrule
    Environment &  PF & MKF &  PF & MKF & PF & MKF & PF & MKF & PF & MKF \\
    \cmidrule(lr){1-1} \cmidrule(lr){2-3} \cmidrule(lr){4-5} \cmidrule(lr){6-7} \cmidrule(lr){8-9} \cmidrule(lr){10-11}
        \textit{world 10} & 5.37 (3.53) & 4.86 (3.76) & 5.13 (3.91) & 4.71 (2.78) & 4.93 (3.87) & 4.65 (2.44) & 4.85 (3.31) & 4.57 (2.05) & 4.8 (2.82) & 4.55 (1.98) \\
        \textit{n-world 10} & 3.07 (3.63) & 0.24 (1.26) & 1.91 (3.25) & 0.03 (0.07) & 1.40 (2.88) & 0.43 (1.67) & 0.04 (0.21) & 0.11 (0.70) &0.03 (0.02) & 0.03 (0.02) \\
        \textit{World 18} & 10.10 (6.49) & 9.22 (7.87) & 9.40 (7.72) & 8.41 (7.22) & 8.89 (8.20) & 8.38 (5.94) & 8.47 (7.79) & 8.34 (3.62) & 8.38 (6.64) & 8.34 (3.18) \\
        \textit{n-World 18} & 6.56 (6.94) & 3.87 (6.37) & 4.91 (6.89) & 1.70 (4.54) & 3.91 (6.46) & 1.30 (3.75) & 1.83 (4.86) & 1.13 (2.99) & 1.24 (3.77) & 1.15 (2.94) \\
        \textit{WORLD 27} & 13.41 (7.50) & 11.14 (7.62)  & 11.49 (7.48) & 7.26 (6.33) & 10.19 (7.21) & 5.96 (5.55) & 7.13 (6.27) & 4.89 (4.42) & 5.71 (5.47) & 4.80 (4.23) \\
        \textit{n-WORLD 27} & 11.84 (8.60) & 9.35 (8.41) & 9.97 (8.32) & 5.77 (6.86) & 8.68 (8.03) & 4.34 (5.89) & 5.66 (6.75) & 2.91 (4.39) & 3.95 (5.56) & 2.83 (4.25) \\
        \emph{Labyrinth} & 10.45 (10.11) & 1.83 (5.65) & 6.84 (9.53) & 0.06 (0.68) & 4.83 (8.49) & 0.03 (0.02) & 1.00 (4.23) & 0.03 (0.02) & 0.11 (1.17) & 0.03 (0.02) \\
        \bottomrule
    \end{tabular}
    \end{adjustbox}
    \label{tab::variance_comparison}
\end{table}


% As it is seen in the Figures the particle filter performance essentially depends both on the number of particles and the noise level added. Kalman particle filter requires much less particles to show comparable results both for MSE and final points. Moreover, it shows weak dependence on the noise level, which is a significant circumstance as it doesn't need an additional exploration stage of hyperparameter matching.

% The results for best cases (i.e. $4\Sigma$) for particle and Kalman particle  filters at various particles number are used as summarized in Table \ref{tab::variance_comparison}. In all cases, \textbf{PFKU} outperforms the particle filter in terms of mean final error and its standard deviation. 

\paragraph{Runtime comparison.}
In the previous sections, we demonstrate the performance of the proposed filtering method in terms of the required number of particles for convergence of MSE and FSE and the smaller variance of these quantities compared to the particle filter.
Here, we provide the runtime comparison of the proposed filtering method and the particle filter.
In this experiment, we simulate $10000$  trajectories in the considered environments and provide the total runtime of such a simulation. 
Since the runtime of both compared methods significantly depends on the used number of particles, we consider $2000$ and $5000$ particles in the particle filter simulations and report the resulting FSE values.
Then, we tune the number of particles in the MKF such that the resulting FSE values are the same or slightly smaller than the corresponding FSE in the particle filter simulations.
The measured runtime, FSE, and the numbers of particles are shown in Table~\ref{tab::time_comparison}.
From this Table follows that the proposed multiparticle Kalman filter is typically 3-4 times faster than the particle filter.
This observation indicates that the gain from the reduction of the number of particles dominates the increasing per-iteration complexity of the proposed method.

% Here computations were performed with particle filter at $2000$ and $5000$ number of particles. 
% Later the \textbf{PFKU} has been run at different numbers of particles till the final coordinate reaches the same error or lower. 
% The required number of particles and computational times are compared. 
% The new algorithm needs at least an order of magnitude fewer particles than the standard particle filter and is typical $3-4$ times faster, which compensates for its extra complexity, which has been discussed earlier.


\begin{table}[!h]
%\fontsize{7pt}{7pt}\selectfont
    \centering
    \caption{Comparison of the total runtime of particle filter (PF) and the proposed multiparticle Kalman filter (MKF) to simulate 10000 trajectories in the considered environments. The number of particles required for the MKF is set such that it achieves the same or slightly smaller FSE compared to values from PF simulations.}
    \begin{adjustbox}{width=0.8\columnwidth,center}
    \begin{tabular}{ccccccc}
    \toprule
     & \multicolumn{3}{c}{PF} & \multicolumn{3}{c}{MKF} \\
    \cmidrule(lr){2-4} \cmidrule(lr){5-7} 
    Environment & \# particles & FSE & Time, s & \# particles & FSE & Time, s\\
    \midrule
        \textit{world 10} & 2000 & 5.03 & 156 & 100 & 4.92 & 58 \\
        \textit{n-world 10} & 2000 & 0.87 & 151 & 100 & 0.80 & 46 \\
        \textit{World 18} & 2000 & 9.26 & 186 & 200 & 8.96 & 80 \\
        \textit{n-World 18} & 2000 & 4.87 & 168 & 100 & 4.61 & 50 \\
        \textit{WORLD 27} & 2000 & 11.0 & 204 & 200 & 10.62 & 86 \\
        \textit{n-WORLD 27} & 2000 & 9.58 & 211 & 150 & 9.45 & 70 \\
        \emph{Labyrinth} & 2000 & 4.8 & 139 & 100 & 3.02 & 48 \\
        \midrule
        \textit{world 10} & 5000 & 4.72 & 368 & 150 & 4.84 & 63 \\
        \textit{n-world 10} & 5000 & 0.30 & 334 & 250 & 0.25 & 90 \\
        \textit{World 18} & 5000 & 8.81 & 415 & 200 & 0.20 & 75 \\
        \textit{n-World 18} & 5000 & 3.49 & 412 & 300 & 3.20 & 111 \\
        \textit{WORLD 27} & 5000 & 9.50 & 489 & 400 & 9.00 & 153 \\
        \textit{n-WORLD 27} & 5000 & 7.94 & 473 & 400 & 7.54 & 151 \\
        \emph{Labyrinth} & 5000 & 2.69 & 323 & 150 & 1.86 & 61 \\
        \bottomrule
    \end{tabular}
    \end{adjustbox}
    \label{tab::time_comparison}
\end{table}

% \begin{figure}[!ht]
% %\centering
%     \hspace{2.8cm}
%     \includegraphics[width=7cm, clip]
%     {pics/world_labfin_performance.pdf}
%     \caption{Dependence of an absolute error for a final coordinate on the number of particles in the \emph{Labyrinth}}
%     \label{fig::final_error_lab}
% \end{figure}


% \begin{table}[!h]
% %\fontsize{7pt}{7pt}\selectfont
%     \centering
%     \caption{Comparison of particle filter (PF) and the proposed Kalman particle filter (MKF). Final point mean and standard deviation are shown. $4\Sigma$ level is used for twice bigger noise.}
%     \begin{adjustbox}{width=\columnwidth,center}
%     \begin{tabular}{c|cc|cc|cc|cc|cc}
%     Number of particles & \multicolumn{2}{c|}{$N=100$} & \multicolumn{2}{c|}{$N=500$} & \multicolumn{2}{c|}{$N=1000$} & \multicolumn{2}{c|}{$N=4000$} & \multicolumn{2}{c}{$N=10000$}\\
%     \hline
%     Environment & PF & MKF & PF & MKF & PF & MKF & PF & MKF & PF & MKF \\
%     \hline
%         world 10 & 5.37 (3.53) & 4.86 (3.76) & 5.13 (3.91) & 4.71 (2.78) & 4.93 (3.87) & 4.65 (2.44) & 4.85 (3.31) & 4.57 (2.05) & 4.8 (2.82) & \textbf{4.55 (1.98)} \\
%         World 18 & 10.10 (6.49) & 9.22 (7.87) & 9.40 (7.72) & 8.41 (7.22) & 8.89 (8.20) & 8.38 (5.94) & 8.47 (7.79) & 8.34 (3.62) & 8.38 (6.64) & \textbf{8.34 (3.18)} \\
%         WORLD 27 & 13.41 (7.50) & 11.14 (7.62)  & 11.49 (7.48) & 7.26 (6.33) & 10.19 (7.21) & 5.96 (5.55) & 7.13 (6.27) & 4.89 (4.42) & 5.71 (5.47) & \textbf{4.80 (4.23)} \\
%         world 10 a & 2.17 (3.55) & 0.24 (1.26) & 0.71 (2.21) & 0.03 (0.07) & 0.33 (1.49) & 0.03 (0.02) & 0.04 (0.21) & 0.03 (0.02) &0.03 (0.02) & \textbf{0.03 (0.02)} \\
%         World 18 a & 6.56 (6.94) & 3.87 (6.37) & 4.91 (6.89) & 1.70 (4.54) & 3.91 (6.46) & 1.30 (3.75) & 1.83 (4.86) & 1.13 (2.99) & 1.24 (3.77) & \textbf{1.15 (2.94)} \\
%         WORLD 27 a & 11.84 (8.60) & 9.35 (8.41) & 9.97 (8.32) & 5.77 (6.86) & 8.68 (8.03) & 4.34 (5.89) & 5.66 (6.75) & 2.91 (4.39) & 3.95 (5.56) & \textbf{2.83 (4.25)} \\
%         \emph{Labyrinth} & 10.45 (10.11) & 1.83 (5.65) & 6.84 (9.53) & 0.06 (0.68) & 4.83 (8.49) & 0.03 (0.02) & 1.00 (4.23) & 0.03 (0.02) & 0.11 (1.17) & \textbf{0.03 (0.02)} 
%     \end{tabular}
%     \end{adjustbox}
%     \label{tab::variance_comparison}
% \end{table}

% As it is seen in Figures the particle filter performance essentially depends both on the number of particles and noise level added. Kalman particle filter requires much less particles to show comparable results both for MSE and final points. Moreover, it shows weak dependence on the noise level, which is very important circumstance as doesn't need additional exploration stage of hyperparameter matching.

% The results for best cases (i.e. $4\Sigma$) for particle and Kalman particle  filters at various particles number used as summarized in Table \ref{tab::variance_comparison}. In all cases Kalman particle outperforms particle filter in terms of mean final error and its standard deviation. This algorithm needs about an order of magnitude less particles than standard particle filter, which compensates its extra complexity, which has been discussed earlier.

\section{Conclusion}

% This method is based on the proper combination of the Extended Kalman filter and particle filter.
% It successfully addresses the problem of uncertainty in the initial state of the object and outperforms the particle filter in both symmetric and non-symmetric environments.
% Our numerical experiments demonstrate that the proposed filtering method requires fewer particles for convergence in terms of both MSE and FSE quality criteria. Although every iteration of the proposed method is more costly compared with the particle filter, the runtime of the proposed method is significantly smaller due to the smaller number of particles necessary for convergence to the same error rates. Also, we illustrate that the proposed approach is more robust to the level of measurement noise than the classical particle filter

% Also, we illustrate that the proposed approach is more robust to the level of measurement noise than the classical particle filter.

In the presented study, we consider the object localization problem with an unknown initial state in both symmetric and non-symmetric environments. 
We demonstrate that the standard particle filter algorithm performs poorly in highly symmetrical environments. 
We propose a novel multiparticle Kalman filter (MKF) based on the combination of the extended Kalman filter and particle filter. 
The MKF successfully addresses the problem of uncertainty in the object's initial state and outperforms the particle filter in all considered environments.
Our numerical experiments demonstrate that MKF requires fewer particles to achieve convergence in terms of both MSE and FSE quality criteria. 
Although every iteration of the proposed method is more costly compared to the particle filter, MKF converges faster since fewer particles are required to achieve the same error rates.
Also, we show that MKF is more robust to the level of measurement noise than the classical particle filter.

The authors would like to acknowledge funding through the SNSF Sinergia grant called "Robust Deep Density Models for High-Energy Particle Physics and Solar Flare Analysis (RODEM)" with funding number CRSII$5\_193716$, the SNSF project grant 200020\_212127 called "At the two upgrade frontiers: machine learning and the ITk Pixel detector", and the Alexander von Humboldt foundation Feodor Lynen fellowship programme.

\bibliographystyle{unsrt}
\bibliography{lib}

\end{document}