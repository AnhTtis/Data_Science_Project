%%%%%%%%% TITLE
%\section*{Supplementary Material for \texttt{NAISR}}
\section*{\centering{Supplementary Material for \texttt{NAISR}}}


% Remove page # from the first page of camera-ready.
\ificcvfinal\thispagestyle{empty}\fi
This supplementary material provides additional model illustrations, implementation details, experimental results, and ablation studies.

\section{Method}
In this section, we investigate how the displacement field and inverse consistency loss of Eq.~\eqref{eq.inv_loss} work. $\mathbf{z}$ is omitted in the derivations because it stays the same in every equation.
 
\subsection{Velocity Field}
\label{sec.vec_field}

\begin{figure}[htbp!]
    \centering
    \includegraphics[width=1.0\columnwidth]{figs/vec_field.pdf}
    \caption{Velocity field. The velocity field $\mathbf{v}_i(\mathbf{p}(t), t)$ carries point $\mathbf{p}$ from the green shape (at state $c_1$) to the cyan shape (at state $c_2$) and then to the purple shape (at state $c_3$) along the red trajectory. In our case, $c_1$, $c_2$ and $c_3$ are different values of a covariate and $\mathbf{p}(c_1)$, $\mathbf{p}(c_2)$ and $\mathbf{p}(c_3)$ are points on shapes whose covariate values are $c_1$, $c_2$ and $c_3$ respectively. }
    \label{fig.vec_field}
\end{figure}


 Suppose there is a velocity field $\mathbf{v}_i(\cdot)$ for each covariate $c_i$ which can carry points to shapes with different values of the covariate $c_i$. As shown in Fig~\ref{fig.vec_field}, the velocity field $\mathbf{v}_i(\cdot)$ carries the point $\mathbf{p}(c_1)$ from the green shape at $c_1$ to $\mathbf{p}(c_2)$ on the cyan shape at $c_2$ and then to $\mathbf{p}(c_3)$ on the purple shape at $c_3$.  Suppose the displacement field from source space to template space controlled by covariate $c_i$ can be represented by the integration of the velocity field $\mathbf{v}_i(\mathbf{p}(t), t)$, which is determined by the current point coordinate $\mathbf{p}(t)$ and current time $t$, as
\begin{equation}
\begin{aligned}
\mathbf{d}_i &= \mathbf{d}_{c_i \rightarrow 0}(\mathbf{p}(c_i))= \int_{c_i}^{0}\mathbf{v}_i(\mathbf{p}(t), t)\,d t\,,
\end{aligned}
\label{eq.vect_field}
\end{equation}

where $\mathbf{d}_i=\mathbf{d}_{c_i \rightarrow 0}(\mathbf{p}(c_i))$ represents the displacement to transform point $\mathbf{p}(c_i)$ from source space ($c_i$) to template space ($0$). As illustrated Sec.~\ref{subsec.model_f}, $g_i(\cdot)$ captures the displacement $\mathbf{d_i}$ between spaces (e.g., source space to template space) by

\begin{equation}
\mathbf{d}_i =  \mathbf{d}_{c_i \rightarrow 0}(\mathbf{p}(c_i))= g_{c_i \rightarrow 0}(\mathbf{p}(c_i)) = g_i(\mathbf{p}(c_i), c_i, 0)\,.
\end{equation}


The inverse displacement from $\mathbf{p}(0)$ on template space ($0$) to source space ($c_i$) is

\begin{equation}
\begin{aligned}
\mathbf{d}_i^{inv} & = g_{0 \rightarrow c_i}(\mathbf{p}(0)) = g_i(\mathbf{p}(0), 0, c_i) \\
&= \int_{0}^{c_i}\mathbf{v}_i(\mathbf{p}(t), t)\,d t\,.
\end{aligned}
\end{equation}

Therefore, a velocity field should allow for 
\begin{equation}
\mathbf{d}_i =-\mathbf{d}_i^{inv} = -g_i(\mathbf{p}(0), 0, c_i)\,.
\end{equation}

Considering that $\mathbf{p}(0)=\mathbf{d}_i + \mathbf{p}(c_i)$,  we can use the inverse consistency loss of Eq.~\eqref{eq.inv_loss_supp} for the displacements $\{g_i(\cdot)\}$ to encourage $g_i$ to compatible with the requirements for integrating an underlying velocity field $\mathbf{v_i}$.

\begin{equation}
\mathcal{L}_{\mathrm{inv}}(\{g_i\}) = \lambda_7\int_{\mathcal{S}} \Sigma_{i=1}^{N} \|(-\mathbf{d}_i) - g_{0 \rightarrow c_i}(\mathbf{p+\mathbf{d}_i})\| d \mathbf{p} \,.
\label{eq.inv_loss_supp}
\end{equation}

\section{Dataset}
Our dataset includes 229 cross-sectional observations (where a patient was only imaged once) and 34 longitudinal observations.  Tab.~\ref{tab.num_of_observations} shows the distribution of the number of observations across patients. Most patients in the dataset only have one observation; only 22 patients have $\geq 3$ observation times. Tab.~\ref{tab.dataset_exp_paitent} shows the airway shapes and the demographic information of an example patient. Tab.~\ref{tab.vis_demo_dataset} and Tab.~\ref{tab.vis_demo_training} show the shapes and demographic information at different age percentiles for the whole data set and training set respectively. The time span of the training set is close but slightly smaller than the whole dataset. We observe that the time span of our longitudinal data for each patient is far shorter than the time span across the entire dataset, indicating the challenge of capturing spatiotemporal dependencies over large time spans between shapes while accounting for individual differences between patients.

\begin{table}[htbp!]
\begin{tabular}{|l |r |r |r |r |r |r |r |r |r |r} 
\hline
    \# observations     &  1 & 2 & 3 & 4 & 5 & 6 & 7 & 9 & 11\\ 
\hline\hline
    \# patients   &  229 & 12 &  6 & 8 & 3 & 2 & 1 & 1 & 1 \\
\hline
\end{tabular}
\caption{Number of patients for a given number of observations. For example, the 1st column indicates that there are 229 patients who were only observed once. }
\label{tab.num_of_observations}
\end{table}

\begin{table*}
\centering

\begin{tabular}{p{0.06\textwidth}p{0.08\textwidth}p{0.08\textwidth}p{0.08\textwidth}p{0.08\textwidth}p{0.08\textwidth}p{0.08\textwidth}p{0.08\textwidth}p{0.08\textwidth}p{0.08\textwidth}}
\toprule
\# time &      0&      1  &      2  &      3  &      4  & 5 & 6 & 7 & 8 \\
\midrule
$\{\mathcal{S}^t\}$&
\includegraphics[width=0.2\columnwidth]{figs/dataset_case/0.png} &  
\includegraphics[width=0.2\columnwidth]{figs/dataset_case/1.png} &  
\includegraphics[width=0.2\columnwidth]{figs/dataset_case/2.png} &  
\includegraphics[width=0.2\columnwidth]{figs/dataset_case/3.png} &  
\includegraphics[width=0.2\columnwidth]{figs/dataset_case/4.png} & 
\includegraphics[width=0.2\columnwidth]{figs/dataset_case/5.png} &  
\includegraphics[width=0.2\columnwidth]{figs/dataset_case/6.png} &  
\includegraphics[width=0.2\columnwidth]{figs/dataset_case/7.png} &  
\includegraphics[width=0.2\columnwidth]{figs/dataset_case/8.png} \\

\bottomrule
\end{tabular}
\begin{tabular}{p{0.06\textwidth}p{0.08\textwidth}p{0.08\textwidth}p{0.08\textwidth}p{0.08\textwidth}p{0.08\textwidth}p{0.08\textwidth}p{0.08\textwidth}p{0.08\textwidth}p{0.08\textwidth}}
\toprule
\# time &      0&      1  &      2  &      3  &      4  & 5 & 6 & 7 & 8 \\
\midrule

age         &    84.00 &    85.00 &    87.00 &    91.0 &    95.00 &    98.00 &   101.00 &   104.00 &   120.00 \\
weight      &    20.40 &    20.40 &    21.00 &    21.9 &    22.80 &    22.90 &    23.50 &    24.90 &    28.50 \\
sex         &     male &    male  &    male &    male &    male &    male  &    male  &     male &    male  \\
m-vol &    30.07 &    32.18 &    48.95 &    33.8 &    44.87 &    42.29 &    28.42 &    40.92 &    61.36 \\
\bottomrule

\end{tabular}
\caption{Visualization and demographic information of observations of a patient in our 3D airway shape dataset. Shapes are plotted with their covariates (age/month, weight/kg, sex) printed in the table. M-vol (measured volume) is the volume ($cm^3$) of the gold standard shapes based on the actual imaging. }
\label{tab.dataset_exp_paitent}
\end{table*}



\begin{table*}

\resizebox{\textwidth}{!}{%
\begin{tabular}{p{0.05\textwidth}p{0.06\textwidth}p{0.06\textwidth}p{0.06\textwidth}p{0.06\textwidth}p{0.06\textwidth}p{0.06\textwidth}p{0.06\textwidth}p{0.06\textwidth}p{0.06\textwidth}p{0.06\textwidth}p{0.06\textwidth}}
\toprule
P- &      0&      10  &      20  &      30  &      40  & 50 & 60 & 70 & 80 & 90 & 100\\
\midrule
$\{\mathcal{S}^t\}$&
\includegraphics[width=0.2\columnwidth]{figs/all_perct/0.png} &  
\includegraphics[width=0.2\columnwidth]{figs/all_perct/1.png} &  
\includegraphics[width=0.2\columnwidth]{figs/all_perct/2.png} &  
\includegraphics[width=0.2\columnwidth]{figs/all_perct/3.png} &  
\includegraphics[width=0.2\columnwidth]{figs/all_perct/4.png} & 
\includegraphics[width=0.2\columnwidth]{figs/all_perct/5.png} &  
\includegraphics[width=0.2\columnwidth]{figs/all_perct/6.png} &  
\includegraphics[width=0.2\columnwidth]{figs/all_perct/7.png} &  
\includegraphics[width=0.2\columnwidth]{figs/all_perct/8.png} &  
\includegraphics[width=0.2\columnwidth]{figs/all_perct/9.png} &
\includegraphics[width=0.2\columnwidth]{figs/all_perct/10.png} \\

\bottomrule
\end{tabular}}

\begin{tabular}{p{0.05\textwidth}p{0.06\textwidth}p{0.06\textwidth}p{0.06\textwidth}p{0.06\textwidth}p{0.06\textwidth}p{0.06\textwidth}p{0.06\textwidth}p{0.06\textwidth}p{0.06\textwidth}p{0.06\textwidth}p{0.06\textwidth}}
\toprule
P- &      0&      10  &      20  &      30  &      40  & 50 & 60 & 70 & 80 & 90 & 100\\
\midrule

age         &     1.00 &    23.00 &    55.00 &    71.00 &    89.00 &   111.00 &   129.00 &   161.00 &   179.00 &   199.00 &   233.00 \\
weight      &     3.90 &    14.20 &    20.10 &    21.80 &    19.70 &    32.85 &    44.80 &    21.30 &    59.00 &    93.90 &    75.60 \\
sex         &     male &     male &     female &     female &     male &     male &     male &     female &     female &     female &     male \\
m-vol &     4.56 &    16.84 &    29.53 &    28.91 &    27.31 &    70.90 &    71.23 &    43.34 &    78.63 &   102.35 &   113.84 \\

\bottomrule

\end{tabular}
\caption{Visualization and demographic information of our 3D airway shape dataset. Shapes of $\{0, 10, 20, 30, 40, 50, 60, 70, 80, 90, 100\}$-th age percentiles 
 are plotted with their covariates (age/month, weight/kg, sex) printed in the table. M-vol (measured volume) is the volume ($cm^3$) of the gold standard shapes based on the actual imaging. }
\label{tab.vis_demo_dataset}
\end{table*}



\begin{table*}
\resizebox{\textwidth}{!}{%
\begin{tabular}{p{0.05\textwidth}p{0.06\textwidth}p{0.06\textwidth}p{0.06\textwidth}p{0.06\textwidth}p{0.06\textwidth}p{0.06\textwidth}p{0.06\textwidth}p{0.06\textwidth}p{0.06\textwidth}p{0.06\textwidth}p{0.06\textwidth}}
\toprule
P- &      0&      10  &      20  &      30  &      40  & 50 & 60 & 70 & 80 & 90 & 100\\
\midrule
$\{\mathcal{S}^t\}$&
\includegraphics[width=0.2\columnwidth]{figs/train_perct/0.png} &  
\includegraphics[width=0.2\columnwidth]{figs/train_perct/1.png} &  
\includegraphics[width=0.2\columnwidth]{figs/train_perct/2.png} &  
\includegraphics[width=0.2\columnwidth]{figs/train_perct/3.png} &  
\includegraphics[width=0.2\columnwidth]{figs/train_perct/4.png} & 
\includegraphics[width=0.2\columnwidth]{figs/train_perct/5.png} &  
\includegraphics[width=0.2\columnwidth]{figs/train_perct/6.png} &  
\includegraphics[width=0.2\columnwidth]{figs/train_perct/7.png} &  
\includegraphics[width=0.2\columnwidth]{figs/train_perct/8.png} &  
\includegraphics[width=0.2\columnwidth]{figs/train_perct/9.png} &
\includegraphics[width=0.2\columnwidth]{figs/train_perct/10.png} \\

\bottomrule
\end{tabular}}

\begin{tabular}{p{0.05\textwidth}p{0.06\textwidth}p{0.06\textwidth}p{0.06\textwidth}p{0.06\textwidth}p{0.06\textwidth}p{0.06\textwidth}p{0.06\textwidth}p{0.06\textwidth}p{0.06\textwidth}p{0.06\textwidth}p{0.06\textwidth}}
\toprule
P- &      0&      10  &      20  &      30  &      40  & 50 & 60 & 70 & 80 & 90 & 100\\
\midrule

age         &     3.00 &    22.00 &    55.00 &    69.00 &    88.00 &   106.00 &   124.00 &   160.00 &   179.00 &   199.00 &   230.00 \\
weight      &     6.75 &    12.05 &    20.10 &    13.30 &    26.70 &    32.70 &    42.00 &    48.80 &    59.00 &    93.90 &    70.10 \\
sex         &     male &     female &     female &     female &     male &     male &     male &     male &     female &     female &     male \\
m-vol &     9.40 &    18.80 &    29.53 &    37.96 &    33.98 &    73.66 &    40.72 &   133.84 &    78.63 &   102.35 &   123.31 \\


\bottomrule

\end{tabular}
\caption{Visualization and demographic information of the training set. The time span of the training set is close but slightly smaller than for the entire dataset. Shapes of $\{0, 10, 20, 30, 40, 50, 60, 70, 80, 90, 100\}$-th age percentiles of age are plotted with their covariates (age/month, weight/kg, sex) printed in the table. M-vol (measured volume) is the volume ($cm^3$) of the gold standard shapes based on the actual imaging.}
\label{tab.vis_demo_training}
\end{table*}


\section{Experiments}

Sec.~\ref{subsec:implementation_details} describes implementation details. Sec.~\ref{subsec:shape_reconstruction},~\ref{subsec:shape_transfer}, and~\ref{subsec:disentangled_shape_evolution} show additional experimental results for shape reconstruction, shape transfer, and disentangled shape evolution respectively.

\subsection{Implementation Details}
\label{subsec:implementation_details}

Each subnetwork, including the template network $\mathcal{T}$ and the displacement networks $\{g_i\}$, are all parameterized with an 8-layer MLP using $sine$ activations. The network parameter initialization follows SIREN~\cite{sitzmann2020siren}. There are 256 hidden units in each layer. The architecture of $\mathcal{T}$ follows SIREN~\cite{sitzmann2020siren}. The architecture of the $\{g_i\}$ follows DeepSDF~\cite{park2019deepsdf}, in which a skip connection is used to concatenate the input of $(\mathbf{p}, c_i)$ to the input of the middle layer. We use a latent code $\mathbf{z}$ of dimension 256 ($L=256$).

For each training iteration, the number of points sampled from each shape is 750 ($N=750$), of which 500 are on-surface points ($N_{on} = 500$) and the others are off-surface points ($N_{off} = 250$). We train \texttt{NAISR} for 3000 epochs using Adam~\cite{kingma2014adam} with a learning rate \(5e-5\) and batch size of 64. Also, we jointly optimize the latent code $\mathbf{z}$ with \texttt{NAISR} using Adam~\cite{kingma2014adam} with a learning rate of \(1e-3\). 

During training, the hyperparameters in $\mathcal{L}_{reconstruction}$ and  $\mathcal{L}_{zero\_padding}$ are the same. In $\mathcal{L}_{reconstruction}$, $\lambda_1 =\lambda_5 = \frac{100}{N}$; $\lambda_2 = \frac{300}{N_{on}}$; $\lambda_3 = \frac{100}{N_{on}}$, $\lambda_4 = \frac{1000}{N_{off}}$. For $\mathcal{L}_{lat}$, $\lambda_6 = \frac{2e-4}{L}$; $\sigma=0.01$ (following DeepSDF~\cite{park2019deepsdf}). For $\mathcal{L}_{inv}$,  $\lambda_7 = \frac{10}{N_{on}}$. During inference, the latent codes are optimized for 800 iterations with a learning rate of $5e-3$. $\mathcal{L}_{zero\_padding}$ is not used when the covariates are given as input as described by Eq.~\eqref{eq.infer_z}.


\subsection{Shape Reconstruction}
\label{subsec:shape_reconstruction}
Tab.~\ref{tab.recons_rst_train} shows the reconstruction performance of different methods for shapes in the training set. We can see that DeepSDF~\cite{park2019deepsdf} and A-SDF~\cite{mu2021asdf} perform much better than other methods on the training set. The deformable methods perform similarly on the training set to the testing set. The non-deformable methods perform better on the training set than on the testing set. For example, A-SDF~\cite{mu2021asdf} performs exceptionally well on the training set; however, it cannot generalize to the testing set. This is because our dataset is more challenging than the RBO dataset~\cite{martin2019rbo} used in A-SDF~\cite{mu2021asdf}, in which sufficient longitudinal data is available for the network to learn the interactions between geometries and articulations.

\begin{table*}[!htbp]
\resizebox{\textwidth}{!}{%
\begin{tabular}{lcrrrrrrr}
\toprule
Methods & Deformable & \# params &  CD mean &  CD median &  EMD mean &  EMD median &  HD mean &  HD median \\
\midrule
DeepSDF~\cite{park2019deepsdf} & \XSolidBrush & 2.24M &    0.021 &      0.017 &     0.941 &       0.907 &    6.503 &      5.710 \\
  A-SDF~\cite{mu2021asdf} &  \XSolidBrush & 1.98M &  0.013 &      0.011 &     0.804 &       0.768 &    4.962 &      4.353 \\
    DIT~\cite{zheng2021DIT} & \Checkmark& 3.70M &    0.056 &      0.038 &     1.162 &       1.095 &   10.426 &      9.422 \\
 NDF\cite{simeonovdu2021ndf} & \Checkmark & 0.34M  & 0.072 &      0.053 &     1.406 &       1.335 &   11.277 &     10.116 \\
 NAISR &   \Checkmark & 2.33M  &0.053 &      0.042 &     1.288 &       1.241 &    9.925 &      8.940 \\
\bottomrule
\end{tabular}}
\caption{Comparison of reconstruction performance on the training set using different methods. The deformable methods perform similarly on the training set to the testing set. The non-deformable methods perform better on the training set than on the testing set (see Tab.~\ref{tab.recons} for shape reconstruction results on the testing set).}
\label{tab.recons_rst_train}
\end{table*}




\subsection{Shape Transfer}
\label{subsec:shape_transfer}

\begin{table*}[!htbp]
\begin{tabular}{cc}
\begin{minipage}[Another example of shape transfer using \textbf{NAISR with Cov.}]{.4\linewidth}
\begin{tabular}{lllll}%{p{0.05\textwidth}p{0.05\textwidth}p{0.08\textwidth}p{0.08\textwidth}p{0.08\textwidth}}
\toprule
\# time &      0&      1  &      2  &      3  \\
\midrule
$\{\mathcal{S}^t\}$&
\includegraphics[width=0.2\columnwidth]{figs/1391_transport/0.png} &  
\includegraphics[width=0.2\columnwidth]{figs/1391_transport/1.png} & 
\includegraphics[width=0.2\columnwidth]{figs/1391_transport/2.png}& 
\includegraphics[width=0.2\columnwidth]{figs/1391_transport/3.png}\\
\bottomrule

\toprule
\# time &      0&      1  &      2  &      3   \\
\midrule
weight      &  20.30 &  20.60 &  27.50 &  27.90 \\
age         &  62.00 &  65.00 &  89.00 &  94.00 \\
sex         &   male &    male &    male&    male \\
p-vol &  61.20 &  61.91 &  69.14 &  70.66 \\
m-vol&  49.61 &  58.87 &  55.91 &  62.07 \\
\bottomrule
\end{tabular}
\label{tab.another_transfer_wo}
\end{minipage} 

\hspace{0.6in}

\begin{minipage}[a]{.4\linewidth}
%\hfill\includegraphics[width=2.\columnwidth]{figs/1181_comp.png}
%{c{0.05\textwidth}c{0.06\textwidth}c{0.05\textwidth}c{0.05\textwidth}c{0.05\textwidth}c{0.05\textwidth}c{0.05\textwidth}c{0.05\textwidth}c{0.05\textwidth}c{0.05\textwidth}c{0.05\textwidth}c{0.05\textwidth}}
\begin{tabular}{lllll}%{p{0.05\textwidth}p{0.08\textwidth}p{0.08\textwidth}p{0.08\textwidth}p{0.08\textwidth}}
\toprule
\# time &      0&      1  &      2  &      3  \\
\midrule
$\{\mathcal{S}^t\}$&
\includegraphics[width=0.2\columnwidth]{figs/1391_general_transport/0.png} &  
\includegraphics[width=0.2\columnwidth]{figs/1391_general_transport/1.png} & 
\includegraphics[width=0.2\columnwidth]{figs/1391_general_transport/2.png}& 
\includegraphics[width=0.2\columnwidth]{figs/1391_general_transport/3.png}\\
\bottomrule

\toprule
\# time &      0&      1  &      2  &      3   \\
\midrule
weight      &  20.30 &  20.60 &  27.50 &  27.90 \\
age         &  62.00 &  65.00 &  89.00 &  94.00 \\
sex         &   male &    male &    male&    male \\
p-vol &  61.94 &  62.31 &  66.17 &  66.80 \\
m-vol&  49.61 &  58.87 &  55.91 &  62.07 \\
\bottomrule
\end{tabular}
%\caption{Another example of shape transfer using \textbf{NAISR w/o Cov.}}

\end{minipage}

\end{tabular}
\caption{Another shape transfer example. Left: using \textbf{NAISR with Cov.} based on Eq.~\eqref{eq.infer_z}; right: using \textbf{NAISR w/o Cov.} based on Eq.~\eqref{eq.infer_cz}. Blue: gold standard shapes; red: transferred shapes with \textbf{NAISR}. The table below lists the covariates (age/month, weight/kg, sex) for the shapes above. P-vol(predicted volume) is the volume ($cm^3$) of the transferred shape by NAISR with covariates following Eq.~\eqref{eq.infer_z}. M-vol (measured volume) is the volume ($cm^3$) of the shapes based on the actual imaging. }
\label{tab.another_transfer}
\end{table*}

\begin{table*}
%\hfill\includegraphics[width=2.\columnwidth]{figs/1181_comp.png}
%{c{0.05\textwidth}c{0.06\textwidth}c{0.05\textwidth}c{0.05\textwidth}c{0.05\textwidth}c{0.05\textwidth}c{0.05\textwidth}c{0.05\textwidth}c{0.05\textwidth}c{0.05\textwidth}c{0.05\textwidth}c{0.05\textwidth}}
\resizebox{\textwidth}{!}{%
\begin{tabular}{p{0.05\textwidth}p{0.06\textwidth}p{0.06\textwidth}p{0.06\textwidth}p{0.06\textwidth}p{0.06\textwidth}p{0.06\textwidth}p{0.06\textwidth}p{0.06\textwidth}p{0.06\textwidth}p{0.06\textwidth}p{0.06\textwidth}}
\toprule
\# time &      0&      1  &      2  &      3  &      4  &      5  &      6  &      7  &      8  &      9  &      10 \\
\midrule
$\{\mathcal{S}^t\}$&
\includegraphics[width=0.2\columnwidth]{figs/1181_transport/0.png} &  
\includegraphics[width=0.2\columnwidth]{figs/1181_transport/1.png} & 
\includegraphics[width=0.2\columnwidth]{figs/1181_transport/2.png}& 
\includegraphics[width=0.2\columnwidth]{figs/1181_transport/3.png}& 
\includegraphics[width=0.2\columnwidth]{figs/1181_transport/4.png}& 
\includegraphics[width=0.2\columnwidth]{figs/1181_transport/5.png}& 
\includegraphics[width=0.2\columnwidth]{figs/1181_transport/6.png}& 
\includegraphics[width=0.2\columnwidth]{figs/1181_transport/7.png}& 
\includegraphics[width=0.2\columnwidth]{figs/1181_transport/8.png}& 
\includegraphics[width=0.2\columnwidth]{figs/1181_transport/9.png}& 
\includegraphics[width=0.2\columnwidth]{figs/1181_transport/10.png}\\
\bottomrule
\end{tabular}}

\begin{tabular}{p{0.05\textwidth}p{0.06\textwidth}p{0.06\textwidth}p{0.06\textwidth}p{0.06\textwidth}p{0.06\textwidth}p{0.06\textwidth}p{0.06\textwidth}p{0.06\textwidth}p{0.06\textwidth}p{0.06\textwidth}p{0.06\textwidth}}
\toprule
\# time &      0&      1  &      2  &      3  &      4  &      5  &      6  &      7  &      8  &      9  &      10 \\
\midrule
weight      &   55.20 &   60.90 &   64.30 &   65.25 &   59.25 &   59.20 &   65.30 &   68.00 &   77.10 &   75.60 &   75.60 \\
age         &  154.00 &  155.00 &  157.00 &  159.00 &  163.00 &  164.00 &  167.00 &  170.00 &  194.00 &  227.00 &  233.00 \\
sex         &    male &    male &    male &  male &  male  &  male &  male &  male &  male &   male &  male \\
p-vol  &   89.97 &   92.09 &   93.80 &   94.76 &   94.23 &   94.56 &   97.42 &   99.18 &  107.96 &  114.66 &  116.01 \\
m-vol &   86.33 &   82.66 &   63.23 &   90.65 &   98.11 &   84.35 &   94.14 &  127.45 &   98.81 &  100.17 &  113.84 \\

\bottomrule

\end{tabular}
\caption{Shape transfer using covariates as input for the patient shown in the main text. Blue: gold standard shapes; red: transferred shapes with \textbf{NAISR}. The table below lists the covariates (age/month, weight/kg, sex) for the shapes above. P-vol(predicted volume) is the volume ($cm^3$) of the transferred shape by NAISR with covariates following Eq.~\eqref{eq.infer_z}. M-vol (measured volume) is the volume ($cm^3$) of the shapes based on the actual imaging. The predicted shapes from \textbf{NAISR with Cov.} looks consistent to the shapes from \textbf{NAISR w/o Cov.}.}
\label{tab.transp_for_case_with_cov}
\end{table*}


Tab.~\ref{tab.transp_for_case_with_cov} shows the transferred shapes using \texttt{NAISR} with covariates as input (following Eq.~\eqref{eq.infer_z}). The predicted shapes from Eq.~\eqref{eq.infer_cz} and Eq.~\eqref{eq.infer_z} are consistent in terms of appearance, volume and development tendency. Tab.~\ref{tab.another_transfer} shows the transferred shapes from \texttt{NAISR} of another patient following Eq.~\eqref{eq.infer_z} and Eq.\eqref{eq.infer_cz} respectively.

\subsection{Disentangled Shape Evolution}
\label{subsec:disentangled_shape_evolution}

Fig.~\ref{fig.another_manifold} shows another example of shape extrapolation in covariate space. Fig.~\ref{fig.average_manifold} visualizes shape extrapolation of an average airway shape. Specifically, we randomly select 10 latent codes $\{\mathbf{z}_i\}$ of airway shapes in the training set and take the average of them as,
\begin{equation}
\Bar{\mathbf{z}} = \frac{1}{10}\sum_{i=1}^{10}\mathbf{z}_i \,.
\label{eq.averagez}
\end{equation}

By comparing Fig.~\ref{fig.another_manifold} and Fig.~\ref{fig.average_manifold}, we observe that extrapolating shapes with an average latent code $\Bar{\mathbf{z}}$ produces a larger range of shapes while extrapolating shapes with just one latent code $\mathbf{z}_i$ tends to produce less shape variation. This might be because a particular latent code $\mathbf{z}_i$ should preserve patient-specific information and thus should shrink the shape space by producing similar shapes in covariate space. While using an average latent code $\mathbf{z}$, the population trend dominates patient-specific information. As a result, the extrapolated shape space from $\Bar{\mathbf{z}}$ is able to reflect the overall shape development tendency in the large population better; it corresponds better with the population shape development shown in Tab.~\ref{tab.vis_demo_dataset}. We also observe that shapes extrapolated from a single latent code $\mathbf{z}_i$ look more realistic and smooth across covariates while $\Bar{\mathbf{z}}$ sometimes produces noisy shapes, especially when extrapolating shapes in unseen covariate space. We will attempt to alleviate this problem by using an efficient invertible displacement field in our future work to encourage reasonable shape extrapolation in unseen covariate space. 


\begin{figure}
\centering
\includegraphics[width=1.\columnwidth]{figs/manifold_caseA/ori.pdf}
\caption{Shape extrapolation in covariate space (for a different patient in the testing set than the one in the main text). Example shapes in the age-weight space are visualized with their volumes ($cm^3$) below. Cyan points represent male and purple points female children in the dataset. The points represent the covariates of all children in the dataset. The colored shades represent the age and weight distributions stratified by sex. The latent code $\mathbf{z}$ is kept constant to create an individualized covariate shape space. The shapes in the green and yellow boxes are plotted with $\{\Phi_i\}$ (see Sec.~\ref{sec.testing}), representing the disentangled shape evolutions along weight and age respectively. Shapes extrapolated from $\mathbf{z}_i$ look realistic and smooth across different covariates. (Best viewed zoomed.) }
\label{fig.another_manifold}
\end{figure}


\begin{figure}
\centering
\includegraphics[width=1.\columnwidth]{figs/average_manifold/ori.pdf}
\caption{Shape extrapolation in covariate space for an average airway shape from $\Bar{\mathbf{z}}$ (from Eq.~\eqref{eq.averagez}). Example shapes in the age-weight space are visualized with their volumes ($cm^3$) below. Cyan points represent male and purple points female children in the dataset. The points represent the covariates of all children in the dataset. The colored shades represent the age and weight distributions stratified by sex. The latent code $\mathbf{z}$ is kept constant to create an individualized covariate shape space. The shapes in the green and yellow boxes are plotted with $\{\Phi_i\}$ (see Sec.~\ref{sec.testing}), representing the disentangled shape evolutions along weight and age respectively. The extrapolated shape space from $\Bar{\mathbf{z}}$ is able to reflect the overall shape development tendency in the large population; it corresponds better with the population shape development shown in Tab.~\ref{tab.vis_demo_dataset}. (Best viewed zoomed.) }
\label{fig.average_manifold}
\end{figure}


\section{Ablation Studies}
In this section, we investigate how the zero padding strategy and the inverse consistency loss influence \texttt{NAISR}'s performance in terms of shape reconstruction performance and shape extrapolation.  

\subsection{Shape Reconstruction}
\begin{table*}
\begin{center}
\resizebox{\textwidth}{!}{%
\begin{tabular}{lrrrrrrrrr}
\toprule
Methods & with Cov. & with I.C. & with Z.P. &CD mean &  CD median &  EMD mean &  EMD median &  HD mean &  HD median \\
\midrule
 NAISR &  \Checkmark &  \Checkmark &  \Checkmark &  0.059 &      0.031 &     1.200 &       1.109 &   10.334 &      8.287 \\ 
 NAISR & \Checkmark & \XSolidBrush  &  \Checkmark &    0.071 &      0.033 &     1.246 &       1.129 &   10.972 &      8.771 \\
NAISR & \Checkmark &  \Checkmark &  \XSolidBrush  &  0.064 &      0.032 &     1.272 &       1.167 &   10.576 &      8.143 \\
NAISR & \Checkmark   &  \XSolidBrush  &  \XSolidBrush &  0.058 &      0.031 &     1.131 &       1.063 &   10.735 &      8.554 \\
\midrule
NAISR &  \XSolidBrush  &  \Checkmark &  \Checkmark &     0.056 &      0.030 &     1.165 &       1.061 &    9.968 &      8.104 \\
NAISR & \XSolidBrush &  \XSolidBrush  &  \Checkmark &    0.059 &      0.029 &     1.179 &       1.082 &   10.078 &      8.140 \\
NAISR  &  \XSolidBrush &  \Checkmark &  \XSolidBrush  &     0.060 &      0.032 &     1.249 &       1.161 &    9.950 &      7.830 \\
NAISR &    \XSolidBrush &  \XSolidBrush  &  \XSolidBrush &    0.054 &      0.027 &     1.086 &       1.009 &   10.216 &      7.992 \\
\bottomrule
\end{tabular}}
\end{center}
\caption{Shape reconstruction performance for different experiment settings. With Cov. denotes that covariates are used as input as described in Eq.~\eqref{eq.infer_z}. With I.C. indicates that the inverse consistency loss is used. With Z.P. denotes that the zero-padding strategy is used. The reconstruction performs best when there are no regularizations. The inverse consistency loss becomes helpful to better reconstruct shapes when using zerp-padding strategy. }
\label{tab.abl_recons}
\end{table*}


Tab.~\ref{tab.abl_recons} shows quantitative evaluation results for shape reconstruction on the testing set using \texttt{NAISR} with different settings. We observe that not providing covariates as input during the testing stage produces better reconstruction results. This might be because the missing covariates leave the network more freedom to estimate the latent code $\mathbf{z}$. The reconstruction performs best when there are no regularizations (i.e. neither zero padding nor inverse consistency). However, when using the zero-padding strategy to encourage the interpretability of the network, the inverse consistency loss becomes helpful to better reconstruct shapes. 


\subsection{Shape Disentanglement and Evolution}

Tab.~\ref{fig.abl_another_manifold} visualizes the shape extrapolation results from the latent code $\mathbf{z}_i$ for a patient using different training settings. We observe that \texttt{NAISR} is able to produce realistic shapes with all experiment settings. However, the network has difficulty evolving shapes along weight when the zero-padding strategy is not used; as shown in Fig.~\ref{fig.abl_another_manifold}(b,d). 

Tab.~\ref{fig.average_manifold} visualizes the shape extrapolation results from the average latent code $\Bar{\mathbf{z}}$ (from Eq.~\eqref{eq.averagez}) for an average airway shape. We can observe that all settings work well when extrapolating shapes near the template airway, and all settings are able to produce tiny shapes for baby airways. Also, all settings can evolve the airway shapes reasonably if there are enough data points in the covariate space, indicating that \texttt{NAISR} is able to capture the airway development tendency of the population. However, all settings may produce unrealistic airways for unseen covariates. For example, the shapes in the left corner and right corner in Fig.\ref{fig.abl_average_manifold}(a,b,c,d) look unrealistic. Shapes in the left corner in Fig.\ref{fig.abl_average_manifold}(a,b,c,d) are baby airways of high weight which do not exist in reality while shapes in the right corner do not belong to pediatric airways (age $\geq$ 216 months) and few cases with those covariates exist in the dataset. 

From Fig.~\ref{fig.abl_average_manifold}, we see that the zero-padding strategy is necessary for the network to learn correct knowledge from data. For example, when not using the zero-padding strategy, as illustrated in Fig.~\ref{fig.abl_average_manifold}(b,d), the network tends to infer a large volume increase for a simple weight increase when the patient is already an adult. However, this is less likely to happen in clinical observation as age plays a more important role~\cite{luscan2020developmental}. %\texttt{NAISR} w/o inverse consistency loss tends to produce curvy airways in unseen covariate space as shown in Fig.~\ref{fig.abl_average_manifold}(c), indicating that our inverse consistency loss is able to regularize the displacement field. 


\begin{figure*}[htbp!]
\centering
 \subfigure [\textbf{NAISR} with I.C. with Z.P.] {\includegraphics[scale=0.28]{figs/manifold_caseA/ori.pdf}}\quad
 \subfigure [\textbf{NAISR} with I.C., w/o Z.P.] {\includegraphics[scale=0.28]{figs/manifold_caseA/nozp.pdf}}
 \subfigure [\textbf{NAISR} w/o I.C., with Z.P.] {\includegraphics[scale=0.28]{figs/manifold_caseA/noinv.pdf}}\quad
 \subfigure [\textbf{NAISR} w/o I.C., w/o Z.P.] {\includegraphics[scale=0.28]{figs/manifold_caseA/noinvzp.pdf}}
\caption{Comparisons of shape extrapolation for different experiment settings from $\mathbf{z}_i$ for one patient. I.C. denotes that the inverse consistency loss was used. Z.P. denotes that the zero-padding strategy was used to encourage interpretability. 
 Example shapes in the age-weight space are visualized with their volumes ($cm^3$) below. Cyan points represent male and purple points female children in the dataset. The points represent the covariates of all children in the dataset. The colored shades represent the age and weight distributions stratified by sex. The latent code $\mathbf{z}$ is kept constant to create an individualized covariate shape space. The shapes in the green and yellow boxes are plotted with $\{\Phi_i\}$ (see Sec.~\ref{sec.testing}), representing the disentangled shape evolutions along weight and age respectively. \textbf{NAISR} is able to produce realistic shapes with all experiment settings. However, the network has difficulty evolving shapes along weight when the zero-padding strategy is not used.}
\label{fig.abl_another_manifold}
\vskip-0.1in
\end{figure*}



\begin{figure*}[htbp!]
\centering
 \subfigure [\textbf{NAISR} with I.C. with Z.P.] {\includegraphics[scale=0.28]{figs/average_manifold/ori.pdf}}\quad
 \subfigure [\textbf{NAISR} with I.C., w/o Z.P.]  {\includegraphics[scale=0.28]{figs/average_manifold/nozp.pdf}}
 \subfigure [\textbf{NAISR} w/o I.C., with Z.P.] {\includegraphics[scale=0.28]{figs/average_manifold/noinv.pdf}}\quad
 \subfigure [\textbf{NAISR} w/o I.C., w/o Z.P.] {\includegraphics[scale=0.28]{figs/average_manifold/noinvzp.pdf}}
\caption{Comparisons of shape extrapolation for different experiment settings from $\Bar{\mathbf{z}}$ for an average airway shape. I.C. denotes that the inverse consistency loss was used. Z.P. denotes that the zero-padding strategy was used to encourage interpretability.  Example shapes in the age-weight space are visualized with their volumes ($cm^3$) below. Cyan points represent male and purple points female children in the dataset. The points represent the covariates of all children in the dataset. The colored shades represent the age and weight distributions stratified by sex. The latent code $\mathbf{z}$ is kept constant to create an individualized covariate shape space. The shapes in the green and yellow boxes are plotted with $\{\Phi_i\}$ (see Sec.~\ref{sec.testing}), representing the disentangled shape evolutions along weight and age respectively. All settings can evolve the airway shapes reasonably if there are enough data points in the covariate space, indicating that \textbf{NAISR} is able to capture the airway development tendency of the population. However, all settings may produce unrealistic airways for unseen covariates.
(Best viewed zoomed.) }
\label{fig.abl_average_manifold}
\vskip-0.1in
\end{figure*}

%-------------------------------------------------------------------------
\clearpage
