% ------- petri nets
\usetikzlibrary{petri}
\tikzstyle{binary place}=[place,circle, double]
\tikzstyle{node}=[circle,draw=black,thick,minimum size=9mm]
\tikzstyle{dest}=[circle,draw=black!50,fill=black!20,thick,minimum
  size=9mm]
\tikzstyle{post}=[->,thick]
\tikzstyle{pre}=[<-,thick]
\tikzstyle{every transition}=[fill,minimum width=1cm,minimum height=2mm]
\tikzstyle{Atransition}=[transition,fill,minimum width=1cm,minimum height=2mm]
\tikzstyle{Otransition}=[transition,fill=white,minimum width=1cm,minimum height=2mm]
\tikzstyle{THtransition}=[transition,fill=white,minimum width=4mm,minimum height=1cm]
% ------- others

% discontinuities:
\newcommand{\emdot}[1]{\fill[black] #1 circle (0.5mm) node[left] {};%
  \fill[white] #1 circle (0.2mm) node[left] {};%
}

\newcommand{\fidot}[1]{\fill[black] #1 circle (0.5mm) node[left] {};}

% functions:
% e.g \eeline{from}{to}{functionvalue}
\newcommand{\eeline}[3]{
  \path [draw,line width=0.5mm,-] (#1,#3) -- (#2,#3);%
  \emdot{(#1,#3)}%
  \emdot{(#2,#3)}%
}

\newcommand{\efline}[3]{
  \path [draw,line width=0.5mm,-] (#1,#3) -- (#2,#3);%
  \emdot{(#1,#3)}%
  \fidot{(#2,#3)}%
}


\newcommand{\feline}[3]{
  \path [draw,line width=0.5mm,-] (#1,#3) -- (#2,#3);%
  \fidot{(#1,#3)}%
  \emdot{(#2,#3)}%
}

\newcommand{\ffline}[3]{
  \path [draw,line width=0.5mm,-] (#1,#3) -- (#2,#3);%
  \fidot{(#1,#3)}%
  \fidot{(#2,#3)}%
}

\newcommand{\nfline}[3]{
  \path [draw,line width=0.5mm,-] (#1,#3) -- (#2,#3);%
  \fidot{(#2,#3)}%
}

\newcommand{\neline}[3]{
  \path [draw,line width=0.5mm,-] (#1,#3) -- (#2,#3);%
  \emdot{(#2,#3)}%
}

\newcommand{\fnline}[3]{
  \path [draw,line width=0.5mm,-] (#1,#3) -- (#2,#3);%
  \fidot{(#1,#3)}%
}

\newcommand{\enline}[3]{
  \path [draw,line width=0.5mm,-] (#1,#3) -- (#2,#3);%
  \emdot{(#1,#3)}%
}


% signals and dependence graph
\newcommand{\tcross}[2]{%
  \draw plot[mark=x, mark size=2mm] coordinates{#1} node[above=1mm]{#2};%
}

\newcommand{\tcrossAR}[2]{%
  \draw plot[mark=x, mark size=2mm] coordinates{#1} node[above right=1mm]{#2};%
}

%\newcommand{\tdepends}[2]{%
%  \path #1 edge [->,bend left,semithick,shorten >=2mm] #2;%
%}

\tikzstyle{Tdelay} = [draw, rectangle, rounded corners,
    minimum height=4mm, minimum width=20mm]

\tikzstyle{Tfunction} = [draw, rectangle,
    minimum height=4mm, minimum width=4mm]

\tikzstyle{Tsignal} = [draw,fill=black,circle, size=1mm]

\newcommand{\QueueZ}[4]{%           \QueueZ(x,y,name,label)
  \path[draw,-] (#1,#2) -- ++(1,0) -- ++(0,1) -- ++(-1,0) -- ++(0,-1);
  %\path[draw,-] (#1,#2) -- ++(0.2,0) -- ++(0,1);
  %\path[draw,-] (#1,#2) -- ++(0.8,0) -- ++(0,1);
  \path[draw] (#1,#2)+(0.5,0.5) node (#4) {#3};
}

\newcommand{\BQueueZ}[5]{%           \BQueueZ(x,y,name,label,bound)
  \path[draw,-] (#1,#2) -- ++(1,0) -- ++(0,1) -- ++(-1,0) -- ++(0,-1);
  %\path[draw,-] (#1,#2) -- ++(0.2,0) -- ++(0,1);
  %\path[draw,-] (#1,#2) -- ++(0.8,0) -- ++(0,1);
  \path[draw] (#1,#2)+(0.5,0.5) node (#4) {#3};
  \path[draw] (#1,#2) -- ++(0.25,0) -- ++(0,0.25) -- ++(-0.25,0) -- ++(0,-0.25);
  \path[draw] (#1,#2)+(0.1,0.1) node (l) {{\scriptsize #5}};
}

\newcommand{\GJoinZ}[5]{%  \GJoinZ{x}{y}{height}{name}{label}
  \path[draw,-] (#1,#2)-- ++(1,0)-- ++(0,#3)-- ++(-1,0)-- ++(0,-#3);
  \path (#1,#2)+(0.5,0.5) node (#5) {#4}; 
}

%\tikzstyle{ra} = [draw,double,-latex]
\tikzstyle{ra} = [draw,thick,double,double distance=1.0pt,->]
\tikzstyle{r} = [draw,->,line width=0.5pt]

\newcommand{\ForkZ}[5]{%  \ForkZ{x}{y}{height}{name}{label}
  \path[draw,-] (#1,#2)-- ++(1,0)-- ++(0,#3)-- ++(-1,0)-- ++(0,-#3);
  \path (#1,#2)+(0.5,0.5) node (#5) {#4}; 
}

