
\def\figanvth{0.4}
\def\figaninstart{0.1}
\def\figaninlen{0.28}
\def\figandelay{0.05}
\def\figanfup[#1]{(1-exp(-(#1)/0.15))*(1-exp(-(#1)/0.16))}
\def\figanfdown[#1]{(1-(1-exp(-((#1)^2)/0.01))*(1-exp(-((#1)^2)/0.1)))}
\def\figancaptpos{0.9}

\def\tikzfigureanalogy{
\begin{tikzpicture}[
xscale=4.4,
yscale=1,
>=latex'
]

\coordinate (origin) at (0,0);

% Axes
\path [draw,->]
(0,0) -- (0,1.1) node (xaxis) [left] {\small $u(t)$};
\path [draw,->]
(0,0) -- (1.05,0)   node (yaxis) [above] {\small $t$};

% Vth
\path [draw, dashdotted, very thin, name path=vth]
(0,\figanvth) node [left] {\small $V_{th}$} -- (1,\figanvth);

% ui
\path [draw, densely dotted, line width=0.3pt]
(\figaninstart,0) -- ++(0,1) node [above, xshift=-1pt, yshift=-1pt] {\small $u_i$} --
++(\figaninlen,0) -- ++(0,-1) node (dt3) [coordinate] {};

% ud
\path [draw, densely dashed, very thin]
({\figaninstart+\figandelay},0) -- ++(0,1) node [above, xshift=3pt, yshift=-1pt] {\small $u_d$} --
++(\figaninlen,0) -- ++(0,-1);

% ur
%   fup
\path [draw, thick, xshift={(\figaninstart+\figandelay)*1cm}, name path=fup]
plot[smooth, domain=0:\figaninlen] (\x,{\figanfup[\x]});
\path [draw, thick, dotted, xshift={(\figaninstart+\figandelay)*1cm}, name path=fupdot]
plot[smooth, domain=\figaninlen:{1-\figaninstart-\figandelay}] (\x,{\figanfup[\x]});
%  fdown - hack to find the inverse
\path [name path=func]
plot[smooth, domain=0:1] (\x,{\figanfdown[\x]});
\path [name path=const]
(0,{\figanfup[\figaninlen]}) -- +(1,0);
\path [name intersections={of=func and const}]
(intersection-1) node (finversept) [coordinate] {};
\newdimen\finverse
\pgfextractx{\finverse}{\pgfpointanchor{finversept}{center}}
%  fdown
\pgfmathsetlengthmacro{\figanfdownshift}{(\figaninstart+\figandelay+\figaninlen)*1cm-\finverse}
\path [draw, thick, dotted, xshift=\figanfdownshift]
plot[smooth, domain=0:{\finverse/1cm}] (\x,{\figanfdown[\x]});
\path [draw, thick, xshift=\figanfdownshift, name path=fdown]
plot[smooth, domain={\finverse/1cm}:{(1cm-\figanfdownshift)/1cm}] (\x,{\figanfdown[\x]});

% uo
\path [draw, thin, name intersections={of=fup and vth, by={up}}, name intersections={of=fdown and vth, by={down}}]
(origin -| up) node (dt2) [coordinate] {} -- ++(0,1) node [above, xshift=3pt, yshift=-1pt] {\small $u_o$} -| 
(origin -| down) node (dt4) [coordinate] {};

% fup/down captions
\path [name path=fcapts]
(\figancaptpos,0) -- ++(0,1);
\path [name intersections={of=fcapts and fdown, by={fd}}, name intersections={of=fcapts and fupdot, by={fu}}]
(fd) node [above] {\small $f_\downarrow$}
(fu) node [below] {\small $f_\uparrow$};

% T & delta captions
\foreach \dt/\name in {{(\figaninstart,0)/1}, {(dt2)/2}, {(dt3)/3}, {(dt4)/4}} {
\path [draw, very thin, shorten <= 1pt, shorten >=-2pt]
\dt -- ++(0,-0.1) node (smth-\name) [coordinate] {};
}

\path [draw, <-, very thin]
(smth-1) -- node [anchor=north] {\tiny $T_1$} (smth-1 -| origin);
\path [draw, <->, very thin, shorten <=0.6pt]
(smth-1) -- node [anchor=north] {\tiny $\delta_\uparrow(T_1)$} (smth-2);
\path [draw, <->, very thin, shorten <=0.6pt]
(smth-2) -- node [anchor=north] {\tiny $T_2$} (smth-3);
\path [draw, <->, very thin, shorten <=0.6pt]
(smth-3) -- node [anchor=north] {\tiny $\delta_\downarrow(T_2)$} (smth-4);
\end{tikzpicture}
}





\def\figcirampin{38}
\def\figcirampinlen{0.25}
\def\figcirampvoltlen{0.27}
\def\figcirampcap{0.3}
\def\figcirdista{0.45}
\def\figcirdistb{0.55}
\def\figcirdistc{1}
\def\figciroutlen{0.25}

\def\tikzfigurecircuit{
\begin{tikzpicture}[
cmp/.style={draw, isosceles triangle, isosceles triangle stretches, minimum width=1.1cm, minimum height=1.0cm},
xscale=0.8,
yscale=0.5,
>=latex'
]

% comparator1
\begin{scope}[color=black!40,text=black!40]
\path
(0,0) node (cmp1) [cmp] {};
\path [draw]
(cmp1.{180-\figcirampin}) node [anchor=west] {\footnotesize $+$} --
++(-\figcirampinlen,0) node (cmp1-west) [coordinate] {};
\path [draw]
(cmp1.{180+\figcirampin}) node [anchor=west] {\footnotesize $-$} -- ++(-\figcirampinlen,0);
\end{scope}


% delay
\path [draw, shorten <=-0.5pt]
(cmp1.east) -- node [anchor=south] {\small $u_i$}
++(\figcirdista,0) node (del) [anchor=west, draw, minimum height=0.4cm, minimum width=0.8cm,
                      inner sep=1pt, rounded corners=0.2cm] {\small $T_p$};

% slew rate limiter
\path [draw]
(del.east) -- node [anchor=south] {\small $u_d$}
++(\figcirdistb,0) node (srl) [draw, minimum size=0.8cm, anchor=west] {};
\path [draw, scale=0.45]
(srl) ++(-0.5,-0.5) .. controls +(0.3,0) and +(-0.8,0) .. +(1,1);

% comparator2
\path [draw]
(srl.east) -- node [anchor=south] {\small $u_r$}
++(\figcirdistc,0) node (cmp2) [cmp, anchor={180-\figcirampin}] {}
                     node [anchor=west] {\footnotesize $+$};
\path [draw]
(cmp2.{180+\figcirampin}) node [anchor=west] {\footnotesize $-$} --
++(-\figcirampinlen,0) node [anchor=east, inner sep=1pt] {\small $V_{th}$};
\path [draw, shorten <=-0.5pt]
(cmp2.east) --
++(\figciroutlen,0) node [anchor=290] {\small $u_o$};

% 1v 0v and infty for comparator 2
\foreach \cmpx in {cmp2} {
\path [draw]
(\cmpx.50) -- ++(0,\figcirampvoltlen) node (\cmpx-1v) [anchor=south] {\small $1V$};
\path [draw]
(\cmpx.310) -- ++(0,-\figcirampvoltlen) node (\cmpx-0v) [anchor=north] {\small $0V$};
\path
($(\cmpx)+(\figcirampcap,0)$) node {\small $\infty$};
}

% for cmp1 in grey:
\begin{scope}[black!40]
\foreach \cmpx in {cmp1} {
\path [draw]
(\cmpx.50) -- ++(0,\figcirampvoltlen) node (\cmpx-1v) [anchor=south] {\small $1V$};
\path [draw]
(\cmpx.310) -- ++(0,-\figcirampvoltlen) node (\cmpx-0v) [anchor=north] {\small $0V$};
\path
($(\cmpx)+(\figcirampcap,0)$) node {\small $\infty$};
}
\end{scope}


% hiding cmp1 -- transparency not allowed by ACM
%\node at ($ (cmp1.east)-(0.17,0) $) (seast) {};
%
%\node[fill=white, fill opacity=0.6, inner sep=0, fit=(cmp1-1v) (cmp1-0v) (cmp1-west) (seast)] {};
\end{tikzpicture}
}

\def\figurangle{300}
\def\figurlen{1.2}
\def\figurdist{4.5}

\def\tikzfigureunrolling{
\begin{tikzpicture}[
gate/.style={draw, fill, circle, minimum size=3pt, inner sep=0},
xscale=1,
yscale=0.6,
>=latex'
]

% C name
%\path
%(\figurangle:{\figurlen/2}) ++(-0.5,0) node [anchor=east] {$C$};

% C nodes
\path
(0,0) node (I) [gate] {}
+(0:\figurlen) node (A) [gate] {}
++(\figurangle:\figurlen) node (B) [gate] {}
++(0:\figurlen) node (C) [gate] {}
++(0:\figurlen) node (O) [gate] {};

% C edges
\foreach \from/\to in {I/A, A/C, I/B, B/C, C/O} {
\path [draw, ->] (\from) -- (\to);
}
\path [draw, <-, scale=0.8]
(B) .. controls +(-1,-1) and +(1,-1) .. (B);

% C captions
\foreach \what/\anchor in {I/south, A/south, B/south, C/north, O/north} {
\node [anchor=\anchor] at (\what) {\footnotesize $\what$};
}

% a small hack for a larger hspace... ;-)
\pgftransformshift{\pgfpoint{+0.5cm}{0cm}} 

% C2(o) nodes
\path
(\figurdist,0) node (i1) [gate] {}
++(0:\figurlen) node (x1) [gate] {}
++(\figurangle:\figurlen) node (c1) [gate] {}
++(0:\figurlen) node (o1) [gate] {}
(c1) ++(0:-\figurlen) node (y1) [gate] {}
++(0:-\figurlen) node (y2) [gate] {}
+(\figurangle:-\figurlen) node (ot2) [gate] {};

% C2(o) edges
\foreach \from/\to in {ot2/y2, y2/y1,
                       y1/c1, i1/y1, i1/x1, x1/c1, c1/o1, i1/y2} {
\path [draw, ->] (\from) -- (\to);
}

% C2(o) captions
\foreach \where/\anchor/\text in
{ot2/south/$X_0$,
y2/north/$B^{(1)}$, i1/south/$I$, x1/south/$A^{(2)}$,
y1/north/$B^{(2)}$, o1/north/$O^{(3)}$, c1/north/$C^{(3)}$} {
\node [anchor=\anchor] at (\where) {\footnotesize \text};
}
\end{tikzpicture}
}






\def\figchdeltadist{0.3}
\def\figchtimedist{0.35}
\def\figsigheight{0.7}
\def\figsigoffset{0.1}
\def\figchdist{1.5}

\def\tikzfigurechannelintro{
\begin{tikzpicture}[>=latex',scale=0.8,every node/.style={transform shape}]

\coordinate (origin1) at (0,0);
\coordinate (start1) at (0,\figsigoffset);
\coordinate (origin2) at (0,-\figchdist);
\coordinate (start2) at (0,-\figchdist+\figsigoffset+\figsigheight);
\coordinate (timeorigin) at (0,{-\figchdist-\figchtimedist});
% axes
\foreach \plot/\xcapt in {1/$\text{in}(t)$, 2/$\text{out}(t)$} {
\path [draw,->]
(origin\plot) -- +(0,1.1) node (xaxis) [left,yshift=-5pt] {\small \xcapt};
\path [draw,->]
(origin\plot) -- +(8.00,0) node (yaxis) [above] {\small $t$};
}
%in signal
\draw[very thick] (start1) -- ++(4,0) -- ++(0,\figsigheight) -- ++(4,0);
%out signal
\draw[very thick] (start2) -- ++(3,0) -- ++(0,-\figsigheight) -- ++(3,0) --
++(0,\figsigheight) -- ++(2,0);
%stricheln
\draw[densely dashed] ($(origin1)+(4,0)$) -- ($(origin2)+(4,-3*\figsigoffset)$);
\draw[densely dashed] ($(origin2)+(3,0)$) -- ++(0,-3*\figsigoffset);
\draw[densely dashed] ($(origin2)+(6,0)$) -- ++(0,-3*\figsigoffset);
\node (T) at ($(origin2) +(3.5,-2*\figsigoffset) $) {$T$};
\node (dT) at ($(origin2) +(5,-2*\figsigoffset) $) {$\delta(T)$};
\draw[->] (T) -- ($(origin2)+(4,-2*\figsigoffset)$);
\draw[->] (T) -- ($(origin2)+(3,-2*\figsigoffset)$);
\draw[->] (dT) -- ($(origin2)+(6,-2*\figsigoffset)$);
\draw[->] (dT) -- ($(origin2)+(4,-2*\figsigoffset)$);
%delay arrow
\draw[->,densely dashed,shorten >=2pt,shorten <=2pt] ($(start1) + (4,\figsigheight)$) -- ($(start2) +
(6,0)$);
\draw[->,densely dashed,shorten >=2pt,shorten <=2pt] ($(start1) + (0,0.5*\figsigheight)$) -- ($(start2) +
(3,0)$);

%\clip(-1,1.5) rectangle ($(origin2) + (9,-4*\figsigoffset) $);
\end{tikzpicture}
}


\def\figchdeltadist{0.3}
\def\figchtimedist{0.35}
\def\figsigheight{0.7}
\def\figsigoffset{0.1}
\def\figchdist{1.5}

\def\tikzfigurechannelintroSmall{
\begin{tikzpicture}[>=latex',scale=0.8,xscale=0.6]

\coordinate (origin1) at (0,0);
\coordinate (start1) at (0,\figsigoffset+\figsigheight);
\coordinate (origin2) at (0,-\figchdist);
\coordinate (start2) at (0,-\figchdist+\figsigoffset+\figsigheight);
\coordinate (timeorigin) at (0,{-\figchdist-\figchtimedist});

% axes
\foreach \plot/\xcapt in {1/$\text{in}(t)$, 2/$\text{out}(t)$} {
\path [draw,->]
(origin\plot) -- +(0,1.1) node (xaxis) [left,yshift=-5pt] {\small \xcapt};
\path [draw,->]
(origin\plot) -- +(6.90,0) node (yaxis) [above] {\small $t$};
}
%in signal
\draw[very thick] (start1) -- ++(0.5,0) -- ++(0,-\figsigheight) -- ++(3.5,0) -- ++(0,\figsigheight) -- ++(2.5,0);
%out signal
\draw[very thick] (start2) -- ++(3.3,0) -- ++(0,-\figsigheight) -- ++(2.8,0) --
++(0,\figsigheight) -- ++(0.4,0);
%stricheln
\draw[densely dashed] ($(origin1)+(4,0)$) -- ($(origin2)+(4,-5*\figsigoffset)$);
\draw[densely dashed] ($(origin2)+(3.3,0)$) -- ++(0,-5*\figsigoffset);
\draw[densely dashed] ($(origin2)+(6.1,0)$) -- ++(0,-5*\figsigoffset);
\node (T) at ($(origin2) +(3.65,-2*\figsigoffset) $) {\small $T$};
\node (dT) at ($(origin2) +(5.05,-4*\figsigoffset) $) {\small $\delta(T)$};
\draw[->] (T) -- ($(origin2)+(4,-2*\figsigoffset)$);
\draw[->] (T) -- ($(origin2)+(3.3,-2*\figsigoffset)$);
\draw[->] (dT) -- ($(origin2)+(6.1,-4*\figsigoffset)$);
\draw[->] (dT) -- ($(origin2)+(4,-4*\figsigoffset)$);
%delay arrow
\draw[->,densely dashed,shorten >=2pt,shorten <=2pt] ($(start1) + (0.5,0)$) -- ($(start2) +
(3.3,0)$);
\draw[->,densely dashed,shorten >=2pt,shorten <=2pt] ($(start1) + (4,0)$) -- ($(start2) +
(6.1,0)$);

\clip(-1.9,1.1) rectangle ($(origin2) + (7.1,-9*\figsigoffset) $);
%\draw (-1.9,1.1) rectangle ($(origin2) + (7.1,-9*\figsigoffset) $);
\end{tikzpicture}
}


%\def\figchdeltadist{0.6}
%\def\figchtimedist{0.35}
%\def\figsigheight{1.0}
%\def\figsigoffset{0.1}

\def\tikzfigurechannelsection{
\begin{tikzpicture}[>=latex',scale=0.8,every node/.style={transform shape}]

\coordinate (origin1) at (0,0);
\coordinate (start1) at (0,\figsigoffset+\figsigheight);
\coordinate (origin2) at (0,-\figchdist);
\coordinate (start2) at (0,-\figchdist+\figsigoffset+\figsigheight);
\coordinate (timeorigin) at (0,{-\figchdist-\figchtimedist});

% axes
\foreach \plot/\xcapt in {1/$\text{in}(t)$, 2/$\text{out}(t)$} {
\path [draw,->]
(origin\plot) -- +(0,1.1) node (xaxis) [left,yshift=-5pt] {\small \xcapt};
\path [draw,->]
(origin\plot) -- +(8.00,0) node (yaxis) [above] {\small $t$};
}
%in signal
\draw[very thick] (start1) -- ++(1.5,0) -- ++(0,-\figsigheight) -- ++(2,0) --
++(0,\figsigheight) -- ++(4.5,0);
%out signal
\draw[very thick] (start2) -- ++(5,0) -- ++(0,-\figsigheight) -- ++(1,0) --
++(0,\figsigheight) -- ++(2,0);
%stricheln
\draw[densely dashed] ($(origin1)+(3.5,0)$) -- ($(origin2)+(3.5,-7*\figsigoffset)$);
\draw[densely dashed] ($(origin2)+(5,0)$) -- ++(0,-3*\figsigoffset);
\draw[densely dashed] ($(origin2)+(6,0)$) -- ++(0,-7*\figsigoffset);
\node (T) at ($(origin2) +(4.25,-2*\figsigoffset) $) {$(-T)$};
\node (dT) at ($(origin2) +(4.75,-6*\figsigoffset) $) {$\delta(T)$};
\draw[->] (T) -- ($(origin2)+(5,-2*\figsigoffset)$);
\draw[->] (T) -- ($(origin2)+(3.5,-2*\figsigoffset)$);
\draw[->] (dT) -- ($(origin2)+(6,-6*\figsigoffset)$);
\draw[->] (dT) -- ($(origin2)+(3.5,-6*\figsigoffset)$);
%delay arrow
\draw[->,densely dashed,shorten >=2pt,shorten <=2pt] ($(start1) + (3.5,0)$) -- ($(start2) +
(6,0)$);
\draw[->,densely dashed,shorten >=2pt,shorten <=2pt] ($(start1) + (1.5,0)$) -- ($(start2) +
(5,0)$);

%\clip(-1.9,1.1) rectangle ($(origin2) + (7.1,-7*\figsigoffset) $);
%\draw (-1.9,1.1) rectangle ($(origin2) + (7.1,-7*\figsigoffset) $);
%\clip(-1,1.5) rectangle ($(origin2) + (9,-4*\figsigoffset) $);
\end{tikzpicture}
}

\def\tikzfigurechannelsectionSmall{
\begin{tikzpicture}[>=latex',scale=0.8,xscale=0.6]

\coordinate (origin1) at (0,0);
\coordinate (start1) at (0,\figsigoffset+\figsigheight);
\coordinate (origin2) at (0,-\figchdist);
\coordinate (start2) at (0,-\figchdist+\figsigoffset+\figsigheight);
\coordinate (timeorigin) at (0,{-\figchdist-\figchtimedist});
% axes
\foreach \plot/\xcapt in {1/$\text{in}(t)$, 2/$\text{out}(t)$} {
\path [draw,->]
(origin\plot) -- +(0,1.1) node (xaxis) [left,yshift=-5pt] {\small \xcapt};
\path [draw,->]
(origin\plot) -- +(5.30,0) node (yaxis) [above] {\small $t$};
}
%in signal
\draw[very thick] (start1) -- ++(0.5,0) -- ++(0,-\figsigheight) -- ++(1.4,0) --
++(0,\figsigheight) -- ++(3.1,0);
%out signal
\draw[very thick] (start2) -- ++(3.3,0) -- ++(0,-\figsigheight) -- ++(0.5,0) --
++(0,\figsigheight) -- ++(1.2,0);
%stricheln
\draw[densely dashed] ($(origin1)+(1.9,0)$) -- ($(origin2)+(1.9,-7*\figsigoffset)$);
\draw[densely dashed] ($(origin2)+(3.3,0)$) -- ++(0,-3*\figsigoffset);
\draw[densely dashed] ($(origin2)+(3.8,0)$) -- ++(0,-7*\figsigoffset);
\node (T) at ($(origin2) +(2.6,-2*\figsigoffset) $) {\footnotesize $(-T)$};
\node (dT) at ($(origin2) +(2.85,-6*\figsigoffset) $) {\small $\delta(T)$};
\draw[->] (T) -- ($(origin2)+(3.3,-2*\figsigoffset)$);
\draw[->] (T) -- ($(origin2)+(1.9,-2*\figsigoffset)$);
\draw[->] (dT) -- ($(origin2)+(3.8,-6*\figsigoffset)$);
\draw[->] (dT) -- ($(origin2)+(1.9,-6*\figsigoffset)$);
%delay arrow
\draw[->,densely dashed,shorten >=2pt,shorten <=2pt] ($(start1) + (1.9,0)$) -- ($(start2) +
(3.8,0)$);
\draw[->,densely dashed,shorten >=2pt,shorten <=2pt] ($(start1) + (0.5,0)$) -- ($(start2) +
(3.3,0)$);

\clip(-1.9,1.4) rectangle ($(origin2) + (5.6,-9*\figsigoffset) $);
%\draw (-1.9,1.4) rectangle ($(origin2) + (5.6,-9*\figsigoffset) $);
\end{tikzpicture}
}



\def\figchinup{0.1}
\def\figchindown{0.45}
\def\figchoutup{0.6}
\def\figchoutdown{0.75}
%\def\figchdist{1.9}
%\def\figchdeltadist{0.4}
%\def\figchtimedist{0.35}

\def\tikzfigurechannel{
\begin{tikzpicture}[
open/.style={draw, fill=white, circle, inner sep=0, minimum size=2.5pt},
closed/.style={draw, fill, circle, inner sep=0, minimum size=2.5pt},
xscale=6,
yscale=0.75,
>=latex'
]

\coordinate (origin1) at (0,0);
\coordinate (origin2) at (0,-\figchdist);
\coordinate (timeorigin) at (0,{-\figchdist-\figchtimedist});

% axes
\foreach \plot/\xcapt in {1/$\text{in}(t)$, 2/$\text{out}(t)$} {
\path [draw,->]
(origin\plot) -- +(0,1.3) node (xaxis) [left,yshift=-5pt] {\small \xcapt};
\path [draw,->]
(origin\plot) -- +(1.05,0) node (yaxis) [above] {\small $t$};
}

% signals
\foreach \plot/\up/\down in {
1/\figchinup/\figchindown, 2/\figchoutup/\figchoutdown} {
\path [draw, thick]
(origin\plot) --
++(\up,0) node (u-0-\plot) [open] {}
++(0,1) node (u-1-\plot) [closed] {} --
++({\down-\up},0) node (d-1-\plot) [open] {}
++(0,-1) node (d-0-\plot) [closed] {} --
++({1-\down},0);
}

% delta(T)
\path [draw, very thin, shorten >=-1mm] 
(d-1-2) -- ++(0,{-\figchdeltadist+\figchdist+0.1}) coordinate (tmp);
\path [draw, very thin, <->]
(tmp) -- node [anchor=north] {\small $\delta(T)$}
(tmp -| d-0-1) coordinate (tmp);
\path [draw, very thin, shorten >=-1mm]
(d-1-1) -- (tmp);

% T
\path [draw, very thin, shorten >=-1mm]
(u-1-2) -- ($(u-1-2)!0.5!(u-1-2 |- origin1)$) coordinate (tmp);
\path [draw, very thin, <->]
(tmp) -- node [anchor=north] {\small $-T$}
(tmp -| d-0-1) coordinate (tmp);
\path [draw, very thin, shorten >=-1mm]
(d-0-1) -- (tmp);

% arrows
\foreach \tran in {u, d} {
\path [draw, densely dotted, thin, ->]
(\tran-0-1) -- (\tran-0-2);
}

% times
\foreach \where/\capt/\sh in {
d-0-1/$t$/0, u-0-2/$t'$/0, d-0-2/$t+\delta(T)$/0.3cm} {
\path [draw, very thin]
(\where) -- (\where |- timeorigin)
node [anchor=north, xshift=\sh, text height=\baselineskip, yshift=7pt] {\small \capt};
}
\end{tikzpicture}
}



\def\figgawidth{2.8}
\def\figgaheight{1.7}
\def\figgayshift{0.15}
\def\figgaboolwidth{0.75}
\def\figgaboolheight{0.9}
\def\figgabooltomux{1.0}
\def\figgaconstwidth{0.5}
\def\figgaconstheight{0.5}
\def\figgaconsttomux{0.3}
\def\figgamuxwidth{0.45}
\def\figgamuxheight{1}
\def\figgamuxindist{50}
\def\figgamuxtoout{0.25}
\def\figgaoutlen{0.25}
\def\figgarstlen{0.15}
\def\figgainlen{0.25}

\def\tikzfiguregate{
\begin{tikzpicture}[
gate/.style={draw, inner sep=0},
xscale=1,
yscale=0.4,
>=latex'
]

\coordinate (origin) at (0,0);
\coordinate (topright) at (\figgawidth,\figgaheight);

% gate border
\path [draw] (origin) rectangle (topright);

% mux & output
\path [draw]
($(origin -| topright)!0.5!(topright) + (0,\figgayshift)$)
+(\figgaoutlen,0) node [anchor=west] {\small $v$} --
+(-\figgamuxtoout,0) node (mux) [draw, trapezium, anchor=top side,
shape border rotate=270, minimum width=\figgamuxheight*1cm, 
minimum height=\figgamuxwidth*1cm, trapezium stretches body] {};

% internal gates
\foreach \muxin/\muxincpt/\name/\cpt in {1/0/bool/$b$, -1/1/const/$I$} {
\path [draw] 
(mux.{180+\figgamuxindist*\muxin}) node [anchor=west, xshift=-1.5pt] {\footnotesize \muxincpt} --
++(-\csname figga\name tomux\endcsname, 0)
node (\name) [gate, anchor=east, minimum width=\csname figga\name width\endcsname*1cm,
minimum height=\csname figga\name height\endcsname*1cm] {\small \cpt};
}

% reset
\path [draw, densely dashed, thin]
(mux.north) -- (mux.north |- topright) --
++(0, \figgarstlen) node [anchor=south, inner sep=1.5pt] {\small\em reset};

% inputs
\foreach \pos in {0.125, 0.375, 0.625, 0.875} {
\path [draw]
($(bool.north west)!\pos!(bool.south west)$) coordinate (tmp) --
(tmp -| origin) -- ++(-\figgainlen,0);
}
\end{tikzpicture}
}


\newcommand\myBoolean[1]{
% the mux  
\path [draw]
(0,0)
+(\figgaoutlen,0) --
+(-\figgamuxtoout,0) node (mux) [draw, trapezium, anchor=top side,
shape border rotate=270, minimum width=\figgamuxheight*1cm, 
minimum height=\figgamuxwidth*1cm, trapezium stretches body] {};

% reset in mux
\path [draw]
(mux.south) -- ++(0,-0.2)
 node [below] {\small R};

% the out channel
\path[draw]
(mux.east)
++(\figgaoutlen,0)
node [gate] {} node [anchor=north, yshift=20pt, text height=\baselineskip] {${#1}$};

% the gate
\path[draw]
($(mux) + (-1.25,-0.55)$)
node (outc1) [gate2, anchor=west, minimum width=\figgaboolwidth*1cm,
  minimum height=\figgaboolheight*0.8cm,fill=white] {\small $f_{#1}$};

\path [draw] ($(outc1.east) + (0,0.2)$) -- ++(0.25,0); % out
\path [draw] ($(outc1.west) + (0,0.2)$) -- ++(-0.25,0); % in
\path [draw] ($(outc1.west) + (0,-0.2)$) -- ++(-0.25,0); % in


% internal gates
\foreach \muxin/\muxincpt/\name in {1/0/bool, -1/1/const} {
\path [draw] 
(mux.{180+\figgamuxindist*\muxin}) node [anchor=west, xshift=-1.5pt] {\footnotesize \muxincpt};
node (\name) {}; 
}

% the inital I
\path [draw]
($(mux) + (-0.5,0.35)$)
node (huhu) [gate2, anchor=east, minimum width=\figgaconstwidth*1cm,
  minimum height=\figgaconstheight*1cm] {\small $I_{#1}$};

\path [draw]
(huhu.east) -- ++(0.25,0);
}



% the branching channs
\def\tikzfigurechannelbranch{
\begin{tikzpicture}[
gate/.style={draw, fill, circle, minimum size=3pt, inner sep=0},
gate2/.style={draw, inner sep=0},
xscale=1,
yscale=0.8,
>=latex'
]

\coordinate (origin) at (0,0);
\coordinate (topright) at (\figgawidth,\figgaheight);

\myBoolean{v}

\pgftransformshift{\pgfpoint{4cm}{0.75cm}}
\myBoolean{w}

\pgftransformshift{\pgfpoint{0cm}{-2cm}}
\myBoolean{z}

% ----------- the channels between ------------------------------

\path [draw]
(-4,1.252) -- ++(2.25,0)
node (c1) [pos=0.5,draw,fill=white,rectangle, rounded corners, minimum width=\figgamuxheight*1cm, 
  minimum height=\figgamuxwidth*1cm] {$c_1$};

\path [draw]
(-4,1.252) -- ++(0,-1.6)
           -- ++(2.25,0)
node (c1) [pos=0.5,draw,fill=white,rectangle, rounded corners, minimum width=\figgamuxheight*1cm, 
  minimum height=\figgamuxwidth*1cm] {$c_2$};

% ----------- left part -----------------------------------------
\pgftransformshift{\pgfpoint{-8.2cm}{1cm}} 

% xyz nodes
\path
(0,0) node (x) [gate] {} node [below] {$v$}
-- +(1,0.5) node (y) [gate] {} node [anchor=north, yshift=20pt, text height=\baselineskip] {$w$}
 node[pos=0.4,above] {$c_1$};

\path
(x) 
-- +(1,-0.5) node (z) [gate] {} node [below=-7pt, text height=\baselineskip] {$z$}
 node[pos=0.4,below] {$c_2$};

% edges
\foreach \from/\to in {x/y, x/z} {
\path [draw, ->] (\from) -- (\to);
}

\draw (1.6,0)
node {$\equiv$};

\end{tikzpicture}
}


%%% Local Variables: 
%%% mode: latex
%%% TeX-master: "paper"
%%% End: 
