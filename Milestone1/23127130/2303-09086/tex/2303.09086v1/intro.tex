\section{Introduction}

\begin{figure*}[!t]
    \hspace{0.06\textwidth}
  \begin{subfigure}[t]{0.48\textwidth}
    \centering
    \includegraphics[width=0.9\textwidth]{figures/intro_figure_b.pdf}
  \end{subfigure}
  \begin{subfigure}[t]{0.32\textwidth}
    \centering
    \includegraphics[width=0.9\textwidth]{figures/web_graph_intervention_a.pdf}\label{fig_web_count}
  \end{subfigure}\vfill
  \hspace{0.06\textwidth}
  \begin{subfigure}[t]{0.48\textwidth}
    \centering
    \includegraphics[width=0.9\textwidth]{./figures/intro_figure_a.pdf}
  \end{subfigure}
  \begin{subfigure}[t]{0.32\textwidth}
    \centering
    \includegraphics[width=0.9\textwidth]{./figures/web_graph_intervention_b.pdf}\label{fig_web_sin}
  \end{subfigure}
  \caption{Comparison of our algorithm (namely, Frank-Wolfe-EC) and several existing approaches, including K-EdgeSelection, Weighted Reduction, and Uniform Reduction (See Section \ref{sec_exp_setup} for a description of these approaches). 
  On the top panel, we report the number of infections (whose scale should be multiplied by $10^3$), averaging over fifty simulations.
  We observe that our approach can be used to reduce the number of infections and the largest singular value of the weight matrix of the diffusion process.
  At a high level, our approach works by:
  (i) Connecting the edge centrality score with the gradient of the sum of the top singular values; See Lemma \ref{prop_grad}.
  (ii) Showing that each iteration of the Frank-Wolfe algorithm can be solved efficiently with a greedy selection procedure; See Lemma \ref{thm_optimal_descent}.
  Moreover, we show that this approach applies to both static and time-varying graphs.
  }\label{fig_intro}
\end{figure*}


Suppose there is a spreading process, such as an epidemic propagating through a graph. Denote the graph as $G = (V, E)$. How would we reduce the number of affected nodes from $V$ during the spreading process? Many studies have considered this question in the network immunization literature \cite{chen2015node,chen2016eigen}, motivated by considerations for controlling the outcome of the diffusion process \cite{pastor2015review}.
A principal approach from the existing literature is to optimize spectral properties of $G$ with edge removal procedures.
For example, \citet{tong2012gelling} design algorithms to reduce the largest eigenvalue of $G$'s adjacency matrix by removing a budgeted number of edges.
\citet{le2015met} further study how to reduce the largest $r$ eigenvalues under a budget constraint of edge removals.
In this work, we revisit the spectral optimization approach on weighted graphs.
Let $W$ denote a non-negative weight matrix corresponding to the edge weights of $G$.
We consider edge-weight reduction with a budgeted amount of $B$ that will create the most drop in the largest $r$ eigenvalues of $WW^{\top}$.

For example, weighted graphs have appeared in recent studies about the pandemic.
\citet{chang2021mobility} study the counterfactual outcome of implementing edge-weight reduction strategies in mobility networks.
Reducing edge weights in mobility networks corresponds to restricting the mobility of population groups.

An effective algorithm for reducing the top singular values  of a graph is by removing edges with the highest centrality scores \cite{tong2012gelling}.
Let $\lambda_1(W)$ denote the largest singular value of $W$ (notice that the largest eigenvalue of $WW^{\top}$ is equal to the square of $\lambda_1(W)$).
Let $\vec u_1$ and $\vec v_1$ denote the left and right singular vectors corresponding to $\lambda_1(W)$, respectively.
The edge centrality score of an edge $(i, j)$ is equal to $\vec u_1(i)\cdot\vec v_1(j)$, where $\vec u_1(i)$ is the $i$-th entry of $\vec u_1$ and $\vec v_1(j)$ is the $j$-th entry of $\vec v_1$.
\citet{tong2012gelling} show that removing edges with the highest edge centrality scores effectively reduces $\lambda_1(W)$.
\citet{chen2018network} further quantifies the approximation ratio of this greedy algorithm using submodular optimization techniques (see also \citet{saha2015approximation}).
These works focus on the case of unweighted graphs, for which the spectral optimization problem given a budgeted amount of edge removals is NP-hard \cite{chen2016eigen}.
Notice that in the case of weighted graphs, the weight of an edge can be reduced by a fraction. 
\citet{yu2021potion} apply gradient-based optimization for targeted diffusion, which also applies to weighted graphs, with a stopping criterion until the gradient gets close to zero.

To motivate our approach, we begin by observing that the edge centrality score from the work of \citet{tong2012gelling} is equal to the gradient of the largest singular value of $W$ squared, up to a scaling of $2\lambda_1(W)$ (See Lemma \ref{prop_grad} for the full statement):
{\[ \frac{\partial\Big(\big(\lambda_1(W)\big)^2\Big)}{\partial W_{i, j}} = 2\lambda_1(W) \cdot \vec u_1(i) \cdot \vec v_1(j). \]}%
Notice that the above corresponds to the rank-$1$ SVD of $W$.
More generally, for any rank $r$, the gradient of the largest $r$ singular values can be efficiently computed via a rank-$r$ SVD of $W$.
Based on the connection between edge centrality and gradients, we minimize the largest $r$ eigenvalues of $WW^{\top}$ via the Frank-Wolfe algorithm, which involves direction finding and line search.
We show an efficient way to find the descent direction by reducing edges with the highest generalized edge centrality score (see Lemma \ref{thm_optimal_descent}).
We then recompute the eigenscores at each iteration, which is also related to the approach of \citet{le2015met}.
By comparison, our algorithm adapts to weighted graphs and is guaranteed to converge to the global optimum (see Theorem \ref{prop_continuous}).

With the connection between edge centrality and gradients, we extend our algorithm to time-varying networks, which include a sequence of graphs with evolving weight matrices.
We provide the generalized eigenscore for each edge of every graph in the sequence and design an algorithm for optimizing the largest $r$ eigenvalues of the product of all weight matrices in the sequence (cf. \citet[Sec. 4.2]{prakash2010virus}).

We evaluate our algorithms by simulating an epidemic model on eleven weighted graphs.
In the static case, our approach achieves, on average, $\mathbf{25.5\%}$ improvement over baselines during SEIR model simulations (cf. Section \ref{sec_prelim} for descriptions).
The largest singular value decreases by an average of $\mathbf{25.1\%}$ more than the baselines.
See Figure \ref{fig_intro} for an illustration.
Meanwhile, our approach is also effective for SIR and SIS models (see Appendix \ref{sec_epi_models}, where we describe both models).
Further, our algorithm reduces the number of infections by over $\mathbf{6.9\%}$ for several time-varying networks.

\smallskip
\noindent\textbf{Organization.} The rest of our paper is organized as follows.
In Section \ref{sec_prelim}, we formally define the spectral optimization problem on weighted graphs.
Then in Section \ref{sec_alg}, we develop two algorithms for this problem on static and time-varying networks.
We validate our approach with extensive experiments in Section \ref{sec_exp}.
Lastly, we discuss several related pieces of literature in Section \ref{sec_related} and questions for future work in Section \ref{sec_discuss}.







