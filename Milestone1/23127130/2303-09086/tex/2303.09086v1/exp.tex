\section{Experiments}\label{sec_exp}

We evaluate our proposed approaches on various weighted graphs and mobility networks.
Our experiments seek to address the following questions:
First, does our approach reduce the infections and the largest singular values well compared to methods from prior works? 
Second, what are the effects of each component in our approach, e.g., setting the rank $r$, running iterative greedy selection, and setting the budget?
Third, does our approach run efficiently in practice? 
We present positive results to answer these three questions, validating the practical benefit of our algorithm. {The code repository for reproducing our results can be found online at \url{https://github.com/NEU-StatsML-Research/Designing-Intervention-on-Mobility-Networks}.}

\begin{algorithm}[!t]
	\caption{Frank-Wolfe for Time-Varying Networks}\label{alg_edgecen_temporal}
	\begin{footnotesize}
		\begin{algorithmic}[1]
		    \Input A sequence of graphs with weight matrix $\cW$ in $s$ steps.
            \Param Same as the static case.
		    \Output A sequence of matrices $\cM$ modified from $\cW$.
			\Procedure{\LPTV}{$\cW, B; T, H$}
			    \State Let $\cM_0 = \cW$
			    \For {$t = 0, 1, \dots, T-1$}
			        \State $\mathcal{G}_t = \{G_t^{\star(i)}\}_{i=1}^s$ = \edgecenTV($\cW, B; \cM_t$)
			        \State Set $\eta_t$ by minimizing $f\big((1 - \eta_k)\cM_t + \eta_t \cG_k\big)$ for $\eta_k \in H$
			        \State $\cM_{t+1} = \{M_{t+1}^{(i)} = (1 - \eta_t) M_t^{(i)} + \eta_t  G_t^{\star(i)}: 1\le i\le s\}$
			    \EndFor
			    \If {there is unused budget in $\cM_T$}
			        \State $B' = B - \sum_{i=1}^{s}\textup{sum}(W^{(i)} - M_T^{(i)})$
			        \State $\cM^{\star}$ = \edgecenTV($\cM_T, B'; \cM_T$)
			    \EndIf
			    \State \Return $\cM^{\star}$
			\EndProcedure
			\vspace{0.0325in}
			\Procedure{\edgecenTV}{$\cW, B; \cM$}
			    \State Let $\tilde X_r$ be the rank-$r$ SVD of $X = \prod_{i=1}^{s}M^{(i)}$ \label{alg_ec_temporal}
			    %\State Let the edge centrality score for edge $(i, j)$ in $\cE^{(t)}$ as  $\big(\tilde M_r\prod_{t'\neq t}M^{(t')}\big)_{i, j}$
			    \State Sort the edges in the \emph{union} of $\cE^{(1)}, \cE^{(2)}, \dots, \cE^{(s)}$ by their edge centrality scores (cf. Eq. \ref{eq_tv_ec}); let $k$ be the first value such that the total top-$k$ edge weights from $\cW$ exceed $B$
			    \State Reduce the first $k-1$ edges' weight to zero and the last edge's weight by the remaining budget 
			    %\State Update the weight of $i$th edge in $S$ by Equation \ref{eq:update_weight_i}. 
			    \State \Return the updated $\cW$
			\EndProcedure	
	\end{algorithmic}
	\end{footnotesize}
\end{algorithm}

\subsection{Experimental setup}\label{sec_exp_setup}

We use three weighted graphs in our model simulations on static networks: (i) An airport traffic network of flights among commercial airports worldwide.  (ii) A trust network of users on the Advogato platform; (iii) A trust network of users on a Bitcoin platform.
The edge weights in the Airport network denote the number of flight routes between two airports.
Edge weights in the last two networks denote different levels of declared trust among users. The edge weights on the Advogato network are between $0$ and $1$. The edge weights on the Bitcoin network range from $-10$ to $10$. We scale the weights to positive by $\exp(w/5)$. 
The statistics of the networks are listed in Appendix \ref{sec_add_setup}.

\vspace{0.03in}


Besides, we use eight mobility networks constructed with the procedure described in \citet{chang2021mobility}. 
We generate the mobility networks based on the mobility patterns of eight cities. The edge weights denote the population that moves from a group to a location from March 2, 2020, to May 10, 2020. 
The mobility patterns cover 25,341 census block groups with over 65 million people and 147,638 points of interest. We report the statistics of the mobility networks in Table \ref{tab:num_infected}.
We defer a comprehensive discussion of the construction procedure to their paper.
% The construction uses the mobility patterns, geometry, and population census datasets from the SafeGraph Consortium.

\vspace{0.03in}

We use two sequences of weighted trust networks from Bitcoin-Alpha and Bitcoin-OTC platforms for time-varying networks. Each sequence contains ten trust relationship networks corresponding to five periods. The edge weights are processed in the same way as in the static Bitcoin network.
We also construct time-varying mobility networks corresponding to ten weeks of the same period above for Chicago and Houston.
We describe network data sources in Appendix \ref{sec_add_setup}. 
% Additionally, we will use the reported cases of COVID-19 infections from The New York Times to calibrate the SEIR model.

\begin{table*}[t!]
\centering
\caption{
\textbf{Top:} Dataset statistics for eight mobility networks.
\textbf{Middle:} Comparison of the largest singular value of the edge-weight reduced matrix.
\textbf{Bottom:} Comparison of the total number of infected populations ($\times 10^3$) in SEIR model simulations. 
% We modify the edge weights using the strategy in each row.
We report the average number of infections from fifty independent simulations.}\label{tab:num_infected}
\begin{scriptsize}
\begin{tabular}{@{}lcccccccccc@{}}
\toprule
Graphs & AT & CH & DA & HO & MI & NY & PH & DC \\ \midrule
Nodes  & 11,232     & 32,390     & 19,069     & 38,895                  & 17,858        & 34,216   & 18,649 & 10,590             \\
%\# CBGs  & 2,799     & 5,784     & 4,069     & 4,029                  & 2,279        & 10,170   & 3,547                              & 2,564             \\
%\# POIs  & 8,433     & 26,606    & 15,000    & 34,866                 & 15,559      &  24,046   & 15,102                            & 8,026           \\
Edges & 154,729   & 439,262   & 283,928   & 671,217                & 276,109      & 463,719  & 260,279                            & 107,733           \\
% Avg. node weight & 2,400 & 1,593 & 2,069 & 2,395 & 2,219 & 1,578 & 1,568 & 2,060 \\ 
Avg. edge weight & 5.258 & 4.659 & 4.921 & 4.951 & 4.833 & 4.749 & 4.864 & 4.848 \\
\midrule\midrule
Largest singular value & AT & CH & DA & HO & MI & NY & PH & DC \\ \midrule
No Intervention     &  5526 & 1296 & 2093 & 14677 & 555 & 2413 & 12032 & 1406 \\
Uniform Reduction  &  5250 & 1231 & 1988 & 1394 & 527 & 2292 & 1143 & 1336 \\
Weighted Reduction &  1254 &  302 &  564 &  420 & 213 &  4818 &  374 &  365 \\
Max Capping  &  5250 & 1231 & 1988 & 1394 & 527 & 2292 & 1143 & 1336 \\
POI Category       &  5526 & 1295 & 2073 & 1467 & 555 & 2270 & 1202 & 1375 \\ 
K-EdgeDeletion     &  1565 &  257 &  417 &  447 & 216 &  355 &  282 &  227 \\
\edgecenshort{}    &  1565 &  257 &  417 &  447 & 216 &  355 &  282 &  226 \\
\textbf{Ours (Alg. \ref{alg_edgecen})}         &  \textbf{1191} & \textbf{125} & \textbf{308} & \textbf{235} & \textbf{169} & \textbf{197} & \textbf{190} & \textbf{188} \\\midrule \midrule
Infected populations                & AT & CH & DA & HO & MI & NY & PH & DC \\ \midrule
No Intervention  &  48$\pm$3 & 1858$\pm$46 & 91$\pm$21 & 366$\pm$26 & 752$\pm$26   & 3146$\pm$21 & 492$\pm$20 & 41$\pm$2 \\
Uniform Reduction   &  46$\pm$2 & 1762$\pm$64 & 84$\pm$11 & 312$\pm$26 & 671$\pm$23  & 2996$\pm$40 & 463$\pm$12 & 41$\pm$1 \\
Weighted Reduction    &  43$\pm$2 & 782$\pm$86 & 66$\pm$3 & 194$\pm$18 & 43$\pm$12  & 1336$\pm$60 & 342$\pm$10 & 40$\pm$1 \\
Max Capping    &  44$\pm$2 & 1741$\pm$65 & 82$\pm$8  & 315$\pm$33 & 675$\pm$26   & 2990$\pm$45 & 455$\pm$15 & 41$\pm$1 \\
POI Category    &  46$\pm$3 & 1728$\pm$62 & 77$\pm$8  & 283$\pm$31 & 687$\pm$25   & 2950$\pm$38 & 458$\pm$17 & 41$\pm$1 \\
K-EdgeDeletion & 44$\pm$2 & 346$\pm$40 & 64$\pm$2 & 186$\pm$18 & 78$\pm$8 & 352$\pm$27 & 185$\pm$10 & 39$\pm$1 \\
\edgecenshort{} &  45$\pm$3 & 355$\pm$46 & 64$\pm$2 & 187$\pm$21 & 78$\pm$7 & 362$\pm$36 & 178$\pm$11 & 39$\pm$1 \\
\textbf{Ours (Alg. \ref{alg_edgecen})}      &  \textbf{40$\pm$1} & \textbf{166$\pm$16} & \textbf{62$\pm$2} & \textbf{86$\pm$10} & \textbf{8$\pm$2} & \textbf{301$\pm$88} & \textbf{129$\pm$13}	& \textbf{39$\pm$1} 
\\\bottomrule
\end{tabular}
\end{scriptsize}
\end{table*}

\smallskip
\noindent\textbf{Baseline methods.}
The experiments of spreading on static networks involve the following baseline methods: 
% \begin{itemize}[leftmargin=0.15in]
(1) K-EdgeDeletion: Delete a set of edges with the highest edge centrality scores according to the best rank-1 approximation of $W$ \cite{tong2012gelling}. 
(2) Weighted reduction: Reduce the weight of every edge by a ratio that is proportional to its weight. (3) Uniform reduction: Uniformly reduce the weight of every edge by the same fraction.
(4) Max occupancy capping: Reduce the cumulative weights at each POI proportional to its max occupancy.
(5) Capping by POI category: Cap the maximum occupancy of a particular category of POIs.
% \end{itemize}
The last three baselines are adapted from \citet{chang2021mobility}.



We consider a similar set of baseline methods for time-varying networks, including uniform reduction,  weighted reduction, and the K-EdgeDeletion method \cite{tong2012gelling}. 
The difference from methods on static networks is that edge weight reduction strategies are applied to all edges in the sequence of networks. 
% Besides, we consider the K-EdgeDeletion method  \cite{tong2012gelling} to modify each network weight matrix in the sequence and vary the budget allocation to each network with following allocation schemes: 
% (1) First week only: Assign all the edge-weight reduction budget to the first week of the sequence.
% (2) Uniform allocation: Distribute the budget uniformly among every network in the sequence. 
% (3) Exponential allocation: Distribute the budget proportional to $\exp({-s})$, decaying exponentially over time.

%The second set of baseline and competing methods focuses on edge-weight reduction strategies for (a sequence of) temporal networks.
% For the experiments on a sequence of temporal networks, we will only use Algorithm \ref{alg_edgecen} to modify the network weight matrix while varying the budget allocation scheme.
% We consider the following list of allocation schemes: 
% \begin{itemize}[leftmargin=0.15in]
% (1) First week only: Assign all the edge-weight reduction budget to the first week of the sequence.
% (2) Uniform allocation: Distribute the budget uniformly among every network in the sequence. 
% (3) Exponential allocation: Distribute the budget proportional to $\exp({-t})$, decaying exponentially over time.
%    \item Weighted allocation: Distribute the budget $B$ proportional to the total sum of edge weights of every weekly network. 
% \end{itemize}



\medskip
\noindent\textbf{Implementation.}
We simulate an SEIR model on each weighted network.
On weighted graphs, a node can get infected by its infectious neighbors with a probability equal to the edge weight times the transmission rate. 
We use a transmission rate of 0.05 and an initially exposed ratio of 0.01.
We follow the procedure of \citet{chang2021mobility} on mobility networks to simulate a metapopulation SEIR model in each network where one SEIR model is instantiated for each CBG. 
We calibrate the parameters of SEIR models so that the simulated cases approximate the reported cases from New York Times COVID-19 data. 
Besides, we also evaluate our algorithm on other variants of epidemic models, including SIR and SIS with the same parameters. 
We describe the simulation setup details in Appendix \ref{sec_add_setup}.
For completeness, a brief description of the epidemic models is provided in Appendix \ref{sec_epi_models}. 

In Algorithm \ref{alg_edgecen} and \ref{alg_edgecen_temporal}, we search the rank parameter $r$ in $[1, 50]$ and the number of iterations in $[5, 30]$. For each result reported in Section \ref{sec_exp}, we search the two hyper-parameters 50 times. 
We use an edge-weight reduction budget of 5\% of the total edge weights. Results of using other budget amounts are consistent and are discussed in Section \ref{sec_ablation}.
We use 30 values from the range of $[10^{-3}, 10^{-1}]$ as the range of learning rate $H$. 
% In each iteration, we conduct a grid search over the learning rates and choose the one that leads to the smallest object value. 
For weighted graphs, we directly use the weight matrix as $W$. We compose the weight matrix $W$ for mobility networks by multiplying the bipartite network matrix and its transpose.
All the experiments are conducted on an AMD 24-Core CPU machine.



\subsection{Experimental results}\label{sec:main_results}%\hfill
%\noindent 
Our algorithms effectively control infections by reducing the largest singular value on a range of static and time-varying networks. We observe consistent results across various epidemic models, including SEIR, SIR, and SIS.
% We apply our approach to simulated SEIR models on static networks and demonstrate its advantage over previous methods in reducing the largest singular value and controlling infections. We show that our approach on time-varying networks also help reduce the number of infections. Besides, our approach consistently improves over baseline methods on SIR and SIS epidemic models.
\vspace{-0.025in}
\begin{itemize}[leftmargin=0.15in]
\setlength\itemsep{0.0em}
\item {\bf Drop in the largest singular value:}
Figure \ref{fig_intro} illustrates the largest singular value of the modified weight matrix of the three weighted graphs.
\LPshort{} reduces the largest singular value more than baselines by \textbf{11.4}\% on average.
Additionally, Table \ref{tab:num_infected} reports the largest singular value of modified mobility networks.
\LPshort{} is $\mathbf{30.7\%}$ more effective than the best baseline  on average.

\item {\bf Reduced number of infections:}
Figure \ref{fig_intro} compares our algorithm to baseline intervention strategies on three weighted graphs.
Overall, our algorithm reduces the number of infected nodes by $\mathbf{10.4\%}$ more than baselines on average.
Table \ref{tab:num_infected} compares the total number of infected populations on eight mobility networks.
Note that ours outperform other baselines by $\mathbf{30.1\%}$ on average and up to $\mathbf{80.3\%}$.
% Additionally, we observe that the trend is consistent with Table \ref{tab:num_infected} during the entire spreading process.

\item {\bf Results for time-varying networks:}
On time-varying networks, \LPTVshort{} also outperforms other baselines.
The number of infections is smaller by $\mathbf{6.9\%}$ averaged over both  time-varying weighted graphs and mobility networks.

\item {\bf Simulation using SIS and SIR:} 
Our approach also helps reduce infections in SIR and SIS epidemic models. We observe that \LPshort{} reduces the number of infections by $\mathbf{14.7\%}$ and $\mathbf{10.8\%}$ more on average over the eight static mobility networks.
\end{itemize}


\subsection{Ablation studies}\label{sec_ablation}%\hfill
%
%\noindent 
We ablate the parameters in our approach and provide further insights into the properties of our algorithm. 
% We show that our approach benefits from using higher ranks and leveraging iterative updates. Additionally, our approach remains effective across various budget settings.
\begin{itemize}[leftmargin=0.15in]
\setlength\itemsep{0.0em}

\item {\itshape Benefit of choosing ranks:}
Recall that our algorithm requires specifying the rank $r$--the number of top singular values--in Equation \ref{eq_convex}. 
We hypothesize that varying the rank $r$ would lead to different intervention results. 
We ablate the performance of our algorithm by using different $r$ in a range of $[1, 50]$. The results show that the performance of the best choice $r$ outperforms using $r=1$ by \textbf{40.2\%} averaged over all networks.
This result justifies our formulation of the network intervention problem as an optimization for the sum of largest-$r$ singular values instead of only the largest single value.

\item {\itshape Benefit of being iterative:} The greedy selection algorithm \edgecenshort{} can be viewed as a special case of  \LPshort{} with $T = 1$.
Notice that our iterative approach is necessary to achieve the observed performance.
In Table \ref{tab:num_infected},  \LPshort{} outperforms \edgecenshort{} by \textbf{31.4}\% on average, and the largest singular value is reduced by \textbf{33.1}\% more.

\item {\itshape Varying budget $B$:}
We have also observed similar results by varying the budget for mobility reduction. 
We vary the budget from 1\% to 20\% using the New York mobility network.
Our algorithm outperforms the baselines consistently using different budgets, similarly for the largest singular value.
Interestingly, when the budget level is small (e.g., 1\%), \LPshort{} reduces the largest singular value more significantly than baseline methods.
\end{itemize}

\subsection{Runtime report}\label{sec:scalability}%\hfill
%
%\noindent
Across all eleven graphs, our approach converges within 30 iterations (or 17 on average).
Each iteration requires an SVD step that takes less than 3 seconds.
The other steps in each iteration require less than 2.7 seconds.
For larger graph instances, we run our method on seven graphs with the number of edges included: com-Orkut (117M), com-LiveJournal (34M), wiki-topcats (28M), web-BerkStan (7.6M), web-Google (5.1M), web-Stanford (2.3M), and web-NotreDame (1.4M) from the SNAP datasets.
Figure \ref{fig:runtime_per_iteration} reports the runtime for one iteration of our algorithm.
Notice that the runtime scales almost linearly with the number of edges. Our algorithm takes 4943 seconds on the largest graph with 117M edges and 3M nodes.
These results show that our algorithm runs efficiently on large-scale graphs.

\begin{figure}[!h]
	\centering
	%\vspace{-0.05in}
	\includegraphics[width=0.3234\textwidth]{figures/plot_running_time.pdf}%\vspace{-0.15in}
	%\vspace{-0.15in}
	\caption{Runtime of Frank-Wolfe-EC in log-log scale for one iteration. The number of  edges ranges from $10^4$ to $10^8$, and the number of  nodes ranges from $10^3$ to $10^6$.}
	\label{fig:runtime_per_iteration}
	%\vspace{-0.2in}
\end{figure}
% \begin{table*}[!t]
% \centering
% \caption{Basic statistics for eight mobility networks constructed from SafeGraph's monthly patterns data.}\label{table_dataset}
% %\vspace{-0.1in}
% \begin{scriptsize}
% \begin{tabular}{@{}lcccccccccc@{}}
% \toprule
%                 & AT & CH & DA & HO &  MI & NY & PH &  DC \\ \midrule
% Number of census block groups  & 2,799     & 5,784     & 4,069     & 4,029                  & 2,279        & 10,170   & 3,547                              & 2,564             \\
% Number of points of interest  & 8,433     & 26,606    & 15,000    & 34,866                 & 15,559      &  24,046   & 15,102                            & 8,026           \\
% Number of nonzero weighted edges & 154,729   & 439,262   & 283,928   & 671,217                & 276,109      & 463,719  & 260,279                            & 107,733           \\
% Average edge weight & 5.258 & 4.659 & 4.921 & 4.951 & 4.833 & 4.749 & 4.864 & 4.848 \\
% Average population per group & 2400.402 & 1593.152 & 2069.406 & 2395.407 & 2219.864 & 1578.068 & 1568.618 & 2060.778 \\
% \bottomrule
% \end{tabular}
% \end{scriptsize}
% \end{table*}