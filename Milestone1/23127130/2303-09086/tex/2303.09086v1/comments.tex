

	\end{subfigure}
	\hfill
	\begin{subfigure}[b]{0.45\textwidth}
		\centering
		\includegraphics[width=\textwidth]{figures/beta09log}
		\caption{log scale}
	\end{subfigure}
	\caption{$\beta=0.9, \max_t{|I_t|}\approx 943.7,
		\max_t|R_t|\approx 8312.0$}
	\label{figure2}
\end{figure}


\begin{figure}[h]
	\centering
	\begin{subfigure}[b]{0.3\textwidth}
		\centering
		\includegraphics[width=\textwidth]{figures/beta042}
		\caption{$\beta=0.42, \\
			\max_t{|I_t|}\approx 0.5335,\\
			\max_t|R_t|\approx 30.96$}
	\end{subfigure}
	\hfill
	\begin{subfigure}[b]{0.3\textwidth}
		\centering
		\includegraphics[width=\textwidth]{figures/beta044}
		\caption{$\beta=0.44, \\
			\max_t{|I_t|}\approx 1.211,\\
			\max_t|R_t|\approx 214.1$}
	\end{subfigure}
	\hfill
	\begin{subfigure}[b]{0.3\textwidth}
		\centering
		\includegraphics[width=\textwidth]{figures/beta05}
		\caption{$\beta=0.5,\\
			\max_t{|I_t|}\approx 51.45,\\
			\max_t|R_t|\approx 2573.3$}
	\end{subfigure}
	\caption{Threshold figures}
	\label{fig:three graphs}
\end{figure}

In these pictures, notice that the number of infected nodes ticks up initially when $\beta > 0.433$, and ticks down when later $\beta\frac{|S|}{N} < 0.433$. For $\beta= 0.42$, the number of infected nodes ticks down immediately.