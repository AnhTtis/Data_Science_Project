\section{Related Work}\label{sec_related}

There is a significant amount of work on diffusion processes on networks.
A detailed survey from an epidemic perspective can be found in  \citet{pastor2015review}.
A key result in the literature is that the largest eigenvalue of the adjacency matrix (a.k.a.~the spectral radius) characterizes the epidemic threshold for many propagation models \cite{wang2003epidemic,ganesh2005effect,prakash2012threshold}.
An important implication of this result is that the epidemic dies out if the spectral radius decreases, and this has motivated many works on epidemic control \cite{van2011decreasing,le2015met,chen2016eigen}.
Because eigen-optimization problems via edge additions or deletions are NP-hard \cite{khalil2014scalable}, both heuristic solutions and principled approximation algorithms have been investigated.
A practical approach in the literature is following the greedy algorithm with a node centrality \cite{chen2015node} or edge centrality notion \cite{tong2012gelling} (see also several alternative edge centrality notions in link recommendation \cite{parotsidis2016centrality} and distance sketching \cite{zhang2019pruning}).
Our notion of edge centrality follows the edge centrality notion studied in \citet{tong2012gelling}.
A related literature studies diffusion control in the Firefighter problem \cite{anshelevich2009approximation}.
Besides epidemic spreading, diffusion processes are also studied in social networks (e.g., \cite{matsubara2012rise,goel2015note,haghtalab2017monitoring}), and financial transaction networks \cite{goel2014connectivity}.

Our work applies the Frank-Wolfe algorithm, a classic algorithm for constrained optimization  \cite{frank1956algorithm,nocedal2006numerical} to study graph spectral optimization.
The Frank-Wolfe algorithm and its theoretical property are well-studied in the machine learning and optimization literature (see, e.g., \citet{jaggi2013revisiting}, \citet{tajima2021frank}, and the references therein). %
We observe a connection between edge centrality and gradients which significantly speeds up the Frank-Wolfe algorithm compared with a naive implementation using a linear program solver.
One relevant application for our approach is to consider node-level intervention measures.
For mobility networks, reducing the weight of a node means restricting a particular group or location's mobility.
Our approach can naturally extend to node-level reduction by similarly deriving node centrality scores as gradients.
Besides, there are also methods for speeding up eigenscore computation on dynamic graphs \cite{chen2017eigen,zhang2016approximate}. It is conceivable that one could combine this method with our approach to achieve the best of both worlds.
Finally, there are studies on the design of vaccine distribution for pandemic control \cite{zhang2014scalable,sambaturu2020designing} and optimization for network robustness \cite{chan2016optimizing}.
It would be interesting to use the new tools developed in this paper to study these related problems.



