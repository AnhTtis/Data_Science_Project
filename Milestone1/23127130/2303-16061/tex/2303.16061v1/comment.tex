\documentclass{article}

\usepackage[english]{babel}
\usepackage{amsmath,amssymb,amsfonts,amsthm}
\usepackage{hyperref}
\usepackage{graphicx}
\usepackage{url}
\usepackage{acronym}
\usepackage{subcaption}
\usepackage{colortbl}
\usepackage{xfrac}
\usepackage[inline]{enumitem}
\usepackage[numbers]{natbib}
\usepackage{textcomp}
\usepackage[htt]{hyphenat}
\usepackage{enumitem}

\newtheorem{example}{Example}

\let\emptyset\varnothing

\definecolor{Gray}{gray}{0.80}
\definecolor{nfyellow}{rgb}{1.00, 0.75, 0.0}

%\let\cite\citep
%\renewcommand{\theequation}{\thesection.\arabic{equation}} 
%\numberwithin{equation}{section} 


\newcommand{\quotes}[1]{``.1"}
\renewcommand{\acffont}[1]{\textsl{#1}}
\newcommand{\overbar}[1]{\mkern 1.5mu\overline{\mkern-1.5mu#1\mkern-1.5mu}\mkern 1.5mu}


\begin{document}

\title{A comment to ``A General Theory of IR Evaluation Measures''}

\author{Fernando Giner\thanks{E.T.S.I. Inform\'{a}tica UNED, C/ Juan del Rosal, 16, 28040-Madrid, Spain (email: fginer3@gmail.com)}}


\maketitle

\begin{abstract}
The paper ``\emph{A General Theory of IR Evaluation Measures}'' develops a formal framework to determine whether IR evaluation measures are interval scales. This comment shows some limitations about its conclusions.
\end{abstract}

\section{Introduction}
Recently, the paper ``\emph{A General Theory of IR Evaluation Measures}''~\cite{FerranteEtAl2018b} has provided a formal framework, based on the \acf{RTM}~\citep{KrantzEtAl1971,LuceEtAl1990,SuppesEtAl1989}, for both set-based and rank-based IR evaluation measures as well as both  binary and multi-graded relevance, determining whether retrieval measures are interval scales.

However, some of its conclusions, such as ``\emph{Precision, Recall, and F-measure are interval scales, independently from the adopted order}'', only apply to some cases since there are several orderings, where these retrieval measures are not interval scales. The purpose of this comment is to point out some aspects that influence on the scale type of a retrieval measure when the \ac{RTM} is assumed. It is shown that the property of being an interval scale depends on the set where the retrieval measure is defined, more specifically, on the ordering relationship defined on this set\footnote{Note that this comment does not challenge other statements of the same authors about the scale type of retrieval measures, such as 'interval scales are equispaced mappings'~\cite{FerranteEtAl2021c}, which has been recently supported by providing a well documented discussion with the foundational works in the area of measurement~\cite{ferrante2022response}. This comment only points out that the arguments provided in~\citeauthor{FerranteEtAl2018b} are not generalisable.}.

On the other hand, I want to thank the great research line of work developed by the authors~\cite{FerranteEtAl2015b,FerranteEtAl2017b,FerranteEtAl2017,FerranteEtAl2018b,FerranteEtAl2019c,FerranteEtAl2021c}, which provides a better understanding of the theoretical foundations of IR evaluation. 

\section{Background on the RTM}
\label{sec:rtm}
The \ac{RTM} was initiated by~\citeauthor{von1887zahlen} and~\citeauthor{Holder1901} at the end of the 19th century. The \emph{opus magnum} of the \ac{RTM} can be found in the three volume work of Foundation of Measurement~\citep{KrantzEtAl1971,LuceEtAl1990,SuppesEtAl1989}; here, some of the main concepts are briefly summarized. 

The \ac{RTM} states that the numbers that are obtained through measurement represent empirical relationships. Two systems are considered: (i) the \acf{ERS}, which is a set of entities of the real world modelled with some relationships\slash operations; and (ii) the \acf{NRS}, which is commonly represented by the real numbers. Measurement consists on establishing a mapping between both systems, in such a way that the relationships between entities are matched by relationships between numbers. As there are different types of relationships, which can be represented on the \ac{NRS}, then there is a taxonomy on the representations, which leads to the notion of \emph{scale types}. 

Four main categories of scale types can be distinguished~\cite{Stevens1946} depending on the types of relationships\slash operations that are preserved: (i) a \emph{nominal scale} preserves every entity of the \ac{ERS} (one-to-one mappings); (ii) an \emph{ordinal scale} preserves an ordering relationship among entities (monotone increasing mappings); (iii) an \emph{interval scale} preserves a relationship among differences or intervals (linear positive mappings); and (iv) a \emph{ratio scale} preserves a relationship among ratios (similarity transformations). 

We formalise the concept of ordinal scale, as an example that will be referred later. Consider an \ac{ERS} endowed with an ordering relationship, $(X, \preceq)$, a real-mapping, $f$, is an ordinal scale if it preserves the ordering relationship on the \ac{ERS}, i.e., if it verifies: $x \preceq y$ iff $f(x) \leq f(y)$, $\forall x$, $y \in X$. Thus, the property of being an ordinal scale depends on the ordering relationship, $\preceq$, where the mapping is defined, and the same can be said for interval scales.

One of the main contributions of the \ac{RTM} are the \emph{representational theorems}, which determine the properties of the relationships\slash operations defined on the \ac{ERS}, such that they guarantee the existence and uniqueness of a mapping that preserves these relationships. For instance, in the particular case of interval scales, all the necessary and sufficient properties to obtain an interval scale are determined by a \emph{difference structure}, which is an \ac{ERS} endowed with an ordering relationship that verifies some specific properties~\citep[p.~59]{Rossi2014}.

\section{Brief Structure of the Paper}
\label{sec:structure}
The paper ``\emph{A General Theory of IR Evaluation Measures}''~\cite{FerranteEtAl2018b} considers measurement as a process of mapping lists of assessed documents (\ac{ERS}) to real numbers (\ac{NRS}), in such a way that the relationships defined on the ERS are represented or preserved as numerical properties. To determine whether retrieval measures are interval scales, a generic procedure is considered, which can be summarized with the following steps: 
\begin{enumerate}[label=(\arabic*)]
\item An ordering relationship among ranked lists of documents is defined. Thus, the \ac{ERS} is the set of lists of documents endowed with this relationship.
\item It is demonstrated that this \ac{ERS} is a difference structure.
\item One interval scale defined on this \ac{ERS} is identified\footnote{Note that the representation theorem of the \ac{RTM} assures that at least there exists one interval scale since the \ac{ERS} is a difference structure}.
\item Making use of the uniqueness of the representation theorem, it is checked whether a retrieval measure is an interval scale or not.
\end{enumerate}
The paper is structured and developed with the following four main sections: 
\begin{enumerate}[label=(\alph*)]
\item The set-based case is considered, \textbf{one} total order is defined by the authors, and the generic procedure is performed to deduce whether retrieval measures are interval scales on this \ac{ERS}.
\item The set-based case is considered again, but now, \textbf{one} partial order is defined, and the generic procedure is performed.
\item The rank-based case is considered, \textbf{one} total order is defined, and the generic procedure is performed to deduce whether retrieval measures are interval scales.
\item Finally, the rank-based case is considered again, \textbf{one} partial order is defined, and the generic procedure is performed.
\end{enumerate}
For each one of these four specific ordering relationships defined on the \ac{ERS}, conclusions are then drawn.

\section{Discussion}

\subsection{Main Issue}
Considering a retrieval measure, its property of being ordinal scale depends on the ordering defined on the \ac{ERS} since the retrieval measure has to preserve this ordering (see Section \ref{sec:rtm}). Thus, statements about the ordinal property of a retrieval measure, which do not specify the ordering on the \ac{ERS}, such as ``\emph{this retrieval measure is an ordinal scale.}'', have to be understood as considering all possible orderings on the \ac{ERS}, i.e., they should be interpreted as  ``\emph{this retrieval measure is an ordinal scale on all possible orderings on the \ac{ERS}}''. 

Otherwise, when a specific ordering on the \ac{ERS} is established or deduced from the context, then statements about the ordinal scale property of a retrieval measure should specify the considered ordering. In this case, statements of the form ``\emph{this retrieval measure is an ordinal scale.}'' are partially specified since they have to indicate the ordering on which the retrieval measure is defined. This statement should be expressed as ``\emph{this retrieval measure is an ordinal scale on} that \emph{empirical domain}'', where ``that'' is a specific \ac{ERS}.

These considerations about the ordinal property of a retrieval measure are also valid for the interval property since every interval scale is an ordinal scale. As noted in Section \ref{sec:structure},~\citeauthor{FerranteEtAl2018b} consider four scenarios: one total/partial order at the set and rank based cases, i.e., four specific orderings on the \ac{ERS} are considered. Thus, in this setting, it should not be made general statements of the form ``\emph{this retrieval measure is an interval scale.}'', neither it should be stated that the interval scale property of a retrieval measure is independent of the ordering on the \ac{ERS}.

However, some conclusions of the considered paper seem to express generic statements of the form: ``\emph{this retrieval measure is an interval scale.}''. In fact, these statements have been made clear in subsequent papers, such as in Ferrante et al.~\cite{FerranteEtAl2021c}: ``\emph{Recently, Ferrante et al. [40], [41] have theoretically shown that some of the most known and used IR measures, like Average Precision (AP) or Discounted Cumulative Gain (DCG), are not interval-scales.}''. 

The main issue about the setting of the considered paper consists in its lack of generalisation. When an ordering has been established on the empirical domain of a retrieval measure, then statements about its scale property should specify the \ac{ERS}, i.e., which are the assumptions on this set or its relationships and\slash or operators.

\subsection{Concerns at the Conclusions Section}
One of the conclusions about the set-based case on a total order claims that ``\emph{In the case of set-based \ac{IR} measures, binary relevance: Precision, Recall, and F-measure are interval scales, independently from the adopted order}''. However, as indicated in the previous subsection, the scale property of a retrieval measure is conditioned by the ordering on the \ac{ERS}. In fact, the generic procedure described in Section \ref{sec:structure} is not independent of the adopted order since it establishes a specific order at the step (1). 

Specifically, in the context of the quoted conclusion of the previous paragraph, i.e., in the case of set-based \ac{IR} measures with binary relevance, there are many orderings (some of them total orders) among lists of assessed documents where \ac{P}, \ac{R}, and F-measure are not interval scales. For instance, when the length of the retrieved lists is two ($N=2$), then it can be considered the following total order on the \ac{ERS}\footnote{Denoting by ``$1$'' a relevant document an by ``$0$'' a non-relevant one.}:
\[
\{0, 1\} \preceq \{0, 0\} \preceq \{1, 1\}. 
\]
However, \ac{P}, \ac{R}, and F-measure are not ordinal scales on this \ac{ERS} since \ac{P}$(\{0, 0\}) < $ \ac{P}$(\{0, 1\})$ (see definition of ordinal scale in Section \ref{sec:rtm}).

The main concern with the proposed procedure is its lack of generalisation since it only considers one ordering defined at the step (1). Once this ordering relationship has been established, then the interval scales deduced with this procedure, at the steps (3) and (4), are interval scales derived from this specific ordering. A retrieval measure may have different scale properties, depending on the empirical domain on which is defined. For instance, as it can be checked in the considered paper, the \ac{RBP} with $p=0.5$ is an interval scale when it is defined on the total ordering of the subsection 6.1, but not when it is defined on the partial ordering of the subsection 6.2. Therefore, there is a strong dependence between the scale type and the \ac{ERS} where the numerical mapping is defined.

Another statement made at the conclusion section is as follows: ``\emph{in the case of rank-based \ac{IR} measures, binary relevance: when using a total order, \ac{RBP} is an interval scale only if $p=\frac{1}{2}$ while all the other measures - namely AP, DCG, ERR, and RBP for other values of p - are not. When using a partial order, none of these measures is an interval scale.}''. The same comments can be made here, it is only considered one total order and one partial order, which are the orderings defined at the step (1), among the huge quantity of possible orderings. However, the conclusion claim is generic, and considers all possible total and partial orders (``\emph{when using \textbf{a} total order}'' or ``\emph{when using \textbf{a} partial order}''). Thus, the considered paper have only demonstrated that the previous retrieval measures are interval scales or not when they are defined on \textbf{one} specific ordering, but not that they are interval scales in general.

The distinction between the different ways of making statements about the scale properties of retrieval measures can be seen with the following two settings. On the one hand, in the foundational works of the \ac{RTM}~\citep{KrantzEtAl1971,LuceEtAl1990,SuppesEtAl1989}, the \ac{ERS} is initially endowed with a generic ordering relationship, i.e., no assumptions are adopted on this ordering. Thus, it can be considered that the \ac{ERS} is endowed with any possible ordering relationship. In this setting it is possible to make generic statements of the form ``\emph{this retrieval measure is an interval scale.}''. On the other hand, in~\citeauthor{FerranteEtAl2018b}'s paper, the \ac{ERS} is endowed with the ordering defined at step (1). Thus, the initial \ac{ERS} from which interval scales are deduced is an empirical domain endowed with a particular ordering. Therefore, the deduced interval scales, are interval scales defined on this specific \ac{ERS}. 

\subsection{Effect of these concerns}
Consider an \ac{IR} setting, where the \ac{ERS} is the set of possible system outputs rankings endowed with one of the ordering relationships of~\citeauthor{FerranteEtAl2018b}'s paper, $\preceq$. Pairs of system outputs rankings can be compared according to the users' information needs. If these comparisons between pairs of rankings do not hold, or hold only in some cases regarding $\preceq$, then it should be expected that the inferences drawn from the numerical values not to hold, or to hold only in some cases. As a consequence, incorrectly classifying the scale type of an IR evaluation measure leads to wrong conclusions about the obtained results.

Establishing an ordering on the \ac{ERS}, according to an intuitive and commonly agreeable relationship, as it has been done in the considered paper, limits the generalisation of the deduced interval scales.

\subsection{Other Concerns}
As a minor issue, the paper assumes that all rankings have a fixed number of documents, $N$, and conclusions about the scale properties of retrieval measures are based on this fact. However, different \ac{IR} systems can retrieve a different number of documents; thus, the analysis has a limited application. 

\section{Conclusions}
\label{sec:conclusion}
Guided by the goal ``\emph{to theoretically ground our evaluation methodology [...] in order to aim for more robust and generalizable inferences}''~\cite{ferrante2022response}, this comment has shown the difficulty of making generic statements of the form ``this retrieval measure is an interval scale.'' since, when the \ac{RTM} is assumed, the scale properties of retrieval measures depend on the \ac{ERS}. 

We think that this issue could be addressed by considering all the possible empirical domains where a retrieval measure is defined, or by letting the empirical domain to be specified by the retrieval measure itself, which is an interesting topic for future works.

\input{acronimi-no-lista}

\begin{thebibliography}{27}
\providecommand{\natexlab}[1]{#1}
\providecommand{\url}[1]{\texttt{#1}}
\expandafter\ifx\csname urlstyle\endcsname\relax
  \providecommand{\doi}[1]{doi: #1}\else
  \providecommand{\doi}{doi: \begingroup \urlstyle{rm}\Url}\fi

\bibitem[Ferrante et~al.(2015)Ferrante, Ferro, and Maistro]{FerranteEtAl2015b}
M.~Ferrante, N.~Ferro, and M.~Maistro.
\newblock {Towards a Formal Framework for Utility-oriented Measurements of
  Retrieval Effectiveness}.
\newblock In J.~Allan, W.~B. Croft, A.~P. de~Vries, C.~Zhai, N.~Fuhr, and
  Y.~Zhang, editors, \emph{{Proc. 1st ACM SIGIR International Conference on the
  Theory of Information Retrieval (ICTIR 2015)}}, pages 21--30. {ACM Press, New
  York, USA}, 2015.
  
\bibitem[Ferrante et~al.(2017)Ferrante, Ferro, and Pontarollo]{FerranteEtAl2017b}
M.~Ferrante, N.~Ferro, and S.~Pontarollo.
\newblock {An Interval-Like Scale Property for IR Evaluation Measures}.
\newblock In \emph{{Proc. 8th International Workshop on Evaluating Information Access (EVIA 2017)}}, pages 10--15. {EVIA@ NTCIR, Tokio, Japan}, 2017.

\bibitem[Ferrante et~al.(2017)Ferrante, Ferro, and
  Pontarollo]{FerranteEtAl2017}
M.~Ferrante, N.~Ferro, and S.~Pontarollo.
\newblock {Are IR Evaluation Measures on an Interval Scale?}
\newblock In J.~Kamps, E.~Kanoulas, M.~de~Rijke, H.~Fang, and E.~Yilmaz,
  editors, \emph{{Proc. 3rd ACM SIGIR International Conference on the Theory of
  Information Retrieval (ICTIR 2017)}}, pages 67--74. {ACM Press, New York,
  USA}, 2017.

\bibitem[Ferrante et~al.(2019)Ferrante, Ferro, and
  Pontarollo]{FerranteEtAl2018b}
M.~Ferrante, N.~Ferro, and S.~Pontarollo.
\newblock {A General Theory of IR Evaluation Measures}.
\newblock \emph{{IEEE Transactions on Knowledge and Data Engineering (TKDE)}},
  31\penalty0 (3):\penalty0 409--422, March 2019.

\bibitem[Ferrante et~al.(2020)Ferrante, Ferro, and Losiouk]{FerranteEtAl2019c}
M.~Ferrante, N.~Ferro, and E.~Losiouk.
\newblock {How do interval scales help us with better understanding IR
  evaluation measures?}
\newblock \emph{{Information Retrieval Journal}}, 23\penalty0 (3):\penalty0
  289--317, June 2020.

\bibitem[Ferrante et~al.(2021)Ferrante, Ferro, and Fuhr]{FerranteEtAl2021c}
M.~Ferrante, N.~Ferro, and N.~Fuhr.
\newblock {Towards Meaningful Statements in IR Evaluation. Mapping Evaluation
  Measures to Interval Scales}.
\newblock \emph{{IEEE Access}}, 9:\penalty0 136182--136216, 2021.

\bibitem[Ferrante et~al.(2022)Ferrante, Ferro, and Fuhr]{ferrante2022response}
M.~Ferrante, N.~Ferro, and N.~Fuhr.
\newblock {Response to Moffat's Comment on ``Towards Meaningful Statements in IR Evaluation: Mapping Evaluation Measures to Interval Scales''}.
\newblock \emph{{arXiv}}, 10.48550/ARXIV.2212.11735, 2022.

\bibitem[H{\"o}lder(1901)]{Holder1901}
O.~H{\"o}lder.
\newblock {Die Axiome der Quantit{\"a}t und die Lehre vom Mass}.
\newblock \emph{{Berichte {\"u}ber die Verhandlungen der K{\"o}niglich
  S{\"a}chsischen Gesellschaft der Wissenschaften zu Leipzig,
  Mathematisch-Physikaliche Classe}}, 53:\penalty0 1--64, 1901.

\bibitem[Krantz et~al.(1971)Krantz, Luce, Suppes, and Tversky]{KrantzEtAl1971}
D.~H. Krantz, R.~D. Luce, P.~Suppes, and A.~Tversky.
\newblock \emph{{Foundations of Measurement. Additive and Polynomial
  Representations}}, volume~1.
\newblock {Academic Press, New York, USA}, 1971.

\bibitem[Luce et~al.(1990)Luce, Krantz, Suppes, and Tversky]{LuceEtAl1990}
R.~D. Luce, D.~H. Krantz, P.~Suppes, and A.~Tversky.
\newblock \emph{{Foundations of Measurement. Representation, Axiomatization,
  and Invariance}}, volume~3.
\newblock {Academic Press, New York, USA}, 1990.

\bibitem[Rossi(2014)]{Rossi2014}
G.~B. Rossi.
\newblock \emph{{Measurement and Probability. A Probabilistic Theory of
  Measurement with Applications}}.
\newblock {Springer-Verlag, New York, USA}, 2014.

\bibitem[Stevens(1946)]{Stevens1946}
S.~S. Stevens.
\newblock {On the Theory of Scales of Measurement}.
\newblock \emph{{Science, New Series}}, 103\penalty0 (2684):\penalty0 677--680,
  June 1946.

\bibitem[Suppes et~al.(1989)Suppes, Krantz, Luce, and Tversky]{SuppesEtAl1989}
P.~Suppes, D.~H. Krantz, R.~D. Luce, and A.~Tversky.
\newblock \emph{{Foundations of Measurement. Geometrical, Threshold, and
  Probabilistic Representations}}, volume~2.
\newblock {Academic Press, New York, USA}, 1989.

\bibitem[von Helmholtz et~al.(1887)]{von1887zahlen}
H.~von Helmholtz.
\newblock \emph{Z{\"a}hlen und Messen Erkenntnis--theoretisch betrachtet, Philosophische Aufs{\"a}tze Eduard Zeller gewidmet}
\newblock {Fuess}, Leipzig, 1887.

\end{thebibliography}


\end{document}
