\documentclass[lettersize, journal]{IEEEtran}
\usepackage{amsthm}
\usepackage{amsmath}
\usepackage{amssymb}
\usepackage{graphicx}
\usepackage{indentfirst}
\usepackage{algorithm,algorithmic}
%\usepackage[ruled, vlined]{algorithm2e}
\usepackage[top=1.6cm, bottom=2cm, left=2cm, right=2cm]{geometry}
\usepackage{threeparttable}
\usepackage{multirow}
\usepackage[usenames]{color}
% \pagestyle{empty}
\usepackage{subfigure}
\usepackage{hyperref}

% \usepackage[numbers]{natbib}
% \newcommand{\cites}[1]{\citeauthor{#1} \cite{#1}}
%\newenvironment{sloppypar}{\par\sloppy}{\par}
\usepackage{mathtools} 
\DeclarePairedDelimiter{\ceil}{\lceil}{\rceil} 
\DeclarePairedDelimiter\floor{\lfloor}{\rfloor}
\renewcommand{\algorithmicrequire}{\textbf{Input:}} 
\renewcommand{\algorithmicensure}{\textbf{Output:}}
\usepackage[dvipsnames]{xcolor}
\newcommand{\dc}[1]{\textcolor{red}{#1}}
\newcommand{\cm}[1]{\textcolor{gray}{#1}}
\newcommand{\alg}{\text{ALG}}
\newcommand{\opt}{\text{OPT}}
\newcommand{\hy}{\widehat{y}}
\makeatletter
\newcommand*{\rom}[1]{\expandafter\@slowromancap\romannumeral #1@}
\makeatother
\usepackage{xurl}
\usepackage{booktabs}
\usepackage{tikz}
\newcommand{\mycaption}[1]{\stepcounter{figure}\raisebox{-7pt}
  {\footnotesize Fig. \thefigure.\hspace{3pt} #1}}

\providecommand{\keywords}[1]
{
%	\small	
	\textbf{\textit{Keywords---}} #1
}

\theoremstyle{definition}
\newtheorem{lemma}{Lemma}
\newtheorem{claim}{Claim}
\newtheorem{theorem}{Theorem}
\newtheorem{remark}{Remark}
\newtheorem{definition}{Definition}
\newtheorem{objective}[theorem]{Objective}
\newtheorem{problem}{Problem}

\usepackage{multirow}
\usepackage{array}
\newcommand{\PreserveBackslash}[1]{\let\temp=\\#1\let\\=\temp}
\newcolumntype{C}[1]{>{\PreserveBackslash\centering}p{#1}}
\newcolumntype{R}[1]{>{\PreserveBackslash\raggedleft}p{#1}}
\newcolumntype{L}[1]{>{\PreserveBackslash\raggedright}p{#1}}
\usepackage{amssymb}% http://ctan.org/pkg/amssymb
\usepackage{pifont}% http://ctan.org/pkg/pifont
\newcommand{\cmark}{\ding{51}}%
\newcommand{\xmark}{\ding{55}}%
% \usepackage[parfill]{parskip}
\usepackage[font=footnotesize]{caption} 
% \newcommand{\dc}[1]{\textcolor{red}{#1}}
\usepackage{lipsum}
\usepackage{threeparttable}
 \newcommand{\revise}[1]{\textcolor{black}{#1}}

\allowdisplaybreaks
\title{A Survey of Machine Learning-Based Ride-Hailing Planning}
%\author{}

\author{
Dacheng Wen,~\IEEEmembership{Student Member,~IEEE},
Yupeng Li,~\IEEEmembership{Member,~IEEE},
Francis C.M. Lau
\IEEEcompsocitemizethanks{
		\IEEEcompsocthanksitem
            Dacheng Wen is with The University of Hong Kong, Hong Kong (e-mail: wdacheng@connect.hku.hk). Work done while Dacheng Wen was under the supervision of Yupeng Li and Francis C.M. Lau.\protect
            \IEEEcompsocthanksitem Yupeng Li (corresponding author) is with Hong Kong Baptist University, Hong Kong (e-mail: ivanypli@gmail.com).\protect
            \IEEEcompsocthanksitem Francis C.M. Lau is with The University of Hong Kong, Hong Kong (email: fcmlau@cs.hku.hk).
	}
}

\begin{document}
%	\markboth{Journal of \LaTeX\ Class Files}%,~Vol.~XX, No.~X, XX~2022
%	{Shell \MakeLowercase{\textit{et al.}}: A Sample Article Using IEEEtran.cls for IEEE Journals}
	
	\maketitle
	
% 	\documentclass{scrartcl}
\usepackage{tikz,pgfplots}
\usepackage{filecontents}
\begin{document}
\pgfkeys{/pgf/number format/.cd,1000 sep={\,}}

\begin{filecontents}{data.csv}
year,count
2016,998
2015,1000
2014,900
2013,837
2012,826
2011,784
2010,801
2009,731
2008,703
2007,632
2006,629
2005,516
2004,512
2003,476
2002,444
2001,497
2000,478
1999,400
1998,393
1997,399
1996,387
\end{filecontents}

\begin{tikzpicture}
\begin{axis}[ xlabel=Count, ylabel=Year]


\addplot[color=blue,mark=*] table[x=year, y=count, col sep=comma]{data.csv};

 \end{axis} 
 \end{tikzpicture}
\end{document}
	

Over the past few years, there has been a significant amount of research focused on studying the ReLU activation function, with the aim of achieving neural network convergence through over-parametrization. However, recent developments in the field of Large Language Models (LLMs) have sparked interest in the use of exponential activation functions, specifically in the attention mechanism.

Mathematically, we define the neural function $F: \R^{d \times m} \times  \mathbb{R}^d \rightarrow \mathbb{R}$ using an exponential activation function. Given a set of data points with labels $\{(x_1, y_1), (x_2, y_2), \dots, (x_n, y_n)\} \subset \mathbb{R}^d \times \mathbb{R}$ where $n$ denotes the number of the data. Here $F(W(t),x)$ can be expressed as $F(W(t),x) := \sum_{r=1}^m a_r \exp(\langle w_r, x \rangle)$, where $m$ represents the number of neurons, and $w_r(t)$ are weights at time $t$. It's standard in literature that $a_r$ are the fixed weights and it's never changed during the training. We initialize the weights $W(0) \in \mathbb{R}^{d \times m}$ with random Gaussian distributions, such that $w_r(0) \sim \mathcal{N}(0, I_d)$ and initialize $a_r$ from random sign distribution for each $r \in [m]$.

Using the gradient descent algorithm, we can find a weight $W(T)$ such that $\| F(W(T), X) - y \|_2 \leq \epsilon$ holds with probability $1-\delta$, where $\epsilon \in (0,0.1)$ and $m = \Omega(n^{2+o(1)}\log(n/\delta))$. To optimize the over-parametrization bound $m$, we employ several tight analysis techniques from previous studies [Song and Yang arXiv 2019, Munteanu, Omlor, Song and Woodruff ICML 2022]. 

 

	
	\begin{IEEEkeywords}
		Ride-hailing, machine learning, matching, repositioning, collective planning, distributed planning.
		%Article submission, IEEE, IEEEtran, journal, \LaTeX, paper, template, typesetting.
	\end{IEEEkeywords}
	
	\section{Introduction}
\label{sec:introduction}
% \begin{itemize}
%     % Diffusion of FL
%     \item {\st{Diffusion of FL}}
%     % Security threats to FL
%     \item {\st{Security threats to FL with particular focus on model poisoning}}
%     % Limitations of existing countermeasures
%     \item {\st{Current countermeasures (e.g., KRUM) and their limitations}}
%     % Proposed method and its advantages
%     \item {\st{Intuitive description of the proposed method and its difference (i.e., advantages) w.r.t. state of the art}}
%     % Main contributions
%     \item {\st{Summary of the main contributions of this work}}
%     % Paper's structure and organization
%     \item {\st{Paper's structure and organization}}
% \end{itemize}

% Diffusion of FL
Recently, {\em federated learning} (FL) has emerged as the leading paradigm for training distributed, large-scale, and privacy-preserving machine learning (ML) systems~\cite{mcmahan2017googleai,mcmahan2017aistats}. 
The core idea of FL is to allow multiple edge clients to collaboratively train a shared, global model without disclosing their local private training data.
%Specifically, an FL system consists of a central server and many edge clients; 
A typical FL round involves the following steps: {\em(i)} the server randomly picks some clients and sends them the current, global model; {\em(ii)} each selected client locally trains its model with its own private data; then, it sends the resulting local model to the server;\footnote{Whenever we refer to global/local model, we mean global/local model {\em parameters}.} {\em(iii)} the server updates the global model by computing an \emph{aggregation function}, usually the average (FedAvg), on the local models received from clients.
% \begin{enumerate}
%     \item[{\em(i)}] the server sends the current, global model to the clients and appoints some of them for training;
%     \item[{\em(ii)}] each selected client locally trains its copy of the global model with its own private data; then, it sends the resulting local model back to the server;\footnote{Whenever we refer to global/local model, we mean global/local model {\em parameters}.}
%     \item[{\em(iii)}] the server updates the global model by computing an \emph{aggregation function} on the local models received from clients (by default, the average, also referred to as FedAvg~\cite{mcmahan2017aistats}).
% \end{enumerate}
This process goes on until the global model converges. %(e.g., after a certain number of rounds or other similar stopping criteria).
%\\
% The advantages of FL over the traditional, centralized learning paradigm are undoubtedly clear in terms of flexibility/scalability (clients can join/disconnect from the FL network dynamically), network communications (only model weights\footnote{We will use \textit{parameters} and \textit{weights} interchangeably.} are exchanged between clients and server), and privacy (each client's private training data is kept local at the client's end and not uploaded to the server).
\\
% Security threats to FL
%However, the growing adoption of FL also raises security concerns~\cite{costa2022covert}, particularly about its confidentiality, integrity, and availability.
Although its advantages over standard ML, FL also raises security concerns~\cite{costa2022covert}. %, particularly about its confidentiality, integrity, and availability~\cite{costa2022covert}.
% OLD, LONG VERSION
% Indeed, some work deals with privacy leakage that may expose the local data of some clients~\cite{melis2019sp}. 
% A large body of work, instead, investigates attacks that usually aim to detriment the predictive accuracy of the learned global model. For instance, \emph{data poisoning} attacks achieve this goal by letting an adversary pollute the training set of some corrupt FL clients with maliciously crafted examples~\cite{jagielski2018sp}.
% Similarly, in \emph{model poisoning} the attacker attempts to tweak the global model weights~\cite{bhagoji2019pmlr} by directly perturbing the local model's weights of some infected FL clients before these are sent to the central server for aggregation, usually via so-called Byzantine attacks. 
% It turns out that Byzantine model poisoning attacks severely impact standard FedAvg; therefore, more robust aggregation functions must be designed to make FL systems secure.
Here, we focus on \emph{untargeted model poisoning} attacks~\cite{bhagoji2019pmlr}, where an adversary attempts to tweak the global model weights %\footnote{We will use the terms \textit{parameters} and \textit{weights} interchangeably.} 
by directly perturbing the local model's parameters of some infected clients before these are sent to the central server for aggregation.
In doing so, the adversary aims to jeopardize the global model \textit{indiscriminately} at inference time.
Such model poisoning attacks severely impact standard FedAvg; therefore, more robust aggregation functions must be designed to secure FL systems.
\\
% In this paper, we focus on designing a novel robust aggregation scheme at the server's end to contrast the effect of Byzantine model poisoning attacks.
%
% Current countermeasures and their limitations
%Several countermeasures have been proposed in the literature to combat model poisoning attacks on FL systems.
% Some methods use simple statistics more robust than plain average to smooth the impact of malicious updates (e.g., Trimmed Mean and FedMedian~\cite{yin2018icml}). 
% Other defenses implement outlier detection techniques to discard malicious updates from the aggregation performed at the server's end. Those are either based on heuristics (e.g., Krum/Multi-Krum~\cite{blanchard2017nips} and Bulyan~\cite{mhamdi2018pmlr}) or data-driven approaches (e.g., K-means clustering~\cite{shen2016acm} or DnC via spectral analysis~\cite{shejwalkar2021ndss}). 
% Finally, some strategies rely on a centralized ``source of trust'' to spot potential malicious updates (e.g., FLTrust~\cite{cao2020fltrust}).
% Several countermeasures have been proposed in the literature to combat model poisoning attacks on FL systems, i.e., to discard possible malicious local updates from the aggregation performed at the server's end. 
% These techniques range from simple statistics more robust than plain average (e.g., Trimmed Mean and FedMedian~\cite{yin2018icml}) to outlier detection heuristics (e.g., Krum/Multi-Krum~\cite{blanchard2017nips} and Bulyan~\cite{mhamdi2018pmlr}) or data-driven approaches (e.g., spectral analysis via K-means clustering~\cite{shen2016acm} or spectral analysis), or methods based on ``source of trust'' (e.g., FLTrust~\cite{cao2020fltrust}).
% OLD, LONG VERSION
%Several countermeasures have been proposed in the literature to combat Byzantine model poisoning attacks on FL systems.
% Descriptive statistics
% For example, Trimmed Mean and FedMedian aggregate local model updates using more robust statistics than standard average~\cite{yin2018icml}.
%
% % Heuristics for outlier detection
% Many existing Byzantine-resilient strategies implement some outlier detection heuristics to discard the model updates sent by potentially malicious clients from the input of the aggregation function.
% One of the most popular heuristics is Krum~\cite{blanchard2017nips}.
% This strategy tries to mitigate the impact of Byzantine attacks by selecting as a global model the local model with the smallest sum of Euclidean distances to {\em all} the other local models.
% Although powerful, Krum requires the server to know (or, at least, estimate) the number of malicious FL clients upfront, which is generally impossible in a realistic attack scenario. %
% Moreover, Krum may become ineffective for complex, high-dimensional model parameter spaces due to the curse of dimensionality.
% Bulyan~\cite{mhamdi2018pmlr} tries to overcome this issue by combining Krum with a variant of Trimmed Mean.
% % Data-driven outlier detection
% Other strategies use data-driven outlier detection techniques -- e.g., via K-means clustering~\cite{shen2016acm} -- to spot potential malicious local model updates. 
% %For instance, Shen et al. propose to cluster local model updates with K-means and thus identify outliers.
%
% % Other techniques
% As far as the server is concerned, any local model received can be from a potential malicious client. 
% FLTrust~\cite{cao2020fltrust} assumes the server acts as a client, i.e., trains a local model on an additional {\em trustworthy} dataset at the server's end and compares it against all the local models from other clients. 
% This way, the server can rely on some ``source of trust'' when discarding potentially malicious clients.
%\\
% Limitations of existing Byzantine-resilient strategies
Unfortunately, existing defense mechanisms either rely on simple heuristics (e.g., Trimmed Mean and FedMedian by~\cite{yin2018icml}) or need strong and unrealistic assumptions to work effectively (e.g., foreknowledge or estimation of the number of malicious clients in the FL system, as for Krum/Multi-Krum~\cite{blanchard2017nips} and Bulyan~\cite{mhamdi2018pmlr}, which, however, cannot exceed a fixed threshold).
Furthermore, outlier detection methods using K-means clustering~\cite{shen2016acm} or spectral analysis like DnC~\cite{shejwalkar2021ndss} do not directly consider the temporal evolution of local model updates received.
Finally, strategies like FLTrust~\cite{cao2020fltrust} require the server to collect its own dataset and act as a proper client, thereby altering the standard FL protocol.
\\
% OLD, LONG VERSION
% Overall, existing Byzantine-resilient strategies are either simple heuristics (e.g., FedMedian) or, if they are more complex, they rely on strong and unrealistic assumptions to work effectively (e.g., knowing the number of malicious clients in the FL system in advance, as for Krum and alike).
% Furthermore, data-driven outlier detection methods do not consider the temporary evolution of local model updates received (e.g., K-means clustering). 
% Finally, strategies like FLTrust requires the server to collect its own dataset and act as a proper client, thereby altering the standard FL protocol.
%
% Description of the proposed method
This work introduces a novel pre-aggregation \textit{filter} robust to untargeted model poisoning attacks. Notably, this filter $(i)$ operates without requiring prior knowledge or constraints on the number of malicious clients and $(ii)$ inherently integrates temporal dependencies. 
The FL server can employ this filter as a preprocessing step before applying \textit{any} aggregation function, be it standard like FedAvg or robust like Krum or Bulyan.
Specifically, we formulate the problem of identifying corrupted updates as a multidimensional (i.e., matrix-valued) time series anomaly detection task. 
The key idea is that legitimate local updates, resulting from well-calibrated iterative procedures like stochastic gradient descent (SGD) with an appropriate learning rate, show \textit{higher predictability} compared to malicious updates. This hypothesis stems from the fact that the sequence of gradients (thus, model parameters) observed during legitimate training exhibit regular patterns, as validated in Section~\ref{subsec:intuition}. %until convergence. 
%This regularity may be more pronounced for smooth convex loss functions, but it can still be captured within an appropriate time window, even for more complex and convoluted loss surfaces. 
%We provide evidence of this claim in Appendix~B, where we show that the average mutual information (i.e., ``predictability''), calculated over pairs of legitimate model updates sent at different FL rounds, is significantly higher than the corresponding computation for a malicious client.
\\
Inspired by the matrix autoregressive (MAR) framework for multidimensional time series forecasting~\cite{chen2021je}, we propose the FLANDERS ({\em \textbf{F}ederated \textbf{L}earning meets \textbf{AN}omaly \textbf{DE}tection for a \textbf{R}obust and \textbf{S}ecure}) filter.
The main advantages of FLANDERS over existing strategies like FLDetector~\cite{zhao2020multivariate} are its resilience to large-scale attacks, where $50\%$ or more FL participants are hostile, and the capability of working under realistic non-iid scenarios.
We attribute such a capability to two key factors: $(i)$ FLANDERS works without knowing a priori the ratio of corrupted clients, and $(ii)$ it embodies temporal dependencies between intra- and inter-client updates, quickly recognizing local model drifts caused by evil players. Below, we summarize our main contributions:

\begin{itemize}
\item[{\em(i)}]
We provide empirical evidence that the sequence of models sent by legitimate clients is more predictable than those of malicious participants performing untargeted model poisoning attacks.
\\
\item[{\em(ii)}] 
We introduce FLANDERS, the first pre-aggregation filter for FL robust to untargeted model poisoning based on multidimensional time series anomaly detection.
\\
\item[{\em(iii)}] 
We integrate FLANDERS into Flower,\footnote{\scriptsize{\url{https://flower.dev/}}} a popular FL simulation framework for reproducibility.
\\
\item[{\em(iv)}] 
We show that FLANDERS improves the robustness of the existing aggregation methods under multiple settings: different datasets, client's data distribution (non-iid), models, and attack scenarios.
\\
\item[{\em(v)}] 
We publicly release all the implementation code of FLANDERS along with our experiments.\footnote{\scriptsize{\url{https://anonymous.4open.science/r/flanders_exp-7EEB}}}
\end{itemize}

% Paper's structure and organization
The remainder of the paper is structured as follows. %some related work and the current state-of-the-art solutions to security issues that FL entails. 
Section~\ref{sec:background} covers background and preliminaries. 
In Section~\ref{sec:related}, we discuss related work.
Section~\ref{sec:problem} and Section~\ref{sec:method} describe the problem formulation and the method proposed. % to tackle it. 
Section~\ref{sec:experiments} gathers experimental results. %, and Section~\ref{sec:limitations} discusses some limitations of this work.
Finally, we conclude in Section~\ref{sec:conclusion}.
 %discusses the limitations of this work and draws future research directions.
%reports conclusions and draws perspectives for future research directions.

%%%%%%% OLD %%%%%%%
%to overcome the resilience of Byzantine failures in distributed Stochastic Gradient Descent computations. 
% The strength of Krum is its time complexity, which is linear in the gradient dimension. 
% However, the robustness of the approach is guaranteed for gradient-based learning applications only when the majority of the clients are not compromised. 
% Besides, the aggregation mechanism of Krum, as well as that of similar methods, is robust from a coarse-grained perspective and does not provide solutions to errors and perturbations that may occur at inference time.
%A related approach to~\cite{blanchard2017nips} is the work of Su et al.~\cite{su2016dc}. Here, the authors propose an iterated approximate agreement to tackle a multi-layer scenario attacked by Byzantine agents. 
%However, the method works efficiently on the sole discrete context and it is inapplicable to continuous state environments.
%\gabri{Maybe, we should just talk about the main limitations of existing countermeasures without digging into their details (or, we can just mention Krum as this is the most popular one). I will move the description of all these methods to the Related Work section.}
	\section{Background on Network Calculus}
\label{sec: background}


\begin{figure*}[tbh]
\centering
\begin{subfigure}[b]{0.3\textwidth}
    \centering
    \includegraphics[width=\linewidth]{images/in-out.png}
    \caption{Arrival and departure data and their relation with delay $d(t)$ and backlog $b(t)$. For a FIFO system, the delay is the horizontal distance between $R(t)$ and $R^*(t)$ but some other multiplexing techniques may shift the data to a later priority, causing a longer delay.}
    \label{fig: data in-out}
\end{subfigure}
\hfill
\begin{subfigure}[b]{0.35\textwidth}
    \centering
    \includegraphics[width=\linewidth]{images/arrival-service.png}
    \caption{Characteristics of an arrival curve and a service curve. From any point of observation, the arriving data never exceeds its arrival curve; the departure data is also never less than the service curve with respect to the data arrival.}
    \label{fig: arrival-service curves}
\end{subfigure}
\hfill
\begin{subfigure}[b]{0.33\textwidth}
    \centering
    \includegraphics[width=\linewidth]{images/bound.png}
    \caption{Delay and backlog bounds of a system. Backlog is the maximum vertical distance between $\alpha(t)$ and $\beta(t)$; FIFO delay is their maximum horizontal distance; but for arbitrary multiplexing, the delay guarantee is when the system clears its buffer, thus it's the intersection of $\alpha(t)$ and $\beta(t)$.}
    \label{fig: system bounds}
\end{subfigure}
\caption{Network calculus framework. We let $R(t)$ and $R^*(t)$ be the arrival and departure data flow of a system; $\alpha(t)$ be the piecewise linear concave arrival curve and $\beta(t)$ be the piecewise linear convex service curve of a system.}
% \hossein{Better to show piece-wise linear concave arrival curve and piece-wise linear convex service curve instead of token-bucket and rate-latency.}}
\end{figure*}

We recall some of the network calculus essentials for a better understanding of the framework used in Saihu. In the following context, we use the following notation: $\mbb{R}^+$ is the set of non-negative real numbers; $[x]_+$ denotes $\max(0, x)$

The data flow is by convention modeled as a left-continuous wide-sense increasing function $R(t): \mbb{R}^+ \mapsto \mbb{R}^+$ with respect to time $t$~\cite{ncbook2001leboudec}. 

A system $\mcal{S}$ receives arrival data described as a cumulative function $R(t)$ and delivers departure data as another cumulative function $R^*(t)$. Figure~\ref{fig: data in-out} illustrates such a system $\mcal{S}$. The benefit of representing a system like this is that we can observe system backlog and delay with such a model. 

\begin{definition}[Backlog and Delay~\cite{ncbook2001leboudec}]
    The backlog of a system at time~$t$ is
    \begin{equation}
        b(t) = R(t) - R^*(t)
    \end{equation}
    
    The virtual delay of a FIFO system at time $t$ is
    \begin{equation}
        d_{FIFO}(t) = \inf \lbp \tau \geq 0 : R(t) \leq R^*(t+\tau) \rbp
    \end{equation}
\end{definition}



The backlog of a system can be viewed as the vertical distance between $R$ and $R^*$. The FIFO (\textit{First-in First-out}) delay is the horizontal distance between $R$ and $R^*$. One may obtain other delay values if the multiplexing technique is not FIFO.

% \begin{figure}
%     \centering
%     \includegraphics[width=0.9\linewidth]{images/in-out.png}
%     \caption{In/out data flow; delay and backlog}
%     \label{fig: data in-out}
% \end{figure}

Since we are interested in the system guarantee instead of a single instance of data flow, we would like to have general bounds to the arrival and departure data flows. Therefore, we define \textit{arrival curve} and \textit{service curve} as the bounds of arrival and departure data flows.

\begin{definition}[Arrival Curve~\cite{ncbook2001leboudec}]
    Given a wide-sense increasing function $\alpha: \mbb{R}^+ \mapsto \mbb{R}^+$, we say that a flow $R(t)$ is $\alpha$-constrained if and only if for all $s \leq t$:
    \begin{equation}
        R(t) - R(s) \leq \alpha(t-s)
    \end{equation}
    We say $R(t)$ has $\alpha$ as an arrival curve.
\end{definition}

\begin{definition}[Service Curve~\cite{ncbook2001leboudec}]
    Given a wide-sense increasing function $\beta: \mbb{R}^+ \mapsto \mbb{R}^+$ and $\beta(0) = 0$. A system $\mcal{S}$ having $R(t)$ and $R^*(t)$ as its arrival and departure flows. We say $\mcal{S}$ offers a service curve $\beta$ if and only if
    \begin{equation}
        R^*(t) \geq (R \otimes \beta)(t) =: \inf_{s \leq t} \lbp R(s) + \beta(t-s) \rbp
    \end{equation}
    where $\otimes$ denotes the min-plus convolution
\end{definition}

Figure~\ref{fig: arrival-service curves} illustrates the arrival and service curves. Any segment of arrival flow $R(t)$ is constrained by arrival curve $\alpha$ and the output curve $R^*(t)$ is always no less than the curve $R\otimes\beta$. As a result, an arrival curve upper bounds the incoming traffic, and a service curve lower bounds the outgoing traffic.

% \begin{figure}
%     \centering
%     \includegraphics[width=\linewidth]{images/arrival-service.png}
%     \caption{Arrival/Service curve}
%     \label{fig: arrival-service curves}
% \end{figure}

We consider 2 special types of curves throughout this paper, \textit{token-bucket} (or sometimes called \textit{leaky-bucket}) curve and \textit{rate-Latency} curve.

\begin{definition}[Token-bucket and Rate-latency~\cite{ncbook2001leboudec}]
    A token-bucket curve $\gamma_{r,b}$ with arrival rate $r$ and burst $b$ is defined as
    \begin{equation}
        \gamma_{r,b}(t) = b + rt
    \end{equation}

    A rate-latency curve $\beta_{R,T}$ with service rate $R$ and latency $T$ is defined as
    \begin{equation}
        \beta_{R,T}(t) = R \lb t - T \rb_+
    \end{equation}
\end{definition}

A token-bucket curve is determined by a burst $b$ and an arrival rate~$r$. Burst represents the maximum possible data volume that can arrive simultaneously, and arrival rate represents the maximum long-term data rate~\cite{bouillard2022tradeoff}.
A rate-latency curve is determined by a latency~$T$ and a service rate~$R$. Latency represents the time a server needs before starting to process the incoming data, and service rate represents the minimum rate to process data after the initial latency.

With the help of arrival and service curves, we can derive delay and backlog bounds for a system $\mcal{S}$ illustrated in Figure~\ref{fig: system bounds}. Suppose a system $\mcal{S}$ has arrival curve $\alpha$ and service curve~$\beta$, its worst-case backlog $b^*$ is the maximum vertical distance between~$\alpha$ and~$\beta$. Similarly, depending on the multiplexing technique applied to the system, its worst-case delay bound $d^*$ is the maximum horizontal distance between $\alpha$ and $\beta$ if $\mcal{S}$ is a FIFO system. If we don't have any information about its multiplexing technique, referred to as arbitrary multiplexing, the best we can say is that when $\alpha$ and $\beta$ intersect each other, where all data has been delivered out of the system. Consequently, the worst-case delay bound for arbitrary multiplexing is the time required for $\mcal{S}$ to clear its buffer.

% \begin{figure}
%     \centering
%     \includegraphics[width=\linewidth]{images/bound.png}
%     \caption{System delay/backlog bounds}
%     \label{fig: system bounds}
% \end{figure}

While a service curve captures the slowest possible output speed of a system, a link's transmission capacity limits the speed as well. Hence, we model this phenomenon using a \textit{greedy shaper} with a sub-additive function $\sigma: \mbb{R}^+ \mapsto \mbb{R}^+$ concatenated with a server. We consider a concatenation as shown in Figure \ref{fig: system}. By convention we assume $\sigma(0) = 0$ and $\beta(t) \leq \sigma(t), \forall t \in \mbb{R}^+$, meaning that the buffer is cleared at the beginning and the service never exceed its physical limitation. With the above definition, such greedy shaper conserves the service provided by the system due to theorem \ref{thm: shaping}.

\begin{figure}[thb]
    \centering
    \includegraphics[width=0.7\linewidth]{images/system.png}
    \caption{Shaping of departure data. A flow that has an arrival curve $\alpha$ feeds into a server with an arrival data flow $R(t)$. The server having service curve $\beta$ takes $R(t)$ and gives a departure data flow $R^*(t)$ to a shaper with shaping function $\sigma$. The shaper takes $R^*(t)$ and shape the data flow as another departure $D(t)$.}
    \label{fig: system}
\end{figure}


\begin{theorem}[Shaping conserves service \cite{ncbook2001leboudec}]
\label{thm: shaping}
Following the system shown in Figure \ref{fig: system}, we have
\begin{equation}
     D = R^* \otimes \sigma \geq \lp R \otimes \beta \rp \otimes \sigma = R \otimes \lp \beta \otimes \sigma \rp = R \otimes \beta
\end{equation}
\end{theorem}

In the following context, we model the shaping function $\sigma$ as a token-bucket curve $\gamma_{C,L}$ with transmission capacity $C$ and the packet size $L$ to capture the link capacity and packetization~\cite{bouillard2022tradeoff}.

	

\section{Review of ML-based ride-hailing planning}
\label{sec:review}
\revise{In this section, we review matching, repositioning, and joint matching and repositioning in Sec.~\ref{sec:review-matching} Sec.~\ref{sec:review-repositioning}, and Sec.~\ref{sec:review-joint}, respectively.}
In each part, we discuss the collective and the distributed strategy separately.
Fig.~\ref{fig:review-outline} gives an outline of the review.

\begin{figure*}[h]
	\centering
	\includegraphics[width=0.8\linewidth]{figs/survey-taxonomy.pdf}
	\caption{\revise{A taxonomy of the ride-hailing planning literature. %is summarized.
	In each category, we discuss three works as representative examples.}}
	\label{fig:review-outline}
\end{figure*}

\subsection{Matching}
\label{sec:review-matching}

\subsubsection{Collective Matching}
\revise{RL is a promising technique for solving the matching problem.
Chen et al.~\cite{chen2020order} propose an RL-based solution in which
a deep evaluation network, which is a plain feed-forward neural network, is used to calculate a score for each pair of driver and rider based on the predicted detour distance, vehicle's seat utilization rate, and profit achieved if they get matched.
For each new ride request, the vehicle with the highest score will be assigned to serve the rider.
When the trip of the ride request is finished, the observed reward, i.e., the sum of the increased profit of the driver and, if any, the reduced cost of the rider through sharing the ride with others, is used to guide the learning process of the deep evaluation network.
Agussurja et al.~\cite{agussurja2019state} formulate the matching problem as a two-stage planning process.
In the first stage, ride requests to be scheduled are selected from all the unserved ones, the problem of which is modeled as a Markov Decision Process.
An approximated value iteration algorithm is used to learn the value function for the matching actions.
In the second stage, the final matching decision is made between the selected ride requests and all vehicles based on the learned value function.
\revise{Kullman et al.~\cite{kullman2022dynamic} apply deep RL to develop matching policies whose decisions leverage the Q-value approximations learned by deep neural networks.}
Multi-hop ride-hailing can improve the efficiency of a ride-hailing system.
To find the transfer points for each transferring trip in the multi-hop ride-hailing service, Xu et al.~\cite{xu2020highly} use a multi-layer feed-forward network to predict the reachable areas of vehicles, based on which the search space of possible vehicle pairs and transfer points for transferring riders is pruned.
In this way, the transfer points searching process can be more efficient.
	Wang et al.~\cite{wang2023optimization} also consider the scenario where riders are allowed to transfer between vehicles.
	They leverage RL to learn a policy that estimates the values of all the vehicles, which are then used to compute the optimal matching decisions by integer-linear programming.
The lengths of the time-intervals between the matching decisions can have critical impact in the matching outcomes.
	Specifically, the efficiency of matching may be improved substantially if the matching is delayed by adaptively adjusting the matching time-intervals according to the real-time situation of the riders and drivers.
	Wang et al.~\cite{wang2019adaptive} find that, if riders are willing to wait for a certain amount of time even if there are available vehicles that can serve them right away, the ride-hailing system can achieve better results, for example, in terms of the total vehicle miles traveled.
	In their solution, 
	 they propose to use an RL policy to decide for each rider, at each time step, whether to conduct matching for her/him, or %leave her/him alone and
	wait for the next time step. 
	Similarly, Qin et al.~\cite{qin2021optimizing} leverage RL in solving the ride-hailing matching problem with dynamic matching time-intervals.}

\revise{Clustering techniques are frequently used in ride-hailing planning.
Hong et al.~\cite{hong2017commuter} propose to use a density-based clustering algorithm, specifically DBSCAN \cite{parimala2011survey}, to identify riders that share similar itineraries based on their historical traveling trajectories. 
To alleviate the computational overhead caused by the large number of distance queries in the matching process, Zhang et al.~\cite{zheng2018order} propose a new clustering algorithm that groups the geographical locations in the road network into different clusters.
Then, the distance between any two nodes is approximated by the distance between the centers of the clusters they belong to. 
Shen et al.~\cite{shen2019roo} propose a spatial-temporal distance metric that measures the similarity of each pair of ride requests.
The ride requests are grouped by a clustering process based on the proposed distance metric.
Then, shared-rides are computed within each group of ride requests.
Another clustering algorithm is proposed by \cite{trasarti2011mining} to extract the mobility profiles from riders' and drivers' historical itineraries.
The matching between riders and drivers is determined based on the similarities between their profiles.}


\revise{An increasing number of collective matching solutions leverage various other ML techniques in planning.
Most of them take social factors of drivers and riders into consideration \cite{mitropoulos2021systematic}.
To mitigate the social barriers in the ride-hailing process, especially in shared-rides, Yatnalkar et al.~\cite{yatnalkar2020enhanced} and Narman et al.~\cite{narman2021enhanced} use Support Vector Machine (SVM) to predict the user social types, e.g., chatty, safety, or punctuality, based on their registered user characteristics.
Riders with similar social characteristics %are more likely
would be more willing to share a trip.
%on their closest available vehicle.
Levinger et al.~\cite{levinger2020human} use a feed-forward neural network to predict rider satisfaction levels according to their profile and trip information.
They proposed a stochastic algorithm to compute the matching decision with rider satisfaction level maximization as the objective.
Montazery and Wilson \cite{montazery2016learning, montazery2018new} propose to take into account the user preference in evaluating the weight (benefit) of the matching between each pair of rider and driver, which is given by their proposed support vector machine-based score function.
With the value calculated, the final matching can be obtained by solving an optimization problem in which the sum of the weights of those matched pairs is maximized.
Tang et al.~\cite{tang2020efficient} model various types of information (e.g., driver, rider, travel time, and activity) and their relationships within a ride-hailing system using a Heterogeneous Information Network (HIN) \cite{sun2012mining}.
Each driver or rider is projected to a multi-dimensional embedding (vector) using the skip-gram model \cite{mikolov2013efficient}.
Moreover, the skip-gram is conducted on node sequences obtained by meta path-based random walks originating from the corresponding node within the HIN \cite{dong2017metapath2vec}.
The cosine similarity between the embeddings of each driver-rider pair is then used to identify possible matching.
Zhang et al.~\cite{zhang2017taxi} consider a scenario where each rider is assigned to multiple drivers (to improve the order answer rate), and riders are free from having to enter the details of destinations (to improve the user experience).
They first leverage historical data to model the probability distribution of destinations of each rider based on his/her departure time and location with Bayesian rules, which is followed by predicting the acceptance probability between the rider and available drivers with logistic regression \cite{friedman2001elements} and gradient boosted decision tree \cite{mason1999boosting}.
They propose a hill climbing-based algorithm to solve the matching problem, which is formulated as an NP-hard combinatorial optimization with maximizing the success rate of matching as the objective.
Schleibaum and M{\"u}ller \cite{schleibaum2020human} advocate taking the determinants of user satisfaction and explainable matching decisions into consideration.
One of their future studies is to find out whether increasing the explainability can improve user satisfaction level or not.}

\revise{It is worth mentioning that many ML-based collective matching strategies take advantage of the Kuhn-Munkres (KM) bipartite matching algorithm as a component of their decision-making pipelines \cite{jonker1986improving}.
Drivers and riders are usually regarded as the two sets of vertices in the target bipartite graph.
To guide the matching between ride requests and ride offers,  Guo et al.~\cite{guo2020spatiotemporal} propose spatial-temporal Thermo, which is used to reflect the demand density of different places and times.
They use Random Forest Regression \cite{breiman2001random} to map multiple features of spatial, temporal, and meteorological dimensions to Thermo.
The weight of each pair of driver and rider in the bipartite graph is estimated by Thermo.
A KM algorithm is then used to calculate the final matching decisions according to the constructed bipartite graph.
Similarly, Xu et al.~\cite{xu2018large} derive their matching decisions using the KM algorithm.
In contrast to \cite{guo2020spatiotemporal}, Xu et al.~\cite{xu2018large} leverage a policy evaluation algorithm to learn a value function which maps each pair of driver and rider to a score.
The KM algorithm calculates the final matching between drivers and riders based on the scores.
Guo and Xu \cite{guo2020deep} also conduct the matching planning using the KM algorithm.
The weight between each pair of driver and rider is obtained from a value function learned by a convolutional neural network-based Double Q-learning (Double DQN) algorithm \cite{van2016deep}.}

\subsubsection{Distributed Matching}
\revise{RL is also a powerful technique for distributed matching \cite{sutton1999reinforcement}.
Gu{\'e}riau and Dusparic \cite{gueriau2018samod} use the Q-learning algorithm to train a policy for each agent (driver) to choose the pickup or rebalancing action based on the environment state, including the status of itself and current distribution of supply and demand.
If pickup action is chosen, then the vehicle will go and pick up the nearest rider.
In their follow-up work \cite{gueriau2020shared}, they extend the method to consider traffic congestion when agents are making decisions.
Wang et al.~\cite{wang2018deep} propose to use the DQN \cite{mnih2015human}, in which a deep neural network is employed to estimate the state-action value function from a single driver's perspective.
Many methods of distributed matching allow the decisions to be determined individually while the matching policy is trained collectively.
For example, De Lima et al.~\cite{de2020efficient} follow the QMIX framework proposed in \cite{rashid2018qmix}, in which the coordinated planning policies are trained by learning a joint action-value function for multiple vehicles and riders aiming at optimizing a global objective.
In the execution process, the matching decision of each vehicle is made in a distributed  manner following its own component in the learned action-value function.
By ensuring the monotonicity of the relationship between the global action-value and the action-value of each passenger, the objectives of distributed planning decisions are ensured to coincide with the centralized decisions during the training process.
Similar to \cite{de2020efficient}, Li et al.~\cite{li2019efficient} adopt the framework where the matching policy is trained in a centralized manner and executed in a distributed manner.
Specifically, they adopt the actor-critic RL framework, where actor and critic are two different networks used to decide and evaluate the action for each driver, respectively.
The coordination among drivers in the matching policy is enabled by the critic network.
It adopts the mean field approximation to model the interactions of drivers by calculating an average on the actions taken by their neighborhoods, which is then considered in the process of evaluating each driver's action.
\revise{In \cite{zhou2019multi}, another centralized training process is proposed, in which a Kullback–Leibler divergence optimization is used to balance the supply and demand and to enable coordination among the vehicles.}
In the execution phase, each driver chooses an action based on their own action-value functions.}

\revise{Some distributed matching strategies leverage other ML techniques.
They mostly determine the matching decisions based on the similarities between the riders and drivers in ride-hailing.
For example, Bicocchi and Mamei \cite{bicocchi2014investigating} use the bag-of-words model to summarize users' frequently visited places as vector representations, which are then fed to the Latent Dirichlet Allocation (LDA) \cite{blei2003latent} model to identify their patterns of daily travel routine behaviors.
Given a rider or a driver, his/her potential participants of shared-rides can be found by calculating the similarities between his/her daily travel routine and those of the other riders and drivers.
Lasmar et al.~\cite{lasmar2019rsrs} propose to leverage a multi-layer Perceptron model to learn user preferences based on their responses to the questionnaires.
For each rider, a ranking list of potential partners for shared-rides is generated according to the similarities between the predicted preferences of her/him and other riders.}

    
\subsection{Repositioning}
\label{sec:review-repositioning}
\subsubsection{Collective Repositioning}
\revise{Some collective repositioning methods leverage RL techniques.
Ride-hailing repositioning for electric vehicles is studied in \cite{liang2020mobility, tang2020online}, in which the state of charge of the electric vehicles is an important factor to be considered.
Liang et al.~\cite{liang2020mobility} develop a solution method utilizing deep RL combined with binary linear programming to obtain a regional joint planning policy for electric vehicles with their state of charge considered.
Using binary linear programming, each vehicle repositioning action is modeled as a binary decision variable, and its weight in the objective is obtained by the value function learned by the policy iteration method.
Similarly, Tang et al.~\cite{tang2020online} also combine RL with combinatorial optimization, in which the RL learned policy is used to advise decision making in the optimization step.
Liang et al.~\cite{liang2021integrated} adopt temporal-difference (TD) learning to obtain action-value function.
Different from \cite{de2020efficient}, the settings in \cite{liang2021integrated} do not allow factorization of the joint action-value function into individual ones while guaranteeing global maximization.
Thus, they formulate two linear programming instances to collectively find the decisions for the vehicles.
To improve the stability of the training process in RL, Fluri et al.~\cite{fluri2019learning} propose a cascading multi-level learning model.
In this model, the area concerned is split in halves as the number of levels of learning increases.
The policy training process proceeds in a top-down manner, i.e., from less to more fine-grained area partitioning. 
The motivation behind is that the policy trained from a coarse level can serve as guidance to the finer levels, which avoids the instability caused by directly training a policy with a large state size (w.r.t. the number of regions).
Fluri et al.~\cite{fluri2019learning} propose to leverage the Lloyd K-means algorithm \cite{lloyd1982least} to partition the area concerned into multiple smaller regions.
Deng et al.~\cite{deng2020multi} leverage the Proximal Policy Optimization algorithm (PPO) \cite{schulman2017proximal} to learn the joint repositioning policy for vehicles, in which the value- and policy-function are approximated by neural networks.
Shi et al.~\cite{shi2019optimal} use Deep Deterministic Policy Gradient (DDPG) \cite{silver2014deterministic} to learn the grid-based multiple vehicles repositioning policy with the objective of total profits maximization.
\revise{In \cite{shou2020reward}, a mean-filed multi-agent RL approach is leveraged to collectively relocate the vehicles in ride-hailing.}}


%In collective repositioning, 
\revise{Some other collective repositioning solutions leverage various ML techniques to predict future information of a ride-hailing system, which plays an important role in guiding the platforms to make better repositioning decisions \cite{chen2022h}.
Riley et al.~\cite{riley2020real} leverage Vector autoregression to forecast the future demand from region to region.
The predicted demand and current system status are then fed into two mixed-integer programming instances to find the desired distribution of vehicles and the assignment of vehicles to regions, respectively.
Iglesias et al.~\cite{iglesias2018data} use a Long Short-Term Memory (LSTM) neural network to predict the future ride requests for each pair of origin and destination within a certain time period.
The predicted information is then used as input to their proposed mixed-integer linear programming instance, which is solved to find the optimal rebalancing actions.
Xu et al.~\cite{xu2018taxi} use two LSTM-based and Mixture Density Network (MDN)-based models to predict the distributions of origins and destinations of future requests, respectively.
With a prediction on the distributions, the repositioning decisions are then obtained by solving a mixed-integer programming problem with total idle driving distance minimization as the objective.
Cheng et al.~\cite{cheng18taxis} leverage a multilevel logistic regression model to predict the likelihood of ride requests occurring at different times and places.
The online repositioning planning decisions of drivers are obtained by leveraging a centralized multi-period stochastic optimization model with both the real-time and predicted demand considered.
Li et al.~\cite{li2020data} and Gao et al.~\cite{gao2020learning} formulate the repositioning task as a two-stage stochastic programming problem.
The source of the stochasticity is the underlying uncertainty of the future demands, the probability distribution of which is obtained by kernel density estimation and a deep learning model combining the LSTM and MDN in \cite{li2020data} and \cite{gao2020learning}, respectively.
Pouls et al.~\cite{pouls2020idle} propose a forecast-driven repositioning solution framework, the core of which is a mixed-integer programming problem with the demand predictions as inputs.
Moreover, it is solved by an off-the-shelf solver called Gurobi \cite{gurobi}.
Note that, in practice, not all planning decisions can be successfully executed by the drivers at the end.
Xu et al.~\cite{xu2020recommender} take the first step to predict the failure possibility of repositioning tasks in the decision-making process, including situations where drivers disobey the planning or end up being unmatched for an unexpectedly long time even though they follow the repositioning planning decisions accordingly.
In the latter case, drivers will be compensated.
The failure rate of each repositioning task is predicted by XGBoost \cite{chen2016xgboost} with both driver- and environment-related features as inputs.
\revise{The problem of multi-vehicle collaboration optimization aiming at maximizing the platform's profit is converted into a minimum cost flow problem, which is solved by an off-the-shelf method called GNU Linear Programming Kit (GLPK) \cite{makhorin2008glpk}. }}


\subsubsection{Distributed Repositioning}
Geographical regions or grids (i.e., abstracts of individual locations) are usually used to model the road networks in the problem of ride-hailing repositioning.
Different from most of the repositioning methods (e.g., \cite{lin2018efficient, riley2020real, ke2019optimizing, li2019efficient, zhou2019multi}) in which the region of interest is divided into predefined and static geographic zones, Castagna et al.~\cite{castagna2020demand, castagna2021demand} leverage the Expectation-Maximization clustering algorithm to derive zones for rebalancing vehicles in an online manner.
They leverage the Proximal Policy Optimization algorithm (PPO) \cite{schulman2017proximal} to train a policy for each vehicle to decide whether to make a pick-up, drop-off, or repositioning action.
Specifically, similar to \cite{tang2021value}, the repositioning destination is also sampled from a probability distribution over all potential positions, which is determined by the number of unserved requests.
Different from \cite{castagna2020demand, castagna2021demand}, Verma et al.~\cite{verma2017augmenting} propose an iterative method to dynamically split the zones based on their expected revenue (Q-values).
The iterative splitting process does not terminate until the historical data is exhausted for the Q-values learning.
\revise{Different from most of the works that model the drivers as agents, Jin et al.~\cite{jin2019coride} regard each geographical region as an agent.}
By hierarchically partitioning the target areas into regions with different granularities, they perform hierarchical RL where the multi-head attention mechanism is used to capture the impacts among the neighboring agents.
Guo et al.~\cite{guo2021multi} try various methods (e.g., Support Vector Regression, Random Forest Regression, and k-Nearest Neighbors regression) to predict future demand density, which is then used to evaluate each region for their spatial-temporal value.
Each available vehicle chooses to stay still or relocate to a neighbor region in a probabilistic manner based on their spatial-temporal values, which can help avoid over-saturation of supply.
In \cite{provoostdemandprop}, the region of interest is represented as a graph.
They build two neural networks to predict the demand on vertices and the passenger flows on edges, respectively.
The proposed repositioning algorithm aims at satisfying the demand on edges in the decreasing order with the nearest vehicles found by backward traversing.

\revise{However, in spite of the various grid-based methods as discussed in most of the related works mentioned above, e.g., \cite{lin2018efficient, guo2021multi}, Jiao et al.~\cite{jiao20deep, jiao2021real} argue that grid-based repositioning policies are not satisfactory in practice because of the excessively-simplified and overlooked non-stationarity in the environment caused by the dynamic environment and the large number of vehicles when coarse-grained region-wise decisions are considered.}
They put forward the process of carrying out repositioning %algorithms
in industrial production by combining offline learning, i.e., batch RL, and online planning stages, i.e., decision-time planning \cite{sutton1999reinforcement}.
To counter the issues of coarse-grained decisions, Kim and Kim \cite{kim2020optimizing} uses a graph to model the road networks which is more realistic.
They build a Graph Neural Network to predict the future demands.
The repositioning destination of each driver is decided greedily based on a function of the predicted demand, the number of excessive vehicles, and the distance information to each candidate position.

\revise{Some other works also spend special effort on tackling the non-stationarity.
With the observation that the actions of drivers are independent (based on self interests), Chaudhari et al.~\cite{chaudhari2020learn} propose a vanilla RL framework where each driver, based on a probabilistic value denoting the extent to which coordination is needed, stochastically chooses to perform an action guided by the independent or coordinated policy.
Note that, although vehicles execute repositioning decisions sequentially in this solution framework, coordination in the latter policy is explicitly considered by solving a minimum cost flow problem for the optimal rebalancing flow of vehicles among all the regions (which is similar to \cite{xu2020recommender}).
In addition, the independence between different repositioning policies learned by the drivers concurrently also contributes to the non-stationarity of the environment.
In this regard, Verma et al.~\cite{verma2019entropy} propose a method for each driver to learn the information of other vehicles in order to make a better planning decision.}
%Concretely,
The principle of maximum entropy \cite{jaynes1957information} is leveraged to improve the predictability of the distribution of drivers even with only limited knowledge available, e.g., the local density of supply.
To tackle the non-stationary challenge in online ride-hailing as well as the catastrophic forgetting of RL \cite{kemker2018measuring}, Haliem et al.~\cite{haliem2020adapool, haliem2021adapool} propose to learn multiple repositioning policies to deal with different contexts of environments (e.g., peak/non-peak hours and weekends/weekdays).
When changes in the distribution of experiences are identified by their proposed change point detection algorithm, switching among those different policies is enabled so as to enhance adaptability to the dynamic environment.
Lei et al.~\cite{lei2019optimal} define the concept of stochastic relocation matrix.
The element in the $i$-th row and the $j$-th column within the matrix represents the probability that an empty vehicle located in the $i$-th region should relocate to the $j$-th region.
\revise{To circumvent the curse of dimensionality, they leverage low-rank approximation to project the original matrix onto a low-dimensional vector.}
They propose a deep convolution-LSTM model to learn how to predict the approximation vector based on the system status.
To alleviate the instability of the state-value function approximator caused by the large scale of its states, Tang et al.~\cite{tang2019deep} propose to bound its outputs by regularizing its worst-case variation w.r.t. %any
changes in its inputs (i.e., states).
Transfer learning proposed in \cite{wang2018deep} is applied to increase the adaptability of the trained model across different cities.

\revise{Besides the traditional ML techniques discussed above, RL, being another well-known technique for decision making in non-stationary environments, has been a key technology in distributed repositioning \cite{khetarpal2022towards,xie2021deep,mao2021near}.}
\revise{Liu et al.~\cite{liu2022deep} propose a single-agent deep RL approach which relocates vacant vehicles to regions with a large demand gap in advance.}
Nguyen et al.~\cite{nguyen2018policy} propose to use the RL framework to train a homogeneous repositioning policy for all agents, i.e., vehicles.
\revise{The policy is trained in a centralized manner with collective behaviors of drivers considered while executing in a distributed manner.}
He and Shin \cite{he2019spatio} leverage Double DQN with their proposed spatial-temporal capsule-based neural network as the state-action value approximator.
The inputs of the network proposed include the location of the vehicle to be relocated, distribution of other vehicles and riders, ride preferences, and some external factors that have impacts on supply and demand, e.g., weather conditions and holiday events.
With all those information processed, the estimated value for each candidate position given the current state of the target vehicle is obtained, and the final decision can be decided in a probabilistic manner.
A more elaborate analysis is presented in their follow-up study \cite{he2020spatio}.
Yu et al.~\cite{yu2019markov} formulate the single-vehicle repositioning planning problem as a Markov Decision Process.
They propose to leverage parallelized matrix operations to re-formulate the Bellman equation \cite{sutton1999reinforcement}, thus reducing the computational complexity in finding optimal planning policy.
Multi-hop ride-hailing repositioning is considered in \cite{singh2019reinforcement, singh2021distributed}.
Similar to \cite{al2019deeppool}, they predict the number of vehicles in each region for certain time slots ahead of time using an estimated time of arrival (ETA) model.
Double DQN is adopted for each vehicle to choose the best neighbor region to move forward based on the current status of all the vehicles and the predicted demand and supply.
    
\subsection{Joint Matching and Repositioning}
\label{sec:review-joint}
In this part, we review methods that jointly optimize matching and repositioning with ML techniques. 
Note that all of them belong to the category of distributed planning. The research works in this part leverage RL to guide the decision making process.
Different from the review given by Qin et al.~\cite{qin2021reinforcement}, we focus on the works that jointly decide matching and repositioning.

Haliem et al.~\cite{haliem2020distributed-a, haliem2021distributed} propose to consider both the matching and repositioning in the ride-hailing planning process.
In their ride-hailing systems, each vehicle conduct initial matching by greedily searching the nearest requests, after which an insertion-based method is used to finalize the potential request list.
Then each driver, based on the value function learned by the DQN, weighs the requests in the final list.
The riders who receive those proposed ride offers can decide whether to accept the offers and join the trips where shared-rides are allowed.
The trips can be solo-ride or shared-ride.
Drivers are repositioned in parallel with the matching process.
Each driver takes actions indicated by his/her trained RL agent, i.e., the decision-making policy, independently.
Their proposed solution framework learns an optimal policy for each driver as opposed to those RL-based methods with collective planning scheme where a central policy is used, e.g., \cite{oda2018movi}.
Note that, in some works, although each driver makes decisions independently (e.g., \cite{haliem2020distributed-a, haliem2021distributed}), all drivers share one trained policy (e.g., \cite{manchella2020passgoodpool, manchella2021flexpool}).
Manchella et al.~\cite{manchella2020passgoodpool, manchella2021flexpool} propose to collectively optimize the system objectives, e.g., minimizing the waiting times and routing times.
Nevertheless, they allow distributed inference at the level of individual drivers. 
Their proposed model can be used by each vehicle independently.
It helps decrease computational costs associated with the growth of distributed systems. 
Specifically, they utilize a Double DQN with the experience relay mechanism.
Their model learns a probabilistic dependence between drivers' actions and the reward function.
The trained policy indicates a destination for each driver if s/he is not matched with any rider according to their proposed heuristic matching algorithm. 
Similar to \cite{xu2020highly, singh2019reinforcement, singh2021distributed}, multi-hop transit is enabled in their solutions.
Wang et al.~\cite{wang2018deep} model the matching and repositioning problems as a Markov Decision Process and propose learning solutions based on DQNs to optimize the trained policy for the drivers.
\revise{Their solution uses a temporal and spatial expanded action search strategy to accommodate the scenarios where there is only sparse training data, e.g., certain remote regions in the middle of the night.}
Besides, to increase the learning adaptability and
efficiency, they propose to use a transfer learning method to leverage the knowledge across both spatial and temporal spaces.

Besides \cite{haliem2020distributed-a, haliem2021distributed, manchella2020passgoodpool, manchella2021flexpool, wang2018deep},
DQN is used in other works as well, e.g., \cite{al2019deeppool, guo2022deep, tang2021value, li2020balancing}.
In \cite{al2019deeppool}, each vehicle decides its action by learning the impact of its action on the reward using a DQN model without coordinating with other vehicles.
In \cite{guo2022deep}, the vehicle repositioning procedure is formulated as a Markov Decision Process.
By sampling the future riders based on the historical probability distribution, the proactive relocation of vehicles is realized via a deep RL framework, which is composed of a Convolutional Neural Network and a Double DQN module. \revise{Then a request-vehicle assignment scheme is presented based on the value function attained from the vehicle repositioning process.}
\revise{Similarly, Tang et al.~\cite{tang2021value} propose a planning framework for tackling both the matching and repositioning tasks, the core of which is a unified value function which is trained offline using abundant historical data and is updated during the online phase.}
With the value function learned, the matching problem is then solved by the method proposed in \cite{xu2018large}, while the reposition destination of each idle vehicle is determined in a probabilistic manner following the distribution given by the discounted long-term values of all the candidate positions.
Li and Allan \cite{li2020balancing} also leverage a global value function for both the tasks of matching and repositioning, which is learned by the value iteration algorithm with historical data of ride requests.



	\begin{table*}[tp]
\centering
{\color{black}\begin{threeparttable}
\caption{\revise{Summary of open-source trip related data sets.}}
\begin{tabular}{@{}C{2.2cm}C{2.4cm}cccm{3.8cm}@{}}
\toprule
\textbf{Source}& \textbf{City} &\textbf{Type} & \textbf{\#~of Records} & \textbf{Time} & \multicolumn{1}{>{\centering\arraybackslash}m{3.8cm}}{\textbf{Data Features}}\\ \midrule
\midrule
Uber Pickups \cite{p668-gy46-22} & New York, United States& Pickups &20M & \begin{tabular}[c]{@{}c@{}}April -- September 2014,\\January -- June 2015 \end{tabular} & Timestamp and location.\\ \midrule
NYC TLC \cite{tlctrip} & New York, United States& Ride request &500M&January 2009 -- July 2021 &Passenger count, start time, end time, origin, destination, distance, fee, etc.\\ \midrule
DiDi GAIA \cite{gaiadata}& Haikou and Chengdu, China& Ride requests  &18M&\begin{tabular}[c]{@{}c@{}}November 2016,\\May -- October 2017\end{tabular}&Trip ID, fee, start time, end time, origin, destination, etc.\\ \midrule
Chicago Data Portal \cite{chicagoalll} & Chicago, United States & Ride requests &199M &January 2013 -- December 2021&Trip ID, taxi ID, start time, end time, fee, origin, destination, etc.\\ \midrule
T-drive \cite{tdrive, yuan2010t, yuan2011driving} & Beijing, China& Trajectories &15M&February 2008&Taxi ID, location, and timestamp.\\ \midrule
GeoLife \cite{geolifetraj, zheng2008learning, zheng2008understanding, zheng2010understanding} & Beijing, China& Trajectories &18,670&April 2007 -- August 2012 &Timestamp, location, transportation mode, etc. \\ \midrule
Beijing Taxi Trajectories \cite{lian2018one} & Beijing, China & Trajectories &129M&May 2019& Taxi ID, location, timestamp, speed, occupancy indicator, etc.\\ \midrule
DiDi GAIA \cite{gaiadata} & Chengdu and Xi'an, China &Trajectories &-\tnote{*}&-&\multicolumn{1}{>{\centering\arraybackslash}m{3.8cm}}{\textbf{-}}\\ \midrule
ECML PKDD 2015 \cite{portokaggle} & Porto, Portugal  & Trajectories  & 2M&July 2013 -- June 2014 &Trip ID, taxi ID, the sequence of locations of the trajectory, etc.\\ \midrule
CRAWDAD EPFL \cite{c7j010-22}  & San Francisco, United States & Trajectories  &11M&May -- June 2008& Taxi ID, location, timestamp, and occupancy indicator.\\\midrule
CRAWDAD Roma \cite{c7qc7m-22}  & Roma, Italy & Trajectories  &22M&February -- March 2014& Taxi ID, location, and timestamp.\\\midrule
Jeju Vehicular Trajectories \cite{y8vk-wj40-22}  & Jeju, South Korea & Trajectories  &8M& - & Vehicle ID, location, speed, lane, etc.\\\midrule
Grab-Posisi \cite{grabsource, huang2019grab}  &Singapore and Jakarta, Indonesia& Trajectories  &84,000&April 2019&Trajectory ID, location, timestamp, speed, etc.\\\midrule
Shanghai Taxi Trajectories \cite{2877-mk46-19, liu2020optimization}  &Shanghai, China& Trajectories  &61M&April 1, 2018&Taxi ID, location, timestamp, speed, occupancy indicator, driving status, etc.\\\midrule
Foursqure \cite{foursquare} & Global & Check-ins &33M&April 2012 -- September 2013 & Venue ID, timestamp, location, etc.\\ \midrule
Brightkite \cite{brightkite} & - & Check-ins &4M&April 2008 -- October 2010&User ID, timestamp, location, etc.\\ \midrule
Gowalla \cite{gowalla} & -  & Check-ins &6M&February 2009 -- October 2010&User ID, timestamp, location, etc.\\ \midrule
LTA of Singapore \cite{singaporedata} & Singapore & \begin{tabular}[c]{@{}c@{}}Real-time\\locations\end{tabular}  &-&-&Location. \\\midrule
Uber \cite{traveltimedata} &  Global &Travel times  &-&-&Origin, destination, travel time, etc.\\\bottomrule
\end{tabular}
\label{table:data}
\begin{tablenotes}
\item[*] \revise{``-'' indicates that the corresponding information is not attainable.}
\end{tablenotes}
\end{threeparttable}}
\end{table*}

\begin{table*}[t]
\centering
{\color{black}\begin{threeparttable}
\caption{Summary of the major simulators.}
\begin{tabular}{@{}cccm{9.6cm}@{}}
\toprule
\textbf{Name} & \textbf{Open-Source} & \begin{tabular}[c]{@{}c@{}}\textbf{Sample Applications}\\\textbf{in Ride-hailing} \end{tabular}& \multicolumn{1}{>{\centering\arraybackslash}m{9.6cm}}{\textbf{Description}} \\ \midrule
\midrule
 DiDi \cite{didisimulation} & \xmark &\cite{tang2021value}&An online simulation platform to evaluate matching and repositioning algorithms.  Evaluation is run with DiDi's real-world data.\\ \midrule
 AMoDeus \cite{ruch2018amodeus} & \cmark&\cite{ruch2020quantifying}&A tool that uses an agent-based transportation simulation framework to simulate arbitrarily configured mobility-on-demand systems with static/dynamic demand. It includes standard benchmark algorithms and a graphical user interface.\\ \midrule
 AMoD2 \cite{amod2} & \cmark&\cite{li2021optimal}&A high-capacity ride-sharing simulator. It uses map data and taxi data from Manhattan. Three matching algorithms and one simple rebalancing algorithm are implemented. \\ \midrule
 Mod-abm \cite{abm-1.0, abm-2.0} & \cmark&\cite{wen2017rebalancing}&A platform for simulation of large-scale mobility-on-demand operations. It supports city-level systems in any urban setting.\\ \midrule
 MATSim \cite{horni2016multi} & \cmark&\cite{tsao2019model}&An open-source framework for implementing large-scale agent-based transport simulations. It consists of several modules which can be combined or used in stand-alone mode, for demand-modeling, traffic flow simulation, re-planning; and a controller to iteratively run simulations, and methods to analyze outputs generated by the modules.\\ \midrule
 SUMO \cite{lopez2018microscopic} & \cmark&\cite{castagna2021multi,zhu2021shared}&A microscopic and space-continuous traffic simulation platform that is suitable for the generation, evaluation, and validation of traffic scenarios of real-world size. It supports road network customization and demand modeling.\\ \midrule
 CityFlow \cite{zhang2019cityflow} & \cmark&-\tnote{*}&A multi-agent RL environment for large scale city traffic scenario. It supports flexible definitions for road network and traffic flow. It provides faster simulation than SUMO.\\ \midrule
 STaRS \cite{ota2016stars} & \xmark &-&A simulation framework for analyzing diverse ride-sharing scenarios, considering the platforms' needs and constraints. Its real-time trip assignments utilize a linear optimization algorithm, efficient indexing, and parallelization for scalability.\\ \midrule
 STaRS+ \cite{mounesan2021fleet} & \xmark &-&A simulation framework based on an integer linear programming model, using heuristic optimization and a novel shortest-path caching scheme for scalability. It supports the simulation of full-city scale ride-sharing with meeting points.\\ \midrule
 UberSim \cite{khalil2022realistic} & \cmark & \cite{salman2023quantifying} &A digital twin transportation simulation model for Birmingham, Alabama, incorporating various transportation modes to analyze the impact of ride-hailing services on urban traffic. It supports policy learning through reinforcement learning.\\ \midrule
 NYC-Yellow-Taxi-V0 \cite{chaudhari2020learn} & \cmark&-&A multi-agent RL environment based on the OpenAI Gym environment \cite{openai-gym}, which offers a toolkit for developing and comparing RL-based ride-hailing fleet management algorithms.\\ \midrule
 SMART-eFlo \cite{liu2022smart} & \cmark & - &An integrated framework that combines the SUMO simulator with multi-agent Gym for reinforcement learning studies, enabling researchers to easily design traffic scenarios and implement RL algorithms for electric fleet management problems\\ \midrule
\end{tabular}
\label{table:simulator}
\begin{tablenotes}
\item[*] ``-'' indicates no sample application in ride-hailing using the corresponding simulator.
\end{tablenotes}
\end{threeparttable}}
\end{table*}



\section{Resources for empirical studies}
\label{sec:resource}
Real-world data are essential in studying ML-based ride-hailing planning, e.g., to train a proposed model.
Further, in order to deploy those proposed planning strategies in practice, it is necessary to utilize a ride-hailing service (process) simulator to validate their performance with real-world data.
\revise{In this section, we present publicly available %multiple
open-source real-world data sets and several related simulators in Sec.~\ref{sec:resource-data} and Sec.~\ref{sec:resource-simulator}, respectively.}

\subsection{Data}
\label{sec:resource-data}
Historical records of ride requests are the most frequently used data in the literature.
New York City Taxi and Limousine Commission \cite{tlctrip} provides more than 11 years of records of ride requests with many data fields, e.g., the pickup and drop-off timestamps, pickup and drop-off locations, trip fare, and ride distance.
About 19 million Uber's pick-up records obtained from this data set %\cite{tlctrip}
are summarised in \cite{p668-gy46-22}.
Chicago Data Portal \cite{chicagoalll} provides a large data set consisting of trip records with various data fields recorded from 2013.
Ride request data of Haikou and Chengdu in China are publicly available in the DiDi GAIA program \cite{gaiadata}.
Besides, there are some other data sets that consist of people's check-in records, e.g., those collected from Foursquare \cite{foursquare}, Brightkite \cite{brightkite}, and Gowalla \cite{gowalla}.
They are informative in revealing ride demands as the data have recorded riders' target places they had traveled to.
In addition to the records of ride requests, some data of the supply side are also available.
\revise{Trajectory data recorded by sampling drivers' locations at a certain frequency are also commonly used in the literature, including trajectories recorded in Beijing (in three different data sets: T-drive \cite{tdrive, yuan2010t, yuan2011driving}, GeoLife \cite{geolifetraj, zheng2008learning, zheng2008understanding, zheng2010understanding}, and Beijing Taxi Trajectories \cite{lian2018one}), Chengdu \cite{gaiadata}, Xi'an \cite{gaiadata}, Porto \cite{portokaggle}, Jakarta \cite{grabsource, huang2019grab}, Singapore \cite{grabsource, huang2019grab}, San Francisco \cite{c7j010-22}, Roma \cite{c7qc7m-22}, Jeju \cite{y8vk-wj40-22}, and Shanghai \cite{2877-mk46-19, liu2020optimization}.}
The Land Transport Authority (LTA) of Singapore \cite{singaporedata} publicizes many APIs for accessing various kinds of transport-related data, including monthly statistics of taxi supply and real-time coordinates of all taxis that are currently available for hire. %offering ride service. 
\revise{Besides, the travel times among different locations in various cities across the world can be obtained in \cite{traveltimedata}.}
The data sets mentioned above are summarized in Table.~\ref{table:data}.
\revise{Note that, although abundant public data sets are available for use, none of them record those ride requests that ended up unserved (because of, for example, excessive waiting time).}
If both the served and unserved ride requests are treated in the ride-hailing simulation, the simulation results would be more realistic than if only
the served ride requests are considered. 

In addition to the trip-related data sets mentioned above, there are several important public data sources that could provide good values
to ride-hailing planning studies.
\revise{OpenStreetMap \cite{openstreetmap} captures worldwide road networks, which can be easily accessed by APIs (e.g., OSMnx) using Python \cite{boeing2017osmnx}.}
Travel times and speeds information of road networks of several well known cities have been made available in \cite{traveltimedata}.
Finally, historical data of weather conditions can be found in \cite{weatherdata}.
Note that weather conditions can be considered in ride-hailing planning
as they could affect the traffic in a major way
\cite{he2019spatio}.
    
    
    
\subsection{Simulators}
\label{sec:resource-simulator}
With the aforementioned real-world data, proposed planning strategies can be evaluated through simulators.
DiDi \cite{didisimulation} makes public an industrial-level simulation platform recently, which is used in the KDD CUP 2020 \cite{kddcup20}. 
The platform evaluates submitted matching and repositioning strategies using real-world data from the GAIA program.
Wen \cite{abm-1.0} develops an agent-based modeling platform for simulating autonomous mobility-on-demand systems, which is later upgraded to a version with better scalability and extensibility \cite{abm-2.0}.
Based on \cite{abm-2.0}, a high-capacity on-demand ride-sharing simulator is proposed in \cite{amod2} with several built-in matching and repositioning algorithms.
Ruch et al.~\cite{ruch2018amodeus} have released an open-source simulator named AMoDeus for accurate and quantitative analysis of matching and repositioning algorithms in the ride-hailing system.
Examples of its usages can be found in \cite{fluri2019learning, carron2019scalable}.
AMoDeus is built upon the open-source microscopic multi-agent transportation simulation environment MATSim \cite{horni2016multi}.
MATSim is also used to evaluate ride-hailing planning strategies, a case study of which can be found in \cite{bischoff2016simulation}.
Besides MATSim, there are similar public traffic simulation tools, including SUMO \cite{lopez2018microscopic} and CityFlow \cite{zhang2019cityflow}.
A more detailed comparison and analysis of the performance of traffic simulation tools can be found in \cite{allan2015benchmark}.
\revise{STaRS is a scalable simulation framework that takes the needs and constraints of platforms into consideration \cite{ota2016stars}. 
STaRS+ extends STaRS to support the simulation of ride-sharing with meeting points \cite{mounesan2021fleet}.}
\revise{For those RL-based ride-hailing planning methods, their policies can be trained using CityFlow \cite{zhang2019cityflow}, UberSim \cite{khalil2022realistic}, NYC-Yellow-Taxi-V0 \cite{chaudhari2020learn}, or SMART-eFlo \cite{liu2022smart}.}
The aforementioned simulators are summarized in Table.~\ref{table:simulator}.

    
	
\section{Conclusion and Future Work}
\label{sec:future}
% Whenever a new solution concept is defined, one of the most intuitive questions is how it can be related to other properties? 
We presented a practical solution to the problem of leximin optimization when only an approximate single-objective solver is available. 
The algorithm is guaranteed to terminate in polynomial time, and its approximation ratio degrades gracefully as a function of the approximation ratio of the single-objective solver.

Currently, our algorithm handles two main settings. First, when inaccuracies in the single-objective solver stem from numeric errors.
Second, when the problem is convex and satisfy several assumptions.
It may be interesting to study more settings in which the inaccuracies stem from computational hardness of the single-objective problem.
% Currently, our algorithm handles settings in which the inaccuracies in the single-objective solver stem from numeric errors.
% It may be interesting to study settings in which the inaccuracies stem from computational hardness of the single-objective problem.
%
%

% identify problems in which an appropriate approximate solver can be designed. 
In particular, to approximate the egalitarian welfare, it is common to model the problem as an integer program or as an exponential sized linear program (e.g., \cite{bansal2006santa, kawase_max-min_2020}) and then approximate the program using different techniques.
% rounding techniques or methods for convex optimization (such as the ellipsoid method).
Can these algorithms be generalized to consider the additional constraints described in Section \ref{sec:algo-short}? This will allow approximating leximin using the approach in this paper.
% In particular, in the problem of stochastic allocations (in Section \ref{sec:app}), to extend the approximation algorithm for the egalitarian welfare, we had to change some steps within.
% What if an algorithm for egalitarian welfare is provided as a black box --- could it be used to design the appropriate procedure to approximate leximin?

% In the context of fair division, this study assumes that there is an access to the true valuations of the agents involved. 
% In reality, people may lie about their valuations.
% Can our definition of approximate-leximin be related to some approximate version of truthfulness?

Another question is whether it is possible to obtain a better approximation factor for leximin, given an $(\multApprox, \additiveApprox)$-approximation algorithm for the single-objective problem.
Specifically, can an $(\multApprox, \additiveApprox)$-approximation to leximin can be obtained in polynomial time? 
If not, what would be the best possible approximation in this case?
% \erel{Mention the tightness of our results}


\iffalse % EREL: removed for the submission. To clarify later
Further, the algorithm suggested in Section \ref{sec:algo-short} tend to work very well if the single-objective optimization problems are convex, and in particular if they are linear programs. 
\erel{Why? the algorithm of \textcite{Ogryczak2004TelecommunicationsND} works even for non-convex programs.}
Can we find an algorithm that works with general approximation algorithms? For example, naive algorithms such as \emph{Next-fit}?
\fi

% \begin{itemize}
%     \item Meaning of approximately leximin in the context of other characteristics like truthfulness.
    
%     \item Solving more problems.
    
%     \item Is it possible to obtained a better approximation factor for leximin maximization in polynomial time? given that $(1-\beta)$ is the best possible for the egalitarian maximization, is it possible to obtain a $(1-\beta)$ approximately leximin optimal solution? what is the best possible approximation  in this case?
    
%     \item The algorithm works very well if the single-objective problem is convex, in particular if it is a linear programs, but can it be modify to work with general approximation algorithms? such as algorithms for makespan minimization?
% \end{itemize}
	\section{Conclusion}\label{sec:conclusion}
In this work, we focus on addressing the fundamental challenge of OOD detection tasks, which is how to fully understand the semantic discrepancy between the ID/OOD samples. We reveal that the key to success in the realistic SCOOD task is to allocate as many ID samples in the unlabeled set correctly as possible. To this end, we propose a novel uncertainty-aware optimal transport scheme that introduces class-specific energy scores as guidance for effective label assignment. Experimental results show that our method achieves better performance than previous state-of-the-art methods on SCOOD benchmarks.

\textbf{Limitations.} In addition to temperature scaling, other techniques such as feature clipping applied in ReAct~\cite{sun2021react} also enhance the performance of energy score, so how to obtain an OOD score that best fits the SCOOD task can be further explored. Moreover, a setting highly related to SCOOD has been proposed in \cite{katz2022training} and formulated as a constrained optimization problem. We will also theoretically analyze these practical OOD settings in our feature work.

% \section*{Acknowledgments}
\textbf{Acknowledgments.} 
This work is supported by National Key R\&D Program of China under Grant 2020AAA0105701, National Natural Science Foundation of China (NSFC) under Grants 61872327, Major Special Science and Technology Project of Anhui, National Natural Science Foundation of China (62033012) and Ant Group through Ant Research Intern Program.

	
% \newpage
\bibliographystyle{IEEEtran}
%\bibliography{ref}
% \newpage
 \documentclass[lettersize, journal]{IEEEtran}
\usepackage{amsthm}
\usepackage{amsmath}
\usepackage{amssymb}
\usepackage{graphicx}
\usepackage{indentfirst}
\usepackage{algorithm,algorithmic}
%\usepackage[ruled, vlined]{algorithm2e}
\usepackage[top=1.6cm, bottom=2cm, left=2cm, right=2cm]{geometry}
\usepackage{threeparttable}
\usepackage{multirow}
\usepackage[usenames]{color}
% \pagestyle{empty}
\usepackage{subfigure}
\usepackage{hyperref}

% \usepackage[numbers]{natbib}
% \newcommand{\cites}[1]{\citeauthor{#1} \cite{#1}}
%\newenvironment{sloppypar}{\par\sloppy}{\par}
\usepackage{mathtools} 
\DeclarePairedDelimiter{\ceil}{\lceil}{\rceil} 
\DeclarePairedDelimiter\floor{\lfloor}{\rfloor}
\renewcommand{\algorithmicrequire}{\textbf{Input:}} 
\renewcommand{\algorithmicensure}{\textbf{Output:}}
\usepackage[dvipsnames]{xcolor}
\newcommand{\dc}[1]{\textcolor{red}{#1}}
\newcommand{\cm}[1]{\textcolor{gray}{#1}}
\newcommand{\alg}{\text{ALG}}
\newcommand{\opt}{\text{OPT}}
\newcommand{\hy}{\widehat{y}}
\makeatletter
\newcommand*{\rom}[1]{\expandafter\@slowromancap\romannumeral #1@}
\makeatother
\usepackage{xurl}
\usepackage{booktabs}
\usepackage{tikz}
\newcommand{\mycaption}[1]{\stepcounter{figure}\raisebox{-7pt}
  {\footnotesize Fig. \thefigure.\hspace{3pt} #1}}

\providecommand{\keywords}[1]
{
%	\small	
	\textbf{\textit{Keywords---}} #1
}

\theoremstyle{definition}
\newtheorem{lemma}{Lemma}
\newtheorem{claim}{Claim}
\newtheorem{theorem}{Theorem}
\newtheorem{remark}{Remark}
\newtheorem{definition}{Definition}
\newtheorem{objective}[theorem]{Objective}
\newtheorem{problem}{Problem}

\usepackage{multirow}
\usepackage{array}
\newcommand{\PreserveBackslash}[1]{\let\temp=\\#1\let\\=\temp}
\newcolumntype{C}[1]{>{\PreserveBackslash\centering}p{#1}}
\newcolumntype{R}[1]{>{\PreserveBackslash\raggedleft}p{#1}}
\newcolumntype{L}[1]{>{\PreserveBackslash\raggedright}p{#1}}
\usepackage{amssymb}% http://ctan.org/pkg/amssymb
\usepackage{pifont}% http://ctan.org/pkg/pifont
\newcommand{\cmark}{\ding{51}}%
\newcommand{\xmark}{\ding{55}}%
% \usepackage[parfill]{parskip}
\usepackage[font=footnotesize]{caption} 
% \newcommand{\dc}[1]{\textcolor{red}{#1}}
\usepackage{lipsum}
\usepackage{threeparttable}
 \newcommand{\revise}[1]{\textcolor{black}{#1}}

\allowdisplaybreaks
\title{A Survey of Machine Learning-Based Ride-Hailing Planning}
%\author{}

\author{
Dacheng Wen,~\IEEEmembership{Student Member,~IEEE},
Yupeng Li,~\IEEEmembership{Member,~IEEE},
Francis C.M. Lau
\IEEEcompsocitemizethanks{
		\IEEEcompsocthanksitem
            Dacheng Wen is with The University of Hong Kong, Hong Kong (e-mail: wdacheng@connect.hku.hk). Work done while Dacheng Wen was under the supervision of Yupeng Li and Francis C.M. Lau.\protect
            \IEEEcompsocthanksitem Yupeng Li (corresponding author) is with Hong Kong Baptist University, Hong Kong (e-mail: ivanypli@gmail.com).\protect
            \IEEEcompsocthanksitem Francis C.M. Lau is with The University of Hong Kong, Hong Kong (email: fcmlau@cs.hku.hk).
	}
}

\begin{document}
%	\markboth{Journal of \LaTeX\ Class Files}%,~Vol.~XX, No.~X, XX~2022
%	{Shell \MakeLowercase{\textit{et al.}}: A Sample Article Using IEEEtran.cls for IEEE Journals}
	
	\maketitle
	
% 	\documentclass{scrartcl}
\usepackage{tikz,pgfplots}
\usepackage{filecontents}
\begin{document}
\pgfkeys{/pgf/number format/.cd,1000 sep={\,}}

\begin{filecontents}{data.csv}
year,count
2016,998
2015,1000
2014,900
2013,837
2012,826
2011,784
2010,801
2009,731
2008,703
2007,632
2006,629
2005,516
2004,512
2003,476
2002,444
2001,497
2000,478
1999,400
1998,393
1997,399
1996,387
\end{filecontents}

\begin{tikzpicture}
\begin{axis}[ xlabel=Count, ylabel=Year]


\addplot[color=blue,mark=*] table[x=year, y=count, col sep=comma]{data.csv};

 \end{axis} 
 \end{tikzpicture}
\end{document}
	

Over the past few years, there has been a significant amount of research focused on studying the ReLU activation function, with the aim of achieving neural network convergence through over-parametrization. However, recent developments in the field of Large Language Models (LLMs) have sparked interest in the use of exponential activation functions, specifically in the attention mechanism.

Mathematically, we define the neural function $F: \R^{d \times m} \times  \mathbb{R}^d \rightarrow \mathbb{R}$ using an exponential activation function. Given a set of data points with labels $\{(x_1, y_1), (x_2, y_2), \dots, (x_n, y_n)\} \subset \mathbb{R}^d \times \mathbb{R}$ where $n$ denotes the number of the data. Here $F(W(t),x)$ can be expressed as $F(W(t),x) := \sum_{r=1}^m a_r \exp(\langle w_r, x \rangle)$, where $m$ represents the number of neurons, and $w_r(t)$ are weights at time $t$. It's standard in literature that $a_r$ are the fixed weights and it's never changed during the training. We initialize the weights $W(0) \in \mathbb{R}^{d \times m}$ with random Gaussian distributions, such that $w_r(0) \sim \mathcal{N}(0, I_d)$ and initialize $a_r$ from random sign distribution for each $r \in [m]$.

Using the gradient descent algorithm, we can find a weight $W(T)$ such that $\| F(W(T), X) - y \|_2 \leq \epsilon$ holds with probability $1-\delta$, where $\epsilon \in (0,0.1)$ and $m = \Omega(n^{2+o(1)}\log(n/\delta))$. To optimize the over-parametrization bound $m$, we employ several tight analysis techniques from previous studies [Song and Yang arXiv 2019, Munteanu, Omlor, Song and Woodruff ICML 2022]. 

 

	
	\begin{IEEEkeywords}
		Ride-hailing, machine learning, matching, repositioning, collective planning, distributed planning.
		%Article submission, IEEE, IEEEtran, journal, \LaTeX, paper, template, typesetting.
	\end{IEEEkeywords}
	
	\section{Introduction}
\label{sec:introduction}
% \begin{itemize}
%     % Diffusion of FL
%     \item {\st{Diffusion of FL}}
%     % Security threats to FL
%     \item {\st{Security threats to FL with particular focus on model poisoning}}
%     % Limitations of existing countermeasures
%     \item {\st{Current countermeasures (e.g., KRUM) and their limitations}}
%     % Proposed method and its advantages
%     \item {\st{Intuitive description of the proposed method and its difference (i.e., advantages) w.r.t. state of the art}}
%     % Main contributions
%     \item {\st{Summary of the main contributions of this work}}
%     % Paper's structure and organization
%     \item {\st{Paper's structure and organization}}
% \end{itemize}

% Diffusion of FL
Recently, {\em federated learning} (FL) has emerged as the leading paradigm for training distributed, large-scale, and privacy-preserving machine learning (ML) systems~\cite{mcmahan2017googleai,mcmahan2017aistats}. 
The core idea of FL is to allow multiple edge clients to collaboratively train a shared, global model without disclosing their local private training data.
%Specifically, an FL system consists of a central server and many edge clients; 
A typical FL round involves the following steps: {\em(i)} the server randomly picks some clients and sends them the current, global model; {\em(ii)} each selected client locally trains its model with its own private data; then, it sends the resulting local model to the server;\footnote{Whenever we refer to global/local model, we mean global/local model {\em parameters}.} {\em(iii)} the server updates the global model by computing an \emph{aggregation function}, usually the average (FedAvg), on the local models received from clients.
% \begin{enumerate}
%     \item[{\em(i)}] the server sends the current, global model to the clients and appoints some of them for training;
%     \item[{\em(ii)}] each selected client locally trains its copy of the global model with its own private data; then, it sends the resulting local model back to the server;\footnote{Whenever we refer to global/local model, we mean global/local model {\em parameters}.}
%     \item[{\em(iii)}] the server updates the global model by computing an \emph{aggregation function} on the local models received from clients (by default, the average, also referred to as FedAvg~\cite{mcmahan2017aistats}).
% \end{enumerate}
This process goes on until the global model converges. %(e.g., after a certain number of rounds or other similar stopping criteria).
%\\
% The advantages of FL over the traditional, centralized learning paradigm are undoubtedly clear in terms of flexibility/scalability (clients can join/disconnect from the FL network dynamically), network communications (only model weights\footnote{We will use \textit{parameters} and \textit{weights} interchangeably.} are exchanged between clients and server), and privacy (each client's private training data is kept local at the client's end and not uploaded to the server).
\\
% Security threats to FL
%However, the growing adoption of FL also raises security concerns~\cite{costa2022covert}, particularly about its confidentiality, integrity, and availability.
Although its advantages over standard ML, FL also raises security concerns~\cite{costa2022covert}. %, particularly about its confidentiality, integrity, and availability~\cite{costa2022covert}.
% OLD, LONG VERSION
% Indeed, some work deals with privacy leakage that may expose the local data of some clients~\cite{melis2019sp}. 
% A large body of work, instead, investigates attacks that usually aim to detriment the predictive accuracy of the learned global model. For instance, \emph{data poisoning} attacks achieve this goal by letting an adversary pollute the training set of some corrupt FL clients with maliciously crafted examples~\cite{jagielski2018sp}.
% Similarly, in \emph{model poisoning} the attacker attempts to tweak the global model weights~\cite{bhagoji2019pmlr} by directly perturbing the local model's weights of some infected FL clients before these are sent to the central server for aggregation, usually via so-called Byzantine attacks. 
% It turns out that Byzantine model poisoning attacks severely impact standard FedAvg; therefore, more robust aggregation functions must be designed to make FL systems secure.
Here, we focus on \emph{untargeted model poisoning} attacks~\cite{bhagoji2019pmlr}, where an adversary attempts to tweak the global model weights %\footnote{We will use the terms \textit{parameters} and \textit{weights} interchangeably.} 
by directly perturbing the local model's parameters of some infected clients before these are sent to the central server for aggregation.
In doing so, the adversary aims to jeopardize the global model \textit{indiscriminately} at inference time.
Such model poisoning attacks severely impact standard FedAvg; therefore, more robust aggregation functions must be designed to secure FL systems.
\\
% In this paper, we focus on designing a novel robust aggregation scheme at the server's end to contrast the effect of Byzantine model poisoning attacks.
%
% Current countermeasures and their limitations
%Several countermeasures have been proposed in the literature to combat model poisoning attacks on FL systems.
% Some methods use simple statistics more robust than plain average to smooth the impact of malicious updates (e.g., Trimmed Mean and FedMedian~\cite{yin2018icml}). 
% Other defenses implement outlier detection techniques to discard malicious updates from the aggregation performed at the server's end. Those are either based on heuristics (e.g., Krum/Multi-Krum~\cite{blanchard2017nips} and Bulyan~\cite{mhamdi2018pmlr}) or data-driven approaches (e.g., K-means clustering~\cite{shen2016acm} or DnC via spectral analysis~\cite{shejwalkar2021ndss}). 
% Finally, some strategies rely on a centralized ``source of trust'' to spot potential malicious updates (e.g., FLTrust~\cite{cao2020fltrust}).
% Several countermeasures have been proposed in the literature to combat model poisoning attacks on FL systems, i.e., to discard possible malicious local updates from the aggregation performed at the server's end. 
% These techniques range from simple statistics more robust than plain average (e.g., Trimmed Mean and FedMedian~\cite{yin2018icml}) to outlier detection heuristics (e.g., Krum/Multi-Krum~\cite{blanchard2017nips} and Bulyan~\cite{mhamdi2018pmlr}) or data-driven approaches (e.g., spectral analysis via K-means clustering~\cite{shen2016acm} or spectral analysis), or methods based on ``source of trust'' (e.g., FLTrust~\cite{cao2020fltrust}).
% OLD, LONG VERSION
%Several countermeasures have been proposed in the literature to combat Byzantine model poisoning attacks on FL systems.
% Descriptive statistics
% For example, Trimmed Mean and FedMedian aggregate local model updates using more robust statistics than standard average~\cite{yin2018icml}.
%
% % Heuristics for outlier detection
% Many existing Byzantine-resilient strategies implement some outlier detection heuristics to discard the model updates sent by potentially malicious clients from the input of the aggregation function.
% One of the most popular heuristics is Krum~\cite{blanchard2017nips}.
% This strategy tries to mitigate the impact of Byzantine attacks by selecting as a global model the local model with the smallest sum of Euclidean distances to {\em all} the other local models.
% Although powerful, Krum requires the server to know (or, at least, estimate) the number of malicious FL clients upfront, which is generally impossible in a realistic attack scenario. %
% Moreover, Krum may become ineffective for complex, high-dimensional model parameter spaces due to the curse of dimensionality.
% Bulyan~\cite{mhamdi2018pmlr} tries to overcome this issue by combining Krum with a variant of Trimmed Mean.
% % Data-driven outlier detection
% Other strategies use data-driven outlier detection techniques -- e.g., via K-means clustering~\cite{shen2016acm} -- to spot potential malicious local model updates. 
% %For instance, Shen et al. propose to cluster local model updates with K-means and thus identify outliers.
%
% % Other techniques
% As far as the server is concerned, any local model received can be from a potential malicious client. 
% FLTrust~\cite{cao2020fltrust} assumes the server acts as a client, i.e., trains a local model on an additional {\em trustworthy} dataset at the server's end and compares it against all the local models from other clients. 
% This way, the server can rely on some ``source of trust'' when discarding potentially malicious clients.
%\\
% Limitations of existing Byzantine-resilient strategies
Unfortunately, existing defense mechanisms either rely on simple heuristics (e.g., Trimmed Mean and FedMedian by~\cite{yin2018icml}) or need strong and unrealistic assumptions to work effectively (e.g., foreknowledge or estimation of the number of malicious clients in the FL system, as for Krum/Multi-Krum~\cite{blanchard2017nips} and Bulyan~\cite{mhamdi2018pmlr}, which, however, cannot exceed a fixed threshold).
Furthermore, outlier detection methods using K-means clustering~\cite{shen2016acm} or spectral analysis like DnC~\cite{shejwalkar2021ndss} do not directly consider the temporal evolution of local model updates received.
Finally, strategies like FLTrust~\cite{cao2020fltrust} require the server to collect its own dataset and act as a proper client, thereby altering the standard FL protocol.
\\
% OLD, LONG VERSION
% Overall, existing Byzantine-resilient strategies are either simple heuristics (e.g., FedMedian) or, if they are more complex, they rely on strong and unrealistic assumptions to work effectively (e.g., knowing the number of malicious clients in the FL system in advance, as for Krum and alike).
% Furthermore, data-driven outlier detection methods do not consider the temporary evolution of local model updates received (e.g., K-means clustering). 
% Finally, strategies like FLTrust requires the server to collect its own dataset and act as a proper client, thereby altering the standard FL protocol.
%
% Description of the proposed method
This work introduces a novel pre-aggregation \textit{filter} robust to untargeted model poisoning attacks. Notably, this filter $(i)$ operates without requiring prior knowledge or constraints on the number of malicious clients and $(ii)$ inherently integrates temporal dependencies. 
The FL server can employ this filter as a preprocessing step before applying \textit{any} aggregation function, be it standard like FedAvg or robust like Krum or Bulyan.
Specifically, we formulate the problem of identifying corrupted updates as a multidimensional (i.e., matrix-valued) time series anomaly detection task. 
The key idea is that legitimate local updates, resulting from well-calibrated iterative procedures like stochastic gradient descent (SGD) with an appropriate learning rate, show \textit{higher predictability} compared to malicious updates. This hypothesis stems from the fact that the sequence of gradients (thus, model parameters) observed during legitimate training exhibit regular patterns, as validated in Section~\ref{subsec:intuition}. %until convergence. 
%This regularity may be more pronounced for smooth convex loss functions, but it can still be captured within an appropriate time window, even for more complex and convoluted loss surfaces. 
%We provide evidence of this claim in Appendix~B, where we show that the average mutual information (i.e., ``predictability''), calculated over pairs of legitimate model updates sent at different FL rounds, is significantly higher than the corresponding computation for a malicious client.
\\
Inspired by the matrix autoregressive (MAR) framework for multidimensional time series forecasting~\cite{chen2021je}, we propose the FLANDERS ({\em \textbf{F}ederated \textbf{L}earning meets \textbf{AN}omaly \textbf{DE}tection for a \textbf{R}obust and \textbf{S}ecure}) filter.
The main advantages of FLANDERS over existing strategies like FLDetector~\cite{zhao2020multivariate} are its resilience to large-scale attacks, where $50\%$ or more FL participants are hostile, and the capability of working under realistic non-iid scenarios.
We attribute such a capability to two key factors: $(i)$ FLANDERS works without knowing a priori the ratio of corrupted clients, and $(ii)$ it embodies temporal dependencies between intra- and inter-client updates, quickly recognizing local model drifts caused by evil players. Below, we summarize our main contributions:

\begin{itemize}
\item[{\em(i)}]
We provide empirical evidence that the sequence of models sent by legitimate clients is more predictable than those of malicious participants performing untargeted model poisoning attacks.
\\
\item[{\em(ii)}] 
We introduce FLANDERS, the first pre-aggregation filter for FL robust to untargeted model poisoning based on multidimensional time series anomaly detection.
\\
\item[{\em(iii)}] 
We integrate FLANDERS into Flower,\footnote{\scriptsize{\url{https://flower.dev/}}} a popular FL simulation framework for reproducibility.
\\
\item[{\em(iv)}] 
We show that FLANDERS improves the robustness of the existing aggregation methods under multiple settings: different datasets, client's data distribution (non-iid), models, and attack scenarios.
\\
\item[{\em(v)}] 
We publicly release all the implementation code of FLANDERS along with our experiments.\footnote{\scriptsize{\url{https://anonymous.4open.science/r/flanders_exp-7EEB}}}
\end{itemize}

% Paper's structure and organization
The remainder of the paper is structured as follows. %some related work and the current state-of-the-art solutions to security issues that FL entails. 
Section~\ref{sec:background} covers background and preliminaries. 
In Section~\ref{sec:related}, we discuss related work.
Section~\ref{sec:problem} and Section~\ref{sec:method} describe the problem formulation and the method proposed. % to tackle it. 
Section~\ref{sec:experiments} gathers experimental results. %, and Section~\ref{sec:limitations} discusses some limitations of this work.
Finally, we conclude in Section~\ref{sec:conclusion}.
 %discusses the limitations of this work and draws future research directions.
%reports conclusions and draws perspectives for future research directions.

%%%%%%% OLD %%%%%%%
%to overcome the resilience of Byzantine failures in distributed Stochastic Gradient Descent computations. 
% The strength of Krum is its time complexity, which is linear in the gradient dimension. 
% However, the robustness of the approach is guaranteed for gradient-based learning applications only when the majority of the clients are not compromised. 
% Besides, the aggregation mechanism of Krum, as well as that of similar methods, is robust from a coarse-grained perspective and does not provide solutions to errors and perturbations that may occur at inference time.
%A related approach to~\cite{blanchard2017nips} is the work of Su et al.~\cite{su2016dc}. Here, the authors propose an iterated approximate agreement to tackle a multi-layer scenario attacked by Byzantine agents. 
%However, the method works efficiently on the sole discrete context and it is inapplicable to continuous state environments.
%\gabri{Maybe, we should just talk about the main limitations of existing countermeasures without digging into their details (or, we can just mention Krum as this is the most popular one). I will move the description of all these methods to the Related Work section.}
	\section{Background on Network Calculus}
\label{sec: background}


\begin{figure*}[tbh]
\centering
\begin{subfigure}[b]{0.3\textwidth}
    \centering
    \includegraphics[width=\linewidth]{images/in-out.png}
    \caption{Arrival and departure data and their relation with delay $d(t)$ and backlog $b(t)$. For a FIFO system, the delay is the horizontal distance between $R(t)$ and $R^*(t)$ but some other multiplexing techniques may shift the data to a later priority, causing a longer delay.}
    \label{fig: data in-out}
\end{subfigure}
\hfill
\begin{subfigure}[b]{0.35\textwidth}
    \centering
    \includegraphics[width=\linewidth]{images/arrival-service.png}
    \caption{Characteristics of an arrival curve and a service curve. From any point of observation, the arriving data never exceeds its arrival curve; the departure data is also never less than the service curve with respect to the data arrival.}
    \label{fig: arrival-service curves}
\end{subfigure}
\hfill
\begin{subfigure}[b]{0.33\textwidth}
    \centering
    \includegraphics[width=\linewidth]{images/bound.png}
    \caption{Delay and backlog bounds of a system. Backlog is the maximum vertical distance between $\alpha(t)$ and $\beta(t)$; FIFO delay is their maximum horizontal distance; but for arbitrary multiplexing, the delay guarantee is when the system clears its buffer, thus it's the intersection of $\alpha(t)$ and $\beta(t)$.}
    \label{fig: system bounds}
\end{subfigure}
\caption{Network calculus framework. We let $R(t)$ and $R^*(t)$ be the arrival and departure data flow of a system; $\alpha(t)$ be the piecewise linear concave arrival curve and $\beta(t)$ be the piecewise linear convex service curve of a system.}
% \hossein{Better to show piece-wise linear concave arrival curve and piece-wise linear convex service curve instead of token-bucket and rate-latency.}}
\end{figure*}

We recall some of the network calculus essentials for a better understanding of the framework used in Saihu. In the following context, we use the following notation: $\mbb{R}^+$ is the set of non-negative real numbers; $[x]_+$ denotes $\max(0, x)$

The data flow is by convention modeled as a left-continuous wide-sense increasing function $R(t): \mbb{R}^+ \mapsto \mbb{R}^+$ with respect to time $t$~\cite{ncbook2001leboudec}. 

A system $\mcal{S}$ receives arrival data described as a cumulative function $R(t)$ and delivers departure data as another cumulative function $R^*(t)$. Figure~\ref{fig: data in-out} illustrates such a system $\mcal{S}$. The benefit of representing a system like this is that we can observe system backlog and delay with such a model. 

\begin{definition}[Backlog and Delay~\cite{ncbook2001leboudec}]
    The backlog of a system at time~$t$ is
    \begin{equation}
        b(t) = R(t) - R^*(t)
    \end{equation}
    
    The virtual delay of a FIFO system at time $t$ is
    \begin{equation}
        d_{FIFO}(t) = \inf \lbp \tau \geq 0 : R(t) \leq R^*(t+\tau) \rbp
    \end{equation}
\end{definition}



The backlog of a system can be viewed as the vertical distance between $R$ and $R^*$. The FIFO (\textit{First-in First-out}) delay is the horizontal distance between $R$ and $R^*$. One may obtain other delay values if the multiplexing technique is not FIFO.

% \begin{figure}
%     \centering
%     \includegraphics[width=0.9\linewidth]{images/in-out.png}
%     \caption{In/out data flow; delay and backlog}
%     \label{fig: data in-out}
% \end{figure}

Since we are interested in the system guarantee instead of a single instance of data flow, we would like to have general bounds to the arrival and departure data flows. Therefore, we define \textit{arrival curve} and \textit{service curve} as the bounds of arrival and departure data flows.

\begin{definition}[Arrival Curve~\cite{ncbook2001leboudec}]
    Given a wide-sense increasing function $\alpha: \mbb{R}^+ \mapsto \mbb{R}^+$, we say that a flow $R(t)$ is $\alpha$-constrained if and only if for all $s \leq t$:
    \begin{equation}
        R(t) - R(s) \leq \alpha(t-s)
    \end{equation}
    We say $R(t)$ has $\alpha$ as an arrival curve.
\end{definition}

\begin{definition}[Service Curve~\cite{ncbook2001leboudec}]
    Given a wide-sense increasing function $\beta: \mbb{R}^+ \mapsto \mbb{R}^+$ and $\beta(0) = 0$. A system $\mcal{S}$ having $R(t)$ and $R^*(t)$ as its arrival and departure flows. We say $\mcal{S}$ offers a service curve $\beta$ if and only if
    \begin{equation}
        R^*(t) \geq (R \otimes \beta)(t) =: \inf_{s \leq t} \lbp R(s) + \beta(t-s) \rbp
    \end{equation}
    where $\otimes$ denotes the min-plus convolution
\end{definition}

Figure~\ref{fig: arrival-service curves} illustrates the arrival and service curves. Any segment of arrival flow $R(t)$ is constrained by arrival curve $\alpha$ and the output curve $R^*(t)$ is always no less than the curve $R\otimes\beta$. As a result, an arrival curve upper bounds the incoming traffic, and a service curve lower bounds the outgoing traffic.

% \begin{figure}
%     \centering
%     \includegraphics[width=\linewidth]{images/arrival-service.png}
%     \caption{Arrival/Service curve}
%     \label{fig: arrival-service curves}
% \end{figure}

We consider 2 special types of curves throughout this paper, \textit{token-bucket} (or sometimes called \textit{leaky-bucket}) curve and \textit{rate-Latency} curve.

\begin{definition}[Token-bucket and Rate-latency~\cite{ncbook2001leboudec}]
    A token-bucket curve $\gamma_{r,b}$ with arrival rate $r$ and burst $b$ is defined as
    \begin{equation}
        \gamma_{r,b}(t) = b + rt
    \end{equation}

    A rate-latency curve $\beta_{R,T}$ with service rate $R$ and latency $T$ is defined as
    \begin{equation}
        \beta_{R,T}(t) = R \lb t - T \rb_+
    \end{equation}
\end{definition}

A token-bucket curve is determined by a burst $b$ and an arrival rate~$r$. Burst represents the maximum possible data volume that can arrive simultaneously, and arrival rate represents the maximum long-term data rate~\cite{bouillard2022tradeoff}.
A rate-latency curve is determined by a latency~$T$ and a service rate~$R$. Latency represents the time a server needs before starting to process the incoming data, and service rate represents the minimum rate to process data after the initial latency.

With the help of arrival and service curves, we can derive delay and backlog bounds for a system $\mcal{S}$ illustrated in Figure~\ref{fig: system bounds}. Suppose a system $\mcal{S}$ has arrival curve $\alpha$ and service curve~$\beta$, its worst-case backlog $b^*$ is the maximum vertical distance between~$\alpha$ and~$\beta$. Similarly, depending on the multiplexing technique applied to the system, its worst-case delay bound $d^*$ is the maximum horizontal distance between $\alpha$ and $\beta$ if $\mcal{S}$ is a FIFO system. If we don't have any information about its multiplexing technique, referred to as arbitrary multiplexing, the best we can say is that when $\alpha$ and $\beta$ intersect each other, where all data has been delivered out of the system. Consequently, the worst-case delay bound for arbitrary multiplexing is the time required for $\mcal{S}$ to clear its buffer.

% \begin{figure}
%     \centering
%     \includegraphics[width=\linewidth]{images/bound.png}
%     \caption{System delay/backlog bounds}
%     \label{fig: system bounds}
% \end{figure}

While a service curve captures the slowest possible output speed of a system, a link's transmission capacity limits the speed as well. Hence, we model this phenomenon using a \textit{greedy shaper} with a sub-additive function $\sigma: \mbb{R}^+ \mapsto \mbb{R}^+$ concatenated with a server. We consider a concatenation as shown in Figure \ref{fig: system}. By convention we assume $\sigma(0) = 0$ and $\beta(t) \leq \sigma(t), \forall t \in \mbb{R}^+$, meaning that the buffer is cleared at the beginning and the service never exceed its physical limitation. With the above definition, such greedy shaper conserves the service provided by the system due to theorem \ref{thm: shaping}.

\begin{figure}[thb]
    \centering
    \includegraphics[width=0.7\linewidth]{images/system.png}
    \caption{Shaping of departure data. A flow that has an arrival curve $\alpha$ feeds into a server with an arrival data flow $R(t)$. The server having service curve $\beta$ takes $R(t)$ and gives a departure data flow $R^*(t)$ to a shaper with shaping function $\sigma$. The shaper takes $R^*(t)$ and shape the data flow as another departure $D(t)$.}
    \label{fig: system}
\end{figure}


\begin{theorem}[Shaping conserves service \cite{ncbook2001leboudec}]
\label{thm: shaping}
Following the system shown in Figure \ref{fig: system}, we have
\begin{equation}
     D = R^* \otimes \sigma \geq \lp R \otimes \beta \rp \otimes \sigma = R \otimes \lp \beta \otimes \sigma \rp = R \otimes \beta
\end{equation}
\end{theorem}

In the following context, we model the shaping function $\sigma$ as a token-bucket curve $\gamma_{C,L}$ with transmission capacity $C$ and the packet size $L$ to capture the link capacity and packetization~\cite{bouillard2022tradeoff}.

	

\section{Review of ML-based ride-hailing planning}
\label{sec:review}
\revise{In this section, we review matching, repositioning, and joint matching and repositioning in Sec.~\ref{sec:review-matching} Sec.~\ref{sec:review-repositioning}, and Sec.~\ref{sec:review-joint}, respectively.}
In each part, we discuss the collective and the distributed strategy separately.
Fig.~\ref{fig:review-outline} gives an outline of the review.

\begin{figure*}[h]
	\centering
	\includegraphics[width=0.8\linewidth]{figs/survey-taxonomy.pdf}
	\caption{\revise{A taxonomy of the ride-hailing planning literature. %is summarized.
	In each category, we discuss three works as representative examples.}}
	\label{fig:review-outline}
\end{figure*}

\subsection{Matching}
\label{sec:review-matching}

\subsubsection{Collective Matching}
\revise{RL is a promising technique for solving the matching problem.
Chen et al.~\cite{chen2020order} propose an RL-based solution in which
a deep evaluation network, which is a plain feed-forward neural network, is used to calculate a score for each pair of driver and rider based on the predicted detour distance, vehicle's seat utilization rate, and profit achieved if they get matched.
For each new ride request, the vehicle with the highest score will be assigned to serve the rider.
When the trip of the ride request is finished, the observed reward, i.e., the sum of the increased profit of the driver and, if any, the reduced cost of the rider through sharing the ride with others, is used to guide the learning process of the deep evaluation network.
Agussurja et al.~\cite{agussurja2019state} formulate the matching problem as a two-stage planning process.
In the first stage, ride requests to be scheduled are selected from all the unserved ones, the problem of which is modeled as a Markov Decision Process.
An approximated value iteration algorithm is used to learn the value function for the matching actions.
In the second stage, the final matching decision is made between the selected ride requests and all vehicles based on the learned value function.
\revise{Kullman et al.~\cite{kullman2022dynamic} apply deep RL to develop matching policies whose decisions leverage the Q-value approximations learned by deep neural networks.}
Multi-hop ride-hailing can improve the efficiency of a ride-hailing system.
To find the transfer points for each transferring trip in the multi-hop ride-hailing service, Xu et al.~\cite{xu2020highly} use a multi-layer feed-forward network to predict the reachable areas of vehicles, based on which the search space of possible vehicle pairs and transfer points for transferring riders is pruned.
In this way, the transfer points searching process can be more efficient.
	Wang et al.~\cite{wang2023optimization} also consider the scenario where riders are allowed to transfer between vehicles.
	They leverage RL to learn a policy that estimates the values of all the vehicles, which are then used to compute the optimal matching decisions by integer-linear programming.
The lengths of the time-intervals between the matching decisions can have critical impact in the matching outcomes.
	Specifically, the efficiency of matching may be improved substantially if the matching is delayed by adaptively adjusting the matching time-intervals according to the real-time situation of the riders and drivers.
	Wang et al.~\cite{wang2019adaptive} find that, if riders are willing to wait for a certain amount of time even if there are available vehicles that can serve them right away, the ride-hailing system can achieve better results, for example, in terms of the total vehicle miles traveled.
	In their solution, 
	 they propose to use an RL policy to decide for each rider, at each time step, whether to conduct matching for her/him, or %leave her/him alone and
	wait for the next time step. 
	Similarly, Qin et al.~\cite{qin2021optimizing} leverage RL in solving the ride-hailing matching problem with dynamic matching time-intervals.}

\revise{Clustering techniques are frequently used in ride-hailing planning.
Hong et al.~\cite{hong2017commuter} propose to use a density-based clustering algorithm, specifically DBSCAN \cite{parimala2011survey}, to identify riders that share similar itineraries based on their historical traveling trajectories. 
To alleviate the computational overhead caused by the large number of distance queries in the matching process, Zhang et al.~\cite{zheng2018order} propose a new clustering algorithm that groups the geographical locations in the road network into different clusters.
Then, the distance between any two nodes is approximated by the distance between the centers of the clusters they belong to. 
Shen et al.~\cite{shen2019roo} propose a spatial-temporal distance metric that measures the similarity of each pair of ride requests.
The ride requests are grouped by a clustering process based on the proposed distance metric.
Then, shared-rides are computed within each group of ride requests.
Another clustering algorithm is proposed by \cite{trasarti2011mining} to extract the mobility profiles from riders' and drivers' historical itineraries.
The matching between riders and drivers is determined based on the similarities between their profiles.}


\revise{An increasing number of collective matching solutions leverage various other ML techniques in planning.
Most of them take social factors of drivers and riders into consideration \cite{mitropoulos2021systematic}.
To mitigate the social barriers in the ride-hailing process, especially in shared-rides, Yatnalkar et al.~\cite{yatnalkar2020enhanced} and Narman et al.~\cite{narman2021enhanced} use Support Vector Machine (SVM) to predict the user social types, e.g., chatty, safety, or punctuality, based on their registered user characteristics.
Riders with similar social characteristics %are more likely
would be more willing to share a trip.
%on their closest available vehicle.
Levinger et al.~\cite{levinger2020human} use a feed-forward neural network to predict rider satisfaction levels according to their profile and trip information.
They proposed a stochastic algorithm to compute the matching decision with rider satisfaction level maximization as the objective.
Montazery and Wilson \cite{montazery2016learning, montazery2018new} propose to take into account the user preference in evaluating the weight (benefit) of the matching between each pair of rider and driver, which is given by their proposed support vector machine-based score function.
With the value calculated, the final matching can be obtained by solving an optimization problem in which the sum of the weights of those matched pairs is maximized.
Tang et al.~\cite{tang2020efficient} model various types of information (e.g., driver, rider, travel time, and activity) and their relationships within a ride-hailing system using a Heterogeneous Information Network (HIN) \cite{sun2012mining}.
Each driver or rider is projected to a multi-dimensional embedding (vector) using the skip-gram model \cite{mikolov2013efficient}.
Moreover, the skip-gram is conducted on node sequences obtained by meta path-based random walks originating from the corresponding node within the HIN \cite{dong2017metapath2vec}.
The cosine similarity between the embeddings of each driver-rider pair is then used to identify possible matching.
Zhang et al.~\cite{zhang2017taxi} consider a scenario where each rider is assigned to multiple drivers (to improve the order answer rate), and riders are free from having to enter the details of destinations (to improve the user experience).
They first leverage historical data to model the probability distribution of destinations of each rider based on his/her departure time and location with Bayesian rules, which is followed by predicting the acceptance probability between the rider and available drivers with logistic regression \cite{friedman2001elements} and gradient boosted decision tree \cite{mason1999boosting}.
They propose a hill climbing-based algorithm to solve the matching problem, which is formulated as an NP-hard combinatorial optimization with maximizing the success rate of matching as the objective.
Schleibaum and M{\"u}ller \cite{schleibaum2020human} advocate taking the determinants of user satisfaction and explainable matching decisions into consideration.
One of their future studies is to find out whether increasing the explainability can improve user satisfaction level or not.}

\revise{It is worth mentioning that many ML-based collective matching strategies take advantage of the Kuhn-Munkres (KM) bipartite matching algorithm as a component of their decision-making pipelines \cite{jonker1986improving}.
Drivers and riders are usually regarded as the two sets of vertices in the target bipartite graph.
To guide the matching between ride requests and ride offers,  Guo et al.~\cite{guo2020spatiotemporal} propose spatial-temporal Thermo, which is used to reflect the demand density of different places and times.
They use Random Forest Regression \cite{breiman2001random} to map multiple features of spatial, temporal, and meteorological dimensions to Thermo.
The weight of each pair of driver and rider in the bipartite graph is estimated by Thermo.
A KM algorithm is then used to calculate the final matching decisions according to the constructed bipartite graph.
Similarly, Xu et al.~\cite{xu2018large} derive their matching decisions using the KM algorithm.
In contrast to \cite{guo2020spatiotemporal}, Xu et al.~\cite{xu2018large} leverage a policy evaluation algorithm to learn a value function which maps each pair of driver and rider to a score.
The KM algorithm calculates the final matching between drivers and riders based on the scores.
Guo and Xu \cite{guo2020deep} also conduct the matching planning using the KM algorithm.
The weight between each pair of driver and rider is obtained from a value function learned by a convolutional neural network-based Double Q-learning (Double DQN) algorithm \cite{van2016deep}.}

\subsubsection{Distributed Matching}
\revise{RL is also a powerful technique for distributed matching \cite{sutton1999reinforcement}.
Gu{\'e}riau and Dusparic \cite{gueriau2018samod} use the Q-learning algorithm to train a policy for each agent (driver) to choose the pickup or rebalancing action based on the environment state, including the status of itself and current distribution of supply and demand.
If pickup action is chosen, then the vehicle will go and pick up the nearest rider.
In their follow-up work \cite{gueriau2020shared}, they extend the method to consider traffic congestion when agents are making decisions.
Wang et al.~\cite{wang2018deep} propose to use the DQN \cite{mnih2015human}, in which a deep neural network is employed to estimate the state-action value function from a single driver's perspective.
Many methods of distributed matching allow the decisions to be determined individually while the matching policy is trained collectively.
For example, De Lima et al.~\cite{de2020efficient} follow the QMIX framework proposed in \cite{rashid2018qmix}, in which the coordinated planning policies are trained by learning a joint action-value function for multiple vehicles and riders aiming at optimizing a global objective.
In the execution process, the matching decision of each vehicle is made in a distributed  manner following its own component in the learned action-value function.
By ensuring the monotonicity of the relationship between the global action-value and the action-value of each passenger, the objectives of distributed planning decisions are ensured to coincide with the centralized decisions during the training process.
Similar to \cite{de2020efficient}, Li et al.~\cite{li2019efficient} adopt the framework where the matching policy is trained in a centralized manner and executed in a distributed manner.
Specifically, they adopt the actor-critic RL framework, where actor and critic are two different networks used to decide and evaluate the action for each driver, respectively.
The coordination among drivers in the matching policy is enabled by the critic network.
It adopts the mean field approximation to model the interactions of drivers by calculating an average on the actions taken by their neighborhoods, which is then considered in the process of evaluating each driver's action.
\revise{In \cite{zhou2019multi}, another centralized training process is proposed, in which a Kullback–Leibler divergence optimization is used to balance the supply and demand and to enable coordination among the vehicles.}
In the execution phase, each driver chooses an action based on their own action-value functions.}

\revise{Some distributed matching strategies leverage other ML techniques.
They mostly determine the matching decisions based on the similarities between the riders and drivers in ride-hailing.
For example, Bicocchi and Mamei \cite{bicocchi2014investigating} use the bag-of-words model to summarize users' frequently visited places as vector representations, which are then fed to the Latent Dirichlet Allocation (LDA) \cite{blei2003latent} model to identify their patterns of daily travel routine behaviors.
Given a rider or a driver, his/her potential participants of shared-rides can be found by calculating the similarities between his/her daily travel routine and those of the other riders and drivers.
Lasmar et al.~\cite{lasmar2019rsrs} propose to leverage a multi-layer Perceptron model to learn user preferences based on their responses to the questionnaires.
For each rider, a ranking list of potential partners for shared-rides is generated according to the similarities between the predicted preferences of her/him and other riders.}

    
\subsection{Repositioning}
\label{sec:review-repositioning}
\subsubsection{Collective Repositioning}
\revise{Some collective repositioning methods leverage RL techniques.
Ride-hailing repositioning for electric vehicles is studied in \cite{liang2020mobility, tang2020online}, in which the state of charge of the electric vehicles is an important factor to be considered.
Liang et al.~\cite{liang2020mobility} develop a solution method utilizing deep RL combined with binary linear programming to obtain a regional joint planning policy for electric vehicles with their state of charge considered.
Using binary linear programming, each vehicle repositioning action is modeled as a binary decision variable, and its weight in the objective is obtained by the value function learned by the policy iteration method.
Similarly, Tang et al.~\cite{tang2020online} also combine RL with combinatorial optimization, in which the RL learned policy is used to advise decision making in the optimization step.
Liang et al.~\cite{liang2021integrated} adopt temporal-difference (TD) learning to obtain action-value function.
Different from \cite{de2020efficient}, the settings in \cite{liang2021integrated} do not allow factorization of the joint action-value function into individual ones while guaranteeing global maximization.
Thus, they formulate two linear programming instances to collectively find the decisions for the vehicles.
To improve the stability of the training process in RL, Fluri et al.~\cite{fluri2019learning} propose a cascading multi-level learning model.
In this model, the area concerned is split in halves as the number of levels of learning increases.
The policy training process proceeds in a top-down manner, i.e., from less to more fine-grained area partitioning. 
The motivation behind is that the policy trained from a coarse level can serve as guidance to the finer levels, which avoids the instability caused by directly training a policy with a large state size (w.r.t. the number of regions).
Fluri et al.~\cite{fluri2019learning} propose to leverage the Lloyd K-means algorithm \cite{lloyd1982least} to partition the area concerned into multiple smaller regions.
Deng et al.~\cite{deng2020multi} leverage the Proximal Policy Optimization algorithm (PPO) \cite{schulman2017proximal} to learn the joint repositioning policy for vehicles, in which the value- and policy-function are approximated by neural networks.
Shi et al.~\cite{shi2019optimal} use Deep Deterministic Policy Gradient (DDPG) \cite{silver2014deterministic} to learn the grid-based multiple vehicles repositioning policy with the objective of total profits maximization.
\revise{In \cite{shou2020reward}, a mean-filed multi-agent RL approach is leveraged to collectively relocate the vehicles in ride-hailing.}}


%In collective repositioning, 
\revise{Some other collective repositioning solutions leverage various ML techniques to predict future information of a ride-hailing system, which plays an important role in guiding the platforms to make better repositioning decisions \cite{chen2022h}.
Riley et al.~\cite{riley2020real} leverage Vector autoregression to forecast the future demand from region to region.
The predicted demand and current system status are then fed into two mixed-integer programming instances to find the desired distribution of vehicles and the assignment of vehicles to regions, respectively.
Iglesias et al.~\cite{iglesias2018data} use a Long Short-Term Memory (LSTM) neural network to predict the future ride requests for each pair of origin and destination within a certain time period.
The predicted information is then used as input to their proposed mixed-integer linear programming instance, which is solved to find the optimal rebalancing actions.
Xu et al.~\cite{xu2018taxi} use two LSTM-based and Mixture Density Network (MDN)-based models to predict the distributions of origins and destinations of future requests, respectively.
With a prediction on the distributions, the repositioning decisions are then obtained by solving a mixed-integer programming problem with total idle driving distance minimization as the objective.
Cheng et al.~\cite{cheng18taxis} leverage a multilevel logistic regression model to predict the likelihood of ride requests occurring at different times and places.
The online repositioning planning decisions of drivers are obtained by leveraging a centralized multi-period stochastic optimization model with both the real-time and predicted demand considered.
Li et al.~\cite{li2020data} and Gao et al.~\cite{gao2020learning} formulate the repositioning task as a two-stage stochastic programming problem.
The source of the stochasticity is the underlying uncertainty of the future demands, the probability distribution of which is obtained by kernel density estimation and a deep learning model combining the LSTM and MDN in \cite{li2020data} and \cite{gao2020learning}, respectively.
Pouls et al.~\cite{pouls2020idle} propose a forecast-driven repositioning solution framework, the core of which is a mixed-integer programming problem with the demand predictions as inputs.
Moreover, it is solved by an off-the-shelf solver called Gurobi \cite{gurobi}.
Note that, in practice, not all planning decisions can be successfully executed by the drivers at the end.
Xu et al.~\cite{xu2020recommender} take the first step to predict the failure possibility of repositioning tasks in the decision-making process, including situations where drivers disobey the planning or end up being unmatched for an unexpectedly long time even though they follow the repositioning planning decisions accordingly.
In the latter case, drivers will be compensated.
The failure rate of each repositioning task is predicted by XGBoost \cite{chen2016xgboost} with both driver- and environment-related features as inputs.
\revise{The problem of multi-vehicle collaboration optimization aiming at maximizing the platform's profit is converted into a minimum cost flow problem, which is solved by an off-the-shelf method called GNU Linear Programming Kit (GLPK) \cite{makhorin2008glpk}. }}


\subsubsection{Distributed Repositioning}
Geographical regions or grids (i.e., abstracts of individual locations) are usually used to model the road networks in the problem of ride-hailing repositioning.
Different from most of the repositioning methods (e.g., \cite{lin2018efficient, riley2020real, ke2019optimizing, li2019efficient, zhou2019multi}) in which the region of interest is divided into predefined and static geographic zones, Castagna et al.~\cite{castagna2020demand, castagna2021demand} leverage the Expectation-Maximization clustering algorithm to derive zones for rebalancing vehicles in an online manner.
They leverage the Proximal Policy Optimization algorithm (PPO) \cite{schulman2017proximal} to train a policy for each vehicle to decide whether to make a pick-up, drop-off, or repositioning action.
Specifically, similar to \cite{tang2021value}, the repositioning destination is also sampled from a probability distribution over all potential positions, which is determined by the number of unserved requests.
Different from \cite{castagna2020demand, castagna2021demand}, Verma et al.~\cite{verma2017augmenting} propose an iterative method to dynamically split the zones based on their expected revenue (Q-values).
The iterative splitting process does not terminate until the historical data is exhausted for the Q-values learning.
\revise{Different from most of the works that model the drivers as agents, Jin et al.~\cite{jin2019coride} regard each geographical region as an agent.}
By hierarchically partitioning the target areas into regions with different granularities, they perform hierarchical RL where the multi-head attention mechanism is used to capture the impacts among the neighboring agents.
Guo et al.~\cite{guo2021multi} try various methods (e.g., Support Vector Regression, Random Forest Regression, and k-Nearest Neighbors regression) to predict future demand density, which is then used to evaluate each region for their spatial-temporal value.
Each available vehicle chooses to stay still or relocate to a neighbor region in a probabilistic manner based on their spatial-temporal values, which can help avoid over-saturation of supply.
In \cite{provoostdemandprop}, the region of interest is represented as a graph.
They build two neural networks to predict the demand on vertices and the passenger flows on edges, respectively.
The proposed repositioning algorithm aims at satisfying the demand on edges in the decreasing order with the nearest vehicles found by backward traversing.

\revise{However, in spite of the various grid-based methods as discussed in most of the related works mentioned above, e.g., \cite{lin2018efficient, guo2021multi}, Jiao et al.~\cite{jiao20deep, jiao2021real} argue that grid-based repositioning policies are not satisfactory in practice because of the excessively-simplified and overlooked non-stationarity in the environment caused by the dynamic environment and the large number of vehicles when coarse-grained region-wise decisions are considered.}
They put forward the process of carrying out repositioning %algorithms
in industrial production by combining offline learning, i.e., batch RL, and online planning stages, i.e., decision-time planning \cite{sutton1999reinforcement}.
To counter the issues of coarse-grained decisions, Kim and Kim \cite{kim2020optimizing} uses a graph to model the road networks which is more realistic.
They build a Graph Neural Network to predict the future demands.
The repositioning destination of each driver is decided greedily based on a function of the predicted demand, the number of excessive vehicles, and the distance information to each candidate position.

\revise{Some other works also spend special effort on tackling the non-stationarity.
With the observation that the actions of drivers are independent (based on self interests), Chaudhari et al.~\cite{chaudhari2020learn} propose a vanilla RL framework where each driver, based on a probabilistic value denoting the extent to which coordination is needed, stochastically chooses to perform an action guided by the independent or coordinated policy.
Note that, although vehicles execute repositioning decisions sequentially in this solution framework, coordination in the latter policy is explicitly considered by solving a minimum cost flow problem for the optimal rebalancing flow of vehicles among all the regions (which is similar to \cite{xu2020recommender}).
In addition, the independence between different repositioning policies learned by the drivers concurrently also contributes to the non-stationarity of the environment.
In this regard, Verma et al.~\cite{verma2019entropy} propose a method for each driver to learn the information of other vehicles in order to make a better planning decision.}
%Concretely,
The principle of maximum entropy \cite{jaynes1957information} is leveraged to improve the predictability of the distribution of drivers even with only limited knowledge available, e.g., the local density of supply.
To tackle the non-stationary challenge in online ride-hailing as well as the catastrophic forgetting of RL \cite{kemker2018measuring}, Haliem et al.~\cite{haliem2020adapool, haliem2021adapool} propose to learn multiple repositioning policies to deal with different contexts of environments (e.g., peak/non-peak hours and weekends/weekdays).
When changes in the distribution of experiences are identified by their proposed change point detection algorithm, switching among those different policies is enabled so as to enhance adaptability to the dynamic environment.
Lei et al.~\cite{lei2019optimal} define the concept of stochastic relocation matrix.
The element in the $i$-th row and the $j$-th column within the matrix represents the probability that an empty vehicle located in the $i$-th region should relocate to the $j$-th region.
\revise{To circumvent the curse of dimensionality, they leverage low-rank approximation to project the original matrix onto a low-dimensional vector.}
They propose a deep convolution-LSTM model to learn how to predict the approximation vector based on the system status.
To alleviate the instability of the state-value function approximator caused by the large scale of its states, Tang et al.~\cite{tang2019deep} propose to bound its outputs by regularizing its worst-case variation w.r.t. %any
changes in its inputs (i.e., states).
Transfer learning proposed in \cite{wang2018deep} is applied to increase the adaptability of the trained model across different cities.

\revise{Besides the traditional ML techniques discussed above, RL, being another well-known technique for decision making in non-stationary environments, has been a key technology in distributed repositioning \cite{khetarpal2022towards,xie2021deep,mao2021near}.}
\revise{Liu et al.~\cite{liu2022deep} propose a single-agent deep RL approach which relocates vacant vehicles to regions with a large demand gap in advance.}
Nguyen et al.~\cite{nguyen2018policy} propose to use the RL framework to train a homogeneous repositioning policy for all agents, i.e., vehicles.
\revise{The policy is trained in a centralized manner with collective behaviors of drivers considered while executing in a distributed manner.}
He and Shin \cite{he2019spatio} leverage Double DQN with their proposed spatial-temporal capsule-based neural network as the state-action value approximator.
The inputs of the network proposed include the location of the vehicle to be relocated, distribution of other vehicles and riders, ride preferences, and some external factors that have impacts on supply and demand, e.g., weather conditions and holiday events.
With all those information processed, the estimated value for each candidate position given the current state of the target vehicle is obtained, and the final decision can be decided in a probabilistic manner.
A more elaborate analysis is presented in their follow-up study \cite{he2020spatio}.
Yu et al.~\cite{yu2019markov} formulate the single-vehicle repositioning planning problem as a Markov Decision Process.
They propose to leverage parallelized matrix operations to re-formulate the Bellman equation \cite{sutton1999reinforcement}, thus reducing the computational complexity in finding optimal planning policy.
Multi-hop ride-hailing repositioning is considered in \cite{singh2019reinforcement, singh2021distributed}.
Similar to \cite{al2019deeppool}, they predict the number of vehicles in each region for certain time slots ahead of time using an estimated time of arrival (ETA) model.
Double DQN is adopted for each vehicle to choose the best neighbor region to move forward based on the current status of all the vehicles and the predicted demand and supply.
    
\subsection{Joint Matching and Repositioning}
\label{sec:review-joint}
In this part, we review methods that jointly optimize matching and repositioning with ML techniques. 
Note that all of them belong to the category of distributed planning. The research works in this part leverage RL to guide the decision making process.
Different from the review given by Qin et al.~\cite{qin2021reinforcement}, we focus on the works that jointly decide matching and repositioning.

Haliem et al.~\cite{haliem2020distributed-a, haliem2021distributed} propose to consider both the matching and repositioning in the ride-hailing planning process.
In their ride-hailing systems, each vehicle conduct initial matching by greedily searching the nearest requests, after which an insertion-based method is used to finalize the potential request list.
Then each driver, based on the value function learned by the DQN, weighs the requests in the final list.
The riders who receive those proposed ride offers can decide whether to accept the offers and join the trips where shared-rides are allowed.
The trips can be solo-ride or shared-ride.
Drivers are repositioned in parallel with the matching process.
Each driver takes actions indicated by his/her trained RL agent, i.e., the decision-making policy, independently.
Their proposed solution framework learns an optimal policy for each driver as opposed to those RL-based methods with collective planning scheme where a central policy is used, e.g., \cite{oda2018movi}.
Note that, in some works, although each driver makes decisions independently (e.g., \cite{haliem2020distributed-a, haliem2021distributed}), all drivers share one trained policy (e.g., \cite{manchella2020passgoodpool, manchella2021flexpool}).
Manchella et al.~\cite{manchella2020passgoodpool, manchella2021flexpool} propose to collectively optimize the system objectives, e.g., minimizing the waiting times and routing times.
Nevertheless, they allow distributed inference at the level of individual drivers. 
Their proposed model can be used by each vehicle independently.
It helps decrease computational costs associated with the growth of distributed systems. 
Specifically, they utilize a Double DQN with the experience relay mechanism.
Their model learns a probabilistic dependence between drivers' actions and the reward function.
The trained policy indicates a destination for each driver if s/he is not matched with any rider according to their proposed heuristic matching algorithm. 
Similar to \cite{xu2020highly, singh2019reinforcement, singh2021distributed}, multi-hop transit is enabled in their solutions.
Wang et al.~\cite{wang2018deep} model the matching and repositioning problems as a Markov Decision Process and propose learning solutions based on DQNs to optimize the trained policy for the drivers.
\revise{Their solution uses a temporal and spatial expanded action search strategy to accommodate the scenarios where there is only sparse training data, e.g., certain remote regions in the middle of the night.}
Besides, to increase the learning adaptability and
efficiency, they propose to use a transfer learning method to leverage the knowledge across both spatial and temporal spaces.

Besides \cite{haliem2020distributed-a, haliem2021distributed, manchella2020passgoodpool, manchella2021flexpool, wang2018deep},
DQN is used in other works as well, e.g., \cite{al2019deeppool, guo2022deep, tang2021value, li2020balancing}.
In \cite{al2019deeppool}, each vehicle decides its action by learning the impact of its action on the reward using a DQN model without coordinating with other vehicles.
In \cite{guo2022deep}, the vehicle repositioning procedure is formulated as a Markov Decision Process.
By sampling the future riders based on the historical probability distribution, the proactive relocation of vehicles is realized via a deep RL framework, which is composed of a Convolutional Neural Network and a Double DQN module. \revise{Then a request-vehicle assignment scheme is presented based on the value function attained from the vehicle repositioning process.}
\revise{Similarly, Tang et al.~\cite{tang2021value} propose a planning framework for tackling both the matching and repositioning tasks, the core of which is a unified value function which is trained offline using abundant historical data and is updated during the online phase.}
With the value function learned, the matching problem is then solved by the method proposed in \cite{xu2018large}, while the reposition destination of each idle vehicle is determined in a probabilistic manner following the distribution given by the discounted long-term values of all the candidate positions.
Li and Allan \cite{li2020balancing} also leverage a global value function for both the tasks of matching and repositioning, which is learned by the value iteration algorithm with historical data of ride requests.



	\begin{table*}[tp]
\centering
{\color{black}\begin{threeparttable}
\caption{\revise{Summary of open-source trip related data sets.}}
\begin{tabular}{@{}C{2.2cm}C{2.4cm}cccm{3.8cm}@{}}
\toprule
\textbf{Source}& \textbf{City} &\textbf{Type} & \textbf{\#~of Records} & \textbf{Time} & \multicolumn{1}{>{\centering\arraybackslash}m{3.8cm}}{\textbf{Data Features}}\\ \midrule
\midrule
Uber Pickups \cite{p668-gy46-22} & New York, United States& Pickups &20M & \begin{tabular}[c]{@{}c@{}}April -- September 2014,\\January -- June 2015 \end{tabular} & Timestamp and location.\\ \midrule
NYC TLC \cite{tlctrip} & New York, United States& Ride request &500M&January 2009 -- July 2021 &Passenger count, start time, end time, origin, destination, distance, fee, etc.\\ \midrule
DiDi GAIA \cite{gaiadata}& Haikou and Chengdu, China& Ride requests  &18M&\begin{tabular}[c]{@{}c@{}}November 2016,\\May -- October 2017\end{tabular}&Trip ID, fee, start time, end time, origin, destination, etc.\\ \midrule
Chicago Data Portal \cite{chicagoalll} & Chicago, United States & Ride requests &199M &January 2013 -- December 2021&Trip ID, taxi ID, start time, end time, fee, origin, destination, etc.\\ \midrule
T-drive \cite{tdrive, yuan2010t, yuan2011driving} & Beijing, China& Trajectories &15M&February 2008&Taxi ID, location, and timestamp.\\ \midrule
GeoLife \cite{geolifetraj, zheng2008learning, zheng2008understanding, zheng2010understanding} & Beijing, China& Trajectories &18,670&April 2007 -- August 2012 &Timestamp, location, transportation mode, etc. \\ \midrule
Beijing Taxi Trajectories \cite{lian2018one} & Beijing, China & Trajectories &129M&May 2019& Taxi ID, location, timestamp, speed, occupancy indicator, etc.\\ \midrule
DiDi GAIA \cite{gaiadata} & Chengdu and Xi'an, China &Trajectories &-\tnote{*}&-&\multicolumn{1}{>{\centering\arraybackslash}m{3.8cm}}{\textbf{-}}\\ \midrule
ECML PKDD 2015 \cite{portokaggle} & Porto, Portugal  & Trajectories  & 2M&July 2013 -- June 2014 &Trip ID, taxi ID, the sequence of locations of the trajectory, etc.\\ \midrule
CRAWDAD EPFL \cite{c7j010-22}  & San Francisco, United States & Trajectories  &11M&May -- June 2008& Taxi ID, location, timestamp, and occupancy indicator.\\\midrule
CRAWDAD Roma \cite{c7qc7m-22}  & Roma, Italy & Trajectories  &22M&February -- March 2014& Taxi ID, location, and timestamp.\\\midrule
Jeju Vehicular Trajectories \cite{y8vk-wj40-22}  & Jeju, South Korea & Trajectories  &8M& - & Vehicle ID, location, speed, lane, etc.\\\midrule
Grab-Posisi \cite{grabsource, huang2019grab}  &Singapore and Jakarta, Indonesia& Trajectories  &84,000&April 2019&Trajectory ID, location, timestamp, speed, etc.\\\midrule
Shanghai Taxi Trajectories \cite{2877-mk46-19, liu2020optimization}  &Shanghai, China& Trajectories  &61M&April 1, 2018&Taxi ID, location, timestamp, speed, occupancy indicator, driving status, etc.\\\midrule
Foursqure \cite{foursquare} & Global & Check-ins &33M&April 2012 -- September 2013 & Venue ID, timestamp, location, etc.\\ \midrule
Brightkite \cite{brightkite} & - & Check-ins &4M&April 2008 -- October 2010&User ID, timestamp, location, etc.\\ \midrule
Gowalla \cite{gowalla} & -  & Check-ins &6M&February 2009 -- October 2010&User ID, timestamp, location, etc.\\ \midrule
LTA of Singapore \cite{singaporedata} & Singapore & \begin{tabular}[c]{@{}c@{}}Real-time\\locations\end{tabular}  &-&-&Location. \\\midrule
Uber \cite{traveltimedata} &  Global &Travel times  &-&-&Origin, destination, travel time, etc.\\\bottomrule
\end{tabular}
\label{table:data}
\begin{tablenotes}
\item[*] \revise{``-'' indicates that the corresponding information is not attainable.}
\end{tablenotes}
\end{threeparttable}}
\end{table*}

\begin{table*}[t]
\centering
{\color{black}\begin{threeparttable}
\caption{Summary of the major simulators.}
\begin{tabular}{@{}cccm{9.6cm}@{}}
\toprule
\textbf{Name} & \textbf{Open-Source} & \begin{tabular}[c]{@{}c@{}}\textbf{Sample Applications}\\\textbf{in Ride-hailing} \end{tabular}& \multicolumn{1}{>{\centering\arraybackslash}m{9.6cm}}{\textbf{Description}} \\ \midrule
\midrule
 DiDi \cite{didisimulation} & \xmark &\cite{tang2021value}&An online simulation platform to evaluate matching and repositioning algorithms.  Evaluation is run with DiDi's real-world data.\\ \midrule
 AMoDeus \cite{ruch2018amodeus} & \cmark&\cite{ruch2020quantifying}&A tool that uses an agent-based transportation simulation framework to simulate arbitrarily configured mobility-on-demand systems with static/dynamic demand. It includes standard benchmark algorithms and a graphical user interface.\\ \midrule
 AMoD2 \cite{amod2} & \cmark&\cite{li2021optimal}&A high-capacity ride-sharing simulator. It uses map data and taxi data from Manhattan. Three matching algorithms and one simple rebalancing algorithm are implemented. \\ \midrule
 Mod-abm \cite{abm-1.0, abm-2.0} & \cmark&\cite{wen2017rebalancing}&A platform for simulation of large-scale mobility-on-demand operations. It supports city-level systems in any urban setting.\\ \midrule
 MATSim \cite{horni2016multi} & \cmark&\cite{tsao2019model}&An open-source framework for implementing large-scale agent-based transport simulations. It consists of several modules which can be combined or used in stand-alone mode, for demand-modeling, traffic flow simulation, re-planning; and a controller to iteratively run simulations, and methods to analyze outputs generated by the modules.\\ \midrule
 SUMO \cite{lopez2018microscopic} & \cmark&\cite{castagna2021multi,zhu2021shared}&A microscopic and space-continuous traffic simulation platform that is suitable for the generation, evaluation, and validation of traffic scenarios of real-world size. It supports road network customization and demand modeling.\\ \midrule
 CityFlow \cite{zhang2019cityflow} & \cmark&-\tnote{*}&A multi-agent RL environment for large scale city traffic scenario. It supports flexible definitions for road network and traffic flow. It provides faster simulation than SUMO.\\ \midrule
 STaRS \cite{ota2016stars} & \xmark &-&A simulation framework for analyzing diverse ride-sharing scenarios, considering the platforms' needs and constraints. Its real-time trip assignments utilize a linear optimization algorithm, efficient indexing, and parallelization for scalability.\\ \midrule
 STaRS+ \cite{mounesan2021fleet} & \xmark &-&A simulation framework based on an integer linear programming model, using heuristic optimization and a novel shortest-path caching scheme for scalability. It supports the simulation of full-city scale ride-sharing with meeting points.\\ \midrule
 UberSim \cite{khalil2022realistic} & \cmark & \cite{salman2023quantifying} &A digital twin transportation simulation model for Birmingham, Alabama, incorporating various transportation modes to analyze the impact of ride-hailing services on urban traffic. It supports policy learning through reinforcement learning.\\ \midrule
 NYC-Yellow-Taxi-V0 \cite{chaudhari2020learn} & \cmark&-&A multi-agent RL environment based on the OpenAI Gym environment \cite{openai-gym}, which offers a toolkit for developing and comparing RL-based ride-hailing fleet management algorithms.\\ \midrule
 SMART-eFlo \cite{liu2022smart} & \cmark & - &An integrated framework that combines the SUMO simulator with multi-agent Gym for reinforcement learning studies, enabling researchers to easily design traffic scenarios and implement RL algorithms for electric fleet management problems\\ \midrule
\end{tabular}
\label{table:simulator}
\begin{tablenotes}
\item[*] ``-'' indicates no sample application in ride-hailing using the corresponding simulator.
\end{tablenotes}
\end{threeparttable}}
\end{table*}



\section{Resources for empirical studies}
\label{sec:resource}
Real-world data are essential in studying ML-based ride-hailing planning, e.g., to train a proposed model.
Further, in order to deploy those proposed planning strategies in practice, it is necessary to utilize a ride-hailing service (process) simulator to validate their performance with real-world data.
\revise{In this section, we present publicly available %multiple
open-source real-world data sets and several related simulators in Sec.~\ref{sec:resource-data} and Sec.~\ref{sec:resource-simulator}, respectively.}

\subsection{Data}
\label{sec:resource-data}
Historical records of ride requests are the most frequently used data in the literature.
New York City Taxi and Limousine Commission \cite{tlctrip} provides more than 11 years of records of ride requests with many data fields, e.g., the pickup and drop-off timestamps, pickup and drop-off locations, trip fare, and ride distance.
About 19 million Uber's pick-up records obtained from this data set %\cite{tlctrip}
are summarised in \cite{p668-gy46-22}.
Chicago Data Portal \cite{chicagoalll} provides a large data set consisting of trip records with various data fields recorded from 2013.
Ride request data of Haikou and Chengdu in China are publicly available in the DiDi GAIA program \cite{gaiadata}.
Besides, there are some other data sets that consist of people's check-in records, e.g., those collected from Foursquare \cite{foursquare}, Brightkite \cite{brightkite}, and Gowalla \cite{gowalla}.
They are informative in revealing ride demands as the data have recorded riders' target places they had traveled to.
In addition to the records of ride requests, some data of the supply side are also available.
\revise{Trajectory data recorded by sampling drivers' locations at a certain frequency are also commonly used in the literature, including trajectories recorded in Beijing (in three different data sets: T-drive \cite{tdrive, yuan2010t, yuan2011driving}, GeoLife \cite{geolifetraj, zheng2008learning, zheng2008understanding, zheng2010understanding}, and Beijing Taxi Trajectories \cite{lian2018one}), Chengdu \cite{gaiadata}, Xi'an \cite{gaiadata}, Porto \cite{portokaggle}, Jakarta \cite{grabsource, huang2019grab}, Singapore \cite{grabsource, huang2019grab}, San Francisco \cite{c7j010-22}, Roma \cite{c7qc7m-22}, Jeju \cite{y8vk-wj40-22}, and Shanghai \cite{2877-mk46-19, liu2020optimization}.}
The Land Transport Authority (LTA) of Singapore \cite{singaporedata} publicizes many APIs for accessing various kinds of transport-related data, including monthly statistics of taxi supply and real-time coordinates of all taxis that are currently available for hire. %offering ride service. 
\revise{Besides, the travel times among different locations in various cities across the world can be obtained in \cite{traveltimedata}.}
The data sets mentioned above are summarized in Table.~\ref{table:data}.
\revise{Note that, although abundant public data sets are available for use, none of them record those ride requests that ended up unserved (because of, for example, excessive waiting time).}
If both the served and unserved ride requests are treated in the ride-hailing simulation, the simulation results would be more realistic than if only
the served ride requests are considered. 

In addition to the trip-related data sets mentioned above, there are several important public data sources that could provide good values
to ride-hailing planning studies.
\revise{OpenStreetMap \cite{openstreetmap} captures worldwide road networks, which can be easily accessed by APIs (e.g., OSMnx) using Python \cite{boeing2017osmnx}.}
Travel times and speeds information of road networks of several well known cities have been made available in \cite{traveltimedata}.
Finally, historical data of weather conditions can be found in \cite{weatherdata}.
Note that weather conditions can be considered in ride-hailing planning
as they could affect the traffic in a major way
\cite{he2019spatio}.
    
    
    
\subsection{Simulators}
\label{sec:resource-simulator}
With the aforementioned real-world data, proposed planning strategies can be evaluated through simulators.
DiDi \cite{didisimulation} makes public an industrial-level simulation platform recently, which is used in the KDD CUP 2020 \cite{kddcup20}. 
The platform evaluates submitted matching and repositioning strategies using real-world data from the GAIA program.
Wen \cite{abm-1.0} develops an agent-based modeling platform for simulating autonomous mobility-on-demand systems, which is later upgraded to a version with better scalability and extensibility \cite{abm-2.0}.
Based on \cite{abm-2.0}, a high-capacity on-demand ride-sharing simulator is proposed in \cite{amod2} with several built-in matching and repositioning algorithms.
Ruch et al.~\cite{ruch2018amodeus} have released an open-source simulator named AMoDeus for accurate and quantitative analysis of matching and repositioning algorithms in the ride-hailing system.
Examples of its usages can be found in \cite{fluri2019learning, carron2019scalable}.
AMoDeus is built upon the open-source microscopic multi-agent transportation simulation environment MATSim \cite{horni2016multi}.
MATSim is also used to evaluate ride-hailing planning strategies, a case study of which can be found in \cite{bischoff2016simulation}.
Besides MATSim, there are similar public traffic simulation tools, including SUMO \cite{lopez2018microscopic} and CityFlow \cite{zhang2019cityflow}.
A more detailed comparison and analysis of the performance of traffic simulation tools can be found in \cite{allan2015benchmark}.
\revise{STaRS is a scalable simulation framework that takes the needs and constraints of platforms into consideration \cite{ota2016stars}. 
STaRS+ extends STaRS to support the simulation of ride-sharing with meeting points \cite{mounesan2021fleet}.}
\revise{For those RL-based ride-hailing planning methods, their policies can be trained using CityFlow \cite{zhang2019cityflow}, UberSim \cite{khalil2022realistic}, NYC-Yellow-Taxi-V0 \cite{chaudhari2020learn}, or SMART-eFlo \cite{liu2022smart}.}
The aforementioned simulators are summarized in Table.~\ref{table:simulator}.

    
	
\section{Conclusion and Future Work}
\label{sec:future}
% Whenever a new solution concept is defined, one of the most intuitive questions is how it can be related to other properties? 
We presented a practical solution to the problem of leximin optimization when only an approximate single-objective solver is available. 
The algorithm is guaranteed to terminate in polynomial time, and its approximation ratio degrades gracefully as a function of the approximation ratio of the single-objective solver.

Currently, our algorithm handles two main settings. First, when inaccuracies in the single-objective solver stem from numeric errors.
Second, when the problem is convex and satisfy several assumptions.
It may be interesting to study more settings in which the inaccuracies stem from computational hardness of the single-objective problem.
% Currently, our algorithm handles settings in which the inaccuracies in the single-objective solver stem from numeric errors.
% It may be interesting to study settings in which the inaccuracies stem from computational hardness of the single-objective problem.
%
%

% identify problems in which an appropriate approximate solver can be designed. 
In particular, to approximate the egalitarian welfare, it is common to model the problem as an integer program or as an exponential sized linear program (e.g., \cite{bansal2006santa, kawase_max-min_2020}) and then approximate the program using different techniques.
% rounding techniques or methods for convex optimization (such as the ellipsoid method).
Can these algorithms be generalized to consider the additional constraints described in Section \ref{sec:algo-short}? This will allow approximating leximin using the approach in this paper.
% In particular, in the problem of stochastic allocations (in Section \ref{sec:app}), to extend the approximation algorithm for the egalitarian welfare, we had to change some steps within.
% What if an algorithm for egalitarian welfare is provided as a black box --- could it be used to design the appropriate procedure to approximate leximin?

% In the context of fair division, this study assumes that there is an access to the true valuations of the agents involved. 
% In reality, people may lie about their valuations.
% Can our definition of approximate-leximin be related to some approximate version of truthfulness?

Another question is whether it is possible to obtain a better approximation factor for leximin, given an $(\multApprox, \additiveApprox)$-approximation algorithm for the single-objective problem.
Specifically, can an $(\multApprox, \additiveApprox)$-approximation to leximin can be obtained in polynomial time? 
If not, what would be the best possible approximation in this case?
% \erel{Mention the tightness of our results}


\iffalse % EREL: removed for the submission. To clarify later
Further, the algorithm suggested in Section \ref{sec:algo-short} tend to work very well if the single-objective optimization problems are convex, and in particular if they are linear programs. 
\erel{Why? the algorithm of \textcite{Ogryczak2004TelecommunicationsND} works even for non-convex programs.}
Can we find an algorithm that works with general approximation algorithms? For example, naive algorithms such as \emph{Next-fit}?
\fi

% \begin{itemize}
%     \item Meaning of approximately leximin in the context of other characteristics like truthfulness.
    
%     \item Solving more problems.
    
%     \item Is it possible to obtained a better approximation factor for leximin maximization in polynomial time? given that $(1-\beta)$ is the best possible for the egalitarian maximization, is it possible to obtain a $(1-\beta)$ approximately leximin optimal solution? what is the best possible approximation  in this case?
    
%     \item The algorithm works very well if the single-objective problem is convex, in particular if it is a linear programs, but can it be modify to work with general approximation algorithms? such as algorithms for makespan minimization?
% \end{itemize}
	\section{Conclusion}\label{sec:conclusion}
In this work, we focus on addressing the fundamental challenge of OOD detection tasks, which is how to fully understand the semantic discrepancy between the ID/OOD samples. We reveal that the key to success in the realistic SCOOD task is to allocate as many ID samples in the unlabeled set correctly as possible. To this end, we propose a novel uncertainty-aware optimal transport scheme that introduces class-specific energy scores as guidance for effective label assignment. Experimental results show that our method achieves better performance than previous state-of-the-art methods on SCOOD benchmarks.

\textbf{Limitations.} In addition to temperature scaling, other techniques such as feature clipping applied in ReAct~\cite{sun2021react} also enhance the performance of energy score, so how to obtain an OOD score that best fits the SCOOD task can be further explored. Moreover, a setting highly related to SCOOD has been proposed in \cite{katz2022training} and formulated as a constrained optimization problem. We will also theoretically analyze these practical OOD settings in our feature work.

% \section*{Acknowledgments}
\textbf{Acknowledgments.} 
This work is supported by National Key R\&D Program of China under Grant 2020AAA0105701, National Natural Science Foundation of China (NSFC) under Grants 61872327, Major Special Science and Technology Project of Anhui, National Natural Science Foundation of China (62033012) and Ant Group through Ant Research Intern Program.

	
% \newpage
\bibliographystyle{IEEEtran}
%\bibliography{ref}
% \newpage
 \documentclass[lettersize, journal]{IEEEtran}
\usepackage{amsthm}
\usepackage{amsmath}
\usepackage{amssymb}
\usepackage{graphicx}
\usepackage{indentfirst}
\usepackage{algorithm,algorithmic}
%\usepackage[ruled, vlined]{algorithm2e}
\usepackage[top=1.6cm, bottom=2cm, left=2cm, right=2cm]{geometry}
\usepackage{threeparttable}
\usepackage{multirow}
\usepackage[usenames]{color}
% \pagestyle{empty}
\usepackage{subfigure}
\usepackage{hyperref}

% \usepackage[numbers]{natbib}
% \newcommand{\cites}[1]{\citeauthor{#1} \cite{#1}}
%\newenvironment{sloppypar}{\par\sloppy}{\par}
\usepackage{mathtools} 
\DeclarePairedDelimiter{\ceil}{\lceil}{\rceil} 
\DeclarePairedDelimiter\floor{\lfloor}{\rfloor}
\renewcommand{\algorithmicrequire}{\textbf{Input:}} 
\renewcommand{\algorithmicensure}{\textbf{Output:}}
\usepackage[dvipsnames]{xcolor}
\newcommand{\dc}[1]{\textcolor{red}{#1}}
\newcommand{\cm}[1]{\textcolor{gray}{#1}}
\newcommand{\alg}{\text{ALG}}
\newcommand{\opt}{\text{OPT}}
\newcommand{\hy}{\widehat{y}}
\makeatletter
\newcommand*{\rom}[1]{\expandafter\@slowromancap\romannumeral #1@}
\makeatother
\usepackage{xurl}
\usepackage{booktabs}
\usepackage{tikz}
\newcommand{\mycaption}[1]{\stepcounter{figure}\raisebox{-7pt}
  {\footnotesize Fig. \thefigure.\hspace{3pt} #1}}

\providecommand{\keywords}[1]
{
%	\small	
	\textbf{\textit{Keywords---}} #1
}

\theoremstyle{definition}
\newtheorem{lemma}{Lemma}
\newtheorem{claim}{Claim}
\newtheorem{theorem}{Theorem}
\newtheorem{remark}{Remark}
\newtheorem{definition}{Definition}
\newtheorem{objective}[theorem]{Objective}
\newtheorem{problem}{Problem}

\usepackage{multirow}
\usepackage{array}
\newcommand{\PreserveBackslash}[1]{\let\temp=\\#1\let\\=\temp}
\newcolumntype{C}[1]{>{\PreserveBackslash\centering}p{#1}}
\newcolumntype{R}[1]{>{\PreserveBackslash\raggedleft}p{#1}}
\newcolumntype{L}[1]{>{\PreserveBackslash\raggedright}p{#1}}
\usepackage{amssymb}% http://ctan.org/pkg/amssymb
\usepackage{pifont}% http://ctan.org/pkg/pifont
\newcommand{\cmark}{\ding{51}}%
\newcommand{\xmark}{\ding{55}}%
% \usepackage[parfill]{parskip}
\usepackage[font=footnotesize]{caption} 
% \newcommand{\dc}[1]{\textcolor{red}{#1}}
\usepackage{lipsum}
\usepackage{threeparttable}
 \newcommand{\revise}[1]{\textcolor{black}{#1}}

\allowdisplaybreaks
\title{A Survey of Machine Learning-Based Ride-Hailing Planning}
%\author{}

\author{
Dacheng Wen,~\IEEEmembership{Student Member,~IEEE},
Yupeng Li,~\IEEEmembership{Member,~IEEE},
Francis C.M. Lau
\IEEEcompsocitemizethanks{
		\IEEEcompsocthanksitem
            Dacheng Wen is with The University of Hong Kong, Hong Kong (e-mail: wdacheng@connect.hku.hk). Work done while Dacheng Wen was under the supervision of Yupeng Li and Francis C.M. Lau.\protect
            \IEEEcompsocthanksitem Yupeng Li (corresponding author) is with Hong Kong Baptist University, Hong Kong (e-mail: ivanypli@gmail.com).\protect
            \IEEEcompsocthanksitem Francis C.M. Lau is with The University of Hong Kong, Hong Kong (email: fcmlau@cs.hku.hk).
	}
}

\begin{document}
%	\markboth{Journal of \LaTeX\ Class Files}%,~Vol.~XX, No.~X, XX~2022
%	{Shell \MakeLowercase{\textit{et al.}}: A Sample Article Using IEEEtran.cls for IEEE Journals}
	
	\maketitle
	
% 	\documentclass{scrartcl}
\usepackage{tikz,pgfplots}
\usepackage{filecontents}
\begin{document}
\pgfkeys{/pgf/number format/.cd,1000 sep={\,}}

\begin{filecontents}{data.csv}
year,count
2016,998
2015,1000
2014,900
2013,837
2012,826
2011,784
2010,801
2009,731
2008,703
2007,632
2006,629
2005,516
2004,512
2003,476
2002,444
2001,497
2000,478
1999,400
1998,393
1997,399
1996,387
\end{filecontents}

\begin{tikzpicture}
\begin{axis}[ xlabel=Count, ylabel=Year]


\addplot[color=blue,mark=*] table[x=year, y=count, col sep=comma]{data.csv};

 \end{axis} 
 \end{tikzpicture}
\end{document}
	

Over the past few years, there has been a significant amount of research focused on studying the ReLU activation function, with the aim of achieving neural network convergence through over-parametrization. However, recent developments in the field of Large Language Models (LLMs) have sparked interest in the use of exponential activation functions, specifically in the attention mechanism.

Mathematically, we define the neural function $F: \R^{d \times m} \times  \mathbb{R}^d \rightarrow \mathbb{R}$ using an exponential activation function. Given a set of data points with labels $\{(x_1, y_1), (x_2, y_2), \dots, (x_n, y_n)\} \subset \mathbb{R}^d \times \mathbb{R}$ where $n$ denotes the number of the data. Here $F(W(t),x)$ can be expressed as $F(W(t),x) := \sum_{r=1}^m a_r \exp(\langle w_r, x \rangle)$, where $m$ represents the number of neurons, and $w_r(t)$ are weights at time $t$. It's standard in literature that $a_r$ are the fixed weights and it's never changed during the training. We initialize the weights $W(0) \in \mathbb{R}^{d \times m}$ with random Gaussian distributions, such that $w_r(0) \sim \mathcal{N}(0, I_d)$ and initialize $a_r$ from random sign distribution for each $r \in [m]$.

Using the gradient descent algorithm, we can find a weight $W(T)$ such that $\| F(W(T), X) - y \|_2 \leq \epsilon$ holds with probability $1-\delta$, where $\epsilon \in (0,0.1)$ and $m = \Omega(n^{2+o(1)}\log(n/\delta))$. To optimize the over-parametrization bound $m$, we employ several tight analysis techniques from previous studies [Song and Yang arXiv 2019, Munteanu, Omlor, Song and Woodruff ICML 2022]. 

 

	
	\begin{IEEEkeywords}
		Ride-hailing, machine learning, matching, repositioning, collective planning, distributed planning.
		%Article submission, IEEE, IEEEtran, journal, \LaTeX, paper, template, typesetting.
	\end{IEEEkeywords}
	
	\section{Introduction}
\label{sec:introduction}
% \begin{itemize}
%     % Diffusion of FL
%     \item {\st{Diffusion of FL}}
%     % Security threats to FL
%     \item {\st{Security threats to FL with particular focus on model poisoning}}
%     % Limitations of existing countermeasures
%     \item {\st{Current countermeasures (e.g., KRUM) and their limitations}}
%     % Proposed method and its advantages
%     \item {\st{Intuitive description of the proposed method and its difference (i.e., advantages) w.r.t. state of the art}}
%     % Main contributions
%     \item {\st{Summary of the main contributions of this work}}
%     % Paper's structure and organization
%     \item {\st{Paper's structure and organization}}
% \end{itemize}

% Diffusion of FL
Recently, {\em federated learning} (FL) has emerged as the leading paradigm for training distributed, large-scale, and privacy-preserving machine learning (ML) systems~\cite{mcmahan2017googleai,mcmahan2017aistats}. 
The core idea of FL is to allow multiple edge clients to collaboratively train a shared, global model without disclosing their local private training data.
%Specifically, an FL system consists of a central server and many edge clients; 
A typical FL round involves the following steps: {\em(i)} the server randomly picks some clients and sends them the current, global model; {\em(ii)} each selected client locally trains its model with its own private data; then, it sends the resulting local model to the server;\footnote{Whenever we refer to global/local model, we mean global/local model {\em parameters}.} {\em(iii)} the server updates the global model by computing an \emph{aggregation function}, usually the average (FedAvg), on the local models received from clients.
% \begin{enumerate}
%     \item[{\em(i)}] the server sends the current, global model to the clients and appoints some of them for training;
%     \item[{\em(ii)}] each selected client locally trains its copy of the global model with its own private data; then, it sends the resulting local model back to the server;\footnote{Whenever we refer to global/local model, we mean global/local model {\em parameters}.}
%     \item[{\em(iii)}] the server updates the global model by computing an \emph{aggregation function} on the local models received from clients (by default, the average, also referred to as FedAvg~\cite{mcmahan2017aistats}).
% \end{enumerate}
This process goes on until the global model converges. %(e.g., after a certain number of rounds or other similar stopping criteria).
%\\
% The advantages of FL over the traditional, centralized learning paradigm are undoubtedly clear in terms of flexibility/scalability (clients can join/disconnect from the FL network dynamically), network communications (only model weights\footnote{We will use \textit{parameters} and \textit{weights} interchangeably.} are exchanged between clients and server), and privacy (each client's private training data is kept local at the client's end and not uploaded to the server).
\\
% Security threats to FL
%However, the growing adoption of FL also raises security concerns~\cite{costa2022covert}, particularly about its confidentiality, integrity, and availability.
Although its advantages over standard ML, FL also raises security concerns~\cite{costa2022covert}. %, particularly about its confidentiality, integrity, and availability~\cite{costa2022covert}.
% OLD, LONG VERSION
% Indeed, some work deals with privacy leakage that may expose the local data of some clients~\cite{melis2019sp}. 
% A large body of work, instead, investigates attacks that usually aim to detriment the predictive accuracy of the learned global model. For instance, \emph{data poisoning} attacks achieve this goal by letting an adversary pollute the training set of some corrupt FL clients with maliciously crafted examples~\cite{jagielski2018sp}.
% Similarly, in \emph{model poisoning} the attacker attempts to tweak the global model weights~\cite{bhagoji2019pmlr} by directly perturbing the local model's weights of some infected FL clients before these are sent to the central server for aggregation, usually via so-called Byzantine attacks. 
% It turns out that Byzantine model poisoning attacks severely impact standard FedAvg; therefore, more robust aggregation functions must be designed to make FL systems secure.
Here, we focus on \emph{untargeted model poisoning} attacks~\cite{bhagoji2019pmlr}, where an adversary attempts to tweak the global model weights %\footnote{We will use the terms \textit{parameters} and \textit{weights} interchangeably.} 
by directly perturbing the local model's parameters of some infected clients before these are sent to the central server for aggregation.
In doing so, the adversary aims to jeopardize the global model \textit{indiscriminately} at inference time.
Such model poisoning attacks severely impact standard FedAvg; therefore, more robust aggregation functions must be designed to secure FL systems.
\\
% In this paper, we focus on designing a novel robust aggregation scheme at the server's end to contrast the effect of Byzantine model poisoning attacks.
%
% Current countermeasures and their limitations
%Several countermeasures have been proposed in the literature to combat model poisoning attacks on FL systems.
% Some methods use simple statistics more robust than plain average to smooth the impact of malicious updates (e.g., Trimmed Mean and FedMedian~\cite{yin2018icml}). 
% Other defenses implement outlier detection techniques to discard malicious updates from the aggregation performed at the server's end. Those are either based on heuristics (e.g., Krum/Multi-Krum~\cite{blanchard2017nips} and Bulyan~\cite{mhamdi2018pmlr}) or data-driven approaches (e.g., K-means clustering~\cite{shen2016acm} or DnC via spectral analysis~\cite{shejwalkar2021ndss}). 
% Finally, some strategies rely on a centralized ``source of trust'' to spot potential malicious updates (e.g., FLTrust~\cite{cao2020fltrust}).
% Several countermeasures have been proposed in the literature to combat model poisoning attacks on FL systems, i.e., to discard possible malicious local updates from the aggregation performed at the server's end. 
% These techniques range from simple statistics more robust than plain average (e.g., Trimmed Mean and FedMedian~\cite{yin2018icml}) to outlier detection heuristics (e.g., Krum/Multi-Krum~\cite{blanchard2017nips} and Bulyan~\cite{mhamdi2018pmlr}) or data-driven approaches (e.g., spectral analysis via K-means clustering~\cite{shen2016acm} or spectral analysis), or methods based on ``source of trust'' (e.g., FLTrust~\cite{cao2020fltrust}).
% OLD, LONG VERSION
%Several countermeasures have been proposed in the literature to combat Byzantine model poisoning attacks on FL systems.
% Descriptive statistics
% For example, Trimmed Mean and FedMedian aggregate local model updates using more robust statistics than standard average~\cite{yin2018icml}.
%
% % Heuristics for outlier detection
% Many existing Byzantine-resilient strategies implement some outlier detection heuristics to discard the model updates sent by potentially malicious clients from the input of the aggregation function.
% One of the most popular heuristics is Krum~\cite{blanchard2017nips}.
% This strategy tries to mitigate the impact of Byzantine attacks by selecting as a global model the local model with the smallest sum of Euclidean distances to {\em all} the other local models.
% Although powerful, Krum requires the server to know (or, at least, estimate) the number of malicious FL clients upfront, which is generally impossible in a realistic attack scenario. %
% Moreover, Krum may become ineffective for complex, high-dimensional model parameter spaces due to the curse of dimensionality.
% Bulyan~\cite{mhamdi2018pmlr} tries to overcome this issue by combining Krum with a variant of Trimmed Mean.
% % Data-driven outlier detection
% Other strategies use data-driven outlier detection techniques -- e.g., via K-means clustering~\cite{shen2016acm} -- to spot potential malicious local model updates. 
% %For instance, Shen et al. propose to cluster local model updates with K-means and thus identify outliers.
%
% % Other techniques
% As far as the server is concerned, any local model received can be from a potential malicious client. 
% FLTrust~\cite{cao2020fltrust} assumes the server acts as a client, i.e., trains a local model on an additional {\em trustworthy} dataset at the server's end and compares it against all the local models from other clients. 
% This way, the server can rely on some ``source of trust'' when discarding potentially malicious clients.
%\\
% Limitations of existing Byzantine-resilient strategies
Unfortunately, existing defense mechanisms either rely on simple heuristics (e.g., Trimmed Mean and FedMedian by~\cite{yin2018icml}) or need strong and unrealistic assumptions to work effectively (e.g., foreknowledge or estimation of the number of malicious clients in the FL system, as for Krum/Multi-Krum~\cite{blanchard2017nips} and Bulyan~\cite{mhamdi2018pmlr}, which, however, cannot exceed a fixed threshold).
Furthermore, outlier detection methods using K-means clustering~\cite{shen2016acm} or spectral analysis like DnC~\cite{shejwalkar2021ndss} do not directly consider the temporal evolution of local model updates received.
Finally, strategies like FLTrust~\cite{cao2020fltrust} require the server to collect its own dataset and act as a proper client, thereby altering the standard FL protocol.
\\
% OLD, LONG VERSION
% Overall, existing Byzantine-resilient strategies are either simple heuristics (e.g., FedMedian) or, if they are more complex, they rely on strong and unrealistic assumptions to work effectively (e.g., knowing the number of malicious clients in the FL system in advance, as for Krum and alike).
% Furthermore, data-driven outlier detection methods do not consider the temporary evolution of local model updates received (e.g., K-means clustering). 
% Finally, strategies like FLTrust requires the server to collect its own dataset and act as a proper client, thereby altering the standard FL protocol.
%
% Description of the proposed method
This work introduces a novel pre-aggregation \textit{filter} robust to untargeted model poisoning attacks. Notably, this filter $(i)$ operates without requiring prior knowledge or constraints on the number of malicious clients and $(ii)$ inherently integrates temporal dependencies. 
The FL server can employ this filter as a preprocessing step before applying \textit{any} aggregation function, be it standard like FedAvg or robust like Krum or Bulyan.
Specifically, we formulate the problem of identifying corrupted updates as a multidimensional (i.e., matrix-valued) time series anomaly detection task. 
The key idea is that legitimate local updates, resulting from well-calibrated iterative procedures like stochastic gradient descent (SGD) with an appropriate learning rate, show \textit{higher predictability} compared to malicious updates. This hypothesis stems from the fact that the sequence of gradients (thus, model parameters) observed during legitimate training exhibit regular patterns, as validated in Section~\ref{subsec:intuition}. %until convergence. 
%This regularity may be more pronounced for smooth convex loss functions, but it can still be captured within an appropriate time window, even for more complex and convoluted loss surfaces. 
%We provide evidence of this claim in Appendix~B, where we show that the average mutual information (i.e., ``predictability''), calculated over pairs of legitimate model updates sent at different FL rounds, is significantly higher than the corresponding computation for a malicious client.
\\
Inspired by the matrix autoregressive (MAR) framework for multidimensional time series forecasting~\cite{chen2021je}, we propose the FLANDERS ({\em \textbf{F}ederated \textbf{L}earning meets \textbf{AN}omaly \textbf{DE}tection for a \textbf{R}obust and \textbf{S}ecure}) filter.
The main advantages of FLANDERS over existing strategies like FLDetector~\cite{zhao2020multivariate} are its resilience to large-scale attacks, where $50\%$ or more FL participants are hostile, and the capability of working under realistic non-iid scenarios.
We attribute such a capability to two key factors: $(i)$ FLANDERS works without knowing a priori the ratio of corrupted clients, and $(ii)$ it embodies temporal dependencies between intra- and inter-client updates, quickly recognizing local model drifts caused by evil players. Below, we summarize our main contributions:

\begin{itemize}
\item[{\em(i)}]
We provide empirical evidence that the sequence of models sent by legitimate clients is more predictable than those of malicious participants performing untargeted model poisoning attacks.
\\
\item[{\em(ii)}] 
We introduce FLANDERS, the first pre-aggregation filter for FL robust to untargeted model poisoning based on multidimensional time series anomaly detection.
\\
\item[{\em(iii)}] 
We integrate FLANDERS into Flower,\footnote{\scriptsize{\url{https://flower.dev/}}} a popular FL simulation framework for reproducibility.
\\
\item[{\em(iv)}] 
We show that FLANDERS improves the robustness of the existing aggregation methods under multiple settings: different datasets, client's data distribution (non-iid), models, and attack scenarios.
\\
\item[{\em(v)}] 
We publicly release all the implementation code of FLANDERS along with our experiments.\footnote{\scriptsize{\url{https://anonymous.4open.science/r/flanders_exp-7EEB}}}
\end{itemize}

% Paper's structure and organization
The remainder of the paper is structured as follows. %some related work and the current state-of-the-art solutions to security issues that FL entails. 
Section~\ref{sec:background} covers background and preliminaries. 
In Section~\ref{sec:related}, we discuss related work.
Section~\ref{sec:problem} and Section~\ref{sec:method} describe the problem formulation and the method proposed. % to tackle it. 
Section~\ref{sec:experiments} gathers experimental results. %, and Section~\ref{sec:limitations} discusses some limitations of this work.
Finally, we conclude in Section~\ref{sec:conclusion}.
 %discusses the limitations of this work and draws future research directions.
%reports conclusions and draws perspectives for future research directions.

%%%%%%% OLD %%%%%%%
%to overcome the resilience of Byzantine failures in distributed Stochastic Gradient Descent computations. 
% The strength of Krum is its time complexity, which is linear in the gradient dimension. 
% However, the robustness of the approach is guaranteed for gradient-based learning applications only when the majority of the clients are not compromised. 
% Besides, the aggregation mechanism of Krum, as well as that of similar methods, is robust from a coarse-grained perspective and does not provide solutions to errors and perturbations that may occur at inference time.
%A related approach to~\cite{blanchard2017nips} is the work of Su et al.~\cite{su2016dc}. Here, the authors propose an iterated approximate agreement to tackle a multi-layer scenario attacked by Byzantine agents. 
%However, the method works efficiently on the sole discrete context and it is inapplicable to continuous state environments.
%\gabri{Maybe, we should just talk about the main limitations of existing countermeasures without digging into their details (or, we can just mention Krum as this is the most popular one). I will move the description of all these methods to the Related Work section.}
	\section{Background on Network Calculus}
\label{sec: background}


\begin{figure*}[tbh]
\centering
\begin{subfigure}[b]{0.3\textwidth}
    \centering
    \includegraphics[width=\linewidth]{images/in-out.png}
    \caption{Arrival and departure data and their relation with delay $d(t)$ and backlog $b(t)$. For a FIFO system, the delay is the horizontal distance between $R(t)$ and $R^*(t)$ but some other multiplexing techniques may shift the data to a later priority, causing a longer delay.}
    \label{fig: data in-out}
\end{subfigure}
\hfill
\begin{subfigure}[b]{0.35\textwidth}
    \centering
    \includegraphics[width=\linewidth]{images/arrival-service.png}
    \caption{Characteristics of an arrival curve and a service curve. From any point of observation, the arriving data never exceeds its arrival curve; the departure data is also never less than the service curve with respect to the data arrival.}
    \label{fig: arrival-service curves}
\end{subfigure}
\hfill
\begin{subfigure}[b]{0.33\textwidth}
    \centering
    \includegraphics[width=\linewidth]{images/bound.png}
    \caption{Delay and backlog bounds of a system. Backlog is the maximum vertical distance between $\alpha(t)$ and $\beta(t)$; FIFO delay is their maximum horizontal distance; but for arbitrary multiplexing, the delay guarantee is when the system clears its buffer, thus it's the intersection of $\alpha(t)$ and $\beta(t)$.}
    \label{fig: system bounds}
\end{subfigure}
\caption{Network calculus framework. We let $R(t)$ and $R^*(t)$ be the arrival and departure data flow of a system; $\alpha(t)$ be the piecewise linear concave arrival curve and $\beta(t)$ be the piecewise linear convex service curve of a system.}
% \hossein{Better to show piece-wise linear concave arrival curve and piece-wise linear convex service curve instead of token-bucket and rate-latency.}}
\end{figure*}

We recall some of the network calculus essentials for a better understanding of the framework used in Saihu. In the following context, we use the following notation: $\mbb{R}^+$ is the set of non-negative real numbers; $[x]_+$ denotes $\max(0, x)$

The data flow is by convention modeled as a left-continuous wide-sense increasing function $R(t): \mbb{R}^+ \mapsto \mbb{R}^+$ with respect to time $t$~\cite{ncbook2001leboudec}. 

A system $\mcal{S}$ receives arrival data described as a cumulative function $R(t)$ and delivers departure data as another cumulative function $R^*(t)$. Figure~\ref{fig: data in-out} illustrates such a system $\mcal{S}$. The benefit of representing a system like this is that we can observe system backlog and delay with such a model. 

\begin{definition}[Backlog and Delay~\cite{ncbook2001leboudec}]
    The backlog of a system at time~$t$ is
    \begin{equation}
        b(t) = R(t) - R^*(t)
    \end{equation}
    
    The virtual delay of a FIFO system at time $t$ is
    \begin{equation}
        d_{FIFO}(t) = \inf \lbp \tau \geq 0 : R(t) \leq R^*(t+\tau) \rbp
    \end{equation}
\end{definition}



The backlog of a system can be viewed as the vertical distance between $R$ and $R^*$. The FIFO (\textit{First-in First-out}) delay is the horizontal distance between $R$ and $R^*$. One may obtain other delay values if the multiplexing technique is not FIFO.

% \begin{figure}
%     \centering
%     \includegraphics[width=0.9\linewidth]{images/in-out.png}
%     \caption{In/out data flow; delay and backlog}
%     \label{fig: data in-out}
% \end{figure}

Since we are interested in the system guarantee instead of a single instance of data flow, we would like to have general bounds to the arrival and departure data flows. Therefore, we define \textit{arrival curve} and \textit{service curve} as the bounds of arrival and departure data flows.

\begin{definition}[Arrival Curve~\cite{ncbook2001leboudec}]
    Given a wide-sense increasing function $\alpha: \mbb{R}^+ \mapsto \mbb{R}^+$, we say that a flow $R(t)$ is $\alpha$-constrained if and only if for all $s \leq t$:
    \begin{equation}
        R(t) - R(s) \leq \alpha(t-s)
    \end{equation}
    We say $R(t)$ has $\alpha$ as an arrival curve.
\end{definition}

\begin{definition}[Service Curve~\cite{ncbook2001leboudec}]
    Given a wide-sense increasing function $\beta: \mbb{R}^+ \mapsto \mbb{R}^+$ and $\beta(0) = 0$. A system $\mcal{S}$ having $R(t)$ and $R^*(t)$ as its arrival and departure flows. We say $\mcal{S}$ offers a service curve $\beta$ if and only if
    \begin{equation}
        R^*(t) \geq (R \otimes \beta)(t) =: \inf_{s \leq t} \lbp R(s) + \beta(t-s) \rbp
    \end{equation}
    where $\otimes$ denotes the min-plus convolution
\end{definition}

Figure~\ref{fig: arrival-service curves} illustrates the arrival and service curves. Any segment of arrival flow $R(t)$ is constrained by arrival curve $\alpha$ and the output curve $R^*(t)$ is always no less than the curve $R\otimes\beta$. As a result, an arrival curve upper bounds the incoming traffic, and a service curve lower bounds the outgoing traffic.

% \begin{figure}
%     \centering
%     \includegraphics[width=\linewidth]{images/arrival-service.png}
%     \caption{Arrival/Service curve}
%     \label{fig: arrival-service curves}
% \end{figure}

We consider 2 special types of curves throughout this paper, \textit{token-bucket} (or sometimes called \textit{leaky-bucket}) curve and \textit{rate-Latency} curve.

\begin{definition}[Token-bucket and Rate-latency~\cite{ncbook2001leboudec}]
    A token-bucket curve $\gamma_{r,b}$ with arrival rate $r$ and burst $b$ is defined as
    \begin{equation}
        \gamma_{r,b}(t) = b + rt
    \end{equation}

    A rate-latency curve $\beta_{R,T}$ with service rate $R$ and latency $T$ is defined as
    \begin{equation}
        \beta_{R,T}(t) = R \lb t - T \rb_+
    \end{equation}
\end{definition}

A token-bucket curve is determined by a burst $b$ and an arrival rate~$r$. Burst represents the maximum possible data volume that can arrive simultaneously, and arrival rate represents the maximum long-term data rate~\cite{bouillard2022tradeoff}.
A rate-latency curve is determined by a latency~$T$ and a service rate~$R$. Latency represents the time a server needs before starting to process the incoming data, and service rate represents the minimum rate to process data after the initial latency.

With the help of arrival and service curves, we can derive delay and backlog bounds for a system $\mcal{S}$ illustrated in Figure~\ref{fig: system bounds}. Suppose a system $\mcal{S}$ has arrival curve $\alpha$ and service curve~$\beta$, its worst-case backlog $b^*$ is the maximum vertical distance between~$\alpha$ and~$\beta$. Similarly, depending on the multiplexing technique applied to the system, its worst-case delay bound $d^*$ is the maximum horizontal distance between $\alpha$ and $\beta$ if $\mcal{S}$ is a FIFO system. If we don't have any information about its multiplexing technique, referred to as arbitrary multiplexing, the best we can say is that when $\alpha$ and $\beta$ intersect each other, where all data has been delivered out of the system. Consequently, the worst-case delay bound for arbitrary multiplexing is the time required for $\mcal{S}$ to clear its buffer.

% \begin{figure}
%     \centering
%     \includegraphics[width=\linewidth]{images/bound.png}
%     \caption{System delay/backlog bounds}
%     \label{fig: system bounds}
% \end{figure}

While a service curve captures the slowest possible output speed of a system, a link's transmission capacity limits the speed as well. Hence, we model this phenomenon using a \textit{greedy shaper} with a sub-additive function $\sigma: \mbb{R}^+ \mapsto \mbb{R}^+$ concatenated with a server. We consider a concatenation as shown in Figure \ref{fig: system}. By convention we assume $\sigma(0) = 0$ and $\beta(t) \leq \sigma(t), \forall t \in \mbb{R}^+$, meaning that the buffer is cleared at the beginning and the service never exceed its physical limitation. With the above definition, such greedy shaper conserves the service provided by the system due to theorem \ref{thm: shaping}.

\begin{figure}[thb]
    \centering
    \includegraphics[width=0.7\linewidth]{images/system.png}
    \caption{Shaping of departure data. A flow that has an arrival curve $\alpha$ feeds into a server with an arrival data flow $R(t)$. The server having service curve $\beta$ takes $R(t)$ and gives a departure data flow $R^*(t)$ to a shaper with shaping function $\sigma$. The shaper takes $R^*(t)$ and shape the data flow as another departure $D(t)$.}
    \label{fig: system}
\end{figure}


\begin{theorem}[Shaping conserves service \cite{ncbook2001leboudec}]
\label{thm: shaping}
Following the system shown in Figure \ref{fig: system}, we have
\begin{equation}
     D = R^* \otimes \sigma \geq \lp R \otimes \beta \rp \otimes \sigma = R \otimes \lp \beta \otimes \sigma \rp = R \otimes \beta
\end{equation}
\end{theorem}

In the following context, we model the shaping function $\sigma$ as a token-bucket curve $\gamma_{C,L}$ with transmission capacity $C$ and the packet size $L$ to capture the link capacity and packetization~\cite{bouillard2022tradeoff}.

	

\section{Review of ML-based ride-hailing planning}
\label{sec:review}
\revise{In this section, we review matching, repositioning, and joint matching and repositioning in Sec.~\ref{sec:review-matching} Sec.~\ref{sec:review-repositioning}, and Sec.~\ref{sec:review-joint}, respectively.}
In each part, we discuss the collective and the distributed strategy separately.
Fig.~\ref{fig:review-outline} gives an outline of the review.

\begin{figure*}[h]
	\centering
	\includegraphics[width=0.8\linewidth]{figs/survey-taxonomy.pdf}
	\caption{\revise{A taxonomy of the ride-hailing planning literature. %is summarized.
	In each category, we discuss three works as representative examples.}}
	\label{fig:review-outline}
\end{figure*}

\subsection{Matching}
\label{sec:review-matching}

\subsubsection{Collective Matching}
\revise{RL is a promising technique for solving the matching problem.
Chen et al.~\cite{chen2020order} propose an RL-based solution in which
a deep evaluation network, which is a plain feed-forward neural network, is used to calculate a score for each pair of driver and rider based on the predicted detour distance, vehicle's seat utilization rate, and profit achieved if they get matched.
For each new ride request, the vehicle with the highest score will be assigned to serve the rider.
When the trip of the ride request is finished, the observed reward, i.e., the sum of the increased profit of the driver and, if any, the reduced cost of the rider through sharing the ride with others, is used to guide the learning process of the deep evaluation network.
Agussurja et al.~\cite{agussurja2019state} formulate the matching problem as a two-stage planning process.
In the first stage, ride requests to be scheduled are selected from all the unserved ones, the problem of which is modeled as a Markov Decision Process.
An approximated value iteration algorithm is used to learn the value function for the matching actions.
In the second stage, the final matching decision is made between the selected ride requests and all vehicles based on the learned value function.
\revise{Kullman et al.~\cite{kullman2022dynamic} apply deep RL to develop matching policies whose decisions leverage the Q-value approximations learned by deep neural networks.}
Multi-hop ride-hailing can improve the efficiency of a ride-hailing system.
To find the transfer points for each transferring trip in the multi-hop ride-hailing service, Xu et al.~\cite{xu2020highly} use a multi-layer feed-forward network to predict the reachable areas of vehicles, based on which the search space of possible vehicle pairs and transfer points for transferring riders is pruned.
In this way, the transfer points searching process can be more efficient.
	Wang et al.~\cite{wang2023optimization} also consider the scenario where riders are allowed to transfer between vehicles.
	They leverage RL to learn a policy that estimates the values of all the vehicles, which are then used to compute the optimal matching decisions by integer-linear programming.
The lengths of the time-intervals between the matching decisions can have critical impact in the matching outcomes.
	Specifically, the efficiency of matching may be improved substantially if the matching is delayed by adaptively adjusting the matching time-intervals according to the real-time situation of the riders and drivers.
	Wang et al.~\cite{wang2019adaptive} find that, if riders are willing to wait for a certain amount of time even if there are available vehicles that can serve them right away, the ride-hailing system can achieve better results, for example, in terms of the total vehicle miles traveled.
	In their solution, 
	 they propose to use an RL policy to decide for each rider, at each time step, whether to conduct matching for her/him, or %leave her/him alone and
	wait for the next time step. 
	Similarly, Qin et al.~\cite{qin2021optimizing} leverage RL in solving the ride-hailing matching problem with dynamic matching time-intervals.}

\revise{Clustering techniques are frequently used in ride-hailing planning.
Hong et al.~\cite{hong2017commuter} propose to use a density-based clustering algorithm, specifically DBSCAN \cite{parimala2011survey}, to identify riders that share similar itineraries based on their historical traveling trajectories. 
To alleviate the computational overhead caused by the large number of distance queries in the matching process, Zhang et al.~\cite{zheng2018order} propose a new clustering algorithm that groups the geographical locations in the road network into different clusters.
Then, the distance between any two nodes is approximated by the distance between the centers of the clusters they belong to. 
Shen et al.~\cite{shen2019roo} propose a spatial-temporal distance metric that measures the similarity of each pair of ride requests.
The ride requests are grouped by a clustering process based on the proposed distance metric.
Then, shared-rides are computed within each group of ride requests.
Another clustering algorithm is proposed by \cite{trasarti2011mining} to extract the mobility profiles from riders' and drivers' historical itineraries.
The matching between riders and drivers is determined based on the similarities between their profiles.}


\revise{An increasing number of collective matching solutions leverage various other ML techniques in planning.
Most of them take social factors of drivers and riders into consideration \cite{mitropoulos2021systematic}.
To mitigate the social barriers in the ride-hailing process, especially in shared-rides, Yatnalkar et al.~\cite{yatnalkar2020enhanced} and Narman et al.~\cite{narman2021enhanced} use Support Vector Machine (SVM) to predict the user social types, e.g., chatty, safety, or punctuality, based on their registered user characteristics.
Riders with similar social characteristics %are more likely
would be more willing to share a trip.
%on their closest available vehicle.
Levinger et al.~\cite{levinger2020human} use a feed-forward neural network to predict rider satisfaction levels according to their profile and trip information.
They proposed a stochastic algorithm to compute the matching decision with rider satisfaction level maximization as the objective.
Montazery and Wilson \cite{montazery2016learning, montazery2018new} propose to take into account the user preference in evaluating the weight (benefit) of the matching between each pair of rider and driver, which is given by their proposed support vector machine-based score function.
With the value calculated, the final matching can be obtained by solving an optimization problem in which the sum of the weights of those matched pairs is maximized.
Tang et al.~\cite{tang2020efficient} model various types of information (e.g., driver, rider, travel time, and activity) and their relationships within a ride-hailing system using a Heterogeneous Information Network (HIN) \cite{sun2012mining}.
Each driver or rider is projected to a multi-dimensional embedding (vector) using the skip-gram model \cite{mikolov2013efficient}.
Moreover, the skip-gram is conducted on node sequences obtained by meta path-based random walks originating from the corresponding node within the HIN \cite{dong2017metapath2vec}.
The cosine similarity between the embeddings of each driver-rider pair is then used to identify possible matching.
Zhang et al.~\cite{zhang2017taxi} consider a scenario where each rider is assigned to multiple drivers (to improve the order answer rate), and riders are free from having to enter the details of destinations (to improve the user experience).
They first leverage historical data to model the probability distribution of destinations of each rider based on his/her departure time and location with Bayesian rules, which is followed by predicting the acceptance probability between the rider and available drivers with logistic regression \cite{friedman2001elements} and gradient boosted decision tree \cite{mason1999boosting}.
They propose a hill climbing-based algorithm to solve the matching problem, which is formulated as an NP-hard combinatorial optimization with maximizing the success rate of matching as the objective.
Schleibaum and M{\"u}ller \cite{schleibaum2020human} advocate taking the determinants of user satisfaction and explainable matching decisions into consideration.
One of their future studies is to find out whether increasing the explainability can improve user satisfaction level or not.}

\revise{It is worth mentioning that many ML-based collective matching strategies take advantage of the Kuhn-Munkres (KM) bipartite matching algorithm as a component of their decision-making pipelines \cite{jonker1986improving}.
Drivers and riders are usually regarded as the two sets of vertices in the target bipartite graph.
To guide the matching between ride requests and ride offers,  Guo et al.~\cite{guo2020spatiotemporal} propose spatial-temporal Thermo, which is used to reflect the demand density of different places and times.
They use Random Forest Regression \cite{breiman2001random} to map multiple features of spatial, temporal, and meteorological dimensions to Thermo.
The weight of each pair of driver and rider in the bipartite graph is estimated by Thermo.
A KM algorithm is then used to calculate the final matching decisions according to the constructed bipartite graph.
Similarly, Xu et al.~\cite{xu2018large} derive their matching decisions using the KM algorithm.
In contrast to \cite{guo2020spatiotemporal}, Xu et al.~\cite{xu2018large} leverage a policy evaluation algorithm to learn a value function which maps each pair of driver and rider to a score.
The KM algorithm calculates the final matching between drivers and riders based on the scores.
Guo and Xu \cite{guo2020deep} also conduct the matching planning using the KM algorithm.
The weight between each pair of driver and rider is obtained from a value function learned by a convolutional neural network-based Double Q-learning (Double DQN) algorithm \cite{van2016deep}.}

\subsubsection{Distributed Matching}
\revise{RL is also a powerful technique for distributed matching \cite{sutton1999reinforcement}.
Gu{\'e}riau and Dusparic \cite{gueriau2018samod} use the Q-learning algorithm to train a policy for each agent (driver) to choose the pickup or rebalancing action based on the environment state, including the status of itself and current distribution of supply and demand.
If pickup action is chosen, then the vehicle will go and pick up the nearest rider.
In their follow-up work \cite{gueriau2020shared}, they extend the method to consider traffic congestion when agents are making decisions.
Wang et al.~\cite{wang2018deep} propose to use the DQN \cite{mnih2015human}, in which a deep neural network is employed to estimate the state-action value function from a single driver's perspective.
Many methods of distributed matching allow the decisions to be determined individually while the matching policy is trained collectively.
For example, De Lima et al.~\cite{de2020efficient} follow the QMIX framework proposed in \cite{rashid2018qmix}, in which the coordinated planning policies are trained by learning a joint action-value function for multiple vehicles and riders aiming at optimizing a global objective.
In the execution process, the matching decision of each vehicle is made in a distributed  manner following its own component in the learned action-value function.
By ensuring the monotonicity of the relationship between the global action-value and the action-value of each passenger, the objectives of distributed planning decisions are ensured to coincide with the centralized decisions during the training process.
Similar to \cite{de2020efficient}, Li et al.~\cite{li2019efficient} adopt the framework where the matching policy is trained in a centralized manner and executed in a distributed manner.
Specifically, they adopt the actor-critic RL framework, where actor and critic are two different networks used to decide and evaluate the action for each driver, respectively.
The coordination among drivers in the matching policy is enabled by the critic network.
It adopts the mean field approximation to model the interactions of drivers by calculating an average on the actions taken by their neighborhoods, which is then considered in the process of evaluating each driver's action.
\revise{In \cite{zhou2019multi}, another centralized training process is proposed, in which a Kullback–Leibler divergence optimization is used to balance the supply and demand and to enable coordination among the vehicles.}
In the execution phase, each driver chooses an action based on their own action-value functions.}

\revise{Some distributed matching strategies leverage other ML techniques.
They mostly determine the matching decisions based on the similarities between the riders and drivers in ride-hailing.
For example, Bicocchi and Mamei \cite{bicocchi2014investigating} use the bag-of-words model to summarize users' frequently visited places as vector representations, which are then fed to the Latent Dirichlet Allocation (LDA) \cite{blei2003latent} model to identify their patterns of daily travel routine behaviors.
Given a rider or a driver, his/her potential participants of shared-rides can be found by calculating the similarities between his/her daily travel routine and those of the other riders and drivers.
Lasmar et al.~\cite{lasmar2019rsrs} propose to leverage a multi-layer Perceptron model to learn user preferences based on their responses to the questionnaires.
For each rider, a ranking list of potential partners for shared-rides is generated according to the similarities between the predicted preferences of her/him and other riders.}

    
\subsection{Repositioning}
\label{sec:review-repositioning}
\subsubsection{Collective Repositioning}
\revise{Some collective repositioning methods leverage RL techniques.
Ride-hailing repositioning for electric vehicles is studied in \cite{liang2020mobility, tang2020online}, in which the state of charge of the electric vehicles is an important factor to be considered.
Liang et al.~\cite{liang2020mobility} develop a solution method utilizing deep RL combined with binary linear programming to obtain a regional joint planning policy for electric vehicles with their state of charge considered.
Using binary linear programming, each vehicle repositioning action is modeled as a binary decision variable, and its weight in the objective is obtained by the value function learned by the policy iteration method.
Similarly, Tang et al.~\cite{tang2020online} also combine RL with combinatorial optimization, in which the RL learned policy is used to advise decision making in the optimization step.
Liang et al.~\cite{liang2021integrated} adopt temporal-difference (TD) learning to obtain action-value function.
Different from \cite{de2020efficient}, the settings in \cite{liang2021integrated} do not allow factorization of the joint action-value function into individual ones while guaranteeing global maximization.
Thus, they formulate two linear programming instances to collectively find the decisions for the vehicles.
To improve the stability of the training process in RL, Fluri et al.~\cite{fluri2019learning} propose a cascading multi-level learning model.
In this model, the area concerned is split in halves as the number of levels of learning increases.
The policy training process proceeds in a top-down manner, i.e., from less to more fine-grained area partitioning. 
The motivation behind is that the policy trained from a coarse level can serve as guidance to the finer levels, which avoids the instability caused by directly training a policy with a large state size (w.r.t. the number of regions).
Fluri et al.~\cite{fluri2019learning} propose to leverage the Lloyd K-means algorithm \cite{lloyd1982least} to partition the area concerned into multiple smaller regions.
Deng et al.~\cite{deng2020multi} leverage the Proximal Policy Optimization algorithm (PPO) \cite{schulman2017proximal} to learn the joint repositioning policy for vehicles, in which the value- and policy-function are approximated by neural networks.
Shi et al.~\cite{shi2019optimal} use Deep Deterministic Policy Gradient (DDPG) \cite{silver2014deterministic} to learn the grid-based multiple vehicles repositioning policy with the objective of total profits maximization.
\revise{In \cite{shou2020reward}, a mean-filed multi-agent RL approach is leveraged to collectively relocate the vehicles in ride-hailing.}}


%In collective repositioning, 
\revise{Some other collective repositioning solutions leverage various ML techniques to predict future information of a ride-hailing system, which plays an important role in guiding the platforms to make better repositioning decisions \cite{chen2022h}.
Riley et al.~\cite{riley2020real} leverage Vector autoregression to forecast the future demand from region to region.
The predicted demand and current system status are then fed into two mixed-integer programming instances to find the desired distribution of vehicles and the assignment of vehicles to regions, respectively.
Iglesias et al.~\cite{iglesias2018data} use a Long Short-Term Memory (LSTM) neural network to predict the future ride requests for each pair of origin and destination within a certain time period.
The predicted information is then used as input to their proposed mixed-integer linear programming instance, which is solved to find the optimal rebalancing actions.
Xu et al.~\cite{xu2018taxi} use two LSTM-based and Mixture Density Network (MDN)-based models to predict the distributions of origins and destinations of future requests, respectively.
With a prediction on the distributions, the repositioning decisions are then obtained by solving a mixed-integer programming problem with total idle driving distance minimization as the objective.
Cheng et al.~\cite{cheng18taxis} leverage a multilevel logistic regression model to predict the likelihood of ride requests occurring at different times and places.
The online repositioning planning decisions of drivers are obtained by leveraging a centralized multi-period stochastic optimization model with both the real-time and predicted demand considered.
Li et al.~\cite{li2020data} and Gao et al.~\cite{gao2020learning} formulate the repositioning task as a two-stage stochastic programming problem.
The source of the stochasticity is the underlying uncertainty of the future demands, the probability distribution of which is obtained by kernel density estimation and a deep learning model combining the LSTM and MDN in \cite{li2020data} and \cite{gao2020learning}, respectively.
Pouls et al.~\cite{pouls2020idle} propose a forecast-driven repositioning solution framework, the core of which is a mixed-integer programming problem with the demand predictions as inputs.
Moreover, it is solved by an off-the-shelf solver called Gurobi \cite{gurobi}.
Note that, in practice, not all planning decisions can be successfully executed by the drivers at the end.
Xu et al.~\cite{xu2020recommender} take the first step to predict the failure possibility of repositioning tasks in the decision-making process, including situations where drivers disobey the planning or end up being unmatched for an unexpectedly long time even though they follow the repositioning planning decisions accordingly.
In the latter case, drivers will be compensated.
The failure rate of each repositioning task is predicted by XGBoost \cite{chen2016xgboost} with both driver- and environment-related features as inputs.
\revise{The problem of multi-vehicle collaboration optimization aiming at maximizing the platform's profit is converted into a minimum cost flow problem, which is solved by an off-the-shelf method called GNU Linear Programming Kit (GLPK) \cite{makhorin2008glpk}. }}


\subsubsection{Distributed Repositioning}
Geographical regions or grids (i.e., abstracts of individual locations) are usually used to model the road networks in the problem of ride-hailing repositioning.
Different from most of the repositioning methods (e.g., \cite{lin2018efficient, riley2020real, ke2019optimizing, li2019efficient, zhou2019multi}) in which the region of interest is divided into predefined and static geographic zones, Castagna et al.~\cite{castagna2020demand, castagna2021demand} leverage the Expectation-Maximization clustering algorithm to derive zones for rebalancing vehicles in an online manner.
They leverage the Proximal Policy Optimization algorithm (PPO) \cite{schulman2017proximal} to train a policy for each vehicle to decide whether to make a pick-up, drop-off, or repositioning action.
Specifically, similar to \cite{tang2021value}, the repositioning destination is also sampled from a probability distribution over all potential positions, which is determined by the number of unserved requests.
Different from \cite{castagna2020demand, castagna2021demand}, Verma et al.~\cite{verma2017augmenting} propose an iterative method to dynamically split the zones based on their expected revenue (Q-values).
The iterative splitting process does not terminate until the historical data is exhausted for the Q-values learning.
\revise{Different from most of the works that model the drivers as agents, Jin et al.~\cite{jin2019coride} regard each geographical region as an agent.}
By hierarchically partitioning the target areas into regions with different granularities, they perform hierarchical RL where the multi-head attention mechanism is used to capture the impacts among the neighboring agents.
Guo et al.~\cite{guo2021multi} try various methods (e.g., Support Vector Regression, Random Forest Regression, and k-Nearest Neighbors regression) to predict future demand density, which is then used to evaluate each region for their spatial-temporal value.
Each available vehicle chooses to stay still or relocate to a neighbor region in a probabilistic manner based on their spatial-temporal values, which can help avoid over-saturation of supply.
In \cite{provoostdemandprop}, the region of interest is represented as a graph.
They build two neural networks to predict the demand on vertices and the passenger flows on edges, respectively.
The proposed repositioning algorithm aims at satisfying the demand on edges in the decreasing order with the nearest vehicles found by backward traversing.

\revise{However, in spite of the various grid-based methods as discussed in most of the related works mentioned above, e.g., \cite{lin2018efficient, guo2021multi}, Jiao et al.~\cite{jiao20deep, jiao2021real} argue that grid-based repositioning policies are not satisfactory in practice because of the excessively-simplified and overlooked non-stationarity in the environment caused by the dynamic environment and the large number of vehicles when coarse-grained region-wise decisions are considered.}
They put forward the process of carrying out repositioning %algorithms
in industrial production by combining offline learning, i.e., batch RL, and online planning stages, i.e., decision-time planning \cite{sutton1999reinforcement}.
To counter the issues of coarse-grained decisions, Kim and Kim \cite{kim2020optimizing} uses a graph to model the road networks which is more realistic.
They build a Graph Neural Network to predict the future demands.
The repositioning destination of each driver is decided greedily based on a function of the predicted demand, the number of excessive vehicles, and the distance information to each candidate position.

\revise{Some other works also spend special effort on tackling the non-stationarity.
With the observation that the actions of drivers are independent (based on self interests), Chaudhari et al.~\cite{chaudhari2020learn} propose a vanilla RL framework where each driver, based on a probabilistic value denoting the extent to which coordination is needed, stochastically chooses to perform an action guided by the independent or coordinated policy.
Note that, although vehicles execute repositioning decisions sequentially in this solution framework, coordination in the latter policy is explicitly considered by solving a minimum cost flow problem for the optimal rebalancing flow of vehicles among all the regions (which is similar to \cite{xu2020recommender}).
In addition, the independence between different repositioning policies learned by the drivers concurrently also contributes to the non-stationarity of the environment.
In this regard, Verma et al.~\cite{verma2019entropy} propose a method for each driver to learn the information of other vehicles in order to make a better planning decision.}
%Concretely,
The principle of maximum entropy \cite{jaynes1957information} is leveraged to improve the predictability of the distribution of drivers even with only limited knowledge available, e.g., the local density of supply.
To tackle the non-stationary challenge in online ride-hailing as well as the catastrophic forgetting of RL \cite{kemker2018measuring}, Haliem et al.~\cite{haliem2020adapool, haliem2021adapool} propose to learn multiple repositioning policies to deal with different contexts of environments (e.g., peak/non-peak hours and weekends/weekdays).
When changes in the distribution of experiences are identified by their proposed change point detection algorithm, switching among those different policies is enabled so as to enhance adaptability to the dynamic environment.
Lei et al.~\cite{lei2019optimal} define the concept of stochastic relocation matrix.
The element in the $i$-th row and the $j$-th column within the matrix represents the probability that an empty vehicle located in the $i$-th region should relocate to the $j$-th region.
\revise{To circumvent the curse of dimensionality, they leverage low-rank approximation to project the original matrix onto a low-dimensional vector.}
They propose a deep convolution-LSTM model to learn how to predict the approximation vector based on the system status.
To alleviate the instability of the state-value function approximator caused by the large scale of its states, Tang et al.~\cite{tang2019deep} propose to bound its outputs by regularizing its worst-case variation w.r.t. %any
changes in its inputs (i.e., states).
Transfer learning proposed in \cite{wang2018deep} is applied to increase the adaptability of the trained model across different cities.

\revise{Besides the traditional ML techniques discussed above, RL, being another well-known technique for decision making in non-stationary environments, has been a key technology in distributed repositioning \cite{khetarpal2022towards,xie2021deep,mao2021near}.}
\revise{Liu et al.~\cite{liu2022deep} propose a single-agent deep RL approach which relocates vacant vehicles to regions with a large demand gap in advance.}
Nguyen et al.~\cite{nguyen2018policy} propose to use the RL framework to train a homogeneous repositioning policy for all agents, i.e., vehicles.
\revise{The policy is trained in a centralized manner with collective behaviors of drivers considered while executing in a distributed manner.}
He and Shin \cite{he2019spatio} leverage Double DQN with their proposed spatial-temporal capsule-based neural network as the state-action value approximator.
The inputs of the network proposed include the location of the vehicle to be relocated, distribution of other vehicles and riders, ride preferences, and some external factors that have impacts on supply and demand, e.g., weather conditions and holiday events.
With all those information processed, the estimated value for each candidate position given the current state of the target vehicle is obtained, and the final decision can be decided in a probabilistic manner.
A more elaborate analysis is presented in their follow-up study \cite{he2020spatio}.
Yu et al.~\cite{yu2019markov} formulate the single-vehicle repositioning planning problem as a Markov Decision Process.
They propose to leverage parallelized matrix operations to re-formulate the Bellman equation \cite{sutton1999reinforcement}, thus reducing the computational complexity in finding optimal planning policy.
Multi-hop ride-hailing repositioning is considered in \cite{singh2019reinforcement, singh2021distributed}.
Similar to \cite{al2019deeppool}, they predict the number of vehicles in each region for certain time slots ahead of time using an estimated time of arrival (ETA) model.
Double DQN is adopted for each vehicle to choose the best neighbor region to move forward based on the current status of all the vehicles and the predicted demand and supply.
    
\subsection{Joint Matching and Repositioning}
\label{sec:review-joint}
In this part, we review methods that jointly optimize matching and repositioning with ML techniques. 
Note that all of them belong to the category of distributed planning. The research works in this part leverage RL to guide the decision making process.
Different from the review given by Qin et al.~\cite{qin2021reinforcement}, we focus on the works that jointly decide matching and repositioning.

Haliem et al.~\cite{haliem2020distributed-a, haliem2021distributed} propose to consider both the matching and repositioning in the ride-hailing planning process.
In their ride-hailing systems, each vehicle conduct initial matching by greedily searching the nearest requests, after which an insertion-based method is used to finalize the potential request list.
Then each driver, based on the value function learned by the DQN, weighs the requests in the final list.
The riders who receive those proposed ride offers can decide whether to accept the offers and join the trips where shared-rides are allowed.
The trips can be solo-ride or shared-ride.
Drivers are repositioned in parallel with the matching process.
Each driver takes actions indicated by his/her trained RL agent, i.e., the decision-making policy, independently.
Their proposed solution framework learns an optimal policy for each driver as opposed to those RL-based methods with collective planning scheme where a central policy is used, e.g., \cite{oda2018movi}.
Note that, in some works, although each driver makes decisions independently (e.g., \cite{haliem2020distributed-a, haliem2021distributed}), all drivers share one trained policy (e.g., \cite{manchella2020passgoodpool, manchella2021flexpool}).
Manchella et al.~\cite{manchella2020passgoodpool, manchella2021flexpool} propose to collectively optimize the system objectives, e.g., minimizing the waiting times and routing times.
Nevertheless, they allow distributed inference at the level of individual drivers. 
Their proposed model can be used by each vehicle independently.
It helps decrease computational costs associated with the growth of distributed systems. 
Specifically, they utilize a Double DQN with the experience relay mechanism.
Their model learns a probabilistic dependence between drivers' actions and the reward function.
The trained policy indicates a destination for each driver if s/he is not matched with any rider according to their proposed heuristic matching algorithm. 
Similar to \cite{xu2020highly, singh2019reinforcement, singh2021distributed}, multi-hop transit is enabled in their solutions.
Wang et al.~\cite{wang2018deep} model the matching and repositioning problems as a Markov Decision Process and propose learning solutions based on DQNs to optimize the trained policy for the drivers.
\revise{Their solution uses a temporal and spatial expanded action search strategy to accommodate the scenarios where there is only sparse training data, e.g., certain remote regions in the middle of the night.}
Besides, to increase the learning adaptability and
efficiency, they propose to use a transfer learning method to leverage the knowledge across both spatial and temporal spaces.

Besides \cite{haliem2020distributed-a, haliem2021distributed, manchella2020passgoodpool, manchella2021flexpool, wang2018deep},
DQN is used in other works as well, e.g., \cite{al2019deeppool, guo2022deep, tang2021value, li2020balancing}.
In \cite{al2019deeppool}, each vehicle decides its action by learning the impact of its action on the reward using a DQN model without coordinating with other vehicles.
In \cite{guo2022deep}, the vehicle repositioning procedure is formulated as a Markov Decision Process.
By sampling the future riders based on the historical probability distribution, the proactive relocation of vehicles is realized via a deep RL framework, which is composed of a Convolutional Neural Network and a Double DQN module. \revise{Then a request-vehicle assignment scheme is presented based on the value function attained from the vehicle repositioning process.}
\revise{Similarly, Tang et al.~\cite{tang2021value} propose a planning framework for tackling both the matching and repositioning tasks, the core of which is a unified value function which is trained offline using abundant historical data and is updated during the online phase.}
With the value function learned, the matching problem is then solved by the method proposed in \cite{xu2018large}, while the reposition destination of each idle vehicle is determined in a probabilistic manner following the distribution given by the discounted long-term values of all the candidate positions.
Li and Allan \cite{li2020balancing} also leverage a global value function for both the tasks of matching and repositioning, which is learned by the value iteration algorithm with historical data of ride requests.



	\begin{table*}[tp]
\centering
{\color{black}\begin{threeparttable}
\caption{\revise{Summary of open-source trip related data sets.}}
\begin{tabular}{@{}C{2.2cm}C{2.4cm}cccm{3.8cm}@{}}
\toprule
\textbf{Source}& \textbf{City} &\textbf{Type} & \textbf{\#~of Records} & \textbf{Time} & \multicolumn{1}{>{\centering\arraybackslash}m{3.8cm}}{\textbf{Data Features}}\\ \midrule
\midrule
Uber Pickups \cite{p668-gy46-22} & New York, United States& Pickups &20M & \begin{tabular}[c]{@{}c@{}}April -- September 2014,\\January -- June 2015 \end{tabular} & Timestamp and location.\\ \midrule
NYC TLC \cite{tlctrip} & New York, United States& Ride request &500M&January 2009 -- July 2021 &Passenger count, start time, end time, origin, destination, distance, fee, etc.\\ \midrule
DiDi GAIA \cite{gaiadata}& Haikou and Chengdu, China& Ride requests  &18M&\begin{tabular}[c]{@{}c@{}}November 2016,\\May -- October 2017\end{tabular}&Trip ID, fee, start time, end time, origin, destination, etc.\\ \midrule
Chicago Data Portal \cite{chicagoalll} & Chicago, United States & Ride requests &199M &January 2013 -- December 2021&Trip ID, taxi ID, start time, end time, fee, origin, destination, etc.\\ \midrule
T-drive \cite{tdrive, yuan2010t, yuan2011driving} & Beijing, China& Trajectories &15M&February 2008&Taxi ID, location, and timestamp.\\ \midrule
GeoLife \cite{geolifetraj, zheng2008learning, zheng2008understanding, zheng2010understanding} & Beijing, China& Trajectories &18,670&April 2007 -- August 2012 &Timestamp, location, transportation mode, etc. \\ \midrule
Beijing Taxi Trajectories \cite{lian2018one} & Beijing, China & Trajectories &129M&May 2019& Taxi ID, location, timestamp, speed, occupancy indicator, etc.\\ \midrule
DiDi GAIA \cite{gaiadata} & Chengdu and Xi'an, China &Trajectories &-\tnote{*}&-&\multicolumn{1}{>{\centering\arraybackslash}m{3.8cm}}{\textbf{-}}\\ \midrule
ECML PKDD 2015 \cite{portokaggle} & Porto, Portugal  & Trajectories  & 2M&July 2013 -- June 2014 &Trip ID, taxi ID, the sequence of locations of the trajectory, etc.\\ \midrule
CRAWDAD EPFL \cite{c7j010-22}  & San Francisco, United States & Trajectories  &11M&May -- June 2008& Taxi ID, location, timestamp, and occupancy indicator.\\\midrule
CRAWDAD Roma \cite{c7qc7m-22}  & Roma, Italy & Trajectories  &22M&February -- March 2014& Taxi ID, location, and timestamp.\\\midrule
Jeju Vehicular Trajectories \cite{y8vk-wj40-22}  & Jeju, South Korea & Trajectories  &8M& - & Vehicle ID, location, speed, lane, etc.\\\midrule
Grab-Posisi \cite{grabsource, huang2019grab}  &Singapore and Jakarta, Indonesia& Trajectories  &84,000&April 2019&Trajectory ID, location, timestamp, speed, etc.\\\midrule
Shanghai Taxi Trajectories \cite{2877-mk46-19, liu2020optimization}  &Shanghai, China& Trajectories  &61M&April 1, 2018&Taxi ID, location, timestamp, speed, occupancy indicator, driving status, etc.\\\midrule
Foursqure \cite{foursquare} & Global & Check-ins &33M&April 2012 -- September 2013 & Venue ID, timestamp, location, etc.\\ \midrule
Brightkite \cite{brightkite} & - & Check-ins &4M&April 2008 -- October 2010&User ID, timestamp, location, etc.\\ \midrule
Gowalla \cite{gowalla} & -  & Check-ins &6M&February 2009 -- October 2010&User ID, timestamp, location, etc.\\ \midrule
LTA of Singapore \cite{singaporedata} & Singapore & \begin{tabular}[c]{@{}c@{}}Real-time\\locations\end{tabular}  &-&-&Location. \\\midrule
Uber \cite{traveltimedata} &  Global &Travel times  &-&-&Origin, destination, travel time, etc.\\\bottomrule
\end{tabular}
\label{table:data}
\begin{tablenotes}
\item[*] \revise{``-'' indicates that the corresponding information is not attainable.}
\end{tablenotes}
\end{threeparttable}}
\end{table*}

\begin{table*}[t]
\centering
{\color{black}\begin{threeparttable}
\caption{Summary of the major simulators.}
\begin{tabular}{@{}cccm{9.6cm}@{}}
\toprule
\textbf{Name} & \textbf{Open-Source} & \begin{tabular}[c]{@{}c@{}}\textbf{Sample Applications}\\\textbf{in Ride-hailing} \end{tabular}& \multicolumn{1}{>{\centering\arraybackslash}m{9.6cm}}{\textbf{Description}} \\ \midrule
\midrule
 DiDi \cite{didisimulation} & \xmark &\cite{tang2021value}&An online simulation platform to evaluate matching and repositioning algorithms.  Evaluation is run with DiDi's real-world data.\\ \midrule
 AMoDeus \cite{ruch2018amodeus} & \cmark&\cite{ruch2020quantifying}&A tool that uses an agent-based transportation simulation framework to simulate arbitrarily configured mobility-on-demand systems with static/dynamic demand. It includes standard benchmark algorithms and a graphical user interface.\\ \midrule
 AMoD2 \cite{amod2} & \cmark&\cite{li2021optimal}&A high-capacity ride-sharing simulator. It uses map data and taxi data from Manhattan. Three matching algorithms and one simple rebalancing algorithm are implemented. \\ \midrule
 Mod-abm \cite{abm-1.0, abm-2.0} & \cmark&\cite{wen2017rebalancing}&A platform for simulation of large-scale mobility-on-demand operations. It supports city-level systems in any urban setting.\\ \midrule
 MATSim \cite{horni2016multi} & \cmark&\cite{tsao2019model}&An open-source framework for implementing large-scale agent-based transport simulations. It consists of several modules which can be combined or used in stand-alone mode, for demand-modeling, traffic flow simulation, re-planning; and a controller to iteratively run simulations, and methods to analyze outputs generated by the modules.\\ \midrule
 SUMO \cite{lopez2018microscopic} & \cmark&\cite{castagna2021multi,zhu2021shared}&A microscopic and space-continuous traffic simulation platform that is suitable for the generation, evaluation, and validation of traffic scenarios of real-world size. It supports road network customization and demand modeling.\\ \midrule
 CityFlow \cite{zhang2019cityflow} & \cmark&-\tnote{*}&A multi-agent RL environment for large scale city traffic scenario. It supports flexible definitions for road network and traffic flow. It provides faster simulation than SUMO.\\ \midrule
 STaRS \cite{ota2016stars} & \xmark &-&A simulation framework for analyzing diverse ride-sharing scenarios, considering the platforms' needs and constraints. Its real-time trip assignments utilize a linear optimization algorithm, efficient indexing, and parallelization for scalability.\\ \midrule
 STaRS+ \cite{mounesan2021fleet} & \xmark &-&A simulation framework based on an integer linear programming model, using heuristic optimization and a novel shortest-path caching scheme for scalability. It supports the simulation of full-city scale ride-sharing with meeting points.\\ \midrule
 UberSim \cite{khalil2022realistic} & \cmark & \cite{salman2023quantifying} &A digital twin transportation simulation model for Birmingham, Alabama, incorporating various transportation modes to analyze the impact of ride-hailing services on urban traffic. It supports policy learning through reinforcement learning.\\ \midrule
 NYC-Yellow-Taxi-V0 \cite{chaudhari2020learn} & \cmark&-&A multi-agent RL environment based on the OpenAI Gym environment \cite{openai-gym}, which offers a toolkit for developing and comparing RL-based ride-hailing fleet management algorithms.\\ \midrule
 SMART-eFlo \cite{liu2022smart} & \cmark & - &An integrated framework that combines the SUMO simulator with multi-agent Gym for reinforcement learning studies, enabling researchers to easily design traffic scenarios and implement RL algorithms for electric fleet management problems\\ \midrule
\end{tabular}
\label{table:simulator}
\begin{tablenotes}
\item[*] ``-'' indicates no sample application in ride-hailing using the corresponding simulator.
\end{tablenotes}
\end{threeparttable}}
\end{table*}



\section{Resources for empirical studies}
\label{sec:resource}
Real-world data are essential in studying ML-based ride-hailing planning, e.g., to train a proposed model.
Further, in order to deploy those proposed planning strategies in practice, it is necessary to utilize a ride-hailing service (process) simulator to validate their performance with real-world data.
\revise{In this section, we present publicly available %multiple
open-source real-world data sets and several related simulators in Sec.~\ref{sec:resource-data} and Sec.~\ref{sec:resource-simulator}, respectively.}

\subsection{Data}
\label{sec:resource-data}
Historical records of ride requests are the most frequently used data in the literature.
New York City Taxi and Limousine Commission \cite{tlctrip} provides more than 11 years of records of ride requests with many data fields, e.g., the pickup and drop-off timestamps, pickup and drop-off locations, trip fare, and ride distance.
About 19 million Uber's pick-up records obtained from this data set %\cite{tlctrip}
are summarised in \cite{p668-gy46-22}.
Chicago Data Portal \cite{chicagoalll} provides a large data set consisting of trip records with various data fields recorded from 2013.
Ride request data of Haikou and Chengdu in China are publicly available in the DiDi GAIA program \cite{gaiadata}.
Besides, there are some other data sets that consist of people's check-in records, e.g., those collected from Foursquare \cite{foursquare}, Brightkite \cite{brightkite}, and Gowalla \cite{gowalla}.
They are informative in revealing ride demands as the data have recorded riders' target places they had traveled to.
In addition to the records of ride requests, some data of the supply side are also available.
\revise{Trajectory data recorded by sampling drivers' locations at a certain frequency are also commonly used in the literature, including trajectories recorded in Beijing (in three different data sets: T-drive \cite{tdrive, yuan2010t, yuan2011driving}, GeoLife \cite{geolifetraj, zheng2008learning, zheng2008understanding, zheng2010understanding}, and Beijing Taxi Trajectories \cite{lian2018one}), Chengdu \cite{gaiadata}, Xi'an \cite{gaiadata}, Porto \cite{portokaggle}, Jakarta \cite{grabsource, huang2019grab}, Singapore \cite{grabsource, huang2019grab}, San Francisco \cite{c7j010-22}, Roma \cite{c7qc7m-22}, Jeju \cite{y8vk-wj40-22}, and Shanghai \cite{2877-mk46-19, liu2020optimization}.}
The Land Transport Authority (LTA) of Singapore \cite{singaporedata} publicizes many APIs for accessing various kinds of transport-related data, including monthly statistics of taxi supply and real-time coordinates of all taxis that are currently available for hire. %offering ride service. 
\revise{Besides, the travel times among different locations in various cities across the world can be obtained in \cite{traveltimedata}.}
The data sets mentioned above are summarized in Table.~\ref{table:data}.
\revise{Note that, although abundant public data sets are available for use, none of them record those ride requests that ended up unserved (because of, for example, excessive waiting time).}
If both the served and unserved ride requests are treated in the ride-hailing simulation, the simulation results would be more realistic than if only
the served ride requests are considered. 

In addition to the trip-related data sets mentioned above, there are several important public data sources that could provide good values
to ride-hailing planning studies.
\revise{OpenStreetMap \cite{openstreetmap} captures worldwide road networks, which can be easily accessed by APIs (e.g., OSMnx) using Python \cite{boeing2017osmnx}.}
Travel times and speeds information of road networks of several well known cities have been made available in \cite{traveltimedata}.
Finally, historical data of weather conditions can be found in \cite{weatherdata}.
Note that weather conditions can be considered in ride-hailing planning
as they could affect the traffic in a major way
\cite{he2019spatio}.
    
    
    
\subsection{Simulators}
\label{sec:resource-simulator}
With the aforementioned real-world data, proposed planning strategies can be evaluated through simulators.
DiDi \cite{didisimulation} makes public an industrial-level simulation platform recently, which is used in the KDD CUP 2020 \cite{kddcup20}. 
The platform evaluates submitted matching and repositioning strategies using real-world data from the GAIA program.
Wen \cite{abm-1.0} develops an agent-based modeling platform for simulating autonomous mobility-on-demand systems, which is later upgraded to a version with better scalability and extensibility \cite{abm-2.0}.
Based on \cite{abm-2.0}, a high-capacity on-demand ride-sharing simulator is proposed in \cite{amod2} with several built-in matching and repositioning algorithms.
Ruch et al.~\cite{ruch2018amodeus} have released an open-source simulator named AMoDeus for accurate and quantitative analysis of matching and repositioning algorithms in the ride-hailing system.
Examples of its usages can be found in \cite{fluri2019learning, carron2019scalable}.
AMoDeus is built upon the open-source microscopic multi-agent transportation simulation environment MATSim \cite{horni2016multi}.
MATSim is also used to evaluate ride-hailing planning strategies, a case study of which can be found in \cite{bischoff2016simulation}.
Besides MATSim, there are similar public traffic simulation tools, including SUMO \cite{lopez2018microscopic} and CityFlow \cite{zhang2019cityflow}.
A more detailed comparison and analysis of the performance of traffic simulation tools can be found in \cite{allan2015benchmark}.
\revise{STaRS is a scalable simulation framework that takes the needs and constraints of platforms into consideration \cite{ota2016stars}. 
STaRS+ extends STaRS to support the simulation of ride-sharing with meeting points \cite{mounesan2021fleet}.}
\revise{For those RL-based ride-hailing planning methods, their policies can be trained using CityFlow \cite{zhang2019cityflow}, UberSim \cite{khalil2022realistic}, NYC-Yellow-Taxi-V0 \cite{chaudhari2020learn}, or SMART-eFlo \cite{liu2022smart}.}
The aforementioned simulators are summarized in Table.~\ref{table:simulator}.

    
	
\section{Conclusion and Future Work}
\label{sec:future}
% Whenever a new solution concept is defined, one of the most intuitive questions is how it can be related to other properties? 
We presented a practical solution to the problem of leximin optimization when only an approximate single-objective solver is available. 
The algorithm is guaranteed to terminate in polynomial time, and its approximation ratio degrades gracefully as a function of the approximation ratio of the single-objective solver.

Currently, our algorithm handles two main settings. First, when inaccuracies in the single-objective solver stem from numeric errors.
Second, when the problem is convex and satisfy several assumptions.
It may be interesting to study more settings in which the inaccuracies stem from computational hardness of the single-objective problem.
% Currently, our algorithm handles settings in which the inaccuracies in the single-objective solver stem from numeric errors.
% It may be interesting to study settings in which the inaccuracies stem from computational hardness of the single-objective problem.
%
%

% identify problems in which an appropriate approximate solver can be designed. 
In particular, to approximate the egalitarian welfare, it is common to model the problem as an integer program or as an exponential sized linear program (e.g., \cite{bansal2006santa, kawase_max-min_2020}) and then approximate the program using different techniques.
% rounding techniques or methods for convex optimization (such as the ellipsoid method).
Can these algorithms be generalized to consider the additional constraints described in Section \ref{sec:algo-short}? This will allow approximating leximin using the approach in this paper.
% In particular, in the problem of stochastic allocations (in Section \ref{sec:app}), to extend the approximation algorithm for the egalitarian welfare, we had to change some steps within.
% What if an algorithm for egalitarian welfare is provided as a black box --- could it be used to design the appropriate procedure to approximate leximin?

% In the context of fair division, this study assumes that there is an access to the true valuations of the agents involved. 
% In reality, people may lie about their valuations.
% Can our definition of approximate-leximin be related to some approximate version of truthfulness?

Another question is whether it is possible to obtain a better approximation factor for leximin, given an $(\multApprox, \additiveApprox)$-approximation algorithm for the single-objective problem.
Specifically, can an $(\multApprox, \additiveApprox)$-approximation to leximin can be obtained in polynomial time? 
If not, what would be the best possible approximation in this case?
% \erel{Mention the tightness of our results}


\iffalse % EREL: removed for the submission. To clarify later
Further, the algorithm suggested in Section \ref{sec:algo-short} tend to work very well if the single-objective optimization problems are convex, and in particular if they are linear programs. 
\erel{Why? the algorithm of \textcite{Ogryczak2004TelecommunicationsND} works even for non-convex programs.}
Can we find an algorithm that works with general approximation algorithms? For example, naive algorithms such as \emph{Next-fit}?
\fi

% \begin{itemize}
%     \item Meaning of approximately leximin in the context of other characteristics like truthfulness.
    
%     \item Solving more problems.
    
%     \item Is it possible to obtained a better approximation factor for leximin maximization in polynomial time? given that $(1-\beta)$ is the best possible for the egalitarian maximization, is it possible to obtain a $(1-\beta)$ approximately leximin optimal solution? what is the best possible approximation  in this case?
    
%     \item The algorithm works very well if the single-objective problem is convex, in particular if it is a linear programs, but can it be modify to work with general approximation algorithms? such as algorithms for makespan minimization?
% \end{itemize}
	\section{Conclusion}\label{sec:conclusion}
In this work, we focus on addressing the fundamental challenge of OOD detection tasks, which is how to fully understand the semantic discrepancy between the ID/OOD samples. We reveal that the key to success in the realistic SCOOD task is to allocate as many ID samples in the unlabeled set correctly as possible. To this end, we propose a novel uncertainty-aware optimal transport scheme that introduces class-specific energy scores as guidance for effective label assignment. Experimental results show that our method achieves better performance than previous state-of-the-art methods on SCOOD benchmarks.

\textbf{Limitations.} In addition to temperature scaling, other techniques such as feature clipping applied in ReAct~\cite{sun2021react} also enhance the performance of energy score, so how to obtain an OOD score that best fits the SCOOD task can be further explored. Moreover, a setting highly related to SCOOD has been proposed in \cite{katz2022training} and formulated as a constrained optimization problem. We will also theoretically analyze these practical OOD settings in our feature work.

% \section*{Acknowledgments}
\textbf{Acknowledgments.} 
This work is supported by National Key R\&D Program of China under Grant 2020AAA0105701, National Natural Science Foundation of China (NSFC) under Grants 61872327, Major Special Science and Technology Project of Anhui, National Natural Science Foundation of China (62033012) and Ant Group through Ant Research Intern Program.

	
% \newpage
\bibliographystyle{IEEEtran}
%\bibliography{ref}
% \newpage
 \documentclass[lettersize, journal]{IEEEtran}
\usepackage{amsthm}
\usepackage{amsmath}
\usepackage{amssymb}
\usepackage{graphicx}
\usepackage{indentfirst}
\usepackage{algorithm,algorithmic}
%\usepackage[ruled, vlined]{algorithm2e}
\usepackage[top=1.6cm, bottom=2cm, left=2cm, right=2cm]{geometry}
\usepackage{threeparttable}
\usepackage{multirow}
\usepackage[usenames]{color}
% \pagestyle{empty}
\usepackage{subfigure}
\usepackage{hyperref}

% \usepackage[numbers]{natbib}
% \newcommand{\cites}[1]{\citeauthor{#1} \cite{#1}}
%\newenvironment{sloppypar}{\par\sloppy}{\par}
\usepackage{mathtools} 
\DeclarePairedDelimiter{\ceil}{\lceil}{\rceil} 
\DeclarePairedDelimiter\floor{\lfloor}{\rfloor}
\renewcommand{\algorithmicrequire}{\textbf{Input:}} 
\renewcommand{\algorithmicensure}{\textbf{Output:}}
\usepackage[dvipsnames]{xcolor}
\newcommand{\dc}[1]{\textcolor{red}{#1}}
\newcommand{\cm}[1]{\textcolor{gray}{#1}}
\newcommand{\alg}{\text{ALG}}
\newcommand{\opt}{\text{OPT}}
\newcommand{\hy}{\widehat{y}}
\makeatletter
\newcommand*{\rom}[1]{\expandafter\@slowromancap\romannumeral #1@}
\makeatother
\usepackage{xurl}
\usepackage{booktabs}
\usepackage{tikz}
\newcommand{\mycaption}[1]{\stepcounter{figure}\raisebox{-7pt}
  {\footnotesize Fig. \thefigure.\hspace{3pt} #1}}

\providecommand{\keywords}[1]
{
%	\small	
	\textbf{\textit{Keywords---}} #1
}

\theoremstyle{definition}
\newtheorem{lemma}{Lemma}
\newtheorem{claim}{Claim}
\newtheorem{theorem}{Theorem}
\newtheorem{remark}{Remark}
\newtheorem{definition}{Definition}
\newtheorem{objective}[theorem]{Objective}
\newtheorem{problem}{Problem}

\usepackage{multirow}
\usepackage{array}
\newcommand{\PreserveBackslash}[1]{\let\temp=\\#1\let\\=\temp}
\newcolumntype{C}[1]{>{\PreserveBackslash\centering}p{#1}}
\newcolumntype{R}[1]{>{\PreserveBackslash\raggedleft}p{#1}}
\newcolumntype{L}[1]{>{\PreserveBackslash\raggedright}p{#1}}
\usepackage{amssymb}% http://ctan.org/pkg/amssymb
\usepackage{pifont}% http://ctan.org/pkg/pifont
\newcommand{\cmark}{\ding{51}}%
\newcommand{\xmark}{\ding{55}}%
% \usepackage[parfill]{parskip}
\usepackage[font=footnotesize]{caption} 
% \newcommand{\dc}[1]{\textcolor{red}{#1}}
\usepackage{lipsum}
\usepackage{threeparttable}
 \newcommand{\revise}[1]{\textcolor{black}{#1}}

\allowdisplaybreaks
\title{A Survey of Machine Learning-Based Ride-Hailing Planning}
%\author{}

\author{
Dacheng Wen,~\IEEEmembership{Student Member,~IEEE},
Yupeng Li,~\IEEEmembership{Member,~IEEE},
Francis C.M. Lau
\IEEEcompsocitemizethanks{
		\IEEEcompsocthanksitem
            Dacheng Wen is with The University of Hong Kong, Hong Kong (e-mail: wdacheng@connect.hku.hk). Work done while Dacheng Wen was under the supervision of Yupeng Li and Francis C.M. Lau.\protect
            \IEEEcompsocthanksitem Yupeng Li (corresponding author) is with Hong Kong Baptist University, Hong Kong (e-mail: ivanypli@gmail.com).\protect
            \IEEEcompsocthanksitem Francis C.M. Lau is with The University of Hong Kong, Hong Kong (email: fcmlau@cs.hku.hk).
	}
}

\begin{document}
%	\markboth{Journal of \LaTeX\ Class Files}%,~Vol.~XX, No.~X, XX~2022
%	{Shell \MakeLowercase{\textit{et al.}}: A Sample Article Using IEEEtran.cls for IEEE Journals}
	
	\maketitle
	
% 	\input{contents/test}
	\input{contents/abstract}
	
	\begin{IEEEkeywords}
		Ride-hailing, machine learning, matching, repositioning, collective planning, distributed planning.
		%Article submission, IEEE, IEEEtran, journal, \LaTeX, paper, template, typesetting.
	\end{IEEEkeywords}
	
	\input{contents/introduction}
	\input{contents/background}
	\input{contents/review}
	\input{contents/resources}
	\input{contents/future}
	\input{contents/conclusion}
	
% \newpage
\bibliographystyle{IEEEtran}
%\bibliography{ref}
% \newpage
 \input{RS_survey-arxiv.bbl}

\vspace{-0.4cm}

\begin{IEEEbiography}[{\includegraphics[width=1in,height=1.25in,clip,keepaspectratio]{bioPhoto/dc.jpeg}}]{Dacheng Wen}
(Student Member, IEEE) received the M.S.~degree in computer science from The University of Hong Kong. 
He is currently pursing the Ph.D.~degree with the Department of Computer Science, The University of Hong Kong.
His research interests include machine learning and intelligent transportation systems.
He is also excited about the emerging blockchain-driven revolution of various computer science and information technology.
\end{IEEEbiography}

\vspace{-0.4cm}

\begin{IEEEbiography}[{\includegraphics[width=1in,height=1.25in,clip,keepaspectratio]{bioPhoto/ivan.png}}]{Yupeng Li} (Member, IEEE) received the Ph.D. degree in computer science from The University of Hong Kong. He was with the University of Toronto and is currently with Hong Kong Baptist University. His research interests are in general areas of network science and, in particular, algorithmic decision making and machine learning problems, which arise in networked systems. %, such as information networks and ride-sharing platforms. 
He is also excited about interdisciplinary research that applies algorithmic techniques to edging problems. Recently, he has worked on robust online machine learning for the application of data classification, and he has extended these techniques to modern areas in networking and social media. Dr. Li has been awarded the Rising Star in Social Computing Award by CAAI and the distinction of Distinguished Member of the IEEE INFOCOM Technical Program Committee in 2022. He serves on the technical committees of some top conferences in computer science. His works have been published in prestigious venues, such as \textsc{IEEE INFOCOM}, \textsc{ACM MobiHoc}, \textsc{IEEE Journal on Selected Areas in Communications}, and \textsc{IEEE/ACM Transactions on Networking}. He is a member of ACM and IEEE.
\end{IEEEbiography}

\vspace{-0.4cm}

\begin{IEEEbiography}[{\includegraphics[width=1in,height=1.25in,clip,keepaspectratio]{bioPhoto/lau.jpg}}]{Francis C.M. Lau} received the Ph.D. degree from the Department of Computer Science, University of Waterloo. He is currently an Honorary Professor and Associate Head (part-time) of the Department of Computer Science at The University of Hong Kong, China. His research interests include computer systems, networks, programming languages, and application of computing in arts. He was the Editor-in-Chief of the Journal of Interconnection Networks from 2011 to 2020. 
\end{IEEEbiography}
 
\end{document}


\vspace{-0.4cm}

\begin{IEEEbiography}[{\includegraphics[width=1in,height=1.25in,clip,keepaspectratio]{bioPhoto/dc.jpeg}}]{Dacheng Wen}
(Student Member, IEEE) received the M.S.~degree in computer science from The University of Hong Kong. 
He is currently pursing the Ph.D.~degree with the Department of Computer Science, The University of Hong Kong.
His research interests include machine learning and intelligent transportation systems.
He is also excited about the emerging blockchain-driven revolution of various computer science and information technology.
\end{IEEEbiography}

\vspace{-0.4cm}

\begin{IEEEbiography}[{\includegraphics[width=1in,height=1.25in,clip,keepaspectratio]{bioPhoto/ivan.png}}]{Yupeng Li} (Member, IEEE) received the Ph.D. degree in computer science from The University of Hong Kong. He was with the University of Toronto and is currently with Hong Kong Baptist University. His research interests are in general areas of network science and, in particular, algorithmic decision making and machine learning problems, which arise in networked systems. %, such as information networks and ride-sharing platforms. 
He is also excited about interdisciplinary research that applies algorithmic techniques to edging problems. Recently, he has worked on robust online machine learning for the application of data classification, and he has extended these techniques to modern areas in networking and social media. Dr. Li has been awarded the Rising Star in Social Computing Award by CAAI and the distinction of Distinguished Member of the IEEE INFOCOM Technical Program Committee in 2022. He serves on the technical committees of some top conferences in computer science. His works have been published in prestigious venues, such as \textsc{IEEE INFOCOM}, \textsc{ACM MobiHoc}, \textsc{IEEE Journal on Selected Areas in Communications}, and \textsc{IEEE/ACM Transactions on Networking}. He is a member of ACM and IEEE.
\end{IEEEbiography}

\vspace{-0.4cm}

\begin{IEEEbiography}[{\includegraphics[width=1in,height=1.25in,clip,keepaspectratio]{bioPhoto/lau.jpg}}]{Francis C.M. Lau} received the Ph.D. degree from the Department of Computer Science, University of Waterloo. He is currently an Honorary Professor and Associate Head (part-time) of the Department of Computer Science at The University of Hong Kong, China. His research interests include computer systems, networks, programming languages, and application of computing in arts. He was the Editor-in-Chief of the Journal of Interconnection Networks from 2011 to 2020. 
\end{IEEEbiography}
 
\end{document}


\vspace{-0.4cm}

\begin{IEEEbiography}[{\includegraphics[width=1in,height=1.25in,clip,keepaspectratio]{bioPhoto/dc.jpeg}}]{Dacheng Wen}
(Student Member, IEEE) received the M.S.~degree in computer science from The University of Hong Kong. 
He is currently pursing the Ph.D.~degree with the Department of Computer Science, The University of Hong Kong.
His research interests include machine learning and intelligent transportation systems.
He is also excited about the emerging blockchain-driven revolution of various computer science and information technology.
\end{IEEEbiography}

\vspace{-0.4cm}

\begin{IEEEbiography}[{\includegraphics[width=1in,height=1.25in,clip,keepaspectratio]{bioPhoto/ivan.png}}]{Yupeng Li} (Member, IEEE) received the Ph.D. degree in computer science from The University of Hong Kong. He was with the University of Toronto and is currently with Hong Kong Baptist University. His research interests are in general areas of network science and, in particular, algorithmic decision making and machine learning problems, which arise in networked systems. %, such as information networks and ride-sharing platforms. 
He is also excited about interdisciplinary research that applies algorithmic techniques to edging problems. Recently, he has worked on robust online machine learning for the application of data classification, and he has extended these techniques to modern areas in networking and social media. Dr. Li has been awarded the Rising Star in Social Computing Award by CAAI and the distinction of Distinguished Member of the IEEE INFOCOM Technical Program Committee in 2022. He serves on the technical committees of some top conferences in computer science. His works have been published in prestigious venues, such as \textsc{IEEE INFOCOM}, \textsc{ACM MobiHoc}, \textsc{IEEE Journal on Selected Areas in Communications}, and \textsc{IEEE/ACM Transactions on Networking}. He is a member of ACM and IEEE.
\end{IEEEbiography}

\vspace{-0.4cm}

\begin{IEEEbiography}[{\includegraphics[width=1in,height=1.25in,clip,keepaspectratio]{bioPhoto/lau.jpg}}]{Francis C.M. Lau} received the Ph.D. degree from the Department of Computer Science, University of Waterloo. He is currently an Honorary Professor and Associate Head (part-time) of the Department of Computer Science at The University of Hong Kong, China. His research interests include computer systems, networks, programming languages, and application of computing in arts. He was the Editor-in-Chief of the Journal of Interconnection Networks from 2011 to 2020. 
\end{IEEEbiography}
 
\end{document}


\vspace{-0.4cm}

\begin{IEEEbiography}[{\includegraphics[width=1in,height=1.25in,clip,keepaspectratio]{bioPhoto/dc.jpeg}}]{Dacheng Wen}
(Student Member, IEEE) received the M.S.~degree in computer science from The University of Hong Kong. 
He is currently pursing the Ph.D.~degree with the Department of Computer Science, The University of Hong Kong.
His research interests include machine learning and intelligent transportation systems.
He is also excited about the emerging blockchain-driven revolution of various computer science and information technology.
\end{IEEEbiography}

\vspace{-0.4cm}

\begin{IEEEbiography}[{\includegraphics[width=1in,height=1.25in,clip,keepaspectratio]{bioPhoto/ivan.png}}]{Yupeng Li} (Member, IEEE) received the Ph.D. degree in computer science from The University of Hong Kong. He was with the University of Toronto and is currently with Hong Kong Baptist University. His research interests are in general areas of network science and, in particular, algorithmic decision making and machine learning problems, which arise in networked systems. %, such as information networks and ride-sharing platforms. 
He is also excited about interdisciplinary research that applies algorithmic techniques to edging problems. Recently, he has worked on robust online machine learning for the application of data classification, and he has extended these techniques to modern areas in networking and social media. Dr. Li has been awarded the Rising Star in Social Computing Award by CAAI and the distinction of Distinguished Member of the IEEE INFOCOM Technical Program Committee in 2022. He serves on the technical committees of some top conferences in computer science. His works have been published in prestigious venues, such as \textsc{IEEE INFOCOM}, \textsc{ACM MobiHoc}, \textsc{IEEE Journal on Selected Areas in Communications}, and \textsc{IEEE/ACM Transactions on Networking}. He is a member of ACM and IEEE.
\end{IEEEbiography}

\vspace{-0.4cm}

\begin{IEEEbiography}[{\includegraphics[width=1in,height=1.25in,clip,keepaspectratio]{bioPhoto/lau.jpg}}]{Francis C.M. Lau} received the Ph.D. degree from the Department of Computer Science, University of Waterloo. He is currently an Honorary Professor and Associate Head (part-time) of the Department of Computer Science at The University of Hong Kong, China. His research interests include computer systems, networks, programming languages, and application of computing in arts. He was the Editor-in-Chief of the Journal of Interconnection Networks from 2011 to 2020. 
\end{IEEEbiography}
 
\end{document}
