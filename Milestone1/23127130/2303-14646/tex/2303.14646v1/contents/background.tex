\section{Background} 
\label{sec:background}
    In this part, we start with the main components of the ride-hailing system and their interactions in Sec.~\ref{sec:background-system}. 
    Then we discuss the ride-hailing problems and their challenges in Sec.~\ref{sec:background-planning}.
    \revise{Our proposed taxonomy for the solution scheme of the methods of existing ML-based ride-hailing planning is introduced in Sec.~\ref{sec:background-scheme}.}

\subsection{\revise{Ride-Hailing System}}
\label{sec:background-system}
    % In a ride-hailing system, there are three main types of entities, the ride-hailing platform, drivers, and riders, respectively.
    \revise{There are three types of entities in a ride-hailing system: the ride-hailing platform, drivers, and riders.}
    A ride-hailing platform operates in the system as a transportation service organizer.
    It establishes the connections and communications between service providers (i.e., drivers) and service consumers (i.e., riders).\footnote{In this paper, we regard drivers 
    and riders 
    as the supply and the demand, respectively.}
    \revise{To illustrate the interactions among the three types of entities, we give a visualization of the ride-hailing process in Fig.~\ref{fig:pipeline}.}
    At any time, riders and drivers can enter the system by making ride requests and ride offers, respectively, through the ride-hailing platform.
    Upon receiving ride offers and requests, the platform makes \emph{planning} decisions for the drivers and riders.
    A planning decision includes the matching between ride offers and ride requests and the repositioning of vehicles.\footnote{\revise{Similar to the planning, \emph{pricing} is another important decision-making module in the ride-hailing system. Our survey focuses on the ML-based ride-hailing planning where most existing works investigate matching and/or repositioning. Notably, many worldwide leading ride-hailing
platforms such as DiDi decouple the decision-making process of pricing from the ones of matching and repositioning.}}


\subsection{\revise{Ride-Hailing Planning}}
\label{sec:background-planning}
    \revise{We dig into the problems of matching and repositioning as they show up in the ride-hailing planning in Sec.~\ref{sec:background-planning-matching} and Sec.~\ref{sec:background-planning-repositioning}, respectively.
    For each part, the general forms of the problem with problem inputs, decisions, and objectives are presented, followed by the challenges we may face when designing solutions.}
    
    \subsubsection{\revise{Matching and Its Challenges}}
    \label{sec:background-planning-matching}
    Matching is the problem of assigning ride offers to ride requests \cite{alonso2017demand, dickerson2018allocation, xu2018large}.
    \revise{A ride-hailing platform makes matching decisions with the effect of keeping the ride-hailing system function properly.}
    The inputs of the matching problem include the information of all vehicles and riders.
    \revise{Each rider has an origin and a destination.
   	A vehicle has its real-time location, on-board rider(s), remaining capacity, and a predefined destination of the driver, if applicable (depending on whether the driver has her/his own trip to make).
    Besides the vehicles and riders, other information may also be useful, such as supply and demand patterns that are learned from the historical data (e.g., \cite{alonso2017predictive, liu2019globally, lin2019probabilistic, wang2019data}) and estimated time of traveling along different routes (e.g., \cite{jindal2018optimizing, xu2018large, zhou2019multi, liu2022personalized}).
    \revise{Given all the information mentioned above, the platform can make matching decisions on the assignments of ride offers to ride requests.}
    In the mode of shared-rides, multiple riders w.r.t.~different ride requests can travel together in a vehicle to their destinations, subject to some preset constraints, e.g., the maximum detour distance and latest time of arrival of the riders (see \cite{mourad2019survey} for more constraints in shared-rides).
    In solo-ride mode, riders are served alone in vehicles during their entire trips.
    \revise{If multi-hop riding is allowed, a rider can be matched with multiple vehicles that cover
    %deliver the rider in
    different hops of her/his way to the destination.}
    %Then, the rider is transferred among the set of matched vehicles and is delivered consecutively by these vehicles to her/his destination 
    \cite{masoud2017decomposition, shah2020sride, xu2020highly}.
    There is a variety of objectives for the matching decisions as revealed
    in the literature.}
    \revise{Mourad et al.~\cite{mourad2019survey} divide the objectives into two categories, operational objectives and quality-related objectives.} %respectively.
    The former category is concerned with system-wide metrics, e.g., total vehicle miles traveled and the number of served requests.
    The latter focuses on the quality of ride-hailing services from the perspective of individuals, which is usually measured by riders' waiting time, riders' detour time, drivers' incomes, etc.
    
    \revise{The problem of ride-hailing matching has been proved to be NP-hard
    %if considering the
    even in the offline setting where full knowledge of ride requests and ride offers are known in advance \cite{bei2018algorithms, alonso2017demand}.
    The online nature of ride offers and ride requests makes matching even more challenging since future information is not known or uncertain when decisions are made.
    A matching decision at any time can influence the upcoming ride requests and vehicles.
    A \emph{myopic} strategy, such as those in \cite{alonso2017demand, bei2018algorithms, li2020trip}, that matches drivers and riders with an aim of optimizing the outcomes only at
    the current time-step may not satisfy long-term objectives.
    Besides, the scale of the problem instance is usually overwhelmingly large for many metropolitan areas, and thus efficient solutions are very much sought after %important desiderata
    \cite{agatz2012optimization}.}
    In reality, the uncertainty in the environment, e.g., travel time and traffic condition, could also have an impact on the performance of the matching algorithms. %proposed in the literature.
    
    \subsubsection{\revise{Repositioning and Its Challenges}}
    \label{sec:background-planning-repositioning}
    \revise{Repositioning is the problem of proactively relocating vehicles to certain positions in advance to prepare them for achieving better outcomes in future matching \cite{lin2018efficient, xu2020recommender, chaudhari2020learn}.
    Though there are drivers who are willing to provide ride offers, vehicle resources can still be underutilized, as ride requests can easily end up unserved without an appropriate repositioning strategy.}
    By properly allocating the vehicle resources, repositioning can lead to high utility of vehicles and high fulfillment of riders \cite{chaudhari2020learn}.
    \revise{To solve the problem of repositioning, the input can be current spatial distributions of vehicles and riders.}
    \revise{Additional inputs could include predictions about future supplies and demands \cite{gao2020learning, zhang2020multiple, guo2021multi, zhang2021mlrnn, ke2021joint, chen2022h}.}
    \revise{The output decisions are the positions to relocate to for all idle vehicles.}
    \revise{Examples of commonly adopted %metrics for the
    repositioning objectives in the literature include minimization of the idle time of vehicles \cite{o2021using}, the waiting time of riders \cite{ji2020spatio}, the number of unfulfilled ride requests \cite{liu2019globally}, the overall difference between the supply and demand of all regions being considered \cite{guo2021multi}, etc. 
    There are also some %metrics are maximized in the
    objectives that aim to maximize metrics such as gross merchandise volume of the platform \cite{lin2018efficient} and the cumulative driver incomes \cite{jin2019coride}.}
    
    The challenges of the repositioning problem are threefold.
    \revise{Firstly, similar to the matching problem, a ride-hailing platform typically needs to deal with a large number of vehicles in real time \cite{oda2018distributed}.}
    Simple strategies can lead to poor performance.
    \revise{For example, a method that greedily relocates available vehicles to the places that have a concentration of ride requests might leave other places with insufficient supply of vehicles. }
    % Simple strategies, e.g., greedily relocating available vehicles to areas with high demand, might result in over-saturation or shortages of supply in certain regions.
    Making a satisfactory repositioning decision efficiently is imperative.
    \revise{Secondly, the supply and demand themselves are not only changing dynamically over time but also subject to the impacts caused by the repositioning decisions made at the previous time steps, the relation between which, however, is difficult to be modeled in the decision making process \cite{liu2020context}.
    Thirdly, in practice, drivers can, due to self-interest, deviate from the repositioning decisions given by the platform in order to maximize their own profits \cite{oda2018movi, sadeghiyengejeh2021re}. }
    The inability of complete control over the behaviors of vehicle resources can introduce more uncertainty in planning.
    
\subsection{\revise{Schemes of Ride-Hailing Planning}}
\label{sec:background-scheme}
    \revise{In this part, we introduce a %novel
    taxonomy with which we can divide the literature of ride-hailing planning into two groups based on their solution schemes, namely \emph{collective} planning and \emph{distributed} planning. }
    
   \revise{In collective planning strategies, the matching or repositioning decisions of all drivers are determined jointly.
    The decisions of individuals depend on each other.
    In this way, coordination among the involved drivers can be considered explicitly.
    It helps optimize the decisions globally and hence benefits the performance regarding certain system-wide objectives, such as the maximization of platforms’ incomes and the overall satisfaction of riders’ requirements \cite{he2019spatio}.
    Thus, collective planning is preferable when aiming at the global optimum against the system-wide objectives.}
    
    \revise{As for the distributed scheme, the planning processes of drivers are generally independent of each other.
    Drivers' actions can be decided asynchronously, i.e., in a per driver manner, without any explicit coordination between one another.
    In other words, while planning the decision for a driver, the actions of others are not considered \cite{oda2018movi, al2019deeppool, singh2021distributed, haliem2021distributed}.
    Though it may not be able to achieve the global optimum of certain system-wide parameters as mentioned above in the distributed scheme, it has some advantages in the following aspects over the collective scheme \cite{al2019deeppool, oda2018movi, chau2020decentralized}.
    Firstly, the planning decisions in the distributed scheme can respect each individual's interests, while the collectively planned decisions that aim at the system-wide objectives may not coincide with each driver's or each rider's intentions.
    For example, some drivers might be required to pick up some riders far away from them or relocate to some places with low demand, which are not ideal for the drivers.
    In this regard, drivers may prefer distributed planning to the collective one as they are driven by self interest. 
    Secondly, compared to the collective design, making decisions from the perspective of a single driver reduces the computational complexity in planning. }
    That is, the distributed planning scheme ensures scalability as a result of less coordination between vehicles. 
    This is especially advantageous when dealing with large problem instances in practice, e.g., during peak hours or in crowded cities.
    \revise{Thirdly, higher planning efficiency can reduce the challenges caused by the uncertainty
    of rapidly changing environments, e.g., varying patterns of supply and demand, travel time, and traffic conditions %in optimization,
    due to
    the fact that distributed planning process takes less time than the collective one and can be done within shorter time slots.} %can be regarded as less dynamic.
    

    