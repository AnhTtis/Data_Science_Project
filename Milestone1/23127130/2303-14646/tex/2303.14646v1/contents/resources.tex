\begin{table*}[tp]
\centering
{\color{black}\begin{threeparttable}
\caption{\revise{Summary of open-source trip related data sets.}}
\begin{tabular}{@{}C{2.2cm}C{2.4cm}cccm{3.8cm}@{}}
\toprule
\textbf{Source}& \textbf{City} &\textbf{Type} & \textbf{\#~of Records} & \textbf{Time} & \multicolumn{1}{>{\centering\arraybackslash}m{3.8cm}}{\textbf{Data Features}}\\ \midrule
\midrule
Uber Pickups \cite{p668-gy46-22} & New York, United States& Pickups &20M & \begin{tabular}[c]{@{}c@{}}April -- September 2014,\\January -- June 2015 \end{tabular} & Timestamp and location.\\ \midrule
NYC TLC \cite{tlctrip} & New York, United States& Ride request &500M&January 2009 -- July 2021 &Passenger count, start time, end time, origin, destination, distance, fee, etc.\\ \midrule
DiDi GAIA \cite{gaiadata}& Haikou and Chengdu, China& Ride requests  &18M&\begin{tabular}[c]{@{}c@{}}November 2016,\\May -- October 2017\end{tabular}&Trip ID, fee, start time, end time, origin, destination, etc.\\ \midrule
Chicago Data Portal \cite{chicagoalll} & Chicago, United States & Ride requests &199M &January 2013 -- December 2021&Trip ID, taxi ID, start time, end time, fee, origin, destination, etc.\\ \midrule
T-drive \cite{tdrive, yuan2010t, yuan2011driving} & Beijing, China& Trajectories &15M&February 2008&Taxi ID, location, and timestamp.\\ \midrule
GeoLife \cite{geolifetraj, zheng2008learning, zheng2008understanding, zheng2010understanding} & Beijing, China& Trajectories &18,670&April 2007 -- August 2012 &Timestamp, location, transportation mode, etc. \\ \midrule
Beijing Taxi Trajectories \cite{lian2018one} & Beijing, China & Trajectories &129M&May 2019& Taxi ID, location, timestamp, speed, occupancy indicator, etc.\\ \midrule
DiDi GAIA \cite{gaiadata} & Chengdu and Xi'an, China &Trajectories &-\tnote{*}&-&\multicolumn{1}{>{\centering\arraybackslash}m{3.8cm}}{\textbf{-}}\\ \midrule
ECML PKDD 2015 \cite{portokaggle} & Porto, Portugal  & Trajectories  & 2M&July 2013 -- June 2014 &Trip ID, taxi ID, the sequence of locations of the trajectory, etc.\\ \midrule
CRAWDAD EPFL \cite{c7j010-22}  & San Francisco, United States & Trajectories  &11M&May -- June 2008& Taxi ID, location, timestamp, and occupancy indicator.\\\midrule
CRAWDAD Roma \cite{c7qc7m-22}  & Roma, Italy & Trajectories  &22M&February -- March 2014& Taxi ID, location, and timestamp.\\\midrule
Jeju Vehicular Trajectories \cite{y8vk-wj40-22}  & Jeju, South Korea & Trajectories  &8M& - & Vehicle ID, location, speed, lane, etc.\\\midrule
Grab-Posisi \cite{grabsource, huang2019grab}  &Singapore and Jakarta, Indonesia& Trajectories  &84,000&April 2019&Trajectory ID, location, timestamp, speed, etc.\\\midrule
Shanghai Taxi Trajectories \cite{2877-mk46-19, liu2020optimization}  &Shanghai, China& Trajectories  &61M&April 1, 2018&Taxi ID, location, timestamp, speed, occupancy indicator, driving status, etc.\\\midrule
Foursqure \cite{foursquare} & Global & Check-ins &33M&April 2012 -- September 2013 & Venue ID, timestamp, location, etc.\\ \midrule
Brightkite \cite{brightkite} & - & Check-ins &4M&April 2008 -- October 2010&User ID, timestamp, location, etc.\\ \midrule
Gowalla \cite{gowalla} & -  & Check-ins &6M&February 2009 -- October 2010&User ID, timestamp, location, etc.\\ \midrule
LTA of Singapore \cite{singaporedata} & Singapore & \begin{tabular}[c]{@{}c@{}}Real-time\\locations\end{tabular}  &-&-&Location. \\\midrule
Uber \cite{traveltimedata} &  Global &Travel times  &-&-&Origin, destination, travel time, etc.\\\bottomrule
\end{tabular}
\label{table:data}
\begin{tablenotes}
\item[*] \revise{``-'' indicates that the corresponding information is not attainable.}
\end{tablenotes}
\end{threeparttable}}
\end{table*}

\begin{table*}[t]
\centering
{\color{black}\begin{threeparttable}
\caption{Summary of the major simulators.}
\begin{tabular}{@{}cccm{9.6cm}@{}}
\toprule
\textbf{Name} & \textbf{Open-Source} & \begin{tabular}[c]{@{}c@{}}\textbf{Sample Applications}\\\textbf{in Ride-hailing} \end{tabular}& \multicolumn{1}{>{\centering\arraybackslash}m{9.6cm}}{\textbf{Description}} \\ \midrule
\midrule
 DiDi \cite{didisimulation} & \xmark &\cite{tang2021value}&An online simulation platform to evaluate matching and repositioning algorithms.  Evaluation is run with DiDi's real-world data.\\ \midrule
 AMoDeus \cite{ruch2018amodeus} & \cmark&\cite{ruch2020quantifying}&A tool that uses an agent-based transportation simulation framework to simulate arbitrarily configured mobility-on-demand systems with static/dynamic demand. It includes standard benchmark algorithms and a graphical user interface.\\ \midrule
 AMoD2 \cite{amod2} & \cmark&\cite{li2021optimal}&A high-capacity ride-sharing simulator. It uses map data and taxi data from Manhattan. Three matching algorithms and one simple rebalancing algorithm are implemented. \\ \midrule
 Mod-abm \cite{abm-1.0, abm-2.0} & \cmark&\cite{wen2017rebalancing}&A platform for simulation of large-scale mobility-on-demand operations. It supports city-level systems in any urban setting.\\ \midrule
 MATSim \cite{horni2016multi} & \cmark&\cite{tsao2019model}&An open-source framework for implementing large-scale agent-based transport simulations. It consists of several modules which can be combined or used in stand-alone mode, for demand-modeling, traffic flow simulation, re-planning; and a controller to iteratively run simulations, and methods to analyze outputs generated by the modules.\\ \midrule
 SUMO \cite{lopez2018microscopic} & \cmark&\cite{castagna2021multi,zhu2021shared}&A microscopic and space-continuous traffic simulation platform that is suitable for the generation, evaluation, and validation of traffic scenarios of real-world size. It supports road network customization and demand modeling.\\ \midrule
 CityFlow \cite{zhang2019cityflow} & \cmark&-\tnote{*}&A multi-agent RL environment for large scale city traffic scenario. It supports flexible definitions for road network and traffic flow. It provides faster simulation than SUMO.\\ \midrule
 STaRS \cite{ota2016stars} & \xmark &-&A simulation framework for analyzing diverse ride-sharing scenarios, considering the platforms' needs and constraints. Its real-time trip assignments utilize a linear optimization algorithm, efficient indexing, and parallelization for scalability.\\ \midrule
 STaRS+ \cite{mounesan2021fleet} & \xmark &-&A simulation framework based on an integer linear programming model, using heuristic optimization and a novel shortest-path caching scheme for scalability. It supports the simulation of full-city scale ride-sharing with meeting points.\\ \midrule
 UberSim \cite{khalil2022realistic} & \cmark & \cite{salman2023quantifying} &A digital twin transportation simulation model for Birmingham, Alabama, incorporating various transportation modes to analyze the impact of ride-hailing services on urban traffic. It supports policy learning through reinforcement learning.\\ \midrule
 NYC-Yellow-Taxi-V0 \cite{chaudhari2020learn} & \cmark&-&A multi-agent RL environment based on the OpenAI Gym environment \cite{openai-gym}, which offers a toolkit for developing and comparing RL-based ride-hailing fleet management algorithms.\\ \midrule
 SMART-eFlo \cite{liu2022smart} & \cmark & - &An integrated framework that combines the SUMO simulator with multi-agent Gym for reinforcement learning studies, enabling researchers to easily design traffic scenarios and implement RL algorithms for electric fleet management problems\\ \midrule
\end{tabular}
\label{table:simulator}
\begin{tablenotes}
\item[*] ``-'' indicates no sample application in ride-hailing using the corresponding simulator.
\end{tablenotes}
\end{threeparttable}}
\end{table*}



\section{Resources for empirical studies}
\label{sec:resource}
Real-world data are essential in studying ML-based ride-hailing planning, e.g., to train a proposed model.
Further, in order to deploy those proposed planning strategies in practice, it is necessary to utilize a ride-hailing service (process) simulator to validate their performance with real-world data.
\revise{In this section, we present publicly available %multiple
open-source real-world data sets and several related simulators in Sec.~\ref{sec:resource-data} and Sec.~\ref{sec:resource-simulator}, respectively.}

\subsection{Data}
\label{sec:resource-data}
Historical records of ride requests are the most frequently used data in the literature.
New York City Taxi and Limousine Commission \cite{tlctrip} provides more than 11 years of records of ride requests with many data fields, e.g., the pickup and drop-off timestamps, pickup and drop-off locations, trip fare, and ride distance.
About 19 million Uber's pick-up records obtained from this data set %\cite{tlctrip}
are summarised in \cite{p668-gy46-22}.
Chicago Data Portal \cite{chicagoalll} provides a large data set consisting of trip records with various data fields recorded from 2013.
Ride request data of Haikou and Chengdu in China are publicly available in the DiDi GAIA program \cite{gaiadata}.
Besides, there are some other data sets that consist of people's check-in records, e.g., those collected from Foursquare \cite{foursquare}, Brightkite \cite{brightkite}, and Gowalla \cite{gowalla}.
They are informative in revealing ride demands as the data have recorded riders' target places they had traveled to.
In addition to the records of ride requests, some data of the supply side are also available.
\revise{Trajectory data recorded by sampling drivers' locations at a certain frequency are also commonly used in the literature, including trajectories recorded in Beijing (in three different data sets: T-drive \cite{tdrive, yuan2010t, yuan2011driving}, GeoLife \cite{geolifetraj, zheng2008learning, zheng2008understanding, zheng2010understanding}, and Beijing Taxi Trajectories \cite{lian2018one}), Chengdu \cite{gaiadata}, Xi'an \cite{gaiadata}, Porto \cite{portokaggle}, Jakarta \cite{grabsource, huang2019grab}, Singapore \cite{grabsource, huang2019grab}, San Francisco \cite{c7j010-22}, Roma \cite{c7qc7m-22}, Jeju \cite{y8vk-wj40-22}, and Shanghai \cite{2877-mk46-19, liu2020optimization}.}
The Land Transport Authority (LTA) of Singapore \cite{singaporedata} publicizes many APIs for accessing various kinds of transport-related data, including monthly statistics of taxi supply and real-time coordinates of all taxis that are currently available for hire. %offering ride service. 
\revise{Besides, the travel times among different locations in various cities across the world can be obtained in \cite{traveltimedata}.}
The data sets mentioned above are summarized in Table.~\ref{table:data}.
\revise{Note that, although abundant public data sets are available for use, none of them record those ride requests that ended up unserved (because of, for example, excessive waiting time).}
If both the served and unserved ride requests are treated in the ride-hailing simulation, the simulation results would be more realistic than if only
the served ride requests are considered. 

In addition to the trip-related data sets mentioned above, there are several important public data sources that could provide good values
to ride-hailing planning studies.
\revise{OpenStreetMap \cite{openstreetmap} captures worldwide road networks, which can be easily accessed by APIs (e.g., OSMnx) using Python \cite{boeing2017osmnx}.}
Travel times and speeds information of road networks of several well known cities have been made available in \cite{traveltimedata}.
Finally, historical data of weather conditions can be found in \cite{weatherdata}.
Note that weather conditions can be considered in ride-hailing planning
as they could affect the traffic in a major way
\cite{he2019spatio}.
    
    
    
\subsection{Simulators}
\label{sec:resource-simulator}
With the aforementioned real-world data, proposed planning strategies can be evaluated through simulators.
DiDi \cite{didisimulation} makes public an industrial-level simulation platform recently, which is used in the KDD CUP 2020 \cite{kddcup20}. 
The platform evaluates submitted matching and repositioning strategies using real-world data from the GAIA program.
Wen \cite{abm-1.0} develops an agent-based modeling platform for simulating autonomous mobility-on-demand systems, which is later upgraded to a version with better scalability and extensibility \cite{abm-2.0}.
Based on \cite{abm-2.0}, a high-capacity on-demand ride-sharing simulator is proposed in \cite{amod2} with several built-in matching and repositioning algorithms.
Ruch et al.~\cite{ruch2018amodeus} have released an open-source simulator named AMoDeus for accurate and quantitative analysis of matching and repositioning algorithms in the ride-hailing system.
Examples of its usages can be found in \cite{fluri2019learning, carron2019scalable}.
AMoDeus is built upon the open-source microscopic multi-agent transportation simulation environment MATSim \cite{horni2016multi}.
MATSim is also used to evaluate ride-hailing planning strategies, a case study of which can be found in \cite{bischoff2016simulation}.
Besides MATSim, there are similar public traffic simulation tools, including SUMO \cite{lopez2018microscopic} and CityFlow \cite{zhang2019cityflow}.
A more detailed comparison and analysis of the performance of traffic simulation tools can be found in \cite{allan2015benchmark}.
\revise{STaRS is a scalable simulation framework that takes the needs and constraints of platforms into consideration \cite{ota2016stars}. 
STaRS+ extends STaRS to support the simulation of ride-sharing with meeting points \cite{mounesan2021fleet}.}
\revise{For those RL-based ride-hailing planning methods, their policies can be trained using CityFlow \cite{zhang2019cityflow}, UberSim \cite{khalil2022realistic}, NYC-Yellow-Taxi-V0 \cite{chaudhari2020learn}, or SMART-eFlo \cite{liu2022smart}.}
The aforementioned simulators are summarized in Table.~\ref{table:simulator}.

    