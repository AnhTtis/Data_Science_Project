\section{Introduction}

\revise{
Ride-hailing, powered by platforms (operators) such as DiDi and Uber, has amassed huge user bases in recent years.\footnote{DiDi: \url{https://www.didiglobal.com};  Uber: \url{https://www.uber.com}.}
As has been reported, DiDi has more than $550$ million registered users while Uber has over $5$ million drivers worldwide \cite{jye19transport, iqbal22uber}.
In {ride-hailing}, the platforms manage intelligently their vehicle resources to fulfil riders' traveling requests.
Compared to the traditional street-hailing mode in which the drivers operate all on their own, ride-hailing is more efficient.
In street-hailing, without any intelligent strategies, an idle driver typically would just pick up the first rider s/he runs into.
But with ride-hailing, the platforms can leverage advanced planning algorithms to manage their fleets to achieve higher efficiency in terms of metrics such as total vehicle miles traveled \cite{ma2017morning}, vehicle capacity utilization rate \cite{cramer2016disruptive}, and rider waiting time \cite{feng2020we, hyland2020operational}.
}
\revise{
	Ride-hailing has greatly reduced the need to own a private vehicle due to its high efficiency in transporting people door-to-door, %traveling services,
 which consequently would also
lessen traffic congestion, energy consumption, and environmental pollution \cite{kpmg20smart, taiebat2022sharing}.
}

\begin{figure*}[t]
	\centering
	\captionsetup{justification=centering}
	\includegraphics[width=0.75\linewidth]{figs/pipeline_.pdf}
	\caption{The pipeline of a ride-hailing system.}
	\label{fig:pipeline}
\end{figure*}

A ride-hailing system consists of three types of entities: the platform, drivers, and riders. 
%The interactions between them as well as 
\revise{The operation pipeline of a ride-hailing system and the interaction between the entities are depicted in Fig.~\ref{fig:pipeline}.}
Riders and drivers can enter or exit the ride-hailing system at any time.
\revise{Given the information of riders and drivers, e.g., the origins and destinations of the riders and the real-time locations of the drivers, the ride-hailing platform can make a planning decision accordingly, which includes the \emph{matching} between drivers and riders and the \emph{repositioning} of the vehicles \cite{holler2018deep, holler2019deep, qin2019deep, jin2019coride}. }
% An example of ride-hailing matching and repositioning in Manhattan, New York is shown in Fig.~\ref{fig:planning}.

\begin{figure}[h]
	\centering
	\subfigure[\revise{Solo-ride}]{
		\includegraphics[width=.4\linewidth]{figs/solo_3.pdf}
	}        
	\subfigure[\revise{Shared-ride}]{
		\includegraphics[width=.4\linewidth]{figs/shared_3.pdf}
	}
	\subfigure[\revise{Single-hop}]{
		\includegraphics[width=.4\linewidth]{figs/single-hop_3.pdf}
	}        
	\subfigure[\revise{Multi-hop}]{
		\includegraphics[width=.4\linewidth]{figs/multi-hop_3.pdf}
	}
	\subfigure{
	    \centering
		\includegraphics[width=0.8\linewidth]{figs/legend_.pdf}
	}
	\caption{\revise{Solo-ride, shared-ride, single-hop, and multi-hop trips.}}
	\label{fig:matching-type}
\end{figure}

Specifically, matching is to assign available vehicles to riders who have yet to be picked up. 
\revise{Different types of matching are 
offered by existing ride-hailing platforms \cite{tafreshian2020frontiers}.
On the one hand, depending on whether the rider is to travel alone or not, a rider's trip can be a \emph{shared-} or \emph{solo-ride}.
\revise{In the shared-ride mode, multiple riders who share similar routes
can be matched with the same vehicle %in the meantime
and thus travel together to their destinations.}
In contrast, riders travel alone in solo-ride mode.
\revise{On the other hand, a rider's trip can be \emph{multi-hop} if the rider gets 
transferred from one vehicle to another during the trip; otherwise, it is \emph{single-hop}.}
A visualized explanation of the {shared-ride}, {solo-ride}, multi-hop, and single-hop trips is given in Fig.~\ref{fig:matching-type}.
Accordingly, the matching can be divided into four types: one-to-one (i.e., one hop and solo-ride) \cite{agatz2011dynamic, lloret2017peer, ta2017efficient, long2018ride}, one-to-many %one-to-many
(i.e., single-hop and shared-ride) \cite{stiglic2016making, regue2016car2work, bei2018algorithms, tamannaei2019carpooling, noruzoliaee2022one}, many-to-one (i.e., multi-hop and solo-ride) \cite{masoud2017real}, and many-to-many (i.e., multi-hop and shared-ride) \cite{agatz2010sustainable, masoud2017decomposition}.}

\begin{figure}[h]
	\centering
	\subfigure[Case 1 (w/o repositioning)]{
		\includegraphics[width=0.98\linewidth]{figs/c1.pdf}
	}        
	\subfigure[Case 2 (w/ repositioning)]{
		\includegraphics[width=0.98\linewidth]{figs/c2.pdf}
	}
    \caption{A scenario that consists of three vehicles, three ride requests, and two time steps. In Case 1 (w/o repositioning), the vehicles stay still and wait for new riders at $t=1$.
    \revise{The three ride requests appear in the region circled with red at $t=2$.
    At $t=2$, the riders cannot be served because the vehicles are too far away from the riders.} In contrast, in Case 2 (w/ repositioning), the vehicles relocate to the region at $t=1$ in advance. At $t=2$, the riders can be served.}
	\label{fig:repositioning}
\end{figure}

The purpose of repositioning is to proactively relocate vehicles to certain positions so that these vehicles become better prepared for serving the upcoming riders.
\revise{The repositioning decisions have short-term effects on the driver availability.
%and thus the matching outcomes. 
In the long run, 
repositioning can substantially change the distribution of drivers across the city.
Thus, repositioning can have a critical impact on how well the potential riders can be matched in terms of certain metrics, e.g., the pickup rate and the waiting time.
\revise{Examples of repositioning are shown in Fig.~\ref{fig:repositioning}.}
We can see from these examples that, with an appropriate repositioning strategy, more riders can be served than if drivers simply stay put while waiting for new riders.}







Planning is the most important task for a ride-hailing platform \cite{holler2018deep, holler2019deep, qin2019deep}.
However, it is a computationally intractable task \cite{alonso2017demand}.
\revise{Specifically, when the size of the planning problem is non-trivial, i.e., the numbers of drivers and riders respectively are large, exact optimization algorithms (e.g., those based on branch-and-cut \cite{liu2015branch}, branch-and-price \cite{qu2015branch, parragh2015dial}, column generation \cite{parragh2012models}, dynamic programming \cite{hame2015maximum}, and Benders decomposition \cite{cortes2010pickup}) all turn out to be infeasible due to their intolerably long running time \cite{ordonez2017dynamic, molenbruch2017typology}.
For example, even when dealing with a small instance that consists of two drivers, six riders, and a city of $1000 \times 1000$ squared distance units, a solution based on the traditional branch-and-bound method takes more than thirty minutes of computation time on an average machine 
\cite{cortes2010pickup}, which is unacceptable for the platforms, drivers, or riders in practice.
Instead, to achieve near-optimal performance of ride-hailing planning in real-time, approximate solutions that can solve problems of real-world sizes, e.g., those using heuristics \cite{luo2011online, chassaing2016els, ritzinger2016dynamic, dutta2018hashing, SHARIFAZADEH2022128, de2022influence, zhou2022scalable, peng2022investigating} and  machine learning (ML) \cite{nguyen2018policy, al2019deeppool, shi2021learning, oda2021equilibrium, haliem2021distributed}, have been proposed.}

\revise{Several literature reviews of ride-hailing planning have been conducted over the past decade.
Some of them focus on matching between drivers and riders, and propose exact or heuristic optimization solutions \cite{agatz2012optimization, furuhata2013ridesharing, mourad2019survey, yan2020dynamic, tafreshian2020frontiers, martins2021optimizing}.
\revise{However, these solutions skip over the vehicle repositioning problem which is a critical component in a ride-hailing system.}
Besides, the superiority of ML techniques in dealing with complex planning tasks has led to an increased number of ML-based methods for ride-hailing planning (e.g., \cite{holler2018deep, lin2018efficient, tang2019deep, holler2019deep, qin2019deep}), which are not covered in the reviews mentioned above. %the aforementioned reviews.
%overlooked by the surveys aforementioned.
\revise{There exist several tutorials and surveys that concentrate on
%pay attention to the field of
ML-based transportation \cite{veres2019deep, farazi2020deep, haghighat2020applications, chakraborty2020review}, but they cover only a limited subset of the literature on
ride-hailing planning.}
Qin et al.~\cite{qin2021reinforcement, qin2022reinforcement} focus on a single specific branch of ML, reinforcement learning (RL), in the context of ride-hailing.
%They summarize RL-based ride-hailing solutions,  
%%summarize some works of ride-hailing planning that leverage reinforcement learning (RL) techniques.
%but RL is only one specific branch of ML.
Our survey presents a more comprehensive overview of the literature on ML-based ride-hailing planning.}
To sum up, our work has the following contributions:



\begin{enumerate}
	
	\item \emph{New taxonomy.} 
	\revise{To set the stage for a clear review, we create a %new
	taxonomy for the literature of ML-based ride-hailing planning. 
 First, we categorize the literature into groups according to their tasks of the planning, including matching, repositioning, and joint matching and repositioning.
Second, we divide the works in each group based on their solution scheme into two divisions, including \emph{collective planning} and \emph{distributed planning}. 
	In the collective scheme, the planning decisions of all drivers and riders are determined jointly. }
%	On the other hand, distributed scheme differs from the collective one in that, in their solutions, 
	\revise{Whereas in the distributed scheme, each vehicle makes its own decisions of planning without coordinating with other vehicles.
%	The collective planning scheme globally optimizes the decisions and hence benefits the performance of certain objectives, e.g., platforms' revenues \cite{he2019spatio}.
	Existing methods of ride-hailing planning are either collective or distributed.}
	\revise{By treating collective and distributed plannings separately, we can understand better their different decision making processes and the corresponding advantages.
 At last, in each division, we further classify the works based on their different ML techniques and foci and discuss them in detail accordingly.}
%	That is, their actions are not decided in a collective manner. 
%	More details will be presented in the following section.
	
	\item \emph{Comprehensive review.} 
	\revise{As mentioned above, existing surveys did not include a comprehensive coverage of the works on ML-based ride-hailing planning.}
	\revise{We fill this gap by providing an extensive overview on %that can help readers better understand
 the latest developments of this topic. }
	
	\item \emph{Abundant resources for empirical studies.} 
	\revise{We examine many public benchmark datasets for ride-hailing planning that are critical for developing and evaluating data-driven ML-based solutions.
	Tools that can be used to simulate the ride-hailing process are important for performance evaluation.
	Thus, we also review the relevant open-source simulators in this survey.}
	
	\item \emph{Future directions.} 
%	After looking into the state-of-the-art literature in ML based ride-hailing planning, 
	\revise{We suggest several promising research problems as future directions, which include drivers' and riders' heterogeneity, malicious behaviors, and fairness in ML-based ride-hailing planning.}
	Besides, emerging ML techniques that have been overlooked in ride-hailing planning are discussed.
% 	which include topics from three different aspects, emerging ML techniques that have been overlooked in ride-hailing planning, heterogeneous drivers and riders that can leads to higher planning efficiency, and ethical principals of ride-hailing planning.
\end{enumerate}

The rest of this survey is organized as follows.
\revise{The background of ride-hailing planning is given in Sec.~\ref{sec:background}.
%Following the taxonomy mentioned above, 
%A detailed survey of the literature is presented in
Sec.~\ref{sec:review} presents a detailed survey of the literature.
Resources useful for empirical studies of ride-hailing planning are reviewed in Sec.~\ref{sec:resource}.}
Finally, we point out some future directions and draw our conclusion in Sec.~\ref{sec:future} and Sec.~\ref{sec:conclusion}, respectively.


