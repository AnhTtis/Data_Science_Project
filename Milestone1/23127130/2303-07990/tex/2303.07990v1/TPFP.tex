%16 229 entries
%Tp count 6224 (both) (38.4%)
%Tp for product names 10189 (62.8%)
\section{}
In order to test the effectiveness of our approach, we ran the program on every entries from 2020, a total of 16 229 distinct CVE entries. We first excluded entries for which no CPE was present. For the remaining entries, we applied our method and extracted any vendor or product names from the summary. These names were then compared to the CPE.

We counted an entry as a True Positive (TP) if the name of the vendor and\textbackslash or that of the product, as recorded in the CPE, are both are present in the summary. This means that our method would have correctly extracted the name of the software from the summery, thus  informing  security professionals of whether or not  the vulnerability  is of concern to them, despite the absence of a CPE. Conversely, we considered that a result is a False Positive  (FP) if the summery contained a pair of vendor and product names present in the CPE dictionary that did not appear in the CPE entry for that CVE.

An examination of our results showed that they were cluttered by short many one or two letter product and vendor names. For example, many products contain the letter ‘X’ as a stand alone word. This letter may also occur alone in the text of a CVE’s summary, a situation that led a to a large number of false positives. We thus decided to elide one and two letter product and vendor names from our experiment. 

After eliding short names, applying the method to 16,229 entries resulted in only 830 false positives (less than 5\%). Furthermore, in many cases, even a false positive provides useful information about the underlying vulnerability, in the sense that it provides additional precision with respect to the software in which the vulnerability occurs. 

For example, there are 19 entries for which the vendor is recorded as ‘Microsoft’ and the product is ‘hyper’. These refer to vulnerabilities in a Windows component called Hyper-V Virtual, and are reported as a false positives because the CPEs for the corresponding vulnerabilities record the product as Windows. Since Hyper-V Virtual can be disabled to reduce the attack surface if it is unused, even this false positive provides useful information that allows security  professional to quickly determine if a given CVE report is relevant to his organization.  The same thing occurs with several other Windows components, as well as with other systems including Android and Kubernetes.

In several other cases, a vulnerability in one product allowed the exploitation of data maintained by another product by the same vendor, if both products were run simultaneously and interacted during their execution. Since the summery  contained the name of the vendor as well as that of both products, while the CPE only listed the product in which the vulnerability actually occurred, this resulted in a FP. However, in this case also, our method provides useful information about whether or not the vulnerability is exploitable on a given system.

TP

It is important to stress that this is a considerable over-count of the potential of false positives. 