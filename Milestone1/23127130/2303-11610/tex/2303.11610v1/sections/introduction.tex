%%%%%%%%%%%%%%%%%%%%%%%%%%%%%%%%%%%%%%%%%%%%%%%%%%%%%%
%%%%%%%%%%%%%%%%%%%%%%%%%%%%%%%%%%%%%%%%%%%%%%%%%%%%%%
%%%%%%%%%%%%%%%%%%%%%%%%%%%%%%%%%%%%%%%%%%%%%%%%%%%%%%
\section{Introduction}\label{sec:intro}

As humans, we are fairly skilled in organising new visual knowledge (novelty) into homogeneous groups, even when we do not know what we are observing. However, machines cannot perform this task satisfactorily without our supervision. The challenge here is mainly in the formulation of discriminative latent representations of the real world and in the quantification of the uncertainty when the novelty is observed \cite{han2019learning,zhong2021neighborhood,zhao2022novel}. The work of Han et al.~\cite{han2019learning} pioneered the formulation of the Novel Class Discovery (NCD) problem by defining it as the task that aims to classify the samples of an unlabelled dataset into different classes, i.e.~the \textit{novel samples}, by exploiting the knowledge of a set of labelled samples, i.e.~the \textit{base samples}. Note that the classes in the labelled and unlabelled datasets are disjoint.

NCD has been explored in the 2D image domain for classification \cite{han2019learning,fini2021unified,zhong2021neighborhood} and later for semantic segmentation \cite{zhao2022novel}. In particular, Zhao et al.~\cite{zhao2022novel} presented the first approach for NCD for 2D semantic segmentation. Two key assumptions were made by the authors. Firstly, at most one novel class is allowed in each image. Secondly, the new class belongs to a foreground object that can be found via saliency detection (e.g.~a man on a bicycle, where the bicycle is the novel class). Thanks to these assumptions, the authors could pool the features of each image into a single latent representation and cluster the representations of the whole dataset to discover clusters of novel classes. We argue that these two are important constraints that are not applicable to generic 3D data, in particular to point clouds captured with LiDAR sensors in real-world city-scale scenarios. One point cloud can contain more than one novel class, and the saliency for 3D data cannot be leveraged in the same way as that for 2D data. Although they are both related to the attraction of human fixations, 3D saliency is more related to the regional importance of 3D surfaces rather than the foreground/background distinction~\cite{Ran2021}. Our motivation in exploring NCD for the 3D setting is mainly driven by addressing these shortcomings.

In this paper, we explore the new problem of NCD for 3D point cloud semantic segmentation (see Fig.~\ref{fig:teaser}).
Given a partially human-annotated dataset, we jointly learn base and novel semantic classes by clustering unlabelled points with similar semantic features.
We adapt the method of Zhao et al.~\cite{zhao2022novel} (Entropy-based Uncertainty Modeling and Self-training - EUMS) for point cloud data and use it as our baseline.
We go beyond their formulation and, inspired by~\cite{caron2020unsupervised}, we integrate batch-level (online) clustering in our method and update prototypes during training in order to make clustering computationally tractable.
Cluster assignments are then used as training pseudo-labels.
We also exploit over-clustering to achieve a higher clustering accuracy as in EUMS.
Because point clouds contain multiple semantic classes, we cannot guarantee that all the classes appear in the point clouds within each batch, some will be missing.
Therefore, we design a queuing strategy to store important features over training time, which will be used for pseudo-labelling in the case of missing classes.
We further introduce a strategy for exploiting the pseudo-label uncertainty to promote the creation of reliable prototypes that we then exploit to produce higher-quality pseudo-labels.
Lastly, we produce two augmented views of the same point cloud and impose pseudo-label consistency amongst them.
We evaluate our approach on SemanticKITTI \cite{behley2019semantickitti, geiger2012cvpr, behley2021ijrr} and SemanticPOSS \cite{pan2020semanticposs}, introducing an evaluation protocol for NCD and point cloud segmentation that can be adopted in future works.
We empirically show that our approach largely outperforms our baseline in both datasets.
We also perform an extensive ablation study to demonstrate the importance of the different components of our method.

\vspace{1mm}
\noindent To summarise, our contributions are:
\setlist{nolistsep}
\begin{itemize}
    \item We address the new problem of NCD for 3D semantic segmentation;
    \item We show that the transposition of the only existing NCD method for 2D semantic segmentation \cite{zhao2022novel} to 3D data is suboptimal;
    \item We present a new method for NCD based on online clustering and uncertainty estimation, which we name it NOPS (NOvel Point Segmentation);
    \item We introduce a new evaluation protocol to assess the performance of NCD for 3D semantic segmentation.
\end{itemize}

