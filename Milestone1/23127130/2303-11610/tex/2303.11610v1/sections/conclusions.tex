\section{Conclusions}\label{sec:conclusions}

We explored the new problem of novel class discovery for 3D point cloud segmentation.
Firstly, we transposed the only NCD method for 2D semantic segmentation to 3D point cloud data, and experimentally found that it has several limitations.
We discussed that extending 2D NCD approaches to 3D data (point clouds) is not trivial because the assumptions made for 2D data are not easily transferable to 3D.
Secondly, we presented \ourmethod, to tackle NCD for point cloud segmentation by using online clustering and exploiting uncertainty quantification to produce pseudo-labels for the novel points.
Lastly, we introduced a new evaluation protocol to asses the performance of NCD for point cloud segmentation.
Experiments on two different segmentation dataset showed that \ourmethod outperforms the compared baselines by a large margin.
Future research directions could investigate the extension of our method when base annotations are fewer and/or weakly labelled.

\noindent \textbf{Limitations}
\ourmethod limitations include the prior knowledge on the number of novel classes $C_n$ to discover.
This could be a limitation when $C_n$ is not known a-priori and novel classes may appear in an incremental manner.
We believe that a solution may be to learn novel classes incrementally.
Another limitation is the loss we use to handle class unbalancing.
More recent techniques to handle this drawback could be further explored.

