\section{Dataset splits for 3D NCD} \label{sec:dataset_splits}

To evaluate the performances of NOPS, we divide SemanticKITTI \cite{behley2019semantickitti} and SemanticPOSS \cite{pan2020semanticposs} into four different splits. We name these splits as SemanticKITTI-$n^i$ and SemanticPOSS-$n^i$, respectively, where $n$ is the number of novel classes contained in each split and $i$ indexes the split. 
In each set, the novel classes and the base classes correspond to unlabelled and labelled points.  
These splits are selected based on two principles, i.e.~balancing the distribution of the novel classes in each split, and including semantic relationships between base and novel classes within the same split.
See details about the splits in Fig.~\ref{fig:dataset_splits}.
The first principle allows us to avoid the case in which the most frequent novel class affects the other classes, thus in turn affecting the learning of the unsupervised points. 
The second principle encourages the model to exploit the supervised knowledge over some base classes to discover the unsupervised novel classes, as in the case of the novel class \textit{rider} in SemanticPOSS-$3^3$, whose discovery can be facilitated by the presence of the class \textit{bike}, that is considered as base class in this specific split.

\begin{figure}[t]
\centering
    \begin{tabular}{c}
        \begin{overpic}[width=0.95\columnwidth]{images/splits_hist/hist_SemanticKITTI.pdf}
        \end{overpic} \\
        \begin{overpic}[width=0.95\columnwidth]{images/splits_hist/hist_SemanticPOSS.pdf}
        \end{overpic}\\
    \end{tabular}
    \vspace{-.3cm}
    \caption{Histograms representing the number of points belonging to each class in SemanticKITTI \cite{behley2019semantickitti} and SemanticPOSS \cite{pan2020semanticposs}. Each class has been assigned the colour of the split in which it has to be considered novel (unlabelled).}
    \label{fig:dataset_splits}
\end{figure}