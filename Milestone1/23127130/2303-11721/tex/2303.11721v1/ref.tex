\newpage
\fontsize{13}{14}\selectfont{
\section*{References}
\justify 
\begin{itemize}[label={},itemindent=-3em,leftmargin=3em]
\setlength\itemsep{-0.5em}

\item \label{AK2018}
Armstrong, T. B., \& Kolesár, M. (2018). Optimal Inference in a Class of Regression Models. \emph{Econometrica, 86}(2), 655-683.

\item \label{ref:11}
Athey, S., \& Imbens, G. (2016). Recursive partitioning for heterogeneous causal effects. \emph{Proceedings of the National Academy of Sciences, 113}(27), 7353–7360.

\item \label{ref:12}
Athey, S., Tibshirani J., \& Wager S. (2019). Generalized random forests. \emph{Annals of Statistics, 47}(2), 1148–1178.

\item \label{BBRW2009}
Battistin, E., Brugiavini, A., Rettore E., \& Weber, G. (2009). The retirement consumption puzzle: Evidence from a regression discontinuity approach. \emph{American Economic Review, 99}(5), 2209-2226.

\item \label{B1999}
Black, S. E. (1999). Do better schools matter? Parental valuation of elementary education. \emph{Quarterly Journal of Economics, 114}(2), 577–599. 

\item \label{ref:23}
Breiman, L. (2001). Random forests. \emph{Machine Learning, 45}(1), 5–32.

\item \label{CCFT2019}
Calonico, S., Cattaneo, M. D., Farrell, M. H., Titiunik, R. (2019). Regression discontinuity designs using covariates. \emph{ Review of Economics and Statistics, 101}(3), 442–451.
 
\item \label{ref:7240-1}
Calonico, S., Cattaneo, M. D., \& Titiunik R. (2014). Robust nonparametric
confidence intervals for regression-discontinuity designs. \emph{Econometrica, 82}(6), 2295–
2326.
 
\item \label{ref:8}
Cattaneo, M. D., \& Escanciano, J. C. (Eds.). (2017). \emph{Regression Discontinuity Designs: Theory and Applications} (Vol. 38). Emerald Publishing Limited.

\item\label{CL2018}
Choi, J., \& Lee, M. (2018). Regression discontinuity with multiple running variables allowing partial effects. \emph{Political Analysis, 26}(3), 258-274.

\item \label{CM2014}
Clark, D., \& Martorell, P. (2014). The signaling value of a high school diploma. \emph{Journal of Political Economy, 122}(2), 282–318.
 
\item \label{E2014}
Efron, B. (2014). Estimation and accuracy after model selection. \emph{Journal of the
American Statistical Association, 109}(507), 991-1007.
 
\item \label{FTAW2021}
Friedberg, R., Tibshirani, J., Athey, S. \& Wager, S. (2021). Local linear forests. \emph{Journal of Computational and Graphical Statistics, 30}(2), 503-517.
 
\item \label{ref:5}
Friedman, J. H. (2001). Greedy function approximation: A gradient boosting machine.
\emph{Annals of Statistics, 29}(5), 1189–1232.
 
\item \label{ref:7240-2}
Ghosal, I., \& Hooker, G. (2021). Boosting random forests to reduce bias; One-step boosted forest and its variance estimate. \emph{Journal of Computational and Graphical Statistics, 30}(2), 493-502.

\item \label{ref:14}
Gu, X. (2018). Estimation of heterogeneous treatment effects in regression discontinuity designs. \emph{Essays in Econometrics} [Unpublished doctoral dissertation]. University of Texas at Austin.
 
\item \label{ref:2}
Hahn, J., Todd, P., \& van der Klaauw, W. (2001). Identification and estimation of treatment
effects with a regression-discontinuity design. \emph{Econometrica, 69}(1), 201-209.

\item \label{H1968}
Hájek, J. (1968). Asymptotic normality of simple linear rank statistics under alternatives. \emph{Annals of Mathematical Statistics, 39}(2), 325–346.
 
\item \label{ref:21}
Hastie T, Tibshirani R, Friedman J. (2016). Kernel Smoothing Methods. In: Hastie T, Tibshirani R, Friedman J, editors. \emph{The Elements of Statistical Learning: Data Mining, Inference, Prediction} (2nd ed.). Springer Series in Statistics. New York, NY: Springer. p. 200.
 

 
\item \label{ref:6}
Heckman, J. J., Lalonde, R. J., \& 
Smith, J. A. (1999). The economics and econometrics of active labor market programs. In O. Ashenfelter \& D. Card (Eds.), \emph{Handbook of Labor Economics} (pp. 1865–2097). Elsevier Science.
 

 
\item \label{H1948}
Hoeffding, W. (1948). A class of statistics with asymptotically normal distribution. \emph{Annals of Mathematical Statistics, 19}(3), 293–325.
 

 
\item \label{ref:4}
Imbens, G. W., \& Kalyanaraman, K. (2012). Optimal bandwidth choice for the regression discontinuity estimator. \emph{Review of Economic Studies, 79}(3), 933-959.
 

 
\item \label{ref:7}
Imbens, G., \& Lemieux, T. (2008). Regression discontinuity designs: A guide to practice.
\emph{Journal of Econometrics, 142}(2), 615-635.
 

 
\item \label{IW2019}
Imbens, G., \& Wager, S. (2019). Optimized regression discontinuity designs.
\emph{Review of Economics and Statistics, 101}(2), 264–278.


\item\label{JL2004}
Jacob, B. A., \& Lefgren, L. (2004). Remedial education and student achievement: A regression-discontinuity analysis. \emph{Review of Economics and Statistics, 86}(1): 226–244.

 
\item \label{ref:20}
Keele, L. J., \& Titiunik, R. (2015) Geographic boundaries as regression discontinuities.
\emph{Political Analysis, 23}(1), 127–155.
 

 
\item \label{L2008}
Lee, D. S. (2008). Randomized experiments from non-random selection in US House elections.
\emph{Journal of Econometrics, 142}(2), 675–697.
 

 
\item \label{ref:3}
Lee, D. S., \& Lemieux, T. (2010). Regression discontinuity designs in economics. \emph{Journal of
Economic Literature, 48}(2), 281-355.

\item\label{M2008}
McCrary, J. (2008). Manipulation of the running variable in the regression discontinuity design: A density test. \emph{Journal of Econometrics, 142}(2), 698-714.

 
\item \label{ref:22}
Matsudaira, J. D. (2008). Mandatory summer school and student achievement. \emph{Journal of
Econometrics, 142}(2), 829–850.
 

 
\item \label{N1923}
Neyman, J. (1923). Sur les applications de la théorie des probabilités aux experiences agricoles: Essai des principes. \emph{Roczniki Nauk Rolniczych, 10}, 1–51.
 

 
\item \label{PWM2011}
Papay, J. P., Willett, J. B., \&  Murnane, R. J. (2011). Extending the regression discontinuity approach to multiple assignment variables. \emph{Journal of
Econometrics, 161}(2), 203–207.
 

 
\item \label{ref:9}
Porter, J. (2003). \emph{Estimation in the regression discontinuity model}. Working Paper, University
of Wisconsin.
 

 
\item \label{RR2012}
Reardon, S. F. \&  Robinson, J. P. (2012).  Regression discontinuity designs with multiple rating-score variables. \emph{Journal of Research on Educational Effectiveness, 5}(1), 83–104.
 

 
\item \label{R1974}
Rubin, D. B. (1974).  Estimating causal effects of treatments in randomized and nonrandomized studies. \emph{Journal of Educational Psychology, 66}(5), 688–701.


\item \label{SWB2012}
Schmieder, J. F., von Wachter, T., \& Bender, S. (2012). The effects of extended unemployment insurance over the business cycle: Evidence from regression discontinuity estimates over 20 years. \emph{Quarterly Journal of Economics, 127}(2), 701–752.
 
\item \label{SL2009}
Sexton, J. \& Laake, P. (2009). Standard errors for bagged and random forest estimators. \emph{Computational Statistics and Data Analysis 53}(3), 801-811.


 
\item \label{ref:1}
Thistlethwaite, D. L., \& Campbell, D. T. (1960). Regression-discontinuity analysis: An alternative to the ex-post facto experiment. \emph{Journal of Educational Psychology, 51}(6), 309-317.
 

 
\item \label{ref:13}
Wager, S. \& Athey, S. (2018). Estimation and inference of heterogeneous treatment effects
using random forests. \emph{Journal of the American Statistical Association, 113}(523), 1228-1242.
 

 
\item \label{WHE2014}
Wager, S., Hastie, T., \& Efron, B. (2014). Confidence intervals for random forests:
The jackknife and the infinitesimal jackknife. \emph{Journal of Machine Learning Research, 15}(48), 1625-1651.

\item \label{Z2012}
Zajonc, T. (2012). Regression discontinuity design with multiple forcing variables.  \emph{Essays on Causal Inference for Public Policy} [Unpublished doctoral dissertation]. Harvard University.
\end{itemize}
}
