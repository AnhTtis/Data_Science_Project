\section{Clustering of tiles}
\label{sec:clusters}

As discussed in \secref{sec:discussion}, tilings by the hat polykite
are composed of certain clusters of tiles.  These clusters can be 
used to define simplified tile shapes that we call \textit{metatiles}.
The metatiles inherit matching rules from the boundaries of the hats
that they contain. Furthermore, through a set of substitution rules 
they form larger, combinatorially equivalent
supertiles that fit together following the same matching rules.  In
this section, we give a precise definition of how tiles are assigned
to clusters, and a computer-assisted proof by case analysis that
this assignment does result in the clusters claimed, fitting together
in accordance with the matching rules given.  

\begin{figure}[ht!]
\captionsetup{margin=0pt}%
\begin{center}
\subfloat[Cluster $T$]{%
  \begin{tikzpicture}[x=5mm,y=5mm]
  \draw[\colcluster,ultra thick] \vcoords{2}{-2} -- \vcoords{8}{-2} --
    \vcoords{2}{4} -- cycle;
  \tileA{60}{0}{0}{$T_1$};
  \draw \vcoords{2}{3} -- \vcoords{2}{4};
  \draw \vcoords{7}{-1} -- \vcoords{8}{-2};
  \markpt{2}{-2};
  \markpt{8}{-2};
  \markpt{2}{4};
  \vctxt{0}{2}{$B^+$};
  \vctxt{5}{-3}{$A^-$};
  \vctxt{5}{2}{$A^-$};
\end{tikzpicture}%
} \qquad \subfloat[Cluster $H$]{%
\begin{tikzpicture}[x=5mm,y=5mm]
  \draw[\colcluster,ultra thick] \vcoords{2}{0} -- \vcoords{4}{-2} --
    \vcoords{12}{-2} -- \vcoords{12}{0} -- \vcoords{4}{8} --
    \vcoords{2}{8} -- cycle;
  \tileAr{240}{0}{2}{$H_1$};
  \tileA{60}{-1}{2}{$H_2$};
  \tileA{60}{1}{1}{$H_3$};
  \tileA{300}{0}{1}{$H_4$};
  \markpt{2}{0};
  \markpt{4}{-2};
  \markpt{10}{-2};
  \markpt{12}{-2};
  \markpt{12}{0};
  \markpt{6}{6};
  \markpt{4}{8};
  \markpt{2}{8};
  \markpt{2}{2};
  \vctxt{3}{-2.5}{$X^+$};
  \vctxt{8}{-3}{$B^-$};
  \vctxt{11.5}{-3}{$X^-$};
  \vctxt{14}{-2}{$X^+$};
  \vctxt{9}{4}{$B^-$};
  \vctxt{5.5}{7.5}{$X^-$};
  \vctxt{3}{9.5}{$X^+$};
  \vctxt{0}{5.5}{$A^+$};
  \vctxt{1}{1.5}{$X^-$};
\end{tikzpicture}%
} \\ \subfloat[Cluster $P$]{%
\begin{tikzpicture}[x=5mm,y=5mm]
  \draw[\colcluster,ultra thick] \vcoords{0}{0} -- \vcoords{4}{-4} --
    \vcoords{12}{-4} -- \vcoords{8}{0} -- cycle;
  \tileA{0}{0}{0}{$P_1$};
  \tileA{60}{1}{0}{$P_2$};
  \draw \vcoords{11}{-3} -- \vcoords{12}{-4};
  \markpt{0}{0};
  \markpt{2}{-2};
  \markpt{4}{-4};
  \markpt{6}{-4};
  \markpt{12}{-4};
  \markpt{10}{-2};
  \markpt{8}{0};
  \markpt{6}{0};
  \vctxt{0.5}{-1.5}{$L$};
  \vctxt{2.5}{-3.5}{$X^-$};
  \vctxt{5}{-5}{$X^+$};
  \vctxt{9}{-5}{$A^-$};
  \vctxt{11.5}{-2.5}{$L$};
  \vctxt{9.5}{-0.5}{$X^-$};
  \vctxt{7}{1}{$X^+$};
  \vctxt{3}{1.5}{$B^+$};
\end{tikzpicture}%
} \qquad \subfloat[Cluster $F$]{%
\begin{tikzpicture}[x=5mm,y=5mm]
  \draw[\colcluster,ultra thick] \vcoords{0}{0} -- \vcoords{4}{-4} --
    \vcoords{10}{-4} -- \vcoords{10}{-2} -- \vcoords{8}{0} -- cycle;
  \tileA{0}{0}{0}{$F_1$};
  \tileA{60}{1}{0}{$F_2$};
  \markpt{0}{0};
  \markpt{2}{-2};
  \markpt{4}{-4};
  \markpt{6}{-4};
  \markpt{8}{-4};
  \markpt{10}{-4};
  \markpt{10}{-2};
  \markpt{8}{0};
  \markpt{6}{0};
  \vctxt{0.5}{-1.5}{$L$};
  \vctxt{2.5}{-3.5}{$X^-$};
  \vctxt{5}{-5}{$X^+$};
  \vctxt{7.5}{-5}{$L$};
  \vctxt{9.5}{-5}{$X^-$};
  \vctxt{11.5}{-3.5}{$F^+$};
  \vctxt{9.5}{-0.5}{$F^-$};
  \vctxt{7}{1}{$X^+$};
  \vctxt{3}{1.5}{$B^+$};
\end{tikzpicture}%
}%
\end{center}
\caption{The four clusters}
\label{fig:tileaclusters}
\end{figure}

The clusters and their associated metatiles
are shown in Figure~\ref{fig:tileaclusters}.
Each metatile is a convex polyiamond outlined in
\textcolor{\colcluster}{\textbf{lime}}; its hats are overlaid, and 
each is given a unique label.
The union of the polykites in a cluster approximates the shape of
its metatile, but with some triangular indentations and protrusions along its
boundary.  At two corners of cluster~$T$, and one of cluster~$P$, an
additional line is drawn from a corner of a polykite to a corner of
the boundary of the polyiamond; this line clarifies  how an indentation to a
corner of the polyiamond is uniquely associated with one of that
polyiamond's sides.  

The boundaries of the four metatiles
are divided into labelled segments by marked points.
The labels represent matching rules to be obeyed in tilings 
by the metatiles.  To satisfy the matching rules, the four
metatiles must form a tiling using copies that are only rotated and
not reflected; edge segments marked $A^+$ and~$A^-$ must adjoin on
adjacent tiles of the tiling; likewise, edge segments $B^+$ and~$B^-$,
$X^+$ and~$X^-$, $F^+$ and~$F^-$, and $L$ and~$L$ must adjoin.
We will show in \secref{sec:subst} that any
tiling by the metatiles has a substitution structure: the tiles may be
grouped (after bisecting some tiles) into supertiles that satisfy
combinatorially equivalent matching rules.  This grouping process
implies that that no tiling by the metatiles is periodic.  Furthermore,
the
substitution structure allows the metatiles to tile arbitrarily
large regions of the plane, and hence the whole plane, implying that 
they form an aperiodic set.

In this section we establish the following result:

\begin{theorem}
\label{thm:clusters}
Any tiling by the hat polykite can be divided into the clusters shown
in Figure~\ref{fig:tileaclusters} (or reflections thereof, but not
mixing reflected and non-reflected clusters), satisfying the given
matching rules, with the resulting tiling by metatiles having the
same symmetries as the original tiling by polykites.
\end{theorem}

Since inspection
of the cluster shapes shows that, conversely, any tiling by metatiles
induces one by the hat polykite (for example, $A^+$ and $A^-$ are
equal and opposite modifications to the shape of an edge and are
consistent wherever they appear in the clusters), the division into
clusters suffices as part of showing that the hat polykite is an
aperiodic monotile.

The proof of Theorem~\ref{thm:clusters} is computer-assisted.  
We define rules (\secref{sec:clusters:rules}) for
assigning the labels from \fig{fig:tileaclusters} 
to tiles in any tiling by the hat polykite.  Those rules
assign a label to a tile based only on its immediate neighbours.
Because no arbitrary choices are involved in the rules, they preserve
all symmetries of the tiling.  It then remains to show that (a)~the
labels assigned do induce a division into the clusters shown, and
(b)~the clusters adjoin other clusters in accordance with the matching
rules.  Because the matching rules do not permit a reflected cluster
to adjoin a non-reflected cluster, it then follows that either no
clusters are reflected or all clusters are reflected. Without
loss of generality we assume in \secref{sec:subst} that no clusters
are reflected.

Both (a) and~(b) may be demonstrated by a case analysis of $2$-patches
of hats.  Ideally, we would restrict our analysis to precisely those
$2$-patches that appear in tilings by the hat.  Such an approach is
unrealistic, however, as it requires foreknowledge of the space of
tilings we are attempting to understand.  In practice the list of
$2$-patches can include false positives that do not occur in any
tilings, as long as our analysis produces valid results for them
as well (and as long as the list contains every $2$-patch that can
occur in a tiling).

For the purposes of our proof we worked with the~$188$
``surroundable $2$-patches'': $2$-patches that can be surrounded
at least once more to form a $3$-patch.  We generated this set of~$188$
patches computationally.  Specifically,
we modified Kaplan's SAT-based software~\cite{Kaplan} to
enumerate all distinct $3$-patches of hats, and extracted the unique 
$2$-patches in their centres.  We validated this list by creating an
independent implementation based on brute-force search with backtracking;
the source code for this implementation is available with our article.
This list certainly includes false positives---a more sophisticated case
analysis shows that at most $63$ of the $188$ surroundable $2$-patches
can actually appear in a tiling by hats. However, all $188$ of them 
satisfy the conditions given in this section, allowing us to obtain
the results we need with simpler and more transparent algorithms.

It is also possible to demonstrate both (a) and~(b) by a shorter 
case analysis using only $1$-patches.  
However, an analysis based on $1$-patches is
more complicated because the classification rules 
in \secref{sec:clusters:rules} assume that all the
neighbours of a tile are known.  Those rules can therefore not be applied
directly to the outer tiles in a $1$-patch, making it necessary to work
with partial information about which labels are consistent with such a
tile.  For more details of this alternative case analysis, see
Appendix~\ref{sec:patches}. 

An analysis of tilings based on the enumeration of patches 
depends on the
assumption that it is only necessary to consider tilings where all
polykites are aligned to the same underlying $[3.4.6.4]$ Laves
tiling.  This assumption is not in fact obvious for tilings by
polykites or other polyforms in general; it is justified in
Appendix~\ref{sec:align}.

For each of the 188 surroundable $2$-patches, the classification
rules of \secref{sec:clusters:rules} determine labels for the tiles
in the patch's interior (comprising the central tile and its
neighbours).  We may then demonstrate~(a) by verifying that when
the central tile of a patch has a given label from one of the
clusters shown in \fig{fig:tileaclusters}, its neighbours in that
cluster appear with the correct labels in the expected positions
and orientations within the patch.  This ``within-cluster''
verification process is explained in detail in
\secref{sec:clusters:within}.  Similarly, in \secref{sec:clusters:between}
we describe a ``between-cluster'' verification process that
demonstrates~(b).  In particular, we show that when a patch's central
tile is adjacent to a tile with a label from a different cluster,
their adajcency relationship is consistent with the labelled edge 
segments that define the matching rules for the clusters.
The reference software mentioned above performs
all of these checks on the 188 surroundable $2$-patches.

\FloatBarrier

\subsection{Classification rules for the hat polykite}
\label{sec:clusters:rules}

\fig{fig:class} presents the eight classification rules for tiles.
Each rule shows a (labelled) central tile and some of its neighbours.  The 
order
of the rules is significant: the first rule that matches determines
the label on the central tile.  For each rule, if all the neighbours
shown are present, and no previous rule matched, the tile acquires
the label indicated.  The last rule is not constrained by any
neighbours, and therefore always matches if no previous rule did.
Thus every tile is assigned some label.

These rules do not distinguish between the labels~$P_1$ and~$F_1$:
the last rule assigns all such tiles the common label~$FP_1$.  The
within-cluster and between-cluster checks that follow are all expressed
in terms of this composite label.  An~$FP_1$ tile can always
be relabelled as either~$P_1$ or~$F_1$ later, depending on whether it
has a neighbour labelled~$P_2$ or~$F_2$ in the correct position and
orientation.

\input{auto-tilea-class}

\FloatBarrier

\subsection{Within-cluster matching checks for the hat polykite}
\label{sec:clusters:within}

Let~$L_1$ and~$L_2$ be the labels of neighbouring tiles in one of the
clusters shown in \fig{fig:tileaclusters}.  To verify that
tiles can be grouped uniquely into copies of these clusters, we must
show that when the central tile of a surroundable $2$-patch has the
label~$L_1$, it has a neighbour labelled~$L_2$ in the expected position
and orientation shown in the cluster.  In practice, we do not need to
check all such pairs of labels---it suffices to choose a subset of
labels that define spanning trees of the neighbour relationships within each
cluster.  For~$H$, we choose the spanning tree that connects~$H_1$ to
its three neighbours.

\fig{fig:within} presents the eight within-cluster checks that must be
applied to each of the surroundable $2$-patches.  For each rule and each
patch, if the rule's shaded tile has the same label as the patch's
central tile, then the patch must also include the neighbour shown in the
rule.  As noted above, these rules do not distinguish between~$P_1$ and~$F_1$; 
it suffices to check that an~$FP_1$ tile has either of~$P_2$ or~$F_2$ as its
neighbour.  Because these rules hold for all surroundable $2$-patches,
the labels assigned in \secref{sec:clusters:rules} induce a division
of the tiles in any hat tiling into the~$H$, $T$, $P$, and~$F$ clusters.

\input{auto-tilea-within}

\FloatBarrier

\subsection{Between-cluster matching checks for the hat polykite}
\label{sec:clusters:between}

Let~$C$ be one of the four clusters in \fig{fig:tileaclusters}, and 
let~$E$ be any of its marked edge segments.  We can enumerate  all 
combinations of an edge segment~$E'$, belonging to a cluster~$C'$,
which are permitted to adjoin~$E$ according to the matching rules.
If any one tile in~$C'$ that adjoins~$E'$ is in the correct position and 
orientation relative to any one tile in~$C$ that adjoins~$E$, it follows as a
result of the within-cluster checks that the entire edge segment properly 
matches between the two clusters.  Furthermore, because the matching 
rules on the
boundaries of $F_1$ and~$P_1$ are identical, it suffices to handle both
using the single label~$FP_1$.  So for each~$E$ we
pick one tile in~$C$, and for each choice of~$E'$, we pick one tile
in~$C'$ that would be a neighbour of the tile picked in~$C$.  We then
check that, in each surroundable $2$-patch
whose central tile has the label of the tile picked in~$C$, there is a
neighbour in a position and orientation and with a label that matches
one of the possibilities for a tile picked in~$C'$ for one choice of~$E'$.

Figure~\ref{fig:between} presents the between-cluster checks that must be
applied to each of the surroundable $2$-patches.  Each diagram shows a 
shaded tile from cluster~$C$ 
and its neighbour from cluster~$C'$, with labels on both, and represents a
tile on one
side of a cluster edge and some options for a tile on the other side
of that edge.  In some cases, there are two alternatives listed
for the same edge, with separate figures for each, marked in the form
``(alternative~$k$ of~2)''.  Also, in some cases there are multiple 
options for the labels on one or both tiles, shown in a single figure.  The
central tile in every $2$-patch that can occur in a tiling should be
checked against all figures shown here with that central tile's label
as one of the options for the shaded tile; if, for all such
$2$-patches, one of the alternatives listed for that edge is present
with one of the labels indicated, then the clusters adjoin other
clusters in accordance with the matching rules.  (Where multiple
alternatives are listed for the same edge, only one of those
alternatives needs to pass the check.)

\input{auto-tilea-between}

\FloatBarrier
