% !TEX root = einstein.tex
\section{Introduction}
\label{sec:intro}



\begin{figure}[ht]
\begin{center}
	\includegraphics[width=.8\textwidth]{polykite_tiling_raster.pdf}
\end{center}
\caption{\label{fig:polykite} The grey ``hat'' polykite tile
is an aperiodic monotile, also known as an ``einstein''.
Copies of this tile may be assembled into tilings of the plane
(the tile ``admits'' tilings), but none of those tilings can have 
translational symmetry.  In fact, the hat admits uncountably many tilings. In Sections~\ref{sec:substitution},~\ref{sec:clusters}, and~\ref{sec:subst} we describe how these tilings all arise from substitution rules, and thus all have the same local structure.}
\end{figure}

Given a set of two-dimensional tiles,  the nature of the planar
tilings that they admit arises from a deep interaction between the
local and the global.  Constraints on the ways that two neighbouring
tiles interlock can reverberate through the global structure of a
tiling at every scale.  Local constraints
encoded in a set of tiles determine the larger space of tilings they admit in
subtle ways.

{\em Aperiodic} sets of tiles walk a fine line between order and disorder, admitting tilings, but only  those without the simple repetition of
translational symmetry. 
Their study dates to Wang's work on the then remaining open cases of Hilbert's {\it Entscheidungsproblem}~\cite{Wang}. Wang encoded logical fragments by what are now known as \emph{Wang tiles}---congruent squares with coloured edges---to
be tiled by translation only with colours matching on adjoining edges.
He conjectured that every set of Wang tiles that admits a tiling (possibly
using only a subset of the tiles) must also
admit a periodic tiling, and showed that this would imply the decidability
of the \textit{tiling problem} (or \textit{domino problem}): the question
of whether a given set of Wang tiles admits any tilings at all.
The algorithm would consist of enumerating, for each positive integer $n$,
the finite set of all legal~$n\times n$ blocks of tiles.
If there is no tiling by the tiles, there must be some~$n$
for which no such block exists (by the
Extension Theorem~\cite[Theorem~3.8.1]{GS}, which ultimately depends on the
compactness of spaces of patches), and we will eventually encounter the
smallest such~$n$.
On the other hand, if there is a fundamental
domain for a periodic tiling, we will eventually discover it in  
a block.  If Wang's conjecture held and aperiodic sets of tiles did not 
exist, this algorithm would always terminate.

Berger~\cite{Berger} then showed that it was undecidable whether a set
of Wang tiles admits a tiling of the plane. He 
constructed the first aperiodic set of  $20426$~Wang tiles, which he used 
as a kind of scaffolding for encoding finite but unbounded runs of arbitrary computation.

Subsequent decades have spawned a rich literature on aperiodic tiling, touching many  different mathematical and scientific settings; we do not attempt a broad survey here. Yet there remain remarkably few really distinct methods of proving aperiodicity in the plane, despite or due to the underlying undecidability of the tiling problem. 

Berger's initial set comprised thousands of tiles, naturally prompting
the question of how small a set of tiles could be while still forcing
aperiodicity.   
Professional and amateur mathematicians produced successively smaller aperiodic sets, culminating in discoveries by Penrose~\cite{Penrose}
and others of several consisting of just two tiles.  Surveys of these 
sets appear in Chapters~10 and~11 of Gr\"unbaum and Shephard~\cite{GS} and
in an account of the Trilobite and Cross tiles~\cite{ChaimTC}.
A recent table appears in the work of Greenfeld and Tao~\cite{GT1}, counting tiles by translation classes (tiles in different orientations are counted as distinct).


The obvious conclusion of this reduction in size would be to arrive
at an \textit{aperiodic monotile}, a single shape that can form tilings (is a monotile) but can only form non-periodic ones (is aperiodic).  Such a shape is also sometimes referred to as an
``einstein'' (a pun from the German ``ein stein'', roughly ``one
shape'', popularized by Danzer).  In the present article we reserve
these terms for two-dimensional closed topological disks that tile
aperiodically purely by virtue of their geometry, without the need
for any kind of non-geometric matching rules that further constrain
tile adjacencies.  It has long been an open question whether such
a tile exists.  Can one tile embody enough complexity
to forcibly disrupt  periodic order at all scales?

\subsection{The search for an einstein}
\label{sec:search}

Several candidate tiles have been proposed as einsteins, but they all challenge in some way the concepts
of ``tile'', ``tiling'', or ``aperiodic''.

Gummelt~\cite{Gummelt} and Jeong and Steinhardt~\cite{SteinhardtJeong,JeongSteinhardt}
describe a single regular decagon that can cover the plane with copies that are allowed to overlap by prescribed rules, but only non-periodically, in a manner tightly coupled to the Penrose tiling. 
Senechal~\cite{Senechalpersonalcommunication} similarly describes simple rules that allow copies of the Penrose dart to overlap and cover the plane, but never periodically. The result is an ingenious route to aperiodicity, but not a 
tiling in the usual sense.

Tiles are often endowed with \textit{matching rules} that constrain
their placement.  Matching rules have taken a variety of different forms
in the literature.  They sometimes act as a symbolic proxy for neighbour
relationships that could easily be encoded geometrically, but they can 
also determine more complex relationships between tiles.
The Taylor--Socolar tile~\cite{ST1} is a regular hexagon with matching
rules in the form of markings in the interiors of tiles.  The matching rules
force aperiodicity, but they require non-adjacent tiles to exchange
information.  As a result, it is impossible to reduce the behaviour of
the tile to the shape of a closed two-dimensional topological
disk.  The matching rules can be expressed
purely geometrically, but doing so requires either a disconnected tile, a
 tile with cutpoints, or a three-dimensional shape that aperiodically tiles a thickened plane ${\mathbb R}^2\times [0,1]$~\cite{ST2}. 

The structure of the Taylor--Socolar
tiling is closely related to Penrose's $1+\epsilon+\epsilon^2$
tiling \cite{penrose_epsilon, baakegahlergrimm2012,taylornotes}. Like the Trilobite and Crab tiles~\cite{ChaimTC}, these can be adjusted so that an arbitrarily high fraction of the area lies in copies of just one kind of tile. 
But no matter how thin or small they become, the other tiles remain necessary.

Loosely speaking, it is often possible to shift the complexity in
a construction from the tiles to the matching rules or vice versa.
For example, if we use a finite atlas of finite configurations as
our allowed matching rules, even the lowly $2\times1$ rectangle is
an aperiodic monotile!\footnote{Beginning with an aperiodic set of
tiles with, say, geometric matching rules, pixelate pictures of the
tiles and how they fit together, in some black and white bit-map.
Take an atlas of these pictures, splitting black pixels vertically
and white ones horizontally into identical rectangles. The rectangle
is an aperiodic monotile with this atlas of matching rules.} Walton
and Whittaker recently described a hexagonal tile that, like the
Taylor--Socolar tile, achieves aperiodicity via a system of
markings~\cite{WW21}.  These ``orientational'' rules are edge-to-edge,
in that they only constrain a tile's relationships to its immediate
neighbours.  However, this tile's behaviour also cannot be expressed
as pure geometry.

Moving to higher dimensional space permits richer forms of aperiodicity
to arise.  The Schmitt--Conway--Danzer tile~\cite[Section 7.2]{Senechal}
tiles $\mathbb{R}^3$, with tilings that never have translations as
symmetries; none of its tilings have compact fundamental domains.
However, the tile does admit a tiling whose symmetry group contains a screw
motion, and hence an infinite cyclic subgroup of screw motions.  We
refer to such a tile as \emph{weakly aperiodic}.
The ``weak'' label is appropriate, as such tiles 
appear readily in the
hyperbolic plane and other non-amenable spaces.  As early as 1974,
B\"or\"oczky exhibited a weakly aperiodic monotile in the hyperbolic
plane~\cite{Boroczsky}, the elegantly simple basis of the ``binary
tilings''~\cite{BlockWeinberger,regprod,MargulisMozes,Mozes97}.

Following Mozes~\cite{Mozes97}, we say a set of tiles is \emph{strongly
aperiodic} if it admits tilings but none with any infinite cyclic
symmetry.  In the Euclidean plane, a set of ``normal'' tiles is weakly 
aperiodic if and only if it is strongly aperiodic~\cite[Theorem~3.7.1]{GS},
leaving us with a single notion of aperiodicity there.

 Recently, Greenfeld and Tao~\cite{GT2} showed that for a  sufficiently 
high number $n$ of dimensions, a single tile, tiling \emph{only by translation}, can be
aperiodic in~$\mathbb{Z}^n$ (and thus in~$\mathbb{R}^n$); Greenfeld and
Kolountzakis~\cite{greenfeld2023tiling} strengthened this result by showing
that the tile can be connected.
Greenfeld and Tao also showed that it is undecidable whether a single
tile, again tiling by translation, admits a tiling of a periodic
subset of $\mathbb{Z}^2 \times G$ for some nonabelian
group~$G$~\cite{GT1}, and subsequently proved this for tiling a
periodic subset of $\mathbb{Z}^n$ (where $n$~is one of the inputs to
the decision problem and not fixed)~\cite{greenfeld2023undecidability}.
Translational aperiodicity is known to be impossible in~$\mathbb{R}^2$.
Kenyon~\cite{Kenyon,Kenyonerratum,Kenyon2}, building on the work of
Girault-Beauquier and Nivat~\cite{GiraultBeauquierNivat}, showed that %ok
any topological disk that admits a tiling by translation also admits a
periodic tiling.  Bhattacharya~\cite{Bhattacharya} showed the
same for any finite set in~$\mathbb{Z}^2$.

Little is known about limits on what sorts of shapes could potentially be
aperiodic monotiles.  Rao~\cite{Rao} showed through a computer
search that the list of 15 known families of convex pentagons that tile the
plane is complete, thereby eliminating any remaining possibility
that a convex polygon could be an einstein.
Jeandel and Rao~\cite{JeandelRao} showed that the smallest aperiodic set
of Wang tiles is of size~$11$.

Even when a single tile admits periodic tilings, that periodicity
may be more or less abstruse, in a way that offers tantalizing hints
about aperiodicity.  The \emph{isohedral number} of a tile is the
minimum number of transitivity classes in any tiling it admits; a
tile is \emph{anisohedral} if its isohedral number is greater than
one.  The second part of Hilbert's 18th problem~\cite{Hilbert} asked
whether there exist anisohedral polyhedra in $\mathbb{R}^3$.
Gr\"unbaum and Shephard suggest \cite[Section~9.6]{GS} that this
question was asked in $\mathbb{R}^3$ because Hilbert assumed that
no such tiles exist in the plane.  But Reinhardt~\cite{Reinhardt}
found an example of such a polyhedron, and Heesch~\cite{Heesch}
then gave an example of such a tile in the plane.  Many anisohedral
prototiles are known today. The computer enumeration by Myers~\cite{Myers}
furnished numerous anisohedral polyominoes, polyhexes, and polyiamonds,
including a record-holding $16$-hex that tiles with a minimum of
ten transitivity classes.  It is unknown whether there is an upper
bound on isohedral numbers of monotiles.\footnote{The problem of
determining whether or not a given set of tiles admits a periodic
tiling is also undecidable, at least for larger sets of
tiles~\cite{Gurevich}. If we enumerate sets of tiles, and define
$I(n)$ to be the isohedral number of the $n$th set if it admits a
periodic tiling, and $-1$ otherwise, then $I(n)$ cannot
be bounded by any computable function. This defies our imagination.}

Related insights can be gleaned from the study of shapes that do not tile
the plane.  A tile's \emph{Heesch number} is the largest possible 
combinatorial radius of any patch formed by copies of the shape (or
equivalently, the maximum number of complete concentric rings that can 
be constructed around it).  A shape that tiles the plane is said to have a
Heesch number of $\infty$.  Heesch first exhibited a shape with Heesch
number~1, and a few isolated examples with Heesch numbers up to~3 were
discovered thereafter~\cite{Mann2004}.  Mann and Thomas discovered marked
polyforms with Heesch numbers up to~5 through a brute-force computer 
search~\cite{MT2016}.  Kaplan conducted a search on unmarked 
polyforms~\cite{Kaplan}, yielding examples with Heesch numbers up to~4.
Ba{\v{s}}i{\'c} discovered the current record holder, a shape with Heesch
number~6~\cite{Basic2021}.  \emph{Heesch's problem} asks which 
positive integers can be Heesch numbers; beyond specific examples with
Heesch numbers up to~6, nothing is known about the solution.
An upper bound on finite Heesch numbers would imply the decidability of 
the tiling problem for a single shape. The algorithm would simply consist of
generating all possible concentric rings around a central tile; eventually
one will either fail (in which case the shape does not tile the plane) or
exceed the upper bound on Heesch numbers (in which case it must tile the plane).

\subsection{Main result}

\begin{figure}[ht!]
\begin{center}
\begin{tikzpicture}[x=1cm,y=1cm]
  \node[draw,text width=4cm] at (0,10.3) {Polykites with periodic tilings
    have aligned periodic tilings (Lemma~\ref{lemma:polykitealign})};
  \node[draw,text width=4cm] at (0,7.3) {Polyforms with aligned weakly
    periodic tilings have aligned strongly periodic tilings (similar
    to \cite[Theorem~3.7.1]{GS})};
  \node[draw,text width=4cm] at (0,4.3) {The hat polykite does not have
    aligned strongly periodic tilings (\secref{sec:coupling})};
  \node[draw,text width=4cm] at (0,2.3) {Clusters of hat polykites can form
    metatiles (\secref{sec:discussion})};
  \node[draw,text width=4cm] at (0,0) {Metatiles have a substitution
    system forming combinatorially equivalent supertiles
    (Sections \ref{sec:discussion} and~\ref{sec:subst})};
  \node[draw,text width=4cm] at (5.5,0) {Metatiles tile the plane};
  \node[draw,text width=4cm] at (5.5,2) {Hat polykites tile the plane};
  \node[draw,text width=4cm,ultra thick] at (5.5,4) {The hat polykite is strongly
    aperiodic};
  \node[draw,text width=4cm,ultra thick] at (5.5,6.5) {All $\mathrm{Tile}(a, b)$ for
    positive $a \ne b$ are strongly aperiodic};
  \node[draw,text width=4cm] at (5.5,10) {Tilings by $\mathrm{Tile}(a,b)$ are
    combinatorially equivalent to those by the hat polykite for
    positive $a \ne b$ (\secref{sec:family})};
  \draw[arrows={->[length=2mm]}] (2.2,0) -- (3.3,0);
  \draw[arrows={->[length=2mm]}] (2.2,2) -- (3.3,2);
  \draw[arrows={->[length=2mm]}] (2.2,4) -- (3.3,4);
  \draw[arrows={->[length=2mm]}] (2.2,7.3) -- (3.3,4.6);
  \draw[arrows={->[length=2mm]}] (2.2,10.3) -- (3.5,4.6);
  \draw[arrows={->[length=2mm]}] (5.3,0.4) -- (5.3,1.4);
  \draw[arrows={->[length=2mm]}] (5.3,2.6) -- (5.3,3.4);
  \draw[arrows={->[length=2mm]}] (5.3,4.6) -- (5.3,5.6);
  \draw[arrows={->[length=2mm]}] (5.3,8.3) -- (5.3,7.4);
\end{tikzpicture}
\end{center}
\caption{The high-level structure of the first proof of aperiodicity in
  this paper}
\label{fig:proofstructure}
\end{figure}

\begin{figure}[ht!]
\begin{center}
\begin{tikzpicture}[x=1cm,y=1cm]
  \node[draw,text width=4cm] at (0,6.4) {Polykites with periodic tilings
    have aligned periodic tilings (Lemma~\ref{lemma:polykitealign})};
  \node[draw,text width=4cm] at (0,3.3) {Clusters of hat polykites
    must form metatiles, adjoining in accordance with matching rules
    (\secref{sec:clusters} and Appendix~\ref{sec:patches})};
  \node[draw,text width=4cm] at (0,-0.3) {Metatiles must follow a
    substitution system forming combinatorially equivalent supertiles
    (Sections \ref{sec:discussion} and~\ref{sec:subst})};
  \node[draw,text width=4cm] at (5.5,0) {The metatiles are strongly aperiodic};
  \node[draw,text width=4cm,ultra thick] at (5.5,2) {The hat polykite is strongly
    aperiodic};
  \node[draw,text width=4cm,ultra thick] at (5.5,4.5) {All $\mathrm{Tile}(a, b)$ for
    positive $a \ne b$ are strongly aperiodic};
  \node[draw,text width=4cm] at (5.5,8) {Tilings by $\mathrm{Tile}(a,b)$ are
    combinatorially equivalent to those by the hat polykite for
    positive $a \ne b$ (\secref{sec:family})};
  \draw[arrows={->[length=2mm]}] (2.2,0) -- (3.3,0);
  \draw[arrows={->[length=2mm]}] (2.2,2) -- (3.3,2);
  \draw[arrows={->[length=2mm]}] (2.2,6.3) -- (3.3,2.6);
  \draw[arrows={->[length=2mm]}] (5.3,0.6) -- (5.3,1.4);
  \draw[arrows={->[length=2mm]}] (5.3,2.6) -- (5.3,3.6);
  \draw[arrows={->[length=2mm]}] (5.3,6.3) -- (5.3,5.4);
\end{tikzpicture}
\end{center}
\caption{The high-level structure of the second proof of aperiodicity in
  this paper}
\label{fig:proofstructure2}
\end{figure}

In this paper, we prove the following:  

\begin{theorem}
\label{thm:main}
The shape shown shaded in \fig{fig:polykite}, a polykite that we call
the ``hat'', is an aperiodic monotile.
\end{theorem}

The shape is almost mundane in its simplicity. It is a \textit{polykite}:
the union of eight kites in the Laves tiling $[3.4.6.4]$ (drawn in 
thin lines in \fig{fig:polykite}), the dual
to the $(3.4.6.4)$ Archimedean tiling.  No special qualifications
or additional matching rules are required: as shown, this shape
tiles the plane, but never with any translational symmetries.  



We provide two different proofs of aperiodicity, both with novel
aspects.  The first proof follows the structure shown in
\fig{fig:proofstructure}, centred on a 
new approach in \secref{sec:coupling} for proving aperiodicity in the plane.
We observe that
any tiling by the hat corresponds to tilings by two different
polyiamonds, one with two thirds the area of the other.  If there were a
strongly periodic tiling by the hat, the other two tilings would also
be strongly periodic. We prove that if so, the lattices of
translations in the polyiamond tilings would necessarily be related by a
similarity; but
no similarity between lattices of translations on the regular
triangular tiling can have the scale factor~$\sqrt{2}$ required by the
ratio of the areas.  This argument does not show that a tiling exists, and
must be combined with an explicit construction of a tiling (outlined
in \secref{sec:discussion} and given in detail in Sections
\ref{sec:clusters} and~\ref{sec:subst}) to complete the proof of
aperiodicity.

The second proof presented (but the first one found) follows the structure
shown in \fig{fig:proofstructure2}.  Here we generally adhere to
Berger's approach, but we must begin with a novel step
to get to the point where such a proof is possible.  
We first show that in any tiling by
the hat polykite, every tile belongs uniquely to one of four distinct
clusters called \textit{metatiles} (\secref{sec:clusters}), which inherit
matching rules from the geometry of the hats that make them up.
The metatiles abstract away the details of individual hats, and support
a standard style of hierarchical construction.
We then proceed with a Berger-style inductive proof of non-periodicity
in \secref{sec:subst}.  We show that any
tile in any tiling by these four metatiles lies in a unique hierarchy
of \textit{supertiles}---effectively combinatorial copies
of the metatiles---at larger and larger scales. The proof is
constructive.  We show that every metatile belongs uniquely to a 
level-$1$ supertile, and that these supertiles 
have the same combinatorial structure as the metatiles.  The 
level-$1$ supertiles must therefore lie uniquely within 
level-$2$ supertiles with the same combinatorics, and so on
for subsequent levels.
This construction proves that a tiling by copies of the
metatiles must be non-periodic, because if it contained a
translational symmetry, then these hierarchies of supertiles could not be
unique.\footnote{In particular, if a tiling had a translational
symmetry, then for sufficiently large $k$ there would exist a level-$k$
supertile that overlaps its image under this translation.
Any metatile in the intersection of these two supertiles would then 
lie within both of their infinite hierarchies, contradicting the
supposed uniqueness of those hierarchies~\cite[Theorem 10.1.1]{GS}.}
It also shows that the metatiles (and hence the hats) admit 
tilings of the plane, because we
construct clusters of arbitrary size~\cite[Theorem 3.8.1]{GS}.
We are not aware of past work that uses a metatile-like construction
as an intermediate stage towards a proof of aperiodicity.

Because of the combinatorial complexity of the hat polykite, 
a significant fraction of our second proof relies on exhaustive enumeration
of cases, which we carried out and cross-checked with two 
independent software implementations developed by two of the authors
in isolation.  These calculations are necessarily ad hoc, and are essentially
unenlightening.  This case
analysis is only needed to show that all tilings follow the
substitution structure; it is not needed for showing that a tiling
exists, and thus is not needed to show that the tile is aperiodic,
given the proof in \secref{sec:coupling} that no periodic tiling exists.

In Section~\ref{sec:clusters} we learn that every tiling by hats
necessarily contains a mixture of reflected and unreflected tiles.
Thus the hat's status as a monotile depends on
whether one considers a shape and its reflection to be congruent.
By longstanding tradition in the tiling literature (indeed, going
back to Euclid’s \textit{Elements}), shapes are considered congruent
if they are equivalent under any Euclidean isometry, including those
that reverse orientation.  The hat is therefore rightly considered a monotile.
 Still, this potential caveat emphasizes the importance of considering the
setting in which a tiling problem is defined: the geometric space
in which we are working, conditions on the tiles and their matching
rules, and the specific families of isometries that we are allowed
to use.  The diversity of ideas discussed in \secref{sec:search}
illustrates how context can colour the problem of aperiodicity.
We revisit the question of tiling aperiodically without
reflections in \secref{sec:conclusion}.

We close this introduction with definitions of the essential terminology
we will need for the rest of the article.
In \secref{sec:discussion},
we then present a compendium of provisional observations about this polykite,
including an explicit construction of a tiling and aspects of its structure
that deserve further study.  Our two proofs of aperiodicity follow:
we show that there are no periodic tilings (\secref{sec:coupling}),
then that tiles must group into clusters that define metatiles
equipped with matching rules (\secref{sec:clusters}),
and finally that metatiles must compose into
supertiles with combinatorially equivalent matching rules
(\secref{sec:subst}).
In \secref{sec:family}, we offer additional remarks about the continuum
of tiles that contains the hat polykite.  As noted there,
computer search shows that the hat is the smallest aperiodic polykite.

\subsection{Terminology}
\label{sec:terminology}

Terminology used for tilings generally follows that of
Gr\"unbaum and Shephard~\cite{GS}.

A \emph{tile} in a metric space is a closed set of points from that
space. A \emph{tiling} by a set of tiles is a collection of images of
tiles from that set under isometries, the interiors of which are
pairwise disjoint and the union of which is the whole space; we
say a set of tiles \emph{admits} the tiling, or in the case of a single tile
that it admits the tiling.  For most purposes, it is convenient for
tiles to be nonempty compact sets that are the closures of their
interiors; the tiles considered here are polygons, or more generally
closed topological disks.  A
tiling is \emph{monohedral} if all its tiles are congruent (where
congruences can incorporate mirror reflections).  All
tilings considered here are also \emph{locally finite}: every circular
disk meets only finitely many tiles. Every
monohedral plane tiling by closed topological disks is locally
finite.

In any locally finite tiling of the plane by closed topological disks,
the connected components of the intersection of two or more tiles are
isolated points, which are called \emph{vertices} of the tiling, and
Jordan arcs, which are called \emph{edges} of the tiling, and the
boundary of any tile is divided into finitely many edges, alternating
with vertices. Each edge lies on the boundary of exactly two tiles,
which we refer to as lying on opposite sides of the edge.  Two
distinct tiles are \emph{neighbours} if they share any point of their
boundaries, and \emph{adjacents} if they share an edge.

When a (closed topological disk) tile has a polygonal boundary, we
refer to it as having \emph{sides} (maximal straight line segments
lying on that boundary) and \emph{corners} (between two sides), to
distinguish these features from the edges and vertices of a tiling.  We rely on
context to distinguish the meanings of ``side'' as referring to sides
of a polygon or the two sides of an edge of a tiling.  A tiling by
polygons is \emph{edge-to-edge} if the corners and sides of the
polygons coincide with the vertices and edges of the tiling.

A \textit{patch} of tiles is a collection of non-overlapping tiles whose
union is a topological disk.  More specifically, a \textit{$0$-patch}
is a patch containing a single tile, and an \textit{$(n+1)$-patch} is 
a patch formed from the union of an $n$-patch $P$ and a set $S$ of
additional tiles, so that $P$ lies in the interior of the patch and no
proper subset of~$S$ yields a patch with $P$ in its interior.  (In
other words, an $n$-patch is
a tile surrounded by $n$ concentric rings of tiles.)  Every tile in a 
fixed tiling generates an $n$-patch for all finite $n$, by recursively
constructing an $(n-1)$-patch and adjoining all its neighbours in the
tiling, along with any other tiles required to fill in holes left by
adding neighbours.

Given a tiling~$\mathcal{T}$\!, a \emph{poly-$\mathcal{T}$-tile} is a
closed topological disk that is the union of finitely many tiles
from~$\mathcal{T}$; in other words, it is the union of the tiles in
a patch within~$\mathcal{T}$. Poly-$\mathcal{T}$-tiles are also referred to
generically as \emph{polyforms}.  Poly-$\mathcal{T}$-tiles may also
be defined so that they are permitted to have holes.
Because we are mainly concerned with tiles
that admit monohedral tilings, it is not generally significant for the
purposes of this paper whether shapes with holes are allowed or not.

The \emph{symmetry group} of a tiling is the group of those isometries
that act as a permutation on the tiles of the tiling.  A tiling is
\emph{weakly periodic} if its symmetry group has an element of
infinite order; in the plane, this means it includes a nonzero
translation.\footnote{Some authors such as Greenfeld and
Tao~\cite{GT0} have used the term ``weakly periodic'' to refer to a
tiling that is a finite union of sets of tiles, each of which is
weakly periodic in the sense used here.}  A tiling is \emph{strongly
periodic} if the symmetry
group has a discrete subgroup with cocompact action on the space
tiled. In Euclidean space, all strongly periodic
tilings are also weakly periodic.  A set of tiles (or a single tile)
is \emph{weakly aperiodic} if it admits a tiling but does not admit a
strongly periodic tiling, and \emph{strongly aperiodic} if it admits a
tiling but does not admit a weakly periodic tiling.

Any finite set of polygons in the plane that admits a weakly periodic
edge-to-edge tiling also admits a strongly periodic
tiling~\cite[Theorem~3.7.1]{GS}. A similar but simpler argument
shows the same to be the case for a finite set of
poly-$\mathcal{T}$-tiles where $\mathcal{T}$ is itself a strongly
periodic tiling and the weakly periodic tiling consists of copies 
of the
tiles all aligned to the same underlying copy of~$\mathcal{T}$,
instead of being edge-to-edge.  Thus in such contexts it is not
necessary to distinguish weak and strong aperiodicity and we refer to
tiles and sets of tiles simply as \emph{aperiodic}.

A \emph{uniform tiling}~\cite[Section~2.1]{GS} is an edge-to-edge
tiling by regular polygons with a vertex-transitive symmetry group.
In the Euclidean plane, a uniform tiling can be described by listing 
the sequence of regular polygons around each vertex, yielding notation
such as $(3.4.6.4)$.  A \emph{Laves
tiling}~\cite[Section~2.7]{GS} is an edge-to-edge monohedral tiling by
convex polygons with regular vertices (all angles between consecutive
edges at a vertex equal) and a tile-transitive symmetry group.  Analogous
notation such as $[3.4.6.4]$ is used for Laves tilings, listing the
sequence of vertex degrees round each tile, and in an appropriate
sense Laves tilings are dual to uniform tilings.
