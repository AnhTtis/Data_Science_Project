% !TEX root = einstein.tex




\section{Aperiodicity via coupling of polyiamond tilings}
\label{sec:coupling}


In Theorem~\ref{thm:subst_tiling},  we proved that the hat polykite 
is a monotile: it admits tilings of the plane.  Our proof used the 
metatile substitution system of \secref{sec:discussion}, described  
in detail in Sections~\ref{sec:clusters} and~\ref{sec:subst}. 

In those sections we also use a computer-assisted case
analysis to show  that every tiling by the hat polykite arises from
the substitution rules. For that reason the hat polykite does not
admit periodic tilings, completing a proof of Theorem~\ref{thm:main}
using a standard approach going back to Berger~\cite{Berger}.



In this section we give a more direct proof of aperiodicity
that exploits the hat's
membership in the $\mathrm{Tile}(a,b)$ continuum introduced in 
\secref{sec:discussion}.
As noted in \secref{sec:terminology}, a planar tile that does not admit
strongly periodic tilings also cannot admit weakly periodic tilings.  Building
on that fact and Theorem~\ref{thm:subst_tiling}, the following establishes 
Theorem~\ref{thm:main}.

\begin{theorem}
\label{thm:coupling:aperiodic}
Let $\mathcal{T}$ be a tiling by the hat polykite.  Then $\mathcal{T}$
is not strongly periodic.
\end{theorem}

\begin{figure}
\begin{center}
\includegraphics[width=\textwidth]{figures/t_t4_t8_new.pdf}
\end{center}
\caption{\label{fig:tt4t8}A patch from a hat tiling $\mathcal{T}$ (centre),
	together with corresponding patches from a chevron tiling
	$\mathcal{T}_4$ (left) and a comet tiling $\mathcal{T}_8$ (right).
	Corresponding reference tiles are marked in each patch.
	Edges of length~$1$ are shown in black;
	tiles and edges of length~$\sqrt{3}$ are given distinct colours 
	according to their orientations, with mirrored tiles shaded darker. 
	$\mathcal{T}_4$ and $\mathcal{T}_8$ are obtained by contracting the
	black and coloured edges of $\mathcal{T}$\!, respectively.
	$\mathcal{T}$ is shown at half scale to fit more of it into the figure.
	We assume that these combinatorially equivalent tilings are periodic
	in order to derive a contradiction.
}
\end{figure}


We suppose throughout this section that there is a strongly periodic
tiling~$\mathcal{T}$ by the hat polykite, described as $\mathrm{Tile}(1,\sqrt{3})$ in \secref{sec:discussion}, and derive a contradiction. 

In the tiling $\mathcal{T}$\!, tiles are necessarily aligned to 
an underlying $[3.4.6.4]$ Laves tiling.  This claim is justified
by Lemma~\ref{lemma:tileaalign},
which shows that all tilings by the hat polykite are aligned in that way.

Contracting the sides of length 
$1$ or~$2$ to length~$0$ in $\mathcal{T}$ produces a strongly periodic
tiling~$\mathcal{T}_4$ by chevrons, tiles of the form $\mathrm{Tile}(0,\sqrt{3})$.
Each chevron is the union of four equilateral triangles
of side length~$\sqrt{3}$, and must therefore have area~$3\sqrt{3}$.
Similarly, contracting the sides of length~$\sqrt{3}$ to length~$0$
produces a strongly periodic tiling~$\mathcal{T}_8$ by comets of
the form $\mathrm{Tile}(1,0)$, which have area~$2\sqrt{3}$.
(Because this contraction process is well-defined around any tile,
edge or vertex, it yields a combinatorial tiling of the plane, and a
combinatorial tiling corresponds to a geometrical tiling of the entire
plane~\cite[Lemma~1.1]{regprod}.)  
\fig{fig:tt4t8} (centre) shows a patch from an example tiling
$\mathcal{T}$, together
with corresponding patches from $\mathcal{T}_4$ (left) and $\mathcal{T}_8$
(right).  This mapping to tiles of
different side lengths is discussed in more detail in
\secref{sec:family}.

Because both~$\mathcal{T}_4$ and~$\mathcal{T}_8$ are strongly
periodic, there must exist an affine map~$g$ that acts
as a bijection between the translation symmetries of~$\mathcal{T}_4$
and~$\mathcal{T}_8$.
Recall that a \textit{similarity} is an affine map that 
preserves shape but not necessarily size (that is, it
scales uniformly in every direction). We will first show that~$g$
is not a similarity. We will then prove that it must be,
obtaining a contradiction.

The tilings $\mathcal{T}$\!, $\mathcal{T}_4$, and $\mathcal{T}_8$ 
are coupled, in the sense that there are bijections
between their tiles, with corresponding tiles in corresponding
orientations and translation symmetries of any one mapping directly to
translation symmetries of the others.  They also have close combinatorial
relationships: any neighbours in the original
polykite tiling correspond to neighbours in both polyiamond tilings. 
The affine map~$g$ defined above transforms every translation symmetry
of~$\mathcal{T}_4$ into a corresponding translation symmetry
of~$\mathcal{T}_8$ 
(one between corresponding pairs of tiles).  Given the areas of the
chevrons and comets,~$g$ must scale areas by $2/3$.

If~$g$ were a similarity, then we could deduce immediately that it
must scale lengths in every direction by~$\sqrt{2/3}$.  However, 
a similarity with this scale factor cannot also map translations 
in~$\mathcal{T}_4$ to translations in~$\mathcal{T}_8$.
Consider the regular tiling by
equilateral triangles, positioned to include a unit edge from $(0,0)$ to
$(1,0)$.  Every vertex of this tiling is given by $m(1,0)+n(1/2,\sqrt{3}/2)$
for integers~$m$ and~$n$, meaning that vectors joining vertices 
(including all possible vectors defining translation symmetries) must
have this form as well.  It follows that 
any distance~$d$ between two vertices must have $d^2$ of the form 
$m^2+mn+n^2$, in which case $d^2$ has an even number
of factors of~$2$.  Therefore a scale factor
of~$\sqrt{2}$ is not possible between translations on two triangular
tilings with the same edge length, and a scale factor of~$\sqrt{2/3}$
is not possible between translations on two triangular tilings with
edge lengths in a ratio of $\sqrt{3}$ to~$1$.


Using the fact that the six translation classes of kites must appear with equal frequency in
any aligned tiling by polykites, we now proceed to show that $g$  must 
in fact be a similarity, which gives the required contradiction.

A polykite is a union of kites from a $[3.4.6.4]$ Laves tiling, and
so its constituent kites are constrained to six possible orientations.
It happens that the hat uses four of those six kite orientations once
each, and the other two orientations (which are related by a halfturn)
twice each.  In any aligned
hat tiling, there are twelve possible tile orientations.  Tiles can
therefore be partitioned into three ``classes'' of four orientations
each, based on the orientations of their repeated kites.
\fig{fig:kitetwocol} (left)
shows the four hat orientations that make up one such class;  within
each hat, kites in orientations that appear more than once are
shaded darker.  Because of these repeated kite orientations, hats 
in each class claim the Laves tiling's
kites in an unbalanced way, favouring two kite orientations over
the other four.  In an infinite hat tiling all kite orientations must
be used in equal proportion, and so to restore balance the tiling
must use tiles from the three classes in equal proportion as well
(meaning that in any patch with perimeter~$x$, the imbalance between
the numbers of polykites with orientations from any two of the sets
is~$O(x)$).  In \fig{fig:tt4t8} (centre), copies of the hat 
with the same orientation class and handedness are coloured the same way.

\begin{figure}[htp!]
\begin{center}
\includegraphics[width=.85\hsize]{figures/imbalance}
\end{center}
\caption{Four orientations of the hat polykite that make up one
	orientation class, based on the repeated kite orientations they
	contain (left).  After contracting edges, these hats give rise to
	corresponding sets of chevrons (centre) and comets (right).}
\label{fig:kitetwocol}
\end{figure}



In the centre of  \fig{fig:kitetwocol}, we contract the sides of
the hat of length $1$ and~$2$, shown in black, to form chevrons in the
same orientations.
(This chevron is symmetric, so two orientations of the polykite in one of those sets
can give rise to identical-looking chevrons.  Those should still be
considered as different orientations of the chevron, as if it were
given an asymmetric marking.)
At right we contract the
sides of length~$\sqrt{3}$, shown coloured, to produce comets.
In Figures~\ref{fig:tt4t8} and~\ref{fig:istrips}
these tiles are coloured by orientation, matching the hats from
which they originated.


Note that given any two chevron sides in~$\mathcal{T}_4$, the
corresponding vector in~$\mathcal{T}_8$ between those two sides is
well defined: a side of a tile in~$\mathcal{T}_4$ corresponds to a
point on the boundary of the corresponding tile in~$\mathcal{T}_8$
(and adjoining sides on adjacent tiles correspond to the same point on
the boundaries of two neighbouring tiles in~$\mathcal{T}_8$), so the
vector is just the vector between those corresponding points. 

We will also make use of a tiling $\mathcal{T}_4'$, derived from
$\mathcal{T}_4$ by dividing every chevron into two congruent rhombi.
All chevrons associated with a single orientation class of hats divide
into rhombi in the same two orientations, as shown in 
\fig{fig:kitetwocol} (centre).  The balance of orientation
classes in $\mathcal{T}$ therefore implies that rhomb orientations
will occur with equal proportion in $\mathcal{T}_4'$.  (In fact,
based on tile adjacencies in hat tilings, it can be shown that 
$\mathcal{T}_4'$ is the Laves tiling $[3.6.3.6]$.)

\begin{figure}[t]
\begin{center}
\includegraphics[width=\textwidth]{istrips_new.pdf}
\end{center}
\caption{\label{fig:istrips} Taking the green segments to be parallel to the lines in $\mathcal{L}_1$,  the light coloured rhombs at left form  $1$-worms in  $\mathcal{T}_4'$.  At right,  the corresponding $1$-strips are shown in  $\mathcal{T}_4$.}
\end{figure}

To show that the period-preserving affine map~$g$ 
must be a similarity, thereby deriving a contradiction, we examine
how~$g$ behaves on the partitions of~$\mathcal{T}_4$ into
structures we call ``$i$-strips''. The edges of the equilateral
triangles in the regular triangular tiling underlying $\mathcal{T}_4$
and~$\mathcal{T}_4'$ lie in three sets of parallel lines. Call those
sets $\mathcal{L}_1$, $\mathcal{L}_2$, and~$\mathcal{L}_3$. Segments
in these directions are coloured green, red, and blue in the figures.
For each $i\in\{1,2,3\}$ we can now identify a set of ``$i$-worms'' in
$\mathcal{T}_4'$.  These are pairwise disjoint, two-way infinite sequences
of rhombi, in which consecutive rhombi in one worm are adjacent 
along an edge parallel to those in~$\mathcal{L}_i$. 
\fig{fig:istrips} (left) illustrates the $1$-worms in $\mathcal{T}_4'$.
Note that the $i$-worms for any given~$i$ 
will collectively use~$2/3$ of the rhombi in $\mathcal{T}_4'$.

Clearly, the $i$-worms for a given $i$ cannot cross, and any line
parallel to those in~$\mathcal{L}_i$ passes through the $i$-worms
in the same order as any other such line passes
through them.  Furthermore, any translation symmetry preserves both 
$i$-worms themselves and the ordering of $i$-worms.   

Every $i$-worm in $\mathcal{T}_4'$ defines an $i$-strip in
$\mathcal{T}_4$, by assigning each chevron to the same strip as one
of its rhombi.  If a chevron's rhombi both belong to $i$-worms for
a given~$i$ in $\mathcal{T}_4'$, then they must be in the \emph{same}
$i$-strip in~$\mathcal{T}_4$, because the line segment between those
two rhombi lies on a line in~$\mathcal{L}_i$.  This assignment must
constitute a partition of the tiles in $\mathcal{T}_4$.  \fig{fig:istrips}
(right) illustrates the $1$-strips in $\mathcal{T}_4$.


Let $\mathbf{v}_i$ be a vector between two consecutive lines
in~$\mathcal{L}_i$, orthogonal to those lines, chosen so the pairwise
angles between those vectors are all~$120^\circ$.  Let $\mathbf{v}'_i$
be a vector orthogonal to~$\mathbf{v}_i$ and with length $1/\sqrt{3}$
times that of~$\mathbf{v}_i$, again chosen so the pairwise angles
between those vectors are all~$120^\circ$.  Note that $\sum_i
\mathbf{v}_i = 0$ and $\sum_i \mathbf{v}'_i = 0$.  Considering the
sides of rhombi in an $i$-worm in~$\mathcal{T}_4'$ that lie in
consecutive lines of~$\mathcal{L}_i$, the vector between the midpoints
of such sides is $\mathbf{v}_i \pm \mathbf{v}'_i$, where the sign
depends on the orientation of the rhomb.  Thus, if the vector
between the midpoints of any two $\mathcal{L}_i$-aligned rhomb sides in an
$i$-worm is $a \mathbf{v}_i + b \mathbf{v}'_i$,
then between those two sides there are $(a+b)/2$ rhombi of one
orientation and $(a-b)/2$ of the other orientation.

The translation symmetries of the strongly periodic tiling~$\mathcal{T}$ 
correspond to a subgroup of the symmetries of~$\mathcal{T}_4'$.  
There are only finitely many orbits of rhombi
under the action of the subgroup, so in any
$i$-worm~$\mathcal{S}$ there must be two rhombi in the same orbit.
The translation mapping one rhomb to the other is a translation symmetry of
$\mathcal{T}_4'$, and therefore maps $i$-worms to $i$-worms. Because it maps
$\mathcal{S}$ to itself and preserves the ordering of $i$-worms, it
must map every $i$-worm to itself.  If that translation is by a
vector $a \mathbf{v}_i + b \mathbf{v}'_i$, it follows that $b=0$,
because otherwise rhombi of the two orientations that make up
these $i$-worms would appear in the tiling in different proportions.

Thus for each $i$ we have some positive integer~$a_i$, such that a
translation by $a_i \mathbf{v}_i$ is a symmetry of both~$\mathcal{T}_4'$
and~$\mathcal{T}_4$.  In~$\mathcal{T}_4$, translation by this vector 
must map every $i$-strip to itself. 
Let~$a$ be the lowest common multiple of the~$a_i$.  Translation by 
$a \mathbf{v}_i$ is also a symmetry of $\mathcal{T}_4$ that 
sends each $i$-strip to itself.

By construction, translation by 
$a \mathbf{v}_i$ in $\mathcal{T}_4$ 
corresponds to a translation symmetry of $\mathcal{T}$\!,
and therefore also to some translation symmetry of~$\mathcal{T}_8$.
We may calculate the precise translation vectors in~$\mathcal{T}$ 
corresponding to each $a \mathbf{v}_i$ based on the tiles in
any $i$-strip in~$\mathcal{T}_4$ (between any two lines
in~$\mathcal{L}_i$ related by a translation by that vector). Every
such $i$-strip (and choice of lines) must produce the same vector
in~$\mathcal{T}_8$.  \fig{fig:kitetrans} shows the
corresponding translations between pairs of chevron edges, 
oriented by way of example to be parallel to  $\mathcal{L}_1$.
  For each such pair, first the
vector within the chevron is indicated, then the corresponding
comet vector between points on the boundary of the comet, then that vector
decomposed into components parallel to and orthogonal to the chevron 
sides between
which the vector is drawn.  The corresponding hats are shown to aid in 
verifying the calculation. 
Rotating, reflecting or reversing the
direction of the chevron vector has the same effect on the
comet  vector.

\begin{figure}[htp!]
\begin{center}
\includegraphics[width=\hsize]{figures/kitetrans}
\end{center}
\caption{Corresponding translations for the chevron and comet}
\label{fig:kitetrans}
\end{figure}

Note that in the first case, the comet vector is parallel to the
sides between which the chevron vector is drawn; the second and
third cases have equal components orthogonal to those sides.  For the
orthogonal component of the corresponding translations
in~$\mathcal{T}_8$ to be equal for all $i$-strips, it follows that
every $i$-strip must have the same proportion of the second and third
cases relative to the first case.  As the first case corresponds
exactly to one of the three sets of orientations that occur in equal
proportions in any tiling, the first case must thus be a third of the
chevrons in any $i$-strip, while the second and third cases (which
together correspond to the other two sets of orientations; however,
each case does not correspond to a single set of orientations) in that
figure must add to two thirds of the chevrons.

The chevrons  in the first case have orthogonal
translation vector $2{\bf v}_i$ in $\mathcal{T}_4$,  zero in
$\mathcal{T}_8$. The remaining two thirds of the chevrons have
orthogonal translation ${\bf v}_i$ in both $\mathcal{T}_4$ and
$\mathcal{T}_8$.  Thus if $a{\bf v}_i$ is a period of the $i$-strips
in $\mathcal{T}_4$, then each of its $i$-strips has  $a/4$ chevrons
from the first case and $a/2$ in the other two cases. This period
corresponds to  a translation symmetry of  $(a/2){\bf v}_i$ in
$\mathcal{T}_8$.  Since the sum of ${\bf v}_i$ over $i=1,2,3$ is
zero (as noted above), the sum of those three orthogonal components
of translation vectors in $\mathcal{T}_8$ is also zero.

Therefore their parallel components must also add to
zero.  But $\sum_i b_i \mathbf{v}'_i = 0$ if and only if all the
$b_i$~are equal; say they all equal~$b$.  That means the three
translation vectors in~$\mathcal{T}_8$ (which are~${\frac{a}{2}
\mathbf{v}_i + b \mathbf{v}'_i}$) are at $120^\circ$~angles to each
other.  In that case, the period-preserving affine transformation~$g$ 
must scale uniformly in every
direction, and is therefore a similarity.  But we know from the discussion
above that~$g$ cannot be a similarity, and so we arrive at a contradiction,
ruling out the initial supposition that~$\mathcal{T}$ was strongly periodic.
