\documentclass[11pt,letterpaper]{article}
\usepackage[margin=1in]{geometry}
% \usepackage[latin1]{inputenc}
\usepackage[british]{babel}
\usepackage[all]{xy}
\usepackage{amscd}
\usepackage{amssymb}
\usepackage{amsthm}
\usepackage{enumitem}
\usepackage{mathrsfs,bbm}
\usepackage{xcolor,graphicx}
\usepackage{graphics}
\usepackage{soul}
\usepackage{comment}
\usepackage[all]{xy}
\usepackage{amscd}
\usepackage{amssymb,amsmath,latexsym}
\usepackage{amsthm}
\usepackage{enumitem}
\usepackage{mathrsfs,bbm}
\usepackage{dsfont}
\usepackage{tikz-cd}
\usepackage[T1]{fontenc}
\usepackage[utf8]{inputenc}  
 %
%%%%%%%%%%%%%%%%%%%%%%%%%%%%%%%%%%
%pagestyle
%%%%%%%%%%%%%%%%%%%%%%%%%%%%%%%%%%
%\pagestyle{plain}
\textwidth=430pt
\headsep=.7cm
\evensidemargin=15pt
\oddsidemargin=15pt
\leftmargin=0cm
\rightmargin=0cm
%%
%%%%%%%%%%%%%%%%%%%%%%%
\newcommand*\fixitem {\item[]%
  \refstepcounter{enumi}\hskip-\leftmargin\labelenumi\hskip\labelsep}
\newtheorem*{mainthm}{Main Theorem}
\newtheorem*{mainthm1}{Theorem}
\newtheorem*{maincor}{Corollary}
\usepackage[colorlinks=true]{hyperref}
\DeclareMathOperator{\Forall}{\forall}
\DeclareMathOperator{\Exists}{\exists}
\DeclareMathOperator{\ord}{ord}
\newcommand{\phiD}{\varphi_D}
\newcommand{\phiDI}{\varphi_{\mathbf{D}_I}}
\newcommand{\phiDIj}{\varphi_{\mathbf{D}_I (j)}}
\newcommand{\phiH}{\varphi_H}
\newcommand{\phiTimes}{\phiD \otimes \phiH}
\newcommand{\phiTimesDI}{\varphi_{\mathbf{D}_I} \otimes \phiH}
\newcommand{\R}{\mathscr{A}}
\newcommand{\X}{\mathscr{X}}
\newcommand{\Xf}{\mathscr{X}_{(k_0 ,i)}[r_0]}
\newcommand{\Xfr}{\mathscr{X}_{(k_0,i)}[r]}
\newcommand{\hotimes}{\widehat{\otimes}}
\newcommand{\C}{\mathbb{C}_p}
\newcommand{\V}{\mathscr{V}}
\newcommand{\B}{\mathscr{B}}
\newcommand{\dualD}{\mathfrak{D}}
\newcommand{\Dg}{\mathbf{D}}
\newcommand{\DD}{\mathcal{D}^0}
\newcommand{\DDg}{\mathcal{D}}
\newcommand{\DV}{\mathcal{D}}
\newcommand{\W}{\mathscr{W}_N}
\newcommand{\Ao}{\mathbf{A}^\circ}
\newcommand{\AoK}{\mathbf{A}^\circ_{\K}}
\newcommand{\AK}{\mathbf{A}_{/\K}}
\newcommand{\OOO}{\mathscr{A}^\circ}
\newcommand{\K}{\mathcal{K}} 
\newcommand{\OK}{\mathcal{O}_{\K}}
\newcommand{\varprojlog}[1]{\underleftarrow{\log\!^{#1}}}
\newcommand{\T}{\mathscr{T}}
\newcommand{\TT}{\mathbf{T}}
\newcommand{\VV}{\mathbf{V}}
\newcommand{\HH}{\mathcal{H}}
\newcommand{\hh}{\mathcal{H}^+}
\newcommand{\HG}[2]{\mathcal{H}_{#1}(#2)}
\newcommand{\hhl}{\mathcal{H}^{+,[l]}}
\newcommand{\hhj}{\mathcal{H}^{+,[j]}}
\newcommand{\hhjj}{\mathcal{H}^{+,[l,l']}}
\newcommand{\GS}{G_{\mathbb{Q},S}}
\newcommand{\Rf}{R_{(k_0 ,i)}[r_0]}
\newcommand{\Rfr}{R_{(k_0 ,i)}[r]}
\newcommand{\parT}{\langle T\rangle}
\newcommand{\Zf}{Z_{(k_0 ,i)}[r_0]}
\newcommand{\Zfr}{\mathscr{Z}_{(k_0 ,i)}[r]}
\newcommand{\ZFf}{\mathscr{Z}_{(k_0 ,i)}[r_0]}
\newcommand{\ZFfr}{\mathscr{Z}_{(k_0 ,i)}[r]}
\newcommand{\ZF}{\mathscr{Z}}
\definecolor{purple}{rgb}{1, 0, 1}

\newcommand{\ie}{\emph{i.e.,}\xspace}
\newcommand{\eg}{\emph{e.g.,}\xspace}
\newcommand{\abr}{\emph{abbr.}\xspace}
\newcommand{\ea}{\emph{et al.}\xspace}
\newcommand{\gensync}{\emph{GenSync}\xspace}
\newcommand{\colosseum}{\emph{Colosseum}\xspace}
\newcommand{\srep}{\emph{SREP}\xspace} % Set Reconciliation Enhances
\newcommand{\srepsim}{\emph{SREPSim}\xspace}
% Propagation
\newcommand{\esrep}{\emph{E-SREP}\xspace}
\newcommand{\epsrep}{\emph{EP-SREP}\xspace}
\newcommand{\mesrep}{\emph{ME-SREP}\xspace}
\newcommand{\mempoolsync}{\emph{MempoolSync}}

\newcommand{\fref}[1]{Fig.~\ref{#1}}
\newcommand{\tref}[1]{Table~\ref{#1}}
\newcommand{\aref}[1]{Algorithm~\ref{#1}}
\newcommand{\procref}[1]{Procedure~\ref{#1}}
\newcommand{\sref}[1]{Section~\ref{#1}}
\newcommand{\lineref}[1]{line~\ref{#1}}
\newcommand{\appref}[1]{Appendix~\ref{#1}}

% Change \eqref
\LetLtxMacro{\originaleqref}{\eqref}
\renewcommand{\eqref}{Eq.~\originaleqref}

% Theorems and corollaries
\newcounter{theoremcount}
\setcounter{theoremcount}{0}
\DeclareRobustCommand{\theorem}[1]{%
  \refstepcounter{theoremcount}%
  \noindent\textit{\textbf{Theorem \thetheoremcount\label{theorem:#1}: }}%
}
\DeclareRobustCommand{\theoremref}[1]{Theorem~\ref{theorem:#1}}

\DeclareRobustCommand{\proof}{\emph{Proof:}\xspace}
\DeclareRobustCommand{\qqed}{\hfill$\blacksquare$}

\newcounter{corollcount}
\setcounter{corollcount}{0}
\DeclareRobustCommand{\coroll}[1]{%
  \refstepcounter{corollcount}%
  \noindent\textit{\textbf{Corollary \thecorollcount\label{coroll:#1}: }}%
}
\DeclareRobustCommand{\corollref}[1]{Corollary~\ref{coroll:#1}}

\newcounter{lemmacount}
\setcounter{lemmacount}{0}
\DeclareRobustCommand{\lemma}[1]{%
  \refstepcounter{lemmacount}%
  \noindent\textit{\textbf{Lemma \thelemmacount\label{lemma:#1}: }}%
}
\DeclareRobustCommand{\lemmaref}[1]{Lemma~\ref{lemma:#1}}

\newcounter{definitioncount}
\setcounter{definitioncount}{0}
\DeclareRobustCommand{\definition}[1]{%
  \refstepcounter{definitioncount}%
  \noindent\textit{\textbf{Definition \thedefinitioncount\label{definition:#1}: }}%
}
\DeclareRobustCommand{\defref}[1]{Definition~\ref{definition:#1}}

%notes of different authors
\newif\ifnotes
\notestrue
\notesfalse

\newif\ifdiff
\difftrue
\difffalse

\newcommand{\anote}[1]{\ifnotes $\ll$\textsf{\textcolor{purple}{Ari: {#1}}}$\gg$ \fi}
\newcommand{\nnote}[1]{\ifnotes $\ll$\textsf{\textcolor{orange}{Novak: {#1}}}$\gg$ \fi}
\newcommand{\diff}[1]{\ifdiff\textcolor{orange}{#1}\else#1\fi}

%%% Local Variables:
%%% mode: latex
%%% TeX-master: "main"
%%% End:

\usepackage{float}
\usepackage{comment}
\usepackage[suppress]{color-edits}
\addauthor{sa}{teal}
\addauthor{kms}{blue}
\addauthor{ab}{purple}
\addauthor{om}{red}
\begin{document}
\title{Agnostic Multi-Robust Learning Using ERM\footnote{Authors are ordered alphabetically.}}
\author{Saba Ahmadi$^\dagger$}
\author{Avrim Blum$^\dagger$}
\author{Omar Montasser$^\ddagger$} 
\author{Kevin Stangl$^\dagger$}


\affil{$^\dagger$Toyota Technological Institute at Chicago\\
$^\ddagger$University of California, Berkeley\\
{\small\texttt{\{saba,avrim,omar,kevin\}@ttic.edu}}}

\maketitle



\begin{abstract}
A fundamental problem in robust learning is asymmetry: a learner needs to correctly classify every one of exponentially-many perturbations that an adversary might make to a test-time natural example. In contrast, the attacker only needs to find one successful perturbation. ~\citet{xiang2022patchcleanser} proposed an algorithm that in the context of patch attacks for image classification, reduces the effective number of perturbations from an exponential to a polynomial number of perturbations and learns using an ERM oracle. However, to achieve its guarantee, their algorithm requires the natural examples to be robustly realizable. 
This prompts the natural question; can we extend their approach to the non-robustly-realizable case where there is no classifier with zero robust error?

Our first contribution is to answer this question affirmatively by reducing this problem to a setting in which an algorithm proposed by ~\citet{DBLP:conf/colt/FeigeMS15} can be applied, and in the process extend their guarantees. Next, we extend our results to a multi-group setting and introduce a novel agnostic multi-robust learning problem where the goal is to learn a predictor that achieves low robust loss on a (potentially) rich collection of subgroups.
\end{abstract}

\section{Introduction}
\label{sec:introduction}
% \begin{itemize}
%     % Diffusion of FL
%     \item {\st{Diffusion of FL}}
%     % Security threats to FL
%     \item {\st{Security threats to FL with particular focus on model poisoning}}
%     % Limitations of existing countermeasures
%     \item {\st{Current countermeasures (e.g., KRUM) and their limitations}}
%     % Proposed method and its advantages
%     \item {\st{Intuitive description of the proposed method and its difference (i.e., advantages) w.r.t. state of the art}}
%     % Main contributions
%     \item {\st{Summary of the main contributions of this work}}
%     % Paper's structure and organization
%     \item {\st{Paper's structure and organization}}
% \end{itemize}

% Diffusion of FL
Recently, {\em federated learning} (FL) has emerged as the leading paradigm for training distributed, large-scale, and privacy-preserving machine learning (ML) systems~\cite{mcmahan2017googleai,mcmahan2017aistats}. 
The core idea of FL is to allow multiple edge clients to collaboratively train a shared, global model without disclosing their local private training data.
%Specifically, an FL system consists of a central server and many edge clients; 
A typical FL round involves the following steps: {\em(i)} the server randomly picks some clients and sends them the current, global model; {\em(ii)} each selected client locally trains its model with its own private data; then, it sends the resulting local model to the server;\footnote{Whenever we refer to global/local model, we mean global/local model {\em parameters}.} {\em(iii)} the server updates the global model by computing an \emph{aggregation function}, usually the average (FedAvg), on the local models received from clients.
% \begin{enumerate}
%     \item[{\em(i)}] the server sends the current, global model to the clients and appoints some of them for training;
%     \item[{\em(ii)}] each selected client locally trains its copy of the global model with its own private data; then, it sends the resulting local model back to the server;\footnote{Whenever we refer to global/local model, we mean global/local model {\em parameters}.}
%     \item[{\em(iii)}] the server updates the global model by computing an \emph{aggregation function} on the local models received from clients (by default, the average, also referred to as FedAvg~\cite{mcmahan2017aistats}).
% \end{enumerate}
This process goes on until the global model converges. %(e.g., after a certain number of rounds or other similar stopping criteria).
%\\
% The advantages of FL over the traditional, centralized learning paradigm are undoubtedly clear in terms of flexibility/scalability (clients can join/disconnect from the FL network dynamically), network communications (only model weights\footnote{We will use \textit{parameters} and \textit{weights} interchangeably.} are exchanged between clients and server), and privacy (each client's private training data is kept local at the client's end and not uploaded to the server).
\\
% Security threats to FL
%However, the growing adoption of FL also raises security concerns~\cite{costa2022covert}, particularly about its confidentiality, integrity, and availability.
Although its advantages over standard ML, FL also raises security concerns~\cite{costa2022covert}. %, particularly about its confidentiality, integrity, and availability~\cite{costa2022covert}.
% OLD, LONG VERSION
% Indeed, some work deals with privacy leakage that may expose the local data of some clients~\cite{melis2019sp}. 
% A large body of work, instead, investigates attacks that usually aim to detriment the predictive accuracy of the learned global model. For instance, \emph{data poisoning} attacks achieve this goal by letting an adversary pollute the training set of some corrupt FL clients with maliciously crafted examples~\cite{jagielski2018sp}.
% Similarly, in \emph{model poisoning} the attacker attempts to tweak the global model weights~\cite{bhagoji2019pmlr} by directly perturbing the local model's weights of some infected FL clients before these are sent to the central server for aggregation, usually via so-called Byzantine attacks. 
% It turns out that Byzantine model poisoning attacks severely impact standard FedAvg; therefore, more robust aggregation functions must be designed to make FL systems secure.
Here, we focus on \emph{untargeted model poisoning} attacks~\cite{bhagoji2019pmlr}, where an adversary attempts to tweak the global model weights %\footnote{We will use the terms \textit{parameters} and \textit{weights} interchangeably.} 
by directly perturbing the local model's parameters of some infected clients before these are sent to the central server for aggregation.
In doing so, the adversary aims to jeopardize the global model \textit{indiscriminately} at inference time.
Such model poisoning attacks severely impact standard FedAvg; therefore, more robust aggregation functions must be designed to secure FL systems.
\\
% In this paper, we focus on designing a novel robust aggregation scheme at the server's end to contrast the effect of Byzantine model poisoning attacks.
%
% Current countermeasures and their limitations
%Several countermeasures have been proposed in the literature to combat model poisoning attacks on FL systems.
% Some methods use simple statistics more robust than plain average to smooth the impact of malicious updates (e.g., Trimmed Mean and FedMedian~\cite{yin2018icml}). 
% Other defenses implement outlier detection techniques to discard malicious updates from the aggregation performed at the server's end. Those are either based on heuristics (e.g., Krum/Multi-Krum~\cite{blanchard2017nips} and Bulyan~\cite{mhamdi2018pmlr}) or data-driven approaches (e.g., K-means clustering~\cite{shen2016acm} or DnC via spectral analysis~\cite{shejwalkar2021ndss}). 
% Finally, some strategies rely on a centralized ``source of trust'' to spot potential malicious updates (e.g., FLTrust~\cite{cao2020fltrust}).
% Several countermeasures have been proposed in the literature to combat model poisoning attacks on FL systems, i.e., to discard possible malicious local updates from the aggregation performed at the server's end. 
% These techniques range from simple statistics more robust than plain average (e.g., Trimmed Mean and FedMedian~\cite{yin2018icml}) to outlier detection heuristics (e.g., Krum/Multi-Krum~\cite{blanchard2017nips} and Bulyan~\cite{mhamdi2018pmlr}) or data-driven approaches (e.g., spectral analysis via K-means clustering~\cite{shen2016acm} or spectral analysis), or methods based on ``source of trust'' (e.g., FLTrust~\cite{cao2020fltrust}).
% OLD, LONG VERSION
%Several countermeasures have been proposed in the literature to combat Byzantine model poisoning attacks on FL systems.
% Descriptive statistics
% For example, Trimmed Mean and FedMedian aggregate local model updates using more robust statistics than standard average~\cite{yin2018icml}.
%
% % Heuristics for outlier detection
% Many existing Byzantine-resilient strategies implement some outlier detection heuristics to discard the model updates sent by potentially malicious clients from the input of the aggregation function.
% One of the most popular heuristics is Krum~\cite{blanchard2017nips}.
% This strategy tries to mitigate the impact of Byzantine attacks by selecting as a global model the local model with the smallest sum of Euclidean distances to {\em all} the other local models.
% Although powerful, Krum requires the server to know (or, at least, estimate) the number of malicious FL clients upfront, which is generally impossible in a realistic attack scenario. %
% Moreover, Krum may become ineffective for complex, high-dimensional model parameter spaces due to the curse of dimensionality.
% Bulyan~\cite{mhamdi2018pmlr} tries to overcome this issue by combining Krum with a variant of Trimmed Mean.
% % Data-driven outlier detection
% Other strategies use data-driven outlier detection techniques -- e.g., via K-means clustering~\cite{shen2016acm} -- to spot potential malicious local model updates. 
% %For instance, Shen et al. propose to cluster local model updates with K-means and thus identify outliers.
%
% % Other techniques
% As far as the server is concerned, any local model received can be from a potential malicious client. 
% FLTrust~\cite{cao2020fltrust} assumes the server acts as a client, i.e., trains a local model on an additional {\em trustworthy} dataset at the server's end and compares it against all the local models from other clients. 
% This way, the server can rely on some ``source of trust'' when discarding potentially malicious clients.
%\\
% Limitations of existing Byzantine-resilient strategies
Unfortunately, existing defense mechanisms either rely on simple heuristics (e.g., Trimmed Mean and FedMedian by~\cite{yin2018icml}) or need strong and unrealistic assumptions to work effectively (e.g., foreknowledge or estimation of the number of malicious clients in the FL system, as for Krum/Multi-Krum~\cite{blanchard2017nips} and Bulyan~\cite{mhamdi2018pmlr}, which, however, cannot exceed a fixed threshold).
Furthermore, outlier detection methods using K-means clustering~\cite{shen2016acm} or spectral analysis like DnC~\cite{shejwalkar2021ndss} do not directly consider the temporal evolution of local model updates received.
Finally, strategies like FLTrust~\cite{cao2020fltrust} require the server to collect its own dataset and act as a proper client, thereby altering the standard FL protocol.
\\
% OLD, LONG VERSION
% Overall, existing Byzantine-resilient strategies are either simple heuristics (e.g., FedMedian) or, if they are more complex, they rely on strong and unrealistic assumptions to work effectively (e.g., knowing the number of malicious clients in the FL system in advance, as for Krum and alike).
% Furthermore, data-driven outlier detection methods do not consider the temporary evolution of local model updates received (e.g., K-means clustering). 
% Finally, strategies like FLTrust requires the server to collect its own dataset and act as a proper client, thereby altering the standard FL protocol.
%
% Description of the proposed method
This work introduces a novel pre-aggregation \textit{filter} robust to untargeted model poisoning attacks. Notably, this filter $(i)$ operates without requiring prior knowledge or constraints on the number of malicious clients and $(ii)$ inherently integrates temporal dependencies. 
The FL server can employ this filter as a preprocessing step before applying \textit{any} aggregation function, be it standard like FedAvg or robust like Krum or Bulyan.
Specifically, we formulate the problem of identifying corrupted updates as a multidimensional (i.e., matrix-valued) time series anomaly detection task. 
The key idea is that legitimate local updates, resulting from well-calibrated iterative procedures like stochastic gradient descent (SGD) with an appropriate learning rate, show \textit{higher predictability} compared to malicious updates. This hypothesis stems from the fact that the sequence of gradients (thus, model parameters) observed during legitimate training exhibit regular patterns, as validated in Section~\ref{subsec:intuition}. %until convergence. 
%This regularity may be more pronounced for smooth convex loss functions, but it can still be captured within an appropriate time window, even for more complex and convoluted loss surfaces. 
%We provide evidence of this claim in Appendix~B, where we show that the average mutual information (i.e., ``predictability''), calculated over pairs of legitimate model updates sent at different FL rounds, is significantly higher than the corresponding computation for a malicious client.
\\
Inspired by the matrix autoregressive (MAR) framework for multidimensional time series forecasting~\cite{chen2021je}, we propose the FLANDERS ({\em \textbf{F}ederated \textbf{L}earning meets \textbf{AN}omaly \textbf{DE}tection for a \textbf{R}obust and \textbf{S}ecure}) filter.
The main advantages of FLANDERS over existing strategies like FLDetector~\cite{zhao2020multivariate} are its resilience to large-scale attacks, where $50\%$ or more FL participants are hostile, and the capability of working under realistic non-iid scenarios.
We attribute such a capability to two key factors: $(i)$ FLANDERS works without knowing a priori the ratio of corrupted clients, and $(ii)$ it embodies temporal dependencies between intra- and inter-client updates, quickly recognizing local model drifts caused by evil players. Below, we summarize our main contributions:

\begin{itemize}
\item[{\em(i)}]
We provide empirical evidence that the sequence of models sent by legitimate clients is more predictable than those of malicious participants performing untargeted model poisoning attacks.
\\
\item[{\em(ii)}] 
We introduce FLANDERS, the first pre-aggregation filter for FL robust to untargeted model poisoning based on multidimensional time series anomaly detection.
\\
\item[{\em(iii)}] 
We integrate FLANDERS into Flower,\footnote{\scriptsize{\url{https://flower.dev/}}} a popular FL simulation framework for reproducibility.
\\
\item[{\em(iv)}] 
We show that FLANDERS improves the robustness of the existing aggregation methods under multiple settings: different datasets, client's data distribution (non-iid), models, and attack scenarios.
\\
\item[{\em(v)}] 
We publicly release all the implementation code of FLANDERS along with our experiments.\footnote{\scriptsize{\url{https://anonymous.4open.science/r/flanders_exp-7EEB}}}
\end{itemize}

% Paper's structure and organization
The remainder of the paper is structured as follows. %some related work and the current state-of-the-art solutions to security issues that FL entails. 
Section~\ref{sec:background} covers background and preliminaries. 
In Section~\ref{sec:related}, we discuss related work.
Section~\ref{sec:problem} and Section~\ref{sec:method} describe the problem formulation and the method proposed. % to tackle it. 
Section~\ref{sec:experiments} gathers experimental results. %, and Section~\ref{sec:limitations} discusses some limitations of this work.
Finally, we conclude in Section~\ref{sec:conclusion}.
 %discusses the limitations of this work and draws future research directions.
%reports conclusions and draws perspectives for future research directions.

%%%%%%% OLD %%%%%%%
%to overcome the resilience of Byzantine failures in distributed Stochastic Gradient Descent computations. 
% The strength of Krum is its time complexity, which is linear in the gradient dimension. 
% However, the robustness of the approach is guaranteed for gradient-based learning applications only when the majority of the clients are not compromised. 
% Besides, the aggregation mechanism of Krum, as well as that of similar methods, is robust from a coarse-grained perspective and does not provide solutions to errors and perturbations that may occur at inference time.
%A related approach to~\cite{blanchard2017nips} is the work of Su et al.~\cite{su2016dc}. Here, the authors propose an iterated approximate agreement to tackle a multi-layer scenario attacked by Byzantine agents. 
%However, the method works efficiently on the sole discrete context and it is inapplicable to continuous state environments.
%\gabri{Maybe, we should just talk about the main limitations of existing countermeasures without digging into their details (or, we can just mention Krum as this is the most popular one). I will move the description of all these methods to the Related Work section.}
\section{Minimizing Robust Loss Using an ERM Oracle \label{erm}}
\label{sec:non-realizable-oracle}

First, we show an example where the approach of \citet{xiang2022patchcleanser} of calling $\ERM_\calH$ on the inflated dataset, i.e., original training points plus all possible perturbations resulting from the allowed masking operations, fails by obtaining %the largest possible gap 
a multiplicative gap of $k-1$ %in true robust loss and apparent 0-1 loss on the inflated dataset.
in the robust loss between the optimal robust classifier and the classifer returned by $\ERM_\calH$.

\begin{exmpl}
\label{exmpl:ERM-failure}
We exhibit an example that achieves a multiplicative gap of $k-1$ in the robust loss gap between 
the inflated $\ERM$ solution and the optimal robust classifier, where $k$ is the size of the perturbation sets.
This gap exists because $\ERM$ can exhibit a solution that incorrectly classifies at least one perturbation per natural example, while there is a robust classifier that concentrates error on one natural example, thus getting low robust loss. %Our only/primary constraint for this problem is that the perturbation balls need to be finite. 
We can even show this type of example in one dimension with $x \in [-2,2]$ in Figure \ref{fig:example}.

\begin{figure}[ht]
\label{fig:example}
\centering
\includegraphics[width=0.9\textwidth]{figure.pdf}
\caption{Inflated ERM failure mode: Blacks points indicate adversarial perturbations. There are $2N$ examples, where half of them are true positives and the rest are true negatives. Each example has exactly $k=N$ perturbations. For $N-1$ of the true negatives, all their perturbations  are in $[-1,-1/2]$ and thus are benign. However, for one of the true negatives $x_{neg}^{*}$, all $N$ of its perturbations land exactly at $0$. 
For the positive case, this is flipped. There is one example $x_{pos}^{*}$ where all its perturbations are in $[1/2,1]$. For every other true positive example, however, one perturbation is at $0$ and the rest are in $[1/2,1]$.  $\ERM$ classifies the points at $0$ as negative points and exhibits $h_{ERM}$. This induces a high robust loss since %there is a perturbation targeting $N-1$ positives. 
$N-1$ of the true positives are not robustly classified.
Thus $h_{ERM}$ has robust loss $\frac{N-1}{2N} \sim 50 \%$ while $h_{ROB}^{*}$ correctly classifies all points (natural and adversarial) other than $x_{neg}^{*}$ and so has robust loss
$\frac{1}{2N}$, which obtains the worst case multiplicative gap for when $k=N$.
}
\end{figure}

\end{exmpl}

Next, we present our first contribution: an algorithm to learn a predictor that is simultaneously robust to a set of (polynomially many) masking operations, using an $\ERM_\calH$ oracle. The algorithm is based on prior work due to \citet{DBLP:conf/colt/FeigeMS15}, but the analysis and application are novel in this work. The main interesting feature of this algorithm is that it achieves stronger robustness guarantees in the non-realizable regime when $\OPT_\calH \gg 0$, where the approach of \citet{xiang2022patchcleanser} can fail as mentioned in~\prettyref{exmpl:ERM-failure}.

\begin{algorithm}%[H]
\begin{algorithmic}
\caption{\citet*{DBLP:conf/colt/FeigeMS15}}
\label{alg:FMS}
  \INPUT weight update parameter $\eta>0$, number of rounds $T$, and training dataset $S=\{(x_1,y_1),\dots, (x_m,y_m)\}$.
  %and corresponding weights $p_1,\dots,p_m$.}
  \STATE Set $w_1(z, (x,y)) = 1$, for each $(x,y)\in S, z\in \calU(x)$.\\
  \STATE Set $P^1(z,(x,y)) = \frac{w_1(z,(x,y))}{\sum_{z'\in\calU(x)} w_1(z',(x,y))}$, for each $(x,y)\in S, z\in \calU(x)$.\\
\FOR{each $t\in \{1,\cdots T\}$}
\STATE Call $\ERM$ on the empirical weighted distribution: 
\STATE {\small\[h_t = \argmin_{h\in\mathcal{H}} \sum_{(x,y)\in S} \sum_{z\in \calU(x)} \frac{1}{m} P^t(z,(x,y)) \ind\insquare{h_{t}(z)\neq y}\]}%.\\
\FOR{each $(x,y)\in S$ and $z\in\calU(x)$}
\STATE {\small $w_{t+1}(z,(x,y)) = (1+\eta \ind\insquare{h_{t}(z)\neq y}) \cdot w_{t}(z, (x,y))$}%
\STATE $P^{t+1}(z,(x,y))=\frac{w_t(z,(x,y))}{\sum_{z'\in\calU(x)} w_t(z',(x,y))}$.
\ENDFOR
\ENDFOR
%\KwOutput{

%\STATE {\bfseries Output:}
\OUTPUT The majority-vote predictor $\MAJ(h_1,\dots, h_T)$. \\
\end{algorithmic}
\end{algorithm}

\begin{thm}
\label{thm:generalization-FMS}
Set $T(\eps) = \frac{32 \ln k}{\eps^2}$ and $m(\eps, \delta) = O\inparen{\frac{\vc(\calH)(\ln k)^2}{\eps^4}\ln \inparen{\frac{\ln k}{\eps^2}}+\frac{\ln(1/\delta)}{\eps^2}}$. Then, for any distribution $\calD$ over $\calX\times \calY$, with probability at least $1-\delta$ over $S\sim \calD^{m(\eps,\delta)}$, running \prettyref{alg:FMS} for $T(\eps)$ rounds produces $h_1,\dots,h_{T(\eps)}$ satisfying:

\small\[\Ex_{(x,y)\sim \calD} \insquare{ \max_{z\in \calU(x)} \ind\insquare{\MAJ(h_1,\dots, h_{T(\eps)})(z)\neq y} } 
\leq 2\OPT_{\calH} + \eps\]
\end{thm}

\paragraph{Comparison with prior related work} As presented, \cite{DBLP:conf/colt/FeigeMS15} only considered \emph{finite} hypothesis classes $\calH$ and provided generalization guarantees depending on $\log\abs{\calH}$. On the other hand, we consider here infinite classes $\calH$ with bounded VC dimension and provide tighter robust generalization bounds. The robust learning guarantee \citep[][Theorem 2]{attias2022improved} assumes access to a \emph{robust} $\ERM$ oracle, which minimizes the robust loss on the training dataset. On the other hand, at the expense of higher sample complexity, we provide a robust learning guarantee using only an $\ERM$ oracle in the challenging \emph{non-realizable} setting. Prior work due to \citet{DBLP:conf/nips/MontasserHS20} considered using an $\ERM$ oracle for robust learning but only in the simpler realizable setting (when $\OPT_\calH=0$).


Before proceeding with the proof of \prettyref{thm:generalization-FMS}, we describe at a high-level the proof strategy. The main insight is to solve a finite zero-sum game. In particular, our goal is to find a mixed-strategy over the hypothesis class that is approximately close to the value of the game:
\[\OPT_{S,\calH} \triangleq \min_{h\in \calH} \frac{1}{m}\sum_{i=1}^{m} \max_{z_i\in \calU(x_i)} \ind\insquare{h(z_i)\neq y_i}.\]
 

We observe that \prettyref{alg:FMS} due to \citep{DBLP:conf/colt/FeigeMS15} solves a similar finite zero-sum game (see \prettyref{lem:FMS}), and then we relate it to the value of the game we are interested in (see \prettyref{lem:opt}). Combined together, this only establishes that we can minimize the robust loss on the empirical dataset using an $\ERM$ oracle. We then appeal to uniform convergence guarantees for the robust loss in \prettyref{lem:unif-robloss} to show that, with large enough training data, our output predictor achieves robust risk that is close to the value of the game. 

\begin{lem}
\label{lem:opt}
For any dataset $S %=\SET{(x_1,y_1),\dots,(x_m,y_m)}\in (\calX\times\calY)^m$,
=\{(x_1,y_1),\dots,(x_m,y_m)\}\in (\calX\times\calY)^m$,
\begin{align*}
    &\OPT_{S,\calH} = \min_{h\in \calH} \frac{1}{m}\sum_{i=1}^{m} \max_{z_i\in \calU(x_i)} \ind\insquare{h(z_i)\neq y_i} \geq
    \min_{Q\in \Delta(\calH)} \max_{\substack{P_{1}\in \Delta(\calU(x_1)),\\ \dots\\ P_{m} \in \Delta(\calU(x_m))}} \frac{1}{m} \sum_{i=1}^{m} \Ex_{z_i\sim P_i } \Ex_{h\sim Q} \ind\insquare{h(z_i)\neq y_i}
\end{align*}
\end{lem}




\begin{lem} [\citet*{DBLP:conf/colt/FeigeMS15}]
\label{lem:FMS}
For any data set $S %=\SET{(x_1,y_1),\dots,(x_m,y_m)}\in (\calX\times\calY)^m$,
=\{(x_1,y_1),\dots,(x_m,y_m)\}\in (\calX\times\calY)^m$, running \prettyref{alg:FMS} for $T$ rounds produces a mixed-strategy $\hat{Q} = \frac{1}{T} \sum_{t=1}^{T} h_t \in \Delta(\calH)$ satisfying:
{\small
\begin{align*}
    &\max_{\substack{P_1\in \Delta(\calU(x_1)),\\ \dots,\\ P_m\in \Delta(\calU(x_m))}} \frac{1}{m}\sum_{i=1}^{m} \Ex_{z_i\sim P_i} \frac{1}{T} \sum_{t=1}^{T} \ind\insquare{h_t(z_i)\neq y_i} \leq
    \min_{Q\in \Delta(\calH)} \max_{\substack{P_{1}\in \Delta(\calU(x_1)),\\ \dots,\\ P_{m} \in \Delta(\calU(x_m))}} \frac{1}{m} \sum_{i=1}^{m} \Ex_{z_i\sim P_i } \Ex_{h\sim Q} \ind\insquare{h(z_i)\neq y_i} + 2\sqrt{\frac{\ln k}{T}}
\end{align*}
}%
\end{lem}


\removed{
\begin{lem} [Extension to weighted samples]
For any data set $S %=\SET{(x_1,y_1),\dots,(x_m,y_m)}\in (\calX\times\calY)^m$
=\{(x_1,y_1),\dots,(x_m,y_m)\}\in (\calX\times\calY)^m$ and any corresponding weights $p_1,\dots, p_m > 0$ such that $\sum_{i=1}^{m} p_i = 1$, running \prettyref{alg:FMS} for $T$ rounds produces a mixed-strategy $\hat{Q} = \frac{1}{T} \sum_{t=1}^{T} h_t \in \Delta(\calH)$ satisfying:
\begin{align*}
    \max_{P_1\in \Delta(\calU(x_1)),\dots,P_m\in \Delta(\calU(x_m))} &\sum_{i=1}^{m} p_i\cdot \Ex_{z_i\sim P_i} \frac{1}{T} \sum_{t=1}^{T} \ind\insquare{h_t(z_i)\neq y_i} \leq\\
    &\min_{Q\in \Delta(\calH)} \max_{P_{1}\in \Delta(\calU(x_1)),\dots, P_{m} \in \Delta(\calU(x_m))} \sum_{i=1}^{m} p_i\cdot \Ex_{z_i\sim P_i } \Ex_{h\sim Q} \ind\insquare{h(z_i)\neq y_i} + 2\sqrt{\frac{\ln k}{T}}.
\end{align*}
\label{lem:extension-FMS-weights}
\end{lem}
}


\begin{lem} [VC Dimension for the Robust Loss \citep{attias2022improved}]
\label{lem:unif-robloss}
For any class $\calH$ and any $\calU$ such that $\sup_{x\in\calX}\abs{\calU(x)}\leq k$, denote the robust loss class of $\calH$ with respect to $\calU$ by
\[\calL^{\calU}_{\calH} = \{(x,y)\mapsto \max_{z\in\calU(x)} \ind\insquare{h(z)\neq y}: h\in\calH\}.\]
Then, it holds that $\vc(\calL^{\calU}_{\calH})\leq O(\vc(\calH) \log(k))$. 
\end{lem}

We are now ready to proceed with the proof of \prettyref{thm:generalization-FMS}.

\begin{proof}[Proof of~\prettyref{thm:generalization-FMS}]
Let $S\sim \calD^m$ be an iid sample from $\calD$, where the size of the sample $m$ will be determined later. By invoking \prettyref{lem:FMS} and \prettyref{lem:opt}, we observe that running \prettyref{alg:FMS} on $S$ for $T$ rounds, produces $h_1,\dots, h_{T}$ satisfying

{\small
\[\max_{\substack{P_1\in \Delta(\calU(x_1)),\\ \dots,\\ P_m\in \Delta(\calU(x_m))}} \frac{1}{m}\sum_{i=1}^{m} \Ex_{z_i\sim P_i} \frac{1}{T} \sum_{t=1}^{T} \ind\insquare{h_t(z_i)\neq y_i} \leq \OPT_{S,\calH} + \frac{\varepsilon}{4}
\]
}
Next, the average robust loss for the majority-vote predictor $\MAJ(h_1,\dots, h_T)$ can be bounded from above as follows:
\begin{align*}
    &\frac{1}{m} \sum_{i=1}^{m} \max_{z_i\in \calU(x_i)} \ind\insquare{\MAJ(h_1,\dots,h_T)(z_i)\neq y_i}\\
    &\leq \frac{1}{m} \sum_{i=1}^{m} \max_{z_i\in \calU(x_i)}  2 \Ex_{t\sim [T]}\ind\insquare{h_t(z_i)\neq y_i}\\
    &= 2 \frac{1}{m} \sum_{i=1}^{m} \max_{z_i\in \calU(x_i)} \frac{1}{T} \sum_{t=1}^{T} \ind\insquare{h_t(z_i)\neq y_i}\\
    &\leq 2 \max_{\substack{P_1\in \Delta(\calU(x_1)),\\ \dots,\\ P_m\in \Delta(\calU(x_m))}} \frac{1}{m}\sum_{i=1}^{m} \Ex_{z_i\sim P_i} \frac{1}{T} \sum_{t=1}^{T} \ind\insquare{h_t(z_i)\neq y_i}\\
    &\leq 2 \OPT_{S,\calH} + \frac\varepsilon2.
\end{align*}
Next, we invoke \prettyref{lem:unif-robloss} to obtain a uniform convergence guarantee on the robust loss. In particular, we apply \prettyref{lem:unif-robloss} on the \emph{convex-hull} of $\calH$: $\calH^{T} = \{\MAJ(h_1,\dots, h_T): h_1,\dots, h_T\in \calH\}$. By a classic result due to \citet{blumer:89}, it holds that $\vc(\calH^T)=O(\vc(\calH)T\ln T)$. Combining this with \prettyref{lem:unif-robloss} and plugging-in the value of $T= \frac{32 \ln k}{\varepsilon^2}$, we get that the VC dimension of the robust loss class of $\calH^T$ is bounded from above by
\[\vc(\calL_{\calH^T}^\calU) \leq O\inparen{\frac{\vc(\calH)(\ln k)^2}{\varepsilon^2}\ln\inparen{\frac{\ln k}{\varepsilon^2}}}.\]
Finally, using Vapnik's ``General Learning'' uniform convergence \citep{vapnik:82}, with probability at least $1-\delta$ over $S\sim \calD^m$ where $m =  O\inparen{\frac{\vc(\calH)(\ln k)^2}{\varepsilon^4}\ln \inparen{\frac{\ln k}{\varepsilon^2}}+\frac{\ln(1/\delta)}{\varepsilon^2}}$, it holds that
\begin{align*}
&\forall f\in \calH^T: \Ex_{(x,y)\sim \calD} \insquare{\max_{z\in\calU(x)}\ind\insquare{f(z)\neq y}} 
\leq \frac{1}{m}\sum_{i=1}^{m} \max_{z_i\in\calU(x_i)}\ind\insquare{f(z_i)\neq y_i} + \frac\varepsilon4
\end{align*}
This also applies to the particular output $\MAJ(h_1,\dots, h_T)$ of \prettyref{alg:FMS}, and thus
\begin{align*}
    &\Ex_{(x,y)\sim \calD} \insquare{ \max_{z\in \calU(x)} \ind\insquare{\MAJ(h_1,\dots, h_{T(\varepsilon)})(z)\neq y} } \\
    &\leq \frac{1}{m} \sum_{i=1}^{m} \max_{z_i\in \calU(x_i)} \ind\insquare{\MAJ(h_1,\dots,h_T)(z_i)\neq y_i} + \frac{\varepsilon}{4}\\
    &\leq 2\OPT_{S,\calH} + \frac\varepsilon2 + \frac\varepsilon4.
\end{align*}

Finally, by applying a standard Chernoff-Hoeffding concentration inequality, we get that $\OPT_{S,\calH} \leq \OPT_\calH + \frac\varepsilon8$. Combining this with the above inequality concludes the proof.
\end{proof}
\section{Multi-robustness guarantees on a set of groups}
\label{sec:multi-robustness}
In this section, we consider the problem of learning a predictor that has low robust loss across multiple groups. This objective can be justified from a fairness perspective. For instance, when the groups correspond to sensitive social or demographic groups like gender and race, ensuring reliable performance on each group is crucial. In this setting, if we use \prettyref{alg:FMS} to minimize the overall robust loss, it may result in concentrating the overall robust loss on a few groups, instead of spreading the loss across many groups. We propose a boosting algorithm that learns a predictor with low robust loss on all the groups simultaneously. We present our algorithm for the case of disjoint groups in~\prettyref{sec:multi-robust}. In the case of overlapping groups, we show a reduction to the case of disjoint groups in~\prettyref{sec:overlapping-groups-reduction}.

Suppose that the training dataset $S$ is partitioned into $g$ groups $\calG=\{G_1,\dots,G_g\}$. Robust loss of a predictor $h$ on group $G_j$ is defined as follows:
\begin{align}
&\RLoss_j(h)=\frac{1}{|G_j|}\sum_{(x,y)\in G_j}\max_{z\in\mathcal{U}(x)}\ind[h(z)\neq y]\label{eqn:unweighted-robust-loss}
\end{align}


The learning benchmark that we compete with on a dataset $S$ for the robust loss on each group is $\OPT^{S}_{\max}$ that is defined as follows:
%\begin{defn}
%\label{defn: optmax}
    %\[ 
\begin{align}
&\OPT^{S}_{\max}=\min_{h\in \mathcal{H}}\max_{j\in[g]}\frac{1}{|G_j|}\sum_{(x,y)\in G_j}\max_{z\in\mathcal{U}(x)}\ind[h(z)\neq y]
\label{defn: optmax}
\end{align}
    %\]
%\end{defn}


In the following, we formalize the notion of multi-robustness over a set of groups $\mathcal{G}=\{G_1,\dots,G_g\}$:

\begin{defn}[Multi-Robustness]
\label{defn:multirob}
A hypothesis $h$ is multi-robust on a dataset $S$ if it achieves the following guarantee:
\begin{align*}
&\max_{j\in[g]}\frac{1}{|G_j|}\sum_{(x,y)\in G_j}\max_{z\in\mathcal{U}(x)}\ind[h(z)\neq y]\leq \OPT^S_{\max}+\eps 
%\label{eq:multi-robustness}
\end{align*}
\label{def:multi-robustness}
\end{defn}
A hypothesis $h$ satisfies Definition \ref{defn:multirob} if it is within $\varepsilon$ robust loss of the min-max optimal classifier where the adversary gets to have two maximization options, e.g. maximizing over the worst-off group and for each example $x$ in that group, picking a worst-case perturbation. 

\begin{defn}[$\beta$-Multi-Robustness]
A hypothesis $h$ is $\beta$-multi-robust on a dataset $S$ if it achieves the following guarantee:
\begin{align*}
&\max_{j\in[g]}\frac{1}{|G_j|}\sum_{(x,y)\in G_j}\max_{z\in\mathcal{U}(x)}\ind[h(z)\neq y]\leq \beta(\OPT^{S}_{\max}+\eps) \label{eq:beta-multi-robustness}
\end{align*}
\end{defn}

\begin{defn}[Multi-Robustness on Average]
A set of hypotheses $\mathcal{H'}=\{h_1,\dots,h_T\}$ is multi-robust on a dataset $S$ on average if the the following property holds:
\[\frac{1}{T}\max_{j\in [g]}\sum_{t=1}^T\RLoss_j(h_t)\leq \OPT^{S}_{\max}+\eps\]
\label{def:avg-multi-robustness}
\end{defn}

\begin{rem}
~\prettyref{def:multi-robustness} is a stronger notion of multi-robustness compared to~\prettyref{def:avg-multi-robustness}. 
\end{rem}

In~\prettyref{sec:multi-robust}, we propose~\prettyref{alg:boosting} which is a two-layer boosting algorithm that achieves multi-robustness on the dataset $S$. %~\prettyref{alg:boosting} returns a set of hypotheses $\calH'=\{h_1,\dots,h_T\}$. 
First, we show that $\calH'=\{h_1,\dots,h_T\}$ returned by~\prettyref{alg:boosting} is multi-robust on average (\prettyref{thm:randomized-multi-robustness}).~\prettyref{thm:deterministic-multi-robustness} exhibits that the majority-vote classifier over $\calH'$, i.e. $\MAJ(h_1,\dots,h_T)$, obtains $\beta$-multi-robustness for $\beta<12$. We remark that although~\prettyref{thm:randomized-multi-robustness} achieves a tighter upper bound on the multi-robustness guarantee,~\prettyref{thm:deterministic-multi-robustness} gives a guarantee for the stronger notion of multi-robustness. Our boosting algorithm makes oracle calls to an extension of~\prettyref{alg:FMS} to weighted samples that is proposed in~\prettyref{alg:weighted-FMS}. In~\prettyref{lem:extension-FMS-weights}, we show an extension of~\prettyref{lem:FMS} holds for~\prettyref{alg:weighted-FMS}.

\begin{algorithm}
\caption{Extension of \citet*{DBLP:conf/colt/FeigeMS15} to Weighted Samples}
\label{alg:weighted-FMS}
\begin{algorithmic}[1]
  \INPUT weight update parameter $\eta>0$, training dataset $S=\{(x_1,y_1),\cdots, (x_m,y_m)\}$ and corresponding weights $p_1,\cdots,p_m$.
  \STATE Set $w_1(z, (x,y)) = 1$ for each $(x,y)\in S, z\in \calU(x)$.
  \STATE Set $P^1(z,(x,y)) = \frac{w_1(z,(x,y))}{\sum_{z'\in\calU(x)} w_1(z',(x,y))}$.
\FOR{$t\in 1,\dots,T$}

\STATE Call $\ERM$ on the empirical \emph{weighted distribution}: 
{\small\[h_t = \argmin_{h\in\mathcal{H}} \sum_{(x,y)\in S} \sum_{z\in \calU(x)} p_{(x,y)} P^t(z,(x,y)) \ind\insquare{h_{t}(z)\neq y}\]}%
\FOR{each $(x,y)\in S$ and $z\in\calU(x)$}
\STATE {\small $w_{t+1}(z,(x,y)) = (1+\eta \ind\insquare{h_{t}(z)\neq y}) \cdot w_{t}(z, (x,y))$}%
\STATE $P^{t+1}(z,(x,y))=\frac{w_t(z,(x,y))}{\sum_{z'\in\calU(x)} w_t(z',(x,y))}$.
\ENDFOR
\ENDFOR
\OUTPUT The majority-vote predictor $\MAJ(h_1,\dots, h_T)$. 
\end{algorithmic}
\end{algorithm}

 


\begin{lem} [Extension to weighted samples]
For any dataset $S =\{(x_1,y_1),\dots,(x_m,y_m)\}\in (\calX\times\calY)^m$ and any corresponding weights $p_1,\dots, p_m > 0$ such that $\sum_{i=1}^{m} p_i = 1$, running
%FMS
Algorithm \ref{alg:weighted-FMS}
for $T$ rounds produces a mixed-strategy $\hat{Q} = \frac{1}{T} \sum_{t=1}^{T} h_t \in \Delta(\calH)$ satisfying:
{\small
\begin{align*}
   &\max_{\substack{P_1\in \Delta(\calU(x_1)),\\ \dots,\\P_m\in \Delta(\calU(x_m))}} \sum_{i=1}^{m} p_i\cdot \Ex_{z_i\sim P_i} \frac{1}{T} \sum_{t=1}^{T} \ind\insquare{h_t(z_i)\neq y_i} \leq
    \min_{Q\in \Delta(\calH)} \max_{\substack{P_{1}\in \Delta(\calU(x_1)),\\ \dots, \\P_{m} \in \Delta(\calU(x_m))}} \sum_{i=1}^{m} p_i\cdot \Ex_{z_i\sim P_i } \Ex_{h\sim Q} \ind\insquare{h(z_i)\neq y_i} 
    + 2\sqrt{\frac{\ln k}{T}}
\end{align*}
}%
\label{lem:extension-FMS-weights}
\end{lem}

\subsection{Boosting algorithm achieving multi-robustness guarantees:}
\label{sec:multi-robust} 
In this section, we present~\prettyref{alg:boosting} that obtains %both multi-robustness on average and $\beta$-multi-robustness for $\beta<12$
multi-robustness guarantees. The algorithm follows the idea proposed by~\citet{freund1996game} that obtains boosting by playing a repeated game. Initially a sample set $S=\{(x_1,y_1),\dots,(x_m,y_m)\}$ partitioned into a set of disjoint groups $\mathcal{G}=\{G_1,\dots,G_{g}\}$ is received as input. %First, the weight of each group $G_j$ is set to $1/g$. At each round of boosting, initially, group weights are normalized. 
$P_j^t$ shows the normalized weight of group $G_j$ in step $t$. Initially, for each group $G_j$, $P^t_j=1/g$.
In each round $t$, the weight of each group gets split between its examples equally: $p_i = P^t_j/|G_j|$ where $(x_i,y_i)\in G_j$. Subsequently, an oracle call is made to~\prettyref{alg:weighted-FMS} with sample weights $p_1,\dots,p_m$.~\prettyref{lem:avg-robust-loss-upper-bound} shows that at each iteration $t$,~\prettyref{alg:weighted-FMS} returns a hypothesis $h_t$ such that its average robust loss across the groups is at most $\OPT^{S}_{\max}+\eps$. In the next iteration $t+1$, for each group $G_j$, the weights of examples in $G_j$ get decreased by a multiplicative factor of 
$1-\delta m_j^{\text{rob}}(h_t)$ where $m_j^{\text{rob}}(h_t)=1-\RLoss_j(h_t)$ and $\delta=\sqrt{{\ln g}/{T}}$.~\prettyref{thm:randomized-multi-robustness} exhibits that after %the average multi-robustness guarantee: for 
$T=\calO({\ln g}/{\eps^2})$ rounds,~\prettyref{alg:boosting} outputs
a set of hypotheses $\mathcal{H'}=\{h_1,\dots,h_T\}$ such that for each group $G_j$ the average multi-robustness guarantee is obtained, i.e., $\frac{1}{T}\sum_{t=1}^T \RLoss_j(h_t)\leq \OPT^S_{\max}+\eps$.~\prettyref{thm:deterministic-multi-robustness} provides that $\MAJ(h_1,\dots,h_t)$ achieves
$\beta$-multi-robustness guarantee for $\beta<12$.



\begin{algorithm}[H]
\caption{Boosting Algorithm Achieving Multi-Robustness Guarantee}
\label{alg:boosting}
\begin{algorithmic}
    \INPUT training dataset $S=\{(x_1,y_1),\dots,(x_m,y_m)\}$ partitioned into a set of groups groups $\{G_1,\cdots,G_g\}$
    \STATE Initially, $\forall 1\leq j\leq g: P_j^t = 1/g$
    \FOR{$t=1,\dots,T$}
    \STATE $p_i = P^t_j/|G_j|$ where $(x_i,y_i)\in G_j$
    \STATE Call~\prettyref{alg:weighted-FMS} on $S$ with weights $(p_1,\dots,p_m)$ and get back a predictor $h_t$ such that:
    \STATE \[\E_{j\sim P^t}[\ell^{rob}_j(h_t)]=\sum_{j\in[g]}P_j^{t}\ell_j^{rob}(h_t)\leq \OPT^S_{\max}+\eps\]
    \STATE Update $P^t_j,  \text{ for all }j\in[g]$:
    %\STATE \[ P^{t+1}_j=\frac{P^t_j\cdot\Biggl(1-\delta\Bigl(1-\RLoss_j(h_t)\Bigr)\Biggr)}{Z_t}\]
    \STATE \[ P^{t+1}_j=\frac{P^t_j\cdot\left(1-\delta m_j^{\text{rob}}(h_t)\right)}{Z_t}\]
    where $m_j^{\text{rob}}(h_t)=1-\RLoss_j(h_t)$, $Z_t$ is a normalization factor, and     $\delta=\sqrt{\frac{\ln g}{T}}$.
    \ENDFOR
    \OUTPUT 
    %a predictor $h$ selected uniformly at random from 
    $\mathcal{H'}=\{h_1,\cdots,h_T\}$%, and the majority-vote predictor over $\calH'$: $\MAJ(h_1,\dots,h_T)$
\end{algorithmic}
\end{algorithm}

\begin{rem}
We remark that the output of~\prettyref{alg:boosting} is a set of majority-vote classifiers over $\calH$:
\[\calH'=\Big\{\MAJ(h_{1,1},\dots, h_{1,T'}),\dots,\MAJ(h_{T,1},\dots, h_{T,T'}): \forall i\in[T], \forall j\in[T'], h_{i,j}\in \calH\Big\}\]
\end{rem}

Before proving the multi-robustness guarantees, we show~\prettyref{lem:avg-robust-loss-upper-bound} holds. Next, we restate the guarantee of the Multiplicative Weights algorithm that is a generalization of \emph{Weighted Majority} algorithm~\citep{littlestone1994weighted} and is equivalent to \emph{Hedge} developed by~\citet{freund1997decision}.

\begin{lem}
In each round $t$ of~\prettyref{alg:boosting}, by making an oracle-call to~\prettyref{alg:weighted-FMS} after $T'=\frac{4\ln k}{\eps^2}$ rounds, %we can find 
a hypothesis $h_t$ is outputted such that $\E_{j\sim P^t}[\ell^{rob}_j(h_t)]=\sum_{j\in[g]}P_j^{t}\ell_j^{rob}(h_t)\leq \OPT^S_{\max}+\eps$.
\label{lem:avg-robust-loss-upper-bound}
\end{lem}

\begin{thm}[Mutiplicative Weights Algorithm \citep{kale2007efficient}]
\label{thm:MW_alg}
For any sequence of costs of experts $\vec{m}_1,\cdots,\vec{m}_T$ revealed by nature where all the costs are in $[0,1]$, the sequence of mixed strategies $\vec{p}_1,\cdots,\vec{p}_T$ produced by the Multiplicative Weights algorithm satisfies:
\[\sum_{t=1}^T \vec{m}_t\cdot \vec{p}_t\leq (1+\delta)\min_{\vec{p}}\sum_{t=1}^T\vec{m}_t\cdot \vec{p}+\frac{\ln n}{\delta}\]
where $n$ is the number of experts.
\end{thm}

\begin{thm}
\label{thm:avg-empirical-boosting}
When $T=\calO(\frac{\ln g}{\eps^2})$,~\prettyref{alg:boosting} computes a set of hypotheses $\mathcal{H}'=\{h_1,\cdots,h_T\}$, such that for each group $G_j$, 
$\frac{1}{T}\sum_{t=1}^T\RLoss_j(h_t)\leq \OPT^S_{\max}+\eps$.
\label{thm:randomized-multi-robustness}
\end{thm}
\begin{proof}

In each iteration $t$, we define average loss and reward terms as follows:

\[L(P^t,h_t)=\E_{j\sim P_t}\Big[\RLoss_j(h_t)\Big]=\sum_{j\in[g]}P^t_j\RLoss_j(h_t), \quad M(P^t,h_t)=\E_{j\sim P_t}\Big[m_j^{\text{rob}}(h_t)\Big]\]
Substituting $\RLoss_j(h_t)=1-m_j^{\text{rob}}(h_t)$ provides:
\[M(P^t,h_t)=\sum_{j\in[g]}P^t_j(1-\RLoss_j(h_t))=1-\sum_{j\in[g]}P^t_j\RLoss_j(h_t)=1-L(P^t,h_t)\]
Now by setting $T=\frac{9\ln g}{\eps^2}$ which implies that $\delta=\sqrt{\frac{\ln g}{T}}=\frac{\eps}{3}$, and by using the guarantee of~\prettyref{thm:MW_alg}, the following bound is obtained.
\begin{align*}
&\frac{1}{T}\sum_{t=1}^T M(P^t,h_t)\leq \frac{(1+\delta)}{T}\min_{j\in[g]}\sum_{t=1}^T M(j,h_t)+\frac{\ln g}{\delta T}\\
&\rightarrow \frac{1}{T}\sum_{t=1}^T M(P^t,h_t)\leq \frac{1}{T}\min_{j\in[g]}\sum_{t=1}^T M(j,h_t)+\delta+\frac{\ln g}{\delta T}\\
%&\rightarrow \frac{1}{T}\sum_{t=1}^T M(P^t,h_t)\leq \frac{1}{T}\min_{j\in[g]}\sum_{t=1}^T M(j,h_t)+2\sqrt{\frac{\ln g}{T}}
&\rightarrow \frac{1}{T}\sum_{t=1}^T M(P^t,h_t)\leq \frac{1}{T}\min_{j\in[g]}\sum_{t=1}^T M(j,h_t)+\frac{2\eps}{3}
\end{align*}
where $M(j,h_t)$ is the reward term when the whole probability mass is concentrated on group $G_j$. Therefore for each group $j\in[g]$:
\begin{align}
&\frac{1}{T}\sum_{t=1}^T M(j,h_t)\geq \frac{1}{T}\sum_{t=1}^T M(P^t,h_t)-\frac{2\eps}{3}
\label{eqn:randomized-boosting-eq1}
\end{align}

\prettyref{lem:avg-robust-loss-upper-bound} provides that in each iteration $t$, $L(P^t,h_t)\leq \OPTSMax+\eps/3$ given that~\prettyref{alg:weighted-FMS} is 
executed for $T'=\frac{36\ln k}{\eps^2}$ rounds.
Thus, at each iteration $t$, $M(P^t,h_t)\geq 1-(\OPTSMax+\eps/3)$. Therefore, $\frac{1}{T}\sum_{t=1}^T M(P^t,h_t)\geq 1-(\OPTSMax+\eps/3)$; combining with~\prettyref{eqn:randomized-boosting-eq1} implies that:
\begin{align*}
&\frac{1}{T}\sum_{t=1}^T M(j,h_t)\geq \frac{1}{T}\sum_{t=1}^T M(P^t,h_t)-\frac{2\eps}{3}
\geq 1-(\OPT^S_{\max}+\frac{\eps}{3})-\frac{2\eps}{3}=1-(\OPT^S_{\max}+\eps)
\end{align*}
Plugging in the definition of $L(P^t,h_t)$ implies that:
\[\frac{1}{T}\sum_{t=1}^T L(j,h_t)\leq \OPT^S_{\max}+\eps\]
Which concludes the proof.
\end{proof}

\begin{cor}
~\prettyref{thm:randomized-multi-robustness} implies that if for each example a predictor is picked uniformly at random from $\calH'$ to predict its label, then for each group $G_j\in \calG$, the expected robust loss is at most $\OPT^S_{\max}+\eps$.
\label{cor:interpret-avg-loss}
\end{cor}

\begin{thm}
When $T=\calO(\frac{\ln g}{\eps^2})$,~\prettyref{alg:boosting} computes a set of hypotheses $\calH'=\{h_1,\dots,h_T\}$ such that for each group $G_j$, $\RLoss_j(\MAJ(h_1,\cdots,h_T))\leq 12(\OPT^S_{\max}+\eps)$.
\label{thm:deterministic-multi-robustness}
\end{thm}

\begin{proof}
By~\prettyref{thm:randomized-multi-robustness}, after $T=\calO(\frac{\ln g}{\eps^2})$ rounds, for each group $G_j$, $\frac{1}{T}\sum_{t=1}^T \RLoss_j(h_t)\leq \OPT^S_{\max}+\eps$. Therefore, at most $1/c$ of the classifiers $\mathcal{H'}=\{h_1,\cdots,h_T\}$ have robust loss greater than $c(\OPT^S_{\max}+\eps)$ on $G_j$. Let's call the set of these classifiers $\mathcal{H}''$. The total number of robustness mistakes on $G_j$ across all the classifiers in $\mathcal{H}'\setminus\mathcal{H}''$ is at most $T(1-1/c)c(\OPT^S_{\max}+\eps)|G_j|$ which is equal to:
{\small
\[T(1-1/c)c(\OPT^S_{\max}+\eps)|G_j|=(T/2-T/c).\frac{2c(c-1)}{c-2}(\OPT^S_{\max}+\eps)|G_j|\]
}%
Therefore, the fraction of examples in $G_j$ that at least $T/2-T/c$ of classifiers in $\mathcal{H}'\setminus\mathcal{H}''$ make a robustness mistake on is at most $\frac{2c(c-1)}{c-2}(\OPT^S_{\max}+\eps)$. Let $T_j$ denote the set of these examples. Thus, for each example in $G_j\setminus T_j$, at least $T-T/c-(T/2-T/c-1)=T/2+1$ of the classifiers in $\calH'\setminus \calH''$ are making no robustness mistakes on them, i.e., classifying all their perturbations correctly.
Hence, the fraction of examples in $G_j$ that are not robustly classified by the majority-vote classifier is at most $\frac{2c(c-1)}{c-2}(\OPT^S_{\max}+\eps)$. To find the best value of $c$, we solve the following optimization problem:
\begin{align*}
&\min \frac{2c(c-1)}{c-2}\\
&\text{s.t. } c>2
\end{align*}
Which gives $c\approx 3.41421$, and $\frac{2c(c-1)}{c-2}\approx 11.6569$.
\end{proof}

\subsection{Reduction from overlapping groups to disjoint groups}
\label{sec:overlapping-groups-reduction}
When the groups are overlapping, we reduce it to the case of disjoint groups. The reduction is as follows: for an input instance $\calI(\calG=\{G_1,\dots,G_g\}, S)$ of overlapping groups, create a new instance $\calI'(\calG'=\{G'_1,\dots,G'_g\},S')$ as follows. Initially, for all $G'_j\in \calG'$, $G'_j$ is an empty set. For each example $(x_i,y_i)\in S$ that belongs to a set of groups $\calG_i=\{G_{i,1},\cdots, G_{i,|\calG_i|}\}\subseteq \calG$ in $\calI$, create identical copies of $(x_i,y_i)$ and assign each copy including the original example to exactly one of the groups in $\calG'_i=\{G'_{i,1},\cdots, G'_{i,|\calG'_i|}\}$. Now we have an instance $\calI'$ with disjoint groups. By executing~\prettyref{alg:boosting} on $\calI'$, it returns a predictor $h$ that achieves %$\beta$-multi-robustness guarantee of at most $\beta\cdot \OPT^{\calI'}_{\max}$ on it. 
a $\beta$-multi-robustness guarantee. First, we argue that if $h$ is used on $\calI$, it achieves a multi-robustness guarantee of $\beta\cdot \OPT^{\calI'}_{\max}$. This is the case since either $h$ makes a robustness mistake on all copies of an example or does not make any robustness mistakes on any of them. Next, we show that $\OPT^{\calI'}_{\max}\leq \OPT^{\calI}_{\max}$. Consider a predictor $h^*\in \calH$ that achieves multi-robustness of $\OPT^{\calI}_{\max}$ on $\calI$. If $h^*$ is used on $\calI'$, for each example $(x,y)\in S$ that $h^*$ has zero robust loss on, it does not make any mistakes on any of its copies in $\calI'$. Additionally, if $h^*$ makes a robustness mistake on $(x,y)$, then it makes a robustness mistake on all its copies in $\calI'$. Thus, $h^*$ achieves a multi-robustness guarantee of $\OPT^{\calI}_{\max}$ on $\calI'$. Therefore, $\OPT^{\calI'}_{\max}\leq \OPT^{\calI}_{\max}$, and a $\beta\cdot \OPT^{\calI'}_{\max}$ multi-robustness guarantee on $\calI$ implies $\beta\cdot \OPT^{\calI}_{\max}$ multi-robustness. A similar argument holds for the average multi-robustness guarantee.

\begin{rem}
When $|\calG|$ is large, this reduction becomes computationally inefficient, since in the worst case, the number of samples gets increased by a multiplicative factor of $|\calG|$. However, this reduction is equivalent to keeping only one copy of each sample $(x_i,y_i)\in S$ and when executing~\prettyref{alg:boosting}, in each iteration $t$, assigning it a weight of $p_i=\sum_{j\in[g]:(x_i,y_i)\in G_j}P^t_j/|G_j|$.
\end{rem}

\subsection{Generalization Guarantees}
\label{sec:generalization-guarantees}
In this section, we derive generalization guarantees for multi-robustness. First,~\prettyref{lem:vc-robustloss-groups} shows how to bound the VC-Dimension of the intersection of robust loss and groups. We can then invoke this Lemma to get uniform convergence guarantees that will allow us to get concentration for the conditional robust loss across groups (see \prettyref{def:multi-robustness}).

\begin{lem} [VC Dimension of Intersection of Robust Loss and Groups]
\label{lem:vc-robustloss-groups}
For any class $\calH$, any perturbation set $\calU$, and any group class $\calG$, denote the intersection function class by
\[\calF^\calU_{\calH,\calG} \triangleq \{ (x,y)\mapsto \max_{z\in\calU(x)} \ind\insquare{h(z)\neq y} \wedge g(x): h\in\calH, g\in\calG \}.\]
Then, it holds that $\vc(\calF^\calU_{\calH,\calG}) \leq \Tilde{O}\inparen{\vc(\calL^{\calU}_{\calH}) + \vc(\calG)}$.
\end{lem}



\begin{thm}[Generalization guarantees for average multi-robustness]
\label{thm:generalization-multi-groups}
With $T=\calO(\ln g/\varepsilon^2)$ and $m= \Tilde{O}\inparen{\frac{\vc(\calH)\ln^2(k)}{\eps^4}+\frac{\vc(\calG) + \ln(1/\delta)}{\varepsilon^2}}$,
~\prettyref{alg:boosting} computes a set of hypotheses $\calH'=\{h_1,\dots, h_T\}$, such that $\forall G_j\in \calG$, 
\begin{align*}
&\frac{1}{T}\sum_{t=1}^T\Prob_{(x,y)\in \calD}\Big[\exists z\in \calU(x): h_t(z)\neq y \mid x\in G_j \Big]\leq\\
&~~~~~~~~~~~~~~~~~\inparen{1 + \frac{\varepsilon}{\Prob_{\calD}(x\in G_j)}}\inparen{\OPTSMax + \varepsilon}+\frac{\varepsilon}{\Prob_{\calD}(x\in G_j)}
\end{align*}
\end{thm}


\begin{thm}[Generalization guarantees for $\beta$-multi-robustness]
\label{thm:generalization-multi-groups-deterministic}
With $T=\calO(\ln g/\varepsilon^2)$ and 

$m=\Tilde{O}\inparen{\frac{\vc(\calH)\ln(g)\ln^2(k)}{\varepsilon^6}+\frac{\vc(\calG) + \ln(1/\delta)}{\varepsilon^2}}$,
~\prettyref{alg:boosting} computes a set of hypotheses $\calH'=\{h_1,\dots, h_T\}$, such that $\forall G_j\in \calG$, 
\begin{align*}
&\Prob_{(x,y)\in \calD}\Big[\exists z\in \calU(x): \MAJ(h_1,\dots,h_T)(z)\neq y \mid x\in G_j \Big]\leq\\
&~~~~~~~~~~~~\inparen{1 + \frac{\varepsilon}{\Prob_{\calD}(x\in G_j)}}\inparen{\beta(\OPTSMax + \varepsilon)}+\frac{\varepsilon}{\Prob_{\calD}(x\in G_j)}
\end{align*}
\end{thm}

\begin{rem}
In~\prettyref{sec:proof-generalization-guarantees-deterministic}, we show how to achieve generalization guarantees in terms of $\OPTDMax$ instead of $\OPTSMax$.
\end{rem}

\section{Conclusion}
In this work, we introduced a  counter-example showing how inflated $\ERM$ in the non-realizable setting can fail in robust learning. Then, we provided a %two-layer boosting 
``boosting-style'' algorithm that uses $\ERM$ %differently 
and obtains strong robust learning guarantees in the challenging non-realizable regime. We have also introduced a new multi-robustness objective and provided algorithms that obtain robustness guarantees simultaneously across groups. Future work can continue to explore the multi-robustness objective, connections with multi-calibration (and related `multi' notions), and other ways to use $\ERM$ oracles to learn robust classifiers. 


\subsection*{Acknowledgements}
This work was supported in part by the National Science Foundation under grants CCF-2212968 and ECCS-2216899, by the Simons Foundation under the Simons Collaboration on the Theory of Algorithmic Fairness, and by the Defense Advanced Research Projects Agency under cooperative agreement HR00112020003. The views expressed in this work do not necessarily reflect the position or the policy of the Government and no official endorsement should be inferred. Approved for public release; distribution is unlimited. This work was done when OM was a PhD student at the Toyota Technological Institute at Chicago.
\bibliography{ref}
\bibliographystyle{plainnat}
\clearpage
\appendix
\section{Appendix for Proofs}

\paragraph{Proof of Theorem \ref{thm:main}.}

\begin{proof}
\label{proof:main}
Our proof has two steps. In Step 1, we will show that SimCLR is equivalent to minimizing the cross entropy loss defined in Eqn.~(\ref{eqn:cross-entropy}). 
In Step 2, we will show  that minimizing the cross-entropy loss 
is equivalent to spectral clustering on $\bfpi$. 
Combining the two steps together, we have proved our theorem. 

\textbf{Step 1: } SimCLR is equivalent to minimizing the cross entropy loss.

The cross-entropy loss takes expectation over 
$\bfW_\bfX\sim \mathbb{P}(\cdot ; \bfpi)$, 
which means $\bfW_\bfX$ has exactly one non-zero entry in each row $i$. By Lemma~\ref{lem:multinomial}, we know every row $i$ of $\bfW_\bfX$ is independent of other rows. Moreover, 
$\bfW_{\bfX,i}\sim \mathcal{M}(1, \bfpi_i/\sum_j \bfpi_{i,j})=\mathcal{M}(1, \bfpi_i)$, because $\bfpi_i$ itself is a probability distribution.
Similarly, we know $\bfW_\bfZ$ also has the row-independent property by sampling over $\mathbb{P}(\cdot;\bfK_\bfZ)$.
Therefore, by Lemma~\ref{lem:cross_split}, we know Eqn.~(\ref{eqn:cross-entropy}) is equivalent to:
\[
 -\sum_{i=1}^n \mathbb{E}_{\bfW_{\bfX,i}}[\log \mathbb{P}(\bfW_{\bfZ,i}=\bfW_{\bfX,i};\bfK_\bfZ)],
\]

This expression takes expectation over $\bfW_{\bfX,i}$ for the given row $i$. Notice that 
$\bfW_{\bfX,i}$ has exactly one non-zero entry, which equals $1$ (same for $\bfW_{\bfZ,i}$). 
As a result
we expand the above expression to be:
\begin{equation}
 -\sum_{i=1}^n \sum_{j\neq i} \Pr(\bfW_{\bfX,i,j}=1)\log \Pr(\bfW_{\bfZ,i,j}=1).
\label{eqn:detailed-expansion}    
\end{equation}


By Lemma~\ref{lem:multinomial}, $\Pr(\bfW_{\bfZ,i,j}=1)=\bfK_{\bfZ,i,j}/\|\bfK_{\bfZ,i}\|_1$ for $j\neq i$. Recall that $\bfK_\bfZ=(k(\bfZ_i-\bfZ_j))_{(i,j)\in[n]^2}$, which means 
$\bfK_{\bfZ,i,j}/\|\bfK_{\bfZ,i}\|_1=\frac{\exp(-\|\bfZ_i-\bfZ_j\|^2/{2\tau})}{\sum_{k\neq i}
\exp(-\|\bfZ_i-\bfZ_k\|^2/{2\tau})
}$ for $j\neq i$, when $k$ is the Gaussian kernel with variance $\tau$. 

Notice that $\bfZ_i=f(\bfX_i)$, so we know
\begin{equation}
-\log \Pr(\bfW_{\bfZ,i,j}=1)=
-\log \frac{\exp(-\|f(\bfX_i)-f(\bfX_j)\|^2/{2\tau})}{\sum_{k\neq i}
\exp(-\|f(\bfX_i)-f(\bfX_k)\|^2/{2\tau}),
}
\label{eqn:infonce-equivalence}    
\end{equation}


The right hand side is exactly the InfoNCE loss defined in Eqn.~(\ref{eqn:infonce}).
Inserting Eqn.~(\ref{eqn:infonce-equivalence}) into Eqn.~(\ref{eqn:detailed-expansion}), we get the SimCLR algorithm, which first samples augmentation pairs $(i,j)$ with $\Pr(\bfW_{\bfX,i,j}=1)$ for each row $i$, and then optimize the InfoNCE loss. 

\textbf{Step 2: } minimizing the cross entropy loss 
is equivalent to spectral clustering on $\bfpi$.


By Lemma~\ref{lem:convert_to_spectral}, we may further convert the loss to 
\begin{equation}
\label{eqn:main-theorem-repul-attr}
\min_{\bfZ}
-\sum_{(i,j)\in [n]^2} \mathbf{P}_{i,j}
\log k (\bfZ_i-\bfZ_j)+\log \mathbf{R}(\bfZ).
\end{equation}
Since $k$ is the Gaussian kernel, this reduces to \[
\min_\bfZ \mathrm{tr}(\bfZ^\top \mathbf{L}(\bfpi) \bfZ)
+\log \mathbf{R}(\bfZ),
\]

where we use the fact that $\mathbb{E}_{\bfW_\bfX\sim \mathbb{P}(\cdot; \bfpi)}[\mathbf{L}(\bfW_\bfX)]
=\mathbf{L}(\bfpi)
$, because the Laplacian operator is linear and $
\mathbb{E}_{\bfW_\bfX\sim \mathbb{P}(\cdot; \bfpi)}(\bfW_\bfX)=\bfpi
$.
\end{proof}

\paragraph{Proof of Theorem \ref{thm:clip}.}
\begin{proof}
Since $\bfW_\bfX\sim \mathbb{P}(\cdot;\bfpi_{\mathbf{A}, \mathbf{B}})$, we know 
$\bfW_\bfX$ has exactly one non-zero entry in each row, denoting the pair that got sampled. 
A notable difference compared to the previous proof is we now have $n_\mathcal{A}+n_\mathcal{B}$ objects in our graph. CLIP deals with this by taking a mini-batch of size $2N$, 
such that $n_\mathcal{A}=n_\mathcal{B}=N$, and adding the $2N$ InfoNCE losses together. We label the objects in $\mathcal{A}$ as $[n_\mathcal{A}]$, and the objects in $\mathcal{B}$ as $\{n_\mathcal{A}+1, \cdots, n_\mathcal{A}+n_\mathcal{B}\}$. 

Notice that $\bfpi_{\mathbf{A}, \mathbf{B}}$ is a bipartite graph, so the edges of objects in $\mathcal{A}$ will only connect to object in $\mathcal{B}$ and vice versa. We can define the similarity matrix in $\cZ$ as $\bfK_\bfZ$, 
where $\bfK_\bfZ(i, j+n_\mathcal{A})=\bfK_\bfZ(j+n_\mathcal{A},i)= k(\bfZ_i-\bfZ_j)$ for $i\in [n_\mathcal{A}], j\in [n_\mathcal{B}]$, and otherwise we set $\bfK_\bfZ(i,j)=0$. 
The rest is same as the previous proof. 
\end{proof}

\paragraph{Proof of Theorem \ref{thm:exponential}.}

\begin{proof}
\label{proof:exponential}
Since the objective function consists of a linear term combined with an entropy regularization, which is a strongly concave function, the maximization problem is a convex optimization problem. Owing to the implicit constraints provided by the entropy function, the problem is equivalent to having only the equality constraint. We then introduce the Lagrangian multiplier $\lambda$ and obtain the following relaxed problem:

$$
\widetilde{E}(\boldsymbol{\alpha})=\psi_{1}-\sum_{i=1}^n \alpha_{i} \psi_{i}+\tau \sum_{i=1}^n \alpha_{i}\log \alpha_{i}+\lambda\left(\boldsymbol{\alpha}^{\top} \mathbf{1}_n-1\right).
$$

As the relaxed problem is unconstrained, taking the derivative with respect to $\alpha_{i}$ yields

$$
\frac{\partial \widetilde{E}(\boldsymbol{\alpha})}{\partial \alpha_{i}}=-\psi_{i}+\tau\left(\log \alpha_{i}+\alpha_{i} \frac{1}{\alpha_{i}}\right)+\lambda=0.
$$

Solving the above equation implies that $\alpha_{i}$ takes the form
$
\alpha_{i}=\exp \left(\frac{1}{\tau} \psi_{i}\right) \exp \left(\frac{-\lambda}{\tau}-1\right).
$ Since $\alpha_{i}$ lies on the probability simplex, the optimal $\alpha_{i}$ is explicitly given by
$
\alpha^{*}_{i}=\frac{\exp \left(\frac{1}{\tau} \psi_{i}\right)}{\sum_{i^{\prime}=1}^n \exp \left(\frac{1}{\tau} \psi_{i^{\prime}}\right)} .
$ Substituting the optimal point into the objective function, we obtain
$$
\begin{aligned}
E\left(\boldsymbol{\alpha}^*\right)  &=\psi_1-\sum_{i=1}^n \frac{\exp \left(\frac{1}{\tau} \psi_{i}\right)}{\sum_{i^{\prime}=1}^n \exp \left(\frac{1}{\tau} \psi_{i^{\prime}}\right)} \psi_{i}+\tau \sum_{i=1}^n \frac{\exp \left(\frac{1}{\tau} \psi_{i}\right)}{\sum_{i^{\prime}=1}^n \exp \left(\frac{1}{\tau} \psi_{i^{\prime}}\right)}\log \frac{\exp \left(\frac{1}{\tau} \psi_{i}\right)}{\sum_{i^{\prime}=1}^n \exp \left(\frac{1}{\tau} \psi_{i^{\prime}}\right)} \\
& =\psi_1 - \tau \log \left(\sum_{i=1}^n \exp \left(\frac{1}{\tau} \psi_{i}\right)\right).
\end{aligned}
$$
Thus, the Lagrangian dual function is given by
\begin{equation*}
-E\left(\boldsymbol{\alpha}^*\right)= -\tau \log \frac{\exp \left(\frac{1}{\tau} \psi_{1}\right)}{\sum_{i=1}^n \exp \left(\frac{1}{\tau} \psi_{i}\right)}.\qedhere
\end{equation*}
\end{proof}



\section{More on Experiments} \label{section: experiment_details}

\paragraph{CIFAR-10 and CIFAR-100} CIFAR-10 ~\citep{krizhevsky2009learning} and CIFAR-100 ~\citep{krizhevsky2009learning} are well-known classic image classification datasets. Both CIFAR-10 and CIFAR-100 contain a total of 60k $32 \times 32$ labeled images of different classes, with 50k for training and 10k for testing. CIFAR-10 is similar to CIFAR-100, except there are 10 different classes in CIFAR-10 and 100 classes in CIFAR-100.

\paragraph{TinyImageNet} TinyImageNet ~\citep{le2015tiny} is a subset of ImageNet ~\citep{deng2009imagenet}. There are 200 different object classes in TinyImageNet, with 500 training images, 50 validation images, and 50 test images for each class. All the images in TinyImageNet are colored and labeled with a size of $64 \times 64$.

\textbf{Pseudo-code.} Algorithm \ref{alg:Training Procedure} presents the pseudo-code for our empirical training procedure.

\begin{algorithm}[!htbp]
\caption{Training Procedure}
\label{alg:Training Procedure}
\begin{algorithmic}[1]
\REQUIRE trainable encoder network $f$, batch size $N$, augmentation strategy \textit{aug}, loss function $L$ with hyperparameters \textit{args}
\FOR {sampled minibatch ${x_i}_{i=1}^N$}
\FORALL{$i \in { 1, ..., N }$}
\STATE draw two augmentations $t_i = \textit{aug}\left(x_i\right) $, $t_i' = \textit{aug}\left(x_i\right) $
\STATE $z_i = f\left(t_i\right)$, $z_i' = f\left(t_i'\right)$
\ENDFOR
\STATE compute loss $\mathcal{L} = L(N, z, z', \textit{args})$
\STATE update encoder network $f$ to minimize $\mathcal{L}$
\ENDFOR
\STATE \textbf{Return} encoder network $f$
\end{algorithmic}
\end{algorithm}

We also provide the pseudo-code for our core loss function used in the training procedure in Algorithm \ref{alg:Core loss}. The pseudo-code is almost identical to SimCLR's loss function, with the exception of an extra parameter $\gamma$.

\begin{algorithm}[!htbp]
\caption{Core loss function $\mathcal{C}$}
\label{alg:Core loss}
\begin{algorithmic}[1]
\REQUIRE batch size $N$, two encoded minibatches $z_1, z_2$, $\gamma$, temperature $\tau$
\STATE $z = \textit{concat}\left(z_1, z_2\right)$
\FOR {$i \in {1, ..., 2N }, j \in {1, ..., 2N}$ }
\STATE $s_{i,j} = \Vert z_i - z_j \Vert_2^{\gamma}$
\ENDFOR
\STATE \textbf{define} $l(i, j)$ \textbf{as} $l(i, j) = - \log \frac{exp\left(s_{i,j}/\tau \right)}{\sum_{k=1}^{2N} \mathbf{1}{[k \ne i]} exp\left(s{i, j} / \tau \right)} $
\STATE \textbf{Return} $\frac{1}{2N} \sum_{k=1}^N\left[l(i, i+N) + l(i+N, i)\right]$
\end{algorithmic}
\end{algorithm}

Utilizing the core loss function $\mathcal{C}$, we can define all kernel loss functions used in our experiments in Table \ref{table: loss definition}. For all $z_i \in z$ with even dimensions $n$, we define $z_{L_i} = z_i\left[0:n/2\right]$ and $z_{R_i} = z_i\left[n/2:n\right]$.

\begin{table}[ht]
\centering
\begin{tabular}{{@{}l|l@{}}}
Kernel  &  Loss function \\ \midrule
Laplacian & $\mathcal{C}\left(N, z, z', \gamma=1, \tau\right)$\\ \midrule
Sum       & $\lambda * \mathcal{C}\left(N, z, z', \gamma=1, \tau_1\right) + (1-\lambda) * \mathcal{C}\left(N, z, z', \gamma=2, \tau_2\right)$  \\ \midrule
Concatenation Sum&$\lambda * \mathcal{C}\left(N, z_L, z'_L, \gamma=1, \tau_1\right) + (1-\lambda) * \mathcal{C}\left(N, z_R, z'_R, \gamma=2, \tau_2\right)$\\ \midrule
$\gamma = 0.5$ & $\mathcal{C}\left(N, z, z', \gamma=0.5, \tau\right)$          \\ 

\end{tabular}

\caption{Definition of kernel loss functions in our experiments}
\label {table: loss definition}
\end{table}

\textbf{Baselines.} We reproduce the SimCLR algorithm using PyTorch Lightning~\citep{PytorchLightning}.

\textbf{Encoder details.}
The encoder $f$ consists of a backbone network and a projection network. We employ ResNet50~\citep{ResNet} as the backbone and a 2-layer MLP (connected by a batch normalization~\citep{ioffe2015batch} layer and a ReLU \cite{nair2010rectified} layer) with hidden dimensions 2048 and output dimensions 128 (or 256 in the concatenation kernel case).

\textbf{Encoder hyperparameter tuning.}
For each encoder training case, we randomly sample 500 hyperparameter groups (sample details are shown in Table \ref{table: Hyperparameter sample}) and train these samples simultaneously using Ray Tune ~\citep{RayTune}, with the ASHA scheduler~\citep{li2018massively}. Ultimately, the hyperparameter group that maximizes the online validation accuracy (integrated in PyTorch Lightning) within 5000 validation steps is chosen for the given encoder training case.

\begin{table}[ht]
\centering

\begin{tabular}{@{}l|l|l@{}}
\midrule
Hyperparameter  & Sample Range & Sample Strategy \\ \midrule
start learning rate & $\left[10^{-2}, 10\right]$ & log uniform \\ \midrule
$\lambda$       & $\left[0, 1\right]$ & uniform \\ \midrule
$\tau$, $\tau_1$, $\tau_2$ & $\left[0, 1\right]$ & log uniform \\ \midrule
\end{tabular}

\caption{Hyperparameters sample strategy}
\label {table: Hyperparameter sample}
\end{table}

\textbf{Encoder training.} 
We train each encoder using the LARS optimizer~\citep{LARSOptimizer}, LambdaLR Scheduler in PyTorch, momentum 0.9, weight decay $10^{-6}$, batch size 256, and the aforementioned hyperparameters for 400 epochs on a single A-100 GPU.

\textbf{Image transformation.} The image transformation strategy, including augmentation, is identical to the default transformation strategy provided by PyTorch Lightning.

\textbf{Linear evaluation.}
The linear head is trained using the SGD optimizer with a cosine learning rate scheduler, batch size 64, and weight decay $10^{-6}$ for 100 epochs. The learning rate starts at $0.3$ and ends at $0$.

\textbf{Moco Experiments.} We also tested our method based on MoCo~\citep{he2019moco}. The results are summarized in Table \ref{tab:results-moco}. Here we choose ResNet18~\citep{ResNet} as the backbone and set a temperature of $0.1$ as default. For our simple sum kernel, we set $\lambda=0.8$. The results show that our method outperforms the original MoCo method.

\begin{table}[thb]
\centering
\caption{MoCo Experiment Results on CIFAR-10 and CIFAR-100.}
\label{tab:results-moco}
\resizebox{\textwidth}{!}{%
\begin{tabular}{@{}c|ccc|ccc@{}}
\toprule
\multirow{3}{*}{Method} & \multicolumn{3}{c|}{CIFAR-10} & \multicolumn{3}{c}{CIFAR-100} \\ \cmidrule(lr){2-4} \cmidrule(lr){5-7} 
                        & 200 epochs & 400 epochs    & 1000 epochs   & 200 epochs & 400 epochs & 1000 epochs         \\ \midrule
MoCo (repro.)         & $76.41 \pm 0.12$    & $80.01 \pm 0.15$          & $84.45 \pm 0.08$    & $\mathbf{47.02 \pm 0.11}$ & $52.50 \pm 0.07$ & $57.62 \pm 0.15$            \\
\midrule
Laplacian Kernel        & ${78.09 \pm 0.10}$    & $\mathbf{83.85 \pm 0.09}$          & $\mathbf{88.34 \pm 0.16}$    & $46.12 \pm 0.22$   & $53.44 \pm 0.17$ & $59.10 \pm 0.14$        \\
Simple Sum Kernel & $\mathbf{78.12 \pm 0.15}$   & $83.23 \pm 0.18$ & $87.50 \pm 0.20$ & $46.65 \pm 0.06$ & $\mathbf{53.62 \pm 0.19}$ & $\mathbf{59.83 \pm 0.12}$\\
\bottomrule
\end{tabular}
}
\end{table}



\section{More Experiments on Synthetic Data}


Consider a scenario with $n$ clusters, each containing $k$ vertices. Let the probability of vertices $u$ and $v$ from the same cluster belonging to $\bfpi$ be $p$. Conversely, for vertices $u$ and $v$ from different clusters, let the probability of belonging to $\pi$ be $q$. We generate the graph $\bfpi$ randomly, based on $p$ and $q$. We experiment with values of $k=100$ and $n=6$ for ease of visualization, embedding all points in a two-dimensional space. Each vertex's initial position originates from a normal distribution. In each iteration, we sample a subgraph of $\bfpi$ uniformly, ensuring each vertex has an out-degree of $1$. We then optimize the corresponding vectors using InfoNCE loss with an SGD optimizer and iterate until convergence. Our experimental setup consists of an SGD learning rate of $1$, an InfoNCE loss temperature of $0.5$, and a batch size of $50$. We evaluate two scenarios with different $p$ and $q$ values: $p=1$, $q=0$, and $p=0.75$, $q=0.2$. The results of these experiments are visualized in Figure \ref{fig:vis-spectral-cluster}. The obtained embeddings exhibit the hallmark pattern of spectral clustering of graph $\bfpi$.

\begin{figure}[!tb]
\centering
\subfigure{
\includegraphics[width=1\textwidth]{Figures/cluster_pi.png}
\label{fig:vis-cluster}
}
\subfigure{
\includegraphics[width=1\textwidth]{Figures/noised_cluster_pi.png}
\label{fig:vis-noised-cluster}
}
\caption{Visualizations of the optimization process using InfoNCE Loss on the vectors corresponding to $\bfpi$. Points of identical color belong to the same cluster within $\bfpi$. To showcase the internal structure of $\bfpi$, we randomly select 10 vertices from each cluster to display the edge distribution of $\bfpi$.}
\label{fig:vis-spectral-cluster}
\end{figure}



\end{document}
