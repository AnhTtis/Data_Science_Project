\def\year{2021}\relax
%File: formatting-instructions-latex-2021.tex
%release 2021.2
\documentclass[letterpaper]{article} % DO NOT CHANGE THIS
\usepackage{aaai21}  % DO NOT CHANGE THIS
\usepackage{times}  % DO NOT CHANGE THIS
\usepackage{helvet} % DO NOT CHANGE THIS
\usepackage{courier}  % DO NOT CHANGE THIS
\usepackage[hyphens]{url}  % DO NOT CHANGE THIS
\usepackage{graphicx} % DO NOT CHANGE THIS
\urlstyle{rm} % DO NOT CHANGE THIS
\def\UrlFont{\rm}  % DO NOT CHANGE THIS
\usepackage{natbib}  % DO NOT CHANGE THIS AND DO NOT ADD ANY OPTIONS TO IT
\usepackage{caption} % DO NOT CHANGE THIS AND DO NOT ADD ANY OPTIONS TO IT
\frenchspacing  % DO NOT CHANGE THIS
\setlength{\pdfpagewidth}{8.5in}  % DO NOT CHANGE THIS
\setlength{\pdfpageheight}{11in}  % DO NOT CHANGE THIS
%\nocopyright

% Packages
\usepackage{type1cm} % type1 computer modern font
\usepackage{graphicx} % advanced figures
\usepackage{xspace} % fix space in macros
\usepackage{balance} % to better equalize the last page
\usepackage{booktabs} % nicer tables
\usepackage{multirow} % multi rows for tables
\usepackage[tableposition=top]{caption} % captions on top for tables  font={bf},
\usepackage{bold-extra} % bold + {small capital, italic}
\usepackage{siunitx} % \num for decimal grouping
\usepackage[vlined,linesnumbered,ruled,noend]{algorithm2e} % algorithms
\usepackage{microtype} % compress text
\usepackage{xfrac} % nicer slanted fractions
\usepackage{mathtools} % amsmath++
%\usepackage{amssymb} % math symbols
%\usepackage[hyphens]{url} % handle long urls
%\usepackage[bookmarks, pdftex, colorlinks=false]{hyperref} % clickable references
\usepackage[capitalise]{cleveref}
% \usepackage[square,numbers]{natbib} % better references
% \usepackage{hyphenat} % name in single line
\usepackage{chato-notes}
\usepackage{subfig}
\usepackage[dvipsnames]{xcolor}
%\usepackage{xcolor}
\graphicspath{{../img/}}
\thispagestyle{plain}
\pagestyle{plain}

% Paragraphs
\newcommand{\spara}[1]{\smallskip\noindent\textbf{#1}}
\newcommand{\mpara}[1]{\medskip\noindent\textbf{#1}}
\newcommand{\para}[1]{\noindent\textbf{#1}}

% squishlist
\newenvironment{squishlist}
{\begin{list}{$\bullet$}
 {\setlength{\itemsep}{0pt}
	\setlength{\parsep}{3pt}
	\setlength{\topsep}{3pt}
	\setlength{\partopsep}{0pt}
	\setlength{\leftmargin}{1.5em}
	\setlength{\labelwidth}{1em}
	\setlength{\labelsep}{0.5em} } }
{\end{list}}

\newcommand{\YM}[1]{{\color{red} (\underline{Y:} {#1})}}
\newcommand{\GDFM}[1]{{\color{blue} (\underline{G:} {#1})}}
\newcommand{\KK}[1]{{\color{purple} (\underline{K:} {#1})}}

\hyphenation{non-phar-ma-ceu-ti-cal}

%PDF Info Is REQUIRED.
% For /Author, add all authors within the parentheses, separated by commas. No accents or commands.
% For /Title, add Title in Mixed Case. No accents or commands. Retain the parentheses.
\pdfinfo{
/Title (Authority without Care: Moral Values behind the Mask Mandate Response)
/Author (Anonymous)
/TemplateVersion (2021.2)
} %Leave this
% /Title ()
% Put your actual complete title (no codes, scripts, shortcuts, or LaTeX commands) within the parentheses in mixed case
% Leave the space between \Title and the beginning parenthesis alone
% /Author ()
% Put your actual complete list of authors (no codes, scripts, shortcuts, or LaTeX commands) within the parentheses in mixed case.
% Each author should be only by a comma. If the name contains accents, remove them. If there are any LaTeX commands,
% remove them.

\setcounter{secnumdepth}{0} %May be changed to 1 or 2 if section numbers are desired.

%%
%% The "title" command has an optional parameter,
%% allowing the author to define a "short title" to be used in page headers.
\title{Authority without Care:\\Moral Values behind the Mask Mandate Response}
% Author order: Yelena, Kyriaki, Gianmarco

\author{
    Yelena Mejova,\textsuperscript{\rm 1}
    Kyriaki Kalimeri,\textsuperscript{\rm 1}
    Gianmarco De Francisci Morales\textsuperscript{\rm 2}\\
}
\affiliations{
    \textsuperscript{\rm 1}ISI Foundation, V. Chisola 5, Turin, Italy,
    \textsuperscript{\rm 2}CentAI, C. Inghilterra 3, Turin, Italy\\
    yelenamejova@acm.org, kyriaki.kalimeri@isi.it, gdfm@acm.org
    % %Afiliations
    % \textsuperscript{\rm 1}Association for the Advancement of Artificial Intelligence\\
    % %If you have multiple authors and multiple affiliations
    % % use superscripts in text and roman font to identify them.
    % %For example,

    % % Sunil Issar, \textsuperscript{\rm 2}
    % % J. Scott Penberthy, \textsuperscript{\rm 3}
    % % George Ferguson,\textsuperscript{\rm 4}
    % % Hans Guesgen, \textsuperscript{\rm 5}.
    % % Note that the comma should be placed BEFORE the superscript for optimum readability

    % 2275 East Bayshore Road, Suite 160\\
    % Palo Alto, California 94303\\
    % % email address must be in roman text type, not monospace or sans serif
    % publications21@aaai.org

    % % See more examples next
}
%%
%% end of the preamble, start of the body of the document source.
\begin{document}

\maketitle

\begin{abstract}
Face masks are one of the cheapest and most effective non-pharmaceutical interventions available against airborne diseases such as COVID-19.
Unfortunately, they have been met with resistance by a substantial fraction of the populace, especially in the U.S.
In this study, we uncover the latent moral values that underpin the response to the mask mandate, and paint them against the country's political backdrop.
We monitor the discussion about masks on Twitter, which involves almost 600K users in a time span of 7 months.
By using a combination of graph mining, natural language processing, topic modelling, content analysis, and time series analysis, we characterize the responses to the mask mandate of both those in favor and against them.
We base our analysis on the theoretical frameworks of Moral Foundation Theory and Hofstede's cultural dimensions.
Our results show that, while the anti-mask stance is associated with a conservative political leaning, the moral values expressed by its adherents diverge from the ones typically used by conservatives.
In particular, the expected emphasis on the values of authority and purity is accompanied by an atypical dearth of in-group loyalty.
We find that after the mandate, both pro- and anti-mask sides decrease their emphasis on care about others, and increase their attention on authority and fairness, further politicizing the issue.
In addition, the mask mandate reverses the expression of Individualism-Collectivism between the two sides, with an increase of individualism in the anti-mask narrative, and a decrease in the pro-mask one.
We argue that monitoring the dynamics of moral positioning is crucial for designing effective public health campaigns that are sensitive to the underlying values of the target audience.
\end{abstract}

\section{Introduction}

The COVID-19 pandemic has been an unprecedented event, which has also brought about an infodemic that makes public health response difficult.
COVID denial, anti-vaccine sentiment, and other flavors of theories (from doubts to full-blown conspiracies) have been documented in social media \cite{patwa2021fighting}.
Among the many controversies, surprisingly, wearing a mask has become extremely politicized and contentious. 
%% In this paper, we argue that there are important moral dimensions to the discourse around mask-wearing, that they change in response to the government mask mandates, and that these need to be taken into account when communicating about public health interventions. %, especially in the cultural context of the U.S.
%As such, it is an interesting litmus test for social behavior under stress in a globally connected environment.
%% Mask-wearing is a simple, low-cost, non-pharmaceutical intervention that has proven to be very effective.
%% By May 9th, 2020, in countries with universal mask-wearing, per-capita COVID mortality increased by just 5.4\% each week, compared with 48\% each week in other countries~\cite{Leffler2020}.
Inconsistent messaging by public health organizations seriously undermined public compliance to this simple measure.
Initially, the WHO recommended that masks should be worn only by professionals or those taking care of a sick person~\cite{who2020advice}, and
%Soon after, a study found that masks help reduce COVID's spread~\cite{chu2020physical}, and the recommendations started to shift. 
in the U.S., the Center for Disease Control and Prevention (CDC) recommended the use of masks only for healthcare workers and people who were sick.
Only on April 3rd, 2020, the CDC officially recommended the wearing of non-medical cloth face coverings when in public places~\cite{landsverk2020cdc}. %\footnote{\url{https://www.businessinsider.com/cdc-recommends-health-workers-use-bandanas-face-masks-crisis-2020-3}}. 
% Alternative article: \footnote{\url{https://www.reuters.com/article/uk-factcheck-outdated-masks/fact-check-medical-discharge-document-includes-outdated-cdc-guidance-on-face-masks-idUSKBN2441WN}}.
Soon after, speculations that masks do more harm than good~\cite{gillespie2020mask} and that they foster a false sense of security~\cite{khan2020mask} started to proliferate.
%% The communication by politicians further complicated matters, with some Republicans hesitating to mandate mask-wearing in their districts, and then-President Trump mocking the Presidential hopeful Joe Biden for wearing one~\cite{aratani2020masks}.
%%%% Opinions often fell along political lines: according to PEW, when surveyed, Republicans were much more likely to call masks unnecessary, ineffective, oppressive, or unfair and were much less likely than Democrats to be concerned about unknowingly spreading COVID-19 to others~\cite{kessel2020republicans}.
%As of August 4th, 34 states and the District of Columbia have implemented health orders requiring the wearing of face coverings in most public spaces.\footnote{\url{https://masks4all.co/what-states-require-masks}}
%Public eschewals of masks continued to generate news headlines, whether these occurred at Trump presidential campaign rallies or on the floor of the U.S. Senate.
%As of August 4th, 34 states and the District of Columbia have implemented health orders requiring the wearing of face coverings in most public spaces.
% Good section on history of mask-wearing: https://journals.sagepub.com/doi/pdf/10.1177/20539517211023533
%The spirited debate about mask-wearing reveals the fissures in the U.S. political and social fabric, with each side confident in the correctness of their view.
Public health experts have acknowledged that it has been a challenge to communicate with groups holding strong values of freedom~\cite{hartsoe2020experts}.
Indeed, globally, research has found that the value of individualism is associated with higher COVID-19 mortality rates~\cite{maaravi2021tragedy} and less adherence to the prevention measures~\cite{biddlestone2020cultural}.
However, how personal moral values evolve around governmental messaging concerning these preventative measures is not yet understood.

% \mynote{Connect mask wearing decision to the facets we analyze. We know that political ideology plays a big role in identity definition and decision making.
% List references about moral values connected to ideology in the wider sense (conservative/liberal) and decision making. Note that in this case there is an altruistic component in mask-wearing, so collectivist-individualist angle is important to look at.
% Information sources -- from peers within the same community, or from outside sources -- as opinion-formation determinants.}

Mask-wearing behavior and attitudes can be influenced by multiple factors.
There is extensive evidence that political ideology and identity influences attitudes, judgments, and behaviors~\cite{van2018partisan}.
At the same time, political identity is deeply intertwined with moral values~\cite{graham2009liberals,hatemi2019ideology}.
Finally, and perhaps obviously, moral values are directly linked to moral decision making~\cite{karandikar2019predicting}, which completes the triad.
In particular, there is an altruistic component in mask-wearing, as their main purpose is to protect others, which is related to solidarity towards the in-group in the face of an out-group threat~\cite{campbell1965ethnocentric}.
For this reason, within the moral values we pay particular attention to the individualist-collectivist angle.
Finally, there is ample literature on the division and polarization of news and information sources in general\footnote{\url{https://www.pewresearch.org/journalism/2020/03/04/about-one-fifth-of-democrats-and-republicans-get-political-news-in-a-kind-of-media-bubble}}.  %~\cite{jurkowitz2020one}.
As users increasingly rely on social media to satisfy their information needs\footnote{\url{https://www.pewresearch.org/journalism/2021/11/15/news-on-twitter-consumed-by-most-users-and-trusted-by-many}},  %~\cite{mitchell2021news}, 
the lines between professional reporting and personal opinions begin to blur.
For this reason, we contextualize the information environment in which the mask debate takes place in terms of information sources from peers---within the Twitter community---and from external sources.


In this work, we provide a fine-grained analysis of the moral values of those expressing opinions around masking by applying the Moral Foundations Theory to a dataset of Twitter posts spanning the beginning of the pandemic, from January to July 2020. 
In particular, we ask:
\emph{What is the anatomy of the collective discussion on mask wearing around the mask mandate on Twitter?}
In particular, focusing on the US, we analyze different facets of this discussion:
\begin{squishlist}
\item[1.] How does their stance relate to their political leaning;
%\item[2.] To what extent do they express collectivist vs. individualist tendencies; -- included in moral values above
\item[2.] What moral values do adherents to pro- or anti-masking stances hold;
\item[3.] What is the information environment around their arguments? % environment = what other peers say on SNS + what outside sources of info are cited
\end{squishlist}

Our findings confirm the known political divisions, with liberals supporting the pro-mask and conservatives the anti-mask stance.
The moral narrative of the two sides also differs: those on the anti-mask side invoke the values of \emph{authority} and \emph{purity}, while those promoting mask-wearing emphasize \emph{care} and \emph{loyalty}.
The introduction of the mask mandate shifts these values from an emphasis on \emph{care} to an increased attention on \emph{authority} and \emph{fairness}, escalating the politicization of the issue.
%Interestingly, our findings are in contrast to the typical characterization of conservatives as valuing \emph{authority}, \emph{purity}, and in-group \emph{loyalty}~\cite{kivikangas2021moral,fulgoni2016empirical}, as the data shows that after the mask mandate, their emphasis on \emph{loyalty} is lesser compared to mask advocates.  %shifts their focus from a less to a more individualistic view.  <<< Not the main observation, cut Hofstede because of space considerations
Content analysis shows that the increasingly individualistic views espoused by the anti-mask side are accompanied by the allegations of the dangers and ineffectiveness of mask-wearing, supported by resources from social media and sometimes even governmental and scientific sources.
We argue that the mask-wearing debate has a dynamic moral landscape that should be carefully monitored to design effective public messaging campaigns that reflect the shifting values of their target audience.
Such studies are especially important, given the rising prominence of social media as a platform for public policy discussion and influence~\cite{shapiro2017politicians}. %atkins2020black,

% accompanied by motivated reasoning which uses resources from social media, and sometimes even governmental and scientific sources, to argue the dangers of mask-wearing and their purported ineffectiveness.

% Should we mention controversy research?

% Hypothesis:\\
% The use of face mask M depends on two causal variables in an AND relationship: the perceived risk of covid R and how prosocial/collective C the person/community is.
% If the behavior M gets imposed when both R and C are ``false'', then there could be an anti-establishment sentiment spiking.\\

% Research Questions:\\
% Recommendations by WHO switched -- what is the impact of this switch on perception of mask-wearing?\\
% How does it affects the trust in authority (WHO)?\\
% Collectively (plotting) or individually (were they antiestablishment before?)\\
% How do these responses relate to prevailing cultural dimension of individualistic vs collective?\\
% Also to political inclination?\\
% Are there influential political/other figures who may cause some actions/opinions?\\
% Do people change their mind when official recommendations flip? How does this relate to (1) individualistic vs collectivistic (on country scale) and (2) pro- vs anti Trump in U.S. (anti or pro-establishment interpretation) (on individual scale)? The (2) works especially in U.S.: pro-Trump = anti-establishment.\\

\section{Related Work}

\subsubsection{Masking Behavior.}
Mask mandates have been shown to be effective in decreasing COVID rates in the U.S., both in urbanized and rural areas \cite{krishnamachari2021role}. 
%Further, in the case of Australia, \cite{costantino2021impact} find that early mask use averts more cases than mask usage that is only implemented closer to the peak. % Cut due to space constraints
%% A survey of 1000 German residents has shown that mask-wearing has been positively related to the knowledge of mask effectiveness and that it protects others~\cite{betsch2020infographic}. 
Early studies have tested the effect of mask adoption, with simulations showing that even relatively ineffective face coverings could meaningfully reduce community transmission~\cite{eikenberry2020mask}. 
%% Though the masking measures must be timely, as in the countries in which a mask was mandated from 16 to 30 days after infection onset had per-capita mortality several orders of magnitude less than those that waited until 60 days after~\cite{leffler2020association}.
Unfortunately, the public response to the masking mandates has been documented to be partial, with only a 12\% increase in mask-wearing immediately after the CDC masking guideline on April 3, 2020~\cite{goldberg2020mask}.
A survey of residents of 10 states in the U.S. relates mask-wearing to COVID cases in the state, political party in power, and individual measures of ``social capital'', as well as some demographics~\cite{hao2021understanding}. They find that respondents are more likely to wear a mask if there are more COVID-related deaths in the state (Odds Ratio of $1.26$), the Democratic party is in power (OR $=2.0$), and if the respondents more often speak to their friends and family (OR $=1.16$). 
Although misinformation has been blamed for the resistance to mask-wearing and social distancing, a nationally representative survey has found that the beliefs about the consequences of these behaviors are more predictive of whether people will actually comply~\cite{hornik2021association}. % also: "a communication intervention directly addressing misinformation beliefs is likely to be less promising than one directly addressing the specific behavioral beliefs around the protective behaviors."
Especially in the U.S., beliefs, and trust in authority are often strongly related to the people's political affiliations. For instance, Republicans, conservatives, and nationalists are less likely to trust that the World Health Organization (WHO) can effectively manage the pandemic \cite{bayram2021trusts}.

%\KK{I found the following studies: \cite{maaravi2021tragedy}: more individualism, more dealths, \cite{biddlestone2020cultural}: more individualism, less social distancing and \cite{lu2021collectivism}: more collectivism more mask wearing.} %% Incorporated into Intro

%In Italy, regional mask mandates and free mask giveaways were shown to contribute to faster recovery, even beyond the lockdown measures \cite{pedersen2021data}. % Use Change Points

% \cite{leffler2020association} Although a complete analysis of the later period in the pandemic is beyond the scope of the current model, we might note the coronavirus-related mortality among early-masking countries, on August 9, 2020.7 Among the 24 countries which initiated public mask-wearing within 20 days of the onset of their outbreak (Supplemental Table A1), the average coronavirus-related mortality was 4.7 per million (SD 6.1) on August 9. Moreover, among the additional 17 countries which recommended masks within 30 days of the start of their outbreak (Supplemental Table A1), the average coronavirus-related mortality was 26.6 per million (SD 36.2) on August 9. By contrast, the per-capita coronavirus mortality in the United States was 502 per million on August 9.

\subsubsection{Social Media \& Masking.} A study of tweets during the early days of the COVID pandemic (Feb-March 2020) identified methods for decreasing the spread of COVID as one of the main themes and the wearing of masks as one of those most associated with positive sentiment~\cite{abd2020top}. 
However, authors of a later study spanning January to October 2020 of geolocated tweets (about 1-2\% of total Twitter stream) showed that the output of automated sentiment analysis tools corresponds poorly with the mask-related sentiment expressed in the tweets due to the richness of the language used \cite{he2021people}. Instead, they perform manual coding of a sample of the tweets, identifying several major categories of concerns around masking, including physical discomfort, effectiveness concerns, being unnecessary for certain circumstances, and political beliefs. 
A network analysis of the mask-related tweets has shown the pro-mask activists exist in a kind of an ``echo chamber'', and that they tend to ignore the subversive rhetoric of the anti-mask fringe \cite{lang2021maskon}.
Instead, a recent Twitter study found a focus on ongoing news \cite{cotfas2021unmasking}.
This fringe is more likely to use toxic language, including insults and profanity, than the pro-mask ones; an analysis of mask-related tweets during 2020 found \cite{pascual2021toxicity}.
The authors link this behavior to either the vociferous protestations of a minority group \cite{miller2009expressing} or potential signaling as an in-group behavior and a marker of personal identification.
Alongside this toxicity, other studies find widespread misinformation and misunderstandings in the social media discussions around mask use. These include the beliefs that COVID19 is over-hyped by the media, that masks are ineffective, and that they do more harm than good \cite{keller2021social}. These beliefs were then shown to impact the mask-wearing (and social distancing) behaviors \cite{hornik2021association}.
%In aggregate, these studies attest to a heated discourse around masks on social media, some of which may reflect or potentially affect the real-world masking behaviors of its users.
In this study, we introduce a dimension of \emph{moral values}, which we argue underlie some of the disagreements on the appropriate use of masks during the epidemic, and which may shed some light on the moral reasoning behind the rhetoric. 

% YouGov findings - which?



\subsubsection{Theoretical Framework.}
Our study of the rhetoric around masks during COVID-19 is grounded in the theory of Moral Foundations, its manifestation in interpersonal and inter-group communication, and its reflection at the societal level as an individualist or collectivist cultural dimension. 
According to the social identity theory, members of an in-group will seek to find negative aspects of an out-group, thus enhancing their self-image~\cite{tajfel1979integrative}.
% Groups give us a sense of social identity: a sense of belonging to the social world.
% We instinctively \textit{categorize} objects, symbols, but also people; we create \textit{social identity} by adopting the identity of the group we categorized, and finally, once we have categorized ourselves as part of a specific group, we refer to that to create \textit{social comparisons}~\cite{oakes1980social}.
% Such reasoning reflects on a broad range of attitudes related to stereotype formations and more extreme expressions to discrimination, social biases, or even racism~\cite{miller2010self}.
Strong in-group and out-group reasoning at a societal level determines whether a society is individualistic or collectivist (IC),
%, in this study, we aim to unveil the cooperative or competitive attitudes of people with respect to the mandate, and possible drivers of the societal polarization around the mask-wearing behavior.
as defined by Hofstede~\cite{hofstede2001culture}. 
The IC dimension considers the degree to which societies are integrated into groups and their perceived obligations and dependence on groups.
Individualism indicates there is a greater importance placed on attaining personal goals. %A person's self-image in this category is defined as ``I''.
Collectivism indicates there is a greater importance placed on the goals and well-being of the group. %A person's self-image in this category is defined as ``We''.
People in collectivist societies generally distinguish sharply between in- and out-groups, while people in individualistic societies treat everyone as a potential in-group member and thus apply universal values to everyone.
Cross-cultural research has demonstrated that the United States is the prototypical individualist culture based on the IC dimension
\cite{hofstede1984culture,kim1994individualism}.
At the same time, the U.S. show a measurable variation on this dimension~\cite{vandello1999patterns} at a state level.

The societal dimensions of collectivism and individualism can be related to the individuals' adherence to specific moral dimensions, as postulated by the Moral Foundations Theory~\cite{Graham2012,samuel2016authoritariansim}, which include:
%We employ the Moral Foundations Theory (MFT), which expresses the psychological basis of morality in terms of innate intuitions, and defines the following five foundations:
 \emph{care/harm}, fundamental concerns for the suffering of others,  including virtues of caring and compassion;
 \emph{fairness/cheating},  concerns about unfair treatment,  inequality,  and more abstract notions of justice;
 \emph{loyalty/betrayal},  concerns related to obligations of group membership;
 \emph{authority/subversion},  concerns related to social order and the obligations of hierarchical relationships such as obedience,  respect,  and proper role fulfillment; and
 \emph{purity/degradation}, with concerns about physical and spiritual contagion,  including virtues of chastity,  wholesomeness, and control of desires.
These foundations are shown to underlie human judgements and decision-making \cite{Weber2015} on societal topics ranging from vaccine hesitancy~\cite{kalimeri2019human,Amin2017}, to politics~\cite{iyer2012understanding}, religion, and social cooperation~\cite{haidt2012righteous, curry2016morality}.
Here, we place the focal point on the linguistic analysis of values expressed via the Twitter platform, and aim to clarify peoples' dispositions and attitudes towards interpersonal and inter-group processes related to persuasion and communication narratives.

The moral dimensions of MFT have also been linked to political ideologies \cite{day2014shifting,kugler2014another}, with conservatives emphasizing the in-group relationships and tradition and liberals endorsing fairness and equal opportunity. 
In the United States, the mask-wearing measure has also been strongly associated with the partisan divide. 
Social identity is a primary reason behind people's decision whether to wear a mask during the pandemic \cite{powdthavee2021face}, and surveys show that faith in President Trump is a strong predictor of refusal to social distance, and its effect is largest among individuals high in binding foundations
\cite{graham2020faith}. 
Indeed, the U.S. counties that showed strong support for Trump in 2016 practiced significantly lower mask-wearing in 2020 \cite{kahane2021politicizing}.
In this work, we illustrate the strong connection between the attitudes expressed towards mask-wearing on Twitter and the the political leaning of those expressing them, and show the shifts in moral emphasis after the mask mandates are introduced.

%In August 2020, large majorities of adults in every region of the U.S. said they usually wear masks in stores and other business, according to the PEW survey. Mask-wearing is least common in the West North Central region, which covers states like Missouri, Kansas and Minnesota. But even in this region, 70\% of adults still said they wear masks regularly in stores and other businesses, up from 47\% in June~\href{https://www.pewresearch.org/fact-tank/2020/08/27/more-americans-say-they-are-regularly-wearing-masks-in-stores-and-other-businesses/}{Link}.\\
%Vandello et al.~\cite{vandello1999patterns} assessed the individualism/collectivism index (IC, hereafter)  for each U.S. state. The higher the index of a state, the more collective behaviour it exhibits.
%Hofstede's model characterises individualism versus collectivism as the society's position  reflected in whether people's self-image is defined in terms of ``I'' or ``we''.


\section{Data}

We begin by collecting tweets mentioning the keywords ``mask'', ``facemask'', ``ffp3'', and ``n95'' (the latter two refer to popular kinds of masks), spanning the dates of January 1st to July 30th, 2020, using the GOT3 library~\cite{getoldtweets}.
These keywords were chosen by considering the special Twitter Covid-19, stream\footnote{\url{https://developer.twitter.com/en/docs/twitter-api/tweets/covid-19-stream/filtering-rules}} and picking the most common English keywords related to masks.
This collection results in \num{18245298} tweets from \num{5935103} users.
Following recommendations from existing literature, we then perform several filtering steps in order to ensure that the tweets can be used to assess the stance of the user on masking, and that the account is likely to belong to a human living in U.S.:


\begin{enumerate}
\item Relevance classifier (details next).
\item Exclude users whose Location cannot be mapped to one of U.S. states or Washington DC.
\item Exclude those not having at least 1 English tweet~\cite{garimella2016quantifying}.
\item Exclude users with only 1 tweet~\cite{garimella2016quantifying}.
\item Exclude top 0.1\% of users by the number of tweets (having higher posting rate)~\cite{des2022detecting}.
\item Exclude users whose friends to followers ratio is $>$10~\cite{aggarwal2012phishari}.
\end{enumerate}

We proceed by making sure the tweets we collect are indeed about mask-wearing due to the COVID-19 epidemic.
Upon manual examination, we find several other topics captured, such as advice on masking during protests, sports-related wear, and beauty products.
To remove such content, we use a set of \emph{distant} labels to identify the non-relevant content.\footnote{We use lists compiled by finding top bigrams (considering 300 most frequent) in the dataset and then manually labeling them for non-relevance. The irrelevant list is \{`hair mask',`gas mask',`sleeping mask', `majora's mask',`ski mask',`eye mask',`clay mask',`tear gas'\} and the relevant list is \{`covid',`coronavirus',`sars-cov-2',`sars-cov2',`social distanc',`socialdistanc'\}.}
We manually annotate a balanced random selection of 300 tweets associated with a distant label and find 96\% accuracy in distinguishing between relevant and non-relevant tweets (expected accuracy of a random baseline is 50\%).
We use \num{430568} tweets with the respective distant labels to train the relevance classifier.
The model is a logistic regression, trained on tf-idf-weighted unigram counts extracted from the tweets and with inverse-proportional class weighting to mitigate class imbalance.
As text pre-processing, we remove URLs, numbers, handles, hashtag and retweet indicators (the \# symbols), and stopwords, keep whole words, and perform snowball stemming.
The accuracy in 5-fold cross-validation is 99.7\%.
We draw another balanced sample of 300 tweets thus classified and find the accuracy of the classifier to be 82.3\% (inter-annotator agreement overlap of 90\%, Cohen's kappa of 0.76 between 3 annotators). 

Next, we geolocate the tweets identified as relevant by mapping the user Location strings to the Geonames ID by using custom string matching,\footnote{Code is available at [Anonymized]} and to U.S. zip codes by using the uszipcode library.\footnote{\url{https://pypi.org/project/uszipcode/}} %\url{https://sites.google.com/site/yelenamejova/resources}
We apply basic pre-processing (i.e., stopword and non-ASCII character removal).
First, we filter locations assigned to the U.S. territory by the uszipcode and then the remaining by the Geonames library, aiming to recover the GPS coordinates of the smallest possible administrative area.
We are able to locate \num{1383729} of the users within a U.S. state or Washington DC.
Our sample is representative of the U.S. population distribution (2019 Census estimates) with a Pearson correlation of $0.96$ ($n$=51), suggesting that at least there is little bias in the sampling of the state in terms of number of users.

The latter four filtering steps aim to exclude users with either too little engagement on the topic or those who post so much that they are likely to be business or automated accounts.
We do not use the popular tool Botometer, as a recent study on mask-related tweets shows it to mostly find active human Twitter users \cite{he2021people}.
At the end of this process, we are left with \num{647730} users.
Finally, we use the Twitter API Friends call to collect the information about whom these users follow (``followees'' or ``friends''), thus resulting in the coverage of \num{598792} users.


\begin{table}[t]
    \caption{Number of tweets and associated users at different stages of data collection.} % Differences between before and after the mandate are statistically significant according to a Student's t-test.
    \label{tab:dataset}
    \centering
    \begin{tabular}{l r r}
    \toprule
    Dataset stage & N tweets & N users \\
    \midrule
    Keyword-based collection & \num{18245298} & \num{5935103} \\
    Geo-location filter & \num{5685866} & \num{1383729} \\
    Engagement \& network filter & \num{3614343} & \num{598792} \\
    \bottomrule
\end{tabular}
\end{table}



\section{Results}

\subsubsection{Stance Classification.}
\begin{figure}[t]
\centering
\vspace{-10pt}
\includegraphics[width=0.8\linewidth]{master_wcc_filter_20210217_trim.png} %,angle=-75,origin=c,clip,trim=5cm 4cm 1cm 1cm
\caption{GCC of follower network, colored by METIS score.}
\label{fig:gcc_plot}
\end{figure}
Following previous work on identifying controversial topics on social media~\cite{garimella2016quantifying,garimella2018quantifying}, we look for a bi-partitioning of the network that would indicate polarization.
We look for two sides of the debate because, ultimately, the decision whether to wear a mask is binary.
We start our analysis by considering the follower network induced by the users captured in our data, shown in Figure~\ref{fig:gcc_plot}.
In total, we are able to retrieve \num{509755293} follower-followee relationships, such that followers are all users whose tweets relate to masks.
When we constrain the followees to the set of users in our tweet dataset, the network contains $|V| = \num{598792}$ users and $|E| = \num{35763336}$ edges.
We use the graph partitioning algorithm METIS~\cite{karypis1998fast} to partition the network into two groups, repeatedly $N=100$ times with different random seeds, so to get an ensemble of partition assignments for each node, and use the average partition assignment for each node across the $N$ repetitions as a polarity score $p \in [0,1]$~\citep{cossard2020falling}.\footnote{Out of the two possible relabeling of each partition assignment, we choose the one that minimizes the distance from the current node labeling to maintain the partition identity fixed.}
We tune the relative size of the partitions by maximizing the number of users within \num{95}\% confidence interval of either extreme, and find the optimal proportion to be $1.1$:$1$ of pro-mask to anti-mask users.
As the algorithm does not indicate the actual stances of the users (in fact, it uses no content information whatsoever), each partition is assigned its stance manually by examining 10 sampled users on each side.
All users in the same partition expressed the same stance, which suggests a good separation.
The sampled users' stance is then propagated to all users within their partition, $0$ for pro-mask and $1$ for anti-mask.

We also verify the robustness of our result by looking at the community structure of the network.
The top-4 communities found by the Louvain algorithm~\cite{blondel2008fast}, which include more than $90\%$ of the nodes of the network, have median polarity scores of $0.2, 1.0, 0.0, 1.0$, which indicate clear separation and polarization in their composition.
The largest community, which includes \num{18708} users and is mainly pro-mask, has a slightly larger spread of polarity scores, which may indicate a larger divergence of opinions (and possibly doubts) in the population.
This result can also be seen in Figure~\ref{fig:metis_hist_score}, the histogram of polarity scores of the network, which shows a smoother decline on the left side than on the right side.
Finally, we assign a discrete label to each node according to its polarity score.
We heuristically choose the threshold $0.05$, which allows us to assign a label to $56.4\%$ of users: $28.8\%$ with 0 and $27.6\%$ with 1, thus leaving $43.5\%$ of users with an unknown label.\footnote{The 95\% Wald confidence interval upper bound for binomial proportions for a polarity score of $0.05$ is $0.10$, but we prefer to err on the side of caution and use a stricter threshold of $0.05$.}
The roughly equal shares of pro- and anti-mask users are in line with recent other manual annotation work on this topic \cite{cotfas2021unmasking}.

The two sides are colored as blue and red, and unknown as grey in Figure~\ref{fig:gcc_plot}.
%We heuristically choose the threshold $0.2$, which allows us to assign $72.7\%$ of users with a label in $\{0,1\}$, thus leaving only $27.3\%$ of users with an unknown label.
To assess the accuracy of the approach described so far, we sample 60 users from those assigned a stance label by METIS and annotate them manually.
To determine the manual label, we use all captured tweets by each user, making in aggregate 1866 tweets to consider.
The annotation was done by three authors, all knowledgeable in the U.S. mask debate.
Twelve percent of the data was used for inter-annotator agreement, which shows perfect agreement.
The exercise revealed an overall precision of $86.4\%$ (95\% confidence interval [0.76, 0.94]), with perfect precision for pro-mask class, but only $72.4\%$ for anti-mask case, with several users, incorrectly labeled as anti-mask by the algorithm.
The network structure suggests some connection of people who express doubts, although are not clearly anti-mask, with more extreme positions.
Indeed, it is likely that the anti-mask stance is more diverse than the pro-mask one, and cannot be easily represented with a single stance, similarly to the vaccine hesitancy case~\citep{cossard2020falling}.

\begin{figure}
\centering
    \includegraphics[width=.9\linewidth]{volume2.pdf}
    \caption{Daily time series of number of tweets by users classified by their mask stance. Vertical line on mandate date.}
\label{fig:volume}
\end{figure}

Figure~\ref{fig:volume} shows the daily volume of tweets by users classified as either pro- or anti-mask.
The two groups have a highly correlated time series, with a Pearson correlation of $r = 0.938$.
The engagement begins roughly at the time of the first CDC recommendation for wearing a mask on 2020-04-03.
The increase in volume confirms previous study which finds increased ``appetite'' to share opinions after major mask-related news \cite{cotfas2021unmasking}.
The subsequent peaks often revolve around major news stories involving masks, such as one when the U.S. Vice-President Pence visited a hospital without wearing a mask towards the end of April,\footnote{https://www.washingtonpost.com/politics/pence-meets-with-mayo-clinic-patients-staff-while-not-wearing-face-mask/2020/04/28/57c4200c-897e-11ea-9dfd-990f9dcc71fc\_story.html} continuous comparisons of the masking behavior of the two contenders for the U.S. Presidency in late May,\footnote{https://edition.cnn.com/2020/05/26/opinions/biden-mask-trump-ghitis/index.html} and subsequent adjustments to the guidelines by the public health officials who were trying to ``correct'' their previous messaging in mid-July.\footnote{https://edition.cnn.com/2020/07/12/politics/jerome-adams-surgeon-general-mask-mandate/index.html}
Given the high correlation of the two time series, we ask whether this effect is endogenous, i.e., if there is a causal feedback loop whereby one of the two stances answers the other.
A Granger causality test on these two time series for time lags from 1 to 14 days finds no strong relationship in either direction.
The likely implication of this negative result is that the volume of both stances has a common cause that is exogenous: external events that get discussed on Twitter.
Although we cannot exclude an effect with lag shorter than one day, the fact that the networks of two stances are well separated makes this hypothesis less likely.

\begin{figure}[t]
\centering
\includegraphics[width=0.75\linewidth]{Antivaxers_Map.png}\hspace{0.5cm}\includegraphics[width=0.085\linewidth]{Antivaxers_Map_Bar.png}
\caption{Fraction of anti-mask users over the total geolocated users in the 48 US states.}\label{fig:antimaskers_distribution}
\end{figure}

Figure~\ref{fig:antimaskers_distribution} shows the fraction of anti-mask users in each state. We find a particularly high concentration in the south, as well as some western states. These findings largely agree with contemporaneous studies of mask-wearing around the U.S.~\cite{katz2020map}.

\begin{figure}[t]
\centering
    \subfloat[Polarity score distribution]{\includegraphics[width=0.49\linewidth]{METIS_score_hist_all2.pdf}\label{fig:metis_hist_score}}
    \subfloat[Pol. leaning per mask stance]{\includegraphics[width=0.48\linewidth]{political_score_vs_metis_binarized_s2.pdf}\label{fig:metis_hist_politics}}
    \caption{Distributions of (a) polarity scores of users (computed via METIS), and (b) political leaning of users (inferred via follower relationships on Twitter) grouped by inferred stance on mask-wearing.}
\label{fig:metis_hist}
\end{figure}

\subsubsection{Political Leaning.}
As outlined in the Introduction, masking regulations have been strongly politicized. 
According to PEW, when surveyed in 2020, Republicans were more likely to call masks unnecessary, ineffective, oppressive, or unfair and were much less likely than Democrats to be concerned about unknowingly spreading COVID-19 to others.\footnote{\url{https://www.pewresearch.org/fact-tank/2020/10/29/both-republicans-and-democrats-cite-masks-as-a-negative-effect-of-covid-19-but-for-very-different-reasons/}}   %~\cite{kessel2020republicans}

We compare the mask stance of the users we are able to classify to their political affiliation, which can be glimpsed via their Twitter social network. 
In particular, we consider the Twitter accounts with known political leaning that the users follows.
In accordance to previous work, we assume that users mostly follow accounts that are in agreement with their political views \cite{golbeck2011computing}, and create a list of prominent political accounts in order to propagate their leaning to their followers.
The list, which we make available to the research community,\footnote{Available at https://tinyurl.com/poliaccounts} includes 501 accounts of members of the U.S. Congress, 79 governors, 70 party entities, and 67 Attorney Generals, as well as 157 media accounts from allsides.com,\footnote{\url{https://www.allsides.com/media-bias/media-bias-ratings}} and 67 journalists from politico.com.\footnote{\url{https://www.politico.com/blogs/media/2015/04/twitters-most-influential-political-journalists-205510}}
In total, it has a rather balanced set of 487 right-leaning and 454 left-leaning accounts.
Note that we do not use prominent politicians such as the U.S. President, as such accounts may be followed purely out of their popularity or importance.
For each user in our mask-related dataset, we count the number of accounts they follow in our list: right-leaning $N_R$ and left-leaning $N_L$.
We consider only users who follow at least 5 accounts in our list from either side, and calculate the aggregated political leaning score as $S_{PL} = (N_R - N_L) / (N_R + N_L)$, which results in $S_{PL} \in [-1,1]$ with $1$ the most right-leaning score.
Thus, we are able to identify the political leaning of \num{18422} users.

Figure~\ref{fig:metis_hist_politics} presents the distribution of political leaning scores for the two categories of users based on their stances on masks.
Pro-mask users are more likely to be following left-leaning accounts, and anti-mask ones the right-leaning ones, with almost no users existing in the middle political ground.
In fact, we find a strong polarization at the extremes of the political spectrum, especially for anti-mask users.



\subsubsection{Moral Values.} Moral values are directly linked to decision making~\cite{karandikar2019predicting}. Since there is an altruistic component to mask wearing, we ask what are the moral values expressed by the holders of the two stances.
We assess the moral narratives by employing the MoralStrength lexicon~\cite{araque2020moralstrength}, which holds the state-of-the-art performance in moral text prediction.
MoralStrength lexicon provides, along with each lemma, the \textit{Moral Valence score}, a numeric assessment that indicates both the polarity and the intensity of the lemma in each moral foundation.
According to this lexicon, the Moral Valence is expressed in a Likert-scale from 1 to 9, with 5 to be considered as neutral. If lower than 5 (middle of the scale) the scores reflect notions closer to Harm, Cheating, Betrayal, Subversion, and Degradation, while values higher than 5 indicate Care, Fairness, Loyalty, Authority, and Purity, respectively.
For each lemma in a tweet and for each foundation, we obtain a moral valence score which is then averaged for each tweet.
Negation correction was not applied, as foundation polarities do not directly translate as opposites (e.g., “not care” is not the same as “harm”).
The MoralStrength lexicon has a limited linguistic coverage; as a result only the 41.5\% of the tweets were found to express a moral foundation. For all the rest, we assigned the value 5, neutral point of the Likert scale.
This approach pushes the observed mean towards the center of the scale, but captures the variability of the value across all documents (we discuss the implications of this methodological step in Discussion).
%The main research question is \emph{whether moral narratives differ between the two sides: pro-mask and anti-mask supporters after the mandate}.
%We perform a Student's t-test on the scores obtained from tweets written before and after the mandate date (2020-04-03) for each moral dimension. 
To assess statistical significance, we use a Student's t-test (with the Benjamini-Hochberg correction for multiple hypothesis testing) on the scores obtained from tweets written before and after the mandate date (2020-04-03) for each moral dimension.

Figure~\ref{fig:morals_proanti} shows the mean moral value scores of each side in the periods before and after the mandate.
Before, the two sides display comparably similar values, except for \emph{care}, which is by far higher for the pro-mask side (significant at $p < 0.001$).
%The other values are not statistically different between the two sides during that period.
However, there is a clear shift in the moral narratives expressed after the mandate by both sides of the debate. 
To understand the context of these morally-charged expressions, we examine a sample of tweets for each side and value. 
%These findings show that there was indeed a shift in moral narratives after the mandate.
%All results are statistically significant with $p<0.001$.

First, we find an increase in the valence of \emph{authority} for the anti-mask side ($p<0.001$), which is mostly accompanied by criticism and mistrust of the decisions made by the authorities.
For instance, the anti-mask side associates wearing a mask with weakness and lack of leadership (\textit{``@JoeBiden Real leadership? With that thing on you look feeble''}). % Original: @JoeBiden Real leadership? Take off the mask you look weak
% Alternative: With that goofy black mask on your face, I can honestly say, you surely don't project real leadership, Joey
Conversely, the pro-mask side sees a lack of leadership in former President Trump's refusal to wear a mask (\textit{``Real leaders won't mind when the mask smears your absurd orange makeup''}). % Original: True leadership don't worry if a mask might wipe off his ridiculous orange makeup
Most of the examples we find in this moral category are indeed criticism of the authorities, which nevertheless signifies that authority is held in high importance, especially for the anti-mask side (it is the value with the highest valence).
The fact that post-mandate the authority-related keywords have higher valence on the anti-mask side suggests stronger criticism of the authorities than the pro-mask side (for whom the increase is significant only at $p=0.004$ before the correction).

\begin{figure}[t]
\centering
%\includegraphics[width=0.85\linewidth]{img/morals_proanti_after_text.pdf}
%\includegraphics[width=\linewidth]{img/morals_proanti_text.pdf}
\includegraphics[width=0.85\linewidth]{morals_proanti_text_new_c.pdf}

\caption{Moral valence in narratives expressed by pro-mask and anti-mask users in the periods before (lighter points) and after (darker points) the mandate.
%A Mann-Whitney U test per moral dimension on the two periods showed that all the differences are statistically significant with $p<0.001$. 
Dot represents the median value while the whiskers represent 5-95\% quantiles.}
\label{fig:morals_proanti}
\end{figure}
In terms of \emph{care}, both sides have a downwards shift after the mandate.
For the pro-mask side, this shift is accompanied by an increase in \emph{fairness} and \emph{loyalty}, which can be interpreted as a shift in focus from personal choice based on caring for others to complying with the mandate.
Also, in this case, the spotlight is often on the opposite side (\textit{``Masks protect others, which Trump doesn't care about. He cares only about himself.''}). % Original: The mask is for other people. Trump doesn't care about others. Just himself.
Conversely, anti-mask supporters express themselves by prioritizing much less the notion of \emph{care}, explicitly showing disregard for the protection of others, or simply stating that they do not care about being criticized for not wearing a mask %(\textit{``I am being told that I don't care about others bc I won't wear a mask. I'm still not wearing a mask'', or %``Not a mask wearer, and won't be, not that I don't care as you put it Gov, again, my choice, my reasons!'', 
(\textit{``I'm tired of being called a murderer because I don't wear a mask. Cannot understand how people are such thoughtless idiots''}). % Original: Im SICK of being called a murderer for not wearing a mask. I no longer understand how people are such mindless idiots

In addition, pro-mask supporters express significantly more \emph{loyalty} in their messaging after the mandate ($p<0.0001$).
Upon examining a sample of posts, we find the increase is primarily related to criticism of those not wearing masks as loyalists to a political affiliation or to Trump personally (\textit{``People don't wear masks because of Trump. Dear leader doesn't wear one, loyal followers go along''}). % Original: I think most of this mask-resistance behavior can be traced directly to Trump. Dear leader doesn't wear a mask, loyal followers follow suit

After the mandate, the \emph{fairness} value increases for both sides (both at $p<0.0001$).
Narratives about the fair treatment of individuals appear to be present equally on both sides, focusing on negotiation when wearing masks is reasonable (pro-mask: \textit{``What if a person not wearing it understands the risks?''}), % Original: Would it be fair to say the person who chose not to wear it understands their risks?
comparison with other rights violations (pro-mask: \textit{``The only ``rights'' that are violated is just not wearing a mask? They don't appreciate the nice life they have.''}), % Original: Can you imagine if the only way our ``rights'' were being violated was by being made to wear a mask? What a nice life those unappreciative wenches have
or whether the criticism is fair (anti-mask: \textit{``Colorado Governor Says People Who Refuse to Wear Mask are ``Selfish Bastards'' OK, and anybody who wants others to harm their immune system is a POS Marxist propagandist''}). % Original: Colorado Gov Says Anyone Who Refuses to Wear Mask is ``Selfish Bastard'' Fair enough, and anybody who wants people to harm their immune system is a POS Marxist propagandist
Instead, the valence of \emph{purity} does not change substantially.
We find that the notion of purity is not tied to religious views, but to what people consider `natural' and `healthy' (e.g. \textit{``My 8 y.o. niece will be heading back to school but will have to wear a mask! How is it physically or emotionally healthy?''}). % Original: My 9-year-old niece will be heading back to school this Fall in WI but has to wear a mask! How is it physically or emotionally healthy for a child to wear a mask?
%, pro-mask: \textit{``Just wear a mask already, it won't hurt you, and it could really help everyone.  Thanks Gelson's for setting a healthy example''}).

% \begin{figure}[t]
% \centering
% \includegraphics[width=0.85\linewidth]{img/morals_whole_text.pdf}
% \caption{Moral differences in narratives in the periods before and after the CDC national mask mandate (2020-04-03).
% For each tweet we estimate the moral valence and then we average over each foundation.
% A Mann-Whitney U test per moral dimension on the two periods showed that all the differences are statistically significant with $p<0.001$. 
% Dot represents the mean value while the whiskers represent the 5-95\% quantiles.}
% \label{fig:morals_whole_text}
% \end{figure}

%Figure~\ref{fig:morals_proanti} depicts this comparison, focusing only on the period after the mandate.
%\footnote{In this study, we employ the term "community" as defined in a network science, referring to a group of people with similar behaviors, and not as in social sciences where a community is meant as a group of people that interact with each other.}
%A Mann-Whitney U test comparison between the two sides shows that pro-mask users discuss more about care, loyalty, and purity presumably focusing on the concepts of protections of others and promoting unity in society.
%Notions of purity are not necessarily tied to religious views, but can be connected to what people consider ``natural''; praying and preaching may also be influencing the purity scores.



We further examine the evolution of moral values expressed by the two sides in time and model it via an Interrupted Time Series (ITS) linear model. 
Figure~\ref{fig:moral_its} depicts the evolution of the average moral foundation score per side, with a vertical dashed black line indicating the date of the official mask mandate.
Complementing our previous analysis, the interrupted time series model shows that for all the moral dimensions, after the mandate, there is an evident change in behavior by both sides. 
Perhaps the most interesting moral dimension is \emph{loyalty}, whose signal is evidently diverging for two sides exactly after the mandate date and continues the same trend until the end of our data collection. 
%This foundation is related to our long history as tribal creatures and our ability to form coalitions and compete~\cite{Graham2012}. The underlying virtue of this foundation is the self-sacrifice for the group~\cite{haidt2012righteous}. 
We also observe that not only does the value of \emph{care} decreases, the trend is downward over time, signaling a progressive shift in the debate.
Similarly, the value of \emph{purity} has a progressively negative trend for pro-mask side over time.
% Another interesting observation emerging from the ITS analysis is the expression of fairness. 
% From the analysis of variance without considering the time dimension, fairness appears to be stable for the pro-mask side while slightly increasing for the anti-mask population. 
% However, from the ITS we notice that the time evolution actually shows a steady increase inside the pro-mask side which comes to surpass the quite stable trend of the anti-mask side.
Thus, we find that the temporal dimension of the data can be instructive about the evolution of the rhetoric in terms of divergence between the two sides of conversation and changes in emphasis.

\begin{figure}[t]
\centering
    %\includegraphics[width=0.85\linewidth]{moral_values_its_all.pdf}
    \includegraphics[width=0.90\linewidth]{moral_values_its_all_mean5.pdf}
    \caption{Time series of moral value scores of pro-mask and anti-mask users, along with an interrupted time series analysis model.}
\label{fig:moral_its}
\end{figure}




\subsubsection{Collectivism vs Individualism.}
\begin{table}[t]
    \caption{Use of singular and plural personal pronouns in a tweet by side, before and after the mask mandate. } % Differences between before and after the mandate are statistically significant according to a Student's t-test.
    \label{tab:i_we}
    \centering
    \begin{tabular}{l c c c c}
    \toprule
    Pronoun        &  \multicolumn{2}{c}{Singular} & \multicolumn{2}{c}{Plural} \\
            \cmidrule(lr){2-3} \cmidrule(lr){4-5}
    Mandate	&  before & after  & before & after \\
    	\midrule
Pro-mask &  0.55 &  0.48 &  0.11 &  0.10 \\
Anti-mask & 0.53 &  0.60 &  0.09 &  0.10 \\
    \bottomrule
\end{tabular}
\end{table}

One of the main purposes of mask wearing is the protection of others, an expression of solidarity within the in-group against an external threat.
Thus, we turn to the Individualism-Collectivism (IC) dimension \cite{hofstede2001culture}, which captures the standing of individuals as interdependent members of a collective. % which was developed to describe precisely the social patterns formed when individuals describe themselves as interdependent members of a collective, at the expense of their autonomy and independence.
We operationalize it via the personal pronouns used in the tweets, mainly first-person singular (``I'', ``me'',``mine'' etc.) and first-person plural (``we'', ``us'', ``ours'' etc), following existing literature~\cite{twenge2013changes}.

Table~\ref{tab:i_we} shows the average usage of the two sets of pronouns in the tweets posted by the two sides of the debate, separately before and after the mask mandate. 
In the period before mask mandate, anti-mask supporters use \emph{I} and other singular pronouns at 0.53, and after the prevalence increases to 0.60 ($+0.07$, $p<0.0001$), which points to an increased focus on the individual. 
Instead, pro-mask supporters decrease their usage of singular pronouns ($-0.07$, $p<0.0001$).
The mention of \emph{We} and other plural pronouns by both pro- and anti-mask supporters remain at the same levels ($\pm 0.01$), thus indicating no significant shifts in the focus on the self-identified group.
%The mention of \emph{We} by the anti-mask supporters increases slightly after the law and decreases for pro-mask supporters, thus indicating a minor shift in the focus on the self-identified group.
Thus, although having comparatively similar usages of singular pronouns before the mandate, the debate after the government's messaging becomes more individualistic for anti-mask side and less so for advocates of masking.
%Next, we examine the arguments both sides put forward to justify their views. 



\subsubsection{Information Environment.} % Content Analysis
%social ->pro conspiracy
%social ->no emotion
%To better understand the two sides' concerns, we dive deeper into the analysis of the textual content of their tweets. 
%Below, we apply NLP techniques to quantitatively and qualitatively analyze the textual content of the tweets posted by both sides of the argument. For this, 

%\subsubsection{Text analysis}
% At this point it is crucial to understand the different narratives expressed not just in terms of moral content, but of the actual concepts expressed.

Finally, we turn to the information context where the debate takes place, in form of links to external sources and information from peers.
We begin by performing basic normalization steps on the original tweet text, including removing punctuation, accents, contractions, and stopwords, substituting emojis with a text description,\footnote{Using \texttt{emoji} library \url{https://pypi.org/project/emoji/}.} and finally by performing lemmatization.
We then identify the most distinguishing words used by each side by calculating the odds ratio of using particular words (lemmas) by one side compared to the other before and after the mask mandate.
We filter these words by their total frequency since otherwise rare terms would emerge as most distinguishing.  
Because the periods before and after the mandate have significant differences in volumes, we set the word frequency threshold to $100$ and $800$ for the two periods, respectively.
%After this filtering, we have \num{623} and \num{1079} words to examine for the periods before and after the mandate, respectively.
%%Table~\ref{tab:odds_after} lists the top $10$ words that are more likely to be used by the anti-mask and pro-mask sides by log odds ratio in the period after the mandate.
Politician names appeared in the top terms and were grouped together, hence resulting in two-word terms. 
The resulting top terms are:
\begin{itemize}
    \item Pro-mask: \emph{trumpvirus, Louie Gohmert, Putin, Herman Cain, wearadamnmask, penny, Mayo, GOP, DeSantis, covidiots}
    \item Anti-mask: \emph{riot, micron, virtue, leftish, loot, sheeple, unhealthy, bacterium, MSM, antifa}
\end{itemize}


% \begin{table}[t]
%     \caption{Top distinguishing words between pro- and anti-mask users, by odds ratio in the period after the mask mandate. Politician names that appear in the top terms are grouped together.}
%     \label{tab:odds_after}
%     \centering
%     \begin{tabular}{l l}
%     \toprule
%     Top 10 Pro-mask Terms & Top 10 Anti-mask Terms \\
%     \midrule
%       trumpvirus   &  riot \\
%       Louie Gohmert  & micron\\
%       Putin & virtue \\
%       Herman Cain & leftish\\
%       wearadamnmask & loot\\
%       penny &  sheeple\\
%       Mayo & unhealthy\\
%       GOP  & bacterium \\
%       DeSantis & MSM \\
%       covidiots & antifa\\
%     \bottomrule
%     \end{tabular}
% \end{table}

% \begin{figure}[t]
%     \centering
%     \includegraphics[width=\linewidth]{./img/odds_ratio_morals.pdf}
%     \caption{Moral differences of pro-mask and anti-mask side in the odds ratio lemmas. For each foundation, we present the average and standard deviation of 
%   \KK{here I do not know the number of tweets that these words correspond to.}  }
%     \label{fig:odds morals}
% \end{figure}

% We compare the moral content of these words by employing the MoralStrenght lexicon.
% For each of the \num{1079} terms, we obtain the respective moral valence score from the lexicon. %, and considered the term as a pro-mask term if the odds ratio was negative and an anti-mask term otherwise.
% Figure~\ref{fig:odds morals} shows the average of moral valence per side for each moral value.
% It emerges that the pro-mask side is more concerned with concepts of fairness, signaling that their priority is the protection of others. At the same time they tend to express slightly more pronounced concerns of in-group loyalty in their messaging.
% In-group loyalty underlines virtues of self-sacrifice for the group or side which the individual belongs to.
% The differences are statistically significant with $p<0.001$ according to a Mann-Whitney U test. 


The most distinctive words by each side are highly politicized: aside from references to politicians (similar to other recent findings around masking \cite{sanders2021unmasking}), we find several words used to attack the other side.
For instance, pro-maskers often use the term `trumpvirus' to refer to COVID-19, % (technically, SARS-COV-2), <<< not sure why we have this technicality here
as a political response against the term `chinavirus' used by President Trump: 
\textit{``$@$realDonaldTrump  \#HermanCainRIP one more death due to \#TrumpVirus . Just think. If only he wore a mask and NOT attended the Tulsa \#CoronavirusRally''}. % Original: $@$realDonaldTrump The \#TrumpVirus has killed one of his supporters, \#HermanCainRIP. Just think. If he had only worn a mask and NOT attended the \#CoronavirusRally in Tulsa
The focus on individuals is worth of notice, as in the case of Luie Gohmert, who strongly supported the use of hydroxychloroquine and attended a House Judiciary Committee hearing without wearing a mask, or in the case of Herman Cain, a Republican politician who opposed the mask mandate and later died of COVID-19.\footnote{\url{https://en.wikipedia.org/wiki/Herman_Cain#Health_and_death}}
% \textit{"$@$MikePence $@$realDonaldTrump How about a mask mandate to slow the spread of the Trumpvirus, 150,000 dead, the economy is the worst it has ever been, gassing protesters for a cheap photo op, voter suppression, please tell BunkerBoy to change the voting date we need Pelosi as President now."} \\
Similarly, the derogative term `covidiots' is used to describe anti-mask supporters.
%\textit{``Just want to say how proud I am of my wife for wearing a mask at the grocery store.  For several weeks now, covidiots have laughed, coughed, and generally hassled her, but she has stayed tough''}. % cont:  Social pressure is such a powerful force. We need to lead by example, like she does

The opposite side instead focuses on the `riots' that would happen if lockdown and mask mandates are enforced, and on the claim that masks can stop a `bacterium' but not a `micron'-size virus. 
%For instance: \textit{``An N95 mask, by definition, is 95\% efficient in stopping 0.3 micron particles, but is less than 95\% efficient in stopping 0.1 micron particles (such as individual virus particles)''}. 
There are also claims that masks are `unhealthy' as they impede breathing, and are instead just `virtue signaling' devices: 
\textit{``Virtue signaling: mask and gloves. People lived with coronaviruses for 100+ years. Turn off the TV.''} % Original: `The newest virtue signal: mask and gloves. People have been walking around with coronaviruses for 100+ years. Turn off the TV
%And we were worried about Islamists forcing headscarves on is \#ConqueredFromWithin''}).
% \textit{``$@$tedcruz Ted Cruz is a champions!!!  Snowflakes are going nuts. Keep the mask off senator! And you leftish are the main biggest cause for the virus spreading by protesting like retards and the whole country understands it on Nov 3rd! VOTE RED!!!''}
A right-wing, anti-establishment sentiment can be inferred from the reference to `sheeple': \textit{``And the CDC telling everyone to mask up is just another test to see how long the sheeple will obey. \#Globalists''}, % original: And the CDC now saying for everyone to wear a mask is just another test to see how long the sheeple will obey. \#Globalists
and from references to supposedly derogative terms such as `leftish' and `antifa'.
%
Nonetheless, what is labeled as the anti-mask side is also more varied in its opinions, ranging from just hesitant to fully conspiratorial, somewhat similar to the spread of opinions around vaccines~\cite{cossard2020falling}.

%\subsubsection{Topic modeling}


\begin{table}[t]
    \centering
    \caption{LDA topics with the most representative words, extracted separately from tweets for each mask debate side. The topic prevalence is reported in parenthesis.}
    \label{tab:lda_topics}
    \footnotesize
    \begin{tabular}{p{0.01\linewidth}p{0.05\linewidth}p{0.75\linewidth}}
    \toprule
    \parbox[t]{0mm}{\multirow{4}{*}{\rotatebox[origin=c]{90}{pro\hspace{-2mm}}}}  
    &T1 (.35) & distance, wear, social, day, mandate, today, friend, hope, state, stay\\
    & T2 (.33)&  wear, people, virus, covid, bad, protect, medical, infect, spread, science\\
    &T3 (.31) & wear, Trump, Cain, rally, covid, Herman, die, Tulsa, death, kill\\
    \midrule
    \parbox[t]{0mm}{\multirow{4}{*}{\rotatebox[origin=c]{90}{anti\hspace{-2mm}}}}     
    & T1 (.46) & wear, people, covid, virus, work, distance, die, thing, protect, spread\\
    & T2 (.31) & wear, mandate, people, state, store, vote, business, leave, today, order\\
    & T3 (.22) & man, Trump, rally, school, eye, fuck, kid, hope, stupid, big\\
    \bottomrule
    \end{tabular}
\end{table}

% For Pro T1 .33, T2 .35, T3 .31; Anti: T1 .22 , T2 .31, T3 .46. 

Moving to higher-level constructs, we aim to uncover common patterns in the argumentation proposed by both sides by applying a topic modeling approach base on  latent Dirichlet allocation (LDA)~\cite{blei2003latent}.
Limitations of LDA clustering of short text are known, still, it offers a good compromise between clustering performance and computational cost~\cite{qiang2020short}
\footnote{Using the Gensim library~\url{https://radimrehurek.com/gensim}.}.
% ~\cite{rehurek2011gensim}
To derive the optimum number of topics $k$, we optimize the topic coherency ($C_v$ metric~\cite{roder2015exploring}) models with $k \in [2,10]$.
For both the pro-mask and anti-mask sides, the model with $k=3$ is the best fit.
Table~\ref{tab:lda_topics} presents the salient keywords that form the corresponding topics.
The topics are ranked according to their prevalence, with T1 the most prevalent one, and similarly, the terms are ranked by descending importance for the specific topic. 

From the emergent topics, the most prominent one on the pro-mask side concerns the various interventions, including \emph{social distancing} and \emph{wearing} a mask.
%\emph{protection} against spreading, including references to \emph{medicine} and \emph{science} 
(\textit{``COVID is not a flu! Everyone needs to wear a mask to protect others from these germs.''}), % Original: Coronavirus IS NOT A FLU! EVERYONE needs to wear a mask to protect others from each other's germs
echoing our earlier finding of higher care value of this side.
The second topic includes references to \emph{medicine} and \emph{science}, while the third centers around (Republican) political figures.
On the anti-mask side, the most prominent topic also concerns the interventions, but instead focuses on whether interventions \emph{work} against the \emph{spread}.
%(\textit{``It's stupid because the mask does not protect you. Only a n95 mask can protect you from Airborne diseases. But wait, even with a mask, you still exposing the mucous membranes in your eyes.''}).
%The topic also includes tweets that express concerns about schools and kids, both in terms of insecurity about sending children back to school and as a necessity to reopen their businesses.
The second one puts the mandate in the context of the \emph{businesses} and \emph{stores}, in contrast with T1 from the pro-mask side which speaks about other interventions such as social \emph{distancing} and \emph{staying} at home.
The third topic is about political \emph{rallies}, and argues that masks are not useful (since the \emph{eyes} are exposed to the virus: \textit{``Still, even with a mask, you are uncovering the mucous membranes in your eyes''}). % Original: But wait, even with a mask, you still exposing the mucous membranes in your eyes

% Example tweets per topics. Promask topic 1
% \textit{$@$BostonFact @connierainie @ZackBornstein 1. Coronavirus IS NOT A FLU! 2. EVERYONE needs to wear a mask to protect others from each other's germs. Not just the young. 3.If u hate to wear a mask, you're really going to hate being on a ventilalor. 4. Show some respect to your readers...proofread your work. Have some pride."}
% Promask topic 2
% \textit{"$@$JillWineBanks Police should issue waivers: "By refusing to wear a mask and social distance I hereby agree to not seek medical care should I contract Covid-19, therein endangering the lives of health care staff.""}

% Antimask topic 1:
% \textit{"$@$RedWaveArmy1 CA did the same shit, it's stupid because the mask does not protect you. Only a n95 mask can protect you from Airborne diseases. But wait, even with a man, you still exposing the mucous membranes in your eyes."}

% topic 2:
% \textit{"$@$TexasLaurieD $@$RealRonPhillips $@$realBradStaggs While I agree with you ... In part I have put my jobs on the line against this mask issue... Its not a business job to inforce the stupid lockdown maskhystaria rules... Mass noncompliance is our only way out of this. I do where 1 on specic ocqssions of My choosing..thats my issue"}

% topics 3:
% \textit{"$@$believeland79 The mask doesn't help the social distance, come on man."}

% Promask topics
% [(0,
%   '0.109*"wear" + 0.026*"people" + 0.014*"virus" + 0.013*"covid" + 0.013*"bad" + 0.013*"protect" + 0.012*"medical" + 0.010*"infect" + 0.010*"spread" + 0.010*"science"'),\\
%  (1,
%   '0.030*"distance" + 0.027*"wear" + 0.023*"social" + 0.017*"day" + 0.015*"mandate" + 0.011*"today" + 0.010*"friend" + 0.008*"hope" + 0.008*"state" + 0.008*"stay"'),\\
%  (2,
%   '0.066*"wear" + 0.048*"trump" + 0.033*"cain" + 0.031*"rally" + 0.029*"covid" + 0.026*"herman" + 0.026*"die" + 0.016*"tulsa" + 0.013*"death" + 0.012*"kill"')]\\
  
%   antimask topics\\
%   [(0,
%   '0.017*"man" + 0.015*"trump" + 0.013*"rally" + 0.010*"school" + 0.010*"eye" + 0.010*"fuck" + 0.008*"kid" + 0.008*"hope" + 0.007*"stupid" + 0.007*"big"'),\\
%  (1,
%   '0.062*"wear" + 0.033*"mandate" + 0.015*"people" + 0.012*"state" + 0.011*"store" + 0.010*"vote" + 0.009*"business" + 0.009*"leave" + 0.008*"today" + 0.008*"order"'),\\
%  (2,
%   '0.096*"wear" + 0.030*"people" + 0.024*"covid" + 0.017*"virus" + 0.015*"work" + 0.011*"distance" + 0.010*"die" + 0.009*"thing" + 0.009*"protect" + 0.008*"spread"')]\\


\begin{table}[t]
    \centering
    \caption{Counts of the top 30 URL domains posted by pro- and anti-mask users. Domains colored by class: news and news aggregators (black), social media and social media automator/aggregators (\textcolor{BrickRed}{red}), business platforms (\textcolor{Blue}{blue}), medical organization (\textcolor{OliveGreen}{green}).}
    \label{tab:top_urls}
    \footnotesize
    \begin{tabular}{r l r l}
    
% Pro-mask (n=74985)  Anti-mask (n=66770) 
% Colors: https://www.overleaf.com/learn/latex/Using_colours_in_LaTeX#Reference_guide
\multicolumn{2}{c}{Pro-mask } & \multicolumn{2}{c}{Anti-mask}   \\
\textcolor{Black}{rawstory.com} & 2317 & \textcolor{BrickRed}{youtube.com} & 3485 \\
\textcolor{Black}{cnn.com} & 2000 & \textcolor{Black}{thegatewaypundit.com} & 1341 \\
\textcolor{BrickRed}{youtube.com} & 1751 & \textcolor{Blue}{etsy.me} & 1210 \\
\textcolor{Black}{washingtonpost.com} & 1393 & \textcolor{BrickRed}{instagram.com} & 912 \\
\textcolor{Black}{a.msn.com} & 935 & \textcolor{Black}{foxnews.com} & 903 \\
\textcolor{Black}{apple.news} & 872 & \textcolor{Blue}{zazzle.com} & 796 \\
\textcolor{Black}{huffpost.com} & 758 & \textcolor{Black}{breitbart.com} & 781 \\
\textcolor{Black}{news.yahoo.com} & 686 & \textcolor{Black}{nypost.com} & 472 \\
\textcolor{Black}{flip.it} & 630 & \textcolor{Blue}{fineartamerica.com} & 453 \\
\textcolor{Black}{nytimes.com} & 587 & \textcolor{Black}{fxn.ws} & 393 \\
\textcolor{Black}{nbcnews.com} & 573 & \textcolor{Black}{westernjournal.com} & 362 \\
\textcolor{Black}{thehill.com} & 527 & \textcolor{BrickRed}{dlvr.it} & 344 \\
\textcolor{Black}{dailykos.com} & 486 & \textcolor{Blue}{pixels.com} & 317 \\
\textcolor{BrickRed}{instagram.com} & 458 & \textcolor{BrickRed}{buff.ly} & 288 \\
\textcolor{Black}{businessinsider.com} & 449 & \textcolor{Black}{theblaze.com} & 282 \\
\textcolor{Black}{thedailybeast.com} & 402 & \textcolor{Black}{bizpacreview.com} & 268 \\
\textcolor{Black}{newsweek.com} & 371 & \textcolor{Black}{infowars.com} & 250 \\
\textcolor{Black}{theguardian.com} & 343 & \textcolor{Blue}{etsy.com} & 246 \\
\textcolor{Black}{usatoday.com} & 337 & \textcolor{BrickRed}{ift.tt} & 236 \\
\textcolor{Black}{yahoo.com} & 333 & \textcolor{Black}{cnn.com} & 231 \\
\textcolor{Black}{cnbc.com} & 307 & \textcolor{Black}{twitchy.com} & 217 \\
\textcolor{Black}{politico.com} & 301 & \textcolor{Blue}{ebay.us} & 202 \\
\textcolor{Black}{newsbreakapp.com} & 295 & \textcolor{OliveGreen}{ncbi.nlm.nih.gov} & 196 \\
\textcolor{BrickRed}{buff.ly} & 260 & \textcolor{BrickRed}{facebook.com} & 192 \\
\textcolor{Black}{npr.org} & 252 & \textcolor{OliveGreen}{nejm.org} & 191 \\
\textcolor{Black}{politicususa.com} & 236 & \textcolor{BrickRed}{newsbreakapp.com} & 175 \\
\textcolor{Black}{apnews.com} & 231 & \textcolor{Black}{a.msn.com} & 173 \\
\textcolor{Black}{nypost.com} & 223 & \textcolor{Black}{google.com} & 162 \\
\textcolor{Black}{latimes.com} & 209 & \textcolor{Black}{dennismichaellynch.com} & 152 \\
\textcolor{Black}{mol.im} & 208 & \textcolor{OliveGreen}{aapsonline.org} & 142\\
    \end{tabular}
    \vspace{-.10pt}
\end{table}




%\subsubsection{URL domains}

Finally, the two sides of the debate have about the same proportion of tweets with URLs (around 23\%).
Earlier works have shown a polarization in terms of news sources accompanying political polarization~\cite{garimella2021political}, however here we find differences in behavior beyond sharing news media.
Table~\ref{tab:top_urls} shows the top domains of the URLs posted by pro- and anti-mask users, along with the counts.
Pro-mask users overwhelmingly post URLs pointing to news websites or aggregators.
YouTube and Instagram feature prominently in both lists, though anti-mask users favor YouTube more than twice the second most popular domain.
Anti-mask tweets also link to a variety of business platforms, including Etsy and Ebay, and lesser-known ones such as Zazzle, a platform for custom-designed products.
In addition, anti-mask users link to the governmental agency National Center for Biotechnology Information (NCBI), the New England Journal of Medicine (NEJM), and the Association of American Physicians and Surgeons (AAPS).
These findings stand in contrast to a smaller recent study of geolocated-only tweets~\cite{he2021people} that finds that anti-mask tweets were less likely to share external information from public health authorities.
An explicit comparison of the captured tweet sets would be necessary to resolve these findings.




\section{Discussion \& Conclusions}

%Our analysis sheds light on various facets of the debate around mask-wearing and reveals several specific insights worth carefully considering.

Our analysis reveals that the government messaging about mask wearing provoked --instead of the intended focus on the benefits to the communities and society at large-- a marked shift in the moral values towards higher politicization of the issue, with an increased focus on authority and fairness.
We argue that interventions targeted to those resistant to mask-wearing should center around these values, instead of appealing to those valued less as debate goes on.

It is no surprise that we find the mask debate captured on Twitter to be highly politically polarized. %, with pro-mask supporters being generally more left-leaning and anti-mask ones more right-leaning.
It is well-known that polarization in the U.S. political scene has been growing in the last decades\footnote{\url{https://www.pewresearch.org/politics/2021/11/09/beyond-red-vs-blue-the-political-typology-2/}},   %~\citep{pew2021beyond}, 
and this growing polarization has important effects on several areas, including public health.
As polarization and health behaviors intersect, it is crucial to understand their interaction to design effective health policies.
For instance, the ideological and moral facets of health intervention perceptions are closely related to the compliance in the population, and the resulting effectiveness, as shown for instance in recent statistics relating vaccination uptake to political leaning\footnote{\url{https://acasignups.net/21/07/20/update-us-covid19-vaccination-levels-county}}.  % ~\cite{gaba2021covid}.
However, shifting established associations between health-related beliefs and political leaning may be challenging. 
For instance, the attendees of Donald Trump's rally in August 2021 booed him after he voiced support for the COVID-19 vaccination drive,\footnote{\url{https://www.theguardian.com/us-news/2021/aug/22/donald-trump-rally-alabama-covid-vaccine}} thus illustrating that even the forerunners of the Republican party may have difficulty reaching their constituents.
As the pandemic develops, and more mandates are issued by the governments, constant polling and monitoring is essential in establishing the public response to these measures (as of mid-2021, the attitudes toward masking are still highly polarized)\footnote{\url{https://www.politico.com/news/2021/08/02/poll-americans-back-return-of-masking-502144}}.
%Further supporting the political polarization frame, our results highlight that a large part of the narrative across the moral values consists of pointing fingers and blaming the opposing side.

% Much of this polarization is often attributed to misinformation, which spreads easily in the informational echo chambers supported by the current technology~\citep{nikolov2020right}
% Indeed, an `infodemic' of misinformation has been declared as an exacerbating factor in the COVID-19 pandemic \citep{kearney2020twitter}.    
% We find both sides to link to other social media platforms, most notably YouTube (video sharing) and Instagram (image and video), although the anti-mask content favors social media sources substantially more.
% Contrary to expectations, we find that anti-mask users link to URLs to medical organizations (NCBI, AAPS) and are thus aware of the scientific literature.
% Nevertheless, they reach different conclusions, which may suggest motivated reasoning and biased assimilation are at play~\citep{lord1979biased}.
%\mynote{Can use this as a link to the next part on moral values, e.g., the information they have access to is the same, but their moral values and judgement might differ.}

We find that different moral values underpin this reasoning emphasized by the two camps.
Pro-mask arguments highlight loyalty and fairness while criticizing the opposing leadership.
The anti-mask ones, on the other hand, focus on the authoritarian and oppressive aspects of the mandate and show a lack of concern for the effects of their actions on others.
In accordance with these results, when examining this phenomenon through the lenses of the Collectivism-Individualism theory, we notice a decisive shift of the anti-mask community towards individualism, with more intense use of first-person personal pronouns.
We note the lack of loyalty among the values emphasized by the anti-mask side, which tends to hold a conservative political view, and differs from the commonly observed ones associated with conservatism: authority, loyalty, and purity~\cite{kivikangas2021moral,fulgoni2016empirical}. 
This may be a response to the pro-social framing of the mask intervention, thus leading its opponents to de-emphasize the in-group narrative usually common to their side.
This interpretation may point to motivated reasoning, wherein the desired conclusion modifies the worldview usually taken.
Interestingly, the ITS analysis shows divergence over time between the two groups on the value of loyalty, which is alarming since such disagreement, if not adequately addressed, can lead to severe societal polarization.

%Our study contributes to the observed connection between the cultural dimensions of individualism-collectivism and the willingness of individuals to take actions that contribute to the societal good.
%\mynote{How do our results resonate with these findings?}

This polarization seems to have been accompanied by a lively commercial activity. 
When examining the links posted by the two sides of the issue, we find a prominent existence of commerce-related platforms including \emph{etsy.com}, \emph{ebay.us}, as well as platforms for custom creation of merchandise such as \emph{zazzle.com}. 
Indeed, a brief search on these websites reveals an assortment of t-shirts, coffee mugs, face masks, and baby bibs with political messages from both sides of the debate. 
Sometimes historical symbols were used to make a stance, such as the use of the yellow star --like those forced on Jews by Nazi Germany-- which was sold by a Nashville store protesting against the vaccination campaign.
After community criticism, the item was removed\footnote{\url{https://www.bbc.com/news/world-us-canada-57297902}}.  %~\cite{bbc2021nashville}.
Our findings suggest that there is an active development of symbolism and aesthetics of the resistance movement, and it would be a fascinating subject of research to uncover the non-verbal representations of the self and the group, expressions of values, and calls to action \cite{mcgarry2019aesthetics,awad2020protest}.
Awareness of such symbolism and self-conceptualization is vital for crafting appropriate messages and fostering communication between the two sides.

%There are preliminary experiments that attempt to integrate the sentiment found in the COVID19-related deliberation on social media into agent-based modeling in order to understand the decisions of a population \cite{dorr2021mining}.
%https://books.google.it/books?hl=en&lr=&id=SMI2EAAAQBAJ&oi=fnd&pg=PA46&dq=%22social+media%22+mask&ots=Ok5NJC0k65&sig=_1ykNJN_PC2Zo7TNd_SLUf7RKVA&redir_esc=y#v=onepage&q=%22social%20media%22%20mask&f=false

% Limitations
Despite the volume and span of the data analyzed here, this study has some limitations as most social media studies. 
Especially when concerning politically charged topics, often a minority of vocal users dominates the conversation~\cite{Mustafaraj2011}, thus making the opinions of the ``silent majority'' challenging to discern.
The observed relationships between the stances on masking, moral values, and political stance are limited to those who choose to express themselves vocally.
In addition, we use a high-precision vocabulary to measure moral values, which has the side effect of pushing the average valence in our results close to the neutral point, due to a large number of tweets for which we cannot extract moral valence.
Traditional surveys are necessary to reach those not as comfortable expressing their opinions online; however, even those have selection biases.
The automated tools utilized in this study are not perfect: after manual examination, we find that the network-based classifier achieved an accuracy of 72.4\% for the anti-mask class, thus introducing noise in the subsequent analysis.
However, from the experience of manually labeling the users, we postulate that the complexities of human expression, including humor and sarcasm, may limit the best possible performance of such a classifier.
Similarly, the inference of moral values from the text may struggle with vocabulary mismatch, sarcasm, and self-censorship. 
Finally, the generalizability of this study is limited by the unique circumstances of an unprecedented global pandemic happening in a hyper-connected world during a politically polarized social environment.
These considerations must be taken into account when comparing our findings to new scenarios.


\subsection{Ethical Considerations}

The dataset presented here contains only tweets which were publicly available at the time of the collection.
We make the dataset publicly available to the research community in compliance with the Twitter Terms of Service\footnote{\url{https://developer.twitter.com/en/developer-terms/}}, that is, sharing only the tweet IDs of the collected posts, which will have to be re-collected. 
In this paper, we have rephrased all quoted tweets to prevent re-identification of their authors.
This will ensure that the tweets which have been removed (either by the user or the platform) will not be available.
Although large, this dataset does not include users with particular disabilities which may disallow them to interact with the platform, as well as minors and those blocked by Twitter.
On the other hand, the content collected here affects not only those who have posted it, but also those who viewed or interacted with it, which may be orders of magnitude more users, since most users are ``lurkers'' who consume social media content without posting \cite{van20141onepercent}.
Ultimately, the masking decisions made by people engaging in this deliberation may directly affect the health and life of vulnerable people, such as those with autoimmune disorders or other conditions making them especially vulnerable to COVID-19. 
Additionally, the sometimes aggressive rhetoric in this material may not be suitable for young Twitter users, or those dealing with mental health issues. 
Also, although the focus of this study is the moral dimension of the debate, we caution public health communicators not to overemphasize moral or emotional dimensions of their message (or attempt to emotionally manipulate its audience), but provide the clearest and most informative messaging possible.
Further, we would like to discourage the tools used in this study to be used for targeting individuals espousing particular opinions for harassment or undue surveillance, and to follow the AAAI ethical guidelines\footnote{\url{https://www.aaai.org/Conferences/code-of-ethics-and-conduct.php}} in the application of these findings.

% From another paper

% Major efforts are ongoing to better understand the reasons for vaccine hesitancy in Europe and around the world. 
% The European Centre for Disease Prevention and Control (ECDC) conducts regular surveys to measure COVID-19 behaviors, and supports the inclusion of additional data sources to understand the beliefs and expectations around vaccination \cite{ecdc2021facilitating}.
% The surveillance tools and opinion community detection methods presented in this paper may assist in better understanding the popular sentiments, associations, questions, and misunderstandings that circulate on one of the most popular social media in Italy.
% Although this dataset captures a small portion of Italian-speakers, most social media users do not engage in posting, and use platforms for entertainment and as a source of information \cite{van20141onepercent}, thus the potential audience of the content captured here may be an order of magnitude larger than the users we analyze.
% Still, we should remember that there are other groups which are not captured in this data, including people not having access to internet and the platform, and those with disabilities not allowing them to interact with it.

% While the dataset collected for this study contains only publicly posted tweets, it is possible that it contains posts from vulnerable groups, including those with serious or chronic health conditions relevant to COVID-19 and the vaccination campaigns, those emotionally or psychologically vulnerable, and family members and friends who are concerned for the wellbeing of their loved ones, among others. 
% In order to preserve the privacy of the individual users, we reveal the Twitter handles only of public figures or anonymous accounts in this paper, and otherwise do not use identifiable information in the analysis of the data. 
% Furthermore, we will abide by Twitter's Terms of Service\footnote{\url{https://developer.twitter.com/en/developer-terms/}} by making the IDs of the collected available upon request, such that those which have been deleted by their authors will not be available when the metadata is re-collected (unfortunately, limiting reproducibility somewhat).
% However, to support the transparency of this work, we make the code available to the research community, especially that pertaining community detection, random walk controversy score computation, and community-aware topic modeling [anonymized].
% Finally, as opinion surveillance methods used in this work may be applied to identify other communities and individuals therein, we urge the research community to strictly abide by the code of ethics for the research and application of these tools (such as one by AAAI\footnote{\url{https://www.aaai.org/Conferences/code-of-ethics-and-conduct.php}}) in order to minimize harm to the subjects of research.



%\section{Appendix}

% \subsection{Individualism vs Mask usage around the world}

% \begin{figure}
% \centering
% \includegraphics[width=\linewidth]{img/individualism-coefficient.pdf}
% \caption{Individualism vs after-intervention coefficient from ITS analysis for different countries.}
% \label{fig:ind-coeff}
% \end{figure}

% To validate the robustness of our findings w.r.t. the relationship between individualism and the adoption of masks, we look at the phenomenon across the globe.
% We use the survey data from YouGov to assess the impact of mask mandates for several countries.
% For each country, we split the data in before and after the mask mandate date (referred to as \emph{intervention}), and compute the fraction of respondents that reply `Frequently' or `Always' to the question ``How often have you worn a face mask outside your home (e.g. when on public transport, going to a supermarket, going to a main road)'' for each week, thus creating a time series which measures reported mask usage.
% We drop countries with fewer than $50$ observations before or after the intervention.\footnote{Which leave us 14 countries: Australia, France, Hong Kong, India, Italy, Malaysia, Mexico, Netherlands, Saudi Arabia, Singapore, Spain, United Arab Emirates, United Kingdom, United States.}
% Then, we run an ITS analysis by fitting a binomial regression model, which represents the log odds of mask-wearing in each split of the data as a function of it being before or after the intervention, thus creating a piece-wise constant approximation to the original data.
% All the coefficients for the `after-intervention' variable are significant at $p<0.001$ level, apart from India, Hong Kong, and Australia.
% Finally, we cross-tabulate these coefficients with data about the individualism level of each country taken from~\cite{kim1994individualism}.
% \Cref{fig:ind-coeff} shows the relationship between these quantities: as the individualism score increases, the log odds of mask-wearing decreases steadily, with a significant Pearson correlation coefficient of $\rho = -0.409$.\footnote{We obtain similar results by omitting the three non-significant countries from the analysis.}
% This results corroborates the finding that more collectivist groups are more receptive to mask mandates.

% \subsection{Distribution of Promaskers and Antimaskers in the US}
%\bibliography{masks}

\begin{thebibliography}{68}
\providecommand{\natexlab}[1]{#1}
\providecommand{\url}[1]{\texttt{#1}}
\providecommand{\urlprefix}{URL }
\expandafter\ifx\csname urlstyle\endcsname\relax
  \providecommand{\doi}[1]{doi:\discretionary{}{}{}#1}\else
  \providecommand{\doi}{doi:\discretionary{}{}{}\begingroup
  \urlstyle{rm}\Url}\fi

\bibitem[{Abd-Alrazaq et~al.(2020)Abd-Alrazaq, Alhuwail, Househ, Hamdi, and
  Shah}]{abd2020top}
Abd-Alrazaq, A.; Alhuwail, D.; Househ, M.; Hamdi, M.; and Shah, Z. 2020.
\newblock Top concerns of tweeters during the COVID-19 pandemic: infoveillance
  study.
\newblock \emph{Journal of medical Internet research} 22(4): e19016.

\bibitem[{Aggarwal, Rajadesingan, and Kumaraguru(2012)}]{aggarwal2012phishari}
Aggarwal, A.; Rajadesingan, A.; and Kumaraguru, P. 2012.
\newblock PhishAri: Automatic realtime phishing detection on twitter.
\newblock In \emph{2012 eCrime Researchers Summit}, 1--12. IEEE.

\bibitem[{Amin et~al.(2017)Amin, Bednarczyk, Ray, Melchiori, Graham,
  Huntsinger, and Omer}]{Amin2017}
Amin, A.~B.; Bednarczyk, R.~A.; Ray, C.~E.; Melchiori, K.~J.; Graham, J.;
  Huntsinger, J.~R.; and Omer, S.~B. 2017.
\newblock Association of moral values with vaccine hesitancy.
\newblock \emph{Nature Human Behaviour} 1: 873--880.

\bibitem[{Araque, Gatti, and Kalimeri(2020)}]{araque2020moralstrength}
Araque, O.; Gatti, L.; and Kalimeri, K. 2020.
\newblock MoralStrength: Exploiting a moral lexicon and embedding similarity
  for moral foundations prediction.
\newblock \emph{Knowledge-based systems} 191: 105184.

\bibitem[{Awad and Wagoner(2020)}]{awad2020protest}
Awad, S.~H.; and Wagoner, B. 2020.
\newblock Protest symbols.
\newblock \emph{Current Opinion in Psychology} 35: 98--102.

\bibitem[{Bayram and Shields(2021)}]{bayram2021trusts}
Bayram, A.~B.; and Shields, T. 2021.
\newblock Who Trusts the WHO? Heuristics and Americans' Trust in the World
  Health Organization During the COVID-19 Pandemic.
\newblock \emph{Social Science Quarterly} .

\bibitem[{Biddlestone, Green, and Douglas(2020)}]{biddlestone2020cultural}
Biddlestone, M.; Green, R.; and Douglas, K.~M. 2020.
\newblock Cultural orientation, power, belief in conspiracy theories, and
  intentions to reduce the spread of COVID-19.
\newblock \emph{British Journal of Social Psychology} 59(3): 663--673.

\bibitem[{Blei, Ng, and Jordan(2003)}]{blei2003latent}
Blei, D.~M.; Ng, A.~Y.; and Jordan, M.~I. 2003.
\newblock Latent dirichlet allocation.
\newblock \emph{JMLR} 3: 993--1022.

\bibitem[{Blondel et~al.(2008)Blondel, Guillaume, Lambiotte, and
  Lefebvre}]{blondel2008fast}
Blondel, V.~D.; Guillaume, J.-L.; Lambiotte, R.; and Lefebvre, E. 2008.
\newblock Fast unfolding of communities in large networks.
\newblock \emph{Journal of Statistical Mechanics: Theory and Experiment}
  2008(10): P10008.

\bibitem[{Campbell(1965)}]{campbell1965ethnocentric}
Campbell, D.~T. 1965.
\newblock Ethnocentric and other altruistic motives.
\newblock In \emph{Nebraska symposium on motivation}, volume~13, 283--311.

\bibitem[{Cossard et~al.(2020)Cossard, De~Francisci~Morales, Kalimeri, Mejova,
  Paolotti, and Starnini}]{cossard2020falling}
Cossard, A.; De~Francisci~Morales, G.; Kalimeri, K.; Mejova, Y.; Paolotti, D.;
  and Starnini, M. 2020.
\newblock Falling into the Echo Chamber: the Italian Vaccination Debate on
  Twitter.
\newblock In \emph{ICWSM}.

\bibitem[{Cotfas et~al.(2021)Cotfas, Delcea, Gherai, and
  Roxin}]{cotfas2021unmasking}
Cotfas, L.-A.; Delcea, C.; Gherai, R.; and Roxin, I. 2021.
\newblock Unmasking People?s Opinions behind Mask-Wearing during COVID-19
  Pandemic?A Twitter Stance Analysis.
\newblock \emph{Symmetry} 13(11): 1995.

\bibitem[{Curry(2016)}]{curry2016morality}
Curry, O.~S. 2016.
\newblock Morality as cooperation: A problem-centred approach.
\newblock In \emph{The evolution of morality}, 27--51. Springer.

\bibitem[{Day et~al.(2014)Day, Fiske, Downing, and Trail}]{day2014shifting}
Day, M.~V.; Fiske, S.~T.; Downing, E.~L.; and Trail, T.~E. 2014.
\newblock Shifting liberal and conservative attitudes using moral foundations
  theory.
\newblock \emph{Personality and Social Psychology Bulletin} 40(12): 1559--1573.

\bibitem[{des Mesnards et~al.(2022)des Mesnards, Hunter, el~Hjouji, and
  Zaman}]{des2022detecting}
des Mesnards, N.~G.; Hunter, D.~S.; el~Hjouji, Z.; and Zaman, T. 2022.
\newblock Detecting bots and assessing their impact in social networks.
\newblock \emph{Operations Research} 70(1): 1--22.

\bibitem[{Eikenberry et~al.(2020)Eikenberry, Mancuso, Iboi, Phan, Eikenberry,
  Kuang, Kostelich, and Gumel}]{eikenberry2020mask}
Eikenberry, S.~E.; Mancuso, M.; Iboi, E.; Phan, T.; Eikenberry, K.; Kuang, Y.;
  Kostelich, E.; and Gumel, A.~B. 2020.
\newblock To mask or not to mask: Modeling the potential for face mask use by
  the general public to curtail the COVID-19 pandemic.
\newblock \emph{Infectious Disease Modelling} .

\bibitem[{Fulgoni et~al.(2016)Fulgoni, Carpenter, Ungar, and
  Preo{\c{t}}iuc-Pietro}]{fulgoni2016empirical}
Fulgoni, D.; Carpenter, J.; Ungar, L.; and Preo{\c{t}}iuc-Pietro, D. 2016.
\newblock An empirical exploration of moral foundations theory in partisan news
  sources.
\newblock In \emph{Proceedings of the Tenth International Conference on
  Language Resources and Evaluation (LREC'16)}, 3730--3736.

\bibitem[{Garimella et~al.(2016)Garimella, De~Francisci~Morales, Gionis, and
  Mathioudakis}]{garimella2016quantifying}
Garimella, K.; De~Francisci~Morales, G.; Gionis, A.; and Mathioudakis, M. 2016.
\newblock Quantifying {Controversy} in {Social} {Media}.
\newblock In \emph{WSDM}, 33--42.

\bibitem[{Garimella et~al.(2018)Garimella, De~Francisci~Morales, Gionis, and
  Mathioudakis}]{garimella2018quantifying}
Garimella, K.; De~Francisci~Morales, G.; Gionis, A.; and Mathioudakis, M. 2018.
\newblock Quantifying {Controversy} on {Social} {Media}.
\newblock \emph{ACM Transactions on Social Computing} 1(1): 3.

\bibitem[{Garimella et~al.(2021)Garimella, Smith, Weiss, and
  West}]{garimella2021political}
Garimella, K.; Smith, T.; Weiss, R.; and West, R. 2021.
\newblock Political polarization in online news consumption.
\newblock \emph{ICWSM} 152--162.

\bibitem[{Gillespie(2020)}]{gillespie2020mask}
Gillespie, C. 2020.
\newblock Does wearing a face mask reduce oxygen? and can it increase CO2
  levels? Here?s what experts say.
\newblock https://www.health.com/condition/
  infectious-diseases/coronavirus/does-wearing-face-maskincrease-co2-levels.
\newblock Health. (Accessed on April 28, 2021).

\bibitem[{Golbeck and Hansen(2011)}]{golbeck2011computing}
Golbeck, J.; and Hansen, D. 2011.
\newblock Computing political preference among twitter followers.
\newblock In \emph{SIGCHI conference on human factors in computing systems},
  1105--1108.

\bibitem[{Goldberg et~al.(2020)Goldberg, Gustafson, Maibach, Ballew, Bergquist,
  Kotcher, Marlon, Rosenthal, and Leiserowitz}]{goldberg2020mask}
Goldberg, M.~H.; Gustafson, A.; Maibach, E.~W.; Ballew, M.~T.; Bergquist, P.;
  Kotcher, J.~E.; Marlon, J.~R.; Rosenthal, S.~A.; and Leiserowitz, A. 2020.
\newblock Mask-wearing increased after a government recommendation: A natural
  experiment in the US during the COVID-19 pandemic.
\newblock \emph{Frontiers in Communication} 5: 44.

\bibitem[{Graham et~al.(2020)Graham, Cullen, Pickett, Jonson, Haner, and
  Sloan}]{graham2020faith}
Graham, A.; Cullen, F.; Pickett, J.; Jonson, C.; Haner, M.; and Sloan, M. 2020.
\newblock Faith in Trump, Moral Foundations, and Social Distancing Defiance
  During the Coronavirus Pandemic.
\newblock \emph{Moral Foundations, and Social Distancing Defiance During the
  Coronavirus Pandemic (April 22, 2020)} .

\bibitem[{Graham et~al.(2012)Graham, Haidt, Koleva, Motyl, Iyer, Wojcik, and
  Ditto}]{Graham2012}
Graham, J.; Haidt, J.; Koleva, S.; Motyl, M.; Iyer, R.; Wojcik, S.~P.; and
  Ditto, P.~H. 2012.
\newblock Moral foundations theory: The pragmatic validity of moral pluralism.
\newblock \emph{Advances in Experimental Social Psychology, Forthcoming} .

\bibitem[{Graham, Haidt, and Nosek(2009)}]{graham2009liberals}
Graham, J.; Haidt, J.; and Nosek, B.~A. 2009.
\newblock Liberals and conservatives rely on different sets of moral
  foundations.
\newblock \emph{Journal of Personality and Social Psychology} 96(5): 1029.

\bibitem[{Haidt(2012)}]{haidt2012righteous}
Haidt, J. 2012.
\newblock \emph{The righteous mind: Why good people are divided by politics and
  religion}.
\newblock Vintage.

\bibitem[{Hao, Shao, and Huang(2021)}]{hao2021understanding}
Hao, F.; Shao, W.; and Huang, W. 2021.
\newblock Understanding the influence of contextual factors and individual
  social capital on American public mask wearing in response to COVID--19.
\newblock \emph{Health \& Place} 68: 102537.

\bibitem[{Hartsoe(2020)}]{hartsoe2020experts}
Hartsoe, S. 2020.
\newblock Experts urge empathetic approach when talking with anti-maskers.
\newblock
  https://globalhealth.duke.edu/news/experts-urge-empathetic-approach-when-talking-anti-maskers.
\newblock (Accessed on Dec 2, 2021).

\bibitem[{Hatemi, Crabtree, and Smith(2019)}]{hatemi2019ideology}
Hatemi, P.~K.; Crabtree, C.; and Smith, K.~B. 2019.
\newblock Ideology justifies morality: Political beliefs predict moral
  foundations.
\newblock \emph{American Journal of Political Science} 63(4): 788--806.

\bibitem[{He et~al.(2021)He, He, Reynolds, Bai, Huang, Li, Zheng, and
  Chen}]{he2021people}
He, L.; He, C.; Reynolds, T.~L.; Bai, Q.; Huang, Y.; Li, C.; Zheng, K.; and
  Chen, Y. 2021.
\newblock Why do people oppose mask wearing? A comprehensive analysis of US
  tweets during the COVID-19 pandemic.
\newblock \emph{Journal of the American Medical Informatics Association} 28(7):
  1564--1573.

\bibitem[{Hofstede(1984)}]{hofstede1984culture}
Hofstede, G. 1984.
\newblock \emph{Culture's consequences: International differences in
  work-related values}, volume~5.
\newblock Sage publications.

\bibitem[{Hofstede(2001)}]{hofstede2001culture}
Hofstede, G. 2001.
\newblock \emph{Culture's consequences: Comparing values, behaviors,
  institutions and organizations across nations}.
\newblock Sage publications.

\bibitem[{Hornik et~al.(2021)Hornik, Kikut, Jesch, Woko, Siegel, and
  Kim}]{hornik2021association}
Hornik, R.; Kikut, A.; Jesch, E.; Woko, C.; Siegel, L.; and Kim, K. 2021.
\newblock Association of COVID-19 misinformation with face mask wearing and
  social distancing in a nationally representative US sample.
\newblock \emph{Health communication} 36(1): 6--14.

\bibitem[{Iyer et~al.(2012)Iyer, Koleva, Graham, Ditto, and
  Haidt}]{iyer2012understanding}
Iyer, R.; Koleva, S.; Graham, J.; Ditto, P.; and Haidt, J. 2012.
\newblock Understanding libertarian morality: The psychological dispositions of
  self-identified libertarians.
\newblock \emph{PloS one} 7(8): e42366.

\bibitem[{Kahane(2021)}]{kahane2021politicizing}
Kahane, L.~H. 2021.
\newblock Politicizing the mask: Political, economic and demographic factors
  affecting mask wearing behavior in the usa.
\newblock \emph{Eastern economic journal} 47(2): 163--183.

\bibitem[{Kalimeri et~al.(2019)Kalimeri, G.~Beir{\'o}, Urbinati, Bonanomi,
  Rosina, and Cattuto}]{kalimeri2019human}
Kalimeri, K.; G.~Beir{\'o}, M.; Urbinati, A.; Bonanomi, A.; Rosina, A.; and
  Cattuto, C. 2019.
\newblock Human values and attitudes towards vaccination in social media.
\newblock In \emph{Companion Proceedings of the World Wide Web Conference},
  248--254.

\bibitem[{Karandikar et~al.(2019)Karandikar, Kapoor, Fernandes, and
  Jonason}]{karandikar2019predicting}
Karandikar, S.; Kapoor, H.; Fernandes, S.; and Jonason, P.~K. 2019.
\newblock Predicting moral decision-making with dark personalities and moral
  values.
\newblock \emph{Personality and Individual Differences} 140: 70--75.

\bibitem[{Karypis and Kumar(1998)}]{karypis1998fast}
Karypis, G.; and Kumar, V. 1998.
\newblock A fast and high quality multilevel scheme for partitioning irregular
  graphs.
\newblock \emph{SIAM Journal on scientific Computing} 20(1): 359--392.

\bibitem[{Katz, Sanger-Katz, and Quealy(2020)}]{katz2020map}
Katz, J.; Sanger-Katz, M.; and Quealy, K. 2020.
\newblock A Detailed Map of Who Is Wearing Masks in the US.
\newblock https://www.nytimes.com/interactive/2020/07/17/upshot/
  coronavirus-face-mask-map.html.
\newblock (Accessed on Dec 1, 2021).

\bibitem[{Keller, Honea, and Ollivant(2021)}]{keller2021social}
Keller, S.~N.; Honea, J.~C.; and Ollivant, R. 2021.
\newblock How Social Media Comments Inform the Promotion of Mask-Wearing and
  Other COVID-19 Prevention Strategies.
\newblock \emph{Int. J. of Environmental Research and Public Health} 18(11):
  5624.

\bibitem[{Khan(2020)}]{khan2020mask}
Khan, A. 2020.
\newblock ?Wearing a mask can give a false sense of security?: What the
  experts say about wearing face masks.
\newblock https://news.yahoo.com/facemask-proper-use-necessary221717984.html.
\newblock (Yahoo! News, 30 March. Accessed on April 28, 2021).

\bibitem[{Kim et~al.(1994)Kim, Triandis, K{\^a}{\u{g}}it{\c{c}}iba{\c{s}}i,
  Choi, and Yoon}]{kim1994individualism}
Kim, U.~E.; Triandis, H.~C.; K{\^a}{\u{g}}it{\c{c}}iba{\c{s}}i, {\c{C}}.~E.;
  Choi, S.-C.~E.; and Yoon, G.~E. 1994.
\newblock \emph{Individualism and collectivism: Theory, method, and
  applications.}
\newblock Sage Publications, Inc.

\bibitem[{Kivikangas et~al.(2021)Kivikangas, Fern{\'a}ndez-Castilla,
  J{\"a}rvel{\"a}, Ravaja, and L{\"o}nnqvist}]{kivikangas2021moral}
Kivikangas, J.~M.; Fern{\'a}ndez-Castilla, B.; J{\"a}rvel{\"a}, S.; Ravaja, N.;
  and L{\"o}nnqvist, J.-E. 2021.
\newblock Moral foundations and political orientation: Systematic review and
  meta-analysis.
\newblock \emph{Psychological Bulletin} 147(1): 55.

\bibitem[{Krishnamachari et~al.(2021)Krishnamachari, Morris, Zastrow, Dsida,
  Harper, and Santella}]{krishnamachari2021role}
Krishnamachari, B.; Morris, A.; Zastrow, D.; Dsida, A.; Harper, B.; and
  Santella, A.~J. 2021.
\newblock The role of mask mandates, stay at home orders and school closure in
  curbing the COVID-19 pandemic prior to vaccination.
\newblock \emph{American journal of infection control} .

\bibitem[{Kugler, Jost, and Noorbaloochi(2014)}]{kugler2014another}
Kugler, M.; Jost, J.~T.; and Noorbaloochi, S. 2014.
\newblock Another look at moral foundations theory: Do authoritarianism and
  social dominance orientation explain liberal-conservative differences in
  ?moral? intuitions?
\newblock \emph{Social Justice Research} 27(4): 413--431.

\bibitem[{Landsverk(2020)}]{landsverk2020cdc}
Landsverk, G. 2020.
\newblock The CDC says healthcare workers who can't get a mask should use a
  bandana or scarf as a 'last resort' as supplies run short.
\newblock
  https://www.businessinsider.com/cdc-recommends-health-workers-use-bandanas-face-masks-crisis-2020-3.
\newblock (Accessed on April 28, 2021).

\bibitem[{Lang, Erickson, and Jing-Schmidt(2021)}]{lang2021maskon}
Lang, J.; Erickson, W.~W.; and Jing-Schmidt, Z. 2021.
\newblock \# MaskOn!\# MaskOff! Digital polarization of mask-wearing in the
  United States during COVID-19.
\newblock \emph{PloS one} 16(4): e0250817.

\bibitem[{Maaravi et~al.(2021)Maaravi, Levy, Gur, Confino, and
  Segal}]{maaravi2021tragedy}
Maaravi, Y.; Levy, A.; Gur, T.; Confino, D.; and Segal, S. 2021.
\newblock ``The tragedy of the commons'': How individualism and collectivism
  affected the spread of the COVID-19 pandemic.
\newblock \emph{Frontiers in Public Health} 9: 37.

\bibitem[{McGarry et~al.(2019)McGarry, Erhart, Eslen-Ziya, Jenzen, and
  Korkut}]{mcgarry2019aesthetics}
McGarry, A.; Erhart, I.; Eslen-Ziya, H.; Jenzen, O.; and Korkut, U. 2019.
\newblock \emph{The aesthetics of global protest: Visual culture and
  communication}.
\newblock Amsterdam University Press.

\bibitem[{Miller and Morrison(2009)}]{miller2009expressing}
Miller, D.~T.; and Morrison, K.~R. 2009.
\newblock Expressing deviant opinions: Believing you are in the majority helps.
\newblock \emph{Journal of Experimental Social Psychology} 45(4): 740--747.

\bibitem[{Mottl(2021)}]{getoldtweets}
Mottl, D. 2021.
\newblock GetOldTweets3.
\newblock https://pypi.org/project/ GetOldTweets3/.
\newblock (Accessed on April 1, 2021).

\bibitem[{Mustafaraj et~al.(2011)Mustafaraj, Finn, Whitlock, and
  Metaxas}]{Mustafaraj2011}
Mustafaraj, E.; Finn, S.; Whitlock, C.; and Metaxas, P.~T. 2011.
\newblock {Vocal Minority versus Silent Majority: Discovering the Opinions of
  the Long Tail}.
\newblock In \emph{International Conference on Social Computing}, 103--110.
  IEEE.

\bibitem[{Pascual-Ferr{\'a} et~al.(2021)Pascual-Ferr{\'a}, Alperstein, Barnett,
  and Rimal}]{pascual2021toxicity}
Pascual-Ferr{\'a}, P.; Alperstein, N.; Barnett, D.~J.; and Rimal, R.~N. 2021.
\newblock Toxicity and verbal aggression on social media: Polarized discourse
  on wearing face masks during the COVID-19 pandemic.
\newblock \emph{Big Data \& Society} 8(1).

\bibitem[{Patwa et~al.(2021)Patwa, Sharma, Pykl, Guptha, Kumari, Akhtar, Ekbal,
  Das, and Chakraborty}]{patwa2021fighting}
Patwa, P.; Sharma, S.; Pykl, S.; Guptha, V.; Kumari, G.; Akhtar, M.~S.; Ekbal,
  A.; Das, A.; and Chakraborty, T. 2021.
\newblock Fighting an Infodemic: COVID-19 Fake News Dataset.
\newblock In Chakraborty, T.; Shu, K.; Bernard, H.~R.; Liu, H.; and Akhtar,
  M.~S., eds., \emph{Combating Online Hostile Posts in Regional Languages
  during Emergency Situation}, 21--29. Cham: Springer International Publishing.
\newblock ISBN 978-3-030-73696-5.

\bibitem[{Powdthavee et~al.(2021)Powdthavee, Riyanto, Wong, Yeo, and
  Chan}]{powdthavee2021face}
Powdthavee, N.; Riyanto, Y.~E.; Wong, E.~C.; Yeo, J.~X.; and Chan, Q.~Y. 2021.
\newblock When face masks signal social identity: Explaining the deep face-mask
  divide during the COVID-19 pandemic.
\newblock \emph{PloS one} 16(6): e0253195.

\bibitem[{Qiang et~al.(2020)Qiang, Qian, Li, Yuan, and Wu}]{qiang2020short}
Qiang, J.; Qian, Z.; Li, Y.; Yuan, Y.; and Wu, X. 2020.
\newblock Short text topic modeling techniques, applications, and performance:
  a survey.
\newblock \emph{IEEE Transactions on Knowledge and Data Engineering} .

\bibitem[{R{\"o}der, Both, and Hinneburg(2015)}]{roder2015exploring}
R{\"o}der, M.; Both, A.; and Hinneburg, A. 2015.
\newblock Exploring the space of topic coherence measures.
\newblock In \emph{WSDM}, 399--408.

\bibitem[{Samuel et~al.(2016)}]{samuel2016authoritariansim}
Samuel, J.; et~al. 2016.
\newblock Authoritariansim and Collectivism: Antecedents and Consequences Among
  College Students.
\newblock \emph{Honors Undergraduate Theses} (12).

\bibitem[{Sanders et~al.(2021)Sanders, White, Severson, Ma, McQueen, Paulo,
  Zhang, Erickson, and Bennett}]{sanders2021unmasking}
Sanders, A.~C.; White, R.~C.; Severson, L.~S.; Ma, R.; McQueen, R.; Paulo, H.
  C.~A.; Zhang, Y.; Erickson, J.~S.; and Bennett, K.~P. 2021.
\newblock Unmasking the conversation on masks: Natural language processing for
  topical sentiment analysis of COVID-19 Twitter discourse.
\newblock In \emph{AMIA Annual Symposium Proceedings}, volume 2021, 555.
  American Medical Informatics Association.

\bibitem[{Shapiro and Hemphill(2017)}]{shapiro2017politicians}
Shapiro, M.~A.; and Hemphill, L. 2017.
\newblock Politicians and the policy agenda: Does use of Twitter by the US
  Congress direct New York Times content?
\newblock \emph{Policy \& internet} 9(1): 109--132.

\bibitem[{Tajfel et~al.(1979)Tajfel, Turner, Austin, and
  Worchel}]{tajfel1979integrative}
Tajfel, H.; Turner, J.~C.; Austin, W.~G.; and Worchel, S. 1979.
\newblock An integrative theory of intergroup conflict.
\newblock \emph{Organizational identity: A reader} 56(65): 9780203505984--16.

\bibitem[{Twenge, Campbell, and Gentile(2013)}]{twenge2013changes}
Twenge, J.~M.; Campbell, W.~K.; and Gentile, B. 2013.
\newblock Changes in pronoun use in American books and the rise of
  individualism, 1960-2008.
\newblock \emph{Journal of cross-cultural psychology} 44(3): 406--415.

\bibitem[{Van~Bavel and Pereira(2018)}]{van2018partisan}
Van~Bavel, J.~J.; and Pereira, A. 2018.
\newblock The partisan brain: An identity-based model of political belief.
\newblock \emph{Trends in cognitive sciences} 22(3): 213--224.

\bibitem[{Van~Mierlo(2014)}]{van20141onepercent}
Van~Mierlo, T. 2014.
\newblock The 1\% rule in four digital health social networks: an observational
  study.
\newblock \emph{J Med Internet Res} 16(2): e2966.

\bibitem[{Vandello and Cohen(1999)}]{vandello1999patterns}
Vandello, J.~A.; and Cohen, D. 1999.
\newblock Patterns of individualism and collectivism across the United States.
\newblock \emph{Journal of personality and social psychology} 77(2): 279.

\bibitem[{Weber et~al.(2015)Weber, Eden, Huskey, Mangus, and Falk}]{Weber2015}
Weber, R.; Eden, A.; Huskey, R.; Mangus, J.~M.; and Falk, E. 2015.
\newblock Bridging Media Psychology and Cognitive Neuroscience.
\newblock \emph{Journal of Media Psychology} .

\bibitem[{{WHO}(2020)}]{who2020advice}
{WHO}. 2020.
\newblock Coronavirus disease (COVID-19) advice for the public: When and how to
  use masks.
\newblock https://archive.vn/cwkJ3.
\newblock (Archived on March 31, 2020, Accessed on April 28, 2021).

\end{thebibliography}

\newpage
\end{document}
\endinput
%%
%% End of file `sample-acmtog.tex'.
