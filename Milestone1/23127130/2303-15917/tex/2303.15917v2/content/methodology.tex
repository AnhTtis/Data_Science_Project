\section{Methodology}
\label{sec:methodology}

Based on the proposed concept and prototype implementation, we conducted a controlled experiment to investigate the influence of motion synchronization on participants' trust formation toward an \simNonHumRobot{}. More specifically, we investigated the following research question:

%\todo{embodied AI system - replace everywhere with prototype / machine / simple non-humanoid robot}

\begin{description}
	\item[RQ] Does the synchronized movement of the robot change the feeling of trust toward an \simNonHumRobot{}?
\end{description}

%We differentiated the movement pattern of the robot between participants choosing from two unsynchronized and one synchronized movement. We decided on task-free interaction because, in a task-oriented setup, you can measure confidence in the machine (cognitive dimension, e.g., reliability, efficiency), but you can't measure the relationship with the machine (affective dimension, e.g., feelings, emotions). Moreover, in task-oriented, it would be difficult to manipulate the synchronization so that it does not affect the quality of the execution of the task.

%Trust toward robots was mainly measured using questionnaires \cite{malle2021multidimensional}. However, also an experimental measure, the trust game, was used to measure trust \citet{oksanen2020trust}. We decided to use both measures. 

%To summarize, first, the participant freely explored the movement with the robot, then filled out the questionnaire and finally had the option to play in the trust game. 

In the following section, we report on the methodology of our experiment.

\subsection{Design and Task}
\label{sec:methodology:design}

To answer the research questions, we conducted a controlled experiment in which participants interacted with an \simNonHumRobot{}, as described in section \ref{sec:prototype}. We varied the movement pattern of the robot in a between-subjects design with three levels (two unsynchronized and one synchronized movement pattern). We explained to the participants that we were working on a prototype for human-machine interaction. To avoid biasing the participants, we did not give any further information about the purpose and goal of the prototype. We chose not to give our participants any specific tasks to perform or goals to achieve. Instead, we asked them to interact with the robot in a natural and spontaneous way without any constraints or instructions. For this, participants were allowed to move around the room at will for 3 minutes (the pilot studies showed that interest in the task lasted for about 3 minutes). We opted for this approach to assess the effects of synchronization on trust in a more natural and realistic setting without the potential confounds that would arise from using a specific task or context. More specifically, a task-oriented scenario would allow measuring the confidence in the machine (cognitive dimension, e.g., reliability, efficiency) but would not allow measuring the relationship with the machine (affective dimension, e.g., feelings, emotions). Moreover, a task-oriented scenario would make it difficult to manipulate the synchronization so that it does not affect the quality of the execution of the task. 

After this, we asked participants to fill out a questionnaire and handed them their compensation of about \trustGameMoney{} (in local currency) as five coins. As the last step, we gave our participants the option to optionally play a version of the trust game \cite{BERG1995122} with the prototype: Participants could deposit any portion of the coins they received (from 0 to 5 coins) into the prototype (if someone refused to participate in the game, we counted it as depositing 0 coins). We informed the participants that the prototype would be credited with the tripled amount of their deposit and subsequently, at will, would pay a share (i.e., between 0\% and 300\% of the inserted money) of this sum back.

Following the results of our literature review and informal pretests, we expected that the type of movement performed by the prototype during the interaction would affect the participants' sense of trust in interacting with the \simNonHumRobot{}. Therefore, we varied the \factorMove{} as an independent variable with three levels, namely:

\begin{description}
	\item[\factorMoveLvlSynchronized{}] as a prototype with movements synchronized to the participant's movements. The participant's movements are mimicked to achieve a specific form of synchronization - delayed in time and possibly spatially transformed. So participant movements are transformed into movements of the prototype using a formula for the robot's kinematics (see fig. \ref{fig:insync:prototype:math}) with added movement delay by limiting maximum speed and acceleration as described in section \ref{sec:prototype:construction} and adding a rotational transformation (by 10 degrees) of the bending direction \(\theta\).
	\item[\factorMoveLvlRandom{}] as a prototype with random movements using Brownian motion \cite{hida1980brownian} to program it. We chose the parameters of the Brownian motion in a way that the amplitude and frequency of the motions were comparable to the motions emitted by the prototype in synchronization mode.
	\item[\factorMoveLvlSimple{}] as a simple, recurring pattern of movement. We program it as sinusoidal, waving sideways in one plane. 
\end{description}

We varied the independent variable in a between-subjects design by assigning each participant to one of the three conditions. For each condition, we measured the following dependent variables:

\begin{description}
	\item[\dvTPA{}] To further gain insight into the trust relationship between the prototype and the participant, we employed the widely used \ac{TPA} checklist as proposed by \citet{Jian2000}. The \ac{TPA} consists of twelve items with 7-point Likert scales each. It measures trust and distrust as polar opposites along a single dimension. Therefore, the output may be a single all-encompassing trust value or separate values for the trust and distrust dimensions \citet{kohn2021measurement}.
	\item[\dvMoney{}] Similar to previous work in assessing trust in robots \cite{mota2016playing,ALARCON2023103858,oksanen2020trust,zorner2021immersive}, we used the amount of money staked in the trust game \cite{BERG1995122} as a measure of the trust participants placed in the prototype.
%	\item[\dvMovement{}] In addition to the measures quantifying trust, we recorded the participant's spatial movement and orientation during the exploration phase. We did this to uncover correlations between the user's movements - and the system's movements stimulated by them - and the trust placed in the system.
\end{description}

Based on the study design, we obtained ethics approval from our institution before the experiment, which had no objections. 


%we need a task that

%- stimulates movement
%- has the AI as a potential helpful item
%- 


%- trust game? Has no movement. 
%- we need something that has a collaborative aspect (AI as partner)

%DV:
%- System Trust Scale Jian et al.
%- 

\subsection{Study Setup and Apparatus}

For the experiment, we used the prototype as described in section \ref{sec:prototype}. We modified the prototype to support the trust game as described above. For this, we built a typical coin acceptor into the prototype, which, as expected, turned out to be a good affordance for the action of inserting coins into it. The coin acceptor also counted the number of inserted coins. The operation of the coin acceptor was coordinated with the operation of the coin returner made by us, which, 10 seconds after the participant had thrown in the last coin, started the process of returning an adequate number of coins by the prototype (see fig. \ref{fig:insync:prototype}D-E).

Further, we used a data recorder placed on the technical table in a far-away corner of the room (see fig. \ref{fig:insync:setup:plan}). We developed a data recorder device to ensure the quality of data collected on paper by the investigator. We recorded the following data: movement tracking data, number of inserted coins, research condition number, timestamp, and time of every experiment stage. For the construction, we used the same chip ESP32 (see section \ref{sec:prototype}) with added microSD card driver. For communication with the prototype, we used the ESP-NOW communication protocol (see section \ref{sec:prototype}).

Besides this, we also used Sony digital camera model no. DSC-HX5V for video recording captures general situation and participant movements for further analysis (see fig. \ref{fig:insync:methodology:studysetup}). In the same room, we also prepared a table for filing a consent form before the study and a questionnaire and compensation confirmation form after the study. We decided to have this table in the same room because after giving compensation, the moderator proposes a trust game with the prototype, so we want to have the prototype in close range all the time (see fig. \ref{fig:insync:setup:plan}).

We decided to put the prototype in the corner of the room, with the front rotated 45 degrees to the wall. This arrangement emphasizes the position of the front of the prototype while allowing the participant to walk around the prototype freely. We left space of 120 cm between the sides of the prototype and the side walls to make it possible. The initial position of the participant for each study is the same, i.e., 130 cm from the front of the prototype. See fig. \ref{fig:insync:setup:plan}, where the participant’s starting position is marked with a circle with a dashed brown line.

\begin{figure*}[t!]
 \centering
 \begin{subfigure}[b]{0.49\textwidth}
    \centering
    \includegraphics[width=\textwidth]{img/blobs/PAPER_study_pic.pdf}
    \caption{}
    \label{fig:insync:setup:foto}
\end{subfigure}
\hfill
\begin{subfigure}[b]{0.49\textwidth}
    \centering
    \includegraphics[width=\textwidth]{img/blobs/PAPER_study_setup.pdf}
    \caption{}
    \label{fig:insync:setup:plan}
\end{subfigure}
\caption{The study setup details. (a) Two frames from a video camera showing participants during the free exploration phase. (b) Floor plan of the study setup with labels and dimensions in centimeters.}
\label{fig:insync:methodology:studysetup}
\Description{A two-part figure of the study setup details. (a) Two frames from a video camera showing participants during the free exploration phase. (b) Floor plan of the study setup with labels and dimensions in centimeters.}
\end{figure*}%\todo{DONE! to we have a photo of this also? We could have it side by side, then this does not capture such a big space.}


%We used three custom designed and developed devices, based on Espressif Systems ESP32 chips with Xtensa® 32-bit LX6 microprocessors: (1) Prototype as described in section \ref{sec:prototype} to interact with participant, (2) IMU sensor attached to participant back, and (3) data recorder placed on technical table in a far away corner of the room. Beside this we also used Sony digital camera model no. DSC-HX5V for video recording to capture general situation and participant movements for further analysis (see fig. \ref{fig:insync:prototype:mathpic}). 

\subsection{Procedure}

%\todo{DONE! correct some details}

After welcoming the participants, we introduced them to the general topic of the experiment and led them to the room with the prototype and other stuff (see fig. \ref{fig:insync:methodology:studysetup}). Then, we asked them to fill out a consent form. After that, we read the following instructions: \emph{\enquote{The aim of the research is to help create machines controlled by artificial intelligence and other physical systems cooperating with humans. Here you can see installations controlled by artificial intelligence. Your task is to check how the installation reacts to your movements. In a moment, I will leave you with the installation so that you can freely explore its possibilities. You have 3 minutes to do so. Please do not touch the installation. After 3 minutes, I will come back and ask you to fill in the questionnaire in which you will evaluate the installation}}. We further instructed the participants that we would measure their physiological body signals and informed them that we would need to attach a recording device to their bodies for this purpose. In fact, we attached the orientation sensor device (see fig. \ref{fig:insync:prototype}J) as described in the previous section. We did not give the participants further instructions about their actions and let them freely explore the prototype in their chosen way. The investigator asked the participant to stand approximately 1.3 meters in front of the prototype (see fig. \ref{fig:insync:setup:plan}) and started video recording. Then investigator started the exploratory experiment phase and left the room, leaving the participant alone with the prototype.

After participants had 3 minutes to explore the prototype and its responses, the investigator returned to the room and stopped the prototype. Then, the investigator asked the participant to sit at the form-filling table (see fig. \ref{fig:insync:setup:plan}) and complete the \ac{TPA} questionnaire and then the demographic data. Subsequently, we continued with the trust game. We handed out the compensation of roughly \trustGameMoney{} as five coins in local currency to the participants. We informed them about the trust game by reading the following instructions: \emph{``Here is your compensation for participating in the study. You have the option to play with the installation. You can give the installation any part of the compensation, and the installation will decide whether to multiply your compensation. It can up to triple the given amount, but it can also choose to keep the entire amount''}. And told them that their participation was completely voluntary. If the participant decided to participate in the game, the investigator opened the coin acceptor cover (see fig. \ref{fig:insync:prototype}B-D) and read the following message: \emph{``The installation accepts coins here. After inserting the coins, please wait 10 seconds for the installation's decision''}.  Participants had as much time as they wanted to decide how much of their compensation they wanted to wager. The investigator stayed in the room for potential assistance. The prototype every time paid back one more coin than the participant inserted. If the participant did not put in any coin or decided not to play, the researcher equalized the participant’s compensation so that each participant ended the study with the same compensation. Finally, the investigator was open to collecting voluntary feedback or comments from participants to use them as inspiration for future work or improvements.

\subsection{Hygiene Measures}

All participants and the investigator were vaccinated against COVID-19 and tested negative using an antigen test on the same day. We ensured that only the investigator and the participant were present in the room. Both the investigator and the participants wore medical face masks throughout the experiment. We disinfected the experimental setup between participants, and all surfaces touched and ventilated the room for 30 minutes. 

\subsection{Analysis}

For the non-parametric analysis of the recorded data, we used the Kruskal–Wallis 1-way analysis of variance with Dunn's tests for multiple comparisons for post-hoc comparisons, correct with Bonferroni's method. We further report the eta-squared \etasquared{} as an estimate of the effect size, classified using Cohen's suggestions as small ($>.0099$), medium ($>.0588$), or large ($>.1379$)~\cite{Cohen1988}. For the analysis of the \dvMoney{}, we employed Shapiro-Wilk’s test and Bartlett's test to check the data for violations of the assumptions of normality and homogeneity of variances, respectively. As the test indicated that the assumption of normality was violated, we continued with a non-parametric analysis as described above.

\subsection{Participants}

We recruited a total of 51 participants (29 identified as female, 22 as male) from our university's mailing list. The participants were aged between 19 and 51 ($\mu = 23.5$, $\sigma = 5.4$). We divided the participants into the three experimental conditions in such a way that they were roughly equally distributed with respect to age and gender, resulting in 17 participants per condition. The participants received around \trustGameMoney{} in the local currency as compensation, which they could use in the trust game as part of the study.