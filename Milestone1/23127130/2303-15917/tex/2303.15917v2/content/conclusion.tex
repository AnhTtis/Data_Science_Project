\section{Conclusion}
\label{sec:conclusion}

\balance

In this paper, we explored movement synchronization as a dynamical approach that can be applied in the design of \simNonHumRobots{} to increase trust. We contributed by design and implementation of a prototype system and the results of a controlled experiment with 51 participants exploring our concept in a between-subjects design. We found significantly higher ratings on trust between people and automation in an established questionnaire. However, we could not find an influence on the willingness to spend money in a trust game. Taken together, our results strongly suggest a positive effect of synchronized movement on the participants' feeling of trust toward an \simNonHumRobot{}. The presented result is also important for designers because it points to dynamics as a design strategy and shows synchronization as a tool for designing trustworthy \simNonHumRobots{}.