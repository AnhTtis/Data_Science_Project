\section{Discussion}
\label{sec:discussion}

The results of our controlled experiment suggest that movement synchronization can positively influence the feeling of trust of humans toward an \simNonHumRobot{}. In the following section, we discuss our results with regard to the research questions presented above.


% tpyical structure of each subsection in the discussion:

% 1. What? -> a short summary of what we found
% 2. Why? -> why did the results come out as they did? How can we explain what we saw (based on RW and own interpretation from what we saw).
% 3. So What? What do these results mean? What follows from this?

\subsection{Synchronized Movement Increases Perceived Trust}

We found that \factorMoveLvlSynchronized{} movements resulted in significantly higher trust ratings on the widely used \acl{TPA} questionnaire compared to both other types of movement, \factorMoveLvlRandom{} (Brownian-like) and \factorMoveLvlSimple{} (sinusoidal waving sideways in one plane). We could not find any significant differences between \factorMoveLvlSimple{} and \factorMoveLvlRandom{} conditions. We attribute similar performance to them belonging to the same underlying category, i.e., unsynchronized movements. To gain a deeper understanding, we analyzed how individual subscales of the \dvTPA{} questionnaire contributed to the result. We found significantly higher ratings for \factorMoveLvlSynchronized{} movements on the perceived dependability, integrity, and familiarity, as well as lower ratings for the wariness and expected underhanded behavior of the prototype. We hypothesize that this is caused by participants feeling bonded with the robot through synchronisation and perceiving a smoother interaction~\cite{stel2010mimicry}.  


Our results demonstrate a distinct influence of the motion patterns on the perceived trust toward the prototype. We collected these results using an abstract prototype to rule out the influence of a possible anthropomorphization, which has been suggested in previous work \cite{Salem2015, Langer2019, deVisser2017}. Thus, we reject the null hypothesis (i.e., the absence of an influence of synchronized motion on perceived trust). These results are in line with previous work investigating the influence of motion synchronization between people on perceived trust (see section \ref{sec:relatedwork}).

Therefore, in this paper, we have demonstrated a new strategy based on synchronization to establish a trusting relationship between humans and robots. In contrast to solutions proposed by previous work \cite{calvo2020effects, naneva2020systematic, salem2015would, christoforakos2021can}, our approach does not depend on the robot's external form or complex behavior. It is thus also suitable for \simNonHumRobots{}. We consider our research to present a significant contribution as, to the best of our knowledge, there is no prior work exploring synchronization as a way to increase trust between humans and \simNonHumRobots{}.

%From prior works on the effect of synchronization in humans, we know that synchronization promotes a feeling of liking and trust. In this study, we focused only on the robot movement dynamics, leaving aside other properties of robot well researched in related works. Taking into account the presented related works and the statistically significant result of the analysis of \ac{TPA} in this study, we find that our hypothesis is confirmed. We consider our research a significant contribution as, to the best of our knowledge, there is no prior work exploring synchronization as a way to increase trust between humans and simple non-humanoid robots.

%Based on this findings and related work on synchronization or trust in humanoid robots we hypothesised that to achieve significant trust increase is enough to have simple non-humanoid robot form not capable of reasoning nor sophisticated pattern of behavior. In this study we also wanted to focus only on the robot movement dynamics, leaving aside other properties of robot well researched in related works. Taking into account the presented related works and the statistically significant result of the analysis of \ac{TPA} in this study, we find that our hypothesis is confirmed. We consider our research a significant contribution as, to the best of our knowledge, there is no prior work exploring synchronization as a way to increase trust between humans and simple non-humanoid robots.



%However, we could not find a significant effect on the reliability. We explain this missing effect by the differences between affective and congitive trust as proposed by typ et al.

%In related works we can find similar result, for more sophisticated robots, explained by divide between affective and cognitive trust. We also find that synchronization have the most significant effect on subscale connected to familiarity. It can be interpreted that synchronization makes people feel more familiar with the system. This might be caused by them knowing the movements and understanding what the system will probably do next. We also look on the level of individual subscales for significant differences between research conditions. We confirmed significant differences between the \factorMoveLvlSynchronized{} and \factorMoveLvlRandom{} conditions for subscale no: 2, 8, 9, 11, 12 and between the \factorMoveLvlSynchronized{} and \factorMoveLvlSimple{} for subscale no: 8, 12. We have found that the effect of synchronized movement is more similar to that of simple movement than that of random movement. A likely explanation is that both simple movement and synchronized movement are more predictable than random movement. 


%Thus we show that trust in a non-humanoid robot as in a humanoid robot can be more complex that is, not only refer to factors such as, for example, reliability. It can extend to affective trust through the use of synchronization. In a broader aspect, we show the importance of robot dynamics for building trust, which is possible thanks to dynamics even with the limitations of the complexity of the robot's form or the spectrum of possible complex behaviors.

%We believe that the presented result is also important for designers. We pointing on dynamics as a design strategy for simple non-humanoid robots and showing synchronization as a tool for designer. Thus, we shows how to go beyond the reliability or skills of the robot when designing a trustworthy, simple non-humanoid robot.



%\todo[inline]{-> we found that synchronized movement resulted in significantly higher trust ratings compared to both other types of movement in the questionnaire.
%	-> talk about the individual subscales here were we found singificant result (integrity, dependability, familarity...) 
%	-> this supports our hypothesis and matches the results from RW for higher trust through synchronized movement between humans
%	-> this means that leveraging synchronization to establish a trust relationship between humans and machines is a viable path.}

\subsection{Synchronized Movement Does Not Affect the Willingness to Spend Money in Our Version of the Trust Game}

We found the highest number of coins inserted for the \factorMoveLvlSynchronized{} condition. Still, we could not find a significant influence of the \factorMove{} of the prototype on the number of coins inserted. Also, we found no significant correlations between the \dvTPA{} and the \dvMoney{}.

We attribute this lack of a significant effect on the \dvMoney{}  to several factors. First, our observations and informal interviews showed that people treated the game as gambling. We assume that this is explained by the fact that the amount of compensation for participation was not high enough. Thus, the risk was low, so participants did not differentiate the amounts they wagered on the game. This is in line with findings from prior work indicating that people's averse to gambling can influence the results of the trust game~\cite{chetty2021trust}.

Second, we speculate that the time between interacting with the prototype and the trust game was too long because participants had to complete the \dvTPA{} in between. During the research procedure design process, we considered making the trust game a part of the interaction phase. We dismissed this idea later, however, as the gambling results would potentially override the subtle effect of synchronization on trust and influence the result of \dvTPA{}. As a result, participants would assess not so much trust as satisfaction with the number of coins returned by the robot. Further, from our observation, we conclude that the participants were surprised by the proposal to insert coins into the prototype. The situation was strange and incomprehensible to them. Finally, the investigator was present in the room, so we may have measured confidence toward the investigator rather than the prototype.

In future studies, we recommend leaving the participant alone with the prototype during this study phase. We also recommend conducting the trust game directly after interacting with the prototype or making it a part of the free interaction phase if the trust game is the only measure.

%\todo[inline]{-> we could not find an effect on the trust game. 
%-> why was that? Refer to RW on the trust game and what worked/did not work for them. Also our speculations here.
%-> What does it mean? What follows for future interaction? And how do we explain the difference between this result and the questionnaire result?}
