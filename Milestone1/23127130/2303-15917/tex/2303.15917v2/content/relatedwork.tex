\section{Related Work}
\label{sec:relatedwork}

Our work was heavily influenced by a large body of prior works in the areas of synchronization and trust in human-robot interaction and synchronization and trust in humans.

\subsection{Trust in Human-Robot Interaction}

% general introduction of the term trust in the area of human/computer interaction
% What influences trust? How is it measured?
% What is the difference between disembodied systems and robots/embodied systems?
% What did RW do to increase trust? 
% Why is the stuff RW did not enough to solve all issues? What problems remain (-> non-humanoid robots, …)

Trust is the primary factor shaping social interactions~\cite{doi:10.1080/21515581.2012.708496}, from intimate relations~\cite{Rosenthal1981}, through work relations~\cite{Brown2015} and consumer behavior~\cite{Ding2013} to conflict~\cite{Malhotra2011}. Most definitions of trust rely on the notion that trust describes a relation between a trustor (subject) and a trustee (object) that are interdependent in the sense that the action of one has some consequences for the other in a situation containing risks for the trustor \cite{pytlikzillig2016consensus}. Further, trust can be understood as a form of reliance that is based on the judgment that the partner has the relevant competence, motivation, and opportunity~\cite{de2021defining}.

Following the pivotal role of trust in interpersonal relationships, trust has also emerged as an increasingly relevant area of research in human-robot interaction~\cite{Hancock2011,gompei2018factors,hancock2021evolving,baker2018toward,plaks2022identifying}. In this field, many studies explored the antecedents of trust toward robots. They found that trust toward robots depends on human personality factors, features of the situation, history of interactions with robots, and characteristics of the robots~\cite{Hancock2011,Kok2020,naneva2020systematic}. Further, the research found that previous interactions with robots result in higher trust \cite{sanders2017trust, correia2016just}. However, these effects of prior experience with robots depend on how the robot behaves in the interaction. In a meta-analysis of trust in HRI, \citet{Hancock2011} found that robot performance-based factors (e.g., reliability, false-alarm rate, failure rate) had the greatest influence on developing trust in the robot. As another perspective on trust toward robots, \citet{gompei2018factors} explored cognitive and affective trust as two dimensions of trust toward robots. Cognitive trust is semi-rational and strongly influenced by the perception of the reliability of the robot, its intention, and its goals. On the other hand, affective trust is emotional in nature and less affected by the robot’s mistakes than cognitive trust but by the robot’s appearance (human-like appearance leads to more trust). As an example of such affective trust, it recently demonstrated that robots with music-driven emotional prosody and gestures were perceived as more trustworthy than robots without such features \citet{savery2019establishing}.


Recently, we have seen a shift in focus of the research on trust in robotics, from studying the causes and effects of trust in robots to the strategies for robots to actively gain and maintain the trust of humans. Most of this work concentrates on two domains, making robots more humanoid with increased social characteristics such as emotional facial expression ~\cite{calvo2020effects} or increasing their reasoning capacity ~\cite{naneva2020systematic,Salem2015,christoforakos2021can}. These directions, however, have important limitations as they do not transfer to non-humanoid and simple robots, which do not resemble humans and are not capable of verbal communication and high reasoning capacity. Therefore, this work focuses on strategies to establish trust toward \aclp{SNHR}.

%Trust toward robots was mainly measured using questionnaires \cite{malle2021multidimensional}, however, also an experimental measures, trust game was used to measure trust \cite{oksanen2020trust}.  

%The question which motivates our research is how to design simple, non-humanoid robots so they will be trusted by humans. The answer to this question is relatively straightforward with respect to the cognitive dimension – the robot should look reliable and should not make errors. This question is, however, much more interesting with respect to the affective dimension of trust, which arguably is even more important, because affective trust increases the willingness to cooperate but also to forgive errors. Research on trust in humans suggests that the ability to synchronize with the human partner may be an important factor in perceiving the robot as trustworthy. 

\subsection{Synchronization in Humans and Human-Robot Interaction}

% What is synchronization?
% How does it relate to trust?
%last part: To the best of our knowledge, however, the influence of synchronization has not yet been systematically studied between humans and machines.Therefore, in this paper, we do x/y/z.

In physics, synchronization is the alignment of the rhythms of two or more oscillators due to mostly weak interactions~\cite{pikovsky2001synchronization} and, therefore, the coordination in time among the states or dynamics of the elements comprising the system \cite{schmidt2008dynamics}. Beyond physics, the notion of synchronization has been adapted for other systems, such as periodic dynamics in cardiac~\cite{Rosenblum1998} or nervous systems~\cite{Lestienne2001}. \citet{pikovsky2001synchronization} provide a description of specific types of synchronization. The most prominent type of synchronization is mimicry \cite{chartrand2009human}, also known as the chameleon effect \cite{chartrand1999chameleon}. In mimicry, one interaction partner mimics with some time delay and possibly some variation, movements, postures, rate of speech, or the tone of voice of the other interaction partner.

Similarly to the usage in other domains, synchronization plays a key role in social interactions~\cite{yun2012interpersonal,Nowak2017,Nowak2020}. In such social interactions, synchronization of behavior binds individuals into higher-level functional units, such as dyads or social groups~\cite{Nowak2017,Nowak2020}, that can perform a common action aimed at the achievement of a common goal, for example, moving a piece of furniture, servicing a car in a race, prepare a meal in a picnic or singing a song together. For the interaction between two individuals to proceed smoothly, the overt behavior and internal states (e.g., activation level, emotions, goals) of the individuals must achieve synchronization \cite{newtson1994perception,fusaroli2014dialog}. This synchronization needs to occur at various levels, including motor behavior, but it also involves cognitive and emotional dynamics \cite{nowak2000modeling}. Synchronization does not need to be the same in all modalities. Two individuals involved in a conversation, for example, in an antiphase manner, synchronize their speech while they synchronize their facial expressions and body movements in-phase manner \cite{stel2010mimicry}. In this context, antiphase means that when the former speaks, the latter is silent, and vice versa. In-phase means that the former's body movements and facial expressions are in unison with the latter's body movements and facial expressions. The synchronization on the emotional level is related to the synchronization on the behavioral level because facial expressions tend to induce the corresponding emotional state in each partner of the conversation \cite{laird1974self,strack1988inhibiting}. Interpersonal synchronization protects against the antisocial consequences of frustration \cite{dybowski2022interpersonal}. Further, synchronization has important consequences on social relations  \cite{baron1994local, newtson1994perception, nowak1998computational, marsh2009social, miles2009rhythm}. Synchronization other leads to the formation of social ties and promotes a feeling of connectedness and liking \cite{chartrand1999chameleon, lakin2003using, dijksterhuis2005we, hove2009s}, while the failure to achieve synchronization evokes feelings of separateness \cite{nowak2007dynamical}. Highly related, \citet{launay2013synchronization} found that synchronization increases trust toward others \cite{launay2013synchronization}. Further, recent works indicated that synchronization of group decisions resulted in higher trust \cite{daudi2016effects}, and synchronization on the brain level in a trust game is correlated with higher investment and, thus, higher trust \cite{Cheng2022}.


Further, there is a large body of work in the area of HCI on human interaction mediated by technology, exploring mediated trust and mediated synchronization. For example, \citet{bos2002effects} found trust formation was slower, and the trust achieved was more fragile in mediated conditions. \citet{riegelsberger2003researcher} propose a methodological foundation to assess mediated trust. \citet{slovak2011exploring} suggest that even subtle differences in video-conference design may significantly impact mediated trust. \citet{brave1998tangible} explore the physical synchronization of states of distant, identical objects to create the illusion of shared physical objects across distance. \citet{rinott2022designing} indicate the potential of using interpersonal motor synchronization in HCI thanks to its pro-social consequences. \citet{slovak2014exploring} connect changes in skin conductance synchrony to changes in emotional engagement. \citet{scissors2008linguistic} demonstrated that forms of linguistic mimicry are associated with establishing trust between strangers in a text-chat environment. In this following, we specifically focus on HRI and the relationship between synchronization and trust.

%\subsection{Synchronization and trust in Human-Robot Interaction}

In recent years, research started to explore synchronization in human-robot interaction. For example, \citet{hofree2014bridging} found that humans will mimic the facial expressions of a robot, even if they are fully aware that their interaction partner is non-human. \citet{shen2015can} found that the motor coordination mechanism improved humans' overall perception of the humanoid robot. Other work has explored robotic drumming, rhythmic HRI, and human-robot musical synchronization \cite{crick2006synchronization,hoffman2011interactive,weinberg2009leader}. \citet{mortl2014rhythm} developed the concept of goal-directed synchronization behavior for robotic agents in repetitive joint action tasks and implemented it in an anthropomorphic robot.
Further, \citet{hashimoto2009effects} explored emotional synchronization and found that human feelings became comfortable when the robot made the synchronized facial expression with human emotions. Further works, again involving humanoid robots, found that the presence of a physical, embodied robot enabled more interaction, better drumming, and turn-taking, as well as enjoyment, especially when the robot used gestures~\cite{kose2009effects}. However, to the best of our knowledge, there exists no prior work exploring synchronization as a means to increase trust between humans and \simNonHumRobots{}. 
