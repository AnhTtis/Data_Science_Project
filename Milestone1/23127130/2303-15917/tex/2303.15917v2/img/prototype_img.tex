\begin{figure*}[ht!]
 \centering
\includegraphics[width=\textwidth]{img/blobs/PAPER_prototype.pdf}
\caption{Prototype design details. (A) The prototype’s general view. We see starting from the top: a single-segment continuum robot; a body of the prototype with electronics, actuators, and a coin acceptor; a plinth in the form of a pipe with a triangular foot. (B) The body with a covered coin acceptor. (C) Coin acceptor cover opening. (D) A body with a visible coin acceptor. (E) A coin returner with a coin cup refilling before each experiment. Below the tray where the returned coins fall. (F) A hidden drawer in which the coins thrown by the participant are collected. Coins are taken out after each experiment. (G) Zooming in on the elements of a single-segment continuum robot construction. Visible: fiberglass backbone, tendons made of braided flexible steel lines, separators made of PET-G, keeping the tendons at a proper distance from the backbone. (H) The experiment stage change button, operated by the investigator, is unnoticeable to the participant. (I) The arrow points to the indicator of condition number and stage of the experiment. The indicator facilitates the operation by the investigator. In addition, the photo shows the actuators and part of the control electronics. (J) IMU sensor to place on the participant's back during the exploratory phase of the experiment. (K) Zooming in on the single-segment continuum robot.}
\Description{Collage of 11 small pictures showing prototype design details. (A) The prototype’s general view. We see starting from the top: a single-segment continuum robot; a body of the prototype with electronics, actuators, and a coin acceptor; a plinth in the form of a pipe with a triangular foot. (B) The body with a covered coin acceptor. (C) Coin acceptor cover opening. (D) A body with a visible coin acceptor. (E) A coin returner with a coin cup was refilled before each experiment. Below the tray where the returned coins fall. (F) A hidden drawer in which the coins thrown by the participant are collected. Coins are taken out after each experiment. (G) Zooming in on the elements of a single-segment continuum robot construction. Visible: fiberglass backbone, tendons made of braided flexible steel lines, separators made of PET-G, keeping the tendons at a proper distance from the backbone. (H) The experiment stage change button operated by the investigator is unnoticeable to the participant. (I) The arrow points to the indicator of condition number and stage of the experiment. The indicator facilitates the operation by the investigator. In addition, the photo shows the actuators and part of the control electronics. (J) IMU sensor to place on the participant’s back during the exploratory phase of the experiment. (K) Zooming in on the single-segment continuum robot.}
\label{fig:insync:prototype}
\end{figure*}