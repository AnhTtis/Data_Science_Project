\section{Introduction}
\label{sec:introduction}

From entertainment~\cite{Pasquali2021, Aaltonen2017} or health care~\cite{Olaronke2017, Tuisku2019} to sex~\cite{Troiano2020, CoxGeorge2018, Richardson2016, Su2019}: with robots evolving from the assembly line to social robots, they are increasingly becoming part of our everyday lives. Thus, the question of how people perceive robots and what attitudes they have toward robots becomes crucial to define our relationships with them~\cite{naneva2020systematic}. Research has shown that trust is one of the key factors influencing the quality of interactions of humans with robots~\cite{safety8030048}, including a user's willingness to interact with a robot, take its advice into account~\cite{Langer2019}, and delegate tasks to robots~\cite{Hancock2011, Salem2015}. Lack of trust, hence, strains our relationship with social robots and may thus hinder or impede their future proliferation as ubiquitous everyday helpers.

In recent years, research has explored strategies by which robots can gain and maintain the trust of human users. This research concentrates mostly on robots’ features that influence how they interact with human users, pointing out such features as the capacity to reason~\cite{Kok2020}, the realistic facial expression of emotions~\cite{Salem2015}, or personalization~\cite{Langer2019}. \citet{kirkpatrick201710}  found that trust toward robots requires that the robot is perceived as having core human features such as agency and intention. Therefore, the more human a robot appears to be, the more trustworthy it can appear. If a robot displays human characteristics, it is likely to be perceived in interaction, in an anthropomorphic way, as a human~\cite{Langer2019, deVisser2017}. Building on this, research proposed a variety of human-like robot systems that can, for example, accurately express emotions through facial expressions or make eye-contact. The strategy of maximally anthropomorphizing robots, however, has its limitations. First, it raises unrealistic expectations that the robot will behave in a fully human way, which ultimately leads to frustration~\cite{Welge2016}. Second, it leaves aside the question of trust toward robots that do not have a humanoid appearance and advanced capacities for interaction with humans (e.g., reasoning). We refer to such kind of robot as an \simNonHumRobot{}. Simple as not having advanced capacities for interaction, and non-humanoid as not having a humanoid appearance.

To overcome these limitations, in this work, we go beyond state-of-the-art and add to the body of research in human-robot interaction by exploring a novel strategy to establish trust between humans and \simNonHumRobots{}. Drawing on recent findings in social psychology and neuropsychology, we propose to leverage the effect of synchronization: When we go for a walk with friends, we can observe that our movements - from stride length to arm movements - unconsciously align; they synchronize. This synchronization is a critical feature of positive social relations~\cite{Semin2008, Nowak2017, Nowak2020} and trust~\cite{Cheng2022}. 

%Suppose a robot is perceived in an anthropomorphic way. In that case, the results of research on human-human interactions showing that synchronization leads to an increase in trust can be generalized to human-robot interactions. Whether synchronization with non-human robots increases trust still needs to be answered. The goal of our research was to answer this question empirically. 

Generalizing the findings in social psychology, we hypothesize that the synchronization of an \simNonHumRobot{} with the human user increases the trust toward the \simNonHumRobot{}. Because synchronization is a basic dynamical feature of interaction, it does not assume a humanoid shape of a robot and advanced capacities for interaction. As a first step to evaluate the feasibility of establishing such a physical synchronization between humans and robots, we explore our idea using an intentionally abstract object looking more like an art installation than a robot to exclude possible confounding factors. For conducting the study, we built a prototype implementation of such an object, which tracks the upper body movements of people in its vicinity and can synchronize its movements with their movements (see fig \ref{fig:teaser}).

The contribution of this paper is two-fold. First, we illustrate the design process and technical implementation of a robot prototype that allows physical synchronization with human users. Second, we contribute the results of a controlled experiment with 51 participants assessing the influence of synchronized movement compared to simple or random movement patterns of a robot on the perceived trust of users and propose a set of guidelines for the future use of such interfaces.
