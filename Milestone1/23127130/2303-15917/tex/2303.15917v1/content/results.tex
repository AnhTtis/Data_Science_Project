\section{Results}
\label{sec:results}

%\begin{figure*}[ht!]
%	\subfloat[System Trust Scale\label{fig:insync:results:sts}]
%	{\includegraphics[width=.49\linewidth]{img/blobs/PAPER_sts}}\hfill
%	\subfloat[Money Task\label{fig:insync:results:money}]
%	{\includegraphics[width=.49\linewidth]{img/blobs/PAPER_money}}\\
%	\includegraphics[width=0.99\textwidth]{example-image-c}\hfill
%	\caption{The aggregated results of the \acl{STS} (a) and the mean number of coins (b) as meassured in our experiment. All error bars depict the standard error.}
%\end{figure*}

\begin{figure*}[t!]
	\centering
	\includegraphics[width=\textwidth]{img/blobs/PAPER_sts_money_corr}\hfill
	\vspace{-1em}
	\begin{minipage}[t]{.33\linewidth}
		\centering
		\subcaption{\acl{TPA}}\label{fig:insync:results:sts}
	\end{minipage}%
	\begin{minipage}[t]{.33\linewidth}
		\centering
		\subcaption{Money Task}\label{fig:insync:results:money}
\end{minipage}%
	\begin{minipage}[t]{.33\linewidth}
		\centering
		\subcaption{Correlation between \textsc{tpa} and Coin Task}\label{fig:insync:results:correlation}
\end{minipage}%

	\caption{The aggregated results of the (a) \acl{TPA} checklist and (b) the mean number of coins as measured in our experiment. (c) depicts the correlation between the two measurements for the three experimental groups. All error bars depict the standard error.}
	\Description{A three-part figure with charts of the aggregated results of the (a) Trust between People and Automation checklist and (b) the mean number of coins as measured in our experiment. (c) depicts the correlation between the two measurements for the three experimental groups. All error bars depict the standard error.}
	
\end{figure*}

In the following section, we report the results of the controlled experiment as described above.

\subsection{\acl{TPA}}

We evaluated the participant's trust in the system using the \ac{TPA} questionnaire.  This 12-item set of Likert scales includes a variety of items sampling trust but also distrust, such as perception of the automation’s deceptive nature or the likelihood of harmful outcomes if it is used. Items that sample distrust must be reverse-coded if used to create a singular trust score \citet{kohn2021measurement}. We reverse-coded items 1 to 5 as they address the distrust dimension. The formula used for a single trust dimension is as follows: \(t=\frac{\sum_{i=1}^5 (8-I_i) + \sum_{i=6}^{12} I_i}{12}\).

We found trust ratings on the \ac{TPA} ranging from \val{3.59}{.74} (\factorMoveLvlRandom{}) over \val{3.90}{.91} (\factorMoveLvlSimple{}) to \val{4.62}{.80} (\factorMoveLvlSynchronized{}), see fig. \ref{fig:insync:results:sts}. A Kruskal-Wallis test indicated a significant (\kruskalwallis{2}{11.18}{<.01}) influence of the \ivMovement{} on the perceived trust with a \efETAsquared{0.19} effect size. Dunn's post-hoc test corrected for multiple comparisons using the Bonferroni method confirmed significantly higher trust ratings for \factorMoveLvlSynchronized{} compared to both \factorMoveLvlSimple{} (\ztest{-2.48}{<.05}) and \factorMoveLvlRandom{} (\ztest{-3.18}{<.01}). We could not find any significant differences between \factorMoveLvlSimple{} and \factorMoveLvlRandom{} (\ztest{.71}{>.05}). 

% Please add the following required packages to your document preamble:
% \usepackage{multirow}
% \usepackage{graphicx}
% \usepackage[normalem]{ulem}
% \useunder{\uline}{\ul}{}
\begin{table*}
	\resizebox{\textwidth}{!}{%
		\begin{tabular}{llllllllllccc}
			\hline
			\multicolumn{1}{c}{\multirow{3}{*}{\textbf{Question}}}                                                   & \multicolumn{2}{c}{\textbf{simp}}                       & \multicolumn{2}{c}{\textbf{rand}}                       & \multicolumn{2}{c}{\textbf{sync}}                       & \multicolumn{3}{c}{\textbf{Kruskal-Wallis}}                                          & \multicolumn{3}{c}{\textbf{Dunn's Test}}                                       \\
			\multicolumn{1}{c}{}                                                                                     & \multirow{2}{*}{$\widetilde{x}$} & \multirow{2}{*}{IQR} & \multirow{2}{*}{$\widetilde{x}$} & \multirow{2}{*}{IQR} & \multirow{2}{*}{$\widetilde{x}$} & \multirow{2}{*}{IQR} & \multirow{2}{*}{$\chi^2(2)$} & \multirow{2}{*}{$p$} & \multirow{2}{*}{$\eta_{}^{2}$} & \multicolumn{1}{l}{simp} & \multicolumn{1}{l}{simp} & \multicolumn{1}{l}{rand} \\
			\multicolumn{1}{c}{}                                                                                     &                                  &                      &                                  &                      &                                  &                      &                              &                      &                                & rand                     & sync                     & sync                     \\ \hline
			The system is deceptive.                                                                                 & 2                                & 3                    & 4                                & 4                    & 3                                & 3                    & .74                          & \textgreater{}.05    &                                & \textbf{}                & \textbf{}                & \textbf{}                \\
			\begin{tabular}[c]{@{}l@{}}The system behaves in an\\ underhanded manner.\end{tabular}                   & 3                                & 3                    & 5                                & 3                    & 2                                & 2                    & 7.67                         & \textless{}.05       & .12                            & \textbf{}                & \textbf{}                & *                        \\
			\begin{tabular}[c]{@{}l@{}}I am suspicious of the system's\\ intent, actions or outputs.\end{tabular}    & 3                                & 4                    & 4                                & 2                    & 4                                & 4                    & 1.05                         & \textgreater{}.05    &                                & \textbf{}                & \textbf{}                & \textbf{}                \\
			I am wary of the system.                                                                                 & 5                                & 1                    & 5                                & 2                    & 3                                & 1                    & 6.67                         & \textless{}.05       & .10                            & \textbf{}                & \textbf{}                & \textbf{}                \\
			\begin{tabular}[c]{@{}l@{}}The system's actions will have\\ a harmful or injurious outcome.\end{tabular} & 1                                & 1                    & 1                                & 1                    & 2                                & 2                    & 2.07                         & \textgreater{}.05    &                                & \textbf{}                & \textbf{}                & \textbf{}                \\
			I am confident in the system.                                                                            & 4                                & 1                    & 3                                & 2                    & 4                                & 0                    & 3.73                         & \textgreater{}.05    &                                & \textbf{}                & \textbf{}                & \textbf{}                \\
			The system provides security.                                                                            & 5                                & 3                    & 5                                & 2                    & 5                                & 2                    & .72                          & \textgreater{}.05    &                                & \textbf{}                & \textbf{}                & \textbf{}                \\
			The system has integrity.                                                                                & 4                                & 1                    & 3                                & 2                    & 4                                & 1                    & 11.78                        & \textless{}.01       & .20                            &                          & *                        & **                       \\
			The system is dependable.                                                                                & 3                                & 2                    & 2                                & 2                    & 4                                & 1                    & 10.37                        & \textless{}.01       & .17                            & \textbf{}                & \textbf{}                & **                       \\
			The system is reliable.                                                                                  & 3                                & 2                    & 2                                & 1                    & 3                                & 2                    & 4.03                         & \textgreater{}.05    &                                & \textbf{}                & \textbf{}                & \textbf{}                \\
			I can trust the system.                                                                                  & 3                                & 2                    & 3                                & 2                    & 4                                & 2                    & 9.18                         & \textless{}.05       & .15                            & \textbf{}                & \textbf{}                & *                        \\
			I am familiar with the system.                                                                           & 1                                & 1                    & 1                                & 1                    & 5                                & 2                    & 27.10                        & \textless{}.001      & .52                            & \textbf{}                & ***                      & ***                      \\ \hline
		\end{tabular}%
	}
	\caption{The participant's answers to the individual subscales of the \acl{TPA}. Asterisks refer to the assumed significance levels $p<.05$ (*), $p<.01$ (**) and $p<.001$ (***).}
\label{tab:insync:results:sts}
\end{table*}

To gain further insights into the participant's attitudes toward the \factorMove{}, we analyzed the individual subscales of the \ac{TPA}. For six subscales, a Kruskal-Wallis test indicated significant differences (see table \ref{tab:insync:results:sts}). Post-hoc tests confirmed significant differences between the \factorMoveLvlSynchronized{} and \factorMoveLvlRandom{} conditions for five subscales. Additionally, we found significant differences between \factorMoveLvlSynchronized{} and \factorMoveLvlSimple{} for two subscales. We could not find significant differences between \factorMoveLvlSimple{} and \factorMoveLvlRandom{} for any subscales. Table~\ref{tab:insync:results:sts} lists the test result for the individual subscales. Further, figure \ref{fig:insync:results:likert} provides a breakdown of the internal distribution of the measured variables for all subscales with significant differences.

\begin{figure*}[ht!]
 \centering
\includegraphics[width=\textwidth]{img/blobs/PAPER_likert.pdf}
\caption{Participants' answers to the \acf{TPA} questions. The figure depicts the six statements that provoked significant differences between the three groups.}
\label{fig:insync:results:likert}
\Description{A six-part figure showing participants’ answers to the trust between People and Automation (TPA) questions. The figure depicts the six statements that provoked significant differences between the three groups.}
\end{figure*}

\subsection{The Trust Game}

As an additional measurement of the participants' trust toward the prototype, we adapted a method of the trust game as described in section \ref{sec:methodology:design}. All but one participant participated in the game. We found the highest number of inserted coins for the \factorMoveLvlSynchronized{} condition (\val{2.82}{1.67}), followed by \factorMoveLvlRandom{} (\val{2.47}{1.66}) and \factorMoveLvlSimple{} (\val{2.41}{1.33}), see fig. \ref{fig:insync:results:money}. As Shapiro-Wilk's test indicated a violation of the assumption of normality of the residuals that could not be resolved by transforming the data on the log scale, we continued with a non-parametric analysis. However, a subsequent Kruskal-Wallis test did not reveal a significant (\kruskalwallis{2}{0.85}{>.05}) influence of the \factorMove{} of the system on the number of coins inserted.

Further, we analyzed the correlation between the \dvTPA{} and the \dvMoney{} grouped by the \ivMovement{}. We found no significant correlations over all groups. While we could not find a trend for \factorMoveLvlSimple{} (\pearson{.005}{>.05}) and \factorMoveLvlRandom{} (\pearson{-.023}{>.05}) movements, \factorMoveLvlSynchronized{} movements indicated a non-significant trend toward a positive relationship (\pearson{.211}{>.05}). Figure \ref{fig:insync:results:correlation} depicts the pairs and fitted correlation lines.

%\subsection{Movement Synchronization}

%\subsection{Qualitative Results}

%Analyzing the video recordings of the free exploration phase, one can observe much greater excitement about the course of the interaction in the synchronized condition.

%After the trust game phase, qualitative feedback and the experimenter's observation showed that the participants were surprised by the proposal to insert coins into the prototype. The situation was strange and incomprehensible to them.