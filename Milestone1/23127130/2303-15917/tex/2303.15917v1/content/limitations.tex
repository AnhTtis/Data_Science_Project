\section{Limitations and Future Work}
\label{sec:limitions}

While we are convinced that our results present a viable contribution to the body of research on trust in human-robot interaction, the study design, as well as the results, impose limitations and directions for future work. In the following section, we discuss them.

\subsection{External Validity and Real-World Applicability}

We deliberately decided to use a form as abstract as possible in our experiment. We opted for this approach to avoid anthropomorphizing the robot’s form and associations with other animate forms carrying specific attitudes or emotions. We also avoided association with all kinds of popular robots. We are confident that the results enable a broad range of real-world application areas because everywhere where the form or complex behavior of the robot is limited, we can use dynamic, in this case, synchronization, as a means to design a trustworthy robot. However, it is an open question on how the results translate to other robot forms. For example, a robotic arm associated with an industrial robot will not eliminate the synchronization effect. Future work is necessary to conclude these challenges.

Further, while reviewing video recordings, we found that in the \factorMoveLvlSynchronized{} condition, at the moment when the participant discovered that the robot synchronizes with their movements, most of them had a strong positive reaction (broad and long smile) and started exploring the prototype more actively. But over time, there was boredom with the lack of new prototype behavior. This is in line with the works of \citet{ravreby2022liking}, who found that novelty or more complex behavior, even at the cost of losing or having more difficulty in achieving synchronization, produces much better results than keeping synchronization. It would be interesting to investigate whether this effect can be reproduced in interaction with the \simNonHumRobot{}.

Considering a broad body of works in HCI on human interactions mediated by technology, especially those through physical objects \cite{brave1998tangible}, it is an intriguing open research question of how our prototype would be perceived in such interactions. The basic idea is that the participant's moves will be mimicked by a remote prototype in front of another person and vice versa. In such a setup, we could, for example, manipulate the coupling parameter of synchronization of two humans remotely interacting, looking at their perception of the relation between them \cite{vallacher2005dynamics}.

%A very promising general direction of research that we want to undertake is also the control of novelty and complexity of traffic based on physical synchronization. Inspired by the effect in the human-human intersection described in a more recent work by Ravreby at al. \cite{ravreby2022liking}. It is also interesting to look for synchronization models exploring this idea based on machine learning models.

%\todo[inline]{how do our findings translate from the artificial setup to the real world? What is still to be researched in this area? Why did we do such an artificial setup?}

\subsection{Synchronization during Collaborative Tasks}

Besides the abstract body of our prototype, we further opted to build a system that basically serves no purpose other than being there and moving with the participant. We chose this path to provide a solid baseline and exclude influencing factors from an actual task of the system. Therefore, we analyzed synchronization as a tool for shaping the attitude toward the robot. However, it is still an open challenge how our results apply to collaborative scenarios where we as humans are working together with a system in a collaborative task. 

Here we see interesting questions in many areas. For example, for speakers with voice assistants, synchronization could increase the confidence in the assistant by giving them significant physical properties through the dynamics of movement. In such a scenario, the assistant synchronizes with the dynamics of human movements to increase affective trust. Compared to prior work, the goal is not to provide fitting gesticulation in harmony with her voice but to establish synchronization with the user. Further, with autonomous cars on the horizon, synchronization of movements of visual and physical elements of the car cockpit and other interfaces could help establish a relationship of trust, increasing the sense of comfort with unfamiliar situations and understanding the intentions of the car. 

Further, we did not enforce or restrict any movements of the participants and, thus, deliberately left it up to the participants how and if they wanted to interact with the system. We chose this approach to avoid biasing the provided baseline in any direction by external influences and to recreate a situation in which people interact freely with a system without a specific goal. While we did not collect the movement quantities, we hypothesise, based on our observations, that the amount of movement could have an influence on the perception of trust and - vice versa - the strength of trust could have an influence on the movement of the participants.

Lastly, in some safety-critical areas, it could also be valuable to decrease trust. It may be interesting to research if the user unconsciously wants to do potentially harmful actions to himself or the machine. Then breaking synchronization can signal the user to focus attention and revise the action and, as a further consequence, decrease trust in a potentially dangerous machine. Future work is needed to conclude these challenges.

%Automotive industry - building a relationship of trust with the car by communicating intentions and internal states of the car by means of the dynamics of movement synchronized with the dynamics of the driver's and passengers' movements. Achieved by adding moving elements in tune with the visual character of the car but constituting a kind of physical interface that communicates through the dynamics of movement.
%Industrial or home robot arms - ...

%\todo[inline]{Why did we chose a device that basically does not serve a purpose besides being there and moving? How do our results translate to devices that do something, that we want to collaborate with?}

\subsection{Synchronicity Beyond Upper-body Movements}

We explored synchronization only with upper-body movements. We opted for this approach to explore the simplest possible patterns leading to synchronization at the level of body movement.

Thus, it remains an open question how other body movements, such as head nodding, gesturing with the hands, or leg movements, can be mapped to induce synchronization. Beyond body movements, there are different body signals like breathing rate or heartbeat, which could also be used to establish synchronization.

On the robot side, we have only investigated a direct mapping of the participant’s movement, i.e., a chameleon-like rendering. It is an additional open question whether we can transfer synchronization patterns to other output modalities by keeping their dynamics, for example, mapping body movements to the brightness of a lamp. Further work is needed to answer these questions.

\subsection{Trustworthy Robot and Ethics}

In HRI, trust plays a crucial role and is strongly linked to persuasiveness. Under-trust or over-trust may have severe or even dangerous consequences~\cite{Salem2015}. Our results indicate that the synchronization of \simNonHumRobots{} to the movement patterns of human users can induce a sense of trust in people. These results can, on the one hand, provide the basis for developing systems that use this finding to more easily and quickly establish a relationship of trust with users. On the other hand, however, like different approaches to gaining and maintaining trust in robots, such as anthropomorphization, this effect can be exploited by malicious actors and lead people to trust untrustworthy entities. We consider the investigation of ethical aspects and assessing the possible dangers a vital field of research. Our work, by highlighting the role of synchronization, provides useful starting points for future work in these fields.

As a possible solution, certification and regulation by independent entities~\cite{shneiderman2020bridging} could increase the trustworthiness of interacting with such systems. Even though synchronization can influence the relationship with the machine on an affective level (e.g., feelings, emotions), such regulation could help to ensure that robots are regarded more as tools that we use to improve our own skills and accelerate progress along our own paths~\cite{bryson2010robots}.

%, as synchronization can influence the relationship with the machine (affective dimension, e.g., feelings, emotions), we want to highlight that robots should rather be seen as tools that we use to enhance our own capabilities and accelerate the progress on our own path~\cite{bryson2010robots}. Robots should not be treated as our peers or partners.




\begin{comment}
	- synchronizacja emocji - com vis - expresja twarzy - nadal prosta forma robota i wyrazanie emocji tylko dynamiką ruchu 
	- synchranozacja z gestykulacja
	- sybcronizacja z potakiwaniem głową w raozmowie
	- i w koncu pruba zanezienia sposobu na synchronizacje ze ruchem wszyskich czesic ciała jednocześnie prze zastosowanie ML do nadania znaczenia błożonym ruchom człowieka ale przy zacowania prostoty robota i jego reakcji wyłącznie przez dynamikę ruchu prostych elemetów. Przy czym w tej sytuacji róznież zastosowanie ML do kontroli ruchu elastycznego ciała robota. 
	
	\todo[inline]{why did we do this style of movement? What else is there (breathing,...) and how do we think our results apply?}
	
\end{comment}
