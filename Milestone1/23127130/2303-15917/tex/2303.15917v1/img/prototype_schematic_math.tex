\begin{figure*}[t!]
 \centering
 \begin{subfigure}[b]{0.22\textwidth}
    \centering
    \includegraphics[width=\textwidth]{img/blobs/PAPER_prototype_model.pdf}
    \caption{}
    \label{fig:insync:prototype:model}
\end{subfigure}
\hfill
\begin{subfigure}[b]{0.3\textwidth}
    \centering
    \includegraphics[width=\textwidth]{img/blobs/PAPER_prototype_schematics_math.pdf}
    \caption{}
    \label{fig:insync:prototype:sche}
\end{subfigure}
\hfill
\begin{subfigure}[b]{0.3\textwidth}
    \centering
    \begin {align*}
        l_1 = (-r\cos \theta)\phi
        \\
        \\
        l_2 = \left(\frac{1}{2}r\cos \theta-\frac{\sqrt{3}}{2}r\sin \theta\right)\phi
        \\
        \\
        l_3 = \left(\frac{1}{2}r\cos \theta-\frac{\sqrt{3}}{2}r\sin \theta\right)\phi
    \end {align*}
    \caption{}
    \label{fig:insync:prototype:math}
\end{subfigure}
\caption{(a) The model of a one-section cable-driven continuum robot used in the prototype: A - three cables/tendons; B - inextensible backbone; C - five spacer structures; D - base. (b) The layout of a one-section cable-driven continuum robot used in kinematic computing. \(l_0\) length of inextensible backbone; \(l_1, l_2, l_3\) length of cables driven by actuators; \(\phi\) angle of backbone bending; \(\theta\) angle of backbone bending direction; \(x, y, z\) backbone base coordinate system; \(x', y', z'\) backbone tip  coordinate system. (c) The formula provides the length of cables. Based on that length, actuators shorten or lengthen cables accordingly.}
\label{fig:insync:prototype:mathsche}
\Description{A three-part figure: (a) The model of a one-section cable-driven continuum robot used in the prototype with marked parts: A - three cables/tendons; B - backbone; C - five spacer structures; D - base. (b) The layout of a one-section cable-driven continuum robot used in kinematic computing. With marked elements: l0 length of inextensible backbone; l1, l2, l3 length of cables driven by actuators; phi angle of backbone bending; theta angle of backbone bending direction; x, y, z backbone base coordinate system; xprime, yprime, zprime backbone tip coordinate system. (c) The formula provides the length of cables. Based on that length actuators shorten or lengthen cables accordingly.}
\end{figure*}


