\section{Introduction}\label{sec:Introduction}

Remote piloted drones, also known as \acp{UAV}, have become increasingly important in recent years,
having already had a major impact on different applications, 
% such as surveillance and security, precision agriculture, and parcel delivery~\cite{NatureFutureDrones,UAV_Communications_for_5G_and_Beyond,8660516,9768113}.
such as surveillance and security, precision agriculture, and parcel delivery~\cite{UAV_Communications_for_5G_and_Beyond,9768113}.
In May 2021, 
Morgan Stanley predicted that, 
by 2050,
the entire \ac{UAM} market, 
including air taxis, delivery, and patrol drones, 
could reach a value up to \$19 trillion, 
accounting for 10 to 11\%  of the projected United States global gross domestic product (GDP)~\cite{MorganStanley}.
In addition, 
it is expected that the intrinsic flexibility of \acp{UAV} will enable new disruptive industries and markets that are currently beyond our imagination.

%--- TLC
The use of \acp{UAV} in communication networks can be categorized into two main categories:
\emph{i)} UAV-aided networks, 
where \acp{UAV} act as flying base stations, or relays,
and \emph{ii)} \acp{CCUAV}, 
where \acp{UAV} connect to the network as flying \ac{UE}. 
In both categories, 
supporting \acp{UAV} with a reliable connection is essential for safe and effective operation.
Cellular network connectivity provides a promising solution to this challenge,
allowing \acp{UAV} to communicate with ground control stations over long distances \ac{BVLoS}.

Given a fourth and/or fifth generation (4G/5G) cellular network,
to provide a minimum \ac{QoS} with reliability guarantees, 
e.g., 100~kbps rate and 50~ms latency at 3 nines of reliability for the command and control (C\&C) channel of a \ac{CCUAV}~\cite{3GPP36777},
most of the research has focused on the optimization of the trajectory of the \ac{CCUAV}~\cite{Cherif_disconnecivityAware,esrafilian20203d, Challita_PathPlanningIntereferenceAwareRL}.
%--- Trajectory Optimization Paper
% As an example, 
% the authors in~\cite{Cherif_disconnecivityAware} propose a trajectory planning framework by formulating a multi-objective optimization problem considering energy consumption and the handover rates. 
% %
% In~\cite{esrafilian20203d},
% an optimal trajectory design for \acp{CCUAV} is fine tuned while considering an accurate  3D environment map and the corresponding radio propagation modelling.
% %
% In~\cite{Challita_PathPlanningIntereferenceAwareRL},
% the authors entered the realm of machine learning (ML) and investigated a reinforcement learning (RL) framework to jointly minimise \acp{CCUAV} mission time and communication latency, as well as network interference.

% Despite the importance of such research,
Despite the importance of UAV trajectory optimization,
to support the significant growth and expansion of \ac{UAV} applications,
authorities and industries are working towards the creation of an organised system of \ac{UAV} highways in the sky to facilitate operation management and ensure reliable connectivity on predetermined aerial routes planned according to government and/or business criteria~\cite{PilotsHandbook_FAA, 3DAerialHighway_Magazine}.
Thus, optimizing 4G, 5G networks to support a minimum \ac{QoS} with reliability guarantees over a limited segregated airspace may be a more feasible and practical approach than route optimization over a given network.

The research community has begun to adopt such a complementary approach. 
However, only a few pioneering works exist in the literature. 
%
In~\cite{karimi2023analysis},
the authors carried out a mathematical analysis of the \ac{RSS} perceived by \acp{CCUAV} flying on aerial corridors, 
while being served by a ground cellular network.
%
In~\cite{chowdhury2021ensuring}, 
the authors explored the deployment of a new set of base stations with uptilted antennas to specifically serve aerial highways.
They also propose an \ac{eICIC} technique to mitigate interference to/from the aerial corridors. 
%
Similarly, in~\cite{PlacementofmmWaveAntenna}, 
the authors proposed a framework to optimize the deployment of uptilted \ac{mmWave} access points to serve \acp{CCUAV} on aerial highways.
%
In our previous work~\cite{10001469},
instead of deploying new base stations for \acp{CCUAV},
we developed a stochastic ADAM-based optimization algorithm to fine-tune the downtilt of an existing 4G macrocellular network to maximize the \ac{CCUAV} and ground UE rates, 
while providing a minimum SINR performance on the predefined aerial highways.
%

In recent years, 
various other solutions based on, e.g., null steering, \ac{D2D} communications, have been investigated to ensure a \ac{CCUAV} reliable connectivity provided a cellular network~\cite{8528463, 10008629, 10001193}.
%
However, none of the mentioned frameworks have investigated the importance of \ac{CCUAV} cell association to the ground macrocellular network. 
Given that multiple \acp{CCUAV} will be closely located over the aerial highway, 
selecting the serving cell that provides the largest \ac{RSRP} may be suboptimal as it may not allow to efficiently exploit \ac{mMIMO} multiplexing capabilities. 
In this paper, 
we investigate a new metric to drive the \ac{CCUAV} cell association process to a \ac{mMIMO} 5G network.
Our proposed methodology uses planning information collected across the aerial highway to extrapolate the \ac{mMIMO} multiplexing capabilities of a cell over a given route.
This information, 
in addition to the \ac{RSRP},
is then incorporated into the \ac{CCUAV} cell selection logic to select the best server among the suitable set of candidates.

The rest of the paper is organized as follows.
In Section~\ref{sec:SystemModel}, 
we introduce the adopted system model. 
In Section~\ref{sec:CellAssociationandMetrics}, 
we define the investigated cell selection metric for \acp{CCUAV}. 
In Section~\ref{sec:Results}, 
we discuss our experiments and results, and finally, 
in Section~\ref{sec:Conclusion}, the conclusions are drawn.