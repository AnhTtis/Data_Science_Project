\section{Evaluation and Discussion}\label{sec:Results}
In this section, 
we evaluate the performance achieved by our proposed metrics when adopted to drive the cell selection process of \acp{CCUAV}. 
%
We adopt the scenario presented in Section \ref{sec:SystemModel}.
Note that $N_{\mathrm{ccuav}}=5$ \acp{CCUAV} are deployed on each aerial highway. 
The values of other parameters adopted in the numerical analysis are reported in Table~\ref{table:evaluation_paramerters}.
%
\begin{table}[!t]
% \scriptsize
\centering
\caption{Summary of the parameters used.\label{table:evaluation_paramerters}}
\begin{tabular}{|c c| c c |} 
 \hline
 \textbf{Param} & \textbf{Value} & \textbf{Param} & \textbf{Value} \\ [0.5ex] 
 \hline\hline
 A & 0.22~Km$^2$&      
 $N_{\rm MS}$ & 3\\        
 $M$ & 64 &
 $M_h$, $M_v$ & 4,4\\
 $\lambda_p$ & 8.57~cm&
%  $G_0$ & 8~dBi \\
$f_c$ & 3.5~GHz \\ 
 $N_g$ & 4 &
%  $f_c$ & 3.5~GHz \\ 
$L_r$ & 400~m \\
 
 $R$ & 18 &
%  $L_r$ & 400~m \\
 $N_w$ & 400 \\
 $\Delta_\phi$ & 10$^\circ$ &
%  $N_w$ & 400 \\
$h_a$ & 1.5~m\\
  $d_\mathrm{ccuav}$ & 50~m & 
$h_a$ &  100~m\\
 $N_\mathrm{ccuav}$ & $\left\{1,\ldots, 7 \right\}$&
 $N_\mathrm{Drop}$ & 1000 \\
 $N_\mathrm{PRB}$ & 100&
  $P_{T_x}^\mathrm{Tot}$ & 46~dBm\\
 $P_{T_x}$ & 26~dBm&
 $K$ & 14.22 ~dB \\
 $\lambda_{Th}$ & 0.10&  
$\alpha$ & 0.50 \\
%  $\alpha$ & 0.50 \\
 \hline
\end{tabular}
\vspace{-2.5em}
\end{table}
%
Note that 
due to space constraints, 
we discuss the results obtained over a single aerial route,
that with $\Delta_\phi=90^\circ$.
However, similar results were obtained for all routes.

\subsection{Cell Selection Rate}\label{subsec:AssociationProbability}

In this section, 
we analyse the impact of the selected cell selection metric on the cell selection rate,
defined as the probability that an arbitrary \ac{CCUAV} selects a given sector as a serving one.

Figure~\ref{fig:AssociationProbability} shows the cell selection rate of the 5 deployed \acp{CCUAV} 
when considering three metrics: \ref{eq:SUM_Metric}, \ref{eq:CAP_Metric}, and \ac{RSRP}.
The first two metrics are the proposed ones, 
and the last one, noted as \ac{RSRP}, is the metric typically used in traditional networks,
which is used here as a benchmark.

\begin{figure}[!t]
    \centering
    \subfigure[Cell selection rates using RSRP metric.]{
    \includegraphics[width=0.43\textwidth]{Figure/SingleAssociationPlot/Association_Probability_RSRP.eps}
    \label{subfig:associationProbRSRP}
    }
    \subfigure[Cell selection rates with summation based metric eq.~\eqref{eq:SUM_Metric}.]{
    \includegraphics[width=0.43\textwidth]{Figure/SingleAssociationPlot/Association_Probability_M1.eps}
    \label{subfig:associationProbM1}
    }
    \subfigure[Cell selection rates with capacity based metric eq.~\eqref{eq:CAP_Metric}.]{
    \includegraphics[width=0.43\textwidth]{Figure/SingleAssociationPlot/Association_Probability_M2.eps}
    \label{subfig:associationProbM2}
    }
\caption{Cell selection rates of 5 CCUAVs on the vertical aerial highway (i.e., route rotated by $90^\circ$). \label{fig:AssociationProbability}}
\vspace{-1.5em}
% \vspace{-2em}
\end{figure}

As shown in Figure~\ref{subfig:associationProbRSRP},
when using the \ac{RSRP} metric,
some of the \acp{CCUAV} tend to associate to the south sector $\mathrm{MS}_{\mathrm{SO}}$, 
as it provides the strongest \ac{RSRP}.
For instance, 
$\mathrm{ccuav}_4$ connects $67.10\%$ of the time to the south sector $\mathrm{MS}_{\mathrm{SO}}$. 
When considering the other two studied metrics \ref{eq:SUM_Metric} and \ref{eq:CAP_Metric}, 
such rates dramatically change to $0.00\%$ and $18.60\%$, respectively.
Even if the south sector, $\mathrm{MS}_{\mathrm{SO}}$, may provide the largest received power,
the new eigenscore-based metrics lead to associations with sectors possessing better spatial resolution capabilities, 
i.e., $\mathrm{MS}_{\mathrm{WE}}$ and $\mathrm{MS}_{\mathrm{NE}}$,
as their \ac{mMIMO} panels have a better geometry with respect to the route,
As shown in Figure~\ref{fig:EigenvaluesEigenscore_evolution}, 
for the selected route, 
note that the south sector $\mathrm{MS}_{\mathrm{SO}}$ has the worst eigenscore, 
meaning that it cannot discern as many \acp{AoA} on the route as the other two sectors $\mathrm{MS}_{\mathrm{WE}}$ and $\mathrm{MS}_{\mathrm{NE}}$. 
Thus, associating more \acp{CCUAV} with the former sector leads to reduced spatial resolution, 
and ultimately reduced performance, 
as demonstrated in the following section.
%
%For $\mathrm{ccuav}_4$,
%the association rate with sectors located at
%south, west, and north-east (respectively, $\mathrm{MS_\mathrm{SO}}, \mathrm{MS_\mathrm{WE}}, \mathrm{MS_\mathrm{NE}}$
%
%in Figure~\ref{fig:2DNetworkLayout}) changes from $67.10\%, 21.50\%$, $11.40\%$, to $0.00\%, 60.6\%, 39.4\%$ and $18.60\%,49.60\%, 31.80\%$
%when adopting metrics \ref{eq:SUM_Metric} and \ref{eq:CAP_Metric}, respectively.
%
%
%Although the south sector, $\mathrm{MS}_{\mathrm{SO}}$, may provide the largest received power for $\mathrm{ccuav}_4$,
%the new eigenscore-based metrics lead
%to association with sectors possessing better spatial resolution capabilities, i.e., $\mathrm{MS}_{\mathrm{WE}}$ and $\mathrm{MS}_{\mathrm{NE}}$.
%In particular, as shown in Figure~\ref{fig:EigenvaluesEigenscore_evolution},
%for the selected route,
%the south sector, $\mathrm{MS}_{\mathrm{SO}}$, has the worst eigenscore, meaning that it cannot distinguish as many \ac{AoA} as the other sectors. Therefore, associating with such sector leads to reduced spatial resolution, and thus reduced performance, as shown in the next section.%, resulting in worse performance.
%for the serving power
%and second, due the higher correlation,
%in disruptive inter-cell interference for other flying CCUAVs.

%--- Comment on HO
It is worth highlighting that the results of this study also suggest that the two proposed metrics can be leveraged to reduce the number of candidate serving sectors for \acp{CCUAV}, 
resulting in fewer handovers and improved network stability. 
This can help to reduce overhead and handover failures, 
and thus enhance performance of the network.

%It is worth highlighting that these results suggest that the two proposed metrics allows to reduce the set of candidate sectors for the  CCUVAs.
%This implies that the proposed metrics reduce the occurrence of multiple connections and handovers,
%leading to improved network stability and reduced overhead.
%----

%----- SINR PERFORMANCE 
\subsection{SINR Performance}

In the following, 
we analyse the \ac{UE} \ac{SINR} performance to further highlight the benefits of the proposed metrics when adopted to drive the \ac{CCUAV} cell selection process.

Figure~\ref{fig:SINR_dist_Vertical_route} shows the \ac{SINR} distribution of the 5 deployed \acp{CCUAV} 
when considering the three metrics discussed earlier: \ref{eq:SUM_Metric}, \ref{eq:CAP_Metric}, and \ac{RSRP}.

\begin{figure}[!t]
\vspace{-1em}
    \centering
    \includegraphics[width=0.49\textwidth]{Figure/Route90deg_5uav_SINR_dB.eps}
    \caption{SINR comparison of CCUAVs placed on the vertical route.}
    \label{fig:SINR_dist_Vertical_route}
\vspace{-1.5em}
\end{figure}

The results show that when the proposed eigenscore-based metrics are adopted for driving the cell selection process,
an important increase in both average and 5\%-tile \acp{SINR} is achieved.
Specifically,
metric \ref{eq:SUM_Metric} achieves a gain of $3.30$\,dB and $3.13$\,dB with respect to the \ac{RSRP} metric at the average and 5\%-tile \acp{SINR}, respectively,
while the respective gains of metric \ref{eq:CAP_Metric} are $2.36$\,dB and $1.66$\,dB.
A summary of the \ac{CCUAV} \acp{SINR} and gains can be found in Table~\ref{table:SINR_dist_Results}.
\begin{table}
\vspace{1em}
\scriptsize
\centering
\caption{Summary of SINR results for different association metric.\label{table:SINR_dist_Results}}
\vspace{-0.5em}
\begin{tabular}{|c |c c c |} 
\hline 
  & \multicolumn{3}{|c|}{\textbf{Association Metric}} \\
 \textbf{ } & \textbf{M1} & \textbf{M2} & \textbf{RSRP}   \\% [0.5ex] 
 \hline
 \hline
 \textbf{Aerial 5\%-tile SINR [dB]}& 0.16 & -1.31 & -2.97   \\ %\hline 
 \textbf{Gain 5\%-tile to RSRP [dB]}& 3.13 & 1.66 & --    \\ %\hline % %\hline 
 %
 \hline
 \textbf{Aerial mean SINR [dB]}& 7.79 & 6.85 & 4.49  \\ %\hline  
\textbf{ Gain to RSRP [dB]}& 3.30 & 2.36 & --    \\ %\hline % %\hline 

 
\hline 
\end{tabular}

\vspace{-2em}
\end{table}

To further analyse the benefits of such metrics to support the reliable connectivity of multiple closely located \acp{CCUAV}, 
Figure~\ref{fig:5Percentile_VS_NumberofCCUAVs} shows how the 5\%-tile \ac{SINR} evolves when more and more \acp{CCUAV} fly on the same aerial highway 
(up to 7 \acp{CCUAV} with an inter-\ac{CCUAV} distance of $d_\mathrm{ccuav}=50$~m).
%The results show that our introduced metrics for serving sector association 
%allow better performances when more than a single CCUAV is present, with a maximum gain of $3.70$\,dB for metric~\eqref{eq:CAP_Metric} when considering 3 CCUAVs, and $4.18$\,dB for metric~\eqref{eq:SUM_Metric} when 7 drones are placed on the aerial highway.
The results show that using the proposed metrics results in a significantly improved 5\%-tile SINR when compared to the \ac{RSRP} metric. 
Metrics~\ref{eq:CAP_Metric} and ~\ref{eq:SUM_Metric} achieve a maximum gain of $3.70$\,dB and $4.18$\,dB  for the case with 3 \acp{CCUAV} and 7 \acp{CCUAV}, respectively.  
A summary of the \ac{CCUAV} 5\%-tile SINR and respective gains can be found in Table~\ref{table:5Perc_Results}.

In summary, 
the improved performance when using the proposed metrics can be attributed to two factors. 
Firstly, 
the eigenscore enables the identification of cells that are more capable of effectively resolving \acp{AoA}, 
resulting in a higher multiplexing gain with \ac{mMIMO}.
Secondly, narrowing the pool of serving cell candidates reduces inter-cell interference. 

\begin{figure}[!t]
    \centering
    \includegraphics[width=0.49\textwidth]{Figure/5Perc_MultiUAV_idx17.eps}
    \caption{5\%-tile SINR versus the number of CCUAVs.}
    \label{fig:5Percentile_VS_NumberofCCUAVs}
\vspace{-0.5em}
\end{figure}

\begin{table}[!t]
\scriptsize
\centering
\caption{Summary of 5\%-tile SINR results for different association metric and number of CCUAVs.\label{table:5Perc_Results}}
\vspace{-0.5em}
\begin{tabular}{|c |c c c c c c c|} 
\hline 
  & \multicolumn{7}{|c|}{\textbf{Number of CCUAVs}} \\
\textbf{Metric}                         & \textbf{1} & \textbf{2} & \textbf{3} & \textbf{4} & \textbf{5} & \textbf{6} & \textbf{7} \\ \hline \hline
\textbf{M1 [dB]}                       & 9.61  & 6.37 & 4.69 & 2.62 & 0.17  & 0.16  & -1.14 \\
\textbf{M2 [dB]}                       & 9.61 & 6.37  & 4.21 & 2.19 & -1.20 & -1.31 & -5.57 \\
\textbf{RSRP [dB]}                     & 9.76  & 5.46 & 0.09 & 0.09 & -2.82 & -2.97 & -5.32 \\
\textbf{Gain M1-RSRP [dB]}      & -0.14 & 0.91 & 2.53 & 2.53 & 2.99  & 3.13  & 4.18 \\
\textbf{Gain M2-RSRP [dB]}     & -0.14 & 0.91 & 2.10 & 2.10 & 1.62  &  1.66 & -0.25 \\

\hline 
\end{tabular}
\vspace{-2.5em}
\end{table}




% --- Final Remark
The results show that using \ac{RSRP} as the sole metric for choosing the serving cell among multiple suitable candidates is not the optimal solution. 
Integrating our proposed eigenscore into the cell association metric allows for better and fairer SINR performance in \ac{mMIMO}-based networks. 
In our results, 
using metric~\ref{eq:SUM_Metric} instead of~\ref{eq:CAP_Metric} yields better results. 
This suggests to network operators,
which plan to integrate aerial highway systems into their network, 
that they should incorporate such an eigenscore as an offset value to drive their cell selection and potentially handover processes.
Further studies are needed. 