\section{System Model}\label{sec:SystemModel}

%--- Macrocellular sectors
To illustrate the main concept behind the proposed metric, 
we consider a sub-6GHz downlink scenario comprised of $N_\mathrm{MS}=3$ macrocellular sectors covering an area of $A$~km$^2$.
Each sector operates at carrier frequency $f_c$, 
and is equipped with a planar \ac{mMIMO} antenna panel with $M$ active antennas, 
located at a height of $h_s=25$\,m.
To provide more detail, 
the \ac{mMIMO} antenna panel is composed of $M_h$ antennas on the horizontal axis and $M_v$ on the vertical one, 
i.e., $M=M_h \times M_v$. 
The distance between each antenna element is $\lambda_p / 2$, 
where $\lambda_p $ is the design wavelength. 
The \ac{mMIMO} antenna panel is directed toward the center of the mentioned covered area.
% Figure~\ref{fig:2DNetworkLayout} depicts the layout of the network.
%
The total transmit power of each sector and the transmit power allocated to each \ac{PRB} by each sector are equal to $P_{T_x}^\mathrm{Tot}$ and $P_{T_x}$, respectively. 
Note that in this paper, 
we consider that the total transmit power $P_{T_x}^\mathrm{Tot}$ of the sector is equally divided among the $N_\mathrm{PRB}$ \acp{PRB} managed by the sector, 
i.e., $P_{T_x} = P_{T_x}^\mathrm{Tot} / N_\mathrm{PRB}$.

%--- Aerial Highways 
A set of $R$ highways of length $L_r$ are deployed at the center of the scenario at an altitude $h_a$. 
Such highways share the same center point, 
and are symmetrically rotated from each other by an angle $\Delta_\phi$.
Each highway is defined by $N_w$ equidistant aerial waypoints (i.e., reference points),
with an inter-waypoint distance $d_w=1$~m.

%--- CCUAVs and Ground Users
$N_\mathrm{ccuav}$ single-antenna \acp{CCUAV} are then deployed on the aerial waypoints of each highway.  
Following the recommendations in~\cite{Vinogradov_reducingSafeUAVDistance},
the chosen inter-\ac{CCUAV} distance is $d_\mathrm{ccuav}=50$~m.
In addition to the \acp{CCUAV}, 
$N_g$ single-antenna \ac{gUE} are also randomly deployed within the coverage area of each sector.
%
Figure~\ref{fig:2DNetworkLayout} depicts the network layout and the deployed aerial highways with \acp{CCUAV} positioned on one of them.
% Figure~\ref{fig:2DNetworkLayout} also shows the deployed aerial highways and the \acp{CCUAV} positioned on one of them.

%--- Notation
For the sake of clarity,
let us denote by $\mathbb{B}$ the set of sectors, 
with $\mathrm{Card}\left\{\mathbb{B}\right\} = N_\mathrm{MS}$,
and by $\mathbb{D}$, $\mathbb{G}$ and $\mathbb{U}$ the sets of \acp{CCUAV}, \acp{gUE} and all \acp{UE} in the network, respectively,
such that $\mathrm{Card}\left\{\mathbb{D}\right\} = N_\mathrm{ccuav}$, $\mathrm{Card}\left\{\mathbb{G}\right\} = N_g$ and $\mathrm{Card}\left\{\mathbb{U}\right\} = N_\mathrm{ccuav} + N_g$, respectively.

%--- Multiplexing
For the sake of simplicity, 
let us assume that a sector multiplexes all its connected \acp{gUE} and \acp{CCUAV} across its bandwidth ---all its \acp{PRB}.
%Matteo: Is the above sentence essential? or can be eliminated for the sake of space?

%Note that, $N_\mathrm{Drop}$ realizations of the gUEs positions are performed to guarantee the robustness of the numerical results. 
%Nicola: I'm not sure if the last sentence should be here or in the numerical results section.

\begin{figure}[!t]
    \centering
    \includegraphics[width=0.48\textwidth]{Figure/2DNetworkLayout.eps}
    \caption{2D Network Layout with 3 MS with $N_g$ gUEs each, $R$ routes and $N_\mathrm{ccuav}$ CCUAVs positioned on the horizontal lane.}
    \label{fig:2DNetworkLayout}
\vspace{-1.5em}
\end{figure}

\subsection{Channel Model}
% \subsection{CHANNEL MODEL}

%--- Large Scale Channel 
The large-scale channel characteristics, 
\ac{LoS} probability, path loss, and shadow fading, 
for \acp{gUE} and \acp{CCUAV} are modelled according to the \ac{3GPP} urban macro scenarios in~\cite{3GPP38901} and~\cite{3GPP36777}, respectively.
We also integrate 2D spatial correlation features for the stochastic lognormal shadowing between \acp{UE} and sectors. 
To compute such realizations, 
we adopt the sum of sinusoids approach presented in \cite{ShadowCorrelation_Xiaodong}.

%--- Small Scale Channel 
The complex channel, $\textbf{h}_{u, b} \in \mathbb{C}^{1XM}$, between a \ac{UE} $u \in \mathbb{U}$ and the $M$ antennas of the \ac{mMIMO} panel of sector $b \in \mathbb{B}$ is modeled as a Rician random variable, i.e.,
\begin{equation}\label{eq:General_RicianChannel}
\textbf{h}_{u, b} = \sqrt{\frac{K}{1+K}}\; \textbf{h}_{u, b}^{\mathrm{LOS}} + \sqrt{\frac{1}{1+K}}\; \textbf{h}_{u, b}^{\mathrm{NLOS}},
\end{equation}
with 
\begin{equation}
    \textbf{h}_{u, b}^\mathrm{LOS} = e^{j \frac{2 \pi}{\lambda_c} \textbf{d}_{u,b}},
\end{equation}
and
\begin{equation}
     \textbf{h}_{u, b}^\mathrm{NLOS} \sim  \mathbb{CN}^M(0,1) \,\,,
\end{equation}
where $\textbf{d}_{u,b}= \left[d_{u,b}^0, \ldots, d_{u,b}^M \right]$ is the distance between a \ac{UE} $u$ and an antenna element $m$ of the \ac{mMIMO} panel of sector $b$,
and $K$ is the $K$-Rician factor, 
whose value is specified in~\cite{3GPP36777}.

\subsection{Signal quality model}

%--- Signal
The signal $y_{u}$ received at \ac{UE} $u$ from its serving sector $s$ is formulated as
\begin{gather}\label{eq:received signal}
    y_u = \sqrt{ \beta_{u,s} }  \,  
    \textbf{h}^H_{u,s}  \textbf{w}_{u,s} + \\  \nonumber
    \sqrt{\beta_{u,s}}  
    \sum_{p \in \mathbb{U}_s \setminus u} \textbf{h}^H_{u,s}  \textbf{w}_{p,s} + \\  \nonumber
    \sum_{b \in \mathbb{B} \setminus s } \sqrt{\beta_{u,b}}  
    \sum_{i \in \mathbb{U}_b} \textbf{h}^H_{u,b} \textbf{w}_{i,b} + n_u \,,
\end{gather}
with 
\begin{equation}\label{eq:LargeScale_gain}
    \beta_{u,b} = P_{T_x} \, G_{u,b} \, \rho_{u,b} \, \tau_{u,b} \,,
\end{equation}
where $\mathbb{U}_b \subset \mathbb{U}$ is the set of \acp{UE} connected to sector $b$, 
$s \in \mathbb{B}$ is the serving sector of \ac{UE} $u$, 
$G_{u,b}$, $\rho_{u,b}$ and $\tau_{u,b}$ are the antenna, path and shadow fading gains between \ac{UE} $u$ and sector $b$, respectively,
$\textbf{w}_{u,b} \in \mathbb{C}^{M \times 1}$ is the precoding vector devised to serve \ac{UE} $u$ at sector $b$,
%Note that such precoding vector $\textbf{w}_{u,b}$ is computed using \ac{ZF},
%and then normalized to satisfy total power constraints. 
and $n_u$ is the thermal noise.

%--- Precoding
At a given sector $b$, 
and for a given channel matrix
$\textbf{H}_b = \left[  \textbf{h}_{1,b}, \textbf{h}_{2,b}, \ldots, \textbf{h}_{N_b^\mathrm{in},b}  \right]^T$,
the precoding matrix $\textbf{W}_b = \left[  \textbf{w}_{1,b}, \textbf{w}_{2,b}, \ldots, \textbf{h}_{N_b^\mathrm{in},b}  \right]$
is derived using \ac{ZF} as~\cite{8528463}
\begin{equation}\label{eq:ZeroForcing}
    \textbf{W}_b = \hat{\textbf{H}}_b^H \left( \hat{\textbf{H}}_b \; \hat{\textbf{H}}_b^H \right)^{-1} \textbf{D}_b^{-1/2},
\end{equation}
where $N_b^\mathrm{in}$ is the cardinality of set $\mathbb{U}_b$ 
(i.e., the number of \acp{UE} served by sector $b$),
$\hat{\textbf{H}}_b$ is the estimated channel matrix,
and $\textbf{D}_b^{-1/2}$ is the diagonal normalization matrix defined to satisfy the transmit power constraints, 
with an equal transmit power allocation for each \ac{UE} in this case.
%
%
% In this work, 
% we assume pilot Reuse 3, 
% in which the set of pilot signals used for channel estimation at each sector is orthogonal with respect to those used at the other 2 sectors. 
% Since the amount of pilots is larger than the amount of served \acp{UE}, 
% we assume perfect \ac{CSI}, 
% and thus the estimated channel matrix $\hat{\textbf{H}}_b$ and the real channel matrix coincide perfectly, i.e., $\hat{\textbf{H}}_b = \textbf{H}_b$.
%

In this work, 
we assume that the set of pilot signals used for channel estimation at each sector is orthogonal with respect to those used at the other 2 sectors. 
This results in perfect \ac{CSI}, 
and thus the estimated and the real channel matrices coincide perfectly, 
i.e., $\hat{\textbf{H}}_b = \textbf{H}_b$.

%--- SINR
Finally, 
the resulting \ac{SINR} at \ac{UE} $u$ when associated to sector $s$ is defined as
\begin{equation}
    \gamma_u = \frac{ P_u  }  
    {I_u + N_u} = \frac{ P_u  }  
    {\left( I_u^{\mathrm{intra}} + I_u^{\mathrm{inter}}\right) + N_u},
\end{equation}
with
\begin{equation}\label{eq:UsefulPowerFromula}
    P_u = \beta_{u,s} 
    \left|\textbf{h}^H_{u,s}  \textbf{w}_{u,s} \right|^2 \,,
\end{equation}
and
\begin{gather} \label{eq:InterferenceFormula}
    I_u = I_u^{\mathrm{intra}} + I_u^{\mathrm{inter}} = \\ \nonumber
    \sqrt{\beta_{u,s}}  
    \sum_{p \in \mathbb{U}_s \setminus u} \left|\textbf{h}^H_{u,s}  \textbf{w}_{p,s}\right|^2 +
    \sum_{b \in \mathbb{B} \setminus s } \sqrt{\beta_{u,b}}  
    \sum_{i \in \mathbb{U}_b} \left| \textbf{h}^H_{u,b} \textbf{w}_{i,b}\right|^2 \,,
\end{gather}
where $P_u$ is the useful received power at \ac{UE} $u$, 
$I_u$ is the total interference composed of the intra-cell interference $I_u^{\mathrm{intra}}$ and the inter-cell interference $I_u^{\mathrm{inter}}$, 
whereas $N_u$ is the thermal noise power.
Note that, 
when using \ac{ZF} and under perfect \ac{CSI},
 $I_u^{\mathrm{intra}}=0$ if $N_b^\mathrm{in} \leq M$.