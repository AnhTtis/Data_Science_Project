\documentclass[fleqn,10pt]{wlscirep}
\usepackage[utf8]{inputenc}
\usepackage[T1]{fontenc}


\title{Individual differences in knowledge network navigation}

\author[1,*]{Manran Zhu}
\author[2, 3]{Taha Yasseri}
\author[1]{János Kertész}
\affil[1]{Central European University, Department of Network and Data Science, Vienna, 1100, Austria}
\affil[2]{University College Dublin, School of Sociology, Dublin 4, D04 V1W8, Ireland}
\affil[3]{University College Dublin, Geary Institute for Public Policy, Dublin 4, D04 V1W8, Ireland}

\affil[*]{Zhu\_Manran@phd.ceu.edu}

\keywords{Human navigation, knowledge networks, Wikipedia, online experiment, online game}

\begin{abstract}
As online information accumulates at an unprecedented rate, it is becoming increasingly important and difficult to navigate the web efficiently. To create an easily navigable cyberspace for individuals across different age groups, genders, and other characteristics, we first need to understand how they navigate the web differently. Previous studies have revealed individual differences in spatial navigation, yet very little is known about their differences in knowledge space navigation. To close this gap, we conducted an online experiment where participants played a navigation game on Wikipedia and filled in questionnaires about their personal information. Our analysis shows that participants' navigation performance in the knowledge space declines with age and increases with foreign language skills. The difference between male and female performance is, however, not significant in our experiment. Participants' characteristics that predict success in finding routes to the target do not necessarily indicate their ability to find innovative routes.
\end{abstract}
\begin{document}

\flushbottom
\maketitle
\thispagestyle{empty}

\section*{Introduction}\label{sec:Introduction}

Online technologies have fundamentally changed information provision and acquisition in our societies. While, in principle, the digital information ecosystem is horizontal, easy to navigate, and egalitarian in providing access to information, in practice, the networks of information repositories have become so complex that successful navigation has turned into a real challenge~\cite{savolainen2015cognitive}. More importantly, access to information is not equally provided to all citizens: not only inequalities in access to the infrastructure, such as broadband Internet connection or smart devices are seen as privileges available to certain groups in societies, individual characteristics, such as familiarity with the technologies, digital literacy~\cite{dutton2014cultures}, education level~\cite{van2014digital} and even personality traits~\cite{ho2005exploratory} all play a role in determining how much an individual can benefit from the open ocean of information available online~\cite{savolainen2015cognitive}. Finally, political decisions such as information sanctions or government censorship also challenge the idea of online information being "free for all" at a macro-level~\cite{gill2015characterizing}. To address the issue of inequality in information access, we must first gain an understanding of how individuals use the web and the unique ways in which they navigate it. This way, we can provide personalized support that caters to the specific needs of each user.

Previous research has demonstrated that the same neural regions that are responsible for navigation in physical space are also involved in navigating a knowledge space: the hippocampus and entorhinal cortex, which contain cells that encode spatial information and enable spatial navigation, also play essential roles in other neural processes such as social cognition and memory~\cite{tavares2015map, olafsdottir2015hippocampal}. Various individual differences have been observed in spatial navigation: spatial abilities decline linearly with age~\cite{anguera2013video, newcombe2018individual}; males generally perform better than females at spatial navigation tasks~\cite{nazareth2019meta, newcombe2018individual}; and people growing up outside cities are generally better at spatial navigation~\cite{coutrot2022entropy}. As for navigation in knowledge space, it was shown that older adults experience more difficulties when seeking information online due to cognitive ability disadvantages~\cite{sharit2008investigating, chevalier2015strategy}, and that males tend to be more confident about their ability to navigate the web than females~\cite{mcdonald2000gender}. Unlike physical space, however, knowledge space is a small world~\cite{zlatic2010knowledge}, consisting of a hierarchy of knowledge and shortcuts between them, making efficient navigation on it possible~\cite{kleinberg2000navigation}. Given the connections and differences between knowledge space and physical space, it is important to study if the individual differences in navigation in physical space are also present in knowledge space.

In an effort to gain insights into online navigation behaviors, researchers conducted a series of studies using Wikipedia as an observational setting~\cite{yasseri2012dynamics, yasseri2012practical} and utilized its well-documented network of articles as the framework for navigation studies~\cite{lamprecht2017structure, arora2022wikipedia}. The wide range of topics represented in Wikipedia (https://en.wikipedia.org/) and the platform's popularity make it a prime candidate for investigating empirical navigation behavior. In a popular online navigation game on Wikipedia, implemented in several versions such as the Wikispeedia (https://dlab.epfl.ch/wikispeedia/play/) and the Wikigame (https://www.thewikigame.com/), players try to go from one Wikipedia article (source) to another (target) through the hyperlinks of other articles within the Wikipedia website. Several navigation patterns on the Wikipedia knowledge network have been discovered: players typically first navigate to more general and popular articles and then narrow down to articles that are semantically closer to the target~\cite{west2012human}; players' search is not Markovian, meaning that a navigation step depends on the previous steps taken by the players~\cite{singer2014detecting}. When it comes to individual differences in navigation, there is still a lack of understanding, as the navigation patterns discovered so far have not taken into account personal information such as age and gender. This means that the current research does not necessarily reflect the behaviors and preferences of different demographic groups. As such, further investigations are needed to understand better how these factors may influence navigation patterns.

To gain a better understanding of how online navigation on the knowledge network is affected by individual characteristics, we conducted an online experiment where we hired 445 participants from the US to play nine rounds of Wikipedia navigation games (illustration in Fig. \ref{fig:Illustration}) and fill in a survey afterward about their personal information such as age, gender, and answer questions which enabled us to characterize their big five personality traits~\cite{cobb2012stability} (details in Methods). We sought to discover if individuals with certain characteristics possess an advantage over others in our navigation tasks. We introduced a uniqueness measure to study if certain players are more creative than others, meaning that they not only tend to win the navigation games but also take unusual routes to the target. 

\begin{figure}[!htb]
\centering
\includegraphics[width=1\textwidth]{figures/Illustration.png}
\footnotesize
\caption{ {\bf Illustration of the Wikipedia navigation game.} In the Wikipedia navigation game, players need to go from one Wikipedia article (source page) to another (target page) through the links of other Wikipedia articles on the current page in 7 steps (Least-clicks game) or 150 seconds (Speed-race game). The figure shows two possible navigation paths from the source page \textsc{Barack Obama} to the target page \textsc{Vincent van Gogh}: 1) \textsc{Barack Obama} to \textsc{Emmanuel Macron} to \textsc{France} to \textsc{Vincent van Gogh} (solid arrows); and 2) \textsc{Barack Obama} to \textsc{Bachelor of Arts} to \textsc{Art} to \textsc{Vincent van Gogh} (dotted arrows). Participants each played nine rounds of games whose source page and target page are shown in the figure. The games are divided into three sessions A, O, and B, with three games in each session. The order of the games is randomized in each game session, and the order of sessions A and B are randomized to reduce the effect of the games' order on performance.
}
\label{fig:Illustration}
\end{figure}

\section*{Results}\label{sec:Results}

\noindent\textbf{Effect of age and gender on navigation performance}
Previous studies have found that in spatial navigation tasks, younger people have an advantage over older people and males over females~\cite{nazareth2019meta}. In our experiment, we observed that participants' knowledge space navigation ability, measured as their ratio of games won (see Methods), also correlates negatively with age (r=$-.31$, p \textless $0.001$). In the bivariate linear regression analysis, age alone explains 9.5\% of the variance of navigation performance (Supplementary Table S$1$). After controlling for factors such as past experience or familiarity with the navigation game, age is still a significant ($p < 0.001$) predictor of navigation success (Table \ref{tab:RegMult} and Supplementary Table S$2$-$3$) and explains 12.1\% of the total variance of performance explained by the personal features as is shown in Fig. \ref{fig:Violinplots}i (see details in Methods). Fig. \ref{fig:Violinplots}a shows the distribution of the participants' navigation performance in different age groups. The age-dependence of performance, however is non-linear, and the decline accelerates for participants in the age groups larger than 30.

As for gender, to our surprise, the difference in navigation ability is not significant (Hedges' g = 0.16, $95\%$ CI = [-0.04  0.36]), although the male group does perform better on average than the female group, as is shown in Fig. \ref{fig:Violinplots}b. Controlling for other factors that might affect game performance, gender is still not a significant predictor for navigation success, as is shown in Table \ref{tab:RegMult}.

\begin{figure}[!htb]
\centering
\includegraphics[width=1\textwidth]{figures/Violinplots.png}
\footnotesize
\caption{ {\bf Performance in subgroups.} The figure shows the navigation performance distribution for participants with different characteristics, where performance is measured as the ratio of games won by each participant. Fig. a-h shows the distribution of navigation performance with respect to the participants' eight characteristics: a) age, b) gender, c) foreign language skills, d) ethnic background, e) political view, f) spatial navigation skills (the first principal component of the spatial navigation related questions $Spatial_{1}$), g) prior experience with the Wikipedia navigation game and h) prior experience with Wikipedia (the first principal component of the Wikipedia related questions $Wikipedia_{1}$). Fig. i shows the percentage of variance explained by each variable as they were added as covariates to the regression model (see details in Methods), normalized by the total variance explained by the multiple regression of all individual characteristics in Table \ref{tab:RegMult}. }
\label{fig:Violinplots}
\end{figure}

\vspace{\baselineskip}\noindent\textbf{Effect of language skills on navigation performance}
Language skills were previously shown to be related to spatial representation, especially the use of landmarks~\cite{shustermana2011cognitive}, but their role in knowledge space navigation is still unclear. In our experiment, we found that speaking a foreign language is a prominent predictor of navigation performance. As is shown in Fig. \ref{fig:Violinplots}c, it correlates with performance positively (Hedges' g = 0.57, $95\%$ CI = [0.37 0.77]), and it alone explains 7.5\% of the variance of performance (Supplementary Table S$1$). After controlling other factors, it is still a significant ($p < 0.001$) predictor (Supplementary Table S$2$ and Table \ref{tab:RegMult}) and explains 22.1\% of the performance variance explained by all the personal features as is shown in Fig. \ref{fig:Violinplots}i. Could the advantage of speaking a foreign language be explained by the correlation with the participants' familiarity with international cultures and specifically non-US persons or content? For each of the nine games, we computed the correlation between speaking a foreign language and winning the game. We found that those games involving non-US source or target pages do not show a higher correlation with better navigation performance (Supplementary Table S$6$).

\vspace{\baselineskip}\noindent\textbf{Effect of other individual characteristics}
Aside from age, gender, and language skills, the participants' ethnic background, spatial navigation skills, and personality traits also affect navigation performance. After controlling the covariates, being Asian, scoring high on "conscientious" and having good spatial navigation skills are significant predictors of success ($p < 0.05$) (Table \ref{tab:RegMult}). Distributions of the navigation performance in different ethnicity, political view, and spatial navigation ability subgroups are shown in Fig. \ref{fig:Violinplots}d-f.

Among the control features, we found that having previously heard of or played the Wikipedia navigation games gives participants an advantage in the experiment. It alone explains 10.4\% of the variance of the navigation performance and is still significant ($p < 0.001$) after controlling for other features (Table \ref{tab:RegMult}). The effect is also illustrated in Fig. \ref{fig:Violinplots}g, where it is seen that participants who had previously played the game were twice as likely to win than those with no prior experience. Aside from that, the participants that use Wikipedia more often and for a broader range of purposes ($Wikipedia_1$ in Table \ref{tab:Loadings}) tend to perform better as shown in Fig. \ref{fig:Violinplots}h and Table \ref{tab:RegMult}. Participants having better spatial navigation skills ($Spatial_1$ in Table \ref{tab:Loadings}) also tend to perform better in the knowledge space navigation task, which conforms with the previous studies that navigation behavior in physical and non-physical space are related~\cite{epstein2017cognitive}. Other control features, such as participants' prior experience with computer games, educational background, and employment status, do not significantly predict the navigation performance in our experiment (Table \ref{tab:RegMult}). 

\vspace{\baselineskip}\noindent\textbf{Interplay between success and uniqueness}
The standard definition of creativity requires both originality and effectiveness\cite{runco2012standard, campbell2022partial}. Are participants with specific characteristics more creative than the others in that they not only tend to win the navigation games (effectiveness) but also tend to take unusual routes (originality)? Defining the uniqueness of a navigation path as the average distance to other paths in the same game (details in Methods), we observed that the performance and creativity of the participants both vary: some participants won all the games while others none (Fig. \ref{fig:Distribution}a); some participants took mainstream routes while others took unique routes not often taken by others (Fig. \ref{fig:Distribution}b). As is shown in Fig. \ref{fig:Distribution}b, failed navigation paths (orange), on average, are more unique than successful navigation paths (green), probably because getting lost and deviating from the target leads to an unsuccessful navigation path and high uniqueness score of the path at the same time, therefore shifting the uniqueness distribution.

\begin{figure}[!htb]
\centering
\includegraphics[width=.8\textwidth]{figures/Distribution.png}
\footnotesize
\caption{ {\bf Distribution of Performance and Uniqueness.} Figure a) shows the distribution of the total number of games the participants won in the experiment. Figure b) shows the distribution of the uniqueness score of the participants' successful and unsuccessful navigation paths, measured as the standardized average distance of the path to the rest of the navigation paths in the same game. See Methods for more details on the definition and calculation of the uniqueness score. }
\label{fig:Distribution}
\end{figure}

Now we turn to the relationship between the participants' characteristics and the uniqueness of their selected paths. Using Hedges' g measure, we compared the average value of each personal feature in the creative subgroup (definition in Methods) and the rest. Positive Hedges' g value implies that the value of the respective personal feature is higher in the creative group, and the larger the absolute g value, the stronger the effect. Fig. \ref{fig:Uniqueness} shows that personal features that are significant indicators of performance (marked by green and orange) do not necessarily indicate creative performance. As is shown in Fig. \ref{fig:Uniqueness}a, for games with time constraints, having less age (being young) and having good spatial navigation skills, though both significant at predicting good navigation performance, have opposite effects on the creativity of the navigation routes: participants in the creative group is younger than the rest for all eight games (M=-0.34, SD=0.22) but have worse spatial navigation skills ($Spatial_1$) for seven out of eight games (M=-0.26, SD=0.22). For games with the number of steps constraints shown in Fig. \ref{fig:Uniqueness}b, age and spatial navigation skills no longer show strong group differences concerning creativity, but the proficiency of computer games ($Computer_1$), though not a significant indicator of performance, became the feature with the prominent difference between creative group and the rest (M=0.33, SD=0.19). For the rest of the variables, the Hedges' g value varies greatly across the games and therefore does not show a consistent effect over creativity.

% fig:Uniqueness
\begin{figure}[!htb]
\centering
\includegraphics[width=1\textwidth]{figures/Uniqueness.png}
\footnotesize
\caption{{\bf Uniqueness statistics.} The figure shows the Hedges' g value for each individual characteristic variable between the creative group and the rest for games with time constraints (left) and the number of steps constraints (right), respectively. The order of the characteristics in both figures are sorted by the median Hedges' g value over eight games. The variables that are significant ($p < 0.01$) in predicting the navigation performance in the multivariate regression analysis (Table \ref{tab:RegMult}) are plotted in orange (negative effect) and green (positive effect), and the rest of the variables in white. The Hedges' g values are computed for eight games separately and for successful navigation paths only.}
\label{fig:Uniqueness}
\end{figure}

\section*{Discussion}\label{sec:Discussion}

Our research investigates the way individuals navigate on the knowledge network. By conducting an online experiment, we have established a link between an individual's navigation behavior and their characteristics. We have found that similarly to navigation in physical space, the way we navigate on the knowledge network is affected by our characteristics, such as age and foreign language skills. However, we didn't observe a significant effect of gender on navigation performance, indicating that the results of navigation in physical space can not be generalized directly to knowledge space for certain traits. Other factors that predict better navigation performance include being Asian, having better spatial navigation skills and scoring high for the "conscientious" personality trait. An analysis of the uniqueness of the navigation paths suggests that the individual characteristics that predict success do not necessarily predict a creative approach to it. Under time constraints, younger participants tend to be more successful and creative simultaneously. In contrast, participants who are confident about their spatial navigation skills tend to take more mainstream routes. Proficient computer game players tend to take unusual navigation routes if they find a route, although they are not more likely to find successful routes than others.

While some of the predictors of success, such as age and prior experience with the navigation game, are intuitive, there are surprising traits, such as the ability to speak a foreign language, correlating non-trivially with the success rate. This indicates the importance of understanding the navigation patterns at the personal level to inform website designs and information repositories towards a more egalitarian information provision.

Our results have far-reaching implications. When it comes to government practices of digital services, the concept of "online only" has already been challenged by scholars relying on the fact that people of certain characteristics, particularly age, are less likely to be able to get online, and therefore there must be alternatives available to them~\cite{hunsaker2018review}. While this notion is becoming more dilute as Internet penetration reaches close to 100 percent in developed countries, it is still essential to note that being online means different things for different people based on their characteristics. If something is "up on the Internet", it does not necessarily mean everyone can find it.

In previous studies, game settings were frequently used to test the spatial navigation ability of both humans and rodents~\cite{coutrot2022entropy, spiers2021explaining, newcombe2018individual}. In our work, we also studied navigation behavior in a game setting. Further analysis is needed to generalize our results to real-life navigation situations. Our analysis could benefit from increasing the number of games in the experiment, which would enable a more solid statistical analysis of the creativity of the participants. It is also beneficial to conduct a repeated experiment to avoid the biases of a one-time experiment. 

Our work extended the previous research on navigation in knowledge space by adding individual variation to the analysis. A natural next research step should be to develop mathematical models incorporating personal traits and explaining participants' navigation behavior. It is also plausible to assist individuals with certain characteristics in future experiments to study if and how our navigation experience can be improved. 

\section*{Methods}\label{sec:Methods}
\textbf{The experiment}
We conducted an online experiment where we hired 445 participants (404 participants after removing participants that didn't finish the experiment and dropping data that had recording errors) from the United States on the online crowdsourcing platform Prolific (https://www.prolific.co/) to play nine rounds of the Wikipedia navigation game and fill in a survey on the survey platform Qualtrics (https://www.qualtrics.com/uk/). At the end of the experiment, each participant received a fixed rate base payment of 5 pounds and a bonus payment of 0.5 pounds for each game they won. To get a balanced population, we applied the following prescreening conditions: i) participants are from the United States, ii) an equal number of female and male participants, iii) participants with White, Asian, Hispanic, and African ethnicity consist $\sim$50\%, $\sim$17\%, $\sim$17\% and $\sim$17\% of the sample respectively. 

In the game sessions, players are given two Wikipedia pages as the source and the target in each game. The players start from the source page and navigate to the target page by clicking on the hyperlinks to other Wikipedia articles on the page. To win each game, they should reach the target page in at most 7 steps (Least-click game) or within 150 seconds (Speed-race game). Each participant plays nine rounds of games grouped into three sessions with a one-minute break between the sessions. After the game sessions, participants first finished a 50-question Big Five personality test (https://openpsychometrics.org/tests/IPIP-BFFM/) measuring their five personality traits: openness to experience, conscientiousness, extroversion, agreeableness, and neuroticism. To control other factors that may affect navigation performance, we then asked six groups of questions about their \textit{i}) employment status, \textit{ii}) education background, \textit{iii}) spatial navigation habit, prior experience with \textit{iv}) the Wikipedia navigation game, \textit{v}) their use of Wikipedia website and \textit{vi}) computer games. Lastly, we asked participants demographic questions about their age, gender, ethnicity, political position, and language skills. See the Supplementary Material for a complete list of the questions in the survey. One of the games with the source page "Alexander the Great" and target page "Tim Burton" turned out to be much more difficult than the other games ($> 3\sigma$), and is therefore counted as an outlier and excluded from our analysis. 

\vspace{\baselineskip}\noindent\textbf{Individual characteristics}
Encoding the participants' answers to the questions in the survey (see encoding details in the Supplementary Material), we end up with 18 control variables characterizing the participants by the six groups of questions and 11 independent variables describing the participants' big five personality traits, age, gender, ethnicity background, political position, and foreign/native language skills. To reduce the strong correlation and anti-correlation present among the control variables, we conducted a principal components analysis (PCA)~\cite{dunteman1989principal} in each question group and summarized 80\% of the variance by a reduced set of variables (principal components). The final list of the 14 control variables and their respective loadings from the original variables are shown in Table \ref{tab:Loadings}. Encoding the categorical variables with one-hot encoding, we end up with 31 individual characteristics variables in total. 

\vspace{\baselineskip}\noindent\textbf{Navigation paths}
A navigation path of a participant refers to the sequence of Wikipedia articles, or Wikipages, clicked by the participant in a game. Representing the hyperlinking structure of the English Wikipedia as a directed graph $G = (V, E)$, with $V = \{a_{k}\}$ denoting the set of all the Wikipages $a_{k}$ and $E = \{H_{kl}\}$ denoting the set of all the existing hyperlinks $H_{kl}$ from $a_{k}$ to $a_{l}$, the navigation path with $N$ steps for the $n$th participant in the $i$th navigation game $g_{i}$ can then be represented as a sequence $P_{n}^{i} = (a_{k})_{k=0}^{N}$ on the Wikipedia graph $G$, where $i = 1, 2, ..., 8$ and $n = 1, 2, ..., 404$. Denoting the source and target Wikipages of the game $g_{i}$ by $A_{s}^{i}$ and $A_{t}^{i}$, the navigation path $P_{n}^{i} = (a_{k})_{k=0}^{N}$ for the $n$th participant in the $i$th game is successful if it starts from the source and reaches the target, i.e. $a_{0} = A_{s}^{i}$ and $a_{N} = A_{t}^{i}$, and not successful if $a_{0} = A_{s}^{i}$ and $a_{N} \neq A_{t}^{i}$. Given the navigation paths of all the participants in all the games, we measure the success of the $n$th participant in the $i$th game by a binary variable $s_{n}^{i}$, which takes the value $1$ if the navigation path $P_{n}^{i}$ is successful otherwise $0$. 

\vspace{\baselineskip}\noindent\textbf{Quantifying the navigation performance of the participants}
To measure the overall performance of a participant in all the Wikipedia navigation games, we define the success of the $n$th participant $S_{n}$ as the ratio of games he/she won in total, where $s_{n}^{i}$ is the binary success measure of the navigation path $P_{n}^{i}$ defined previously.
\begin{equation}
    S_{n} = \frac{1}{8}\sum_{i=1}^{8} s_{n}^{i}
\end{equation}

\vspace{\baselineskip}\noindent\textbf{Quantifying the uniqueness of the navigation paths}
To understand how the navigation paths differ, we first trained a 64-dimensional graph embedding for each Wikipage $a_{i}$ over the English Wikipedia graph $G$ using the DeepWalk~\cite{perozzi2014deepwalk} algorithm. The graph embedding assigns a 64-dimensional number vector $\vec v_{i}$ to each Wikipage $a_{i}$, using which we constructed a semantic distance measure between the pairs of the Wikipages:

\begin{equation}
    d(a_i, a_j) = 1 - \frac{\vec v_i \cdot \vec v_j}{\|\vec v_i\|\|\vec v_j\|},
\end{equation}

\noindent where the semantic distance $d(a_i, a_j)$ between the Wikipages $a_i$ and $a_j$ is defined as the cosine distance between their graph embeddings $\vec v_i$ and $\vec v_j$. Given two navigation paths $P_{m}^{i} = (a_{m})_{m=0}^{M}$ and $P_{n}^{i} = (a_{n})_{n=0}^{N}$ of the $m$th and $n$th participants in the game $g_i$, we define the distance between the two paths as the Hausdorff distance~\cite{rockafellar2009variational} between the two sets of Wikipages:

\begin{equation}
    D_{H}(P_{m}^{i}, P_{n}^{i}) = Max\{\sup_{a_m \in P_{m}^{i}}d(a_m, P_{n}^{i}), \sup_{a_n \in P_{n}^{i}}d(P_{m}^{i}, a_n)\}
\end{equation}

\noindent where $d(a_m, P_{n}^{i}) = \inf_{a_n \in P_{n}^{i}} d(a_m, a_n)$ quantifies the distance from the Wikipage $a_m$ to the navigation path $P_{n}^{i}$, defined as the smallest distance from $a_m$ to any Wikipage $a_n$ in $P_{n}^{i}$. Given the distance between any two navigation paths, we defined the uniqueness of a successful navigation path $P_{n}^{i}$ of the $n$th participant in the game $g_i$ as its average distance to all the other successful navigation paths in the same game:

\begin{align}
    u_{n}^{i} &= \frac{1}{K_{i} - 1} \sum_{s_{k}^{i}=1, k \neq n}^{} D_{H} (P_{n}^{i}, P_{k}^{i})\\
    \Tilde{u}_{n}^{i} &= \frac{u_{n}^{i} - \mu}{\sigma}  
\end{align}

\noindent where $K_{i}$ is the total number of successful navigation paths in the $i$ th game. Standardizing the uniqueness of the navigation paths by the average uniqueness score $\mu$ and standard deviation $\sigma$ of all the successful navigation paths within the game, we get the standardized uniqueness $\Tilde{u}_{n}^{i}$ for the $n$th participant in the $i$th game.

\vspace{\baselineskip}\noindent\textbf{Quantifying uniqueness differences in subgroups}
To study if participants with certain characteristics are more or less likely to take unique navigation paths, we first identified the creative paths for each game as the successful navigation paths with the standardized uniqueness score $\Tilde{u}_{n}^{i}$ larger than 1, which consists of around 17\% of all the successful routes in each game. Using Hedges' g measure, we compared the average value of each individual characteristic in the creative group and the rest. Positive Hedges' g value implies that the value of the respective individual characteristic is higher in the creative group, and the larger the absolute g value, the stronger the effect. Statistics for eight games when it's played under time constraints (Speed-race game) and the number of steps constraints (Least-clicks game) are shown in Fig. \ref{fig:Uniqueness} on the left and right, respectively.

\vspace{\baselineskip}\noindent\textbf{Linear regression model}
We conducted a bivariate linear regression analysis between each individual characteristic and the navigation performance measured as the ratio of games won in total. Results of the bivariate regressions are shown in Supplementary Table S$1$. To further study if the significant navigation performance predictors are due to covariation with other variables, we started from the bivariate linear regression results and picked the characteristic that explains the largest amount of variation of the navigation performance (largest $R^{2}$) as the covariate. For each of the other individual characteristics, we again conducted a linear regression to predict the navigation performance, but this time with the chosen covariate. Repeating the covariate-choosing procedure, we expanded the covariates set by one variable at a time until no variable was significant ($p < 0.001$) in predicting the navigation performance anymore. Results of the regressions with covariates are shown in Supplementary Table S$2$-$5$. As the last step, we did a multivariate regression analysis of all the individual characteristics. We picked the category with the most predicting power for the three categorical variables: i) Asian ethnicity, ii) liberal political view, and iii) male as the representative of the respective characteristic. Results of the multivariate regression are shown in Table \ref{tab:RegMult}. For all the linear regressions, we adopted the ordinary least square (OLS) multivariate regression model:

\begin{equation}
    S_{n} = \beta \times C_{n} + \epsilon_{n}
\end{equation}

Where $S_{n}$ measures the navigation performance of the $n$th participant defined as the ratio of games won in total, $C_{n}$ denotes the individual characteristics of the $n$th participant, and $\epsilon_{n}$ is an error term.

\begin{table}[!htp]\centering
\caption{\textbf{Multivariate regression result. } The table shows the multivariate linear regression result with the ratio of games won as the dependent variable. For categorical independent variables, we did one hot encoding and picked the encoded feature that explained the largest amount of variance in the bivariate regression analysis. Note that all features are standardized with $M=0$ and $SD=1$. Significance thresholds: $^{*}p<0.05; ^{**}p<0.01; ^{***}p<0.001.$ }
\label{tab:RegMult}
\scriptsize
\begin{tabular}
{p{.2\textwidth}p{.1\textwidth}p{.15\textwidth}p{.1\textwidth}}\toprule
{} &   $\beta$ &               $95\% CI$ &            $p$ \\
                    &           &                         &                \\
\midrule
$Age$               &  $-0.069$ &  ($-0.099$ to $-0.038$) &  $0.000^{***}$ \\
$Male$              &   $0.005$ &   ($-0.023$ to $0.034$) &     $0.710^{}$ \\
$Asian$             &   $0.046$ &    ($0.017$ to $0.075$) &   $0.002^{**}$ \\
$Liberal$           &   $0.026$ &   ($-0.002$ to $0.055$) &     $0.068^{}$ \\
$Native$            &  $-0.022$ &   ($-0.052$ to $0.007$) &     $0.142^{}$ \\
$Foreign$           &   $0.064$ &    ($0.035$ to $0.093$) &  $0.000^{***}$ \\
$Agreeableness$     &  $-0.010$ &   ($-0.037$ to $0.017$) &     $0.481^{}$ \\
$Conscientiousness$ &   $0.031$ &    ($0.005$ to $0.058$) &    $0.021^{*}$ \\
$Extraversion$      &  $-0.019$ &   ($-0.046$ to $0.008$) &     $0.175^{}$ \\
$Neuroticism$       &   $0.007$ &   ($-0.020$ to $0.034$) &     $0.626^{}$ \\
$Openness$          &  $-0.005$ &   ($-0.032$ to $0.022$) &     $0.701^{}$ \\
$Wikigame$          &   $0.059$ &    ($0.030$ to $0.089$) &  $0.000^{***}$ \\
$Wikipedia_{1}$     &   $0.046$ &    ($0.017$ to $0.075$) &   $0.002^{**}$ \\
$Wikipedia_{2}$     &   $0.010$ &   ($-0.017$ to $0.037$) &     $0.456^{}$ \\
$Computer_{1}$      &  $-0.007$ &   ($-0.037$ to $0.023$) &     $0.647^{}$ \\
$Computer_{2}$      &  $-0.007$ &   ($-0.035$ to $0.020$) &     $0.610^{}$ \\
$Spatial_{1}$       &   $0.039$ &    ($0.011$ to $0.067$) &   $0.006^{**}$ \\
$Spatial_{2}$       &   $0.001$ &   ($-0.026$ to $0.028$) &     $0.937^{}$ \\
$Spatial_{3}$       &  $-0.010$ &   ($-0.037$ to $0.017$) &     $0.453^{}$ \\
$Spatial_{4}$       &   $0.008$ &   ($-0.019$ to $0.035$) &     $0.576^{}$ \\
$Education_{1}$     &  $-0.007$ &   ($-0.037$ to $0.023$) &     $0.653^{}$ \\
$Education_{2}$     &  $-0.003$ &   ($-0.030$ to $0.024$) &     $0.821^{}$ \\
$Employment_{1}$    &   $0.001$ &   ($-0.028$ to $0.030$) &     $0.945^{}$ \\
$Employment_{2}$    &  $-0.018$ &   ($-0.045$ to $0.009$) &     $0.194^{}$ \\
$Employment_{3}$    &  $-0.020$ &   ($-0.048$ to $0.007$) &     $0.144^{}$ \\
\multicolumn{4}{c}{ } \\
Observations & \multicolumn{3}{c}{404} \\
$R^{2}$/$R^{2}$ adjusted & \multicolumn{3}{c}{0.303/0.257} \\
\bottomrule
\end{tabular}
\end{table}

\begin{table}[!htp]
\centering
\caption{\textbf{Loadings of the principle components} The table shows the encoded variables (first column) and their respective loadings to the leading principal components in each category of questions preserving at least $80\%$ of the variance of the data in the category. }
\label{tab:Loadings}
\begin{tabular}{p{.2\textwidth}p{.15\textwidth}p{.15\textwidth}p{.15\textwidth}p{.15\textwidth}}\hline
\textbf{Variables} & \multicolumn{4}{c}{\textbf{Principle Components and Factor Loadings}}\\
\hline
 &$\mathbf{Wikigame_{1}}$ & & & \\
\hspace*{.5em}$WG_{prior}$ &1.00 & & & \\[.5cm]
 &$\mathbf{Wikipedia_{1}}$ &$\mathbf{Wikipedia_{2}}$ & & \\
\hspace*{.5em}$W_{purpose}$ &0.71 &0.71 & & \\
\hspace*{.5em}$W_{frequency}$ &0.71 &-0.71 & & \\[.5cm]
 &$\mathbf{Computer_{1}}$ &$\mathbf{Computer_{2}}$ & & \\
\hspace*{.5em}$C_{frequency}$ &0.57 &-0.61 & & \\
\hspace*{.5em}$C_{good}$ &0.54 &0.79 & & \\
\hspace*{.5em}$C_{like}$ &0.61 &-0.13 & & \\[.5cm]
 &$\mathbf{Spatial_{1}}$ &$\mathbf{Spatial_{2}}$ &$\mathbf{Spatial_{3}}$ &$\mathbf{Spatial_{4}}$ \\
\hspace*{.5em}$S_{good}$ &0.59 &-0.14 &-0.21 &-0.34 \\
\hspace*{.5em}$S_{learn}$ &0.44 &-0.49 &-0.31 &0.64 \\
\hspace*{.5em}$S_{unknown}$ &0.58 &0.20 &0.15 &-0.44 \\
\hspace*{.5em}$S_{known}$ &0.33 &0.64 &0.37 &0.54 \\
\hspace*{.5em}$S_{left}$ &0.06 &-0.54 &0.84 &-0.01 \\[.5cm]
 &$\mathbf{Education_{1}}$ &$\mathbf{Education_{2}}$ & & \\
\hspace*{.5em}$ED_{years}$ &0.71 &-0.71 & & \\
\hspace*{.5em}$ED_{highest}$ &0.71 &0.71 & & \\[.5cm]
 &$\mathbf{Employment_{1}}$ &$\mathbf{Employment_{2}}$ &$\mathbf{Employment_{3}}$ & \\
\hspace*{.5em}$EM_{status}$ &0.30 &-0.55 &0.77 & \\
\hspace*{.5em}$EM_{mental}$ &0.59 &-0.07 &-0.20 & \\
\hspace*{.5em}$EM_{physical}$ &0.08 &0.75 &0.57 & \\
\hspace*{.5em}$EM_{intensive}$ &0.54 &0.36 &-0.10 & \\
\hspace*{.5em}$EM_{creative}$ &0.51 &-0.10 &-0.21 & \\[.5cm]
\hline
\end{tabular}
\end{table}

\clearpage

\section*{Acknowledgements}

We are grateful to Csaba Pleh, Peter Kardos, and Markus Strohmaier for their valuable advice. This project was supported by the Humboldt Foundation within the Research Group Linkage Program. JK and MZ were partially supported through ERC grant No. 810115-DYNASET. JK acknowledges further support from EU H2020 ICT48 project "Humane AI Net", grant No. 952026, and Horizon 2020 "INFRAIA-01-2018-2019" project "SoBigData++", grant No. 871042.

\section*{Author contributions statement}

All authors contributed to the conception and design. MZ designed, led the experiment, and collected the data, all authors analyzed the data and wrote the paper.

\section*{Data availability}

Raw data for the online experiment has restricted access and can be provided upon consultation. Request for data should be directed to the corresponding authors.

\section*{Competing interests}

The authors declare no competing interests.

\section*{Ethics declarations}

All subjects gave their informed consent for inclusion before they participated in the study. The protocol of the study was approved by the Ethics Committee of Central European University (reference number: 2022-2023/1/EX). All methods of the study were carried out in accordance with the principles of the Belmont Report. 

\section*{Supplementary information}
The supplementary materials are included.


\begin{thebibliography}{31}
\bibitem{savolainen2015cognitive} Savolainen, Reijo. "Cognitive barriers to information seeking: A conceptual analysis." Journal of Information Science 41.5 (2015): 613-623.

\bibitem{dutton2014cultures} Dutton, William H., and Grant Blank. "Cultures of the internet: Five clusters of attitudes and beliefs among users in Britain." (2014).

\bibitem{van2014digital} Van Deursen, Alexander JAM, and Jan AGM Van Dijk. "The digital divide shifts to differences in usage." New media \& society 16.3 (2014): 507-526.

\bibitem{ho2005exploratory} Ho, Shuk Ying. "An exploratory study of using a user remote tracker to examine web users' personality traits." Proceedings of the 7th international conference on Electronic commerce. 2005.

\bibitem{gill2015characterizing} Gill, Phillipa, et al. "Characterizing web censorship worldwide: Another look at the opennet initiative data." ACM Transactions on the Web (TWEB) 9.1 (2015): 1-29.

\bibitem{tavares2015map} Tavares, Rita Morais, et al. "A map for social navigation in the human brain." Neuron 87.1 (2015): 231-243.

\bibitem{olafsdottir2015hippocampal} Ólafsdóttir, H. Freyja, et al. "Hippocampal place cells construct reward related sequences through unexplored space." Elife 4 (2015): e06063.

\bibitem{anguera2013video} Anguera, Joaquin A., et al. "Video game training enhances cognitive control in older adults." Nature 501.7465 (2013): 97-101.

\bibitem{newcombe2018individual} Newcombe, Nora S. "Individual variation in human navigation." Current Biology 28.17 (2018): R1004-R1008.

\bibitem{nazareth2019meta} Nazareth, Alina, et al. "A meta-analysis of sex differences in human navigation skills." Psychonomic bulletin \& review 26.5 (2019): 1503-1528.

\bibitem{hunsaker2018review} Hunsaker, Amanda, and Eszter Hargittai. "A review of Internet use among older adults." New media \& society 20.10 (2018): 3937-3954.

\bibitem{coutrot2022entropy} Coutrot, Antoine, et al. "Entropy of city street networks linked to future spatial navigation ability." Nature 604.7904 (2022): 104-110.

\bibitem{sharit2008investigating} Sharit, Joseph, et al. "Investigating the roles of knowledge and cognitive abilities in older adult information seeking on the web." ACM Transactions on Computer-Human Interaction (TOCHI) 15.1 (2008): 1-25.

\bibitem{chevalier2015strategy} Chevalier, Aline, Aurélie Dommes, and Jean-Claude Marquié. "Strategy and accuracy during information search on the Web: Effects of age and complexity of the search questions." Computers in human behavior 53 (2015): 305-315.

\bibitem{mcdonald2000gender} McDonald, Sharon, and Linda Spencer. "Gender differences in web navigation." Women, Work and Computerization. Springer, Boston, MA, 2000. 174-181.

\bibitem{zlatic2010knowledge} Zlatic, Vinko. "Knowledge Networks: The Case Of Wikipedia", in: Encyclopedia of Life Support Systems (EOLSS): Complex Networks, Chapter 6. Ed: Guido Caldarelli. 2010, UNESCO Eolss Publishers, Oxford.

\bibitem{kleinberg2000navigation} Kleinberg J M. Navigation in a small world[J]. Nature, 2000, 406(6798): 845-845.

\bibitem{yasseri2012dynamics} Yasseri, Taha, et al. "Dynamics of conflicts in Wikipedia." PloS one 7.6 (2012): e38869.

\bibitem{yasseri2012practical} Yasseri, Taha, András Kornai, and János Kertész. "A practical approach to language complexity: a Wikipedia case study." PloS one 7.11 (2012): e48386.

\bibitem{lamprecht2017structure} Lamprecht, Daniel, et al. "How the structure of Wikipedia articles influences user navigation." New Review of Hypermedia and Multimedia 23.1 (2017): 29-50.

\bibitem{arora2022wikipedia} Arora, Akhil, et al. "Wikipedia reader navigation: When synthetic data is enough." Proceedings of the Fifteenth ACM International Conference on Web Search and Data Mining. 2022.

\bibitem{west2012human} West, Robert, and Jure Leskovec. "Human wayfinding in information networks." Proceedings of the 21st international conference on World Wide Web. 2012.

\bibitem{singer2014detecting} Singer, Philipp, et al. "Detecting memory and structure in human navigation patterns using Markov chain models of varying order." PloS one 9.7 (2014): e102070.

\bibitem{cobb2012stability} Cobb-Clark, Deborah A., and Stefanie Schurer. "The stability of big-five personality traits." Economics Letters 115.1 (2012): 11-15.

\bibitem{shustermana2011cognitive} Shusterman A, Lee S A, Spelke E S. Cognitive effects of language on human navigation[J]. Cognition, 2011, 120(2): 186-201.

\bibitem{epstein2017cognitive} Epstein, Russell A., et al. "The cognitive map in humans: spatial navigation and beyond." Nature neuroscience 20.11 (2017): 1504-1513.

\bibitem{runco2012standard} Runco M A, Jaeger G J. The standard definition of creativity[J]. Creativity research journal, 2012, 24(1): 92-96.

\bibitem{campbell2022partial} Campbell, Chelsea M., Eduardo J. Izquierdo, and Robert L. Goldstone. "Partial copying and the role of diversity in social learning performance." Collective Intelligence 1.1 (2022): 26339137221081849.

\bibitem{spiers2021explaining} Spiers, Hugo J., Antoine Coutrot, and Michael Hornberger. "Explaining World‐Wide Variation in Navigation Ability from Millions of People: Citizen Science Project Sea Hero Quest." Topics in cognitive science (2021).

\bibitem{dunteman1989principal} Dunteman, George H. Principal components analysis. No. 69. Sage, 1989.

\bibitem{perozzi2014deepwalk} Perozzi, Bryan, Rami Al-Rfou, and Steven Skiena. "Deepwalk: Online learning of social representations." Proceedings of the 20th ACM SIGKDD international conference on Knowledge discovery and data mining. 2014.

\bibitem{rockafellar2009variational} Rockafellar, R. Tyrrell, and Roger J-B. Wets. Variational analysis. Vol. 317. Springer Science \& Business Media, 2009.


\end{thebibliography}

\end{document}