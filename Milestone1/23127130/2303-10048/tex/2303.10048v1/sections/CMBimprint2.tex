\section{Voids CMB lensing cross-correlations}\label{sec:CMBxvoids}

Similarly to the correlation that have been observed between the CMB radiation and the overdensities identified in the foreground matter field \cite{baxter2015,Madhavacheril2015,planck2016,geach2017,baxter2018}, cosmic voids show, due to their underdensed nature, an anti-correlation (or a de-magnification) imprint in the lensing signal of the CMB. We will take advantage of the so-called stacking methodology \cite{krause2013,davies2018,higuchi2013} to reach high accuracy and a detectable signal-to-noise level, as the signal detected at a single void position can be noise-dominated \cite{krause2013}. This correlation between the CMB and voids has already been measured both in simulated and real catalogues \cite{cai,vielzeuf2019,Raghunathan2019}, and recent analysis have also shown a moderate discrepancy between observations and massless neutrinos $\Lambda$CDM simulations ($\sim 2-3\sigma$ lower signal in the observed data) \cite{hang2021,kovacs2022}. Furthermore, if confirmed, such a discrepancy could be related to the other tensions such as the correlation of super-voids and the CMB/ISW signal in simulated massless neutrinos $\Lambda$CDM mocks and observations \cite{granett2008,cai,Y1_ISW,Y3_ISW}.  
Our aim is to verify if this correlation signal can be affected by the neutrino mass, and if the presence of massive neutrinos could change the correlation signal itself, in the same direction as recently suggested by the aforementioned tensions.
Beside that, it is also important to stress that one of the powerful feature of the lensing signal comes from the fact that it is not directly influenced by the bias of matter tracers, since it is directly related to the true matter distribution inside the identified voids. Furthermore, in terms of cosmological probes, CMB lensing correlations with foreground objects also presents advantages with respect to background galaxy lensing: indeed as mentioned before, the peak of the CMB lensing kernel is around $z\sim 1.5$ and offers the opportunity to explore a wider and higher range of redshifts with respect to the lensing of background galaxies (see Fig.1 of \cite{farbod2016} for a comparison of the different lensing kernels). These correlations also present a particular interest in the sense that they are not affected by the main systematic errors that have to be considered when one estimate the lensing of background galaxy (such as intrinsic alignments or shear bias). 

\subsection{Imprint in CMB lensing map and stacking methodology}\label{sec:stacking_meth}

We construct a new estimator for the projected void density, in line with the previous definition of CMB convergence (Eq.~\ref{eq:kappa}). In fact, by looking at the position of a single underdense regions aligned with the CMB lensing reconstructed map, we would expect a negative convergence or a "de-lensing" signal, and this signal will be directly related to the underlying matter field, and thus to the projected void density profile. The convergence signal can then be expressed as a function of the projected void density in the angular direction $\boldsymbol\theta$:
\begin{equation}
    \kappa_{\rm v}(\boldsymbol\theta)=\Sigma_{\rm v}(\boldsymbol\theta)/\Sigma_{\rm crit};
\end{equation}
with the critical surface density for voids being defined as:
\begin{equation}
    \Sigma_{\rm crit}=\frac{c^2}{4\pi G}\frac{f_{\rm CMB}}{f_{\rm voids}f_{\rm CMB-voids}} ,
 \end{equation}
where $f_{\rm CMB}$, $f_{\rm voids}$ and $f_{\rm CMB-voids}$ are respectively the angular diameter distances to the CMB, to the considered void and between the CMB and the void. $\Sigma_{\rm v}(\boldsymbol\theta)$ represents the projected underlying distribution of matter as a function of the void centre, different from the density profile presented in Section~\ref{sec:dens} where we measured the density profile of the voids using dark matter haloes, which are biased objects of the underlying matter field.

In the literature, different stacking methods \cite{nadathur2016,Raghunathan2019} have been proposed;
in the context of this work, we decided to smooth the CMB with a Gaussian kernel with a Full Width at Half Maximum (FWHM) of 1 degree that has shown to be a good compromise to optimise the detection level \cite{vielzeuf2019}. To estimate the stacked correlation profile, we applied the same methodology presented in \cite{vielzeuf2019}, i.e. after cutting patches of 5 times the void radius ($R_{\rm v}$) in the smoothed CMB \healpix{} maps centreed at the position of the voids in our catalogues, we re-scaled the patches in order to have regions of similar sizes, and stacked them pixel by pixel. Once the stacked image is computed, it is possible to reconstruct the averaged convergence profile of the voids by averaging the pixel value in concentric shells around the image centre. Furthermore, \cite{nadathur2017} has shown using simulations that these imprints change according to the void population. In fact, the lensing imprint is directly related to the true underlying gravitational potential, and the inner density of cosmic voids - which depends on size and voids definition \cite{Hamaus2014,dai2015} - will impact on the strength of this correlation signal.





In order to estimate the error in our measurements, similarly to \cite{vielzeuf2019}, we will generate $1,000$ random CMB lensing maps using the \texttt{synfast} and \texttt{anafast} module from \healpix{} package, with the same power spectrum as the original DEMNUni map presented in Figure~\ref{fig:cls_kcmb_dem}. We then can compute the covariance of the cross-correlation signal using the stacking method on the void catalogues and these CMB lensing maps. We do not consider additional noise due to observational systematics, since our goal is to disentangle the role of neutrinos in cross-correlations from a pure physical point of view; we leave a more realistic measurement for forthcoming analysis. Nevertherless, we also verified the importance of non-gaussian terms in the noise of the CMB lensing reconstruction by building the covariance applying 1000 random rotations to our void catalogue and stacked them. The resulting errors have shown to be consistent with the Gaussian noise realisations method described above.




\subsection{Measurement of the voids-CMB lensing cross-correlation signal in the DEMNUni simulations}

As mentioned before, one of the advantage of imprints of structures on the CMB lensing field resides in the fact that the resulting signal is directly dependant on the underlying matter field. Once one includes massive neutrino to the cosmological model, imprints of the different matter can be affected in two ways:
\begin{itemize}
    \item the lensing imprint due to neutrinos alone, which changes the overall lensing signal amplitude due to their mass. In fact, we are adding a non-relativistic component to the matter field, an additional particle that will enhance lensing to the background radiation. One expects this effect to occurs at scales larger than the free-streaming scale, i.e. at scales where neutrinos fall in potential wells.
    \item The change in the lensing signal due to the slowing-down of clustering of cold dark matter particles, caused by the presence of massive neutrinos.
\end{itemize}
In this section we will explore both effects for the different neutrino masses recipes.

\subsubsection{Neutrino contribution to the correlation signal}\label{sec:neutr_contr}

First, we investigate the contribution of massive neutrinos themselves to the correlation signals. In other words, we look at the correlation signals of the different elements (CDM and massive neutrinos) separately for the different parametrizations of the void finder. We have applied our stacking methodology presented in Section~\ref{sec:stacking_meth} to both lensing maps: the lensing signal of CDM-only and the CMB lensing map due to massive neutrino only, for the void catalogue identified in the DEMNUNi simulation with $m_\nu=0.53$eV.

In Figure~\ref{fig:neutrinocontrib_im}, we show the stacked images of the two lensing imprints, CDM-only (top panels) and massive neutrinos-only (bottom panels) for the three smoothing scales considered (from left to right). As expected, we observe a negative imprint in the stacked lensing signal at the void positions for both the cold dark matter and neutrino field: voids are in fact underdense regions in both the fields. We note that, since neutrino density perturbations are much lower than the CDM ones, their imprint on the CMB lensing map is smaller than for CDM. In the bottom panel of Figure~\ref{fig:neutrinocontrib_im}, we show the corresponding lensing profiles as a function of the distance to the void center (normalized with the void radius $R_{\rm v}$) for the contribution of massive neutrinos (dashed lines) w.r.t. CDM (solid lines, and for the different smoothing scales considered (10, 20 and 30 $h^{-1}$Mpc). The neutrino imprint signal has been multiplied by a factor of 10 for visualisation in the same Figure as the cold dark matter one. The ratio of massive neutrinos to CDM as a function of distance to the void center ($\kappa_{m_\nu}/\kappa_{\rm CDM}$) is shown in the insight panel of Figure~\ref{fig:neutrinocontrib_im}. As expected, the neutrino-only contribution to the total void-CMB lensing cross-correlation represents few percents of the total signal. The figure also shows that as we increase the smoothing parameter of the void finder, the relative contributions of both neutrinos and CDM to the signal tends to increase. This is consistent with \cite{vielzeuf2019}, and can been seen as the consequence of a selection effect induced by the smoothing process in the void identification. Indeed, the smoothing kernel applied to the density field will force the algorithm to neglect structure with scales below the smoothing parameter. These structure will anyway lie inside our voids, thus including more nonlinear modes inside the voids, and these nonlinear fluctuations will boost the amplitude of the correlation signal making the voids deeper as the smoothing scale increases. 
We stress that this behaviour seems to be opposite to the one of the two point correlation function shown in Figure~\ref{fig:dens_prof_20}. However, this is only partially due to the difference between the density profiles measured in the halo-traced void field and CDM-traced void field, respectively, while the main effect comes from the smoothing length of our void-finder.
Moreover, while in Figure~\ref{fig:dens_prof_20} the profile of the halo-traced voids becomes slightly shallower as we increase the smoothing length, the void profile measured in the lensing-convergence field becomes deeper.
A similar trend can be observed also the neutrino-convergence void-profile, meaning that similarly to CDM, neutrinos seems to be less present in voids identified using large smoothing scales, and thus we can measure a stronger void-lensing signal if we increase the smoothing scale. 
However, if we look at the insight plot, we see that as we increase the smoothing scale, the ratio of the signals from massive neutrinos and CDM increases. This implies that, by increasing the smoothing scale, the abundance of massive neutrinos inside the voids decreases faster than the CDM abundance. This could be explained by a possible reduction of the massive neutrino abundance at small scales, w.r.t the CDM one, but higher resolution simulations would be required to confirm such results.
\begin{figure}[htbp]
\begin{center}
\hspace*{-1cm}\includegraphics[keepaspectratio,height=20cm, width=20cm]{./figs/IMAGE_SEPARATE_EFFEC2.pdf}
\includegraphics[keepaspectratio,height=8.5cm, width=8.5cm]{./figs/separate_profile_053_fwhm_err.pdf}

\caption{{\it Top panel:} Stacked image of the imprint of the different void catalogues on CMB lensing, CDM-only component (top row) and massive neutrino only component with $m_\nu=0.53$ eV (bottom row). {\it Bottom panel:} Contribution of neutrino in the stacking imprint of cosmic voids (dash line) compared to the CDM contribution (solid line) for the three different smoothing scale considered. Note that the amplitude of the neutrino contribution on this figure has been amplified by a factor of 10. The shaded regions represent the error computed via 1,000 realisations of CMB lensing maps as explained in Section~\ref{sec:stacking_meth}.} 
\label{fig:neutrinocontrib_im}
\end{center}
\end{figure}

Moreover, as shown in Figure~\ref{fig:FS}, massive neutrinos will fall in potential wells of different sizes and this will be redshift dependent. In Appendix \ref{sec:appendix_A} we have tested such a behaviour by binning our sample in different radius bins and redshifts. Even though the redshift evolution of the separate signals of massive neutrino and CDM is difficult to observe, we could detect a variation of these two contributions once we consider different void sizes, suggesting, in agreement with theoretical predictions, a stronger presence of massive neutrinos in smaller voids.


\subsubsection{Cross-correlations CMBL and voids}
In this section, we analyse the full cross-correlation signals (from neutrino plus CDM particles) obtained for all voids catalogues and the different neutrino masses presented before. Figure~\ref{fig:FULL_SIGNAL} shows the correlation profile for the different void catalogues. Similarly to previous results, the strength of the lensing imprint on the CMB caused by cosmic voids will be larger as we increase the smoothing parameter in the void finder. Moreover, the differences in this signal due to massive neutrinos appear larger for larger smoothing scales. As expected, as we increase the neutrino mass, the lensing amplitude at the void center decreases, implying a slowing down of matter density perturbations caused by free-streaming neutrinos. While we can observe differences between the different cases inside the voids, moving further outside from the void centre all signal seems to converge. 
\begin{figure*}
    \centering
    \includegraphics[width=50mm]{figs/FINAL_plot_10Full.pdf}
    \includegraphics[width=50mm]{figs/FINAL_plot_20Full.pdf}
    \includegraphics[width=50mm]{figs/FINAL_plot_30Full.pdf}
    \caption{Imprint of cosmic voids for different massive neutrino cosmologies, combined CDM and neutrinos. From left to right, three smoothing scales: 10 $h^{-1}$Mpc, 20 $h^{-1}$Mpc, 30 $h^{-1}$Mpc .}
    \label{fig:FULL_SIGNAL}
\end{figure*}
We want to quantify the sensitivity level of each measurement: to this purpose we consider the CDM+$\nu$ lensing-convergence within the void region where it varies the most ($R<R_{\rm v}/2$), and compute its ratio for different neutrino masses. In this way we define $\delta\kappa_{\rm in}$, i.e. a sensitivity parameter to the neutrino mass of the lensing-covergence void profile :
\begin{equation}\label{eq:sens_param}
    \delta\kappa_{in}=\frac{\sum_0^{r<R_{\rm v}/2}\kappa_{m_\nu=0.16{\rm eV},0.32 {\rm eV},0.53{\rm eV}}}{\sum_0^{r<R_{\rm v}/2}\kappa_{\Lambda {\rm CDM}}},
\end{equation}
where $\delta\kappa_{in}$ stands for the amplitude ratio of the signal with and without massive neutrinos in the inner region of the void ($R<R_{\rm v}/2$)\footnote{Note that this amplitude parameter is similar to the lensing amplitude parameter $A_\kappa=\kappa_{\rm obs}/\kappa_{\rm sims}$ used in the literature (see e.g. \cite{vielzeuf2019,kovacs2022}) to evaluate the agreement between the observed correlation signal, $\kappa_{\rm obs}$, and the one measured in $\Lambda$CDM simulations, $\kappa_{\rm sim}$.}. Figure~\ref{fig:FULL_sensiv} shows the sensitivity parameter of Equation~\ref{eq:sens_param} as a function of the smoothing scale of the void finder. As already observed in Figure~\ref{fig:FULL_SIGNAL}, the increase in the smoothing scale in the void finder results in a boost of the intensity of the cross-correlation signal amplitude, and this boost seems to be dependant on the mass of the neutrinos present in the simulations. In other words, as we increase the smoothing scale of the void finder, we measure a larger difference in the correlation signal of massive neutrino simulations with respect to the massless neutrino $\Lambda$CDM cosmology. The errorbars in Figure~\ref{fig:FULL_sensiv} have been estimated by propagating the errors of our stacking measurement described in Section~\ref{sec:stacking_meth}, thus not considering any extra systematic errors. 
\begin{figure}
    \centering
    \includegraphics[width=.5\columnwidth]{figs/FINAL_sensiv_Full.pdf}
    \caption{Sensitivity parameter $\delta\kappa_{in}$ - Eq.~\eqref{eq:sens_param} - for different massive neutrino masses as a function of the void finder smoothing scale.}
    \label{fig:FULL_sensiv}
\end{figure}
The measure of this reduction in the lensing signal inside cosmic voids due to the presence of massive neutrinos is in particular interesting as it is consistent with the tensions in the recently observed lensing signal and massless neutrinos $\Lambda$CDM simulations \cite{hang2021,kovacs2022}. Namely, in both analysis voids have been identified using the 2D void finder described previously with a smoothing scale of $20 h^{-1}$Mpc and $10 h^{-1}$Mpc respectively, resulting in a observed signal of the correlation of cosmic voids with the Planck 2018 lensing convergence map \cite{planck2018} about $2\sigma$ lower than the one measured in massless neutrinos $\Lambda$CDM simulation without massive neutrinos. The direction of this tension is thus in line with the decrease of the lensing imprint of cosmic voids caused by the presence of massive neutrinos in our simulations.

\subparagraph{Void redshift evolution:}

We then divide our void catalogues in different redshift bins from $z=0.2$ to $z=2$ and apply our stacking methodology to each bin, combining both CDM and neutrino maps. We show in the top panel of Figure~\ref{fig:zevol} the profiles measured for the different smoothing scales, while in the bottom panels we show $\delta\kappa_{in}$ as a function of redshift for the different massive neutrino cosmologies and the different smoothing scales. 
At low redshifts, although neutrinos will fall in large potential wells, we expect the smaller fluctuations to be smoothed. Namely, from Figure~\ref{fig:FS} we can see that as the redshift decreases, smaller scales will be affected. On the other hand, as we increase the redshift, the difference in the void population is also increasing when neutrinos are more massive (see Figure~\ref{fig:radius_dist}); the amount of small voids will be larger in the massless neutrino simulations with respect to the massive ones. These small structures would be more smoothed in their centre due to their size, on similar scales for which massive neutrinos will smooth the matter field. However, in the medium redshift range, the sensitivity parameter decreases with the neutrino mass. This is consistent with the slight tension claimed by \cite{hang2021}, where a lower signal in the observation appears in the DESI Legacy survey observation at $0.6 < z < 0.8$.
\begin{figure*}

\centering

\includegraphics[width=.32\textwidth]{./figs/FINAL_plot_10binz.pdf}
\includegraphics[width=.32\textwidth]{./figs/FINAL_plot_20binz.pdf}
\includegraphics[width=.32\textwidth]{./figs/FINAL_plot_30binz.pdf}

\includegraphics[width=.32\textwidth]{./figs/FINAL_sensiv_10binz.pdf}
\includegraphics[width=.32\textwidth]{./figs/FINAL_sensiv_20binz.pdf}
\includegraphics[width=.32\textwidth]{./figs/FINAL_sensiv_30binz.pdf}


\caption{{\it Top panel:} Redshift evolution of the lensing profile and imprint on CMB convergence map. All voids catalogues considered in this work are divided in different redshift bins for the different massive neutrinos cosmologies analysed here, $m_\nu=$0 eV (blue), $m_\nu=$0.16 eV (red), $m_\nu=$0.32 eV (green) and $m_\nu=$0.53 eV (yellow). {\it Bottom panel:} corresponding sensitivity parameter in  different redshift bin. From left to right, a different smoothing scale (10, 20, 30 $h^{-1}$Mpc) is considered for the void finder. }
\label{fig:zevol}
\end{figure*}




\subparagraph{Void radius evolution:}
Previously, we have measured that medium redshift ranges are showing more differences in void lensing imprints (Figure~\ref{fig:zevol}). In addition to that, it is also possible to prune the void catalogues in order to select voids that show a stronger lensing imprint. Moreover, as explained before and confirmed above, we expect the smaller voids to be more affected by the presence of massive neutrino since neutrino free-streaming will reduce the clustering at the scales corresponding to their sizes, making them less underdensed. Consequently, similarly to what we did in Sect.~\ref{sec:neutr_contr}, we split the void catalogues into different radius bins and measure the stacked lensing signal of all the sub-samples, separately.


Figure~\ref{fig:Revol.pdf} shows the different void CMB lensing profiles (top panels) and the evolution of the sensitivity parameter as a function of the void radius (bottom panels), for the different neutrino masses and smoothing scales considered. Smaller voids have less pronounced lensing signal than medium radius voids, and the relative difference in the massive neutrino cosmologies seems to be enhanced as the object radius decreases and the smoothing scale increases. However, we note that - in particular for the lowest smoothing scale of 10 $h^{-1}$Mpc - the neutrino imprint seems to decrease; this maybe related to the fact that for small smoothing scales matter underdensities within the voids are less nonlinear, that is with smaller amplitude, and they become more linear as the neutrino mass increases.


\begin{figure*}

\includegraphics[width=.32\textwidth]{./figs/FINAL_plot_10binR.pdf}
\includegraphics[width=.32\textwidth]{./figs/FINAL_plot_20binR.pdf}
\includegraphics[width=.32\textwidth]{./figs/FINAL_plot_30binR.pdf}\quad

\includegraphics[width=.32\textwidth]{./figs/FINAL_sensiv_10binR.pdf}
\includegraphics[width=.32\textwidth]{./figs/FINAL_sensiv_20binR.pdf}
\includegraphics[width=.32\textwidth]{./figs/FINAL_sensiv_30binR.pdf}


\caption{Radius evolution of the void imprint on the CMB convergence map and lensing profile. All voids catalogues considered in this work are divided in different radius bins for the different massive neutrinos cosmologies; same description as Figure~\ref{fig:zevol}.}
\label{fig:Revol.pdf}
\end{figure*}

Finally, we went one step further in Appendix \ref{sec:appendix_B} where we have reiterated the correlation measurement applying a double binning in void radius and redshift.





