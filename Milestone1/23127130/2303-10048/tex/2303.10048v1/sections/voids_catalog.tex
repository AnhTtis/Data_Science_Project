\subsection{The void catalogue analysis}\label{sec:void_cat}

First, we investigate the properties of the void catalogues in order to understand the effect of massive neutrinos on voids.
In this section, we look at the the signature of massive neutrinos for different cosmic void parameters: i) the total void number, ii) void size function and iii) void density profile. The aim is to quantify how the void population changes when we add massive neutrino to the cosmic recipe.
We have then identified cosmic voids using the 2D void finder presented in Section~\ref{sec:void_finding} with three different smoothing parameters, $10 h^{-1}$Mpc, $20 h^{-1}$Mpc and $30 h^{-1}$Mpc using the DEMNUnii DM halo catalogues as tracers of the matter field, for the different $\Lambda$CDM cosmologies with and without massive neutrinos.



\subsubsection{Total void number and size}\label{sec:vsf}
Void statistics - and specifically the number of identified voids with a given size - has been exploited to constrain the cosmological model. In particular, the distribution of underdensed region with respect to their size have shown to be particularly sensitive to dynamical cosmological models~\citep{verza_2019}, and modified gravity models~\citep{barreira2015,Li2012,cai2015,Voivodic2017,Contarini2020}. Moreover, recent efforts measured the void abundance using a spherical model and the excursion principle, to perform cosmological analysis and set constraints on parameters~\cite{ronconi2017,contarini2019,contarini2022} and the fisrt cosmological constraint from void abundance have been recently performed using the BOSS DR12 observations~\cite{contarini2022boss}. Here, we investigate how the presence of massive neutrinos can impact void features and statistics, rather than modeling this effect.
Intuitively, one expects the presence of massive neutrinos to increase the total number of identified voids in the underlying matter field. In fact, massive neutrinos slowing down clustering process will in turn slow down void merging as well, resulting in a larger number of small voids. This has been observed in previous analysis~\cite{Kreisch2019,Schuster2019}, where the authors - using \texttt{VIDE}'s 3D voids identifier on the CDM field of the DEMNUni N-body simulation - have identified more voids while they were considering higher neutrino masses. However, \cite{Kreisch2019,Schuster2019} have shown that changing the matter tracer from CDM particles to the CDM halo field can in fact invert this trend. This can be explain by the fact that the massive neutrinos - as they are slowing down the clustering processes - modify as well the overall tracer density once we employ haloes as matter tracers (see Figure~\ref{fig:halo_mass_dist})\footnote{We note that this is not the case for CDM particles for which the overall density of particles will remain constant.}. This can directly impact the resulting void catalogues, since the sparsity of the tracer used to identify cosmic voids impacts both the void abundance and size~\cite{sutter2014}. Moreover, the direction of the change in the density of the tracers depends on the resolution of our simulation: higher resolution simulation will include smaller haloes, in even greater numbers once one accounts for massive neutrinos, resulting in a denser tracer sample. Lower resolution simulations, instead, will reach a turn-over point in the number of smaller haloes identified. In this work, we consider a void-tracer population with minimum mass $M_h\simeq 2.5\times 10^{12}h^{-1}M_\odot$, i.e. the minimum halo mass of the DEMNUni FoF catalogues, whose mass function, and therefore total density, have been shown to decrease due to the free streaming and growth suppression by massive neutrinos.


In Table~\ref{tab:voidnumber} we show the total number of voids identified for each smoothing parameter, and their ratio w.r.t. the massless neutrino case. First, we note that differences in the smoothing scale parameter of the void finder will impact the void properties: the smaller the smoothing scale, the larger the number of small voids (see~\cite{vielzeuf2019} and Figure~\ref{fig:central_dens_radii}). Then, in Table~\ref{tab:voidnumber} we measure a decrease in the total number of voids traced by haloes with mass $M_h>2.5\times 10^{12} h^{-1} M_\odot$ while the neutrino mass increases for the $10 h^{-1}$Mpc and $20 h^{-1}$Mpc smoothing parameter; this is consistent to what have been observed in~\citep{massara2015}. This effect decreases while we use higher smoothing scales, and it is, in fact, the opposite when we consider a $30 h^{-1}$Mpc smoothing scale. 

\begin{table}

 \centering   
\begin{tabular}{|l|r|r|r|}
\hline
\diaghead{\theadfont Diag ColumnmnHead II}%
{}{smoothing scale} & \text{$10 h^{-1}$ Mpc} & \text{$20 h^{-1}$ Mpc} & \text{$30 h^{-1}$ Mpc} \\ \hline
\text{$n_{\Lambda{\rm CDM}} $}                                                                        & 144,594             & 68,221              & 30,055              \\ \hline
\text{$n_{\Lambda{\rm CDM} + m_\nu=0.16{\rm eV}}$}                                                          & 129,957             & 65,563              & 30,767              \\ \hline
\text{$n_{\Lambda{\rm CDM} + m_\nu=0.32{\rm eV}}$}                                                          & 114,046             & 61,945             & 30,986              \\ \hline
\text{$n_{\Lambda{\rm CDM} + m_\nu=0.53{\rm eV}}$ }                                                                  & 98,658              & 58,016              & 30,814              \\ \hline
\text{$n_{\Lambda{\rm CDM} + m_\nu=0.16{\rm eV}}/n_{\Lambda{\rm CDM}}$}                                      & 0.90               & 0.96      & 1.02               \\ \hline
\text{$n_{\Lambda{\rm CDM} + m_\nu=0.32{\rm eV}}/n_{\Lambda{\rm CDM}}$}                                      & 0.80               & 0.91      & 1.03               \\ \hline
\text{$n_{\Lambda{\rm CDM} + m_\nu=0.53 {\rm eV}}/n_{\Lambda{\rm CDM}}$}
& 0.68               & 0.85               & 1.02              \\ \hline
\end{tabular}
\caption{Total number of cosmic voids (and ratio w.r.t. the massless neutrino $\Lambda$CDM case, bottom rows) traced by haloes with mass $M_h > 2.5\times 10^{12} h^{-1} M_\odot$, found in the DEMNUni simulations for different void finder smoothing scales and different  neutrino masses.}\label{tab:voidnumber}

\end{table}


The top panels of Figure~\ref{fig:radius_dist} show the 2D void abundances as a function of void radius for both massless neutrinos $\Lambda$CDM and $\Lambda{\rm CDM} + m_\nu$ simulations in various redshift bins, while the bottom panels show the ratio with respect to the massless neutrino cosmology. Within our definition of 2D voids, DM halo tracers show a drop in the number of small voids ($R_{\rm v}<50 h^{-1}$Mpc) if the neutrino are massive particles, and that this effect is even more pronounced for higher redshifts, which is consistent with the fact that at higher redshifts the range of scales affected by massive neutrino is larger than at lower ones (cf. Figure~\ref{fig:FS}). Moreover, similarly to Table~\ref{tab:voidnumber}, the choice of the smoothing scale parameter of the finder is related on how massive neutrinos are affecting the void size function. In fact, as we increase the smoothing parameter the effect of massive neutrino in the size function decreases: this effect could be explained since, as we increase the smoothing scale the finder tends to merge small voids into larger structures, and thus shifting the size function towards large voids, i.e. towards scales where neutrinos become non-relativistic, fall in potential wells, hence avoiding underdensed regions.


\begin{figure*}[htbp]

\centering

\includegraphics[width=\columnwidth]{figs/VSF_FULL_Mpch.pdf}

\caption{{\it Top panel}: abundances of 2D voids at different redshifts, as a function of void radius in Mpc/$h$, for massless (solid line) and massive neutrino (dashed line) simulations. From left to right,  the different smoothing scales considered. {\it Bottom panels}: ratio of the void abundances in neutrino cosmologies w.r.t. the massless neutrinos $\Lambda$CDM case.} 
\label{fig:radius_dist}
\end{figure*}




\subsubsection{Void density profiles}\label{sec:dens}


\begin{figure*}%[ht]
\centering

\includegraphics[width=1\columnwidth]{./figs/dens_10_loopz.pdf}
\includegraphics[width=1\columnwidth]{./figs/dens_20_loopz.pdf}
\includegraphics[width=1\columnwidth]{./figs/dens_30_loopz.pdf}
\caption{Void-halo two point correlation functions, 2PCF. As in the void density profiles, each row shows a different smoothing scale, while each column a different redshift bin. The bottom panels in each box is the ratio between the massive neutrino 2PCF w.r.t. the massless neutrinos $\Lambda$CDM case.}
\label{fig:dens_prof_20}
\end{figure*}



The general density profile of cosmic voids has already been studied in several works and different models have been proposed in the literature~\citep{Ricciardelli2014,Hamaus2014,barreira2015}. However, these studies have highlighted the complexity of finding a general definition for this profile, due to the fact that it depends on the void definition itself (e.g. the choice of tracers of the matter field, void finder, smoothing scales, etc.). Nevertheless, in all these studies, cosmic voids can be described as underdensed regions at the void centre, surrounded by a more or less pronounced positive density shell at the void's edge: the so-called compensation wall. This wall is associated to filaments, while the depth of the central region and the size of the compensation wall will depend on the size of the considered objects. In this work, we will quantify how the presence of massive neutrinos can affect the density profile of the voids in the halo field. In fact, the density profile of cosmic voids, due to their scales, have shown to be  particularly sensitive to massive neutrino. In particular, from what have been observed in~\cite{Zhang2019} in simulations, the halo mass function inside voids is more affected by the presence of massive neutrinos than in other regions in the sky. This implies that the clustering process inside underdense regions, and consequently the void density profile, will be different while considering massive neutrinos in the cosmic budget.
In~\cite{massara2015}, by looking at the CDM density profile of 3D cosmic voids of a given size, they showed how the presence of massive neutrino will smooth the density profile by decreasing the size of the compensation wall and by making the void less empty at the void centre, and that this effect is more significant at low redshifts than higher ones. These effects are direct consequences of the slowing-down of clustering due to massive neutrinos.
However, it is also interesting to note that in~\cite{Contarini2020} the authors have shown that effects on the density profile of 3D voids in modified gravity models tend to be cancelled out when neutrinos are massive. 


In the context of this work, our void definition allows us to identify voids much larger than the 3D voids studied in previous works, so that it becomes important to investigate the density profile of our 2D voids. In particular, changes in the underlying density profile of the objects are directly related to their imprint signals on the CMB convergence map. Thus, analysing the actual profile of our voids could give us insights to characterize their imprints in the CMB lensing map (see Section~\ref{sec:CMBxvoids}). 

The density profile of cosmic voids can be defined as the number of tracers at a given angular distance from the void centre, compared to the mean distribution of tracers at redshift $z$. We can analogously relate it to the void-halo two point cross-correlation function (2PCF), which refers to the measurement of pairs void/halo at different angular separations (see Eq.~4 in~\cite{Khoraminezhad2021}). Therefore, to estimate the density profile of our cosmic voids in the halo field, we have measured the void-halo 2PCF using the publicly available \texttt{GUNDAM} toolkit~\cite{gundam}. The \texttt{GUNDAM} pipeline measures the 2PCF $\xi_{ij}(r)$ directly in the ligthcone, using the Davis-Peebles estimator~\citep{DP}:
\begin{equation}
    \xi_{ij}(r)=D_iD_j(r)/D_iR(r)-1,
\end{equation}
where $D_{i}D_{j}(r)$ and $D_iR(r)$ represent the count of pairs at the distance $r$ between the object $i$ and the object $j$ or the random distribution of point $R$, respectively. 
Note that, in order to be in line with future experiments, we computed the density profile of our voids using DM haloes as tracers of the density, meaning that the resulting profile will be also related to the void size function of Figure~\ref{fig:radius_dist}.

In Figure~\ref{fig:dens_prof_20} we show the measured density profile for the void catalogue, divided in three redshift bins ($0.2<z<0.5$, $0.8<z<1.2$, $1.6<z<2.0$) for the three smoothing scales considered in this work (from top panel to bottom panel). We observe from the correlation function that once we consider larger smoothing scales, the identified voids tend to be smoother, with a central density less negative with respect to what is observed for smaller smoothing scales, which is consistent with what we saw in Figure~\ref{fig:central_dens_radii}.  
Moreover in all panels, we noticed that larger neutrino masses tend to make voids slightly deeper than in the massless neutrino $\Lambda$CDM case, which is an opposite trend to what has been observed in~\cite{massara2015}. However, the analysis of~\citep{massara2015} is using the non-observable dark matter particles as void-tracers, while in this work we use halo as void-tracers. It is well known that voids identified in the CDM field are and behave differently from voids identified in halo/galaxy distributions, which are biased tracers of the underlying dark matter field. In this respect, the profile amplitude of the halo-traced voids becomes deeper as the halo-bias increases with the neutrino mass~\cite{Marulli_Carbone_2011}, contrary to what we will observe in the next Section for the void profile measured in the lensing-convergence field, which decreases as massive neutrinos smooth the CDM density perturbations below their free-streaming scale. Since more biased objects are also sparser, the observed void profiles are thus in agreement with the decrease in the number density of gravitationally bound haloes with a mass greater than the minimum mass of the mock catalogues (see Section~\ref{sec:hmf}). Therefore, the more the neutrinos are massive, the more the voids will be devoid of massive objects. Similarly to what happens with the halo mass function, the effect of massive neutrino is greater at high redshifts than at low ones, the latter being also consistent with the fact that, at higher redshifts, larger scales will be affected by massive neutrinos (cf. Figure~\ref{fig:FS}) and more biased haloes will become even rarer. We stress that the density profile presented here is computed using DM haloes, i.e. biased objects as tracers of the matter field, as they are more realistic than the CDM particles which can be measured only in simulations. 

