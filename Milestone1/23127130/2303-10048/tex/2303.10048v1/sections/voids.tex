\section{The void lightcone}


\subsection{The void finder}
\label{sec:void_finding}
During the past decades, various methodologies have been developed to identify underdensed regions of the matter field (see for example a comparison of void finders in~\cite{Colberg2008}). In the context of massive neutrino cosmologies, mainly 3-dimensional voids finders which implement Voronoi tessellation techniques have been  applied. In particular,~\cite{massara2015,Kreisch2019,Schuster2019} have quantified the effect of different massive neutrino cosmology on cosmic void properties and statistics using the DEMNUni simulations with the \texttt{VIDE}~\cite{sutter2015} 3-dimensional definition of voids. However, in~\cite{cautun2018} the authors have shown the potential of voids identified in 2-dimensional projected density fields to increase the detection level of void-lensing signals. 
In~\cite{Carles_void} - see also~\cite{vielzeuf2019,kovacs2019,kovacs2020,fang2019,kovacs2022} - the authors presented a void finder that identifies 2-dimensional underdensities in the matter field. Specifically, this void finder has been developed in the context of photometric observations in order to encompassed the photometric redshift errors, and it has been shown to optimize the strength of the lensing signal at void positions. Such results are in agreement with~\cite{cautun2018}, which argued that elongated structures on the line of sight (such are 2D voids by definition) will have a stronger lensing signal than spherical, 3 dimensional voids. Thus in this analysis, we have used this 2D void finder to identify cosmic voids in the halo catalogs from the DEMNUni simulation presented in Sect.~\ref{sec:DEMNUni}. The void finder can identify underdense regions in a tracer field following these steps:

\begin{enumerate}
    \item Divide the tracer sample in redshift slices, of a predefined comoving size $s_v$. In the context of this work, we have followed~\cite{Carles_void,vielzeuf2019} where the slice size is set to $s_v= 100 h^{-1}$Mpc, a value that has been shown to minimize error on redshift estimation in photometric survey;
    \item Convert each redshift slice in an \healpix{} map, and count the number of tracers located in each pixel of the map;
    \item Smooth the \healpix{} map with a gaussian kernel, the smoothing scale being left as free parameter of our methodology; 
    \item Identify the most underdensed pixel in the density map; %\MC{Is it a slice or the most underdensed pixel in the \healpix{} map? Is the most underdensed pixel the void's centre?};
    \item Compute the density in concentric shells around the void centre in the pixelized density maps until the shell density reaches the mean density of the slice itself; %\MC{as before, it's not clear to me if concentric shells are done in the 3d distribution within a slice or on 2d healpix maps}.
    \item Repeat the process for the next most underdensed pixels in the slice.
    
\end{enumerate}

The output catalogue will contain information on different voids parameters such as the void radius $R_{\rm v}$ - the radius at which the void density reaches the mean density of the slice - the mean void density $\Bar{\rho}$ - the averaged value of all pixel located inside the void radius - and the central density $\rho_{1/4}$, being the density of the central part of the identified void ($r<\frac{1}{4}R_{\rm v}$). These features in the output catalogue will depend on the different input parameters ({\it e.g.} matter tracer density, smoothing parameter, etc.).



\subsubsection{Void finder parameters}
Void finder parameters will impact the void properties and, crucially, the analysis itself. 
For example, in~\cite{Y1_ISW} and~\cite{Y3_ISW} using the same 2D void definition presented here, the authors chose a large gaussian smoothing parameter ($50 h^{-1}$ Mpc) that forces the detection of large underdense region, {\it supervoids}, with radius of the order of $R_{\rm v}\sim100 h^{-1}$ Mpc, for which a tension with $\Lambda$CDM without massive neutrinos simulations have been observed~\cite{granett2008,cai}. However, in the case of void signal in the CMB lensing map, the objects showing a stronger signal are medium-size voids, $R_{\rm v}\sim 40-80 \, h^{-1}$Mpc~\cite{vielzeuf2019}. 
As the aim of this work is to 
evaluate the impact of massive neutrinos in such signals, we have used three different smoothing prescriptions: 10, 20, 30 $h^{-1}$Mpc. 
%Thus, we have run the finder using these three smoothing scales on the four DM Halo lightcones presented in the Section~\ref{sec:lightcone}.
In order to illustrate the effect of different smoothing, we show in Figure~\ref{fig:central_dens_radii} the distribution in central density ($\delta_{1/4}$) versus the void radius $R_{\rm v}$ computed for the void catalogues with different smoothing parameters, for the different neutrino masses prescriptions. Eventhough the effect of massive neutrino is difficult to detect, in the Figure we can see that increasing the smoothing scale of the void finder will result, on average, in voids smoother and larger. Namely, the most numerous population of voids will have radius of $R_{\rm v}\sim 30 h^{-1}$Mpc and a central density of $\delta_{1/4}\sim -0.8$ when the smoothing scale of the void finder is $10 h^{-1}$Mpc; with a smoothing scale of $30h^{-1}$Mpc we identify more voids with $R_{\rm v}\sim 75 h^{-1}$Mpc and $\delta_{1/4}\sim -0.7$.

\begin{figure}
\begin{center}
\includegraphics[width=.4\columnwidth]{./figs/smoothing_param_effect.pdf}
\includegraphics[width=.4\columnwidth]{./figs/smoothing_param_effect016.pdf}
\includegraphics[width=.4\columnwidth]{./figs/smoothing_param_effect032.pdf}
\includegraphics[width=.4\columnwidth]{./figs/smoothing_param_effect053.pdf}
\end{center}
\caption{Void distribution in the radius / central density plane $(R_{\rm v}, \delta_{1/4})$ for different smoothing scales, each panel representing the different massive neutrino fields in the DEMNUni simulations.}
\label{fig:central_dens_radii}
\end{figure}

Cosmic voids can be separated in different subgroups, in particular~\cite{sheth2004} presented two different scenarios at the origin of void formation:
\begin{itemize}
    \item \textit{voids-in-voids}: generally large underdensity, localized in an underdensed environment that are usually showing a negative relative bias with respect to the matter field~\cite{hamausbias}; 
    \item \textit{voids-in-clouds}: density minimum residing in larger overdensities, these voids are usually smaller objects compared to the previous ones.
\end{itemize}
Due to the local environment in which their reside, intrinsic properties of these two classes differ; in particular, their clustering properties will differ as in~\cite{zhao2016}, where the power spectrum of small voids ($R_{\rm v}<12h^{-1}$Mpc) has been shown to be correlated to the matter field, while large voids showed anti-correlation signal, {\it i.e.} small voids tend to reside in denser regions while larger voids will be identified in less populated regions. 
Therefore, we should expect differences in the correlation profile due to the environment; we check weather massive neutrinos could also affect the correlation signal differently, depending on the void population considered.
 Differentiating these populations is not straightforward, e.g. in~\cite{sheth2004} the authors used the void radius to separate the two type of voids, defining a \textit{characteristic void size} evolving with cosmic time. Moreover, as mentioned before, the presence of massive neutrino will smooth the density field up to a given scale, we expect effects due to their presence to depend both on the radius of the void considered and on their redshift. 


