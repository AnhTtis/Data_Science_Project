\section{CMBL maps and halo lightcone}\label{sec:DEMNUni}

\subsection{The DEMNUni simulations}
The bulk of this work is based on numerical simulations from the ``Dark Energy and Massive Neutrino Universe'' 
(\href{https://www.researchgate.net/project/DEMN-Universe-DEMNUni}{DEMNUni})\cite{DEMNUni2}.
The DEMNUni simulations have been produced with the aim of investigating large-scale structures in the presence of massive neutrinos and dynamical dark energy, and they were conceived for the nonlinear analysis and modelling of different probes, including dark matter, halo, and galaxy clustering~\cite{DEMNUni1,Moresco2017,Zennaro2018,Ruggeri2018,Bel2019,Parimbelli2021,Parimbelli2022, Baratta_2022,Guidi_2022, SHAM-Carella_in_prep}, weak lensing, CMB lensing, SZ and ISW effects~\cite{Roncarelli2015,DEMNUni_simulations,fabbian2018, Beatriz_2023}, cosmic void statistics~\cite{Kreisch2019,Schuster2019,verza_2019,Verza_2022a,Verza_2022b}, and cross-correlations among these probes~\cite{Cuozzo2022_inprep}.
The DEMNUni simulations combine a good mass resolution with a large volume to include perturbations both at large and small scales. They are characterised by, a softening length $\varepsilon=20\, h^{-1}$ kpc, a comoving volume of $(2 \: h^{-1}\mathrm{Gpc})^3$ filled with $2048^3$ dark matter particles and, when present, $2048^3$ neutrino particles. The simulations are initialised at $z_{\rm in}=99$ with Zel'dovich initial conditions. The initial power spectrum is rescaled to the initial redshift via the rescaling method developed in~\cite{zennaro_2017}. Initial conditions are then generated with a modified version of the \texttt{N-GenIC} software, assuming Rayleigh random amplitudes and uniform random phases.
The DEMNUni simulations were run using the tree particle mesh-smoothed particle hydrodynamics (TreePM-SPH) code \gadget{} \citep{Springel_2005}, specifically modified as in \citep{Viel_2010} to account for the presence of massive neutrinos. This modified version of \gadget{} follows the evolution of cold dark matter (CDM) and neutrino particles, treating them as two separated collisionless components. For each simulation we have produced 63 output logarithmically equispaced in the scale factor $a=1/(1+z)$, 49 of which lay between $z=0$ and $z=10$.
The DEMNUni suite accounts for 15 different cosmological models with different neutrino masses and dynamical dark energy. However, in this work we focus on four separate numerical simulations: one in a standard neutrino massless $\Lambda$CDM model, and three in a modified $\Lambda$CDM cosmology characterised by the presence of massive neutrinos with total mass $m_{\nu}=$ $0.16$ eV, $0.32$ eV, $0.53$ eV. All these simulation share the same baseline, {\it Planck-2013} cosmology \citep{Planck2013_XVI}:
\begin{equation*}
\{ \Omega_{dm},  \Omega_{b},  \Omega_{\Lambda}, n_s, \sigma_8, H_0
  \}  = \{ 0.27, 0.05, 0.68, 0.96, 0.83, 67 \: \rm{ Km / s /
    Mpc} \}. 
%\begin{split}
%\{ \Omega_{dm},  \Omega_{b},  \Omega_{\Lambda}, & n_s, \sigma_8, H_0
%  \}  = \\
%  & \{ 0.27, 0.05, 0.68, 0.96, 0.83, 67 \: \rm{ Km / s /
%    Mpc} \}. 
%\end{split}
\end{equation*}


In this work we use the friend-of-friend (FoF) halo catalogues, built from each of the 63 particle snapshots via the  FoF algorithm included in \gadget{}~\citep{springel01,dolag09}, setting to 32 the minimum number of CDM particles, thus fixing the halo minimum mass to $m_{\rm FoF}\simeq2.5\times 10^{12}h^{-1}M_\odot$. 


\subsection{CMBL reconstruction}
\label{sec:lightcone}

The lensing observables maps are extracted with a post-processing procedure acting on the N-body particle snapshots to create a full {\it lightcone}. This procedure follows the approaches of \citep{fosalba08,dasbode2008}, and was developed to perform high-resolution CMB lensing simulations \citep{fabbian2018,Hilbert2020}, in order to implement a multiple-lens ray-tracing algorithm in spherical coordinates on the full sky. 

The current version of the code reconstructs a full-sky, backward lightcone around an observer  using the particle snapshots out to the comoving distance of the highest redshift available from the simulation, following \cite{calabrese2015}. To overcome the finite size of an N-body simulation box, the code replicates the box volume the number of times in space necessary to fill the entire volume between the observer and the source plane. Moreover, the code can randomize the particle positions, as described in \cite{carbone2008, carbone2009}, throughout flips and shifts, to minimize any numerical artifacts due to the repetition of the same structures along the line of sight.
The backward lightcone is then sliced into 62 full-sky spherical shells with the following scheme: the median comoving distance spanned by each shell coincides with the comoving distance at the redshift of each N-body snapshot. Any particle inside each of these shells is projected onto 2D spherical maps; the resulting surface mass density $\Sigma$ on each sphere is thus defined on a two-dimensional grid. For each pixel of the $i$-th sphere one has
\begin{equation}
\Sigma^{(i)}(\boldsymbol{\theta}) = \frac{n \; m_{X}}{A_{\rm pix}}\,,
\label{eq:surfmass}
\end{equation}
where $n$ is the number of particles in the pixel, $A_{\rm pix}$ is the pixel area in steradians and $m_X$ is the particle mass of type $X$ (dark matter, or neutrino) from the N-body simulation. 
For this work, the algorithm produces for each N-body simulation a full-sky convergence map on a \healpix{}\footnote{\url{http://healpix.sourceforge.net}} grid \citep{Hp} with $n_\text{side} = 4096$, which corresponds to a pixel resolution of $0.85$ arcmin. 
Finally, the lensing convergence of a source plane at redshift $z_S$ is computed in the Born approximation as the weighted sum of surfaces mass density:
\begin{equation}
\label{eq:kappa}
\begin{split}
 \kappa (\boldsymbol{\theta}, \chi_S) &= 
\frac{4 \pi G}{c^2}\frac{1}{f_S}\!
\sum_{i} 
(1 + z_D^{(i)}) \frac{f_{DS}^{(i)}}{f_D^{(i)}}
\left[\Sigma^{(i)}( \boldsymbol{\theta})\!-\!\bar{\Sigma}^{(i)}\right].
\end{split}
\end{equation}
Eq. \eqref{eq:kappa} follows the standard notation in the literature of weak lensing observables \citep[see, e.g.,][ for reviews]{2001PhR...340..291B,2015RPPh...78h6901K, 2018ARA&A..56..393M}, for the convergence field $\kappa$ at an angular position $\boldsymbol{\theta}$ of a source at comoving line-of-sight distance $\chi_S$ and redshift $z_S = z(\chi_S)$. Note that $f_{DS} = f_K(\chi_S - \chi_D)$, $f_D = f_K(\chi_D)$ and $f_S = f_K(\chi_S)$, where $f_K(\chi)$ denotes the comoving angular diameter distance for comoving  line-of-sight distance $\chi$, and $z_D = z(\chi_D)$ is the redshift corresponding to comoving line-of-sight distance $\chi_D$. Finally, $\Sigma^{(i)}$ represents the angular surface mass density, while $\bar{\Sigma}^{(i)}$ is the mean angular surface mass density of the $i$-th shell; $f_{DS}^{(i)}$ and $f_D^{(i)}$ are the corresponding distances at the redshift of the $i$-th shell. $\Sigma^{(i)}$ is extracted directly from the N-body simulation with the map-making procedure described before. The angular position of the centre of each \healpix{} pixel coincides with the direction of propagation of the rays in the Born approximation. Several lensing observables can be then constructed by changing the source distance (or redshift $z_S$), which will affect the geometrical weight in the sum of Eq. \eqref{eq:kappa}. 

In this work, we have constructed convergence maps for CMB lensing ($\kappa$CMB), i.e. setting the source plane at the last scattering surface ($z_S$=1089). Four lightcones have been produced using particles snapshots in the four different cosmological scenarios: $\Lambda$CDM, and $\Lambda{\rm CDM} + m_\nu$ simulations with total neutrino masses $m_\nu=0.16$ eV, $m_\nu=0.32$ eV  and $m_\nu=0.53$ eV. In Figure~\ref{fig:cls_kcmb_dem} we show the angular power spectra from CMB convergence maps, both in terms of actual spectra (top panel) and ratio w.r.t. the $\Lambda$CDM case (bottom panel). Besides, signals are also compared with the semi-analytical realizations of \texttt{pyCAMB}\footnote{\url{https://camb.readthedocs.io/en/latest/}}, for all the different cosmologies considered. The bottom panel of this Figure highlights the effects of neutrino masses on the spectra, reducing the power especially at high multipoles where neutrino physics is more relevant. Points with errorbars are measurements on the reconstructed lightcone convergence ($\kappa$CMB) maps, while lines are semi-analytical realizations with \texttt{pyCAMB}. N-body simulations - and their related lightcone maps - are in line with theoretical expectations, especially for multipoles $\ell > 50$, whereas larger scales are affected by cosmic variance as seen in the top panel. Vertical lines in the Figure are computed directly from the free-streaming length of the different neutrinos masses considered - see Eq. \eqref{eq:FS} - translated into an (average) multipole $\langle \ell_{\rm FS} \rangle$, which represents the scale where neutrino effects could become more relevant. Multipoles are averaged on the CMB weak-lensing kernel (basically, the geometrical weight of Eq. \eqref{eq:kappa}), as:
\begin{equation}\label{eq:elleFS}
    \langle \ell_{\rm FS} \rangle = \frac{\int_{0}^{z_{\rm survey}}W^{\kappa \rm CMB}(z)\, 2\pi / \lambda_{\rm FS}(z) \chi(z) {\rm d}z}{\int_{0}^{z_{\rm survey}}W^{\kappa \rm CMB}(z) {\rm d}z};
\end{equation}
note how the free-streaming length is converted into multipole by the Limber approximation, $\ell_{\rm FS}  = 2 \pi / \lambda_{\rm FS} \cdot \chi$, and $z_{\rm survey}=2.0$ as the maximum redshift of the halo and void catalogues. The CMB lensing kernel is a function of the redshift of the lensed object, and can be expressed as follows:
\begin{equation}\label{eq:cmblensingkernel}
    W^{\kappa \rm CMB}=\frac{3 \Omega_m H_0^2}{2c}\frac{1+z}{H(z)}\chi(z)\frac{\chi(z_{\rm CMB})-\chi(z)}{\chi(z_{\rm CMB})}, 
\end{equation}
with the kernel peaking at $z \sim 1.5$. 

\begin{figure}[htbp]

\includegraphics[width=1.\columnwidth]{figs/kCMB_cls_demnunii.png}

\caption{{\it Top panel:} CMB convergence angular power spectrum,  for $\Lambda$CDM (blue line) and $\Lambda{\rm CDM} + m_\nu$ simulations with total neutrino masses $m_\nu=0.16$ eV (red, dot-dashed line), $m_\nu=0.32$ eV (green, dotted line) and $m_\nu=0.53$ eV (orange, dashed lines). Black, dashed line is the semi-analytical realization with \texttt{pyCAMB} for the DEMNUni $\Lambda$CDM cosmology. {\it Bottom panel:} fractional difference for the angular power spectra with respect to the $\Lambda$CDM case. Points with errorbars refer to measurements from N-body simulations via the lightcone convergence maps; signals have been binned in multipoles, error bars representing the variance in each bin. Lines are semi-analytical realizations with \texttt{pyCAMB} in the different cosmologies. Vertical lines are the (average) FS multipole - $\langle \ell_{\rm FS} \rangle$ - as computed by Eq.~\eqref{eq:elleFS}.}

\label{fig:cls_kcmb_dem}
\end{figure}

\subsection{The halo catalogue: construction and  measurements}\label{sec:hmf}
The halo sample is constructed using the same lightcone prescription for the lensing convergence maps, as described in the previous Section. The FoF sample extracted from the DEMNUni simulations is the basis for a full-sky, 3D-halo catalogue, where each object is placed around a central observer. In this case, each FoF halo behaves as a (CDM or neutrino) particle in the lensing lightcone, i.e. they are replicated a number of times in space necessary to fill the entire volume, which encompasses several spherical slices representing the Universe at different stages of its evolution. These slices follow the prescription and randomisation procedure described in Section~\ref{sec:lightcone} for CMB lensing. Therefore, we have produced for each considered cosmology a full-sky, 3D halo catalogue where each object is defined by its coordinates ($r, \theta, \phi$), where $r$ is the distance from the central observer at $z=0$ (thus $r$ it is also a measure of redshift), and $\theta$, $\phi$ are the standard coordinates on the celestial sphere. This procedure is then applied to all N-body simulations, building halo catalogues in four different cosmological scenarios.

Figure~\ref{fig:halo_mass_dist} shows the DM halo mass function measured on the DEMNUni halo catalogue for both massive and massless neutrinos cosmologies. The Figure highlights how the presence of massive neutrino tends to lower the halo abundance, and that this effect is more pronounced as the mass of the neutrinos increases. This is expected, since the free-streaming of massive neutrino induces counter effects to gravitational collapse and therefore a slow-down of the growth of structures. Moreover, in agreement with previous works (see for e.g. \cite{Brandbyge2010,marulli2011,Castorina}), this suppression is particularly evident for the heaviest haloes ($M_h>10^{14}h^{-1}\, M_\odot$). Bottom panels of Figure~\ref{fig:halo_mass_dist}, in fact, show the relative ratio of the halo mass function in the presence of massive neutrino w.r.t. the case where the neutrino are massless, per redshift bin and mass bin. In particular, we notice that this effect is stronger as the redshift increases: as we go to higher redshift the free-streaming scale of massive neutrinos increases, and consequently the presence of massive neutrinos will smooth the matter field at higher scales.
%\pauline{comment on the sigma8 degeneracy that causes this}.

The effect of massive neutrinos in the halo mass function (and more generally in the matter clustering) will have an impact on the void population identified in the next step of the analysis. On the one hand, as shown in Figure~\ref{fig:halo_mass_dist}, the number of DM haloes formed in the presence of massive neutrinos decreases as the mass of the neutrinos increases, for the fixed minimum mass of the DEMNUni simulations; thus, the density of the considered tracers identifying voids will decrease as the neutrino mass increases. On the other hand, since the presence of massive neutrino slows down the clustering process, the trend is the opposite for haloes with smaller minimum mass, which can have densities higher in massive neutrino cosmologies than in the massless neutrino case \cite{liu2018}. Nonetheless from an observational point of view, halo number density of the DEMNUni simulations is consistent with galaxy densities from future survey observation as Euclid and LSST.

The decrease in the number of matter tracers per unit volume has a direct impact on the size distribution of voids: by using a sparser tracer sample, one should expect the resulting catalogue to contain fewer small structures, that will be eventually merged in larger ones \cite{sutter2014}\footnote{Note that while this trend can be seen for voids identified in the halo field, it can be different if CDM particles are used as void tracers (see \cite{massara2015} as an example).}. On the other hand, the fact that the clustering of the matter will be less effective due to massive neutrinos implies that the distribution of the underlying matter field will be sparser, and we therefore can expect to find shallower and larger underdensities w.r.t. the standard massless neutrino case.


\begin{figure}[htbp]
\begin{center}
\includegraphics[width=.8\columnwidth]{./figs/HMF_FULL3.pdf}
\end{center}
\caption{{\it Top panel:} Halo mass function in redshift bins for both $\Lambda$CDM (error bars) and $\Lambda{\rm CDM} + m_\nu$ simulations with the three degenerate neutrino masses studied here, $m_\nu=0.16$ eV (dashed line), $m_\nu=0.32$ eV (dash-dotted line) and $m_\nu=0.53$ eV (dotted lines). {\it Bottom panels:} ratio of the halo mass function in massive neutrino cosmologies w.r.t. the $\Lambda$CDM case. Errorbars are derived assuming a Poisson distribution in each of the mass bins.} 

\label{fig:halo_mass_dist}
\end{figure}

