\section{Conclusions}
The aim of this work is to study the cosmic void imprint on the CMB-convergence in massive neutrinos cosmologies. To this purpose, we identify two ways in which massive neutrinos can alter this CMB void-lensing signal. First, massive neutrinos can induce selection effects in the void identification process as they affect the density of tracers used to identify them. Second, massive neutrinos suppress matter density perturbations inside cosmic voids, implying that void-lensing signals from voids with similar size might differ according to the neutrino mass considered.  In order to fully understand these two effects, we analysed how the presence of massive neutrinos modifies the void finding process, and thus the intrinsic properties of the void catalogues. Moreover, we studied the cross-correlation signal of cosmic voids with CMB lensing. For this work, we exploited a set of N-body cosmological simulations, the DEMNUni suite, in various massive neutrino scenarios (0 eV, 0.16 eV, 0.32 eV and 0.53 eV).


CMB-convergence full-sky maps were built in Born approximation via ray-tracing of CMB-photons moving across the dark matter distribution (CDM and neutrinos) of the DEMNuni simulations, combined with a stacking technique~\cite{Fabbian_2018,Hilbert2020} of comoving particle snapshots . With the same stacking technique, for each considered cosmology, halo catalogues in comoving snapshots, obtained from the DEMNUni simulations via a FoF algorithm, were organised in a full-sky lightcone. In each lightcone of DM haloes, we identified 2D voids using the void finder presented in~\cite{Carles_void}, and applied different levels of Gaussian smoothing to the density field, in order to probe void catalogues with different properties (such as the mean void radius or voids densities). By looking at the intrinsic void features, we have shown that: \begin{itemize}
    \item in terms of their abundances, shown in Figure~\ref{fig:radius_dist}, the presence of massive neutrinos tends to decrease the total number density of voids traced by haloes with a minimum mass of $M_h=2.5 \times 10^{12} h^{-1}M_\odot$. We note that the number density of small void radius tends to be more reduced than the large radius ones. Such a behaviour in the void size function can be directly explained by the decrease in the tracer density, due to structure formation suppression by free-streaming neutrinos, which in turn induces a merging of the smaller structures in the void identification procedure. As we increase the Gaussian smoothing in the void finder - and thus trace larger structures - we can reduce this effect.

    \item with regards to void profiles traced by the halo-void cross-correlation, we observe in Figure~\ref{fig:dens_prof_20} that voids in massive neutrino cosmologies seem to be slightly deeper than in the massless neutrino case. In fact, massive neutrinos slow down structure formation and therefore the overall density of haloes with mass larger than $M_h=2.5 \times 10^{12} h^{-1}M_\odot$ is decreased. Consequently, the density of halo tracers inside voids is reduced as well, making void profiles, traced by the halo-void cross-correlation, to look deeper in massive neutrino cosmologies.
\end{itemize} 
In this respect, 2D voids should be treated as potential tools to constrain neutrino masses, by looking at the void size function and the void-halo cross-correlation.



Besides that, in this work we consider also the void-CMB lensing cross-correlation and show that it can be considered as well an important observable to probe the neutrino mass. In fact, contrary to void-halo clustering, this observable has only a linear dependence on the void bias (which in turn depends on the population of haloes used to identify the voids), while it is independent from the halo bias, since CMB convergence maps depend directly on the underlying matter field (both CDM and neutrinos). Indeed, the void-CMB lensing cross-correlation could be used together with void-void clustering to constrain the void bias. In this work, we looked at two kinds of void-CMB lensing cross-correlations: the cross-correlation between cosmic voids and  CMB lensing deflections caused by the CDM field alone (i.e. void profiles traced by void-$\kappa_{\rm CDM}$ cross-correlation), and the cross-correlation between voids and CMB lensing deflections caused by the the neutrino field alone (i.e. void profiles traced by void-$\kappa_{m_\nu}$ cross-correlation), for the case of a high neutrino mass of $m_\nu=0.53$ eV. In fact, as voids are large structures (especially the one traced by haloes with quite a large mass) we expect neutrino free-streaming to be less effective than in galaxy clusters, which are characterised by smaller scales. We measured an anti-correlation signal in both neutrino and CDM convergence maps within voids.
In Figure~\ref{fig:neutrinocontrib_im}  we observe first a large increase in the amplitude of the anti-correlation signals (both for CDM and $\nu$) for increasing smoothing scales; this can be explained by the fact that with our void-finder we are identifying density fluctuations that in amplitude are larger than the smoothing threshold chosen in the void-finder. If this happens at large scales (i.e. large smoothing scales), this means that the identified density fluctuation is already nonlinear at those scales, and therefore we will observe even more nonlinear fluctuations on scales smaller than the smoothing one, scales which enclosed within the void profile. Therefore, in terms of nonlinear underdensities, we will observe deeper voids as the smoothing scale increases, as shown in Figure~\ref{fig:neutrinocontrib_im} for the unbinned total signal and $m_\nu=0.53$ eV, and in Figure~\ref{fig:Revol.pdf} for different radius bins and neutrino masses.

In addition, in Figure~\ref{fig:neutrinocontrib_im} we can also observe that the ratio, $\kappa_{m_\nu} /\kappa_{CDM}$, i.e. between the void profiles in the neutrino and CDM fields respectively, increases with the smoothing scale. However, the signal is noisy and larger resolution simulations will be needed to study this effect in detail. We leave it for future work.



Since the effects of massive neutrinos depend on the scales and redshifts considered, we split our catalogues into various subsets of voids of similar sizes at different redshifts, in order to measure the void-lensing profiles while these parameters evolve. In Figure~\ref{fig:Zevol_separate.pdf} for the smallest smoothing scale we observe a slight increase of $\kappa_{m_\nu}$/$\kappa_{\rm CDM}$ as the redshift increases. The trend is inverted for the largest smoothing scale. However, this behaviour is hardly distinguishable due to the large errorbars and again we leave it for future work. Analogously, comparing the left and right panels of Figure~\ref{fig:Zevol_separate.pdf}, we observe that increasing the smoothing scale inverts the redshift evolution of the void-lensing profiles: larger voids are deeper for larger redshifts (high-$z$ voids will enclose more nonlinear perturbations for larger smoothing scales, boosting thus the anti-correlation amplitude) and smaller smoothing scale voids will be deeper for smaller redshifts (the smoothing technique selection affects only very small scales and we can observe the non linear growth of underdensities via gravitational instability).

Moreover, we have measured different correlation signals between voids and CMB convergence maps for the full void sample as well as for various sub-samples, at different redshifts in different neutrino cosmologies. 
For the considered void populations (traced by haloes with mass larger than $M_h=2.5\times 10^{12} h^{-1}M_\odot$), we note in Figure~\ref{fig:FULL_SIGNAL} that the presence of massive neutrinos tends to decrease the amplitude of the void-CMB lensing cross-correlation w.r.t. the massless neutrino case, i.e. to produce  shallower void profiles. We note that this effect is more enhanced as the neutrino mass increases. This can be explained by the theory of cosmological perturbations in the presence of massive neutrinos which suppress the nonlinear evolution of matter density perturbations, both overdensities and underdensities (such as voids). Therefore, in massive neutrinos scenarios the lensing convergence amplitude is suppressed below the free-streaming scale and consequently, for such scales, void-lensing profiles will be less deep as the neutrino mass increases. Similar trends are also showed in Figures~\ref{fig:zevol}-\ref{fig:Revol.pdf} as functions of redshift and bin in radius. Worth of note, these results could point neutrinos as the explanation for the claim from recent observation campaigns~\cite{hang2021,kovacs2022} of shallower void profiles measured via the cross-correlation between voids and CMB lensing. 

Our simulated measurements show the potential power of using in future particular setups in the void identification pipeline (eg larger structures) to increase the void sensitivity as a neutrino mass probe. Indeed, one of the main results of this work is represented by the trend of the sensitivity parameter, $\delta \kappa_{in}$ shown in Figure~\ref{fig:FULL_sensiv}. It shows that exploiting a large smoothing scale in the void search could help in distinguish very clearly between different massive neutrino cosmological scenarios.

Finally, we can claim that we observed a clear dependence on the neutrino mass in the void-lensing signal when the full void catalogue is considered, and this is in particular encouraging for the next-generation surveys, that will provide a unprecedentedly large catalogues of cosmic voids up to redshift $z=2$.
A possible extension  of this work could be to exploit simulated galaxy catalogues and to include observational systematics, in order to verify the ability of future surveys to detect the cosmological dependence on the neutrino mass in void-lensing correlation signals. 
