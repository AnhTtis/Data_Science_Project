\appendix


\section{The massive neutrinos behaviour at different scales and redshifts}\label{sec:appendix_A}
\subparagraph{Redshift evolution:}\label{sec:redshift_evol_neutrinoeffect}

We want to verify whether or not massive neutrino present in cosmic voids are showing variation in their CMB imprints at different redshift, that is to say if the contribution of massive neutrino in the lensing imprints of voids evolves with time. To do so, we divide our sample in three different redshift bins ($0.2<z<0.5$; $0.8<z<1.2$ and $1.6<z<2.0$) and perform the stacking measurement of the signal in both neutrino-only lensing maps and CDM-only ones, for all the redshift sub-samples. Figure~\ref{fig:Zevol_separate.pdf} shows the separate imprints of neutrinos (dashed lines) and CDM (solid lines) in CMB lensing maps at different redshifts, for the three different smoothing scales. In the CDM only map, the amplitude of the correlation at the void centre seems to follow the behaviour of the CMB kernel with a peak at redshift $z\sim 1.5$. 
If we compare the difference in the signal within the three smoothing scales, in agreement with Figure~\ref{fig:neutrinocontrib_im}, we can observe that the redshift evolution of the lensing signal is more pronounced for both the CDM-only signal and the neutrino-only one as we increase the smoothing scale. In fact, if we look at Figure~\ref{fig:FS} and \ref{fig:central_dens_radii}, we see that the mean void radius (and most of the voids) in the 10 $h^{-1}$Mpc smoothing catalogue is below the free-streaming length considered here. Consequently, we do not expect a strong redshift evolution in the neutrinos lensing signal. However, once we increase the smoothing scale, the objects identified will reach sizes greater than the free-streaming length of neutrinos, thus these structures could be affected at different redshifts.
The insight plots of Figure~\ref{fig:Zevol_separate.pdf} show the ratio between CDM and neutrino contributions in the lensing signal at the void centre; however, the contribution signal is quite noisy, and we cannot observe a significant evolution in redshift for any of the smoothing scales used, with an average value for the ratio close to $2\%$. Given a specific neutrino mass, the lensing imprint directly caused by massive neutrino follows a similar redshift evolution as the imprint generated by CDM.


\begin{figure*}[h!]

\includegraphics[width=150mm]{./figs/separate_effect_redshift_sub_mpc.pdf}


\caption{Redshift evolution. Imprint caused by neutrinos {with $m_\nu=0.53$ eV} (dashed lines) and CDM only (solid line) for voids in three different redshift bins for the three smoothing scale 10 $h^{-1}$Mpc ({\it left panel}), 20 $h^{-1}$Mpc ({\it middle panel}), 30 $h^{-1}$Mpc ({\it right panel}). The shaded region represents the fluctuations of the signal measured in 1,000 randomly-generated CMB lensing map (see Section~\ref{sec:stacking_meth}). The insight plot in each panel is the ratio of the signal induced by neutrinos w.r.t. the one induced by CDM.}
\label{fig:Zevol_separate.pdf}
\end{figure*}


\subparagraph{Radius evolution:}\label{sec:Radius_evol_neutrinoeffect}
The effects of massive neutrinos on the structures of the universe is two-fold. At scales smaller than the free-streaming scale $\lambda_{\rm FS}$, due to their velocities, the neutrinos will travel over the different fluctuations in the potential field without being affected,  smoothing in fact the inhomogeneities. At scales larger than $\lambda_{\rm FS}$, the neutrinos will fall into the gravitional potentials. Consequently, we do expect voids catalogues including scales similar or larger than $\lambda_{\rm FS}$ to be more {\it devoid} of neutrinos than the ones with smaller structures.
Thus, we want to investigate the abundance of massive neutrinos in underdensed structure as a function of their sizes: for this purpose, we have divided our voids catalogues in 6 radius bins, %($[20,40],[40,60],[60,80],[80,100],[100,120],[120,160]Mpc/h$) 
and we have measured the lensing signal in each of these sub-sampled catalogues by applying the stacking technique to both the lensing signal due to CDM particles and the lensing signal due to massive neutrinos\footnote{Note that we have discard the lower radius bin ($20<R_{\rm v}(h^{-1}{\rm Mpc})<40$) in the 30$h^{-1}$Mpc smoothing scale due the low numerosity of the sample.}. Results are shown in Figure~\ref{fig:Revol_separate} for the three different smoothing parameters (from left to right panels). A stronger "de-lensing" signal due to massive neutrinos is observed for larger voids: the neutrinos are, as expected, less present in the largest objects. The insight plots in the Figure show the ratio between the lensing signal due to CDM-only to massive neutrino-only at the void centre.  
The contribution of massive neutrinos on the lensing signal seems for all the cases to be stronger as one increases the radius of the lensed voids, for both massive neutrinos and CDM imprints. Such behaviours suggest that, as we decrease the void radius, cosmic voids identified in the matter field will be less underdensed with neutrinos, which is consistent with the fact that neutrino will fall in potential wells at scales larger than $\lambda_{\rm FS}$, while travelling through density fluctuations at smaller scales. 
\begin{figure}[h!]

\includegraphics[width=150mm]{figs/separate_effect_radius_sub_mpc.pdf}
\caption{Radius evolution. Imprint caused by neutrinos with $m_\nu=0.53$ eV (dashed lines) and CDM only (solid line) for voids in six different radius bins for the three different smoothing scales: 10 $h^{-1}$Mpc ({\it left panel}), 20 $h^{-1}$Mpc ({\it middle panel}), 30 $h^{-1}$Mpc ({\it right panel}). The shaded regions represent the fluctuations of the signal measured in 1,000 randomly-generated CMB lensing map (see Section~\ref{sec:stacking_meth}). The insight plot in each panel is the ratio of the signal induced by neutrinos w.r.t. the one induced by CDM.}
\label{fig:Revol_separate}
\end{figure}

\section{Combined binning, in redshift and radius}
\label{sec:appendix_B}
The lensing imprint detected in massive neutrino simulations w.r.t. the massless neutrinos $\Lambda$CDM model depends on the redshift and the radius of the lenses used. In this section, we intend to measure such signal in different sub-samples binned both in radius and redshift. We have  binned our sample in the same ranges in redshift and radius as previously, and proceed to the stacking of these sub-samples in the CMB lensing maps of our simulation. The lensing profiles and sensitivity evolution for the different bins and neutrino cosmologies are shown in Figure~\ref{fig:finalevol_zoom.pdf} and \ref{fig:finalratio}, respectively. We added intermediate bins in Figure~\ref{fig:finalratio} in order to look in details at the redshift and radius evolution of the sensitivity parameter. The vertical lines in Figure~\ref{fig:finalratio} correspond to the value of the free-streaming scale $\lambda_{\rm FS}$ (computed for the average redshift in the bin). We note that in the lower redshift range, the number of stacked objects decreases significantly, and thus the correlation profiles becomes too noisy to disentangle the mass of neutrino species. However, in the higher redshift bins, due to the increase in the observed volume, we reach a number of voids higher enough to appreciate differences in the stacked profiles among the different cosmologies. In particular, we note that for the higher redshift bins, small and medium size voids show a relatively lower CMB lensing imprint once one considers massive neutrinos. On the contrary, larger structures (scales of the order of $\sim$ 100 $h^{-1}$Mpc) are showing sensitivity parameters close to $1$, suggesting no significant imprints in the correlation profile due to neutrinos.  
\begin{figure*}
\begin{center}

\includegraphics[width=1.\columnwidth]{figs/FINAL_plot_10binR_binz.pdf}
\includegraphics[width=1.\columnwidth]{./figs/FINAL_plot_20binR_binz.pdf}
\includegraphics[width=1.\columnwidth]{./figs/FINAL_plot_30binR_binz.pdf}
\end{center}
\caption{Lensing imprint of cosmic voids, with a combined  binning in both redshift and radius, for the different smoothing scale (from top to bottom: 10, 20, 30 $h^{-1}$Mpc). The shade region is the error computed using the methodology described in Section~\ref{sec:stacking_meth}. Different line-styles refer to the radius bin, while the redshift evolution is shown from left to right panels. Colours discriminate between massive and massless neutrino cosmologies.}
\label{fig:finalevol_zoom.pdf}
\end{figure*}


\begin{figure*}
\begin{center}



\includegraphics[width=.32\textwidth]{figs/FINAL_sensiv_10binR_binz.pdf}
\includegraphics[width=.32\textwidth]{./figs/FINAL_sensiv_20binR_binz.pdf}
\includegraphics[width=.32\textwidth]{./figs/FINAL_sensiv_30binR_binz.pdf}
\end{center}
\caption{Sensitivity parameter $\delta\kappa_{in}$ as a function of mean void radius for each redshift bin considered (from top to bottom); in each column the different smoothing scale considered. The vertical dashed lines represent the free-streaming length $\lambda_{\rm FS}$ of the neutrinos - Eq.~\eqref{eq:FS} - for the redshift bin.}
\label{fig:finalratio}
\end{figure*}



