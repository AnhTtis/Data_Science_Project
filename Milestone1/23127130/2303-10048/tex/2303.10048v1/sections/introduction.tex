% !TEX root = ../main_cleaned.tex

\section{Introduction}


The upcoming generation of galaxy surveys, such as Euclid\footnote{\url{https://www.euclid-ec.org/}}, DES\footnote{\url{https://www.darkenergysurvey.org/}}, DESI\footnote{\url{https://www.desi.lbl.gov/}} or LSST\footnote{\url{https://www.lsst.org/}}, will map the recent universe on large sky fractions with unprecedented precision, and will constrain the cosmological model using, among other probes, the clustering of billion galaxies and their gravitational lensing. The correlation of these probes with observations of the Cosmic Microwave Background (CMB) anisotropies (through the Interaged Sachs Wolfe or Sunyaev-Zeldovich effect) as well as their gravitational lensing (CMB lensing) will also provide complementary source of information and additional constraining power on cosmology and systematics of the large-scale structures (LSS) probes~\cite[see e.g][ for some recent overview]{ilic2022,wenzl2022,zhang2022}. \\*
The imprint of LSS in CMB-based observables, being it that of  matter overdensities such as galaxy clusters and cosmic filaments~\cite{he2018} or  underdensities such as voids and troughs~\cite{gruen2016,brouwer2018}, has been detected several times in the past and compared to predictions of numerical simulations to assess its intrepretation. 
Since observations of neutrino oscillations~\cite{Becker-Szendy1992,Fukuda1998,ahmed2004} suggest the presence of at least two species of neutrinos with a non-zero mass, future experiments in cosmology (on the CMB and galaxy surveys side) will try to constrain the properties of massive neutrinos from their imprint at cosmological scales. As such, the effect of massive neutrinos on the whole set of cosmological observables analyzed by those experiments has to be carefully studied if reliable information has to be extracted from those probes. 


The presence of massive neutrinos in our universe has an impact in both background evolution and structure formation~\cite{Lesgourgues2006}. In particular, the evolution of the large structure in the cosmic web is directly sensitive to  the neutrino mass at scales of the order of the size of cosmic voids. Several  analyses have shown how cosmic voids could be exploited to set constraints on neutrino physics~\cite{Kreisch2019,bayer2021,contarini2021}. 
While at small scales, due to their non-zero velocity, massive neutrinos will travel across density fluctuations and effectively smooth them, at scales comparable to cosmic voids massive neutrinos will fall in the potential wells. We then expect cosmic voids to be particularly affected by the presence of massive neutrinos because the typical size of voids (10 to 100s of $h^{-1}$Mpc) approaches the free-streaming length ($\lambda_{\rm FS}$) of massive neutrinos, which can be expressed as function of redshift and neutrino species mass~\cite{Lesgourgues2012,Lesgourgues2013}:
\begin{equation}\label{eq:FS}
    \lambda_{\rm FS}(m_\nu,z)\sim 8.1 \frac{H_0(1+z)}{H(z)}\left(\frac{1eV}{m_\nu}\right)h^{-1}{\rm Mpc},
\end{equation}
with $H(z)$ and $H_0$ being the Hubble parameter and its value at $z=0$, respectively. As an example in Figure~\ref{fig:FS} we show the evolution of $\lambda_{\rm FS}$ as a function of redshift for the different neutrino masses that will be considered in this work.
\begin{figure}
    \centering
    \includegraphics[width=0.6\columnwidth]{figs/Freestreaming_neutrino_comoving.pdf}
    \caption{Comoving Free-Streaming length $\lambda_{\rm FS}$ of massive neutrinos - Eq.~\eqref{eq:FS} - as a function of redshift, for the different neutrino masses considered in this work.}
    \label{fig:FS}
\end{figure}
Since the neutrino field is less clustered compared to the total matter field, the ratio between matter and neutrino density is higher at the maximum of the potential field, i.e. in cosmic voids. We thus expect to observe stronger effects of the presence of massive neutrinos in voids than in denser regions of the universe and~\cite{Zhang2019} has in fact shown that the halo mass function of dark matter haloes selected in cosmic voids is more affected by the presence of massive neutrino than by the full sample of the dark matter halo present in their volume. 
The overall presence of massive neutrinos therefore slows down the evolution of cosmic voids with respect to the massless neutrino case and induce modifications in the density profile of voids. We thus expect the signature of massive neutrinos to be also observable in the gravitational lensing effect~\cite{massara2015}. % on CMB photons in the presence of massive neutrinos.

%The large structure at low redshift is traced by the luminous matter, i.e. galaxies that are biased objects with respect to the total underlying matter field. %Nevertheless, gravitational lensing of background structures by the foreground matter field is directly sensitive to the underlying matter field. 

The deflection of photons (coming from CMB and background galaxies alike) induced by gravitational potentials along the line of sight tend to converge light rays onto denser regions and divert them away in the case of voids. Such {\it negative-lensing}  signal of cosmic voids has already been detected on the galaxy shape~\cite{clampitt2015,melchior2014,Carles_void,fang2019}, as well than in the CMB anisotropies and lensing 
\cite{cai,planck2016isw,vielzeuf2019,Raghunathan2019}. The most recent results from  DESI~\cite{hang2021} and DES~\cite{kovacs2022} have reported a lower signal in the observed lensing signal in the CMB at cosmic void positions w.r.t. the one estimated from $\Lambda$CDM simulations with massless neutrinos. Since the presence of massive neutrinos has been advocated as a way to reconcile such observational results with those observed in simulations, in this work we try to assess the validity of this hypothesis and study the gravitational lensing caused by cosmic voids modeled with N-body simulation with different neutrino mass on the CMB lensing field. 

This paper will be organized as follow, in Section~\ref{sec:DEMNUni} we will present the DEMNUni simulations halo samples and CMB-lensing (CMBL) maps used in this analysis, then in Sections~\ref{sec:void_finding} and \ref{sec:void_cat} the void finding methodology and the different void catalogues used in this analysis will be exposed. And finally Section~\ref{sec:CMBxvoids} will present the effect of massive neutrinos on the correlation of cosmic voids with CMB lensing signal.
