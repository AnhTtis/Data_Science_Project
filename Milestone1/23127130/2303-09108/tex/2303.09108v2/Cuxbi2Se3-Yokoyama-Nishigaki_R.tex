\documentclass[aps,prl,preprint,groupedaddress]{revtex4-1}
%\documentclass[aps,prb,groupedaddress,twocolumn]{revtex4-2}
%\documentclass[aps,prl,groupedaddress,twocolumn]{revtex4-1}
%\documentclass[aps,prl,nofootinbib,groupedaddress]{revtex4-1}
%\bibliographystyle{apsrev4-1}

%\documentclass[aps,prb,preprint,superscriptaddress]{revtex4}
%\documentclass[aps,prl,twocolumn,groupedaddress]{revtex4-1}
\usepackage[pdftex]{graphicx}
%\usepackage[dvipdfmx]{graphicx}
%\usepackage{graphicx}
%\usepackage{graphicx}	% Include figure files
\usepackage{dcolumn}	% Align table columns on decimal point
\usepackage{bm}			% bold math
\usepackage{here}
%\usepackage{color}
% You should use BibTeX and apsrev.bst for references
% Choosing a journal automatically selects the correct APS
% BibTeX style file (bst file), so only uncomment the line
% below if necessary.
\bibliographystyle{apsrev4}


%\begin{document}
%	\preprint{APS/123-QED}

\usepackage[dvipdfm]{}
\usepackage{graphicx}
\usepackage{epstopdf}
\usepackage{mathrsfs}
\usepackage{amsmath}
\def\tc{$T_{\rm c}$}
\def\tl{$1/T_1$} 
\def\kb{$k_{\rm B}T_{\rm c}$}
\def\tlt{$1/T_1T$}
\def\pb{PbTaSe$_2$}
\def\cx{Cu$_x$Bi$_2$Se$_3$}
\def\sx{Sr$_x$Bi$_2$Se$_3$}
\def\nx{Nb$_x$Bi$_2$Se$_3$}
\def\BS{Bi$_2$Se$_3$}
\def\hct{$H_{\rm c2}$}
\def\dv{\textbf {d}-vector}
\begin{document}

\title{Manipulating the  nematic director  by magnetic fields in the  spin-triplet  superconducting state of  Cu$_{x}$Bi$_2$Se$_3$} %  }
%\author{
\author{M. Yokoyama}
\thanks{Partly based on Master degree theses by M. Yokoyama (Feb. 2021) and H. Nishizaki (Feb. 2022), Okayama University.}
\affiliation{Department of Physics, Okayama University, Okayama 700-8530, Japan}
\author{H. Nishigaki}
\thanks{Partly based on Master degree theses by M. Yokoyama (Feb. 2021) and H. Nishizaki (Feb. 2022), Okayama University.}
\affiliation{Department of Physics, Okayama University, Okayama 700-8530, Japan}
\author{S. Ogawa}
\affiliation{Department of Physics, Okayama University, Okayama 700-8530, Japan}
\author{S. Nita}
\affiliation{Department of Physics, Okayama University, Okayama 700-8530, Japan}
\author{H. Shiokawa}
\affiliation{Department of Physics, Okayama University, Okayama 700-8530, Japan}
\author{K. Matano}
\affiliation{Department of Physics, Okayama University, Okayama 700-8530, Japan}
%\affiliation{Institute of Physics, Chinese Academy of Sciences, and Beijing National Laboratory for Condensed Matter Physics,  Beijing 100190, China}
%\affiliation{School of Physical Sciences, University of Chinese Academy of Sciences, Beijing 100190, China}
\author{Guo-qing Zheng}
%\thanks{To whom correspondence should be addressed; E-mail:  zheng@psun.phys.okayama-u.ac.jp}
\affiliation{Department of Physics, Okayama University, Okayama 700-8530, Japan}
%\affiliation{Institute of Physics, Chinese Academy of Sciences, and Beijing National Laboratory for Condensed Matter Physics,  Beijing 100190, China}
%}
%\email[]{zheng@psun.phys.okayama-u.ac.jp}%\thanks{}%\altaffiliation{}
%\affiliation{
%$^1$
%Department of Physics, Okayama University, Okayama 700-8530, Japan}
%$^2$Institute of Physics, Chinese Academy of Sciences, and Beijing National Laboratory for Condensed Matter Physics, Beijing 100190, China
%}
%\date{\today}

\begin{abstract}
	Electronic nematicity, a consequence of  rotational symmetry breaking, is an emergent phenomenon in various new materials. %sub-fields of physics.    %Therefore, nematicity is at the heart of  materials- and condensed-matter physics, and 
	%
	In order to fully  utilize the functions of these materials,  ability of  
	tuning them  through a knob, the nematic director,  is desired.
	%In spin-triplet superconductors, the vector order-parameter $\textbf{{d}}$  acts as a nematic director. 
	Here we report a successful manipulation  of the nematic director, the vector order-parameter ($\textbf{{d}}$-vector),  in the  spin-triplet superconducting state of  Cu$_x$Bi$_2$Se$_3$ by  magnetic fields. % with low dopings and  less disorder. %($x$=0.15,  0.20 and 0.27)
	  % We measure the field- and direction dependence of the Meissner diamagnetism through   ac susceptibility  in the superconducting state. 
	At  $H$ = 0.5 T, the ac susceptibility related to the upper critical field  shows a %nearly %sum of two types of 
	two-fold symmetry in the basal plane. %with respect to the angle between the field and crystal $a$-axis. %, in contrast to the trigonal crystal structure. %, with 90 degree phase difference between the two. 
	At $H$ = 1.5  T, however, the susceptibility %diamagnetism 
	shows a six-fold symmetry, which has never been reported before in any superconductor. These results indicate that the $\textbf{{d}}$-vector initially pinned to a certain direction is unlocked  by  %the magnetic field above 
	a threshold field   to respect the trigonal crystal symmetry. %
	%to trace the field. 
	We further reveal that the superconducting gap in different crystals converges to $p_x$ symmetry at high fields, although it differs at low fields.
%	Our work further reveals the intrinsic gap symmetry of Cu$_x$Bi$_2$Se$_3$;although these crystals show different  gap symmetry at low fields, they converge to the same $p_x$  symmetry at high fields.   %revealed  interactions of  the $\textbf{{d}}$ vector with lattice  and 
	%demonstrates for the first time the high tunability of the nematic director by the magnetic field.
\end{abstract}
\maketitle
%\clearpage

%\textbf{Teaser:}
%Magnetic-field tuning of the nematic director in Cu$_{x}$Bi$_2$Se$_3$

%74.25.N- 	Response to electromagnetic fields
%71.70.Ej 	Spin-orbit coupling, Zeeman and Stark splitting, Jahn-Teller effect
%74.25.Jb 	Electronic structure (photoemission, etc.)
%76.60.Cq 	Chemical and Knight shifts
%74.25.nj   Nuclear magnetic resonance
%74.70.Dd 	Ternary, quaternary, and multinary compounds (including Chevrel phases, borocarbides, etc.)
%\pacs{74.25.N−, 71.70.Ej, 74.25.Jb, 76.60.Cq}

%\clearpage
%\section{Introduction}

Skyrmion spin textures of magnets %ic materials
\cite{Pflei},  the normal state of iron-pnictides \cite{Fernandes-Shmalien-Chubukov}, and the superconducting states of %doped topological insulator 
spin-triplet  superconductors \cite{MatanoKrienerSegawaEtAl2016,Yang} and magic-angle graphene \cite{YCao}, are all nematic. Furthermore, the excitation in the vortex cores of spin-triplet superconductors can form nematic skymion-type texture \cite{Babaev}.
%Among these materials, 
%Odd-parity spin-triplet  superconductors such as Cu$_{0.3}$Bi$_2$Se$_3$ and K2Cr3As3, 
%Exploring topological materials  and  their electronic functions are among the front-most topics of  current condensed matter physics.
%In particular,   much attention has been paid in recent years to topological superconductors where
% 
In particular, nematic  spin-triplet superconducting states  are topological \cite{FuPRL}, where
Majorana fermions (excitations) are expected to appear on edges or in the vortex cores\cite{QiZhang,Ivanov},  %Such novel edge states  
which can  potentially  be applied  to  fault tolerant
non-Abelian quantum computing \cite{TopologicalQuantumComputation,quantumcomputer_KITAEV20032}.
% 
However,  bulk spin-triplet  superconductors are still very rare. % However, research on bulk topological superconductors progresses much more slowly. 
Carrier-doped topological insulator Cu$_{0.3}$Bi$_2$Se$_3$ \cite{MatanoKrienerSegawaEtAl2016} and ferromagnetically correlated electron system K$_2$Cr$_3$As$_3$ \cite{Yang} are recently-established spin-triplet superconductors, with the superconducting transition temperature $T_c$ as high as 6.5 K. Along with the   uranium-based candidates such as  UTe$_2$ \cite{UTe2}, they provide good platforms for the study of topological quantum phenomena. % and industrial application. 
In order to implement these compounds in  applications, however, one still needs to better understand the physics of the spin-triplet states in these materials.

 




In contrast to spin-singlet state, %A more generally-used term associated with the gap in %In a more broader context, the most important component pertinent to
 a spin-triplet superconducting state is described by the vector order parameter \textbf{{d}},  % \dv %$d$-,
 whose direction is perpendicular to the direction of paired  spins  and whose magnitude is the gap size \cite{Balian}. 
 In superfluid $^3$He, the \dv\ rotates freely \cite{Leggett}. In a solid, however,  the \dv\ can be  pinned to a certain direction which results in spin-rotation symmetry breaking as first found in Cu$_{0.3}$Bi$_2$Se$_3$  \cite{MatanoKrienerSegawaEtAl2016}. The pinning of the \dv\ is the origin of the observed   nematic responses \cite{MatanoKrienerSegawaEtAl2016,Yonezawa_Natphys,Sr_dope_2fold_PanNikitinAraiziEtAl2016,Asaba_Sr_PhysRevX.7.011009,Sr_dope_2fold_Du2017,FengDL,Kawai}, so that  the $\textbf{{d}}$-vector is the nematic director.
 When the   \dv\   rotates or is flipped, the magnetic response also changes and the pairing symmetry can even change, % \cite{Yang}. 
 giving rise to a  transition from a phase with one symmetry to another  with different symmetry.  Cu$_x$Bi$_2$Se$_3$    exemplifies such intriguing property,  where  carrier-concentration  tunes the superconducting phase from one to another with different \dv\ direction. For low doping level with $x<$0.46, the \dv\ lies in the basal plane, while for high dopings with $x\geq$0.46, the \dv\ rotates to the $c$-axis direction \cite{chiral_PhysRevB.94.180504,Kawai}, accompanying a possible nematic-to-chiral phase transition.
 Thus, a thorough understanding of the $\textbf{{d}}$-vector  and its interaction with the environment and external perturbations is important. 
  






In this paper, we report a successful manipulation of  the $\textbf{{d}}$-vector in Cu$_x$Bi$_2$Se$_3$ by a magnetic field as small as 1 Tesla.
We synthesized  Cu-doped \BS\ single crystals with low doping rate by the electrochemical intercalating method.% and characterized the superconducting properties by magnetic susceptibility.  
Through the measurements of $^{77}$Se nuclear magnetic resonance (NMR), we confirm the small carrier concentration and less disorder/defects caused by doping of the new crystals.
We measure the ac-susceptibility in the superconducting state by rotating the sample in a magnet to change the angle between the magnetic field and the crystal $a$-axis. We further reveal the intrinsic gap symmetry of Cu$_x$Bi$_2$Se$_3$.
 
Single crystals of \cx\ were prepared by intercalating Cu into \BS\ by the electrochemical doping method described in Ref.[\onlinecite{Kawai,Kriener_PRB2011}].
First, single crystals of \BS\ were grown by melting stoichiometric mixtures of 
elemental Bi (99.9999\%) and Se (99.999\%) at 850 $^{\rm o}$C for 48 hours in sealed evacuated quartz tubes. 
After melting, the sample was slowly cooled down to 550 $^{\rm o}$C over 48 hours
and kept at the same temperature for 24 hours.
%Elemental starting materials of Bi (99.9999\% purity) and Se (99.999\%) with a stoichiometric ratio were placed in a quarts tube,
%sealed under vacuum, heated in 850$^{\rm o}$C for 48 hours, 
%cooled slowly down to 550 $^{\rm o}$C for 48 hours, and kept at same temperature for 24 hour.
Those melt-grown \BS\ single crystals were cleaved into smaller rectangular pieces of about 14 mg.
They were wound by bare copper wire (dia. 0.05 mm), and used as a working electrode.
A Cu wire with diameter of 0.5 mm  was used both as the counter (CE) and the reference electrode (RE).
We applied a current of 10 $\mu$A in a saturated solution of CuI powder (99.99\%) in acetonitrile (CH$_3$CN).
The obtained crystals samples were then annealed at 560 $^{\rm o}$C for 1 hour in sealed evacuated quartz tubes, and quenched into water.
After quenching, the samples were covered with epoxy (STYCAST 1266) to  avoid deterioration.
We have confirmed that the epoxy does not have extrinsic effect on the physical properties such as $T_{\rm c}$ or upper critical field $H_{\rm c2}$. 
The Cu concentration $x$ was determined from the mass increment of the samples.
To check the superconducting properties, dc susceptibility measurements were performed using a superconducting quantum interference device (SQUID) with
the vibrating sample magnetometer (VSM).
%NMR measurements were carried out by using a phase-coherent spectrometer.
%
%\noindent
%\textbf{NMR measurements.}
The $^{77}$Se-NMR spectra were obtained by the fast Fourier
transformation of the spin-echo  at a field of $H_0 $ = 1.5 T. % obtained with a standard $\pi/2-\tau - \pi$ sequence. 
The Knight shift $K$ was calculated using nuclear gyromagnetic ratio $\gamma_{\rm N}$ = 8.118 MHz/T for $^{77}$Se.
%NMR measurements were performed.
%
%\noindent
%\textbf{Angle-resolved ac-susceptibility  measurements.}
The
ac susceptibility  was measured by the inductance of an in-situ NMR coil. 
Angle-dependent measurements 
were performed by using a piezo-driven rotator (Attocube  ANR51) equipped with Hall  sensors  to determine the angle between  magnetic field and  crystal axis. We estimate that the error  in the angle determination is less than 1 degree.



%\section{Results}
\begin{figure}[htbp]
	\includegraphics[clip,width=80mm]{Fig1.pdf}
	\caption{%\label{SHandHc2}% 
		%\textbf{Sample characterizations}.
		(color online)(a) Superconducting transition and the shielding fraction of Cu$_{x}$Bi$_2$Se$_3$ ($x$=0.15, 0.20 and 0.27). Arrows indicate $T_c$ for the samples. (b) ac susceptibility for the sample with $x$=0.20. A small hump around $T$=2.1 K is due to the environmental change associated with the superfluid transition of liquid Helium. 
		(c) upper critical field $H_{\rm c2}$ for $H\parallel a$ for the three samples, (d)  $H_{\rm c2}$ for $H\parallel c$ for the three samples.
	}
\end{figure}
%
%
%\noindent
%\textbf{Sample characterization.} %In this paper, we report on four samples with different Cu content $x$ = 0.28, 0.37, 0.46, and 0.54 picked up from some samples we synthesized.
Figure 1(a) shows the superconducting transition  of the three samples Cu$_{x}$Bi$_2$Se$_3$ ($x$=0.15, 0.20 and 0.27) obtained by  dc susceptibility measurements. 
%In our previous work, we only focused on $x\geq$0.28.  
The $T_c$ is 3.6, 3.8 and 3.6 K for $x$=0.15, 0.20 and 0.27, respectively, which is 
 higher than  $T_c$ of other $x$ concentrations reported for $x\geq$0.28, following a general trend that $T_c$ increases with decreasing $x$ \cite{Kriener_PRB2012,Kawai}.
The shielding fraction at $T$=1.8 K for $x$=0.20 and 0.27 exceeds 40\% which is among the highest value reported so far. 
In our case, demagnetization is negligible as the magnetic field is applied parallel to the plate.
Figure 1(b) shows the data  for $x$=0.20 as a representative example for  ac susceptibility. 
Figure 1(c) and Fig.1(d) show the upper critical field $H_{\rm c2}$ for $H\parallel a$ and $H\parallel c$, respectively.  The $H_{\rm c2}$ for $x$=0.15 and 0.27 is comparable to that reported previously by Kriener et al \cite{Kriener_PRB2012}, but $H_{\rm c2}$ for $x$=0.20 is much higher.
\begin{figure}[htbp]
	\includegraphics[clip,width=70mm]{NMRspec.pdf}
	\caption{\label{spec}%(color online) 
		%\textbf{
		(color online)	The $^{77}$Se Knight shift as a function of nominal Cu-content $x$. The inset shows the NMR spectrum  for $x$=0.15, 0.20 and  $x$=0.28. The arrow indicates the gravity center of the spectrum from which the Knight shift was extracted.
	}
\end{figure}
%
In Fig. \ref{spec}, we show the $^{77}$Se Knight shift as a function of Cu-content $x$ for these samples, together with the data for other samples reported previously \cite{MatanoKrienerSegawaEtAl2016,Kawai}. The Knight shift, which is proportional to the density of states, decreases with decreasing $x$, %indicating 
being consistent with a smaller carrier concentration for the new crystals  than that for the previous ones with $x\geq$0.28.
 The $^{77}$Se-NMR spectrum taken at $H$=1.5 T and $T$=3.0 K above $T_{c}(H)$ was shown in the inset to the figure. 
The spectrum  becomes sharper  with decreasing $x$, with the full width at half maximum (FWHM) of 14.4 kHz and 14.7 kHz for $x$=0.15 and $x$=0.20, respectively,   which is narrower than that for  $x$=0.27 (16.5 kHz) and $x\geq$0.28 \cite{MatanoKrienerSegawaEtAl2016,Kawai} and thus ensures that these low-doping samples have a less disorder.   %Also, the Knight shift for $x$=0.15 and 0.20 is smaller than $x$=0.28 reported previously, ensuring that the carrier doping of these two samples is indeed smaller.
%The obtained \tc\ and  shielding fraction (SF) for most samples are close to the values reported by Kriener et al.\cite{Kriener_PhysRevB.84.054513}. The SF for  $x$=0.46 is the highest (56.2\%) among those reported so far.
%In Table 1 we list the properties for the five samples that we will discuss in this paper. 
%


%\vspace{1cm}
%\noindent
%\textbf{Evolution of the symmetry with magnetic fields.} % in the angle dependence of the magnetic susceptibility.}
Figures   \ref{x=0.15}-\ref{x=0.27} %and \ref{polarplot2} 
show the main results of this work.
Figures   \ref{x=0.15}(a),  Fig.\ref{x=0.2}(a) and Fig.\ref{x=0.27}(a) depict the angle $\varphi$ dependence of the diamagnetism measured by ac susceptibility at $T$=1.4 K (see Fig.1(b) for an example) under various magnetic fields. 
Here $\varphi$ is the angle between the crystal $a$-axis and the magnetic field. The pre-determined $a$-axis direction is set to be $\varphi$=0 degree.
The plotted quantity is related to  $H_{\rm c2}$; the larger  $H_{\rm c2}$, the larger diamagnetism at a fixed temperature and field.
At $H$= 0.5 T, a (nearly) two-fold symmetry is observed, in agreement with previous Knight shift \cite{MatanoKrienerSegawaEtAl2016} and $H_{\rm c2}$ \cite{MatanoKrienerSegawaEtAl2016,Yonezawa_Natphys,Kawai} measurements. 
At $H$=1.15 T (1.25 T for $x$=0.20 and 1 T for $x$=0.27), new components emerge in the oscillation, and surprisingly, there emerges six minima in the oscillation   at $H$=1.5 T. The situation is better visualized in the polar plots in Fig.   \ref{x=0.15}(b), Fig.   \ref{x=0.2}(b) and Fig. \ref{x=0.27}(b). A  six-fold symmetry is clearly seen at  $H$=1.5 T.
%	
\begin{figure}[htbp]
	\includegraphics[clip,width=90mm]{Fig3.pdf}
	\caption{\label{x=0.15}(color online) 
		%\textbf{
			Field- and angle-evolution of the diamagnetism and $\textbf{{d}}$-vector revealed by the ac susceptibility measurement for $x$=0.15. (a) In-plane angle dependence of the diamagnetism  at $T$=1.4 K measured by the ac susceptibility under various fields.	(b) Polar plot of the ac susceptibility at $H$=0.5 T and 1.5 T, respectively. The purple curve in the lower panel is a simulation of an anisotropic $H_{\rm c2}$ formula (see text). In the upper panel, the black curve connects the outer envelop of the three ellipses.  (c) Illustration of the  $\textbf{{d}}$-vector(s) by arrow(s) at different fields. The purple balls depict Bi, and the red and green balls represent Se in the bird-viewed basal planes.
	}
\end{figure}
%
\begin{figure}[htbp]
	\includegraphics[clip,width=90mm]{Fig4.pdf}
	\caption{\label{x=0.2}(color online) 
		%\textbf{Field- and angle-evolution of the diamagnetism and $\textbf{{d}}$-vector  for $x$=0.20.}
		(a) In-plane angle dependence of the ac susceptibility of the $x$=0.20 crystal at $T$=1.4 K for various fields.	(b) Polar plot of the ac susceptibility at $H$=0.5 T and 1.5 T. In the lower panel, red and blue curves are the simulations of the ac $\chi$ from two different domains. The thick purple curve is the outer envelop of the two ellipses. For other versions of the simulation curves, see Supplemental Material \cite{Supp} (c) Illustration of the  $\textbf{{d}}$-vector direction(s).
	}
\end{figure}

%

\begin{figure}[htbp]
	\includegraphics[clip,width=90mm]{Fig5.pdf}
	\caption{\label{x=0.27}%color online) 
		%\textbf{
		(color online)	Field- and angle-evolution of the diamagnetism and $\textbf{{d}}$-vector  for $x$=0.27.
		 (a) In-plane angle dependence of the diamagnetism  at $T$=1.8 K measured by the ac susceptibility under various fields for $x$=0.27.	(b) Polar plot of the ac susceptibility at $H$=0.5 T and 1.5 T, respectively. The captions for the curves are the same as Fig. 4(b). (c) Illustration of the  $\textbf{{d}}$-vector direction(s) at different fields.
	}
\end{figure}





%\section{Discussion} 
%\noindent
%\textbf{Sub-dominant component.}
The data in the lower panel of  Fig.\ref{x=0.15}(b) can be fitted by a phenomenological formula, 
%\begin{eq}
	-$\chi$ = - $\frac{|\chi_{max}|}{\sqrt{cos^2(\varphi-\theta)+c\cdot sin^2(\varphi-\theta)}}$.
%	\end{eq}
Here, $|\chi_{max}|$ is the largest diamagnetic susceptibility and  $\theta$ is the angle between the elliptical direction and the $a$-axis, and $c$ is a number parameter that determines the shape of the  ellipse. The parameter $c$ obtained from the fittings in  Fig.\ref{x=0.15}(b), Fig.\ref{x=0.2}(b) and Fig.\ref{x=0.27}(b) is 1.3, 1.2 and 1.4, respectively. A smaller $c$ makes the shape more peanuts-shape like while a larger $c$ makes the shape more ellipse-like. This formula was originally developed  to describe the $H_{\rm c2}$ anisotropy \cite{Hc2-anisotropy} and was also adapted in Ref. \cite{Sr_dope_2fold_PanNikitinAraiziEtAl2016}.

Although the data for $x$=0.15 %can be fitted by 
agree well with the simulation of a single ellipse (a single component), the data for $x$=0.20 and 0.27 do not. There is a small second component, as can be seen in the lower panels of  Fig.\ref{x=0.2}(b) and Fig.\ref{x=0.27}(b). For $x$=0.20, the $\theta$ for the small, second component is  340$^{\circ}$, while it is 80$^{\circ}$ for the main component. For $x$=0.27,  the  $\theta$ is  355$^{\circ}$  for the main component, while it is 95$^{\circ}$ for the second component.
In either case, the sub-dominant component disappears at high fields beyond 1.5 T,  which indicates that it arises from a different crystal domain with a lower 
$H_{\rm c2}$  (see below for more discussion). 
The single domain seen in $x$=0.15 can be understood as due to its lower doping which makes the crystal more homogeneous as evidenced by the sharper NMR spectrum.
Theoretically, two degenerate gap states corresponding to $p_x$ and $p_y$ were proposed by Fu \cite{Fu_CuxBi2Se3_PhysRevB.90.100509}.
The main component for $x$=0.15 and $x$=0.20 is compatible with  $p_x$, but that for $x$=0.27 is compatible with  $p_y$ symmetry. In the previous work, $H_{\rm c2}$ measurements found that  $x$=0.3 and 0.37 correspond to $p_x$ while $x$=0.28 and 0.36 correspond to $p_y$ \cite{MatanoKrienerSegawaEtAl2016,Kawai}. 

%\vspace{1cm}
%\noindent
%\textbf{$\textbf{{d}}$-vector tilted from high-symmetry direction  and phonon interaction.}
Looking into the detail of the lower panels of Fig. \ref{x=0.15}(b), Fig. \ref{x=0.2}(b) and Fig. \ref{x=0.27}(b), one notices that the elliptical axis is tilted away from the high-symmetry (crystal axis) direction. As the  $\textbf{{d}}$-vector is perpendicular to the main axis of the  $H_{\rm c2}$ ellipse %ellipse direction 
\cite{MatanoKrienerSegawaEtAl2016,Kawai},  we  illustrate the corresponding $\textbf{{d}}$-vector for the present three crystals in the lower panels of Fig. 3(c), Fig. 4(c) and Fig. 5(c). For $x$=0.20, the $\textbf{{d}}$-vector for the main component is tilted from the high symmetry axis by $\delta$=-10$^{\circ}$, while the tilting angle is  $\delta$=10$^{\circ}$ for the sub-dominant component. 
So the two components can be regarded as belonging to the same gap symmetry in view of the trigonal crystal symmetry.
For $x$=0.27, the $\textbf{{d}}$-vectors is  85$^{\circ}$, which is close to 90$^{\circ}$, for the main component, while it is  5$^{\circ}$ for the minor component. Therefore,  the two components correspond to different symmetries that are orthogonal to each other.
%Second,  there exists a tiny sub-dominant component, in addition to the two-fold symmetry.  As can be seen in \ref{x=0.2two-comp}, the data can be well reproduced by a main component directing to the 79$^{\circ}$  ($\sim \frac{5}{12}\pi$) direction, and a sub-dominant component directing to the 330$^{\circ}$ ($\sim \frac{11}{12}\pi$) direction. For $x$=0.15, the angle is  80$^{\circ}$ and for $x$=0.27, the angle is 168.5-348$^{\circ}$, which is close $\sim \frac{11}{12}\pi$. 
%
%\begin{figure}[htbp]
%	\includegraphics[clip,width=90mm]{x=0.2_two-comp_fitted.eps}
%	\caption{\label{x=0.2two-comp}%color online) 
%		H-evolution of the ac susceptibility.  
%	}
%\end{figure}
%
%There are two possible causes for such results. The first, and obvious, possibility is that there are two domains. However, that would mean that the $H_{\rm c2}$ is quite different for the two domains which should result in  two different $T_c$. As no indication of such, we believe that the results stem from an intrinsic property of the material. And
%
We believe that the   cause for the tilting %deviation 
from the high-symmetry axis direction is  phonon-mediated interaction. % induced distortion.
%
Hecker and  Fernandes \cite{phonon} recently proposed that the competition between a quadratic phonon-mediated interaction $E_{\rm nem-ph}$ and the cubic
nematic anharmonicity can make the nematic director %(the long axis of the elliptical shape of  $H_{\rm c2}$) 
deviate from the high-symmetry direction. The intrinsic nematic anharmonic cubic interaction $E_{\rm nem}$ favors the nematic director to align parallel to the high-symmetry directions, while the phonon-mediated non-analytic quadratic interaction 
prefers the nematic director to
align to the directions farthest away from
the high-symmetry directions. 


Hecker and  Fernandes further 
pointed out that the phonon interaction and thus the nematic director ($\textbf{{d}}$-vector) rotation will have an impact on domain formations \cite{phonon}. If the $\textbf{{d}}$-vector is along the high symmetry axis, the three equivalent directions have the same angular separation,
and  the surface energies between any two domains
are equal. As a result, when one direction among the three is
chosen as the majority-domain $\textbf{{d}}$-vector, the other two directions will be
randomly picked as minority-domains $\textbf{{d}}$-vector \cite{phonon}. However, 
%the subdomains will correlate with the major domain due to surface energy; 
if the $\textbf{{d}}$-vector is rotated by an angle $\delta$ away from the crystal axis, in order to minimize the surface energy, then there will only be one type of minor domain %instead of two otherwise, 
with the   $\textbf{{d}}$-vector rotated by an angle of -$\delta$. This is exactly we have found for the $x$=0.20  crystal.
%The angle $\phi$=79$^{\circ}$ is close to the anti-symmetric direction $\frac{5}{12}\pi$, so is the tiny component along 330$^{\circ}$ ($\sim \frac{11}{12}\pi$).
%Once the the phonon interaction exceeds a threshold value, the tilt of the   nematic director away from the high symmetry direction will occur, and as a result, the  $H_{\rm c2}$ shape will no longer be two-fold symmetric but a subdominant component will inevitally emerge.  

%different scenario:

%x=0.15 single domain

%x=0.20 two domains, larger one: 4x state, smaller one: 4y-state, but at high fields: 4x-state only

%measured at PF6?

%x=0.27 three domains, two larger one (a and b): 4y state. a,  tilt to become 4x but not depinned  even at 3.5T. b,  4y state,  depinned and becomes 4x. c, along 0-90 degree at H=0, tinny

%Question: is x=0.27 same as Kawai paper in terms of Hc2? 

%\vspace{1cm}
%\noindent
%\textbf{Depinning of the $\textbf{{d}}$-vector.}
At low fields, the two-fold nematic behavior in the physical quantities is well understood as due to the \dv\ pinning to a certain direction by, {\it e.g.}, spin-orbit coupling promoted by disorder or defects,
%the interations such as spin-orbit coupling, 
although there are three equivalent directions favored by the $\textbf{{d}}$-vector.    
 % impurity or crystal distortion.
 In superfluid $^3$He, the $\textbf{{d}}$-vector rotates freely as there is no lattice.
As can be seen in Figs.3-5, the  oscillation in $H_{\rm c2}$ restores a  six-fold symmetry at high fields of $H\geq$1.5 T, compatible with 
%as expected by 
the trigonal crystal-lattice  symmetry. Namely, the three $a$-axis directions become equivalent at high fields, as 
% The corresponding $\textbf{{d}}$-vector directions are 
 depicted in the upper panels of Fig. 3(c), Fig.4(c) and Fig. 5(c).
 This means that the  $\textbf{{d}}$-vector is depinned by the application of a large magnetic field. 
 We should note that a possible change in vortex lattice structure cannot explain the observed symmetry change. Firstly, any vortex lattice cannot give rise to a two-fold symmetry  in physical quantities. Secondly, a higher order effect could lead to a response of six-fold symmetry at high fields in principle, but in such case a circular-like shape should appear at low fields \cite{Ichioka},  which is not seen in our experiments.  
 
 The simplest interpretation of the symmetry transition in  $H_{\rm c2}$ is through the total energy $E_{\rm tot} = E_{\rm nem} + E_{\rm nem-ph}+ E_{\rm Zeeman}$ including Zeeman energy $E_{\rm Zeeman}$. Above a threshold value $H_{\rm pin}$, the $\textbf{{d}}$-vector %always
  traces the rotating magnetic field as to gain Zeeman energy which is the largest when the $\textbf{{d}}$-vector is perpendicular to the field. 
 The energy gain by $E_{\rm nem}$ is the largest when the $\textbf{{d}}$-vector is along the three high-symmetry directions.
  The $H_{\rm pin}$ is 1.0$\sim$1.2 T in the case of Cu$_{x}$Bi$_2$Se$_3$. This is the first case, to our knowledge, that the $\textbf{{d}}$-vector can be manipulated by the magnetic field.
  Also, a six-fold symmetry of  $H_{\rm c2}$ or its related physical quantities has never been observed before in  any superconductor, although theories have pointed out that a trigonal superconductor with a two-component order-parameter may show such property under certain conditions \cite{Fu-Hc2,Mineev}. Therefore, our result is supplemental to the Knight shift measurement which revealed a  spin-triplet state in Cu$_{x}$Bi$_2$Se$_3$.
 

%\vspace{1cm}
%\noindent
%\textbf{Gap symmetry.}
The most intriguing feature is that,  although the long axis direction  of the ellipse for the angle-dependent Meissner diamagnetism is tilted way from the high symmetry axis directions  of the  trigonal crystal lattice at low fields, %$H\leq$0.5 T, 
with $\theta$=80$^{\circ}$ for $x$=0.15 and  $x$=0.20, 
 it is restored to the high symmetry axis directions at high fields $H\geq$1.5 T. 
 This means that the phonon-induced interaction  $E_{\rm rem-ph}$ is also smaller than the Zeeman interaction. % which, in terms of magnetic field, is $H_{\rm pin}$ =  1.0$\sim$1.2 T. 
%More importantly, s
Such symmetry at high magnetic fields is compatible with the so-called $\Delta_{4x}$ ($p_x$) state proposed \cite{Fu_CuxBi2Se3_PhysRevB.90.100509}. %, but not the $\Delta_{4y}$ state.
In contrast, the ellipse for $x$=0.27 at low fields has an elliptical direction along  $\theta$=355$^{\circ}$ and thus the gap
 is compatible with the other state, the so-called $\Delta_{4y}$  ($p_y$) state. %, which is degenerate with $\Delta_{4x}$ at zero field. 
The previously reported crystals of $x$=0.28 and 0.36 also showed such symmetry.
Most strikingly and surprisingly,  the $\Delta_{4y}$  ($p_y$)  compatible elliptical shape at low fields %of the diamagnetism 
is also restored to the symmetry of  $\Delta_{4x}$ ($p_x$) at high fields, as seen in the upper panels of Fig.5(b) and Fig.5(c).
We have investigated more than 16 crystals with 0.15$\leq x \leq$0.40, and found that all of them become  compatible with the $\Delta_{4x}$ state after the \dv\ is depinned at high magnetic fields, although  the  $\theta$ is different from crystal to crystal at low fields and some of them  are consistent with the $\Delta_{4y}$ symmetry.
This implies that the intrinsic gap symmetry is $\Delta_{4x}$. We speculate that the  $\Delta_{4y}$ appearing  in some cases  is accidental due to local defects caused by the quenching process. This is an open question that needs to be addressed in the future. %crystal  environment.



%\section{Conclusion}
%\section{summary} 
%\textit{Conclusion.} 
In conclusion, for the first time,  we have successfully manipulated the nematic director by magnetic fields.  
The diamagnetism in single crystal samples of  \cx\ measured by  ac susceptibility shows a two-fold symmetry with respect to the angle between the field and the crystal $a$ axis at a low field $H$ = 0.5 T, with the direction of largest diamagnetism slight deviated from the crystal axis. %, with 90 degree phase difference between the two. 
At high fields of
 $H \geq$ 1.5 T, however, the ac susceptibility shows a six-fold symmetry, exactly matching the   crystal axes. These results indicate that the $\textbf{{d}}$-vector initially pinned to a certain direction is unlocked  by  the magnetic fields above a threshold value to trace the field. The six-fold symmetry in  $H_{\rm c2}$ or its related physical quantities  was expected for a spin-triplet superconductor with two-component order parameter, but has never been observed before. Thus, our results are supplemental to the Knight shift result which found  that Cu$_x$Bi$_2$Se$_3$ is a spin-triplet superconductor.
 Our work further reveals the $p_x$ gap  symmetry at high fields for all samples, irrespective of different symmetries at low fields, indicating that this is the intrinsic gap symmetry of Cu$_x$Bi$_2$Se$_3$.

%Four samples with different Cu-content $x$ = 0.28, 0.37, 0.46 and 0.54 were successfully synthesized.
%By NMR measurements, we found for the first time that the carrier density increased further by Cu doping beyond $x$=0.37.
%By magnetic susceptibility measurements, we found that the in-plane \hct\ shows a clear two-fold oscillation for the samples with $x$ = 0.28, 0.36 and 0.37 which have similar carrier density as evidenced by the Knight shift.
%However, the angle at which \hct\ becomes minimal is different by 90 degrees %$^{\circ}$ 
%between different samples. This indicates that the  direction of the \dv\ is different from crystal to crystal
%due to a different local structure caused by  strain during the quenching process,  dopant-induced crystal distortion, etc.. 
%For one sample the angle of the minimum \hct\ is in the Se-Se crystallographic axis direction
%and the other is in perpendicular to Se-Se crystallographic axis direction.
%In the samples with $x$ = 0.46 and 0.54, the %we discovered the suppression of
%two-fold oscillation is suppressed, which indicates a gap symmetry change from nematic to isotropic as carrier density increases. These findings enriched the contents of topological superconductivity in doped Bi$_2$Se$_3$, and we hope that our work will stimulate further studies on possibly even more exotic superconducting state (possible chiral state) in the high-doped region of \cx\  as well as on bulk topological superconductors in general.
%Two different minimum directions and suppression of two-fold oscillation indicate the sample dependence of pinning direction of \dv\
%due to structural origin such as distortion, dopant position, etc.

%\section{Materials and Methods}
%\noindent
%\textbf{Single crystal growth and characterization.}

%The ac-$\chi$ vs $H$ data in the normal state were fitted by a linear function (a constant line). \hct\  was defined as a point off the straight line. A typical example is shown in Supplementary Information.

%\noindent
%\textbf{Magnetoresistance measurements}. The angle-dependent electrical resistance was measured by the standard four-electrode method in a Physical Properties Measurement System (PPMS, Quantum Design) with a mechanical rotating probe. The %removing of the epoxy and the 
%building of  electrodes were carried out in a glove box filled with high-purity Ar gas to prevent sample from degradation. 
%The electrodes were made such that the current direction is %approximately 
%along the $a$-axis. %The sample was exposed to air for less than 20s during transferring the sample to PPMS. 
%The excitation currents are 0.1-1 mA to make a compromise of the Joule heating and the measurement accuracy. 

%\begin{acknowledgments}
\section{Acknowledgments}
We thank Y. Inada for help in Laue diffraction measurements, S. Kambe  for advice in crystal growing,  S. Kawasaki for help in susceptibility measurements, and M. Ichioka and R. Fernandes   for useful discussions. %for help in some of the \hct\  measurements. %, Markus Kriener and T. Mizushima  for useful discussions. 
This work was supported in part by the JSPS Grants No. 19H00657, No. 20K03862 and No. 22H0448 (Grant-in-Aid for Scientific Research on Innovative Areas “Quantum Liquid Crystals”).
%\end{acknowledgments}
%
% \section{Authors contributions}
%G.-q.Z planned and supervised the project. Yokoyama  and H. Nishigaki   synthesized  the  single crystals. % under the supervision of Y.A.
%Yokoyama,  H. Nishigaki, Ogawa, Nitta and K.M.  performed magnetic susceptibility  measurements,    Shiokawa, Yokoyama,  H. Nishigaki, and  K.M conducted NMR measurements. G.-q.Z wrote the manuscript.  All authors discussed the results and interpretation.  %All authors have discussed the results and the interpretation.


%\section{Competing financial interests}
%The authors declare no competing  interests.

%\section{Materials and Correspondence}
%Supplementary information is available in the online version of the paper.
%Correspondence and requests for materials should be addressed to G.-q.Z.
%

%\section{Data availability}
%The data that support the findings of this study are available on reasonable request.
%\bibliography{E:/papers/topological,E:/papers/SC_discover,E:/papers/Bi2Se3,E:/papers/my,E:/papers/PbTaSe2}
\begin{thebibliography}{10}
	
\bibitem{Pflei}
U. K. R\"{o}{\ss}ler, A. N. Bogdanov, C. Pfleiderer, 
Spontaneous skyrmion ground states in magnetic metals. 
\textit{Nature} \textbf{442}, 797 (2006).

\bibitem{Fernandes-Shmalien-Chubukov}	
R. M. Fernandes, A. V. Chubukov, J. Schmalian, 
What drives nematic order in iron-based superconductors? 
\textit{Nature Physics} \textbf{10}, 97–104 (2014).

\bibitem{MatanoKrienerSegawaEtAl2016}
K. Matano, M. Kriener, K.  Segawa, Y. Ando, G.-q.  Zheng,  
Spin-rotation symmetry breaking in the superconducting state of Cu$_x$Bi$_2$Se$_3$.
\textit{Nature Physics} \textbf{12}, 852 (2016).

\bibitem{Yang}
J. Yang, J. Luo, C. J. Yi, Y. G. Shi, Y. Zhou, G.-q.Zheng,
Spin-triplet superconductivity in K$_2$Cr$_3$As$_3$. 
\textit{Sci. Adv.} \textbf{7}, eabl4432 (2021).

\bibitem{YCao}
Y. Cao, D. Rodan-Legrain, J. M. Park, N. F. Q. Yuan, K. Watanabe,
Taniguchi, T.,  Fernandes, R. M.,  Fu, L., \& Jarillo-Herrero, P.,
Nematicity and competing orders in superconducting
magic-angle graphene, 
\textit{Science} \textbf{372}, 264–271 (2021).
		
 	
\bibitem{Babaev}
A. A. Zyuzin, J. Garaud,  \& E. Babaev, 
Nematic Skyrmions in Odd-Parity Superconductors. 
\textit{Phys. Rev. Lett.} \textbf{119}, 167001 (2017).

\bibitem{FuPRL}
L. Fu,  E. Berg,  
Odd-Parity Topological Superconductors: Theory and Application to Cu$_x$Bi$_2$Se$_3$, 
{\it Phys. Rev. Lett.} {\bf 105}, 097001 (2010).


\bibitem{QiZhang}
%X.-L.	Qi,  and S.-C. Zhang, % Topological insulators and superconductors.
% Rev. Mod.	Phys. {\bf 83}, 1057 (2011).
X.-L. Qi, S.-C. Zhang,
Topological insulators and superconductors.
\textit{Rev. Mod. Phys.} \textbf{83}, 1057-1110 (2011). 

\bibitem{Ivanov}
D. A. Ivanov,
Non-Abelian Statistics of Half-Quantum Vortices in $p$-Wave Superconductors.
\emph{Phys. Rev. Lett.}  {\bf 86}, 268 (2001).

%\bibitem{Sato_Ando_review_2017}
%M. Sato and Y. Ando, Reports on Progress in Physics \textbf{80}, 076501 (2017).
%Sato, M. \& Ando, Y. Topological superconductors: a review. \textit{Reports on Progress in Physics} \textbf{80}, 076501 (2017).

%\bibitem{Fu_PhysRevLett.100.096407}
%L. Fu and C. L. Kane, Phys. Rev. Lett. \textbf{100}, 096407 (2008).
%Fu, L. \& Kane, C. L. Superconducting Proximity Effect and Majorana Fermions at the Surface of a Topological Insulator. \textit{Phys. Rev. Lett.} \textbf{100}, 096407 (2008).

%\bibitem{MajoranaFermion_Qbit_PhysRevLett.94.166802}
%K. T. Law, P. A. Lee, and T. K. Ng, Phys. Rev. Lett. \textbf{103}, 237001 (2009).

\bibitem{TopologicalQuantumComputation}
M. H. Freedman,  A. Kitaev, M. J. Larsen,  Z. Wang, 
Topological quantum computation.
\textit{Bull. Amer. Math. Soc.} \textbf{40} 31-38 (2003).

%\bibitem{PhysRevLett.94.166802}
%S. Das Sarma, M. Freedman, and C. Nayak, Phys. Rev. Lett. \textbf{94}, 166802 (2005).

\bibitem{quantumcomputer_KITAEV20032}
A. Kitaev, 
Fault-tolerant quantum computation by anyons.
\textit{Annals of Physics} \textbf{303}, 2 (2003).



\bibitem{UTe2}
S. Ran, C. Eckberg, Q-P. Ding, Y. Furukawa, T.  Metz, S. R. Saha, I-L. Liu, M.  Zic, H. Kim, J.  Paglione, N. P. Butch,
Nearly ferromagnetic spin-triplet superconductivity.
\emph{Science} {\bf 365}, 684 (2019).

\bibitem{Balian}
R. Balian, N. R. Werthamer,
Superconductivity with Pairs in a relative p wave.
\emph{Phys. Rev.} \textbf{131}, 1553 (1963).


\bibitem{Leggett}
A. J. Leggett, , 
A theoretical description of the new phases of liquid $^{3}$He.
\emph{Rev. Mod. Phys.} \textbf{47}, 331 (1975).

\bibitem{Yonezawa_Natphys}
S. Yonezawa, K. Tajiri, S.  Nakata, Y.  Nagai,  Z. Wang, K.  Segawa, Y. Ando , Y. Maeno, , 
Thermodynamic evidence for nematic superconductivity in Cu$_x$Bi$_2$Se$_3$.
\textit{Nature Physics}, \textbf{13}, 123 (2016).
%Yonezawa, S., Tajiri, K., Nakata, S., Nagai, Y., Wang, Z., Segawa, K., Ando, Y. \& Maeno, Y. Thermodynamic evidence for nematic superconductivity in CuxBi2Se3. \textit{Nature Physics} \textbf{13}, 123- (2016).

\bibitem{Sr_dope_2fold_PanNikitinAraiziEtAl2016}
%Y. Pan, A. M. Nikitin, G. K. Araizi, Y. K. Huang, Y. Matsushita, T. Naka, and A. de Visser,
%Scientific Reports \textbf{6}, 28632 (2016).
Y. Pan, A. M. Nikitin, G. K. Araizi, Y. K. Huang, Y. Matsushita, T. Naka,  A. de Visser,  
Rotational symmetry breaking in the topological superconductor Sr$_x$Bi$_2$Se$_3$ probed by upper-critical field experiments. 
\textit{Scientific Reports} \textbf{6}, 28632- (2016).

%\bibitem{Nikitin_PhysRevB.94.144516}
%A. M. Nikitin, Y. Pan, Y. K. Huang, T. Naka, and A. de Visser, Phys. Rev. B \textbf{94}, 144516
%(2016).
%Nikitin, A. M., Pan, Y., Huang, Y. K., Naka, T. \& de Visser, A. High-pressure study of the basal-plane anisotropy of the upper critical field of the topological superconductor Sr$_x$Bi$_2$Se$_3$. \textit{Phys. Rev. B} \textbf{94}, 144516 (2016).

\bibitem{Asaba_Sr_PhysRevX.7.011009}
T. Asaba, B. J. Lawson, C. Tinsman, L. Chen, P.  Corbae, G. Li, Y. Qiu, Y. S. Hor, L. Fu, L. Li,  
Rotational Symmetry Breaking in a Trigonal Superconductor Nb-doped Bi$_2$Se$_3$.
\textit{Phys. Rev. X} \textbf{7}, 011009 (2017).
%Asaba, T., Lawson, B. J., Tinsman, C., Chen, L., Corbae, P., Li, G., Qiu, Y., Hor, Y. S., Fu, L. \& Li, L. Rotational Symmetry Breaking in a Trigonal Superconductor Nb-doped Bi$_2$Se$_3$. \textit{Phys. Rev. X} \textbf{7}, 011009 (2017).

\bibitem{Sr_dope_2fold_Du2017}
G. Du, Y. Li, J. Schneeloch, R. D. Zhong, G.  Gu, H. Yang, H. Lin, H.-H. Wen, ,
Superconductivity with two-fold symmetry in topological superconductor SrxBi2Se3.
\textit{Science China Physics, Mechanics \& Astronomy} \textbf{60}, 037411 (2017).
%Du, G., Li, Y., Schneeloch, J., Zhong, R. D., Gu, G., Yang, H., Lin, H. \& Wen, H.-H. Superconductivity with two-fold symmetry in topological superconductor SrxBi2Se3. \textit{Science China Physics, Mechanics \& Astronomy} \textbf{60}, 037411 (2017).

\bibitem{FengDL}
%Ran Tao, Ya-Jun Yan, Xi Liu, Zhi-Wei Wang, Yoichi Ando, Qiang-Hua Wang, Tong Zhang, and Dong-Lai Feng
%Phys. Rev. X {\bf 8}, 041024 (2018).
R. Tao, Y.-J. Yan, X. Liu, Z.-W. Wang, Y. Ando, Q.-H. Wang, T. Zhang,  D.-L. Feng
Direct Visualization of the Nematic Superconductivity in Cu$_x$Bi$_2$Se$_3$. \textit{Phys. Rev. X} \textbf{8}, 041024 (2018).

\bibitem{Kawai}
T. Kawai, C.G.  Wang, Y. Kandori, Y. Honoki, K.  Matano,  T. Kambe, G.-q. Zheng, , 
Direction and symmetry transition of the vector order parameter in topological superconductors Cu$_x$Bi$_2$Se$_3$,
\textit{Nat. Commun. }\textbf{11}, 235 (2020).

\bibitem{chiral_PhysRevB.94.180504}
%J. W. F. Venderbos, V. Kozii, and L. Fu, Phys. Rev. B \textbf{94}, 180504 (2016).
J. W. F. Venderbos, V. Kozii,  L. Fu,  
Odd-parity superconductors with two-component order parameters: Nematic and chiral, full gap, and Majorana node.
\textit{Phys. Rev. B} \textbf{94}, 180504 (2016).



\bibitem{Kriener_PRB2011}
%M. Kriener, K. Segawa, Z. Ren, S. Sasaki, S. Wada, S. Kuwabata, and Y. Ando, Phys. Rev. B
%\textbf{84}, 054513 (2011).
M. Kriener, K. Segawa, Z. Ren, S. Sasaki, S. Wada, S. Kuwabata,  Y. Ando, 
Electrochemical synthesis and superconducting phase diagram of Cu$_x$Bi$_2$Se$_3$.
\textit{Phys. Rev. B} \textbf{84}, 054513 (2011).

\bibitem{Kriener_PRB2012}
%M. Kriener, K. Segawa, S. Sasaki, \& Y. Ando, Phys. Rev. B \textbf{86}, 180505 (2012).
M. Kriener, K. Segawa,  Sasaki, S. Y. Ando,  
Anomalous suppression of the superfluid density in the Cu$_x$Bi$_2$Se$_3$ superconductor upon progressive Cu intercalation.
\textit{Phys. Rev. B} \textbf{86}, 180505 (2012).

%\bibitem{Kouwen}
%V. Mourik, K. Zuo, S. M. Frolov, S. R. Plissard, E. P. A. M. Bakkers, L. P. Kouwenhoven, Science {\bf 336}, 1003 (2012).
%Mourik, V., Zuo, K., Frolov, S. M., Plissard, S. R., Bakkers, E. P. A. M. \& Kouwenhoven, L. P. Signatures of Majorana Fermions in Hybrid Superconductor-Semiconductor Nanowire Devices. \textit{Science} \textbf{336}, 1003-1007 (2012).

%\bibitem{Yazdani}
%S. Nadj-Perge, I. K. Drozdov, J. Li, H. Chen, S. Jeon, J. Seo, A. H. MacDonald, B. A. Bernevig, A. Yazdani,
%Science  {\bf 346}, 602 (2014).
%Nadj-Perge, S., Drozdov, I. K., Li, J., Chen, H., Jeon, S., Seo, J., MacDonald, A. H., Bernevig, B. A. \& Yazdani, A. Observation of Majorana fermions in ferromagnetic atomic chains on a superconductor. \textit{Science} \textbf{346}, 602-607 (2014).

%\bibitem{Jia}
%Hao-Hua Sun, Kai-Wen Zhang, Lun-Hui Hu, Chuang Li, Guan-Yong Wang, Hai-Yang Ma, Zhu-An Xu, Chun-Lei Gao, Dan-Dan Guan, Yao-Yi Li, Canhua Liu, Dong Qian, Yi Zhou, Liang Fu, Shao-Chun Li, Fu-Chun Zhang, and Jin-Feng Jia, Phys. Rev. Lett. {\bf 116}, 257003 (2016).
%Sun, H.-H., Zhang, K.-W., Hu, L.-H., Li, C., Wang, G.-Y., Ma, H.-Y., Xu, Z.-A., Gao, C.-L., Guan, D.-D., Li, Y.-Y., Liu, C., Qian, D., Zhou, Y., Fu, L., Li, S.-C., Zhang, F.-C. \& Jia, J.-F. Majorana Zero Mode Detected with Spin Selective Andreev Reflection in the Vortex of a Topological Superconductor. \textit{Phys. Rev. Lett.} \textbf{116}, 257003 (2016).

%\bibitem{Ding}
%D. Wang, L. Kong, P. Fan, H. Chen, S. Zhu, W. Liu, L. Cao, Y. Sun, S. Du, J. Schneeloch, R. Zhong, G. Gu, L. Fu, H. Ding, and H.-J. Gao,
%Science {\bf 362}, 333 (2018).
%Wang, D., Kong, L., Fan, P., Chen, H., Zhu, S., Liu, W., Cao, L., Sun, Y., Du, S., Schneeloch, J., Zhong, R., Gu, G., Fu, L., Ding, H. \& Gao, H.-J. Evidence for Majorana bound states in an iron-based superconductor. \textit{Science} \textbf{362}, 333-335 (2018).

%\bibitem{Senthil}
%T. Senthil,J. B. Marston and M. P. A. Fisher, Phys. Rev. B {\bf 60}, 4245 (1999).
%Senthil, T., Marston, J. B. \& Fisher, M. P. A. Spin quantum Hall effect in unconventional superconductors. \textit{Phys. Rev. B} \textbf{60}, 4245-4254 (1999).

%\bibitem{Wilczek}
%Wilczek, F. Majorana returns. \textit{Nat. Phys.} \textbf{5}, 614 (2009).
%\bibitem{Qi_TRS_SC_PhysRevLett.102.187001}
%Qi, X.-L., Hughes, T. L., Raghu, S. \& Zhang, S.-C. Time-Reversal-Invariant Topological Superconductors and Superfluids in Two and Three Dimensions. \textit{Phys. Rev. Lett.} \textbf{102}, 187001 (2009).

%\bibitem{Nishiyama}
%Nishiyama, M.,  Inada, Y. \& Zheng, G.-q.,
%Spin Triplet Superconducting State due to Broken Inversion Symmetry in Li$_2$Pt$_3$B.
%\textit{Phys. Rev. Lett.} \textbf{98}, 047002 (2007).


%\bibitem{Linter_Ferro_interface_PhysRevLett.104.067001}
%J. Linder, Y. Tanaka, T. Yokoyama, A. Sudbø, and N. Nagaosa, Phys. Rev. Lett. \textbf{104}, 067001(2010).

%\bibitem{sato_PhysRevB.79.094504}
%M. Sato and S. Fujimoto, Phys. Rev. B \textbf{79}, 094504 (2009).
%Sato, M. \& Fujimoto, S. Topological phases of noncentrosymmetric superconductors: Edge states, Majorana fermions, and non-Abelian statistics. \textit{Phys. Rev. B} \textbf{79}, 094504 (2009).

%\bibitem{tanaka_PhysRevLett.105.097002}
%Y. Tanaka, Y. Mizuno, T. Yokoyama, K. Yada, and M. Sato, Phys. Rev. Lett. \textbf{105}, 097002 (2010).
%Tanaka, Y., Mizuno, Y., Yokoyama, T., Yada, K. \& Sato, M. Anomalous Andreev Bound State in Noncentrosymmetric Superconductors. \textit{Phys. Rev. Lett.} \textbf{105}, 097002 (2010).

%\bibitem{sato_fujimoto_non-centro_PhysRevB.79.094504}
%M. Sato and S. Fujimoto, Phys. Rev. B \textbf{79}, 094504 (2009).
%\bibitem{Fu_Berg_PhysRevLett.105.097001}
%L. Fu and E. Berg, Phys. Rev. Lett. \textbf{105}, 097001 (2010).
%Fu, L. \& Berg, E. Odd-Parity Topological Superconductors: Theory and Application to Cu$_x$Bi$_2$Se$_3$. \textit{Phys. Rev. Lett.} \textbf{105}, 097001 (2010).

%\bibitem{sato_odd_paritySC_PhysRevB.81.220504}
%M. Sato, Phys. Rev. B \textbf{81}, 220504 (2010).
%Sato, M. Topological odd-parity superconductors. \textit{Phys. Rev. B} \textbf{81}, 220504 (2010).


%\bibitem{Nikitin_PhysRevB.94.144516}
%Pan, Y., Nikitin, A. M., Araizi, G. K., Huang, Y. K., Matsushita, Y., Naka, T. \& de Visser, A. Rotational symmetry breaking in the topological superconductor SrxBi2Se3 probed by upper-critical field experiments. \textit{Scientific Reports} \textbf{6}, 28632 (2016).

%\bibitem{Hor_PhysRevLett.104.057001}
%Y. S. Hor, A. J. Williams, J. G. Checkelsky, P. Roushan, J. Seo, Q. Xu, H. W. Zandbergen,
%A. Yazdani, N. P. Ong, and R. J. Cava, Phys. Rev. Lett. \textbf{104}, 057001 (2010).
%Hor, Y. S., Williams, A. J., Checkelsky, J. G., Roushan, P., Seo, J., Xu, Q., Zandbergen, H. W., Yazdani, A., Ong, N. P. \& Cava, R. J. Superconductivity in Cu$_x$Bi$_2$Se$_3$ and its Implications for Pairing in the Undoped Topological Insulator. \textit{Phys. Rev. Lett.} \textbf{104}, 057001 (2010).

%\bibitem{SrxBi2Se3_discover_PhysRevB.92.020506}
%Shruti, V. K. Maurya, P. Neha, P. Srivastava, and S. Patnaik, Phys. Rev. B \textbf{92}, 020506 (2015).

%\bibitem{Sr-dope_arXiv:1512.03519}
%Y. Qiu, K. N. Sanders, J. Dai, J. E. Medvedeva, W. Wu, P. Ghaemi, T. Vojta, and Y. S. Hor,
%arXiv:1512.03519 (2015).

%\bibitem{Sasaki_PhysRevLett.107.217001}
%Sasaki, S., Kriener, M., Segawa, K., Yada, K., Tanaka, Y., Sato, M. \& Ando, Y. Topological Superconductivity in Cu$_x$Bi$_2$Se$_3$. \textit{Phys. Rev. Lett.} \textbf{107}, 217001 (2011).

%\bibitem{Peng_NoZBCP_PhysRevB.88.024515}
%H. Peng, D. De, B. Lv, F. Wei, and C.-W. Chu, Phys. Rev. B \textbf{88}, 024515 (2013).
%Peng, H., De, D., Lv, B., Wei, F. \& Chu, C.-W. Absence of zero-energy surface bound states in Cu$_x$Bi$_2$Se$_3$ studied via Andreev reflection spectroscopy. \textit{Phys. Rev. B} \textbf{88}, 024515 (2013).

%\bibitem{STM_PhysRevLett.110.117001}
%N. Levy, T. Zhang, J. Ha, F. Sharifi, A. A. Talin, Y. Kuk, and J. A. Stroscio, Phys. Rev. Lett.
%\textbf{110}, 117001 (2013).
%Levy, N., Zhang, T., Ha, J., Sharifi, F., Talin, A. A., Kuk, Y. \& Stroscio, J. A. Experimental Evidence for $s$-Wave Pairing Symmetry in Superconducting Cu$_x$Bi$_2$Se$_3$ Single Crystals Using a Scanning Tunneling Microscope. \textit{Phys. Rev. Lett.} \textbf{110}, 117001 (2013).

%\bibitem{ARPES}
%L. Wray et al, arXiv:0912.3341.
%Wray, L., Xu, S., Xiong, J., Xia, Y., Qian, D., Lin, H., Bansil, A., Hor, Y., Cava, R. J. \& Hasan, M. Z. Observation of unconventional band topology in a superconducting doped topological insulator, Cu$_x$-Bi$_2$Se$_3$: Topological Superconductor or non-Abelian superconductor? arXiv:0912.3341 (2009).

%\bibitem{class}
%A. P. Schnyder,  S. Ryu, A. Furusaki, and A. W. W. Ludwig,
%Classification of topological insulators and superconductors in three spatial dimensions.
%Phys. Rev. B {\bf 78}, 195125 (2008).
%A. Altland and M.R. Zimbauer, Phys. Rev. B 55, 1142 (1997).
%Altland, A. \& Zirnbauer, M. R. Nonstandard symmetry classes in mesoscopic normal-superconducting hybrid structures. \textit{Phys. Rev. B} \textbf{55}, 1142-1161 (1997).

\bibitem{Supp}
See Supplemental Material for the curves of the individual ellipse and the outer envelop alone.

\bibitem{Fu_CuxBi2Se3_PhysRevB.90.100509}
%L. Fu, Phys. Rev. B \textbf{90}, 100509 (2014).
L. Fu, Odd-parity topological superconductor with nematic order: Application to Cu$_x$Bi$_2$Se$_3$. \textit{Phys. Rev. B} \textbf{90}, 100509 (2014).

\bibitem{Hc2-anisotropy}
R. C. Morris, R. V. Coleman,  R. Bhandarit,  
Superconductivity and Magnetoresistance in NbSe$_2$
\textit{Phys. Rev. B} \textbf{5}, 895 (1972).





%\bibitem{Smylie_Nb_PhysRevB.94.180510}
%M. P. Smylie, H. Claus, U. Welp, W.-K. Kwok, Y. Qiu, Y. S. Hor, and A. Snezhko, Phys. Rev.
%B \textbf{94}, 180510 (2016).
%Smylie, M. P., Claus, H., Welp, U., Kwok, W.-K., Qiu, Y., Hor, Y. S. \& Snezhko, A. Evidence of nodes in the order parameter of the superconducting doped topological insulator Nb$_x$Bi$_2$Se$_3$ via penetration depth measurements. \textit{Phys. Rev. B} \textbf{94}, 180510 (2016).

%\bibitem{Smylie_Nb_PhysRevB.96.115145}
%M. P. Smylie, K. Willa, H. Claus, A. Snezhko, I. Martin, W.-K. Kwok, Y. Qiu, Y. S. Hor,
%E. Bokari, P. Niraula, A. Kayani, V. Mishra, and U. Welp, Phys. Rev. B \textbf{96}, 115145 (2017).
%Smylie, M. P., Willa, K., Claus, H., Snezhko, A., Martin, I., Kwok, W.-K., Qiu, Y., Hor, Y. S., Bokari, E., Niraula, P., Kayani, A., Mishra, V. \& Welp, U. Robust odd-parity superconductivity in the doped topological insulator Nb$_x$Bi$_2$Se$_3$. \textit{Phys. Rev. B} \textbf{96}, 115145 (2017).

%\bibitem{Nb_dope_STM_PhysRevB.98.094523}
%A. Sirohi, S. Das, P. Neha, K. S. Jat, S. Patnaik, and G. Sheet, Phys. Rev. B \textbf{98}, 094523
%(2018).
%Sirohi, A., Das, S., Neha, P., Jat, K. S., Patnaik, S. \& Sheet, G. 
%Low-energy excitations and non-BCS superconductivity in Nb$_x$Bi$_2$Se$_3$ \textit{Phys. Rev. B} \textbf{98}, 094523 (2018).




%\bibitem{Tou_PhysRevLett.80.3129}
%H. Tou, Y. Kitaoka, K. Ishida, K. Asayama, N. Kimura, Y.
%Onuki, E. Yamamoto, Y. Haga, and K. Maezawa, Phys. Rev. Lett. \textbf{80}, 3129 (1998).

%\bibitem{IshidaMukudaKitaokaEtAl1998}
%K. Ishida, H. Mukuda, Y. Kitaoka, K. Asayama, Z. Q. Mao, Y. Mori, and Y. Maeno, Nature \textbf{396}, 658 (1998).

%\bibitem{Nishiyama_PhysRevLett.98.047002}
%M. Nishiyama, Y. Inada, and G.-q. Zheng, Phys. Rev. Lett. \textbf{98}, 047002 (2007).



%\bibitem{Sr_position_PhysRevMaterials.2.014201}
%Z. Li, M. Wang, D. Zhang, N. Feng, W. Jiang, C. Han, W. Chen, M. Ye, C. Gao, J. Jia, J. Li, S. Qiao, D. Qian, B. Xu, H. Tian, and B. Gao, 
%Phys. Rev. Materials \textbf{2}, 014201 (2018).
%Li, Z., Wang, M., Zhang, D., Feng, N., Jiang, W., Han, C., Chen, W., Ye, M., Gao, C., Jia, J., Li, J., Qiao, S., Qian, D., Xu, B., Tian, H. \& Gao, B. Possible structural origin of superconductivity in Sr-doped Bi$_2$Se$_3$. \textit{Phys. Rev. Materials} \textbf{2}, 014201 (2018).



%\bibitem{Hashimoto_JPSJ.82.044704}
%T. Hashimoto, K. Yada, A. Yamakage, M. Sato, and Y. Tanaka, J. Phys. Soc. Jpn. \textbf{82}, 044704 (2013).

%\bibitem{quench_condiction_PhysRevB.91.144506}
%J. A. Schneeloch, R. D. Zhong, Z. J. Xu, G. D. Gu, and J. M. Tranquada, Phys. Rev. B \textbf{91},
%144506 (2015).
%Schneeloch, J. A., Zhong, R. D., Xu, Z. J., Gu, G. D. \& Tranquada, J. M. Dependence of superconductivity in $Cu_xBi_2Se_3$ on quenching conditions. \textit{Phys. Rev. B} \textbf{91}, 144506 (2015).



%\bibitem{Kuntsevich_distortion}
%A. Y. Kuntsevich, M. A. Bryzgalov, V. A. Prudkoglyad, V. P. Martovitskii, Y. G. Selivanov,
%and E. G. Chizhevskii, New Journal of Physics \textbf{20}, 103022 (2018).
%Kuntsevich, A. Y., Bryzgalov, M. A., Prudkoglyad, V. A., Martovitskii, V. P., Selivanov, Y. G. \& Chizhevskii, E. G. Structural distortion behind the nematic superconductivity in Sr$_x$Bi$_2$Se$_3$. \textit{New Journal of Physics} \textbf{20}, 103022 (2018).

%\bibitem{Yip}
%P.T. How and S.-K. Yip, arXiv: 1901.11237.
%How, P.T., \& Yip, S.-K. 
%Signatures of Nematic Superconductivity in Doped Bi$_2$Se$_3$ under Applied Stress
%\textit{Phys. Rev. B} {\bf 100}, 134508 (2019).

%\bibitem{Bonn}
%Ramshaw, B. J. ,  Day, J., Vignolle, B,  LeBoeuf, D,  Dosanjh, P.,  Proust, C.,  Taillefer, L.,  Liang, R., Hardy,  W. N., \&   Bonn, D. A.
%Vortex lattice melting and $H_{c2}$ in underdoped YBa2Cu3Oy.
%Phys. Rev. B 86, 174501 (2012).

%\bibitem{Clark}


%\bibitem{Iwasa}
%Saito, Y.,  Nojima,T.  \&  Iwasa, Y.
%Quantum phase transitions in highly crystalline
%two-dimensional superconductors.
%NATURE COMMUNICATIONS {\bf 9},778 (2018).



%\bibitem{WangMartin}
%Wu, F. \& Martin, I. Nematic and chiral superconductivity induced by odd parity
%fluctuations. Phys. Rev. B {\bf 96}, 144504 (2017).

%\bibitem{mizushima_2018arXiv180906989U}
%H. Uematsu, T. Mizushima, A. Tsuruta, S. Fujimoto, and J. A. Sauls, arXiv e-prints ,
%arXiv:1809.06989 (2018).
%Uematsu, H., Mizushima, T., Tsuruta, A., Fujimoto, S. \& Sauls, J. A. Chiral Higgs Mode in Nematic Superconductors. \textit{Phys. Rev. Lett.} \textbf{123}, 237001 (2019).

%\bibitem{Chiral_PRB}
%L. Chirolli,
%PHYSICAL REVIEW B 98, 014505 (2018).
%Chirolli, L. Chiral superconductivity in thin films of doped Bi$_2$Se$_3$. \textit{Phys. Rev. B} \textbf{98}, 014505 (2018).

%\bibitem{WangQH}
%L. Yang and Q.-H. Wang, arXiv:1902.10915
%Yang, L., and Wang, Q.-H. 
%The direction of the $d$-vector in a nematic triplet superconductor
%arXiv:1902.10915 (2019)

%\bibitem{Arpes}
%E. Lahoud, E. Maniv, M. S. Petrushevsky, M. Naamneh, A. Ribak, S. Wiedmann, L. Petaccia, Z. Salman,
%K. B. Chashka, Y. Dagan, and A. Kanigel,
%Phy. Rev. B {\bf 88}, 195107 (2013).
%Lahoud, E., Maniv, E., Petrushevsky, M. S., Naamneh, M., Ribak, A., Wiedmann, S., Petaccia, L., Salman, Z., Chashka, K. B., Dagan, Y. \& Kanigel, A. Evolution of the Fermi surface of a doped topological insulator with carrier concentration. \textit{Phys. Rev. B} \textbf{88}, 195107 (2013).

%\bibitem{LiLu}
%B. J. Lawson, G. Li, F. Yu, T. Asaba, C. Tinsman, T. Gao, W. Wang, Y. S. Hor, and Lu Li, Phy. Rev. B {\bf 90}, 195141 (2014).
%Quantum oscillations in ${\mathrm{Cu}}_{x}{\mathrm{Bi}}_{2}{\mathrm{Se}}_{3}$ in high magnetic fields
%Lawson, B. J., Li, G., Yu, F., Asaba, T., Tinsman, C., Gao, T., Wang, W., Hor, Y. S. \& Li, L. Quantum oscillations in Cu$_x$Bi$_2$Se$_3$ in high magnetic fields. \textit{Phys. Rev. B} \textbf{90}, 195141 (2014).

\bibitem{phonon}
M. Hecker, R. M. Fernandes, 
Phonon-induced rotation of the electronic nematic director in superconducting Bi$_2$Se$_3$, 
\textit{Phys. Rev. B} {\bf 105}, 174504 (2022).

\bibitem{Ichioka}
M. Ichioka, N. Enomoto, K. Machida, Vortex lattice structure in a $d_{x^2-y^2}$-wave superconductor.
J. Phys. Soc. Jpn.  \textbf{66}, 3928 (1997).

\bibitem{Fu-Hc2}
J. W. F. Venderbos, V. Kozii, and L. Fu,
Identication of nematic superconductivity from the upper critical field.
Phys. Rev. B \textbf{94}, 094522 (2016).

\bibitem{Mineev}
P. L. Krotkov and V. P. Mineev, 
Upper critical field in a trigonal unconventional superconductor: UPt$_3$.
Phys. Rev. B \textbf{65}, 224506 (2002).

\end{thebibliography}
\end{document}
