%%%%%%%%%%%%%%%%%%%%%%% file template.tex %%%%%%%%%%%%%%%%%%%%%%%%%
%
% This is a general template file for the LaTeX package SVJour3
% for Springer journals.          Springer Heidelberg 2010/09/16
%
% Copy it to a new file with a new name and use it as the basis
% for your article. Delete % signs as needed.
%
% This template includes a few options for different layouts and
% content for various journals. Please consult a previous issue of
% your journal as needed.
%
%%%%%%%%%%%%%%%%%%%%%%%%%%%%%%%%%%%%%%%%%%%%%%%%%%%%%%%%%%%%%%%%%%%
%
% First comes an example EPS file -- just ignore it and
% proceed on the \documentclass line
% your LaTeX will extract the file if required
\begin{filecontents*}{example.eps}
%!PS-Adobe-3.0 EPSF-3.0
%%BoundingBox: 19 19 221 221
%%CreationDate: Mon Sep 29 1997
%%Creator: programmed by hand (JK)
%%EndComments
gsave
newpath
  20 20 moveto
  20 220 lineto
  220 220 lineto
  220 20 lineto
closepath
2 setlinewidth
gsave
  .4 setgray fill
grestore
stroke
grestore
\end{filecontents*}
%
\RequirePackage{fix-cm}
%
%\documentclass{svjour3}                     % onecolumn (standard format)
%\documentclass[smallcondensed]{svjour3}     % onecolumn (ditto)
%\documentclass[smallextended]{svjour3}       % onecolumn (second format)
\documentclass[twocolumn]{svjour3}          % twocolumn
%
\smartqed  % flush right qed marks, e.g. at end of proof
%
\usepackage{graphicx}
%
% \usepackage{mathptmx}      % use Times fonts if available on your TeX system
%
% insert here the call for the packages your document requires
%\usepackage{latexsym}
% etc.
%
% please place your own definitions here and don't use \def but
% \newcommand{}{}
%
% Insert the name of "your journal" with
% \journalname{myjournal}
%
% \usepackage[latin1]{inputenc}
\usepackage[british]{babel}
\usepackage[all]{xy}
\usepackage{amscd}
\usepackage{amssymb}
\usepackage{amsthm}
\usepackage{enumitem}
\usepackage{mathrsfs,bbm}
\usepackage{xcolor,graphicx}
\usepackage{graphics}
\usepackage{soul}
\usepackage{comment}
\usepackage[all]{xy}
\usepackage{amscd}
\usepackage{amssymb,amsmath,latexsym}
\usepackage{amsthm}
\usepackage{enumitem}
\usepackage{mathrsfs,bbm}
\usepackage{dsfont}
\usepackage{tikz-cd}
\usepackage[T1]{fontenc}
\usepackage[utf8]{inputenc}  
 %
%%%%%%%%%%%%%%%%%%%%%%%%%%%%%%%%%%
%pagestyle
%%%%%%%%%%%%%%%%%%%%%%%%%%%%%%%%%%
%\pagestyle{plain}
\textwidth=430pt
\headsep=.7cm
\evensidemargin=15pt
\oddsidemargin=15pt
\leftmargin=0cm
\rightmargin=0cm
%%
%%%%%%%%%%%%%%%%%%%%%%%
\newcommand*\fixitem {\item[]%
  \refstepcounter{enumi}\hskip-\leftmargin\labelenumi\hskip\labelsep}
\newtheorem*{mainthm}{Main Theorem}
\newtheorem*{mainthm1}{Theorem}
\newtheorem*{maincor}{Corollary}
\usepackage[colorlinks=true]{hyperref}
\DeclareMathOperator{\Forall}{\forall}
\DeclareMathOperator{\Exists}{\exists}
\DeclareMathOperator{\ord}{ord}
\newcommand{\phiD}{\varphi_D}
\newcommand{\phiDI}{\varphi_{\mathbf{D}_I}}
\newcommand{\phiDIj}{\varphi_{\mathbf{D}_I (j)}}
\newcommand{\phiH}{\varphi_H}
\newcommand{\phiTimes}{\phiD \otimes \phiH}
\newcommand{\phiTimesDI}{\varphi_{\mathbf{D}_I} \otimes \phiH}
\newcommand{\R}{\mathscr{A}}
\newcommand{\X}{\mathscr{X}}
\newcommand{\Xf}{\mathscr{X}_{(k_0 ,i)}[r_0]}
\newcommand{\Xfr}{\mathscr{X}_{(k_0,i)}[r]}
\newcommand{\hotimes}{\widehat{\otimes}}
\newcommand{\C}{\mathbb{C}_p}
\newcommand{\V}{\mathscr{V}}
\newcommand{\B}{\mathscr{B}}
\newcommand{\dualD}{\mathfrak{D}}
\newcommand{\Dg}{\mathbf{D}}
\newcommand{\DD}{\mathcal{D}^0}
\newcommand{\DDg}{\mathcal{D}}
\newcommand{\DV}{\mathcal{D}}
\newcommand{\W}{\mathscr{W}_N}
\newcommand{\Ao}{\mathbf{A}^\circ}
\newcommand{\AoK}{\mathbf{A}^\circ_{\K}}
\newcommand{\AK}{\mathbf{A}_{/\K}}
\newcommand{\OOO}{\mathscr{A}^\circ}
\newcommand{\K}{\mathcal{K}} 
\newcommand{\OK}{\mathcal{O}_{\K}}
\newcommand{\varprojlog}[1]{\underleftarrow{\log\!^{#1}}}
\newcommand{\T}{\mathscr{T}}
\newcommand{\TT}{\mathbf{T}}
\newcommand{\VV}{\mathbf{V}}
\newcommand{\HH}{\mathcal{H}}
\newcommand{\hh}{\mathcal{H}^+}
\newcommand{\HG}[2]{\mathcal{H}_{#1}(#2)}
\newcommand{\hhl}{\mathcal{H}^{+,[l]}}
\newcommand{\hhj}{\mathcal{H}^{+,[j]}}
\newcommand{\hhjj}{\mathcal{H}^{+,[l,l']}}
\newcommand{\GS}{G_{\mathbb{Q},S}}
\newcommand{\Rf}{R_{(k_0 ,i)}[r_0]}
\newcommand{\Rfr}{R_{(k_0 ,i)}[r]}
\newcommand{\parT}{\langle T\rangle}
\newcommand{\Zf}{Z_{(k_0 ,i)}[r_0]}
\newcommand{\Zfr}{\mathscr{Z}_{(k_0 ,i)}[r]}
\newcommand{\ZFf}{\mathscr{Z}_{(k_0 ,i)}[r_0]}
\newcommand{\ZFfr}{\mathscr{Z}_{(k_0 ,i)}[r]}
\newcommand{\ZF}{\mathscr{Z}}
\begin{document}
\begin{sloppypar}

\title{Efficient Execution of SPARQL Queries with OPTIONAL and UNION Expressions%\thanks{Grants or other notes
%about the article that should go on the front page should be
%placed here. General acknowledgments should be placed at the end of the article.}
}
% \subtitle{Do you have a subtitle?\\ If so, write it here}

%\titlerunning{Short form of title}        % if too long for running head

\author{Lei Zou \and Yue Pang \and M. Tamer Özsu \and Jiaqi Chen}

%\authorrunning{Short form of author list} % if too long for running head

\institute{Lei Zou \at
              Peking University, Beijing, China\\\email{zoulei@pku.edu.cn}
           \and
           Yue Pang \at Peking University, Beijing, China\\\email{michelle.py@pku.edu.cn}
           \and
           M. Tamer Özsu \at University of Waterloo, Waterloo, Canada\\\email{tamer.ozsu@uwaterloo.ca}
           \and
           Jiaqi Chen \at Peking University, Beijing, China\\\email{chenjiaqi93@pku.edu.cn}}

\date{Received: date / Accepted: date}
% The correct dates will be entered by the editor


\maketitle

\begin{abstract}
The proliferation of RDF datasets has resulted in studies focusing on optimizing SPARQL query processing. Most existing work focuses on basic graph patterns (BGPs) and ignores other vital operators in SPARQL, such as \texttt{UNION} and \texttt{OPTIONAL}. SPARQL queries with these operators, which we abbreviate as SPARQL-UO, pose serious query plan generation challenges. In this paper, we propose techniques for executing SPARQL-UO queries using BGP execution as a building block, based on a novel \underline{B}GP-based \underline{E}valuation (BE)-Tree representation of query plans. On top of this, we propose a series of \emph{cost-driven BE-tree transformations} to generate more efficient plans by reducing the search space and intermediate result sizes, and a \emph{candidate pruning} technique that further enhances efficiency at query time. Experiments confirm that our method outperforms the state-of-the-art by orders of magnitude.
\keywords{Graph database \and SPARQL query optimization \and OPTIONAL expressions \and UNION expressions}
% \PACS{PACS code1 \and PACS code2 \and more}
% \subclass{MSC code1 \and MSC code2 \and more}
\end{abstract}

\section{Introduction}
\label{sec:introduction}
% \begin{itemize}
%     % Diffusion of FL
%     \item {\st{Diffusion of FL}}
%     % Security threats to FL
%     \item {\st{Security threats to FL with particular focus on model poisoning}}
%     % Limitations of existing countermeasures
%     \item {\st{Current countermeasures (e.g., KRUM) and their limitations}}
%     % Proposed method and its advantages
%     \item {\st{Intuitive description of the proposed method and its difference (i.e., advantages) w.r.t. state of the art}}
%     % Main contributions
%     \item {\st{Summary of the main contributions of this work}}
%     % Paper's structure and organization
%     \item {\st{Paper's structure and organization}}
% \end{itemize}

% Diffusion of FL
Recently, {\em federated learning} (FL) has emerged as the leading paradigm for training distributed, large-scale, and privacy-preserving machine learning (ML) systems~\cite{mcmahan2017googleai,mcmahan2017aistats}. 
The core idea of FL is to allow multiple edge clients to collaboratively train a shared, global model without disclosing their local private training data.
%Specifically, an FL system consists of a central server and many edge clients; 
A typical FL round involves the following steps: {\em(i)} the server randomly picks some clients and sends them the current, global model; {\em(ii)} each selected client locally trains its model with its own private data; then, it sends the resulting local model to the server;\footnote{Whenever we refer to global/local model, we mean global/local model {\em parameters}.} {\em(iii)} the server updates the global model by computing an \emph{aggregation function}, usually the average (FedAvg), on the local models received from clients.
% \begin{enumerate}
%     \item[{\em(i)}] the server sends the current, global model to the clients and appoints some of them for training;
%     \item[{\em(ii)}] each selected client locally trains its copy of the global model with its own private data; then, it sends the resulting local model back to the server;\footnote{Whenever we refer to global/local model, we mean global/local model {\em parameters}.}
%     \item[{\em(iii)}] the server updates the global model by computing an \emph{aggregation function} on the local models received from clients (by default, the average, also referred to as FedAvg~\cite{mcmahan2017aistats}).
% \end{enumerate}
This process goes on until the global model converges. %(e.g., after a certain number of rounds or other similar stopping criteria).
%\\
% The advantages of FL over the traditional, centralized learning paradigm are undoubtedly clear in terms of flexibility/scalability (clients can join/disconnect from the FL network dynamically), network communications (only model weights\footnote{We will use \textit{parameters} and \textit{weights} interchangeably.} are exchanged between clients and server), and privacy (each client's private training data is kept local at the client's end and not uploaded to the server).
\\
% Security threats to FL
%However, the growing adoption of FL also raises security concerns~\cite{costa2022covert}, particularly about its confidentiality, integrity, and availability.
Although its advantages over standard ML, FL also raises security concerns~\cite{costa2022covert}. %, particularly about its confidentiality, integrity, and availability~\cite{costa2022covert}.
% OLD, LONG VERSION
% Indeed, some work deals with privacy leakage that may expose the local data of some clients~\cite{melis2019sp}. 
% A large body of work, instead, investigates attacks that usually aim to detriment the predictive accuracy of the learned global model. For instance, \emph{data poisoning} attacks achieve this goal by letting an adversary pollute the training set of some corrupt FL clients with maliciously crafted examples~\cite{jagielski2018sp}.
% Similarly, in \emph{model poisoning} the attacker attempts to tweak the global model weights~\cite{bhagoji2019pmlr} by directly perturbing the local model's weights of some infected FL clients before these are sent to the central server for aggregation, usually via so-called Byzantine attacks. 
% It turns out that Byzantine model poisoning attacks severely impact standard FedAvg; therefore, more robust aggregation functions must be designed to make FL systems secure.
Here, we focus on \emph{untargeted model poisoning} attacks~\cite{bhagoji2019pmlr}, where an adversary attempts to tweak the global model weights %\footnote{We will use the terms \textit{parameters} and \textit{weights} interchangeably.} 
by directly perturbing the local model's parameters of some infected clients before these are sent to the central server for aggregation.
In doing so, the adversary aims to jeopardize the global model \textit{indiscriminately} at inference time.
Such model poisoning attacks severely impact standard FedAvg; therefore, more robust aggregation functions must be designed to secure FL systems.
\\
% In this paper, we focus on designing a novel robust aggregation scheme at the server's end to contrast the effect of Byzantine model poisoning attacks.
%
% Current countermeasures and their limitations
%Several countermeasures have been proposed in the literature to combat model poisoning attacks on FL systems.
% Some methods use simple statistics more robust than plain average to smooth the impact of malicious updates (e.g., Trimmed Mean and FedMedian~\cite{yin2018icml}). 
% Other defenses implement outlier detection techniques to discard malicious updates from the aggregation performed at the server's end. Those are either based on heuristics (e.g., Krum/Multi-Krum~\cite{blanchard2017nips} and Bulyan~\cite{mhamdi2018pmlr}) or data-driven approaches (e.g., K-means clustering~\cite{shen2016acm} or DnC via spectral analysis~\cite{shejwalkar2021ndss}). 
% Finally, some strategies rely on a centralized ``source of trust'' to spot potential malicious updates (e.g., FLTrust~\cite{cao2020fltrust}).
% Several countermeasures have been proposed in the literature to combat model poisoning attacks on FL systems, i.e., to discard possible malicious local updates from the aggregation performed at the server's end. 
% These techniques range from simple statistics more robust than plain average (e.g., Trimmed Mean and FedMedian~\cite{yin2018icml}) to outlier detection heuristics (e.g., Krum/Multi-Krum~\cite{blanchard2017nips} and Bulyan~\cite{mhamdi2018pmlr}) or data-driven approaches (e.g., spectral analysis via K-means clustering~\cite{shen2016acm} or spectral analysis), or methods based on ``source of trust'' (e.g., FLTrust~\cite{cao2020fltrust}).
% OLD, LONG VERSION
%Several countermeasures have been proposed in the literature to combat Byzantine model poisoning attacks on FL systems.
% Descriptive statistics
% For example, Trimmed Mean and FedMedian aggregate local model updates using more robust statistics than standard average~\cite{yin2018icml}.
%
% % Heuristics for outlier detection
% Many existing Byzantine-resilient strategies implement some outlier detection heuristics to discard the model updates sent by potentially malicious clients from the input of the aggregation function.
% One of the most popular heuristics is Krum~\cite{blanchard2017nips}.
% This strategy tries to mitigate the impact of Byzantine attacks by selecting as a global model the local model with the smallest sum of Euclidean distances to {\em all} the other local models.
% Although powerful, Krum requires the server to know (or, at least, estimate) the number of malicious FL clients upfront, which is generally impossible in a realistic attack scenario. %
% Moreover, Krum may become ineffective for complex, high-dimensional model parameter spaces due to the curse of dimensionality.
% Bulyan~\cite{mhamdi2018pmlr} tries to overcome this issue by combining Krum with a variant of Trimmed Mean.
% % Data-driven outlier detection
% Other strategies use data-driven outlier detection techniques -- e.g., via K-means clustering~\cite{shen2016acm} -- to spot potential malicious local model updates. 
% %For instance, Shen et al. propose to cluster local model updates with K-means and thus identify outliers.
%
% % Other techniques
% As far as the server is concerned, any local model received can be from a potential malicious client. 
% FLTrust~\cite{cao2020fltrust} assumes the server acts as a client, i.e., trains a local model on an additional {\em trustworthy} dataset at the server's end and compares it against all the local models from other clients. 
% This way, the server can rely on some ``source of trust'' when discarding potentially malicious clients.
%\\
% Limitations of existing Byzantine-resilient strategies
Unfortunately, existing defense mechanisms either rely on simple heuristics (e.g., Trimmed Mean and FedMedian by~\cite{yin2018icml}) or need strong and unrealistic assumptions to work effectively (e.g., foreknowledge or estimation of the number of malicious clients in the FL system, as for Krum/Multi-Krum~\cite{blanchard2017nips} and Bulyan~\cite{mhamdi2018pmlr}, which, however, cannot exceed a fixed threshold).
Furthermore, outlier detection methods using K-means clustering~\cite{shen2016acm} or spectral analysis like DnC~\cite{shejwalkar2021ndss} do not directly consider the temporal evolution of local model updates received.
Finally, strategies like FLTrust~\cite{cao2020fltrust} require the server to collect its own dataset and act as a proper client, thereby altering the standard FL protocol.
\\
% OLD, LONG VERSION
% Overall, existing Byzantine-resilient strategies are either simple heuristics (e.g., FedMedian) or, if they are more complex, they rely on strong and unrealistic assumptions to work effectively (e.g., knowing the number of malicious clients in the FL system in advance, as for Krum and alike).
% Furthermore, data-driven outlier detection methods do not consider the temporary evolution of local model updates received (e.g., K-means clustering). 
% Finally, strategies like FLTrust requires the server to collect its own dataset and act as a proper client, thereby altering the standard FL protocol.
%
% Description of the proposed method
This work introduces a novel pre-aggregation \textit{filter} robust to untargeted model poisoning attacks. Notably, this filter $(i)$ operates without requiring prior knowledge or constraints on the number of malicious clients and $(ii)$ inherently integrates temporal dependencies. 
The FL server can employ this filter as a preprocessing step before applying \textit{any} aggregation function, be it standard like FedAvg or robust like Krum or Bulyan.
Specifically, we formulate the problem of identifying corrupted updates as a multidimensional (i.e., matrix-valued) time series anomaly detection task. 
The key idea is that legitimate local updates, resulting from well-calibrated iterative procedures like stochastic gradient descent (SGD) with an appropriate learning rate, show \textit{higher predictability} compared to malicious updates. This hypothesis stems from the fact that the sequence of gradients (thus, model parameters) observed during legitimate training exhibit regular patterns, as validated in Section~\ref{subsec:intuition}. %until convergence. 
%This regularity may be more pronounced for smooth convex loss functions, but it can still be captured within an appropriate time window, even for more complex and convoluted loss surfaces. 
%We provide evidence of this claim in Appendix~B, where we show that the average mutual information (i.e., ``predictability''), calculated over pairs of legitimate model updates sent at different FL rounds, is significantly higher than the corresponding computation for a malicious client.
\\
Inspired by the matrix autoregressive (MAR) framework for multidimensional time series forecasting~\cite{chen2021je}, we propose the FLANDERS ({\em \textbf{F}ederated \textbf{L}earning meets \textbf{AN}omaly \textbf{DE}tection for a \textbf{R}obust and \textbf{S}ecure}) filter.
The main advantages of FLANDERS over existing strategies like FLDetector~\cite{zhao2020multivariate} are its resilience to large-scale attacks, where $50\%$ or more FL participants are hostile, and the capability of working under realistic non-iid scenarios.
We attribute such a capability to two key factors: $(i)$ FLANDERS works without knowing a priori the ratio of corrupted clients, and $(ii)$ it embodies temporal dependencies between intra- and inter-client updates, quickly recognizing local model drifts caused by evil players. Below, we summarize our main contributions:

\begin{itemize}
\item[{\em(i)}]
We provide empirical evidence that the sequence of models sent by legitimate clients is more predictable than those of malicious participants performing untargeted model poisoning attacks.
\\
\item[{\em(ii)}] 
We introduce FLANDERS, the first pre-aggregation filter for FL robust to untargeted model poisoning based on multidimensional time series anomaly detection.
\\
\item[{\em(iii)}] 
We integrate FLANDERS into Flower,\footnote{\scriptsize{\url{https://flower.dev/}}} a popular FL simulation framework for reproducibility.
\\
\item[{\em(iv)}] 
We show that FLANDERS improves the robustness of the existing aggregation methods under multiple settings: different datasets, client's data distribution (non-iid), models, and attack scenarios.
\\
\item[{\em(v)}] 
We publicly release all the implementation code of FLANDERS along with our experiments.\footnote{\scriptsize{\url{https://anonymous.4open.science/r/flanders_exp-7EEB}}}
\end{itemize}

% Paper's structure and organization
The remainder of the paper is structured as follows. %some related work and the current state-of-the-art solutions to security issues that FL entails. 
Section~\ref{sec:background} covers background and preliminaries. 
In Section~\ref{sec:related}, we discuss related work.
Section~\ref{sec:problem} and Section~\ref{sec:method} describe the problem formulation and the method proposed. % to tackle it. 
Section~\ref{sec:experiments} gathers experimental results. %, and Section~\ref{sec:limitations} discusses some limitations of this work.
Finally, we conclude in Section~\ref{sec:conclusion}.
 %discusses the limitations of this work and draws future research directions.
%reports conclusions and draws perspectives for future research directions.

%%%%%%% OLD %%%%%%%
%to overcome the resilience of Byzantine failures in distributed Stochastic Gradient Descent computations. 
% The strength of Krum is its time complexity, which is linear in the gradient dimension. 
% However, the robustness of the approach is guaranteed for gradient-based learning applications only when the majority of the clients are not compromised. 
% Besides, the aggregation mechanism of Krum, as well as that of similar methods, is robust from a coarse-grained perspective and does not provide solutions to errors and perturbations that may occur at inference time.
%A related approach to~\cite{blanchard2017nips} is the work of Su et al.~\cite{su2016dc}. Here, the authors propose an iterated approximate agreement to tackle a multi-layer scenario attacked by Byzantine agents. 
%However, the method works efficiently on the sole discrete context and it is inapplicable to continuous state environments.
%\gabri{Maybe, we should just talk about the main limitations of existing countermeasures without digging into their details (or, we can just mention Krum as this is the most popular one). I will move the description of all these methods to the Related Work section.}
%!TEX root = ../main.tex

\section{Inductive Conformal Prediction}
\label{sec:pre:icp}
Given a set $\{ z_i = ( x_i, y_i ) \}_{i=1}^l$ with observation $x_i \in \calX$ and label $y_i \in \calY$ such that each $z_i \in \calZ := \calX \times \calY $ is drawn i.i.d. from an \emph{unknown} distribution on $\calZ$, inductive conformal prediction (ICP) provides 
% a simple yet powerful framework to learn 
a \emph{set prediction} $\Feps(x) \subseteq \calY$, parameterized by an error rate $0 < \epsilon <1$, such that given a new sample $z_{l+1} = (x_{l+1},y_{l+1})$ satisfying an \emph{exchangeability} condition (elaborated in Theorem~\ref{thm:icp-validity}), we have
\bea\label{eq:icpmiscoverage}
\probof{ y_{l+1} \in \Feps(x_{l+1}) } \geq 1-\epsilon, 
\eea
\ie, the prediction set $\Feps$ guarantees to contain the true label $y_{l+1}$ with probability at least $1-\epsilon$. 

% In order to achieve the probabilistic coverage in~\eqref{eq:icpmiscoverage}, ICP performs the following three steps.

{\bf Training}. We start by dividing the dataset into a \emph{proper training set} $\{ z_1,\dots,z_m \}$ and a \emph{calibration set} $\{ z_{m+1},\dots,z_{l} \}$. We shorthand $n = l - m$ as the size of the calibration set.
We learn a prediction function $f: \calX \rightarrow \tcalY$ from the proper training set using \emph{any} architecture, which allows us to fully exploit the power of modern deep learning. The prediction space $\tcalY$ can be the same as the label space $\calY$, or can contain auxiliary information such as a heuristic notion of uncertainty (\eg, softmax scores in classification or a heatmap in the case of keypoint detection). 

{\bf Conformal calibration}. 
% Leveraging the learned $f$, 
We define a \emph{nonconformity} function $S: \calZ^{m} \times \calZ \rightarrow \Real{}$ to measure how well a given sample $z = (x,y)$ \emph{conforms} to the proper training set. A popular instance of $S$ leverages the learned prediction $f$:
\bea \label{eq:nonconformity}
S\parentheses{\cbrace{z_1,\dots,z_m},(x,y)} \stackrel{\eg}{=} r(y,f(x)),
\eea
where $r: \calY \times \tcalY \rightarrow \Real{}$ is a measure of disagreement between the label $y$ and the prediction $f(x)$. For example, consider $\calY = \tcalY = \Real{}$, one can design $r(y,f(x)) = \abs{y - f(x)}$: if $(x,y)$ poorly conforms to the training set, $f$ will incur large errors.   
While the function $S$ can be arbitrary (\eg, a learnable neural network~\cite{stutz22iclr-learnconformal}), \eqref{eq:nonconformity} is a convenient definition since $f$ is implicitly dependent on $\{z_i\}_{i=1}^m$ and $r$ can incorporate domain-specific knowledge.
We then compute the nonconformity scores on the calibration set as $\alpha_i = r(y_i,f(x_i)), i = m+1,\dots,l$,
and sort them in \emph{nonincreasing} order $\alpha_{\pi(1)}\geq\dots \geq \alpha_{\pi(n)}$, where $\pi(i) \in \{m+1,\dots,l\}$ is an index permutation.
 % (offset by $m$).

{\bf Conformal prediction}. Given a new observation $x_{l+1}$ (with an unknown $y_{l+1}$) and a user-specified $\epsilon \in (0,1)$, we compute the inductive conformal prediction (ICP) set as
\bea\label{eq:icpcompute}
\Feps \parentheses{x_{l+1}} = \cbrace{y \in \calY \mid \alpha^y \leq \alpha_{\pi(\floor{(n+1)\epsilon})}},
\eea
where $\alpha^y = r(y,f(x_{l+1}))$
is the nonconformity score of the new sample when fixing the true label to be $y$. In other words, the ICP set~\eqref{eq:icpcompute} outputs the set of all labels that make the nonconformity score of the new sample no greater than $\alpha_{\pi(\floor{(n+1)\epsilon})}$ -- the $\floor{(n+1)\epsilon}$-th largest nonconformity score in the calibration set. 
% By doing so, ICP ensures that there are at least $\floor{(n+1)\epsilon}$ samples in the calibration set that are less conforming than the new sample. 
We have the following result stating the probabilistic coverage of the ICP set~\eqref{eq:icpcompute}.
% provides a valid statistical coverage of the true label $y_{l+1}$.

\begin{theorem}[Validity of ICP Coverage {\cite{vovk05book-conformal,lei18jasa-conformal,vovk12acml-icpconditional}}] \label{thm:icp-validity}
If $z_{m+1},\dots,z_l$, $z_{l+1} = (x_{l+1},y_{l+1})$ are exchangeable, \ie, their distribution is invariant under permutation, then
\bea\label{eq:icpvalidity}
1 - \epsilon \leq \probof{y_{l+1} \in \Feps(x_{l+1})} \leq 1 - \epsilon + 1/(n+1)
\eea
for any $\epsilon \in (0,1)$. Furthermore, when conditioned on the calibration set, calling $h = \floor{(n+1)\epsilon}$, we have
\begin{equation}\label{eq:beta}
\hspace{-4mm}\probof{y_{l+1}\!\in\!\Feps(x_{l+1})\!\mid\!\{z_{m+1},\dots,z_l\}}\!\sim\!\mathrm{Beta}(n+1\!-\!h,h).
\end{equation}
\end{theorem}
A few remarks are in order about Theorem~\ref{thm:icp-validity}.
First, asking $z_{m+1},\dots,z_l,z_{l+1}$ to be exchangeable is weaker than asking them to be independent. However, this assumption typically fails when the calibration set is a single video sequence, where the image frames $\{z_{m+1},\dots,z_l\}$ are temporally correlated~\cite{luo21arxiv-conformalsafety}. Fortunately, as we detail in Section~\ref{sec:experiments}, the way the LineMOD Occlusion dataset~\cite{brachmann14eccv-linemodocc} was collected makes the exchangeability condition easily satisfied, which also suggests best practices to make the exchangeability condition hold in computer vision. 
Second, the lower bound in~\eqref{eq:icpvalidity} can be intuitively proved because under exchangeability, $\alpha_{l+1} := r(y_{l+1},f(x_{l+1}))$ --the nonconformity score of the new sample with the true label-- is \emph{exchangeable} with the nonconformity scores of the calibration samples, and hence \emph{equally likely} to fall in anywhere between the scores $\{ \alpha_{\pi(i)}\}_{i=1}^n$. Consequently, $\probof{y_{l+1} \in \Feps(x_{l+1})} = \probof{\alpha_{l+1} \leq \alpha_{\pi(\floor{(n+1)\epsilon})}} = 1 - \floor{(n+1)\epsilon}/(n+1) \geq 1 - \epsilon$. The upper bound in \eqref{eq:icpvalidity} states that $1-\epsilon$ is not overly conservative (indeed tight if $n$ is large). 
Lastly, the probabilistic guarantee in \eqref{eq:icpvalidity} is \emph{marginal} over the randomness of the calibration set, meaning if one chooses an infinite number of calibration sets,  the \emph{average} empirical coverage will converge to $1-\epsilon$. This, however, implies that the empirical coverage given one calibration set is a random variable that fluctuates as the Beta distribution~\eqref{eq:beta}. Fig.~\ref{fig:beta-distribution} plots the Beta distribution at $\epsilon=0.1$ with different sizes of the calibration set. We observe that as $n$ increases the empirical coverage becomes more concentrated at $1-\epsilon$. Our experiments show that even with a small ($n=200$) calibration set, the empirical coverage is close to, and mostly higher than, $1-\epsilon$.

% \begin{proposition}[Conditional Validity of ICP {\cite{vovk12acml-icpconditional}}] \label{prop:icp-conditional-validity}
% \red{To be filled out}
% \end{proposition}


% Proposition~\ref{prop:icp-validity} states that, if the new observation $z_{l+1}$ is exchangeable with the calibration set (which is a weaker condition than requiring $z_{l+1}$ is jointly i.i.d. with the calibration set), then no matter which prediction function $f$ has been learned from the proper training set, and which function $A$ has been chosen to compute the nonconformity score, we have at least $1-\epsilon$ confidence that the ICP $\Feps$ defined in \eqref{eq:icp} contains the true label. Of course, the caveat here is that the quality of the learned prediction function $f$ and the nonconformity function $A$ will decide the conservativeness of the ICP $\Feps$. For example, if $f$ has poor predictive power, then the set $\Feps$ may be arbitrarily large so that it tells little information about the true label $y$. \red{Fortunately, as we will show in experiments, with modern deep learning architectures for learning $f$, we can obtain ICPs that are both confident and tight.}

%!TEX root = ../main.tex
% \begin{figure}
% \hspace{-4mm}\includegraphics[width=1.1\columnwidth]{icp-overview-half.pdf}
% \caption{Given a learned prediction function and a calibration set of $n$ samples, conformal calibration uses a nonconformity function~\eqref{eq:nonconformity} to compute and sort nonconformity scores $\{ \alpha_{\pi(i)}\}_{i=1}^n$. Given a new observation and an error rate $\epsilon$, conformal prediction~\eqref{eq:icpcompute} outputs a prediction set of all labels under which the nonconformity score of the new sample is no larger than $\alpha_{\pi(\floor{(n+1)\epsilon})}$.
% \label{fig:icp-overview}}
% \end{figure}
\begin{figure}
\vspace{-4mm}
\begin{center}
\includegraphics[width=0.6\columnwidth]{beta.pdf}
\end{center}
\vspace{-6mm}
\caption{Beta distribution of the conditional coverage in~\eqref{eq:beta} with $\epsilon=0.1$ and different $n$. Notice how the conditional probability becomes more concentrated around $1-\epsilon$ when $n$ increases.
\label{fig:beta-distribution}}
\vspace{-7mm}
\end{figure}
The {\tt Baseline} module is by design the core of the {\tt pygwb} stochastic analysis. %
Its main role is to manage the cross-correlation between {\tt Interferometer} data products, combine these into a single cross-spectrum, which represents the point estimate of the analysis, and calculate the associated error, as introduced in Sec.~\ref{sec: GWB analysis}.

The standard initialization of a {\tt Baseline} object simply requires a pair of {\tt Interferometer} objects. %
\begin{Verbatim}[commandchars=\\\{\},frame=leftline,framesep=1.5ex,framerule=0.8pt,fontsize=\small]
\PY{k+kn}{from} \PY{n+nn}{pygwb} \PY{k+kn}{import} \PY{n}{baseline}
\PY{n}{H2H2\PYZus{}baseline} \PY{o}{=} \PY{n}{baseline}\PY{o}{.}\PY{n}{Baseline}\PY{p}{(}\PY{l+s+s2}{\PYZdq{}}\PY{l+s+s2}{H1\PYZhy{}H2}\PY{l+s+s2}{\PYZdq{}}\PY{p}{,} \PY{n}{H1}\PY{p}{,} \PY{n}{H2}\PY{p}{)}
\end{Verbatim}

Here {\tt H1} and {\tt H2} are {\tt Interferometer} objects. % 
It is also possible to load a previously stored {\tt Baseline} object in {\tt pickle} format by calling the relevant class method. %

The data loaded into the {\tt Interferometer} objects are automatically imported into the {\tt Baseline} object upon initialization. % 
The {\tt Baseline} object relies on the {\tt spectral} module to calculate cross-correlations between the data streams, following the methodology shown in Sec.~\ref{sec:spectral}. %
Similarly, it relies on the {\tt postprocessing} module to obtain the point estimate $\hat{\Omega}^\alpha_{\rm ref}$ and its variance $\sigma^\alpha_{\rm ref}$, as described in Eqs.~(\ref{eq:ptest_postpoc}--\ref{eq:sigma_postpoc}). %
The user may choose to calculate point estimate and sigma spectra or point values; in the latter case, the spectra are automatically stored to facilitate subsequent analyses. %

Calculating $\hat{\Omega}^\alpha_{\rm ref}$, as well as performing parameter estimation on the GWB spectrum, requires the two-detector \gls{orf}, $\gamma_{IJ}$, shown in Eq.~\eqref{eq:orf}. %
The \gls{orf} is calculated at {\tt Baseline} object initialization, then stored as an attribute. %
By default, we assume \gls{gr}, which presents two independent degrees of freedom for the strain field, typically $A=\{+, \times\}$ in the transverse-traceless gauge. For a precise derivation of this function and detector response definitions, see for example~\cite{Romano_2017}. %

The {\tt Baseline} object is also equipped to probe circularly polarized backgrounds~\cite{Seto:2007tn}, and non-\gls{gr} polarizations in the \gls{gwb}, such as scalar and vector backgrounds~\cite{TeVeS}. %
This requires selecting a different choice of $A$, according to the chosen polarization type, which can be declared when calculating $\hat{\Omega}^\alpha_{\rm ref}$ or the \gls{orf} directly. %
Details on the expressions for non-\gls{gr} $\gamma_{IJ}$ functions may be found in the appendix of~\cite{TeVeS}.



\section{Method}
\label{sec:method}

% \ml{``Inconsistent'' to ``large variation''}

% In this section, we propose our methods based on the observations in Section \ref{sec:motivation}.
In this section, we propose two techniques to further enhance the strong baseline to capture the variation of activation distributions better.
We first introduce spatial re-scaling to adapt the network to pixel-to-pixel variation.
We then propose channel-wise shifting and re-scaling to better capture the channel-to-channel variation.
Meanwhile, as both of the two methods are image-dependent, the image-to-image variation can be captured naturally.
By combining the two methods with our strong baseline, we build our enhanced BNN for SR, named EBSR.

% Because the activation distributions among pixels, channels and images have large variations \red{**are highly inconsistent} in SR networks, we introduce spatial re-scaling to adapt to pixel-wise variations and channel shift and re-scaling to adapt to channel-wise variations. And both of them are image-dependent to adapt to image-wise variations, which means during inference our network re-scales and shifts the distributions of activations flexibly for different input images. Based on these methods, we build an enhanced binary neural network for image super-resolution (EBSR).

% According to [3], the difference of activation magnitudes indicates different scaling factors are needed for each pixel.

\subsection{Spatial Re-scaling}
% It is better to use different scaling factors for different pixels to reduce the quantization error and retain more detailed information for image super-resolution. 

% \ml{In the main method, we do not need to introduce the previous works but can focus on introducing our own method. Channel rescaling in Real-to-binary Net is not relevant in this context.}

% Re-scaling the output of binary convolutions was proposed at the birth of BNN in XNOR-Net \cite{rastegari2016xnor} to reduce quantization error and improve accuracy for image classification tasks.
% It is computed as below:
% \begin{equation}
% \mathcal{A} * \mathcal{W} \approx(\operatorname{sign}(\mathcal{A}) \circledast \operatorname{sign}(\mathcal{W})) \odot \mathcal{K} \alpha
% \label{eq:xnor-net rescale}
% \end{equation}
% where $\circledast$ denotes the binary convolution and $\odot$ denotes the element-wise multiplication.
% $\mathcal{A}$, $\mathcal{W}$, $\alpha$, and $\mathcal{K}$ denote the activation, weight, weight scaling factor, and activation scaling factor, respectively.
%  Later in XNOR-Net++ \cite{bulat2019xnor}, Bulat et al. fuse the activation and weight scaling factors into a single one that is learned end-to-end based on gradients and this improves the classification accuracy on ImageNet dataset.

% % It is computed as Eq.~\ref{eq:xnor-net rescale}, where $\circledast$ denotes 
% %  the binary convolution and $\odot$ denotes the element-wise multiplication. The binary convolution of $\mathcal{A}$ and $\mathcal{W}$ is rescaled by the weight scaling factor $\alpha$ and the activation scaling factor $\mathcal{K}$, both of which are calculated analytically.


% \zc{Similarly, you should explain the meaning of A, W and the operators $\circledast$ in the formula}
% Then in Real-to-binary Net \cite{martinez2020training}, Martinez et al. used a data-driven channel re-scaling module that takes the pre-convolution activations as input to predict the activation scaling factor. Unlike that in XNOR-Net++ \cite{bulat2019xnor}, these scaling factors are not fixed during inference but rather inferred from data. By doing this, they further improved the classification accuracy on ImageNet over XNOR-Net++. 
As is shown in Figure \ref{fig:pixel}, activation distributions have large pixel-to-pixel variation in SR networks
and the difference of activation magnitudes indicates different scaling factors are preferred for different pixels.
Inspired by \cite{martinez2020training}, we propose spatial re-scaling to better adapt the network to the spatial variation
of activation distributions in SR networks.
% fit the various pixel-wise distributions in SR networks.
We take the real-valued activations $A$ before convolution as input and predict pixel-wise scaling factors $S(A)$, which re-scale the binary convolution output. Spatial re-scaling process can be formulated as follows:
\begin{equation}
A * W \approx(\operatorname{sign}(A) \circledast \operatorname{sign}(W)) \odot \alpha \odot S(A)
\label{eq:spatial rescale}
\end{equation}
where $\circledast$ denotes 
the binary convolution and $\odot$ denotes the element-wise multiplication. $A$, $W$, $\alpha$, and $S\left(A\right)$ denote real-valued activations, weights, the scaling factor of weights, and the spatial-wise scaling factor of activations respectively. $S\left(A\right) \in \mathbb{R}^{1\times H\times W}$ can be calculated with a convolution and a sigmoid function.
% as $\sigma\left( CONV\left(A\right)\right)$. 
As shown in Figure \ref{fig:method}(a), real-valued activations first go through a convolution layer,
which has an input channel of $C$ and an output channel of 1, 
and then pass through a sigmoid function to produce the scaling factors $S(A)$ along the spatial dimension.
During inference, the scaling factor will change dynamically according to different input feature maps.
By re-scaling binary convolution output using $S(A)$, we can reduce the quantization error and the original pixel-wise information in FP activation
will be preserved much better.
Spatial re-scaling leads to a large PSNR improvement of 0.24 dB (from 30.30 dB to 31.54 dB) on Set5 and 0.22 dB (from 25.09 dB to 25.31 dB)
on Urban100 compared with our strong baseline. 

\subsection{Channel-wise Shifting and Re-scaling}

\begin{table}[!tb]
\centering
\caption{Comparison between whether to fuse channel-wise shifting and re-scaling or not based on our baseline with spatial re-scaling. }
\label{tab:fusing}

\scalebox{0.65}{
\begin{tabular}{c|cc|cc|cc}
\hline
\multirow{2}{*}{Method}     & \multirow{2}{*}{OPs} & \multirow{2}{*}{Params} & \multicolumn{2}{c|}{Set5} & \multicolumn{2}{c}{Urban100} \\ \cline{4-7} 
                            &                      &                         & PSNR        & SSIM        & PSNR          & SSIM         \\ \hline
Baseline + spatial re-scale & 2.16G                & 0.05M                   & 31.54       & 0.883       & 25.31         & 0.759        \\
+ channel-wise shift and re-scale             & 2.34G                & 0.09M                   & 31.61       & 0.885       & 25.35         & 0.761        \\
+ w/ fusing                   & 2.27G                & 0.08M                   & \textbf{31.64}       & \textbf{0.885}       & \textbf{25.36}         & \textbf{0.761}        \\ \hline
\end{tabular}
}
\end{table}

In SR networks, activation distributions exhibit larger channel-to-channel variation (Figure \ref{fig:chl}).
Both the mean and magnitude of the activation distributions vary significantly across channels.
% Thus we use channel-wise shifting and re-scaling to adapt to various channel-wise distributions. 
\cite{martinez2020training} has proposed the data-driven channel re-scaling, 
but our method differs from them in further introducing data-driven thresholds to handle the channel-wise variation of both mean and magnitude.
Since the blocks to generate the scaling factors and thresholds are very similar, we further propose to fuse them into one module.
% and fusing channel-wise shifting and re-scaling into one module.
We evaluate the effect of fusing the two blocks in Table \ref{tab:fusing}.
With channel-wise shifting and re-scaling fused, our models have fewer operations and parameters overhead and slightly higher performance.

For the specific process, we take the real-valued activations as input and predict different thresholds and scaling factors for each channel. They are also image dependent, e.g., $\beta_{i}$ in Eq.\ref{eq:act_binarize} is no longer fixed during inference but generated according to different input feature maps. Channel-wise shifting and re-scaling can be formulated as follows:
\begin{equation}
A * W \approx(\operatorname{sign}(A-C_s(A)) \circledast \operatorname{sign}(W)) \odot \alpha \odot C_r(A)
\label{eq:channel-wise_shift_and_rescale}
\end{equation}
where $\circledast$ denotes 
the binary convolution and $\odot$ denotes the element-wise multiplication. $C_s(A), C_r(A) \in \mathbb{R}^{C\times1\times1}$ denote the channel-wise threshold and scaling factor, respectively. 
We show the block diagram in Figure \ref{fig:method}(b).
The real-valued input feature map is first squeezed to a ${C\times1\times1}$ vector by a global average pooling (GAP) layer.
The subsequent fully connected layers and ReLU learn the channel-wise information and output a ${2C\times1\times1}$ vector.
Then the ${2C\times1\times1}$ vector is split into two ${C\times1\times1}$ vectors.
We use the first $C$ channels as the channel-wise bias and pass the last $C$ channels through a sigmoid layer 
as the channel-wise scaling factor, which are used to shift the real-valued activations and re-scale the binary convolution output, respectively. 


% \ml{We can mention previously, channel-wise re-scale has been proposed. We propose to fuse them. Add the comparison between fuse v.s. no fuse.}

\begin{figure}[!tbp]%
  \centering
    \includegraphics[width=0.4\textwidth]{fig/methods.png}
  
% \subfloat[channel-wise shifting\&re-scale]{
%     \label{subfig:channel-wise shifting and re-scale}
%     \includegraphics[width=0.2\textwidth]{fig/chl shift and rescale.png}
%   }

  \caption{Block diagram for spatial re-scaling, and channel-wise shifting and re-scaling.} 
  % Input A is the real-valued activation tensor and C, H, and W denote its dimension. GAP stands for global average pooling. The reduction ratio r is set to 16 for a better trade-off between the performance and the number of operations and parameters.}
  \label{fig:method}
\end{figure}


\subsection{Network Structure}

Combining the spatial re-scaling and the channel-wise shifting and re-scaling methods, we construct the enhanced convolution layer (E-Conv).
Then we build our EBSR model based on E-Conv.
In Figure \ref{fig:E-conv}, we compare the binary convolution layer used in the baseline network and our proposed E-Conv.
We use spatial and channel-wise scaling factors to re-scale the binary convolution output,
and use channel-wise shifting to learn appropriate thresholds for each channel before binarization.
The scaling factors and threshold used in E-Conv are learnable and depend on the real-valued input activations.
In this way, our proposed EBSR can adapt to pixel-to-pixel, channel-to-channel, and image-to-image variations
to reduce the large binarization error and preserve more details.
% In this way, our proposed E-Conv reduces the large quantization error caused by binarization and keeps the original information of input feature maps to a large extent.


\begin{figure}[!tb]%
  \centering

    \includegraphics[width=0.5\textwidth]{fig/E-conv.png}

  \caption{Comparison of (a) the binary convolution layer with a skip connection used in our baseline network and (b) the proposed E-Conv.}
  \label{fig:E-conv}
\end{figure}


Figure \ref{fig:network} shows the basic block based on the E-Conv and our EBSR composed of the basic blocks. Following existing works, the convolution layers in the head and tail modules are not binarized. We choose the lightweight EDSR which has 16 basic blocks and 64 channels, and EDSR which has 32 basic blocks and 256 channels as our backbones, which correspond to EBSR-light and EBSR, respectively.

\begin{figure}[!tb]%
  \centering
  {
    \includegraphics[width=0.35\textwidth]{fig/network.png}
  }
  
  \caption{The structure of our proposed EBSR.  Convolution layers in purple are real-valued vanilla 3x3 convolutions.}
  \label{fig:network}
\end{figure}
\section{Experimental Results}
\label{sec:experiments}
\subsection{Training Details}
\cite{Kalantari2017DeepHD} provides the first dataset specifically designed for multi-exposure HDR fusion under large motion. It consists of 74 training sets, which we use to supervise the training of our model. We crop the input images to patches of size \(256 \times 256\) at a step size of 64. This totally generates 20128 training samples. To augment training samples, we randomly rotate and flip the training images. The training adopts Adam optimizer. The learning rate is initialized to \(10^{-4}\) and is reduced to \(10^{-5}\) after 20 epochs. It is observed that 40 epochs are sufficient for the training to converge.    

\subsection{Numerical Evaluation}
We numerically measure the performance of our method on the 15 test sets of \cite{Kalantari2017DeepHD}, by Peak Signal-to-Noise Ratio (PSNR) and Structure Similarity, computed in both tonemapping domain (-\(\mu\)) and HDR linear domain (-L). Visual difference metric HDR-VDP-2 is also adopted, where the parameters are set as same as in previous works \cite{wu2018end} and \cite{niu2021hdrgan}. 

Table \ref{table_metrics} compares our model with state-of-the-art models. For \cite{yan2020nonlocal} and \cite{xiong2021hierarchical}, we use the results reported in the publications. Note that \cite{sen2012robust} and \cite{hu2013hdr} are not machine learning based methods. Moreover,  \cite{Kalantari2017DeepHD} and \cite{wu2018end} apply optical flow and homography transformation to preprocess the input images respectively, and hence entail extra computation. 

Table \ref{table_metrics} shows that our method outperforms competing method in terms of PSNR-L, SSIM-$\mu$, SSIM-L and HDR-VDP-2. It ranks the second best in PSNR-$\mu$, being slightly (0.1dB) inferior to \cite{xiong2021hierarchical}. Note that \cite{xiong2021hierarchical} utilizes a pretrained model to detect ghosting regions for training, whereas our method does not require any pretrained model. The high PSNR and SSIM scores varify that our model has strong HDR reconstruction ability and can accurately restore the radiance and structure of the scene in both tonemapping domain and HDR linear domain. Furthermore, its high performance in term of HDR-VDP-2\cite{mantiuk2011hdr} performance indicates that our method can generate HDR image visually close to the target image.

\begin{table*}[ht]
\centering
\begin{tabular}{l|c|c|c|c|c}
\hline
& PSNR-$\mu$ & PSNR-L & SSIM-$\mu$ & SSIM-L & HDR-VDP-2 \\
\hline
\bfseries Sen & 40.97 & 38.36 & 0.9830 & 0.9746 & 60.60\\
\hline
\bfseries Hu  & 35.65 & 30.80 & 0.9725 & 0.9491 & 58.34\\
\hline
\bfseries Kalantari & 42.69 & 41.22 & 0.9888 & 0.9845 & 65.05\\
\hline
\bfseries DeepHDR& 41.99 & 41.22 & 0.9878 & 0.9859 & \underline{65.91}\\
\hline
\bfseries AHDR & 43.62 & 41.03 & 0.9900  &\underline{0.9883} & 63.85 \\
\hline 
\bfseries NHDRRNet& 42.414 & - & 0.9887 & - & 61.21 \\
\hline 
\bfseries HDR-GAN &43.92 & \underline{41.57} &\underline{0.9905} &0.9865 & 65.45\\
\hline 
\bfseries HFNet & \textbf{44.28} & 41.47 & - & - & - \\
\hline 
\bfseries Ours & \underline{44.18} & \textbf{42.19}&\textbf{0.9912} & \textbf{0.9883}& \textbf{67.07} \\
\hline
\end{tabular}
\caption{Numerical performance of the proposed model, evaluated on the dataset by Kalantari-Ramamoorthi. The best and second best results for each metric are marked in \textbf{bold} and \underline{underlined}, respectively}
\label{table_metrics}
\end{table*}

\subsection{Visual Performance Evaluation}

\begin{figure*}[!htb]
\centering
\includegraphics[width=\textwidth]{experiments/kalantari_test.png}
\caption{Visual comparison on the test set of Kalantari-Ramamoorthi dataset. Zoom-in views of reconstruction by each method are presented on the saturated regions that contain moving objects. Our network built with gated Swin Transformer yields noticeably better visual results than other methods.}
\label{fig_kalantari_test}
\end{figure*}
Fig. \ref{fig_kalantari_test} present the visual performance of our method and comparable methods on two examples from \cite{Kalantari2017DeepHD}. We present the zoom-in views of two challenging cases, where large saturated regions contain substantial non-rigid motion in the reference image. The two patch-based methods do not reconstruct the missing details in the saturated regions, as they heavily rely on the details provided by the reference image for registration. Image reconstructed by the optical flow based method \cite{Kalantari2017DeepHD} suffers motion blur artifacts. This is because the convolutions of DeepHDR and HDR-GAN have limited receptive fields, and hence are hampered to repair missing content in misaligned regions by aligned regions. The gating mechanism of AHDR is only applied to low-level features, so the high-level outliers may deteriorate the HDR fusion. In contrast to comparable methods, our model remarkably overcomes the ghosting artifacts.

\begin{figure}[ht]
\centering
\includegraphics[width=\columnwidth]{experiments/sen_test.pdf}
\caption{Visual performance comparison on example images from the dataset by Sen et al. Zoom in views on challenging areas are presented. Although the ground truth is unavailable, it can be clearly observed that our method visually performs better than comparable methods.}
\label{sen_test}
\end{figure}

\begin{figure}[ht]
\centering
\includegraphics[width=\columnwidth]{experiments/tursun_test.pdf}
\caption{Visual performance comparison on example images from the dataset by Tursun et al. Compared to state of the art methods, our method suffers less ghosting artifact.}
\label{tursun_test}
\end{figure}

Fig.\ref{sen_test} and Fig.\ref{tursun_test} present visual performance of our method on two examples from benchmark datasets \cite{sen2012robust} and \cite{tursun2016objective}. As these test datasets   do not provide ground truth image. we mark the visual difference on the results generated by different methods. It can be seen that our method suffers less artifacts than other methods in various scenes with various motion patterns, achieving better visual results. Our method creates high-quality HDR more robustly and generalizes well. 

\subsection{Ablation Study}

\begin{table}[h]
\centering
\resizebox{\columnwidth}{!}{
\begin{tabular}{l|c|c|c|c|c}
\hline
                         & PSNR-$\mu$ & PSNR-l & SSIM-$\mu$ & SSIM-l & HDR-VDP-2 \\ \hline
restormer(w/o ssim loss) & 44.00  & 41.5   & 0.9906 & 0.9873 & 64.72  \\ \hline
Ours(w/o ssim loss)      & 44.07  & 41.83  & 0.9909 & 0.9879 &  64.78  \\ \hline
Ours                     & 44.18  & 42.19  & 0.9912 & 0.9883 & 67.07      \\ \hline
\end{tabular}
}
\caption{Experimental results of ablation study. We compare using Gated Swin Transformer v.s. Gated Transformer, and the combined loss function v.s. the traditional $l_{1}$ norm loss function.}
\label{table_ablation_block_loss}
\end{table}

We verify various components of our method, including Swin Transformer, loss function, and gating mechanism by ablation study.

\subsubsection{Ablation Study on Block Design}
Our model has similar architecture to Restormer, which uses modified Transformer, whereas we use modified Swin Transformer as the building unit. For comparison, we replace the residual modules in each block in our model with multiple transformer layers as in Restormer, with same number of transformer layers. Table \ref{table_ablation_block_loss} presents the results, which show that using Swin Transformer achieves superior performance in all measures. The reason is that the attention module of Restormer is computed channel-wise, but forgoes the cross-exposure spatial dependency to repair the non-aligned area. 

\subsubsection{Ablation Study on Loss Function}
We trained our model under different loss function configurations, as shown in \ref{table_ablation_block_loss}. The results validate that the SSIM loss benefits detail reconstruction.

\subsubsection{Ablation Study on Gating Mechanism}
\begin{table}[h]
\resizebox{\columnwidth}{!}{
\begin{tabular}{l|c|c|c|c|c}
\hline
           & PSNR-$\mu$ & PSNR-l & SSIM-$\mu$ & SSIM-l & HDR-VDP-2 \\ \hline
w/o gating & 43.14  & 41.03  & 0.9904 & 0.9868 &     64.88      \\ \hline
one gating & 43.44  & 41.42  & 0.9909 & 0.9882 &    67.13   \\ \hline
Ours       & 43.61  & 41.74  & 0.9909 & 0.9881 & 66.96     \\ \hline
\end{tabular}
}
\caption{Ablation experimental results to verify the effectiveness of the gating mechanism}
\label{table_ablation_gating}
\end{table}

The gating mechanism is an important component in our model. Ablation study is conducted in the gating mechanism as follows.

\textbf{w/o gating}: The gating mechanism is not used in the feed forward network of all transformer layers in the model, that it, our GST unit degenerate to the vanilla Swin Transformer.

\textbf{one gating}: The gating mechanism is only used in the first Swin Transformer layers subsequent to the embedding layer, but not used for other layers. 

 Table \ref{table_ablation_gating} shows the results of the ablation experiments, where the model is trained for 20 epochs. By removing the gating mechanism, the network relies on self-attention for image alignment, resulting in the lowest performance. On top of it, adding gates to low level layers notably improves the HDR reconstruction. Furthermore, by integrating the gating mechanism with all Swin Transformer layers, the model effectively inpaints information in non-aligned regions and obtains the highest HDR reconstruction results, thus validates the effectiveness of the gating mechanism in our model.

\section{Conclusion}\label{sec:conclusion}
In this work, we focus on addressing the fundamental challenge of OOD detection tasks, which is how to fully understand the semantic discrepancy between the ID/OOD samples. We reveal that the key to success in the realistic SCOOD task is to allocate as many ID samples in the unlabeled set correctly as possible. To this end, we propose a novel uncertainty-aware optimal transport scheme that introduces class-specific energy scores as guidance for effective label assignment. Experimental results show that our method achieves better performance than previous state-of-the-art methods on SCOOD benchmarks.

\textbf{Limitations.} In addition to temperature scaling, other techniques such as feature clipping applied in ReAct~\cite{sun2021react} also enhance the performance of energy score, so how to obtain an OOD score that best fits the SCOOD task can be further explored. Moreover, a setting highly related to SCOOD has been proposed in \cite{katz2022training} and formulated as a constrained optimization problem. We will also theoretically analyze these practical OOD settings in our feature work.

% \section*{Acknowledgments}
\textbf{Acknowledgments.} 
This work is supported by National Key R\&D Program of China under Grant 2020AAA0105701, National Natural Science Foundation of China (NSFC) under Grants 61872327, Major Special Science and Technology Project of Anhui, National Natural Science Foundation of China (62033012) and Ant Group through Ant Research Intern Program.


%\begin{acknowledgements}
%If you'd like to thank anyone, place your comments here
%and remove the percent signs.
%\end{acknowledgements}

% BibTeX users please use one of
% \bibliographystyle{spbasic}      % basic style, author-year citations
\bibliographystyle{spmpsci}      % mathematics and physical sciences
%\bibliographystyle{spphys}       % APS-like style for physics
\bibliography{sample-base}   % name your BibTeX data base

\clearpage
\section{Appendix for Proofs}

\paragraph{Proof of Theorem \ref{thm:main}.}

\begin{proof}
\label{proof:main}
Our proof has two steps. In Step 1, we will show that SimCLR is equivalent to minimizing the cross entropy loss defined in Eqn.~(\ref{eqn:cross-entropy}). 
In Step 2, we will show  that minimizing the cross-entropy loss 
is equivalent to spectral clustering on $\bfpi$. 
Combining the two steps together, we have proved our theorem. 

\textbf{Step 1: } SimCLR is equivalent to minimizing the cross entropy loss.

The cross-entropy loss takes expectation over 
$\bfW_\bfX\sim \mathbb{P}(\cdot ; \bfpi)$, 
which means $\bfW_\bfX$ has exactly one non-zero entry in each row $i$. By Lemma~\ref{lem:multinomial}, we know every row $i$ of $\bfW_\bfX$ is independent of other rows. Moreover, 
$\bfW_{\bfX,i}\sim \mathcal{M}(1, \bfpi_i/\sum_j \bfpi_{i,j})=\mathcal{M}(1, \bfpi_i)$, because $\bfpi_i$ itself is a probability distribution.
Similarly, we know $\bfW_\bfZ$ also has the row-independent property by sampling over $\mathbb{P}(\cdot;\bfK_\bfZ)$.
Therefore, by Lemma~\ref{lem:cross_split}, we know Eqn.~(\ref{eqn:cross-entropy}) is equivalent to:
\[
 -\sum_{i=1}^n \mathbb{E}_{\bfW_{\bfX,i}}[\log \mathbb{P}(\bfW_{\bfZ,i}=\bfW_{\bfX,i};\bfK_\bfZ)],
\]

This expression takes expectation over $\bfW_{\bfX,i}$ for the given row $i$. Notice that 
$\bfW_{\bfX,i}$ has exactly one non-zero entry, which equals $1$ (same for $\bfW_{\bfZ,i}$). 
As a result
we expand the above expression to be:
\begin{equation}
 -\sum_{i=1}^n \sum_{j\neq i} \Pr(\bfW_{\bfX,i,j}=1)\log \Pr(\bfW_{\bfZ,i,j}=1).
\label{eqn:detailed-expansion}    
\end{equation}


By Lemma~\ref{lem:multinomial}, $\Pr(\bfW_{\bfZ,i,j}=1)=\bfK_{\bfZ,i,j}/\|\bfK_{\bfZ,i}\|_1$ for $j\neq i$. Recall that $\bfK_\bfZ=(k(\bfZ_i-\bfZ_j))_{(i,j)\in[n]^2}$, which means 
$\bfK_{\bfZ,i,j}/\|\bfK_{\bfZ,i}\|_1=\frac{\exp(-\|\bfZ_i-\bfZ_j\|^2/{2\tau})}{\sum_{k\neq i}
\exp(-\|\bfZ_i-\bfZ_k\|^2/{2\tau})
}$ for $j\neq i$, when $k$ is the Gaussian kernel with variance $\tau$. 

Notice that $\bfZ_i=f(\bfX_i)$, so we know
\begin{equation}
-\log \Pr(\bfW_{\bfZ,i,j}=1)=
-\log \frac{\exp(-\|f(\bfX_i)-f(\bfX_j)\|^2/{2\tau})}{\sum_{k\neq i}
\exp(-\|f(\bfX_i)-f(\bfX_k)\|^2/{2\tau}),
}
\label{eqn:infonce-equivalence}    
\end{equation}


The right hand side is exactly the InfoNCE loss defined in Eqn.~(\ref{eqn:infonce}).
Inserting Eqn.~(\ref{eqn:infonce-equivalence}) into Eqn.~(\ref{eqn:detailed-expansion}), we get the SimCLR algorithm, which first samples augmentation pairs $(i,j)$ with $\Pr(\bfW_{\bfX,i,j}=1)$ for each row $i$, and then optimize the InfoNCE loss. 

\textbf{Step 2: } minimizing the cross entropy loss 
is equivalent to spectral clustering on $\bfpi$.


By Lemma~\ref{lem:convert_to_spectral}, we may further convert the loss to 
\begin{equation}
\label{eqn:main-theorem-repul-attr}
\min_{\bfZ}
-\sum_{(i,j)\in [n]^2} \mathbf{P}_{i,j}
\log k (\bfZ_i-\bfZ_j)+\log \mathbf{R}(\bfZ).
\end{equation}
Since $k$ is the Gaussian kernel, this reduces to \[
\min_\bfZ \mathrm{tr}(\bfZ^\top \mathbf{L}(\bfpi) \bfZ)
+\log \mathbf{R}(\bfZ),
\]

where we use the fact that $\mathbb{E}_{\bfW_\bfX\sim \mathbb{P}(\cdot; \bfpi)}[\mathbf{L}(\bfW_\bfX)]
=\mathbf{L}(\bfpi)
$, because the Laplacian operator is linear and $
\mathbb{E}_{\bfW_\bfX\sim \mathbb{P}(\cdot; \bfpi)}(\bfW_\bfX)=\bfpi
$.
\end{proof}

\paragraph{Proof of Theorem \ref{thm:clip}.}
\begin{proof}
Since $\bfW_\bfX\sim \mathbb{P}(\cdot;\bfpi_{\mathbf{A}, \mathbf{B}})$, we know 
$\bfW_\bfX$ has exactly one non-zero entry in each row, denoting the pair that got sampled. 
A notable difference compared to the previous proof is we now have $n_\mathcal{A}+n_\mathcal{B}$ objects in our graph. CLIP deals with this by taking a mini-batch of size $2N$, 
such that $n_\mathcal{A}=n_\mathcal{B}=N$, and adding the $2N$ InfoNCE losses together. We label the objects in $\mathcal{A}$ as $[n_\mathcal{A}]$, and the objects in $\mathcal{B}$ as $\{n_\mathcal{A}+1, \cdots, n_\mathcal{A}+n_\mathcal{B}\}$. 

Notice that $\bfpi_{\mathbf{A}, \mathbf{B}}$ is a bipartite graph, so the edges of objects in $\mathcal{A}$ will only connect to object in $\mathcal{B}$ and vice versa. We can define the similarity matrix in $\cZ$ as $\bfK_\bfZ$, 
where $\bfK_\bfZ(i, j+n_\mathcal{A})=\bfK_\bfZ(j+n_\mathcal{A},i)= k(\bfZ_i-\bfZ_j)$ for $i\in [n_\mathcal{A}], j\in [n_\mathcal{B}]$, and otherwise we set $\bfK_\bfZ(i,j)=0$. 
The rest is same as the previous proof. 
\end{proof}

\paragraph{Proof of Theorem \ref{thm:exponential}.}

\begin{proof}
\label{proof:exponential}
Since the objective function consists of a linear term combined with an entropy regularization, which is a strongly concave function, the maximization problem is a convex optimization problem. Owing to the implicit constraints provided by the entropy function, the problem is equivalent to having only the equality constraint. We then introduce the Lagrangian multiplier $\lambda$ and obtain the following relaxed problem:

$$
\widetilde{E}(\boldsymbol{\alpha})=\psi_{1}-\sum_{i=1}^n \alpha_{i} \psi_{i}+\tau \sum_{i=1}^n \alpha_{i}\log \alpha_{i}+\lambda\left(\boldsymbol{\alpha}^{\top} \mathbf{1}_n-1\right).
$$

As the relaxed problem is unconstrained, taking the derivative with respect to $\alpha_{i}$ yields

$$
\frac{\partial \widetilde{E}(\boldsymbol{\alpha})}{\partial \alpha_{i}}=-\psi_{i}+\tau\left(\log \alpha_{i}+\alpha_{i} \frac{1}{\alpha_{i}}\right)+\lambda=0.
$$

Solving the above equation implies that $\alpha_{i}$ takes the form
$
\alpha_{i}=\exp \left(\frac{1}{\tau} \psi_{i}\right) \exp \left(\frac{-\lambda}{\tau}-1\right).
$ Since $\alpha_{i}$ lies on the probability simplex, the optimal $\alpha_{i}$ is explicitly given by
$
\alpha^{*}_{i}=\frac{\exp \left(\frac{1}{\tau} \psi_{i}\right)}{\sum_{i^{\prime}=1}^n \exp \left(\frac{1}{\tau} \psi_{i^{\prime}}\right)} .
$ Substituting the optimal point into the objective function, we obtain
$$
\begin{aligned}
E\left(\boldsymbol{\alpha}^*\right)  &=\psi_1-\sum_{i=1}^n \frac{\exp \left(\frac{1}{\tau} \psi_{i}\right)}{\sum_{i^{\prime}=1}^n \exp \left(\frac{1}{\tau} \psi_{i^{\prime}}\right)} \psi_{i}+\tau \sum_{i=1}^n \frac{\exp \left(\frac{1}{\tau} \psi_{i}\right)}{\sum_{i^{\prime}=1}^n \exp \left(\frac{1}{\tau} \psi_{i^{\prime}}\right)}\log \frac{\exp \left(\frac{1}{\tau} \psi_{i}\right)}{\sum_{i^{\prime}=1}^n \exp \left(\frac{1}{\tau} \psi_{i^{\prime}}\right)} \\
& =\psi_1 - \tau \log \left(\sum_{i=1}^n \exp \left(\frac{1}{\tau} \psi_{i}\right)\right).
\end{aligned}
$$
Thus, the Lagrangian dual function is given by
\begin{equation*}
-E\left(\boldsymbol{\alpha}^*\right)= -\tau \log \frac{\exp \left(\frac{1}{\tau} \psi_{1}\right)}{\sum_{i=1}^n \exp \left(\frac{1}{\tau} \psi_{i}\right)}.\qedhere
\end{equation*}
\end{proof}



\section{More on Experiments} \label{section: experiment_details}

\paragraph{CIFAR-10 and CIFAR-100} CIFAR-10 ~\citep{krizhevsky2009learning} and CIFAR-100 ~\citep{krizhevsky2009learning} are well-known classic image classification datasets. Both CIFAR-10 and CIFAR-100 contain a total of 60k $32 \times 32$ labeled images of different classes, with 50k for training and 10k for testing. CIFAR-10 is similar to CIFAR-100, except there are 10 different classes in CIFAR-10 and 100 classes in CIFAR-100.

\paragraph{TinyImageNet} TinyImageNet ~\citep{le2015tiny} is a subset of ImageNet ~\citep{deng2009imagenet}. There are 200 different object classes in TinyImageNet, with 500 training images, 50 validation images, and 50 test images for each class. All the images in TinyImageNet are colored and labeled with a size of $64 \times 64$.

\textbf{Pseudo-code.} Algorithm \ref{alg:Training Procedure} presents the pseudo-code for our empirical training procedure.

\begin{algorithm}[!htbp]
\caption{Training Procedure}
\label{alg:Training Procedure}
\begin{algorithmic}[1]
\REQUIRE trainable encoder network $f$, batch size $N$, augmentation strategy \textit{aug}, loss function $L$ with hyperparameters \textit{args}
\FOR {sampled minibatch ${x_i}_{i=1}^N$}
\FORALL{$i \in { 1, ..., N }$}
\STATE draw two augmentations $t_i = \textit{aug}\left(x_i\right) $, $t_i' = \textit{aug}\left(x_i\right) $
\STATE $z_i = f\left(t_i\right)$, $z_i' = f\left(t_i'\right)$
\ENDFOR
\STATE compute loss $\mathcal{L} = L(N, z, z', \textit{args})$
\STATE update encoder network $f$ to minimize $\mathcal{L}$
\ENDFOR
\STATE \textbf{Return} encoder network $f$
\end{algorithmic}
\end{algorithm}

We also provide the pseudo-code for our core loss function used in the training procedure in Algorithm \ref{alg:Core loss}. The pseudo-code is almost identical to SimCLR's loss function, with the exception of an extra parameter $\gamma$.

\begin{algorithm}[!htbp]
\caption{Core loss function $\mathcal{C}$}
\label{alg:Core loss}
\begin{algorithmic}[1]
\REQUIRE batch size $N$, two encoded minibatches $z_1, z_2$, $\gamma$, temperature $\tau$
\STATE $z = \textit{concat}\left(z_1, z_2\right)$
\FOR {$i \in {1, ..., 2N }, j \in {1, ..., 2N}$ }
\STATE $s_{i,j} = \Vert z_i - z_j \Vert_2^{\gamma}$
\ENDFOR
\STATE \textbf{define} $l(i, j)$ \textbf{as} $l(i, j) = - \log \frac{exp\left(s_{i,j}/\tau \right)}{\sum_{k=1}^{2N} \mathbf{1}{[k \ne i]} exp\left(s{i, j} / \tau \right)} $
\STATE \textbf{Return} $\frac{1}{2N} \sum_{k=1}^N\left[l(i, i+N) + l(i+N, i)\right]$
\end{algorithmic}
\end{algorithm}

Utilizing the core loss function $\mathcal{C}$, we can define all kernel loss functions used in our experiments in Table \ref{table: loss definition}. For all $z_i \in z$ with even dimensions $n$, we define $z_{L_i} = z_i\left[0:n/2\right]$ and $z_{R_i} = z_i\left[n/2:n\right]$.

\begin{table}[ht]
\centering
\begin{tabular}{{@{}l|l@{}}}
Kernel  &  Loss function \\ \midrule
Laplacian & $\mathcal{C}\left(N, z, z', \gamma=1, \tau\right)$\\ \midrule
Sum       & $\lambda * \mathcal{C}\left(N, z, z', \gamma=1, \tau_1\right) + (1-\lambda) * \mathcal{C}\left(N, z, z', \gamma=2, \tau_2\right)$  \\ \midrule
Concatenation Sum&$\lambda * \mathcal{C}\left(N, z_L, z'_L, \gamma=1, \tau_1\right) + (1-\lambda) * \mathcal{C}\left(N, z_R, z'_R, \gamma=2, \tau_2\right)$\\ \midrule
$\gamma = 0.5$ & $\mathcal{C}\left(N, z, z', \gamma=0.5, \tau\right)$          \\ 

\end{tabular}

\caption{Definition of kernel loss functions in our experiments}
\label {table: loss definition}
\end{table}

\textbf{Baselines.} We reproduce the SimCLR algorithm using PyTorch Lightning~\citep{PytorchLightning}.

\textbf{Encoder details.}
The encoder $f$ consists of a backbone network and a projection network. We employ ResNet50~\citep{ResNet} as the backbone and a 2-layer MLP (connected by a batch normalization~\citep{ioffe2015batch} layer and a ReLU \cite{nair2010rectified} layer) with hidden dimensions 2048 and output dimensions 128 (or 256 in the concatenation kernel case).

\textbf{Encoder hyperparameter tuning.}
For each encoder training case, we randomly sample 500 hyperparameter groups (sample details are shown in Table \ref{table: Hyperparameter sample}) and train these samples simultaneously using Ray Tune ~\citep{RayTune}, with the ASHA scheduler~\citep{li2018massively}. Ultimately, the hyperparameter group that maximizes the online validation accuracy (integrated in PyTorch Lightning) within 5000 validation steps is chosen for the given encoder training case.

\begin{table}[ht]
\centering

\begin{tabular}{@{}l|l|l@{}}
\midrule
Hyperparameter  & Sample Range & Sample Strategy \\ \midrule
start learning rate & $\left[10^{-2}, 10\right]$ & log uniform \\ \midrule
$\lambda$       & $\left[0, 1\right]$ & uniform \\ \midrule
$\tau$, $\tau_1$, $\tau_2$ & $\left[0, 1\right]$ & log uniform \\ \midrule
\end{tabular}

\caption{Hyperparameters sample strategy}
\label {table: Hyperparameter sample}
\end{table}

\textbf{Encoder training.} 
We train each encoder using the LARS optimizer~\citep{LARSOptimizer}, LambdaLR Scheduler in PyTorch, momentum 0.9, weight decay $10^{-6}$, batch size 256, and the aforementioned hyperparameters for 400 epochs on a single A-100 GPU.

\textbf{Image transformation.} The image transformation strategy, including augmentation, is identical to the default transformation strategy provided by PyTorch Lightning.

\textbf{Linear evaluation.}
The linear head is trained using the SGD optimizer with a cosine learning rate scheduler, batch size 64, and weight decay $10^{-6}$ for 100 epochs. The learning rate starts at $0.3$ and ends at $0$.

\textbf{Moco Experiments.} We also tested our method based on MoCo~\citep{he2019moco}. The results are summarized in Table \ref{tab:results-moco}. Here we choose ResNet18~\citep{ResNet} as the backbone and set a temperature of $0.1$ as default. For our simple sum kernel, we set $\lambda=0.8$. The results show that our method outperforms the original MoCo method.

\begin{table}[thb]
\centering
\caption{MoCo Experiment Results on CIFAR-10 and CIFAR-100.}
\label{tab:results-moco}
\resizebox{\textwidth}{!}{%
\begin{tabular}{@{}c|ccc|ccc@{}}
\toprule
\multirow{3}{*}{Method} & \multicolumn{3}{c|}{CIFAR-10} & \multicolumn{3}{c}{CIFAR-100} \\ \cmidrule(lr){2-4} \cmidrule(lr){5-7} 
                        & 200 epochs & 400 epochs    & 1000 epochs   & 200 epochs & 400 epochs & 1000 epochs         \\ \midrule
MoCo (repro.)         & $76.41 \pm 0.12$    & $80.01 \pm 0.15$          & $84.45 \pm 0.08$    & $\mathbf{47.02 \pm 0.11}$ & $52.50 \pm 0.07$ & $57.62 \pm 0.15$            \\
\midrule
Laplacian Kernel        & ${78.09 \pm 0.10}$    & $\mathbf{83.85 \pm 0.09}$          & $\mathbf{88.34 \pm 0.16}$    & $46.12 \pm 0.22$   & $53.44 \pm 0.17$ & $59.10 \pm 0.14$        \\
Simple Sum Kernel & $\mathbf{78.12 \pm 0.15}$   & $83.23 \pm 0.18$ & $87.50 \pm 0.20$ & $46.65 \pm 0.06$ & $\mathbf{53.62 \pm 0.19}$ & $\mathbf{59.83 \pm 0.12}$\\
\bottomrule
\end{tabular}
}
\end{table}



\section{More Experiments on Synthetic Data}


Consider a scenario with $n$ clusters, each containing $k$ vertices. Let the probability of vertices $u$ and $v$ from the same cluster belonging to $\bfpi$ be $p$. Conversely, for vertices $u$ and $v$ from different clusters, let the probability of belonging to $\pi$ be $q$. We generate the graph $\bfpi$ randomly, based on $p$ and $q$. We experiment with values of $k=100$ and $n=6$ for ease of visualization, embedding all points in a two-dimensional space. Each vertex's initial position originates from a normal distribution. In each iteration, we sample a subgraph of $\bfpi$ uniformly, ensuring each vertex has an out-degree of $1$. We then optimize the corresponding vectors using InfoNCE loss with an SGD optimizer and iterate until convergence. Our experimental setup consists of an SGD learning rate of $1$, an InfoNCE loss temperature of $0.5$, and a batch size of $50$. We evaluate two scenarios with different $p$ and $q$ values: $p=1$, $q=0$, and $p=0.75$, $q=0.2$. The results of these experiments are visualized in Figure \ref{fig:vis-spectral-cluster}. The obtained embeddings exhibit the hallmark pattern of spectral clustering of graph $\bfpi$.

\begin{figure}[!tb]
\centering
\subfigure{
\includegraphics[width=1\textwidth]{Figures/cluster_pi.png}
\label{fig:vis-cluster}
}
\subfigure{
\includegraphics[width=1\textwidth]{Figures/noised_cluster_pi.png}
\label{fig:vis-noised-cluster}
}
\caption{Visualizations of the optimization process using InfoNCE Loss on the vectors corresponding to $\bfpi$. Points of identical color belong to the same cluster within $\bfpi$. To showcase the internal structure of $\bfpi$, we randomly select 10 vertices from each cluster to display the edge distribution of $\bfpi$.}
\label{fig:vis-spectral-cluster}
\end{figure}



\end{sloppypar}
\end{document}
% end of file template.tex

