\section{Conclusions}

We present what we believe is the first convex formulation of bodies undergoing
large deformations coupled with articulated rigid bodies through contact and
holonomic constraints.

To achieve this, we introduce a material model in terms of a linearized Green
strain and demonstrate that our formulation is linear with a positive
semi-definite stiffness matrix when the rotational component of the deformation
gradients is lagged. This allows us to incorporate the modeling of large
deformations into our previous work on the convex formulation of contact
\cite{bib:castro2022unconstrained}. We exploit the structure of the problem in
two ways. Firstly, we partition the problem and express it as a smaller
optimization program that only includes constrained variables. Secondly, we show
how the expensive-to-compute Schur complements required in the optimization
problem can be obtained as an intermediate computation in the Cholesky
factorization of the momentum balance equations.

To demonstrate the effectiveness of our method, we present validation results
and simulations relevant to robotics. Specifically, we showcase the ability of
our approach to resolve stable grasps in two challenging manipulation tasks
robustly. 

Unlike other approaches that run a fixed number of iterations to stay within a
specified computational budget, our method always solves the fully constrained
problem to convergence at real-time rates. This is enabled by the strong
convergence guarantees of our method using the SAP solver
\cite{bib:kuppuswamy2020soft}. Therefore, our simulation results always satisfy
the momentum equations and friction laws without introducing difficult-to-detect
artifacts. This aspect is particularly critical for the meaningful
sim-to-real transfer of results.

Profiling reveals that parallelization of geometry and FEM elemental routines is
attractive in cases with a much larger number of degrees of freedom than
constraints. However, cases dominated by the number of constraints are less
amenable to parallelization since they are dominated by the cost of
factorizations and the SAP solver. Our analysis includes a summary of
limitations and future research directions.

Finally, we incorporated our method into the open-source robotics toolkit Drake
\cite{bib:drake} and hope that the simulation and robotics communities can
benefit from our contribution.
