\section{Results and Discussion}
\label{sec:results}

We present a series of simulation test cases to assess the robustness, accuracy,
and performance of our method. All simulations are carried out in a system with
two 16-core Intel\textsuperscript{\textregistered}
Xeon\textsuperscript{\textregistered} Gold 6226R Processors and 192 GB of RAM,
running Ubuntu 20.04. However, all of our tests run in a single thread. We
report scene statistics and performance numbers in Table \ref{tab:statistics}
and timing breakdowns of the most time-consuming routines in
Fig.~\ref{fig:time_breakdown}. We observe that the narrow phase geometry and the
element-wise FEM computations are embarrassingly parallelizable and can
immediately benefit from a parallel implementation. On the other hand, the
computations for the SAP solver and the Schur complement, both requiring
Cholesky factorizations, are less amenable to parallelization and require
careful investigation. The SAP problem for the arch example essentially is block
tridiagonal, and SAP's supernodal solver \cite{bib:davis2016survey} can
effectively exploit this structure. However, the structure of the problem is
very close to dense for the teddy bear example, and SAP factorizations become
the bottleneck. In addition, in less dynamic scenes (such as the masonry arch
and the cylinder press example), the SAP solver benefits from its warm-start
strategy \cite{bib:castro2022unconstrained}. The scalability of SAP with the
number of bodies is studied in \cite{bib:castro2022unconstrained} and with the
number of constraints in \cite{bib:masterjohn2022velocity}. All examples use our
corotational model except for the cylinder press benchmark, which uses a model
to match published results. Unless otherwise specified, the Rayleigh damping
model uses $\alpha = 0$ and $\beta = 0.01$. For all simulation results, SAP runs
to convergence with relative tolerance $\varepsilon_r=10^{-6}$
\cite{bib:castro2022unconstrained}.

\begin{table*}[t]
  \caption{Timing and scene statistics for all examples. We account for both
   rigid and deformable DoFs, with 3 DoFs per mesh vertex. The realtime rate is
   the ratio of simulated time over CPU time. We also record the maximum number
   of constraints (including contact and holonomic) per time step as well as the
   average number over all time steps.}
  \label{tab:statistics}
  \begin{center}
  \begin{tabular}{l|c|c|c|c} 
    \toprule
    Example                          & DoFs & Time Step (sec) & Realtime Rate & Constraints Average (Max) \\
    \hline
    Arch ($\mu=0.2$)                 & 225  & 0.04            & 7.26                  & 259.5 (357)       \\
    \hline
    Arch ($\mu=1.0$)                 & 225  & 0.04            & 6.29                  & 320.2 (364)       \\
    \hline
    Cylinder (4151 elements)         & 3796 & 0.01            & 0.06                  & 248.3 (335)       \\
    \hline
    Allegro hand                     & 410  & 0.01            & 1.01                  & 89.5 (328)       \\
    \hline
    Teddy bear                       & 2043 & 0.02            & 0.37                  & 236.5 (464)       \\
    \bottomrule
  \end{tabular}
\end{center}
\end{table*}

\begin{figure}[!h]
  \centering
  \adjincludegraphics[width=1.0\columnwidth, trim={{0.07\width} 0.0 {0.07\width} 0.0},clip]{figures/time_breakdown.png}
  \caption{\label{fig:time_breakdown} Cost budget (in percentage) of the total
    run time cost for main computation routines (geometry queries, Schur
    complement, element-wise computation for FEM, and SAP solver) for all
    examples. All other computation are grouped under ``other".}
\end{figure}

\subsection{Rubber Cylinder Pressed Between Two Rigid Plates}
To validate our method in resolving contact as well as the internal stress of
deformable bodies, we perform a benchmark extensively studied in the engineering
literature, with well-known numerical solution \cite{bib:bijelonja2005finite,
bib:sussman1987finite, bib:manual2013vm201}. 

A homogeneous cylinder with a 0.4 meter diameter is pressed between two
frictionless rigid plates. The top plate is pressed downward for a displacement
of 0.2 meter. A schematic of the setup is illustrated in
Fig.~\ref{fig:cylinder_press_schematic}. The material of the cylinder is
modeled as Mooney-Rivlin rubber with constants $C_1 = 0.293~\text{MPa}$, $C_2 =
0.177 ~\text{MPa}$, and bulk modulus $1410~\text{MPa}$. In particular, we use
the Mooney-Rivlin formulation described in \cite{bib:kim2020dynamic} for its
rest stability. The mass density of the material is set to
$1000~\text{kg}/\text{m}^3$ and there is no gravity. As in previous literature,
we consider a plane state of strains. To achieve that in our
inherently 3D formulation, we model the cylinder with two layers of linear
tetrahedral elements along the $z$-direction, assume $\partial/\partial z=0$ in
the computation of the deformation gradient $\bm{F}$, and impose zero out of
plane displacements to avoid drift due to round-off errors. We apply a downward
force on the top plate through a prismatic joint and measure the resulting
displacement. The contact constraints set in as the top plate is pressed
downward and are resolved automatically by our method. We increase the magnitude
of the force linearly in time from zero to $f_{\text{max}} =
1.4~\text{MN}/\text{m}$, to match \cite{bib:bijelonja2005finite}. To better
compare with static analysis results in the literature, we reduce inertial
effects by overdamping the cylinder with $\beta = 0.5$ and by slowly applying
the load over a period of 500 seconds. We perform a mesh refinement study using
four progressively finer meshes. The force-displacement curves obtained with our
method are shown in Fig.~\ref{fig:force_displacement_plot} along with the
reference solution from \cite{bib:bijelonja2005finite}. As the spatial
resolution is increased, the force-displacement curves converge to the reference
solution.

\begin{figure}[!h]
    \centering
    \adjincludegraphics[width=0.8\columnwidth]{figures/cylinder_press_schematic.png}
    \caption{\label{fig:cylinder_press_schematic} A solid deformable cylinder
     0.4 meter in diameter is compressed between a rigid plate and the ground.
     A downward force is applied on the plate through a prismatic joint. Plane
     strain assumption is made in the $xy$-plane.}
\end{figure}

\begin{figure}[!h]
    \centering
    \adjincludegraphics[width=0.8\columnwidth, trim={0 {0.0\width} {0.05\width} {0.04\width}},clip]{figures/cylinder_press_plot.png}
    \caption{\label{fig:force_displacement_plot} Force per unit length
      vs. displacement for the cylinder press benchmark illustrated in
      Fig.~\ref{fig:cylinder_press_schematic}. The force-displacement curves
      converge under mesh refinement.}
\end{figure}

\subsection{Masonry Arch}
We test the robustness of our method on the frictionally dependent stable arch
tests from \cite{bib:Kaufman2008,bib:li2020ipc,bib:macklin2019nonsmooth}. In
particular, we take the additionally challenging setup from \cite{bib:li2020ipc}
where the arch has its base balanced on sharp edges. Similar to
\cite{bib:li2020ipc}, we simulate the blocks in the arch as very stiff
deformable materials with a high Young's modulus of $20~\text{GPa}$, Poisson's
ratio $0.3$, and density $2300~\text{kg}/\text{m}^3$. While non-linear models
often struggle to converge in presence of such stiff materials due to
ill-conditioning, our method solves the free-motion momentum balance in
Eq.~\eqref{eq:free_motion_velocities} to machine precision with a single linear solve
thanks to our linearized corotated model. Our method is able to stably resolve
the frictional contacts among the blocks at $40~\text{ms}$ time step. We
simulate the structure for 10 minutes and verify the blocks with friction
coefficient $\mu = 1.0$ form a long-term stable arch. We also confirm that the
structure falls apart with a lower friction coefficient $\mu = 0.2$. Our method
has guaranteed convergence and therefore simulation results always satisfy
momentum balance, the model of Coulomb friction, and the maximum dissipation
principle.

\begin{figure}[!h]
    \centering
    \adjincludegraphics[width=1.0\columnwidth]{figures/arch_side_by_side.png}
    \caption{\label{fig:arch} A self-supporting masonry arch. Each block is
     modeled as a stiff deformable object to stress test the stability of our
     method and showcase its ability to resolve contact constraints among
     deformable bodies. With $\mu = 1.0$, the structure is held stable by
     friction (left) whereas with $\mu = 0.2$, the structure eventually falls
     apart (right).}
\end{figure}

\subsection{Soft-bubble Gripper}
\label{sec:bubble_gripper}
We simulate a \textit{Soft-bubble} gripper \cite{bib:kuppuswamy2020soft}
manipulating a deformable teddy bear, Fig. \ref{fig:teddy}. This
is a Schunk WSG 50 gripper with air filled rubber chambers as fingers providing
highly compliant gripping surfaces. We model the air-inflated latex membranes as
deformable volumetric objects with Young's modulus $50~\text{kPa}$ and Poisson's
ratio $0.3$. The flat base of the bubbles are attached to the gripper fingers
using holonomic constraints. The teddy bear is modeled with Young's modulus
$10~\text{kPa}$ and Poisson's ratio $0.4$. The  gripper is PD-controlled with a
prescribed \textit{close-lift-shake-place} motion sequence. We perform the
shaking motion to stress test the stability of the grasp in simulation, see the
supplemental video. The entire system with 2043 degrees of freedom is highly
coupled via up to 464 constraints. Our solver is able to solve all time steps of
this challenging scenario to convergence.
\begin{figure}[!h]
    \centering
    \adjincludegraphics[width=0.8\columnwidth]{figures/teddy.png}
    \caption{Highly compliant \label{fig:teddy} \emph{Soft-bubble} gripper
    \cite{bib:kuppuswamy2020soft} securing the grasp of a deformable teddy
    bear.}
\end{figure}

\subsection{Interactive Control}
\label{sec:allegro_hand}
A KUKA LBR iiwa arm (7 DoFs) outfitted with an anthropomorphic Allegro hand (16
DoFs) is teleoperated in real-time to manipulate a deformable ball, Fig.
\ref{fig:allegro}. Desired joint positions and velocities are computed using a
differential inverse kinematics controller that targets a desired end effector
pose, controlled interactively with a gamepad. The deformable ball is modeled
with Young’s modulus $25~\text{kPa}$ and Poisson’s ratio $0.4$. As the robot
closes its palm around the ball, contact patches between the ball and the palm,
phalanges, and fingertips of the hand form a secure grasp. We apply a shaking
motion to stress test the stability of the grasp, see supplemental video. Even
with the overhead of the controller, our simulation runs at real-time rates.
