\section{Convex Formulation of Constrained Dynamics}
\label{sec:convex_formulation}

From Eq.~\eqref{eq:scheme_momentum} we define the momentum residual as
\begin{equation}
	\mf{m}(\mf{v}) = 
	\mf{M}(\mf{q}^{\theta}(\mf{v}))(\mf{v}-\mf{v}_0) -
	\delta t\,\mf{k}(\mf{q}^{\theta}(\mf{v}), \mf{v}^{\theta}(\mf{v})).
	\label{eq:m_definition}
\end{equation}
With our choice of linearized corotational model, the balance of momentum for
deformable bodies is linear in $\mf{v}$. For rigid bodies, we follow
\cite{bib:castro2022unconstrained} where inertia and damping terms are treated
implicitly for stability and Coriolis and gyroscopic forces are treated
explicitly. This allows us to write a linearized version of
Eq.~\eqref{eq:scheme_momentum} as
\begin{equation}
  \mf{m}(\mf{v_0}) + \mf{A}(\mf{v} - \mf{v_0}) = \mf{A}(\mf{v}-\mf{v}^*) = \mf{J}^T\bgamma, \label{eq:momentum_linearized}
\end{equation}
where $\mf{A}=\partial\mf{m}/\partial\mf{v}$ is symmetric positive definite
(SPD) and independent of $\mf{v}$ and we define 
\begin{equation}
	\mf{v}^* \defeq \mf{v_0} - \mf{A}^{-1}\mf{m}(\mf{v_0})
	\label{eq:free_motion_velocities}
\end{equation}
as the \emph{free-motion} velocities that would result in the absence of
constraints, when $\bgamma=\mf{0}$ in Eq.~\eqref{eq:momentum_linearized}. Since
$\mf{A}$ is block diagonal, we solve the free-motion velocities for each body
separately. We discuss efficiency considerations in Section
\ref{sec:schur_complement}. From Section \ref{sec:deformables_modeling},
the block in $\mf{A}$ for the $b\text{-th}$ deformable body is given by
\begin{equation}
    \mf{A}_b = \left(1 + \alpha \theta \delta t\right)\mf{M}_b + \theta \delta t ( \theta_{vq} \delta t + \beta) \mf{K}_b.
    \label{eq:tangent_matrix}
\end{equation}
We note that this result is exact. Moreover $\mf{A}_b$ is SPD since $\mf{M}_b$
is SPD and, with our corotational model, $\mf{K}_b$ is positive semi-definite.
We combine the linearized momentum balance \eqref{eq:momentum_linearized} with
constraints \eqref{eq:convex_constraints} to write a convex formulation of the
dynamics
\begin{align}
\begin{split}
  &\mf{A}(\mf{v}-\mf{v}^*) = \mf{J}^T\bgamma,\\
  &\mathcal{C} \ni \bgamma \perp \mf{v}_c-\hat{\mf{v}}_c \in \mathcal{C}^*.
  \label{eq:optimiality_conditions}
\end{split}
\end{align}

Equations~\eqref{eq:optimiality_conditions} are the optimality conditions to the
following convex optimization problem \cite{bib:mazhar2014,
bib:castro2022unconstrained}
\begin{equation}
	\begin{aligned}
	\min_{\mf{v}} \quad &
	\frac{1}{2}\Vert\mf{v}-\mf{v}^*\Vert_{A}^2 \\
	\textrm{s.t.} \quad & \mf{J}\mf{v} + \mf{b}-\hat{\mf{v}}_c \in \mathcal{C}^*,\\
	\end{aligned}
	\label{eq:sap_primal}
\end{equation}
where we used $\mf{v}_c = \mf{J}\mf{v} + \mf{b}$ from
Section~\ref{sec:constraint_kinematics}. To solve Eq.~\eqref{eq:sap_primal}, we
use the Semi-Analytic Primal (SAP) solver from
\cite{bib:castro2022unconstrained} that formulates a regularized version of this
problem with proven global convergence to its unique solution.
