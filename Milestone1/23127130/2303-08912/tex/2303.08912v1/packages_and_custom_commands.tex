\usepackage{amsmath} 
\usepackage{amsfonts}
\usepackage{mathtools} 
\usepackage{amssymb}
\usepackage{graphicx}
\usepackage{bm}  % Bold math
\usepackage[colorinlistoftodos, textsize=footnotesize]{todonotes}
\usepackage[colorlinks=true, allcolors=black]{hyperref}
\usepackage{setspace}
\usepackage{algorithm}      % http://ctan.org/pkg/algorithm
\usepackage{algpseudocode}  % http://ctan.org/pkg/algorithmicx
\usepackage{tikz}
\usepackage{verbatim}
\usepackage{xcolor}
\usepackage{subcaption}

% Package adjustbox: Introduces \adjincludegraphics to include figures allowing
% trimming.
\usepackage{adjustbox}

% Package ulem: The package provides an \ul (underline) command which will break
% over line ends; this technique may be used to replace \em (both in that form
% and as the \emph command), so as to make output look as if it comes from a
% typewriter. The package also offers double and wavy underlining, and striking
% out (line through words) and crossing out (/// over words).
\usepackage[normalem]{ulem}

\usepackage[thinc]{esdiff}

% Package amsthm: Provides the proof environment. N.B. this MUST go before the
% \newtheorem below. Do not change the order.
% Theorems numbered on a per section basis.
\newtheorem{theorem}{Theorem}[section]
\newtheorem{corollary}{Corollary}[theorem]
\newtheorem{prop}{Proposition}
\newtheorem{lemma}{Lemma}

\usepackage{mathrsfs}
\usepackage{booktabs}

\newcommand{\RedHighlight}[1]{{\color{red}\textbf{#1}}}

% Theorems numbered as explained in the IEEEtran HOWTO guide.

\newif\ifcompile
%\compiletrue % uncomment out to compile


% Place this declaration AFTER amsmath is included.
\DeclareMathOperator*{\argmin}{arg\,min}
\DeclareMathOperator{\tr}{tr}

% TikZ commands for drawing schematics.
\newcommand{\tikzmark}[1]{\tikz[overlay, remember picture] \coordinate (#1);}

%For making diagonal entries in a matrix.
\newcommand{\diagentry}[1]{\mathmakebox[1.8em]{#1}}

% Math Shortcuts: Vector Font: For 3D vectors.
\newcommand{\vf}[1]{{\bm{#1}}}
% Matrix Font: for matrices, arrays and concatenation of 3D vectors.
\newcommand{\mf}[1]{{\mathbf{#1}}}

% Bold greek symbols:
\newcommand{\bgamma}{{\bm\gamma}} \newcommand{\btgamma}{{\bm{\tilde\gamma}}}
\newcommand{\bsigma}{{\bm\sigma}}
\newcommand{\bepsilon}{{\bm\epsilon}}
\newcommand{\bxi}{{\bm\xi}}
\newcommand{\bphi}{{\bm\phi}}

% Definition symbol. Section 3 (page 15) of "The Comprehensive LATEX Symbol
% List".
\newcommand\defeq{\stackrel{\text{\tiny def}}{=}}
