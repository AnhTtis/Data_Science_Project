\section{Mathematical Formulation}

The state of our system is described by generalized positions $\mf{q} \in
\mathbb{R}^{n_q}$ and generalized velocities $\mf{v} \in \mathbb{R}^{n_v}$ where
$n_q$ and $n_v$ denote the total number of generalized positions and velocities,
respectively. Time derivatives of the configurations are related to the
generalized velocities by $\dot{\mf{q}} = \mf{N}(\mf{q})\mf{v}$, with
$\mf{N}(\mf{q}) \in \mathbb{R}^{n_q \times n_v}$ being the kinematic map. In
particular, we use joint coordinates to describe articulated rigid bodies and
FEM to spatially discretize deformable bodies. As a result, for deformable
bodies, $\mf{q}$ and $\mf{v}$ are the stacked positions and velocities of the
mesh vertices, and the kinematic map $\mf{N}$ is the identity.

\subsection{Kinematics of Constraints}
\label{sec:constraint_kinematics}

We consider $n_c$ constraints described at the velocity level by the constraint
velocity $\mf{v}_{c,i}\in\mathbb{R}^{r_i}$, with $r_i$ the number of equations
for the $i\text{-th}$ constraint. For contact constraints, $\mf{v}_{c,i}$
corresponds to the relative velocity vector at the $i\text{-th}$ contact pair
and $r_i=3$, see \cite{bib:castro2022unconstrained}. For a holonomic constraint
described by $\mf{g}_i(\mf{q},t)=0$, $\mf{v}_{c,i} = d\mf{g}_i/dt$. We can write
constraint velocities in terms of the generalized velocities of the system
(including both rigid and deformable body degrees of freedom) as
$\mf{v}_{c,i}=\mf{J}_i\mf{v} + \vf{b}_i$, with $\mf{J}_i$ and
$\vf{b}_i=\partial\mf{g}_i/\partial t$ the Jacobian and bias, respectively.
Collecting all constraints, we define the stacked constraint velocity
$\mf{v}_{c}=\mf{J}\mf{v} + \mf{b}$, of size $r=\sum r_i$, where $\mf{J}$ and
$\mf{b}$ are the stacked Jacobian and bias term for all constraints.

\subsection{Discrete Time Stepping Scheme}
\label{sec:discrete_time_stepping}

We follow the scheme and notation in
\cite{bib:castro2022unconstrained} closely. We discretize time into intervals of
fixed size $\delta t$ to advance the state of the system from time $t_n$ to the
next step at $t_{n+1} = t_n + \delta t$. To simplify notation, we use the naught
subscript to denote quantities at $t_n$ while no additional subscript is used
for quantities at $t_{n+1}$. We define quantities evaluated at intermediate time
steps $t^\theta = \theta t^{n+1}+(1-\theta)t^{n}$ in accordance with the
standard $\theta\text{-method}$ using scalar parameters $\theta$ and
$\theta_{vq} \in [0, 1]$
\begin{align}
	\mf{q}^{\theta} &\defeq \theta\mf{q} + (1-\theta)\mf{q}_0,\nonumber\\
	\mf{v}^{\theta} &\defeq \theta\mf{v} + (1-\theta)\mf{v}_0,\nonumber\\
	\mf{v}^{\theta_{vq}} &\defeq \theta_{vq}\mf{v} + (1-\theta_{vq})\mf{v}_0,
	\label{eq:theta_method}
\end{align}
so that we can accommodate backward Euler with $\theta=\theta_{vq}=1$,
symplectic Euler with $\theta=0$, $\theta_{vq}=1$  and the second order midpoint
rule with $\theta=\theta_{vq}=1/2$ in the same formulation. Using these
definitions we can write the constrained dynamics of rigid and deformable bodies
within a unified framework as follows
\begin{flalign}
  % Momentum equation.
  &\mf{M}(\mf{q}^{\theta})(\mf{v}-\mf{v}_0) = \delta t \,
   \mf{k}(\mf{q}^{\theta},\mf{v}^{\theta}) + \mf{J}(\mf{q}_0)^T\mf{\bgamma},
   \label{eq:scheme_momentum}\\
  % Constraints
  &\mathcal{C} \ni \bgamma \perp \mf{v}_c-\hat{\mf{v}}_c \in \mathcal{C}^*,
  \label{eq:convex_constraints}\\
  % Positions update.
  &\mf{q} = \mf{q}_0 + \delta t \mf{N}(\mf{q}^{\theta})\mf{v}^{\theta_{vq}}.
  \label{eq:scheme_q_update}
\end{flalign}

Equation~\eqref{eq:scheme_momentum} collects the momentum equations for both
rigid and deformable degrees of freedom.  For each articulated rigid body and
deformable body with $n_{v_b}$ degrees of freedom, we use $\mf{M}_b$ to denote
its mass matrix and $\mf{k}_b$ to encode external forces and gyroscopic terms
for articulated rigid bodies \cite{bib:castro2022unconstrained} and internal
forces for deformable bodies. We use $\mf{M}$ to denote the block diagonal
matrix with $\mf{M}_b$ as the diagonal blocks and $\mf{k}$ to denote the stacked
$\mf{k}_b$ for all bodies. Rigid and deformable degrees of
freedom are coupled through the constraint Jacobian $\mf{J}(\mf{q}_0)$ whenever
constraint impulses $\bgamma$ are non-zero. Following \cite{bib:mazhar2014},
Eq.~\eqref{eq:convex_constraints} encodes both holonomic constraints and the
convex approximation of contact constraints. The convex set $\mathcal{C} =
\mathcal{C}_1 \times \mathcal{C}_2 \times \cdots \times \mathcal{C}_{n_c}$ is
the Cartesian product of sets $\mathcal{C}_{i}$ for the $i\text{-th}$
constraint. For contact constraints, $\mathcal{C}_{i}$ corresponds to the
friction cone $\mathcal{F}_{i}$ for that contact
\cite{bib:castro2022unconstrained}. For bi-lateral holonomic constraints,
$\mathcal{C}_{i}=\mathbb{R}^{r_i}$ and impulses can take any real value. The
\emph{dual} of $\mathcal{C}$ is denoted with $\mathcal{C}^*$. For contact
constraints, $\hat{\mf{v}}_{c,i}$ encodes information about the contact distance
at the previous time step, see \cite{bib:castro2022unconstrained} for details.
For holonomic constraints $\hat{\mf{v}}_{c,i} = -\mf{g}_i(\mf{q}_0)/\delta t$,
where $\mf{g}_i(\mf{q}_0)$ is the constraint error at the previous time step
\cite{bib:mazhar2014}. Finally, Eq. \eqref{eq:scheme_q_update} is the positions
update in accordance to the $\theta\text{-method}$.

Next we discuss the modeling of deformable bodies and refer the reader to
\cite{bib:castro2022unconstrained} for details on the treatment of articulated
rigid bodies.


\section{Modeling Deformable Bodies}
\label{sec:deformables_modeling}
Here we describe the equations of motion for a single deformable body and drop
the $b$ subscript of $\mf{M}$ and $\mf{k}$ for simplicity.

\subsection{Preliminaries}
\label{sec:deformables_modeling}

We follow a standard formulation of continuum mechanics discretized by FEM. We
describe it briefly to provide context and introduce notation, and refer the
reader to \cite{bib:bonet2021book, bib:belytschko2014book} for further details.
Within this framework, we describe choices specific to our work.

The kinematics of the model is fully specified by the Lagrangian description map
$\bm{\phi}(\bm{X},t)$ which describes the current position of material points in
the solid as a function of their reference position $\bm{X}$ and time $t$. We
consider hyperelastic solids modeled with an energy density function
$\Psi(\mf{F}(\bm{X}))$, where $\mf{F}=\partial\bm{\phi}/\partial\bm{X}$ is the
deformation gradient. The elastic potential energy for the entire solid is given
by
\begin{equation*}
    E = \int \Psi(\mf{F}(\bm{X})) d\bm{X},
\end{equation*}
which induces elastic force $\mf{f}_e = -\partial E / \partial \mf{q}$. The
stiffness matrix is defined as $\mf{K}=-\partial\mf{f}_e/\partial\mf{q}$. We use
the Rayleigh model of damping
\begin{equation}
    \mf{f}_d(\mf{v}) = -\left(\alpha \mf{M} + \beta \mf{K} \right) \mf{v},
\end{equation}
where $\alpha$ and $\beta$ are Rayleigh damping coefficients and $\mf{M}$ is the
mass matrix from FEM discretization.
With these internal forces, $\mf{k}$ in Eq.~\eqref{eq:scheme_momentum} can be
written as
\begin{equation}
\mf{k}(\mf{v}) = \mf{f}_e(\mf{q^\theta}(\mf{v})) + \mf{f}_d(\mf{v^\theta}(\mf{v})) + \mf{f}_{\text{ext}},
\end{equation}
where $\mf{f_{\text{ext}}}$ includes constant external forces, such as a
gravity.

\subsection{Corotational Material Model}
\label{sec:corotational_model}

In principle, our framework can use any hyperelastic energy density
$\Psi(\mf{F}(\bm{X}))$. We refer to \cite{bib:sifakis2012fem} for implementation
notes and to \cite{bib:kim2020dynamic} for a survey of popular choices. However,
the stiffness matrix $\mf{K}$ can often become indefinite and hinder the
convergence of iterative Newton solvers at large deformations
\cite{bib:teran2005robust, bib:kim2020dynamic}. Standard remedies include
projecting the energy density Hessian to be positive semi-definite
\cite{bib:teran2005robust} or clamping eigenvalues to be non-negative
\cite{bib:smith2019analytic}, both associated with additional computational
cost. On the other hand, linear constitutive models introduce unacceptable
artifacts when large rotational deformation is present
\cite{bib:muller2004interactive}. 

We propose a corotational linear model that enjoys the computational efficiency
of the linear model while avoiding the rotational artifacts at the same time. In
particular, we adopt a Saint-Venant-Kirchhoff model with a corotational linear
Green strain $\hat{\mf{E}}$
\begin{equation}\label{eq:energy_density}
    \Psi(\hat{\mf{E}}) = \mu \|\hat{\mf{E}}\|_F^2 + \frac{\lambda}{2} \tr(\hat{\mf{E}})^2
\end{equation}
where $\mu$ and $\lambda$ are Lam\'{e} parameters.
To define the corotational $\hat{\mf{E}}$, we first define a
\emph{corotated deformation gradient}
\begin{equation*}
\hat{\mf{F}}(\bm{X}) \defeq \hat{\mf{R}}(\bm{X})^T \mf{F}(\bm{X})
\end{equation*}
where $\hat{\mf{R}}(\bm{X})$ is a rotation matrix to be chosen to eliminate the
rotational component of the deformation. Letting $\mf{B} = \hat{\mf{F}} -
\mf{I}$, the Green strain is given by 
\begin{align*}
    \mf{E} &= \frac{1}{2}\left(\mf{F}^T \mf{F} - \mf{I}\right) \\
    &= \frac{1}{2}\left(\hat{\mf{F}}^T \hat{\mf{F}} - \mf{I}\right) = \frac{1}{2}\left(\left(\mf{I} + \mf{B}^T\right) \left(\mf{I} + \mf{B}\right) - \mf{I}\right).
\end{align*}

By dropping the non-linear term $\mf{B}^T \mf{B}$, we arrive at the definition
of the linearized Green strain 
\begin{align*}
\hat{\mf{E}} &\defeq \frac{1}{2}(\hat{\mf{F}} + \hat{\mf{F}}^T) - \mf{I} .
\end{align*}

With $\hat{\mf{R}}=\mf{R}$, the rotational component of $\mf{F}$ from its polar
decomposition, $\hat{\mf{E}}$ becomes the strain measure used in
\cite{bib:mcadams2011efficient}. While this eliminates rotational artifacts,
the non-linear $\Psi$ can lead to indefiniteness in $\mf{K}$.
Instead, we approximate $\hat{\mf{R}}$ with $\mf{R}_0$, the rotational component
of the polar decomposition of $\mf{F}_0(\bm{X})$ at the previous time step. With
this choice, our model is linear, rotational artifacts are negligible (see results in
Section \ref{sec:results}), and the stiffness matrix $\mf{K}$ has an analytic
form guaranteed to be positive semi-definite, as proved in the Appendix. Though
similar to \cite{bib:muller2004interactive}, we do not approximate $\mf{K}$ by
dropping non-linear terms in the derivatives \cite{bib:barbic2012exact}.
Our formulation is linear in $\mf{q}$ by design and naturally generalizes to non-tetrahedral elements.
