\subsection{Participating Degrees of Freedom}
\label{sec:constraint_resolution}

SAP \cite{bib:castro2022unconstrained} solves a non-linear system of
equations of size $n_v$, usually large with deformable bodies in the system. It
is often impractical to solve such a large scale non-linear system for
interactive robotics applications, so we aim to exploit the structure of the
program \eqref{eq:sap_primal} to reduce its size. 

The constraint Jacobian $\mf{J}$ is often sparse because the constraint velocity
$\mf{v}_{c,i}$ for the $i\text{-th}$ constraint only involves a local set of
vertices. Moreover, since contact develops at the surface of deformable objects,
most columns of $\mf{J}$ that correspond to inner mesh vertices are zero.
Degrees of freedom (DoFs) associated with non-zero columns in $\mf{J}$ are
herein referred to as \emph{participating} DoFs. All other DoFs are referred to
as \emph{non-participating}. For the rest of this section we work on a single
body and drop the $b$-subscript. We use $m_p$ and $m_n$ to denote the number of
participating and non-participating DoFs for this body and in general use
subscripts $p$ and $n$ for participating and non-participating quantities,
respectively. With this partition, we permute DoFs so that non-participating
DoFs appear before participating ones
\begin{gather}\label{eq:permuted_constraint_equation}
  \begin{bmatrix}  
      \mf{A}_{nn} &  \mf{A}_{np} \\
      \mf{A}_{pn}  &  \mf{A}_{pp} 
  \end{bmatrix}
  \begin{bmatrix}  
    \mf{\Delta v}_{n} \\
    \mf{\Delta v}_{p}
  \end{bmatrix}
  =
  \begin{bmatrix}
    \mf{0} \\
    \mf{J}_{p}^T \bgamma 
  \end{bmatrix}
\end{gather}
with $\mf{\Delta v} \defeq \mf{v} - \mf{v^*}$ and $\mf{J}_p$ being the columns
of $\mf{J}$ corresponding to participating DoFs of the deformable body under
consideration. Notice how this partition leads to an (often large) block of
zeros of size $m_n$ on the right hand side. This allows us to eliminate
$\mf{\Delta v}_{n}$ algebraically
\begin{equation}
 \mf{\Delta v}_{n} = -\mf{A}_{nn}^{-1} \mf{A}_{np} \mf{\Delta v}_p
 \label{eq:dv_n}
\end{equation}
to obtain a system in terms of participating DoFs only
\begin{equation}\label{schur_complement_constraint_equation}
  \hat{\mf{A}}\mf{\Delta v}_p = \mf{J}_{p}^T \bgamma 
\end{equation}
where
$\hat{\mf{A}} = \mf{A}_{pp}  -  \mf{A}_{pn} \mf{A}_{nn}^{-1} \mf{A}_{np} \in
\mathbb{R}^{m_p\times m_p}$ is the Schur complement of $\mf{A}$. Since
constraints in \eqref{eq:sap_primal} only involve participating DoFs, we can now
solve a much smaller optimization problem in terms of participating DoFs
\begin{equation}
	\begin{aligned}
	\min_{\mf{v}} \quad &
	\frac{1}{2}\Vert\mf{v}-\mf{v}_{p}^*\Vert_{\hat{A}}^2 \\
	\textrm{s.t.} \quad & \mf{J}\mf{v} + \mf{b}-\hat{\mf{v}}_c \in \mathcal{C}^*,\\
	\end{aligned}
	\label{eq:sap_primal_participating}
\end{equation}
where $\mf{v}_{p}^*$ corresponds to the free-motion velocities of participating
DoFs only. We then use \eqref{eq:dv_n} to update non-participating DoFs
velocities.

\subsection{Schur Complement Computation}
\label{sec:schur_complement}

Explicitly forming $\mf{A}_{nn}^{-1}$ needed in the Schur complement
$\hat{\mf{A}}$ is computationally expensive and impractical. However, we can
obtain $\hat{\mf{A}}$ as an intermediate result during the factorization of
$\mf{A}$ as required for the computation of free-motion velocities in
Eq.~\eqref{eq:free_motion_velocities}. We start from the factorization
of the permuted $\mf{A}$
\begin{equation}
  \begin{bmatrix}  
      \mf{A}_{nn} &  \mf{A}_{np} \\
      \mf{A}_{pn}  &  \mf{A}_{pp} 
  \end{bmatrix}
  =
  \begin{bmatrix}  
      \mf{L}_{nn} &  \mf{0} \\
      \mf{L}_{pn}  &  \mf{L}_{pp} 
  \end{bmatrix}
  \begin{bmatrix}  
      \mf{L}_{nn} &  \mf{0} \\
      \mf{L}_{pn}  &  \mf{L}_{pp} 
  \end{bmatrix}^T
\end{equation}
to see that
\begin{align*}
\hat{\mf{A}} =~& \mf{A}_{pp}  -  \mf{A}_{pn} \mf{A}_{nn}^{-1} \mf{A}_{pp} \\
= ~& \mf{L}_{pn} \mf{L}_{pn}^T +  \mf{L}_{pp} \mf{L}_{pp}^T 
-\mf{L}_{pn} \mf{L}_{nn}^T (\mf{L}_{nn} \mf{L}_{nn}^T)^{-1} \mf{L}_{nn} \mf{L}_{pn}^T \\
= ~& \mf{L}_{pp} \mf{L}_{pp}^T \\
= ~& \mf{A}_{pp} - \mf{L}_{pn} \mf{L}_{pn}^T,
\end{align*}

Therefore, we factorize the permuted $\mf{A}$ with a right-looking Cholesky
factorization and record the Schur complement matrix $\hat{\mf{A}}$ after block
$\mf{A}_{nn}$ has been factorized. Moreover, the same factorization of $\mf{A}$
is used to solve Eqs.~\eqref{eq:free_motion_velocities} and ~\eqref{eq:dv_n}.
Equations within participating and non-participating partitions can be reordered
to reduce fill-ins in the Cholesky factorization. We observe that AMD ordering
\cite{bib:amestoy1996approximate} within a partition leads to fewer fill-ins
than arbitrary ordering.
