\section{Previous Work}

Early work in the finite elements community focused on the mathematical
formalism and model fidelity of methods for the accurate computation of stresses
and deformations at contacting surfaces. Node-on-segment approaches
\cite{bib:hallquist1985sliding, bib:chaudhary1986solution,
bib:bathe1997constraint} enforce contact such that nodes on a contactor surface
do not penetrate their opposing target segments (or facets in three dimensions).
Two-pass approaches iterate this process by reversing the role of
contactor/target surfaces \cite{bib:bittencourt1998}. However, these methods
proved to lack robustness and lead to locking due to over-constraint. To address
these problems, the mortar formalism \cite{bib:MadayOriginalMortarWork1988} was
introduced in segment-to-segment approaches for contact
\cite{bib:bib:mcdevitt2000mortar, bib:puso2004mortar}.

The computer graphics community has contributed significantly, particularly on
performant methods for visually realistic simulations. Projected Gauss Seidel
(PGS) variants \cite{bib:muller2007position, bib:shinar2008two} are popular
given they can be easily parallelized. However, they lack robustness due to
convergence issues \cite{bib:erleben2007velocity}. In the pursuit of increased
robustness, optimization based solvers were developed
\cite{bib:Kaufman2008,bib:gast2015optimization,bib:li2020ipc}. Work on elements
that can recover from inversion \cite{bib:irving2004invertible,
bib:teran2005robust, bib:stomakhin2012energetically} and on corotational models
\cite{bib:mcadams2011efficient, bib:muller2004interactive} further increased the
robustness of these formulations at large deformations.

Multibody dynamics with frictional contact is complicated by the non-smooth
nature of the solutions. Non-existence of solutions \cite{bib:baraff1993issues},
exponential worst-case complexity \cite{bib:baraff1994fast}, and NP-hardness
\cite{bib:Kaufman2008} have led the community to search alternative
formulations. To improve computational tractability, Anitescu introduces a
\textit{convex approximation} of the contact problem \cite{bib:anitescu2006}.
For a strictly convex formulation with unique solution, \cite{bib:todorov2014}
introduces regularization. A convex formulation of compliant contact is
presented in \cite{bib:castro2022unconstrained} and later extended to allow
continuous surface patches in \cite{bib:masterjohn2022velocity}.

Several open-source software options support deformable object modeling. MuJoCo
\cite{bib:todorov2012mujoco} simulates deformable bodies as a group of connected
rigid bodies with spring-dampers. Chrono \cite{bib:chrono2016}  and Siconos
\cite{bib:acary2019siconos} implement the Finite Element Method (FEM), though
mostly target large scale applications, such as granular media simulation. Game
engines like Bullet \cite{bib:bullet} and PhysX \cite{bib:physx} are popular in
reinforcement learning. SOFA \cite{bib:sofa2012}, initially designed for virtual
reality, has also been successful in soft robotics projects.