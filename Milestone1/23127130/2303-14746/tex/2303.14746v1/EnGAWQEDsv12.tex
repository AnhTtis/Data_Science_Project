%\documentclass[preprint,pra,floatfix,showpacs]{revtex4-1}
\documentclass[aps,pra,showpacs,amsmath,amssymb,amsfonts,tightenlines,twocolumn,lengthcheck]{revtex4-1}
\usepackage{diagbox}
\usepackage{multirow}
\usepackage{amsmath}
\usepackage{amsfonts}
\usepackage{graphicx}
\usepackage{epsfig}
\usepackage{color}
\usepackage{txfonts}
\usepackage[colorlinks,citecolor=blue]{hyperref}

\begin{document}
\title{Giant-atom entanglement in waveguide-QED systems including non-Markovian effect}
\author{Xian-Li Yin}
\affiliation{Key Laboratory of Low-Dimensional Quantum Structures and Quantum Control of Ministry of Education, Key Laboratory for Matter Microstructure and Function of Hunan Province, Department of Physics and Synergetic Innovation Center for Quantum Effects and Applications, Hunan Normal University, Changsha 410081, China}
\author{Jie-Qiao Liao}
\email{Corresponding author: jqliao@hunnu.edu.cn}
\affiliation{Key Laboratory of Low-Dimensional Quantum Structures and Quantum Control of Ministry of Education, Key Laboratory for Matter Microstructure and Function of Hunan Province, Department of Physics and Synergetic Innovation Center for Quantum Effects and Applications, Hunan Normal University, Changsha 410081, China}

\begin{abstract}
We study the generation of quantum entanglement between two giant atoms coupled to a common one-dimensional waveguide. Here each giant atom interacts with the waveguide at two separate coupling points. Within the Wigner-Weisskopf framework for single coupling points, we obtain the time-delayed quantum master equations governing the evolution of the two giant atoms for three different coupling configurations: separated, braided, and nested couplings. For each coupling configuration, we consider both the Markovian and non-Markovian entanglement dynamics of the giant atoms, which are initially in two different separable states: single- and double-excitation states. Our results show that the generated entanglement depends on the phase shift, time delay, atomic initial state, and the coupling configuration. For the single-excitation initial state, there exists the steady-state entanglement for each coupling in both the Markovian and non-Markovian regimes due to the appearance of the dark state. For the double-excitation initial state, we observe entanglement sudden birth via adjusting the phase shift in both regimes. In particular, the maximally achievable entanglement for the nested coupling is about one order of magnitude larger than those of separate and braided couplings. We also find that the maximal entanglement for these three coupling configurations can be enhanced in the case of small time delays. This work can be utilized for the generation and control of entanglement in quantum networks based on giant-atom waveguide-QED systems, which have wide potential applications in quantum information processing.
\end{abstract}

\date{\today}
\maketitle
\section{Introduction}
Quantum entanglement is the key resource for various quantum information applications~\cite{Einstein35,Horodecki09,Duarte21}, such as quantum key distribution~\cite{Ekert91,Grosshans03}, quantum dense coding~\cite{Bennett92}, quantum teleportation~\cite{Bennett93},  and  quantum-computing technology~\cite{DiVincenzo00}.  The generation of quantum entanglement has been theoretically and experimentally studied in a variety of systems, such as optical systems~\cite{Pan12}, trapped ion systems~\cite{Wineland03,Monroe04}, cavity quantum electrodynamics (QED) systems~\cite{Haroche01,Walther06}, circuit-QED systems~\cite{Wallraff21,Blais04,Schoelkopf04}, and waveguide-QED systems~\cite{Roy17,Gu17,Sheremet21}. In particular, waveguide-QED systems provide an outstanding platform for generating long-distance quantum entanglement, and hence they can be regarded as a promising candidate for quantum information processing~\cite{Kimble08,Chang06,Fan07,Law10,Roy11,Fan12,Zoller16,Chang18,Solano20}.

In traditional waveguide-QED systems, the atoms interact with the waveguide at single points and are commonly assumed as point-like objects, called dipole approximation~\cite{Wall08}. This approximation is valid in quantum optics when the atoms are assumed to be much smaller than the wavelength of the coupled fields. However, the recent experimental and theoretical advances on giant atoms~\cite{Kockum20Rev} indicate that this approximation becomes invalid when considering the coupling of the superconducting qubits either to the surface acoustic waves (SAWs) or microwave waveguides at multiple coupling points. Typically, quantum interference will take place in coupled quantum systems with multiple coupling points. It has been shown that the quantum interference effect will cause some interesting physical phenomena, such as frequency-dependent Lamb shifts and relaxation rates~\cite{Kockum14PRA}, decoherence-free interatomic interaction~\cite{Kockum18,Oliver20,Ciccarello2,Kockum22pra}, unconventional bound states~\cite{Guo20prr,WangX21,Vega21,Wang21,Lim23}, non-Markovian decay dynamics~\cite{Guo17PRA,Delsing19,Longhi20,Du21pra1,Du22A,Yin22A,Lu23}, and single-photon scattering~\cite{Wang20,Du21pra2,Jia21,Liao22}.

Recently, many schemes have been proposed to generate long-range quantum entanglement between distant emitters in traditional waveguide-QED systems~\cite{Vidal11,Baranger13,Porras2013,Ballestero14,Facchi16}. In addition, to increase the maximally achievable entanglement, chiral waveguide setups have also been used to generate two-qubit entanglement~\cite{Zoller14,Moreno15,Schotland16,Mok20}. However, compared to the traditional waveguide-QED systems, quantum interference effects are more abundant and adjustable in giant-atom waveguide-QED systems. Consequently, the generation of entanglement between giant atoms can exhibit new features that cannot appear for small atoms~\cite{Wang21pra,Santos23}. Moreover, the non-Markovian retarded effect should be considered when the propagating time of photons between coupling points is comparable to the atomic lifetime. In this scenario, an interesting question is the study of the joint influence of quantum interference effect and the non-Markovian retarded effect on the entanglement generation.

In this paper we study the generation of quantum entanglement between two giant atoms with three different coupling configurations: the separate, braided, and nested couplings~\cite{Kockum18}. Here, each giant atom couples to a common waveguide at two separate coupling points.  Based on the Wigner-Weisskopf theory~\cite{Scullybook} for single coupling points, we obtain the time-delayed quantum master equation of the two giant atoms for three different couplings. It can be shown that the time-delayed quantum master equation will reduce to the local quantum master equation derived through the SLH formalism in Ref.~\cite{Kockum18} under the Markovian limit (i.e., the case of the zero time delay). Concretely, we focus on the entanglement dynamics of the two giant atoms by considering two different initially separable states. For a certain initial state and finite value of the time delay, we find that the entanglement dynamics between the two giant atoms can exhibit different features due to their different coupling configurations. In the case of the Markovian regime and the initial single-excitation state,  the maximally achievable entanglement of  both the braided and nested couplings can exceed $0.5$. However, for the separate coupling, the maximal entanglement can only reach $0.5$, which is consistent with the small-atom case.  When the giant atoms are initially in the double-excitation state, the maximal entanglement of the nested coupling is about one order of magnitude larger than those of the other two couplings. Moreover, we can observe the sudden birth of entanglement for these three couplings by adjusting the phase shift. In the case of the non-Markovian regime and the single-excitation initial state, the generation of entanglement is delayed when the time delay takes a finite value. This indicates that the non-Markovian retarded effect works. Meanwhile, this non-Markovian effect also reduces the value of the stationary entanglement.  For the double-excitation initial state, we find that the maximal entanglement for these three couplings can be enhanced for a small value of the time delay. As the time delay increases to much larger than the lifetime of the giant atoms, there is no entanglement generation in these three couplings for both the single- and double-excitation initial states.

The rest of this paper is organized as follows. In Sec.~\ref{Physical model and Eqs}, we introduce the physical system for two giant atoms coupled to a common waveguide and present the Hamiltonians. In Sec.~\ref{QMEQ of two GAs}, we derive the time-delayed quantum master equations of the two giant atoms for three different coupling configurations and analyze the mechanism of entanglement generation of the two giant atoms. In Sec.~\ref{EnDys}, we study the entanglement dynamics between the two giant atoms in different phase shifts, time delays, and initial states. Finally, we conclude this paper in Sec.~\ref{conclusion}.


%%%%%%%%%%%%%%%%%%%%%%%%%%%%%
\begin{figure}[tbp]
\center\includegraphics[width=0.48\textwidth]{modelv1.eps}
\caption{Schematic of the coupling configurations for double two-level giant atoms with energy separation $\omega_{0}$ interacting with a common waveguide:  (a) separate,  (b) braided,  and (c) nested couplings. The positions of the coupling points are labelled by $x_{jn}$, with $j=a$, $b$ and $n=1$, $2$ referring to the giant atoms the coupling points, respectively. In all panels, the two giant atoms are initially prepared in two different separable states. The $\theta_{0}=k_{0}d$ is the accumulated phase shift when the single photon passes through the neighboring coupling points of the giant atoms with the waveguide.}
\label{modelv1}
\end{figure}
%%%%%%%%%%%%%%%%%%%%%%%%%%%%%

\section{System and Hamiltonians}\label{Physical model and Eqs}
We consider a two-giant-atom waveguide-QED system, in which each giant atom interacts with a common one-dimensional waveguide through two separate coupling points, as shown in Fig.~\ref{modelv1}. By changing the arrangement of the coupling points, there are three different coupling configurations: separated [Fig.~\ref{modelv1}(a)], braided [Fig.~\ref{modelv1}(b)], and nested [Fig.~\ref{modelv1}(c)] couplings. The locations of these coupling points are labelled by the coordinates $x_{jn}$, with $j = a$, $b$ marking the giant atoms and $n = 1$, $2$ denoting the two coupling points of each atom. Under the rotating-wave approximation (RWA), the Hamiltonian of the system reads ($\hbar=1$)~\cite{Liao22}
\begin{equation}
\hat{H}=\hat{H}_{0}+\hat{H}_{I},
\end{equation}
where
\begin{equation}
\hat{H}_{0}=\omega _{0}\sum_{j=a,b}\hat{\sigma}_{j}^{+}\hat{\sigma}_{j}^{-}+\sum_{k}\omega _{k}\hat{c}_{k}^{\dagger }\hat{c}_{k},
\end{equation}
and
\begin{equation}
\hat{H}_{I}=\sum_{j=a,b}\sum_{n=1,2}\sum_{k}(g_{jn}e^{ikx_{jn}}\hat{c}_{k}\hat{\sigma}_{j}^{+}+\text{H.c.}).
\end{equation}
Here $\omega_{0}$ is the transition frequency between the excited state $|e\rangle_{j}$ and the ground state $|g\rangle_{j}$ of the giant atoms. The operator $\hat{\sigma}_{j}^{+}=|e\rangle_{jj}\langle g|$ ($\hat{\sigma}_{j}^{-}=|g\rangle_{jj}\langle e|$) is the raising (lowering) operator of the giant atom $j$, and $\hat{c}_{k}$ ($\hat{c}_{k}^{\dagger}$) is the annihilation (creation) operator of the propagating photons in the waveguide with wave vector $k$ and frequency $\omega_{k}$. The constant term $g_{jn}$ is the coupling strength related to the coupling points $x_{jn}$. For simplicity, we consider the case where the coupling strengthes at each coupling point are equal to $g$.

%%%%%%%%%%%%%%%%%%%%%%%%%%%%%%%
\begin{table*}
\caption{Time nonlocal terms in Eq.~(\ref{unified MEq}) for three different coupling configurations. The superoperators $\hat{\mathcal{L}}_{\text{ind}}\hat{\varrho}(t-nt_{d})$ with $n=1$, $2$, and $3$ represent the individual non-local evolution of the two giant atoms. The superoperators $\hat{\mathcal{L}}_{\text{coll}}\hat{\varrho}(t-nt_{d})$ with $n=1$, $2$, and $3$ describe the non-local exchanging interaction and collective decay of the two giant atoms. }
\begin{tabular}{|c|c|c|c|}
\hline
\diagbox{Superoperator}{$\textrm{Coupling}$}  & Separated coupling & Braided coupling & Nested coupling\tabularnewline
\hline
$\hat{\mathcal{L}}_{\text{ind}}\hat{\varrho}(t-t_{d})$ & $\begin{array}{c}
\underset{j}{\sum}(-i)\gamma\sin\theta_{0}[\hat{\sigma}_{j}^{+}\hat{\sigma}_{j}^{-},\hat{\varrho}(t-t_{d})]\\
+2\gamma\cos\theta_{0}\hat{\mathcal{D}}[\hat{\sigma}_{j}^{-}]\hat{\varrho}(t-t_{d})
\end{array}$ & $0$ & $\begin{array}{c}
-i\gamma\sin\theta_{0}[\hat{\sigma}_{b}^{+}\hat{\sigma}_{b}^{-},\hat{\varrho}(t-t_{d})]\\
+2\gamma\cos\theta_{0}\hat{\mathcal{D}}[\hat{\sigma}_{b}^{-}]\hat{\varrho}(t-t_{d})
\end{array}$\tabularnewline
\hline
$\hat{\mathcal{L}}_{\text{ind}}\hat{\varrho}(t-2t_{d})$  & $0$ & $\begin{array}{c}
\underset{j}{\sum}(-i)\gamma\sin(2\theta_{0})[\hat{\sigma}_{j}^{+}\hat{\sigma}_{j}^{-},\hat{\varrho}(t-2t_{d})]\\
+2\gamma\cos(2\theta_{0})\hat{\mathcal{D}}[\hat{\sigma}_{j}^{-}]\hat{\varrho}(t-2t_{d})
\end{array}$ & $0$\tabularnewline
\hline
$\hat{\mathcal{L}}_{\text{ind}}\hat{\varrho}(t-3t_{d})$ & $0$ & $0$ & $\begin{array}{c}
-i\gamma\sin(3\theta_{0})[\hat{\sigma}_{a}^{+}\hat{\sigma}_{a}^{-},\hat{\varrho}(t-3t_{d})]\\
+2\gamma\cos(3\theta_{0})\hat{\mathcal{D}}[\hat{\sigma}_{a}^{-}]\hat{\varrho}(t-3t_{d})
\end{array}$\tabularnewline
\hline
$\hat{\mathcal{L}}_{\text{coll}}\hat{\varrho}(t-t_{d})$ & $\begin{array}{c}
\underset{i\neq j}{\sum}\frac{-i\gamma}{2}\sin\theta_{0}[\hat{\sigma}_{i}^{+}\hat{\sigma}_{j}^{-},\hat{\varrho}(t-t_{d})]\\
+\gamma\cos\theta_{0}(\hat{\sigma}_{i}^{-}\hat{\varrho}(t-t_{d})\hat{\sigma}_{j}^{+}\\
-\frac{1}{2}[\hat{\sigma}_{i}^{+}\hat{\sigma}_{j}^{-},\hat{\varrho}(t-t_{d})]_{+})
\end{array}$ & $\begin{array}{c}
\underset{i\neq j}{\sum}\frac{-3i\gamma}{2}\sin\theta_{0}[\hat{\sigma}_{i}^{+}\hat{\sigma}_{j}^{-},\hat{\varrho}(t-t_{d})]\\
+3\gamma\cos\theta_{0}(\hat{\sigma}_{i}^{-}\hat{\varrho}(t-t_{d})\hat{\sigma}_{j}^{+}\\
-\frac{1}{2}[\hat{\sigma}_{i}^{+}\hat{\sigma}_{j}^{-},\hat{\varrho}(t-t_{d})]_{+})
\end{array}$ & $\begin{array}{c}
\underset{i\neq j}{\sum}(-i)\gamma\sin\theta_{0}[\hat{\sigma}_{i}^{+}\hat{\sigma}_{j}^{-},\hat{\varrho}(t-t_{d})]\\
+2\gamma\cos\theta_{0}(\hat{\sigma}_{i}^{-}\hat{\varrho}(t-t_{d})\hat{\sigma}_{j}^{+}\\
-\frac{1}{2}[\hat{\sigma}_{i}^{+}\hat{\sigma}_{j}^{-},\hat{\varrho}(t-t_{d})]_{+})
\end{array}$\tabularnewline
\hline
$\hat{\mathcal{L}}_{\text{coll}}\hat{\varrho}(t-2t_{d})$ & $\begin{array}{c}
\underset{i\neq j}{\sum}(-i)\gamma\sin(2\theta_{0})[\hat{\sigma}_{i}^{+}\hat{\sigma}_{j}^{-},\hat{\varrho}(t-2t_{d})]\\
+2\gamma\cos(2\theta_{0})(\hat{\sigma}_{i}^{-}\hat{\varrho}(t-2t_{d})\hat{\sigma}_{j}^{+}\\
-\frac{1}{2}[\hat{\sigma}_{i}^{+}\hat{\sigma}_{j}^{-},\hat{\varrho}(t-2t_{d})]_{+})
\end{array}$ & $0$ & $\begin{array}{c}
\underset{i\neq j}{\sum}(-i)\gamma\sin(2\theta_{0})[\hat{\sigma}_{i}^{+}\hat{\sigma}_{j}^{-},\hat{\varrho}(t-2t_{d})]\\
+2\gamma\cos(2\theta_{0})(\hat{\sigma}_{i}^{-}\hat{\varrho}(t-2t_{d})\hat{\sigma}_{j}^{+}\\
-\frac{1}{2}[\hat{\sigma}_{i}^{+}\hat{\sigma}_{j}^{-},\hat{\varrho}(t-2t_{d})]_{+})
\end{array}$\tabularnewline
\hline
$\hat{\mathcal{L}}_{\text{coll}}\hat{\varrho}(t-3t_{d})$ & $\begin{array}{c}
\underset{i\neq j}{\sum}\frac{-i\gamma}{2}\sin(3\theta_{0})[\hat{\sigma}_{i}^{+}\hat{\sigma}_{j}^{-},\hat{\varrho}(t-3t_{d})]\\
+\gamma\cos(3\theta_{0})(\hat{\sigma}_{i}^{-}\hat{\varrho}(t-3t_{d})\hat{\sigma}_{j}^{+}\\
-\frac{1}{2}[\hat{\sigma}_{i}^{+}\hat{\sigma}_{j}^{-},\hat{\varrho}(t-3t_{d})]_{+})
\end{array}$ & $\begin{array}{c}
\underset{i\neq j}{\sum}\frac{-i\gamma}{2}\sin(3\theta_{0})[\hat{\sigma}_{i}^{+}\hat{\sigma}_{j}^{-},\hat{\varrho}(t-3t_{d})]\\
+\gamma\cos(3\theta_{0})(\hat{\sigma}_{i}^{-}\hat{\varrho}(t-3t_{d})\hat{\sigma}_{j}^{+}\\
-\frac{1}{2}[\hat{\sigma}_{i}^{+}\hat{\sigma}_{j}^{-},\hat{\varrho}(t-3t_{d})]_{+})
\end{array}$ & $0$\tabularnewline
\hline
\end{tabular}
\label{Table1}
\end{table*}
%%%%%%%%%%%%%%%%%%%%%%%%%%%%%%%

In the interaction picture with respect to $H_{0}$, the atom-waveguide coupling Hamiltonian becomes
\begin{equation}
\hat{V}_{I}(t) =g\sum_{j=a,b}\sum_{n=1,2}\left[ \hat{B}(x_{jn},t) \hat{\sigma}_{j}^{+}e^{i\omega _{0}t}+\hat{B}^{\dagger}(x_{jn},t) \hat{\sigma}_{j}^{-}e^{-i\omega _{0}t}\right],
\end{equation}
with $\hat{B}( x_{jn},t) =\sum_{k}e^{ikx_{jn}}e^{-i\omega_{k}t}\hat{c}_{k}$ being the operator associated with the fields.

\section{Quantum master equations of the two giant atoms}\label{QMEQ of two GAs}
To study quantum entanglement of the two giant atoms, we treat the fields in the waveguide as the environment of the atoms and derive quantum master equation to govern the evolution of the two atoms. The formal master equation of the system in the interaction picture reads~\cite{Breuer02}
\begin{equation}
\dot{\hat{\rho}}_{I}(t) =-\int_{0}^{t}ds\text{Tr}_{w}\left \{\left[ \hat{V}_{I}(t) ,\left[ \hat{V}_{I}(s) ,\hat{\rho}_{w}\otimes \hat{\rho}_{I}(s) \right] \right] \right \},
\end{equation}
where $\hat{\rho}_{I}(t)$ is the density matrix of the two atoms, $\hat{\rho}_{w}$ is the density matrix of the fields in the waveguide, and $\text{Tr}_{w}\{\bullet\}$ denotes taking trace over these fields. We consider the case where all the field modes in the  waveguide are initially in the vacuum state $\hat{\rho}_{w}=|\emptyset\rangle\langle\emptyset|$ with $|\emptyset\rangle$ representing the empty states. Then, we have
\begin{equation}
\text{Tr}_{w}[ \hat{B}^{\dagger }( x_{jn},t) \hat{B}(x_{jn},s) \hat{\rho}_{w}] =0.
\end{equation}

Using the Wigner-Weisskopf approximation at each single coupling point and assuming $\omega_{k}\approx\omega_{0}+(k-k_{0})\upsilon_{g}$, with $k_{0}$ ($\upsilon_{g}$) being the wave vector (group velocity) of the field at frequency $\omega_{0}$~\cite{Fan05,Fan09}, the dynamics of the two giant atoms in these three different coupling configurations are governed by the following unified time-delayed quantum master equation
\begin{equation}
\label{unified MEq}
\dot{\hat{\rho}}(t)=\hat{\mathcal{L}}_{\text{loc}}\hat{\rho}(t)+\sum_{n}^{3}(\hat{\mathcal{L}}_{\text{ind}}+\hat{\mathcal{L}}_{\text{coll}})\hat{\varrho}(t-nt_{d}).
\end{equation}
Hereafter, we drop the superscript $``I"$ and always refer to the master equation in the interaction picture. In addition, we introduce the definition
\begin{equation}
\hat{\varrho}(t-nt_{d}) \equiv \hat{\rho}(t-nt_{d})\Theta(t-nt_{d}),
\end{equation}
where $\Theta(t)$ is the Heaviside step function. The superoperator
\begin{equation}
\label{locdissipation}
\hat{\mathcal{L}}_{\text{loc}}\hat{\rho}(t) =2\gamma \hat{\mathcal{D}}[ \hat{\sigma}_{a}^{-}] \hat{\rho}(t)+2\gamma \hat{\mathcal{D}}[ \hat{\sigma}_{b}^{-}] \hat{\rho}(t)
\end{equation}
describes the local dissipation of the giant atoms $a$ and $b$ with the damping rate $\gamma =4\pi g^{2}/\upsilon _{g}$, and
\begin{equation}
\hat{\mathcal{D}}[\hat{o}]\hat{\rho}(t)=\hat{o}\hat{\rho}(t)\hat{o}^{\dagger }-(\hat{o}^{\dagger }\hat{o}\hat{\rho}(t)+\hat{\rho}(t)\hat{o}^{\dagger }\hat{o})/2
\end{equation}
is the standard Lindblad superoperator. The superoperators $\hat{\mathcal{L}}_{\text{ind}}\hat{\varrho}(t-nt_{d})$ with $n=1$, $2$, and $3$ describe the individual non-local evolution of the giant atoms. Here, the time delay $t_{d}=d/\upsilon_{g}$ is introduced, where we assume that the distances between neighboring coupling points are equal to $d$. The superoperators $\hat{\mathcal{L}}_{\text{coll}}\hat{\varrho}(t-nt_{d})$ with $n=1$, $2$, and $3$ describe the non-local exchanging interaction and collective decay of the giant atoms. The specific expressions of  $\hat{\mathcal{L}}_{\text{ind}}\hat{\varrho}(t-nt_{d})$ and $\hat{\mathcal{L}}_{\text{coll}}\hat{\varrho}(t-nt_{d})$ for these three couplings in Fig.~\ref{modelv1} are summarized in Table~\ref{Table1}. Note that the time-delayed quantum master equation for a single giant atom has been derived in Ref.~\cite{Zhu22}.

It is interesting to point out that under the Markovian approximation $nt_{d}\rightarrow0$, the density operator $\hat{\varrho}(t-nt_{d})$ is replaced by $ \hat{\rho}(t)$. In this case, the nonlocal Markovian quantum master equation~(\ref{unified MEq})  is reduced to the following local master equation
\begin{eqnarray}
\label{Meq td0}
\dot{\hat{\rho}}(t) &=&-i[\hat{H}^{\prime },\hat{\rho}(t)]+\sum_{j=a,b}\Gamma _{j}\hat{\mathcal{D}}[\hat{\sigma}_{j}^{-}]\hat{\rho}(t)  \notag \\
&&+\sum_{i\neq j}\Gamma _{\text{coll}}\left( \hat{\sigma}_{i}^{-}\hat{\rho}(t)\hat{\sigma}_{j}^{+}-\frac{1}{2}[\hat{\sigma}_{i}^{+}\hat{\sigma}_{j}^{-},\hat{\rho}(t)]_{+}\right).
\end{eqnarray}
Here the Hamiltonian takes the following form
\begin{equation}
\label{DD interaction}
\hat{H}^{\prime }=\delta \omega _{a}\hat{\sigma}_{a}^{+}\hat{\sigma}_{a}^{-}+\delta \omega _{b}\hat{\sigma}_{b}^{+}\hat{\sigma}_{b}^{-}+g_{ab}(\hat{\sigma}_{a}^{+}\hat{\sigma}_{b}^{-}+\text{H.c.}),
\end{equation}
where $\delta\omega _{a}$ and $\delta\omega _{b}$ are the Lamb shifts of the two giant atoms and $g_{ab}$ is the exchanging interaction strength. The parameters $\Gamma_{j=a,b}$ and $\Gamma_{\text{coll}}$ in Eq.~(\ref{Meq td0}) are the individual and collective decay rates of the giant atoms, respectively. We note that Eq.~(\ref{Meq td0}) is consistent with the quantum master equation derived by the SLH formalism in Ref.~\cite{Kockum18} after returning back to the Schr\"{o}dinger picture.  In addition, we neglect the nonradiative decay ratio $\gamma_{\text{nr}}$ and pure dephasing $\gamma_{\phi}$ of the giant atoms in Eq.~(\ref{unified MEq}), because the decay rates $\gamma_{\text{nr}}$ and $\gamma_{\phi}$ are much smaller than the coupling rate $\gamma$ in realistic physical systems. For example, the quantity $\gamma^{\prime}=\gamma_{\text{nr}}+\gamma_{\phi}$ is generally at least ten times smaller than the decay rate $\gamma$ in superconducting qubit systems~\cite{Delsing13,Kirchmair22}.

Before studying the entanglement generation for a general case in the two-atom space, we first consider the entanglement generation in the Markovian limit $\gamma t_{d}\rightarrow 0$ when the two atoms are initially in the single-excitation state and two-excitation state. When we consider the Markovian limit and assume the system dynamics is restricted into the single-excitation subspace, the jump terms in the local quantum master equation~(\ref{Meq td0}) can be neglected to obtain the non-Hermitian effective Hamiltonian
\begin{eqnarray}
\hat{H}_{\text{eff}} &=&\sum_{j=a,b}\delta \omega _{j}\hat{\sigma}_{j}^{+}\hat{\sigma}_{j}^{-}+\sum_{i\neq j}g\hat{\sigma}_{i}^{+}\hat{\sigma}_{j}^{-}\nonumber \\
&&-\frac{i}{2}\sum_{j=a,b}\Gamma _{j}\hat{\sigma}_{j}^{+}\hat{\sigma}_{j}^{-}-\frac{i}{2}\sum_{i\neq j}\Gamma _{\text{coll}}\hat{\sigma}_{i}^{+}\hat{\sigma}_{j}^{-}.
\end{eqnarray}
In this case, the density matrix can be expressed as $\hat{\rho}(t)=|\psi(t)\rangle\langle\psi(t)|$, and $|\psi(t)\rangle$ is governed by the following Schr\"{o}dinger equation
\begin{equation}
\label{Seq}
i\frac{\partial|\psi (t) \rangle}{\partial t}=\hat{H}_{\text{eff}}| \psi (t) \rangle.
\end{equation}
According to Eq.~(\ref{Seq}), it is straightforward to analytically solve the system dynamics of the two giant atoms by assuming their general state in the single-excitation subspace as
\begin{equation}
\label{Tstate}
\left \vert \psi(t)\right \rangle =c_{eg}(t)\left \vert e\right \rangle _{a}\left \vert g\right \rangle _{b}+c_{ge}(t) \left \vert g\right \rangle _{a}\left \vert e\right \rangle _{b},
\end{equation}
where $c_{eg}(t)$ and $c_{ge}(t)$ are the probability amplitudes. Using the Laplace transform and its inverse, we can obtain the analytical expressions of  $c_{eg}(t)$ and $c_{ge}(t)$ under the corresponding initial conditions.

%%%%%%%%%%%%%%%%%%%%%%%%%%%%%
\begin{figure}[tbp]
\center\includegraphics[width=0.48\textwidth]{energylevel.eps}
\caption{Scheme of the levels and decays for the collective states of the two giant atoms.}
\label{energylevel}
\end{figure}
%%%%%%%%%%%%%%%%%%%%%%%%%%%%%

To study the entanglement generation in containing the two-excitation components, we work in the collective state representation, where the two giant-atom system behaves as a single four-level system with states $|\psi_{2}\rangle=|e\rangle_{a}|e\rangle_{b}$, $|\psi_{0}\rangle=|g\rangle_{a}|g\rangle_{b}$, and $|\psi_{\pm}\rangle$. According to the eigen-equation $\hat{H}^{\prime }|\psi_{\pm }\rangle=E_{\pm }|\psi _{\pm }\rangle$, the expressions of the collective states $|\psi_{\pm}\rangle$ are given by
\begin{equation}
\label{coll states}
|\psi_{\pm }\rangle =N_{\pm }\left( \frac{\delta \omega_{a}-\delta \omega _{b}\pm \Omega }{g_{ab}}|e\rangle_{a}|g \rangle _{b}+2|g\rangle _{a}|e \rangle _{b}\right),
\end{equation}
with the corresponding eigenvalues
\begin{equation}
E_{\pm }=\frac{1}{2}(\delta\omega _{a}+\delta\omega _{b}\pm\Omega),
\end{equation}
where $\Omega =\sqrt{4g_{ab}^{2}+(\delta\omega_{a}-\delta\omega_{b})^{2}}$ is the level shift induced by the exchanging interaction and the difference of the Lamb shifts of the two giant atoms. The normalization constants $N_{\pm }$ in Eq.~(\ref{coll states}) are defined by
\begin{equation}
N_{\pm }=\left[ 4+\frac{1}{g_{ab}^{2}}(\delta \omega _{a}-\delta \omega_{b}\pm \Omega )^{2}\right] ^{-1/2}.
\end{equation}
Figure~\ref{energylevel} shows the energy-level diagram of the double two-level giant atoms, including the levels and transition rates between different levels. To obtain the transition rates between these levels, we use the basis $\{|\psi_{2}\rangle ,|\psi_{+}\rangle,|\psi_{-}\rangle,|\psi_{0}\rangle \}$ to obtain the evolution of the diagonal elements of the quantum master equation~(\ref{Meq td0}) as
\begin{eqnarray}
\label{Eqsforfourlevel}
\dot{\rho}_{22}(t)&=&-(\Gamma _{a}+\Gamma _{b})\rho _{22}(t),  \nonumber \\
\dot{\rho}_{++}(t) &=&\Gamma _{2+}\rho _{22}(t)+\Gamma _{++}\rho _{++}(t)+\Gamma _{+-}\rho _{+-}(t)+\Gamma _{-+}\rho _{-+}(t) ,  \nonumber \\
\dot{\rho}_{--}(t)&=&\Gamma _{2-}\rho _{22}(t)+\Gamma _{--}\rho _{--}(t)+\Gamma _{+-}\rho _{+-}(t) +\Gamma _{-+}\rho _{-+}(t),  \nonumber \\
\dot{\rho}_{00}(t)&=&\Gamma _{+0}\rho _{++}(t)+\Gamma _{+-}\rho _{+-}(t) +\Gamma _{-+}\rho _{-+}(t)+\Gamma _{-0}\rho _{--}(t).  \nonumber \\
&&
\end{eqnarray}

According to Eqs.~(\ref{Meq td0}) and~(\ref{Eqsforfourlevel}), the transition rates are given by
\begin{eqnarray}
\label{D_rates}
\Gamma _{2+} &=&\frac{\Gamma _{b}\alpha _{+}-\Gamma _{a}\alpha _{-}+4g_{ab}\Gamma_{\text{coll}}}{2\Omega },  \nonumber \\
\Gamma _{2-} &=&\frac{\Gamma _{a}\alpha _{+}-\Gamma _{b}\alpha _{-}-4g_{ab}\Gamma_{\text{coll}}}{2\Omega },  \nonumber \\
\Gamma _{+0} &=&-\Gamma _{++}=\frac{\Gamma _{a}\alpha _{+}-\Gamma _{b}\alpha_{-}+4g_{ab}\Gamma _{\text{coll}}}{2\Omega },  \nonumber \\
\Gamma _{-0} &=&-\Gamma _{--}=\frac{\Gamma _{b}\alpha _{+}-\Gamma _{a}\alpha_{-}-4g_{ab}\Gamma _{\text{coll}}}{2\Omega },  \nonumber \\
\Gamma _{+-} &=&\Gamma _{-+}=\frac{\left( \Gamma _{a}-\Gamma _{b}\right)\sqrt{-\alpha _{+}\alpha _{-}}+2g_{ab}\Gamma _{\text{coll}}\left( \sqrt{\frac{-\alpha _{-}}{\alpha _{+}}}-\sqrt{\frac{-\alpha _{+}}{\alpha _{-}}}\right) }{4\Omega }, \nonumber \\
&&
\end{eqnarray}
with $\alpha _{\pm }=( \delta \omega _{a}-\delta \omega _{b}) \pm\Omega$. Equation~(\ref{D_rates}) indicates that the transition rates between the collective states of the two giant atoms depend on the parameters $g$, $\delta\omega_{j}$, $\Gamma_{j}$, and $\Gamma_{\text{coll}}$, which can be adjusted by tuning the phase shift $\theta_{0}$ or designing different coupling configurations. It is straightforward to prove that $\Gamma_{+-}=\Gamma_{-+}=0$ for the separate and braided couplings, and hence there are no $\rho_{+-}(t)$ and $\rho_{-+}(t)$ terms in Eq.~(\ref{D_rates}). After obtaining these transition rates, the entanglement generation will be clarified based on Fig.~\ref{energylevel}. In addition, we would like to point out that the system evolves in the absence of the external pumping. To generate the long-lived maximally entangled states, one may drive the double-giant-atom waveguide-QED system with external fields~\cite{Santos23}. Below, we will study the entanglement dynamics between the two giant atoms for three different coupling configurations in both the Markovian and non-Markovian regimes, in which the time delay is neglected and considered, respectively.

\section{Entanglement dynamics between two giant atoms}\label{EnDys}
In this section, we study the entanglement generation between the two giant atoms for three different coupling configurations shown in Fig.~\ref{modelv1}. To determine the entanglement dynamics of the giant atoms, we need to solve the quantum master equations of the reduced density operator $\hat{\rho}$ describing the two giant atoms. The entanglement of the double two-level giant atoms can be quantified by the concurrence~\cite{Wootters98}, which is defined as
\begin{equation}
C(t) =\max (0,\sqrt{\lambda _{1}}-\sqrt{\lambda _{2}}-\sqrt{\lambda _{3}}-\sqrt{\lambda _{4}}),
\end{equation}
where $\lambda_{i}$ are the eigenvalues (in descending order) of the spin-flipped density matrix $\tilde{\rho}=\hat{\rho}(\hat{\sigma}_{y}\otimes \hat{\sigma}_{y})\hat{\rho}^{\ast }(\hat{\sigma}_{y}\otimes \hat{\sigma}_{y})$, with $\hat{\sigma}_{y}$ being the Pauli spin-flip operator. Note that $C=1$ and $C = 0$ correspond to a maximally entangled state and a separable state, respectively. The time-delayed quantum master equation~(\ref{unified MEq}) can be numerically solved under given initial conditions. For each coupling configuration, we will consider that the two giant atoms are initially in the single-excitation state $|\psi(0)\rangle=|e\rangle _{a}|g\rangle _{b}$ and the double-excitation state $|\psi(0)\rangle=|e\rangle _{a}|e\rangle _{b}$, respectively.

%%%%%%%%%%%%%%%%%%%%%%%%%%%%%
\begin{figure}[tbp]
\center\includegraphics[width=0.48\textwidth]{CS.eps}
\caption{Concurrences $C^{(S)}_{eg}$ and $C^{(S)}_{ee}$  as functions of the scaled evolution time $\gamma t$ and the scaled phase shift $\theta_{0}/\pi$ at given values of $\gamma t_{d}$. In the the left and right columns, the giant atoms are initially in the states $|\psi(0)\rangle=|e\rangle_{a}|g\rangle_{b}$ and $|e\rangle_{a}|e\rangle_{b}$, respectively. In panels (a,b), (c,d), (e,f), and (g,h), we take the time delay $\gamma t_{d}=0$, $0.1$, $1$, and $\infty$, respectively.}
\label{CS}
\end{figure}
%%%%%%%%%%%%%%%%%%%%%%%%%%%%%



\subsection{Entanglement generation between two separate giant atoms}
We begin by considering the two separated giant atoms depicted in Fig.~\ref{modelv1}(a). In this case, the time-delayed quantum master equation is given by
\begin{eqnarray}
\label{TDMEq_Scase}
\dot{\hat{\rho}}(t)  &=&\hat{\mathcal{L}}_{\text{loc}}\hat{\rho}(t) +\hat{\mathcal{L}}_{\text{ind}}\hat{\varrho}(t-t_{d}) +\hat{\mathcal{L}}_{\text{coll}}\hat{\varrho}(t-t_{d})   \notag \\
&&+\hat{\mathcal{L}}_{\text{coll}}\hat{\varrho}( t-2t_{d}) +\hat{\mathcal{L}}_{\text{coll}}\hat{\varrho}(t-3t_{d}),
\end{eqnarray}
where the local dissipation operator $\hat{\mathcal{L}}_{\text{loc}}\hat{\rho}(t)$ is given by Eq.~(\ref{locdissipation}). The superoperator $\hat{\mathcal{L}}_{\text{ind}}\hat{\varrho}(t-t_{d}) $ in Eq.~(\ref{TDMEq_Scase}) represents the non-local time evolution of the two separate giant atoms, with the frequency shift $\gamma\sin\theta_{0}$ and the damping rate $2\gamma\cos\theta_{0}$. The superoperator $\hat{\mathcal{L}}_{\text{coll}}\hat{\varrho}(t-nt_{d})$ describes the non-local exchanging interaction and collective decay of the two giant atoms. For example, when $n=1$, the exchanging interaction strength is $\gamma\sin(\theta_{0})/2$ and the damping rate is  $\gamma\cos\theta_{0}$. In the Markovian limit $\gamma t_{d}\rightarrow0$, the non-local superoperators in Eq.~(\ref{TDMEq_Scase}) becomes local. Then for the two separate giant atoms, we can obtain the Lamb shifts $\delta\omega_{a}=\delta\omega_{b}=\gamma\sin\theta_{0}$, the exchanging coupling strength $g_{ab}=\gamma[\sin\theta_{0}+2\sin(2\theta_{0})+\sin(3\theta_{0})]/2$, the individual decay rates $\Gamma_{a}=\Gamma_{b}=2\gamma(1+\cos\theta_{0})$, and the collective decay rate $\Gamma_{\text{coll}}=\gamma[\cos\theta_{0}+2\cos(2\theta_{0})+\cos(3\theta_{0})]$. To see the entanglement generation, we consider that the two separate giant atoms are initially in two different separable states.

In Fig.~\ref{CS}, we show the time evolution of the concurrences $C^{(S)}_{eg}$ and $C^{(S)}_{ee}$ as functions of the dimensionless quantities $\gamma t$ and $\theta_{0}/\pi$ at various values of the time delay $\gamma t_{d}$. Note that the superscript ``$S$" denotes the separate-coupling case and the subscript ``$eg$" (``$ee$") corresponds the atomic initial state $|\psi(0)\rangle=|e\rangle_{a}|g\rangle_{b}$ ($|e\rangle_{a}|e\rangle_{b}$). The left and right columns in Fig.~\ref{CS} represent the cases of single- and double-excitation initial states, respectively.  Figures~\ref{CS}(a)$-$\ref{CS}(f) show that, for a finite value of $\gamma t_{d}$, both $C^{(S)}_{eg}$ and $C^{(S)}_{ee}$ are modulated by the phase shift $\theta_{0}$. Meanwhile, the dependence of $C^{(S)}_{eg}$ and $C^{(S)}_{ee}$ on $\theta_{0}$ is a $2\pi$-periodic function. For a phase shift $\theta_{0}\in[0,\pi]$, both $C^{(S)}_{eg}$ and $C^{(S)}_{ee}$  satisfy the relation $C_{eg(ee)}^{(S)}(t,\theta_{0})=C_{eg(ee)}^{(S)}(t,2\pi-\theta_{0})$. From Fig.~\ref{CS}(a) we can see that when $|\psi(0)\rangle=|e\rangle_{a}|g\rangle_{b}$ and $\gamma t_{d}=0$, the $C^{(S)}_{eg}$ is zero for $t=0$ since the initial state is separable, and then it increases gradually, except for some special phases $\theta_{0}=(2n+1)\pi$, with an integer $n$. This is because when $\theta_{0}=(2n+1)\pi$, the exchanging interaction strength $g_{ab}$, individual decay rate $\Gamma_{a}$ ($\Gamma_{b}$), and collective decay rate $\Gamma_{\text{coll}}$ become zero. Then the two separate giant atoms are decoupled from the waveguide. Hence there is no entanglement generation between the two giant atoms. When $\theta_{0}=(n+1/2)\pi$ and $2n\pi$, we find that $C^{(S)}_{eg}$ tends asymptotically to a steady-state value $0.5$ in the long-time limit. In this case, the generated entanglement does not decay since the dark state appears. We notice that the exchanging interaction strength  is zero but the individual and collective decay rates are non-zero at $\theta_{0}=(n+1/2)\pi$ and $2n\pi$. For other phase shifts, such as $\theta_{0}=\pi/4$ and $3\pi/4$, it can be seen that the $C^{(S)}_{eg}$ decreases to zero after reaching its maximal value, as shown in the valleys in Fig.~\ref{CS}(a).

According to Eqs.~(\ref{Seq}) and~(\ref{Tstate}), the concurrence $C^{(S)}_{eg}$ in the case of the single-excitation initial state and the Markovian limit can be analytically obtained as
\begin{equation}
\label{CS_eg}
C^{(S)}_{eg}(t)=e^{-2(1+\cos \theta _{0})\gamma t}\left \vert \sinh \left[4e^{2i\theta _{0}}\cos(\theta_{0}/2)^{2}\gamma t\right] \right \vert.
\end{equation}
By substituting $\theta_{0}=\pi/2$ and $2\pi$ into Eq.~(\ref{CS_eg}), the concurrence becomes $C^{(S)}_{eg}(t)=(1-e^{-4\gamma t})/2$ and $C^{(S)}_{eg}(t)=(1-e^{-8\gamma t})/2$, respectively. For the two values of $\theta_{0}$,  it is straightforward to find that $C^{(S)}_{eg}(t)$ approaches a steady-state value $0.5$ at the rates $4\gamma$ and $8\gamma$, respectively. It can be proved that, for the separate coupling, the individual decays and Lamb shifts satisfy the relations $\Gamma_{a}=\Gamma_{b}$ and $\delta\omega_{a}=\delta\omega_{b}$. In this case, the transition rates $\Gamma_{+-}=\Gamma_{+-}=0$ and the states $|\psi_{\pm}\rangle$ are reduced to the symmetric and antisymmetric states $|\pm\rangle=(|e\rangle_{a}|g\rangle_{b}+|g\rangle_{a}|e\rangle_{b})/\sqrt{2}$. In particular, by substituting $\theta_{0}=2\pi+|\epsilon|$ $(\theta_{0}=\pi/2+|\epsilon|)$ with $|\epsilon|\ll1$ into Eq.~(\ref{D_rates}), we obtain the transition rates $\Gamma_{+0}\approx8\gamma$ and $\Gamma_{-0}\approx0$ $(\Gamma_{+0}\approx0$ and $\Gamma_{+0}\approx4\gamma)$. This means that, when $\theta_{0}=2\pi+|\epsilon|$ $(\theta_{0}=\pi/2+|\epsilon|)$, the state $|+\rangle$ $(|-\rangle)$ becomes a dark state, which is completely decoupled from the waveguide. As a result, the concurrence $C^{(S)}_{eg}(t)$ does not decay and it reaches a stationary value $C^{(S)}_{eg}(t\rightarrow\infty)=0.5$. Note that here we let $\theta_{0}$ slightly deviate from $2\pi$ and $\pi/2$ for ensuring $\Omega\neq0$ in Eq.~(\ref{D_rates}).

For the two separate giant atoms initially in the state $|\psi(0)\rangle=|e\rangle_{a}|e\rangle_{b}$, it can be seen from Fig.~\ref{CS}(b) that the entanglement dynamics exhibits some features  different from Fig.~\ref{CS}(a). In this initial state, as shown by the decay process in Fig.~\ref{energylevel}, the two giant atoms first evolve into a mixture of two maximally entangled states $|+\rangle$ and $|-\rangle$ [see Eq.~(\ref{coll states})], and eventually decay to the ground state $|\psi_{0}\rangle=|g\rangle_{a}|g\rangle_{b}$. In general, the mixing of the states $|+\rangle$ and $|-\rangle$ is not an entangled state. From Fig.~\ref{CS}(a), we see that there is no entanglement generation for all values of $\theta_{0}$ at the initial finite time.  The concurrence $C^{(S)}_{ee}$ is created at later times for some phase shifts, due to the asymmetry between the two cascades shown in Fig.~\ref{energylevel}. This phenomenon is known as the entanglement sudden (delayed) birth~\cite{Ficek08,Retamal08,Garraway09}. For this coupling configuration and initial state, the maximal generated entanglement can only reach a small value $C^{(S)}_{ee}\approx0.03$.

When the time delay $\gamma t_{d}$ is taken into account, the time evolution of the concurrences $C^{(S)}_{eg}$ and $C^{(S)}_{ee}$  can also be obtained by numerically solving the time-delayed quantum master equation~(\ref{TDMEq_Scase}). As shown in Figs.~\ref{CS}(c)$-$\ref{CS}(h), $C^{(S)}_{eg}$ and $C^{(S)}_{ee}$ are characterized by different features by adjusting $\gamma t_{d}$ from a small value (i.e., $\gamma t_{d}=0.1$) to a large value (i.e., $\gamma t_{d}=1$). In Figs.~\ref{CS}(c) and~\ref{CS}(d), we take $\gamma t_{d}=0.1$, which corresponds the case where the propagation time $t_{d}$ of photons between the neighboring coupling points is less than the lifetime $1/\gamma$ of each giant atom. As can be seen from Fig.~\ref{CS}(c), when the non-Markovian retarded effect exists, $C^{(S)}_{eg}$ is mainly created around $\theta_{0}=2n\pi$, $(n+1/2)\pi$, and $(2n+1)\pi+|\epsilon|$ with the increase of time. However, the maximally achievable steady-state values of $C^{(S)}_{eg}$ decrease compared with Fig.~\ref{CS}(a). We find that $C^{(S)}_{eg}$ exhibits slight oscillation before achieving its steady-state value at $\theta_{0}=2n\pi$. In addition, the steady-state value of $C^{(S)}_{eg}$ at $\theta_{0}=2n\pi$ is smaller than that at $\theta_{0}=(n+1/2)\pi$ due to different quantum interference effect in these phase shifts, as shown in Fig.~\ref{CS}(c). For the initial state $|\psi(0)\rangle=|e\rangle_{a}|e\rangle_{g}$, the maximal value of $C^{(S)}_{ee}$ in Fig.~\ref{CS}(d) is largely enhanced but decays to zero faster when $\theta_{0}$ is near $2n\pi$.

As the time delay gets a further increase to $\gamma t_{d}=1$, i.e., the propagating time of photons between neighboring coupling points is comparable to the lifetime of the giant atoms. In this case, both the concurrences $C^{(S)}_{eg}$ and $C^{(S)}_{ee}$ exhibit stronger oscillation and more peaks. For the initial state $|\psi(0)\rangle=|e\rangle_{a}|g\rangle_{g}$,  $C^{(S)}_{eg}$ is also created at later times in the presence of the time delay. In particular,  $C^{(S)}_{eg}$ is created even when $\theta_{0}=\pi$, as shown in Fig.~\ref{CS}(e), which is a remarkable symbol of the non-Markovian recovery phenomenon. For the concurrence $C^{(S)}_{ee}$ in Fig.~\ref{CS}(f), it is mainly created around $\theta_{0}=2n\pi$ and its peak value decreases comparing with Fig.~\ref{CS}(d). When we consider the limit case $\gamma t_{d}\rightarrow\infty$, the two giant atoms individually decay. Then $C^{(S)}_{eg}$ and $C^{(S)}_{ee}$ become independent of $\theta_{0}$ and always retain their initial value $C^{(S)}_{eg}=C^{(S)}_{ee}=0$, as shown in Figs.~\ref{CS}(g) and~\ref{CS}(h).


\subsection{Entanglement generation between two braided giant atoms}
We now turn to the case of two braided giant atoms, as shown in Fig.~\ref{modelv1}(b). According to Eq.~(\ref{unified MEq}) and Table~\ref{Table1}, the time-delayed quantum master equation for this coupling configuration is given by
\begin{eqnarray}
\label{TDMEq_B}
\dot{\hat{\rho}}(t) &=&\hat{\mathcal{L}}_{\text{loc}}\hat{\rho}(t)+\hat{\mathcal{L}}_{\text{ind}}\hat{\varrho}(t-2t_{d})+\hat{\mathcal{L}}_{\text{coll}}\hat{\varrho}(t-t_{d})  \notag \\
&&+\hat{\mathcal{L}}_{\text{coll}}\hat{\varrho}(t-3t_{d}),
\end{eqnarray}
where the local dissipation operator $\hat{\mathcal{L}}_{\text{loc}}\hat{\rho}(t)$ for the braided giant atoms is also given by Eq.~(\ref{locdissipation}). The second term on the right-hand side of Eq.~(\ref{TDMEq_B}) represents the non-local time evolution of the braided giant atoms, with the frequency shift $\gamma\sin(2\theta_{0})$ and the damping rate $2\gamma\cos(2\theta_{0})$. The superoperator $\hat{\mathcal{L}}_{\text{coll}}\hat{\varrho}(t-t_{d})$ [$\hat{\mathcal{L}}_{\text{coll}}\hat{\varrho}(t-3t_{d})$] describes the non-local exchanging interaction and collective decay of the braided atoms, with the exchanging interaction strength $3\gamma\sin\theta_{0}/2$ [$\gamma\sin(3\theta_{0})/2$] and the damping rate $3\gamma\cos\theta_{0}$ [$\gamma\cos(3\theta_{0})$]. In the Markovian limit, we can obtain the local quantum master equation for the two braided giant atoms, with the effective exchanging interaction strength $g_{ab}=\gamma[3\sin\theta_{0}+\sin(3\theta_{0})]/2$, the individual decay rates $\Gamma_{a}=\Gamma_{b}=2\gamma[1+\cos(2\theta_{0})]$, and the collective decay rate $\Gamma_{\text{coll}}=\gamma[3\cos\theta_{0}+\cos(3\theta_{0})]$. It has been shown that there exists an interatomic interaction without decoherence at $\theta_{0}=(n+1/2)\pi$~\cite{Kockum18}. Below, we will show that this kind of interaction enables the entanglement dynamics of the two braided giant atoms to exhibit significant difference from those of two other coupling configurations.
%%%%%%%%%%%%%%%%%%%%%%%%%%%%%
\begin{figure}[tbp]
\center\includegraphics[width=0.48\textwidth]{CB.eps}
\caption{Concurrences $C^{(B)}_{eg}$ and $C^{(B)}_{ee}$  as functions of the scaled evolution time $\gamma t$ and the scaled phase shift $\theta_{0}/\pi$ when $\gamma t_{d}$ takes various values. In the the left and right columns, the giant atoms are initially in the states $|\psi(0)\rangle=|e\rangle_{a}|g\rangle_{b}$ and $|e\rangle_{a}|e\rangle_{b}$, respectively. In panels (a,b), (c,d), (e,f), and (g,h), we take the time delay $\gamma t_{d}=0$, $0.1$, $1$, and $\infty$, respectively.}
\label{CB}
\end{figure}
%%%%%%%%%%%%%%%%%%%%%%%%%%%%%

Figures~\ref{CB}(a)$-$\ref{CB}(h) show the concurrences $C^{(B)}_{eg}$ and $C^{(B)}_{ee}$ versus the dimensionless quantities $\gamma t$ and $\theta_{0}/\pi$ when the time delay $\gamma t_{d}$ takes different values. In the left and right columns of Fig.~\ref{CB}, the two braided giant atoms are initially in the states $|\psi(0)\rangle=|e\rangle_{a}|g\rangle_{b}$ and $|e\rangle_{a}|e\rangle_{b}$, respectively. We see that $C^{(B)}_{eg}$ and $C^{(B)}_{ee}$ in Figs.~\ref{CB}(a)$-$\ref{CB}(f) are phase dependent with a period of $\pi$, which is different from the case of the separate coupling. When the phase shift is in the region of $\theta_{0}\in[0,\pi/2]$, we have the relation $C_{eg(ee)}^{(B)}(t,\theta_{0})=C_{eg(ee)}^{(B)}(t,\pi-\theta_{0})$. In the case of $\gamma t_{d}=0$ and $|\psi(0)\rangle=|e\rangle_{a}|g\rangle_{b}$, as shown in Fig.~\ref{CB}(a), the concurrence $C^{(B)}_{eg}$ is characterized by an oscillating process when $\theta_{0}$ is near $(n+1/2)\pi$, whereas it approaches a steady-state value $0.5$ at $\theta_{0}=n\pi$ in the long-time limit.  The oscillation of $C^{(B)}_{eg}$ near $\theta_{0}=(n+1/2)\pi$ is caused by the nonzero exchanging interaction. This is because when $\theta_{0}\rightarrow(n+1/2)\pi$, the collective decay rate and the exchanging interaction strength become $\Gamma_{\text{coll}}\rightarrow0$ and $g_{ab}\rightarrow\gamma$, respectively. In this case, the concurrence $C^{(B)}_{eg}$ is mainly characterized by an oscillating process. However, when $\theta_{0}=2n\pi$, we obtain $g_{ab}\rightarrow0$ and $\Gamma_{\text{coll}}\rightarrow4\gamma$, which leads to a non-oscillatory contribution to $C^{(B)}_{eg}$. According to these analyses, it can be seen that the concurrence $C^{(B)}_{eg}$ depends on the two parameters $g_{ab}$ and $\Gamma_{\text{coll}}$ in different ways. The oscillation of $C^{(B)}_{eg}$ is caused by the exchanging coupling $g_{ab}$, whereas the non-oscillatory contribution of $C^{(B)}_{eg}$ comes from the collective decay $\Gamma_{\text{coll}}$~\cite{Vidal11}. When $\theta\neq n\pi/2$, we see that $C^{(B)}$ exhibits a fast increase followed by a very slow decay [for $\theta_{0}\rightarrow(2n+1)\pi$] or an oscillating decay [for $\theta_{0}\rightarrow(n+1/2)\pi$], as shown by the valleys in Fig.~\ref{CB}(a).

For the braided coupling, we can also obtain the analytical expression of $C^{(B)}_{eg}(t)$ in the Markovian limit. In terms of Eqs.~(\ref{Seq}) and~(\ref{Tstate}), we have
\begin{equation}
\label{CB_eg}
C^{(B)}_{eg}(t)=\frac{|\sinh [(3e^{i\theta _{0}}+e^{3i\theta _{0}})\gamma t]|}{e^{4\gamma t\cos^{2}\theta _{0}}}.
\end{equation}
From Eq.~(\ref{CB_eg}), we find that when $\theta_{0}=(n+1/2)\pi$ and $2n\pi$, the concurrence is reduced to $C^{(B)}_{eg}(t)=|\sin(2\gamma t)|$ and $C^{(B)}_{eg}(t)=(1-e^{-8\gamma t})/2$, respectively. Therefore, $C^{(B)}_{eg}(t)$ exhibits a periodic oscillation in the range from zero to one with a period $\pi/(2\gamma)$ when $\theta_{0}=(n+1/2)\pi$. While it tends asymptotically to a steady-state value $0.5$ at a rate $8\gamma$ when $\theta_{0}=(2n+1)\pi$.

In Fig.~\ref{CB}(b), the two braided giant atoms is initially in the state $|\psi(0)\rangle=|e\rangle_{a}|e\rangle_{b}$, which shows that there is no entanglement generation at earlier times even when $\gamma t_{d}=0$, but at some finite times, the entanglement suddenly begins to create for some values of $\theta_{0}$. Similar to the separate coupling, both the frequency shift and individual decay rate of each giant atom for the braided coupling are also equal, i.e., $\delta\omega_{a}=\delta\omega_{b}=\gamma\sin(2\theta_{0})$ and $\Gamma_{a}=\Gamma_{b}=1+\cos(2\theta_{0})$. In this case, the entangled states $|\psi_{\pm}\rangle$ given by Eq.~(\ref{coll states}) also become the symmetric and antisymmetric states $|\pm\rangle$. According to Eq.~(\ref{D_rates}), the states $|\pm\rangle$ decay to the ground state $|\psi_{0}\rangle=|g\rangle_{a}|g\rangle_{b}$ with different rates. In particular, we find that the maximal value of the $C^{(B)}_{ee}$ can achieve about $0.03$ by adjusting the value of $\theta_{0}$, which is consistent with the case of the separate coupling [see Fig.~\ref{CS}(b)]. However, the value of $\theta_{0}$ corresponding to the maximally achievable $C^{(B)}_{ee}$ and $C^{(S)}_{ee}$ is different due to different coupling configurations. We also find that there is no entanglement generation for $\theta_{0}=\pi$ and $\pi/2$. To explain this phenomenon, we substitute $\theta_{0}=\pi+|\epsilon|$ $ (\pi/2+|\epsilon|)$ into Eq.~(\ref{energylevel}) to obtain $\Gamma_{2+}=\Gamma_{+0}=0$ and $\Gamma_{2-}=\Gamma_{-0}=8\gamma$ $(\Gamma_{2+}=\Gamma_{2-}=\Gamma_{+0}=\Gamma_{-0}=0)$. This means that when $\theta_{0}=\pi+|\epsilon|$, the cascade process on the left-side hand in Fig.~\ref{energylevel} is forbidden. For the cascade process on the right-hand side, even though it is allowed, the population of $|\psi_{2}\rangle=|e\rangle_{a}|e\rangle_{b}$ decays to the ground state $|\psi_{0}\rangle$ very fast. Therefore, there is no entanglement generation. When $\theta_{0}=\pi/2+|\epsilon|$, both the left and right cascade processes are forbidden, and hence the population in the state $|\psi_{2}\rangle$ cannot decay into the states $|\pm\rangle$.

Figures~\ref{CB}(c) and~\ref{CB}(d) show the concurrences $C^{(B)}_{eg}$ and $C^{(B)}_{ee}$ versus $\gamma t$ and $\theta_{0}/\pi$ when $\gamma t_{d}$ takes a small value $0.1$.  When the giant atoms are initially in the state $|\psi(0)\rangle=|e\rangle_{a}|g\rangle_{b}$, we find that the peak values of  $C^{(B)}_{eg}$ decrease gradually as time increases even when $\theta_{0}=(n+1/2)\pi$, as shown in Fig.~\ref{CB}(c). This is because the non-Markovian retarded effect starts to work in this case, where the decoherence-free interaction of the two braided giant atoms is partially suppressed. In addition, the steady-state value of $C^{(B)}_{eg}$ at $\theta_{0}=n\pi$ is less than that of the Markovian case, as shown in Fig.~\ref{CB}(a). Different from the case of $\gamma t_{d}=0$ in Fig.~\ref{CB}(b), the concurrence $C^{(B)}_{ee}$ in Fig.~\ref{CB}(d) is only created around $\theta_{0}=n\pi$, and it achieves larger values but decays to zero faster. From Fig.~\ref{CB}(d), we see that $C^{(B)}_{ee}$ can also be created when $\theta_{0}=n\pi$ in the joint influence of the quantum interference effect and the non-Markovian retarded effect.

By increasing the time delay to $\gamma t_{d}=1$,  the concurrence $C^{(B)}_{eg}$ exhibit stronger oscillation and more peaks when $\theta_{0}$ is near to $2n\pi$ and $(n+1/2)\pi$, as shown in Figs.~\ref{CB}(e) and~\ref{CB}(f). As time increases, $C^{(B)}_{eg}$ is characterized by a slow nonexponential oscillating decay process around $\theta_{0}=(n+1/2)\pi$. However, when $\theta_{0}=2n\pi$, $C^{(B)}_{eg}$ approaches a steady-state value more slowly. For the initial state $|\psi(0)\rangle=|e\rangle_{a}|e\rangle_{b}$, there appear more oscillations and peaks in $C^{(B)}_{ee}$ around $\theta_{0}=2n\pi$. With the increase of time, the peak values of $C^{(B)}_{ee}$ decrease gradually. When the time delay gets a larger value, i.e., $\gamma t_{d}\rightarrow\infty$, both $C^{(B)}_{eg}$ and $C^{(B)}_{ee}$ retain their initial values $C^{(B)}_{eg}=C^{(B)}_{ee}=0$ [Figs.~\ref{CB}(g) and~\ref{CB}(h)], which is consistent with the separate-coupling case.

%%%%%%%%%%%%%%%%%%%%%%%%%%%%%
\begin{figure}[tbp]
\center\includegraphics[width=0.48\textwidth]{CN.eps}
\caption{Concurrences $C^{(N)}_{eg}$ and $C^{(N)}_{ee}$  as functions of the scaled evolution time $\gamma t$ and the scaled phase shift $\theta_{0}/\pi$ at given values of $\gamma t_{d}$. In the the left and right columns, the giant atoms are initially in the states $|\psi(0)\rangle=|e\rangle_{a}|g\rangle_{b}$ and $|e\rangle_{a}|e\rangle_{b}$, respectively. In panels (a,b), (c,d), (e,f), and (g,h), we take the time delay $\gamma t_{d}=0$, $0.1$, $1$, and $\infty$, respectively.}
\label{CN}
\end{figure}
%%%%%%%%%%%%%%%%%%%%%%%%%%%%%
\subsection{Entanglement generation between two nested giant atoms}
Finally, we study the entanglement generation for the nested coupling, as shown in Fig.~\ref{modelv1}(c). According to Eq.~(\ref{unified MEq}) and Table~\ref{Table1}, the time-delayed quantum master equation of the two nested giant atoms is given by
\begin{eqnarray}
\label{TDMEq_N}
\dot{\hat{\rho}}(t)  &=&\hat{\mathcal{L}}_{\text{loc}}\hat{\rho}(t) +\hat{\mathcal{L}}_{\text{ind}}\hat{\varrho}(t-t_{d}) +\hat{\mathcal{L}}_{\text{ind}}\hat{\varrho}(t-3t_{d}) \nonumber \\
&&+\hat{\mathcal{L}}_{\text{coll}}\hat{\varrho}(t-t_{d}) +\hat{\mathcal{L}}_{\text{coll}}\hat{\varrho}(t-2t_{d}).
\end{eqnarray}
The first term in Eq.~(\ref{TDMEq_N}) represents the local dissipation operator of the two nested giant atoms [see Eq.~(\ref{locdissipation})]. The superoperator $\hat{\mathcal{L}}_{\text{ind}}\hat{\varrho}(t-t_{d})$ [$\hat{\mathcal{L}}_{\text{ind}}\hat{\varrho}(t-3t_{d})$]  describes the non-local time evolution of giant atom $a$ ($b$), with the frequency shift $\gamma\sin\theta_{0}$ [$\gamma\sin(3\theta_{0})$] and the decay rate $2\gamma\cos\theta_{0}$ [$2\gamma\cos(3\theta_{0})$], respectively. The superoperator $\hat{\mathcal{L}}_{\text{coll}}\hat{\varrho}(t-t_{d})$ [$\hat{\mathcal{L}}_{\text{coll}}\hat{\varrho}(t-2t_{d})$] denotes the non-local exchanging interaction and collective decay of the nested giant atoms, with the exchanging interaction strength $\gamma\sin\theta_{0}$ [$\gamma\sin(2\theta_{0})$] and the decay rate  $2\gamma\cos\theta_{0}$ [$2\gamma\cos(2\theta_{0})$], respectively. In the Markovian limit, the local quantum master equation for the two nested giant atoms is given by Eq.~(\ref{Meq td0}), with the corresponding $g_{ab}=\gamma[\sin\theta_{0}+\sin(2\theta_{0})]$, $\delta\omega_{a}=\gamma\sin(3\theta_{0})$, $\delta\omega_{b}=\gamma\sin\theta_{0}$, $\Gamma_{a}=2\gamma[1+\cos(3\theta_{0})]$, $\Gamma_{b}=2\gamma(1+\cos\theta_{0})$, and $\Gamma_{\text{coll}}=2\gamma[\cos\theta_{0}+\cos(2\theta_{0})]$. Note that the frequency shifts and the individual decay rates of the two nested giant atoms are not equal. Compared with the other two coupling configurations, we will see that this feature allows the nested coupling to create greater entanglement in the case of double-excitation initial state and Markovian limit.

In Figs.~\ref{CN}(a)$-$\ref{CN}(h) we show the entanglement dynamics of the two nested giant atoms when the time delay $\gamma t_{d}$ takes various values. The initial states are taken as $|\psi(0)\rangle=|e\rangle_{a}|g\rangle_{b}$ and $|e\rangle_{a}|e\rangle_{b}$ in the left and right columns of Fig.~\ref{CN}, respectively. Similar to the separate-coupling case, the concurrences  $C^{(N)}_{eg}$ and $C^{(N)}_{ee}$ are also phase dependent with a period $2\pi$ and satisfy the relation  $C_{eg(ee)}^{(N)}(t,\theta_{0})=C_{eg(ee)}^{(N)}(t,2\pi-\theta_{0})$ for $\theta_{0}\in[0,\pi]$. For the initial state $|\psi(0)\rangle=|e\rangle_{a}|g\rangle_{g}$ and $\theta_{0}=2n\pi$,  we find that $C^{(N)}_{eg}$ has a steady-state value for a finite value of $\gamma t_{d}$, as shown in Figs.~\ref{CN}(a), ~\ref{CN}(c), and~\ref{CN}(e). When the phase shift $\theta_{0}\rightarrow\pi$, the concurrence $C^{(N)}_{eg}$ is characterized by a very slow initial increase, which is same as the separate coupling [see Fig.~\ref{CS}(a)].  However, for the nested coupling, the maximal value of the generated entanglement in Fig.~\ref{CN}(a) can exceed $0.5$ at some values of $\theta_{0}$. For example, it can be seen that the concurrence fast achieves a peak value $C^{(N)}_{eg}\approx0.78$ at $\theta_{0}=\pi/3$, followed by a fast decay to zero. To explain this feature, we take $\theta_{0}=\pi/3$ to obtain $g_{ab}=\sqrt{3}\gamma$, $\Gamma_{a}=0$, $\Gamma_{b}=3\gamma$, and $\Gamma_{\text{coll}}=0$. In this case, $C^{(N)}_{eg}$ exhibits a fast increase (caused by the non-zero exchanging interaction strength) followed by a fast decay (due to the non-zero individual decay rate of giant atom $b$).

For the nested coupling configuration, the analytical expression for $C^{(N)}_{eg}(t)$ can be derived from Eqs.~(\ref{Seq}) and~(\ref{Tstate}) as
\begin{equation}
\label{CN_eg}
C_{eg}^{(N)}(t)=\left\vert\frac{F(t,\theta _{0})}{4\cos \left( \frac{\theta _{0}}{2}\right) \sqrt{[(3\cos \theta _{0}-1) ^{2}+4\sin ^{2}\theta_{0}] }}\right\vert,
\end{equation}
where we introduce the function
\begin{eqnarray}
F(t,\theta _{0}) &=&e^{-(A+D)\gamma t}[(1+e^{A\gamma t})(1-e^{B\gamma t})A
\nonumber \\
&&+2ie^{2i\theta _{0}}(1-e^{B\gamma t})(1-e^{B^{\ast }\gamma t})\sin \theta_{0}],
\end{eqnarray}
with
\begin{eqnarray}
\label{N_effecients}
A &=&\sqrt{(5-2e^{i\theta _{0}}+e^{2i\theta _{0}})(e^{i\theta_{0}}+e^{2i\theta _{0}})^{2}},  \notag \\
B &=&\sqrt{8e^{-4i\theta _{0}}\cos ^{2}\left( \frac{\theta _{0}}{2}\right)(3\cos \theta _{0}+2i\sin \theta _{0}-1)},  \notag \\
D &=&2+\cos \theta _{0}+\cos \left( 3\theta _{0}\right) .
\end{eqnarray}
By substituting $\theta_{0}=2n\pi$ into Eq.~(\ref{CN_eg}), we have $C_{eg}^{(N)}(t)=(1-e^{-8\gamma t})/2$, which approaches a steady-state value $0.5$ at a rate $8\gamma$ in the long-time limit. This feature is consistent with the separate-coupling case. When $\theta_{0}=2n\pi$, we find that the quantities for the two couplings all become $\delta\omega_{a}=\delta\omega_{b}=0$, $g_{ab}=0$, and $\Gamma_{a}=\Gamma_{b}=\Gamma_{\text{coll}}=4\gamma$.

Figure~\ref{CN}(b) depicts the time evolution of the concurrence $C^{(N)}_{ee}$ and its dependence on the phase shift when $\gamma t_{d}=0$. It can be seen that there is no entanglement generation at earlier times, and entanglement suddenly starts to create at some finite time for some values of $\theta_{0}$. In particular, we find that the maximal generated entanglement between the two nested giant atoms initially in the state $|\psi(0)\rangle=|e\rangle_{a}|e\rangle_{b}$ can achieve a maximal value $ C^{(N)}_{ee}\approx0.37$, which is about one order of magnitude larger than those of both the separate and braided couplings. In the previous discussions, we have shown that the two giant atoms have equal frequency shifts for the separate and braided couplings.  However, for the nested giant atoms, their frequency shifts are $\delta\omega_{a}=\gamma\sin(3\theta_{0})$ and $\delta\omega_{b}=\gamma\sin\theta_{0}$, respectively, which are not equal except for $\theta_{0}=n\pi$ and $(2n+1)\pi/4$.  For the initial state $|\psi(0)\rangle=|e\rangle_{a}|e\rangle_{b}$, we also find that, at some finite time, $C^{(N)}_{ee}$ is characterized by a fast initial increase followed by a slow decay when $\theta_{0}\rightarrow\pi$. To explain this feature, we substitute $\theta_{0}=\pi+|\epsilon|$ (such as $0.99\pi$) into Eq.~(\ref{D_rates}) to obtain $\Gamma_{2+}\approx0.004$, $\Gamma_{2-}\approx0.006$, $\Gamma_{+0}\approx0.01$, and $\Gamma_{-0}\approx0.00005$, which satisfy the relation $\Gamma_{+0}>\Gamma_{2-}>\Gamma_{2+}\gg\Gamma_{-0}$. Therefore, when we adjust the phase shift to $\theta_{0}\rightarrow \pi$, the decay process from the state $|\psi_{+}\rangle$ to $|\psi_{0}\rangle$  is faster than that from $|\psi_{2}\rangle$ to $|\psi_{+}\rangle$. However, the decay process from $|\psi_{-}\rangle$ and $|\psi_{0}\rangle$ is much less than that from $|\psi_{2}\rangle$ and $|\psi_{-}\rangle$. The asymmetric decay between the two cascade processes leads to the entanglement generation between the two nested giant atoms.

When the time delay $\gamma t_{d}$ is non-zero, the non-Markovian effect will affect the entanglement dynamics of the two nested giant atoms. For a small value $\gamma t_{d}=0.1$, as shown in Fig.~\ref{CN}(c),  the overall shape of the concurrence $C^{(N)}_{eg}$ remains roughly unchanged, but both the maximally achievable values of $C^{(N)}_{eg}$ at $\theta_{0}\neq2n\pi$ and the steady-state values of $C^{(N)}_{eg}$ at $\theta_{0}=2n\pi$ decrease. However, for the initial state $|\psi(0)\rangle=|e\rangle_{a}|e\rangle_{b}$, as shown in Fig.~\ref{CN}(d), we see that the maximally achievable value of $C^{(N)}_{ee}$ is enhanced due to the joint influence of the quantum interference effect and the non-Markovian retardation effect.

By further increasing the time delay to $\gamma t_{d}=1$, the non-Markovian retarded effect becomes stronger. In this case, the concurrence $C^{(N)}_{eg}$ is created even when $\theta_{0}=\pi$, as shown in Fig.~\ref{CN}(e). Meanwhile, there appear new peaks in the valleys of $C^{(N)}_{eg}$. For the initial state $|\psi(0)\rangle=|e\rangle_{a}|e\rangle_{b}$, $C^{(N)}_{eg}$ is created at later times when $\theta_{0}$ is near $\pi$. In addition, the generation time of entanglement is more delayed compared with the Markovian limit in Fig.~\ref{CN}(b). We also see that the maximally achievable value of $C^{(N)}_{ee}$ [Fig.~\ref{CN}(f)] decreases for the large value $\gamma t_{d}=1$. Finally, we also find from Figs.~\ref{CN}(g) and~\ref{CN}(h) that there is no entanglement generation in the limit $\gamma t_{d}\rightarrow\infty$.

Therefore, the entanglement generation of the two giant atoms depends on the atomic initial state, coupling configuration, and phase shift, when the time delay is within an appropriate range. This requires that the propagating time of photons between neighboring coupling points cannot be much larger than the lifetime of the giant atoms.

\section{Conclusion}\label{conclusion}
In conclusion, we have studied the entanglement generation between two giant atoms with three different coupling configurations. Using the Wigner-Weisskopf approach for single coupling points, we have obtained the time-delayed quantum master equation governing the dynamics of the two giant atoms. By taking into account and neglecting the time delay between the coupling points, we have considered the entanglement generation in both the Markovian and non-Markovian regimes, respectively. In particular, we have analyzed the entanglement generation when the two giant atoms are initially in two different separable states. It has been shown that the entanglement generation between the two giant atoms depends on the coupling configurations, phase shift, time delay, and atomic initial state. For a certain initial state and coupling configuration of the two giant atoms, the entanglement dynamics is affected by the joint influence of quantum interference and non-Markovian effect. However, we would like to remark that the time delay within an appropriate range is a significant condition to generate and control the entanglement. This work will pave the way for quantum information processing in the giant-atom waveguide-QED systems including the non-Markovian effect.


\begin{acknowledgments}
J.-Q.L. was supported in part by National Natural Science Foundation of China (Grants No.~12175061, No.~12247105, and No.~11935006) and the Science and Technology Innovation Program of Hunan Province (Grants No.~2021RC4029 and No.~2020RC4047).
\end{acknowledgments}


\begin{thebibliography}{99}
%%QIT applications
\bibitem{Einstein35} A. Einstein, B. Podolsky, and N. Rosen, Can quantum-mechanical description of physical reality be considered complete? Phys. Rev. \textbf{47}, 777 (1935).
\bibitem{Horodecki09} R. Horodecki, P. Horodecki, M. Horodecki, and K. Horodecki, Quantum entanglement, Rev. Mod. Phys. \textbf{81}, 865 (2009).
\bibitem{Duarte21} F. J. Duarte and T. S. Taylor, \emph{Quantum Entanglement Engineering and Applications} (IOP Publishing, London, 2021).
%%Key distribution
\bibitem{Ekert91} A. K. Ekert, Quantum Cryptography Based on Bell's Theorem, Phys. Rev. Lett. \textbf{67}, 661 (1991).
\bibitem{Grosshans03} F. Grosshans, G. Van Assche, J. Wenger, R. Brouri, N. J. Cerf, and P. Grangier, Quantum key distribution using gaussian-modulated coherent states, Nature (London) \textbf{421}, 238 (2003).

%%Quantum dense coding
\bibitem{Bennett92} C. H. Bennett and S. J. Wiesner, Communication via One- and Two-Particle Operators on Einstein-Podolsky-Rosen States, Phys. Rev. Lett. \textbf{69}, 2881 (1992).
%%Quantum teleportation
\bibitem{Bennett93}  C. H. Bennett, G. Brassard, C. Cr\'{e}peau, R. Jozsa, A. Peres, and W. K. Wootters, Teleporting an Unknown Quantum State via Dual Classical and Einstein-Podolsky-Rosen Channels, Phys. Rev. Lett. \textbf{70}, 1895 (1993).

%%Quantum-computing technology
\bibitem{DiVincenzo00} David P. DiVincenzo, \emph{The Physical Implementation of Quantum Computation}, Fortschr. Phys. \textbf{48}, 771 (2000).
%%Various system for the generation of entanglement
%%Optical system
\bibitem{Pan12} J.-W. Pan, Z.-B. Chen, C.-Y. Lu, H. Weinfurter, A. Zeilinger, and M. \.{Z}ukowski,  Multiphoton entanglement and interferometry, Rev. Mod. Phys. \textbf{84}, 777 (2012).

 %%trapped ions
\bibitem{Wineland03} D. Leibfried, R. Blatt, C. Monroe, and D. Wineland, Quantum dynamics of single trapped ions, Rev. Mod. Phys. \textbf{75}, 281 (2003).
\bibitem{Monroe04} B. B. Blinov, D. L. Moehring, L. M. Duan, and C. Monroe, Observation of entanglement between a single trapped atom and a single photon, Nature (London) \textbf{428}, 153 (2004).
%%Cavity QED systems
\bibitem{Haroche01} J. M. Raimond, M. Brune, and S. Haroche, Colloquium: Manipulating quantum entanglement with atoms and photons in a cavity, Rev. Mod. Phys. \textbf{73}, 565 (2001).
\bibitem{Walther06} H. Walther, B. T. H. Varcoe, B. G. Englert, and T. Becker, Cavity quantum electrodynamics, Rep. Prog. Phys. \textbf{69}, 1325 (2006).
%%Circuit QED systems
\bibitem{Wallraff21} A. Blais, A. L. Grimsmo, S. M. Girvin, and A. Wallraff, Circuit quantum electrodynamics, Rev. Mod. Phys. \textbf{93}, 025005 (2021).
\bibitem{Blais04} A. Blais, R.-S. Huang, A. Wallraff, S. M. Girvin, and R. J. Schoelkopf, Cavity quantum electrodynamics for superconducting electrical circuits: An architecture for quantum computation, Phys. Rev. A \textbf{69}, 062320 (2004).
\bibitem{Schoelkopf04} A. Wallraff, D. I. Schuster, A. Blais, L. Frunzio, R.-S. Huang, J. Majer, S. Kumar, S. M. Girvin, and R. J. Schoelkopf, Strong coupling of a single photon to a superconducting qubit using circuit quantum electrodynamics, Nature (London) \textbf{431}, 162 (2004).
%%Waveguide-QED systems
\bibitem{Roy17} D. Roy, C. M. Wilson, and O. Firstenberg, Colloquium: Strongly interacting photons in one-dimensional continuum, Rev. Mod. Phys. \textbf{89}, 021001 (2017).
\bibitem{Gu17} X. Gu, A. F. Kockum, A. Miranowicz, Y.-x. Liu, and F. Nori, Microwave photonics with superconducting quantum circuits, Phys. Rep. \textbf{718}, 1 (2017).
\bibitem{Sheremet21} A. S. Sheremet, M. I. Petrov, I. V. Iorsh, A. V. Poshakinskiy, and A. N. Poddubny, Waveguide quantum electrodynamics: collective radiance and photon-photon correlations,  Rev. Mod. Phys. \textbf{95}, 015002 (2023).
%%Quantum information process
\bibitem{Kimble08} H. J. Kimble, The quantum internet, Nature (London) \textbf{453}, 1023 (2008).
\bibitem{Chang06} D. E. Chang, A. S. S{\o}rensen, P. R. Hemmer, and M. D. Lukin, Quantum Optics with Surface Plasmons, Phys. Rev. Lett. \textbf{97}, 053002 (2006).
\bibitem{Fan07} J.-T. Shen and S. Fan, Strongly Correlated Two-Photon Transport in a One-Dimensional Waveguide Coupled to a Two-Level System, Phys. Rev. Lett. \textbf{98}, 153003 (2007).
\bibitem{Law10} J.-Q. Liao and C. K. Law, Correlated two-photon transport in a one-dimensional waveguide side-coupled to a nonlinear cavity, Phys. Rev. A \textbf{82}, 053836 (2010).
\bibitem{Roy11} D. Roy, Two-Photon Scattering by a Driven Three-Level Emitter in a One-Dimensional Waveguide and Electromagnetically Induced Transparency, Phys. Rev. Lett. \textbf{106}, 053601 (2011).
\bibitem{Fan12}  E. Rephaeli and S. Fan, Stimulated Emission from a Single Excited Atom in a Waveguide, Phys. Rev. Lett. \textbf{108}, 143602(2012).
\bibitem{Zoller16} H. Pichler and P. Zoller, Photonic Circuits with Time Delays and Quantum Feedback, Phys. Rev. Lett. \textbf{116}, 093601 (2016).
\bibitem{Chang18} D. E. Chang, J. S. Douglas, A. Gonz\'{a}lez-Tudela, C.-L. Hung, and H. J. Kimble, Colloquium: Quantum matter built from nanoscopic lattices of atoms and photons, Rev. Mod. Phys. \textbf{90}, 031002 (2018).
\bibitem{Solano20} K. Sinha, P. Meystre, E. A. Goldschmidt, F. K. Fatemi, S. L. Rolston, and P. Solano, Non-Markovian Collective Emission from Macroscopically Separated Emitters, Phys. Rev. Lett. \textbf{124}, 043603 (2020).
%%Dipole approxiamtion
\bibitem{Wall08}D. F. Walls and G. J. Milburn, \emph{Quantum Optics}, 2nd ed. (Springer, Berlin, 2008).

%%Review for giant atoms
\bibitem{Kockum20Rev}   A. F. Kockum, Quantum optics with giant atoms--the first five years, in \emph{Mathematics for Industry} (Springer Singapore, Singapore, 2021), pp. 125--146.


%%Frequency-dependent decay rates
\bibitem{Kockum14PRA} A. F. Kockum, P. Delsing, and G. Johansson, Designing frequency-dependent relaxation rates and Lamb shifts for a giant artificial atom, Phys. Rev. A \textbf{90}, 013837 (2014).

%%Decoherence-free interaction
\bibitem{Kockum18} A. F. Kockum, G. Johansson, and F. Nori, Decoherence-Free Interaction between Giant Atoms in Waveguide Quantum Electrodynamics, Phys. Rev. Lett. \textbf{120}, 140404 (2018).
    \bibitem{Oliver20} B. Kannan, M. J. Ruckriegel, D. L. Campbell, A. F. Kockum, J. Braum\"{u}ller, D. K. Kim, M. Kjaergaard, P. Krantz, A. Melville, B. M. Niedzielski, A. Veps\"{a}l\"{a}inen, R. Winik, J. L. Yoder, F. Nori, T. P. Orlando, S. Gustavsson, and W. D. Oliver, Waveguide quantum electrodynamics with superconducting artificial giant atoms, Nature (London) \textbf{583}, 775 (2020).
\bibitem{Ciccarello2} A. Carollo, D. Cilluffo, and F. Ciccarello, Mechanism of decoherence-free coupling between giant atoms, Phys. Rev. Research \textbf{2}, 043184 (2020).
\bibitem{Kockum22pra} A. Soro and A. F. Kockum, Chiral quantum optics with giant atoms, Phys. Rev. A \textbf{105}, 023712 (2022).
%%Unconventional bound states
\bibitem{Guo20prr} L. Guo, A. F. Kockum, F. Marquardt, and G. Johansson, Oscillating bound states for a giant atom, Phys. Rev. Research \textbf{2}, 043014 (2020).
\bibitem{WangX21} X. Wang, T. Liu, A. F. Kockum, H.-R. Li, and F. Nori, Tunable Chiral Bound States with Giant Atoms, Phys. Rev. Lett. \textbf{126}, 043602 (2021).
\bibitem{Vega21} C. Vega, M. Bello, D. Porras, and A. Gonz\'{a}lez-Tudela, Qubit-photon bound states in topological waveguides with long-range hoppings, Phys. Rev. A \textbf{104}, 053522 (2021).
\bibitem{Wang21} W. Cheng, Z. Wang, and Y.-X. Liu, Topology and retardation effect of a giant atom in a topological waveguide, Phys. Rev. A \textbf{106}, 033522 (2022).
\bibitem{Lim23} K.  H. Lim, W. K. Mok,  and L. C. Kwek, Oscillating bound states in non-Markovian photonic lattices, Phys. Rev. A \textbf{107}, 023716 (2023).
%%Non-Markovian dynamics
\bibitem{Guo17PRA} L. Guo, A. Grimsmo, A. F. Kockum, M. Pletyukhov, and G. Johansson, Giant acoustic atom: A single quantum system with a deterministic time delay, Phys. Rev. A \textbf{95}, 053821 (2017).
\bibitem{Delsing19} G. Andersson, B. Suri, L. Guo, T. Aref, and P. Delsing, Nonexponential decay of a giant artificial atom, Nat. Phys. \textbf{15}, 1123 (2019).
\bibitem{Longhi20} S. Longhi, Photonic simulation of giant atom decay, Opt. Lett. \textbf{45}, 3017 (2020).
\bibitem{Du21pra1} L. Du, M.-R. Cai, J.-H. Wu, Z. Wang, and Y. Li, Single-photon nonreciprocal excitation transfer with non-Markovian retarded effects, Phys. Rev. A \textbf{103}, 053701 (2021).
\bibitem{Du22A} L. Du, Y.-T. Chen, Y. Zhang, and Y. Li, Giant atoms with time-dependent couplings, Phys. Rev. Research \textbf{4}, 023198 (2022).
\bibitem{Yin22A} X.-L. Yin, W.-B. Luo, J.-Q. Liao, Non-Markovian disentanglement dynamics in double-giant-atom waveguide-QED systems, Phys. Rev. A \textbf{106}, 063703 (2022).
\bibitem{Lu23} Q.-Y. Qiu, Y. Wu, and X.-Y. L\"{u}, Collective radiance of giant atoms in non-Markovian regime, Sci. China Phys. Mech. Astron. \textbf{66}, 224212 (2023).
%%Single-photon scattering
\bibitem{Wang20} W. Zhao and Z. Wang, Single-photon scattering and bound states in an atom-waveguide system with two or multiple coupling points, Phys. Rev. A \textbf{101}, 053855 (2020).
\bibitem{Du21pra2} L. Du and Y. Li, Single-photon frequency conversion via a giant $\Lambda$-type atom, Phys. Rev. A \textbf{104}, 023712 (2021).
\bibitem{Jia21} Q. Y. Cai and W. Z. Jia, Coherent single-photon scattering spectra for a giant-atom waveguide-QED system beyond the dipole approximation, Phys. Rev. A \textbf{104}, 033710 (2021).
\bibitem{Liao22} X.-L. Yin, Y.-H. Liu, J.-F. Huang, and J.-Q. Liao, Single-photon scattering in a giant-molecule waveguide-QED system, Phys. Rev. A \textbf{106}, 013715 (2022).

%%Spontaneous entanglment generation
\bibitem{Vidal11} A. Gonz\'{a}lez-Tudela, D. Martin-Cano, E. Moreno, L. Martin-Moreno, C. Tejedor, and F. J. Garcia-Vidal, Entanglement of Two Qubits Mediated by One-Dimensional Plasmonic Waveguides, Phys. Rev. Lett. \textbf{106}, 020501 (2011).
\bibitem{Baranger13} H. Zheng and H. U. Baranger, Persistent Quantum Beats and Long-Distance Entanglement from Waveguide-Mediated Interactions, Phys. Rev. Lett. \textbf{110}, 113601 (2013).
\bibitem{Porras2013} A. Gonz\'{a}lez-Tudela and D. Porras, Mesoscopic Entanglement Induced by Spontaneous Emission in Solid-State Quantum Optics, Phys. Rev. Lett. \textbf{110}, 080502 (2013).
\bibitem{Ballestero14} C. Gonzalez-Ballestero, E. Moreno, and F. J. G. Vidal, Generation, manipulation, and detection of two-qubit entanglement in waveguide QED, Phys. Rev. A \textbf{89}, 042328 (2014).
\bibitem{Facchi16} P. Facchi, M. S. Kim, S. Pascazio, F. V. Pepe, D. Pomarico, and T. Tufarelli, Bound states and entanglement generation in waveguide quantum electrodynamics, Phys. Rev. A \textbf{94}, 043839 (2016).

%%Entanglement generation in chiral wavegudie
\bibitem{Zoller14} T. Ramos, H. Pichler, A. J. Daley, and P. Zoller, Quantum Spin Dimers from Chiral Dissipation in Cold-Atom Chains, Phys. Rev. Lett. \textbf{113}, 237203 (2014).
\bibitem{Moreno15} C. Gonzalez-Ballestero, A. Gonzalez-Tudela, F. J. GarciaVidal, and E. Moreno, Chiral route to spontaneous entanglement generation, Phys. Rev. B \textbf{92}, 155304 (2015).
\bibitem{Schotland16} I. M. Mirza and J. C. Schotland, Multiqubit entanglement in bidirectional-chiral-waveguide QED, Phys. Rev. A \textbf{94}, 012302 (2016)
\bibitem{Mok20} W. K. Mok, J. B. You, L. C. Kwek, and D. Aghamalyan, Microresonators enhancing long-distance dynamical entanglement generation in chiral quantum networks, Phys. Rev. A \textbf{101}, 053861 (2020).



%%Entanglement generation in giant-atom waveguide-QED systems
\bibitem{Wang21pra}  H. Yu, Z. Wang, and J.-H. Wu, Entanglement preparation and nonreciprocal excitation evolution in giant atoms by controllable dissipation and coupling, Phys. Rev. A \textbf{104}, 013720 (2021).
\bibitem{Santos23} A. C. Santos and R. Bachelard, Generation of Maximally Entangled Long-Lived States with Giant Atoms in a Waveguide, Phys. Rev. Lett. \textbf{130}, 053601 (2023).


%%Wigner-Weisskopf approximation
\bibitem{Scullybook} M. O. Scully and M. S. Zubairy, \emph{Quantum Optics} (Cambridge University Press, Cambridge, 1997).
%%The theory of open quantum system
\bibitem{Breuer02} H.-P. Breuer and F. Petruccione, \emph{The Theory of Open Quantum Systems} (Oxford University Press, New York, 2002).
%%Non-Markovian entanglement between two distance atoms
%\bibitem{Compagno07} B. Bellomo, R. Lo Franco, and G. Compagno, Non-Markovian Effects on the Dynamics of Entanglement, Phys. Rev. Lett. \textbf{99}, 160502 (2007).
%\bibitem{Zheng08} X. Cao and H. Zheng, Non-Markovian disentanglement dynamics of a two-qubit system, Phys. Rev. A \textbf{77}, 022320 (2008).
%\bibitem{Plastina08} S. Maniscalco, F. Francica, R. L. Zaffino, N. L. Gullo, and F. Plastina, Protecting Entanglement via the Quantum Zeno Effect, Phys. Rev. Lett. \textbf{100}, 090503 (2008).
%\bibitem{Mazzola09} L. Mazzola, S. Maniscalco, J. Piilo, K.-A. Suominen, and B. M. Garraway, Sudden death and sudden birth of entanglement in common structured reservoirs, Phys. Rev. A \textbf{79}, 042302 (2009).
%\bibitem{Moreno13} C. Gonzalez-Ballestero, F. J. Garc\'{i}a-Vidal, and E. Moreno, Non-Markovian effects in waveguide-mediated entanglement, New J. Phys. \textbf{15}, 073015 (2013).

%%Linear approximation
\bibitem{Fan05} J.-T. Shen and S. Fan, Coherent Single Photon Transport in a One-Dimensional Waveguide Coupled with Superconducting Quantum Bits, Phys. Rev. Lett. \textbf{95}, 213001 (2005).
\bibitem{Fan09} J.-T. Shen and S. Fan, Theory of single-photon transport in a single-mode waveguide. I. Coupling to a cavity containing a two-level atom, Phys. Rev. A \textbf{79}, 023837 (2009).


%%Time-delayed master equation for a giant atom
\bibitem{Zhu22} Y. T. Zhu, S. Xue, R. B. Wu, W. L. Li, Z. H. Peng, and M. Jiang, Spatial-nonlocality-induced non-Markovian electromagnetically induced transparency in a single giant atom, Phys. Rev. A \textbf{106}, 043710 (2022).
%%Experimental parameters
\bibitem{Delsing13} I.-C. Hoi, C. M. Wilson, G. Johansson, J. Lindkvist, B. Peropadre, T. Palomaki, and P. Delsing, Microwave quantum optics with an artificial atom in one-dimensional open space, New J. Phys. \textbf{15}, 025011 (2013).
\bibitem{Kirchmair22} M. Zanner, T. Orell, C. M. F. Schneider, R. Albert, S. Oleschko, M. L. Juan, M. Silveri, and G. Kirchmair, Coherent control of a multi-qubit dark state in waveguide quantum electrodynamics, Nat. Phys. \textbf{18}, 538 (2022).

%%Measurement of quantum entanglement
\bibitem{Wootters98} W. K. Wootters, Entanglement of Formation of an Arbitrary State of Two Qubits, Phys. Rev. Lett. \textbf{80}, 2245 (1998).

%%Entanglement sudden birth
\bibitem{Ficek08} Z. Ficek and R. Tana\'{s}, Delayed sudden birth of entanglement, Phys. Rev. A \textbf{77}, 054301 (2008).
\bibitem{Retamal08} C. E. Lopez, G. Romero, F. Lastra, E. Solano, and J. C. Retamal, Sudden Birth versus Sudden Death of Entanglement in Multipartite Systems, Phys. Rev. Lett. \textbf{101}, 080503 (2008).
\bibitem{Garraway09} L. Mazzola, S. Maniscalco, J. Piilo, K.-A. Suominen, and B. M. Garraway, Sudden death and sudden birth of entanglement in common structured reservoirs, Phys. Rev. A \textbf{79}, 042302 (2009).


\end{thebibliography}
\end{document}
