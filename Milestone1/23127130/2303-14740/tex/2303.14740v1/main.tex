% mnras_template.tex 
%
% LaTeX template for creating an MNRAS paper
%
% v3.0 released 14 May 2015
% (version numbers match those of mnras.cls)
%
% Copyright (C) Royal Astronomical Society 2015
% Authors:
% Keith T. Smith (Royal Astronomical Society)

% Change log
%
% v3.0 May 2015
%    Renamed to match the new package name
%    Version number matches mnras.cls
%    A few minor tweaks to wording
% v1.0 September 2013
%    Beta testing only - never publicly released
%    First version: a simple (ish) template for creating an MNRAS paper

%%%%%%%%%%%%%%%%%%%%%%%%%%%%%%%%%%%%%%%%%%%%%%%%%%
% Basic setup. Most papers should leave these options alone.
\documentclass[fleqn,usenatbib]{mnras}

% MNRAS is set in Times font. If you don't have this installed (most LaTeX
% installations will be fine) or prefer the old Computer Modern fonts, comment
% out the following line
\usepackage{newtxtext,newtxmath}
% Depending on your LaTeX fonts installation, you might get better results with one of these:
%\usepackage{mathptmx}
%\usepackage{txfonts}
\usepackage{float}
\usepackage{graphicx}
\usepackage{caption}
\usepackage{subcaption}
% Use vector fonts, so it zooms properly in on-screen viewing software
% Don't change these lines unless you know what you are doing
\usepackage[T1]{fontenc}
\usepackage{breqn}
\usepackage{amsmath}
\usepackage{flushend}
% Allow "Thomas van Noord" and "Simon de Laguarde" and alike to be sorted by "N" and "L" etc. in the bibliography.
% Write the name in the bibliography as "\VAN{Noord}{Van}{van} Noord, Thomas"
\DeclareRobustCommand{\VAN}[3]{#2}
\let\VANthebibliography\thebibliography
\def\thebibliography{\DeclareRobustCommand{\VAN}[3]{##3}\VANthebibliography}


%%%%% AUTHORS - PLACE YOUR OWN PACKAGES HERE %%%%%

% Only include extra packages if you really need them. Common packages are:
\usepackage{graphicx}	% Including figure files
\usepackage{amsmath}	% Advanced maths commands
% \usepackage{amssymb}	% Extra maths symbols
\usepackage{comment}
%%%%%%%%%%%%%%%%%%%%%%%%%%%%%%%%%%%%%%%%%%%%%%%%%%

%%%%% AUTHORS - PLACE YOUR OWN COMMANDS HERE %%%%%

% Please keep new commands to a minimum, and use \newcommand not \def to avoid
% overwriting existing commands. Example:
%\newcommand{\pcm}{\,cm$^{-2}$}	% per cm-squared

%%%%%%%%%%%%%%%%%%%%%%%%%%%%%%%%%%%%%%%%%%%%%%%%%%

%%%%%%%%%%%%%%%%%%% TITLE PAGE %%%%%%%%%%%%%%%%%%%

% Title of the paper, and the short title which is used in the headers.
% Keep the title short and informative.
\title[Chaotic dynamics of off-equatorial orbits around pseudo-Newtonian compact objects]{Chaotic dynamics of off-equatorial orbits around Pseudo-Newtonian Schwarzschild and Kerr-like compact objects surrounded by dipolar halo}

% The list of authors, and the short list which is used in the headers.
% If you need two or more lines of authors, add an extra line using \newauthor
\author[S. Das et al.]{
Saikat Das$^{1}$\thanks{E-mail: dassaikat.physics@gmail.com}
and Suparna Roychowdhury$^{2}$\thanks{E-mail:  suparna@sxccal.edu}
\\
% List of institutions
$^{1}$Department of Physics, Indian Institute of Technology Madras, Chennai 600 036, India\\
$^{2}$Department of Physics, St. Xavier’s College, 30 Park Street, Kolkata 700 016, India\\
}

% These dates will be filled out by the publisher
\date{Accepted XXX. Received YYY; in original form ZZZ}

% Enter the current year, for the copyright statements etc.
\pubyear{2023}

% Don't change these lines
\begin{document}
\label{firstpage}
\pagerange{\pageref{firstpage}--\pageref{lastpage}}
\maketitle

% Abstract of the paper
\begin{abstract}
In this paper, we implement a generalized pseudo-Newtonian potential and prescribe a numerical fitting formalism, to study the off-equatorial orbits inclined at a certain angle with the equatorial plane around both Schwarzschild and Kerr-like compact object primaries surrounded by a dipolar halo of matter. The chaotic dynamics of the orbits are detailed for both non-relativistic and special-relativistic test particles. The dependence of the degree of chaos on the rotation parameter $a$ and the inclination angle $i$ is established individually using widely used indicators, such as the Poincar\'e Map and the Lyapunov Characteristic Number. We find that although the chaoticity of the orbits has a positive correlation with $i$, the growth in the chaotic behaviour is not systematic. There exists a threshold value of the inclination angle $i_{\text{c}}$, after which the degree of chaos shows a sharp increase. On the other hand, the chaoticity of the inclined orbits anti-correlates with $a$ at the lower inclination angles. At higher values of $i$, the degree of chaos is maximum for the maximally counter-rotating compact objects, though it has a weak negative, sometimes positive, correlation with $a$ at its higher values. The studies performed with several initial conditions and orbital parameters reveal the intricate nature of the system. 
\end{abstract}

% Select between one and six entries from the list of approved keywords.
% Don't make up new ones.
\begin{keywords}
gravitation -- methods: numerical -- chaos
\end{keywords}

%%%%%%%%%%%%%%%%%%%%%%%%%%%%%%%%%%%%%%%%%%%%%%%%%%

%%%%%%%%%%%%%%%%% BODY OF PAPER %%%%%%%%%%%%%%%%%%

\section{Introduction}
In the past few years, the study of the dynamics of orbits around a single Black Hole (BH hereafter) \citep[e.g.][and references therein]{nag2017influence, polcar2019free, dubeibe2021effect} or a BH binary \citep[e.g.][and references therein]{dubeibe2017pseudo, de2021beyond, alrebdi2022equilibrium} has become very popular. With the instrumental development of the detection and measuring techniques, it is already established that most of the galaxies consist of supermassive BHs at their galactic centres \citep{kormendy1995inward, beckmann2012active}. In almost all cases, the BHs are usually surrounded by a hollow spherical halo of matter and large accretion disks around them \citep{panagia1996nature, meyer1997formation}. Any particle traveling within the disk region or the corona will be influenced by both the central BH and the halo around it, and both of them need to be considered while looking for the locus of the particle. These are the situations where the Core-Shell Model comes into the picture while modeling systems like these \citep{vieira1999relativistic}. Not only the stand-alone BHs or neutron stars with accretion disks around them but even the galaxies can be modeled using the core-shell scheme because of the observational evidence of their structures which consist of huge rings and shells around the supermassive BHs located at their individual galactic centres \citep{sackett1990dark, arnaboldi1993studies, reshetnikov1997global, malin1983catalog, quinn1984formation, dupraz1987dynamical}.

In the present work, we are mostly focusing on the two-body system, where one of them is a rotating or a non-rotating compact object primary (COP hereafter) surrounded by a hollow halo of matter, and the other one is a test particle with a unit mass. While studying the dynamics of the orbit of the test particle around the compact object (CO hereafter), the phenomenon of chaos naturally comes into the picture. Several works have already been done on the chaotic behaviour of the orbits under the influence of the fully relativistic gravitational field of a BH-halo system \citep{vieira1996chaos, letelier1997chaos, vieira1999relativistic, gueron2002geodesic, vogt2003exact, semerak2010free, semerak2012free, sukova2013free, janiuk2011different, witzany2015free}. Incorporating the charge into the system also brings out interesting results where the charged test particle is influenced by the gravitational field as well as the magnetic force of the central BH along with its magnetosphere \citep{kopavcek2014inducing, kopavcek2015regular, takahashi2009chaotic, kovavr2008off, kovavr2010off}. However, as most of the BHs are rotating in nature \citep{miller2009stellar, ziolkowski2010population, daly2011estimates, reynolds2012probing, dotti2012orientation, tchekhovskoy2012prograde, garofalo2013retrograde, healy2014remnant, sesana2014linking}, we are more interested in studying the chaotic dynamics of the orbits around a Kerr-like BH, or a COP in general. 

While many research groups encounter the problem of accretion dynamics with a fully relativistic approach, it is very intensive, and computationally demanding to simulate these systems exactly. Therefore, we can follow a beyond-Newtonian approach, or a Pseudo-Keplerian formalism, where a Newtonian-like potential is designed in such a way that it mimics the actual potential by sustaining the essential aspects of the spacetime around the CO within a feasible limit of error. Certainly, the mimicking potential, known as the Pseudo-Newtonian Potential (PNP hereafter), will not exactly reproduce the fully relativistic scenario because of the nonlinearity in space-time close to the event horizon. Nevertheless, it reproduces spacetime far from the event horizon with a high degree of accuracy as the relativistic nonlinearity weakens in this region. Furthermore, it simplifies the calculation and the computation significantly. Such a PNP was first developed by \citet{paczynsky1980thick} which is applicable for the orbits on or near the equatorial plane of a Schwarzschild-like COP. After the successful implementation of this PNP, many pseudo-potentials were introduced for both Schwarzschild and Kerr-type COs, each with different sets of advantages and drawbacks \citep{nowak1991diskoseismology, chakrabarti1992newtonian, Semerak1999pseudo, mukhopadhyay2002description, mukhopadhyay2003pseudo, ghosh2004rotating, chakrabarti2006studies}. Till now, one of the most popular and widely implemented PNPs for a Kerr-like CO is the one developed in \citet{artemova1996modified} (ABN hereafter). However, all of the mentioned PNPs are applicable to the orbits near the equatorial plane. Even if they are prescribed for thick accretion disks, they do not include the inclination of the orbit in their models. For this reason, hardly any study has been done on the chaotic behaviour of inclined orbits. However, the chaotic nature of the equatorial orbits, and their correlation with the rotation parameter, has been widely studied using the previously available PNPs \citep{gueron2001chaos, gueron2001chaotic, chen2003chaotic, letelier2011chaotic, wang2012dynamics, nag2017influence}. 

The potential developed in \cite{ghosh2007} (GM hereafter) gives us the opportunity to look for the chaotic dynamics of the off-equatorial orbits inclined at a certain angle $i$ with the equatorial plane. The generalized PNP is useful to study the accretion dynamics around rotating, and non-rotating BHs in the off-equatorial planes \citep{ghosh2014newtonian}. The vector potential, derived from the generalized pseudo-Keplerian gravitational force in \cite{ghosh2007}, is also suitable for the hydrodynamical accretion studies of thick accretion disks \citep{bhattacharya2010disk}. Besides its applicability over a wide range of inclination angles ($0^\circ \leq i \leq 30^\circ$), the potential is valid for the entire range of the rotation parameter of the CO ($-1 \leq a \leq 1$). As the PNP is directly developed from the spacetime metric, it reproduces the values of the radius of marginally bound orbit $r_{\text{mb}}$, and the efficiency of unit mass at marginally stable orbit $E_{\text{ms}}$ with minimal error, and the radius of marginally stable orbit $r_{\text{ms}}$ with no error at all. This helps us in implementing this PNP to look at how the chaoticity of the off-equatorial orbits depends on the entire range of the Kerr parameter, as well as the angle of inclination.  

As mentioned earlier, we have used the core-shell model in the present work because it is in close resemblance with the realistic picture of an actual astrophysical BH \citep{vieira1999relativistic}. We have used a dipolar perturbative term signifying the hollow halo of matter around the COP. As opposed to the quadrupolar term used in this regard \citep{wang2012dynamics}, the dipolar term will be predominant when the halo is somewhat asymmetrically placed about the equatorial plane of the CO, although it is axially symmetric about the rotations axis of the system. The asymmetric mass distribution of the halo about the equatorial plane is more practical \citep{binney2008}. The dipolar term corresponding to this asymmetric halo has been used several times in literature \citep{gueron2001chaos, nag2017influence}.

In the current piece of work, we have implemented the PNP presented in \citet{ghosh2007} to study the chaotic behaviour of the off-equatorial orbits around a Schwarzschild and a Kerr-like COP. The paper is organized as follows. In section \ref{ch:description_of_PNP}, we have given a brief description of the PNPs and a numerical scheme to study the off-equatorial orbits. Along with that, we have presented the required equations of motion for both non-relativistic and special relativistic test particles. In section \ref{ch:chaos_poincare}, we use the previously mentioned equations of motion to study the chaotic dynamics of the orbits by generating the Poincar\'e Maps of their phase-space trajectories. Through the analysis, we establish a correlation of the degree of chaos with the inclination angle $i$ of the orbit, and the Kerr parameter $a$ of the COP, in a qualitative manner, for both non-rotating and rotating COs. In section \ref{ch:lyapunov}, we quantify the chaoticity using the Maximum Lyapunov Exponent (MLE hereafter) and Lyapunov Characteristic Number (LCN hereafter). Here, we also implement them to corroborate the qualitative results and analyze the chaotic correlations in more depth. Finally, in section \ref{ch:conclusion}, we conclude our findings and discuss the future directions of our research.

\section{Formulation of the pseudo-Newtonian potentials and the equations of motion} \label{ch:description_of_PNP}
We begin our study by providing a mathematical description of the PNPs, along with the necessary equations of motion to study the orbital dynamics of the system. For the special case of equatorial orbits, we compare the effective potential consisting of GM PNP with that of ABN PNP, another well-established PNP widely used in this regard.

\subsection{Pseudo-Newtonian Potential}
\begin{figure}
        \centering
	\includegraphics[trim={20cm 0 12cm 0}, clip, width=\columnwidth]{1}
    \caption{Different dynamical parameters of an off-equatorial orbit around a COP. The inclined plane consists of a circular orbit along which the test particle is rotating. The orbit is inclined at an angle $i$ with the symmetry axis of the CO (the $z$-axis), as well as its equatorial plane. The spinning parameter of the CO is $a$. $L$ denotes the total angular momentum of the test particle and its direction is along the rotating axis of the inclined orbit.}
    \label{fig:off-equatorial plane}
\end{figure}
If a particle of unit mass is accreting around a rotating COP of mass $M$ and angular momentum $J_\phi$ on a plane inclined at an angle $i$ with the equatorial plane of the COP (Figure \ref{fig:off-equatorial plane}), the generalized gravitational force on the particle, as prescribed by \cite{ghosh2007}, can be mimicked by the expression  
\begin{dmath} \label{eq:GMPNF}
    F_{\text{K} r}=2MA^2\sec^2i \left\{a\sqrt{2M}r^{3/2}\left\{\Delta+2r\left(r-M\right)\right\}+r\Delta\sqrt{ [A+r^4-a^2(\Delta+r^2-3Mr))]\cos2i\sec^2i}\right\}^{-2}
\end{dmath}
where $A=a^4+r^4+2a^2r(r-2M)$ and $\Delta=r^2+a^2-2Mr$. The parameter $a$ (=$J_\phi$/$M$) denotes the rotation parameter of the COP ($-1\leq a \leq 1$). Here, $r$ is the radial distance of the rotating particle from the origin in the spherical polar coordinate $(r,\theta,\phi)$. The gravitational force on the particles, given in equation (\ref{eq:GMPNF}), can be applied successfully by taking the value of $i$ in the range $i\in[0^\circ,30^\circ]$ while keeping the error within reasonable limits \citep{ghosh2007}. The corresponding Pseudo-Kerr potential $V_{\text{GM}}$, for particular values of $a$ and $i$, can be evaluated from the generalised force using the relation
\begin{equation} \label{eq:Pot_Def}
    V_{\text{GM}}(r,a,i)=-\int_r^\infty F_{\text{Kr}}(r',a,i)dr'
\end{equation}

For Schwarzschild-like COs, the value of $a$ becomes $0$. Thus, by putting $a=0$ in the equation (\ref{eq:Pot_Def}), we get the PNP for the inclined orbits around a Schwarzschild-like COP, which is given by
\begin{equation} \label{eq:GMPW}
    V_{\text{GMS}}(r)=-\frac{\sec(2i)}{r-2}
\end{equation}
It should be mentioned that throughout the work, the velocities have been scaled by $c$, the speed of light in vacuum, and the distances are scaled by $r_{\text{g}}=GM/c^2$, where $G$ is the gravitational constant and $M$ is the mass of the COP. We have assumed that $G=M=c=1$.

The potential, given in equation (\ref{eq:GMPW}), is strikingly similar to the PNP presented in \cite{paczynsky1980thick} (PW hereafter). By putting $i=0$, $V_{\text{GMS}}$ gets equal to the PW potential. The fact that $V_{\text{GM}}$ comes down to the PW potential for $a=i=0$ signifies the generic nature of the GM PNP. In a similar way, if we put $i=0$ but $a\neq 0$, we get the PNP for the orbits on the equatorial plane in the Kerr geometry. 

The analytically closed form of the generalized PNP $V_{\text{GM}}$, however, is extremely difficult to derive by a straightforward integration, as per the expression given in equation (\ref{eq:Pot_Def}), because of the complex mathematical form of the gravitational force, given in equation (\ref{eq:GMPNF}). Therefore, we take a numerical approach where we integrate the generalized force $F_{\text{Kr}}$ numerically for several values of $r$ and fit the data with a fitting function of the form 
\begin{equation} \label{eq:GMPNP}
    V_{\text{GMf}}(r)=-\frac{\exp\left(\frac{\gamma_1}{(r-\gamma_2)^{\gamma_3}}+\gamma_4\right)}{(r-\gamma_2)^{\gamma_5}}
\end{equation}
where $\gamma_1,\gamma_2,\gamma_3,\gamma_4,\gamma_5$ are fitting parameters. It is evident that the potential $V_{\text{GMf}}$ is spherically symmetric in nature, similar to the potential $V_{\text{GM}}$, as well as the space around the rotating or non-rotating CO. The fitting function asymptotically approaches zero when the test particle moves towards infinity $\left( V_{\text{GMf}} \rightarrow 0 \text{, when } r \rightarrow \infty \right)$. Also, it tends to infinity when the test particle approaches the event horizon at the radial distance $r=\gamma_2 r_{\text{g}}$ $\left( V_{\text{GMf}} \rightarrow - \infty \text{, when } r \rightarrow \gamma_2 \right)$. These facts satisfy the conditions of a PNP, and it leads us to implement $V_{\text{GMf}}$, given in equation \ref{eq:GMPNP}, as a suitable fitting function. For particular values of $a$ and $i$, the fitting parameters are evaluated so that $V_{\text{GMf}}$ fits with $V_{\text{GM}}$ with a very small relative error. More details about the fitting methodology are presented in section \ref{ch:chaos_poincare}.

As the GM PNP is generic in nature, it is worthwhile to ask how this potential will behave for the orbits on the equatorial plane around a rotating COP ($i=0$ but $a \neq 0$) and how this potential differs from any other PNP which is being applied in this scenario. In literature, there are many well-established PNPs that have been successfully used in the past for equatorial accretion disks (as referred earlier in this paper). We consider the well-known ABN PNP and compare it with GM PNP for $i=0$. 

The ABN PNP, after integrating the free-fall acceleration prescribed in \citet{artemova1996modified}, comes out to be
\begin{equation} \label{eq:APNP}
    V_{\text{ABN}}(r)=-\frac{1}{r_1 (\beta-1)} \left[ \frac{r^{\beta-1}}{(r-r_1)^{\beta -1}} -1 \right]
\end{equation}
Here, $r_1$ represents the radius of the event horizon, given as follows.
\begin{equation} \label{eq:ABN_event_horizon}
    r_1 = 1+\sqrt{1-a^2}
\end{equation}
The value of $\beta$ is given by
\begin{equation} \label{eq:ABN_beta}
    \beta=\frac{r_{\text{in}}}{r_1}-1
\end{equation}
where $r_{\text{in}}$ is the radius of marginally stable orbit such that
\begin{equation} \label{eq:ABN_r_in}
    r_{\text{in}}=3+Z_2 \mp \sqrt{(3-Z_1)(3+Z_1+2Z_2)}
\end{equation}
The values of the parameters $Z_1$ and $Z_2$ are defined as
\begin{equation}
    Z_1=1+(1-a^2)^{\frac{1}{3}} \left[ (1+a)^{\frac{1}{3}}+(1-a)^{\frac{1}{3}} \right] \text{ and }
    Z_2=\sqrt{3a^2+Z_1^2}
\end{equation}
The negative and positive signs in equation (\ref{eq:ABN_r_in}) signify the co-rotating ($a>0$) and counter-rotating ($a<0$) COPs respectively.

\subsection{Equations of motion} 
We move on to the formulation of the orbital dynamics of our system. As there is azimuthal symmetry in the orbital motion of the particle revolving around a CO, it is convenient to write equation (\ref{eq:GMPNP}) in the cylindrical polar coordinates $(\rho,\phi,z)$ as 
\begin{equation} \label{eq:monopole}
    V_{\text{GMf}}(\rho,z)=-\frac{\exp\left(\frac{\gamma_1}{\left(\sqrt{\rho^2+z^2}-\gamma_2\right)^{\gamma_3}}+\gamma_4\right)}{\left(\sqrt{\rho^2+z^2}-\gamma_2\right)^{\gamma_5}}
\end{equation}
The absence of any $\phi$ term in the potential signifies the azimuthal symmetry. 

Along with the monopole term presented in equation (\ref{eq:monopole}) which is conservative in nature, we used a dipolar, non-central, perturbative term which is introduced to simulate a halo around the central object. The distribution of the halo is such that it is axially symmetric about the rotating axis of the COP. If the halo is also symmetric about the equatorial plane, which is the case for the prolate or the oblate geometry, the first non-central, contributing term in the potential will be quadrupolar in nature \citep{vieira1999relativistic, wang2012dynamics}. However, we consider the halo to be asymmetrically placed about the equatorial plane, somewhat deformed in the polar regions \citep{binney2008}. In this case, the non-central, leading term will be dipolar in nature with a form $\alpha z$, $\alpha$ being the dipole-coefficient. The value of $\alpha$ will depend on the asymmetry in the mass distribution of the halo and it will be small for any practical system \citep{vieira1999relativistic, nag2017influence}. Therefore, the net potential that the particle is experiencing while revolving around a CO surrounded by an asymmetrically distributed halo can be represented by
\begin{equation} \label{eq:phi_g}
    \Phi_g(\rho,z)=V_{\text{GMf}}(\rho,z)+\alpha z
\end{equation}

The orbit of the test particle, with total angular momentum $L$, is inclined with respect to the equatorial plane with an inclination angle $i$, and it is precessing about the spinning axis of the COP. This phenomenon is known as the Lense-Thirring Precession \citep{lense1918influence}. The $z$-component of the angular momentum $L_z$ is conserved due to the azimuthal symmetry of the orbit. The locus of the particle will not always make an angle $i$ with the equatorial plane, but the plane of orbital precession will always be inclined at $i$ with the spinning axis of the COP (Figure \ref{fig:off-equatorial plane}). Therefore, $L_z=L \cos(i)$, which implies that $L$ is also conserved \citep{ghosh2007}. 

The motion of the orbiting particle in the $\phi$ direction will induce a centrifugal force on it along the radial direction. This can be taken into account by adding an additional term $\left( \dfrac{L^2}{2\rho^2} \right)$ with the potential $\Phi_g(\rho,z)$ in equation (\ref{eq:phi_g}). Hence, the effective potential in Kerr geometry can be given by the following expression.  
\begin{equation} \label{eq:overall_potential_general}
    \left.V_{\text{eff}}(\rho,z)\right|_{a \neq 0}=-\frac{\exp\left(\frac{\gamma_1}{\left(\sqrt{\rho^2+z^2}-\gamma_2\right)^{\gamma_3}}+\gamma_4\right)}{\left(\sqrt{\rho^2+z^2}-\gamma_2\right)^{\gamma_5}}+\alpha z+\frac{L^2}{2\rho^2}
\end{equation}
The net potential for an off-equatorial orbit in Schwarzschild geometry $(a=0)$ is more straightforward. We can use the potential given in equation (\ref{eq:GMPW}) and write down the net effective potential as follows.
\begin{equation} \label{eq:overall_potential_Sch}
    \left.V_{\text{eff}}(\rho,z)\right|_{a=0}=-\frac{\sec(2i)}{\sqrt{\rho^2+z^2}-2}+\alpha z+\frac{L^2}{2\rho^2}
\end{equation}

The equations of motion can be derived from the Hamiltonian, given by
\begin{equation}
    H = \frac{1}{2} \left( p_\rho^2 + p_z^2 \right) + V_{\text{eff}} (\rho,z)
\end{equation}
where $V_{\text{eff}}$ can be replaced by either the potential given in equation (\ref{eq:overall_potential_general}) for a Kerr-like COP, or the one given in equation (\ref{eq:overall_potential_Sch}) for a Schwarzschild-like COP. Thus, the equations of motion can be formulated as follows.
\begin{subequations} \label{eq:eom_nrel}
	\begin{alignat} {4}
		&\label{eq:rho_dot_nrel}\dot{\rho}=p_\rho\\
		&\label{eq:prho_dot_nrel}\dot{p_\rho}=-\frac{\partial V_{\text{eff}}}{\partial \rho}=-\frac{\partial \Phi_g}{\partial \rho}+\frac{L^2}{\rho^3}\\ 
		&\label{eq:z_dot_nrel}\dot{z}=p_z\\
		&\label{eq:pz_dot_nrel}\dot{p_z}=-\frac{\partial V_{\text{eff}}}{\partial z}=-\frac{\partial \Phi_g}{\partial z}
	\end{alignat}
\end{subequations}
Here, there is no special relativistic correction as the speed of the particle is assumed to be much less than $c$.

However, if the speed of the particle is comparable to the speed of light, special relativistic corrections to the equations in (\ref{eq:rho_dot_nrel}-\ref{eq:pz_dot_nrel}) have to be performed, and the equations will be given by
\begin{subequations} \label{eq:eom_rel}
	\begin{alignat} {4}
		&\label{eq:rho_dot_rel}\dot{\rho}=p_\rho\\
		&\label{eq:prho_dot_rel}\dot{p_\rho}=\frac{1}{\Phi_g-E}\left[ \frac{\partial \Phi_g}{\partial \rho} \left(1-p_\rho^2 \right)-\frac{\partial \Phi_g}{\partial z}p_z p_\rho - \frac{L^2}{(E-\Phi_g) \rho^3} \right]\\ 
		&\label{eq:z_dot_rel}\dot{z}=p_z\\
		&\label{eq:pz_dot_rel}\dot{p_z}=\frac{1}{\Phi_g-E}\left[ \frac{\partial \Phi_g}{\partial z} \left(1-p_z^2 \right)-\frac{\partial \Phi_g}{\partial \rho}p_z p_\rho \right]
	\end{alignat}
\end{subequations}
where $E$ is the total energy of the particle which is conserved and it is governed by the relation
\begin{equation} \label{eq:energy_cons}
    p_\rho^2+p_z^2+\frac{L^2}{(E-\Phi_g)^2 \rho^2}+\frac{1}{(E-\Phi_g)^2}=1
\end{equation}
The equations are exactly similar to the ones derived in \cite{gueron2001chaos} and \cite{nag2017influence}.

The energy conservation equation, given in equation (\ref{eq:energy_cons}), converts the four-dimensional phase space consisting of $(\rho,p_\rho,z,p_z)$ into an effective three-dimensional hypersurface consisting of $(\rho,p_\rho,z)$. The value of $p_z$ is bound to satisfy equation (\ref{eq:energy_cons}) for a particular set of the values of $(\rho,p_\rho,z)$. This provides comprehensibility in analyzing the system and also gives the advantage to minimize the error in numerical calculations.

\subsection{Comparison between GM PNP and ABN PNP for equatorial orbits}
If we compare these two pseudo-potentials, namely the GM PNP and the ABN PNP, and investigate which one of them mimics the actual general relativistic scenario more accurately, both the potentials reproduce the radius of marginally stable orbits ($r_{\text{ms}}$) to be exactly the same as that in the fully relativistic case. Both of them reproduce the values of the radii of marginally bound orbits ($r_{\text{mb}}$) and the efficiency of unit mass at the marginally stable orbits ($E_{\text{ms}}$) with similar accuracy. The margin of dissimilarity being very less, both the PNPs can be implemented efficiently to study the accretion dynamics of equatorial disks. However, the simple mathematical form of ABN PNP makes it more comprehensive and convenient to implement in complicated scenarios, involving the orbits on the equatorial plane. For off-equatorial orbits, the GM PNP is possibly one of the mandatory choices, despite having a complex mathematical form.  

\begin{figure*} 
    
     \centering
     \begin{subfigure}[b]{0.32\textwidth}
         \centering
         \includegraphics[width=\textwidth]{2_1}
         \caption{$a$=0.1}
         \label{fig:comparison_of_potentials_a_1}
     \end{subfigure}
     \hfill
     \begin{subfigure}[b]{0.32\textwidth}
         \centering
         \includegraphics[width=\textwidth]{2_2}
         \caption{$a$=0.5}
         \label{fig:comparison_of_potentials_a_2}
     \end{subfigure}
     \hfill
     \begin{subfigure}[b]{0.32\textwidth}
         \centering
         \includegraphics[width=\textwidth]{2_3}
         \caption{$a$=0.9}
         \label{fig:comparison_of_potentials_a_3}
     \end{subfigure}
     \hfill
     \begin{subfigure}[b]{0.32\textwidth}
         \centering
         \includegraphics[width=\textwidth]{2_4}
         \caption{$a$=-0.1}
         \label{fig:comparison_of_potentials_a_4}
     \end{subfigure}
     \hfill
     \begin{subfigure}[b]{0.32\textwidth}
         \centering
         \includegraphics[width=\textwidth]{2_5}
         \caption{$a$=-0.5}
         \label{fig:comparison_of_potentials_a_5}
     \end{subfigure}
     \hfill
     \begin{subfigure}[b]{0.32\textwidth}
         \centering
         \includegraphics[width=\textwidth]{2_6}
         \caption{$a$=-0.9}
         \label{fig:comparison_of_potentials_a_6}
     \end{subfigure}
        \caption{Comparison between the GMf PNP $\left(V(r)=V_{\text{GMf}}(r)+L^2/2r^2\right)$ and ABN PNP $\left(V(r)=V_{\text{ABN}}(r)+L^2/2r^2\right)$, for $z=0$, $L=4.2$, and different values of the Kerr parameter $a$. }
        \label{fig:comparison_of_potentials_a}
\end{figure*}

If we find out the differences between the two PNPs for higher radial distances far from the event horizon, we can see a trend in their behaviour which has a correlation with the rotation parameter $a$. Let us consider the effective potential $V(r)$ consisting of the GMf PNP, or the ABN PNP, along with the centrifugal contribution $\left( \dfrac{L^2}{2r^2} \right)$. At $a=0$, that is for a Schwarzschild-like CO, both the PNPs come down to the PW PNP. That is why both ABN PNP and GMf PNP are equal to each other for $a=0$. For $a>0$, that is for a co-rotating COP, the effective potential consisting of the ABN PNP is more than that of the GMf PNP (Figure \ref{fig:comparison_of_potentials_a_1}-\ref{fig:comparison_of_potentials_a_3}). For $a<0$, which represents a counter-rotating COP, the situation is completely opposite (Figure \ref{fig:comparison_of_potentials_a_4}-\ref{fig:comparison_of_potentials_a_6}). In this case, the effective potential consisting of the ABN PNP is less compared to that of the GMf PNP. This impacts the chaotic behaviour of the equatorial orbits in some significant way, which has been studied in section \ref{ch:chaos_poincare}.

\section{Stability of orbits and visualization of chaos} \label{ch:chaos_poincare}
We have implemented the method of Poincar\'e Map (PM hereafter) to visualize chaos in the system. In this method, instead of looking at the entire three-dimensional hypersurface $(\rho,p_\rho,z)$, we look at a two-dimensional cross-section of the phase space which consists of the points of intersection of the trajectories on a fixed plane while moving in a particular direction (either $p_z>0$ or $p_z<0$) along the trajectory \citep{berry1978topics}. The stable or regular orbits form systematic patterns in the map corresponding to the concentric Kolmogorov-Arnold-Moser (KAM) tori, whereas the chaotic orbits give rise to a sea of scattered points \citep{berry1978topics, goldstein2001classical}. Usually, the systematic patterns, known as the islands of order, are surrounded by these regions of chaos. This phenomenon is known as soft chaos, where the regular patterns and the chaotic points coexist simultaneously. On the contrary, the hard chaos is defined by the complete absence of the ordered islands. In almost all cases, we get to observe the soft chaos. Hard chaos is very rare in literature.

The PM allows us to look at the chaotic nature of the phase space in a qualitative manner. It gives a visual impression of the overall chaoticity of the system for a given set of parameters. The system is said to be more chaotic when the islands of the regular patterns are less compared to the region of scattered points \citep{goldstein2001classical, nag2017influence}. It is a useful tool to look for how the degree of chaos correlates with different dynamical parameters of a system. If the chaotic behaviour of the system increases, the islands of regular orbits shrink or gradually collapse. The decrease in the allowed volume in the phase space also signifies the suppression of non-linearity in the system. 

In this paper, we have plotted the Poincar\'e sections on a specific $z$-plane, while the orbit is moving out of the plane along its trajectory in the phase space ($p_z>0$). We have considered the initial conditions in such a way that for a particular set of orbital parameters, the coordinates $\rho (0)$ and $z (0)$ remain fixed, and the momentum $p_{\rho} (0)$ is varied at regular intervals. The value of $p_z (0)$ is evaluated from the energy conservation equation (\ref{eq:energy_cons}). We have made the orbits evolve for time steps $n=10^6$. It is possible to conclude the dependence of chaos on a particular parameter through a comparative study of the PM plots by varying that parameter while keeping the rest fixed.  

\subsection{For non-rotating compact object primaries}

\begin{figure*} 
    
     \centering
     \begin{subfigure}[b]{0.49\textwidth}
         \centering
         \includegraphics[width=\textwidth]{3_1}
         \caption{$i=10^\circ$, Non-relativistic}
         \label{fig:PM_with_i10_NonRel}
     \end{subfigure}
     \hfill
     \begin{subfigure}[b]{0.49\textwidth}
         \centering
         \includegraphics[width=\textwidth]{3_2}
         \caption{$i=13^\circ$, Non-relativistic}
         \label{fig:PM_with_i13_NonRel}
     \end{subfigure}
     \hfill
     \begin{subfigure}[b]{0.49\textwidth}
         \centering
         \includegraphics[width=\textwidth]{3_3}
         \caption{$i=14^\circ$, Non-relativistic}
         \label{fig:PM_with_i14_NonRel}
     \end{subfigure}
     \hfill
     \begin{subfigure}[b]{0.49\textwidth}
         \centering
         \includegraphics[width=\textwidth]{3_4}
         \caption{$i=15^\circ$, Non-relativistic}
         \label{fig:PM_with_i15_NonRel}
     \end{subfigure}
        \caption{Poincar\'e maps on the cross-sectional plane $z=0$ for the orbits with $\alpha=2\times 10^{-4}$, $E=0.976$, and $L=4.18$. The plots are evaluated without considering any special relativistic corrections. The angle of inclination $i$ is varied to observe that the orbits are becoming more chaotic as $i$ increases.}
        \label{fig:PM_with_i_NonRel}
\end{figure*}

As the gravitational force, given in equation (\ref{eq:GMPNF}), is generic in nature, we can put $a=0$ in the expression and study the off-equatorial orbits around a static and stationary COP. We have implemented the effective potential, given in equation (\ref{eq:overall_potential_Sch}), and found out the equations of motion as per the equations given in (\ref{eq:rho_dot_nrel}-\ref{eq:pz_dot_nrel}) which does not consider any relativistic correction. We have varied the inclination angle $i$ and generated the PM for each of the angles. We have shown the maps for $i=10^\circ,13^\circ,14^\circ,\text{ and } 15^\circ$ by keeping the total energy $E$, total angular momentum $L$, and dipole coefficient $\alpha$ unchanged (Figure \ref{fig:PM_with_i_NonRel}). We can observe a trend in the chaotic behaviour as the angle $i$ increases. Evidently, the PM of the phase space for $i=15^\circ$ consists of a more chaotic region than that for $i=10^\circ$, in which case most of the orbits are regular in nature. The regular (or, less nonlinear) orbits, originating from the initial conditions $\rho(0)=15$, $p_{\rho} (0) = [0,-0.005, -0.010, -0.015,...,-0.095]$, and $z(0)=0$, in the Poincar\'e map for $i=10^\circ$ gradually become more nonlinear as the inclination angle increases. For $i=14^\circ$, most of the regular patterns corresponding to these orbits convert into scattered points. In the PM plot for $i=15^\circ$, all of these orbits become chaotic in nature. It is to be noted that the allowed volume of phase space also enhances with the increase of the inclination angle. Therefore, this provides qualitative evidence that the degree of chaos has a positive correlation with the angle of inclination of the orbit.

\begin{figure*} 
    
     \centering
     \begin{subfigure}[b]{0.49\textwidth}
         \centering
         \includegraphics[width=\textwidth]{4_1}
         \caption{$i=10^\circ$, Relativistic}
         \label{fig:PM_with_i10_Rel}
     \end{subfigure}
     \hfill
     \begin{subfigure}[b]{0.49\textwidth}
         \centering
         \includegraphics[width=\textwidth]{4_2}
         \caption{$i=13^\circ$, Relativistic}
         \label{fig:PM_with_i13_Rel}
     \end{subfigure}
     \hfill
     \begin{subfigure}[b]{0.49\textwidth}
         \centering
         \includegraphics[width=\textwidth]{4_3}
         \caption{$i=14.6^\circ$, Relativistic}
         \label{fig:PM_with_i14_Rel}
     \end{subfigure}
     \hfill
     \begin{subfigure}[b]{0.49\textwidth}
         \centering
         \includegraphics[width=\textwidth]{4_4}
         \caption{$i=15^\circ$, Relativistic}
         \label{fig:PM_with_i15_Rel}
     \end{subfigure}
        \caption{Poincar\'e maps on the cross-sectional plane $z=0$ for the orbits with $\alpha=2\times 10^{-4}$, $E=0.974$, and $L=4.60$. The plots are evaluated while considering the special relativistic corrections. The angle of inclination $i$ is varied to observe that the orbits are becoming more chaotic as $i$ increases.}
        \label{fig:PM_with_i_Rel}
\end{figure*}

In order to incorporate the relativistic corrections into the equations, we have derived the equations of motion using equations given in (\ref{eq:rho_dot_rel}-\ref{eq:pz_dot_rel}). We solved these equations too to get the PMs corresponding to the phase spaces of the orbits of relativistic test-particle and studied the trend of the degree of chaos with the inclination angle $i$ (Figure \ref{fig:PM_with_i_Rel}). Similar to the non-relativistic case, we have kept the parameter $E$, $L$, and $\alpha$ to be constant. We can observe the exact similar behaviour of the orbits, as we have seen in the non-relativistic case. The only difference is that we have used a smaller value of $E$ and a higher value of $L$ in order to suppress the nonlinear nature of the relativistic equation of motion (\ref{eq:rho_dot_rel}-\ref{eq:pz_dot_rel}). This establishes the fact that the relativistic phase-space trajectories are similar in nature to the corresponding non-relativistic counterpart, but with lower energy and/or higher angular momentum of the orbiting particle. 

\begin{figure*} 
    
     \centering
     \begin{subfigure}[b]{0.33\textwidth}
         \centering
         \includegraphics[width=\textwidth]{5_1}
         \caption{$L=4.3$, Non-relativistic}
         \label{fig:PM_with_L4.3_NonRel}
     \end{subfigure}
     \hfill
     \begin{subfigure}[b]{0.33\textwidth}
         \centering
         \includegraphics[width=\textwidth]{5_2}
         \caption{$L=4.4$, Non-relativistic}
         \label{fig:PM_with_L4.4_NonRel}
     \end{subfigure}
     \hfill
     \begin{subfigure}[b]{0.33\textwidth}
         \centering
         \includegraphics[width=\textwidth]{5_3}
         \caption{$L=4.5$, Non-relativistic}
         \label{fig:PM_with_L4.5_NonRel}
     \end{subfigure}
     \begin{subfigure}[b]{0.33\textwidth}
         \centering
         \includegraphics[width=\textwidth]{5_4}
         \caption{$L=4.3$, Relativistic}
         \label{fig:PM_with_L4.3_Rel}
     \end{subfigure}
     \hfill
     \begin{subfigure}[b]{0.33\textwidth}
         \centering
         \includegraphics[width=\textwidth]{5_5}
         \caption{$L=4.4$, Relativistic}
         \label{fig:PM_with_L4.4_Rel}
     \end{subfigure}
     \hfill
     \begin{subfigure}[b]{0.33\textwidth}
         \centering
         \includegraphics[width=\textwidth]{5_6}
         \caption{$L=4.5$, Relativistic}
         \label{fig:PM_with_L4.5_Rel}
     \end{subfigure}
        \caption{Poincar\'e maps on the cross-sectional plane $z=-5$ for the orbits with $i=8^\circ$, $\alpha=3.9\times 10^{-4}$, and $E=0.972$. The angular momentum of the particle $L$ is varied to observe that the orbits are becoming less chaotic as $L$ increases, implying that there is a negative correlation of chaos with $L$. Also, the relativistic case is always more chaotic than its corresponding non-relativistic counterpart.}
        \label{fig:PM_with_L_Rel_NonRel}
\end{figure*}

Furthermore, we observe that the chaotic behaviour of the orbits has a negative correlation with the angular momentum $L$ for a particular value of $i$ (Figure \ref{fig:PM_with_L_Rel_NonRel}). We studied both the non-relativistic and relativistic cases. One can easily observe that the PM plot for $L=4.3$ (\ref{fig:PM_with_L4.3_Rel}), with almost no regular orbit in the bottom half of the map, is much more chaotic than that for $L=4.4$ (\ref{fig:PM_with_L4.4_Rel}), or $L=4.5$ (\ref{fig:PM_with_L4.5_Rel}). As $L$ increases, we can observe the islands of order getting evolved within the chaotic region. This establishes the fact that the degree of chaos decreases with the increasing value of $L$. This was established earlier for the orbits on the equatorial plane \citep{nag2017influence}. For off-equatorial orbits, the statement holds to be true as well.

One can also observe that the PM plots for the relativistic test particles are more chaotic than the corresponding non-relativistic counterparts (Figure \ref{fig:PM_with_L_Rel_NonRel}). This can be explained by the increase in nonlinearity in the equations of motion for the relativistic case (equation \ref{eq:rho_dot_rel}-\ref{eq:pz_dot_rel}) compared to the non-relativistic ones (equation \ref{eq:rho_dot_nrel}-\ref{eq:pz_dot_nrel}). Combining the earlier findings, we can conclude that this non-linearity can be suppressed either by decreasing the energy $E$, or by increasing the angular momentum $L$.

\subsection{For rotating compact object primaries}
After considering the Schwarzschild geometry, we move on to the Kerr geometry where the Kerr parameter $a$ is non-zero. In this case, for particular values of $a$ and $i$, we first evaluate the PNP $V_{\text{GM}}$ from equation (\ref{eq:Pot_Def}) for several values of $r$, using the generalised force $F_{\text{Kr}}$ given in equation (\ref{eq:GMPNF}). After that, we take the natural log of the potential values to fit the data with the logarithmic form of the fitting function in equation (\ref{eq:GMPNP}), which is given by
\begin{equation} \label{eq:log_GMPNP}
    \ln\left(-V_{\text{GMf}}(r)\right)=\frac{\gamma_1}{(r-\gamma_2)^{\gamma_3}}+\gamma_4 - \gamma_5\ln\left(r-\gamma_2\right)
\end{equation}
This allows us to fit the curve conveniently using the least-square-fitting method with minimum error and evaluate the fitting parameters. After we estimate the values of the fitting parameters, we can use them in equation (\ref{eq:overall_potential_general}) to get the overall potential consisting of the GMf PNP term for particular values of $a$ and $i$, along with the dipolar perturbation term and the centrifugal contribution as well. Thereafter, we implement this potential in the equations (\ref{eq:rho_dot_nrel})-(\ref{eq:pz_dot_nrel}) and equations (\ref{eq:rho_dot_rel})-(\ref{eq:pz_dot_rel}) to get the equations of motion for non-relativistic and relativistic cases, respectively. 

\begin{figure*} 
    
     \centering
     \begin{subfigure}[b]{0.49\textwidth}
         \centering
         \includegraphics[width=\textwidth]{6_1}
         \caption{$V_{\text{GM}}$ and $V_{\text{GMf}}$ for $a=0.4$, $i=20^\circ$}
         \label{fig:GMPNP_fitting_1}
     \end{subfigure}
     \hfill
     \begin{subfigure}[b]{0.49\textwidth}
         \centering
         \includegraphics[width=\textwidth]{6_2}
         \caption{PM plot for $a=0.4$, $i=20^\circ$, Non-relativistic}
         \label{fig:GMPNP_fitting_2}
     \end{subfigure}
     \begin{subfigure}[b]{0.49\textwidth}
         \centering
         \includegraphics[width=\textwidth]{6_3}
         \caption{$V_{\text{GM}}$ and $V_{\text{GMf}}$ for $a=-0.4$, $i=20^\circ$}
         \label{fig:GMPNP_fitting_3}
     \end{subfigure}
     \hfill
     \begin{subfigure}[b]{0.49\textwidth}
         \centering
         \includegraphics[width=\textwidth]{6_4}
         \caption{PM plot for $a=-0.4$, $i=20^\circ$, Non-relativistic}
         \label{fig:GMPNP_fitting_4}
     \end{subfigure}
        \caption{(a),(c) Fitting the GM PNP $V_{\text{GM}}$ with the function $V_{\text{GMf}}$ for $a=\pm 0.4$, and $i=20^\circ$; (b),(d) Poincar\'e maps on the cross-sectional plane $z=0$ for the non-relativistic orbits with $a=\pm 0.4$, $i=20^\circ$, $\alpha=2.2\times 10^{-4}$, $L=4.85$, and $E=0.972$.}
        \label{fig:GMPNP_fitting}
\end{figure*}

We can see the quality of fitting of GMf PNP with GM PNP in Figure \ref{fig:GMPNP_fitting_1} for $a=0.4$ and $i=20^\circ$, and Figure \ref{fig:GMPNP_fitting_3} for $a=-0.4$ and $i=20^\circ$. For the convenience of observation, we have shown fewer points of GM PNP in the figures. Otherwise, we use a large number of points for efficient fitting while performing actual calculations. We have generated the PM plots using the non-relativistic equations of motion for $a=\pm 0.4$, and $i=20^\circ$ (Figure \ref{fig:GMPNP_fitting_2} and \ref{fig:GMPNP_fitting_4}), for particular values of $\alpha$, $L$, and $E$. We can observe that, for $a=0.4$, most of the orbits are regular with small regions of scattered points at the edges. In comparison to this, the PM plot for $a=-0.4$ is more chaotic as it consists of regular orbits surrounded by a significant area of chaotic regions. The weakening of chaos occurs due to the dragging of the inertial frame around a co-rotating COP which is a direct consequence of the Lense-Thirring Precession of the orbits around a Kerr-like COP \citep{lense1918influence, steklain2009stability, wang2012dynamics}. 

\begin{figure*} 
    
     \centering
     \begin{subfigure}[b]{0.49\textwidth}
         \centering
         \includegraphics[width=\textwidth]{7_1}
         \caption{$a=0.2$, GMf PNP, Non-relativistic}
         \label{fig:PM_comparison_GMPNP_APNP_1}
     \end{subfigure}
     \hfill
     \begin{subfigure}[b]{0.49\textwidth}
         \centering
         \includegraphics[width=\textwidth]{7_2}
         \caption{$a=-0.2$, GMf PNP, Non-relativistic}
         \label{fig:PM_comparison_GMPNP_APNP_2}
     \end{subfigure}
     \hfill
     \begin{subfigure}[b]{0.49\textwidth}
         \centering
         \includegraphics[width=\textwidth]{7_3}
         \caption{$a=0.2$, ABN PNP, Non-relativistic}
         \label{fig:PM_comparison_GMPNP_APNP_3}
     \end{subfigure}
     \hfill
     \begin{subfigure}[b]{0.49\textwidth}
         \centering
         \includegraphics[width=\textwidth]{7_4}
         \caption{$a=-0.2$, ABN PNP, Non-relativistic}
         \label{fig:PM_comparison_GMPNP_APNP_4}
     \end{subfigure}
        \caption{Poincar\'e maps on the cross-sectional plane $z=0$ for the orbits following GMf PNP and ABN PNP with $a=\pm 0.2$, $i=0^\circ$, $\alpha=3\times 10^{-4}$, $E=0.975$, and $L=4.0$. The plots are evaluated without considering any special relativistic corrections.}
        \label{fig:PM_comparison_GMPNP_APNP}
\end{figure*}

We have already compared the effective potentials consisting of the GMf PNP with that of the ABN PNP in the previous section. Now, we have to compare the PM plots generated using the two potentials (Figure \ref{fig:PM_comparison_GMPNP_APNP}). Apparently, we are considering the orbits on the equatorial plane in this case ($i=0^\circ$). We can observe the usual frame-dragging effect (FDE hereafter) for GMf PNP (Figure \ref{fig:PM_comparison_GMPNP_APNP_1} and \ref{fig:PM_comparison_GMPNP_APNP_2}). The chaotic nature of the orbits for $a=-0.2$ is significantly high compared to that for $a=0.2$. The FDE is also evident in the PMs corresponding to the orbits derived from ABN PNP. The number of regular orbits, as well as the number of points in the scattered regions, have been decreased significantly in the map for $a=-0.2$ (\ref{fig:PM_comparison_GMPNP_APNP_4}) compared to that for $a=0.2$ (\ref{fig:PM_comparison_GMPNP_APNP_3}). However, it is to be noted that the degree of chaos in the map for $a=0.2$ evaluated from GMf PNP (Figure \ref{fig:PM_comparison_GMPNP_APNP_1}) is more than that derived from ABN PNP (Figure \ref{fig:PM_comparison_GMPNP_APNP_3}). On the contrary, the chaotic nature of the orbits in the map for $a=-0.2$ evaluated from GMf PNP (Figure \ref{fig:PM_comparison_GMPNP_APNP_2}) is less compared to that derived from ABN PNP (Figure \ref{fig:PM_comparison_GMPNP_APNP_4}). This is the consequence of the result that we presented in section \ref{ch:description_of_PNP}, where we did a comparison between the two PNPs (Figure \ref{fig:comparison_of_potentials_a}). The relative shift in the local minima of the effective potentials affects the chaotic behaviour of the orbits in the way we can see in Figure \ref{fig:PM_comparison_GMPNP_APNP} \citep[][]{nag2017influence}. 

\begin{figure*} 
    
     \centering
     \begin{subfigure}[b]{0.495\textwidth}
         \centering
         \includegraphics[width=\textwidth]{8_1_1}
         \caption{$a=-1.0$, $i=10^\circ$, Non-relativistic}
         \label{fig:PM_GMPNP_with_a_1_1}
     \end{subfigure}
     \hfill
     \begin{subfigure}[b]{0.495\textwidth}
         \centering
         \includegraphics[width=\textwidth]{8_1_4}
         \caption{$a=-0.45$, $i=10^\circ$, Non-relativistic}
         \label{fig:PM_GMPNP_with_a_1_2}
     \end{subfigure}
     \hfill
     \begin{subfigure}[b]{0.495\textwidth}
         \centering
         \includegraphics[width=\textwidth]{8_1_6}
         \caption{$a=-0.1$, $i=10^\circ$, Non-relativistic}
         \label{fig:PM_GMPNP_with_a_1_3}
     \end{subfigure}
     \hfill
     \begin{subfigure}[b]{0.495\textwidth}
         \centering
         \includegraphics[width=\textwidth]{8_2_1}
         \caption{$a=0.1$, $i=10^\circ$, Non-relativistic}
         \label{fig:PM_GMPNP_with_a_1_4}
     \end{subfigure}
     \hfill
     \begin{subfigure}[b]{0.495\textwidth}
         \centering
         \includegraphics[width=\textwidth]{8_2_4}
         \caption{$a=0.6$, $i=10^\circ$, Non-relativistic}
         \label{fig:PM_GMPNP_with_a_1_5}
     \end{subfigure}
     \hfill
     \begin{subfigure}[b]{0.495\textwidth}
         \centering
         \includegraphics[width=\textwidth]{8_2_6}
         \caption{$a=1.0$, $i=10^\circ$, Non-relativistic}
         \label{fig:PM_GMPNP_with_a_1_6}
     \end{subfigure}
        \caption{Poincar\'e maps on the cross-sectional plane $z=0$ for the orbits following GMf PNP with different values of the rotation parameter $a$, and fixed values of orbital parameters, such that $i=10^\circ$, $\alpha=3\times 10^{-4}$, $E=0.976$, and $L=4.85$. The plots are evaluated without considering any special relativistic corrections.}
        \label{fig:PM_GMPNP_with_a_1}
\end{figure*}

In order to observe how the chaoticity of the inclined orbits changes with the rotation of the COP, we studied the chaotic nature of the orbits for the whole range of the Kerr parameter $\left( -1 \leq a \leq 1 \right)$. First, we consider the orbits with a relatively smaller value of inclination ($i=10^\circ$), along with particular values of dipole coefficient $\alpha$, energy $E$, and angular momentum $L$. We observe that the orbits around the counter-rotating COPs (Figure \ref{fig:PM_GMPNP_with_a_1_1}-\ref{fig:PM_GMPNP_with_a_1_3}) are more chaotic than that around the co-rotating ones (Figure \ref{fig:PM_GMPNP_with_a_1_4}-\ref{fig:PM_GMPNP_with_a_1_6}). For any negative value of $a$, we have examined the PM plot for the corresponding positive value of $a$ (e.g. Figure \ref{fig:PM_GMPNP_with_a_1_1} and Figure \ref{fig:PM_GMPNP_with_a_1_6} respectively). The former is apparently more chaotic that the latter for all the cases. This occurs because of the usual FDE. We can study the gradual change in the chaoticity from $a=-1$ to $a=+1$ from Figure \ref{fig:PM_GMPNP_with_a_1}. As the value of $a$ is increasing, we can observe the gradual formation of islands of regular orbits in the lower halves of the plots, which shows more chaotic behaviour than the corresponding upper halves. The allowed volume of the phase space decreases as well. The systematic suppression of chaos with respect to the Kerr parameter is evident. The trend in the chaoticity observed in this case has been corroborated quantitatively using the Lyapunov Characteristic Numbers in section \ref{ch:lyapunov}.

\begin{figure*} 
    
     \centering
     \begin{subfigure}[b]{0.495\textwidth}
         \centering
         \includegraphics[width=\textwidth]{8_3_1}
         \caption{$a=-1.0$, $i=25^\circ$, Non-relativistic}
         \label{fig:PM_GMPNP_with_a_2_1}
     \end{subfigure}
     \hfill
     \begin{subfigure}[b]{0.495\textwidth}
         \centering
         \includegraphics[width=\textwidth]{8_3_4}
         \caption{$a=-0.45$, $i=25^\circ$, Non-relativistic}
         \label{fig:PM_GMPNP_with_a_2_2}
     \end{subfigure}
     \hfill
     \begin{subfigure}[b]{0.495\textwidth}
         \centering
         \includegraphics[width=\textwidth]{8_3_5}
         \caption{$a=-0.25$, $i=25^\circ$, Non-relativistic}
         \label{fig:PM_GMPNP_with_a_2_3}
     \end{subfigure}
     \hfill
     \begin{subfigure}[b]{0.495\textwidth}
         \centering
         \includegraphics[width=\textwidth]{8_3_6}
         \caption{$a=-0.01$, $i=25^\circ$, Non-relativistic}
         \label{fig:PM_GMPNP_with_a_2_4}
     \end{subfigure}
     \hfill
     \begin{subfigure}[b]{0.495\textwidth}
         \centering
         \includegraphics[width=\textwidth]{8_4_2}
         \caption{$a=0.25$, $i=25^\circ$, Non-relativistic}
         \label{fig:PM_GMPNP_with_a_2_5}
     \end{subfigure}
     \hfill
     \begin{subfigure}[b]{0.495\textwidth}
         \centering
         \includegraphics[width=\textwidth]{8_4_4}
         \caption{$a=0.6$, $i=25^\circ$, Non-relativistic}
         \label{fig:PM_GMPNP_with_a_2_6}
     \end{subfigure}
        \caption{Poincar\'e maps on the cross-sectional plane $z=0$ for the orbits following GMf PNP with different values of the rotation parameter $a$, and fixed values of orbital parameters, such that $i=25^\circ$, $\alpha=2\times 10^{-4}$, $E=0.9745$, and $L=5.82$. The plots are evaluated without considering any special relativistic corrections.}
        \label{fig:PM_GMPNP_with_a_2}
\end{figure*}

\begin{figure*} 
    
     \centering
     \begin{subfigure}[b]{0.33\textwidth}
         \centering
         \includegraphics[width=\textwidth]{8_5_1}
         \caption{$a=-0.25$, $i=25^\circ$, Relativistic}
         \label{fig:PM_GMPNP_with_a_5_1}
     \end{subfigure}
     \hfill
     \begin{subfigure}[b]{0.33\textwidth}
         \centering
         \includegraphics[width=\textwidth]{8_5_2}
         \caption{$a=-0.15$, $i=25^\circ$, Relativistic}
         \label{fig:PM_GMPNP_with_a_5_2}
     \end{subfigure}
     \hfill
     \begin{subfigure}[b]{0.33\textwidth}
         \centering
         \includegraphics[width=\textwidth]{8_5_3}
         \caption{$a=-0.09$, $i=25^\circ$, Relativistic}
         \label{fig:PM_GMPNP_with_a_5_3}
     \end{subfigure}
        \caption{Poincar\'e maps on the cross-sectional plane $z=0$ for the orbits following GMf PNP around counter-rotating COPs, $i=25^\circ$, $\alpha=2\times 10^{-4}$, $E=0.965$, and $L=5.82$. The plots are evaluated by considering special relativistic corrections.}
        \label{fig:PM_GMPNP_with_a_5}
\end{figure*}

After considering a smaller value of the inclination angle, we move on to study the orbits with a higher angle of inclination. We considered the off-equatorial orbits with an angle of inclination $i=25^\circ$. We fixed the orbital parameters and varied the Kerr parameter $a$ to generate the PMs. For the orbits around a counter-rotating COP ($a<0$), the cross-sectional maps corresponding to some of the values of $a$ are presented in Figure \ref{fig:PM_GMPNP_with_a_2_1}-\ref{fig:PM_GMPNP_with_a_2_4}. We observe that the overall chaotic behaviour is decreasing as the value of $a$ increases from $-1$ (Figure \ref{fig:PM_GMPNP_with_a_2_1}) to $-0.45$ (Figure \ref{fig:PM_GMPNP_with_a_2_2}). However, the trend in the chaoticity seems to be the opposite for the rest of the maps. If we compare the PM for $a=-0.45$ (Figure \ref{fig:PM_GMPNP_with_a_2_2}) and for $a=-0.25$ (Figure \ref{fig:PM_GMPNP_with_a_2_3}), some of the orbits in the upper halves of the map ($p_\rho >0$) are becoming regular, whereas, in the rest of the region the regular orbits are collapsing and giving rise to scattered points corresponding to the chaotic orbits. The scenario is more evident if we compare the PMs for $a=-0.25$ (Figure \ref{fig:PM_GMPNP_with_a_2_3}) and $a=-0.01$ (Figure \ref{fig:PM_GMPNP_with_a_2_4}), in which case the regular orbits in the lower half of the map for $a=-0.25$ completely converts into chaotic orbits for $a=-0.01$, although the overall change in not very significant. For $a=0.25$ (Figure \ref{fig:PM_GMPNP_with_a_2_5}), few regular orbits can be observed in the lower half of the map. However, the points in the chaotic region become more scattered compared to what we observe for $a=-0.01$ in Figure \ref{fig:PM_GMPNP_with_a_2_4}. For $a=0.6$ (Figure \ref{fig:PM_GMPNP_with_a_2_6}), the regular orbits in the lower half of the map have collapsed, and the points in the chaotic region are becoming even more scattered. Starting from $a=-0.01$ (Figure \ref{fig:PM_GMPNP_with_a_2_4}) to $a=0.6$ (Figure \ref{fig:PM_GMPNP_with_a_2_6}), the overall chaotic nature of the upper halves of the map has not changed significantly. It shows that, for higher values of the Kerr parameter $a$, the degree of chaos has a weak positive correlation with the rotation parameter. As the changes are very small, quantitative analysis is required to confirm the correlation. 

We have observed a similar, and clearer trend in chaoticity for the PMs of special-relativistic orbits. The maps are given in Figure \ref{fig:PM_GMPNP_with_a_5}. Most of the regular orbits in Figure \ref{fig:PM_GMPNP_with_a_5_1} are becoming chaotic in Figure \ref{fig:PM_GMPNP_with_a_5_2} and the chaoticity increases in Figure \ref{fig:PM_GMPNP_with_a_5_3}. 

\begin{figure*} 
    
     \centering
     \begin{subfigure}[b]{0.33\textwidth}
         \centering
         \includegraphics[width=\textwidth]{9_1}
         \caption{$i=17^\circ$, $a=-0.2$, Non-relativistic}
         \label{fig:PM_nonrel_GMPNP_with_i_1}
     \end{subfigure}
     \hfill
     \begin{subfigure}[b]{0.33\textwidth}
         \centering
         \includegraphics[width=\textwidth]{9_2}
         \caption{$i=18.5^\circ$, $a=-0.2$, Non-relativistic}
         \label{fig:PM_nonrel_GMPNP_with_i_2}
     \end{subfigure}
     \hfill
     \begin{subfigure}[b]{0.33\textwidth}
         \centering
         \includegraphics[width=\textwidth]{9_3}
         \caption{$i=20^\circ$, $a=-0.2$, Non-relativistic}
         \label{fig:PM_nonrel_GMPNP_with_i_3}
     \end{subfigure}
     \hfill
     \begin{subfigure}[b]{0.33\textwidth}
         \centering
         \includegraphics[width=\textwidth]{9_4}
         \caption{$i=28^\circ$, $a=-0.15$, Non-relativistic}
         \label{fig:PM_nonrel_GMPNP_with_i_4}
     \end{subfigure}
     \hfill
     \begin{subfigure}[b]{0.33\textwidth}
         \centering
         \includegraphics[width=\textwidth]{9_5}
         \caption{$i=29^\circ$, $a=-0.15$, Non-relativistic}
         \label{fig:PM_nonrel_GMPNP_with_i_5}
     \end{subfigure}
     \hfill
     \begin{subfigure}[b]{0.33\textwidth}
         \centering
         \includegraphics[width=\textwidth]{9_6}
         \caption{$i=30^\circ$, $a=-0.15$, Non-relativistic}
         \label{fig:PM_nonrel_GMPNP_with_i_6}
     \end{subfigure}
        \caption{Poincar\'e maps on the cross-sectional plane $z=0$ for the orbits following GMf PNP with (a)-(c) $a=-0.2$, $\alpha=2\times 10^{-4}$, $E=0.972$, $L=4.6$, and (d)-(f) $a=-0.15$, $\alpha=2\times 10^{-4}$, $E=0.965$, $L=5.45$. The inclination angle $i$ is varied. The plots are evaluated without considering any special relativistic corrections.}
        \label{fig:PM_nonrel_GMPNP_with_i}
\end{figure*}

\begin{figure*} 
    
     \centering
     \begin{subfigure}[b]{0.33\textwidth}
         \centering
         \includegraphics[width=\textwidth]{10_1}
         \caption{$i=24^\circ$, $a=-0.15$, Relativistic}
         \label{fig:PM_rel_GMPNP_with_i_1}
     \end{subfigure}
     \hfill
     \begin{subfigure}[b]{0.33\textwidth}
         \centering
         \includegraphics[width=\textwidth]{10_2}
         \caption{$i=25^\circ$, $a=-0.15$, Relativistic}
         \label{fig:PM_rel_GMPNP_with_i_2}
     \end{subfigure}
     \hfill
     \begin{subfigure}[b]{0.33\textwidth}
         \centering
         \includegraphics[width=\textwidth]{10_3}
         \caption{$i=25.2^\circ$, $a=-0.15$, Relativistic}
         \label{fig:PM_rel_GMPNP_with_i_3}
     \end{subfigure}
        \caption{Poincar\'e maps on the cross-sectional plane $z=0$ for the orbits following GMf PNP with $a=-0.15$, $\alpha=2\times 10^{-4}$, $E=0.965$, and $L=5.82$. The inclination angle $i$ is varied. The plots are evaluated while considering special relativistic corrections.}
        \label{fig:PM_rel_GMPNP_with_i}
\end{figure*}

After finding out the dependence of chaos on the Kerr parameter $a$, we studied the change in the chaotic nature of the orbits with respect to the inclination angle $i$ for both non-relativistic (Figure \ref{fig:PM_nonrel_GMPNP_with_i}) and relativistic (Figure \ref{fig:PM_rel_GMPNP_with_i}) equations of motion. We kept the rest of the parameters unchanged for both cases and varied $i$ to study how the PM plots are getting affected as the angle of inclination increases. We have considered the higher values of the angles. In both cases, the chaotic behaviour of the orbits increases with the angle of inclination $i$. For the non-relativistic orbits with higher inclinations (Figure \ref{fig:PM_nonrel_GMPNP_with_i_4}-\ref{fig:PM_nonrel_GMPNP_with_i_6}), the PM plots change drastically from $i=28^\circ$ to $i=29^\circ$, and from $i=29^\circ$ to $i=30^\circ$. The PM plot for $i=29^\circ$ (Figure \ref{fig:PM_nonrel_GMPNP_with_i_5}) consists of four islands of order, which extinguish in the PM plot for $i=30^\circ$ (Figure \ref{fig:PM_nonrel_GMPNP_with_i_6}), signifying the increase in chaos in the system. In comparison to this, the change of the nature of orbits for smaller inclination angles is more gradual (Figure \ref{fig:PM_nonrel_GMPNP_with_i_1}-\ref{fig:PM_nonrel_GMPNP_with_i_3}). We have already observed the change to be gradual at smaller inclination angles in Figure \ref{fig:PM_with_i_NonRel} and Figure \ref{fig:PM_with_i_Rel} when we studied the PMs corresponding to the off-equatorial orbits around the Schwarzschild-like COPs. This phenomenon will also be explained with the quantitative analysis of chaos in the next section. Similar to the non-relativistic case, the chaotic nature of the relativistic orbits has a positive correlation with the inclination angle $i$ (Figure \ref{fig:PM_rel_GMPNP_with_i}). The PM plot for $i=25^\circ$ (Figure \ref{fig:PM_rel_GMPNP_with_i_2}) consists of two clear stable regions, which get extinguished for the $i=25.2^\circ$ case (Figure \ref{fig:PM_rel_GMPNP_with_i_3}). 

\section{Quantification of chaos using Lyapunov exponent} \label{ch:lyapunov}
As we have studied the chaos of the orbits in a qualitative manner, we proceed to look for a quantitative way of studying these chaotic behaviours. The quantitative analysis of chaos is an absolute necessity. It can provide a magnified picture of chaos present in a system by considering a large number of initial conditions at once. There are many indicators of chaos available in the literature. We have used the scheme of Lyapunov Exponent in this regard as it is easy to implement, it can efficiently distinguish between order and chaos in our system, and it is very much effective in studying the chaotic correlations over a long range of the dynamical parameters. One can measure the exponential divergence of two neighboring orbits while their starting point were very close to each other \citep{strogatz2007nonlinear}. The number of exponents will be equal to the number of dimensions of the phase space. However, in long run (for $t \rightarrow \infty$), the maximum of the exponents dominates. This is known as the Maximum Lyapunov Exponent (MLE) \citep{nag2017influence, de2021beyond}. This can be calculated as
\begin{equation} \label{eq:lpnv_variational}
    \Lambda_{\text{max}} = \lim\limits_{\begin{subarray}{c}
      t \to \infty\\
      ||\xi(0)|| \to 0
      \end{subarray}} \left( \frac{1}{t} \ln \frac{||\xi(t)||}{||\xi(0)||} \right)
\end{equation}
where $||\xi(t)||$ is the norm of the deviation vector of the orbits in the phase space at time $t$. It can be evaluated using the Difference-Hamiltonian and the variational equations \citep{skokos2010numerical}. This method is known as the variational method. Although the method is very accurate, the calculation is too much rigorous with the potential used in this work (equation (\ref{eq:monopole})). As an alternative, we have used the two-particle method which is a more convenient and less rigorous way of calculating MLE. 

In the two-particle method, we consider two initial points which are infinitesimally close to each other. Thereafter, the orbits are made to evolve independently starting from those initial conditions and the norm of the final deviation between the orbits is measured \citep{de2021beyond}. The mathematical formula given in equation (\ref{eq:lpnv_variational}), in this case, is modified according to the following.
\begin{equation} \label{eq:lpnv_two_particle}
    \Lambda_{\text{max}} = \lim\limits_{\begin{subarray}{c}
      t \to \infty\\
      ||\delta x(0)|| \to 0
      \end{subarray}} \left( \frac{1}{t} \ln \frac{||\delta x(t)||}{||\delta x(0)||} \right)
\end{equation}
where $||\delta x(t)||$ is the norm of the deviation between the two neighboring orbits in the phase space at time $t$. From the numerical point of view, the mean rate of deviation is calculated using the following formula.
\begin{equation} \label{eq:lpnv_two_particle_mean_rate}
    \Lambda_{\text{max}} = \frac{1}{n \tau} \sum_{k=1}^n \ln \frac{||\delta x(k \tau)||}{||\delta x(0)||}
\end{equation}

In order to achieve accurate results, the initial deviation has to be very small, and the precision of the computational instrument has to be high. Furthermore, the orbits diverge very quickly in the two-particle method. According to the definition of MLE, in order to get legitimate exponents, the deviation $\delta x(t)$ has to remain much within the allowable volume in the phase space. Therefore, we need to re-normalize the deviated orbit after every time step $\tau$ using the Gram-Schmidt renormalization scheme \citep{tancredi2001comparison}. 

Although the MLEs can be implemented to quantify chaos for individual orbits evolving from particular initial conditions, they still cannot be used as an overall quantitative measure of chaos for a set of parameters. As a solution, we can consider a large number of initial conditions evenly distributed over the allowed phase space and calculate MLEs for each one of them. Thereafter, we can take the average of all the exponents and consider this as a rough quantitative measure of the overall chaotic behaviour of the system for a given set of parameters. In practice, this is known as the Lyapunov Characteristic Number (LCN) of the system. Ideally, as we increase the number of initial conditions, we acquire a more accurate value of LCN. 

In the present work, we have calculated the values of each LCN by taking the average of several MLEs corresponding to different sets of initial conditions. For each set of the initial conditions, the initial deviation has been taken to be $\delta x(0)=10^{-8}$, and the calculation has been carried out for $n=10^5$ iterations with renormalization-time-step $\tau=0.1$. The value of $\Lambda_{\text{max}}$ is smaller than $5 \times 10^{-4}$ for regular orbits and it is more than $5 \times 10^{-4}$ for chaotic orbits.   

By implementing LCN as a quantitative indicator of chaos, we have studied the correlation of chaos with the Kerr parameter $a$ of the COP and the inclination angle $i$ of the orbit. First, let us consider the former (Figure \ref{fig:lcn_with_a}). Looking at the overall trends of LCNs, it is evident that the inclined orbits are more chaotic for the maximally counter-rotating COPs, i.e. when the value of $a$ is closer to -1. For lower values of the inclination angle, such as $i=10^\circ$ (Figure \ref{fig:lcn_with_a_1}) or $i=15^\circ$ (Figure \ref{fig:lcn_with_a_2}), the chaoticity anti-correlates with $a$ for the whole range of it. For higher angles of inclination, the correlation becomes nuanced. It turns out that the degree of chaos has a weak dependence on the Kerr parameter $a$ for most of its range. However, the chaoticity rapidly increases below a threshold value of the rotation parameter $a=a_{\text{c}}$. For $i=25^\circ$ (Figure \ref{fig:lcn_with_a_3}), the threshold value is $a_{\text{c}} \approx -0.85$. For $i=27^\circ$ (Figure \ref{fig:lcn_with_a_4}), the threshold value comes out to be $a_{\text{c}} \approx -0.55$. It is happening because, for the highly inclined orbits, the chaoticity of the system is mostly dominated by the angle of inclination when $a>a_{\text{c}}$. That is why the chaoticity is not changing much in this regime even if the value of $a$ is decreasing. When the value of $a$ goes below the threshold value ($a<a_{\text{c}}$), the Kerr parameter starts showing its influence on the system and its chaoticity shows a sharp, negative correlation as we have seen for the lower inclination angles. 

The trend in Figure \ref{fig:lcn_with_a_1} corroborates the correlation established with PMs in Figure \ref{fig:PM_GMPNP_with_a_1}. Similarly, the trend in Figure \ref{fig:lcn_with_a_3} is in agreement with the correlation established in Figure \ref{fig:PM_GMPNP_with_a_2}. It is to be noted that for $i=25^\circ$ (Figure \ref{fig:lcn_with_a_3}), the degree of chaos shows a weak positive correlation for a significant range of the rotation parameter. It was also observed through the PMs in the previous section. Although this weak positive correlation at the higher values of the rotation parameter ($a>a_{\text{c}}$) can be observed for some of the high values of $i$ other than $i=25^\circ$, it is not true for all the highly inclined orbits. For example, if the angle of inclination is $i=27^\circ$ (Figure \ref{fig:lcn_with_a_4}), we can observe that the chaoticity has a weak negative correlation with $a$ when $a>a_{\text{c}}$. Furthermore, it is to be noted that the dependence of chaoticity on the Kerr parameter for $a>a_{\text{c}}$ becomes weaker as $i$ increases from $15^\circ$ to $25^\circ$ to $27^\circ$. For $i=27^\circ$, the change in LCN with respect to $a$ is very small when $a>a_{\text{c}}$ (Figure \ref{fig:lcn_with_a_4}). 

On the other hand, the dependence of chaoticity on the inclination angle $i$ is more straightforward, consistent, and intuitive as well (Figure \ref{fig:lcn_with_i}). The degree of chaos is increasing in the system as the orbits are becoming more inclined with the rotation axis of the COP. It implies that the thicker accretion disks consist of more chaotic orbits compared to the thinner ones. However, the change in the LCN is more rapid at the higher values of the inclination angle. This was also observed earlier while studying the chaotic behaviour of the orbits qualitatively using PM plots (Figure \ref{fig:PM_nonrel_GMPNP_with_i}). 

It is to be noted that there is a sudden increase in the values of LCNs at some threshold value of the inclination angle $i=i_{\text{c}}$ and this is observed for all the possible sets of orbital parameters (Figure \ref{fig:lcn_with_i}). For $i<i_{\text{c}}$, the increase in chaoticity is very gradual. As the value of $i$ just crosses $i_{\text{c}}$, the LCNs start increasing rapidly until it reaches an angle after which the growth slows down a little, though the rate of increment is still more than what it was below $i_{\text{c}}$. Now, if we compare Figure \ref{fig:lcn_with_i_1} and Figure \ref{fig:lcn_with_i_2}, the threshold value has come down from $i_{\text{c}} \approx 21^\circ$ to $i_{\text{c}} \approx 18^\circ$ as the dipole coefficient increases from $\alpha = 2 \times 10^{-4}$ to $\alpha = 2.2 \times 10^{-4}$. Similarly, if we compare Figure \ref{fig:lcn_with_i_1} and Figure \ref{fig:lcn_with_i_3}, the threshold value has increased from $i_{\text{c}} \approx 21^\circ$ to $i_{\text{c}} \approx 23.5^\circ$ as the energy $E$ decreased and the angular momentum $L$ increased. These results imply that the value of $i_{\text{c}}$ has an anti-correlation with the degree of chaos present in the system. As $\alpha$ increases, and/or $E$ increases, and/or $L$ decreases, the nonlinearity in the system gets enhanced, and the value of $i_{\text{c}}$ gets lowered. 

\begin{figure*} 
    
     \centering
     \begin{subfigure}[b]{0.49\textwidth}
         \centering
         \includegraphics[width=\textwidth]{12_1}
         \caption{$i=10^\circ$, $E=0.976$, $L=4.85$, and $\alpha=3\times 10^{-4}$}
         \label{fig:lcn_with_a_1}
     \end{subfigure}
     \hfill
     \begin{subfigure}[b]{0.49\textwidth}
         \centering
         \includegraphics[width=\textwidth]{12_2}
         \caption{$i=15^\circ$, $E=0.976$, $L=4.85$, and $\alpha=3\times 10^{-4}$}
         \label{fig:lcn_with_a_2}
     \end{subfigure}
     \hfill
     \begin{subfigure}[b]{0.49\textwidth}
         \centering
         \includegraphics[width=\textwidth]{12_3}
         \caption{$i=25^\circ$, $E=0.9745$, $L=5.82$, and $\alpha=2\times 10^{-4}$}
         \label{fig:lcn_with_a_3}
     \end{subfigure}
     \hfill
     \begin{subfigure}[b]{0.49\textwidth}
         \centering
         \includegraphics[width=\textwidth]{12_4}
         \caption{$i=27^\circ$, $E=0.9745$, $L=5.82$, and $\alpha=2\times 10^{-4}$}
         \label{fig:lcn_with_a_4}
     \end{subfigure}
        \caption{Lyapunov Characteristic Exponent ($\Lambda_{\text{av}}$) as a function of Kerr Parameter $a$.}
        \label{fig:lcn_with_a}
\end{figure*}

\begin{figure*} 
    
     \centering
     \begin{subfigure}[b]{0.49\textwidth}
         \centering
         \includegraphics[width=\textwidth]{13_1}
         \caption{$a=-0.15$, $E=0.976$, $L=4.85$, and $\alpha=2\times 10^{-4}$}
         \label{fig:lcn_with_i_1}
     \end{subfigure}
     \hfill
     \begin{subfigure}[b]{0.49\textwidth}
         \centering
         \includegraphics[width=\textwidth]{13_2}
         \caption{$a=-0.15$, $E=0.976$, $L=4.85$, and $\alpha=2.2\times 10^{-4}$}
         \label{fig:lcn_with_i_2}
     \end{subfigure}
     \hfill
     \begin{subfigure}[b]{0.49\textwidth}
         \centering
         \includegraphics[width=\textwidth]{13_3}
         \caption{$a=-0.15$, $E=0.974$, $L=4.95$, and $\alpha=2\times 10^{-4}$}
         \label{fig:lcn_with_i_3}
     \end{subfigure}
     \hfill
     \begin{subfigure}[b]{0.49\textwidth}
         \centering
         \includegraphics[width=\textwidth]{13_4}
         \caption{$a=-0.925$, $E=0.9745$, $L=5.82$, and $\alpha=2\times 10^{-4}$}
         \label{fig:lcn_with_i_4}
     \end{subfigure}
        \caption{Lyapunov Characteristic Exponent ($\Lambda_{\text{av}}$) as a function of inclination angle $i$.}
        \label{fig:lcn_with_i}
\end{figure*}

\section{Conclusions} \label{ch:conclusion}
In the present piece of work, we have studied the chaotic behaviour of the off-equatorial orbits around a Pseudo-Newtonian COP using the generalized force presented in \cite{ghosh2007}. Because of the complex mathematical form of the pseudo-Keplerian force, we have prescribed a numerical method in which a fitting function is used to generate the PNP and implement it in the analysis of orbital dynamics. In order to incorporate a more realistic scenario, we have introduced the dipolar perturbative term which corresponds to an asymmetrically placed hollow halo of matter around the COP. To study the chaotic dynamics of the off-equatorial orbits around the COP, we have implemented the Poincar\'e Maps and Lyapunov Characteristic Numbers as the indicators of chaos. Where Poincar\'e maps help to visualize chaos qualitatively, Lyapunov Characteristic Number quantifies the degree of chaos. 

We studied the chaotic dynamics of the equatorial orbits governed by the GMf PNP and compared them with that governed by the ABN PNP. We saw that for the orbits around the co-rotating COPs, the GMf PNP induces more chaos into the system. For the orbits around the counter-rotating COPs, the effect is the opposite, i.e. the ABN PNP makes the equatorial orbits more chaotic compared to the GMf PNP. The FDE is visible in the case of both the PNPs. 

While studying the correlation of chaos with respect to the rotation parameter $a$, we observed that the chaoticity is maximum for the orbits around the maximally counter-rotating COPs. For lower values of the inclination angles, the degree of chaos shows a negative correlation with $a$. Hence, the chaoticity gradually decreases as $a$ increases. However, for higher orbital inclinations, the degree of chaos shows a weak dependence on $a$ at its higher values, and the chaoticity is mostly dominated by the inclination angle in this regime. It is only when $a$ decreases below a threshold value $a_{\text{c}}$, the chaoticity gets dependent on $a$ and it begins to increase as $a$ gets lowered. At $a>a_{\text{c}}$, the chaotic behaviour of the system shows a weak negative correlation for most of the values of $i$. However, for some of the orbital inclinations, the chaoticity shows a weak positive correlation with respect to $a$ at $a>a_{\text{c}}$. Further investigation is required to identify a trend, if there exists one, and predict at which values of $i$ the chaoticity might show a weak positive correlation in the range of $a>a_{\text{c}}$. However, it has been observed that as $i$ increases, the dependence of chaoticity on $a$ at $a>a_{\text{c}}$ becomes weaker. 

We have also established that the chaoticity increases as the orbits become more inclined with the equatorial plane. This has been shown for both Schwarzschild and Kerr-like COPs. The Poincar\'e maps reveal that the change in the chaotic nature of the orbits is more rapid at higher inclination angles than the lower ones in which cases the changes are more gradual. The reason has been explored using the trend in the LCNs. We have observed that the LCNs show a sudden sharp upturn at a certain threshold value of the inclination angle $i=i_{\text{c}}$. For orbits with inclination $i>i_{\text{c}}$, the change in the degree of chaos is more rapid, and for $i<i_{\text{c}}$, the chaoticity changes gradually. Therefore, we can state that for $i>i_{\text{c}}$, even a small change in the inclination angle can affect the chaotic behaviour of the orbits. The value of $i_{\text{c}}$ anti-correlates with the nonlinearity in the system. It increases when the energy $E$ decreases, and/or the angular momentum $L$ increases, and/or $\alpha$ decreases. 

In future, we would like to analyze the fitting function, given in equation \ref{eq:GMPNP}, in more depth. It will be interesting to study how the fitting parameters $\{ \gamma_1,...,\gamma_5 \}$ get affected by the rotation parameter $a$ and the inclination angle $i$. The study can lead to a concrete analytical form of the fitting function, converting it into an independent PNP useful for the inclined orbits around COPs. We would also like to perform an in-depth analysis of the dynamics of chaos in the current system using other indicators of chaos, namely the Fast Lyapunov Index or FLI \citep{froeschle1997fast, wang2012dynamics}, Small Alignment Index or SALI \citep{skokos2001alignment, skokos2004detecting}, and General Alignment Index or GALI \citep{skokos2007geometrical}, which are more sensitive to chaos than MLE, making them more efficient to distinguish between order and chaos. We would also like to investigate how the value of $i_{\text{c}}$ varies with the dipole coefficient $\alpha$, energy $E$, and angular momentum $L$, in a quantitative manner. Because of the generic nature of the GM PNP, it is customary to analyze the chaotic dynamics of the equatorial orbits around a Pseudo-Kerr COP and compare it with the ABN PNP prescribed in \citet{artemova1996modified} in the presence of other perturbative terms corresponding to the quadrupolar and the octapolar halos. We would also like to implement the PNP and the numerical scheme used in this article to study the off-axis motion of a test particle in a restricted three-body system, where the other two bodies are pseudo-Newtonian compact object binaries.

\section*{Acknowledgements}

The authors would like to acknowledge Dr. Sankhasubhra Nag for his helpful suggestions and discussions. The authors would also like to acknowledge Ms. Roopkatha Banerjee and Mr. Samantak Kundu for their help in the computational work.

%%%%%%%%%%%%%%%%%%%%%%%%%%%%%%%%%%%%%%%%%%%%%%%%%%
\section*{Data Availability}

 
No new data were generated or analyzed in support of this research.




%%%%%%%%%%%%%%%%%%%% REFERENCES %%%%%%%%%%%%%%%%%%

% The best way to enter references is to use BibTeX:

\bibliographystyle{mnras}
%\bibliography{reference} % if your bibtex file is called example.bib
\begin{thebibliography}{}
\makeatletter
\relax
\def\mn@urlcharsother{\let\do\@makeother \do\$\do\&\do\#\do\^\do\_\do\%\do\~}
\def\mn@doi{\begingroup\mn@urlcharsother \@ifnextchar [ {\mn@doi@}
  {\mn@doi@[]}}
\def\mn@doi@[#1]#2{\def\@tempa{#1}\ifx\@tempa\@empty \href
  {http://dx.doi.org/#2} {doi:#2}\else \href {http://dx.doi.org/#2} {#1}\fi
  \endgroup}
\def\mn@eprint#1#2{\mn@eprint@#1:#2::\@nil}
\def\mn@eprint@arXiv#1{\href {http://arxiv.org/abs/#1} {{\tt arXiv:#1}}}
\def\mn@eprint@dblp#1{\href {http://dblp.uni-trier.de/rec/bibtex/#1.xml}
  {dblp:#1}}
\def\mn@eprint@#1:#2:#3:#4\@nil{\def\@tempa {#1}\def\@tempb {#2}\def\@tempc
  {#3}\ifx \@tempc \@empty \let \@tempc \@tempb \let \@tempb \@tempa \fi \ifx
  \@tempb \@empty \def\@tempb {arXiv}\fi \@ifundefined
  {mn@eprint@\@tempb}{\@tempb:\@tempc}{\expandafter \expandafter \csname
  mn@eprint@\@tempb\endcsname \expandafter{\@tempc}}}

\bibitem[\protect\citeauthoryear{Alrebdi, Papadakis, Dubeibe  \& Zotos}{Alrebdi
  et~al.}{2022}]{alrebdi2022equilibrium}
Alrebdi H.,  Papadakis K.~E.,  Dubeibe F.~L.,   Zotos E.~E.,  2022, Astron. J.,
  163, 75

\bibitem[\protect\citeauthoryear{Arnaboldi, Capaccioli, Cappellaro, Held  \&
  Sparke}{Arnaboldi et~al.}{1993}]{arnaboldi1993studies}
Arnaboldi M.,  Capaccioli M.,  Cappellaro E.,  Held E.~V.,   Sparke L.,  1993,
  A\&A, 267, 21

\bibitem[\protect\citeauthoryear{Artemova, Bj{\"o}rnsson  \& Novikov}{Artemova
  et~al.}{1996}]{artemova1996modified}
Artemova I.~V.,  Bj{\"o}rnsson G.,   Novikov I.~D.,  1996, ApJ, 461, 565

\bibitem[\protect\citeauthoryear{Beckmann \& Shrader}{Beckmann \&
  Shrader}{2012}]{beckmann2012active}
Beckmann V.,  Shrader C.,  2012, Active galactic nuclei.
John Wiley \& Sons

\bibitem[\protect\citeauthoryear{Berry}{Berry}{1978}]{berry1978topics}
Berry M.~V.,  1978, Am. Inst. Phys. Conf. Proc., 46, 16

\bibitem[\protect\citeauthoryear{Bhattacharya, Ghosh  \&
  Mukhopadhyay}{Bhattacharya et~al.}{2010}]{bhattacharya2010disk}
Bhattacharya D.,  Ghosh S.,   Mukhopadhyay B.,  2010, ApJ, 713, 105

\bibitem[\protect\citeauthoryear{Binney \& Tremaine}{Binney \&
  Tremaine}{2008}]{binney2008}
Binney J.,  Tremaine S.,  2008, Galactic dynamics, 2nd edn.
Princeton University Press, USA

\bibitem[\protect\citeauthoryear{Chakrabarti \& Khanna}{Chakrabarti \&
  Khanna}{1992}]{chakrabarti1992newtonian}
Chakrabarti S.~K.,  Khanna R.,  1992, MNRAS, 256, 300

\bibitem[\protect\citeauthoryear{Chakrabarti \& Mondal}{Chakrabarti \&
  Mondal}{2006}]{chakrabarti2006studies}
Chakrabarti S.~K.,  Mondal S.,  2006, MNRAS, 369, 976

\bibitem[\protect\citeauthoryear{Chen \& Wang}{Chen \&
  Wang}{2003}]{chen2003chaotic}
Chen J.,  Wang Y.,  2003, Class. Quant. Grav., 20, 3897

\bibitem[\protect\citeauthoryear{Daly}{Daly}{2011}]{daly2011estimates}
Daly R.~A.,  2011, MNRAS, 414, 1253

\bibitem[\protect\citeauthoryear{De, Roychowdhury  \& Banerjee}{De
  et~al.}{2021}]{de2021beyond}
De S.,  Roychowdhury S.,   Banerjee R.,  2021, MNRAS, 501, 713

\bibitem[\protect\citeauthoryear{Dotti, Colpi, Pallini, Perego  \&
  Volonteri}{Dotti et~al.}{2012}]{dotti2012orientation}
Dotti M.,  Colpi M.,  Pallini S.,  Perego A.,   Volonteri M.,  2012, ApJ, 762,
  68

\bibitem[\protect\citeauthoryear{Dubeibe, Lora-Clavijo  \&
  Gonz{\'a}lez}{Dubeibe et~al.}{2017}]{dubeibe2017pseudo}
Dubeibe F.,  Lora-Clavijo F.,   Gonz{\'a}lez G.~A.,  2017, Phys. Lett. A, 381,
  563

\bibitem[\protect\citeauthoryear{Dubeibe, Saeed  \& Zotos}{Dubeibe
  et~al.}{2021}]{dubeibe2021effect}
Dubeibe F.~L.,  Saeed T.,   Zotos E.~E.,  2021, ApJ, 908, 74

\bibitem[\protect\citeauthoryear{Dupraz \& Combes}{Dupraz \&
  Combes}{1987}]{dupraz1987dynamical}
Dupraz C.,  Combes F.,  1987, A\&A, 185, L1

\bibitem[\protect\citeauthoryear{Froeschl{\'e}, Lega  \& Gonczi}{Froeschl{\'e}
  et~al.}{1997}]{froeschle1997fast}
Froeschl{\'e} C.,  Lega E.,   Gonczi R.,  1997, Celest. Mech. Dyn. Astr., 67,
  41

\bibitem[\protect\citeauthoryear{Garofalo}{Garofalo}{2013}]{garofalo2013retrograde}
Garofalo D.,  2013, Advances in Astronomy, 2013, 213105

\bibitem[\protect\citeauthoryear{Ghosh}{Ghosh}{2004}]{ghosh2004rotating}
Ghosh S.,  2004, A\&A, 418, 795

\bibitem[\protect\citeauthoryear{Ghosh \& Mukhopadhyay}{Ghosh \&
  Mukhopadhyay}{2007}]{ghosh2007}
Ghosh S.,  Mukhopadhyay B.,  2007, ApJ, 667, 367

\bibitem[\protect\citeauthoryear{Ghosh, Sarkar  \& Bhadra}{Ghosh
  et~al.}{2014}]{ghosh2014newtonian}
Ghosh S.,  Sarkar T.,   Bhadra A.,  2014, MNRAS, 445, 4460

\bibitem[\protect\citeauthoryear{Goldstein, Pool  \& Safko}{Goldstein
  et~al.}{2001}]{goldstein2001classical}
Goldstein H.,  Pool C.,   Safko J.,  2001, Classical Mechanics, 3rd edn.
Addison Wesley, USA

\bibitem[\protect\citeauthoryear{Gu{\'e}ron \& Letelier}{Gu{\'e}ron \&
  Letelier}{2001a}]{gueron2001chaotic}
Gu{\'e}ron E.,  Letelier P.~S.,  2001a, Phys. Rev. E, 63, 035201

\bibitem[\protect\citeauthoryear{Gueron \& Letelier}{Gueron \&
  Letelier}{2001b}]{gueron2001chaos}
Gueron E.,  Letelier P.~S.,  2001b, A\&A, 368, 716

\bibitem[\protect\citeauthoryear{Gu{\'e}ron \& Letelier}{Gu{\'e}ron \&
  Letelier}{2002}]{gueron2002geodesic}
Gu{\'e}ron E.,  Letelier P.~S.,  2002, Phys. Rev. E, 66, 046611

\bibitem[\protect\citeauthoryear{Healy, Lousto  \& Zlochower}{Healy
  et~al.}{2014}]{healy2014remnant}
Healy J.,  Lousto C.~O.,   Zlochower Y.,  2014, Phys. Rev. D, 90, 104004

\bibitem[\protect\citeauthoryear{Janiuk \& Czerny}{Janiuk \&
  Czerny}{2011}]{janiuk2011different}
Janiuk A.,  Czerny B.,  2011, MNRAS, 414, 2186

\bibitem[\protect\citeauthoryear{Kop{\'a}{\v{c}}ek \& Karas}{Kop{\'a}{\v{c}}ek
  \& Karas}{2014}]{kopavcek2014inducing}
Kop{\'a}{\v{c}}ek O.,  Karas V.,  2014, ApJ, 787, 117

\bibitem[\protect\citeauthoryear{Kop{\'a}{\v{c}}ek \& Karas}{Kop{\'a}{\v{c}}ek
  \& Karas}{2015}]{kopavcek2015regular}
Kop{\'a}{\v{c}}ek O.,  Karas V.,  2015, Journal of Physics: Conference Series,
  600, 012070

\bibitem[\protect\citeauthoryear{Kormendy \& Richstone}{Kormendy \&
  Richstone}{1995}]{kormendy1995inward}
Kormendy J.,  Richstone D.,  1995, Annual review of A\&A, 33, 581

\bibitem[\protect\citeauthoryear{Kov{\'a}{\v{r}}, Stuchl{\'\i}k  \&
  Karas}{Kov{\'a}{\v{r}} et~al.}{2008}]{kovavr2008off}
Kov{\'a}{\v{r}} J.,  Stuchl{\'\i}k Z.,   Karas V.,  2008, Class. Quant. Grav.,
  25, 095011

\bibitem[\protect\citeauthoryear{Kov{\'a}{\v{r}}, Kop{\'a}{\v{c}}ek, Karas  \&
  Stuchl{\'\i}k}{Kov{\'a}{\v{r}} et~al.}{2010}]{kovavr2010off}
Kov{\'a}{\v{r}} J.,  Kop{\'a}{\v{c}}ek O.,  Karas V.,   Stuchl{\'\i}k Z.,
  2010, Class. Quant. Grav., 27, 135006

\bibitem[\protect\citeauthoryear{Lense \& Thirring}{Lense \&
  Thirring}{1918}]{lense1918influence}
Lense J.,  Thirring H.,  1918, Phys. Z., 19, 156

\bibitem[\protect\citeauthoryear{Letelier \& Vieira}{Letelier \&
  Vieira}{1997}]{letelier1997chaos}
Letelier P.,  Vieira W.,  1997, Phys. Rev. D, 56, 8095

\bibitem[\protect\citeauthoryear{Letelier, Ramos-Caro  \&
  L{\'o}pez-Suspes}{Letelier et~al.}{2011}]{letelier2011chaotic}
Letelier P.~S.,  Ramos-Caro J.,   L{\'o}pez-Suspes F.,  2011, Phys. Lett. A,
  375, 3655

\bibitem[\protect\citeauthoryear{Malin \& Carter}{Malin \&
  Carter}{1983}]{malin1983catalog}
Malin D.,  Carter D.,  1983, ApJ, 274, 534

\bibitem[\protect\citeauthoryear{Meyer}{Meyer}{1997}]{meyer1997formation}
Meyer F.,  1997, MNRAS, 285, L11

\bibitem[\protect\citeauthoryear{Miller, Reynolds, Fabian, Miniutti  \&
  Gallo}{Miller et~al.}{2009}]{miller2009stellar}
Miller J.~M.,  Reynolds C.~S.,  Fabian A.~C.,  Miniutti G.,   Gallo L.~C.,
  2009, ApJ, 697, 900

\bibitem[\protect\citeauthoryear{Mukhopadhyay}{Mukhopadhyay}{2002}]{mukhopadhyay2002description}
Mukhopadhyay B.,  2002, ApJ, 581, 427

\bibitem[\protect\citeauthoryear{Mukhopadhyay \& Misra}{Mukhopadhyay \&
  Misra}{2003}]{mukhopadhyay2003pseudo}
Mukhopadhyay B.,  Misra R.,  2003, ApJ, 582, 347

\bibitem[\protect\citeauthoryear{Nag, Sinha, Ananda  \& Das}{Nag
  et~al.}{2017}]{nag2017influence}
Nag S.,  Sinha S.,  Ananda D.~B.,   Das T.~K.,  2017, Ap\&SS, 362, 81

\bibitem[\protect\citeauthoryear{Nowak \& Wagoner}{Nowak \&
  Wagoner}{1991}]{nowak1991diskoseismology}
Nowak M.~A.,  Wagoner R.~V.,  1991, ApJ, 378, 656

\bibitem[\protect\citeauthoryear{Paczynsky \& Wiita}{Paczynsky \&
  Wiita}{1980}]{paczynsky1980thick}
Paczynsky B.,  Wiita P.~J.,  1980, A\&A, 88, 23

\bibitem[\protect\citeauthoryear{Panagia, Scuderi, Gilmozzi, Challis, Garnavich
   \& Kirshner}{Panagia et~al.}{1996}]{panagia1996nature}
Panagia N.,  Scuderi S.,  Gilmozzi R.,  Challis P.,  Garnavich P.,   Kirshner
  R.,  1996, ApJ, 459, L17

\bibitem[\protect\citeauthoryear{Polcar, Sukov{\'a}  \& Semer{\'a}k}{Polcar
  et~al.}{2019}]{polcar2019free}
Polcar L.,  Sukov{\'a} P.,   Semer{\'a}k O.,  2019, ApJ, 877, 16

\bibitem[\protect\citeauthoryear{Quinn}{Quinn}{1984}]{quinn1984formation}
Quinn P.~J.,  1984, ApJ, 279, 596

\bibitem[\protect\citeauthoryear{Reshetnikov \& Sotnikova}{Reshetnikov \&
  Sotnikova}{1997}]{reshetnikov1997global}
Reshetnikov V.,  Sotnikova N.,  1997, arXiv preprint astro-ph/9704047

\bibitem[\protect\citeauthoryear{Reynolds, Brenneman, Lohfink, Trippe, Miller,
  Reis, Nowak  \& Fabian}{Reynolds et~al.}{2012}]{reynolds2012probing}
Reynolds C.~S.,  Brenneman L.~W.,  Lohfink A.~M.,  Trippe M.~L.,  Miller J.~M.,
   Reis R.~C.,  Nowak M.~A.,   Fabian A.~C.,  2012, AIP Conference Proceedings,
  1427, 157

\bibitem[\protect\citeauthoryear{Sackett \& Sparke}{Sackett \&
  Sparke}{1990}]{sackett1990dark}
Sackett P.~D.,  Sparke L.~S.,  1990, ApJ, 361, 408

\bibitem[\protect\citeauthoryear{{Semer{\'a}k} \& {Karas}}{{Semer{\'a}k} \&
  {Karas}}{1999}]{Semerak1999pseudo}
{Semer{\'a}k} O.,  {Karas} V.,  1999, A\&A, 343, 325

\bibitem[\protect\citeauthoryear{Semer{\'a}k \& Sukov{\'a}}{Semer{\'a}k \&
  Sukov{\'a}}{2010}]{semerak2010free}
Semer{\'a}k O.,  Sukov{\'a} P.,  2010, MNRAS, 404, 545

\bibitem[\protect\citeauthoryear{Semer{\'a}k \& Sukov{\'a}}{Semer{\'a}k \&
  Sukov{\'a}}{2012}]{semerak2012free}
Semer{\'a}k O.,  Sukov{\'a} P.,  2012, MNRAS, 425, 2455

\bibitem[\protect\citeauthoryear{Sesana, Barausse, Dotti  \& Rossi}{Sesana
  et~al.}{2014}]{sesana2014linking}
Sesana A.,  Barausse E.,  Dotti M.,   Rossi E.~M.,  2014, ApJ, 794, 104

\bibitem[\protect\citeauthoryear{Skokos}{Skokos}{2001}]{skokos2001alignment}
Skokos C.,  2001, Journal of Physics A: Mathematical and General, 34, 10029

\bibitem[\protect\citeauthoryear{Skokos \& Gerlach}{Skokos \&
  Gerlach}{2010}]{skokos2010numerical}
Skokos C.,  Gerlach E.,  2010, Phys. Rev. E, 82, 036704

\bibitem[\protect\citeauthoryear{Skokos, Antonopoulos, Bountis  \&
  Vrahatis}{Skokos et~al.}{2004}]{skokos2004detecting}
Skokos C.,  Antonopoulos C.,  Bountis T.,   Vrahatis M.,  2004, Journal of
  Physics A: Mathematical and General, 37, 6269

\bibitem[\protect\citeauthoryear{Skokos, Bountis  \& Antonopoulos}{Skokos
  et~al.}{2007}]{skokos2007geometrical}
Skokos C.,  Bountis T.,   Antonopoulos C.,  2007, Physica D: Nonlinear
  Phenomena, 231, 30

\bibitem[\protect\citeauthoryear{Steklain \& Letelier}{Steklain \&
  Letelier}{2009}]{steklain2009stability}
Steklain A.,  Letelier P.,  2009, Phys. Lett. A, 373, 188

\bibitem[\protect\citeauthoryear{Strogatz}{Strogatz}{2007}]{strogatz2007nonlinear}
Strogatz S.~H.,  2007, Nonlinear dynamics and chaos: with applications to
  physics, biology, chemistry, and engineering.
Sarat Book House, Kolkata

\bibitem[\protect\citeauthoryear{Sukov{\'a} \& Semer{\'a}k}{Sukov{\'a} \&
  Semer{\'a}k}{2013}]{sukova2013free}
Sukov{\'a} P.,  Semer{\'a}k O.,  2013, MNRAS, 436, 978

\bibitem[\protect\citeauthoryear{Takahashi \& Koyama}{Takahashi \&
  Koyama}{2009}]{takahashi2009chaotic}
Takahashi M.,  Koyama H.,  2009, ApJ, 693, 472

\bibitem[\protect\citeauthoryear{Tancredi, S{\'a}nchez  \& Roig}{Tancredi
  et~al.}{2001}]{tancredi2001comparison}
Tancredi G.,  S{\'a}nchez A.,   Roig F.,  2001, AJ, 121, 1171

\bibitem[\protect\citeauthoryear{Tchekhovskoy \& McKinney}{Tchekhovskoy \&
  McKinney}{2012}]{tchekhovskoy2012prograde}
Tchekhovskoy A.,  McKinney J.~C.,  2012, MNRAS: Letters, 423, L55

\bibitem[\protect\citeauthoryear{Vieira \& Letelier}{Vieira \&
  Letelier}{1996}]{vieira1996chaos}
Vieira W.~M.,  Letelier P.~S.,  1996, Phys. Rev. Lett., 76, 1409

\bibitem[\protect\citeauthoryear{Vieira \& Letelier}{Vieira \&
  Letelier}{1999}]{vieira1999relativistic}
Vieira W.~M.,  Letelier P.~S.,  1999, ApJ, 513, 383

\bibitem[\protect\citeauthoryear{Vogt \& Letelier}{Vogt \&
  Letelier}{2003}]{vogt2003exact}
Vogt D.,  Letelier P.~S.,  2003, Phys. Rev. D, 68, 084010

\bibitem[\protect\citeauthoryear{Wang \& Wu}{Wang \&
  Wu}{2012}]{wang2012dynamics}
Wang Y.,  Wu X.,  2012, Chinese Physics B, 21, 050504

\bibitem[\protect\citeauthoryear{Witzany, Semer{\'a}k  \& Sukov{\'a}}{Witzany
  et~al.}{2015}]{witzany2015free}
Witzany V.,  Semer{\'a}k O.,   Sukov{\'a} P.,  2015, MNRAS, 451, 1770

\bibitem[\protect\citeauthoryear{Ziolkowski}{Ziolkowski}{2010}]{ziolkowski2010population}
Ziolkowski J.,  2010, Memorie della Societ{\`a} Astronomica Italiana, 81, 294

\makeatother
\end{thebibliography}


% Alternatively you could enter them by hand, like this:
% This method is tedious and prone to error if you have lots of references
%\begin{thebibliography}{99}
%\bibitem[\protect\citeauthoryear{Author}{2012}]{Author2012}
%Author A.~N., 2013, Journal of Improbable Astronomy, 1, 1
%\bibitem[\protect\citeauthoryear{Others}{2013}]{Others2013}
%Others S., 2012, Journal of Interesting Stuff, 17, 198
%\end{thebibliography}

%%%%%%%%%%%%%%%%%%%%%%%%%%%%%%%%%%%%%%%%%%%%%%%%%%


% Don't change these lines
\bsp	% typesetting comment
\label{lastpage}
\end{document}

% End of mnras_template.tex