\documentclass[reprint,aps,amsmath,amssymb,prfluids,onecolumn,11pt]{revtex4-2}
\usepackage{tabularx} % nice looking tables
\usepackage{graphicx} % Graphics
\usepackage{bm}       % Bold math
\usepackage{natbib}   % References
\usepackage{tikz}     % Drawings with tikz
\usepackage{amsmath}  % Math expressions
\usepackage{amssymb}  % Math symbols
\usepackage[normalem]{ulem}     % Under line
\usepackage{subcaption} % Subfigures
\usepackage{epstopdf,epsfig}
%\usepackage[cp1252]{inputenc} % Fixes encoding issues of accented letters
\usepackage{textalpha}
\usepackage{setspace} \singlespacing
%\usepackage{todonotes}
\usepackage[normalem]{ulem}

\DeclareRobustCommand\sampleline[1]{%
  \tikz\draw[#1] (0,0) (0,\the\dimexpr\fontdimen22\textfont2\relax)
  -- (2em,\the\dimexpr\fontdimen22\textfont2\relax);%
}


\newcommand{\Rey}{\mathrm{Re}}
\newcommand{\Sto}{\mathrm{St}}
\DeclareRobustCommand\stl{\tikz[baseline]\draw[solid] (0,.5ex)--++(.5,0) ;}
\DeclareRobustCommand\dedash{\tikz[baseline]\draw[densely dashed] (0,.5ex)--++(.5,0) ;}
\DeclareRobustCommand\ddd{\tikz[baseline]\draw[dash dot dot] (0,.5ex)--++(.5,0) ;}
\DeclareRobustCommand\circles{\tikz \draw (0.5,0.1) circle (3pt);}
\DeclareRobustCommand\squares{\tikz \draw (0,0) rectangle (0.2,0.2);}
\DeclareRobustCommand\diamonds{\tikz \draw[rotate=45] (0,0) rectangle (0.2,0.2);}
\DeclareRobustCommand\triangles{\tikz \draw (0,0) -- (0.1,0.2) -- (0.2,0) -- cycle;}

% Custom colors
\colorlet{revision}{black}

\begin{document}

\title{Mechanisms of drag reduction by semi-dilute inertial particles in turbulent channel flow}

\author{Himanshu Dave}
\affiliation{
  School for Engineering of Matter, Transport and Energy, Arizona State University, Tempe, AZ 85281, USA
}

\author{M. Houssem Kasbaoui}
\email{houssem.kasbaoui@asu.edu}
\affiliation{
  School for Engineering of Matter, Transport and Energy, Arizona State University, Tempe, AZ 85281, USA
}
\date{\today}


\begin{abstract}
We investigate the mechanisms by which inertial particles dispersed at semi-dilute conditions cause significant drag-reduction in a turbulent channel flow at $\Rey_\tau = 180$. We consider a series of four-way coupled Euler-Lagrange simulations where particles having friction Stokes number $\Sto^+ = 6$ or 30 are introduced at progressively increasing mass loading from $M=0.2$ to 1.0. The simulations show that $\Sto^+ = 30$ particles cause large drag-reduction by up to 19.74\% at $M=1.0$, whereas $\Sto^+ = 6$ particles cause large drag increase by up to 16.92\% at $M=1.0$. To reveal the mechanisms underpinning drag-reduction or drag-increase, we investigate the stress distribution within the channel and the impact of the dispersed particles on the near-wall coherent structures. We find a distinctive feature of drag-reducing particles which consists in the formation of extremely long clusters, called ropes. These structures align preferentially with the low-speed streaks and contribute to their stabilization and suppression of bursting. Despite the additional stresses due to the particles, the modulation of the near-wall coherent structures leads to a greater reduction of Reynolds shear stresses and partial relaminarization of the near-wall flow. In the case of the drag-increasing particles with $\Sto^+ = 6$, a reduction in Reynolds shear stresses is also observed, however, this reduction is insufficient to overcome the additional particle stresses which leads to drag increase.
\end{abstract}

\pacs{}
\maketitle

\section{Introduction}
\label{sec:introduction}
% \begin{itemize}
%     % Diffusion of FL
%     \item {\st{Diffusion of FL}}
%     % Security threats to FL
%     \item {\st{Security threats to FL with particular focus on model poisoning}}
%     % Limitations of existing countermeasures
%     \item {\st{Current countermeasures (e.g., KRUM) and their limitations}}
%     % Proposed method and its advantages
%     \item {\st{Intuitive description of the proposed method and its difference (i.e., advantages) w.r.t. state of the art}}
%     % Main contributions
%     \item {\st{Summary of the main contributions of this work}}
%     % Paper's structure and organization
%     \item {\st{Paper's structure and organization}}
% \end{itemize}

% Diffusion of FL
Recently, {\em federated learning} (FL) has emerged as the leading paradigm for training distributed, large-scale, and privacy-preserving machine learning (ML) systems~\cite{mcmahan2017googleai,mcmahan2017aistats}. 
The core idea of FL is to allow multiple edge clients to collaboratively train a shared, global model without disclosing their local private training data.
%Specifically, an FL system consists of a central server and many edge clients; 
A typical FL round involves the following steps: {\em(i)} the server randomly picks some clients and sends them the current, global model; {\em(ii)} each selected client locally trains its model with its own private data; then, it sends the resulting local model to the server;\footnote{Whenever we refer to global/local model, we mean global/local model {\em parameters}.} {\em(iii)} the server updates the global model by computing an \emph{aggregation function}, usually the average (FedAvg), on the local models received from clients.
% \begin{enumerate}
%     \item[{\em(i)}] the server sends the current, global model to the clients and appoints some of them for training;
%     \item[{\em(ii)}] each selected client locally trains its copy of the global model with its own private data; then, it sends the resulting local model back to the server;\footnote{Whenever we refer to global/local model, we mean global/local model {\em parameters}.}
%     \item[{\em(iii)}] the server updates the global model by computing an \emph{aggregation function} on the local models received from clients (by default, the average, also referred to as FedAvg~\cite{mcmahan2017aistats}).
% \end{enumerate}
This process goes on until the global model converges. %(e.g., after a certain number of rounds or other similar stopping criteria).
%\\
% The advantages of FL over the traditional, centralized learning paradigm are undoubtedly clear in terms of flexibility/scalability (clients can join/disconnect from the FL network dynamically), network communications (only model weights\footnote{We will use \textit{parameters} and \textit{weights} interchangeably.} are exchanged between clients and server), and privacy (each client's private training data is kept local at the client's end and not uploaded to the server).
\\
% Security threats to FL
%However, the growing adoption of FL also raises security concerns~\cite{costa2022covert}, particularly about its confidentiality, integrity, and availability.
Although its advantages over standard ML, FL also raises security concerns~\cite{costa2022covert}. %, particularly about its confidentiality, integrity, and availability~\cite{costa2022covert}.
% OLD, LONG VERSION
% Indeed, some work deals with privacy leakage that may expose the local data of some clients~\cite{melis2019sp}. 
% A large body of work, instead, investigates attacks that usually aim to detriment the predictive accuracy of the learned global model. For instance, \emph{data poisoning} attacks achieve this goal by letting an adversary pollute the training set of some corrupt FL clients with maliciously crafted examples~\cite{jagielski2018sp}.
% Similarly, in \emph{model poisoning} the attacker attempts to tweak the global model weights~\cite{bhagoji2019pmlr} by directly perturbing the local model's weights of some infected FL clients before these are sent to the central server for aggregation, usually via so-called Byzantine attacks. 
% It turns out that Byzantine model poisoning attacks severely impact standard FedAvg; therefore, more robust aggregation functions must be designed to make FL systems secure.
Here, we focus on \emph{untargeted model poisoning} attacks~\cite{bhagoji2019pmlr}, where an adversary attempts to tweak the global model weights %\footnote{We will use the terms \textit{parameters} and \textit{weights} interchangeably.} 
by directly perturbing the local model's parameters of some infected clients before these are sent to the central server for aggregation.
In doing so, the adversary aims to jeopardize the global model \textit{indiscriminately} at inference time.
Such model poisoning attacks severely impact standard FedAvg; therefore, more robust aggregation functions must be designed to secure FL systems.
\\
% In this paper, we focus on designing a novel robust aggregation scheme at the server's end to contrast the effect of Byzantine model poisoning attacks.
%
% Current countermeasures and their limitations
%Several countermeasures have been proposed in the literature to combat model poisoning attacks on FL systems.
% Some methods use simple statistics more robust than plain average to smooth the impact of malicious updates (e.g., Trimmed Mean and FedMedian~\cite{yin2018icml}). 
% Other defenses implement outlier detection techniques to discard malicious updates from the aggregation performed at the server's end. Those are either based on heuristics (e.g., Krum/Multi-Krum~\cite{blanchard2017nips} and Bulyan~\cite{mhamdi2018pmlr}) or data-driven approaches (e.g., K-means clustering~\cite{shen2016acm} or DnC via spectral analysis~\cite{shejwalkar2021ndss}). 
% Finally, some strategies rely on a centralized ``source of trust'' to spot potential malicious updates (e.g., FLTrust~\cite{cao2020fltrust}).
% Several countermeasures have been proposed in the literature to combat model poisoning attacks on FL systems, i.e., to discard possible malicious local updates from the aggregation performed at the server's end. 
% These techniques range from simple statistics more robust than plain average (e.g., Trimmed Mean and FedMedian~\cite{yin2018icml}) to outlier detection heuristics (e.g., Krum/Multi-Krum~\cite{blanchard2017nips} and Bulyan~\cite{mhamdi2018pmlr}) or data-driven approaches (e.g., spectral analysis via K-means clustering~\cite{shen2016acm} or spectral analysis), or methods based on ``source of trust'' (e.g., FLTrust~\cite{cao2020fltrust}).
% OLD, LONG VERSION
%Several countermeasures have been proposed in the literature to combat Byzantine model poisoning attacks on FL systems.
% Descriptive statistics
% For example, Trimmed Mean and FedMedian aggregate local model updates using more robust statistics than standard average~\cite{yin2018icml}.
%
% % Heuristics for outlier detection
% Many existing Byzantine-resilient strategies implement some outlier detection heuristics to discard the model updates sent by potentially malicious clients from the input of the aggregation function.
% One of the most popular heuristics is Krum~\cite{blanchard2017nips}.
% This strategy tries to mitigate the impact of Byzantine attacks by selecting as a global model the local model with the smallest sum of Euclidean distances to {\em all} the other local models.
% Although powerful, Krum requires the server to know (or, at least, estimate) the number of malicious FL clients upfront, which is generally impossible in a realistic attack scenario. %
% Moreover, Krum may become ineffective for complex, high-dimensional model parameter spaces due to the curse of dimensionality.
% Bulyan~\cite{mhamdi2018pmlr} tries to overcome this issue by combining Krum with a variant of Trimmed Mean.
% % Data-driven outlier detection
% Other strategies use data-driven outlier detection techniques -- e.g., via K-means clustering~\cite{shen2016acm} -- to spot potential malicious local model updates. 
% %For instance, Shen et al. propose to cluster local model updates with K-means and thus identify outliers.
%
% % Other techniques
% As far as the server is concerned, any local model received can be from a potential malicious client. 
% FLTrust~\cite{cao2020fltrust} assumes the server acts as a client, i.e., trains a local model on an additional {\em trustworthy} dataset at the server's end and compares it against all the local models from other clients. 
% This way, the server can rely on some ``source of trust'' when discarding potentially malicious clients.
%\\
% Limitations of existing Byzantine-resilient strategies
Unfortunately, existing defense mechanisms either rely on simple heuristics (e.g., Trimmed Mean and FedMedian by~\cite{yin2018icml}) or need strong and unrealistic assumptions to work effectively (e.g., foreknowledge or estimation of the number of malicious clients in the FL system, as for Krum/Multi-Krum~\cite{blanchard2017nips} and Bulyan~\cite{mhamdi2018pmlr}, which, however, cannot exceed a fixed threshold).
Furthermore, outlier detection methods using K-means clustering~\cite{shen2016acm} or spectral analysis like DnC~\cite{shejwalkar2021ndss} do not directly consider the temporal evolution of local model updates received.
Finally, strategies like FLTrust~\cite{cao2020fltrust} require the server to collect its own dataset and act as a proper client, thereby altering the standard FL protocol.
\\
% OLD, LONG VERSION
% Overall, existing Byzantine-resilient strategies are either simple heuristics (e.g., FedMedian) or, if they are more complex, they rely on strong and unrealistic assumptions to work effectively (e.g., knowing the number of malicious clients in the FL system in advance, as for Krum and alike).
% Furthermore, data-driven outlier detection methods do not consider the temporary evolution of local model updates received (e.g., K-means clustering). 
% Finally, strategies like FLTrust requires the server to collect its own dataset and act as a proper client, thereby altering the standard FL protocol.
%
% Description of the proposed method
This work introduces a novel pre-aggregation \textit{filter} robust to untargeted model poisoning attacks. Notably, this filter $(i)$ operates without requiring prior knowledge or constraints on the number of malicious clients and $(ii)$ inherently integrates temporal dependencies. 
The FL server can employ this filter as a preprocessing step before applying \textit{any} aggregation function, be it standard like FedAvg or robust like Krum or Bulyan.
Specifically, we formulate the problem of identifying corrupted updates as a multidimensional (i.e., matrix-valued) time series anomaly detection task. 
The key idea is that legitimate local updates, resulting from well-calibrated iterative procedures like stochastic gradient descent (SGD) with an appropriate learning rate, show \textit{higher predictability} compared to malicious updates. This hypothesis stems from the fact that the sequence of gradients (thus, model parameters) observed during legitimate training exhibit regular patterns, as validated in Section~\ref{subsec:intuition}. %until convergence. 
%This regularity may be more pronounced for smooth convex loss functions, but it can still be captured within an appropriate time window, even for more complex and convoluted loss surfaces. 
%We provide evidence of this claim in Appendix~B, where we show that the average mutual information (i.e., ``predictability''), calculated over pairs of legitimate model updates sent at different FL rounds, is significantly higher than the corresponding computation for a malicious client.
\\
Inspired by the matrix autoregressive (MAR) framework for multidimensional time series forecasting~\cite{chen2021je}, we propose the FLANDERS ({\em \textbf{F}ederated \textbf{L}earning meets \textbf{AN}omaly \textbf{DE}tection for a \textbf{R}obust and \textbf{S}ecure}) filter.
The main advantages of FLANDERS over existing strategies like FLDetector~\cite{zhao2020multivariate} are its resilience to large-scale attacks, where $50\%$ or more FL participants are hostile, and the capability of working under realistic non-iid scenarios.
We attribute such a capability to two key factors: $(i)$ FLANDERS works without knowing a priori the ratio of corrupted clients, and $(ii)$ it embodies temporal dependencies between intra- and inter-client updates, quickly recognizing local model drifts caused by evil players. Below, we summarize our main contributions:

\begin{itemize}
\item[{\em(i)}]
We provide empirical evidence that the sequence of models sent by legitimate clients is more predictable than those of malicious participants performing untargeted model poisoning attacks.
\\
\item[{\em(ii)}] 
We introduce FLANDERS, the first pre-aggregation filter for FL robust to untargeted model poisoning based on multidimensional time series anomaly detection.
\\
\item[{\em(iii)}] 
We integrate FLANDERS into Flower,\footnote{\scriptsize{\url{https://flower.dev/}}} a popular FL simulation framework for reproducibility.
\\
\item[{\em(iv)}] 
We show that FLANDERS improves the robustness of the existing aggregation methods under multiple settings: different datasets, client's data distribution (non-iid), models, and attack scenarios.
\\
\item[{\em(v)}] 
We publicly release all the implementation code of FLANDERS along with our experiments.\footnote{\scriptsize{\url{https://anonymous.4open.science/r/flanders_exp-7EEB}}}
\end{itemize}

% Paper's structure and organization
The remainder of the paper is structured as follows. %some related work and the current state-of-the-art solutions to security issues that FL entails. 
Section~\ref{sec:background} covers background and preliminaries. 
In Section~\ref{sec:related}, we discuss related work.
Section~\ref{sec:problem} and Section~\ref{sec:method} describe the problem formulation and the method proposed. % to tackle it. 
Section~\ref{sec:experiments} gathers experimental results. %, and Section~\ref{sec:limitations} discusses some limitations of this work.
Finally, we conclude in Section~\ref{sec:conclusion}.
 %discusses the limitations of this work and draws future research directions.
%reports conclusions and draws perspectives for future research directions.

%%%%%%% OLD %%%%%%%
%to overcome the resilience of Byzantine failures in distributed Stochastic Gradient Descent computations. 
% The strength of Krum is its time complexity, which is linear in the gradient dimension. 
% However, the robustness of the approach is guaranteed for gradient-based learning applications only when the majority of the clients are not compromised. 
% Besides, the aggregation mechanism of Krum, as well as that of similar methods, is robust from a coarse-grained perspective and does not provide solutions to errors and perturbations that may occur at inference time.
%A related approach to~\cite{blanchard2017nips} is the work of Su et al.~\cite{su2016dc}. Here, the authors propose an iterated approximate agreement to tackle a multi-layer scenario attacked by Byzantine agents. 
%However, the method works efficiently on the sole discrete context and it is inapplicable to continuous state environments.
%\gabri{Maybe, we should just talk about the main limitations of existing countermeasures without digging into their details (or, we can just mention Krum as this is the most popular one). I will move the description of all these methods to the Related Work section.}

%% Macro setup
\definecolor{purple}{rgb}{1, 0, 1}

\newcommand{\ie}{\emph{i.e.,}\xspace}
\newcommand{\eg}{\emph{e.g.,}\xspace}
\newcommand{\abr}{\emph{abbr.}\xspace}
\newcommand{\ea}{\emph{et al.}\xspace}
\newcommand{\gensync}{\emph{GenSync}\xspace}
\newcommand{\colosseum}{\emph{Colosseum}\xspace}
\newcommand{\srep}{\emph{SREP}\xspace} % Set Reconciliation Enhances
\newcommand{\srepsim}{\emph{SREPSim}\xspace}
% Propagation
\newcommand{\esrep}{\emph{E-SREP}\xspace}
\newcommand{\epsrep}{\emph{EP-SREP}\xspace}
\newcommand{\mesrep}{\emph{ME-SREP}\xspace}
\newcommand{\mempoolsync}{\emph{MempoolSync}}

\newcommand{\fref}[1]{Fig.~\ref{#1}}
\newcommand{\tref}[1]{Table~\ref{#1}}
\newcommand{\aref}[1]{Algorithm~\ref{#1}}
\newcommand{\procref}[1]{Procedure~\ref{#1}}
\newcommand{\sref}[1]{Section~\ref{#1}}
\newcommand{\lineref}[1]{line~\ref{#1}}
\newcommand{\appref}[1]{Appendix~\ref{#1}}

% Change \eqref
\LetLtxMacro{\originaleqref}{\eqref}
\renewcommand{\eqref}{Eq.~\originaleqref}

% Theorems and corollaries
\newcounter{theoremcount}
\setcounter{theoremcount}{0}
\DeclareRobustCommand{\theorem}[1]{%
  \refstepcounter{theoremcount}%
  \noindent\textit{\textbf{Theorem \thetheoremcount\label{theorem:#1}: }}%
}
\DeclareRobustCommand{\theoremref}[1]{Theorem~\ref{theorem:#1}}

\DeclareRobustCommand{\proof}{\emph{Proof:}\xspace}
\DeclareRobustCommand{\qqed}{\hfill$\blacksquare$}

\newcounter{corollcount}
\setcounter{corollcount}{0}
\DeclareRobustCommand{\coroll}[1]{%
  \refstepcounter{corollcount}%
  \noindent\textit{\textbf{Corollary \thecorollcount\label{coroll:#1}: }}%
}
\DeclareRobustCommand{\corollref}[1]{Corollary~\ref{coroll:#1}}

\newcounter{lemmacount}
\setcounter{lemmacount}{0}
\DeclareRobustCommand{\lemma}[1]{%
  \refstepcounter{lemmacount}%
  \noindent\textit{\textbf{Lemma \thelemmacount\label{lemma:#1}: }}%
}
\DeclareRobustCommand{\lemmaref}[1]{Lemma~\ref{lemma:#1}}

\newcounter{definitioncount}
\setcounter{definitioncount}{0}
\DeclareRobustCommand{\definition}[1]{%
  \refstepcounter{definitioncount}%
  \noindent\textit{\textbf{Definition \thedefinitioncount\label{definition:#1}: }}%
}
\DeclareRobustCommand{\defref}[1]{Definition~\ref{definition:#1}}

%notes of different authors
\newif\ifnotes
\notestrue
\notesfalse

\newif\ifdiff
\difftrue
\difffalse

\newcommand{\anote}[1]{\ifnotes $\ll$\textsf{\textcolor{purple}{Ari: {#1}}}$\gg$ \fi}
\newcommand{\nnote}[1]{\ifnotes $\ll$\textsf{\textcolor{orange}{Novak: {#1}}}$\gg$ \fi}
\newcommand{\diff}[1]{\ifdiff\textcolor{orange}{#1}\else#1\fi}

%%% Local Variables:
%%% mode: latex
%%% TeX-master: "main"
%%% End:


%%% Redefining theorem-like environments
\newcounter{environments}

\newcounter{theoremCounter}
\newcounter{lemmaCounter}
\newcounter{definitionCounter}
\newcounter{propositionCounter}
\newcounter{corollaryCounter}
\newcounter{exampleCounter}
\newcounter{remarkCounter}
\newcounter{propertyCounter}
\newcounter{assumptionCounter}
\newcounter{proofCounter}

%\theorempreskip{1pt}
%\theorempostskip{1pt}

\let\proposition\relax
\let\theorem\relax
\let\lemma\relax
\let\definition\relax
\let\corollary\relax
\theoremseparator{.}
\theorembodyfont{\itshape}
\theoremsymbol{$\triangleleft$}
\newtheorem{theorem}[theoremCounter]{Theorem}
\newtheorem{lemma}[lemmaCounter]{Lemma}
\newtheorem{definition}[definitionCounter]{Definition}
\newtheorem{proposition}[propositionCounter]{Proposition}
\newtheorem{corollary}[corollaryCounter]{Corollary}

\let\remark\relax
\let\example\relax
\let\assumption\relax

\theorembodyfont{\normalfont}
\newtheorem{example}[exampleCounter]{Example}
\newtheorem{remark}[remarkCounter]{Remark}

\theoremheaderfont{\itshape}
\theoremsymbol{}
\renewtheorem{property}[remarkCounter]{Property}

\theoremheaderfont{\bfseries}
\theorembodyfont{\itshape}
\newtheorem{assumption}[assumptionCounter]{Assumption}
\theoremheaderfont{\itshape}


\theoremstyle{plain}
\theoremheaderfont{\itshape}
\theorembodyfont{\normalfont}
\let\proof\relax
\theoremseparator{.}
\theoremsymbol{\qedfull}
\newtheorem*{proof}{Proof}
\qedsymbol{\qedfull}

% Reset equation counters for each property!
\makeatletter
\@addtoreset{equation}{property}
\makeatother

% Tikz stuff
\newcommand{\seqarr}
{\begin{tikzpicture}
		\draw[-{Triangle[scale=.7]}] (0,0) --  (.3,0); 
\end{tikzpicture}}

\newcommand{\looparr}
{\begin{tikzpicture}[scale=0.7,baseline=-1.55ex]
		\draw[arrows = {-Stealth[inset=0pt, length=2pt, angle'=60]}] (0,0) arc (102:437:.2cm);
\end{tikzpicture}}

\definecolor{blue-violet}{rgb}{0.54, 0.17, 0.89}
\definecolor{cadmiumorange}{rgb}{0.93, 0.53, 0.18}
\definecolor{yellow-green}{rgb}{0.6, 0.8, 0.2}
\definecolor{green1}{rgb}{0.12, 0.3, 0.17}
\definecolor{byzantium}{rgb}{0.44, 0.16, 0.39}


\section{Structure of a particle-free turbulent channel flow}
\label{sec:unladen}

\begin{figure}
  \begin{subfigure}{0.45\textwidth}
    \centering
    \includegraphics[width=1\linewidth]{plotting/fig2a.pdf}
    \caption{ }
    \label{fig:unladen_vel}
  \end{subfigure}
  \begin{subfigure}{0.45\textwidth}
    \centering
    \includegraphics[width=1\linewidth]{plotting/fig2b.pdf}
    \caption{ }
    \label{fig:unladen_rms}
    \end{subfigure}
    \caption{Unladen flow velocity statistics at $\Rey_\tau =180$: (a) Streamwise mean velocity (b) root-mean square velocity fluctuations in the streamwise ($u_{f,rms}^+$, \stl), normal ($v_{f,rms}^+$, \dedash) and spanwise ($w_{f,rms}^+$, \ddd) directions. The symbols correspond to the data taken from 
    \citet{kimTurbulenceStatisticsFully1987}.}
    \label{fig:unladen_results}
\end{figure}

The particle-free channel represents a baseline reference that we use to highlight flow modifications induced by inertial particles. To that end, we start by reviewing aspects of an unladen turbulent channel flow at $\Rey_\tau=180$ that are relevant to the discussion in \S\ref{sec:particle_laden}.

Figure \ref{fig:unladen_results} shows  profiles of  mean streamwise velocity $\langle u^+_f\rangle$ and root-mean square (rms) velocity  fluctuations. Averaging is performed using 100 snapshots collected over a period of 10 of eddy turnover time once the flow is stationary. Further,  the streamwise and spanwise directions are averaged such that the only variation is in the wall-normal direction. As expected at this Reynolds number, the mean streamwise velocity shows three characteristic regions: viscous layer for $y^+\lesssim 5$, a buffer layer for $5\lesssim y^+\lesssim 30$, and a logarithmic layer for $y^+\gtrsim 30$. Velocity fluctuations in the streamwise direction dominate over the two other components and peak at $y^+\sim 12$ in the buffer layer. These observations are consistent with those of \citet{kimTurbulenceStatisticsFully1987} and general understanding of turbulent channel flow at the Reynolds number considered.

\begin{figure}
    \centering
    \includegraphics[width=5.0in]{plotting/fig3.pdf}
    \caption{Contribution of the Reynolds shear stress(\ddd) and viscous shear stress(\dedash) to the total shear stress(\stl) in the particle-free channel at $\Rey_\tau = 180$.}
    \label{fig:tss_sp}
\end{figure}

The structure of the mean flow results from a balance between pressure gradient $-\langle\partial p/\partial x\rangle$, viscous stress $\mu d\langle u_f\rangle/dy$, and Reynolds shear stress $-\rho_f \langle u'_fv'_f \rangle $. By Reynolds-averaging the fluid momentum equations, one can show that the equation for the streamwise momentum reduces to
\begin{equation}
    -\langle\frac{\partial  p }{\partial x}\rangle = \frac{d}{d y}\left(\mu\frac{d}{dy} \langle u_f\rangle - \rho_{f}\langle u'_{f}v'_{f}\rangle\right).
    \label{eq:RSS_1}
\end{equation}
Given that the pressure gradient in a fully developed channel is constant, the total shear stress (sum of the viscous and Reynolds stresses) must vary linearly across the channel, i.e.,
\begin{equation}
    \mu\frac{d}{dy} \langle u_f\rangle - \rho_{f}\langle u'_{f}v'_{f}\rangle = \tau_{w}\left(1-\frac{y}{h}\right).
    \label{eq:cf_rss}
\end{equation}
This behavior is illustrated in figure \ref{fig:tss_sp} showing the variation of the total shear stress and its viscous and Reynolds contributions as a function of the wall normal distance. In accordance with (\ref{eq:cf_rss}), the total shear stress varies linearly from the wall to the channel center where it vanishes. The viscous shear stress dominates near the wall, and vanishes away from it. Conversely, the contribution of the Reynolds shear stress is small in the viscous sublayer, whereas it dominates in the logarithmic layer.

The Reynolds shear stresses has a direct influence on skin-friction drag. The latter is characterized using the coefficient
\begin{equation}
    C_{f}=\frac{\tau_w}{\frac{1}{2}\rho_f U_{b,f}^{2}},
    \label{eq:Cf}
\end{equation}
where $U_{b,f}=\dot{m}_f/\rho_f$ is the bulk fluid velocity corresponding to the ratio of the cross-sectional average fluid mass flow rate and the fluid density.  While $\tau_w$ is fixed in a channel flow driven by a constant pressure gradient, modulating the Reynolds shear stress is susceptible to change the bulk velocity $U_{b,f}$, and, in turn, the skin-friction drag $C_f$. Double integrating equation (\ref{eq:cf_rss}) clarifies the connection between $C_f$ and Reynold shear stress. The resulting mass flow rate per unit spanwise length is
\begin{equation}
    \frac{\dot{m}_f}{L_z}= \frac{2}{3} \frac{\tau_w h^2}{\nu}\left(1 + \frac{3}{(u_\tau h)^2}\int_{y=0}^{h} \int_{y'=0}^y \langle u'_fv'_f\rangle dy'dy\right).
    \label{eq:sp_mass_flow_rate}
\end{equation}
In this form, it becomes clear that the Reynolds shear stress reduces the mass flow rate, given $\langle u'_fv'_f\rangle<0$, resulting in an increase of $C_f$ compared to the laminar baseline. Therefore, it is not surprising that a large number of prior studies on skin-friction drag reduction in turbulent channel flows focused on reducing the Reynolds shear stress \citep{hetsroniLowspeedStreaksDragreduced1998,minDragReductionPolymer2003,ptasinskiTurbulentChannelFlow2003}.

\begin{figure}
	\centering
  \includegraphics[width=\linewidth]{tikzgraphics/fig4.pdf}
	\caption{Isocontours showing the low speed streaks in the particle-free channel at $y^+ = 10$.
	\label{fig:streak_SP_a}}
\end{figure}

\begin{figure}
	\centering
  \includegraphics[width=5.0in]{plotting/fig5.pdf}
	\caption{Two-point autocorrelation of the streamwise velocity fluctuations in the spanwise direction for particle-free channel at $y^+ = 10$.
	\label{fig:streak_SP_b}}
\end{figure}

From a mechanistic perspective, the Reynolds shear stress arises from coherent flow structures that populate the near-wall region \citep{smitsHighReynoldsNumber2011}. The so-called low-speed streaks, regions of slow moving fluid elongated in the streamwise direction, are among the most significant coherent structures found in a turbulent channel flow \citep{baeLifeCycleStreaks2021}. These streaks are shown in figure \ref{fig:streak_SP_a} visualized using isocontours of streamwise velocity in a wall parallel plane at $y^+=10$.
There has been sustained effort to understand the morphology and dynamics of low-speed streaks, as well as their connection to other coherent structures, such as quasi-streamiwse vorticies and so-called large-scale motions and very large-scale motions (see
\citep{klineStructureTurbulentBoundary1967a,willmarthStructureReynoldsStress1972,smithCharacteristicsLowspeedStreaks1983,jimenezMinimalFlowUnit1991,stretchAutomatedPatternEduction1991,jeongCoherentStructuresWall1997,jimenezCoherentStructuresWallbounded2018,zhuVortexAxisTracking2019,jiangExperimentalStudyLowspeed2020,schoppaCoherentStructureGeneration2002,smitsReynoldsStressScaling2021,baeLifeCycleStreaks2021,zhouInteractionNearwallStreaks2022}). The general consensus is that low-speed streaks in the buffer layer are formed between a pair of quasi-streamwise vortices with opposite orientation. The bursting of low-speed streaks contributes a significant part of the Reynolds shear stress and turbulent energy production \citep{kimProductionTurbulenceSmooth1971,baeLifeCycleStreaks2021}. This occurs when the quasi-streamwise vortices surrounding a low-speed streak become unstable \citep{jeongCoherentStructuresWall1997}, resulting in a lift up and eventual break down of the streak. Prior to their collapse, low-speed streaks in the buffer region have a typical length in the range 200--300 wall units \citep{baeLifeCycleStreaks2021,jeongCoherentStructuresWall1997}, but may extend beyond 1000 wall units \citep{jimenezLargescaleDynamicsNearwall2004}. Remarkably, these structures have a spanwise spacing $\lambda_f^+$ that varies little with Reynolds number, and is about $\lambda_f^+ \sim 100-110$ \citep{klineStructureTurbulentBoundary1967,jimenezMinimalFlowUnit1991,klewickiViscousSublayerFlow1995,jimenezLargescaleDynamicsNearwall2004}. We verify this by computing the two-point autocorrelation of the streamwise velocity as a function of the spanwise separation and wall distance,
\begin{equation}
    R^f_{uu}(\Delta z;y_0)=\frac{\langle{u'_f (x,y_0,z,t)u'_f (x,y_0,z+\Delta z,t)}\rangle}{\langle{u_f'^2}\rangle}.
    \label{eq:lambda_f}
\end{equation}
Figure \ref{fig:streak_SP_b} shows the variation of  $R^f_{uu}$ at $y^+=10$. The streak spanwise spacing $\lambda_f^+$ corresponds to twice the distance between the origin and $\Delta z^+$ where $R^f_{uu}$ reaches a first minimum which yields $\lambda_f^+ =2\times 53= 106$ in the present simulations.

It is noteworthy that the physics of a turbulent channel flow, discussed here at $\Rey_\tau =180$, remain largely the same near the wall, even at much higher Reynolds numbers. \citet{moserDirectNumericalSimulation1999} conduct wall-bounded channel flow simulations at $\Rey_\tau = 590,395$ and $180$. They showed that despite differences in the log region, the dynamics in the viscous and buffer layer regions are similar. Given that inertial particles tend to accumulate in these two regions, we expect that the particle-fluid interactions observed at $\Rey_\tau = 180$ will persist to much higher Reynolds numbers.

%\section{Results and Discussion}
%\label{sec:results}
\section{Effect of inertial particles at semi-dilute conditions}
\label{sec:particle_laden}

Introducing inertial particles at semi-dilute concentration causes a departure from the known characteristics of a particle-free channel flow. In the following, we analyze the flow modulation resulting from the particle feedback force and propose a mechanism based on the interplay between near-wall coherent structures and particle clusters.

\subsection{Flow modulation and impact on skin-friction drag}
\label{sec:modulation}

\begin{figure}
	\centering
	\includegraphics[width=5in]{tikzgraphics/fig6.pdf}
	\caption{Isocontours of streamwise velocity in a wall-normal plane: (top) the reference unladen flow, (middle) flow laden with $\Sto^+=30$ particles at $M=1.0$, and (bottom) flow laden with $\Sto^+=6$ particles at $M=1.0$. Note: To facilitate visual comparison, the domain in the middle has been truncated to the same dimensions as the smaller domain in the bottom case.
	\label{fig:isocontour_u}}
\end{figure}


Figure \ref{fig:isocontour_u} shows isocontours of the streamwise velocity at an arbitrary time after the flow reached a stationary state.
From these instantaneous visualizations, it is immediately clear that semi-dilute inertial particles cause strong modulation of the carrier flow, with the most apparent change being a change in fluid bulk velocity. The latter is visibly increased by $\Sto^+=30$ particles at mass loading $M=1$ compared to the reference particle-free flow.
 In particular, the fluid streamwise velocity near the centerline shows a noticeable increase.
Further, the overall level of turbulence judged by the naked eye is diminished compared to the unladen flow.
In contrast, $\Sto^+=6$ particles at mass loading $M=1$ cause an apparent slow down of the carrier flow. The greatest drop of the fluid velocity is around the centerline.

\begin{figure}\centering
  \begin{subfigure}{0.49\linewidth}
    \centering
    \includegraphics[width=3.2in]{plotting/fig7a.pdf}
    \caption{}
    \label{fig:u_st6}
  \end{subfigure}\hfill
  \begin{subfigure}{0.49\linewidth}
    \centering
    \includegraphics[width=3.2in]{plotting/fig7b.pdf}
    \caption{}
    \label{fig:u_st30}
	\end{subfigure}
  \caption{Average streamwise velocity as a function of the wall normal distance for (a) $\Sto^+ = 6$ and (b) $\Sto^+ = 30$ various mass loadings. The solid black line represents a particle-free channel. Symbols denote Stokes number $\Sto^+=$ 6 (\textcolor{red}{\protect\scalebox{1.75}{$\bullet$}}) or 30 (\textcolor{blue}{$\blacksquare$}). Darker symbols correspond to larger mass loading which varies from 0.2 to 1.0.}
  \label{fig:uavg}
\end{figure}

\begin{figure}\centering
  \begin{subfigure}{0.49\linewidth}
    \centering
    \includegraphics[width=3.20in]{plotting/fig8a.pdf}
    \caption{}
    \label{fig:u_rms_st6}
  \end{subfigure}\hfill
  \begin{subfigure}{0.49\linewidth}
    \centering
    \includegraphics[width=3.20in]{plotting/fig8b.pdf}
    \caption{}
    \label{fig:u_rms_st30}
  \end{subfigure}\\
  \begin{subfigure}{0.49\linewidth}
    \centering
    \includegraphics[width=3.20in]{plotting/fig8c.pdf}
    \caption{}
    \label{fig:v_rms_st6}
  \end{subfigure}\hfill
	\begin{subfigure}{0.49\linewidth}
		\centering
		\includegraphics[width=3.20in]{plotting/fig8d.pdf}
		\caption{}
		\label{fig:v_rms_st30}
	\end{subfigure}\\
	\begin{subfigure}{0.49\linewidth}
		\centering
		\includegraphics[width=3.20in]{plotting/fig8e.pdf}
		\caption{}
		\label{fig:w_rms_st6}
	\end{subfigure}\hfill
	\begin{subfigure}{0.49\linewidth}
		\centering
		\includegraphics[width=3.20in]{plotting/fig8f.pdf}
		\caption{}
		\label{fig:w_rms_st30}
	\end{subfigure}
  \caption{Variation of the fluid root-mean-square velocity fluctuations with the wall normal distance for $\Sto^+ = 6$ (a,c,e) and $\Sto^+ = 30$ (b,d,f) at $M = 0.2-1.0$. The solid black line represents a particle-free channel. Symbols as in figure \ref{fig:uavg}.
  \label{fig:vel_rms}}
\end{figure}

Figure \ref{fig:uavg} shows the mean streamwise velocity profile for the particle-laden cases with $\Sto^+ = 6$ and $\Sto^+ = 30$ at the mass loadings $M=0.2$, 0.6, and 1.0. The velocity profile in the particle-free channel flow is also shown for comparison. In the viscous sublayer, the profiles follow the same linear scaling as the unladen channel. The largest impact of the inertial particles manifests in the buffer and logarithmic layers. In channels laden with $\Sto^+=30$ particles, the fluid velocity profile shifts upward in the logarithmic layer from the reference profile of the unladen channel. This trend is further reinforced with increasing mass loading which leads to greater upward shift of the velocity profile. These observations are in agreement with those of \citet{zhouNonmonotonicEffectMass2020} who found a similar upward shift of the streamwise velocity profile in their simulations at $\Rey_\tau=180$, $\Sto^+=30$, and $M=0.75$. Conversely, the profile of the mean streamwise velocity shifts downward from the reference unladen flow when $\Sto^+=6$ particles are suspended. Similar to the cases with higher inertia particles, increasing the mass loading causes an amplification of the trend observed, i.e., a downward shift of the profile here.

Figures \ref{fig:isocontour_u} and \ref{fig:uavg} provide qualitative and quantitative evidence that $\Sto^+=30$ particles increase the fluid mass flow rate, while $\Sto^+=6$ particles decrease it. As discussed in \S\ref{sec:unladen}, the implication of this flow modulation in a channel driven by a constant pressure gradient is that $\Sto^+=30$ particles decrease skin-friction drag, whereas $\Sto^+=6$ perform the opposite, i.e., increase skin-friction drag.

Figures \ref{fig:vel_rms} shows the variation of the velocity fluctuations root-mean-square (r.m.s.) with Stokes number and mass loading. For $\Sto^+=6$ particles, the profile of the streamwise fluctuations  at $M=0.2$ and $M=0.6$ changes little from the profile of the unladen flow. Only when mass loading is increased to $M = 1.0$ do we see a significant change of the streamwise fluctuations, primarily in the log region. Conversely, $\Sto^+=30$ particles lead to a more pronounced modulation of the streamwise velocity fluctuations at all three mass loadings considered. Generally, the streamwise fluctuations decrease slightly in the viscous layer and increase significantly in the buffer and logarithmic layers. Further, the location of the peak shifts from about $y^+=14$ in the unladen case to $y^+=22$ at $M=1.0$. While $\Sto^+=30$ and $\Sto^+=6$ particles exhibit different modulation characteristics for the fluid streamwise velocity fluctuations, their impact on the wall normal and spanwise velocity fluctuations displays fewer differences. Both particles cause significant dampening of the wall normal and spanwise fluctuations in the viscous, buffer and logarithmic layers. Increasing mass loading leads to larger reduction of these fluctuations.

\begin{figure}
  \includegraphics[width=5.0in]{plotting/fig9.pdf}
  \caption{Relative change of skin-friction drag in particle-laden turbulent channel flows. Positive values indicate drag reduction, whereas negative values indicate drag increase. Symbols as in figure \ref{fig:uavg}.\label{fig:friction_coef}
  }
\end{figure}
By modulating the fluid bulk velocity, the dispersed inertial particles lead to a large change in skin-friction drag. Figure \ref{fig:friction_coef} shows times histories of relative change in skin-friction coefficient for all cases in table \ref{tab:parameters}. The reference value, $C_{f,\mathrm{0}}$, corresponds to the skin-friction drag coefficient from the statistically stationary particle-free channel flow. Figure \ref{fig:friction_coef} illustrates how particle inertia plays a selective role by determining the type of flow modulation obtained, be it drag-reducing or drag-increasing, while mass loading acts as an amplifying factor.
For all these cases, we compute the the drag reduction factor  $\mathrm{DR}$, defined as
\begin{equation}
  \mathrm{DR}= \frac{C_{f,\mathrm{0}} - C_f}{C_{f,\mathrm{0}}},
\end{equation}
which takes positive values in the case of drag reduction, and negative values in the case of drag increase. Further, we compute the change in mass flow rate $(\dot{m}_f-\dot{m}_{f,0})/\dot{m}_{f,0}$, where $\dot{m}_{f,0}$ is the mass flow rate in the reference particle-free channel. We report these values in table \ref{tab:drag_reduction}. The greatest drag reduction is obtained with $\Sto^+=30$ particles at the mass loading $M=1.0$, which yields drag reduction factor $\mathrm{DR}=19.54\%$ and corresponding mass flow rate increase of 11.07\%. The latter value is substantially higher than what has been reported in the literature, in particular, by \citet{liNumericalSimulationParticleladen2001} who found an increase in mass flow rate by about $\sim 5\%$ using particles with Stokes number of $O(100)$. This level of drag reduction shows that inertial particles can induce drag reduction at a level comparable with the one obtained using polymer additives \citep{housiadasPolymerinducedDragReduction2003,thaisTemporalLargeEddy2010}, such as in the polymeric channel flow simulations of \citet{housiadasPolymerinducedDragReduction2003} at $\Rey_\tau=180$  where drag reduction $\mathrm{DR}\simeq 25\%$ is reported. Note the drag reduction effect of $\Sto^+=30$ particles decreases significantly at lower mass loadings. At $M=0.2$, these particles reduce drag by only 4.21\%, and yield a modest mass flow rate increase of 2.22\%. This weaker drag reduction is to be expected because the particle feedback force scales with mass loading, thus, particle-induced flow modulation vanishes as $M$ decreases. This also holds for the drag increasing particles with $\Sto^+=6$ whose effect increases with mass loading resulting in drag increase by $16.92\%$ and mass flow rate decrease by 6.10\% at $M=1.0$.



\begin{table}
  \caption{Variation of percent drag reduction and mass flow rate in the present simulations. \label{tab:drag_reduction}}
  \begin{ruledtabular}
    \begin{tabular}{llll}
      Stokes number ($\Sto^+$) & Mass loading ($M$) & DR(\%) & $\Delta \dot{m}_f/\dot{m}_{f,0}$(\%)  \\\hline
      6             & 0.2          & -1.76      & -0.78      \\
                    & 0.6          & -8.90      & -3.93      \\
                    & 1.0          & -16.92      & -6.10      \\
      30            & 0.2          & 4.21       & 2.22      \\
                    & 0.6          & 12.27       & 7.37      \\
                    & 1.0          & 19.54				& 11.07
    \end{tabular}
  \end{ruledtabular}
\end{table}


\subsection{Shear stress balance in the presence of inertial particles}
\label{sec:stress_two_phase}

As with the unladen flow, the structure of the flow in a particle-laden channel results from a balance of stresses applied on the fluid. However, the presence of particles introduces additional stresses that alter the balance in equation (\ref{eq:RSS_1}). To derive a new balance that takes into account particle stresses, we apply Reynolds-averaging to the momentum equation (\ref{eq:NVS_2}). Assuming that particle clustering does not break the dilute limit locally ($1-\phi\simeq 1$), the resulting balance is
\begin{equation}
  \frac{d}{dy}\left(\mu\frac{d }{dy}\langle u_f\rangle - \rho_{f}\langle u_{f}'v_f'\rangle \right)+ \langle F_{p,x}\rangle= -\bigg\langle \frac{\partial p}{\partial x}\bigg\rangle,
    \label{eq:RSS_2}
\end{equation}
where $\langle F_{p,x}\rangle$ represents the mean streamwise particle stresses.
{\color{revision}
The latter can be related to the particle-phase Reynolds shear stress. To do so, we consider the particle conservation equations in the Eulerian frame. Using the Two-Fluid model discussed in \citep{kasbaouiClusteringEulerEuler2019} under the assumption of mono-kinetic particle velocity distribution, the particle mass and momentum conservation equations read
\begin{eqnarray}
    \frac{\partial }{\partial t}(\rho_p\phi) + \nabla \cdot (\rho_p\phi\bm{u}_p)&=& 0 \\
     \frac{\partial}{\partial t}(\rho_p\phi\bm{u}_p) + \nabla \cdot (\rho_p\phi\bm{u}_p\bm{u}_p) &=& -\bm{F}_p +\bm{C}
\end{eqnarray}
where $\bm{u}_p$ is the $\bm{C}$ represents the collision stresses. Neglecting the latter and averaging the streamwise particle momentum balance yields}
\begin{equation}
	\frac{d}{dy}  \left(\rho_p\langle \phi u''_p v''_p\rangle\right)=-\langle F_{p,x}\rangle. \label{eq:RSS_3}
\end{equation}
Here, $u_p''$ and $v_p''$ refer to the streamwise and wall-normal particle-phase velocity fluctuations with respect to the Favre-averaged particle velocities $\widetilde{u}_p=\langle \phi u_p\rangle/\langle \phi\rangle$ and $\widetilde{v}_p=\langle \phi v_p\rangle/\langle \phi\rangle$. Combining equations (\ref{eq:RSS_2}) and (\ref{eq:RSS_3}) yields
\begin{equation}
   \frac{d}{dy}\left(\mu\frac{d}{dy} \langle u_f\rangle - \rho_{f}\langle u_{f}'v_f'\rangle - \rho_{p}\langle \phi u_{p}''v_{p}'' \rangle \right)= -\bigg\langle\frac{\partial p}{\partial x}\bigg\rangle,
    \label{eq:RSS_5}
\end{equation}
which integrates to
\begin{equation}
  \mu\frac{d \langle u_f\rangle }{d y} - \rho_{f}\langle u_{f}'v_f'\rangle - \rho_p\langle \phi u_{p}''v_{p}'' \rangle=\tau_w \left(1-\frac{y}{h}\right) .
    \label{eq:RSS_6}
\end{equation}
Similar to the particle-free channel, equation (\ref{eq:RSS_6}) shows the total stress varies linearly across the channel provided that the particle-phase Reynolds shear stress $\rho_{p}\langle \phi u_{p}''v_{p}'' \rangle $ is also taken into account. Integrating equation (\ref{eq:RSS_6}) twice, leads to an updated expression for the fluid mass flow rate by unit spanwise length which takes into account the effect of the dispersed particles,
\begin{equation}
  \frac{\dot{m_f}}{Lz} = \frac{2}{3}\frac{\tau_w h^2}{\nu}\left(1+\frac{3}{(u_\tau h)^2}\int_{0}^{h}\left(\int_{0}^y \langle u_{f}'v_f'\rangle + \frac{M}{\phi_0} \langle \phi u_{p}''v_{p}'' \rangle dy'\right) dy\right).
    \label{eq:MFR_2}
\end{equation}
The relationship (\ref{eq:MFR_2}) shows that the particles alter the fluid mass flow rate through two competing effects: (i) a direct effect through the particle-phase Reynolds shear stress $\rho_p\langle \phi u_{p}''v_{p}'' \rangle$ which, like the fluid-phase Reynolds shear stress, tends to reduce the mass flow rate, and (ii) an indirect effect through the modulation of the fluid-phase shear stress $\rho_f\langle u_{f}'v_f'\rangle$. It is only when the fluid-phase shear stress is reduced more than can be balanced by the particle-phase Reynolds shear stress that the fluid mass flow rate is increased.

 \begin{figure}\centering
  \centering
  \begin{subfigure}{0.49\textwidth}
    \centering
    \includegraphics[width=3.20in]{plotting/fig10a.pdf}
    \caption{}
    \label{fig:tsscomparest30}
  \end{subfigure}
  \hfill
  \begin{subfigure}{0.49\textwidth}
    \centering
    \includegraphics[width=3.20in]{plotting/fig10b.pdf}
    \caption{}
    \label{fig:tsscomparest6}
  \end{subfigure}

  \caption{Shear stress contributions as a function of the wall normal distance for the particle-laden turbulent channel flows at (a) $\Sto=30$, $M=1.0$ and (b) $\Sto=6$, $M=1.0$. The viscous shear stress is denoted by (\circles), fluid-phase shear stress by (\squares), particle-phase shear stress by (\diamonds) and the total shear stress by (\triangles). Lines without symbols correspond to the reference single-phase channel as denoted in figure \ref{fig:tss_sp}.}
  \label{fig:tss_comparison}
\end{figure}

Figure \ref{fig:tss_comparison} shows the total shear stress profile and the variations in the fluid and particle stress components for cases $\Sto^+ = 6$ and $\Sto^+ = 30$ at $M = 1.0$. In both cases, the total shear stress varies linearly across the channel as predicted by (\ref{eq:RSS_6}). This first observation validates the two hypotheses underpinning the relationship (\ref{eq:RSS_6}): (i) $1-\phi\simeq 1$ meaning that the particle phase remains dilute even though significant clustering occurs near the walls as we show in \S \ref{sec:mechanism}, and  (ii) collisional stresses are negligible compared to hydrodynamic stresses exerted on particles, even within particle clusters. Considering $\Sto^+ = 30$ particles, figure \ref{fig:tsscomparest30} shows partial relaminarization of the near-wall region. Compared to the reference particle-free flow, the viscous drag increases in the viscous and buffer layers. This modulation is directly linked to the increase in fluid mass flow rate observed in figure \ref{fig:isocontour_u}. Further, the fluid-phase Reynolds shear stresses drops significantly with a peak down to about 39\% of the unladen case, and is shifted further towards the centerline. This drop is partially balanced by  the rise of particle-phase Reynolds shear stress. The latter dominates in the region $0.1\lesssim y/h\lesssim 0.45$ ($18\lesssim y^+\lesssim 81$) and is a comparable to the fluid-phase Reynolds shear stress towards the centerline. Conversely, figure \ref{fig:tsscomparest6} shows that $\Sto^+ = 6$ particles cause a drop of the viscous stress. This is expected since the fluid mass flow rate reduces with these particles. $\Sto^+ = 6$ particles also cause significantly lower fluid-phase Reynolds shear stress, although slightly less than $\Sto^+ = 30$ particles since the peak $\rho_{f}\langle u_{f}'v_f'\rangle$ drops to only 46\% of the unladen case. Further, $\Sto^+ = 6$ particles  cause slightly larger particle-phase shear stress.

Figure \ref{fig:rss_pss_comparison} shows the effect of varying mass loading on the fluid and particle shear stresses. For both $\Sto^+=30$ and $\Sto^+=6$ particles, the particle shear stress rises with increasing mass loading while the fluid Reynolds shear stress drops. As shown by the relationships (\ref{eq:RSS_6}) and (\ref{eq:MFR_2}), the competition between increasing particle shear stress and reducing fluid Reynolds shear stress is what ultimately determines whether the particles increase or decrease the mass flow rate, and \emph{a fortiori}, drag reduction or drag increase, respectively. Figure \ref{fig:rss_pss_sum} shows how increasing mass loading causes a progressive deviation of the total Reynolds shear stress  $\rho_{f}\langle u_{f}'v_f'\rangle + \rho_p \langle \phi u_{p}''v_{p}'' \rangle$ from the single phase Reynolds shear stress. It is clear that $\Sto^+=30$ particles reduce the total Reynolds shear stress, although at a rate that varies little from $M=0.6$ to $M=1.0$ suggesting a possible saturation. With $\Sto^+=6$ particles, there is an increase of total Reynolds shear stress which accentuates with increasing mass loading.


\begin{figure}
  \centering
  \begin{subfigure}{0.49\textwidth}
    \centering
    \includegraphics[width=3.20in]{plotting/fig11a.pdf}
    \caption{}
    \label{fig:rsscomparest6}
  \end{subfigure}
  \hfill
  \begin{subfigure}{0.49\textwidth}
    \centering
    \includegraphics[width=3.20in]{plotting/fig11b.pdf}
    \caption{}
    \label{fig:rsscomparest30}
  \end{subfigure}\\
	\begin{subfigure}{0.49\textwidth}
		\centering
		\includegraphics[width=3.20in]{plotting/fig11c.pdf}
		\caption{}
		\label{fig:psscomparest6}
	\end{subfigure}
	\hfill
	\begin{subfigure}{0.49\textwidth}
		\centering
		\includegraphics[width=3.20in]{plotting/fig11d.pdf}
		\caption{}
		\label{fig:psscomparest30}
	\end{subfigure}
  \caption{Fluid-phase and particle-phase Reynolds shear stress for (a,c) $\Sto^+=6$ and (b,d) $\Sto^+=30$ respectively. Symbols as in figure \ref{fig:uavg}. The solid black line denotes the single-phase Reynolds shear stress.}
  \label{fig:rss_pss_comparison}
\end{figure}

\begin{figure}
	\centering
	\begin{subfigure}{0.49\textwidth}
		\centering
		\includegraphics[width=3.20in]{plotting/fig12a.pdf}
		\caption{}
		\label{fig:sscomparest6}
		\end{subfigure}\hfill
		\begin{subfigure}{0.49\textwidth}
		\centering
		\includegraphics[width=3.20in]{plotting/fig12b.pdf}
		\caption{}
		\label{fig:sscomparest30}
\end{subfigure}
\caption{Total Reynolds shear stress for (a) $\Sto^+=6$ and (b) $\Sto^+=30$ respectively. Symbols as in figure \ref{fig:uavg}. The solid black line denotes the single-phase Reynolds shear stress.}
\label{fig:rss_pss_sum}
\end{figure}


\subsection{Interplay between particle clusters and near-wall coherent structures}
\label{sec:mechanism}
In this section, we show that modulating the skin-friction drag depends to a large extent on how particle clusters interact with near-wall coherent structures.

\begin{figure}
  \includegraphics[width=5in]{tikzgraphics/fig13.pdf}
  \caption{Isocontours of normalized particle volume fraction in a wall-normal plane showing the presence of clusters and the accumulation of particles near the walls. As in figure \ref{fig:isocontour_u}, the larger domain for $\Sto^+ = 30$ particles is truncated to the same size as the domain for $\Sto^+ = 6$ particles to facilitate visual comparison.}
  \label{fig:vfp}
\end{figure}
The distribution of $\Sto^+=6$ and $\Sto^+=30$ particles within the channel is strongly inhomogeneous.
Visualization of normalized particle volume fraction in a wall-normal plane in figure \ref{fig:vfp} shows that the particles concentrate in long filamentous clusters that may span the entire channel height.
$\Sto^+=30$ particles form clusters that are relatively denser and further elongated in the streamwise direction compared to clusters formed by $\Sto^+=6$ particles. Figure \ref{fig:vfp} also shows that the normalized particle volume fraction within the bulk of the channel is lower at mass loading $M=0.1$, compared to the bulk normalized volume fraction at $M=0.6$ and 1.0. This points to a tendency of particles to accumulate near the walls that is stronger at $M=0.1$  than at $M=0.6$ and $M=1$.  Note that the formation of such clusters is expected owing to the fact that the particles considered in this study have significant inertia. As previously discussed by several investigators, inertial particles in wall-bounded turbulent flows tend to form clusters due to two effects, namely, turbophoresis, i.e., the migration of inertial particles to lower turbulence regions near the walls \citep{caporaloniTransferParticlesNonisotropic1975,reeksTransportDiscreteParticles1983,nowbaharTurbophoresisAttenuationTurbulent2013,kuertenTurbulenceModificationHeat2011}, and preferential concentration, i.e., the migration of inertial particles from vortical regions to straining regions of the flow \citep{eatonPreferentialConcentrationParticles1994,marchioliStatisticsParticleDispersion2008,kasbaouiTurbulenceModulationSettling2019,fongVelocitySpatialDistribution2019}. It follows that the particle feedback force is concentrated along these structures, and that the resulting flow modulation depends largely on the cluster morphology and dynamics.

\begin{figure}\centering
  \begin{subfigure}{0.49\linewidth}
    \centering
    \includegraphics[width=3.2in]{plotting/fig14a.pdf}
    \caption{}
    \label{fig:pnd_st6}
  \end{subfigure}\hfill
  \begin{subfigure}{0.49\linewidth}
    \centering
    \includegraphics[width=3.2in]{plotting/fig14b.pdf}
    \caption{}
    \label{fig:pnd_st30}
	\end{subfigure}
  \caption{Particle number density normalized by the average particle number density as a function of the wall normal distance for (a) $\Sto^+ = 6$ and (b) $\Sto^+ = 30$ at various mass loadings. Symbols as in figure \ref{fig:uavg}.}
  \label{fig:PND}
\end{figure}

Although particle clusters can be observed throughout the channel, it is near the walls that the majority of particles accumulate. Figure \ref{fig:PND} shows the variation of the normalized plane-averaged volume fraction ${\langle \phi\rangle}/ \phi_0$ with the wall normal distance. Within the region $y^+ < 10$, the local particle volume fraction is several times larger than the mean volume fraction $\phi_0$, which shows that the majority of the particles accumulate near the walls. $\Sto^+=30$ particles lead to the largest wall accumulation reaching ${\langle \phi\rangle}/ \phi_0\simeq 4.98$ at $M=1$ compared to ${\langle \phi\rangle}/ \phi_0\simeq 2.62$ for $\Sto^+=6$ particles at the same mass loading. Similar observations were made by \citet{nilsenVoronoiAnalysisPreferential2013} and \citet{yuanThreedimensionalVoronoiAnalysis2018} who, despite considering only one-way coupling, found that particles with $\Sto^+=30$ have the greatest wall-accumulation among particles with $\Sto^+$ in the range 1-100. Interestingly, the particle wall accumulation reduces when mass loading increases. At $M=0.1$, the particle volume fraction at the wall rises to $\langle\phi\rangle/\phi_0\simeq 14.96$ and 6.7 for $\Sto^+=30$ and $\Sto^+=6$, respectively. This finding is in agreement with the observation from figure \ref{fig:vfp} that the relative bulk particle volume fraction is lowest at $M=0.1$ as relatively more particles accumulate at the walls with decreasing $M$. This effect likely results from two-way coupling, given that particle-particle collisions are weak in the present semi-dilute regime.

Here, we stress that capturing the particle ropes accurately and the subsequent flow modulation requires much larger domains than those generally used in simulations of particle-laden turbulent channel flows \citep{rousonPreferentialConcentrationSolid2001,zhaoTurbulenceModulationDrag2010a,bernardiniReynoldsNumberScaling2014,costaInterfaceresolvedSimulationsSmall2020,jieExistenceFormationMultiscale2022}. The present large domain used for simulations with $\Sto^+=30$ particles is sufficiently wide to allow a natural development of flow and particle structures in the spanwise direction. However, even with a streamwise length of $12\pi h\sim 38h$, the domain remains too short to properly characterize the average streamwise length of the particle ropes.

\begin{figure}
    \centering
  \begin{subfigure}{\linewidth}
    \centering
    \includegraphics[width=4.5in]{tikzgraphics/fig15a.pdf}
    \caption{}
    \label{fig:cluster_length_st30_2d1}
  \end{subfigure}
  \begin{subfigure}{\linewidth}
    \centering
    \includegraphics[width=4in]{tikzgraphics/fig15b.pdf}
    \caption{}
    \label{fig:cluster_length_st30_2d2}
  \end{subfigure}
  \begin{subfigure}{\linewidth}
    \centering
    \includegraphics[width=4in]{tikzgraphics/fig15c.pdf}
    \caption{}
    \label{fig:cluster_length_st6_2d1}
\end{subfigure}
  \caption{\textcolor{revision}{Isocontours of normalized particle volume fraction at $y^+ = 10$ for (a,b) $\Sto^+ = 30, M = 1.0$ and (c) $\Sto^+ = 6, M = 1.0$. The view in (b) corresponds to the area marked by the red rectangle in (a).}}
  \label{fig:cluster_length}
\end{figure}
With most of the particles concentrating near the walls, clusters found therein have the largest impact on the carrier flow. As shown in figure \ref{fig:cluster_length}, the topology of these structures varies significantly depending on whether the particles are drag-reducing ($\Sto^+=30$) or drag-increasing ($\Sto^+=6$). For better comparison of the scales, figure \ref{fig:cluster_length_st30_2d2} shows a view of the particle volume fraction field for $\Sto^+=30$ particles cropped to the same dimensions as the smaller domain used with $\Sto^+=6$ particles and shown in figure \ref{fig:cluster_length_st6_2d1}. In contrast with $\Sto^+=6$ particles, the higher inertia particles at $\Sto^+=30$ form distinctively long and stable clusters. These structures, which we call \emph{ropes}, span the entire length of the domain in the streamwise direction, i.e, over 6000 wall units. The ropes travel downstream but remain stable and coherent for dynamically significant times. Further, the ropes repeat periodically in the spanwise direction in a fashion reminiscent of low-speed streaks discussed in \S\ref{sec:unladen}. This suggests that formation of these ropes results from the interaction of particle clusters with coherent flow structures in the buffer layer. The fact that no such ropes are observed with $\Sto^+=6$ particles suggests that intermittent flow structures in the buffer layer are capable of breaking down clusters formed by low inertia particles, whereas clusters formed by particles with large inertia retain their spatial and temporal coherence. The stable particle ropes may in turn alter the near-wall coherent flow structures.

\begin{figure}
    \centering
    \begin{subfigure}{\linewidth}
      \centering
      \includegraphics[width=4.5in]{tikzgraphics/fig16a.pdf}
      \caption{$\Sto^+ =30, M = 1.0$}
      \label{fig:riblet_S30M10_large}
    \end{subfigure}
    \begin{subfigure}{\linewidth}
      \centering
      \includegraphics[width=4in]{tikzgraphics/fig16b.pdf}
      \caption{$\Sto^+ =30, M = 1.0$}
      \label{fig:riblet_S30M10}
    \end{subfigure}
    \begin{subfigure}{\linewidth}
    \centering
    \includegraphics[width=4in]{tikzgraphics/fig16c.pdf}
    \caption{$\Sto^+ = 6, M = 1.0$}
    \label{fig:riblet_S6M10}
  \end{subfigure}
  \caption{\textcolor{revision}{Overlay of the isocontours of fluid streamwise velocity, and the contour of the relative particle volume fraction $\phi/\phi_0 = 3$ at $y^+ = 10$ for (a,b) $\Sto=30$, $M=1.0$ and (c) $\Sto^+ = 6, M = 1.0$. The view in (b) corresponds to the area marked by the red rectangle in (a).}}
  \label{fig:riblet_comparison}
\end{figure}
In order to shed light on how particle ropes interact with near-wall coherent flow structures, we report in figure  \ref{fig:riblet_comparison} isocontours of streamwise velocity at $y^+ = 10$ with the iso-level $\phi=3\times\phi_0$ overlayed on top. The latter shows the regions where the particles cluster. For the flow laden with $\Sto^+ = 30$ particles at $M=1$, we observe that the long ropes align well with the low-speed streaks, showing that the dynamics of these two coherent structures are interlinked. Compared to the unladen flow (see figure \ref{fig:streak_SP_a}), the low-speed streaks are visibly further elongated in a way similar to how the particle ropes extend in the streamwise direction. The spanwise spacing of the low-speed streaks also increases and appears comparable to the spanwise spacing of the ropes. In the case of the flow laden with $\Sto^+=6$ particles at $M=1$, the clusters are also primarily found in the low-speed streaks. However, the streamwise length of these clusters is much shorter in comparison with the low-speed streaks and with the ropes formed by $\Sto^+ = 30$ particles. In addition, the streamwise length and spanwise spacing of low-speed streaks increase compared to the particle-free flow, although not to the same extent as with $\Sto^+=30$ particles.



\begin{figure}
	\centering
	\begin{subfigure}{0.45\linewidth}
		\includegraphics[width=\linewidth]{plotting/fig17a.pdf}
		\caption{\label{fig:spacing_st30_f}}
	\end{subfigure}
	\hfill
	\begin{subfigure}{0.45\linewidth}
		\includegraphics[width=\linewidth]{plotting/fig17b.pdf}
		\caption{\label{fig:spacing_st30_p}}
	\end{subfigure}
	\begin{subfigure}{0.45\linewidth}
		\includegraphics[width=\linewidth]{plotting/fig17c.pdf}
		\caption{\label{fig:spacing_st30_p}}
	\end{subfigure}
	\hfill
	\begin{subfigure}{0.45\linewidth}
		\includegraphics[width=\linewidth]{plotting/fig17d.pdf}
		\caption{\label{fig:spacing_st6_p}}
	\end{subfigure}
	\caption{Variation with spanwise spacing of the two-point autocorrelation of the streamwise fluid fluctuations and particle volume fraction fluctuations in the spanwise direction for the (a,b) drag-reducing case $\Sto^+=30$ (\textcolor{blue}{\protect\scalebox{1.25}{$\blacksquare$}}) and (c,d) drag increasing case $\Sto^+ = 6$ (\textcolor{red}{\protect\scalebox{1.75}{$\bullet$}}). Darker symbols correspond to larger mass loading which varies from 0.2 to 1.0. The solid black line represents the particle-free channel flow.
	\label{fig:spacing}}
\end{figure}
To characterize quantitatively the spanwise spacing of particle clusters and their impact on the spanwise spacing of low-speed streaks, we compute the two-point autocorrelation of the particle volume fraction fluctuations,
\begin{equation}
	    R^p_{\phi\phi}(\Delta z;y_0)=\frac{\langle{\phi' (x,y_0,z,t)\phi' (x,y_0,z+\Delta z,t)}\rangle}{\langle{\phi'^2}\rangle},
\end{equation}
and the the two-point autocorrelation of the streamwise velocity fluctuations $R^f_{uu}$. Figure \ref{fig:spacing} shows the variation $ R^p_{\phi\phi}$ and $ R^f_{\phi\phi}$ with spanwise spacing at $y^+=10$. Similar to how the low-speed streak spacing $\lambda_f^+$ is defined, we define $\lambda_p^+$, the spanwise spacing of particle clusters, as twice the distance between the origin and $\Delta z^+$ where $R^p_{\phi\phi}$ reaches a first minimum.

\begin{table}
  \caption{Spanwise spacing of the low-speed streaks and particle ropes. \label{tab:spacing}}
  \begin{ruledtabular}
    \begin{tabular}{llll}
      Stokes number ($\Sto^+$) & Mass loading ($M$) & $\lambda^+_f$ & $\lambda^+_p$ \\\hline
      (Particle-free) & 0 & 106 & $-$ \\
      6             & 0.2          & 125      & 99     \\
                    & 0.6          & 134      & 108    \\
                    & 1.0          & 116      & 90     \\
      30            & 0.2          & 126      & 108    \\
                    & 0.6          & 161      & 130    \\
                    & 1.0          & 170	  & 135    
    \end{tabular}
  \end{ruledtabular}
\end{table}

Table \ref{tab:spacing} shows the values of $\lambda^+_f$ and $\lambda^+_p$ for all cases simulated.
For the drag-reducing cases at $\Sto^+ = 30$, it is clear that as the mass loading is increased from 0.2 to 1.0 the low-speed streak spanwise spacing increases from $\lambda^+_f = 126$ to $170$. These are significant increases compared to the low-speed streak spacing of $\lambda^+_f = 106$ in the particle-free channel. The rope spacing $\lambda_p^+$ increases from $\lambda^+_p = 108$ to $135$ as mass loading is increased. 
\textcolor{revision}{The disparity between $\lambda_p^+$ and $\lambda_f^+$ is likely due to small particle clusters that detach from the main ropes due to the spanwise meandering of ropes and low-speed streaks.}
%
%\textcolor{revision}{We have also reported the difference between the spacing of low speed and particle streaks, i.e., $\lambda_f^+-\lambda_p^+$ in table \ref{tab:spacing}. With the drag-reducing $\Sto^+ = 30$  particles, $\lambda_f^+-\lambda_p^+$ increases from 18 wall units for $M=0.2$ to 35 wall units at $M=1.0$. This suggests that the particle ropes are able to push in and out of the low-speed streaks thanks to the large particle inertia.}
%In doing so, these particles may help suppress coherent structures and relaminarize the near-wall region.
In comparison, $\Sto^+ = 6$ particles lead to substantially lower modulation of the low-speed streaks. As shown in table \ref{tab:spacing}, the spanwise spacing of the low-speed streaks varies between $\lambda^+_f = 116$ and $134$ when $\Sto^+ = 6$ particles are dispersed. The corresponding spacing of particle clusters varies in the range of $\lambda^+_p = 90-108$, \textcolor{revision} {with less disparity between $\lambda^+_p$ and $\lambda^+_f$ compared to the flow laden with $\Sto^+ = 30$ particles. This suggests that $\Sto^+ = 6$ clusters are more closely aligned with the high-strain low-vorticity regions found within the low-speed streaks, likely due to their lower inertia.}
%
%\textcolor{revision}{ but the difference $\lambda_f^+-\lambda_p^+$ remains constant at 26 wall units for all mass loadings considered. This lower disparity between low-speed streaks and particle spacing suggests that  $\Sto^+ = 6$ particles are primarily located close to the low-speed streaks and that their comparatively lower inertia prevents them from escaping further way}.

Note that the two-way coupling plays a critical role in the arrangement of low-speed streaks and particle clusters. In a prior study by \citet{jieExistenceFormationMultiscale2022}, where the authors considered one-way coupled Euler-Lagrange simulations of particle-laden channel flows at $\Rey_\tau$ between 600 and 2000, the absence of feedback force from the particles leads to low-speed streaks that have identical characteristics to those of a particle-free turbulent channel flow. The data presented by the authors further suggests that the particle cluster spanwise spacing varies little with Reynolds number and is about $\lambda_p^+ \sim 114$ for $\Sto^+ = 30$ particles. However, as we have shown in this study,  $\lambda_p^+$ and $\lambda_f^+$ reach considerably higher values when two-way coupling is significant since the dynamics of clusters and near-wall coherent structures become more inter-dependent.

\begin{figure}
  \centering
  \includegraphics[width=4in]{tikzgraphics/fig18.pdf}
  \caption{\textcolor{revision}{Instantaneous velocity vectors overlayed by contour of particle volume fraction $\phi/\phi_0 = 3$, for the case $\Sto^+ = 30, M = 1.0$, show particle ropes forming in the high strain region between quasi-streamwise vortices.}}
  \label{fig:vorticity}
\end{figure}

%The mechanism underpinning the modulation of low-speed streaks relates to how inertial particles interact with quasi-streamwise vortices surrounding these streaks.
Figure \ref{fig:vorticity} shows an example of how $\Sto^+=30$ particles are distributed in the vicinity of a pair of quasi-streamwise vortices. The particles form ropes by concentrating in the straining region between the pair of vortices, consistently with the preferential concentration mechanism. Pockets of particles can be seen ejected upward towards the centerline, which results in a downward feedback force on the fluid. This process is self-sustaining because the ejected particles eventually return to the near-wall region due to turbophoresis, where they accumulate again along particle ropes. The feedback force from these clusters contributes to the the suppression of bursting and stabilization of quasi-streamise vortices. Consequently, low-speed streaks nested in-between quasi-streamwise vortices extend further than possible in particle-free flows. Because bursting events contribute largely to the Reynolds shear stress production \citep{willmarthStructureReynoldsStress1972}, the stabilizing role of  $\Sto^+=30$ particles is likely the main reason these particles reduce the fluid-phase Reynolds shear stress to the extent shown in \S \ref{sec:stress_two_phase}, and \emph{in fine}, skin-friction drag reduction.


\section{Conclusion}\label{sec:conclusion}
In this work, we focus on addressing the fundamental challenge of OOD detection tasks, which is how to fully understand the semantic discrepancy between the ID/OOD samples. We reveal that the key to success in the realistic SCOOD task is to allocate as many ID samples in the unlabeled set correctly as possible. To this end, we propose a novel uncertainty-aware optimal transport scheme that introduces class-specific energy scores as guidance for effective label assignment. Experimental results show that our method achieves better performance than previous state-of-the-art methods on SCOOD benchmarks.

\textbf{Limitations.} In addition to temperature scaling, other techniques such as feature clipping applied in ReAct~\cite{sun2021react} also enhance the performance of energy score, so how to obtain an OOD score that best fits the SCOOD task can be further explored. Moreover, a setting highly related to SCOOD has been proposed in \cite{katz2022training} and formulated as a constrained optimization problem. We will also theoretically analyze these practical OOD settings in our feature work.

% \section*{Acknowledgments}
\textbf{Acknowledgments.} 
This work is supported by National Key R\&D Program of China under Grant 2020AAA0105701, National Natural Science Foundation of China (NSFC) under Grants 61872327, Major Special Science and Technology Project of Anhui, National Natural Science Foundation of China (62033012) and Ant Group through Ant Research Intern Program.


%\appendix
%\section{Old write-up}
%\subsection{Velocity modulation due to particle-fluid interaction}

There has been considerable interest in recent years on drag modulation due to particles within a channel flow. Many studies have been performed using various methods such as the eulerian-lagrangian point particle method \citep{zhouNonmonotonicEffectMass2020,LagrangianStatisticsTurbulent1995,bakerDirectComparisonEulerian2020}, the immersed boundary approach \citep{picanoTurbulentChannelFlow2015,breugemSecondorderAccurateImmersed2012} and the lattice Boltzmann method \citep{pengDirectNumericalInvestigation2019,wangLatticeBoltzmannSimulation2016,zhangLatticeBoltzmannMethod2016}. Studies however have not been able to present a clear picture as to how different particle characteristics (Stokes number, mass loading) affect the drag within a flow. This section aims to understand the velocity statistics for low-moderate mass loading (0.2-0.6) at Stokes numbers of O(1-10) [Cases \textbf{A-C} ($\Sto^+ = 6$) and cases \textbf{D-F} ($\Sto^+ = 30$)], using the eulerian-lagrangian point particle approach, which has been a proven method for particle-laden flows (**reference needed**). 

\begin{figure}[htp]
  \begin{subfigure}{0.45\textwidth}
    \centering
    \includegraphics[width=1\linewidth]{plotting/AVGVEL_ST6.pdf}
    \caption{}
    \label{fig:fvelst6}
  \end{subfigure}
  \begin{subfigure}{0.45\textwidth}
    \centering
    \includegraphics[width=1\linewidth]{plotting/FURMS_ST6.pdf}
    \caption{}
    \label{fig:furmsst6}
  \end{subfigure}
  \caption{(a) Streamwise velocity and (b) streamwise velocity fluctuations plotted as a function of $y^+$ for $St^+ = 6$.} 
  \label{fig:fluid_stats_st6}
\end{figure}

\begin{figure}[htp]
  \begin{subfigure}{0.45\textwidth}
    \centering
    \includegraphics[width=1\linewidth]{plotting/AVGVEL_ST30.pdf}
    \caption{}
    \label{fig:fvelst30}
  \end{subfigure}
  \begin{subfigure}{0.45\textwidth}
    \centering
    \includegraphics[width=1\linewidth]{plotting/FURMS_ST30.pdf}
    \caption{}
    \label{fig:furmsst30}
  \end{subfigure}
  \caption{(a) Streamwise velocity and (b) streamwise velocity fluctuations plotted as a function of $y^+$ for $St^+ = 30$.} 
  \label{fig:fluid_stats_st30}
\end{figure}

Figure \ref{fig:fluid_stats_st6} shows the streamwise velocity $\langle u_f\rangle / u_\tau$ and the streamwise fluctuations $\sqrt{\langle u^{`2}_f \rangle}/u_\tau$ for cases \textbf{A-C}. It is observed that the particles have a weak modulation on the near wall statistics for for $\langle u_f\rangle / u_\tau$ and $\sqrt{\langle u^{`2}_f \rangle}/u_\tau$ at $\Sto^+ = 6$. There is a monotonic reduction observed for $\langle u_f\rangle / u_\tau$ in the outer layer with increasing mass loading. $\sqrt{\langle u^{`2}_f \rangle}/u_\tau$ decreases throughout the channel compared to the single phase flow, with maximum modulation witnessed for case \textbf{C}. 

Cases \textbf{D-F} all present an enhancement in $\langle u_f\rangle / u_\tau$ with maximum increase observed for case \textbf{F} as shown in fig. \ref{fig:fvelst30}. \citet{zhouNonmonotonicEffectMass2020} conduct a similar study as shown for cases \textbf{D-F} at $\Rey_\tau = 180$ and $\Sto^+ = 30$. They observe a similar increase in $\langle u_f\rangle / u_\tau$ within the log-outer layer, with the exception that the present study does not observe the near center-line reduction that was witnessed by \citet{zhouNonmonotonicEffectMass2020}. There is a significant near-wall reduction in $\sqrt{\langle u^{`2}_f \rangle}/u_\tau$, and a monotonic increase in the log-outer layer with increasing mass loading for cases \textbf{D-F} as shown in fig. \ref{fig:furmsst30}. A similar behaviour was observed by \citet{zhouNonmonotonicEffectMass2020}. 

The modulation of the streamwise velocity fluctuations, more specifically within the log-outer layer show a direct correlation to the increase in the overall mass flow rate. As the log-outer layer $\sqrt{\langle u^{`2}_f \rangle}/u_\tau$ is decreased we see a reduction in the mass flow rate (cases \textbf{A-C}),  and as $\sqrt{\langle u^{`2}_f \rangle}/u_\tau$ is increased in the log-outer layer we see an increase in the mass flow rate (cases \textbf{D-F}). 

The wall normal (fig. \ref{fig:fvrms}) and spanwise (fig. \ref{fig:fwrms}) velocity fluctuations show a significant reduction within the flow for all cases compared to the single phase flow. There is a strong relationship between increasing mass loading and decreasing $\sqrt{\langle v^{`2}_f \rangle}/u_\tau$ and $\sqrt{\langle w^{`2}_f \rangle}/u_\tau$ within the range of $\Sto^+ = 6-30$. A similar reduction is also noticed by \citet{zhouNonmonotonicEffectMass2020}, for particle-laden channel flows run at a similar semi-dilute regime at $St^+ = 30$. The mass loading therefore plays a more significant role than the Stokes number in modulating the fluid wall-normal and spanwise velocity fluctuations in a semi-dilute regime in the range of $St^+ = O(1-10)$. 

The modulation of the streamwise fluid velocity within the channel is therefore dependent largely on the streamwise velocity fluctuations. Enhancement of the mass flow rate can be achieved through particles that have a higher inertia ($\Sto^+ = 30$), due to staying closer to the wall. This helps modulate the fluid characteristics near the wall more strongly, primarily $\sqrt{\langle u^{`2}_f \rangle}/u_\tau$,  than particle with a lower inertia ($\Sto^+ = 6$). Particles with a larger inertia tend to stay closer to the wall on average throughout the channel compared to lower inertia paticles which are more spread out within the channel. This is shown clearly in fig. \ref{fig:pnd}, where the particle number density is plotted compared to the distance away from the wall. From fig. \ref{fig:fluid_stats_st6}, \ref{fig:fluid_stats_st30} and \ref{fig:fluid_stats_rms}, we can notice that particles help reduce the velocity fluctuations around them. However, it is observed that greater log-outer layer streamwise fluid velocity fluctuations helps aid the enhancement in the mass flow rate. This is easier done through particles of higher inertia that stay closer to the wall reducing the near wall streamwise fluid fluctuations significantly increasing the log-outer layer fluctuations, enhancing the overall mass flow rate. In contrast lower inertia particles tend to be scattered throughout the channel, not delivering the effect that higher inertia particles do in modulation the fluid flow.       

\begin{figure}[htp]
  \begin{subfigure}{0.45\textwidth}
    \centering
    \includegraphics[width=1\linewidth]{plotting/FVRMS.pdf}
    \caption{}
    \label{fig:fvrms}
  \end{subfigure}
  \begin{subfigure}{0.45\textwidth}
    \centering
    \includegraphics[width=1\linewidth]{plotting/FWRMS.pdf}
    \caption{}
    \label{fig:fwrms}
  \end{subfigure}
  \caption{(a) Wall-normal and (b) spanwise fluctuations plotted as a function of $y^+$.} 
  \label{fig:fluid_stats_rms}
\end{figure}

The present study aims to understand the behaviour of the particle phase compared to the fluid phase as we move away from the wall. Figure \ref{fig:fluid_particle_vel_stats}, \ref{fig:global_fluid_particle_velfluc_stats} and \ref{fig:pc_fluid_particle_velfluc_stats} take a look at the velocity characteristics between the two phases to understand how particle help modulate the fluid due to turbophoresis and preferential concentration. Figure \ref{fig:fluid_particle_vel_stats} compares the streamwise velocity between the particle $\langle u_p \rangle/u_\tau$ and the fluid $\langle u_f \rangle/u_\tau$ phase. Additionally, the fluid streamwise velocity conditioned at particle locations $\langle u_{f\mid p}\rangle/u_\tau$ is also presented. From all the cases it is observed that higher inertial particles ($\Sto^+ = 30$) modulate the fluid by showing a greater streamwise velocity within all regions of the flow i.e. $\langle u_f \rangle/u_\tau$, $\langle u_p \rangle/u_\tau$ and $\langle u_{f\mid p}\rangle/u_\tau$ compared to lower inertial particles ($\Sto^+ = 6$). Furthermore, All particle-laden cases show a similar behaviour overall with the particle phase lagging the fluid phase in the near-wall region. As the mass loading is increased the near wall streamwise velocity difference between the particle phase and the fluid phase is increased. However, the fluid at the particle locations significantly lags the overall flow. The magnitude of $\langle u_{f\mid p}\rangle/u_\tau$ is dependent on the inertia of the particles. Cases \textbf{D-F} show a higher $\langle u_{f\mid p}\rangle/u_\tau$ compared to cases \textbf{A-C} within all layers of the turbulent boundary layer and within the outer region.     

\begin{figure}[htp]
  \begin{subfigure}{0.45\textwidth}
    \centering
    \includegraphics[width=1\linewidth]{plotting/FPFPLU02.pdf}
    \caption{M = 0.2}
    \label{fig:fpfplu02}
  \end{subfigure}
  \begin{subfigure}{0.45\textwidth}
    \centering
    \includegraphics[width=1\linewidth]{plotting/FPFPLU02.pdf}
    \caption{M = 0.4}
    \label{fig:fpflpu04}
  \end{subfigure}
  \begin{subfigure}{0.45\textwidth}
    \centering
    \includegraphics[width=1\linewidth]{plotting/FPFPLU06.pdf}
    \caption{M = 0.6}
    \label{fig:fpflpu06}
  \end{subfigure}
  \caption{Fluid particle velocity comparison} 
  \label{fig:fluid_particle_vel_stats}
\end{figure}

To understand the particle modulation on the fluid further, fig. \ref{fig:global_fluid_particle_velfluc_stats} takes a look at the global fluid and particle velocity fluctuations in comparison. Figure \ref{fig:pfurms_st6} (\textbf{A-C}) and \ref{fig:pfurms_st30} (\textbf{D-F}) show the streamwise velocity fluctuations comparison between the particle and fluid phase. It is observed for the streamwise velocity fluctuations that the particle phase leads the fluid in the near wall region and lags the fluid in the outer layer for all inertial particles. Mass loading does not play a significant role in the modulation of the streamwise velocity fluctuations for particles of inertia $\Sto^+ = 6$, however as the inertia is increased there is monotonic increase in the fluid and particle streamwise velocity fluctuations as the mass loading is increased. The wall normal (fig. \ref{fig:pfvrms_st6},\ref{fig:pfvrms_st30}) and spanwise (fig. \ref{fig:pfwrms_st6},\ref{fig:pfwrms_st30}) velocity fluctuations on the other hand, both have the particle phase lagging the fluid phase and mass loading plays a role for all inertial particles. There is a monotonic decrease in both the fluid and particle wall normal and spanwise velocity fluctiations as mass loading is increased.    

\begin{figure}[htp]
  \begin{subfigure}{0.45\textwidth}
    \centering
    \includegraphics[width=1\linewidth]{plotting/PFURMS_ST6.pdf}
    \caption{$St^+ = 6$}
    \label{fig:pfurms_st6}
  \end{subfigure}
  \begin{subfigure}{0.45\textwidth}
    \centering
    \includegraphics[width=1\linewidth]{plotting/PFURMS_ST30.pdf}
    \caption{$St^+ = 30$}
    \label{fig:pfurms_st30}
  \end{subfigure}
  \begin{subfigure}{0.45\textwidth}
    \centering
    \includegraphics[width=1\linewidth]{plotting/PFVRMS_ST6.pdf}
    \caption{$St^+ = 6$}
    \label{fig:pfvrms_st6}
  \end{subfigure}
  \begin{subfigure}{0.45\textwidth}
    \centering
    \includegraphics[width=1\linewidth]{plotting/PFVRMS_ST30.pdf}
    \caption{$St^+ = 30$}
    \label{fig:pfvrms_st30}
  \end{subfigure}
  \begin{subfigure}{0.45\textwidth}
    \centering
    \includegraphics[width=1\linewidth]{plotting/PFWRMS_ST6.pdf}
    \caption{$St^+ = 6$}
    \label{fig:pfwrms_st6}
  \end{subfigure}
  \begin{subfigure}{0.45\textwidth}
    \centering
    \includegraphics[width=1\linewidth]{plotting/PFWRMS_ST30.pdf}
    \caption{$St^+ = 30$}
    \label{fig:pfwrms_st30}
  \end{subfigure}
  \caption{Global fluid particle velocity fluctuations comparison} 
  \label{fig:global_fluid_particle_velfluc_stats}
\end{figure}

The modulatin of the fluid velocity fluctuations by the particles is better understood by taking a look at the fluid velocity fluctuations at particle locations. Figure \ref{fig:pc_fluid_particle_velfluc_stats} shows the particle velocity fluctuations and the fluid velocity fluctuations at the particle locations. The reduction in the near-wall fluid streamwise velocity fluctuations $\sqrt{\langle u^{`2}_f \rangle}/u_\tau$, is due to the partiles modulating the fluid significantly. $\sqrt{\langle u^{`2}_{f \mid p} \rangle}/u_\tau$ is notably reduced near the wall (i.e. fig. \ref{fig:pcpfurms_st6}, \ref{fig:pcpfurms_st30}), helping reduce the overall $\sqrt{\langle u^{`2}_f \rangle}/u_\tau$ in that region. As observed in fig. \ref{fig:global_fluid_particle_velfluc_stats}, $\sqrt{\langle v^{`2}_p \rangle}/u_\tau$ and $\sqrt{\langle w^{`2}_p \rangle}/u_\tau$ have always been lower than the fluid counterpart. Furthermore, these particles tend to modulate the fluid strongly by reducing the $\sqrt{\langle v^{`2}_f \rangle}/u_\tau$ and $\sqrt{\langle w^{`2}_f \rangle}/u_\tau$ compared to the single phase flow. This modulation is better observed in fig. \ref{fig:pcpfvrms_st6}, \ref{fig:pcpfvrms_st30} (wall-normal velocity fluctuations) and fig. \ref{fig:pcpfwrms_st6}, \ref{fig:pcpfwrms_st30} (spanwise velocity fluctuations). The particles help significantly reduce the fluid velocity fluctuations at particle locations. This in turn helps reduce the global fluid velocity fluctuations. A common observation across all particle-laden cases is that the near-wall modulation of the fluid is stronger than the outer layer modulation. This is due to turbophoresis pushing the majority of the particles closer to the wall. Particles with a higher inertia ($St^+ = 30 > St^+ = 6$) tend to be 'heavier' and stay closer to the wall modulating the fluid more strongly. Figures \ref{fig:fluid_stats_st6} - \ref{fig:pc_fluid_particle_velfluc_stats} show that the more the near-wall $\sqrt{\langle u^{`2}_f \rangle}/u_\tau$ is reduced (in turn increasing the log-outer layer $\sqrt{\langle u^{`2}_f \rangle}/u_\tau$), the stronger the increase in the mass flow rate. Particles with a higher Stokes number tend to be able to perform better in increasing the mass flow rate because they tend the modulate the near-wall statistics more strongly that lower Stokes number particles. On the other hand, lower Stokes number particles tend to modulate the flow weakly throughout the channel, but also tend to modulate the flow in regions (i.e. log-outer layer) where drag enhancement is produced.    

\begin{figure}[htp]
  \begin{subfigure}{0.45\textwidth}
    \centering
    \includegraphics[width=1\linewidth]{plotting/PC_PFURMS_ST6.pdf}
    \caption{$St^+ = 6$}
    \label{fig:pcpfurms_st6}
  \end{subfigure}
  \begin{subfigure}{0.45\textwidth}
    \centering
    \includegraphics[width=1\linewidth]{plotting/PC_PFURMS_ST30.pdf}
    \caption{$St^+ = 30$}
    \label{fig:pcpfurms_st30}
  \end{subfigure}
  \begin{subfigure}{0.45\textwidth}
    \centering
    \includegraphics[width=1\linewidth]{plotting/PC_PFVRMS_ST6.pdf}
    \caption{$St^+ = 6$}
    \label{fig:pcpfvrms_st6}
  \end{subfigure}
  \begin{subfigure}{0.45\textwidth}
    \centering
    \includegraphics[width=1\linewidth]{plotting/PC_PFVRMS_ST30.pdf}
    \caption{$St^+ = 30$}
    \label{fig:pcpfvrms_st30}
  \end{subfigure}
  \begin{subfigure}{0.45\textwidth}
    \centering
    \includegraphics[width=1\linewidth]{plotting/PC_PFWRMS_ST6.pdf}
    \caption{$St^+ = 6$}
    \label{fig:pcpfwrms_st6}
  \end{subfigure}
  \begin{subfigure}{0.45\textwidth}
    \centering
    \includegraphics[width=1\linewidth]{plotting/PC_PFWRMS_ST30.pdf}
    \caption{$St^+ = 30$}
    \label{fig:pcpfwrms_st30}
  \end{subfigure}
  \caption{Fluid conditioned at particle locations and the particle phase velocity fluctuations comparison} 
  \label{fig:pc_fluid_particle_velfluc_stats}
\end{figure}



%\subsection{Particle-fluid interaction}

\begin{figure}[htp]
	\begin{subfigure}{0.45\textwidth}
	    \centering
	    \includegraphics[width=1\linewidth]{plotting/PRE.pdf}
	    \caption{}
	    \label{fig:pre}
	\end{subfigure}
	\begin{subfigure}{0.45\textwidth}
	    \centering
	    \includegraphics[width=1\linewidth]{plotting/PRESCALED.pdf}
	    \caption{}
	    \label{fig:pre_scaled}
	\end{subfigure}		
	\caption{Particle Reynolds number shows that within the log-layer region and even further away from it particles do present a wake and influence the flow around them.}
	\label{fig:pre_stats}
\end{figure} 

Particle Reynolds number, $\Rey_p$ is a good parameter to understand how particles modulate the flow based on their physical properties (i.e. particle diameter, Stokes number and particle density). We begin investigating this particle-phase of the channel flow by measuring the particle Reynolds number, $\Rey_p$ within the flow. Figure \ref{fig:pre} shows that $\Rey_p$ is the highest in the log-layer region. It is known from theory \citep{bagchiDirectNumericalSimulation2001}\citep{richterDragForcesHeat2012} that as $\Rey_p$ increases particles create a greater wake, affecting fluid statistics and also nearby particle movement, especially within the log layer maximum flux in all quantities is observed. Furthermore, for the $\Rey_p$ range observed within fig. \ref{fig:pre}, it is inferred that the drag law proposed by \citet{tennetiDragLawMonodisperse2011a} is appropriate. 

To further understand the modulation of the mean streamwise velocity $U_f^+$, \ref{fig:pre_scaled} shows the scaled particle Reynolds number, $\Rey_p/\sqrt{\Sto^+}$. All cases seem to collapse closely, with case \textbf{C} and \textbf{F} showung a slight increase in the log layer region. This is in line with what is witnessed in fig. \ref{fig:fvelst6} and \ref{fig:fvelst30}, showing that it is the slip velocity that controls the intensity of the modulation performed on the fluid by the inertial particles. However, it is the inertia that control whether we see an increase or decrease in the mass flow rate. The higher the slip velocity within the log layer, the greater the modulation. Figure \ref{fig:fvel} shows that case \textbf{C} has masximum drag and also consequently has the highest slip velocity for particles at $\Sto^+ = 6$. Case \textbf{F} on the other hand mas maximum mass flow rate recorded and consequently also has the higher slip velocity recorded in the log region for particles with $\Sto^+ = 30$.

\begin{figure}[htp]
  \begin{subfigure}{0.8\textwidth}
    \centering
    \includegraphics[width=1\linewidth,trim={0ex 8ex 0ex 8ex},clip]{images/vfp_s6m02.png}
    \caption{$ St^+ = 6, M = 0.2$}
    \label{fig:vfp_s6m02}
  \end{subfigure}
  \begin{subfigure}{0.8\textwidth}
    \centering
    \includegraphics[width=1\linewidth,trim={0ex 8ex 0ex 8ex},clip]{images/vfp_s6m04.png}
    \caption{$St^+ = 6, M = 0.4$}
    \label{fig:vfp_s6m04}
  \end{subfigure}
  \begin{subfigure}{0.8\textwidth}
    \centering
    \includegraphics[width=1\linewidth,trim={0ex 8ex 0ex 8ex},clip]{images/vfp_s6m06.png}
    \caption{$St^+ = 6, M = 0.6$}
    \label{fig:vfp_s6m06}
  \end{subfigure}
  \begin{subfigure}{0.8\textwidth}
    \centering
    \includegraphics[width=1\linewidth,trim={0ex 8ex 0ex 8ex},clip]{images/vfp_s30m02.png}
    \caption{$St^+ =30, M = 0.2$}
    \label{fig:vfp_s30m02}
  \end{subfigure}
  \begin{subfigure}{0.8\textwidth}
    \centering
    \includegraphics[width=1\linewidth,trim={0ex 8ex 0ex 8ex},clip]{images/vfp_s30m04.png}
    \caption{$St^+ =30, M = 0.4$}
    \label{fig:vfp_s30m04}
  \end{subfigure}
  \begin{subfigure}{0.8\textwidth}
    \centering
    \includegraphics[width=1\linewidth,trim={0ex 8ex 0ex 8ex},clip]{images/vfp_s30m06.png}
    \caption{$St^+ =30, M = 0.6$}
    \label{fig:vfp_s30m06}
  \end{subfigure}
  \caption{Particle volume fraction normalized by the average volume fraction showing formation of clusters within the flow in high strain regions and the accumalation of particles near the wall through turbophoresis}
  \label{fig:vfp}
\end{figure}

Particle laden flows with a semi-dilute concentration have a substantial effect on fluid characteristics mainly due to two phenomenons that take place simultaneously within a particle-laden flow. Preferential concentration ejects particles from high vorticity (low strain) regions towards low vorticity (high strain) regions of the flow. This can help remove energy from these high vorticity regions as they get ejected outwards. Moreover, the effect of turboproresis near the wall allows for particles to segregate much closer to the wall with time as the the flow is driven within the channel. This effect of turbophoresis that aids in a higher localized volume fraction near the wall increases the effect of preferential concentration aiding in formations of clusters near the wall. To further take a look into this effect fig. \ref{fig:vfp} shows the particle volume fraction normalized by its average volume fraction and fig. \ref{fig:pnd} shows the particle number density relative to their distance from the wall.
      
\begin{figure}[htp]
    \centering
    \includegraphics[scale=0.5]{plotting/PND.pdf}
    \caption{Particle number density does not show strong dependence on mass loading in the semi-dilute regime as we can see all cases closely resembling each other apart from near the wall where we see maximum accumalation of particler as $\Sto^+$ increases.}
    \label{fig:pnd}
\end{figure}  

All cases show the effects of turbophoresis where maximum accumalation is observed near the wall however, heavier particles (i.e. larger Stokes numbers) tend to stay closer to the wall compared to lighter particles (i.e. smaller Stokes numbers). Previous studies by \citet{sardinaWallAccumulationSpatial2012} show greater accumalation of inertial particles near the wall as the Stokes number is increased. Figure \ref{fig:vfp} shows dense clustering within the channel as the mass loading is increased. Large particle clusters are observed near the wall for cases \textbf{D}, \textbf{E} and \textbf{F} compared to the center of the flow because of their high inertia. Contrastingly cases \textbf{A}, \textbf{B} and \textbf{C} show greater ejection of particles towards the center of the channel by vortical structures away from the wall owing to lower inertia. The length scales of the clusters for particles of $\Sto^+  =6$ stay consistent throughout the channel and become denser as the mass loading is increased, while the length scales for particles of $\Sto^+ = 30$ are higher near the wall than in the center of the flow and become denser as the mass loading is increased. This behaviour can be further witnessed in figure \ref{fig:pnd}. $\Sto^+ = 30$ cases show greater particle number density near the wall compared to $\Sto^+ = 6$ while having lesser particle number density away from the wall. It can therefore be hypothesized that larger Stokes numbers modulate the fluid largely near the wall while smaller Stokes numbers modulate the flow in all regions of the channel. This is consistent from trends we have seen in previous studies by \citet{fongVelocitySpatialDistribution2019}. Particles of $\Sto^+$ of O(10) display greater modulation of the fluid compared to particles of $\Sto^+$ of O(1) because of the large macro-scale focres they apply on the fluid due to the dense clustering near the wall.

\begin{figure}
    \centering
    \includegraphics[width=0.7\linewidth]{images/3driblet.png}
    \caption{Particles forming large riblets throughout the channel at high Stokes numbers ($\Sto^+ = 30, M = 0.6$}
    \label{fig:3driblet}
\end{figure}

To further understand the movement of oartucles near the wall and how inertia and mass loading affect the clustering, a detailed picture is shown in figure \ref{fig:3driblet} and \ref{fig:riblet_comparison}. Figure \ref{fig:3driblet} shows long riblets of particles forming near the wall for case \textbf{F}. Previous experiments have shown that particles arrange themselved in low-speed regions of the flow near the wall. Figure \ref{fig:riblet_comparison} shows the contours of the particles volume fraction for all cases (3 times the average volume fraction for particles of $\Sto^+ = 6$ and 5 times the average volume fraction for particles of $\Sto^+ = 30$) near the wall over the fluid streamwise velocity. The effect of preferential concentration is very evident in all cases \textbf{A}-\textbf{F}. Particles tend to get ejected from high speed (high vorticity) regions to low speed (low vorticity) regions. FOr particles of $\Sto^+ = 6$ (fig. \ref{fig:nribletS6M02}, \ref{fig:nribletS6M04} and \ref{fig:nribletS6M06}), a large number of low-speed regions are present with small clustering of particles near the wall within all these low-speed regions. This is due to the low inertia of the particles, making them `lighter' and allowing them to drift with more ease. Even though, the effect of preferential concentration is evident for particles at $\Sto^+ = 6$, they seem to have a higher level of movement near the wall. This shows that the fluid is strongly modulating the particles, just as equally as the particles are modulating the fluid near the wall when inertia is low. Contrastingly, cases at $\Sto^+ = 30$ show longer more well defined low-speed regions. Particles have larger clusters, multiple length scales bigger than particle clusters at $\Sto^+ = 6$. As the mass loading is increased the long particle clusters (i.e. `riblets') become more defined (fig. \ref{fig:nribletS30M06}). A three dimentional representation of that is shown in fig. \ref{fig:3driblet}. Low speed regions become thinner but also longer (spanning the entire domain). This is consistent with data on particle-laden flows with larger inertia particles. They tend to be more `heavier', meaning less movement near the wall. However the effect of preferential concentration is still visible with particles forming long riblets within the low speed regions, with minimal to almost non in the high speed regions. For particle-laden cases at $\Sto^+ = 30$, there are relatively less number of particle clusters but longer clusters than flows with inertial particles at $\Sto^+ = 6$. Fewer clustering means fewer movement of particles in and out of different vorticity regions of the flow. The particles tend to modulate the fluid more stronglt than the fluid modulating the particles for higher inertia particle-laden cases.   

\begin{figure}[htp]
  \begin{subfigure}{0.75\textwidth}
    \centering
    \includegraphics[width=1\linewidth,trim={0ex 8ex 0ex 8ex},clip]{images/nribletS6M02.png}
    \caption{$ \Sto^+ = 6, M = 0.2$}
    \label{fig:nribletS6M02}
  \end{subfigure}
  \begin{subfigure}{0.75\textwidth}
    \centering
    \includegraphics[width=1\linewidth,trim={0ex 8ex 0ex 8ex},clip]{images/nribletS6M04.png}
    \caption{$\Sto^+ = 6, M = 0.4$}
    \label{fig:nribletS6M04}
  \end{subfigure}
  \begin{subfigure}{0.75\textwidth}
    \centering
    \includegraphics[width=1\linewidth,trim={0ex 8ex 0ex 8ex},clip]{images/nribletS6M06.png}
    \caption{$\Sto^+ = 6, M = 0.6$}
    \label{fig:nribletS6M06}
  \end{subfigure}
  \begin{subfigure}{0.75\textwidth}
    \centering
    \includegraphics[width=1\linewidth,trim={0ex 8ex 0ex 8ex},clip]{images/nribletS30M02.png}
    \caption{$\Sto^+ =30, M = 0.2$}
    \label{fig:nribletS30M02}
  \end{subfigure}
  \begin{subfigure}{0.75\textwidth}
    \centering
    \includegraphics[width=1\linewidth,trim={0ex 8ex 0ex 8ex},clip]{images/nribletS30M04.png}
    \caption{$\Sto^+ =30, M = 0.4$}
    \label{fig:nribletS30M04}
  \end{subfigure}
  \begin{subfigure}{0.75\textwidth}
    \centering
    \includegraphics[width=1\linewidth,trim={0ex 8ex 0ex 8ex},clip]{images/nribletS30M06.png}
    \caption{$\Sto^+ =30, M = 0.6$}
    \label{fig:nribletS30M06}
  \end{subfigure}
  \caption{Particle volume fraction (i.e. 3 times the average for (a)(b) and (c) and 5 times the average for (d)(e) and (f)) plotted over the streamwise velocity showing low speed streaks in the fluid, and preferential concentration of particles into these low speed regions creating long riblet like formations for particles at $\Sto^+ = 30$}
  \label{fig:riblet_comparison}
\end{figure}

%\subsection{Contribution to Reynolds shear stress and fluid turbulent kinetic energy}

The contribution to the total shear stress by the Reynolds shear stress $-\rho_f \langle uv \rangle $, and the viscous shear stress $\mu du_{f}/dy$ within wall bounded flows has been of vital importance to understanding the modulation of the mass flow rate within a single phase flows. Through Reynolds averaging, previous studies have arrived on a single equation that describes the turbulent shear stress profile as shown below in eqn. \ref{eq:RSS_1}. Here the total shear stress $\tau$ is shown as the viscous shear stress $\mu\partial u_f/\partial y$ minus the Reynolds shear stress $\rho\langle u_{f}^{'}v_{f}^{'}\rangle$. $\tau$ here as shown by eqn. \ref{eq:RSS_1} has a constant slope as the distance away from the wall.

\begin{equation}
    -\frac{\partial \langle P \rangle}{\partial x} = \frac{\partial}{\partial y}\left(\mu\frac{\partial u_f}{\partial y} - \rho\langle u_{f}'v_{f}'\rangle\right)
    \label{eq:RSS_1}
\end{equation}

The total shear stress, $\tau$ and the contributions to it (viscous and Reynolds stress) for the single phase channel flow is shown in fig. \ref{fig:tss_sp}. It is clear how $\tau$ has a constant slope from the wall to the center of the flow. The viscous shear stress dominates the near wall region, while as we move away from the wall it tends to exponentially decrease in magnitude. Comparatively, the Reynolds shear stress shows the maximum magnitude in the log layer of the flow, and has a steady reduction as we approach the center of the channel.

\begin{figure}[htp]
    \centering
    \includegraphics[width=\linewidth]{plotting/TSS-SP.pdf}
    \caption{Contribution of the Reynolds shear stress and viscous shear stress to the total shear stress within a single phase channel flow.}
    \label{fig:tss_sp}
\end{figure}

However for the current multiphase (particle-laden) cases, The total shear stress is a contribution of the viscous and Reynolds shear stress, and an additional third term, the particle shear stress. This is due to the extra particle momentum exchange term we see added to the conservation of momentum equation in eqn. \ref{eq:NVS_2}. Performing a similar Reynolds averaging on the particle term $\langle F_p \rangle $ yields a similar equation as eqn. \ref{eq:RSS_1} with an extra particle shear stress term as shown in eqn. \ref{eq:RSS_2}. Keep in mind for constant pressure gradient flows (the force driving the flow for all simulations conducted in this study), the integral of the left hand side of the equation $\partial\langle P\rangle/\partial x$, always remains the same, since the pressure gradient forcing is a constant value. Therefore it is of importance to understand how the viscous, Reynolds and particle shear stress curves are modulated within the flow as Stokes number or the mass loading is varied.

\begin{equation}
    -\frac{\partial \langle P \rangle}{\partial x} = \frac{\partial}{\partial y}\left(\mu\frac{\partial u_f}{\partial y} - \rho\langle u_{f}'v_f'\rangle - m_{p}\langle n_{d} \rangle \langle u_{p}'v_{p}' \rangle \right)
    \label{eq:RSS_2}
\end{equation}

From eqn. \ref{eq:RSS_2}, it is observed that the fluid Reynolds shear stress and the particle shear stress are both subtracted from the viscous shear stress. The fluid Reynolds shear stress and particle shear stress therefore, perform a similar role in affecting the total shear stress within a flow. Previous single phase studies have shown that the Reynolds shear stress is a good indicator of the skin-friction drag near the wall. Reducing the magnitude of the Reynolds shear stress can strongly aid in the reduction of the drag experienced by the fluid and increase the mass flow rate. Therefore, we can hypothesize that if the total contribution of $\rho\langle u_{f}^{'}v_{f}^{'}\rangle$ and $m_{p}\langle n_{d} \rangle \langle u_{p}^{'}v_{p}^{'} \rangle$ if lower than the single phase $\rho\langle u_{f}^{'}v_{f}^{'}\rangle$, we can achieve increase in mass flow rate.  

\begin{figure}[htp]
  \begin{subfigure}{0.45\textwidth}
    \centering
    \includegraphics[width=1\linewidth]{plotting/TSS_COMPARE-S6M06.pdf}
    \caption{$\Sto^+ = 6, M = 0.6$}
    \label{fig:tsscomparest6}
  \end{subfigure}
  \begin{subfigure}{0.45\textwidth}
    \centering
    \includegraphics[width=1\linewidth]{plotting/TSS_COMPARE-S30M06.pdf}
    \caption{$\Sto^+ = 30, M = 0.6$}
    \label{fig:tsscomparest30}
  \end{subfigure}
  \caption{This figure shows the shear stress profile for (a) case \textbf{C} (maximum drag enhancement) and (b) case \textbf{F} (maximum drag reduction), compared to the single phase flow (dashed lines) where (\sampleline{dash pattern=on .7em off .2em on .2em off .2em}) is the total shear stress, (\sampleline{dotted}) is the Reynolds shear stress and (\sampleline{dashed}) is the viscous shear stress} 
  \label{fig:tss_comparison}
\end{figure}

Figure \ref{fig:tss_comparison} shows the stress profile and how the different shear stress terms vary in comparison to the single phase flow. As shown in eqn. \ref{eq:RSS_2} The total shear stress profile will always remain the same. Equation \ref{eq:RSS_2} shows that the particle shear stress and the Reynolds shear stress both produce a similar effect as both are subtracted from the viscous shear stress. Therefore, if the viscous stress is increased, the Reynolds and particle stresses should reduce to satisfy eqn. \ref{eq:RSS_2}. Moreover, we know through previous studies (** insert reference) that as the Reynolds shear stress is reduced there is an improvement in the mass flow rate of the fluid. The near coherent structures are responsible for nearly half of the skin friction drag (Guala, JFM, 2006). Modulating these coherent structures can help reduce the skin-friction drag induced within the flow. Figure \ref{fig:tsscomparest30} shows the stress profile for case \textbf{F} and it is observed that the near wall viscous shear stress profile is higher than that of the single phase. This shows that the higher inertial particles help reduce the Reynolds shear stress, and they themselves induce a lower magnitude of stress. This shows how higher inertial particles can help reduce the induced drag and enhance the mass flow rate by modulating the coherent structures near the wall. They can help reduce the Reynolds shear stress significantly while also keeping the particle shear stress low near the wall such that the sum of the Reynolds and particle shear stress near the wall is lower than the Reynolds shear stress of the single phase flow near the wall. Contrastingly, fig. \ref{eq:tsscomparest6} shows the stress profile for lower inertial particles i.e. case \textbf{C}. The near wall viscous stress profile is almost similar to the single phase case. This means that the sum of the Reynolds and particle shear stress are also equal to the single phase Reynolds shear stress near the wall. Case \textbf{C} shows maximum drag enhancement, this is mostly due to the log-outer layer increase in the particle shear stress, due to a large amount of particles being easily ejected into the middle of the channel slowing down the fluid as a result of this.        

To further understand the modulation of the coherent structures near the wall, the Q-criterion is explored to understand how particles modulate the vortical structures near the wall with changing Stokes number while keeping the mass loading the same. The vortical structures are diminished with time as particles are injected into the system. Inertial particles of $\Sto^+$ of O(10) diminish the vortical structures and modulate the fluid significantly more than particles of $\Sto^+$ of order O(1) as we can see in \ref{fig:qcrit}. Further demonstrating how particle-laden flows do not follow the same rules as unladen flows. Diminishing the vortical structures is not the only essential criteria to decreasing the drag of the system. From the results in fig. \ref{fig:fvelst30} and \ref{fig:tsscomparest30} for higher inertia cases and fig. \ref{fig:fvelst6} and \ref{fig:tsscomparest6} for lower inertia cases, it is observed that the mass flow rate of the fluid is not solely dependent on the diminishing of the coherent structures, but also keeping the particle induced shear stress low enough which higher inertial particles perform better at than lower inertial particles. 

\begin{figure}[htp]
    \centering
    \includegraphics[width=\linewidth]{tikzgraphics/q-crit.pdf}
    \caption{The Q-criterion represents the vortical structures near the wall and is a good depiction of thje Reynolds shear stress modulation near the wall. The figure shows that the vortical structures are diminished for all particle-laden cases with higher inertia particles performing better at diminishing these structres than lower inertia particles.}
    \label{fig:qcrit}
\end{figure}

\par
Another phenomenon that can be explored is preferential concentration and how particles transfer the turbulent energy within the flow. Quadrant analysis \citep{wallaceWallRegionTurbulent1972}\citep{adrianStochasticEstimationOrganized1988} was therefore performed on all cases further understand this. The velocity fluctuations within the flow are divided in 4 quadrants based on their signs, where Q1($+u$,$+v$) and Q3($-u$,$-v$) are the outward and inward motions within the flow which dont reveal much about the vortical structures within the flow. Additionally Q2($-u$,$+v$) and Q4($+u$,$-v$) are the ejection and sweep events which reveal the vortical motions within the flow and are the greatest contributors of Reynolds shear stress within the flow. We can observe from fig. \ref{fig:quadrant} for all $\Sto^+ = 6$ and $\Sto^+ = 30$ cases that as the mass loading is increased and more particles are injected within the system we see a monotonic behaviour that depicts a decrease in the Reynolds shear stress contribution by the Q2 and Q4 events which are the sweep and ejection events and also agree with the previous results shown in fig. \ref{fig:tss_comparison}. Furthermore, looking at the sweep and ejection quadrants further confirms what is observed in fig. \ref{fig:tss_comparison}. The log scale graph shows that for the highest mass loading (maximum modulation), the near wall Reynolds stress reduction is more significant for $\Sto^+ = 30$ than for $\Sto^+ = 6$.    

\begin{figure}%[htp]
  \begin{subfigure}{0.45\linewidth}
    \centering
    \includegraphics[width=1\linewidth]{plotting/Q1.pdf}
    \caption{Quadrant 1}
    \label{fig:q1}
  \end{subfigure}
  \begin{subfigure}{0.45\textwidth}
    \centering
    \includegraphics[width=1\linewidth]{plotting/Q2.pdf}
    \caption{Quadrant 2}
    \label{fig:q2}
  \end{subfigure}
  \begin{subfigure}{0.45\linewidth}
    \centering
    \includegraphics[width=1\linewidth]{plotting/Q3.pdf}
    \caption{Quadrant 3}
    \label{fig:q3}
  \end{subfigure}
  \begin{subfigure}{0.45\linewidth}
    \centering
    \includegraphics[width=1\linewidth]{plotting/Q4.pdf}
    \caption{Quadrant 4}
    \label{fig:q4}
  \end{subfigure}
  \caption{Quadrant analysis for all cases normalized by ${u_\tau}^2$ as a function of $y^+$. The results agree very well to literature where we see the maximum contribution to the Reynolds stress is due to Q2 and Q4.}
  \label{fig:quadrant}
\end{figure}

~




\bibliography{references,references_houssem,final_references}

\end{document}
