\section{Structure of a particle-free turbulent channel flow}
\label{sec:unladen}

\begin{figure}
  \begin{subfigure}{0.45\textwidth}
    \centering
    \includegraphics[width=1\linewidth]{plotting/fig2a.pdf}
    \caption{ }
    \label{fig:unladen_vel}
  \end{subfigure}
  \begin{subfigure}{0.45\textwidth}
    \centering
    \includegraphics[width=1\linewidth]{plotting/fig2b.pdf}
    \caption{ }
    \label{fig:unladen_rms}
    \end{subfigure}
    \caption{Unladen flow velocity statistics at $\Rey_\tau =180$: (a) Streamwise mean velocity (b) root-mean square velocity fluctuations in the streamwise ($u_{f,rms}^+$, \stl), normal ($v_{f,rms}^+$, \dedash) and spanwise ($w_{f,rms}^+$, \ddd) directions. The symbols correspond to the data taken from 
    \citet{kimTurbulenceStatisticsFully1987}.}
    \label{fig:unladen_results}
\end{figure}

The particle-free channel represents a baseline reference that we use to highlight flow modifications induced by inertial particles. To that end, we start by reviewing aspects of an unladen turbulent channel flow at $\Rey_\tau=180$ that are relevant to the discussion in \S\ref{sec:particle_laden}.

Figure \ref{fig:unladen_results} shows  profiles of  mean streamwise velocity $\langle u^+_f\rangle$ and root-mean square (rms) velocity  fluctuations. Averaging is performed using 100 snapshots collected over a period of 10 of eddy turnover time once the flow is stationary. Further,  the streamwise and spanwise directions are averaged such that the only variation is in the wall-normal direction. As expected at this Reynolds number, the mean streamwise velocity shows three characteristic regions: viscous layer for $y^+\lesssim 5$, a buffer layer for $5\lesssim y^+\lesssim 30$, and a logarithmic layer for $y^+\gtrsim 30$. Velocity fluctuations in the streamwise direction dominate over the two other components and peak at $y^+\sim 12$ in the buffer layer. These observations are consistent with those of \citet{kimTurbulenceStatisticsFully1987} and general understanding of turbulent channel flow at the Reynolds number considered.

\begin{figure}
    \centering
    \includegraphics[width=5.0in]{plotting/fig3.pdf}
    \caption{Contribution of the Reynolds shear stress(\ddd) and viscous shear stress(\dedash) to the total shear stress(\stl) in the particle-free channel at $\Rey_\tau = 180$.}
    \label{fig:tss_sp}
\end{figure}

The structure of the mean flow results from a balance between pressure gradient $-\langle\partial p/\partial x\rangle$, viscous stress $\mu d\langle u_f\rangle/dy$, and Reynolds shear stress $-\rho_f \langle u'_fv'_f \rangle $. By Reynolds-averaging the fluid momentum equations, one can show that the equation for the streamwise momentum reduces to
\begin{equation}
    -\langle\frac{\partial  p }{\partial x}\rangle = \frac{d}{d y}\left(\mu\frac{d}{dy} \langle u_f\rangle - \rho_{f}\langle u'_{f}v'_{f}\rangle\right).
    \label{eq:RSS_1}
\end{equation}
Given that the pressure gradient in a fully developed channel is constant, the total shear stress (sum of the viscous and Reynolds stresses) must vary linearly across the channel, i.e.,
\begin{equation}
    \mu\frac{d}{dy} \langle u_f\rangle - \rho_{f}\langle u'_{f}v'_{f}\rangle = \tau_{w}\left(1-\frac{y}{h}\right).
    \label{eq:cf_rss}
\end{equation}
This behavior is illustrated in figure \ref{fig:tss_sp} showing the variation of the total shear stress and its viscous and Reynolds contributions as a function of the wall normal distance. In accordance with (\ref{eq:cf_rss}), the total shear stress varies linearly from the wall to the channel center where it vanishes. The viscous shear stress dominates near the wall, and vanishes away from it. Conversely, the contribution of the Reynolds shear stress is small in the viscous sublayer, whereas it dominates in the logarithmic layer.

The Reynolds shear stresses has a direct influence on skin-friction drag. The latter is characterized using the coefficient
\begin{equation}
    C_{f}=\frac{\tau_w}{\frac{1}{2}\rho_f U_{b,f}^{2}},
    \label{eq:Cf}
\end{equation}
where $U_{b,f}=\dot{m}_f/\rho_f$ is the bulk fluid velocity corresponding to the ratio of the cross-sectional average fluid mass flow rate and the fluid density.  While $\tau_w$ is fixed in a channel flow driven by a constant pressure gradient, modulating the Reynolds shear stress is susceptible to change the bulk velocity $U_{b,f}$, and, in turn, the skin-friction drag $C_f$. Double integrating equation (\ref{eq:cf_rss}) clarifies the connection between $C_f$ and Reynold shear stress. The resulting mass flow rate per unit spanwise length is
\begin{equation}
    \frac{\dot{m}_f}{L_z}= \frac{2}{3} \frac{\tau_w h^2}{\nu}\left(1 + \frac{3}{(u_\tau h)^2}\int_{y=0}^{h} \int_{y'=0}^y \langle u'_fv'_f\rangle dy'dy\right).
    \label{eq:sp_mass_flow_rate}
\end{equation}
In this form, it becomes clear that the Reynolds shear stress reduces the mass flow rate, given $\langle u'_fv'_f\rangle<0$, resulting in an increase of $C_f$ compared to the laminar baseline. Therefore, it is not surprising that a large number of prior studies on skin-friction drag reduction in turbulent channel flows focused on reducing the Reynolds shear stress \citep{hetsroniLowspeedStreaksDragreduced1998,minDragReductionPolymer2003,ptasinskiTurbulentChannelFlow2003}.

\begin{figure}
	\centering
  \includegraphics[width=\linewidth]{tikzgraphics/fig4.pdf}
	\caption{Isocontours showing the low speed streaks in the particle-free channel at $y^+ = 10$.
	\label{fig:streak_SP_a}}
\end{figure}

\begin{figure}
	\centering
  \includegraphics[width=5.0in]{plotting/fig5.pdf}
	\caption{Two-point autocorrelation of the streamwise velocity fluctuations in the spanwise direction for particle-free channel at $y^+ = 10$.
	\label{fig:streak_SP_b}}
\end{figure}

From a mechanistic perspective, the Reynolds shear stress arises from coherent flow structures that populate the near-wall region \citep{smitsHighReynoldsNumber2011}. The so-called low-speed streaks, regions of slow moving fluid elongated in the streamwise direction, are among the most significant coherent structures found in a turbulent channel flow \citep{baeLifeCycleStreaks2021}. These streaks are shown in figure \ref{fig:streak_SP_a} visualized using isocontours of streamwise velocity in a wall parallel plane at $y^+=10$.
There has been sustained effort to understand the morphology and dynamics of low-speed streaks, as well as their connection to other coherent structures, such as quasi-streamiwse vorticies and so-called large-scale motions and very large-scale motions (see
\citep{klineStructureTurbulentBoundary1967a,willmarthStructureReynoldsStress1972,smithCharacteristicsLowspeedStreaks1983,jimenezMinimalFlowUnit1991,stretchAutomatedPatternEduction1991,jeongCoherentStructuresWall1997,jimenezCoherentStructuresWallbounded2018,zhuVortexAxisTracking2019,jiangExperimentalStudyLowspeed2020,schoppaCoherentStructureGeneration2002,smitsReynoldsStressScaling2021,baeLifeCycleStreaks2021,zhouInteractionNearwallStreaks2022}). The general consensus is that low-speed streaks in the buffer layer are formed between a pair of quasi-streamwise vortices with opposite orientation. The bursting of low-speed streaks contributes a significant part of the Reynolds shear stress and turbulent energy production \citep{kimProductionTurbulenceSmooth1971,baeLifeCycleStreaks2021}. This occurs when the quasi-streamwise vortices surrounding a low-speed streak become unstable \citep{jeongCoherentStructuresWall1997}, resulting in a lift up and eventual break down of the streak. Prior to their collapse, low-speed streaks in the buffer region have a typical length in the range 200--300 wall units \citep{baeLifeCycleStreaks2021,jeongCoherentStructuresWall1997}, but may extend beyond 1000 wall units \citep{jimenezLargescaleDynamicsNearwall2004}. Remarkably, these structures have a spanwise spacing $\lambda_f^+$ that varies little with Reynolds number, and is about $\lambda_f^+ \sim 100-110$ \citep{klineStructureTurbulentBoundary1967,jimenezMinimalFlowUnit1991,klewickiViscousSublayerFlow1995,jimenezLargescaleDynamicsNearwall2004}. We verify this by computing the two-point autocorrelation of the streamwise velocity as a function of the spanwise separation and wall distance,
\begin{equation}
    R^f_{uu}(\Delta z;y_0)=\frac{\langle{u'_f (x,y_0,z,t)u'_f (x,y_0,z+\Delta z,t)}\rangle}{\langle{u_f'^2}\rangle}.
    \label{eq:lambda_f}
\end{equation}
Figure \ref{fig:streak_SP_b} shows the variation of  $R^f_{uu}$ at $y^+=10$. The streak spanwise spacing $\lambda_f^+$ corresponds to twice the distance between the origin and $\Delta z^+$ where $R^f_{uu}$ reaches a first minimum which yields $\lambda_f^+ =2\times 53= 106$ in the present simulations.

It is noteworthy that the physics of a turbulent channel flow, discussed here at $\Rey_\tau =180$, remain largely the same near the wall, even at much higher Reynolds numbers. \citet{moserDirectNumericalSimulation1999} conduct wall-bounded channel flow simulations at $\Rey_\tau = 590,395$ and $180$. They showed that despite differences in the log region, the dynamics in the viscous and buffer layer regions are similar. Given that inertial particles tend to accumulate in these two regions, we expect that the particle-fluid interactions observed at $\Rey_\tau = 180$ will persist to much higher Reynolds numbers.