\subsection{Particle-fluid interaction}

\begin{figure}[htp]
	\begin{subfigure}{0.45\textwidth}
	    \centering
	    \includegraphics[width=1\linewidth]{plotting/PRE.pdf}
	    \caption{}
	    \label{fig:pre}
	\end{subfigure}
	\begin{subfigure}{0.45\textwidth}
	    \centering
	    \includegraphics[width=1\linewidth]{plotting/PRESCALED.pdf}
	    \caption{}
	    \label{fig:pre_scaled}
	\end{subfigure}		
	\caption{Particle Reynolds number shows that within the log-layer region and even further away from it particles do present a wake and influence the flow around them.}
	\label{fig:pre_stats}
\end{figure} 

Particle Reynolds number, $\Rey_p$ is a good parameter to understand how particles modulate the flow based on their physical properties (i.e. particle diameter, Stokes number and particle density). We begin investigating this particle-phase of the channel flow by measuring the particle Reynolds number, $\Rey_p$ within the flow. Figure \ref{fig:pre} shows that $\Rey_p$ is the highest in the log-layer region. It is known from theory \citep{bagchiDirectNumericalSimulation2001}\citep{richterDragForcesHeat2012} that as $\Rey_p$ increases particles create a greater wake, affecting fluid statistics and also nearby particle movement, especially within the log layer maximum flux in all quantities is observed. Furthermore, for the $\Rey_p$ range observed within fig. \ref{fig:pre}, it is inferred that the drag law proposed by \citet{tennetiDragLawMonodisperse2011a} is appropriate. 

To further understand the modulation of the mean streamwise velocity $U_f^+$, \ref{fig:pre_scaled} shows the scaled particle Reynolds number, $\Rey_p/\sqrt{\Sto^+}$. All cases seem to collapse closely, with case \textbf{C} and \textbf{F} showung a slight increase in the log layer region. This is in line with what is witnessed in fig. \ref{fig:fvelst6} and \ref{fig:fvelst30}, showing that it is the slip velocity that controls the intensity of the modulation performed on the fluid by the inertial particles. However, it is the inertia that control whether we see an increase or decrease in the mass flow rate. The higher the slip velocity within the log layer, the greater the modulation. Figure \ref{fig:fvel} shows that case \textbf{C} has masximum drag and also consequently has the highest slip velocity for particles at $\Sto^+ = 6$. Case \textbf{F} on the other hand mas maximum mass flow rate recorded and consequently also has the higher slip velocity recorded in the log region for particles with $\Sto^+ = 30$.

\begin{figure}[htp]
  \begin{subfigure}{0.8\textwidth}
    \centering
    \includegraphics[width=1\linewidth,trim={0ex 8ex 0ex 8ex},clip]{images/vfp_s6m02.png}
    \caption{$ St^+ = 6, M = 0.2$}
    \label{fig:vfp_s6m02}
  \end{subfigure}
  \begin{subfigure}{0.8\textwidth}
    \centering
    \includegraphics[width=1\linewidth,trim={0ex 8ex 0ex 8ex},clip]{images/vfp_s6m04.png}
    \caption{$St^+ = 6, M = 0.4$}
    \label{fig:vfp_s6m04}
  \end{subfigure}
  \begin{subfigure}{0.8\textwidth}
    \centering
    \includegraphics[width=1\linewidth,trim={0ex 8ex 0ex 8ex},clip]{images/vfp_s6m06.png}
    \caption{$St^+ = 6, M = 0.6$}
    \label{fig:vfp_s6m06}
  \end{subfigure}
  \begin{subfigure}{0.8\textwidth}
    \centering
    \includegraphics[width=1\linewidth,trim={0ex 8ex 0ex 8ex},clip]{images/vfp_s30m02.png}
    \caption{$St^+ =30, M = 0.2$}
    \label{fig:vfp_s30m02}
  \end{subfigure}
  \begin{subfigure}{0.8\textwidth}
    \centering
    \includegraphics[width=1\linewidth,trim={0ex 8ex 0ex 8ex},clip]{images/vfp_s30m04.png}
    \caption{$St^+ =30, M = 0.4$}
    \label{fig:vfp_s30m04}
  \end{subfigure}
  \begin{subfigure}{0.8\textwidth}
    \centering
    \includegraphics[width=1\linewidth,trim={0ex 8ex 0ex 8ex},clip]{images/vfp_s30m06.png}
    \caption{$St^+ =30, M = 0.6$}
    \label{fig:vfp_s30m06}
  \end{subfigure}
  \caption{Particle volume fraction normalized by the average volume fraction showing formation of clusters within the flow in high strain regions and the accumalation of particles near the wall through turbophoresis}
  \label{fig:vfp}
\end{figure}

Particle laden flows with a semi-dilute concentration have a substantial effect on fluid characteristics mainly due to two phenomenons that take place simultaneously within a particle-laden flow. Preferential concentration ejects particles from high vorticity (low strain) regions towards low vorticity (high strain) regions of the flow. This can help remove energy from these high vorticity regions as they get ejected outwards. Moreover, the effect of turboproresis near the wall allows for particles to segregate much closer to the wall with time as the the flow is driven within the channel. This effect of turbophoresis that aids in a higher localized volume fraction near the wall increases the effect of preferential concentration aiding in formations of clusters near the wall. To further take a look into this effect fig. \ref{fig:vfp} shows the particle volume fraction normalized by its average volume fraction and fig. \ref{fig:pnd} shows the particle number density relative to their distance from the wall.
      
\begin{figure}[htp]
    \centering
    \includegraphics[scale=0.5]{plotting/PND.pdf}
    \caption{Particle number density does not show strong dependence on mass loading in the semi-dilute regime as we can see all cases closely resembling each other apart from near the wall where we see maximum accumalation of particler as $\Sto^+$ increases.}
    \label{fig:pnd}
\end{figure}  

All cases show the effects of turbophoresis where maximum accumalation is observed near the wall however, heavier particles (i.e. larger Stokes numbers) tend to stay closer to the wall compared to lighter particles (i.e. smaller Stokes numbers). Previous studies by \citet{sardinaWallAccumulationSpatial2012} show greater accumalation of inertial particles near the wall as the Stokes number is increased. Figure \ref{fig:vfp} shows dense clustering within the channel as the mass loading is increased. Large particle clusters are observed near the wall for cases \textbf{D}, \textbf{E} and \textbf{F} compared to the center of the flow because of their high inertia. Contrastingly cases \textbf{A}, \textbf{B} and \textbf{C} show greater ejection of particles towards the center of the channel by vortical structures away from the wall owing to lower inertia. The length scales of the clusters for particles of $\Sto^+  =6$ stay consistent throughout the channel and become denser as the mass loading is increased, while the length scales for particles of $\Sto^+ = 30$ are higher near the wall than in the center of the flow and become denser as the mass loading is increased. This behaviour can be further witnessed in figure \ref{fig:pnd}. $\Sto^+ = 30$ cases show greater particle number density near the wall compared to $\Sto^+ = 6$ while having lesser particle number density away from the wall. It can therefore be hypothesized that larger Stokes numbers modulate the fluid largely near the wall while smaller Stokes numbers modulate the flow in all regions of the channel. This is consistent from trends we have seen in previous studies by \citet{fongVelocitySpatialDistribution2019}. Particles of $\Sto^+$ of O(10) display greater modulation of the fluid compared to particles of $\Sto^+$ of O(1) because of the large macro-scale focres they apply on the fluid due to the dense clustering near the wall.

\begin{figure}
    \centering
    \includegraphics[width=0.7\linewidth]{images/3driblet.png}
    \caption{Particles forming large riblets throughout the channel at high Stokes numbers ($\Sto^+ = 30, M = 0.6$}
    \label{fig:3driblet}
\end{figure}

To further understand the movement of oartucles near the wall and how inertia and mass loading affect the clustering, a detailed picture is shown in figure \ref{fig:3driblet} and \ref{fig:riblet_comparison}. Figure \ref{fig:3driblet} shows long riblets of particles forming near the wall for case \textbf{F}. Previous experiments have shown that particles arrange themselved in low-speed regions of the flow near the wall. Figure \ref{fig:riblet_comparison} shows the contours of the particles volume fraction for all cases (3 times the average volume fraction for particles of $\Sto^+ = 6$ and 5 times the average volume fraction for particles of $\Sto^+ = 30$) near the wall over the fluid streamwise velocity. The effect of preferential concentration is very evident in all cases \textbf{A}-\textbf{F}. Particles tend to get ejected from high speed (high vorticity) regions to low speed (low vorticity) regions. FOr particles of $\Sto^+ = 6$ (fig. \ref{fig:nribletS6M02}, \ref{fig:nribletS6M04} and \ref{fig:nribletS6M06}), a large number of low-speed regions are present with small clustering of particles near the wall within all these low-speed regions. This is due to the low inertia of the particles, making them `lighter' and allowing them to drift with more ease. Even though, the effect of preferential concentration is evident for particles at $\Sto^+ = 6$, they seem to have a higher level of movement near the wall. This shows that the fluid is strongly modulating the particles, just as equally as the particles are modulating the fluid near the wall when inertia is low. Contrastingly, cases at $\Sto^+ = 30$ show longer more well defined low-speed regions. Particles have larger clusters, multiple length scales bigger than particle clusters at $\Sto^+ = 6$. As the mass loading is increased the long particle clusters (i.e. `riblets') become more defined (fig. \ref{fig:nribletS30M06}). A three dimentional representation of that is shown in fig. \ref{fig:3driblet}. Low speed regions become thinner but also longer (spanning the entire domain). This is consistent with data on particle-laden flows with larger inertia particles. They tend to be more `heavier', meaning less movement near the wall. However the effect of preferential concentration is still visible with particles forming long riblets within the low speed regions, with minimal to almost non in the high speed regions. For particle-laden cases at $\Sto^+ = 30$, there are relatively less number of particle clusters but longer clusters than flows with inertial particles at $\Sto^+ = 6$. Fewer clustering means fewer movement of particles in and out of different vorticity regions of the flow. The particles tend to modulate the fluid more stronglt than the fluid modulating the particles for higher inertia particle-laden cases.   

\begin{figure}[htp]
  \begin{subfigure}{0.75\textwidth}
    \centering
    \includegraphics[width=1\linewidth,trim={0ex 8ex 0ex 8ex},clip]{images/nribletS6M02.png}
    \caption{$ \Sto^+ = 6, M = 0.2$}
    \label{fig:nribletS6M02}
  \end{subfigure}
  \begin{subfigure}{0.75\textwidth}
    \centering
    \includegraphics[width=1\linewidth,trim={0ex 8ex 0ex 8ex},clip]{images/nribletS6M04.png}
    \caption{$\Sto^+ = 6, M = 0.4$}
    \label{fig:nribletS6M04}
  \end{subfigure}
  \begin{subfigure}{0.75\textwidth}
    \centering
    \includegraphics[width=1\linewidth,trim={0ex 8ex 0ex 8ex},clip]{images/nribletS6M06.png}
    \caption{$\Sto^+ = 6, M = 0.6$}
    \label{fig:nribletS6M06}
  \end{subfigure}
  \begin{subfigure}{0.75\textwidth}
    \centering
    \includegraphics[width=1\linewidth,trim={0ex 8ex 0ex 8ex},clip]{images/nribletS30M02.png}
    \caption{$\Sto^+ =30, M = 0.2$}
    \label{fig:nribletS30M02}
  \end{subfigure}
  \begin{subfigure}{0.75\textwidth}
    \centering
    \includegraphics[width=1\linewidth,trim={0ex 8ex 0ex 8ex},clip]{images/nribletS30M04.png}
    \caption{$\Sto^+ =30, M = 0.4$}
    \label{fig:nribletS30M04}
  \end{subfigure}
  \begin{subfigure}{0.75\textwidth}
    \centering
    \includegraphics[width=1\linewidth,trim={0ex 8ex 0ex 8ex},clip]{images/nribletS30M06.png}
    \caption{$\Sto^+ =30, M = 0.6$}
    \label{fig:nribletS30M06}
  \end{subfigure}
  \caption{Particle volume fraction (i.e. 3 times the average for (a)(b) and (c) and 5 times the average for (d)(e) and (f)) plotted over the streamwise velocity showing low speed streaks in the fluid, and preferential concentration of particles into these low speed regions creating long riblet like formations for particles at $\Sto^+ = 30$}
  \label{fig:riblet_comparison}
\end{figure}
