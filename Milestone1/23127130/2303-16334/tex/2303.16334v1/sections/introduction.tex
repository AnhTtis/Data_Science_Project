
\section{Introduction}
\label{sec:introduction}

Inertial particles introduced in wall-bounded turbulent flows play a significant role in the transport of mass and momentum in many engineering applications. Examples include cyclone separators, fluidized bed risers, sediment transport in pipelines, and dust ingested in engines. In the case of gas-solid flows, \textcolor{revision}{semi-dilute particle concentrations, i.e., particle volume-fraction typically $10^{-6} - 10^{-3}$,  may be sufficient to cause significant modulation of the flow structures \citep{kasbaouiTurbulenceModulationSettling2019}. Provided that the mass loading is $O(1)$, the dynamics of the two phases in the semi-dilute regime are controlled by the two-way coupling between the particles and fluid, whereas, particle-particle collisions play a secondary or negligible role.} In this paper, we show that \textcolor{revision}{semi-dilute} inertial particles introduced in a turbulent channel flow may cause significant skin-friction drag increase  or reduction, depending on particle concentration, inertia, and how particle clusters interact with near-wall coherent flow structures.

Inertial solid particles, or liquid droplets small enough to be dominated by surface tension, dispersed in gas may drastically alter the carrier flow properties. Due to their inability to follow fluid streamlines, these particles exert micro-stresses on the carrier fluid. If the particles are sufficiently concentrated, the collective action of these micro-stresses may amount to a large macroscopic force capable of modifying the carrier flow properties.
Taking Homogeneously Sheared Turbulence (HST) as a simplified proxy for general turbulent shear flows, \citet{kasbaouiTurbulenceModulationSettling2019} and \citet{kasbaouiRapidDistortionTwoway2019} showed that introducing inertial particles at semi-dilute concentration, characterized by an average particle volume fraction $\phi_0 = 10^{-6} - 10^{-3}$ and mass loading $M=O(1)$, may cause an attenuation or augmentation of the turbulent kinetic energy. Whether the latter is increased or decreased depends on particle inertia. \citet{ahmedDirectNumericalSimulation2001} showed that turbulence modulation in HST is due to a reverse cascade of energy, whereby energy injected into the flow by particles at their scale propagates up to the macro-scales. The resulting particle-laden turbulence has distinctively different characteristics from turbulence in single-phase flows as shown by \citet{gualtieriClusteringTurbulenceModulation2013} who found that the energy cascade in particle-laden HST departs from the traditional -5/3 law \citep{kolmogorovLocalStructureTurbulence1941}.

In  wall-bounded turbulent flows, inertial particles are expected to have disproportionally larger impact on near-wall flow structures. Inertial particles tend to migrate to regions of lower turbulent fluctuations, a process called turbophoresis. The latter leads to the accumulation of particles near bounding walls. In simulations of dilute particle-laden turbulent channel flows, \citet{sardinaWallAccumulationSpatial2012}, and later, \citet{nilsenVoronoiAnalysisPreferential2013} and \citet{yuanThreedimensionalVoronoiAnalysis2018}, showed that the particle concentration in the viscous layer may be one or  two orders of magnitude larger than the mean. The highest wall accumulation happens for particles with inertia characterized by  friction Stokes number $\mathrm{St}^+=\tau_p u_\tau^2/\nu$ in the range 10-50. Here, $u_\tau$, $\nu$, and $\tau_p$ refer to the friction velocity, kinematic viscosity, and particle response time. Further, such inertial particles disperse in a highly inhomogeneous way leading to the formation of clusters even in the near-wall region where most particles accumulate \citep{sardinaWallAccumulationSpatial2012}. Clusters found therein tend to be elongated structures that may be several orders of magnitude longer than the particle diameter \citep{jieExistenceFormationMultiscale2022}.
Because inertial particles accumulate into such long clusters, they are able to modulate flow structures on scales as large as the cluster scales, which may exceed even the largest turbulence scales \citep{kasbaouiTurbulenceModulationSettling2019}. Hence, provided that the particle concentration is sufficiently large to yield meaningful feedback force on the flow, the dispersed particles are expected to modulate near-wall flow structures and alter the turbulence structure in wall-bounded flows.

The near-wall coherent flow structures have a large impact on skin-friction drag. The existence of these structures, their evolutionary dynamics and their role in the generation of shear stress in particle-free wall-bounded turbulent flows \textcolor{revision}{have long been under study.} \citet{fiedlerCoherentStructuresTurbulent1988} describes the existence of these structures within the boundary layer as a ``zoo of structures'' ranging from ``horseshoe- and hairpin-eddies, pancake- and surfboard-eddies, typical eddies, vortex rings, mushroom-eddies, arrowhead-eddies, etc''. In turbulent channel flows, \citet{jeongCoherentStructuresWall1997} found that flow structures in the buffer region are dominated primarily by elongated quasi-streamwise vortices. The latter are arranged antisymmetrically with vortices of opposite directions arranged next to each other \citep{schoppaCoherentStructureGeneration2002a,stretchAutomatedPatternEduction1991}. The so-called low-speed streaks are regions of slow moving fluid that have been identified in various studies, and were later shown to be nested in the space between a pair of quasi-streamwise vortices \citep{klineStructureTurbulentBoundary1967a,smithCharacteristicsLowspeedStreaks1983,jiangExperimentalStudyLowspeed2020}. The spanwise spacing of low-speed streaks  is a characteristic of turbulence in channels, since its value of $\sim$100 wall units was found to vary little with Reynolds number \citep{klineStructureTurbulentBoundary1967,jimenezMinimalFlowUnit1991,klewickiViscousSublayerFlow1995,jimenezLargescaleDynamicsNearwall2004}. Bursting occurs when the quasi-streamwise vortices become unstable \citep{jimenezCoherentStructuresWallbounded2018}. The formation and breakdown of these structures is part of a self-sustaining process that repeats periodically. \citet{willmarthStructureReynoldsStress1972} showed that bursting events are among the largest contributors to the Reynolds stress production. Thus, reducing skin-friction drag hinges on the ability to suppress bursting and stabilizing quasi-streamwise vortices as has been shown in drag-reduced polymeric channel flows \citep{mccombDragreducingPolymersTurbulent1978,bermanDragReductionPolymers1978,renardyMechanismDragReduction1995,zhuTransientDynamicsTurbulence2019}.

To the best of our knowledge, parameters leading to reproducible skin-friction drag reduction using inertial particles have not yet been identified. The majority of older studies point to an increase of skin-friction drag or negligible effect \citep{radinDragReductionSolidfluid1975a,gyrDragReductionTurbulence1995a}.
\citet{liNumericalSimulationParticleladen2001} are among the first to provide evidence of skin-friction drag reduction in simulations with the point-particle method. The authors showed that particles with friction Stokes number $\mathrm{St}^+=192$ dispersed in a vertical channel at $\Rey_\tau=u_\tau h/\nu = 125$, where $h$ is the channel half-height, increase the fluid mass flow rate by $\sim 5\%$ for mass loadings as small as $M=0.2$. Note that an increase in fluid mass flow rate is equivalent to a reduction in skin-friction drag. Later, \citet{zhaoTurbulenceModulationDrag2010} showed that inertial particles with $\mathrm{St}^+= 30$ at mass loading $M=0.36$ increase the fluid mass flow rate by approximately 15\% in a turbulent channel flow at  $\Rey_\tau=180$. However, these results may not be representative of a stationary state, since the latter requires much longer integration time than what is reported by \citet{zhaoTurbulenceModulationDrag2010}.  A follow-up study by \citet{zhouNonmonotonicEffectMass2020} in an identical configuration shows negligible drag reduction, about $\sim 0.2\%$ at $M=0.4$ and 2.8\% at $M=0.75$. Recently, \citet{costaNearwallTurbulenceModulation2021} revisited the semi-dilute particle-laden channel flow at $\Rey_\tau=180$ using particle-resolved direct numerical simulations (PR-DNS). Contrary to the aforementioned work,  \citet{costaNearwallTurbulenceModulation2021} found that inertial particles with $\mathrm{St}^+=50$ at $M=0.34$ cause a large increase in skin-friction drag by about $\sim 16\%$ compared to a particle-free channel. One should also note that despite the higher numerical resolution offered by PR-DNS, the greater computational cost constrained \citet{costaNearwallTurbulenceModulation2021} to use significantly smaller computational domain. With volume about 1/4th of that used in prior simulations with the point-particle method \citep{zhaoTurbulenceModulationDrag2010,zhouNonmonotonicEffectMass2020}, the computational box used by  \citet{costaNearwallTurbulenceModulation2021} may be too small to allow a natural development of particle clusters and their interaction with near-wall coherent structures.

Given the conflicting results previously reported, the questions of \emph{whether} inertial particles can induce significant skin-friction drag reduction, and if they do, \emph{how}?, have not been settled yet. 
\textcolor{revision}{
In this paper, we address these questions using Euler-Lagrange simulations of particle-laden turbulent channel flow at $\Rey_\tau=180$ while varying the characteristics of the particle phase. Although there is a multitude of non-dimensional numbers that can be used to characterize the particle-fluid interaction \citep{tanakaClassificationTurbulenceModification2008}, our past work shows that Reynolds number, Stokes number, and Mass loading are the most relevant non-dimensional numbers that control the dynamics in the semi-dilute regime \citep{kasbaouiPreferentialConcentrationDriven2015a,kasbaouiClusteringEulerEuler2019,kasbaouiRapidDistortionTwoway2019,kasbaouiTurbulenceModulationSettling2019,shuaiAcceleratedDecayLamb2022,shuaiInstabilityDustyVortex2022}. For this reason, we focus on varying the Stokes number $\Sto^+$ and mass loading $M$ seperately.}
 In section \ref{sec:setup}, we describe the mathematical framework, numerical methods, and simulation parameters used in this study. In order to highlight the flow modulation induced by inertial particles, we provide a brief review of the characteristics of particle-free turbulent channel flow at $\Rey_\tau = 180$ in section \ref{sec:unladen}, namely, in terms of velocity statistics, stress balance, and coherent flow structures. In section \ref{sec:particle_laden}, we analyze the particle-laden channel flows, in particular, the induced flow modification (\S~\ref{sec:modulation}), stress balance (\S~\ref{sec:stress_two_phase}), and interplay between particle clusters and near-wall coherent flow structures (\S~\ref{sec:mechanism}). Finally, we provide concluding remarks in \S~\ref{sec:conclusion}.
