\section{Effect of inertial particles at semi-dilute conditions}
\label{sec:particle_laden}

Introducing inertial particles at semi-dilute concentration causes a departure from the known characteristics of a particle-free channel flow. In the following, we analyze the flow modulation resulting from the particle feedback force and propose a mechanism based on the interplay between near-wall coherent structures and particle clusters.

\subsection{Flow modulation and impact on skin-friction drag}
\label{sec:modulation}

\begin{figure}
	\centering
	\includegraphics[width=5in]{tikzgraphics/fig6.pdf}
	\caption{Isocontours of streamwise velocity in a wall-normal plane: (top) the reference unladen flow, (middle) flow laden with $\Sto^+=30$ particles at $M=1.0$, and (bottom) flow laden with $\Sto^+=6$ particles at $M=1.0$. Note: To facilitate visual comparison, the domain in the middle has been truncated to the same dimensions as the smaller domain in the bottom case.
	\label{fig:isocontour_u}}
\end{figure}


Figure \ref{fig:isocontour_u} shows isocontours of the streamwise velocity at an arbitrary time after the flow reached a stationary state.
From these instantaneous visualizations, it is immediately clear that semi-dilute inertial particles cause strong modulation of the carrier flow, with the most apparent change being a change in fluid bulk velocity. The latter is visibly increased by $\Sto^+=30$ particles at mass loading $M=1$ compared to the reference particle-free flow.
 In particular, the fluid streamwise velocity near the centerline shows a noticeable increase.
Further, the overall level of turbulence judged by the naked eye is diminished compared to the unladen flow.
In contrast, $\Sto^+=6$ particles at mass loading $M=1$ cause an apparent slow down of the carrier flow. The greatest drop of the fluid velocity is around the centerline.

\begin{figure}\centering
  \begin{subfigure}{0.49\linewidth}
    \centering
    \includegraphics[width=3.2in]{plotting/fig7a.pdf}
    \caption{}
    \label{fig:u_st6}
  \end{subfigure}\hfill
  \begin{subfigure}{0.49\linewidth}
    \centering
    \includegraphics[width=3.2in]{plotting/fig7b.pdf}
    \caption{}
    \label{fig:u_st30}
	\end{subfigure}
  \caption{Average streamwise velocity as a function of the wall normal distance for (a) $\Sto^+ = 6$ and (b) $\Sto^+ = 30$ various mass loadings. The solid black line represents a particle-free channel. Symbols denote Stokes number $\Sto^+=$ 6 (\textcolor{red}{\protect\scalebox{1.75}{$\bullet$}}) or 30 (\textcolor{blue}{$\blacksquare$}). Darker symbols correspond to larger mass loading which varies from 0.2 to 1.0.}
  \label{fig:uavg}
\end{figure}

\begin{figure}\centering
  \begin{subfigure}{0.49\linewidth}
    \centering
    \includegraphics[width=3.20in]{plotting/fig8a.pdf}
    \caption{}
    \label{fig:u_rms_st6}
  \end{subfigure}\hfill
  \begin{subfigure}{0.49\linewidth}
    \centering
    \includegraphics[width=3.20in]{plotting/fig8b.pdf}
    \caption{}
    \label{fig:u_rms_st30}
  \end{subfigure}\\
  \begin{subfigure}{0.49\linewidth}
    \centering
    \includegraphics[width=3.20in]{plotting/fig8c.pdf}
    \caption{}
    \label{fig:v_rms_st6}
  \end{subfigure}\hfill
	\begin{subfigure}{0.49\linewidth}
		\centering
		\includegraphics[width=3.20in]{plotting/fig8d.pdf}
		\caption{}
		\label{fig:v_rms_st30}
	\end{subfigure}\\
	\begin{subfigure}{0.49\linewidth}
		\centering
		\includegraphics[width=3.20in]{plotting/fig8e.pdf}
		\caption{}
		\label{fig:w_rms_st6}
	\end{subfigure}\hfill
	\begin{subfigure}{0.49\linewidth}
		\centering
		\includegraphics[width=3.20in]{plotting/fig8f.pdf}
		\caption{}
		\label{fig:w_rms_st30}
	\end{subfigure}
  \caption{Variation of the fluid root-mean-square velocity fluctuations with the wall normal distance for $\Sto^+ = 6$ (a,c,e) and $\Sto^+ = 30$ (b,d,f) at $M = 0.2-1.0$. The solid black line represents a particle-free channel. Symbols as in figure \ref{fig:uavg}.
  \label{fig:vel_rms}}
\end{figure}

Figure \ref{fig:uavg} shows the mean streamwise velocity profile for the particle-laden cases with $\Sto^+ = 6$ and $\Sto^+ = 30$ at the mass loadings $M=0.2$, 0.6, and 1.0. The velocity profile in the particle-free channel flow is also shown for comparison. In the viscous sublayer, the profiles follow the same linear scaling as the unladen channel. The largest impact of the inertial particles manifests in the buffer and logarithmic layers. In channels laden with $\Sto^+=30$ particles, the fluid velocity profile shifts upward in the logarithmic layer from the reference profile of the unladen channel. This trend is further reinforced with increasing mass loading which leads to greater upward shift of the velocity profile. These observations are in agreement with those of \citet{zhouNonmonotonicEffectMass2020} who found a similar upward shift of the streamwise velocity profile in their simulations at $\Rey_\tau=180$, $\Sto^+=30$, and $M=0.75$. Conversely, the profile of the mean streamwise velocity shifts downward from the reference unladen flow when $\Sto^+=6$ particles are suspended. Similar to the cases with higher inertia particles, increasing the mass loading causes an amplification of the trend observed, i.e., a downward shift of the profile here.

Figures \ref{fig:isocontour_u} and \ref{fig:uavg} provide qualitative and quantitative evidence that $\Sto^+=30$ particles increase the fluid mass flow rate, while $\Sto^+=6$ particles decrease it. As discussed in \S\ref{sec:unladen}, the implication of this flow modulation in a channel driven by a constant pressure gradient is that $\Sto^+=30$ particles decrease skin-friction drag, whereas $\Sto^+=6$ perform the opposite, i.e., increase skin-friction drag.

Figures \ref{fig:vel_rms} shows the variation of the velocity fluctuations root-mean-square (r.m.s.) with Stokes number and mass loading. For $\Sto^+=6$ particles, the profile of the streamwise fluctuations  at $M=0.2$ and $M=0.6$ changes little from the profile of the unladen flow. Only when mass loading is increased to $M = 1.0$ do we see a significant change of the streamwise fluctuations, primarily in the log region. Conversely, $\Sto^+=30$ particles lead to a more pronounced modulation of the streamwise velocity fluctuations at all three mass loadings considered. Generally, the streamwise fluctuations decrease slightly in the viscous layer and increase significantly in the buffer and logarithmic layers. Further, the location of the peak shifts from about $y^+=14$ in the unladen case to $y^+=22$ at $M=1.0$. While $\Sto^+=30$ and $\Sto^+=6$ particles exhibit different modulation characteristics for the fluid streamwise velocity fluctuations, their impact on the wall normal and spanwise velocity fluctuations displays fewer differences. Both particles cause significant dampening of the wall normal and spanwise fluctuations in the viscous, buffer and logarithmic layers. Increasing mass loading leads to larger reduction of these fluctuations.

\begin{figure}
  \includegraphics[width=5.0in]{plotting/fig9.pdf}
  \caption{Relative change of skin-friction drag in particle-laden turbulent channel flows. Positive values indicate drag reduction, whereas negative values indicate drag increase. Symbols as in figure \ref{fig:uavg}.\label{fig:friction_coef}
  }
\end{figure}
By modulating the fluid bulk velocity, the dispersed inertial particles lead to a large change in skin-friction drag. Figure \ref{fig:friction_coef} shows times histories of relative change in skin-friction coefficient for all cases in table \ref{tab:parameters}. The reference value, $C_{f,\mathrm{0}}$, corresponds to the skin-friction drag coefficient from the statistically stationary particle-free channel flow. Figure \ref{fig:friction_coef} illustrates how particle inertia plays a selective role by determining the type of flow modulation obtained, be it drag-reducing or drag-increasing, while mass loading acts as an amplifying factor.
For all these cases, we compute the the drag reduction factor  $\mathrm{DR}$, defined as
\begin{equation}
  \mathrm{DR}= \frac{C_{f,\mathrm{0}} - C_f}{C_{f,\mathrm{0}}},
\end{equation}
which takes positive values in the case of drag reduction, and negative values in the case of drag increase. Further, we compute the change in mass flow rate $(\dot{m}_f-\dot{m}_{f,0})/\dot{m}_{f,0}$, where $\dot{m}_{f,0}$ is the mass flow rate in the reference particle-free channel. We report these values in table \ref{tab:drag_reduction}. The greatest drag reduction is obtained with $\Sto^+=30$ particles at the mass loading $M=1.0$, which yields drag reduction factor $\mathrm{DR}=19.54\%$ and corresponding mass flow rate increase of 11.07\%. The latter value is substantially higher than what has been reported in the literature, in particular, by \citet{liNumericalSimulationParticleladen2001} who found an increase in mass flow rate by about $\sim 5\%$ using particles with Stokes number of $O(100)$. This level of drag reduction shows that inertial particles can induce drag reduction at a level comparable with the one obtained using polymer additives \citep{housiadasPolymerinducedDragReduction2003,thaisTemporalLargeEddy2010}, such as in the polymeric channel flow simulations of \citet{housiadasPolymerinducedDragReduction2003} at $\Rey_\tau=180$  where drag reduction $\mathrm{DR}\simeq 25\%$ is reported. Note the drag reduction effect of $\Sto^+=30$ particles decreases significantly at lower mass loadings. At $M=0.2$, these particles reduce drag by only 4.21\%, and yield a modest mass flow rate increase of 2.22\%. This weaker drag reduction is to be expected because the particle feedback force scales with mass loading, thus, particle-induced flow modulation vanishes as $M$ decreases. This also holds for the drag increasing particles with $\Sto^+=6$ whose effect increases with mass loading resulting in drag increase by $16.92\%$ and mass flow rate decrease by 6.10\% at $M=1.0$.



\begin{table}
  \caption{Variation of percent drag reduction and mass flow rate in the present simulations. \label{tab:drag_reduction}}
  \begin{ruledtabular}
    \begin{tabular}{llll}
      Stokes number ($\Sto^+$) & Mass loading ($M$) & DR(\%) & $\Delta \dot{m}_f/\dot{m}_{f,0}$(\%)  \\\hline
      6             & 0.2          & -1.76      & -0.78      \\
                    & 0.6          & -8.90      & -3.93      \\
                    & 1.0          & -16.92      & -6.10      \\
      30            & 0.2          & 4.21       & 2.22      \\
                    & 0.6          & 12.27       & 7.37      \\
                    & 1.0          & 19.54				& 11.07
    \end{tabular}
  \end{ruledtabular}
\end{table}


\subsection{Shear stress balance in the presence of inertial particles}
\label{sec:stress_two_phase}

As with the unladen flow, the structure of the flow in a particle-laden channel results from a balance of stresses applied on the fluid. However, the presence of particles introduces additional stresses that alter the balance in equation (\ref{eq:RSS_1}). To derive a new balance that takes into account particle stresses, we apply Reynolds-averaging to the momentum equation (\ref{eq:NVS_2}). Assuming that particle clustering does not break the dilute limit locally ($1-\phi\simeq 1$), the resulting balance is
\begin{equation}
  \frac{d}{dy}\left(\mu\frac{d }{dy}\langle u_f\rangle - \rho_{f}\langle u_{f}'v_f'\rangle \right)+ \langle F_{p,x}\rangle= -\bigg\langle \frac{\partial p}{\partial x}\bigg\rangle,
    \label{eq:RSS_2}
\end{equation}
where $\langle F_{p,x}\rangle$ represents the mean streamwise particle stresses.
{\color{revision}
The latter can be related to the particle-phase Reynolds shear stress. To do so, we consider the particle conservation equations in the Eulerian frame. Using the Two-Fluid model discussed in \citep{kasbaouiClusteringEulerEuler2019} under the assumption of mono-kinetic particle velocity distribution, the particle mass and momentum conservation equations read
\begin{eqnarray}
    \frac{\partial }{\partial t}(\rho_p\phi) + \nabla \cdot (\rho_p\phi\bm{u}_p)&=& 0 \\
     \frac{\partial}{\partial t}(\rho_p\phi\bm{u}_p) + \nabla \cdot (\rho_p\phi\bm{u}_p\bm{u}_p) &=& -\bm{F}_p +\bm{C}
\end{eqnarray}
where $\bm{u}_p$ is the $\bm{C}$ represents the collision stresses. Neglecting the latter and averaging the streamwise particle momentum balance yields}
\begin{equation}
	\frac{d}{dy}  \left(\rho_p\langle \phi u''_p v''_p\rangle\right)=-\langle F_{p,x}\rangle. \label{eq:RSS_3}
\end{equation}
Here, $u_p''$ and $v_p''$ refer to the streamwise and wall-normal particle-phase velocity fluctuations with respect to the Favre-averaged particle velocities $\widetilde{u}_p=\langle \phi u_p\rangle/\langle \phi\rangle$ and $\widetilde{v}_p=\langle \phi v_p\rangle/\langle \phi\rangle$. Combining equations (\ref{eq:RSS_2}) and (\ref{eq:RSS_3}) yields
\begin{equation}
   \frac{d}{dy}\left(\mu\frac{d}{dy} \langle u_f\rangle - \rho_{f}\langle u_{f}'v_f'\rangle - \rho_{p}\langle \phi u_{p}''v_{p}'' \rangle \right)= -\bigg\langle\frac{\partial p}{\partial x}\bigg\rangle,
    \label{eq:RSS_5}
\end{equation}
which integrates to
\begin{equation}
  \mu\frac{d \langle u_f\rangle }{d y} - \rho_{f}\langle u_{f}'v_f'\rangle - \rho_p\langle \phi u_{p}''v_{p}'' \rangle=\tau_w \left(1-\frac{y}{h}\right) .
    \label{eq:RSS_6}
\end{equation}
Similar to the particle-free channel, equation (\ref{eq:RSS_6}) shows the total stress varies linearly across the channel provided that the particle-phase Reynolds shear stress $\rho_{p}\langle \phi u_{p}''v_{p}'' \rangle $ is also taken into account. Integrating equation (\ref{eq:RSS_6}) twice, leads to an updated expression for the fluid mass flow rate by unit spanwise length which takes into account the effect of the dispersed particles,
\begin{equation}
  \frac{\dot{m_f}}{Lz} = \frac{2}{3}\frac{\tau_w h^2}{\nu}\left(1+\frac{3}{(u_\tau h)^2}\int_{0}^{h}\left(\int_{0}^y \langle u_{f}'v_f'\rangle + \frac{M}{\phi_0} \langle \phi u_{p}''v_{p}'' \rangle dy'\right) dy\right).
    \label{eq:MFR_2}
\end{equation}
The relationship (\ref{eq:MFR_2}) shows that the particles alter the fluid mass flow rate through two competing effects: (i) a direct effect through the particle-phase Reynolds shear stress $\rho_p\langle \phi u_{p}''v_{p}'' \rangle$ which, like the fluid-phase Reynolds shear stress, tends to reduce the mass flow rate, and (ii) an indirect effect through the modulation of the fluid-phase shear stress $\rho_f\langle u_{f}'v_f'\rangle$. It is only when the fluid-phase shear stress is reduced more than can be balanced by the particle-phase Reynolds shear stress that the fluid mass flow rate is increased.

 \begin{figure}\centering
  \centering
  \begin{subfigure}{0.49\textwidth}
    \centering
    \includegraphics[width=3.20in]{plotting/fig10a.pdf}
    \caption{}
    \label{fig:tsscomparest30}
  \end{subfigure}
  \hfill
  \begin{subfigure}{0.49\textwidth}
    \centering
    \includegraphics[width=3.20in]{plotting/fig10b.pdf}
    \caption{}
    \label{fig:tsscomparest6}
  \end{subfigure}

  \caption{Shear stress contributions as a function of the wall normal distance for the particle-laden turbulent channel flows at (a) $\Sto=30$, $M=1.0$ and (b) $\Sto=6$, $M=1.0$. The viscous shear stress is denoted by (\circles), fluid-phase shear stress by (\squares), particle-phase shear stress by (\diamonds) and the total shear stress by (\triangles). Lines without symbols correspond to the reference single-phase channel as denoted in figure \ref{fig:tss_sp}.}
  \label{fig:tss_comparison}
\end{figure}

Figure \ref{fig:tss_comparison} shows the total shear stress profile and the variations in the fluid and particle stress components for cases $\Sto^+ = 6$ and $\Sto^+ = 30$ at $M = 1.0$. In both cases, the total shear stress varies linearly across the channel as predicted by (\ref{eq:RSS_6}). This first observation validates the two hypotheses underpinning the relationship (\ref{eq:RSS_6}): (i) $1-\phi\simeq 1$ meaning that the particle phase remains dilute even though significant clustering occurs near the walls as we show in \S \ref{sec:mechanism}, and  (ii) collisional stresses are negligible compared to hydrodynamic stresses exerted on particles, even within particle clusters. Considering $\Sto^+ = 30$ particles, figure \ref{fig:tsscomparest30} shows partial relaminarization of the near-wall region. Compared to the reference particle-free flow, the viscous drag increases in the viscous and buffer layers. This modulation is directly linked to the increase in fluid mass flow rate observed in figure \ref{fig:isocontour_u}. Further, the fluid-phase Reynolds shear stresses drops significantly with a peak down to about 39\% of the unladen case, and is shifted further towards the centerline. This drop is partially balanced by  the rise of particle-phase Reynolds shear stress. The latter dominates in the region $0.1\lesssim y/h\lesssim 0.45$ ($18\lesssim y^+\lesssim 81$) and is a comparable to the fluid-phase Reynolds shear stress towards the centerline. Conversely, figure \ref{fig:tsscomparest6} shows that $\Sto^+ = 6$ particles cause a drop of the viscous stress. This is expected since the fluid mass flow rate reduces with these particles. $\Sto^+ = 6$ particles also cause significantly lower fluid-phase Reynolds shear stress, although slightly less than $\Sto^+ = 30$ particles since the peak $\rho_{f}\langle u_{f}'v_f'\rangle$ drops to only 46\% of the unladen case. Further, $\Sto^+ = 6$ particles  cause slightly larger particle-phase shear stress.

Figure \ref{fig:rss_pss_comparison} shows the effect of varying mass loading on the fluid and particle shear stresses. For both $\Sto^+=30$ and $\Sto^+=6$ particles, the particle shear stress rises with increasing mass loading while the fluid Reynolds shear stress drops. As shown by the relationships (\ref{eq:RSS_6}) and (\ref{eq:MFR_2}), the competition between increasing particle shear stress and reducing fluid Reynolds shear stress is what ultimately determines whether the particles increase or decrease the mass flow rate, and \emph{a fortiori}, drag reduction or drag increase, respectively. Figure \ref{fig:rss_pss_sum} shows how increasing mass loading causes a progressive deviation of the total Reynolds shear stress  $\rho_{f}\langle u_{f}'v_f'\rangle + \rho_p \langle \phi u_{p}''v_{p}'' \rangle$ from the single phase Reynolds shear stress. It is clear that $\Sto^+=30$ particles reduce the total Reynolds shear stress, although at a rate that varies little from $M=0.6$ to $M=1.0$ suggesting a possible saturation. With $\Sto^+=6$ particles, there is an increase of total Reynolds shear stress which accentuates with increasing mass loading.


\begin{figure}
  \centering
  \begin{subfigure}{0.49\textwidth}
    \centering
    \includegraphics[width=3.20in]{plotting/fig11a.pdf}
    \caption{}
    \label{fig:rsscomparest6}
  \end{subfigure}
  \hfill
  \begin{subfigure}{0.49\textwidth}
    \centering
    \includegraphics[width=3.20in]{plotting/fig11b.pdf}
    \caption{}
    \label{fig:rsscomparest30}
  \end{subfigure}\\
	\begin{subfigure}{0.49\textwidth}
		\centering
		\includegraphics[width=3.20in]{plotting/fig11c.pdf}
		\caption{}
		\label{fig:psscomparest6}
	\end{subfigure}
	\hfill
	\begin{subfigure}{0.49\textwidth}
		\centering
		\includegraphics[width=3.20in]{plotting/fig11d.pdf}
		\caption{}
		\label{fig:psscomparest30}
	\end{subfigure}
  \caption{Fluid-phase and particle-phase Reynolds shear stress for (a,c) $\Sto^+=6$ and (b,d) $\Sto^+=30$ respectively. Symbols as in figure \ref{fig:uavg}. The solid black line denotes the single-phase Reynolds shear stress.}
  \label{fig:rss_pss_comparison}
\end{figure}

\begin{figure}
	\centering
	\begin{subfigure}{0.49\textwidth}
		\centering
		\includegraphics[width=3.20in]{plotting/fig12a.pdf}
		\caption{}
		\label{fig:sscomparest6}
		\end{subfigure}\hfill
		\begin{subfigure}{0.49\textwidth}
		\centering
		\includegraphics[width=3.20in]{plotting/fig12b.pdf}
		\caption{}
		\label{fig:sscomparest30}
\end{subfigure}
\caption{Total Reynolds shear stress for (a) $\Sto^+=6$ and (b) $\Sto^+=30$ respectively. Symbols as in figure \ref{fig:uavg}. The solid black line denotes the single-phase Reynolds shear stress.}
\label{fig:rss_pss_sum}
\end{figure}
