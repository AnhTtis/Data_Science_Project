\subsection{Velocity modulation due to particle-fluid interaction}

There has been considerable interest in recent years on drag modulation due to particles within a channel flow. Many studies have been performed using various methods such as the eulerian-lagrangian point particle method \citep{zhouNonmonotonicEffectMass2020,LagrangianStatisticsTurbulent1995,bakerDirectComparisonEulerian2020}, the immersed boundary approach \citep{picanoTurbulentChannelFlow2015,breugemSecondorderAccurateImmersed2012} and the lattice Boltzmann method \citep{pengDirectNumericalInvestigation2019,wangLatticeBoltzmannSimulation2016,zhangLatticeBoltzmannMethod2016}. Studies however have not been able to present a clear picture as to how different particle characteristics (Stokes number, mass loading) affect the drag within a flow. This section aims to understand the velocity statistics for low-moderate mass loading (0.2-0.6) at Stokes numbers of O(1-10) [Cases \textbf{A-C} ($\Sto^+ = 6$) and cases \textbf{D-F} ($\Sto^+ = 30$)], using the eulerian-lagrangian point particle approach, which has been a proven method for particle-laden flows (**reference needed**). 

\begin{figure}[htp]
  \begin{subfigure}{0.45\textwidth}
    \centering
    \includegraphics[width=1\linewidth]{plotting/AVGVEL_ST6.pdf}
    \caption{}
    \label{fig:fvelst6}
  \end{subfigure}
  \begin{subfigure}{0.45\textwidth}
    \centering
    \includegraphics[width=1\linewidth]{plotting/FURMS_ST6.pdf}
    \caption{}
    \label{fig:furmsst6}
  \end{subfigure}
  \caption{(a) Streamwise velocity and (b) streamwise velocity fluctuations plotted as a function of $y^+$ for $St^+ = 6$.} 
  \label{fig:fluid_stats_st6}
\end{figure}

\begin{figure}[htp]
  \begin{subfigure}{0.45\textwidth}
    \centering
    \includegraphics[width=1\linewidth]{plotting/AVGVEL_ST30.pdf}
    \caption{}
    \label{fig:fvelst30}
  \end{subfigure}
  \begin{subfigure}{0.45\textwidth}
    \centering
    \includegraphics[width=1\linewidth]{plotting/FURMS_ST30.pdf}
    \caption{}
    \label{fig:furmsst30}
  \end{subfigure}
  \caption{(a) Streamwise velocity and (b) streamwise velocity fluctuations plotted as a function of $y^+$ for $St^+ = 30$.} 
  \label{fig:fluid_stats_st30}
\end{figure}

Figure \ref{fig:fluid_stats_st6} shows the streamwise velocity $\langle u_f\rangle / u_\tau$ and the streamwise fluctuations $\sqrt{\langle u^{`2}_f \rangle}/u_\tau$ for cases \textbf{A-C}. It is observed that the particles have a weak modulation on the near wall statistics for for $\langle u_f\rangle / u_\tau$ and $\sqrt{\langle u^{`2}_f \rangle}/u_\tau$ at $\Sto^+ = 6$. There is a monotonic reduction observed for $\langle u_f\rangle / u_\tau$ in the outer layer with increasing mass loading. $\sqrt{\langle u^{`2}_f \rangle}/u_\tau$ decreases throughout the channel compared to the single phase flow, with maximum modulation witnessed for case \textbf{C}. 

Cases \textbf{D-F} all present an enhancement in $\langle u_f\rangle / u_\tau$ with maximum increase observed for case \textbf{F} as shown in fig. \ref{fig:fvelst30}. \citet{zhouNonmonotonicEffectMass2020} conduct a similar study as shown for cases \textbf{D-F} at $\Rey_\tau = 180$ and $\Sto^+ = 30$. They observe a similar increase in $\langle u_f\rangle / u_\tau$ within the log-outer layer, with the exception that the present study does not observe the near center-line reduction that was witnessed by \citet{zhouNonmonotonicEffectMass2020}. There is a significant near-wall reduction in $\sqrt{\langle u^{`2}_f \rangle}/u_\tau$, and a monotonic increase in the log-outer layer with increasing mass loading for cases \textbf{D-F} as shown in fig. \ref{fig:furmsst30}. A similar behaviour was observed by \citet{zhouNonmonotonicEffectMass2020}. 

The modulation of the streamwise velocity fluctuations, more specifically within the log-outer layer show a direct correlation to the increase in the overall mass flow rate. As the log-outer layer $\sqrt{\langle u^{`2}_f \rangle}/u_\tau$ is decreased we see a reduction in the mass flow rate (cases \textbf{A-C}),  and as $\sqrt{\langle u^{`2}_f \rangle}/u_\tau$ is increased in the log-outer layer we see an increase in the mass flow rate (cases \textbf{D-F}). 

The wall normal (fig. \ref{fig:fvrms}) and spanwise (fig. \ref{fig:fwrms}) velocity fluctuations show a significant reduction within the flow for all cases compared to the single phase flow. There is a strong relationship between increasing mass loading and decreasing $\sqrt{\langle v^{`2}_f \rangle}/u_\tau$ and $\sqrt{\langle w^{`2}_f \rangle}/u_\tau$ within the range of $\Sto^+ = 6-30$. A similar reduction is also noticed by \citet{zhouNonmonotonicEffectMass2020}, for particle-laden channel flows run at a similar semi-dilute regime at $St^+ = 30$. The mass loading therefore plays a more significant role than the Stokes number in modulating the fluid wall-normal and spanwise velocity fluctuations in a semi-dilute regime in the range of $St^+ = O(1-10)$. 

The modulation of the streamwise fluid velocity within the channel is therefore dependent largely on the streamwise velocity fluctuations. Enhancement of the mass flow rate can be achieved through particles that have a higher inertia ($\Sto^+ = 30$), due to staying closer to the wall. This helps modulate the fluid characteristics near the wall more strongly, primarily $\sqrt{\langle u^{`2}_f \rangle}/u_\tau$,  than particle with a lower inertia ($\Sto^+ = 6$). Particles with a larger inertia tend to stay closer to the wall on average throughout the channel compared to lower inertia paticles which are more spread out within the channel. This is shown clearly in fig. \ref{fig:pnd}, where the particle number density is plotted compared to the distance away from the wall. From fig. \ref{fig:fluid_stats_st6}, \ref{fig:fluid_stats_st30} and \ref{fig:fluid_stats_rms}, we can notice that particles help reduce the velocity fluctuations around them. However, it is observed that greater log-outer layer streamwise fluid velocity fluctuations helps aid the enhancement in the mass flow rate. This is easier done through particles of higher inertia that stay closer to the wall reducing the near wall streamwise fluid fluctuations significantly increasing the log-outer layer fluctuations, enhancing the overall mass flow rate. In contrast lower inertia particles tend to be scattered throughout the channel, not delivering the effect that higher inertia particles do in modulation the fluid flow.       

\begin{figure}[htp]
  \begin{subfigure}{0.45\textwidth}
    \centering
    \includegraphics[width=1\linewidth]{plotting/FVRMS.pdf}
    \caption{}
    \label{fig:fvrms}
  \end{subfigure}
  \begin{subfigure}{0.45\textwidth}
    \centering
    \includegraphics[width=1\linewidth]{plotting/FWRMS.pdf}
    \caption{}
    \label{fig:fwrms}
  \end{subfigure}
  \caption{(a) Wall-normal and (b) spanwise fluctuations plotted as a function of $y^+$.} 
  \label{fig:fluid_stats_rms}
\end{figure}

The present study aims to understand the behaviour of the particle phase compared to the fluid phase as we move away from the wall. Figure \ref{fig:fluid_particle_vel_stats}, \ref{fig:global_fluid_particle_velfluc_stats} and \ref{fig:pc_fluid_particle_velfluc_stats} take a look at the velocity characteristics between the two phases to understand how particle help modulate the fluid due to turbophoresis and preferential concentration. Figure \ref{fig:fluid_particle_vel_stats} compares the streamwise velocity between the particle $\langle u_p \rangle/u_\tau$ and the fluid $\langle u_f \rangle/u_\tau$ phase. Additionally, the fluid streamwise velocity conditioned at particle locations $\langle u_{f\mid p}\rangle/u_\tau$ is also presented. From all the cases it is observed that higher inertial particles ($\Sto^+ = 30$) modulate the fluid by showing a greater streamwise velocity within all regions of the flow i.e. $\langle u_f \rangle/u_\tau$, $\langle u_p \rangle/u_\tau$ and $\langle u_{f\mid p}\rangle/u_\tau$ compared to lower inertial particles ($\Sto^+ = 6$). Furthermore, All particle-laden cases show a similar behaviour overall with the particle phase lagging the fluid phase in the near-wall region. As the mass loading is increased the near wall streamwise velocity difference between the particle phase and the fluid phase is increased. However, the fluid at the particle locations significantly lags the overall flow. The magnitude of $\langle u_{f\mid p}\rangle/u_\tau$ is dependent on the inertia of the particles. Cases \textbf{D-F} show a higher $\langle u_{f\mid p}\rangle/u_\tau$ compared to cases \textbf{A-C} within all layers of the turbulent boundary layer and within the outer region.     

\begin{figure}[htp]
  \begin{subfigure}{0.45\textwidth}
    \centering
    \includegraphics[width=1\linewidth]{plotting/FPFPLU02.pdf}
    \caption{M = 0.2}
    \label{fig:fpfplu02}
  \end{subfigure}
  \begin{subfigure}{0.45\textwidth}
    \centering
    \includegraphics[width=1\linewidth]{plotting/FPFPLU02.pdf}
    \caption{M = 0.4}
    \label{fig:fpflpu04}
  \end{subfigure}
  \begin{subfigure}{0.45\textwidth}
    \centering
    \includegraphics[width=1\linewidth]{plotting/FPFPLU06.pdf}
    \caption{M = 0.6}
    \label{fig:fpflpu06}
  \end{subfigure}
  \caption{Fluid particle velocity comparison} 
  \label{fig:fluid_particle_vel_stats}
\end{figure}

To understand the particle modulation on the fluid further, fig. \ref{fig:global_fluid_particle_velfluc_stats} takes a look at the global fluid and particle velocity fluctuations in comparison. Figure \ref{fig:pfurms_st6} (\textbf{A-C}) and \ref{fig:pfurms_st30} (\textbf{D-F}) show the streamwise velocity fluctuations comparison between the particle and fluid phase. It is observed for the streamwise velocity fluctuations that the particle phase leads the fluid in the near wall region and lags the fluid in the outer layer for all inertial particles. Mass loading does not play a significant role in the modulation of the streamwise velocity fluctuations for particles of inertia $\Sto^+ = 6$, however as the inertia is increased there is monotonic increase in the fluid and particle streamwise velocity fluctuations as the mass loading is increased. The wall normal (fig. \ref{fig:pfvrms_st6},\ref{fig:pfvrms_st30}) and spanwise (fig. \ref{fig:pfwrms_st6},\ref{fig:pfwrms_st30}) velocity fluctuations on the other hand, both have the particle phase lagging the fluid phase and mass loading plays a role for all inertial particles. There is a monotonic decrease in both the fluid and particle wall normal and spanwise velocity fluctiations as mass loading is increased.    

\begin{figure}[htp]
  \begin{subfigure}{0.45\textwidth}
    \centering
    \includegraphics[width=1\linewidth]{plotting/PFURMS_ST6.pdf}
    \caption{$St^+ = 6$}
    \label{fig:pfurms_st6}
  \end{subfigure}
  \begin{subfigure}{0.45\textwidth}
    \centering
    \includegraphics[width=1\linewidth]{plotting/PFURMS_ST30.pdf}
    \caption{$St^+ = 30$}
    \label{fig:pfurms_st30}
  \end{subfigure}
  \begin{subfigure}{0.45\textwidth}
    \centering
    \includegraphics[width=1\linewidth]{plotting/PFVRMS_ST6.pdf}
    \caption{$St^+ = 6$}
    \label{fig:pfvrms_st6}
  \end{subfigure}
  \begin{subfigure}{0.45\textwidth}
    \centering
    \includegraphics[width=1\linewidth]{plotting/PFVRMS_ST30.pdf}
    \caption{$St^+ = 30$}
    \label{fig:pfvrms_st30}
  \end{subfigure}
  \begin{subfigure}{0.45\textwidth}
    \centering
    \includegraphics[width=1\linewidth]{plotting/PFWRMS_ST6.pdf}
    \caption{$St^+ = 6$}
    \label{fig:pfwrms_st6}
  \end{subfigure}
  \begin{subfigure}{0.45\textwidth}
    \centering
    \includegraphics[width=1\linewidth]{plotting/PFWRMS_ST30.pdf}
    \caption{$St^+ = 30$}
    \label{fig:pfwrms_st30}
  \end{subfigure}
  \caption{Global fluid particle velocity fluctuations comparison} 
  \label{fig:global_fluid_particle_velfluc_stats}
\end{figure}

The modulatin of the fluid velocity fluctuations by the particles is better understood by taking a look at the fluid velocity fluctuations at particle locations. Figure \ref{fig:pc_fluid_particle_velfluc_stats} shows the particle velocity fluctuations and the fluid velocity fluctuations at the particle locations. The reduction in the near-wall fluid streamwise velocity fluctuations $\sqrt{\langle u^{`2}_f \rangle}/u_\tau$, is due to the partiles modulating the fluid significantly. $\sqrt{\langle u^{`2}_{f \mid p} \rangle}/u_\tau$ is notably reduced near the wall (i.e. fig. \ref{fig:pcpfurms_st6}, \ref{fig:pcpfurms_st30}), helping reduce the overall $\sqrt{\langle u^{`2}_f \rangle}/u_\tau$ in that region. As observed in fig. \ref{fig:global_fluid_particle_velfluc_stats}, $\sqrt{\langle v^{`2}_p \rangle}/u_\tau$ and $\sqrt{\langle w^{`2}_p \rangle}/u_\tau$ have always been lower than the fluid counterpart. Furthermore, these particles tend to modulate the fluid strongly by reducing the $\sqrt{\langle v^{`2}_f \rangle}/u_\tau$ and $\sqrt{\langle w^{`2}_f \rangle}/u_\tau$ compared to the single phase flow. This modulation is better observed in fig. \ref{fig:pcpfvrms_st6}, \ref{fig:pcpfvrms_st30} (wall-normal velocity fluctuations) and fig. \ref{fig:pcpfwrms_st6}, \ref{fig:pcpfwrms_st30} (spanwise velocity fluctuations). The particles help significantly reduce the fluid velocity fluctuations at particle locations. This in turn helps reduce the global fluid velocity fluctuations. A common observation across all particle-laden cases is that the near-wall modulation of the fluid is stronger than the outer layer modulation. This is due to turbophoresis pushing the majority of the particles closer to the wall. Particles with a higher inertia ($St^+ = 30 > St^+ = 6$) tend to be 'heavier' and stay closer to the wall modulating the fluid more strongly. Figures \ref{fig:fluid_stats_st6} - \ref{fig:pc_fluid_particle_velfluc_stats} show that the more the near-wall $\sqrt{\langle u^{`2}_f \rangle}/u_\tau$ is reduced (in turn increasing the log-outer layer $\sqrt{\langle u^{`2}_f \rangle}/u_\tau$), the stronger the increase in the mass flow rate. Particles with a higher Stokes number tend to be able to perform better in increasing the mass flow rate because they tend the modulate the near-wall statistics more strongly that lower Stokes number particles. On the other hand, lower Stokes number particles tend to modulate the flow weakly throughout the channel, but also tend to modulate the flow in regions (i.e. log-outer layer) where drag enhancement is produced.    

\begin{figure}[htp]
  \begin{subfigure}{0.45\textwidth}
    \centering
    \includegraphics[width=1\linewidth]{plotting/PC_PFURMS_ST6.pdf}
    \caption{$St^+ = 6$}
    \label{fig:pcpfurms_st6}
  \end{subfigure}
  \begin{subfigure}{0.45\textwidth}
    \centering
    \includegraphics[width=1\linewidth]{plotting/PC_PFURMS_ST30.pdf}
    \caption{$St^+ = 30$}
    \label{fig:pcpfurms_st30}
  \end{subfigure}
  \begin{subfigure}{0.45\textwidth}
    \centering
    \includegraphics[width=1\linewidth]{plotting/PC_PFVRMS_ST6.pdf}
    \caption{$St^+ = 6$}
    \label{fig:pcpfvrms_st6}
  \end{subfigure}
  \begin{subfigure}{0.45\textwidth}
    \centering
    \includegraphics[width=1\linewidth]{plotting/PC_PFVRMS_ST30.pdf}
    \caption{$St^+ = 30$}
    \label{fig:pcpfvrms_st30}
  \end{subfigure}
  \begin{subfigure}{0.45\textwidth}
    \centering
    \includegraphics[width=1\linewidth]{plotting/PC_PFWRMS_ST6.pdf}
    \caption{$St^+ = 6$}
    \label{fig:pcpfwrms_st6}
  \end{subfigure}
  \begin{subfigure}{0.45\textwidth}
    \centering
    \includegraphics[width=1\linewidth]{plotting/PC_PFWRMS_ST30.pdf}
    \caption{$St^+ = 30$}
    \label{fig:pcpfwrms_st30}
  \end{subfigure}
  \caption{Fluid conditioned at particle locations and the particle phase velocity fluctuations comparison} 
  \label{fig:pc_fluid_particle_velfluc_stats}
\end{figure}


