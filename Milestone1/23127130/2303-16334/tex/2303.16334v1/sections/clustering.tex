\subsection{Interplay between particle clusters and near-wall coherent structures}
\label{sec:mechanism}
In this section, we show that modulating the skin-friction drag depends to a large extent on how particle clusters interact with near-wall coherent structures.

\begin{figure}
  \includegraphics[width=5in]{tikzgraphics/fig13.pdf}
  \caption{Isocontours of normalized particle volume fraction in a wall-normal plane showing the presence of clusters and the accumulation of particles near the walls. As in figure \ref{fig:isocontour_u}, the larger domain for $\Sto^+ = 30$ particles is truncated to the same size as the domain for $\Sto^+ = 6$ particles to facilitate visual comparison.}
  \label{fig:vfp}
\end{figure}
The distribution of $\Sto^+=6$ and $\Sto^+=30$ particles within the channel is strongly inhomogeneous.
Visualization of normalized particle volume fraction in a wall-normal plane in figure \ref{fig:vfp} shows that the particles concentrate in long filamentous clusters that may span the entire channel height.
$\Sto^+=30$ particles form clusters that are relatively denser and further elongated in the streamwise direction compared to clusters formed by $\Sto^+=6$ particles. Figure \ref{fig:vfp} also shows that the normalized particle volume fraction within the bulk of the channel is lower at mass loading $M=0.1$, compared to the bulk normalized volume fraction at $M=0.6$ and 1.0. This points to a tendency of particles to accumulate near the walls that is stronger at $M=0.1$  than at $M=0.6$ and $M=1$.  Note that the formation of such clusters is expected owing to the fact that the particles considered in this study have significant inertia. As previously discussed by several investigators, inertial particles in wall-bounded turbulent flows tend to form clusters due to two effects, namely, turbophoresis, i.e., the migration of inertial particles to lower turbulence regions near the walls \citep{caporaloniTransferParticlesNonisotropic1975,reeksTransportDiscreteParticles1983,nowbaharTurbophoresisAttenuationTurbulent2013,kuertenTurbulenceModificationHeat2011}, and preferential concentration, i.e., the migration of inertial particles from vortical regions to straining regions of the flow \citep{eatonPreferentialConcentrationParticles1994,marchioliStatisticsParticleDispersion2008,kasbaouiTurbulenceModulationSettling2019,fongVelocitySpatialDistribution2019}. It follows that the particle feedback force is concentrated along these structures, and that the resulting flow modulation depends largely on the cluster morphology and dynamics.

\begin{figure}\centering
  \begin{subfigure}{0.49\linewidth}
    \centering
    \includegraphics[width=3.2in]{plotting/fig14a.pdf}
    \caption{}
    \label{fig:pnd_st6}
  \end{subfigure}\hfill
  \begin{subfigure}{0.49\linewidth}
    \centering
    \includegraphics[width=3.2in]{plotting/fig14b.pdf}
    \caption{}
    \label{fig:pnd_st30}
	\end{subfigure}
  \caption{Particle number density normalized by the average particle number density as a function of the wall normal distance for (a) $\Sto^+ = 6$ and (b) $\Sto^+ = 30$ at various mass loadings. Symbols as in figure \ref{fig:uavg}.}
  \label{fig:PND}
\end{figure}

Although particle clusters can be observed throughout the channel, it is near the walls that the majority of particles accumulate. Figure \ref{fig:PND} shows the variation of the normalized plane-averaged volume fraction ${\langle \phi\rangle}/ \phi_0$ with the wall normal distance. Within the region $y^+ < 10$, the local particle volume fraction is several times larger than the mean volume fraction $\phi_0$, which shows that the majority of the particles accumulate near the walls. $\Sto^+=30$ particles lead to the largest wall accumulation reaching ${\langle \phi\rangle}/ \phi_0\simeq 4.98$ at $M=1$ compared to ${\langle \phi\rangle}/ \phi_0\simeq 2.62$ for $\Sto^+=6$ particles at the same mass loading. Similar observations were made by \citet{nilsenVoronoiAnalysisPreferential2013} and \citet{yuanThreedimensionalVoronoiAnalysis2018} who, despite considering only one-way coupling, found that particles with $\Sto^+=30$ have the greatest wall-accumulation among particles with $\Sto^+$ in the range 1-100. Interestingly, the particle wall accumulation reduces when mass loading increases. At $M=0.1$, the particle volume fraction at the wall rises to $\langle\phi\rangle/\phi_0\simeq 14.96$ and 6.7 for $\Sto^+=30$ and $\Sto^+=6$, respectively. This finding is in agreement with the observation from figure \ref{fig:vfp} that the relative bulk particle volume fraction is lowest at $M=0.1$ as relatively more particles accumulate at the walls with decreasing $M$. This effect likely results from two-way coupling, given that particle-particle collisions are weak in the present semi-dilute regime.

Here, we stress that capturing the particle ropes accurately and the subsequent flow modulation requires much larger domains than those generally used in simulations of particle-laden turbulent channel flows \citep{rousonPreferentialConcentrationSolid2001,zhaoTurbulenceModulationDrag2010a,bernardiniReynoldsNumberScaling2014,costaInterfaceresolvedSimulationsSmall2020,jieExistenceFormationMultiscale2022}. The present large domain used for simulations with $\Sto^+=30$ particles is sufficiently wide to allow a natural development of flow and particle structures in the spanwise direction. However, even with a streamwise length of $12\pi h\sim 38h$, the domain remains too short to properly characterize the average streamwise length of the particle ropes.

\begin{figure}
    \centering
  \begin{subfigure}{\linewidth}
    \centering
    \includegraphics[width=4.5in]{tikzgraphics/fig15a.pdf}
    \caption{}
    \label{fig:cluster_length_st30_2d1}
  \end{subfigure}
  \begin{subfigure}{\linewidth}
    \centering
    \includegraphics[width=4in]{tikzgraphics/fig15b.pdf}
    \caption{}
    \label{fig:cluster_length_st30_2d2}
  \end{subfigure}
  \begin{subfigure}{\linewidth}
    \centering
    \includegraphics[width=4in]{tikzgraphics/fig15c.pdf}
    \caption{}
    \label{fig:cluster_length_st6_2d1}
\end{subfigure}
  \caption{\textcolor{revision}{Isocontours of normalized particle volume fraction at $y^+ = 10$ for (a,b) $\Sto^+ = 30, M = 1.0$ and (c) $\Sto^+ = 6, M = 1.0$. The view in (b) corresponds to the area marked by the red rectangle in (a).}}
  \label{fig:cluster_length}
\end{figure}
With most of the particles concentrating near the walls, clusters found therein have the largest impact on the carrier flow. As shown in figure \ref{fig:cluster_length}, the topology of these structures varies significantly depending on whether the particles are drag-reducing ($\Sto^+=30$) or drag-increasing ($\Sto^+=6$). For better comparison of the scales, figure \ref{fig:cluster_length_st30_2d2} shows a view of the particle volume fraction field for $\Sto^+=30$ particles cropped to the same dimensions as the smaller domain used with $\Sto^+=6$ particles and shown in figure \ref{fig:cluster_length_st6_2d1}. In contrast with $\Sto^+=6$ particles, the higher inertia particles at $\Sto^+=30$ form distinctively long and stable clusters. These structures, which we call \emph{ropes}, span the entire length of the domain in the streamwise direction, i.e, over 6000 wall units. The ropes travel downstream but remain stable and coherent for dynamically significant times. Further, the ropes repeat periodically in the spanwise direction in a fashion reminiscent of low-speed streaks discussed in \S\ref{sec:unladen}. This suggests that formation of these ropes results from the interaction of particle clusters with coherent flow structures in the buffer layer. The fact that no such ropes are observed with $\Sto^+=6$ particles suggests that intermittent flow structures in the buffer layer are capable of breaking down clusters formed by low inertia particles, whereas clusters formed by particles with large inertia retain their spatial and temporal coherence. The stable particle ropes may in turn alter the near-wall coherent flow structures.

\begin{figure}
    \centering
    \begin{subfigure}{\linewidth}
      \centering
      \includegraphics[width=4.5in]{tikzgraphics/fig16a.pdf}
      \caption{$\Sto^+ =30, M = 1.0$}
      \label{fig:riblet_S30M10_large}
    \end{subfigure}
    \begin{subfigure}{\linewidth}
      \centering
      \includegraphics[width=4in]{tikzgraphics/fig16b.pdf}
      \caption{$\Sto^+ =30, M = 1.0$}
      \label{fig:riblet_S30M10}
    \end{subfigure}
    \begin{subfigure}{\linewidth}
    \centering
    \includegraphics[width=4in]{tikzgraphics/fig16c.pdf}
    \caption{$\Sto^+ = 6, M = 1.0$}
    \label{fig:riblet_S6M10}
  \end{subfigure}
  \caption{\textcolor{revision}{Overlay of the isocontours of fluid streamwise velocity, and the contour of the relative particle volume fraction $\phi/\phi_0 = 3$ at $y^+ = 10$ for (a,b) $\Sto=30$, $M=1.0$ and (c) $\Sto^+ = 6, M = 1.0$. The view in (b) corresponds to the area marked by the red rectangle in (a).}}
  \label{fig:riblet_comparison}
\end{figure}
In order to shed light on how particle ropes interact with near-wall coherent flow structures, we report in figure  \ref{fig:riblet_comparison} isocontours of streamwise velocity at $y^+ = 10$ with the iso-level $\phi=3\times\phi_0$ overlayed on top. The latter shows the regions where the particles cluster. For the flow laden with $\Sto^+ = 30$ particles at $M=1$, we observe that the long ropes align well with the low-speed streaks, showing that the dynamics of these two coherent structures are interlinked. Compared to the unladen flow (see figure \ref{fig:streak_SP_a}), the low-speed streaks are visibly further elongated in a way similar to how the particle ropes extend in the streamwise direction. The spanwise spacing of the low-speed streaks also increases and appears comparable to the spanwise spacing of the ropes. In the case of the flow laden with $\Sto^+=6$ particles at $M=1$, the clusters are also primarily found in the low-speed streaks. However, the streamwise length of these clusters is much shorter in comparison with the low-speed streaks and with the ropes formed by $\Sto^+ = 30$ particles. In addition, the streamwise length and spanwise spacing of low-speed streaks increase compared to the particle-free flow, although not to the same extent as with $\Sto^+=30$ particles.



\begin{figure}
	\centering
	\begin{subfigure}{0.45\linewidth}
		\includegraphics[width=\linewidth]{plotting/fig17a.pdf}
		\caption{\label{fig:spacing_st30_f}}
	\end{subfigure}
	\hfill
	\begin{subfigure}{0.45\linewidth}
		\includegraphics[width=\linewidth]{plotting/fig17b.pdf}
		\caption{\label{fig:spacing_st30_p}}
	\end{subfigure}
	\begin{subfigure}{0.45\linewidth}
		\includegraphics[width=\linewidth]{plotting/fig17c.pdf}
		\caption{\label{fig:spacing_st30_p}}
	\end{subfigure}
	\hfill
	\begin{subfigure}{0.45\linewidth}
		\includegraphics[width=\linewidth]{plotting/fig17d.pdf}
		\caption{\label{fig:spacing_st6_p}}
	\end{subfigure}
	\caption{Variation with spanwise spacing of the two-point autocorrelation of the streamwise fluid fluctuations and particle volume fraction fluctuations in the spanwise direction for the (a,b) drag-reducing case $\Sto^+=30$ (\textcolor{blue}{\protect\scalebox{1.25}{$\blacksquare$}}) and (c,d) drag increasing case $\Sto^+ = 6$ (\textcolor{red}{\protect\scalebox{1.75}{$\bullet$}}). Darker symbols correspond to larger mass loading which varies from 0.2 to 1.0. The solid black line represents the particle-free channel flow.
	\label{fig:spacing}}
\end{figure}
To characterize quantitatively the spanwise spacing of particle clusters and their impact on the spanwise spacing of low-speed streaks, we compute the two-point autocorrelation of the particle volume fraction fluctuations,
\begin{equation}
	    R^p_{\phi\phi}(\Delta z;y_0)=\frac{\langle{\phi' (x,y_0,z,t)\phi' (x,y_0,z+\Delta z,t)}\rangle}{\langle{\phi'^2}\rangle},
\end{equation}
and the the two-point autocorrelation of the streamwise velocity fluctuations $R^f_{uu}$. Figure \ref{fig:spacing} shows the variation $ R^p_{\phi\phi}$ and $ R^f_{\phi\phi}$ with spanwise spacing at $y^+=10$. Similar to how the low-speed streak spacing $\lambda_f^+$ is defined, we define $\lambda_p^+$, the spanwise spacing of particle clusters, as twice the distance between the origin and $\Delta z^+$ where $R^p_{\phi\phi}$ reaches a first minimum.

\begin{table}
  \caption{Spanwise spacing of the low-speed streaks and particle ropes. \label{tab:spacing}}
  \begin{ruledtabular}
    \begin{tabular}{llll}
      Stokes number ($\Sto^+$) & Mass loading ($M$) & $\lambda^+_f$ & $\lambda^+_p$ \\\hline
      (Particle-free) & 0 & 106 & $-$ \\
      6             & 0.2          & 125      & 99     \\
                    & 0.6          & 134      & 108    \\
                    & 1.0          & 116      & 90     \\
      30            & 0.2          & 126      & 108    \\
                    & 0.6          & 161      & 130    \\
                    & 1.0          & 170	  & 135    
    \end{tabular}
  \end{ruledtabular}
\end{table}

Table \ref{tab:spacing} shows the values of $\lambda^+_f$ and $\lambda^+_p$ for all cases simulated.
For the drag-reducing cases at $\Sto^+ = 30$, it is clear that as the mass loading is increased from 0.2 to 1.0 the low-speed streak spanwise spacing increases from $\lambda^+_f = 126$ to $170$. These are significant increases compared to the low-speed streak spacing of $\lambda^+_f = 106$ in the particle-free channel. The rope spacing $\lambda_p^+$ increases from $\lambda^+_p = 108$ to $135$ as mass loading is increased. 
\textcolor{revision}{The disparity between $\lambda_p^+$ and $\lambda_f^+$ is likely due to small particle clusters that detach from the main ropes due to the spanwise meandering of ropes and low-speed streaks.}
%
%\textcolor{revision}{We have also reported the difference between the spacing of low speed and particle streaks, i.e., $\lambda_f^+-\lambda_p^+$ in table \ref{tab:spacing}. With the drag-reducing $\Sto^+ = 30$  particles, $\lambda_f^+-\lambda_p^+$ increases from 18 wall units for $M=0.2$ to 35 wall units at $M=1.0$. This suggests that the particle ropes are able to push in and out of the low-speed streaks thanks to the large particle inertia.}
%In doing so, these particles may help suppress coherent structures and relaminarize the near-wall region.
In comparison, $\Sto^+ = 6$ particles lead to substantially lower modulation of the low-speed streaks. As shown in table \ref{tab:spacing}, the spanwise spacing of the low-speed streaks varies between $\lambda^+_f = 116$ and $134$ when $\Sto^+ = 6$ particles are dispersed. The corresponding spacing of particle clusters varies in the range of $\lambda^+_p = 90-108$, \textcolor{revision} {with less disparity between $\lambda^+_p$ and $\lambda^+_f$ compared to the flow laden with $\Sto^+ = 30$ particles. This suggests that $\Sto^+ = 6$ clusters are more closely aligned with the high-strain low-vorticity regions found within the low-speed streaks, likely due to their lower inertia.}
%
%\textcolor{revision}{ but the difference $\lambda_f^+-\lambda_p^+$ remains constant at 26 wall units for all mass loadings considered. This lower disparity between low-speed streaks and particle spacing suggests that  $\Sto^+ = 6$ particles are primarily located close to the low-speed streaks and that their comparatively lower inertia prevents them from escaping further way}.

Note that the two-way coupling plays a critical role in the arrangement of low-speed streaks and particle clusters. In a prior study by \citet{jieExistenceFormationMultiscale2022}, where the authors considered one-way coupled Euler-Lagrange simulations of particle-laden channel flows at $\Rey_\tau$ between 600 and 2000, the absence of feedback force from the particles leads to low-speed streaks that have identical characteristics to those of a particle-free turbulent channel flow. The data presented by the authors further suggests that the particle cluster spanwise spacing varies little with Reynolds number and is about $\lambda_p^+ \sim 114$ for $\Sto^+ = 30$ particles. However, as we have shown in this study,  $\lambda_p^+$ and $\lambda_f^+$ reach considerably higher values when two-way coupling is significant since the dynamics of clusters and near-wall coherent structures become more inter-dependent.

\begin{figure}
  \centering
  \includegraphics[width=4in]{tikzgraphics/fig18.pdf}
  \caption{\textcolor{revision}{Instantaneous velocity vectors overlayed by contour of particle volume fraction $\phi/\phi_0 = 3$, for the case $\Sto^+ = 30, M = 1.0$, show particle ropes forming in the high strain region between quasi-streamwise vortices.}}
  \label{fig:vorticity}
\end{figure}

%The mechanism underpinning the modulation of low-speed streaks relates to how inertial particles interact with quasi-streamwise vortices surrounding these streaks.
Figure \ref{fig:vorticity} shows an example of how $\Sto^+=30$ particles are distributed in the vicinity of a pair of quasi-streamwise vortices. The particles form ropes by concentrating in the straining region between the pair of vortices, consistently with the preferential concentration mechanism. Pockets of particles can be seen ejected upward towards the centerline, which results in a downward feedback force on the fluid. This process is self-sustaining because the ejected particles eventually return to the near-wall region due to turbophoresis, where they accumulate again along particle ropes. The feedback force from these clusters contributes to the the suppression of bursting and stabilization of quasi-streamise vortices. Consequently, low-speed streaks nested in-between quasi-streamwise vortices extend further than possible in particle-free flows. Because bursting events contribute largely to the Reynolds shear stress production \citep{willmarthStructureReynoldsStress1972}, the stabilizing role of  $\Sto^+=30$ particles is likely the main reason these particles reduce the fluid-phase Reynolds shear stress to the extent shown in \S \ref{sec:stress_two_phase}, and \emph{in fine}, skin-friction drag reduction.
