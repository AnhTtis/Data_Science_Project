\section{Simulation setup and parameters}
\label{sec:setup}
In this section, we present the parameters and methods used in our Euler-Lagrange simulations of semi-dilute particle-laden turbulent channel flow. Section \ref{sec:formulation} provides an overview of the mathematical framework, while section \ref{sec:configuration} provides details about the configuration and flow parameters in this study.
\subsection{Mathematical formulation}
\label{sec:formulation}

The particle phase is treated in a Lagrangian frame, where each individual particle is tracked. For a particle ``$i$'', the dynamics are given by \citep{maxeyEquationMotionSmall1983}
\begin{eqnarray}
    \frac{d \bm{x}_p^i}{dt}&=&\bm{u}_p^i, \label{eq:MR_1}\\
    \frac{d \bm{u}_p^i}{dt}&=&f_d\frac{(\bm{u}_f(\bm{x}_p^i,t)-\bm{u}_p^i)}{\tau_p}+\frac{1}{\rho_p}\nabla\cdot\bm{\tau}(\bm{x}_p^i,t)+\bm{F}^i_{p\rightarrow p}+\bm{F}^i_{w \rightarrow p},\label{eq:MR_2}
\end{eqnarray}
where $\rho_p$, $d_p$, $\tau_p=\rho_p d_p^2/(18\mu)$, $\bm{x}_p^i$, $\bm{u}_p^i$, $\bm{F}^i_{w \rightarrow p}$, and $\bm{F}^i_{p\rightarrow p}$ are the particle density, diameter, response time, position, velocity, particle-wall collisional acceleration, and particle-particle collisional acceleration, respectively. \textcolor{revision}{The fluid stress tensor $\bm{\tau}$ is given by
\begin{equation}
\bm{\tau} = -p\bm{I}+\mu[\nabla\bm{u}_f + \nabla\bm{u}^T_f - \frac{2}{3}(\nabla\cdot\bm{u}_f)\bm{I}],\label{eq:tau_f}
\end{equation}
where the hydrodynamic $\bm{u}_f$ is the fluid velocity, $p$ is pressure, and $\mu$ is the dynamic viscosity. $\bm{I}$ is the identity tensor.} The first term on the right-hand side of (\ref{eq:MR_2}) accounts for the drag force exerted by the fluid on the particle. In order to accurately capture this force for particles with finite Reynolds number $\Rey_p=|\bm{u}_f-\bm{u}_p|d_p/\nu$ and particles that may be located within clusters, we use a nonlinear drag correction factor $f_d$ modeled with  the correlation by \citet{tennetiDragLawMonodisperse2011a}. The latter accounts for particle Reynolds number $\Rey_p$ and local particle volume fraction $\phi$. The second term on the right-hand side of (\ref{eq:MR_2}) represents the acceleration due to resolved fluid stresses on the particle phase. Although this term is included for completeness, its effect is negligible in the semi-dilute regime due to the high density ratio $\rho_p/\rho_f=O(10^3)$. For the same reason, other hydrodynamic forces are ignored. Note that particle-particle collisions, while typically negligible in dilute flows with average particle volume fraction $\phi_0<10^{-3}$, are included due to the tendency of inertial particles to accumulate into clusters with local volume fraction as high as $\phi\sim 10^{-2}$. \textcolor{revision}{The particle-particle and wall-particle collisions are performed using the soft-sphere collision model described in \citep{capecelatroEulerLagrangeStrategy2013}, and originally proposed \citet{cundallDiscreteNumericalModel1979}, with a restitution coefficient $e=0.9$. The unperturbed fluid velocity at the particle location is computed using the method of \citet{irelandImprovingParticleDrag2017}.} Further, in order to isolate inertial effects, gravity is ignored.
The carrier turbulent flow is described using volume-filtered incompressible Navier-Stokes equations \citep{andersonFluidMechanicalDescription1967,capecelatroEulerLagrangeStrategy2013},
\begin{eqnarray}
\frac{\partial }{\partial t}\left(1-\phi\right) + \nabla\cdot\left((1-\phi)\bm{u}_f\right)&=&0, \label{eq:NVS_1}\\
\rho_f\left( \frac{\partial }{\partial t}\left(\left(1-\phi\right) \bm{u}_f\right)+ \nabla \cdot\left(\left(1-\phi\right)\bm{u}_f\bm{u}_f\right)\right)&=&-\nabla p + \mu\nabla^2\bm{u}_f+(1-\phi)\bm{A}+\bm{F}_p+\nabla\cdot\bm{R}_\mu \label{eq:NVS_2},
\end{eqnarray}
where $\bm{u}_f$ is the fluid velocity, $p$ is pressure, $\bm{F}_p$ is the momentum exchange between the two-phases, and $\bm{A}$ is a constant pressure gradient that drives the flow within the channel. \textcolor{revision}{This forcing is a function of the wall shear stress $\tau_w$ and the channel half height $h$, such that $\bm{A} = (\tau_{w}/h)\bm{e}_x$, where $\bm{e}_x$ is a unitary vector oriented in the streamwise direction.} The tensor $\bm{R}_\mu$ arises from filtering the fluid stress tensor \citep{capecelatroEulerLagrangeStrategy2013}, and is closed using the effective viscosity model $\bm{R}_\mu=\mu_f((1-\phi)^{-2.8}-1)[\bm{\nabla}\bm{u}_f+\bm{\nabla}\bm{u}_f^T-\frac{2}{3}(\bm{\nabla}\cdot\bm{u}_f)I]$ of \citet{gibilaroApparentViscosityFluidized2007}. This term, included here for completeness, is negligible in the semi-dilute regime considered. Likewise, the presence of the volume fraction $\phi$ in equations (\ref{eq:NVS_1}) and (\ref{eq:NVS_2}) accounts for volume excluded by the particle phase \citep{andersonFluidMechanicalDescription1967}.  \textcolor{revision}{This effect is typically neglected in the semi-dilute regime \citep{paksereshtVolumetricDisplacementEffects2019}. However, we retain the volume fraction $\phi$ in our equations since turbophoresis and preferential concentration may lead to local accumulation of the particles resulting in volume fractions one or two orders of magnitude higher than the average \citep{sardinaWallAccumulationSpatial2012,nilsenVoronoiAnalysisPreferential2013,yuanThreedimensionalVoronoiAnalysis2018}.}
Consistently with equation (\ref{eq:MR_2}), the particles exert a feedback force on the fluid given by
\begin{equation}
    \bm{F}_p=-\phi \rho_p f_d \frac{\bm{u}_{f}\vert_p-\bm{u}_p}{\tau_p}-\phi \nabla\cdot\bm{\tau}\vert_p,\label{eq:Fp}
\end{equation}
where $\bm{u}_p(\bm{x},t)$ is the Eulerian particle velocity at the location $\bm{x}$, $\bm{u}_{f}\vert_p$ is the fluid velocity at the particle location, and $\bm{\tau}\vert_p$ is the total fluid stresses at the particle location. These Eulerian fields are computed from the Lagrangian quantities in (\ref{eq:MR_1}) and (\ref{eq:MR_2}) using a filtering procedure that reads
\begin{eqnarray}
    \phi(\bm{x},t)&=&\sum_{i}^N V_{p}q\left(\left|\left|\bm{x}-\bm{x}_p^i\right|\right|\right),\\
    \phi\bm{u}_{p}(\bm{x},t)(\bm{x},t)&=&\sum_{i}^N V_{p}\bm{u}_p^iq\left(\left|\left|\bm{x}-\bm{x}_p^i\right|\right|\right),\\
    \phi\bm{u}_{f}\vert_p(\bm{x},t)&=&\sum_{i}^N V_p\bm{u}_f (\bm{x}_p^i (t),t) q\left(\left|\left|\bm{x}-\bm{x}_p^i\right|\right|\right),
\end{eqnarray}
where $V_p=\pi d_p^3/6$ is the particle volume and $q$ is a Gaussian filter kernel. As discussed above, drag force dominates the momentum exchange in the semi-dilute regime. From a scaling analysis of equation (\ref{eq:Fp}), one can see that the particle feedback force scales with mass loading $M=\phi_0\rho_p/\rho_f$. Consequently, the feedback force from the particle phase onto the fluid phase is negligible if $M\ll 1$. In this case, the flow dynamics are independent from those of the particle phase, essentially behaving as in particle-free conditions. However, as $M$ approaches unity, the coupling between the two-phases strengthens resulting in flow and particle dynamics that are mutually interlinked. The dynamics resulting from the joint evolution of the particle and fluid phases at $M=O(1)$ are the subject of this study.


\subsection{Channel flow configuration}
\label{sec:configuration}
\begin{table}
\caption{Summary of the non-dimensional parameters.\label{tab:parameters}}
\begin{ruledtabular}
\begin{tabular}{l l l l l l l l l}
 Runs & $\Rey\tau$ & $\Sto^+$ & $M$ & $\phi_0$        & $d_p^+$ & $\rho_p/\rho_f$ & $h/d_p$ & $N_p$\\ \hline
 A    & 180        & 6        & 0.2 & $2.4\times 10^{-4}$ & 0.36    & 833             & $500$   & $6.03 \times 10^6$ \\
 B    & 180        & 6        & 0.6 & $7.2\times 10^{-4}$ & 0.36    & 833             & $500$   & $18.1 \times 10^6$ \\
 C    & 180        & 6        & 1.0 & $1.2\times 10^{-3}$ & 0.36    & 833             & $500$   & $30.1 \times 10^6$ \\
 D    & 180        & 30       & 0.2 & $2.4\times 10^{-4}$ & 0.80    & 833             & $225$   & $4.93 \times 10^6$ \\
 E    & 180        & 30       & 0.6 & $7.2\times 10^{-4}$ & 0.80    & 833             & $225$   & $14.8 \times 10^6$ \\
 F    & 180        & 30       & 1.0 & $1.2\times 10^{-3}$ & 0.80    & 833             & $225$   & $24.7 \times 10^6$ \\
\end{tabular}
\end{ruledtabular}
\end{table}

\begin{figure}
  \begin{subfigure}{0.4\textwidth}
    \centering
    \includegraphics[width=1\linewidth]{tikzgraphics/fig1a.pdf}
    \caption{}
    \label{fig:configuration_s}
  \end{subfigure}
  \begin{subfigure}{0.5\textwidth}
    \centering
    \includegraphics[width=1\linewidth]{tikzgraphics/fig1b.pdf}
    \caption{}
    \label{fig:configuration_l}
    \end{subfigure}
    \caption{Schematic of the computational domains used in simulations of channels laden with (a) $\Sto^+=6$ and (b) $\Sto^+=30$ particles.}
    \label{fig:configuration}
\end{figure}

We consider 6 mono-disperse particle-laden turbulent channel flows at varying particle-phase properties. Table \ref{tab:parameters} provides a summary of the flow and simulation parameters. In all these simulations, the friction Reynolds number is fixed at $\Rey_\tau=\rho_f u_\tau h/\mu=180$, where $u_\tau=\sqrt{\tau_w/\rho_f}$ is the friction velocity and $\tau_w$ is the wall shear stress.  Two particle diameters are considered yielding non-dimensional diameters $d_p^+=0.36$ and $0.80$. The superscript $+$ denotes a quantity that has been non-dimensionalized using inner wall scaling.  These particles are sufficiently small to make any finite-size effects negligible. The friction Stokes number, which measures particle inertia, is $\Sto^+=\tau_pu_\tau^2/\nu=6$ and $\Sto^+=30$ for the particles with $d_p^+=0.36$ and $d_p^+=0.80$, respectively. In both cases, particle inertia is significant such that one may expect these particles to form clusters and accumulate near walls due to turbophoresis \citep{sardinaWallAccumulationSpatial2012}. For each of the two Stokes numbers considered, we vary the average particle volume fraction to yield $\phi_0 = 2.4\times 10^{-4}, 7.2\times 10^{-4}$, and $1.2\times 10^{-3}$. With the particle-fluid density ratio fixed at $\rho_p/\rho_f=833$ for all 6 cases, the mass loading $M=\rho_p/\rho_f\phi_0$ is 0.2, 0.6, or 1.0. These parameters correspond to the semi-dilute regime, where the particle phase is dilute, yet the two-way coupling between the particle and fluid phases is strong due to the large mass loading. Thus, particle feedback on the fluid can be expected to lead to significant modulation of the flow, especially for cases with $M=0.6$ and $1.0$.

Figure \ref{fig:configuration} shows a schematic of the domains used in this study. Channel flows laden with  $\Sto^+=6$ particles are simulated in a domain of size $4\pi h$ in the streamwise direction $x$, $2h$ in the wall-normal direction $y$, and $(4/3)\pi h$ in the spanwise direction $z$ as shown in figure \ref{fig:configuration_s}. This domain size is comparable to those used in previous studies \citep{zhaoTurbulenceModulationDrag2010,zhouNonmonotonicEffectMass2020}. Channel flows laden with  $\Sto^+=30$ particles are simulated in a domain 9 times larger with dimensions $12\pi h \times 2h \times 4\pi h$ as shown in figure \ref{fig:configuration_l}. \textcolor{revision}{While computations in such larger domain are significantly more expensive, we have found it necessary to use this larger domain to capture the increased spanwise spacing of the particle and flow structures with $\Sto^+=30$ particles. This aspect is discussed in \S\ref{sec:mechanism}.}

The equations of motion are solved using the flow solver NGA \citep{desjardinsHighOrderConservative2008}, with the Euler-Lagrange strategy of \citet{capecelatroEulerLagrangeStrategy2013}. The fluid mass and momentum equations (\ref{eq:NVS_1}) and (\ref{eq:NVS_2}) are solved on a staggered grid of size $226\times128\times 168$ for the small channel and \textcolor{revision}{$678\times128\times 504$} for the larger one. The grid is stretched in the wall-normal direction using a hyperbolic tangent function such that the minimum mesh spacing in the wall-normal direction is $\Delta y^+_\mathrm{min}=0.5$. In the streamwise and spanwise directions, the mesh spacing is constant at $\Delta x^+=10$ and $\Delta z^+=5$, respectively. In both small and large domains, the values $\Delta x^+$, $\Delta y^+_\mathrm{min}$, and $\Delta z^+$ are identical. The discretization relies on second order finite-volume schemes that preserve mass, momentum and kinetic energy \citep{morinishiFullyConservativeFinite2004,morinishiFullyConservativeHigher1998,desjardinsHighOrderConservative2008}. The fluid equations are advanced in time with a fractional step approach and a Crank-Nicolson scheme introduced by \citet{pierceProgressvariableApproachLargeeddy2004}. Equations (\ref{eq:MR_1}) and (\ref{eq:MR_2}), describing the position and velocity of Lagrangian particle are advanced using a second-order Runge-Kutta scheme. \textcolor{revision}{Soft-sphere particle-particle and particle-wall collisions are handled with the method in \citep{capecelatroEulerLagrangeStrategy2013}.} Depending on the case, a total of \textcolor{revision}{$N=4.9\times 10^6$ to $30.1\times 10^6$} particles are tracked in the simulation domain. Eulerian particle data such as the volume fraction field are computed from Lagrangian data using a Gaussian kernel of width $7d_p$. The method is fully conservative, yields grid-independent solutions in two-way coupled problems, and has been extensively verified against experiments \citep{capecelatroMassLoadingEffects2015,capecelatroNumericalCharacterizationModeling2014,capecelatroEulerianLagrangianModeling2013,wangInertialParticleVelocity2019}, and theoretical calculations \citep{kasbaouiRapidDistortionTwoway2019,kasbaouiTurbulenceModulationSettling2019,shuaiAcceleratedDecayLamb2022,shuaiInstabilityDustyVortex2022}.

The Euler-Lagrange simulations are initialized from auxiliary simulation of unladen channel flow at $\Rey_\tau=180$. Once the single-phase flow reaches a stationary state, the Lagrangian particles are inserted randomly into the channel with velocities matching the fluid velocity interpolated at their locations. To reach a new stationary state, the two-phase flow simulations are carried out for 120 eddy turnover times ($h/u_\tau$). After which, the simulations are run for additional 10 eddy turnover times to collect statistics. \textcolor{revision}{In total, running these simulations required $2.58$M CPU hours (cpu.h) on Intel Xeon Gold 6252  nodes, with each $\Sto^+=6$ simulation requiring 345,600 cpu.h and each $\Sto^+=30$ simulation requiring 518,400 cpu.h.}
