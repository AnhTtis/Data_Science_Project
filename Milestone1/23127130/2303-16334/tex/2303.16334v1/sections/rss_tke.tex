\subsection{Contribution to Reynolds shear stress and fluid turbulent kinetic energy}

The contribution to the total shear stress by the Reynolds shear stress $-\rho_f \langle uv \rangle $, and the viscous shear stress $\mu du_{f}/dy$ within wall bounded flows has been of vital importance to understanding the modulation of the mass flow rate within a single phase flows. Through Reynolds averaging, previous studies have arrived on a single equation that describes the turbulent shear stress profile as shown below in eqn. \ref{eq:RSS_1}. Here the total shear stress $\tau$ is shown as the viscous shear stress $\mu\partial u_f/\partial y$ minus the Reynolds shear stress $\rho\langle u_{f}^{'}v_{f}^{'}\rangle$. $\tau$ here as shown by eqn. \ref{eq:RSS_1} has a constant slope as the distance away from the wall.

\begin{equation}
    -\frac{\partial \langle P \rangle}{\partial x} = \frac{\partial}{\partial y}\left(\mu\frac{\partial u_f}{\partial y} - \rho\langle u_{f}'v_{f}'\rangle\right)
    \label{eq:RSS_1}
\end{equation}

The total shear stress, $\tau$ and the contributions to it (viscous and Reynolds stress) for the single phase channel flow is shown in fig. \ref{fig:tss_sp}. It is clear how $\tau$ has a constant slope from the wall to the center of the flow. The viscous shear stress dominates the near wall region, while as we move away from the wall it tends to exponentially decrease in magnitude. Comparatively, the Reynolds shear stress shows the maximum magnitude in the log layer of the flow, and has a steady reduction as we approach the center of the channel.

\begin{figure}[htp]
    \centering
    \includegraphics[width=\linewidth]{plotting/TSS-SP.pdf}
    \caption{Contribution of the Reynolds shear stress and viscous shear stress to the total shear stress within a single phase channel flow.}
    \label{fig:tss_sp}
\end{figure}

However for the current multiphase (particle-laden) cases, The total shear stress is a contribution of the viscous and Reynolds shear stress, and an additional third term, the particle shear stress. This is due to the extra particle momentum exchange term we see added to the conservation of momentum equation in eqn. \ref{eq:NVS_2}. Performing a similar Reynolds averaging on the particle term $\langle F_p \rangle $ yields a similar equation as eqn. \ref{eq:RSS_1} with an extra particle shear stress term as shown in eqn. \ref{eq:RSS_2}. Keep in mind for constant pressure gradient flows (the force driving the flow for all simulations conducted in this study), the integral of the left hand side of the equation $\partial\langle P\rangle/\partial x$, always remains the same, since the pressure gradient forcing is a constant value. Therefore it is of importance to understand how the viscous, Reynolds and particle shear stress curves are modulated within the flow as Stokes number or the mass loading is varied.

\begin{equation}
    -\frac{\partial \langle P \rangle}{\partial x} = \frac{\partial}{\partial y}\left(\mu\frac{\partial u_f}{\partial y} - \rho\langle u_{f}'v_f'\rangle - m_{p}\langle n_{d} \rangle \langle u_{p}'v_{p}' \rangle \right)
    \label{eq:RSS_2}
\end{equation}

From eqn. \ref{eq:RSS_2}, it is observed that the fluid Reynolds shear stress and the particle shear stress are both subtracted from the viscous shear stress. The fluid Reynolds shear stress and particle shear stress therefore, perform a similar role in affecting the total shear stress within a flow. Previous single phase studies have shown that the Reynolds shear stress is a good indicator of the skin-friction drag near the wall. Reducing the magnitude of the Reynolds shear stress can strongly aid in the reduction of the drag experienced by the fluid and increase the mass flow rate. Therefore, we can hypothesize that if the total contribution of $\rho\langle u_{f}^{'}v_{f}^{'}\rangle$ and $m_{p}\langle n_{d} \rangle \langle u_{p}^{'}v_{p}^{'} \rangle$ if lower than the single phase $\rho\langle u_{f}^{'}v_{f}^{'}\rangle$, we can achieve increase in mass flow rate.  

\begin{figure}[htp]
  \begin{subfigure}{0.45\textwidth}
    \centering
    \includegraphics[width=1\linewidth]{plotting/TSS_COMPARE-S6M06.pdf}
    \caption{$\Sto^+ = 6, M = 0.6$}
    \label{fig:tsscomparest6}
  \end{subfigure}
  \begin{subfigure}{0.45\textwidth}
    \centering
    \includegraphics[width=1\linewidth]{plotting/TSS_COMPARE-S30M06.pdf}
    \caption{$\Sto^+ = 30, M = 0.6$}
    \label{fig:tsscomparest30}
  \end{subfigure}
  \caption{This figure shows the shear stress profile for (a) case \textbf{C} (maximum drag enhancement) and (b) case \textbf{F} (maximum drag reduction), compared to the single phase flow (dashed lines) where (\sampleline{dash pattern=on .7em off .2em on .2em off .2em}) is the total shear stress, (\sampleline{dotted}) is the Reynolds shear stress and (\sampleline{dashed}) is the viscous shear stress} 
  \label{fig:tss_comparison}
\end{figure}

Figure \ref{fig:tss_comparison} shows the stress profile and how the different shear stress terms vary in comparison to the single phase flow. As shown in eqn. \ref{eq:RSS_2} The total shear stress profile will always remain the same. Equation \ref{eq:RSS_2} shows that the particle shear stress and the Reynolds shear stress both produce a similar effect as both are subtracted from the viscous shear stress. Therefore, if the viscous stress is increased, the Reynolds and particle stresses should reduce to satisfy eqn. \ref{eq:RSS_2}. Moreover, we know through previous studies (** insert reference) that as the Reynolds shear stress is reduced there is an improvement in the mass flow rate of the fluid. The near coherent structures are responsible for nearly half of the skin friction drag (Guala, JFM, 2006). Modulating these coherent structures can help reduce the skin-friction drag induced within the flow. Figure \ref{fig:tsscomparest30} shows the stress profile for case \textbf{F} and it is observed that the near wall viscous shear stress profile is higher than that of the single phase. This shows that the higher inertial particles help reduce the Reynolds shear stress, and they themselves induce a lower magnitude of stress. This shows how higher inertial particles can help reduce the induced drag and enhance the mass flow rate by modulating the coherent structures near the wall. They can help reduce the Reynolds shear stress significantly while also keeping the particle shear stress low near the wall such that the sum of the Reynolds and particle shear stress near the wall is lower than the Reynolds shear stress of the single phase flow near the wall. Contrastingly, fig. \ref{eq:tsscomparest6} shows the stress profile for lower inertial particles i.e. case \textbf{C}. The near wall viscous stress profile is almost similar to the single phase case. This means that the sum of the Reynolds and particle shear stress are also equal to the single phase Reynolds shear stress near the wall. Case \textbf{C} shows maximum drag enhancement, this is mostly due to the log-outer layer increase in the particle shear stress, due to a large amount of particles being easily ejected into the middle of the channel slowing down the fluid as a result of this.        

To further understand the modulation of the coherent structures near the wall, the Q-criterion is explored to understand how particles modulate the vortical structures near the wall with changing Stokes number while keeping the mass loading the same. The vortical structures are diminished with time as particles are injected into the system. Inertial particles of $\Sto^+$ of O(10) diminish the vortical structures and modulate the fluid significantly more than particles of $\Sto^+$ of order O(1) as we can see in \ref{fig:qcrit}. Further demonstrating how particle-laden flows do not follow the same rules as unladen flows. Diminishing the vortical structures is not the only essential criteria to decreasing the drag of the system. From the results in fig. \ref{fig:fvelst30} and \ref{fig:tsscomparest30} for higher inertia cases and fig. \ref{fig:fvelst6} and \ref{fig:tsscomparest6} for lower inertia cases, it is observed that the mass flow rate of the fluid is not solely dependent on the diminishing of the coherent structures, but also keeping the particle induced shear stress low enough which higher inertial particles perform better at than lower inertial particles. 

\begin{figure}[htp]
    \centering
    \includegraphics[width=\linewidth]{tikzgraphics/q-crit.pdf}
    \caption{The Q-criterion represents the vortical structures near the wall and is a good depiction of thje Reynolds shear stress modulation near the wall. The figure shows that the vortical structures are diminished for all particle-laden cases with higher inertia particles performing better at diminishing these structres than lower inertia particles.}
    \label{fig:qcrit}
\end{figure}

\par
Another phenomenon that can be explored is preferential concentration and how particles transfer the turbulent energy within the flow. Quadrant analysis \citep{wallaceWallRegionTurbulent1972}\citep{adrianStochasticEstimationOrganized1988} was therefore performed on all cases further understand this. The velocity fluctuations within the flow are divided in 4 quadrants based on their signs, where Q1($+u$,$+v$) and Q3($-u$,$-v$) are the outward and inward motions within the flow which dont reveal much about the vortical structures within the flow. Additionally Q2($-u$,$+v$) and Q4($+u$,$-v$) are the ejection and sweep events which reveal the vortical motions within the flow and are the greatest contributors of Reynolds shear stress within the flow. We can observe from fig. \ref{fig:quadrant} for all $\Sto^+ = 6$ and $\Sto^+ = 30$ cases that as the mass loading is increased and more particles are injected within the system we see a monotonic behaviour that depicts a decrease in the Reynolds shear stress contribution by the Q2 and Q4 events which are the sweep and ejection events and also agree with the previous results shown in fig. \ref{fig:tss_comparison}. Furthermore, looking at the sweep and ejection quadrants further confirms what is observed in fig. \ref{fig:tss_comparison}. The log scale graph shows that for the highest mass loading (maximum modulation), the near wall Reynolds stress reduction is more significant for $\Sto^+ = 30$ than for $\Sto^+ = 6$.    

\begin{figure}%[htp]
  \begin{subfigure}{0.45\linewidth}
    \centering
    \includegraphics[width=1\linewidth]{plotting/Q1.pdf}
    \caption{Quadrant 1}
    \label{fig:q1}
  \end{subfigure}
  \begin{subfigure}{0.45\textwidth}
    \centering
    \includegraphics[width=1\linewidth]{plotting/Q2.pdf}
    \caption{Quadrant 2}
    \label{fig:q2}
  \end{subfigure}
  \begin{subfigure}{0.45\linewidth}
    \centering
    \includegraphics[width=1\linewidth]{plotting/Q3.pdf}
    \caption{Quadrant 3}
    \label{fig:q3}
  \end{subfigure}
  \begin{subfigure}{0.45\linewidth}
    \centering
    \includegraphics[width=1\linewidth]{plotting/Q4.pdf}
    \caption{Quadrant 4}
    \label{fig:q4}
  \end{subfigure}
  \caption{Quadrant analysis for all cases normalized by ${u_\tau}^2$ as a function of $y^+$. The results agree very well to literature where we see the maximum contribution to the Reynolds stress is due to Q2 and Q4.}
  \label{fig:quadrant}
\end{figure}

~
