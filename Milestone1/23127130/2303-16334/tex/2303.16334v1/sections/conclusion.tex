\section{Discussion and conclusion}
\label{sec:conclusion}
We have shown that it is possible to induce significant drag reduction using mono-dispersed spherical particles, provided that their inertia and concentration are tuned appropriately. Using four-way coupled Euler-Lagrange simulations of semi-dilute particle-laden turbulent channel flows at $\Rey_\tau=180$ with mass loading varying between $M=0.2$ and 1.0, we have shown that the particle Stokes number is a determining factor in the type of modulation observed. Among the two types of particles we have considered having friction Stokes number $\Sto^+=6$ or 30, drag increase is observed with the former, and drag reduction is observed with the latter. Mass loading plays an amplifying role in such a way that at $M=0.2$ the drag increase or decrease observed is negligible, whereas these effects become significant at $M=1.0$, resulting in drag reduction factors of up to $\mathrm{DR}=19.74\%$ and $\mathrm{DR}=-16.92\%$ for $\Sto^+ = 30$ and $\Sto^+ = 6$, respectively. A key observation is that particle clusters and coherent structures found in the near-wall region have tightly coupled evolutions. Modifications to the latter by the particle clusters explain in part the observed changes to skin-friction drag.

For the drag-reducing cases considered, the largest drag reduction is achieved for the case $\Sto^+ = 30, M = 1.0$ in which skin friction drag drops by $\mathrm{DR} = 19.54\%$ and mass flow rate increase by $\Delta \dot{m}_f/\dot{m}_{f,0} = 11.07\%$ compared to the reference particle-free flow. A distinct feature visually observed for particles at $\Sto^+ = 30$ is the existence of concentrated clusters along the channel walls with local particle volume fraction several times larger than the mean. These clusters, that we call \emph{ropes}, are very long structures that span the entire domain in the streamwise direction, about $38h$. Further, the ropes appear to preferably align with the low-speed streaks of the flow, and to cause their structure to differ considerably from those found in particle-free flows. The observed modulation which includes a stabilization of the low-speed streaks, reduction in bursting, elongation in the streamwise direction, and increase in spanwise separation result from the collective feedback force from particles located within these concentrated ropes. Using two-point autocorrelations, we found that the low-speed streaks spanwise spacing $\lambda_f^+$ increases from the little varying value $\lambda_f^+=106$ in particle-free flows to $\lambda_f^+=170$ when the flow is laden with $\Sto^+ = 30$ particles at $M = 1.0$. In comparison, the ropes spacing in this case is $\lambda_p^+ = 135$. The disparity between $\lambda_f^+$ and $\lambda_p^+$ 
\textcolor{revision}{is likely due to small clusters that detach from the main ropes due to the spanwise meandering of ropes and low-speed streaks.}
%
%suggests that,
%\textcolor{revision}{clusters formed by the more inertial $\Sto^+=30$ particles are able to push in and out of low-speed streaks more easily than clusters formed by $\Sto^+=6$ particles, and that this may be a key mechanism conferring $\Sto^+=30$ particles their drag reducing character.}
%
%} these particles, owing to their large inertia, are  able to escape low-speed streaks and traverse regions of high-vorticity and high-speed where they exert larger drag on the fluid. 
While dispersed particles cause additional stresses on the fluid, the modulation of near-wall coherent structures by $\Sto^+ = 30$ particles leads to greater reduction in Reynolds shear stress, which ultimately causes a partial relaminarization of the near-wall flow and skin-friction drag reduction.

In contrast to the larger inertia particles, dispersing $\Sto^+ = 6$ particles in the flow causes drag increase. The largest effect is observed at $M = 1.0$ which yields a drag increase of $16.92\%$ and mass flow rate decrease of $6.10\%$. These lower inertia particles do not show the same type of clustering seen with $\Sto^+ = 30$ particles. Particle cluster sizes are smaller and no rope-like clusters spanning the entire length in the streamwise direction are observed. Furthermore, the low-speed streaks also do not seem to widen or elongate at the rate that was observed for the $\Sto^+ = 30$ case. While the low-speed streak spacing increases compared to the particle-free case, from $\lambda^+_f = 106$ to $116$ at the highest drag increasing case at $\Sto^+ = 6$ and $M = 1.0$, the change is significantly lower when compared to the $\Sto^+ = 30$ case. The spanwise spacing of the particle clusters is also significantly lower with $\lambda^+_p = 90-108$. For these $\Sto^+ = 6$ particles, the low-speed streaks and particle clusters are more closely aligned. This is because lower inertial particles are less likely to escape the low-speed regions where they are mostly located. While they do exert a feedback force that reduces near-wall coherent structures, the resulting drop in Reynolds shear stress is not sufficient to balance the additional stresses exerted by the particles, hence leading to drag increase.

We shall note that the mechanisms discussed in this study hold some similarities with those found in polymeric flows. Here, friction Stokes number, ratio of the particle response time and friction time scale, is analogous to the Weissenberg number,  ratio of the polymer elasticity timescale and the friction time scale. Like the Weissenberg number in polymeric flows, the Stokes number determines whether drag reduction or drag increase is achieved. Stresses induced by inertial particles are analogous to stresses resulting from polymers. In both cases, drag reduction is determined by the extent to which the fluid Reynolds shear stress is suppressed in comparison to the additional particle or polymer stresses. However, polymers modulate flow structures through contraction and elongation, whereas inertial particles act on the flow through their drag force. Further, the mechanisms related to particle clustering, rope formation, and interplay with near-wall coherent structures are unique to particle-laden flows.

Finally, the fact that $\Sto^+ = 6$ and  $\Sto^+ = 30$ particles lead to opposite drag modulation suggests that there is a critical Stokes number above which drag reduction is obtained. This threshold may depend on mass loading and density ratio. Moreover, while we have shown close to 20\% drag reduction using $\Sto^+ = 30$ particles at mass loading $M=1$, varying Stokes number may lead to higher drag reduction. Additional simulations are required to find the threshold Stokes number for drag reduction and establish a regime map of drag modulation. 

\section*{Acknowledgement}
Acknowledgement is made to the donors of the American Chemical Society Petroleum Research Fund for partial support of this research (award \#62195-DNI9) and to the US National Science Foundation (award \#2028617, CBET-FD).