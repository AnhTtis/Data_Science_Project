%% 
%% Copyright 2019-2021 Elsevier Ltd
%% 
%% This file is part of the 'CAS Bundle'.
%% --------------------------------------
%% 
%% It may be distributed under the conditions of the LaTeX Project Public
%% License, either version 1.2 of this license or (at your option) any
%% later version.  The latest version of this license is in
%%    http://www.latex-project.org/lppl.txt
%% and version 1.2 or later is part of all distributions of LaTeX
%% version 1999/12/01 or later.
%% 
%% The list of all files belonging to the 'CAS Bundle' is
%% given in the file `manifest.txt'.
%% 
%% Template article for cas-dc documentclass for 
%% double column output.

\documentclass[a4paper,fleqn]{cas-dc}

% If the frontmatter runs over more than one page
% use the longmktitle option.

%\documentclass[a4paper,fleqn,longmktitle]{cas-dc}

\usepackage[numbers]{natbib}
%\usepackage[authoryear]{natbib}
% \usepackage[authoryear,longnamesfirst]{natbib}

\usepackage{caption} 
\usepackage{graphicx}%插入图片
\usepackage{amssymb}%数学符号
\usepackage{subfigure}
\usepackage{soul}

\newcommand{\figref}[1]{Fig.~\ref{#1}}
\newcommand{\tblref}[1]{Table.~\ref{#1}}
\newcommand{\eqtref}[1]{Eq.~\ref{#1}}
% \newcommand{\dgli}[1]{\textcolor[rgb]{0,0.75,0.5}{#1}}
% \newcommand{\lzh}[1]{\textcolor[rgb]{0.75,0,0.5}{#1}}
\newcommand{\dgli}[1]{\textcolor[rgb]{0,0,0}{#1}}
\newcommand{\lzh}[1]{\textcolor[rgb]{0,0,0}{#1}}
\newcommand{\cmmnt}[1]{}

%%%Author macros
\def\tsc#1{\csdef{#1}{\textsc{\lowercase{#1}}\xspace}}
\tsc{WGM}
\tsc{QE}
%%%

% Uncomment and use as if needed
%\newtheorem{theorem}{Theorem}
%\newtheorem{lemma}[theorem]{Lemma}
%\newdefinition{rmk}{Remark}
%\newproof{pf}{Proof}
%\newproof{pot}{Proof of Theorem \ref{thm}}

% \begin{document}
\let\WriteBookmarks\relax
\def\floatpagepagefraction{1}
\def\textpagefraction{.001}

% Short title
\shorttitle{Modeling and Analysis on Efficiency Degradation of Lithium-ion Batteries }    

% Short author
\shortauthors{short author list for running head}  

% Main title of the paper
\title [mode = title]{Modeling and Analysis on Efficiency Degradation of Lithium-ion Batteries}  

% Title footnote mark
% eg: \tnotemark[1]
\tnotemark[1] 

% Title footnote 1.
% eg: \tnotetext[1]{Title footnote text}
\tnotetext[1]{tnote text} 

% First author
%
% Options: Use if required
% eg: \author[1,3]{Author Name}[type=editor,
%       style=chinese,
%       auid=000,
%       bioid=1,
%       prefix=Sir,
%       orcid=0000-0000-0000-0000,
%       facebook=<facebook id>,
%       twitter=<twitter id>,
%       linkedin=<linkedin id>,
%       gplus=<gplus id>]

\author[must]{Zihui Lin}
\fnmark[1]
% Corresponding author indication
% \cormark[1]

% Footnote of the first author
%\fnmark[<footnote mark no>]

% Email id of the first author
% \ead{3220002268@student.must.edu.mo}

% URL of the first author
%\ead[url]{<URL>}

% Credit authorship
% eg: \credit{Conceptualization of this study, Methodology, Software}
\credit{<Credit authorship details>}

% Address/affiliation
\affiliation[must]{organization={Macau University of Science and Technology},
            addressline={Avenida Wai Long, Taipa, Macau, China}, 
            city={Macau},
%          citysep={}, % Uncomment if no comma needed between city and postcode
            postcode={0000}, 
            state={Macao},
            country={China}}

\author[must]{Dagang Li} \corref{cor1}

% Footnote of the second author
% \fnmark[2]

% Email id of the second author
\ead{dgli@must.edu.cn}

% URL of the second author
%\ead[url]{}

% Credit authorship
\credit{}



% Corresponding author text
\cortext[cor1]{Corresponding author}

% Footnote text
%\fntext[1]{}

% For a title note without a number/mark
%\nonumnote{}

% Here goes the abstract
\begin{abstract}
Efficiency of Battery Energy Storage Systems (BESSs) is increasingly critical as renewable energy generation becomes more prevalent on the grid. Therefore, it is necessary to study the energy efficiency of Lithium-ion Batteries (LIBs), which are typically used in BESSs. The purpose of this study is to propose the State of Efficiency (SOE) as a measure of how efficiently LIBs transfer energy, and what factors affect the SOE of a battery throughout its lifetime. Using NASA's data set, we calculate the SOE of NCA LIBs by calculating the ratio of energy generated and consumed during discharge and charge phases. 
%We propose a linear trend hypothesis for the SOE trajectories, and confirm the hypothesis by applying the Mann-Kendall (MK) trend test. 
A linear trend was observed in the SOE trajectories, which is confirmed by the Mann-Kendall (MK) trend test.
Following that, a linear SOE degradation model was presented. %and fitted to the SOE trends of LIBs under various scenarios. 
%Based on the regression results, 
Ambient temperature, discharge current, and cutoff voltage all affect SOE in different ways. Using the SOE and %the degradation model proposed 
its behavior observed in this study, Battery Management Systems (BMS) can improve the energy efficiency of LIBs by adjusting operating conditions or developing better management strategies.
\end{abstract}
\begin{document}
% Use if graphical abstract is present
%\begin{graphicalabstract}
%\includegraphics{}
%\end{graphicalabstract}

% Research highlights
\begin{highlights}
\item SOE is proposed as a state indicator of LIB in terms of energy efficiency.
\item The SOE aging trajectory for NCA LIBs exhibits a linear trend based on publicly available data set.
\item A careful discussion of the phenomena of variation in ambient temperature, discharge current, and cutoff voltage with variation in SOE trend is presented.
\item According to regression analysis, NCA LIBs' energy efficiency performance can be adjusted by altering their operation conditions without memory effects.
\end{highlights}

\maketitle

% Keywords
% Each keyword is seperated by \sep
\begin{keywords}
Lithium-ion battery \sep Energy efficiency \sep SOE
\end{keywords}



% Main text
\section{Introduction}
\label{sec:sec1}

%Since renewable energy sources are intermittent, they are often combined with BESS to provide power on demand \cite{olabi2022renewable}.
Unlike traditional power plants, renewable energy from solar panels or wind turbines needs \lzh{storage solutions} %\dgli{temporal storage systems?}
, such as BESSs to become reliable energy sources and provide power on demand \cite{olabi2022renewable}.
The LIB, which is used as a promising component of BESS \cite{choi2021li} that are intended to store and release energy, has a high energy density and a long energy cycle life \cite{li2022battery}. %Therefore, 
\dgli{The performance of} LIBs has a direct impact on 
%the efficiency of 
both the BESS and renewable energy sources since a reliable and efficient power system must always match power generation and load \cite{koohi2020review}.

However, LIB's performance can be affected by a variety of operating conditions \cite{hausbrand2015fundamental}, and its performance continuously degrades during usage. The degradation can be classified into two parts: when the equipment is at rest (static) and when it is in use (operational) \cite{wohlfahrt2004aging}. In a static condition, the factors that determine degradation are temperature, State of Charge (SOC), and rest time, whereas during dynamic operation, the factors that determine degradation are rate (power), depth of discharge, and heat \cite{goodenough2010challenges, birkl2017degradation}. 

%In terms of 
Regarding LIB as an energy storage device, most studies are currently focused on its capacity management, charging rate, and cycle times \cite{tarascon2001issues}. A BMS of a BESS typically manages the LIBs' State of Health (SOH) and Remaining Useful Life (RUL) in terms of capacity (measured in ampere hour) \cite{tarascon2001issues}. As part of the management of BESS, it has been possible to estimate and predict SOH and RUL. SOH is a key indicator for the estimation of battery life by focusing primarily on the maximum capacity that the battery is currently capable to supply. SOC indicates the capacity that the battery currently can provide at the present time. Estimating the SOC can provide insight into the battery's current capacity, while %predicting 
the SOH trajectory can help predict the battery's life regarding its capacity. %However, in spite of 
Despite the fact that the battery's capacity is one of the most critical performance indicators, little attention has been paid to the battery's %energy 
efficiency as it ages.

%BESS must have efficient batteries in order to increase its productivity while integrating energy production and consumption. 
The efficiency of LIBs greatly affects the efficiency of BESSs, which should minimize energy loss during operations.
This %is especially true 
becomes increasingly important
when more renewable energy sources are connected to the grid and handled by BESSs \cite{choi2021li}. %With the advancements in technology and industry, renewable energy will become a viable and sufficient source of energy in the future. However, the battery, which is used as an energy storage component for renewable energy sources, may not be able to provide the necessary amount of power if its efficiency falls below a critical level. To maintain a stable energy supply and to avoid systemic risks to the grid, it is imperative to optimize battery efficiency.
As the core energy storage component, if the batteries can only return an unexpected small fraction of the energy that is used to charge them, that will become a big problem for both the BESSs and the \dgli{power grid as a whole}.

Coulombic Efficiency (CE) \cite{hobold2021moving} was used as an indicator of LIB efficiency in the reversibility of electrical current \cite{xiao2020understanding}, which actually has a direct relationship with the battery's capacity \cite{yang2018study}. It should be noted, however, that capacity and energy are not equivalent. Since the energy levels of lithium-ions are different during the redox reaction, regeneration requires more electromotive force than discharge due to the different voltage levels  \cite{eftekhari2017energy}. Therefore, even if LIB has a high CE, it may not be energy efficient.

Battery's energy efficiency is critical to the sustainability of the planet due to its broad application \cite{pilkington2011relative}. Researchers have investigated BESS applications to maintain an acceptable level of efficiency \cite{qian2010high}. A study has also been conducted on the energy efficiency of electric vehicles \cite{wu2015electric}. The optimal battery SOC range for battery vehicles has been estimated by analyzing the relationship between battery energy efficiency and battery SOC range \cite{redondo2017impact}. There has also been discussion of the energy efficiency of NiMH batteries as they are applied to grid frequency control at various discharge currents \cite{zurfi2016experimental}. To the best of the author's knowledge, %there is no discussion regarding the trend of energy efficiency trend LIBs over their entire life cycle.
\dgli{there is still no study yet focusing on what factors will, and how they affect the energy efficiency of LIBs in the long run across their entire life cycle.}

The following are the contributions of this study:
\begin{itemize}
\item SOE is proposed as a state indicator of LIB in terms of energy efficiency.
\item The aging trajectory of SOE for NCA LIBs is studied and a linear model is proposed to describe SOE degradations.
%for NCA LIBs exhibits a linear trend based on the publicly available data set.
\item %A careful discussion of the phenomena of variation in ambient temperature, discharge current, and cutoff voltage with variation in SOE trend is presented.
A number of factors \cmmnt{on}\lzh{that affect} SOE have been identified and studied, including ambient temperature, discharge current, and cutoff voltage.
\item %According to regression analysis, NCA LIBs’ energy efficiency performance can be adjusted by altering their operation conditions without memory effects.
%Unlike SOH, we found SOE has very little memory effect when battery ages and condition changes, which can help for efficient BESS design.
\lzh{We found NCA LIBs have very weak memory effect on SOE under changing operating conditions, which is helpful when designing an efficient BESS.}
\end{itemize}

The remainder of this paper is organized as follows. In Section 2, the SOE is presented along with its relationship to CE and SOH, as well as the adopted data set. In Section 3, the SOE trajectories of batteries is analyzed in a variety of application scenarios. We verify the linear relationship between SOE and cycle number by using time series analysis, and present the SOE trend model and its regression results. Finally, in Section 4, we discuss the phenomena caused by changes in ambient temperature, discharge current, and cut-off voltage separately.

% In this study, we propose the SOE as a novel indicator of battery performance, assess the energy efficiency of LIBs in various application scenarios, and discuss how SOE is affected by a variety of operating conditions. In order to gain a deeper understanding of the energy efficiency of LIBs during their aging process, we use a well-known data set provided by NASA and different operating conditions have been taken into account. SOE is defined as a ratio between energy released and consumed per cycle in order to determine the round-trip energy efficiency of the LIB. Next, we calculate SOE using the battery terminal voltage and current provided by the data set. Then, based on the SOE trajectory, we assume that cycle number and SOE have a linear relationship, which we verify through time series analysis. This is followed by the development of the SOE trend model. Finally, in order to understand the variance in the SOE trend under different conditions of use, we discuss the phenomena caused by changes in ambient temperature, discharge current, and cut-off voltage separately.

%% see appendix~\ref{sec:sample:appendix}.

\section{Energy Efficiency of Lithium-ion Battery}
\label{sec:sec2}
%\subsection{Research data}
\subsection{The NASA \lzh{battery degradation} data set}
%\dgli{Why this data is suitable (more suitable than others) for the purpose of this study}

For this study, we use data set which is provided by the Prognostics Center of Excellence (PCoE) at NASA Ames \cite{bds2017nasa}. \lzh{In these experiments \cite{goebel2008prognostics}, 18650-size NCA LIBs continue to work until End of Life (EoL) is reached, thereby providing data over battery lifespan. This is helpful to see how energy efficiency has changed over time. }

\lzh{These experiments, as shown in \tblref{tbl:1}, are designed to reduce battery performance in a controlled way. The test scheme,  as shown in \tblref{tbl:2}, is with a test matrix that included four cut-off voltages (2.7V, 2.5V, 2.2V and 2.5 V), three discharge currents (1A, 2A, and 4A), four ambient temperatures (4°C, 24°C, 43°C), and two stopping criteria (20\% and 30\% capacity fade). In most cases, an individual battery's operating conditions are different from those of other batteries, and remain constant throughout the test. As a result, this data set captures battery degradation data under various scenarios, which is suitable for analyzing battery efficiency under a wide range of operating conditions.}
 

\begin{table}[ht]
\caption{Technical indicators of experimental Li-on batteries.}\label{tbl:1}
\begin{tabular}{ll}
\specialrule{0.05em}{3pt}{3pt}
{Manufacturer}                 & {Idaho National Laboratory} \\
\hline
{Type}                         & {18650}                     \\
\hline
{Cathode/Anode}                & {NCA/Graphite}              \\
\hline
{Rated capacity/Rated voltage} & {2 Ah/3.7 V}                \\
\hline
{charging currents}            & {1A for CC stage}           \\
\hline
{Cutoff voltages}             & {4.2 V to 2/2.2/2.5/2.7 V}  \\
\hline
{Discharging currents}         & {1/2/4A}                    \\
\hline
{Ambient temperature}          & {4/24/43°C}                  \\
\hline
{stop criteria}                & {20/30\% capacity fade}    \\
\hline
% \specialrule{0.05em}{3pt}{3pt}
\end{tabular}
\end{table}

\lzh{Each cycle, batteries are fully charged in CC-CV (Constant Current, Constant Voltage) mode and discharged in CC (Constant Current) mode to a certain depth, which is regulated by cut-off voltage. Throughout the cycle, voltage and current at the battery terminal are collected, which can be used to calculate how efficiently energy is reproduced in the specific cycle. }

\begin{table*}[htbp]
\centering
\caption{Test Scheme of Batteries.}\label{tbl:2}
\setlength{\tabcolsep}{5.7mm}{
\begin{tabular}{|c|cc|cc|c|c|}
\hline
\multicolumn{1}{|l|}{{Ambient Temperature}}                                                          & \multicolumn{2}{c|}{{4°C}}                                   & \multicolumn{2}{c|}{{24°C}}                                  & {43°C}   & {24°C and 44°C}   \\ \hline
\multicolumn{1}{|l|}{{\begin{tabular}[c]{@{}l@{}}Discharge Current \\ Cut-off Voltage\end{tabular}}} & \multicolumn{1}{c|}{{1A}}    & {2A}    & \multicolumn{1}{c|}{{2A}}    & {4A}    & {4A}    & {1A, 2A, and 4 A} \\ \hline
{2.0V}                                                                                               & \multicolumn{1}{c|}{{B0045}} & {B0053} & \multicolumn{1}{c|}{{}}      & {B0033} & {B0029} & {}              \\ \hline
{2.2V}                                                                                               & \multicolumn{1}{c|}{{B0046}} & {B0054} & \multicolumn{1}{c|}{{B0007}} & {B0034} & {B0030} & {B0038}         \\ \hline
{2.5V}                                                                                               & \multicolumn{1}{c|}{{B0047}} & {B0055} & \multicolumn{1}{c|}{{B0006}} & {}      & {B0031} & {B0039}         \\ \hline
{2.7V}                                                                                               & \multicolumn{1}{c|}{{B0048}} & {B0056} & \multicolumn{1}{c|}{{B0005}} & {}      & {B0032} & {B0040}         \\ \hline
\end{tabular}}
\end{table*}

Additionally, the data set includes a group of batteries utilising multiple load current levels (1A, 2A, and 4A) as ambient temperature steps from 24°C to 44 °C. The study will use this group of batteries to investigate the characteristics of energy efficiency. Due to the particular characteristics of this group, the series of cycles has been divided into three continuous parts so that uniform data can be collected. 

According to some studies, a fresh battery's CE may exceed 100\% because the internal electrochemical characteristics are not stable in the initial cycles \cite{xiao2020understanding}. Furthermore, the documentation of the data set indicates that there are abnormalities due to software errors or other unknown causes, such as stopping the test before reaching the specified cut-off voltage. Considering that the purpose of this study is to determine the characteristics of LIB's energy efficiency throughout its lifespan, as well as to identify the trend of SOE and the influence of relevant operating conditions on it, the first cycle as well as some special abnormal cycles for each battery have been excluded.

\subsection{State of Efficiency}

As an energy intermediary, LIBs are used to store and release electric energy. An example of this would be a battery that is used as an energy storage device for renewable energy. The battery receives electricity generated by solar or wind power production equipment. Whenever there is a demand from the grid, the stored electric energy is released. Inevitably, this process involves the dissipation of energy. As a result of polarization, the battery's energy dissipates during the charge-discharge process because coulomb losses from non-productive chemical side reactions and the battery's terminal voltage drops when current flows through  it \cite{zurfi2017electrical}. Therefore, while batteries are in operation, they lose energy during both cycle aging and calendar aging, and the amount varies depending on how they are used. \figref{fig:1} shows energy conservation in the application of battery. Energy dissipates during charging and discharging, and the energy in and out of the battery remains balanced.
\begin{figure}[ht]    
    \centering
    \includegraphics[scale=0.3]{battery_efficiency_model.png}
    \caption{Energy conservation during battery charging and discharging.}
    \label{fig:1}
\end{figure}
\begin{equation} \label{eq:1}
    E_{charge}=E_{discharge}+E_{dissipate},
\end{equation}
where $ E_{discharge} $ and $ E_{dissipate} $ represent the energy released during discharge and the energy dissipated during battery operation. The dissipated energy in operation can be expressed as

\begin{equation} \label{eq:2}
    E_{dissipate}=E_{disCycle}+E_{disCalendar},
\end{equation}
where $ E_{disCycle} $ and $ E_{disCalendar} $ represent the dissipated energy during the cycle aging and calendar aging. \lzh{Existing researches have studied SOH, which refers to the maximum remaining capacity of a battery over its rated capacity, that is, the capacity performance of a battery.} In contrast to SOH, SOE focuses on the battery's efficiency in using energy, as \lzh{discharge energy in a battery is always less than charge energy.}

The USA PNGV Battery Test Manual \cite{shhpngv} gives a \lzh{intuitive} definition of round-trip efficiency, but does not have a strict specific test protocol. PNGV round-trip efficiency is defined as
\begin{equation} \label{eq:3}
    \text{Round-trip Efficiency} = \frac{watt \cdot hours(discharge)}{watt \cdot hours(regen)} \times 100 \%.
\end{equation}

\lzh{Considering batteries are energy storage devices, we propose a calculation of battery efficiency based on the ratio of output energy to input energy.} As this study aims to evaluate the energy efficiency of a complete charging and discharging process, SOE is defined as

\begin{equation} \label{eq:4}
    SOE=\frac{E_{discharged}}{E_{charged}} , 
\end{equation}

where battery efficiency is calculated as the ratio between the amount of energy the battery can supply during discharge and the amount of energy it consumes during charging. The charged energy is the accumulation of power during charging:

\begin{equation} \label{eq:5}
    E_{charged}=\int_{t_0}^{t_n}P_{charge}(t)dt=\int_{t_0}^{t_n}V_{charge}(t)I_{charge}(t)dt ,
\end{equation}
where $ P_{charge} (t) $ is the charging power as a function of time, which is obtained by multiplying $ V_{charge} (t) $ and $ I_{charge} (t) $, that is, the product of voltage and current at the terminal of the battery. It is common practice in industrial applications to sample discretely, then \eqtref{eq:5} can be simplified as

\begin{equation} \label{eq:6}
    E_{charged}=\sum_{i=0}^{n-1}{P_{charge}\Delta t}=\sum_{t=0}^{n-1}{V_{charge}I_{charge}\Delta t_i} ,
\end{equation}

Similarly, we can also measure the energy during discharge. Hence, in this study, SOE is evaluated as

\begin{equation} \label{eq:7}
    SOE=\frac{E_{discharged}}{E_{charged}}=\frac{\sum_{i=0}^{n-1}{V_{discharge}I_{discharge}\Delta t_i}}{\sum_{i=0}^{n-1}{V_{charge}I_{charge}\Delta t_i}}.
\end{equation}

where $ V_{discharge} $, $ I_{discharge} $, $ V_{charge} $ and $ I_{charge} $ are provided in the data set and $ \Delta t_i $ is calculated by their timestamp.

\subsection{Coulombic Efficiency and energy efficiency}

CE as a battery parameter to monitor the magnitude of side reactions \cite{yang2018study} is defined as

\begin{equation} \label{eq:8}
    CE = \frac {C_{discharge}}{C_{charge}},
\end{equation}
where $ C_{discharge} $ is the discharge capacity of a battery at a single cycle, and $ C_{charge} $ is the charge capacity of the battery in the same cycle. By definition, CE is the ratio of discharge capacity over charge capacity of a LIB. Because capacity is measured by the total charge flow to/from the electrode and the total capacity of LIBs is usually cathode-limiting, CE can be expressed as the ratio between the amount of Lithium-ions or electrons returning to cathode and the amount of Lithium-ions or electrons departing from cathode in a full cycle, as indicated in \eqtref{eq:8}.

A battery's CE refers to its ability to regenerate Li+ or electrons. In an ideal battery, where there are no side reactions at the electrodes, Lithium-ions or electrons should flow solely as a result of reversible electrochemical reactions with CE equal to 100\%. Realistic batteries, however, exhibit numerous side reactions between the electrode and the electrolyte. Due to the presence of irreversible side reactions in the battery, the CE is always less than 100\%. Generally, modern LIBs have a CE of at least 99.99\% if more than 90\% capacity retention is desired after 1000 cycles \cite{xiao2020understanding}.

\begin{figure}[htbp]
	\centering  %图片全局居中
	% \subfigbottomskip=2pt %两行子图之间的行间距
	\subfigcapskip=-5pt %设置子图与子标题之间的距离
	\subfigure[100\% energy efficiency]{
		\includegraphics[width=0.48\linewidth]{energy_efficiency_a.png}}
	\subfigure[90\% energy efficiency due to overpotential]{
		\includegraphics[width=0.48\linewidth]{energy_efficiency_b.png}}
	\subfigure[50\% lower energy efficiency, resulting in energy waste]{
		\includegraphics[width=0.48\linewidth]{energy_efficiency_c.png}}
	
	\caption{Three different scenarios \cite{eftekhari2017energy} at 100\% CE.}
        \label{fig:2}
\end{figure}

However, the coulombic efficiency of a battery cannot be equated with its energy efficiency. This is due to the fact that lithium atoms participate in redox reactions at different energy levels. The overpotential that is consumed during charging cannot be recovered during discharging. Due to the internal resistance of the battery, some energy must be consumed to overcome the resistance, which limits the battery's ability to achieve 100\% energy efficiency \cite{eftekhari2017energy}.

\figref{fig:2} illustrates the charging/discharging process with both 100\% CE, corresponding to different electrical energy efficiencies. For the ideal 100\% energy efficiency in (a), the charge/discharge curves are perfectly symmetrical, meaning that the stored lithium ions have the same energy level as in both the charge and discharge phases. Nevertheless, the electrons of the charge phase have a higher energy level than the electrons of the discharge phase in (b), which will result in an energy efficiency of only 90\%. (c) represents a very low energy conversion efficiency of the battery, where 50\% of the energy is wasted in the charging/discharging process despite the fact that 100\% of capacity is available.

\subsection{SOH versus SOE}

Numerous studies have been conducted on SOH for energy storage batteries. Served as a measure of the battery's aging process, SOH is commonly calculated as

\begin{equation} \label{eq:9}
    SOH=\frac{Q_{max}}{Q_{rated}},
\end{equation}
where $ Q_{max} $ and $ Q_{rated} $ represent the battery's maximum discharge capacity with its rated capacity. RUL is defined by the number of cycles a battery can undergo before its capacity deteriorates and reaches EoL which is typically defined as when the battery has only 80\% of its rated capacity left. 

\begin{figure}[htbp]
	\centering  %图片全局居中
	% \subfigbottomskip=2pt %两行子图之间的行间距
	\subfigcapskip=-5pt %设置子图与子标题之间的距离
	\subfigure[B0005 discharged to 2.7V at 24 °C and 2A]{
		\includegraphics[width=0.9\linewidth]{B0005SOESOH.png}}
	\subfigure[B00029 discharged to 2.0V at 43 °C and 4A]{
		\includegraphics[width=0.9\linewidth]{B0029SOESOH.png}}
        \subfigure[B00033 discharged to 2.0V at 24 °C and 4A]{
		\includegraphics[width=0.9\linewidth]{B0033SOESOH.png}}
	\subfigure[B0045 discharged to 2.0V at 4 °C and 1A]{
		\includegraphics[width=0.9\linewidth]{B0045SOESOH.png}}
	
	\caption{Typical SOH and SOE trajectories over cycling.}
        \label{fig:3}
\end{figure}

\figref{fig:3} illustrates typical SOE and SOH trends as the battery ages over a number of cycles. Due to the wide range of use scenarios included in the data set, including different discharge intensities and depths in various ambient temperatures, these batteries exhibit distinctly different trajectories in terms of their SOH and SOE. As an example, when the battery B0005 is tested at an ambient temperature of 24°C, with a discharge current of 2A and a cutoff voltage of 2.7V, its SOE is essentially above 0.83 while its SOH is continuously reduced to less than 0.70. As a comparison, B0033 was also evaluated at 24°C, but is subjected to a higher intensity (4A current) and a deeper (2.0V cutoff voltage) discharge, it's SOE is around 0.7. The efficiency of B0005 was 0.86 when its capacity had decayed to about 0.8, whereas B0033's SOE was only 0.73 with similar SOH. In spite of the high discharge temperature of 43°C, the SOH of B0029 drops very rapidly, but it is almost always above 0.83 in terms of SOE. One of the extreme cases was the testing of B0045 at 4°C. Even though its SOH was very low (less than 0.5 throughout the test), it had a relatively high efficiency of about 0.78. As illustrated in these examples, SOH and SOE differ in how they explain battery performance. Generally, SOH describes the health of a battery in terms of its ability to release Coulombs. While SOE describes the efficiency of a battery as an energy storage medium in terms of the ratio of energy transfer during charging and discharging.

\begin{table}[htbp]
\caption{Typical SOE and SOH values.}\label{tbl:3}
\setlength{\tabcolsep}{8mm}{
\begin{tabular}{lll}
\specialrule{0.05em}{3pt}{3pt}
Batteries & SOH              & SOE              \\
B0005        & 90.23\%          & 87.59\%          \\
B0039                                                  & 88.58\%          & 90.50\% \\
B0005                                                  & 80.08\%          & 86.10\% \\
B0033                                                  & 81.28\%          & 73.73\% \\
B0005                                                  & 70.06\%          & 84.45\% \\
B0045                                                  & 46.40\%          & 77.55\%
\end{tabular}}
\end{table}

To quantitatively analyze the correlation between SOE and SOH, we used the person correlation coefficient (PCC). the PCC is calculated as follows:
\begin{equation} \label{eq:10}
    PCC = \frac{\sum(SOE - \overline{SOE})(SOH - \overline{SOH})}
            {\sqrt{\sum(SOE-\overline{SOE})^{2}\cdot\sum(SOH-\overline{SOH})^{2}}}\\.
        % ~ \\    
\end{equation}

\begin{table}[ht]
\caption{PCC of SOH and SOE.}\label{tbl:4}
\setlength{\tabcolsep}{2.6mm}{
\begin{tabular}{lllll}
\specialrule{0.05em}{3pt}{3pt}
\cellcolor[HTML]{DED1CA}{\begin{tabular}[c]{@{}l@{}}0.5406\\ (B0045)\end{tabular}} & \cellcolor[HTML]{F8CCAE}{\begin{tabular}[c]{@{}l@{}}0.9218\\ (B0046)\end{tabular}} & \cellcolor[HTML]{ECCEBA}{\begin{tabular}[c]{@{}l@{}}0.7705\\ (B0047)\end{tabular}} & \cellcolor[HTML]{E5CFC2}{\begin{tabular}[c]{@{}l@{}}0.6512\\ (B0048)\end{tabular}} & \cellcolor[HTML]{BDD7EE}{\begin{tabular}[c]{@{}l@{}}0.0333\\ (B0053)\end{tabular}} \\
\cellcolor[HTML]{BED7ED}{\begin{tabular}[c]{@{}l@{}}0.0502\\ (B0054)\end{tabular}} & \cellcolor[HTML]{BDD7EE}{\begin{tabular}[c]{@{}l@{}}0.0254\\ (B0055)\end{tabular}} & \cellcolor[HTML]{C3D6E7}{\begin{tabular}[c]{@{}l@{}}0.1265\\ (B0056)\end{tabular}} & \cellcolor[HTML]{F8CBAD}{\begin{tabular}[c]{@{}l@{}}0.9442\\ (B0007)\end{tabular}} & \cellcolor[HTML]{F6CCAF}{\begin{tabular}[c]{@{}l@{}}0.9161\\ (B0006)\end{tabular}} \\
\cellcolor[HTML]{F0CDB5}{\begin{tabular}[c]{@{}l@{}}0.8326\\ (B0005)\end{tabular}} & \cellcolor[HTML]{F6CCAF}{\begin{tabular}[c]{@{}l@{}}0.9183\\ (B0033)\end{tabular}} & \cellcolor[HTML]{E3D0C4}{\begin{tabular}[c]{@{}l@{}}0.6205\\ (B0034)\end{tabular}} & \cellcolor[HTML]{D6D2D2}{\begin{tabular}[c]{@{}l@{}}0.4282\\ (B0029)\end{tabular}} & \cellcolor[HTML]{C7D5E2}{\begin{tabular}[c]{@{}l@{}}0.1962\\ (B0030)\end{tabular}} \\
\cellcolor[HTML]{C2D6E9}{\begin{tabular}[c]{@{}l@{}}0.1075\\ (B0031)\end{tabular}} & \cellcolor[HTML]{D5D3D4}{\begin{tabular}[c]{@{}l@{}}0.3998\\ (B0032)\end{tabular}} & {}    & {}    & {}                                                                                
\end{tabular}}
\end{table}

As shown in the \tblref{tbl:3}, there is no obvious correlation between SOH and SOE during battery aging. However, it is a necessity to evaluate the efficiency of batteries in energy storage scenarios.

\section{Modeling State of Efficiency}
\subsection{Trend of SOE Trajectory}

A battery undergoes several charging and discharging cycles during its aging process. For the data set used, the SOH of each battery decreases over time until it reaches the end-of-experiment condition (20\% or 30\% capacity fade). According to \eqtref{eq:7}, we calculate the SOE for each battery under different operating conditions for every cycle. \figref{fig:4} shows the trajectory of the number of SOE (between 0 and 1) over cycles for each battery during the aging process. These batteries' SOE trajectories all show a flat or slightly downward (decreasing) trend. In other words, %they can be roughly expressed as a linear trend accompanied by fluctuations.
\dgli{a clear linear trend is observed despite different level of fluctuations is also present at different conditions.}

\begin{figure}[htbp]    
    \centering
    \includegraphics[scale=0.25]{SOE_curves.png}
    \caption{SOE trajectory under various operating conditions.}
    \label{fig:4}
\end{figure}

% Observing each battery's SOE trajectory, it is apparent that the SOE trajectories can be roughly expressed as a linear trend accompanied by fluctuations. In other words, most battery SOE trajectories show a downward (decreasing) trend, and this trend approximates a linear trajectory with fluctuations.% It is possible that these fluctuations are caused by capacity regeneration as there are different rest times between cycles.

\subsection{Linearity of the SOE trajectory}

Since the SOE degradation shows a %decreasing 
trend that is approximately linear, we propose the hypothesis that its SOE can be expressed as a time series with a linear model, with the following formula:

\begin{equation} \label{eq:11}
    SOE_t = \alpha t + \eta +\upsilon _t,
\end{equation}
where $ a $ is the slope of the trend, $ t $ is the cycle number, $ \eta $ is the energy efficiency of the battery at the beginning, and $ \upsilon _t $ is the fluctuation around the linear trend with an expectation approximating to zero. 

\lzh{We intend to show that the SOE trajectories of these batteries is flat or consistently decreasing over time. For this purpose, we will proceed to remove their inherent linear trend, by calculating the first-order difference of the SOE time series. If the first-order differences do not decrease or increase over time, then we can conclude that a linear model will fit the SOE trajectory. Now, the} first-order difference for $ SOE_t $ is expressed as

\begin{equation} \label{eq:12}
\begin{aligned}
\Delta _{SOE_t}  & = SOE_t - SOE_{t-1} \\
           &= \alpha t + \eta +\upsilon _t - ( \alpha (t-1) + \eta +\upsilon _{t-1}) \\
  & = \alpha + \nu _t    
\end{aligned},
\end{equation}
where $ \nu _t $ is the random fluctuation of $\Delta _{SOE}$ with the expectation of 0.

\lzh{To determine whether the $\Delta _{SOE}$ shows no trend,} we use The Mann-Kendall (MK) Trend Test.The MK Trend Test \cite{mann1945nonparametric, kendall1948rank} is used to analyze time series data for consistently increasing or decreasing trends (monotonic trends).
MK trend test is also used in trend detection in hydrologic data\cite{hamed2008trend}.
For the time series $\Delta _{SOE}$ with the actual number of cycles performed in the experiment of n (sample size), the MK trend test first calculates S:

\begin{equation} \label{eq:13}
S=\sum_{k=1}^{n-1} \sum_{j=k+1}^{n} \operatorname{sgn}\left(\Delta _{SOE_{j}} - \Delta _{SOE_{k}}\right),
\end{equation}
where $ \operatorname{sgn}(x)=\left\{\begin{array}{rll}

1 & \text { if } & x>0 \\

0 & \text { if } & x=0 \\

-1 & \text { if } & x<0

\end{array}\right. $.

An MK test analyzes the sign of the difference between all data and its predecessors. Throughout the time series, each value is compared with its preceding value, giving $ n(n - 1) / 2 $ pairs of sign values. Intuitively, the number of sign values indicates the tendency of a trend's present. The variance of S is then calculated as

\begin{equation} \label{eq:14}
    \operatorname{VAR}(S)=\frac{1}{18}\left(n(n-1)(2 n+5)-\sum_{k=1}^{p} q_{k}\left(q_{k}-1\right)\left(2 q_{k}+5\right)\right),
\end{equation}
where p represents the number of tie groups of the time series SOE, and $ q_k $ represents the number of data of $ k $ tie groups. For the test statistic, transform S into $ Z_{MK} $ as follows:

\begin{equation} \label{eq:15}
Z_{M K}=\left\{\begin{array}{rll}
\frac{S-1}{V A R(S)} & \text { if } & S>0 \\
0 & \text { if } & S=0 \\
\frac{S+1}{V A R(S)} & \text { if } & S<0
\end{array}\right .
\end{equation}

Using the python package \cite{hussain2019pymannkendall} of MK trend test, the $ \Delta _ {SOE} $ series of each battery were examined and the results are shown in \tblref{tbl:4}. The z-table can be used to calculate the probability (p value). 

\begin{table}[htbp] 
\caption{The p values of Mann-Kendall trend test result.}\label{tbl:5}
\begin{tabular}{|l|l|l|l|l|}
\hline
{B0045}    & {B0046}    & {B0047}    & {B0048}    & {B0053}    \\ \hline
{0.767631} & {0.434133} & {0.689333} & {0.655519} & {0.957178} \\ \hline
{B0054}    & {B0055}    & {B0056}    & {B0005}    & {B0006}    \\ \hline
{0.700847} & {0.882874} & {0.662952} & {0.870048} & {0.579624} \\ \hline
{B0007}    & {B0033}    & {B0034}    & {B0029}    & {B0030}    \\ \hline
{0.774103} & {0.998988} & {0.698799} & {0.919888} & {0.687463} \\ \hline
{B0031}    & {B0032}    & {   -}     & {   -}     & {   -}        \\ \hline
{0.782101} & {0.880084} & {   -}     & {   -}     & {   -}        \\ \hline
\end{tabular}

\end{table}

Assuming a 5\% significance level, if the p value is less then 0.05, then the alternate hypothesis is accepted, which indicates the presence of a trend, whereas if the p value is greater than 0.10, then the null hypothesis is accepted, indicating the absence of a trend, in other words, the $ \Delta _{SOE} $ series, which is the first-order difference of the SOE series for all batteries in the data set, does not exhibit a linear trend. \lzh{Therefore, all batteries in the data set can be fitted by linear models expressed by \eqtref{eq:11}.}

\subsection{SOE linear model and regression results}

\lzh{As discussed earlier, the SOE of a NCA LIB decreases linearly with its age (cycle number). Thus, we can independently represent the trend of SOE of a specific NCA LIB under a certain operating condition} with two parameters: the slope $ \alpha $ and the initial value (the intercept) $ \eta $.

\begin{equation} \label{eq:16}
SOE_t = \alpha t + \eta ,
\end{equation}
where $ t $ is the number of cycles. The least squares method is used to model the SOE over cycle number in the linear model, and the regression yields the slope $ \alpha $ and interception $ \eta $ of the SOE trajectory.

\begin{figure}[htbp]
	\centering  %图片全局居中
	% \subfigbottomskip=2pt %两行子图之间的行间距
	\subfigcapskip=-5pt %设置子图与子标题之间的距离
	\subfigure[B0005 $\alpha$: -0.0003, $\eta$: 0.8784]{
        % \subfigure[B0005 alpha: -0.0003, eta: 0.8784]{
		\includegraphics[width=0.48\linewidth]{B0005_result.png}}
	\subfigure[B00029 $\alpha$: -0.0002, $\eta$: 0.8403]{
		\includegraphics[width=0.48\linewidth]{B0029_result.png}}
        \subfigure[B00033 $\alpha$: -0.0003, $\eta$: 0.7571]{
		\includegraphics[width=0.48\linewidth]{B0033_result.png}}
	\subfigure[B0045 $\alpha$: -0.0003, $\eta$: 0.7647]{
		\includegraphics[width=0.48\linewidth]{B0045_result.png}}
	
	\caption{Typical SOH and SOE trajectories over cycling.}
        \label{fig:5}
\end{figure}

% \subsection{Regression results}

There are several groups of batteries in this data set that have been aged under various operating conditions. A selection of regression results for SOE trends is shown in \figref{fig:5}. The conditions differ mainly in terms of ambient temperature, discharge current, and cutoff voltage. In order to visualize the effect of different operating conditions on the energy efficiency of the battery, the SOE trend model has been used to regress the parameters in order to illustrate the SOE trends during battery aging, and a summary of the results is shown in \figref{fig:6}. 

As shown in \figref{fig:6} (a), a straight line represents a specific operating condition during the battery life cycle. Longer lines indicate a longer battery life. The blue, green, and red lines correspond to ambient temperatures of 4°C, 24°C, and 43°C respectively. A thinner line indicates a lower discharge current, whereas a thicker line indicates a higher discharge current. Dark colors represent deep cutoff voltages, and light colors are for high cutoff voltages. 

\figref{fig:6} (b) shows the SOE ranges of the batteries. Each horizontal bar represents an SOE range from a particular test case. Meanwhile, the rest of the batteries age under constant operation. Right-handed bars have a higher SOE, whereas left-handed bars have a lower SOE. At the beginning of the test, the SOE is located on the right side of the bar. As the battery ages, the SOE shifts to the left. Longer bars indicate a battery with a wider range of SOE during the test, which means these batteries show a more obvious decline in SOE during the test. 


\begin{figure}[htbp]
	\centering  %图片全局居中
	% \subfigbottomskip=2pt %两行子图之间的行间距
	\subfigcapskip=-5pt %设置子图与子标题之间的距离
	\subfigure[SOE trends]{
		\includegraphics[width=0.95\linewidth]{SOE_trend_result.png}}
	\subfigure[SOE ranges]{
		\includegraphics[width=0.95\linewidth]{SOE_trend_result2.png}}
		\caption{SOE trends and ranges under various operating conditions.}
        \label{fig:6}
\end{figure}

\section{Analysis and Discussion}
%\subsection{Regression results of the SOE trends model}

\subsection{SOE trends and ranges under different operating conditions}

The test schema specifies that EoL conditions occur when battery capacity drops below a certain level, at which point the test is terminated. As a result, the batteries undergo a variation in the number of cycles. Depending on the rate at which battery capacity is degraded, SOE trajectories vary in length. Among all the constant operating conditions included in the data set, batteries has the longest life at 24°C, followed by 4°C, and the shortest at 43°C. However, contrary to SOH, the batteries with the highest SOE were tested at 43°C, 1A. This suggests that discharge current may have a greater effect on SOE than ambient temperature.

Continuous cycling decreases the energy efficiency of most batteries. Nevertheless, some batteries, such as those discharged at 4°C ambient temperature at 2A current, retained their SOE or even became more efficient upon cycling. Moreover, their RUL was longer than those discharged at 4°C 1A current. These batteries, despite having a relatively long RUL and no downward trend in SOE, had a relatively lower SOE than those discharged at 4°C 1A. It is possible that the higher discharge current may have contributed to an extended RUL, but resulted in a suppressed SOE for batteries at extra low temperatures. It is interesting to note that these batteries suffer severely from lower cutoff voltages in terms of SOE at 4°C ambient temperature. The SOE of batteries discharged at 4°C 1A with a voltage of 2.0V and 2.2V has a value of approximately 0.75, while other batteries of the same group with a relatively higher cutoff voltage, have a value of approximately 0.8.

Cycled batteries at 43°C exhibit rapid degradation in terms of SOH, but their SOE appears to be least affected by discharge current and cutoff voltage. Among all groups, these batteries show superior SOE when discharging at 1A current, and even when discharging at 4A, relatively intense current, they are superior in terms of SOE.

When tested at 24°C with a 2A discharge current, batteries exhibit a long RUL and a high SOE. In these batteries, the cutoff voltage appears to have a mitigating effect on SOE, and RUL and SOE may be affected by differences in the manufacturing process. When the batteries were tested with 4A at the same temperature, their SOE was again suppressed by the discharge current. It is noteworthy that the data set does not include the 24°C 1A test case. Based on all the results in the data set, it appears that this approach will result in higher efficiency and RUL.

The range of SOE for these batteries is determined by the slope of the trend and the number of cycles. Batteries that have a relatively long RUL and have a high tendency to degrade have a longer SOE range. Batteries operating at 24°C 2A have a high initial SOE and a wide SOE range. These characteristics indicate that the batteries' energy efficiency is relatively good at the beginning of the test and decreases as they age. In contrast, batteries that operate at 43°C have a narrow SOE range, primarily due to their short RUL. 4°C 2A batteries don't show a tendency to decrease with a higher cutoff voltage, so they also have a narrow SOE range.

\begin{figure}[htbp]
    \centering 
    \includegraphics[scale=0.25]{SOE_special_group.png}
    \caption{SOE curves for a group of batteries with varying operating conditions.}
    \label{fig:7}
\end{figure}

The B0038, B0039, and B0040 batteries, on the other hand, had test conditions that changed during the aging period as opposed to the other groups with constant operating conditions. Initially, the ambient temperature is 24°C, but after the first 11 cycles, it changes to 43°C. Discharge current is initially 4A, is reduced to 1A when the temperature rises to 43°C, and then increases to 2A. Consequently, this battery group consists of three periods of constant operation: 24°C, 4A, 43°C, 1A, and 43°C, 2A. \figref{fig:7} shows the SOE curves after removing the first and last cycles, as well as the 11th cycle (the jump from 24°C to 43°C). The diamond-marked lines in \figref{fig:6} correspond to three periods of continuous operation. There is the lowest efficiency (24°C, 4A) in the first period, and the highest efficiency (43°C, 1A) in the second period. The efficiency decreases during the last period (43°C, 2A), but still outperforms the other group operating at 43°C, 4A. This group of batteries may indicate that energy efficiency can be adjusted under different operating conditions without causing memory effects.

In contrast to the other groups with constant operating conditions, one group of batteries (the B0038, B0039 and B0040) had test conditions that changed during the aging process, as shown in \figref{fig:5}. This group of batteries starts with an ambient temperature of 24°C and then changes to 43°C after the first 11 cycles, while the discharge current is initially 4A, then changes to 1A, and finally to 2A at the end of the test. Consequently, this battery group has three periods of constant operating conditions. 24°C, 4A; 43°C, 1A; and 43°C, 2A. After the removal of the first and last cycles as well as the 11th cycle (the jump from 24°C to 43°C), the SOE curves can be seen in Figure 2. As shown in the figure, the lines marked with diamonds correspond to 3 time periods with constant operating conditions. The first period (24°C, 4A) shows the lowest efficiency, while the second period (43°C, 1A) has the highest efficiency. In the last period (43°C, 2A), the efficiency decreases, but still outperforms the other group that operates at 43°C, 4A. This group of batteries may indicate that energy efficiency can be adjusted under different operating conditions without causing memory effects.

\subsection{Influence of operating conditions on SOE}

Operating conditions have a significant impact on the efficiency of a battery. Based on the data set used in this study, ambient temperature, cutoff voltage, charge current, and discharge current were all considered operating conditions, as shown in \figref{fig:8}. Different combinations of ambient temperature, cutoff voltage, and discharge current are tested on different groups of batteries. However, the charge current for all batteries is the same.

\begin{figure}[htbp]
    \centering  
    \includegraphics[scale=0.3]{SOE_impact_factors.png}
    \caption{The operating conditions in battery operation scenario.}
    \label{fig:8}
\end{figure}

\figref{fig:9} (a) shows that a battery with a lower discharge current is more energy efficient. Higher discharge currents allow a battery to operate at higher power, but they may also negatively affect the battery's SOE. A B0034 discharged at 4A has a SOE of roughly 0.73. On the other hand, the B0007 discharged at 2A has an SOE of more than 0.85, at the same ambient temperature and cutoff voltage. 

Battery performance increases at higher ambient temperatures, as shown in \figref{fig:9} (b) . From 4°C to 24°C, the SOE of a 2A discharged battery improved by almost 0.12; from 24°C to 43°C, the SOE of a 4A discharged battery improved by approximately 0.09.

Discharging batteries at different depths can be achieved by using different cutoff voltages. When a battery is discharged to an extended depth, more energy is released during a single discharge cycle. An increase or decrease in discharge depth, for example, from 2.7V to 2.5V, generally has a limited effect on the SOE, as shown in \figref{fig:9} (c). It should be noted, however, that at ambient temperatures of 4 degrees or 24 degrees, 2.2V or 2.0V cutoff voltages will result in a significantly lower SOE than those at higher cutoff voltages. Using the B0029 as an example, at a depth of discharge of 2.0V, the SOE is only 0.76; other batteries in the group, the B0032 and B0032, with cutoff voltages of 2.5V and 2.7V, have SOEs exceeding 0.86.

\subsection{Implications and Suggestions}

\lzh{Based on NASA's experimental data, we identified some interesting patterns and phenomena concerning battery energy efficiency. Some of these patterns or phenomena make sense intuitively, while some are somewhat surprising, but in many cases they seem to be quite different from the currently existing degradation patterns for SOH or SOC. Clearly, it would be worthwhile to delve further into the reasons behind the phenomenon, but the phenomena and patterns themselves can already serve as valuable references when it comes to using these batteries and designing efficient-aware BESSs in the future.}

\lzh{As it turns out, as the number of cycles increases, we observe that SOE also degrades just like SOH because of the effects of performance fading as the battery ages. In general, when a battery is operated under the same conditions throughout its lifespan, the overall energy efficiency degradation is rarely greater than 4\%. However, when compared to the effects of ageing, changes in operating conditions can result in more than 20\% difference in SOE. Surprisingly, the SOE does not seem to exhibit a significant 'memory effect': it appears that, when operating conditions of a battery are improved, the SOE of a battery reaches a higher level immediately, and that is comparable to the SOE of a battery that has been operated under similar conditions for a period of time. In other words, it seems that SOE is less influenced by previous operating conditions.}

\lzh{Therefore, old batteries, which are currently considered obsolete due to capacity loss, may in fact still be useful from a standpoint of energy efficiency. There is still considerable potential for these batteries to provide energy caching for renewable energy in an efficient manner under the right conditions. }

\lzh{It is important to note that the quantity of these old batteries can compensate for their capacity loss. A simple approach is to call on these batteries in turn to achieve a targeted strategy. Additionally, batteries can be clustered in a dissipative or non-dissipative circuit topology, to collectively provide the energy required for energy storage. In this way, retired batteries will enable renewable energy in a cost-effective, environmentally friendly and efficient manner. For BESS, the performance of batteries varies due to production deviations, inhomogeneous aging, and temperature differences within the cluster. With a battery pack containing batteries with differing SOEs and SOHs, the overall energy efficiency of BESS may be adversely affected. In order to maximize the performance and efficiency of a battery pack, active balancing must be implemented in order to equalize the performance of these batteries.}

% \lzh{Due to the fact that SOE does not have a significant memory effect, it is possible to sacrifice efficiency in emergency situations for other reasons without potential repercussions. Optimised use can be quickly resolved without affecting the efficiency of subsequent uses after the emergency energy requirement has been met.}

%\lzh{Operating conditions have a significant impact on SOE, as we mentioned above. The significant negative impact of high current discharge on SOE reminds us that to improve the overall efficiency of a BESS, the discharge intensity of individual cells should be limited in order to minimize energy efficiency losses. }

\lzh{It is true that both low and high temperatures can have a detrimental effect on the capacity of SOH. The effect of higher ambient temperatures on SOE, however, is essentially positive. In other words, the battery's energy efficiency can be improved when the temperature is kept warm within its normal operating range.  It is worth noticing that low temperatures have a specific negative effect on SOE. Deep discharge of a battery under low temperature conditions results in a rapid drop in energy efficiency. When the temperature is not low, however, the SOE is sensitive to being discharged at lower cut-off voltages. }

\lzh{Due to the fact that SOE does not have a significant memory effect, it is possible to sacrifice efficiency in emergency situations for other reasons without potential repercussions. special circumstances may require the battery to undergo more intense operations, such as deep discharge at low temperatures or high-current discharge, which generally only affects the battery's current energy efficiency and does not cause long-term impact of SOE. Optimised use can be quickly resolved without affecting the efficiency of subsequent uses after the emergency energy requirement has been met. Truly, due to the limitations of the data set, evaluating whether intense use of the battery under truly extreme conditions will cause irreversible damage to its energy efficiency performance requires more extensive data. In the future, we intend to continue working on this issue.}


% As we have seen, both aging and operating conditions have an impact on energy efficiency. A lithium-ion battery pack is often connected to a battery management system (BMS). In order to maintain safe and efficient operation, the BMS plays a critical role in ensuring all batteries work within their boundaries. The BMS controller should monitor the parameters of each battery, including terminal voltage, ambient temperature and battery temperature, charge and discharge current, and estimate the SOE with SOC and SOH.

% NCA Batteries perform better at higher ambient temperatures and at lower C-rates in terms of SOE. While the battery is discharged at high temperatures, a smaller discharge current can result in an acceptable energy efficiency. In general, different depths of discharge have no significant effect on energy efficiency, but extreme depths of discharge at low temperatures may negatively affect SOE. Consequently, the C-rate of a single cell can be controlled by maintaining a suitable ambient temperature and arranging the cells in packs to improve BESS' energy efficiency. From the SOE point of view, it is possible to adjust the cut-off voltage so that the cells release different amounts of energy with almost the same energy efficiency, except in cases of extreme low temperatures.

% In BESS, the performance of batteries varies due to production deviations, inhomogeneous aging, and temperature differences within the pack. With a battery pack containing batteries with differing SOEs and SOHs, the overall energy efficiency of BESS may be adversely affected. In order to maximize the performance and efficiency of a battery pack, active balancing must be implemented in order to equalize the batteries in the pack.

\begin{figure}[htbp]
	\centering  %图片全局居中
	% \subfigbottomskip=2pt %两行子图之间的行间距
	\subfigcapskip=-5pt %设置子图与子标题之间的距离
	\subfigure[The influence by discharge current.]{
		\includegraphics[width=0.9\linewidth]{impact_by_current.png}}
	\subfigure[The influence by ambient temperature.]{
		\includegraphics[width=0.9\linewidth]{impact_by_temperature.png}}
	\subfigure[The influence by cutoff voltage.]{
		\includegraphics[width=0.9\linewidth]{impact_by_voltage.png}}
	
	\caption{The influence to SOE by operating conditions.}
        \label{fig:9}
\end{figure}

\section{Conclusions}

Efficiency of batteries, particularly those used in energy storage systems, will have a significant impact on power systems. In this study, we proposed SOE as an indicator of the battery's SOE, and evaluated the SOE of NCA LIBs in the well-known NASA data set.

Our study examined the SOE trends of these batteries under a variety of operating conditions. On the basis of the linear trend hypothesis, we developed a SOE trend model that is applicable to this data set. We also discussed how different operating conditions under different scenarios affected the SOE of the batteries. Results of the regression of the SOE trend model have shown that increasing ambient temperature and decreasing discharge current have a positive impact on energy efficiency. The results also indicated that at low ambient temperatures, the battery's energy efficiency may be significantly reduced when operating at an extremely low cutoff voltage. As both aging and operating conditions have an impact on energy efficiency, BMS controllers should monitor the parameters of each battery, including terminal voltage, ambient temperature, charging and discharge current, so as to ensure performance and efficiency. 

There are a number of limitations to this study that should be acknowledged. The size of the NASA data set currently used restricts further analysis of the impact of a wide range of operating conditions on SOE trends. The data set includes only NCA LIBs with aging data. It may be necessary to develop more complex models to estimate the energy efficiency of different LIBs. Further studies may investigate the modeling and estimation of SOE trajectories for other types of LIBs by using a data set which will provide a broader range of scenarios.

% \begin{equation} \label{eq:}
    
% \end{equation}

% Numbered list
% Use the style of numbering in square brackets.
% If nothing is used, default style will be taken.
%\begin{enumerate}[a)]
%\item 
%\item 
%\item 
%\end{enumerate}  

% Unnumbered list
%\begin{itemize}
%\item 
%\item 
%\item 
%\end{itemize}  

% Description list
%\begin{description}
%\item[]
%\item[] 
%\item[] 
%\end{description}  

% Figure
% \begin{figure}[<options>]
% 	\centering
% 		\includegraphics[<options>]{}
% 	  \caption{}\label{fig1}
% \end{figure}


% \begin{table}[<options>]
% \caption{}\label{tbl1}
% \begin{tabular*}{\tblwidth}{@{}LL@{}}
% \toprule
%   &  \\ % Table header row
% \midrule
%  & \\
%  & \\
%  & \\
%  & \\
% \bottomrule
% \end{tabular*}
% \end{table}

% Uncomment and use as the case may be
%\begin{theorem} 
%\end{theorem}

% Uncomment and use as the case may be
%\begin{lemma} 
%\end{lemma}

%% The Appendices part is started with the command \appendix;
%% appendix sections are then done as normal sections
%% \appendix

% \section{}\label{}

% To print the credit authorship contribution details
\printcredits

%% Loading bibliography style file
% \bibliographystyle{model1-num-names}
% \bibliographystyle{cas-model2-names}
\bibliographystyle{elsarticle-num}

% Loading bibliography database
\bibliography{cas-refs}

% Biography
\bio{}
% Here goes the biography details.
\endbio

% \bio{pic1}
% Here goes the biography details.
\endbio

\end{document}

