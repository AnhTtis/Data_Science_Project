

%%%%%%%%%%%%
\section{Heavy Ion Collisions}
\label{sec:HIC}
%%%%%%%%%%%%

Although heavy ions are collided at a wide range of collision energies, the
concept of a hydrodynamic attractor has emerged from attempts to understand the
behaviour of hadronic matter at highest available energy densities.
In this section we will review some of the relevant kinematics as 
well as the idea of boost-invariance, which plays a key role at early stages of the
collision. 


%%%%%%%%%%%%%%%%%%%%%%%%
\subsection{The spacetime picture of the collision}
\label{subsect:general}
%%%%%%%%%%%%%%%%%%%%%%%%


In all collision systems and for all collision energies the relevant physics is
a challenge for existing theoretical techniques and eludes a direct treatment
based on QCD. A number of approximations and model approaches have emerged, each
taking advantage of special circumstances arising at various stages of evolution.  Those
different theoretical patches merge together into a coherent picture
consisting of following phases (see e.g.~\cite{Strickland:2014pga}):
%
\begin{itemize}
  \item {\bf collective state formation} ($0\leq\tau\lesssim 0.3~\rm fm/c$)\\
      gluon dominated, governed by semi-hard particle scattering; 
  \item {\bf pre-hydrodynamic collective flow} ($0.3\lesssim\tau\lesssim  2~\rm fm/c$)\\ 
      highly anisotropic QGP flow with large pressure gradients;
  \item {\bf hydrodynamic evolution} ($2\lesssim\tau\lesssim~6~\rm fm/c$) \\
      leading up to the QCD crossover followed by hadronisation;
  \item {\bf hot hadron gas } ($6\lesssim\tau\lesssim10~\rm fm/c$)\\ 
      expanding gas of hadrons exhibiting re-scattering processes; 
  \item {\bf freeze-out} ($\tau\gtrsim~10~\rm fm/c$)\\ 
      free-streaming gas of non-interacting hadrons. 
\end{itemize}
%
The sequence of events defined above is schematically pictured in
Fig.~\ref{fig:timescalesHIC}. The hyperbolae represent surfaces of constant
proper time $\tau\equiv\sqrt{t^2-z^2}$, while the nuclei move in the $z$
direction almost along the light cones.  In phenomenological computations,
hydrodynamic models are successfully used already at times around
$\tau\lesssim1$~fm/c, where the system is still highly anisotropic. Therefore,
our focus in this review is on the first two stages, the goal being to
understand how it is that the prehydrodynamic stage of evolution can be
described by fluid-dynamical models. 

%

\begin{figure}[t]
\begin{center}
\includegraphics[height = .4\textheight]{timescales.pdf} 
\caption{Schematic picture of an ultrarelativistic heavy ion collision with estimated time scales. Figure taken from Ref. \cite{Strickland:2014pga}.}
\label{fig:timescalesHIC}
\end{center}
\end{figure}


%%%%%%%%%%%%%%%%%%%%%%%%
\subsection{The initial stages}
\label{subsec:initial_state}
%%%%%%%%%%%%%%%%%%%%%%%%


The initial state of a heavy ion collision remains the most uncertain element of
the theoretical picture described in Sec.~\ref{subsect:general}, as it is the
domain of non-perturbative quantum field theory. Nevertheless, crucial insights
into the relevant physics were formulated already in the early 1980s.  Two heavy
ions approaching one another at ultrarelativistic velocity are highly Lorentz
contracted along the direction of motion, with the factor
$\gamma=1/\sqrt{1-v^2/c^2}\sim100$ typical for RHIC conditions and more than $1000$ for
the LHC. The ultrarelativistic nature of the collisions has critically important
consequences for the physics of the subsequent evolution. 

The fundamental observations originate in
Refs.~\cite{Gottfried:1974yp,Low:1977gq,Anishetty:1980zp} and rely on the notion
of {\em nuclear transparency}, which states that the highly Lorentz-contracted
nuclei essentially pass through each other, creating a central fragmentation
region of energy density high enough for a deconfined state of QCD matter to
form. The contracted nuclei are treated as if they were of infinite transverse
extent, with no dynamics in the transverse plane. The baryon number of the
colliding nuclei is carried away from this region by the receding projectiles,
leaving behind a drop of approximately baryon-neutral plasma.

The physical picture developed in Ref.~\cite{Bjorken:1982qr} envisages matter
moving essentially along the collision axis, which we take to be the z-axis,
with velocity $v=z/t$ in the centre of mass frame, in a manner reminiscent of
the Hubble expansion of the Universe. This assumption is equivalent to
invariance under boosts along the collision axis and it can be tested
experimentally. It implies that the number of charged particles per unit
rapidity $dN_{\rm ch}/d\eta$ is independent of rapidity in the region
$\eta\approx0$. Experimental data from PHOBOS~\cite{Back:2002wb,PHOBOS:2002whd}
shown in \rff{fig:dNchTutorial} demonstrate the emergence of a central plateau
region with increasing collision energy in the range $\sqrt{s}=19.6-200$~GeV in
the Au-Au system. In consequence, at earliest times, longitudinal expansion
dominates the dynamics and the transverse flow builds up only somewhat later.
This effect is strongest for central collisions. 




\begin{figure}
\begin{center}
\includegraphics[width =.9\textwidth]{cent_eta_20_130_200_BW.pdf}
\caption{Emergence of a central plateau region in the charge particle production
rate in Au-Au collisions for increasing collision energy
$\sqrt{s}=19.6,~130~{\rm and}~200~\rm GeV$. The grey band represents most central
collisions in the $0-6\%$ centrality bin. The plots are taken from
Refs.~\cite{Back:2002wb,PHOBOS:2002whd}.
}
\label{fig:dNchTutorial}
\end{center}
\end{figure}

%%%%%%%%%%%%%%%%%%%%%

This idealised picture can be expressed as a set of symmetry assumptions which
define {\em Bjorken flow}. To do this, it is very convenient to use the proper
time $\tau$ and spacetime rapidity $\sr={\rm arctanh}(z/t)$
coordinates\footnote{It is easy to check that for boost-invariant flow 
 $\eta_s = \eta$.}. In terms of these, the Minkowski metric takes the form:
%
\begin{equation}
    ds^2 =-dt^2+dz^2+dx_\perp^2 =-d\tau^2 +\tau^2 d\sr^2 +dx_\perp^2~,
\end{equation}
%
where $x_\perp=(x,y)$ are coordinates in the transverse plane. The physical
idealisations sketched in the previous paragraphs translate to the statement that
the physics is independent of spacetime rapidity $\sr$ as well as the
coordinates in the transverse plane. The components of the relativistic flow
velocity assume the form $(u^\mu) = (1,0,0,0)$, with $u_\mu u^\mu=-1$.  

The fundamental local observable which will be the focus of our considerations
is the energy-momentum tensor.  Under the symmetry assumptions stated above it
can be expressed in terms of three functions of the proper time $\tau$: 
%
\begin{equation}
    T^{\mu}_{\nu}= {\rm diag}\left\{-\edens(\tau),\pL(\tau),\pT(\tau),\pT(\tau)\right\}~,
    \label{eq:Tmn}
\end{equation}
%
where $\edens$ is the energy density in the local rest-frame, and the
eigenvalues $\pL, \pT$ are referred to as the longitudinal and transverse
pressures. The form of Eq.~(\ref{eq:Tmn}) does not rely on the applicability of
a hydrodynamic description, as it is determined only by the symmetry assumptions
reviewed above. 

One can parametrise the eigenvalues $\pL, \pT$
as 
\begin{equation}
    \pL=\PP\left(1-\frac{2}{3}\mathcal{A}\right), \qquad 
    \pT=\PP\left(1+\frac{1}{3}\mathcal{A}\right)~,
    \label{eq:AdefNC}
\end{equation}
where
\be
\PP \equiv \frac{1}{3}\left(\pL + 2\pT\right)
\ee
is naturally interpreted as the average pressure, while $\pa$ reflects the pressure anisotropy
\be
\pa  \equiv \f{\pL-\pT}{\PP}.
\ee
The pressure anisotropy is a measure of distance from equilibrium, or more
precisely, from spatial isotropy, which is a necessary condition for equilibrium
in the absence of external fields.



%%%%%%%%%%%%%%%%%%%%%%%
\subsection{Conformal symmetry}
\label{sec:confsym}
%%%%%%%%%%%%%%%%%%%%%%%%%

Since QCD at high energies is approximately scale invariant, it is natural to
impose conformal symmetry to simplify the mathematical description. This is a
very powerful assumption which requires tracelessness of the energy momentum
tensor $T^\mu_\mu=0$, and implies that $\PP=\edens/3$. The energy-momentum tensor
for conformal Bjorken flow can thus be expressed in terms of two functions of
proper time, $\edens$ and $\pa$.
Conservation of the energy momentum tensor 
%
\begin{equation}
    \label{eq:cons}
    \nabla_\mu T^{\mu\nu} = 0
\end{equation}
relates the pressure anisotropy to the logarithmic derivative of the energy density:
%
\begin{equation}
    \pa(\tau) = 6\left(1+\frac{3}{4}\tau\partial_\tau\ln\edens\right)~.  
    \label{eq:Adef3}
\end{equation}
%

For conformal systems it is also very convenient to introduce the concept of
{\it effective temperature} $T(\tau)$, defined by
%
\begin{equation}
    \edens = C_e T^4~,
    \label{eq:Tdef}
\end{equation}
%
where $C_e$ is a constant which depends on the number of degrees of
freedom. This equation has the form of a conformal equation of state, so that
in an equilibrium state $T$ is the thermodynamic temperature. Away from
equilibrium \rf{eq:Tdef} defines $T$ as equal to the temperature of an equilibrium state
with the same energy density. 

At asymptotically late times the system approaches local thermodynamic
equilibrium, so the pressure anisotropy tends to zero and the energy-momentum
tensor in~\rf{eq:Tmn} approaches the perfect-fluid form. The way this happens is
determined by the microscopic dynamics which governs the evolution of the
pressure anisotropy.  Once $\pa(\tau)$ is known, the energy density is
determined by \rf{eq:Adef3} up to a single integration constant which sets the
scale. In this sense, for Bjorken flow the dynamics is captured by the
pressure anisotropy.  In the late time limit, if we set $\pa\approx 0$, then
\rf{eq:cons} determines the 
effective temperature
%
\begin{equation}
    T = \f{\Lambda}{(\Lambda\tau)^{1/3}} 
\label{eq:bjorken}
\end{equation}
%
where $\Lambda$ is the integration constant containing information about the
initial condition.  This is a consequence of local equilibrium and the
conservation of energy-momentum, so it is valid regardless of any dynamical
details.



%%%%%%%%%%%%%%%%%%%%%%%%%%
\subsection{Prehydrodynamic evolution and the hydrodynamic attractor}
\label{subsec:prehydro}
%%%%%%%%%%%%%%%%%%%%%%%%%



%%%%%%%%%%%%%%%

\begin{figure}[t]
\begin{center}
\includegraphics[height = .3\textheight]{coupling.pdf} 
\caption{Schematic picture of a coupling evolution and the transition to hydrodynamics. Figure taken from Ref. \cite{Moreland:2018gsh}.}
\label{fig:hydrodynamisation}
\end{center}
\end{figure}


%%%%%%%%%%%%%%%


The early stages of QGP dynamics are not well understood at this time. At a
qualitative level one may say that the longitudinally expanding, approximately
boost-invariant initial state begins to build up transverse pressure and evolves
toward local thermal equilibrium. The main challenge is to understand how this
state becomes amenable to a description in terms of hydrodynamics. Consistency
with observation suggests that this happens on a timescale of about
$0.3\leq\tau\leq1~\rm fm/c$, when the system is still very anisotropic, and
hydrodynamics in the usual sense would not be expected to apply.  And yet, one
has to accept as fact that hydrodynamic simulations capture many essential
features of QGP dynamics. 

An important point is that gluon self-interactions are not only responsible for
asymptotic freedom, but also for their proliferation, which leads to a dense
medium. Attempts to describe it in terms of quasiparticles require parameter
values such that the mean free path of constituents cannot be large compared to
their de Broglie wavelength~\cite{Busza:2018rrf}. This implies that despite the
weakness of parton interactions at small distances, strong collective effects
should be expected and are seen as playing a key role in the thermalisation
process~\cite{Blaizot:1987nc,Baier:2000sb}.  The precise way this plays out is
still the subject of current  research, but it is feasible that following a
regime where a field-theoretical description is necessary, the system enters a
stage which can be described by approximately free-streaming quasiparticles (for
recent reviews please see e.g.
Refs.~\cite{Schlichting:2019abc,Berges:2020fwq,Schenke:2021mxx}).  The simplest
way to model this situation is to adopt a "step-function approach" and assume
that particles free stream for some time $\tau_{\rm fs}$, and at that point the
description switches to hydrodynamic evolution at a time when the expanding
plasma system is still far from equilibrium. This is schematically depicted in
Fig.~\ref{fig:hydrodynamisation}.

The successful application of hydrodynamic models in such far-from-equilibrium
situations implies that the complexity of initial states is rapidly reduced
within a very short interval of proper-time.  Since this happens for all initial
states, the system can be said to reach a far-from-equilibrium {\em hydrodynamic
attractor}.  In the context of boost-invariant flow this implies that any
potentially complex dynamics of the pressure anisotropy should give way to
universal features already at very early times, very far from the perfect fluid
domain.  Thus, hydrodynamic attractors enter the picture as an interface to the
hydrodynamic stage. In principle, this attractor could describe free streaming
at the very earliest times, but it is not known whether this is the case or not.
At present we have to resort to various models and uncontrolled approximations,
some of which (such as kinetic theory) imply free streaming, while others do
not.  

In the next seven Sections we will review the early-time dynamics and the
appearance of far-from-equilibrium attractors in various model systems.  We will
also address the important issue of relaxing some of the symmetry assumptions
which we have described in this Section. 


