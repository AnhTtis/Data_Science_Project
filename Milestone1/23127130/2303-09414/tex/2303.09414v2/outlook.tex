
%%%%%%%%%%%%%%%%
\section{Outlook}
\label{sec:Out}
%%%%%%%%%%%%%%%%

In this review we have attempted to present the main ideas behind the hypothesis
that the applicability of fluid dynamics to early phases of QGP dynamics can be
explained by a far-from-equilibrium hydrodynamic attractor. We have emphasised
the role of the kinematic setting specific to heavy-ion collisions, which makes
it plausible that such an attractor occurs also in QCD.  At the conceptual level
this picture is rather compelling. However, Bjorken flow is a very restrictive
setting, and despite some existing applications using the attractor in practical
calculations of phenomenologically interesting observables requires developing
effective methods of working with models with a large number of degrees of
freedom. Progress is likely to come gradually, by learning how to deal with
models of increasing complexity, extending the studies reviewed in Sections
\ref{sec:PhaseSpace} and \ref{sec:beyond}. 

The idea of hydrodynamic attractors has been closely connected with the
divergence of the hydrodynamic gradient expansion. In this review we have not
discussed this connection beyond its utilitarian aspects, but on a conceptual
level there have been some important developments in recent times. This includes
a proof of the generic divergence of the gradient expansion at the linearised
level without any symmetry assumptions, and its connection with the properties
of dispersion relations~\cite{Withers:2018srf,Heller:2020uuy}.  At the nonlinear
level, some of the results for Bjorken flow have been generalised to a much
wider class of flows, called longitudinal flows. In particular, the gradient
expansion has been shown to diverge for this class of
solutions~\cite{Heller:2021oxl}. It was also found that the large order
behaviour of the gradient series can be expressed in terms of new degrees of
freedom, the singulant fields, which track transient
effects~\cite{Heller:2021yjh}. The relevance of these advances to the study of
attractors remains an interesting challenge for the future. 

A number of issues were not addressed in this review. One is the presence of
other degrees of freedom, such as those  connected with chiral symmetry
breaking, and their possible effect on the attractor dynamics of
QGP~\cite{Mitra:2020mei,Mitra:2020hbj} (see also Ref.~\cite{Mitra:2022uhv}).
Another such issue is the role of fluctuations, which has been mostly neglected
in the attractor literature, with the notable exception of
Refs.~\cite{Akamatsu:2016llw,Chen:2022ryi}. 

It would also be very interesting to understand the role of quantum effects in
the early-time dynamics. Of course the hydrodynamic picture implicitly contains
them, but in a rather opaque way. On the other hand, the kinetic theory
description arises from quantum field theory through a series of
approximations~\cite{Mueller:2002gd,Jeon:2004dh} and it should be possible to
study the origin and robustness of the kinetic theory attractor in a framework
which allows for a systematic incorporation of quantum corrections. This is
connected with other approaches to early-time dynamics, including those
involving ideas such as the Color Glass Condensate or non-thermal
attractors~\cite{Berges:2008wm,Mazeliauskas:2018yef,Brewer:2019oha,Brewer:2022ifw,Brewer:2022vkq,Berges:2020fwq}.
It is not yet fully understood how they are related to the ideas reviewed here,
and clarifying this appears to be a very promising avenue for future research. 

Finally, there is the more general question about far-from-equilibrium
attractors in nonequilibrium systems.  In the context of heavy-ion collisions
the specific kinematic circumstances have lead us to consider boost-invariant
expansion where far-from-equilibrium hydrodynamic behaviour was first noted, but
there could be other situations where analogous phenomena might
appear~\cite{Baggioli:2021tzr}, perhaps even in the non-relativistic
domain~\cite{Le:2022ntg}.  Another context where far from equilibrium
hydrodynamic attractors can occur is the dynamics of systems in nontrivial
spacetime backgrounds (see e.g. Ref.~\cite{Vyas:2022hkm}). 


