
%%%%%%%%%%%%
\section{Attractors in hydrodynamic models}
\label{sec:attractors}
%%%%%%%%%%%%

Hydrodynamic attractors were first identified in hydrodynamic models, and
subsequently studied in other models of equilibration such as kinetic theory and
strongly coupled theories amenable to studies based on the AdS/CFT
correspondence.  This Section reviews attractors arising in hydrodynamic
models of Bjorken flow and introduces a number of concepts which will be used in
the remainder of this article.


%%%%%%%%%
\subsection{Bjorken flow in MIS theory}
%%%%%%%%%

We now turn to the description of Bjorken flow in MIS theory, specifically 
the BRSSS version~\cite{Baier:2007ix}.  As reviewed in \rfs{sec:HIC},
the dynamics of the energy-momentum tensor in this case is captured by the
pressure anisotropy $\pa(\tau)$ and the energy density $\edens(\tau)$, or
equivalently the effective temperature $T(\tau)$.  Conservation of the
energy-momentum tensor reduces 
to \rf{eq:Adef3}, which can be written  as the evolution equation for the 
effective temperature:
%
\be 
\label{eq:consbjorken} 
\tau\p_\tau \log T(\tau) = -\f{1}{3} + \f{1}{18} \pa(\tau)~,  
\ee
%
while the MIS relaxation equation~\rf{eq:brsss} becomes an evolution equation
for the pressure anisotropy. To write it down most explicitly one needs to take
full advantage of the constraints of conformal symmetry.   

Conformal symmetry implies that the energy scale is set by the local effective
temperature. The transport coefficients are then determined by dimensional
analysis~\footnote{The other transport coefficients appearing in  \rf{eq:brsss}
    are similarly constrained, but are not relevant for Bjorken flow.}
%
\begin{equation}
\label{eq:tcs}
\tau_\pi=\frac{C_{\tau\Pi}}{T}, \qquad \eta=C_\eta s~,\qquad \lambda_1=\frac{C_{\lambda_1}}{T \eta}~,
\end{equation}
%
where $s=4\edens/3T$ is the entropy density,  
up to dimensionless constants $C_{\tau\Pi},C_\eta,C_{\lambda_1}$. These
constants can be fitted to experiment, or matched to an underlying microscopic
theory in cases where an explicit calculation of the gradient expansion is
feasible. An example of such a calculation for the cases of \symm\ was carried
out in Ref.~\cite{Bhattacharyya:2007vjd,Baier:2007ix} using the AdS/CFT correspondence, with
the result
%
\begin{equation}
C_{\tau\Pi}=\frac{2-\log 2 }{2\pi}~,  \qquad C_\eta=\frac{1}{4\pi}~, \qquad C_{\lambda_1}=\frac{1}{2\pi}~.
\end{equation}
%
These values provide a useful point of reference as well as an order of
magnitude estimate which is sometimes used in hydrodynamic simulations. 

Once the transport coefficients are written in the form \rf{eq:tcs}, the
MIS/BRSSS relaxation equation can be written in the form
%
\be
\label{eq:misbjorken} 
    C_{\tau\Pi} \left(\tau \pa'(\tau) + \f{2}{9} \pa^2(\tau)\right) = 8 C_\eta -\tau T(\tau) \left( \pa(\tau) + \frac{C_\lambda}{12 C_\eta}  \pa(\tau)^2\right).
\ee
%
The system of two coupled ordinary differential equations, \rf{eq:consbjorken}
and \rf{eq:misbjorken} determines the dynamics of Bjorken flow in MIS theory. 


%%%%%%%%%%%%%%%%%%%%%%%%%%%%%
\subsection{Late time asymptotics of Bjorken flow}
%%%%%%%%%%%%%%%%%%%%%%%%%%%%%

The evolution equations,
\rf{eq:consbjorken} and \rf{eq:misbjorken}, can be combined to give a single ODE which
determines the dynamics of the effective temperature
%
\be
    \label{eq:MISTeom}
    C_{\tau\Pi}\tau T'' &+&
    \frac{3}{2}\tau\left(\frac{C_{\lambda_1}\tau}{C_\eta}+\frac{2C_{\tau\Pi}}{T}\right){T'}^2+
    \left(\frac{11C_{\tau\Pi}}{3}+\frac{(C_\eta+C_{\lambda_1})\tau
    T}{C_\eta}\right)T'+\nn\\
    &+& \frac{(2C_\eta+C_{\lambda_1})T^2}{6C_\eta}-
    \frac{4(C_\eta-C_{\tau\Pi})T}{9\tau}=0~.
    \ee
%
It is easy to see that at large proper-times this equation has an asymptotic 
solution of the form
%
\begin{equation}
    T(\tau)=\frac{\Lambda}{(\Lambda\tau)^\frac{1}{3}}
    \left(1-\frac{2C_\eta}{3(\Lambda\tau)^\frac{2}{3}}+
    O\left(\f{1}{(\Lambda\tau)^{4/3}}\right) \right)~,
    \label{eq:MISTasym}
\end{equation}
%
where $\Lambda$ is an integration constant.  Since the initial value problem for
\rf{eq:MISTeom} allows for the choice of two integration constants, namely the
initial temperature and its derivative, it is clear that the asymptotic solution
\rf{eq:MISTasym} contains only half the information encoded in the initial
state. This is a consequence of dissipation. A complete solution would require
augmenting this result with additional terms which depend on the remaining
initial data, but vanish faster than any power of proper time. We will return to
this important point below in \rfs{sec:late}. 

Quite generally, dissipation implies an effective loss of information:
specifically, a partial loss of memory of the initial state of the system. The initial
state can be far from equilibrium and may be characterised by many parameters.
On the other hand, the final state of equilibrium is characterised by very few
parameters. The asymptotic late-time behaviour of the system will thus be
partially independent of the initial state.  This process of ``information
loss'' can be studied using modern asymptotic methods. Furthermore, it lies at
the heart of the idea of hydrodynamic attractors, which -- as we will discuss in
detail -- is fundamentally the observation that generic initial states evolve
into a region of phase space which can be effectively covered by a subset of all
possible initial conditions. 


%%%%%
\subsection{Universal variables}
%%%%%


In the case of conformal Bjorken flow it is possible to make the notion of
information loss described above even sharper by using suitable variables which
are correlated in a universal way: variables in which the asymptotic behaviour
near equilibrium is completely independent of initial conditions.  This is not a
typical situation and is only possible due to the very strong symmetry
assumptions.

Conformal symmetry suggests using the dimensionless pressure anisotropy $\pa$
and introducing the dimensionless variable $w\equiv \tau T$. At late times, when
the temperature follows \rf{eq:bjorken},
$w\sim\tau^{2/3}$, so that it can be thought of as a ``clock variable'': the
proper time in units of local effective temperature. Since the relaxation time
$\tau_\Pi \sim 1/T$, one also has $w\sim\tau/\tau_\Pi$, so one can think of this
variable as the proper time in units of the relaxation time. 
Using these dimensionless variables, the conservation equation~\eqref{eq:consbjorken} can be
written as
%
\be 
\label{eq:consw} 
\f{d\log T}{d \log w} = \f{\pa - \ 6}{\pa + 12}~, 
\ee
%
and the MIS equation \rf{eq:misbjorken} takes the form 
%
\begin{equation}
    \label{eq:MISAeom}
C_{\tau\Pi}\left(1+\frac{\mathcal{A}(w)}{12}\right) \mathcal{A'}(w)+\left(\frac{C_{\tau\Pi}}{3w}+\frac{C_{\lambda_1}}{8C_\eta}\right)\mathcal{A}(w)^2=\frac{3}{2}\left(\frac{8C_\eta}{w}-\mathcal{A}(w)\right)~.
\end{equation}
%
The remarkable point here is that \rf{eq:MISAeom} is a self-contained equation
which can be solved independently of the conservation law \rf{eq:consw}.  Once
solutions $\pa(w)$ are found, they can be used in \rf{eq:consw} to determine the
corresponding evolution of the effective temperature. 

For a perfect fluid $\pa=0$ and either \rf{eq:consbjorken} or \rf{eq:consw}
suffices to determine the solution, leading to Bjorken's \rf{eq:bjorken}.
However, for dissipative systems one must also specify a nontrivial solution of
\rf{eq:MISAeom}, which depends on the microscopic dynamics of the plasma through
the transport coefficients, as well as on the initial state of the system.  If a
solution $\pa(w)$ of \rf{eq:MISAeom} is given, one can integrate
\rf{eq:consw} to solve for the effective temperature as a function of $w$:
%
\begin{equation}
  \label{eq.evol}
   T(w) =  \Phi_\pa (w, w_0)  T(w_0)~,
\end{equation}
%
for some initial condition 
$T(w_0)$, with the function $\Phi_\pa$ being
%
\begin{equation}
  \label{eq.phidef}
  \Phi_\pa (w, w_0) = \exp\left(\int_{w_0}^{w} \f{dx}{x}
  \f{\pa(x) - \ 6}{\pa(x) + 12} \right)~.
\end{equation}
%
The subscript $\pa$ which appears above indicates the functional dependence of this
quantity on the pressure anisotropy.


%%%%%%%%%%%%%%%%%%%%%%%%%%%%%%%%%%%%%%%
\subsection{The hydrodynamic attractor}
%%%%%%%%%%%%%%%%%%%%%%%%%%%%%%%%%%%%%%%

It is straightforward to solve \rf{eq:MISAeom} numerically. As expected, at late
times all solutions tend to zero as equilibrium is approached. However, a rather striking
picture emerges when studying the behaviour of solutions obtained by setting
initial conditions at a sequence of diminishing initial values of
$w$, as seen in \rff{fig:MISattractor}. It is evident that
the solution curves rapidly approach a distinguished locus, which is referred to
as a far-from-equilibrium attractor~\cite{Heller:2015dha}. This attractor curve is 
determined uniquely by this procedure, and will be denoted by $\pa_\star$. 
It is the extension of the hydrodynamic attractor expected near equilibrium into
the early-time, nonequilibrium region. 

%%%%%%%%%%%%%%%%

\begin{figure}
\begin{center}
    \includegraphics[height =.55\textwidth]{attrMIS.pdf}
\caption{Some solutions of \rf{eq:MISAeom} (blue lines) plotted together with
the attractor (red line); the dashed magenta line represents second order viscous hydrodynamics.}
\label{fig:MISattractor}
\end{center}
\end{figure}


%%%%%%%%%%%%%%%


It is physically important that solutions initialised off  the attractor
approach it rapidly while the pressure anisotropy is high and the system is
still far from equilibrium.  This fact leads to a potential explanation of the
early thermalisation puzzle, as we will argue in the following.  Note also that
solutions which start out below the attractor are initially driven {\em away}
from equilibrium toward the attractor. As discussed further below, this is a
consequence of the strong longitudinal expansion. 

The emerging picture is that for a given range of initial conditions,
apart from an initial transient, the function $\pa(w)$ quickly approaches a
universal attractor $\pa_\star(w)$ which is determined by the
microscopic theory under consideration. We
assume that the physically interesting range of initial conditions is in the
basin of attraction of this unique attractor.  This suggests that it should be a
good approximation to replace the form of the pressure anisotropy $\pa(w)$, as
it appears in \rf{eq:consw}, by the attractor $\pa_\star(w)$:
%
\begin{equation}
  \label{eq.approx}
   T(w) \approx  \Phi_{\pa_\star} (w, w_0)  T(w_0)~.
\end{equation}
%
Within such an approximation, the temperature at late times is determined by the
temperature at early times {\em alone}: the remaining dependence on the initial
state is neglected by assuming that the effective dynamics of the system is
captured by its attractor, apart from a negligible initial transient\footnote{
    An example which bears some similarity to what is considered here is the
    idea of an inflationary attractor in cosmology, which also captures the
    effective loss of information about the pre-inflationary features of our
Universe (see e.g.~\cite{Mukhanov:2005sc}).}.

The attractor apparent in the pressure anisotropy $\pa(w)$ is particularly
striking, but it is a manifestation of an intrinsic feature of this dynamical  system, as
well as many other like it, however one chooses to describe them. It also has
implications for other observables,  such as the speed of sound away from
equilibrium~\cite{Cartwright:2022hlg}.

The qualitative picture seen in \rff{fig:MISattractor} is typical of Bjorken
flow in many models of equilibration, including various extensions of MIS
theory, anisotropic hydrodynamics~\cite{Strickland:2017kux,Alqahtani:2022xvo},
kinetic theory as well as strongly coupled \symm\ theory.  Before reviewing some
of them, we will try to understand the features seen in this plot in a
quantitative way, using asymptotic methods to extract the relevant physics from
\rf{eq:MISAeom}. 


%%%%%%%%%%%%%%%%%%%%%%%%%%%%%%
\subsection{Early time behaviour}
\label{sec:MISearly}
%%%%%%%%%%%%%%%%%%%%%%%%%%%%

As it is clear from \rff{fig:MISattractor}, at small
values of $w$ generic solutions are divergent, apart from the attractor which is
regular there. It is straightforward to check that if we assume that the
pressure anisotropy approaches a finite, constant value $\pa_\pm$ as $w
\rightarrow {0}$,
then the only possible values consistent with \rf{eq:MISAeom} are
\begin{equation}
    \pa_\pm = \pm 6\sqrt{C_\eta/C_{\tau\Pi}}~.
\end{equation}
The negative option is unstable, it acts as a repulsor; we will not discuss it 
further here. The positive value
provides the initial condition which can be used to determine the attractor
numerically. 

The early-time behaviour of regular solutions of \rf{eq:MISAeom} can be studied
analytically through a convergent series expansion in powers of
$w$~\cite{Heller:2015dha,Kurkela:2019set,Aniceto:2022dnm} 
%
\begin{equation}
    \paz(w)=\sum_{n=0}^\infty c_n w^n=6\sqrt{\frac{C_\eta}{C_{\tau\Pi}}}-    \frac{9(C_{\lambda_1}+2\sqrt{C_\eta C_{\tau\Pi}})}{C_{\tau\Pi}(2 C_{\tau\Pi} + 9 \sqrt{C_\eta C_{\tau\Pi}})}w+\cdots
\end{equation}
%
In the following we will denote the attractor solution by
$\pa_\star$. 
The remaining solutions of \rf{eq:MISAeom} diverge at $w=0$, but are seen to
approach the attractor rapidly. 
From a physical perspective it is very important to understand how exactly this
happens and what is the reason for it. One can look for solutions of
the form
%
\begin{equation}
    \pa(w) = \paz(w) + \delta\pa(w)
\end{equation}
%
where $\delta\pa(w)$ is dominant for $w$ approaching zero.  
The equation of motion \rf{eq:MISAeom} then takes the approximate form
%
\begin{equation}
    w \delta \pa'(w)+4 \delta\pa(w)=0~,
\end{equation}
%
which gives $\delta\pa\sim w^{-4}$. This result is independent of the transport
coefficients, which suggests a kinematic origin of this phenomenon.  More
specifically, the physical mechanism behind it can be identified with the strong
longitudinal expansion of the system.  The implications of this fact will be
discussed in \rfs{sec:beyond}. 


%%%%%%%%%%%
\subsection{Late time behaviour}
\label{sec:late}
%%%%%%%%%%%

At large values of $w$, all solutions plotted in \rff{fig:MISattractor} approach
the curve corresponding to the leading order of the gradient expansion. This
can be seen directly in \rf{eq:MISAeom} by noting that as $w \rightarrow
\infty$ both terms on the left hand side of are subdominant, so that the leading
asymptotic behaviour is 
%
\begin{equation}
    \pa(w)\sim\frac{8 C_\eta}{w}\,.
    \label{eq:MISlatetimeL}
\end{equation}
%
Just as the late-time solution of \rf{eq:bjorken}, this implies a loss of initial
state information, because \rf{eq:MISAeom} which governs the dynamics of the
pressure anisotropy requires an initial condition, so a general solution would
contain a single integration constant. This information is completely absent
from the asymptotic solution \rf{eq:MISlatetimeL}, which is completely
universal, identical for all initial conditions. 

As an aside, it is amusing to note that the leading asymptotic behaviour of the
pressure anisotropy can be made not just independent of the initial conditions,
but even across different theories, which at this order differ only by the value
of $\eta/s$. Indeed, defining $\tilde{w}\equiv w/8C_\eta$,
the asymptotics of the pressure anisotropy in any conformal theory are simply
$\pa\sim 1/\tilde{w}$~\cite{Heller:2016rtz}.  This observation has found applications in situations
where the late-time behaviours of different theories are compared. 

The leading asymptotic behaviour of the pressure anisotropy captured by
\rf{eq:MISlatetimeL} is corrected by an infinite series of subleading terms:
%
\begin{equation}
 \label{eq:MISlatetime}
    \mathcal{A}(w)= \sum_{k=1}^\infty \f{a_k}{w^k}
\end{equation}
%
with
\begin{equation}
    \label{eq:MISgradevals}
    a_1 = 8 C_\eta, \quad 
    a_2 = \f{16}{3}\, C_\eta( C_{\tau\Pi}-C_{\lambda_1})~.
\end{equation}
%
Each term appearing here corresponds to a specific order of the gradient
expansion.  If this series is truncated, one obtains an
approximation which one would like to identify with the hydrodynamic
prediction for the asymptotic behaviour of $\pa(w)$. There is an important
subtlety however: the series appearing in \rf{eq:MISlatetime} has a vanishing
radius of convergence. This will be discussed at length below, but for the
moment we will adopt a pragmatic attitude and simply truncate the expansion,
keeping only a couple of the leading terms.

It is important to realise that there are corrections to \rf{eq:MISlatetime}
which are not of the form of a power of $1/w$ -- instead, they are damped exponentially
in the limit of large $w$. To see this, one can linearise this equation around
the truncated asymptotic solution 
%
\be
\pa(w) = \f{a_1}{w} + \f{a_2}{w^2} + \delta\pa(w)~,
\ee
%
by treating $\delta\pa$ as small.  This leads to the equation
%
\begin{equation}
\delta\mathcal{A}'(w)+
\left(
\frac{3}{2 C_{\tau\Pi}} +
\frac{2C_{\lambda_1} - C_\eta}{C_{\tau\Pi}}\f{1}{w} + O\left(\f{1}{w^2}\right)
\right)\delta\mathcal{A}=0~,    
\label{eq:deltaA}
\end{equation}
%
whose solution is
%
\begin{equation}
    \delta\mathcal{A}(w)=\sigma w^{\frac{C_\eta-2
    C_{\lambda_1}}{C_{\tau\Pi}}}e^{-\frac{3w}{2C_{\tau\Pi}}} \left(1
    +O\left(\f{1}{w}\right) \right)~,
\end{equation}
%
where $\sigma$ is an integration constant. 
A more systematic analysis along the lines sketched above reveals solutions of
the form of a {\em
transseries}~\cite{Heller:2015dha,Aniceto:2015mto,Basar:2015ava}:
%
\begin{equation}
    \mathcal{A}(w)=\sum_{m=0}^\infty\sigma^me^{-mAw}\Phi_m(w)~,
    \label{eq:MIStrans}
\end{equation}
%
where 
%
\begin{equation}
    \Phi_m(w)=w^{m\beta}\sum_{n=0}^\infty\frac{a^{(m)}_n}{w^n}~,
\end{equation}
%
with
%
\begin{equation}
    \label{eq:singulant}
    A=\frac{3}{2C_{\tau\Pi}}~, \qquad \beta=\frac{C_\eta- 2 C_{\lambda_1}}{C_{\tau\Pi}}~.
\end{equation}
%
and $\Phi_0(w)$ is just the series \rf{eq:MISlatetime}. Each transseries sector
provides a set of corrections weighted by a power of an exponential damping
factor. The damping rate is set by the relaxation time -- the constant factor of
$3/2$ is explained in Ref.~\cite{Janik:2006gp}.  Crucially, each transseries
sector is also weighted by a power of the undetermined transseries parameter --
the integration constant $\sigma$. This integration constant can in principle be
determined by setting an initial condition, but that information is
exponentially dissipated away in the course of evolution. The transient effects
of the nontrivial transseries sectors can actually be seen in numerical
experiments~\cite{Spalinski:2018mqg}.  It is important to note that the presence
of the transseries sectors is a consequence of the presence of nonhydrodynamic
modes in MIS theory. This connection is quite general and will manifest itself a
number of times in the following. 

The transseries structure is a beautiful metaphor of how information about the
initial state is dissipated in the course of evolution as the system approaches
equilibrium: this data is effectively lost due to the exponential damping,
leaving only a universal hydrodynamic tail: the hydrodynamic attractor.  The
early-time $1/w^4$, expansion-driven approach to the attractor is replaced at
later times by the exponential nonhydrodynamic mode decay whose rate is set by
the relaxation time.  

%%%%%%%
\subsection{Determining the attractor}
%%%%%%%

While there exist hydrodynamic models where the attractor can be found
exactly~\cite{Denicol:2019lio,Strickland:2019hff}, in general attractors can be
found be studying the behaviour of multiple solutions obtained by numerical
means. In the simple case of Bjorken flow in conformal MIS theory this can be
done by setting initial conditions at decreasing values of $w$, as illustrated
in \rff{fig:MISattractor}. 

Another approach to capturing the attractor is a variant of the slow-roll
approximation best known in the context of inflationary
cosmology~\cite{Liddle:1994dx,Spalinski:2007kt}. This method is approximate, but
can be pursued analytically. The idea is to treat the derivative term in
\rf{eq:MISAeom} as a perturbation, which ensures the regularity of the obtained
solution at $w=0$.  This can be implemented systematically by inserting a formal
gradient-counting parameter $\epsilon$ into \rf{eq:MISAeom} and seeking a
solution as a series in this quantity. The zeroth order solution is determined
by a quadratic equation.  The attractor solution corresponds to positive root,
and one finds~\cite{Heller:2015dha}
%
\begin{equation}
    \mathcal{A}_{\rm slow roll}(w)=
   \frac{6}{8C_{\tau\Pi}+\frac{3C_{\lambda_1}w}{C_\eta}}\sqrt{64C_\eta C_{\tau\Pi}+24 C_{\lambda_1}w+9w^2}~.
\end{equation}
%
This is just the nullcline of \rf{eq:MISAeom}. Corrections are easily calculated
and provide a very accurate representation of the attractor, but its analytic
form quickly becomes very complex.


Another way to obtain approximate attractors analytically in certain hydrodynamic models was
proposed in~\cite{Jaiswal:2019cju}, where the authors considered 
a family of relaxation equations of the form 
%
\begin{equation}
    \frac{d\pi}{d\tau} = -\frac{\pi}{\tau_\pi} +\frac{1}{\tau}\left[\frac{4}{3}\beta_\pi-\left(\lambda+\frac{4}{3}\right)\pi-\chi\frac{\pi^2}{\beta_\pi}\right]~,
    \label{eq:summary_pi}
\end{equation}
%
where $\pi\equiv\edens\pa$.  By suitable choices of the parameters $\beta_\pi$,
$\tau_\pi$, $\lambda$ and $\chi$ one can describe the original MIS
model~\cite{Israel:1979wp}, the DNMR model~\cite{Denicol:2012cn} or the
"third-order" model of Ref.~\cite{Jaiswal:2013vta}. All three models possess an
attractor solution, but it can only be found numerically. In a conformal theory,
the relaxation time is determined by the effective temperature, i.e.
$\tau_\pi\sim1/T(\tau)$. One can obtain an analytic approximation of the
attractor by treating this dependence in a sort of ``mean field'' spirit.
Instead of keeping the exact temperature dependence the authors of
Ref.~\cite{Jaiswal:2019cju} study three possible options which amount to taking
the temperature to be constant, or taking one or two terms in the expansion
given in  \rf{eq:MISTasym}.  In each of these cases one can obtain a general
analytic solution to \rf{eq:summary_pi}, which depends on an integration
constant. It is possible to choose this integration constant to obtain a
solution regular at $w=0$. This solution provides a rather good approximation to
the numerically calculated attractor, with the error not larger than
$3\%$~\cite{Jaiswal:2019cju}. This approximate attractor solution was used in
practice for the computations of thermal particle production~\cite{Naik:2021yph}.

Further analytic results for boost-invariant attractors can be found in
Refs.~\cite{Denicol:2017lxn,Denicol:2019lio,Blaizot:2020gql}.

We will also describe two systematic approaches to finding attractors in an
approximate way. One is 
based directly on the
gradient expansion, and leads to some very interesting developments which we
review in the following subsection. 
The other, perhaps the most general approach to
identifying attractors, albeit purely numerically, involves studying sets of solutions on time slices of phase space; it will be described in
\rfs{sec:PhaseSpace}. 

%%%%%%%%%%%%%%%%%%%%%%%%%%%%%%%%%%%%%%%%%%%%%
\subsection{The gradient expansion at large orders}
\label{sec:largeorders}
%%%%%%%%%%%%%%%%%%%%%%%%%%%%%%%%%%%%%%%%%%%%%

We now turn to an important point of both mathematical and physical
significance: the infinite series appearing in~\rf{eq:MIStrans} have a vanishing
radius of convergence.  At sufficiently late times, the asymptotic behaviour of
all solutions is given by the leading order of the gradient expansion, which
corresponds to Navier-Stokes theory. In many cases it has been possible to
calculate a large number of terms, which offers the possibility to extend the
late-time approximation of the attractor toward early times. This was studied in
the case of the large proper time expansion of \symm\ in
Ref.~\cite{Heller:2013fn} where the series was found to have a vanishing radius
of convergence. It was subsequently found that such expansions diverge in many
other cases, including models of
hydrodynamics~\cite{Heller:2015dha,Aniceto:2015mto} and kinetic
theory~\cite{Denicol:2016bjh,Heller:2016rtz,Florkowski:2016zsi}.  It has been
demonstrated that in the context of MIS theory the gradient expansion has a
vanishing radius of convergence also beyond the relatively simple setting of
Bjorken flow, and it can only be avoided by fine-tuning of the initial
conditions~\cite{Heller:2020uuy,Heller:2021oxl}. In fact, the only known example
of where the hydrodynamic gradient expansion is convergent for generic initial
conditions occurs for Bjorken flow in the model of an ultrarelativistic  gas of
hard spheres of Ref.~\cite{Denicol:2019lio}.  The implication of these findings
is that the gradient series does not define a unique solution. However, it
captures the asymptotic behaviour of all solutions in the late-time limit.  

The simplest approach to such divergent asymptotic series is truncation at low
order, as we have been tacitly assuming until now. It is known from countless
examples (such as the Stirling formula for the Gamma function $\Gamma(z)$ at
large values of $|z|$) that keeping only the leading terms of a divergent
asymptotic series often gives excellent results also quite far from the
asymptotic limit. This can be made quite precise using the notion of optimal
truncation~\cite{Bender78:AMM}. While this approach is very useful in practice,
from a conceptual point of view it is very interesting and rewarding to examine
the nature of the divergence in more detail, since it reveals the physics behind
it. 

The gradient expansion of the pressure anisotropy is of the form 
%
\begin{equation}
    \mathcal{A}(w)=\sum_{n=0}^\infty\frac{a_n}{w^n}~,
    \label{eq:Agradexp}
\end{equation}
%
where the leading terms can be read off from \rf{eq:MISlatetimeL}. When referring to
this series in the case of MIS, for definiteness we will assume numerical values for the
coefficients $a_n$ given in \rf{eq:MISlatetime}.  It is straightforward to compute hundreds of
these coefficients numerically. Simple convergence tests lead to the conclusion that the series is
divergent factorially (see \rff{fig:divergeMIS}): at large $n$, up to a constant factor, one has 
\be
\label{eq:dingle}
a_n\sim\Gamma(n+\beta) A^{-n}, 
\ee
where $A, \beta$ are constants which carry important information
about the physics. In particular, the quantity $A$ reflects the damping rate ot
transient, nonhydrodynamic effects. Since \rf{eq:dingle} arises in many
contexts, $A$ is referred to by various names. We will follow Dingle and refer
to is as the {\em singulant}~\cite{Dingle}. In the case of MIS theory
$A=3/2C_{\tpi}$, which shows that the divergence originates in the
nonhydrodynamic sector. 


%%%%%%%%%%%%%%%%%%

\begin{figure}[t]
\begin{center}
\includegraphics[width =.6\textwidth]{divergeMIS.pdf}
\caption{The ratio of the coefficients of the gradient expansion \rf{eq:Agradexp}.}
\label{fig:divergeMIS}
\end{center}
\end{figure}

%%%%%%%%%%%%%%%%%%%%%%

There is a large and growing body of work aimed at understating the role of
corrections to asymptotic series such as \rf{eq:Agradexp}, sometime referred to
as ``asymptotics beyond all orders''~\cite{Berry1991}.  An effective approach to
this problem is to consider ``resumming'' the series in Eq.~(\ref{eq:Agradexp}).
By this one means finding a function whose asymptotic expansion matches the
original series (see e.g.~\cite{Bender:2017fyz}). Given a factorially divergent sequence $\{c_n\}$ this can be
done by Borel summation, whose basic idea is captured by the formal manipulation 
\begin{equation}
\sum_{n=0}^{\infty} c_n
=
\sum_{n=0}^{\infty} c_n
\underbrace{
\left(
\frac{1}{n!} \int_0^\infty  t^n e^{-t} dt
\right)}_{1}
=
\int_0^\infty
\underbrace{\left(\sum_{n=0}^{\infty}\frac{c_n}{n!}\right)}_{\mathrm{Borel\ transform}}
t^n e^{-t} dt.
\end{equation}
To implement this idea in practice, one first defines the Borel
transform of the original factorially divergent series 
\rf{eq:Agradexp} by
%
\begin{equation}
    \mathcal{BA}(\xi)=\sum_{n=1}^\infty\frac{a_n}{n!}\xi^{n}~,
    \label{eq:ABorel}
\end{equation}
%
which defines an analytic function inside a disc of radius 
$|A|$ at the origin. 
The Borel sum of the original divergent series is defined by the inverse Borel
transform 
%
\begin{equation}
    \label{eq:Borelsum}
    \mathcal{A}_{\rm sum}(w)=w\int_\mathcal{C}d\xi\, e^{-w\xi}
    \,\widetilde{\mathcal{BA}}(\xi)~.
\end{equation}
%
The tilde over the Borel transform indicates that the domain where the series
\rf{eq:ABorel} is defined will need to be extended by means of analytic
continuation so that one can find a contour $\mathcal{C}$ which extends to
infinity. 

In most cases of interest one cannot carry out this prescription exactly. Typically, the
number of coefficients $a_n$ which are available in practice is finite, and the
coefficients which are available are often given numerically with some finite
precision. One also has to rely on approximate methods of analytic continuation.
The quality of this procedure is also critically important for the accuracy of
the result of the
resummation~\cite{Costin:2020hwg,Costin:2020pcj,Costin:2021bay,Costin:2022hgc}.

The most straightforward and widely-used way to carry out the required analytic continuation 
is to adopt the Pad\'e approximant
%
\begin{equation}
    \mathcal{BA}_{\rm Pade}(\xi)=\frac{P_m(\xi)}{Q_n(\xi)}~,
\end{equation}
%
where $P_m(\xi)$ and $Q_n(\xi)$ are polynomials of degree $m$, $n$ respectively,
with coefficients properly fitted to match the expansion (\ref{eq:ABorel}). Due
to the approximate nature of this procedure, the singularities of the
analytically continued Borel transform can only be poles.  However, given an
adequate number of terms in the series \rf{eq:Agradexp} and with polynomials of
high enough degree, the poles appear in dense sequences accumulating at the
actual branch points (``condensing'', as it were, along branch cuts). 
This procedure can thus provide a quantitative approximation to the true
singularities of the Borel transform. 

In the case of the MIS gradient expansion, the singularities of the analytically
continued Borel transform the are shown in \rff{fig:borelpadeMIS}.  This pattern
indicates the existence of a branch point at $\xi=A$ (given in
\rf{eq:singulant}) and this can be shown to be related to the large order
behaviour expressed by \rf{eq:dingle}. The fact that this branch point is found
on the real axis means that the integration contour in \rf{eq:Borelsum} must be
deformed to run either below or above the real axis. This leads to a complex
ambiguity of the Borel sum. This ambiguity is in fact cancelled once
contributions from nontrivial transseries sectors are included, and the
imaginary part of the transseries parameter is set correctly.  The consistency
of this procedure relies on the phenomenon of resurgence, which is an intricate
relationship between the expansion coefficients appearing in the different
transseries sectors. For details of these matters we refer the Reader to
Refs.~\cite{Heller:2015dha,Aniceto:2015mto,Basar:2015ava,Aniceto:2022dnm} and
for resurgence in general to Ref.~\cite{Aniceto:2018bis,Mitschi:2016fxp}. 

%%%%%%%%%%%%%%%%%%%%%%%%%%%%%

\begin{figure}[t]
\begin{center}
\includegraphics[width =.6\textwidth]{borelpadeMIS.pdf}
\caption{The poles of the Borel transform \rf{eq:ABorel}.}
\label{fig:borelpadeMIS}
\end{center}
\end{figure}

%%%%%%%%%%%%%%%%%%%%%%5


%%%%%%%%%%%%%%%%%
\subsection{The attractor in HJSW hydrodynamics}
%%%%%%%%%%%%%%%%%%


%%%%%%%%%%%%%%%%%%%%%%%%%

\begin{figure}[t]
\begin{center}
\includegraphics[width =.6\textwidth]{hjswattr.pdf}
\caption{The blue curves depict solutions whose initial
  conditions were set at several values of $w$ between $0.05$ and $0.3$. The red
  curve represents the attractor. The parameter values used when making the plot
were those for \symm\ and  $C_\sigma=0$.}
\label{fig:HJSWattractor}
\end{center}
\end{figure}

%%%%%%%%%%%%%%%%%%%%%%%%%%%

So far this Section has focused on the attractor of MIS theory, but the same ideas
can be applied to other hydrodynamic models discussed in \rfs{sec:MIS}. One
point of interest is that in such models one sometimes encounters
higher-dimensional phase spaces.  For example, this happens in the HJSW model
introduced in Ref.~\cite{Heller:2014wfa}, which leads to a second order equation
replacing \rf{eq:MISAeom}. In consequence, the full phase space is three dimensional.
Explicitly, this relaxation equation reads (see also Ref.~\cite{Florkowski:2017olj})
%
\bel{eq:vcp}
\alpha_1 \pa''+ \alpha_2 \, \pa'^2+\alpha_3 \, \pa'+12 \, \pa^3+\alpha_4 \, \pa^2+\alpha_5 \, \pa+\alpha_6 = 0,
\ee
%
where
%
\bel{vcp.coeffs}
\alpha_1 &=&  w^2 \, (\pa+12)^2,\nn\\
\alpha_2 &=& w^2 \, (\pa+12),\nn\\
\alpha_3 &=& 12 \,  w \,  (\pa+12) \, (\pa+3 \, w \,  \Omega_I),\nn\\
\alpha_4 &=& 48 \,  (3 \,  w\, \Omega_I - 1),\nn\\
\alpha_5 &=& 108 \, \left(- 4\,  C_\eta \, C_{\sigma} + 3 \, w^2 \, \Omega ^2\right),\nn\\
\alpha_6 &=& -864 \, C_\eta \, \left(- 2 \, C_{\sigma}+3\, w\, \Omega^2\right).
\ee
%
At early times
there is a unique power series solution regular at $w=0$:
%
\bel{eq:hjsw.smallw}
\pa(w) =  4 + \f{54 \, C_\eta \, |\Omega|^2- 48 \, \Omega_I}{20 - 9  \, C_\eta \,
  C_{\sigma}} \, w + \dots
\ee
%
This is the attractor, as seen in \rff{fig:HJSWattractor}, where this curve is
plotted in the full phase space. 

At large $w$, the gradient expansion takes the form
%
\bel{eq:hjsw.largew}
\pa(w) =  \f{8 C_\eta}{w} +
\f{16 C_\eta ( 2 \Omega_I- C_{\sigma} )}{3 |\Omega|^2 w^2} + \dots
\ee
%
As expected, the first term captures the shear viscosity, as in MIS theory. The
higher order terms differ from the corresponding expansion given in
Eq.~\eqref{eq:MISlatetime}, \eqref{eq:MISgradevals}.  Similarly to the case of
MIS theory, this series has vanishing radius of
convergence~\cite{Aniceto:2015mto}.  One can use this expansion in conjunction
with Borel summation to obtain a useful estimate of the attractor. We will
return to this point in \rfs{sec:attractorHolo}. 


%%%%%%%%%%%%%%%%
\subsection{Attractors in general frame models}
%%%%%%%%%%%%%%%%


%%%%%%%%%%%%%%%%


\begin{figure}[t]
\begin{center}
\includegraphics[width =.6\textwidth]{nssattr.pdf}
\caption{The blue curves depict solutions whose initial conditions were set at
several values of $w$ between $0.05$ and $0.3$. The red curve represents the
attractor. The parameter values used when making the plot were $C_\eta = 0.08,
C_\tau=0.2, C_\varphi=0.01$.}
\label{fig:NSSattractor}
\end{center}
\end{figure}

%%%%%%%%%%%%%%%%%%%

Attractors have also been studied in hydrodynamic models where the Landau frame
condition has not been
imposed~\cite{Shokri:2020cxa,Noronha:2021syv,Pandya:2021ief,Rocha:2022ind}.
Here we wish to highlight an interesting example of an attractor within a
$3$-dimensional phase space which arises in the general-frame MIS theory of
Ref.~\cite{Noronha:2021syv} (see \rfs{sec:gf}). Imposing the symmetries of
Bjorken flow implies that the energy-momentum tensor contains three functions of
proper time (instead of two, as would be the case had the Landau frame condition
been imposed). This leads to a system of coupled equations for two functions of
$w$, denoted by $\xa,\xb$: 
\be
\label{eq:nssA}
\frac{1}{12}(C_\tau - C_\varphi) w (\xa  + 12) \xa'
-\frac{3}{8} w \xa (\xb  - 4)
+\frac{(C_\tau - C_\varphi)}{3} \xa^2
-\frac{9}{2} w \xb - 12 C_\eta &=& 0,\\
\label{eq:nssB}
\f{1}{12} C_\tau w (\xa  + 12) \xb'
+ \f{1}{3}\xa ( C_\tau \xb+  C_\varphi) +
\f{3}{2} w \xb  &=& 0,
\ee
where the prime denotes differentiation with respect to $w$, and $C_\varphi,
C_\tau$ are dimensionless constants. The functions $\xa, \xb$ replace the
pressure anisotropy in parametrising the dissipative part of the general-frame
energy-momentum tensor and are defined in Ref.~\cite{Noronha:2021syv}. In the
special case where $C_\varphi=0$ these equations admit a solution with
$\xb\equiv 0$ and then \rf{eq:nssA} reduces to the equation satisfied by the
pressure anisotropy in MIS theory~\rf{eq:MISAeom}. The late time asymptotics of
solutions are $\xa \sim 8 C_\eta/w$ and $\xb \sim -63 C_\eta C_\varphi/27 w^2$
for all initial conditions.  

The phase space of solutions in this model is
three-dimensional rather than two-dimensional as in MIS theory. As in the examples
discussed earlier, there is a unique
solution regular at $w=0$ which acts as an attractor, as seen in
\rff{fig:NSSattractor}.


