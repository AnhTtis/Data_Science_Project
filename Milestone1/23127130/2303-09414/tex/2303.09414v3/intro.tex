%%%%%%%%%%%%%%%%%%%%%%
\section{Introduction}
\label{sec:intro}
%%%%%%%%%%%%%%%%%%%%%%


The theory of the strong nuclear interactions, Quantum Chromodynamics, is
beautiful on many levels, one being the simplicity of its formulation. This
simplicity hides a richness of phenomena which remains beyond reach even now,
after decades of research. While the spectrum of hadronic states is often viewed
as a problem solved at least at some level by lattice calculations, the
collective properties which are relevant for the physics of hadronic matter at
finite temperature and density are far less well understood. The main motivation
for this review article comes from studies of Quark-Gluon Plasma (QGP) created
in ultrarelativistic nuclear
collisions~\cite{Kolb:2003dz,Heinz:2013th,Busza:2018rrf,Schenke:2021mxx}. Such
enquiries are of great intrinsic interest, as they address the nature of
Yang-Mills theory itself.  They also have wide ranging implications in diverse
areas of physics, such as nonequilibrium statistical physics, nuclear physics
and astrophysics~\cite{Lovato:2022vgq,Sorensen:2023zkk,Achenbach:2023pba}. 

The heavy-ion collision (HIC) programme is an experimental study of the
properties of strongly-interacting matter, currently pursued at RHIC and the
LHC. Extracting physical properties of QCD matter from collider data is a
formidable challenge.  A crucial element of the analysis is the application of
hydrodynamic models; usually these are variants of the Mueller-Israel-Stewart
theory (MIS)~\cite{Muller:1967zza,Israel:1976tn,Israel:1979wp}. The traditional
formulation of relativistic hydrodynamics assumes that the system under
consideration is approximately in a state of local thermodynamic equilibrium.
The leading order description is then the theory of perfect fluids, and
dissipative effects are accounted for by augmenting the perfect fluid evolution
by adding terms involving gradients of the hydrodynamic variables.  In order to
explain the observed signs of fluidity ({\it e.g.} elliptic flow), the
hydrodynamic stage of simulations has to begin at rather early times, after an
interval of less than 1~fm/c, when the system is still very anisotropic.  Even
though the drop of QGP is not at all close to local equilibrium, hydrodynamic
modelling is very
successful~\cite{Heinz:2002un,Romatschke:2007mq,Teaney:2003kp,Schenke:2021mxx}.
This is a puzzle which touches on foundational questions in
relativistic fluid dynamics.  The discovery of far-from-equilibrium attractors
is a possible resolution of this puzzle~\cite{Heller:2015dha}. 

Theoretical analysis of HIC began already in the 1970s (see e.g.
Ref.~\cite{Yagi:2005yb}). A
crucially important step was made in the seminal work of 
Bjorken~\cite{Bjorken:1982qr}, who pointed out that in a certain kinematic
regime one should expect the initial conditions, as well as subsequent dynamics,
to be approximately invariant under Lorentz boosts along the collision axis.
Supplemented with the assumption of conformal invariance, this has opened the
door to analytic calculations in a situation where one might have thought
numerical computations were the only possible approach. The results of these
calculations have limited applicability due to the strong symmetry assumptions
explained in more detail in the following Section, but they have led to a wealth
of insights. One of them, which has emerged in the past few years, is the notion
of hydrodynamic attractors. 

The term ``attractor'' has a number of meanings. 
In the present context it is
best to think of hydrodynamic attractors as submanifolds of the phase space of
the theory under consideration which are approached asymptotically in the course
of dissipative evolution. The appearance of such attractors at late time is
entirely expected, but it was found that in some cases this attractor extends to
early times, when the system is very far from equilibrium~\cite{Heller:2015dha}.
Such far-from-equilibrium attractors have been identified in many model
systems~\cite{Romatschke:2017vte,Strickland:2018ayk,Almaalol:2020rnu,Noronha:2021syv}
and it is essentially clear that their origin at early times is kinematical:
they arise due to the strong longitudinal expansion~\cite{Blaizot:2017ucy}.
This effect appears in any theory or model of equilibration, be it a
hydrodynamic or kinetic theory model, or presumably QCD itself. Since attractor
behaviour eliminates much of the complexity of initial states as well as of the
dynamics, it may be feasible to match the attractor of QCD to the attractor of a
much simpler phenomenological model, such as the widely used MIS model of
hydrodynamics.
This provides a possible explanation of the success of hydrodynamic simulations
in the description of heavy-ion collisions.  At present, one cannot claim this
with a high degree of certainty, since this explanation relies on studies
involving rather strong symmetry assumptions.  They are valid to some degree at
the early stages of QGP evolution, but it is not yet known to what extent their
violation affects the robustness of hydrodynamic attractors. 
Nevertheless, we regard this possibility with a degree of confidence. 

In this article we review the theoretical underpinnings of hydrodynamic
attractors as well as some applications which are directly relevant to
phenomenological studies.  We hope that our article will be somewhat
complementary to existing reviews, such as
Refs.~\cite{Florkowski:2017olj,Berges:2020fwq,Soloviev:2021lhs}.  We begin, in
\rfs{sec:HIC}, with a brief account of the physical setting of heavy-ion
collisions and the emergence of boost-invariance, a symmetry property which
plays a crucial role in the entire picture. In \rfs{sec:MIS} we emphasise the
conceptual difference between hydrodynamics, understood as an asymptotic
statement about equilibrating systems, and hydrodynamic models which provide a
dynamical description with appropriate asymptotics. Attractors are then
introduced, first in the context of hydrodynamic models in \rfs{sec:attractors},
and then in the framework of kinetic theory in \rfs{sec:attractorKT}. In
\rfs{sec:attractorHolo} we turn to the example of \sym\ supersymmetric
Yang-Mills theory (SYM), which has historically played a crucial role in the
paradigm shift which occurred over the last decade, having provided (thanks to
the AdS/CFT correspondence) a theoretical laboratory based on first principles
where the transition to hydrodynamic behaviour could be investigated. In
\rfs{sec:PhaseSpace} we describe the phase space approach to attractors, this
time aiming for a treatment independent of any special choice of variables. Such
an approach is potentially useful in identifying attractors without relying on
simplifying symmetry assumptions.  \rfs{sec:prehydro} reviews some recent
quantitative applications of attractors to the modelling of heavy ion
collisions.  In \rfs{sec:beyond} we summarise what has been learnt from studies
of conformal Bjorken flow and review some results concerning attractors in
models where some of the symmetry assumptions have been relaxed, specifically by
incorporating the breaking of conformal symmetry or the inclusion of transverse
dynamics.  Finally, \rfs{sec:Out} offers some opinions on research directions
one can envisage following from the developments discussed in this review. 


