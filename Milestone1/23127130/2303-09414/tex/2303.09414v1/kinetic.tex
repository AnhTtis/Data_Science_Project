%%%%%%%%%%%%%%%%%%%%%%%%
\section{Attractors from Kinetic Theory}
\label{sec:attractorKT}
%%%%%%%%%%%%%%%%%%%%%%%%

The discovery of attractors in hydrodynamic models can be viewed as a strong
indication that similar phenomena should occur also in more elaborate
microscopic theories. This supposition has by now been confirmed in numerous
studies discussed further in this review.  The simplest class of models, whose
complexity goes beyond what is discussed in the previous Section, are models of
kinetic theory, where attractors have been identified and studied in many interesting cases~\cite{Heller:2016rtz,Blaizot:2017ucy,Romatschke:2017vte,Behtash:2017wqg,Heller:2018qvh,Behtash:2018moe,Strickland:2018ayk,Behtash:2019txb,Denicol:2019lio,Strickland:2019hff,Behtash:2019qtk,Kurkela:2019set,Chattopadhyay:2019jqj,Almaalol:2020rnu,Heller:2020anv,McNelis:2020jrn,Kamata:2020mka,Blaizot:2020gql,Alalawi:2022pmg,Jaiswal:2022udf}. 

Kinetic theory is 
based on the classical notion of a single particle distribution
function $f(x, p)$ obeying the Boltzmann equation
%
\begin{equation}
    p^\mu\partial_\mu f(x,p) = \mathcal{C}[f]~.
    \label{eq:Boltzmann}
\end{equation}
%
The collision kernel appearing on the right-hand side of \rf{eq:Boltzmann} can
in general be very complicated, since in principle it should account for all
scattering processes which can occur in a given theory. In practice, only a
subset is accounted for, or some other form of approximation has to be adopted
to capture essential features of the underlying microscopic theory. Here we will
review kinetic theory attractors assuming one of two options: the relaxation
time approximation (RTA)~\cite{Anderson:1974abc} and the Effective Kinetic
Theory for Quantum Chromodynamics  (EKT)~\cite{Arnold:2002zm}. 


%%%%%%%%%%%%%%%%%%%%%%%%%%%%%%%%%%%%%%%%%%%%%%
\subsection{Boost invariant flow in RTA}
%%%%%%%%%%%%%%%%%%%%%%%%%%%%%%%%%%%%%%%%%%%%%%

A significant simplification, which has been the subject of numerous studies 
is the relaxation time approximation, 
where the collision kernel in~\rf{eq:Boltzmann} is replaced by 
%
\begin{equation}
    \mathcal{C}[f]=p^\mu u_\mu \frac{f-f_{\rm eq}}{\tau_R}~.
    \label{eq:C_RTA}
\end{equation}
%
Here $\tau_R$ is a momentum-independent relaxation time and $f_{\rm eq} =
\exp\left(-\frac{p_\mu u^\mu}{T}\right)$ is the equilibrium distribution
function. The resulting equation is linear in $f(x,p)$ and is much easier to
work with.  Recently, this ansatz has been generalised in various
ways~\cite{Rocha:2021zcw,Rocha:2021lze,Dash:2021ibx,Rocha:2022ind,Denicol:2022bsq}. 

The Boltzmann equation in the RTA applied to Bjorken flow is a
quasi-analytically solvable problem \cite{Baym:1984np,Florkowski:2013lza} which
provides a very useful environment for testing ideas of nonequilibrium dynamics. 
Since the one particle distribution function $f(x,p)$ is a scalar, boost
invariance implies that it may depend only on variables invariant under
longitudinal boosts: 
$\tau$, $p_T$, and $W=t p_L- zE$, 
where $E\equiv p_0=\sqrt{p_T^2+p_L^2+m^2}$ is the particle's energy~\cite{Bialas:1984wv,Bialas:1987en}
\footnote{The boost-invariance of $W$ is a consequence of the transformation law 
$(E,p_L)\mapsto(E\cosh(y)-p_L\sinh(y),p_L\cosh(y)-E\sinh(y))$, and analogously for $(t,z)$.}.
%
With the help of
$W$ one can define $ v(p_T,W,\tau)=Et-p_L z =\sqrt{W^2+(p_T^2+m^2)\tau^2}~,$
which allows us to express energy and longitudinal 
momentum of particles of mass $m$ in terms of boost-invariant variables
%
\begin{equation}
    E=\frac{v t+W z}{\tau^2}~, \qquad 
    p_L=\frac{W t+ vz}{\tau^2}~.
\end{equation}
%
In terms of $\tau$, $W$ and $v$ we can write 
$p^\mu\partial_\mu f=\frac{v}{\tau}\partial_\tau f$,
$p_\mu u^\mu =\frac{v}{\tau}$,
and the boost-invariant Boltzmann equation in the RTA takes the form
\cite{Strickland:2018ayk,Baym:1984np,Florkowski:2013lza}
%
\begin{equation}
    \partial_\tau f(\tau,W,p_T)=\frac{f_{\rm eq}(\tau,W,p_T)-f(\tau,W,p_T)}{\tau_R}~.
    \label{eq:Boltzmann_RTA}
\end{equation}
%
The equilibrium distribution function is explicitly given by
%
\begin{equation}
  \label{biedf}
    f_{\rm eq}(\tau,W,p_T)=\exp\left(-\beta u_\mu p^\mu\right)
    =\exp\left(-\frac{\sqrt{W^2+p_T^2 \tau^2}}{T(\tau) \tau}\right)~,
\end{equation}
%
where we have set $m=0$, since for the time being we will concentrate on models
respecting conformal symmetry.

In order to obtain a closed system of equations one needs a way to determine the
effective temperature $T(\tau)$ appearing in \rf{biedf}. This can be achieved by
imposing the Landau matching condition, which states that local energy density
determined by the function $f(\tau,W,p_T)$ should be equal to the equilibrium
configuration with temperature $T(\tau)$.
In order to do that in a Lorentz invariant way one uses the measure  
%
\begin{equation}
    dP = \frac{d^4p}{(2\pi)^4}2\pi\delta(p^2)2\theta(p^0)=\frac{dp_L}{(2\pi)^3p^0}d^2p_T=\frac{dW d^2p_T}{(2\pi)^3v}~,
\end{equation}
%
and the desired matching condition is expressed as
%
\begin{equation}
    \edens(\tau)= \int dP(p_\mu u^\mu)^2 f(\tau,W,p_T) = \frac{3T(\tau)^4}{\pi^2}~.
    \label{eq:LandauMatching}
\end{equation}
%
A beautiful fact of life is that Eq.~(\ref{eq:Boltzmann_RTA}) admits the general
solution~\cite{Baym:1984np,Florkowski:2013lza} 
%
\begin{equation}
    f(\tau,W,p_T)=D(\tau,\tau_0)f_0(W,p_T)+\int_{\tau_0}^\tau
    \frac{d\tau'}{\tau_R(\tau')} D(\tau,\tau')f_{\rm eq}(\tau',W,p_T)~,
    \label{eq:RTA_solution}
\end{equation}
%
where $f_0(W,p_T)$ is the initial distribution function at $\tau=\tau_0$, and $D(\tau_2,\tau_1)$ is 
given by 
\begin{equation}
    D(\tau_2,\tau_1)=\exp\left[-\int_{\tau_1}^{\tau_2}\frac{dt}{\tau_R(t)}\right]~.
\end{equation}
%
The first term in Eq.~(\ref{eq:RTA_solution}) expresses free streaming, which
dominates at early times, while the second term is captures 
relaxation toward local equilibrium, which is controlled by $\tau_R$.

%%%%%
\subsection{The gradient expansion}
%%%%%

The Boltzmann equation in the RTA \rf{eq:Boltzmann_RTA} can be used to calculate
the distribution function in the gradient expansion. The most direct way to
proceed is to solve it iteratively starting with the equilibrium distribution,
thus implementing the Chapman-Enskog expansion (see
e.g.~\cite{DeGroot:1980dk,Jaiswal:2013npa}). One can then calculate the late
proper time expansion of the effective temperature using \rf{eq:LandauMatching}
and translate it into a series for the pressure anisotropy, wchich can be
written in the form of \rf{eq:MISlatetime}, with the leading coefficients given
by~\cite{Heller:2016rtz} 
%
\begin{equation}
    a_1=\f{8}{5}\ \gamma,\quad
    a_2=\f{32}{105}\ \gamma^2, \quad
    a_3 = - \f{416}{525} \, \gamma^3~.
\end{equation}
%
This can be matched to the gradient expansion of any hydrodynamic
model~\cite{Florkowski:2016zsi}. Depending on the choice of model, one or more terms may be
matched. In the case of MIS theory, a comparison with \rf{eq:MISgradevals} shows
that to match RTA kinetic theory one needs $C_\eta = \gamma/5$. 

The large order behaviour of the gradient expansion reveals a nonhydrodynamic
mode with the expected relaxation time, but the results are actually much more
complex, because of the wealth of possible initial conditions in kinetic theory, where
the initial state is specified by the distribution function at some initial
time. This will not be discussed further here, but some details can be found
in Refs.~\cite{Heller:2016rtz,Heller:2018qvh}.  Note also that the spectrum of
nonhydrodynamic modes in RTA kinetic theory is very different from that of
hydrodynamic models~\cite{Romatschke:2015gic}.


%%%%%
\subsection{The initial value problem}
%%%%%

The additional input needed to evaluate \rf{eq:RTA_solution} is an initial
condition. An important example, used below, is the  
Romatschke-Strickland parametrisation \cite{Romatschke:2003ms}
%
\begin{equation}
    f_0(W,p_T)=
    \exp\left[-\frac{\sqrt{(p\cdot u)^2+\xi_0(z\cdot p)^2}}{\Lambda_0}\right]
    =
    \exp\left[-\frac{\sqrt{(1+\xi_0)W^2+p_T^2 \tau^2_0}}{\Lambda_0\tau_0}\right]~,
    \label{eq:f0RTA}
\end{equation}
%
where $-1<\xi_0<\infty$ measures initial momentum space anisotropy,
$z_\mu=(\frac{z}{\tau},0,0,\frac{t}{\tau}),$ and $\Lambda_0$ determines the
characteristic energy scale. Using this form, one can explicitly evaluate the
initial energy density
%
\begin{equation}
    \edens(\tau_0)=\int dP (p\cdot u)^2 f_0(\tau_0,W,p_T)=
    \frac{3T_0^4}{\pi^2}\frac{\mathcal{H}(\frac{\alpha_0\tau_0}{\tau_0})}{\mathcal{H}(\alpha_0)}~,
\end{equation}
%
where $\alpha_0=(1+\xi_0)^{-\frac{1}{2}}$.  Using the Landau matching condition,
along with the general solution presented in Eq. (\ref{eq:RTA_solution}) one
then 
obtains an integral equation for the $T(\tau)$, i.e., the effective temperature
as a function of proper time $\tau$~\cite{Baym:1984np,Florkowski:2013lza}
%
\begin{equation}
    T(\tau)^4=D(\tau,\tau_0)T_0^4\frac{\mathcal{H}\left(\frac{\alpha_0\tau_0}{\tau}\right)}{\mathcal{H}(\alpha_0)}+\int_{\tau_0}^\tau \frac{d\tau'}{2\tau_{\rm eq}(\tau')}D(\tau,\tau')T(\tau')^4\mathcal{H}\left(\frac{\tau'}{\tau}\right)~,
    \label{eq:Tint}
\end{equation}
%
where 
%
\begin{equation}
    \mathcal{H}(y)=y\int_0^\pi \sin(\phi)\sqrt{y^2\cos^2(\phi)+\sin^2(\phi)} d\phi~.
\end{equation}
%
Equation (\ref{eq:Tint}) can be solved in an iterative manner, with some initial
temperature profile $T(\tau)$ \cite{Strickland:2018ayk}. Knowing the temperature
as a function of $\tau$ one can carry out the integral in \rf{eq:RTA_solution}
to obtain the full distribution function $f(\tau,W,p_T)$.



%%%%%%%%%%%%%%%%%%%%%%%%%%%%%%%%%%%%%%%%%%%%%%%%%%%
\subsection{Attracting behaviour of the distribution function}
\label{subsec:RTAattr}
%%%%%%%%%%%%%%%%%%%%%%%%%%%%%%%%%%%%%%%%%%%%%%%%%%%

To establish the existence of an attractor in kinetic theory one may adopt one of two
approaches. The first is to look at the moments of the distribution function
\cite{Blaizot:2017ucy,Kurkela:2019set,Strickland:2019hff,Almaalol:2020rnu} while the second looks for attractor behaviour of the distribution function itself
\cite{Strickland:2018ayk}.  In this subsection we will follow the latter
approach, while the former will be described in Sec.~\ref{subsec:ETK} in the
context of more realistic approximation to the collisional kernel~\cite{Almaalol:2020rnu}. 

%%%%%%%%%%%%%%%%


\begin{figure}
\begin{center}
\includegraphics[height =.4\textheight]{fAttractor.pdf}
\includegraphics[height =.4\textheight]{fAttractor2.pdf}
\caption{Quantitative approach towards a hydrodynamic attractor $\alpha_0\simeq0.0025$ (solid black line) in the distribution function. Initial conditions are $T_0=1$~GeV at $\tau_0=0.1$~fm/c and $0.1\leq\alpha_0\leq1.5$ for dashed/dotted color lines.   
The first and third rows show  $f(p_T=0,p_L)$,
while second and
fourth show $f(p_T,p_L=0)$.
First two columns are for lower
momenta $p_i/T\leq3$ while second two columns are for higher momenta
$p_i/T\leq40$ ($i=x,y,z$). Different columns represent different time
$\overline{w}$ instances marked in the top. The 
    scaled variable $\overline{w}=\tau/\tau_R = \frac{\tau
T(\tau)}{5\bar{\eta}}$, which differs by a constant factor from the variable
$w=\tau T(\tau)$ introduced earlier. 
Plots from Ref. \cite{Strickland:2018ayk}.}
\label{fig:fAttractor}
\end{center}
\end{figure}


%%%%%%%%%%%%%%%

In order to demonstrate that the full distribution function $f(\tau,W,p_T)$ has
an attractor one numerically solves the RTA Boltzmann equation
(\ref{eq:Boltzmann_RTA}) for the class of initial conditions parametrized by
Eq.~(\ref{eq:f0RTA}) and identifies the attractor by a "slow roll" approximation
\cite{Romatschke:2017vte,Strickland:2018ayk}:
%
\begin{equation}
    \left.\mathcal{A}'(\tau T)\right|_{\tau=\tau_0}\propto\left.\frac{\edens\partial_\tau\edens+\tau\edens\partial^2_\tau\edens-\tau(\partial_\tau\edens)^2}{\edens^2}\right|_{\tau=\tau_0}~,
\end{equation}
%
with $T(\tau_0)=1$~GeV and $\tau_0=0.1$~fm/c \cite{Strickland:2018ayk}.  Solving
$\mathcal{A}'|_{\tau=\tau_0}=0$ for $\alpha_0$ singles out the value of the initial anisotropy
parameter $\alpha_0\approx0.0025$, which determines the attractor solution. 


The approach to the attractor for different anisotropic initial configurations
is shown in Fig.~\ref{fig:fAttractor}.  It is apparent that the infrared part of
the distribution function (the region close to $p=0$) approaches the attractor
earlier than the ultraviolet part, which is a manifestation of the ``bottom-up''
scenario characteristic of weakly coupled systems \cite{Baier:2000sb}.  The
approach to the attractor is also slower in the transverse direction ($p_z=0$)
than in the longitudinal direction ($p_T=0$).  Note also that in some momentum
regions the attractor is approached from below, while in others it is approached
from above.






%%%%%%%%%%%%%%%%%%%%%%%%%%%%%%%%%%%%%%%%%%%%
\subsection{Weakly coupled QCD }
\label{subsec:ETK}
%%%%%%%%%%%%%%%%%%%%%%%%%%%%%%%%%%%%%%%%%%%%

The discussion of previous section relied on the RTA collisional kernel.
An important question is whether similar results can be established within more
realistic models. Recently, this issue was addressed in the context of Effective
Kinetic Theory (ETK) of QCD~\cite{Arnold:2002zm}. The EKT Boltzmann equation
for a pure gluon system reads
%
\begin{equation}
    -\partial_\tau f +\frac{p_z}{\tau}\partial_{p_z} f =
    \mathcal{C}_{1\leftrightarrow2}[f]+\mathcal{C}_{2\leftrightarrow2}[f]~,
    \label{eq:BoltzmannEKT}
\end{equation}
%
where the inelastic $ \mathcal{C}_{1\leftrightarrow2}$ and elastic $
\mathcal{C}_{2\leftrightarrow2}$ collisional terms include physics of dynamical
screening and Landau-Pomaranchuk-Migdal damping. 
Although EKT is does not
account for the full complexity of QCD, for isotropic systems it incorporates
the leading $\alpha_s$-order description and has been extensively used to
address off-equilibrium perturbative QGP dynamics~\cite{Kurkela:2015qoa,Du:2020dvp,Du:2020zqg}.

To study the process of equilibration, Ref.~\cite{Almaalol:2020rnu} considers
the set of moments of the distribution functions defined by
%
\begin{equation}
    \mathcal{M}^{nm}(\tau):=\int\frac{d^3p}{(2\pi)^3}p^{n-1}p_z^{2m}f(\tau,\bf p)~,
\end{equation}
%
where $p=|{\bf p}|$. In terms of these, the energy density of a massless
particle gas is $\edens=\mathcal{M}^{20}$, particle density is
$n=\mathcal{M}^{10}$, while the longitudinal pressure 
reads $P_L=\mathcal{M}^{01}$.
The pressure anisotropy can be expressed in terms of these 
moments as
%
\begin{equation}
    \mathcal{A}=3\frac{\pT(\tau)-\pL(\tau)}{\edens(\tau)}=
    \frac{3}{2}-\frac{9}{2}\frac{\mathcal{M}^{01}(\tau)}{\mathcal{M}^{20}(\tau)}~.
\end{equation}
The distribution function can be obtained numerically by solving the Boltzmann
equation~(\ref{eq:BoltzmannEKT}) utilizing the algorithm described in
Ref.~\cite{AbraaoYork:2014hbk,Kurkela:2015qoa}.

Two classes of initial conditions were 
considered in Ref.~\cite{Almaalol:2020rnu}.  The first one is given by a
spheroidally deformed thermal distribution function given by
%
\begin{equation}
  f_{0,\rm RS}(p)=\frac{1}{\exp\left(\frac{\sqrt{p^2+\xi_0^2p_z^2}}{\Lambda_0}\right)-1},
  \label{eq:f0RS}
\end{equation}
%
where $-1<\xi_0<\infty$, as in the RTA case, parameterises the initial momentum
anisotropy, while $\Lambda_0$ sets the initial energy scale.
The second group of initial conditions is given by the non-thermal CGC-motivated
distribution function, explicitly written as 
%
\begin{equation}
    f_{0, \rm CGC}(p)=\frac{2A}{\lambda_{\rm YM}}\frac{Q_0}{\sqrt{p^2+\xi_0^2p_z^2}}\exp\left(-\frac{2}{3}\frac{p^2+\xi_0^2 p_z^2}{Q_0^2}\right)~,
    \label{eq:f0CGC}
\end{equation}
%
where the scale $Q_0$ is related to the QCD saturation scale $Q_0=\langle
 p_T\rangle_0\approx1.8Q_s$ \cite{Lappi:2011ju}.
Furthermore, $\lambda_{\rm YM}=g_{\rm YM}^2N_c$  is the 't Hooft coupling. 
The normalization constant $A$
is fixed by matching the initial energy density with the predictions of
classical Yang-Mills theory $\tau_0\edens_0=0.358\nu_{\rm
eff}\frac{Q_s^3}{\lambda_{\rm YM}}$ \cite{Lappi:2006hq}. For both sets of
initial conditions, the scales $\Lambda_0$ and $Q_0$
play a  role similar to the temperature in a thermal distribution, i.e., they
determine which portion of momentum space is occupied. The fact that
distribution in \rf{eq:f0CGC} is inversely proportional to $\lambda_{\rm YM}$
reflects the overpopulation of gluons determined by multiple low energy
scatterings at initial times. 

The plots in Fig.~\ref{fig:EKTattractor} show the evolution of three sample moments
with different initial conditions,
parametrized by Eq.~(\ref{eq:f0RS}) and \rf{eq:f0CGC}.  
% The
% horizontal line is in units $\tau/\tau_R(\tau)$ which measures system's age in
% relaxation time units given by $\tau_R=4\pi\overline{\eta}/T(\tau)$. 
As 
seen in the upper panel of Fig. \ref{fig:EKTattractor}, all sampled initial
conditions merge into one universal line, the hydrodynamic attractor, before
they are well approximated by the viscous hydrodynamics. This happens on
the time scale $\tau/\tau_R\sim0.5$ common for all three moments of the
distribution function. Since this result holds also for higher moments, it is a
strong indication that, similarly to the RTA case, the attractor is present in the full
one particle distribution function \cite{Alalawi:2020zbx}.

%%%%%%%%%%%%%%%%%%%%%%%%%%%


\begin{figure}
\begin{center}
\includegraphics[height =.17\textheight]{fig1.pdf}
\includegraphics[height =.17\textheight]{fig2.pdf}
\caption{Evolution of the scaled moments 
    $\overline{\mathcal{M}}^{nm}(\tau)=
    \mathcal{M}^{nm}(\tau)/\mathcal{M}^{nm}_{\rm eq}(\tau)$
computed as functions of rescaled time 
    $\overline{w}=\tau/\tau_R = \frac{\tau T(\tau)}{5\bar{\eta}}$ 
    for various initial conditions. Upper panel: fixed
    initial time different initial momentum space anisotropy. Lower panel:
    different initial times with fixed initial momentum anisotropy. The value
    the 't Hooft coupling used here is $\lambda_{\rm YM}=10$, which corresponds
to shear viscosity $\eta/s\approx0.63$ \cite{Arnold:2003zc,Keegan:2015avk}.  
The plots are taken from Ref.~\cite{Almaalol:2020rnu}.}
\label{fig:EKTattractor}
\end{center}
\end{figure}


%%%%%%%%%%%%%%%%%%%%%%%%%%%%

The plots in the lower panel of Fig.~\ref{fig:EKTattractor} show evolution of
moments initialised at successively smaller initial times $\tau_0$.  As apparent
from the figure, earlier initialisation leads to faster decay to the attractor,
suggesting a scaling dependence on $\tau_0$ at early times.  At late times, both
RS and CGC initial conditions follow the same attractor, showing that details of
hydrodynamic evolution are insensitive not only to the initial pressure
anisotropy but also to microscopic features such as momentum distribution and
initial occupancy. While true at late times, this need not be the case
at very early times where different models predict different attractors, as also
indicated in Fig.~\ref{fig:EKTattractor}. This will be discussed further in
\rfs{sec:prehydro}.






