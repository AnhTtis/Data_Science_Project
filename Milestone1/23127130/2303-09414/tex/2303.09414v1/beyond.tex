%%%%%%%%%%%%%
\section{Beyond conformal Bjorken flow}
\label{sec:beyond}
%%%%%%%%%%%%%


%%%%%%%%%%%%%
\subsection{The origin of attractor behaviour}
%%%%%%%%%%%%%


%%%%%%%%%%%%%%%


\begin{figure}
\centering
\includegraphics[height =.35\textheight]{SlowRoll.pdf}
\caption{Hydrodynamic attractors in different theories.
Plot form Ref.~\cite{Romatschke:2017vte}.
}
\label{fig:attractors}       
\end{figure}

%%%%%%%%%%%%%%%%

We have seen that in the case of Bjorken flow in conformal models of
equilibration one can identify hydrodynamic attractors which extend into the
far-from-equilibrium region. This is illustrated in Fig.~\ref{fig:attractors}
which presents the approximate attractors obtained for three different
microscopic models. The appearance of attractors in these very different systems
suggests that they are a generic feature of conformal theories undergoing
Bjorken flow and are ultimately due to the specific kinematics of the earliest
stages of ultrarelativistic heavy-ion collisions.  These kinematic circumstances
lead to simplifying symmetry assumptions such as boost invariance, conformal
symmetry and the suppression of transverse dynamics.  The resolution of the
early thermalisation puzzle based on the notion of attractors relies on these
assumptions remaining approximately valid for a sufficiently long time. 


The physics of a heavy ion collision can be pictured as a competition between
the longitudinal expansion resulting from the initial conditions and the
interactions which drive the system toward equilibrium~\cite{Blaizot:2017ucy}.
While the kinetic theory perspective makes this very explicit, the presence of
an early-time, expansion dominated phase followed by a late-time regime
interpreted in terms of nonhydrodynamic mode decay is also apparent in
hydrodynamic models, as discussed in \rfs{sec:MISearly}. Recently this point was
amplified in Refs.~\cite{Kurkela:2019set,Heller:2020anv}, where these two
regimes were clearly distinguishable. As shown in \rff{fig:timescales}, at late
times generic solutions approach the attractor exponentially, with the rate set
by the nonhydrodynamic mode frequency at vanishing wave vector.  This
interpretation is consistent with linear response, and has been verified in many
examples.  However, the behaviour at very early times, at least in the case of
kinetic theory and hydrodynamic models, is not exponential, but follows a power
law which is independent of the transport coefficients. The compelling
explanation of this behaviour is that it is a consequence of the longitudinal
expansion dominating over any transverse dynamics.  

To assess the relevance of far-from-equilibrium hydrodynamic attractors to the
physics of QGP at early times one needs to understand primarily the effects of
conformal symmetry breaking and the onset of transverse dynamics. While a full
exploration of these important issues remains a task for the future, a few notable
results concerning these effects already exist in the
literature~\cite{Romatschke:2017acs,Chattopadhyay:2021ive,Jaiswal:2021uvv,Chen:2021wwh,Kamata:2022jrc,Jaiswal:2022udf,Chattopadhyay:2022sxk,Alalawi:2022pmg,Alqahtani:2022dfm} and some of them will be
reviewed in this Section. 

%%%%%%%%%%%%%%%%%%

\begin{figure}
\begin{center}
\includegraphics[height = .4\textheight]{WhatAttracts.pdf} 
\caption{Log-linear plot showing different attraction mechanisms: expansion domination at early times, non-hydro mode decay at late times \cite{Kurkela:2019set,Blaizot:2017ucy}.}
\label{fig:timescales}
\end{center}
\end{figure}

%%%%%%%%%%%%%%%%


%%%%%%%%%%%%
\subsection{Breaking conformal symmetry}
%%%%%%%%%%%%

The assumption of approximate conformal symmetry is valid in QCD at
sufficiently high energies. Systematic theoretical studies and detailed
comparison to available experimental data have provided constraints on the
parameters appearing in the hydrodynamic description of QGP evolution, notably
the bulk viscosity which is a sign of departures from conformality. The results
of a multiparameter Bayesian fit of Ref.~\cite{Nijs:2020ors,Nijs:2020roc} show
that the effects of bulk viscosity are subdominant relative to shear, making the
assumption of conformal invariance at early times look plausible. To quantify
this effect in the pre-hydrodynamic stage the easiest thing to do is to assume
that the free-streaming particles move with some effective velocity $v_{\rm
fs}<1$, where $v_{\rm fs}=1$ is conformal.  Experimental data suggest that
$v_{\rm fs}\approx0.82$ \cite{Nijs:2020ors,Nijs:2020roc}, indicating some
departure from conformality. 
Other recent studies which provide evidence for the effects of conformal
symmetry breaking at the prehydrodynamic stage are reported in
Refs.~\cite{daSilva:2022xwu,NunesdaSilva:2020bfs}.  

%%%%%%%%%%%%

\begin{figure}
\begin{center}
\includegraphics[height = .21\textheight]{etaovers2.pdf} 
\includegraphics[height = .21\textheight]{zetaovers2.pdf}
\caption{Values of the shear (left) and the bulk (right) viscosity extracted from a Bayesian  analysis of a $20$ parameter model of a HIC from Ref.~\cite{Nijs:2020ors,Nijs:2020roc}.
The horizontal black line on the left panel is the holographic prediction~\cite{Kovtun:2004de}.
}
\label{fig:BayesianTransp}
\end{center}
\end{figure}


%%%%%%%%%%%%%%%%%


An important difference between conformal and non-conformal Bjorken flow is that
in the latter case the energy momentum tensor \rf{eq:Tmn} contains an additional
degree of freedom, which (as discussed in \rfs{sec:confsym}) is eliminated by
the tracelessness condition for conformal systems.  Attractors in Bjorken flow
without conformal symmetry were recently investigated in
Refs.~\cite{Chattopadhyay:2021ive,Jaiswal:2021uvv,Chen:2021wwh,Kamata:2022jrc,Chattopadhyay:2022sxk,Alalawi:2022pmg}.
These articles have focused on a particular nonconformal model of kinetic theory
in the RTA, where conformal symmetry is broken due to quasiparticles of
nonvanishing mass $m$~\cite{Jaiswal:2014isa,Denicol:2014vaa}; this model also
assumes a constant relaxation time $\tau_R$.
Refs.~\cite{Chattopadhyay:2021ive,Jaiswal:2021uvv,Chen:2021wwh} have identified
a free-streaming far-from-equilibrium attractor in this model, however the
hydrodynamic description employed there showed only a late-time attractor (in
the near-equilibrium,  hydrodynamic region).  Here we will review the analysis
of Ref.~\cite{Jaiswal:2022udf} which also studies this model, generalising the
very suggestive approach to conformal Bjorken flow in RTA kinetic theory
developed in
Refs.~\cite{Blaizot:2017ucy,Blaizot:2018rft,Blaizot:2019scw,Blaizot:2021cdv}.
This analysis leads to a different hydrodynamic description of the system, which
was found to reproduce the free-streaming attractor seen at the level of kinetic
theory.  

The basic idea of
Refs.~\cite{Blaizot:2017ucy,Blaizot:2018rft,Blaizot:2019scw,Blaizot:2021cdv} is
to convert the Boltzmann kinetic equation to an infinite hierarchy of coupled
ordinary differential equations describing a set of moments of the distribution
function \footnote{In this section we use the notation $\int_{\bf
    p}\equiv\int\frac{d^3p}{(2\pi)^3}$.
}
%
\be
\LL_n &\equiv&  \int_{\bf p} p_0 \ P_{2n}(\cos \psi) \ f(\tau, p)~, 
\qquad \forall n \geq 0
\ee
%
where $\cos\psi=p_z/p_0=v_z$ is constituent's velocity along the $z-$direction,
and $P_{2n}$ are Legendre polynomials of degree $2n$. 
These moments satisfy the following infinite hierarchy of equations
which can be derived from the RTA Boltzmann equation \rf{eq:Boltzmann_RTA} 
%
\begin{subequations}
\label{eq:byh}
\begin{align}
\pd{\LL_0}{\tau} =& -\frac{1}{\tau} \left( a_0 \LL_0 + c_0 \LL_1 \right) \,,\\
\pd{\LL_n}{\tau} =& -\frac{1}{\tau} \left( a_n \LL_n + b_n \LL_{n-1} + c_n \LL_{n+1} \right) 
	- \frac{\left( \LL_n - \LL_n^{\rm eq} \right)}{\taur}  \, ,
\quad \forall n \geq 1 \qquad
\end{align}
\label{eq:L}
\end{subequations}
%
where the coefficients $a_n,b_n,c_n$ are known explicitly~\cite{Blaizot:2017ucy,Jaiswal:2022udf}
%
\begin{equation}
    a_n=\frac{2(14n^2+7n-2)}{(4n-1)(4n+3)}, \quad
    b_n=\frac{2n(2n-1)(3n+3)}{(4n-1)(4n+1)}, \quad
    c_n=\frac{(1-2n)(2n+1)(2n+2)}{(4n+1)(4n+3)}~,
\end{equation}
%
and are determined entirely by the free-streaming part of the kinetic equation. The
moments $\mathcal{L}_n^{\rm eq}$ are computed with the Boltzmann equilibrium
distribution function $f_{\rm eq}(p_0/T)=\exp(-p_0/T)$, so 
for example $\mathcal{L}_1^{\rm eq}=\frac{1}{2}(\edens-3P)$. 
% In the conformal
% case $\mathcal{L}_n^{\rm eq}=0$ for $n\geq1$\cite{Blaizot:2017ucy}.

While in the conformal case the moments $\LL_n$ are sufficient to provide a
representation of the dynamics, in the nonconformal case the
energy-momentum tensor is no longer traceless and this requires
introducing a second set of moments.  Indeed, from the general form of the
energy-momentum tensor \rf{eq:Tmn} for Bjorken flow it follows that 
%
\be
\edens = \LL_0 ,  
\qquad \pL = \frac{1}{3} \left( \LL_0 + 2 \LL_1 \right) , 
\qquad \pT = \frac{1}{3} \left( \LL_0 -\LL_1 -\frac{3}{2} T_\mu^\mu  \right)~.
\ee
%
This motivates the introduction of another set of moments
\cite{Jaiswal:2022udf} in the following way:
%
\be
\MM_n &\equiv& m^2 \int_{\bf p} \frac{1}{p_0} P_{2n}(\cos \psi) \ f(\tau, p), 
	 \qquad \forall n \geq 0~.
\ee
%
For example, the moment $\MM_0$ is equal to the trace of the energy-momentum tensor:
%
\be
\MM_0 = m^2\int_{\bf p} \frac{1}{p_0}f(\tau, p)  = T^\mu_\mu = \edens - \pL-2\pT~,
\ee
%
and controls deviations from conformality.  The RTA Boltzmann equation for
Bjorken flow \rf{eq:Boltzmann_RTA} can now be rewritten as an infinite hierarchy of equations
with \rf{eq:byh} supplemented by 
%
\begin{equation}
    \frac{\partial\mathcal{M}_n}{\partial\tau}=-\frac{1}{\tau}\left(a_n'\mathcal{M}_n+b_n'\mathcal{M}_{n-1}+c_n'\mathcal{M}_{n+1}\right)
    -\frac{\mathcal{M}_n-\mathcal{M}^{\rm eq}_n}{\tau_R}~,
    \label{eq:M}
\end{equation}
%
where $\mathcal{M}^{\rm eq}_n$ are the equilibrium moments, and the coefficients read
%
\begin{equation}
    a_n'=\frac{2(6n^2+3n-1)}{(4n-1)(4n+3)},\quad
    b_n'=\frac{4n^2(2n-1)}{(4n-1)(4n+1)},\quad
    c_n'=-\frac{(2n+1)^2(2n+2)}{(4n+1)(4n+3)}~.
\end{equation}
%
Note that the $\MM_n$ moments are coupled to the $\mathcal{L}_n$ by the
presence of the $\mathcal{M}^{\rm eq}_n$ terms, and decouple in the
collisionless limit of $\tau_R\rightarrow\infty$. The equations \rf{eq:L}
constitute a closed system which has to be solved first, determining the moments
$\LL_n$.

Using this infinite hierarchy of evolution equations one can identify an
early-time far-from-equilibrium attractor which describes free-streaming. This
is in line with results found in this model in
Refs.~\cite{Chattopadhyay:2021ive,Jaiswal:2021uvv,Chen:2021wwh}.  Out of the
three independent components of the energy-momentum tensor (see \rf{eq:AdefNC}),
one has an attractor while the remaining two carry information about the initial
state to the asymptotic late-time region.  This can be contrasted with conformal
systems, where the attractor appears in the pressure anisotropy, and information
about the initial conditions is carried to asymptotically late times by the
energy density alone. 

In \rfc{Jaiswal:2022udf} the quantity which has an attractor is expressed as 
\begin{equation}
    g_0\equiv
    \frac{\tau}{\LL_0}\frac{\partial\LL_0}{\partial\tau} = 
    -1 - \f{\pL}{\edens}.
    \label{eq:g0def}
\end{equation}
In the conformal case this is trivially related to the pressure anisotropy, but
in the absence of conformal symmetry it differs from it due to
bulk pressure. For this reason the function $g_0$ does not satisfy a single ODE, as it
would in a conformal model; instead, 
the pair of functions $g_0, \edens$ satisfy a coupled set of ODEs which
determine their dynamics.

The attractor solution tends to $g_0=-1$ at early
times, which corresponds to free streaming ($\pL=0$). Its late-time behaviour is
much harder to analyse than in the conformal case, since the equations of state
are more involved. In particular, the velocity of sound tends to zero in this
limit, which leads to rather nontrivial
asymptotics~\cite{Kamata:2022jrc,Kamata:2022ola}. The hydrodynamic region is
approached much more slowly and in way which depends of the mass.  This
behaviour is illustrated in \rff{fig:g0}. 

Truncations of the hierarchy of evolution equations provide workable
approximations which capture essential features of the full dynamics.  The most
straightforward truncation, which accounts for all three independent components
of the energy-momentum tensor, consists of keeping $\LL_0$, $\LL_1$ and $\MM_0$.
The higher modes $\LL_2$, and $\MM_1$ coupled with the three lowest ones have to
be modelled in some way, out of which the most simple, but not unreasonable, is
to set $\LL_2=\MM_1=0$.  This system of equations is hydrodynamic in spirit in
the sense that it is a closed set of three ODEs. However, it 
is different from the hydrodynamic model used in
Refs.~\cite{Chattopadhyay:2021ive,Jaiswal:2021uvv,Chen:2021wwh}. The main
virtue of this description is that it captures the early-time attractor
identified in the underlying kinetic theory, as seen in \rff{fig:g0}.





%%%%%%%%%%%%%


\begin{figure}
\begin{center}
\includegraphics[height = .36\textheight]{Fig5.pdf} 
\caption{The attractor in the nonconformal kinetic model of
    Ref.~\cite{Jaiswal:2022udf}, together attractors in effective hydrodynamic
    models (recall that $g_0$ is defined in \rf{eq:g0def}). 
% Here $g_0 \equiv \tau\partial_\tau\ln\edens$.
% Recall that $g_0$ is defined in \rf{eq:g0def}. 
The green, blue, red, and orange curves represent $m/T(\tau_R)=0.001,0.5,1$, and $5$. 
``Two-moments'' refers to 
the truncation with $\LL_2=0$. This truncation clearly captures the early-time attraction, in contrast to the Navier-Stokes, which refers to the truncated gradient expansion.  
This plot is taken from Ref.~\cite{Jaiswal:2022udf}.
}
\label{fig:g0}
\end{center}
\end{figure}

%%%%%%%%%%%%




%%%%%%%%%%%%%%
\subsection{Incorporating transverse dynamics}
%%%%%%%%%%%%%%

Since the longitudinal expansion is believed to be dominant at early times, the attractors seen
in models of Bjorken flow may retain their relevance even in the presence of
transverse dynamics. The persistence of early-time attractors in such
circumstances was recently studied in
Refs.~\cite{Kurkela:2019set,Kurkela:2020wwb,Ambrus:2021sjg} (see also Ref.~\cite{Borrell:2021cmh}). The first of these papers
considered transverse flow in the case of kinetic theory in the RTA.  The
authors have solved the boost-invariant Boltzmann equation numerically for a
choice of realistic initial transverse profiles. It was found that as long as
the transverse gradients remain negligible compared to the longitudinal ones at
initialisation time, arbitrary initial conditions in 3+1D evolve towards the
1+1D attractor. The late-time evolution does however depend on the transverse
profile of energy and transverse momentum, reflecting the fact that the space of
solutions of boost-invariant perfect fluid hydrodynamics has a higher
dimensionality than what is seen in the case of Bjorken flow.  These results
suggests a degree of robustness of early time attractors in the presence of
transverse dynamics.

Other work on attractors with transverse dynamics includes studies of Gubser
flow~\cite{Denicol:2018pak,Behtash:2017wqg,Behtash:2019qtk,Dash:2020zqx} (this
activity was recently reviewed by Soloviev~\cite{Soloviev:2021lhs}).
The applicability of hydrodynamics, including the effects of
the build-up of transverse dynamics at early times was also the subject of
recent 
Refs.~\cite{Ambrus:2022koq,Ambrus:2022qya,Ambrus:2023oyk}.





