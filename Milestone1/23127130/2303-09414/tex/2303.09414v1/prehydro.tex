
%%%%%%%%%%%%%%%%%%%%%%%%%%%%%%%%%%%%%%%%%%%%%
\section{Attractors and prehydrodynamic flow}
\label{sec:prehydro}
%%%%%%%%%%%%%%%%%%%%%%%%%%%%%%%%%%%%%%%%%%%%%

As discussed in Sec.~\ref{sec:HIC}, the standard picture of heavy-ion collisions
involves formation of QGP followed by a stage of nonequilibrium evolution until
a time when conventional fluid dynamics can be applied. The developments
reviewed in the last four Sections suggest that approximate Bjorken symmetry of
the early time dynamics could be responsible for a far-from-equilibrium
attractor capable of providing a bridge between these two stages. This attractor
could then be modelled using some much simpler effective description such as MIS
theory used outside its naive domain of applicability.  In this Section we will
review some efforts aiming to apply this idea to QGP
dynamics~\cite{Giacalone:2019ldn,Jankowski:2020itt}. Other work which makes
practical use of attractors in the context of heavy-ion collisions includes
Refs.~\cite{Du:2020dvp,Du:2020zqg,Du:2022bel,Dore:2020fiq,Dore:2020jye,Dore:2021xqq,Dore:2022qyz,Ambrus:2022qya,Ambrus:2022koq,Ambrus:2023oyk,Coquet:2021cuv,Coquet:2021gms}.

As currently understood (see \rfs{sec:attractors}), the existence of a
nonequilibrium hydrodynamic attractor in Bjorken flow is contingent upon there
being a definite, finite, physically distinguished value $\paz(0)$ of the
pressure anisotropy at $w=0$~\cite{Heller:2015dha,Aniceto:2015mto}. One can
translate this into a statement about the early time behaviour of the energy
density. Indeed, under the above assumptions, the conservation of
energy-momentum~\rf{eq:Adef3} implies that for asymptotically small proper-time
$\tau$
%
\be
\label{eq:mu}
  \edens \sim \f{\mu^4}{(\mu\tau)^{\beta}}~,
\ee
%
where the scale $\mu$ is an integration constant which reflects the initial
conditions, and the exponent $\beta$ is related to the attractor by the relation
%
\be
\label{eq.abeta}
\paz(0) = 6\left(1-\f{3}{4}\beta\right)~.
\ee
%
We will consider $0\leq\beta < 4$ (where $\beta=1$ corresponds to free
streaming).  While different initial conditions will correspond to different
values of $\mu$, the parameter $\beta$ characterises the attractor
itself and is therefore a feature of the particular microscopic theory under
consideration.  For instance, in MIS theory the attractor is
the unique stable solution which is regular at $w=0$, where 
%
\be \label{eq.mis} \paz(0) = 6\sqrt{\frac{C_\eta}{C_{\tau\Pi}}} \iff \beta =
\f{4}{3} \left(1 - \sqrt{\f{C_\eta}{C_{\tau\Pi}}}\right)~.
\ee
%


%%%%%%%%%%%%%%%%%%%
\subsection{Entropy and particle production}
\label{subsect:Entropy}
%%%%%%%%%%%%%%%%%%%%


The hydrodynamic attractor connects early and late-time behaviour of the system,
and this fact makes it possible to relate final state entropy to characteristics
of the initial state.  We will denote the value of $w$ at very early proper time
$\tau_0$  by $w_0$, and its value at late times $\tau_\infty$ by $w_\infty$.
Since at late times the system is approaching thermal equilibrium, one may use
standard thermodynamic relations to write the entropy density as\footnote{Note
that the arguments of this Section do not require invoking any concepts of
entropy far from equilibrium.}
\be
s(\tau_\infty) = \f{4}{3} \f{\edens(\tau_\infty)}{T(\tau_\infty)}
\ee
and then apply \rf{eq.approx} to express the right hand side in terms of
quantities evaluated at proper time $\tau_0$. This leads to the key relation
between the entropy density per unit rapidity at late time and the initial
energy density
%
\be
\label{eq:entrofinal}
s(\tau_\infty) \tau_\infty= h(\beta) \left(\edens(\tau_0) \tau_0^{\beta}\right)^{\frac{2}{4-\beta}}~,
\ee
%
where
%
\be
\label{eq:hbeta}
h(\beta) = \f{4}{3} w_\infty w_0^{\f{2\beta}{\beta-4}} \Phi_\paz(w_\infty, w_0)^2~.
\ee
%
The reason for writing \rf{eq:entrofinal} in this particular way is that the
left hand side as well as both factors on the right hand side are well defined
as $\tau_0\rightarrow 0$ and $\tau_\infty\rightarrow \infty$. Given \rf{eq:mu}
this is obvious for the second (parenthesised) factor, but one can also check
that the function in \rf{eq:hbeta} is finite in this limit because $\Phi_\paz$
in \rf{eq.phidef} diverges for small $w_0$ and vanishes for large $w_\infty$
precisely in such a way that the dependence on the initial and final values of
$w$ drops out, leaving a finite and nonzero result.  This can be shown in
general based on the asymptotic behaviours of the pressure anisotropy. 

The importance of \rf{eq:entrofinal} rests on the fact that it is an explicit
relation between the initial energy density and final entropy density of
expanding plasma, which accounts for entropy production as the
system evolves along the attractor.  Furthermore, it can be translated into a
statement about centrality dependence of observed particle multiplicities by
utilizing the following relation~\cite{Yagi:2005yb}:
%
\begin{equation}
\label{eq:dndy}
    \frac{d N_{\rm ch}}{d \eta}\approx A(s\tau)_{\rm hydro}~,
\end{equation}
%
where $A$ is a constant whose value will not be relevant in the subsequent
analysis. In the context of hydrodynamic attractors such a calculation was
first described in Ref.~\cite{Giacalone:2019ldn} for the special case of
free-streaming attractors, and generalised to any Bjorken attractor in
Ref.~\cite{Jankowski:2020itt}. We will now review these developments. 

Given the entropy density in \rf{eq:entrofinal}, and using \rf{eq:dndy}, the charged particle
multiplicity of a specific event can be
expressed as 
%
\be
\label{eq:dndyEps}
\dndy = A \tau_0^{\frac{2\beta}{4-\beta}} h(\beta) \int d^2\mathbf{x}_\perp
\edens(\tau_0, \mathbf{x}_\perp)^{\frac{2}{4-\beta}}~.
\ee
%
The new element here is allowing for a non-trivial dependence of the
initial energy density on the location in the plane transverse to the collision
axis.  This brings in dependence on the impact parameter of a given event. The
underlying assumptions and applicability of this procedure are discussed in
Ref.~\cite{Giacalone:2019ldn}, where it was introduced. Formula
(\ref{eq:dndyEps}) can be used to estimate the expected multiplicity by
averaging over Monte Carlo generated events.  

For a given event, the calculation of the initial energy density requires a model of the
initial state.  Ref.~\cite{Giacalone:2019ldn} considered free-streaming
attractors and showed that the results are consistent with experiment if one
chooses the dilute-dense
model~\cite{Dumitru:2001ux,Blaizot:2004wu,Gelis:2005pt,Blaizot:2010kh,Schlichting:2019bvy}
to describe the initial energy density profile. We will first review that study,
and then turn to the analysis of Ref.~\cite{Jankowski:2020itt}, where this
approach was generalised by dropping the assumption of free streaming at early
times. 

%%%%%%%
\subsection{The dilute-dense model and free-streaming attractors}
%%%%%%%

A standard approach to quantify fluctuations of nucleon positions is the Glauber
model~\cite{Miller:2007ri}. A basic object used to formulate a description of
the initial state in this approach is the thickness function
$T(\mathbf{x}_\perp)$ \cite{Miller:2007ri,Yagi:2005yb}, which is determined by
the integral of the average nuclear matter density along the longitudinal
direction~\cite{Yagi:2005yb}
%
\begin{equation}
    T({\bf x}_\perp)=\int_{-\infty}^\infty dz~ \rho({\bf x}_\perp,z)~.
\end{equation}
%
The nuclear density is usually parametrised by the Wood-Saxon distribution function
%
\begin{equation}
    \label{eq:saxonw}
    \rho(r)=\rho_0\left[1+\exp\left(\frac{r-R}{a}\right)\right]^{-1}~,
\end{equation}
%
where $\rho_0$ is chosen such that $\rho(r)$ is normalized to the number of
nucleons. For the two systems considered in
Refs.~\cite{Giacalone:2019ldn,Jankowski:2020itt} we have $a_{\rm Pb}=0.55$~fm,
$R_{\rm Pb}=6.62$~fm for $^{208}\rm Pb$ and $a_{\rm
Au}=0.53$~fm and $R_{\rm Au}=6.40$~fm for $^{197}\rm Au$~\cite{deVries:1987atn}.  

To quantify off-central collisions one introduces the impact parameter, that is
a vector ${\bf b}=(b_x,b_y)$ in the transverse plane which connects the centers of
the projectiles. Then, given the thickness functions $T^{A/B}({\bf
x}_\perp)\equiv T({\bf x}_\perp\pm{\bf b}/2)$ of two nuclei A and B colliding at a
given impact parameter $\bf b$, one defines centrality as~\cite{Teaney:2009qa}
%
\begin{equation}
    centrality = \frac{\pi |{\bf b}|^2}{\sigma_{\rm tot}}~,
    \label{eq:centrality}
\end{equation}
%
where $\sigma_{\rm tot}$ is a total inelastic nucleus-nucleus cross section. For
the data considered here we have $\sigma_{\rm tot}=767$
fm$^2$ for Pb-Pb collisions, and $\sigma_{\rm tot}=685$~ fm$^2$ for Au-Au
collisions. In a fluctuating Glauber model, the positions ${\bf x}_i$ of nucleons in
each of the nuclei are sampled with the distribution $\rho(r)$ given in
\rf{eq:saxonw} and their collisions are
determined by the mutual distance not larger than $\sqrt{\sigma_{\rm nn}/\pi}$,
where $\sigma_{\rm nn}$ is a nucleon-nucleon cross section equal to $\sigma_{\rm
nn}\left(\sqrt{s}=200~\rm GeV\right)=4.2$~fm$^2$ and $\sigma_{\rm
nn}\left(\sqrt{s}=2.76~\rm TeV\right)=6.4$~fm$^2$. The thickness function is
then determined on an event-by-event basis by summing up all density profiles of
all participating nucleons
%
\begin{equation}
    T_{A/B}({\bf x_\perp})=\frac{1}{n_{A/B}}\sum_{i=1}^{n_{A/B}}\rho_c\left({\bf x}-{\bf x}_i\pm\frac{\bf b}{2}\right)~,
\end{equation}
%
where $n_{A/B}$ is number of nucleons in the $A/B$ nuclei, and each nucleon is modelled by a Gaussian,
%
\begin{equation}
    \rho_c({\bf x_\perp})=\frac{1}{(2\pi v^2)^{\frac{3}{2}}}\exp\left(-\frac{\bf x^2_\perp}{2v^2}\right)~,
\end{equation}
%
with a fixed width $v=0.5$~fm determining its transverse size.

In the dilute-dense
model~\cite{Giacalone:2019ldn,Dumitru:2001ux,Blaizot:2004wu,Gelis:2005pt,Blaizot:2010kh,Schlichting:2019bvy}
of the initial energy deposition which is used in Ref.~\cite{Giacalone:2019ldn} the
initial energy density is taken as
%
\begin{equation}
    \epsilon_0^{\rm dilute-dense}({\bf x}_\perp) =C T^<({\bf x}_\perp)\sqrt{T^>({\bf x}_\perp)}~,
    \label{eq:eps_diluted}
\end{equation}
%
where $C$ is a constant. 
Centrality dependence enters through the relations
%
\begin{equation}
    T^<({\bf x}_\perp)={\rm min}\left(T({\bf x}_\perp-{\bf b}/2),T({\bf x}_\perp+{\bf b}/2)\right)~,
\end{equation}
%
\begin{equation}
     T^>({\bf x}_\perp)={\rm max}\left(T({\bf x}_\perp-{\bf b}/2),T({\bf x}_\perp+{\bf b}/2)\right)~.
\end{equation}
%
When the resulting energy density is used in \rf{eq:dndyEps}, the results are
consistent with the assumption of a free-streaming
attractor~\cite{Giacalone:2019ldn}. 

%%%%%%%
\subsection{Beyond free-streaming}
%%%%%%%

The dilute-dense model is one of a number of initial state models currently
being explored, and it is interesting to ask to what extent are other models
compatible with a free-streaming attractor. The study of
Ref.~\cite{Jankowski:2020itt} considered two options. The first is defined
by~\cite{Lappi:2006hq,Romatschke:2017ejr}
%
\begin{equation} \epsilon_0^{\rm dense-dense}({\bf x}_\perp)=C T({\bf
x}_\perp-{\bf b}/2)T({\bf x}_\perp+{\bf b}/2)~.  \label{eq:eps_dd}
\end{equation}
%
The second model assumes 
%
\begin{equation} 
    \label{eq:trento1}
    \epsilon_0^{p=-1}({\bf x}_\perp)=C\frac{T({\bf x}_\perp-{\bf
b}/2)T({\bf x}_\perp+{\bf b}/2)}{T({\bf x}_\perp-{\bf b}/2)+T({\bf x}_\perp+{\bf
b}/2)}~, \end{equation}
%
which is a special case of Trento parametrisation~\cite{Moreland:2014oya}.  The
normalisation constant $C$ appearing in these equations is independent of the
impact parameter $\bf{b}$.

%%%%%%%%%%%%%%%%%%%%%%%%%%%
\begin{figure}
\begin{center}
\includegraphics[width=0.7\textwidth]{a3fit.pdf}
\caption{
  Universal centrality dependence of $Q(c,c'{=}20))$, i.e. the number of
  produced charged particles normalized to $20$ centrality for each of the
  three models we consider. Experimental data
  shown for different collision systems: Xe+Xe \cite{ALICE:2018cpu}, Pb+Pb
  \cite{ALICE:2010mlf}, Au+Au \cite{PHOBOS:2010eyu},U+U \cite{PHENIX:2015tbb},
  Cu+Cu \cite{PHOBOS:2010eyu}.
  The plot is taken from Ref.~\cite{Jankowski:2020itt}.
  }
   \label{fig:wideExp} 
  \end{center}
\end{figure}
%%%%%%%%%%%%%%%%%%%%%%%%%%%

The impact of changing the initial state model can be assessed by fitting the
parameter $\beta$ using the observed centrality dependence of the measured
multiplicities. 
The formula \rf{eq:dndyEps} leads to a prediction in each centrality class.  We
then define the following ratios of multiplicities at different centralities 
%
\be Q(c,c') \equiv \f{\langle\dndy\rangle_{c^{\ }}}{\langle\dndy\rangle_{c'}}~,
\label{eq:Qij} \ee
%
where the angle-brackets denote the mean value over events in the specified
centrality class. These quantities are independent of the normalization factors
$C$ entering Eqns.~(\ref{eq:eps_diluted}), (\ref{eq:eps_dd}), (\ref{eq:trento1}); they
are also independent of the factor $h(\beta)$, which depends on the shape of the
presumptive attractor, not just its behaviour at early time.  However, they
retain dependence on the parameter $\beta$ itself, which is related to the
attractor by \rf{eq.abeta}. In this way, for any value of $\beta$, we obtain a
set of numbers $Q(c,c')$ which can be directly compared with  
published experimental results.  The best fit for each of the three models is
found to be 
%
\be 
\beta^{\rm dilute-dense} = 1.12, \quad \beta^{\rm IP} = 1.96, \quad
\beta^{\rm dense-dense} = 0.44  
\ee
%
with statistical errors not exceeding $0.02$. 
These values differ by a factor of almost $4.5$, which shows that if indeed an
attractor determines early time behaviour, it is strongly connected to the
initial state model. The implication here is that if we believe that QGP is
free streaming at early times, then this places a constraint on the initial state
model. Conversely, if we have reasons to favour a particular initial state
model, then this may require accounting for corrections to free streaming at
early times.  These remarks should be relevant for Bayesian
studies~\cite{Nijs:2020ors,Nijs:2020roc,JETSCAPE:2020mzn,Bernhard:2016tnd},
which scan over families of initial state models, but assume free-streaming
prehydrodynamic evolution for all of them.


