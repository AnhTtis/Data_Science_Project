%%%%%%%%%%%% 
\section{Hydrodynamic models of equilibration}
\label{sec:MIS}
%%%%%%%%%%%%


The appearance of attractors at the early stages of QGP dynamics can be
understood most easily in the context of what we refer to here as {\em hydrodynamic
models of equilibration}. This Section reviews the necessary conceptual
framework by clarifying the relationship between hydrodynamic behaviour and this
simplest class of models where its emergence can be studied.  Since this review
is focused on attractors, the aim of this section is not to introduce the
subject of relativistic hydrodynamics, which is well covered by the
existing sources (see e.g.
~\cite{Romatschke:2009im,Florkowski:2017olj,Romatschke:2017ejr}), but rather to
present a perspective which is useful for understanding hydrodynamic attractors.

%%%%%%%%%%%%%%%
\subsection{Conservation laws}
%%%%%%%%%%%%%%%


Hydrodynamic behaviour follows from conservation laws, the
most fundamental ones being those which express spacetime symmetries. In the
relativistic setting they take the form of the conservation law of the
energy-momentum tensor: 
%
\bel{eq:conservation}
\nabla_\mu T^{\mu\nu} = 0~.
\ee
%
In the context of a microscopic theory, such as a quantum field theory,
$T^{\mu\nu}$ above would refer to the expectation value of the energy-momentum
operator in some state, while in a kinetic theory model this would be a suitable
moment of the distribution function (see \rfs{sec:attractorKT}). When the system
is in local equilibrium, this quantity can be expressed in the perfect fluid
form, which is just a constant boost of its value at equilibrium:
\bel{eq:perfect}
    T_{\mu\nu}= \edens u_\mu u_\nu + \PP \Delta_{\mu\nu}\,,
\ee
where $\Delta_{\mu\nu}=g_{\mu\nu}+u_\mu u_\nu$ and $u$ is the boost parameter --
the relativistic velocity. Throughout this review, the metric $g$ is assumed to
be that of flat Minkowski space. The quantities $\edens$ and $\PP$ are scalars
which can be interpreted as the energy density and pressure in the local rest
frame. They are usually expressed in terms of the local effective temperature
$T$ through equations of state. The effective temperature and flow velocity are
then referred to as the {\em hydrodynamic variables}.  Due to the normalisation
condition of the four-velocity ($u\cdot u = -1$), there are four independent
variables. 

If the system is not in global equilibrium, the four hydrodynamic variables
are no longer constant and energy momentum tensor 
will depart from the perfect fluid form
\bel{eq:dissipative}
T_{\mu\nu}= \edens u_\mu u_\nu + \PP \Delta_{\mu\nu} + \pi^{\mu\nu}.
\ee
The correction $\pi^{\mu\nu}$ appearing above will be referred to as the
dissipative tensor.  This tensor vanishes unless the hydrodynamic variables vary
in spacetime, so one expects that sufficiently close to equilibrium it can be
expressed as a series of terms involving derivatives of the hydrodynamic
variables; this series is referred to as the {\em hydrodynamic gradient
expansion}\footnote{Unless explicitly indicated otherwise, we use the terms {\em
gradient} and {\em derivative} to mean derivatives with respect to the spacetime
variables, as opposed to purely spacial derivatives.}.  The gradient expansion
provides an asymptotic description of a given flow sufficiently close to
equilibrium. This asymptotic behaviour is strongly constrained by symmetries and
is thus common to many microscopic systems.  

The definition of the hydrodynamic variables is physically
unambiguous only in global equilibrium.  In general, one can redefine them
according to 
 \be
 \label{eq:frame}
 \edens = \tilde{\edens} + \delta\edens,\quad u^\mu= \tilde{u}^\mu+\delta u^\mu ~.
 \ee
In the context of the gradient expansion the delta-terms appearing above can be
thought of as being of order one or higher. Up to some finite order such
redefinitions can be used to impose so-called hydrodynamic {\em frame
conditions} which eliminate some components of the energy-momentum tensor. A
very convenient requirement of this type is the Landau condition
\begin{equation}
\label{eq:landauc}
    u_\mu \pi^{\mu\nu} = 0\,.
\end{equation}
Unless stated otherwise, in this review we will be assuming that this choice has been made. 

%%%%%%%%%%%%%%%%
\subsection{Modelling hydrodynamics}
%%%%%%%%%%%%%%%%

The basic idea of hydrodynamic models is to adopt the hydrodynamic
variables $(u^\mu)$
and $T$ as independent classical fields in an effective description of the dynamics
of the energy-momentum tensor. Hydrodynamic models then view the conservation
equations \rf{eq:conservation} not as a statement about the expectation value of
energy-momentum in a microscopic theory, but rather as a set of four evolution
equations which determine the dynamics of the four hydrodynamic variables.
With the energy-momentum tensor in the form given in \rf{eq:perfect}, this leads
to the relativistic theory of perfect fluids.  

In order to
incorporate dissipation one needs to express the dissipative tensor
$\pi^{\mu\nu}$ in \rf{eq:dissipative} in terms of
the hydrodynamic variables and their gradients. It is natural to do this by
using the gradient expansion, which from this perspective is the most general
parameterisation of near-equilibrium behaviour, including all the terms allowed
by symmetries. 

In conformal theories it is very convenient to express gradients in terms of
{\em Weyl-covariant derivative} $\mathcal{D}_\mu$ which differs from the ordinary derivative by
terms involving the four-velocity $u$ and its gradient. Its general definition and properties can be found in
Ref.~\cite{Loganayagam:2008is} (see also the appendix E of
Ref.~\cite{Florkowski:2017olj} for a brief summary).  
The simplest possibility is to set
%
\bel{eq:navstokes}
\pi_{\mu\nu} = - \eta \sigma_{\mu\nu} \equiv - \eta \left(\D_{\mu} u_\nu +
    \D_\nu u_\mu\right) =
- \eta \left(\p_\mu u_\nu +
\partial_\nu u_\mu-\frac{2}{3}\Delta_{\mu\nu}\partial_\alpha u^\alpha\right)~,
\ee
%
which is the unique term of first order in gradients which is consistent with
Lorentz and conformal invariance. The coefficient $\eta$ appearing here is the
shear viscosity, which is a scalar function of the effective temperature.  The
resulting model is the relativistic generalisation of Navier-Stokes theory.  In
contrast to non-relativistic case, this theory is acausal, because it possesses
solutions which propagate at arbitrarily large velocities. In consequence, this  
theory is also
unstable~\cite{Hiscock:1983zz,Hiscock:1985zz,Romatschke:2009im,Gavassino:2021owo}. 

To obtain a consistent and practically useful dynamical model one needs to
provide a prescription for augmenting the conservation equations
\rf{eq:conservation} in such a way as to be able to calculate the time evolution
of arbitrary initial data.  This prescription has to guarantee stability under
perturbations of equilibrium, as well as causality of propagation.  It must also
ensure the correct asymptotic behaviour as equilibrium is approached, which is
given by \rf{eq:navstokes}.  These requirements are very strong, and precious
few examples exist where they have been proved to be satisfied (see
Refs.~\cite{Bemfica:2017wps,Bemfica:2019cop,Bemfica:2020xym,Bemfica:2019knx,Bemfica:2020zjp}).
In the remainder of this Section we review the most widely-used approaches,
where they can be satisfied at least at the linearised level.  



%%%%%%%%%%%%
\subsection{The MIS approach}
%%%%%%%%%%%%

The MIS approach~\cite{Israel:1976tn,Israel:1979wp} does not assume an explicit
form of the dissipative tensor in terms of gradients of the hydrodynamic
variables. Instead, it posits a separate set of partial differential equations
for the dissipative tensor. These are formulated in such a way as to possess 
asymptotic solutions in the form of the gradient expansion parameterised in
terms of some finite number of scalar parameters. 

In the simplest variant of MIS theory the dissipative tensor satisfies equations
of the form of a relaxation equation
%
\begin{equation}
   (\tpi \D  +1)\pi_{\mu\nu}= - \eta \sigma^{\mu\nu} + \dots
   \label{eq:MISpi}
\end{equation}
%
where $\D\equiv u^\mu\D_\mu$. The properties of the Weyl-covariant derivative
ensure that 
that the Landau condition is preserved under time
evolution. One may also include additional terms in this equation, as discussed below. 
As written, this model ensures stability as well as causality at the linearised
level, as long as the relaxation time is large enough, satisfying the bound
(see e.g. Ref.\cite{Romatschke:2009im,Florkowski:2017olj})
%
\bel{eq:causal}
T \tau_\pi > 2 \eta/s~. 
\ee
%
Causality and stability at the nonlinear level are much more challenging to
establish, as discussed e.g. in Ref.~\cite{Bemfica:2020zjp}.

The solution to the relaxation equation \rfn{eq:MISpi} can be formally expanded in gradients:
\be
\label{eq:pigradshort}
\pimunu &=& -\eta \sigmamunu +  \tpi  \D\left(\eta\sigmamunu\right) +\ldots 
\ee
where the ellipsis denotes terms of third and higher orders. 
The leading term is of the Navier-Stokes form given in \rf{eq:navstokes}. 
The second and higher order terms are affected by the 
precise set of terms chosen for the right hand side of \rf{eq:MISpi}. 
In order to view a hydrodynamic model of equilibration as an effective
description of some underlying theory, one needs to have a means of matching the
two. This can be done using the gradient expansion which, as a perturbative
series around the state of global equilibrium, can be computed in any dynamical
theory -- at least in principle.  This circumstance makes it possible to match
parameters by comparing terms of the gradient expansion calculated in a
microscopic theory with analogous terms calculated in a hydrodynamic
model~\cite{Bhattacharyya:2007vjd}. For this to be generally possible at a given order in
the gradient expansion, the series in \rf{eq:pigradshort} would have to
include all terms allowed by Lorentz (and conformal) symmetry at this order. 
\rf{eq:MISpi} can match any microscopic model to first order in gradients, but
if one wishes to have the option to match to second order, additional terms are
needed. In Ref.~\cite{Baier:2007ix} the complete
set of second order terms which are consistent with
Lorentz and conformal covariance was determined. They can be matched by the
gradient expansion of the following relaxation equation
%
\be
\label{eq:brsss}
\left(\tau_\pi \D + 1 \right) \pi^{\mu\nu}
= -\eta \sigma^{\mu\nu}  
+ \lambda_1 {\pi^{\langle\mu}}_\lambda \pi^{\nu\rangle\lambda} +
\lambda_2 {\pi^{\langle\mu}}_\lambda \omega^{\nu\rangle\lambda} +
\lambda_3 {\omega^{\langle\mu}}_\lambda \omega^{\nu\rangle\lambda}\,.
\ee
Here $\lambda_1, \lambda_2,\lambda_3$ are additional transport coefficients
which guarantee
matching to second order in gradients~\footnote{We have omitted terms which
vanish in a flat metric background.}, 
%
\be
\omega^{\mu\nu} = \f{1}{2} 
\left(\D^\mu u^\nu - \D^\nu u^\mu \right),
\ee
%
is the kinetic vorticity,
and 
the angular brackets are defined as 
\be
{}^{\langle}   A^{ \mu\nu \rangle}   \equiv A^{\langle \mu\nu \rangle} = \f{1}{2}   \Delta^{\mu\alpha} \Delta^{\nu \beta} \left( A_{\alpha \beta} + A_{\beta\alpha} \right)
- \f{1}{3}  \Delta^{\mu\nu} \Delta^{\alpha \beta} A_{\alpha \beta} .
\ee
In the remainder of this review when talking about MIS theory we will have in
mind the above form of the relaxation equations, sometimes referred to as the
BRSSS equations. 
It is worth pointing out that while \rf{eq:brsss} is general enough so that
its gradient expansion includes all the terms in \rf{eq:pigradshort} with
arbitrary coefficients, it is not unique~\cite{Baier:2007ix}.

Finally, we note that while MIS theory is the most widely-used framework for
building models of hydrodynamics, other approaches exist, such as
anisotropic hydrodynamics (for a review and references see e.g.
Ref.~\cite{Florkowski:2017olj}). 



%%%%%%%%%%%%%
\subsection{Lessons from linear response}
\label{sec:linresponse}
%%%%%%%%%%%%%

Important insights into nonequilibrium dynamics follow from linearisation around
the state of global equilibrium. 
For our purposes it is enough to consider here the state of homogeneous equilibrium (non-rotating, without any external fields).
The hydrodynamic variables which solve the
linearised equations are then 
proportional to the harmonic factor $\exp
\left(-i \omega(k) t + i \vec{k}\cdot\vec{x}\right)$.  
The dispersion relations
which define the different solutions (modes) fall into two categories: the
{\em hydrodynamic modes} whose frequency vanishes with at long wavelengths, 
$\lim_{k\rightarrow0} \omega(k) = 0$, and the {\em nonhydrodynamic modes} which are
gapped: $\lim_{k\rightarrow0} \omega(k) \neq 0$.  This gap -- the frequency at
vanishing wave vector $k$ -- sets the asymptotic damping rate of the transient modes.  The damping
of the hydrodynamic modes diminishes with $k$, so modes of long
wavelengths are weakly damped. 

For example, linearisation of the evolution equations of MIS theory reveals a
set of hydrodynamic sound and shear modes\footnote{The radius of convergence of the series expansions of $\omega(k)$  is set by  singularities in the complexified $k$ place which
reflect mode
collisions~\cite{Withers:2018srf,Grozdanov:2019uhi,Heller:2020hnq,Heller:2020jif}.}
\be
    \omega_{\rm shear} &=& -i \frac{\eta}{s T}k^2 + O(k^4)~,
\\
    \omega_{\rm sound} &=& \pm \f{1}{\sqrt{3}} k -\frac{2i}{3}\frac{\eta}{s T} k^2  + O(k^3)~,
\ee
as well a some nonhydrodynamic modes which are damped
regardless of wavelength: their dispersion relation is $\omega = - i/\tau_\Pi +
O(k^2)$. In the limit when the relaxation time vanishes, the nonhydrodynamic modes
decouple and this theory reduces to Navier-Stokes theory. A calculation of the
velocity of sound (see e.g. Refs.~\cite{Romatschke:2009im,Florkowski:2017olj})
gives
\be
v = \f{1}{\sqrt{3}} \sqrt{1 + 4 \f{\eta/s}{T\tpi}} .
\ee
The condition
\rf{eq:causal} provides a limit on how small the relaxation time can be without
violating causality.  Thus, a natural way to think of nonhydrodynamic modes is
to view them as a regulator~\cite{Spalinski:2016fnj} (somewhat in the spirit of a
``UV-completion'' of quantum field theories), with the relaxation time
playing the role of a regulator parameter. 

This happens not just in MIS-type theories, but in many other hydrodynamic
models which are causal at least at the linear level, such as
BDNK~\cite{Bemfica:2017wps,Bemfica:2019knx,Kovtun:2019hdm,Noronha:2021syv} and
HJSW~\cite{Heller:2014wfa}. Indeed, recent
results~\cite{Heller:2022ejw,Gavassino:2023myj} strongly suggest that the
presence of nonhydrodynamic modes is a necessary condition for causality.  The
existence of hydrodynamic modes follows from conservation laws, while the
nonhydrodynamic modes are required to maintain causality.  The nonhydrodynamic
modes account for transient behaviour, while the long-lived hydrodynamic modes
express a measure of universality in the approach to equilibrium.  





%%%%%%%%%%%%%%%%%%%%%%%%%%%%%%%%%%%%%%%%%
\subsection{Modeling the non-hydrodynamic sector}
\label{sec:HJSW}
%%%%%%%%%%%%%%%%%%%%%%%%%%%%%%%%%%%%%%%%%

The appearance of nonhydrodynamic modes in models of relativistic hydrodynamics
mirrors the structure of microscopic theories. However, the analysis of
linearised perturbations of microscopic models reveals a much more complicated
picture than the simple nonhydrodynamic sector of MIS theory.  This happens in
models of kinetic theory~\cite{Romatschke:2015gic}, as well as strongly coupled
field theories described using methods based on the AdS/CFT
correspondence~\cite{Kovtun:2004de}. In both these cases there is an infinite
number of nonhydrodynamic modes, and in the latter case they are not purely
decaying. Sufficiently close to equilibrium the details of this sector are not
relevant, as the near-equilibrium physics is captured by the hydrodynamic
modes~\cite{Bantilan:2022ech}. However, in practice models of hydrodynamics are
often used further away from equilibrium, so models with different
nonhydrodynamic sectors will a priori lead to different results.  In such
situations one is really probing the physics of the regulator.

This raises the question whether it is possible to engineer hydrodynamic models
which mimic nontrivial nonhydrodynamic sectors. An example of such a model was
put forward in Ref.~\cite{Heller:2014wfa} and will be referred to as the HJSW
model (see also  \cite{Florkowski:2017olj,Aniceto:2015mto,Gavassino:2022roi}).
The motivation behind its formulation was to mimic the behaviour of strongly
coupled $\mathcal{N}=4$ SYM theory, where the least-damped transient modes
depend very weakly on momentum (a phenomenon known as ultralocality).
This leads to an evolution equation for the dissipative tensor of the form 
%
\begin{equation}
    \label{eq:hjsw}
    \left(\frac{1}{T}\mathcal{D}\right)^2\pi_{\mu\nu}
    +2\Omega_I\frac{1}{T}\mathcal{D}\pi_{\mu\nu}+|\Omega|^2\pi_{\mu\nu}=-\eta|\Omega|^2\sigma_{\mu\nu}-C_\sigma\frac{1}{T}\mathcal{D}\left(\eta\sigma_{\mu\nu}\right)+\dots~.
\end{equation}
%
This equation is a replacement for the MIS/BRSSS relaxation equation,
\rf{eq:MISpi}. The parameters $\eta, \Omega_R, \Omega_I, C_\sigma$ play the same
role as the transport coefficients appearing in \rf{eq:MISpi}, and $|\Omega|^2 =
\Omega_R^2+ \Omega_I^2$. The term with parameter $C_\sigma$ was introduced to
broaden the domain where the theory is stable and causal at the linearised
level.  The physical meaning of these parameters is
partially revealed by formally expanding \rf{eq:hjsw} in gradients, which yields
\rf{eq:pigradshort} with the identification 
\be
\tau_\Pi =  \frac{2\Omega_I - C_\sigma}{|\Omega|^2} \f{1}{T}
\ee
and $\eta$ retaining its meaning as the shear viscosity. 

Further insight is gained by calculating the dispersion relations for linear perturbations of equilibrium.
Apart from the standard hydrodynamic modes we see nonhydrodynamic modes
\be
\omega_\pm(k) = - i \Omega_I \pm \Omega_R + O(k)
\ee
whose relaxation rate is set by $\Omega_I$. In contrast to MIS theory, these
modes are not purely decaying: they also oscillate with frequency set by
$\Omega_R$. This captures the patterns of least-damped quasinormal mode of the
black brane appearing in the dual description of \symm\ theory (see
\rfs{sec:attractorHolo}), where  $\Omega_{R} \approx 9.8$ and $\Omega_{I}
\approx 8.6$.  Of course, in the spirit of hydrodynamics, \rf{eq:hjsw} could in
principle apply to any theory with a similar pattern of nonhydrodynamic modes.
Thus, at least at the level of the gradient expansion, this model contains
Navier-Stokes theory in the near-equilibrium limit -- just like MIS -- but
provides a different regulator sector. 
The assumption of ultralocality which has lead to \rf{eq:hjsw} is a useful
simplification, but it is not strictly obeyed in SYM, and can be avoided at the
level of hydrodynamic models~\cite{Heller:2021yjh}. 

The idea of including nonhydrodynamic modes in a deliberate manner has also been
the founding concept of the Hydro+ programme~\cite{Stephanov:2017ghc}, which is
being actively developed in connection with the search for signals of a critical
point in the QCD phase diagram through heavy-ion collisions. Recent work
developing this circle of ideas  includes
Refs.~\cite{Abbasi:2020xli,Ke:2022tqf}.


%%%%%%%%%%%%%%%%%%%%%%%%%%%
\subsection{General frames}
%%%%%%%%%%%%%%%%%%%%%%%%%%%
\label{sec:gf}

Another interesting class of hydrodynamic models was discovered quite recently
by Disconzi, Bemfica, Noronha and Kovtun in
Refs.~\cite{Bemfica:2017wps,Bemfica:2019knx,Kovtun:2019hdm}.  These models,
usually referred to by the acronym BDNK,  deviate from the MIS approach in that
they do not introduce additional hydrodynamic fields beyond those already
present in Navier-Stokes theory and rely only on the conservation equations to
provide the dynamics.  

The basic insight of BDNK was to recognize that the Landau condition,
\rf{eq:landauc}, is not a fundamental requirement, but rather one of many
ways of pinning down the definition of the hydrodynamic variables
off-equilibrium. So instead of \rf{eq:navstokes}, at first order in
gradients one could adopt the following form of the dissipative tensor:
\be
\label{eq:bdnk}
\pi_{\mu\nu} = \tau^{\mu\nu} + \mathcal{C}\left(u_\mu u_\nu+ \frac{1}{3}\Delta_{\mu\nu} \right)+ \mathcal{Q}_\mu u_\nu + \mathcal{Q}_\nu u_\mu
\ee
where
\be
\tau^{\mu\nu} = - \eta \sigma^{\mu\nu},
\quad
\mathcal{Q}^\mu
= -\tau_\psi\Delta^{\mu\lambda}\D_{\lambda}\edens, 
\quad
\mathcal{C}
= -\tau_\phi \mathcal{D}\edens
\label{eq:bdnk2}
\ee
where $\tau_\phi, \tau_\psi$ are new transport coefficients. These  additional
terms in \rf{eq:bdnk} (relative to \rf{eq:navstokes}) could be removed using the
frame freedom \rf{eq:frame}, which would amount to imposing the Landau
condition. No new dynamical fields are introduced: $\tau^{\mu\nu},
\mathcal{Q}^\mu, \mathcal{C}$ are expressed explicitly in terms of the basic
hydrodynamical variables $\edens, u^\mu$. Nevertheless, this theory is causal
and stable~\cite{Bemfica:2020zjp} for suitable choices of parameters, because
relaxing the Landau frame condition  introduces a nonhydrodynamic sector.  The
structure of this sector turns out to be the same as in MIS theories, but the
evolution equations are different and lead to the same physics only close to
equilibrium~\cite{Bantilan:2022ech}.  Models of this type are the subject of a
number of interesting recent
studies~\cite{Hoult:2020eho,Pandya:2021ief,Pandya:2022pif,Pandya:2022sff,Gavassino:2023odx,Gavassino:2023qwl}.

The general-frame concept can be taken further in the spirit of the MIS
approach~\cite{Noronha:2021syv}. The basic idea is to replace \rf{eq:bdnk2} by a
set of relaxation equations\footnote{The published version of
Ref.~\cite{Noronha:2021syv} presents a rather general implementation of this
idea. A conformal implementation, similar to what we review here, can be found
in the original arXiv.org (v1) submission. The relaxation equations \rfn{eq:nss}
contain only the conceptually essential terms; the original reference contains
some additional contributions motivated by entropy considerations.}
\be
\tau_\pi 
\mathcal{D}\pi^{\mu\nu}
+ \pi^{\mu\nu}
&=& - \eta \sigma^{\mu\nu}
\nn\\
\tau_Q
\D\mathcal{Q}^{\mu}
+ \mathcal{Q}^\mu
&=& -\tau_\psi \Delta^{\mu\lambda}\D_{\lambda}\edens
\nn\\
\tau_C
\mathcal{D}\mathcal{C}
+ \mathcal{C}
&=& -\tau_\phi \mathcal{D}\edens
\label{eq:nss}
\ee
with additional transport coefficients $\tau_\pi, \tau_Q, \tau_C$.  The
resulting model has more degrees of freedom and a nonhydrodynamic sector which
is larger than in either MIS or BDNK, and is of some practical as well as
conceptual significance~\cite{Gavassino:2020ubn,Gavassino:2021cli}.  We will return
to it briefly in \rfs{sec:attractors}, since it offers some additional insights
into attractor behaviour.  

