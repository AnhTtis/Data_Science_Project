
%%%%%%%%%%%%%%%%%%%%%%%%%%%%%%%%%%%%%%%%
\section{The phase space perspective}
\label{sec:PhaseSpace}
%%%%%%%%%%%%%%%%%%%%%%%%%%%%%%%%%%%%%%%%

We have seen that some part of the information about the initial state becomes 
suppressed during dissipative evolution. In the case of Bjorken flow this is
especially clearly visible in the behaviour of $\pa(w)$, where at late times the
approach to equilibrium is independent of initial conditions up to exponentially
suppressed corrections.
Furthermore, in this case the attractor locus is one dimensional, and is in fact
a solution of a differential equation.  These features are a consequence of
expressing the dynamics through a set of convenient variables. To explore
hydrodynamic attractors beyond the simplest settings one needs to understand how
to analyse the problem without such special variables, because in more
complicated situations such variables may not be known, or even exist. In this
Section we review a very general approach, which does not rely on any symmetry assumptions such as
those of Bjorken flow~\cite{Heller:2020anv}.  It addresses the
emergence of attractors by considering the behaviour multiple solutions in a
suitably defined phase space.  It is worth emphasising that this perspective can be
applied to any dynamical model of equilibration, including  models formulated in
the language of kinetic theory and the AdS/CFT correspondence. 

%%%%%%%%%%%
\subsection{Dimensionality reduction}
%%%%%%%%%%%


%%%%%%%%%

\begin{figure}[t!]
  \centering
  \includegraphics[width=0.45\columnwidth]{frame-1.pdf}
\hspace{4mm}
  \includegraphics[width=0.45\columnwidth]{frame-3.pdf}
\vspace{1mm}\vfill
  \includegraphics[width=0.45\columnwidth]{frame-5.pdf}
\hspace{4mm}
  \includegraphics[width=0.45\columnwidth]{frame-7.pdf}
\vspace{1mm}\vfill
  \includegraphics[width=0.45\columnwidth]{frame-9.pdf}
\hspace{4mm}
  \includegraphics[width=0.45\columnwidth]{frame-11.pdf}
\caption{A sequence of snapshots expressing the evolution of a point-cloud of
solutions plotted on a proper-time slice in boost-invariant MIS theory. Initially the depicted region is
uniformly filled, but in subsequent plots we see the dimensionality reduced from
$2$ to $1$. The colour of a dot encodes the effective
temperature~\cite{Spalinski:2022cgj}.}
\label{fig:dimred}
\end{figure}

%%%%%%%%%%

To introduce the basic idea we return to the simplest models of equilibration,
formulated in the language of hydrodynamics. Given an equation such as
\rf{eq:MISTeom}, the most generic parametrisation of phase space would be to
use $T(\tau), \dot{T}(\tau)$.  The late time behaviour of the temperature
\rf{eq:MISTasym} shows that any set of solutions whose initial conditions are set on some
proper-time slice $\tau=\tau_0$, in the course of evolution collapses
approximately onto a one-dimensional locus: a curve parametrised by the value of
$\Lambda$, which is the only remnant of the initial state.   In the simplest MIS
model this means that only one combination of two integration constants is still
accessible at late times, but in more complex models the initial state could
carry much more information which is effectively dissipated by the time the
asymptotic form \rf{eq:MISTasym} is reached.  This suggests that even in more
general settings one may view the attractor as a locus of low dimensionality
embedded in a potentially high-dimensional phase space. One can thus say that
evolution toward the hydrodynamic attractor is tantamount to {\em dimensionality
reduction} of sets of solutions viewed on phase space slices. 

We illustrate this perspective by considering the full phase space for Bjorken
flow in  MIS theory, parametrised by $(\tau, T, \dot{T})$ -- the proper time
is included as one of the phase space variables because equations of motion
\rf{eq:MISTeom} depend explicitly on~$\tau$.  The plots in \rff{fig:dimred}
allow us to follow a collection of solutions starting with a
uniformly-distributed set of points on an initial proper-time slice. In the
course of dissipative evolution one sees them all approaching the attractor
locus, which in this case is a straight line, whose alignment depends on 
$\tau$. 

%%%%%%%%%%%%%%%%%%%%%%
\subsection{Machine learning}
%%%%%%%%%%%%%%%%%%%%%%%%

The notion of dimensionality reduction in phase space is a promising perspective
on hydrodynamic attractors, one which is not tied to the simplicity of Bjorken
flow. A key element of this approach is some convenient, but essentially
arbitrary parametrisation of phase space. In situations more generic than
Bjorken flow this will certainly involve some coarse-graining, but one will
still need to consider phase spaces of large dimensionality. This suggests
applying machine learning techniques to this problem. This has not yet been
fully explored in the published literature, but some encouraging pilot studies exist.
Here we will comment briefly on the approach suggested in
Ref.~\cite{Heller:2020anv}, which made use of one of the simplest
dimensionality-reduction methods -- Principal Component Analysis (PCA).  PCA
analyses the variations in a data set in different directions and associates an
\textit{explained variance} with each of them. In this way the number of
principal components of a set of solutions on a proper time slice reflects the
effective dimensionality of this \textit{point cloud}. In the case of Bjorken flow, when
the system is close to equilibrium this cloud will be one-dimensional,
reflecting the single integration constant of the asymptotic Bjorken solution
\rf{eq:bjorken}. 

As a simple illustration, we begin by applying PCA to the two-dimensional phase
space of MIS theory, so as to quantify the pattern of behaviour seen in
\rff{fig:dimred}. On the initial time slice we pick a state $(T,\dot{T})$ and
consider a random set of points within a disc around it. For this set of points,
the two principal components are approximately equal in magnitude. At each time
step we recompute the principal components of this evolving point cloud; their
evolution is shown in Fig.~\ref{fig:brsss_pca}.  Dimensionality reduction is
signalled when one of the components is much smaller than the other one.  It is
significant that the dimensionality reduction splits into two stages: the first
one can interpret as the effect of the longitudinal expansion, and the second as
nonhydrodynamic mode decay. 

%%%%%%%%%%%%%%%%%%%

\begin{figure}[t!]
    \centering
\includegraphics[width=.47\columnwidth]{PCAtop.pdf} 
\includegraphics[width=.47\columnwidth]{PCAbottom.pdf}
\caption{Left: evolution of explained variance ratio of each principal component
  in MIS/BRSSS for circles (radius: $10^{-4}$) of initial conditions with
  centres lying in the middle of initial dots of corresponding color in
  Fig.~\ref{plot:descent}. Right: logarithm of decaying principal components
  plotted as a function of $w$. For large enough values of~$w$ one clearly sees
  persistent exponential decay. Plot from Ref.~\cite{Heller:2020anv}. }
    \label{fig:brsss_pca}
\end{figure}

%%%%%%%%%%%%%%%%%%

A similar picture emerges in the case of the HJSW model discussed in
\rfs{sec:HJSW}, where the phase space s three-dimensional.  In this case the
dimensionality reduction is even more striking, as seen in \rff{dimred:hjsw},
whose bottom panel depicts three stages in the evolution of a uniform cloud of
solutions at three instances of proper time. Initially, the boost-invariant
expansion leads to a rapid, parameter-independent collapse of the
three-dimensional region to a two-dimensional locus -- this is the early-time,
expansion-dominated phase. Subsequently the reduced two-dimensional cloud
evolves until it shrinks to a line, as it must for conformal Bjorken flow. The
dynamics of the second and third stages depend on the parameter values, as
expected on the basis of the interpretation in terms of nonhydrodynamic mode
decay.  The evolution of principal components is shown in the upper part of
Fig.~\ref{dimred:hjsw}, where one can clearly discern the three different stages
with different dimensionality. 

%%%%%%%%%%%%%%%%%%%%%%%%%%%%

\begin{figure}[t!]
    \centering
\includegraphics[width=.95\columnwidth]{HJSWfig2.png}
\caption{In the HJSW model, the evolution of a cloud in phase space can
be split into three stages, corresponding to the dimensionality
of the cloud. The reduction from three to two dimensions
corresponds to a collapse onto the slow region (blue region in plots). Figure performed with
$C_\eta=0.75$, $C_{\tau_\pi}=1.19$, and $C_\omega=9.8$, taken from Ref.~\cite{Heller:2020anv}. }
    \label{dimred:hjsw}
\end{figure}

%%%%%%%%%%%%%%%

This type of analysis can be applied to phase spaces of arbitrary
dimensionality, in any dynamical model. This will in general involve some coarse
graining and truncation of the phase space, which in itself need to be
finite-dimensional. An enlightening example which illustrates these issues is
provided by a model of Bjorken flow in kinetic theory in the RTA which can be
found in the Supplemental Material of Ref.~\cite{Heller:2020anv}. A more
realistic study of the phase space approach to Bjorken flow in the effective
kinetic theory of QCD can be found in Ref.~\cite{Du:2022bel}.  These studies
rely on PCA, but a number of other machine learning techniques exist which may
provide a more refined picture, such as Topological Data Analysis (see e.g.
\cite{chazal}) which has recently been explored in a somewhat related physical
context~\cite{Spitz:2023wmn}. 


%%%%%%%%%%%%%%%%%%%%%
\subsection{The attractor as a slow region}
%%%%%%%%%%%%%%%%%%%%%%


%%%%%%%%%%%%%%%%%%%%

\begin{figure}[t!]
\centering
  \includegraphics[width=.47\columnwidth]{brs31new.pdf}
\includegraphics[width=.47\columnwidth]{brs32new.pdf}
\vspace{3mm}
\includegraphics[width=.47\columnwidth]{brs33new.pdf}
\caption{Three proper time slices of phase space tracking a cloud of
  about 10,000 solutions of MIS theory. The red
  curve denotes the attractor locus
${\cal A}_\star$.  The background color encodes the speed at which
the points move in phase space, with magenta faster than blue according to
the norm of velocity vector~\eqref{eq.Vdef}; the dark blue denotes the
slow region. The plots were made for $C_{\eta} = 0.75$ and
$C_{\tau_{\pi}} = 1$; 
$\tau_{0}$ denotes the initialization time.
 Plot from Ref.~\cite{Heller:2020anv}.}
\label{plot:descent}
\end{figure}

%%%%%%%%%%%%%%%%%%%%%%%%%%%


In models with phase spaces whose dimension is greater than two, the attractor
is not a single solution but rather a region of phase space onto which actual
solutions condense. In conformal examples of Bjorken flow it was possible to
project this region onto a single solution in the $\pa, w$ variables, but this
is not expected to be possible in more general situations. Therefore, a more
general characterisation of the attractor is needed. Intuitively one would
expect that the attractor locus should correspond to a ``slow region'' where the
flow in phase space is slowest, because it takes a long time to escape it, while
the fast regions can be quickly traversed.

In the case of MIS this idea correctly identifies the
attractor on any given proper time slice. A point on such a slice of phase space can be
described by the vector 
$\vec{X}(\tau) = \left(\tau_{0} \,
T(\tau),\tau_0^2 \, \dot{T}(\tau) \right)$; its velocity is given by 
%
\be
\label{eq.Vdef}
  \vec{V}=\tau_{0} \frac{\partial \vec{X}}{\partial \tau} ~,
\ee
%
where the factors of $\tau_{0}$ have been introduced for dimensional reasons. The slow
region can be defined by the Euclidean norm
$V$ of this vector, which has a minimum at
asymptotically late times when the system approaches local thermal equilibrium. 
The resulting picture can be seen in Fig.~\ref{plot:descent}; the
background color is determined by~$V$, where  bluer color implies lower speed.
There is a slow region stretching out from local thermal equilibrium, and the
attractor $\mathcal{A}_{*}(w)$ lies along it. 
The identification of the attractor as a slow region at least in principle
generalizes directly to phase spaces of any dimension. 





