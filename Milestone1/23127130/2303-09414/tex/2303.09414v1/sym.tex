%%%%%%%%%%%%%%%%%%%%%%%%%%%%%%%%%%%%%%%%%%%%%
\section{Attractor behaviour through AdS/CFT} \label{sec:attractorHolo}
%%%%%%%%%%%%%%%%%%%%%%%%%%%%%%%%%%%%%%%%%%%%%


In this Section we review studies of equilibration of Bjorken flow in \symm\
theory, which are possible to carry out in the strong coupling limit by virtue
of the AdS/CFT correspondence \cite{Maldacena:1997re,Witten:1998qj}. Excellent
reviews of the applications of holographic methods to heavy ion collisions
include
Refs.~\cite{DeWolfe:2013cua,Gubser:2010nc,Casalderrey-Solana:2011dxg,vanderSchee:2014qwa};
applications to Bjorken flow are reviewed in
Refs.~\cite{Janik:2010we,Florkowski:2017olj,Berges:2020fwq,Soloviev:2021lhs}. In
this Section we will therefore refrain from discussing the techniques involved,
our focus being on the results of such calculations and their interpretation in
terms of hydrodynamic attractors. 

%%%%%
\subsection{Thermal states in AdS/CFT}
%%%%%

The basic fact which lies at the heart of applying AdS/CFT to nonequilibrium
physics is that the equilibrium state of \sym\ supersymmetric Yang-Mills plasma
in $4$-dimensional Minkowski space corresponds to a black hole in
asymptotically-$AdS_5$ space. This object is often referred to as a black brane
due to the fact that the horizon of the gravitational solution is planar rather
than spherical.  The black brane temperature $T$ is equal to the temperature of
the plasma.  The duality map (sometimes referred to as the holographic
dictionary) leads to a formula for the energy density of the plasma: 
%
\begin{equation}
    \edens = \frac{3\pi^2}{8}N_c^2 T^4.
    \label{eq:EHEOS}
\end{equation}
%
Up to a factor of $3/4$, which is interpreted as
the effect of strong coupling, this coincides with the result for the energy
density of quanta comprising the plasma in the absence of interactions.

The interpretation of the equilibrium state in terms of a black hole immediately
suggests that its perturbations should correspond to perturbations of the black
hole. Their spectrum can therefore be computed by standard methods developed in
studies of general relativity, adapted to the specific challenges of
asymptotically AdS spaces. At the linearised level such perturbations are known
as quasinormal modes (QNM) of the black hole, and they describe damped
oscillations of the black hole horizon. Their  complex frequencies naturally
fall into one of two categories: a finite number of hydrodynamic modes whose
damping rate is proportional to the wave vector, and an infinite series of
nonhydrodynamic modes which are damped even for arbitrarily long-wavelength
perturbations.  This matches directly the picture of perturbations of
equilibrium expected on the basis of linear response theory, as reviewed in
\rfs{sec:linresponse}. 

The process of equilibration can also be described analytically at the nonlinear
level.  This was pioneered in
Refs.~\cite{Janik:2005zt,Janik:2006ft,Heller:2007qt,Heller:2008mb} by studies of
the asymptotic late-time expansion of Bjorken flow (reviewed in
\rfs{sec:HoloAsym} below).  These results were subsequently generalised to
generic near-equilibrium states~\cite{Bhattacharyya:2007vjd}, which are mapped
to perturbed black objects in asymptotically $AdS$ spaces described in a
gradient expansion analogous to what is done in hydrodynamics.  This connection
is sometimes referred to as the fluid-gravity correspondence. It has also led to
an interpretation of off-equilibrium entropy in field theory in terms of
slowly-evolving horizons in the dual gravitational
representation~\cite{Bhattacharyya:2008xc,Booth:2010kr,Booth:2011qy}.  These
analytic studies have provided a number of insights which were later applied to
numerical simulations and have greatly aided the interpretation of their
results.


%%%%%
\subsection{Numerical solutions and early time behaviour}
%%%%%

Numerical approaches to solving the initial value
problem in the gravitational representation of Bjorken flow and translating the
result into field theory
language have also been critically important~\cite{Chesler:2008hg,Chesler:2009cy,Heller:2011ju,Heller:2012je,Chesler:2013lia,Jankowski:2014lna}.
One of the first steps in such calculations is the selection of consistent
initial geometries.  This is a nontrivial task, as it requires satisfying the
constraints following from Einstein equations.  A basic result is that the
early-time behaviour of the energy density on the field theory side has the form
of a Taylor series with only even powers of the proper time~\cite{Beuf:2009cx}: 
%
\begin{equation}
    \label{eq:epsearly}
    \edens = \edens_0 + \edens_2\tau^2 + O(\tau^4)~.
\end{equation}
%
In Refs.~\cite{Heller:2011ju,Heller:2012je,Jankowski:2014lna} the initial conditions were set in
such a way that the leading coefficient $\edens_0\neq 0$. This corresponds to
the initial value of the pressure anisotropy $\pa(0)=6$. A number of such
solutions are plotted in \rff{fig:symnumerics}. It is apparent that the pressure
anisotropy reaches the hydrodynamic attractor while the system is still very
anisotropic. It is not clear however whether an expansion dominated regime
exists at early times. This is partly due to the oscillatory behaviour, which is
interpreted as a consequence of the rich spectrum of nonhydrodynamic modes whose
frequencies are not purely imaginary, in contrast to models of MIS hydrodynamics
or kinetic theory.  A possibly more significant issue is that the effective
phase space of the theory is multidimensional. While two real numbers suffice to
specify the initial data for the equations for Bjorken flow in MIS theory, in
the AdS/CFT calculation the initial data is specified by a function of the
radial (holographic) coordinate in the asymptotically-AdS space, which
defines the initial geometry.  If this were coarse-grained in some way, one
could represent the phase space as a having effectively a finite number of
dimensions, but there is no reason to believe that a two-dimensional truncation
would provide a reasonable
account. The plot of $\pa(w)$ should therefore be viewed as a projection from a
high-dimensional phase space and may obscure the picture at early times. 
An indication of this can be seen in \rff{fig:HJSWattractor} and
\ref{fig:NSSattractor}. 

%%%%%%%%%%%%%%%%%

\begin{figure}[t]
\begin{center}
    \includegraphics[width=.75\textwidth]{Anisotropy.pdf}
\caption{
    Pressure anisotropy as a function of $w$ for
$\mathcal{N}=4$ SYM at strong coupling \cite{Jankowski:2015uva}.
}
\label{fig:symnumerics}
\end{center}
\end{figure}

%%%%%%%%%%%%%%

In contrast with hydrodynamic models, where a regular solution at $w=0$ was a
natural candidate for an attractor, there is no such natural candidate here. In
Ref.~\cite{Romatschke:2017ejr} an attempt was made to find a physical argument
which would single out a special initial condition close to $w=0$. An
interesting feature of this proposed attractor is that it appears to be close to
free-streaming at early times, just as what is found in kinetic theory.  
A somewhat complementary approach to identifying the early-time attractor based
on Borel summation of the gradient expansion is also not conclusive, since it
looses predictivity at very early times, as  
reviewed in \rfs{sec:BorelAtr} below.

The existence of an early-time attractor in this theory is of great interest,
and this question was revisited recently in Ref.~\cite{Kurkela:2019set}.  In
this paper the authors looked for the early, expansion-dominated phase and did
not find evidence for it.  The
evolution of the pressure anisotropy for a number of solutions is shown in
Fig.~\ref{fig:LateHolo}, where it is apparent that the approach to the attractor
is determined by the $1/T$ scale irrespective of the initialization time.

In summary, the status of the early-time attractor in the case of \symm\ is not entirely
clear at this time. It is expected to exist on the basis of general, kinematic arguments,
but it could also be that due to the strong coupling limit the expansion
dominated region is artificially smeared out and effectively only a late time
attractor exists. 


%%%%%%%%%%%%%%%%%%%%%%%%%%%%%%%%%%%%%%%%%

\begin{figure}
\centering
\includegraphics[width =.75\textwidth]{adsevolution4.pdf}
\caption{Evolution of various initial configurations in holography. 
Note that the pressure anisotropy is connected to the quantity in the plot through
the relation 
$\pa=-\frac{3}{2}+\frac{9P_L}{2\epsilon}$.
Plot form Ref.~\cite{Kurkela:2019set}.
}
\label{fig:LateHolo}       
\end{figure}

%%%%%%%%%%%%%%%%%%%%%%%%%%%%%%%%%%%%%%%%%%%



%%%%%%%%%%%%%%%%%%%%%%%%
\subsection{The large proper time expansion}
\label{sec:HoloAsym}
%%%%%%%%%%%%%%%%%%%%%%%%

The emergence of fluid behaviour in boost-invariant \symm\ was first 
demonstrated in Refs.~\cite{Janik:2005zt} by studying the behaviour
of the energy density at large values of the proper time.  
The asymptotic behaviour of the energy density is given by
%
\begin{equation}
    \edens_{\rm hydro}(\tau)\sim \frac{\Lambda^4}{(\Lambda\tau)^{4/3}}{\sum_{n=0}^\infty \edens_n^{(0)}(\Lambda\tau)^{-2n/3}}~,
    \label{eq:epsHydro}
\end{equation}
%
where the energy scale $\Lambda$ reflects the initial conditions, as in other
cases of Bjorken flow, and $\edens_0^{(0)}=1$.  The expansion coefficients
$\edens_n^{(0)}, n>0$ can be determined using the AdS/CFT correspondence. The first
three subleading orders were calculated
analytically~\cite{Janik:2006ft,Heller:2008mb,Booth:2009ct}, and higher orders
numerically~\cite{Heller:2013fn,Aniceto:2018uik}. For $n\gg1$ these coefficients
diverge factorially, i.e.
%
\begin{equation}
    \edens_n^{(0)}\sim\frac{\Gamma(n+\beta_1)}{A_1^{n+\beta}}+{\rm h.c.}~
    \label{eq:hydro_diverge}
\end{equation}
%
where $A_1,\beta_1$ are complex constants; the singulant $A_1$ turns out
to be related to the lowest
nonhydrodynamic quasinormal mode frequency of the dual black hole
$\omega_1=3.1195-{\rm i}2.7467$ \cite{Kovtun:2005ev} by the relation 
$A_1={\rm i}\frac{3}{2}\omega_1$. This connection of large
order behaviour of the gradient expansion to the nonhydrodynamic modes is therefore fully analogous to
what was discussed in \rfs{sec:largeorders} in the case of MIS theory. 

%%%%%%%%%%%%%%%%%%%%%%%%%%%%%%%%%%%%%%%%%

\begin{figure}[t]
\centering
\includegraphics[height =.35\textheight]{BPPertextra.pdf}
\caption{Poles of the Borel-Pad\'e approximant $\mathrm{BP}_{189}\left[\edens_{\rm hydro}\right]$,
           in the complex $\xi-$plane.  We can see the appearance of the
           different fundamental sectors (shown as filled circles) as well as
           mixed sectors (shown as filled purple diamonds).  The predicted
           branch points for each sector are marked by colours:  $A_{1}$ and
           $\overline{A_{1}}$ (blue), $A_{2}$ and $\overline{A_{2}}$ (red),
           $A_{3}$ and $\overline {A_{3}}$ (green). This plot is taken from Ref.~\cite{Aniceto:2018uik}.
}
\label{fig:AdSBorel}       
\end{figure}

%%%%%%%%%%%%%%%%%%%%%%%%%%%%%%%%%%%%%%%%%%%

Equation~(\ref{eq:epsHydro}) cannot be the whole story, since initial states in
$AdS$ contain much more information than just the scale $\Lambda$. In fact, each
nonhydrodynamic mode of \symm\ plasma introduces an exponentially damped
transseries sector, 
with an independent transseries parameter which corresponds to a
piece of initial data.  Thus, the full expansion of $\edens(\tau)$ takes form of
multi-parameter transseries with infinitely many transseries
parameters~\cite{Aniceto:2018uik}: 
%
\begin{equation}
    \edens(\tau)\sim
    \underbrace{ \frac{\Lambda^4}{(\Lambda\tau)^{4/3}}{\sum_{n=0}^\infty
    \edens_n^{(0)}(\Lambda\tau)^{-2n/3}}}_{\edens_{\rm hydro}(\tau)} +
    \frac{\Lambda^4{\sigma_1}}{(\Lambda\tau)^{4/3}}{\sum_{n=0}^\infty
    \edens_n^{(1)}(\Lambda\tau)^{-2n/3}e^{-A_1(\Lambda\tau)^{2/3}}} +{\rm h.c.} +\cdots
    \label{eq:asymEps}
\end{equation}
%
For simplicity, only one nontrivial transseries sector is written explicitly in 
\rf{eq:asymEps}; it is the contribution of the least-damped, transient mode determined
by the complex quasinormal mode frequency $A_1$. The imaginary part of $A_1$
controls the exponential damping, while the real part determines the oscillation
frequency.  The full solution also includes supplementary sectors representing mutual
couplings between distinct modes.  This intricate structure is reflected in
\rff{fig:AdSBorel}, where the branch points correspond to quasinormal modes (as
well as their products)~\cite{Aniceto:2018uik}.

All the coefficients $\edens_n^{(k)}$ appearing in the transseries expansion in
\rf{eq:asymEps} can be determined using the AdS/CFT correspondence; many of them
have been calculated numerically for the most relevant sectors in
Ref.~\cite{Aniceto:2018uik}.  The original hydrodynamics series $\edens_n^{(0)}$
diverges factorially, as in \rf{eq:hydro_diverge}, and so do the series appearing
in each transseries sector. For instance, in the first sector one finds that for
$n\gg1$
\begin{equation}
    \edens_n^{(1)}\sim\f{\Gamma(n+\beta_2)}{A_2^{n+\beta_2}} + \mathrm{h.c.}
\end{equation}
where $A_2,\beta_2$ are complex constants; the singulant $A_2={\rm
i}\frac{3}{2}\omega_2$, where $\omega_2=5.1695-{\rm i}4.7636$ is second least
damped QNM in the sense that $|{\rm Im} \ A_1| < |{\rm Im} \ A_2|$.  This type of
relation between series appearing in different transseries sectors is a
manifestation of resurgence~\cite{Aniceto:2018bis,Mitschi:2016fxp}. The resurgence property of
the transseries implies that the complete structure of the nonhydrodynamic
sectors can be recovered from the original hydrodynamic gradient expansion.  



%%%%%%%%%%%%%%%%%%%%%%%%%%%
\subsection{The attractor by Borel summation} 
\label{sec:BorelAtr}
%%%%%%%%%%%%%%%%%%%%%%%%%%%
 
As reviewed in \rfs{sec:attractors}, one can estimate the location
of the attractor by Borel summation of the hydrodynamic gradient expansion. To
do this in the present case one has to calculate the expansion of the pressure
anisotropy in powers of the $w$ variable using \rf{eq:epsHydro}.  This series is
also factorially divergent and the analytic continuation of the Borel transform
has the same pattern of singularities as seen in \rff{fig:AdSBorel}.  In
particular, there are no singularities on the real axis.  This means that the
Borel sum of this series does not suffer from the complex ambiguity encountered
in the case of MIS theory.  Of course the transseries contributions are still
present, and will become significant for values of $w$ sufficiently far from the
asymptotic region. Thus, at small values of $w$ this approach
looses predictability, since the no longer negligible exponentially suppressed
contributions eventually bring in dependence on the transseries coefficients and
it is not known which values correspond to the attractor. 

%%%%%%%%%%%%%%%

\begin{figure}[t]
\begin{center}
\includegraphics[width=.48\textwidth]{BorelPolesHJSW.pdf}
\includegraphics[width=.48\textwidth]{borela.pdf}
\caption{
        Left panel: poles of the Borel transform of the gradient series of HJSW
    hydrodynamics; the red dots represent the known complex frequencies
    $\Omega$ of the
    nonhydrodynamic modes; the circles show the locations of their multiples. 
Right panel: the numerical attractor in HJSW hydrodynamics projected on the $(w,
\pa)$ plane (red curve) compared with the result of Borel summation of the
gradient expansion (black dots). }
\label{fig:borelpadeHJSW}
\end{center}
\end{figure}

%%%%%%%%%%%%%%%%%%

This procedure has been in tested in the case of the HJSW model
\cite{Heller:2014wfa} which has a similar singularity structure  and where the
true attractor is easily found numerically.  The relevant equation which is
satisfied by $\pa(w)$ is \rf{eq:vcp}, which can easily be solved in a power
series for large $w$; the leading terms appear in \rf{eq:hjsw.largew}.  This
series is factorially divergent and the singularities of the analytic
continuation of its Borel transform resemble those of \symm, as seen in the
right panel of \rff{fig:borelpadeHJSW}.  The series can then be Borel-resummed
at various values of $w$ following the procedure outlined in
\rfs{sec:largeorders}. We can interpret the outcome as an approximation to the
attractor. As seen from \rff{fig:borelpadeHJSW}, the result is in excellent
agreement with the numerical attractor down to $w\approx 0.4$. At earlier times
the transient transseries constributions are no longer negligible. Moreover,
these effects depend on the transseries parameters, and it is not known how to
determine their values so as to reproduce the attractor.  This would require
connecting the late-time transseries with the convergent series describing the
attractor \rf{eq:hjsw.smallw}. The analogous issue was recently discussed in MIS
theory using transasymptotic summation~\cite{Aniceto:2022dnm}, so perhaps this
point could be addressed in the future. 

%%%%%%%%%%%%%%%%%%%%%%%%%%%

\begin{figure}
\begin{center}
\includegraphics[width=.82\textwidth]{AnisotropyParam.pdf}
\caption{
The pressure anisotropy as a function of $w$ for $\mathcal{N}=4$ SYM at strong
coupling, along with the parametrized attractor (thick magenta line), from
Eq.~(\ref{eq:Aparam}). 
}
\label{fig:AttractorParam}
\end{center}
\end{figure}
%

%%%%%%%%%%%%%%%%%%%%%%

In the case of \symm\ theory one can apply Borel summation to the gradient
expansion in exactly the same way~\cite{Spalinski:2017mel}. The resulting
estimate of the attractor can be represented by fitting the numerical result to
a rational function:
%
\begin{equation} 
    A_{\rm attr}(w)=\frac{-276w+2530}{3975 w^2 - 570 w + 120}~.
\label{eq:Aparam} 
\end{equation}
%
It is apparent from \rff{fig:AttractorParam}
that this resummation breaks down below $w\approx0.4$ and therefore sheds little
light on the early-time behaviour. 

To conclude this section, let us mention some results concerning finite coupling
corrections~\cite{Casalderrey-Solana:2017zyh}. Such corrections can be included
by modifying the dual gravitational representation of the leading approximation
discussed so far. The ensuing Bjorken flow gradient expansion can also be
calculated to high order. The analytic continuation of its Borel transform
reveals an interesting structure of singularities which interpolates between the
one found in at infinite coupling \cite{Aniceto:2018uik} and the one familiar
from kinetic theory \cite{Heller:2016rtz}. 







