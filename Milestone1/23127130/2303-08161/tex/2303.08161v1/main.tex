%%
%% This is file `sample-acmsmall.tex',
%% generated with the docstrip utility.
%%
%% The original source files were:
%%
%% samples.dtx  (with options: `acmsmall')
%% 
%% IMPORTANT NOTICE:
%% 
%% For the copyright see the source file.
%% 
%% Any modified versions of this file must be renamed
%% with new filenames distinct from sample-acmsmall.tex.
%% 
%% For distribution of the original source see the terms
%% for copying and modification in the file samples.dtx.
%% 
%% This generated file may be distributed as long as the
%% original source files, as listed above, are part of the
%% same distribution. (The sources need not necessarily be
%% in the same archive or directory.)
%%
%%
%% Commands for TeXCount
%TC:macro \cite [option:text,text]
%TC:macro \citep [option:text,text]
%TC:macro \citet [option:text,text]
%TC:envir table 0 1
%TC:envir table* 0 1
%TC:envir tabular [ignore] word
%TC:envir displaymath 0 word
%TC:envir math 0 word
%TC:envir comment 0 0
%%
%%
%% The first command in your LaTeX source must be the \documentclass
%% command.
%%
%% For submission and review of your manuscript please change the
%% command to \documentclass[manuscript, screen, review]{acmart}.
%%
%% When submitting camera ready or to TAPS, please change the command
%% to \documentclass[sigconf]{acmart} or whichever template is required
%% for your publication.
%%
%%
\documentclass[acmsmall]{acmart}

%%
%% \BibTeX command to typeset BibTeX logo in the docs
\AtBeginDocument{%
  \providecommand\BibTeX{{%
    Bib\TeX}}}

%% Rights management information.  This information is sent to you
%% when you complete the rights form.  These commands have SAMPLE
%% values in them; it is your responsibility as an author to replace
%% the commands and values with those provided to you when you
%% complete the rights form.
\setcopyright{acmcopyright}
\copyrightyear{2018}
\acmYear{2018}
\acmDOI{XXXXXXX.XXXXXXX}


%%
%% These commands are for a JOURNAL article.
\acmJournal{JACM}
\acmVolume{37}
\acmNumber{4}
\acmArticle{111}
\acmMonth{8}


%%% MACROS
\newcommand{\macrospath}{./macros}
\newcommand{\proofspath}{./proofs/}
%\newcommand{\localmacrospath}{../}
%\usepackage{bussproofs}
\usepackage{ifthen}
\usepackage{xspace}

%%%%%%%%%%%%%%%%
%%
%% 		Types of paper
%%
%%%%%%%%%%%%%%%%

\newboolean{talk}
\setboolean{talk}{false}
\newboolean{paper}
\setboolean{paper}{false}



%%%%%%%%%%%%%%%%
%%
%% 		Latex styles
%%
%%%%%%%%%%%%%%%%

\newboolean{IEEEstyle}
\setboolean{IEEEstyle}{false}
\newboolean{lipicsstyle}
\setboolean{lipicsstyle}{false}
\newboolean{eptcsstyle}
\setboolean{eptcsstyle}{false}
\newboolean{entcsstyle}
\setboolean{entcsstyle}{false}
\newboolean{sigplanstyle}
\setboolean{sigplanstyle}{false}
\newboolean{easychairstyle}
\setboolean{easychairstyle}{false}
\newboolean{scrartclstyle}
\setboolean{scrartclstyle}{true}
\newboolean{lncsstyle}
\setboolean{lncsstyle}{false}

\newboolean{acmartstyle}
\setboolean{acmartstyle}{false}

%%%%%%%%%%%%%%%%
%%
%% 		Features to include
%%
%%%%%%%%%%%%%%%%

\newboolean{needstheorems}
\setboolean{needstheorems}{false}
\newboolean{withimages}
\setboolean{withimages}{false}
\newboolean{withproofs}
\setboolean{withproofs}{true}


%%%%%%%%%%%%%%%%
%%
%% 		Languages
%%
%%%%%%%%%%%%%%%%

\newboolean{french}
\setboolean{french}{false}

\setboolean{acmartstyle}{true}
\usepackage{amsthm}
%\theoremstyle{plain}
    \newtheorem{theorem}{Theorem}[section]
    \newtheorem{lemma}[theorem]{Lemma}
    \newtheorem*{lemmanonum}{Lemma}
    \newtheorem{corollary}[theorem]{Corollary}
    \newtheorem{proposition}[theorem]{Proposition}
%\theoremstyle{definition}
    \newtheorem{definition}[theorem]{Definition}
%\theoremstyle{remark}
    \newtheorem{remark}[theorem]{Remark}
    
    \ifthenelse{\boolean{scrartclstyle}}{
  \newtheorem{example}{Example}[section]
}{}


\definecolor{purple}{rgb}{1, 0, 1}

\newcommand{\ie}{\emph{i.e.,}\xspace}
\newcommand{\eg}{\emph{e.g.,}\xspace}
\newcommand{\abr}{\emph{abbr.}\xspace}
\newcommand{\ea}{\emph{et al.}\xspace}
\newcommand{\gensync}{\emph{GenSync}\xspace}
\newcommand{\colosseum}{\emph{Colosseum}\xspace}
\newcommand{\srep}{\emph{SREP}\xspace} % Set Reconciliation Enhances
\newcommand{\srepsim}{\emph{SREPSim}\xspace}
% Propagation
\newcommand{\esrep}{\emph{E-SREP}\xspace}
\newcommand{\epsrep}{\emph{EP-SREP}\xspace}
\newcommand{\mesrep}{\emph{ME-SREP}\xspace}
\newcommand{\mempoolsync}{\emph{MempoolSync}}

\newcommand{\fref}[1]{Fig.~\ref{#1}}
\newcommand{\tref}[1]{Table~\ref{#1}}
\newcommand{\aref}[1]{Algorithm~\ref{#1}}
\newcommand{\procref}[1]{Procedure~\ref{#1}}
\newcommand{\sref}[1]{Section~\ref{#1}}
\newcommand{\lineref}[1]{line~\ref{#1}}
\newcommand{\appref}[1]{Appendix~\ref{#1}}

% Change \eqref
\LetLtxMacro{\originaleqref}{\eqref}
\renewcommand{\eqref}{Eq.~\originaleqref}

% Theorems and corollaries
\newcounter{theoremcount}
\setcounter{theoremcount}{0}
\DeclareRobustCommand{\theorem}[1]{%
  \refstepcounter{theoremcount}%
  \noindent\textit{\textbf{Theorem \thetheoremcount\label{theorem:#1}: }}%
}
\DeclareRobustCommand{\theoremref}[1]{Theorem~\ref{theorem:#1}}

\DeclareRobustCommand{\proof}{\emph{Proof:}\xspace}
\DeclareRobustCommand{\qqed}{\hfill$\blacksquare$}

\newcounter{corollcount}
\setcounter{corollcount}{0}
\DeclareRobustCommand{\coroll}[1]{%
  \refstepcounter{corollcount}%
  \noindent\textit{\textbf{Corollary \thecorollcount\label{coroll:#1}: }}%
}
\DeclareRobustCommand{\corollref}[1]{Corollary~\ref{coroll:#1}}

\newcounter{lemmacount}
\setcounter{lemmacount}{0}
\DeclareRobustCommand{\lemma}[1]{%
  \refstepcounter{lemmacount}%
  \noindent\textit{\textbf{Lemma \thelemmacount\label{lemma:#1}: }}%
}
\DeclareRobustCommand{\lemmaref}[1]{Lemma~\ref{lemma:#1}}

\newcounter{definitioncount}
\setcounter{definitioncount}{0}
\DeclareRobustCommand{\definition}[1]{%
  \refstepcounter{definitioncount}%
  \noindent\textit{\textbf{Definition \thedefinitioncount\label{definition:#1}: }}%
}
\DeclareRobustCommand{\defref}[1]{Definition~\ref{definition:#1}}

%notes of different authors
\newif\ifnotes
\notestrue
\notesfalse

\newif\ifdiff
\difftrue
\difffalse

\newcommand{\anote}[1]{\ifnotes $\ll$\textsf{\textcolor{purple}{Ari: {#1}}}$\gg$ \fi}
\newcommand{\nnote}[1]{\ifnotes $\ll$\textsf{\textcolor{orange}{Novak: {#1}}}$\gg$ \fi}
\newcommand{\diff}[1]{\ifdiff\textcolor{orange}{#1}\else#1\fi}

%%% Local Variables:
%%% mode: latex
%%% TeX-master: "main"
%%% End:

\usepackage{cleveref}

%Where to look \newcommand{\macrospath}{../../../macros}

\newcommand{\Enfpath}{../Enf+Compatibility}
\newcommand{\Nafpath}{../Naf+Compatibility}
\newcommand{\NafEpath}{../Naf-extended+Compatibility}
\newcommand{\NafEXpath}{../Naf-extended-modular-equivalences+Compatibility}
\newcommand{\IdeasMultiCpath}{../Idea-multicontexts-stop-and-go}
\newcommand{\StopAndGoSimulationspath}{../Stop-and-go-simulations}

\renewcommand{\naft}{\mathit{toy}}
\renewcommand{\nafet}{\mathit{net}}
\renewcommand{\nafext}{\mathit{nafex}}
\newcommand{\nett}{\mathit{net}}

\newcommand{\leqnet}{\precsim_{\nett}}
\newcommand{\eqnet}{\simeq_{\nett}}
\renewcommand{\naf}{toy\xspace}
\renewcommand{\Naf}{Toy\xspace}
%\renewcommand{\nafe}{$\mathsf{net}$\xspace}
%\renewcommand{\Nafe}{$\mathsf{Net}$\xspace}
\renewcommand{\nafex}{mir$_{M}$\xspace}
\renewcommand{\Nafex}{Mir$_{M}$\xspace}

\newcommand{\nafexp}[1]{mir$_{#1}$\xspace}

\renewcommand{\leqnafex}{\precsim_{\vscx}}
\renewcommand{\eqnafex}{\simeq_{\vscx}}

\renewcommand{\nafext}{\vscx}


\renewcommand{\net}{net\xspace}
\renewcommand{\Net}{Net\xspace}

%Macros for eta_v mir and net
\newcommand{\vscxeta}{\vscx-\eta_v}
\newcommand{\nafexeta}{mir$_{M}$-$\eta_v$\xspace}
\newcommand{\Nafexeta}{Mir$_{M}$-$\eta_v$\xspace}
\newcommand{\nafexetap}[1]{mir$_{#1}$-\eta_v\xspace}
\newcommand{\leqnafexeta}{\precsim_{\vscxeta}}
\newcommand{\eqnafexeta}{\simeq_{\vscxeta}}
\newcommand{\nafexetat}{\vscxeta}

\newcommand{\netetat}{\mathit{net}-$\eta_v$}
\newcommand{\neteta}{net-$\eta_v$\xspace}
\newcommand{\Neteta}{Net-$\eta_v$\xspace}
\newcommand{\leqneteta}{\precsim_{\netetat}}
\newcommand{\eqneteta}{\simeq_{\netetat}}


\newcommand{\opnafexetap}[1]{\ctxholep{#1}_{\nafexetat}}
\newcommand{\opnafexeta}{\ctxhole_{\nafexetat}}
\newcommand{\relnafexeta}{\opnafexetap\relsym}
\newcommand{\lasrelnafexetasym}{\opnafexetap\lasrelsym}
\newcommand{\lasrelnafexeta}{\, \lasrelnafexetasym \,}


\newcommand{\metalassenopsym}{{\cal L}^{M}_{\eta_v}}
\newcommand{\metalassenop}[1]{{#1}^{\metalassenopsym}}
\newcommand{\metalasrelsym}{\metalassenop\relsym  }
\newcommand{\metalasrel}{\, \metalasrelsym\,  }

\newcommand{\metalasrelnafexetasym}{\opnafexetap\metalasrelsym}
\newcommand{\metalasrelnafexeta}{\, \metalasrelnafexetasym \,}



%equivalence relation

\newcommand{\transitive}[1]{#1^{+}}

\newcommand{\symmetric}[1]{\symfont{sym}(#1)}
%\input{macros-types}


\begin{document}

%%
%% The "title" command has an optional parameter,
%% allowing the author to define a "short title" to be used in page headers.
\title{Normal Form Bisimulations by Value}

%%
%% The "author" command and its associated commands are used to define
%% the authors and their affiliations.
%% Of note is the shared affiliation of the first two authors, and the
%% "authornote" and "authornotemark" commands
%% used to denote shared contribution to the research.
\author{Beniamino Accattoli}
\affiliation{%
  \institution{Inria \& LIX, École Polytechnique, UMR 7161}
%  \streetaddress{1 Th{\o}rv{\"a}ld Circle}
% 	\city{Saclay}
  \country{France}}
\email{beniamino.accattoli@inria.fr}

\author{Claudia Faggian}
\affiliation{%
  \institution{IRIF, CNRS, Université Paris Cité, F-75013 Paris}
%  \streetaddress{1 Th{\o}rv{\"a}ld Circle}
% 	\city{Saclay}
  \country{France}}
\email{}

\author{Adrienne Lancelot}
\affiliation{%
  \institution{Inria \& LIX, École Polytechnique, UMR 7161}
%  \streetaddress{1 Th{\o}rv{\"a}ld Circle}
% 	\city{Saclay}
  \country{France}}
\email{adrienne.lancelot@inria.fr}


%%
%% By default, the full list of authors will be used in the page
%% headers. Often, this list is too long, and will overlap
%% other information printed in the page headers. This command allows
%% the author to define a more concise list
%% of authors' names for this purpose.
\renewcommand{\shortauthors}{Accattoli, Faggian, and Lancelot}

%%
%% The abstract is a short summary of the work to be presented in the
%% article.
\begin{abstract}

HPC applications are critical in various scientific domains ranging from molecular dynamics to chemistry to fluid dynamics. Conjugate Gradient (CG) is a popular application kernel used in iterative linear HPC solvers and has applications in numerous scientific domains. However, the HPCG benchmark shows that the peformance achieved by Top500 HPC systems on CG is a small fraction of the performance achieved on Linpack. While CG applications have significant portions of computations that are dense and sparse matrix multiplications, skewed SpMMs/GEMMs in the HPC solvers have poor arithmetic intensities which makes their execution highly memory bound unlike GEMMs in DNNs which have high arithmetic intensity. The problem of low intensity individual \GEMM also exists in various emerging workloads from other domains like Graph Neural Networks, Transformers etc. In this work we identify various reuse opportunities between the tensors in these solver applications to extract reuse in the entire Directed Acyclic Graph of the tensor operations rather than individual tensor operations. These opportunities essentially depend on the dimensions of the tensors and the structure of the tensor dependency graph. We propose a systematic methodology to determine various kinds of reuse opportunities in the graph of operations and determine the loop order and tiling in the interdependent operations. As a result, we propose a novel mapping strategy \DataflowName that improves reuse of HPC applications on spatial accelerators. We also propose a data organization strategy in the buffer. Our mapping achieves geomean 6.7x reduction in memory accesses. 



%The TOP500 supercomputers achieve only a fraction of performance on HPC solvers compared to Linpack benchmark on which they are ranked. Majority of the computations in this algorithm are GEMMs. Spatial accelerators have gained huge success over acceleration of DNN applications where the data movement is minimized by exploiting reuse within a single matrix multiplication. However, matrix multiplications in many HPC applications like iterative linear solvers have arithmetic intensities. However, in this work we show various reuse and acceleration opportunities within these solver algorithms to minimize the data movement in the entire dataflow graph of the operation. We propose a novel dataflow \DataflowName to maximize reuse in the complete DAG of tensor operations for HPC and other applications from other domains like Machine Learning and Graph that use GEMMs. We observe that \DataflowName reduces the memory accesses by \TODO{xx\%} and inter-node communication by \TODO{xx\%} 

\end{abstract}

%%
%% The code below is generated by the tool at http://dl.acm.org/ccs.cfm.
%% Please copy and paste the code instead of the example below.
%%
\begin{CCSXML}
\end{CCSXML}

%%
%% Keywords. The author(s) should pick words that accurately describe
%% the work being presented. Separate the keywords with commas.
\keywords{Lambda calculus, program equivalence, bisimulations, call-by-value}

%\received{20 February 2007}
%\received[revised]{12 March 2009}
%\received[accepted]{5 June 2009}

%%
%% This command processes the author and affiliation and title
%% information and builds the first part of the formatted document.
\maketitle

\section{introduction}

% 1. importance of TKGs and reasoning on TKGs. 
% 2. low resource languages, main main idea.
% 3. relations and limitations of current works.
% 4. summarize our solutions and contributions.

Temporal Knowledge Graphs (TKGs)~\cite{YAGO,ICEWS18,WIKI,acekg} characterize temporally evolving events, where each event, represented as ({\em subject}, {\em relation}, {\em object}), is associated with temporal information ({\em time}), e.g., ({\em Macron}, {\em reelected}, {\em French president}, {\em 2022}). TKGs has facilitated various knowledge-intensive Web applications with timeliness, such as question answering~\cite{KBQA}, product recommendation~\cite{RippleNet,TKG4Rec,TKG4Rec2,RETE}, and social event forecasting~\cite{KG4Social,DyDiff-VAE,andgan,belief,misinfo,polarization}. 

As new events are continually emerging, modern TKGs are still far from being complete. Conventionally, the TKG construction process relies primarily on information extraction from unstructured corpus~\cite{WIKI,YAGO, EventKG}, which necessitates extensive manual annotations to keep up with changing events. For instance, the recent transition from Trump to Biden as the President of the United States has not been reflected in many TKGs, highlighting the need for timely updates. This spurs research on temporal knowledge graph reasoning to automate evolving events prediction over time~\cite{TA-DistMult,Know-Evolve,Renet,RE-GCN}. Unfortunately, the problem of TKG incompleteness is particularly pronounced in low-resource languages, where it is unable to collect enough corpus and annotations to support robust TKG construction. This results in suboptimal reasoning performance and distinctly unsatisfying accuracy in predicting recent and future events.

% whose performance can degrade significantly in low-resource language TKGs that suffer from severe incompleteness over time. 
% \jingfeng{why don't people  study cross-lingual TKG previously, (i.e. use language alignment to improve TKG). Is it really helpful intuitively to use high resource language to help TKGC? For instance, is it enough to use static langauge-alignment to help KGC, ignoring the temporal information? Are those langauge-alignment changing across time?}



\begin{figure}
    \centering
    \includegraphics[width = 1.0\linewidth]{fig/task.pdf}
    \caption{An illustrative example of cross-lingual reasoning on TKGs. 1) We aim to transfer knowledge from English TKG to Japanese TKG, where the English version provides more complete information; 2) Cross-lingual alignments only cover a small ratio of entities, e.g., Apple Inc; 3) Cross-lingual alignments can be noisy and misleading, e.g., A city called Ventura is linked to new macOS Ventura at $t_2$, introducing noise for reasoning in Japanese.}
    \label{fig:illustration}
    %\vspace{-6mm}
\end{figure}

Inspired by the incompleteness issue facing low-resource languages in constructing TKGs, we introduce a novel task named Cross-Lingual Temporal Knowledge Graph Reasoning (as shown in Figure~\ref{fig:illustration}). This task aims to alleviate the reliance on supervision for TKGs in low-resource languages (referred to as the target language) by transferring temporal knowledge from high-resource languages (referred to as the source language)~\footnote{In this paper, for the sake of brevity, we interchangeably use the terms high-resource/low-resource and source/target.}. In contrast, all the existing efforts are either limited to reasoning in monolingual TKGs (usually high-resource languages, e.g., English)~\cite{TA-DistMult,Know-Evolve,Renet,RE-GCN}, or multilingual static KGs~\cite{KEnS,AlignKGC,SS-AGA}. To the best of our knowledge, cross-lingual TKG reasoning that transfers temporal knowledge between TKGs has not been investigated. 

%Motivated by this, we study a new task named {\em cross-lingual temporal knowledge graph reasoning} as shown in Figure~\ref{fig:illustration}, to alleviate the heavy dependence on supervision for any resource-poor language TKGs by distilling the temporal knowledge from resource-rich ones. Differently, all the existing efforts are either limited to reasoning in monolingual (usually high-resource languages, e.g., English) temporal KGs~\cite{TA-DistMult,Know-Evolve,Renet,RE-GCN}, or multilingual static KG~\cite{KEnS,AlignKGC,SS-AGA}, but neglecting the reasoning in a both temporal and cross-lingual manner that highly requires capturing time-evolving patterns and language discrepancy. To the best of our knowledge, this problem, regarding how to transfer cross-lingual knowledge between TKGs, has still not been formally investigated. 

% Unlike conventional TKG reasoning, 
The fulfillment of this task poses tremendous challenges in two aspects: 1) \textbf{Scarcity of cross-lingual alignment}: as the informative bridge of two separate TKGs, cross-lingual alignment is imperative for cross-lingual knowledge transfer~\cite{AlignKGC,KEnS,SS-AGA}. However, obtaining alignments between languages is a time-consuming and resource-intensive process that heavily relies on human annotations. The transfer of knowledge through a limited number of alignments is often insufficient to fully enhance the TKG in the target language. 2) \textbf{Temporal knowledge discrepancy}: the information associated with two aligned entities is not necessarily identical, especially with regards to temporal patterns. Utilizing a rough approach to equate the aligned entities at all times can result in the transfer of misleading knowledge and negatively impact performance. This becomes more pronounced when the alignments are noisy and unreliable. For example, at the time step $t_2$, a new event about operating system ``{\it Ventura}'' from Apple company occurs in the source English TKG, and meanwhile there is a noisy aligned entity ``{\it Ventura city}'' in the target Japanese TKG. Directly pulling those two entities at this point, can inevitably introduce  noise and fail to predict a set of related events in the target TKG. Therefore, it is crucial to dynamically regulate the alignment strength of each local graph structure over time in order to maximize the effectiveness of cross-lingual knowledge distillation.

% Pulling those entities together cannot augment information in target languages. Small alignment strength is beneficial in the unreliable alignment cases, otherwise the misleading knowledge transferring can even hurt the performance.

% Moreover, in a case that the alignments are not fully reliable, directly pulling the two aligned entities together 


% optimally dynamic alignment strength
% {\em Optimal alignment strength to maximize the benefits of knowledge distillation is difficult to obtain, especially in the temporal manner.} 
% In practical, although the aligned entities can share similar information, they may still differ in other perspectives, including but not limited to frequency, interactions, and temporal patterns. How to adjust the alignment strength (i.e., the distance constrains of the aligned entities in the uni-space) accordingly for different entities at different time is unclear. \zheng{Ruijie TODO: add Ventura case}Moreover, in a case that the alignments are not fully reliable, directly pulling the two aligned entities together can even hurt the performance.



% scarcity of hinders the efficient
% knowledge transfer across languages. 
% {\em Transferring knowledge through a small set of alignments is hard to augment information for all entities.} 

% Aligning the same entities across languages rely heavily on manual labeling or rule-based inference~\cite{EA1,EA2,EA3,selfKG}, which is too time-consuming and impractical to obtain the alignments covering most of the entities in target language. 

% In this paper, we study how to boost the TKG reasoning performance in low-resource languages by explicitly increasing the completeness of those TKGs in history. Instead of improving the underlying information extraction techniques in low-data regime, we propose a new task called {\em Cross-lingual Temporal Knowledge Graph Reasoning}, motivated by the facts that there exists common or complementary knowledge shared by the TKGs in different languages under similar topics. The new task aims to facilitate TKG reasoning in low-resource languages (target languages) by distilling knowledge from a corresponding TKG in high-resource language (source language)  through a small set of entity alignments as bridges~\footnote{In this paper, we interchangeably use the terminology high-resource/low-resource and source/target for briety.}. Figure~\ref{fig:illustration} provides an illustrative example of the proposed task.


% Unfortunately, recent breakthroughs in temporal knowledge graph reasoning model~\cite{TA-DistMult,Know-Evolve,Renet,RE-GCN} highly rely on the completeness of the TKGs, especially for the most recent events. 

% However, the completeness of TKGs varies a lot across different languages, even under similar topics. Conventionally, the TKG construction process relies primarily on information extraction techniques built on the unstructured corpus~\cite{WIKI,YAGO, EventKG}. Therefore, the amount of corpus and human annotations in different languages significantly influence the quality of the corresponding TKGs . 
% Therefore, automatically completing/updating TKGs has been attracting enormous interests in recently years, which aims to predict recent/future events on TKGs based on historical events~\cite{TA-DistMult,Know-Evolve,Renet,RE-GCN}, namely temporal knowledge graph reasoning~\footnote{Broadly speaking, TKG reasoning includes interpolation to predict historical events and extrapolation to predict future events. In this paper, we refer to extrapolation task as TKG reasoning, since it is more vital for time-sensitive downstream tasks.}.


% For languages with large-scale and carefully labeled corpus (we refer to as high-resource languages, e.g., English), the constructed TKGs are more comprehensive than TKGs in other languages that lack the high-quality corpus (we refer to as low-resource languages, e.g., Spanish, Slovene, Danish, etc). Such completeness discrepancy leads to distinctly uneven TKG reasoning performances in different languages, which in turn affects the quality of service of the downstream applications. 


% Compared with the traditional TKG reasoning task, the new task imposes non-trivial challenges. An intuitive solution is to construct a unified graph including two TKGs in both source and target languages, and the knowledge distillation can be fulfilled by pulling the aligned entities from two languages close to each other in the uni-space~\cite{AlignKGC,KEnS}. However, there are still two challenges to be addressed. 

% \zheng{Ruijie TODO, Place this part to related works.}
% Existing works in related areas fail to address the aforementioned challenges. Monolingual reasoning methods on static/temporal knowledge graphs~\cite{TransE,TranR,ComplEX,RotatE,TA-DistMult,Know-Evolve,Renet,RE-GCN} is incapable of the desired knowledge transferring due to the insufficient alignment modeling. Although they can be extended on the cross-lingual scenario by viewing the alignments as a new relation on the merged TKGs, the limited amount of alignments prevent them from augmenting information for most of the entities. Entity alignment methods on KGs~\cite{EA1,EA2,EA3,EA4,EA5,selfKG} can automatically enlarge the alignments by  predicting the correspondence between the two TGs. But most of them, if not all, require the relatively even completeness of two TGs to capture the structural similarities, which can not be satisfied in our case, as target TKGs are far from complete. Some recent works start to study the multilingual TK reasoning on static graphs~\cite{AlignKGC,KEnS,SS-AGA}, which similarly aim to extract knowledge from several source KGs to boost the reasoning performance in the target KG, while they still require a sufficient amount of cross-lingual alignments and totally ignore the temporal perspective in our task.

% to facilitate temporal knowledge graph reasoning in low-resource languages. 
% increase the TKG connection and target TKG capacity
% In light of the mutual benefits, we iteratively generate pseudo alignment pairs and pseudo temporal events to address the co-existing scarcity issue in both cross-lingual alignment and target TKGs. 


In this paper, we propose a novel Mutually-paced Knowledge Distillation (\model) framework, where a teacher network learns more enriched temporal knowledge and reasoning skills from the source TKG to facilitate the learning of a student network in the low-data target one. The knowledge transfer is enabled via an alignment module, which estimates entity correspondence across languages based on temporal patterns. Firstly, to alleviate the limited language alignments (\textbf{Challenge \#1}), such a knowledge distillation process is mutually paced over time. This means, on one hand, we encourage the mutually interactive learning between the teacher and student. Concretely, the alignment module between the teacher and the student learns to generate pseudo alignment between TKGs to maximally expand the upper bound of knowledge transfer. And subsequently, it empowers the student to encode more informative knowledge in target TKG, which can in turn boost the alignment module to explore more reasonable alignments as the bridge across TKGs. One the other hand, inspired by self-paced learning~\cite{spl-1,spl-2}, we make the generations as a progressively easy-to-hard process over time. We start from generating reliable pseudo data with high confidence. As time goes by, we then gradually increase the generation amount by relieving the restriction over time. Secondly, to inhibit the temporal knowledge mismatch (\textbf{Challenge \#2}), the attention module can estimate the graph alignment strength distribution over time. This is achieved by a temporal cross-lingual attention in terms of the local graph structure and temporal-evolving patterns of aligned entities. As such, it can dynamically control the negative effect and suppress noise  propagation from the source TKG. Moreover, we provide a theoretical convergence guarantee for the training objective on both initial ground-truth data and pseudo data. To evaluate \model, we conduct extensive experiments of 12 cross-lingual TKG transfer tasks in multilingual EventKG dataset~\cite{EventKG}. Our empirical results show that the \model method outperforms state-of-the-art baselines in both with and without alignment noise settings, where only $20\%$ of temporal events in the target KG and $10\%$ of cross-lingual alignments are preserved.

% To validate the effectiveness of \model, we conduct extensive experiments of 12 cross-lingual TKG transfer tasks in multilingual EventKG benchmark dataset~\cite{EventKG} . Our experimental results empirically demonstrate the superiority of the \model method over state-of-the-art baselines, ranging from static KG embedding~\cite{TransE,TransR,DistMult,RotatE}, temporal KG reasoning~\cite{TA-DistMult,Renet,RE-GCN} to multilingual KG completion~\cite{KEnS,AlignKGC,SS-AGA}, in both with and without alignment noise settings. We further conduct comprehensive ablation and hyperparameter studies to validate the effectiveness of each design choices. Moreover, we provide theoretical analysis of convergence guarantee for the training objective on both initial groundtruth data and pseudo generative data.



To sum up, our contributions are three-fold:

\begin{itemize}[leftmargin = 15pt]
    \item \textbf{Problem formulation}: We propose the cross-lingual temporal knowledge graph reasoning task, to boost the temporal reasoning performance in target TKG by transferring knowledge from source TKG;
    \item \textbf{Novel framework}: We propose a novel \model framework, which enables the mutually-paced learning between the teacher and student networks, to promote both pseudo alignments and knowledge transfer reliability. Besides, \model involves a dynamic alignment estimation across TKGs that inhibits the influence of temporal knowledge discrepancy.
    \item \textbf{Extensive evaluations}: Empirically, extensive experiments on 12 cross-lingual TKG transfer tasks in multilingual EventKG benchmark dataset demonstrate the effectiveness of \model.
\end{itemize}
% pseudo data generation technique to progressively enhance the training data. The generated pseudo alignments can help the training of the representation modules by the knowledge distillation, and in turn adding pseudo events in the target TKG can improves alignment module by providing high-quality representations. 




% interactively
% TKGs in a source language and a target language are represented by a teacher representation module and a student one into a uni-space, respectively. 
% The knowledge distillation is enabled by a cross-lingual alignment module which pulls the aligned entities close to each other and push other entities far away. 
% To address the challenge caused by the scarcity of cross-lingual alignment, 


% !TEX root = main.tex
\section{Preliminaries}
\paragraph{Contexts} All along the
paper we use (many notions of) \emph{contexts}, \ie terms with exactly one hole, noted $\ctxhole$. Plugging a term $\tm$ in a context $\ctx$, noted $\ctxp{\tm}$, possibly~captures free variables of $\tm$. For instance $(\la\var\ctxhole)\ctxholep\var = \la\var\var$, while $(\la\var\vartwo)\isub\vartwo\var = \la\varthree\var$.


\paragraph{Preorders} We shall mostly deal with simulations, rather than bisimulations, since the result for equivalences shall always follow by simply considering the symmetric notions. Given a preorder/similarity $\precsim_X$ for some $X$, we  denote with $\simeq_X$ the corresponding equivalence/bisimilarity.

\paragraph{(In)Equational Theories and Compatibility} Good program preorder/equivalences are \emph{(in)equational theories}, that is, they contain the reduction of the calculus and they are \emph{compatible}, defined as: if $\tm\precsim\tmp$ then $\ctxp\tm \precsim\ctxp\tmp$ for all contexts $\ctx$, that is, that they are stable by context closure. Reduction is usually trivially included in similarities while compatibility is usually non-trivial to prove.


\paragraph{Contextual Equivalence} The standard of reference for program equivalences is contextual equivalence, that can be defined abstractly as follows.
\begin{definition}[Contextual Preorder and Equivalence] Given a language of terms $\mathcal{T}$ with its associated notion of contexts $\ctx$ and predicate stating the termination of evaluation $\bs{}_{e}$, we define the associated \emph{contextual preorder} $\leqcp{e}$ and \emph{contextual equivalence} $\equivcp{e}$ as follows:
	\begin{itemize}
		\item $\tm \leqcp{e} \tmp$ if $\ctxp\tm ~{\bs{}_{e}}$ implies $\ctxp\tmp ~{\bs{}_{e}}$ for all contexts $\ctx$ such that $\ctxp{\tm}$ and $\ctxp\tmp$ are closed terms. 
		\item $\tm \equivcp{e} \tmp$ is the equivalence relation induced by $\leqc$, that is, $\tm \equivcp{e} \tmp \iff \tm \leqcp{e} \tmp$ and $\tmp \leqcp{e} \tm$.
	\end{itemize}
\end{definition}


A relation $\relsym$ is sound for the contextual preorder $\leqcp{e}$ when $\relsym \subseteq \leqcp{e}$. Soundness follows from compatibility and \emph{adequacy for $\bs{}_e$}, defined as: if $\tm\rel\tmp$  then $\tm \bs{}_e$ implies $\tmp \bs{}_e$.


\begin{proposition}
	\label{prop:congruence-included-contextual-equivalence}
	Let $\precsim$ be a compatible and adequate preorder. Then $\precsim \subseteq \leqcp{e}$.
\end{proposition}

\begin{proof}
	Suppose $\tmtwo \precsim \tmtwop$.
	Let $\ctx$ any closing context for $\tmtwo$ and $\tmtwop$.
	By compatibility, $\ctxp{\tmtwo} \precsim \ctxp{\tmtwop}$.
	By adequacy, $\ctxp{\tmtwo} \bs{}_e$ implies $\ctxp{\tmtwop} \bs{}_e$, that is, $\tmtwo \leqcp e \tmtwop$.
\end{proof}

We shall see that normal form simulations are defined in such a way that they are adequate, so that soundness follows directly from compatibility. Note that soundness without compatibility is useless: the relation $\relsym \defeq \set{(\var\vartwo,\var\vartwo)}$ is sound but not compatible. 

\paragraph{Diamond} A rewriting notion that shall play a role is the \emph{diamond property}, which is the following one-step strengthening of confluence for a reduction $\to$: if $\tm \to \tmtwo_1$, $\tm\to\tmtwo_2$, and $\tmtwo_1 \neq \tmtwo_2$ then exists $\tmthree$ such that $\tmtwo_1 \to \tmthree$ and $\tmtwo_2 \to \tmthree$. Some well-known facts: the diamond property implies confluence but not vice-versa; if $\to$ is diamond and there is a terminating reduction from $\tm$ then there are no diverging reductions from $\tm$; all reductions to normal form, if any, have the same length. Roughly, the diamond property is a relaxed form of determinism, where non-deterministic choices have no impact on the result nor on the length of the evaluation leading to it.

\paragraph{Solvability and Scrutability} A cornerstone of the theory of the (call-by-name) $\l$-calculus is the study of  \emph{meaningless terms}, which can be used to model the \emph{undefined} in the representation of partial recursive functions in the $\l$-calculus. It is well-known that one cannot define meaningless terms as the $\beta$-diverging ones, as this induces an inconsistent theory. One rather needs to define them as the \emph{unsolvable terms}, which is a notion first studied by \citet{Wad:SemPra:71,DBLP:journals/siamcomp/Wadsworth76} and \citet{DBLP:books/daglib/0016519,solvability-barendregt} . We omit the definition of unsolvable terms, but mention that, as proved by Wadsworth, they can be characterized as those terms that diverge with respect to head reduction. An alternative notion of meaningless term is given by scrutable terms.

\begin{definition}[Testing contexts and (in)scrutability]
\label{def:cbn-scrutability}
Testing context are defined by:
\begin{center}
\textsc{Testing contexts}  \ \ \ $\tctx \grameq \ctxhole \mid \tctx \tm \mid (\la\var\tctx)\tm$
\end{center}
A term $\tm$ is \emph{\cbn scrutable} if there is a testing context $\tctx$ and a value $\val$ such that $\tctxp\tm \tob^* \val$, that is, a variable or an abstraction, otherwise $\tm$ is \emph{\cbn inscrutable}.
\end{definition}

For instance, $\Omega$ is inscrutable but $\la\var\Omega$ and $\var\Omega$ are scrutable. Inscrutable terms are the minimal terms with respect to the \cbn contextual preorder (with respect to termination on weak head normal forms). There is a useful characterization of scrutable terms that avoids mentioning testing contexts, due to \citet{parametricBook}, who adapt Wadsworth's for solvability. 
\begin{theorem}[Characterization of \cbn scrutability]
\label{thm:cbn-scrutability-characterization}
A term $\tm$ is scrutable if and only if the weak head reduction of $\tm$ terminates.
\end{theorem}

From the characterization, it follows that every inscrutable term is unsolvable, but not vice-versa.

In \cbv, there are analogous notions of (un)solvable and (in)scrutable terms, thoroughly studied by \citet{DBLP:journals/pacmpl/AccattoliG22}. Surprisingly, \cbv unsolvable terms are not a good notion of meaningless term, as they lead to an inconsistent theory. The right notion of meaningless term in \cbv is given by \cbv inscrutable terms, which are the minimum terms for the \cbv contextual preorder.

% !TEX root = main.tex
\section{Background About Normal Form Bisimulations}
Normal form bisimulations are program equivalences that, instead of comparing terms \emph{externally}, depending on how they behave \emph{in contexts}, they compare them \emph{internally}, by looking at the structure of their (infinitary) \emph{normal forms}. Let us give the simplest possible example.

Let $\towh$ be weak head reduction, also known as \cbn evaluation, which is a deterministic reduction and it is defined as  $(\la\var\tm)\tmtwo \tmthree_1 \ldots \tmthree_k\towh \tm\isub\var\tmtwo \tmthree_1 \ldots \tmthree_k$ with $k\geq 0$. Normal form simulations are usually based on a big-step presentation of $\towh$: we write $\tm\bswh\tmtwo$ if the $\towh$-evaluation of $\tm$ terminates on $\tmtwo$, and $\tm\bswhdiv$ otherwise.
%\adr{ We shall name $\leqcn$ and $\eqcn$ the associated notions of contextual preorder and equivalence.} 
The following notion of simulation was first considered by Sangiorgi \cite{SANGIORGI-normal-form-bisimulation}. Our presentation is slightly different but equivalent.
\begin{definition}[\cbn normal form simulations, \cite{SANGIORGI-normal-form-bisimulation}]
\label{def:cbn-nfs}
	A relation $\relsym$ is a \cbn normal form simulation if $\relsym\subseteq\relcbn$, where $\tm \relcbn \tmp$ holds whenever $\tm,\tmp$ satisfy one of the following clauses:
	\begin{center}
		$\begin{array}{r@{\hspace{.3cm}}r@{\hspace{.3cm}}l@{\hspace{.3cm}}l@{\hspace{.3cm}}lll}
	\textup{(cbn 1)} & && \tm\bswhdiv & \ie ~ \text{has no} \towh \text{-normal form.}
	\\
	\textup{(cbn 2)} & \tm \bswh \var & \text{and} & \tmp \bswh \var &
	\\
	\textup{(cbn 3)} & \tm \bswh \la\var\tm_1 & \text{and} &\tmp \bswh \la\var\tmp_1 & \text{with}~ \tm_1 \rel \tmp_1
	\\
	\textup{(cbn 4)} & \tm \bswh \ntm = \ntmONE\tmtwo&\text{and}& \tmp \bswh \ntmtwo = \ntmONEtwo \tmtwop &\text{with}~\ntmONE\rel\ntmONEtwo ~ \text{and}~ \tmtwo\rel\tmtwop
	\end{array}$
	\end{center}
%	\begin{enumerate}
%		\item $\tm\bswhdiv$;
%		\item $\tm \bswh \var$ and $\tmp \bswh \var$;
%		\item $\tm \bswh \la\var\tm_1$ and $\tmp \bswh \la\var\tmp_1$ with $\tm_1 \rel \tmp_1$;
%		\item $\tm \bswh \ntm = \ntmONE\ntmTWO$ and $\tmp \bswh \ntmtwo = \ntmONEtwo\ntmTWOtwo$ with $\ntmONE\rel\ntmONEtwo$ and $\ntmTWO\rel\ntmTWOtwo$.
%	\end{enumerate}
	\emph{\cbn normal form similarity} $\leqcbn$ is the largest \cbn normal form simulation, that is, it is the union of all normal form simulations.
	\end{definition}
Normal form bisimulations and bisimilarity are the symmetric variants of simulations and similarity, defined as expected.
A simple way of proving soundness of $\leqcbn$ is to show compatibility via the variant of Howe's method developed by Lassen in \cite{lassen1999bisimulation}. 

\paragraph{Partiality and Divergence.} Normal form simulations rely on a partial notion of evaluation (with respect to full $\beta$-reduction), such as weak head reduction $\towh$. The key point is that the partial reduction leaves some sub-terms not evaluated (arguments and abstraction bodies for $\towh$). The derived simulations compare the $\towh$-normal forms $\nfwh(\tm)$ and $\nfwh(\tmtwo)$ of $\tm$ and $\tmtwo$ by 
\begin{itemize}
\item Checking that they have the same structure for the partially evaluated part of the term, and
\item Asking that the non-evaluated sub-terms of $\nfwh(\tm)$ and $\nfwh(\tmtwo)$ in the same positions are pairwise normal form similar.
\end{itemize}
The use of a partial notion of evaluation is crucial, as it allows fine discriminations related to divergence, which would be blurred if one would only consider full normal forms. \cbn normal form similarity, indeed, discriminates between the following forms of divergence:
\begin{enumerate}
\item Looping as $\Omega$;
\item Looping only after having received an argument, as for $\la\var\Omega$;
\item Having a looping argument, as for $\var \Omega$;
\item The finite iterations of 2 and 3, and of their combination;
\item The \emph{infinite} iteration of 2 or 3, that never actually loop as $\Omega$ as they keep producing an infinite amount of head variables and/or abstractions. There exists terms $\tm$, indeed, the $\towh$ normal form of which is, for instance, $\la\var\tmtwo$ or $\var\tmtwo$ or $\var \la\vartwo\tmtwo$, and such that $\tmtwo$ has the same property. Therefore, they give rise to infinite normal forms such as $\la\var\la\var\la\var\ldots$ or $\var (\var (\var \ldots$ or $\var (\la\vartwo \var (\la\vartwo \ldots$.
\end{enumerate}

% \paragraph{Trees.} It turns out that Sangiorgi's notion of bisimilarity coincides with \levy-Longo tree equality, that is, the equality induced on terms by having the same \levy-longo tree, which is a standard notion of infinitary normal form for terms based on weak head reduction.
%
%Sangiorgi's (bi)simulations can be adapted to head reduction and then they no longer distinguish between the first and the second form of divergence above. The adapted notion of bisimilarity corresponds to \bohm tree equality, as shown by Lassen \cite{lassen1999bisimulation}, who also shows how to add $\eta$-equivalence. 


\paragraph{Open Terms.} Normal forms simulations are different from most other notions of program equivalence in that they have to deal with open terms, because, even when the terms to compare are closed, the sub-terms on which the comparison is iterated might be open.%, because the comparison goes under abstractions.



\paragraph{Easier to Use.} With respect to other notions of equivalence such as applicative bisimilarity, normal form bisimilarity is often simpler to establish, because it removes the quantification over arguments. A typical example is the proof of the equivalence of the Curry and Turing fix-point combinators $\curryfixn$ and $\turingfixn$, which is particularly simple with \cbn normal form bisimilarity, as Lassen explained in his first article relating \bohm tree equivalence and head normal form bisimulations \cite{lassen1999bisimulation}, a variant of Sangiorgi's. We can easily adapt his argument on head normal form bisimulations to weak head normal form bisimulations.
Let $\curryfixn = \la\var{\curryfixauxn\curryfixauxn}\text{, where } \curryfixauxn= \la\varthree{\var({\varthree\varthree})}$ and $ \turingfixn = (\la\varthree{\la\var{\var({\varthree\varthree\var})}})(\la\varthree{\la\var{\var({\varthree\varthree\var})}})$. It is easy to check that the following relation $\relsym$ is a \cbn normal form (bi)simulation relating $\curryfixn$ and $\turingfixn$.
\begin{center}
$\rel \defeq \{(\curryfixn,\turingfixn),(\curryfixauxn\curryfixauxn,\var(\turingfixn\var)), (\var(\curryfixauxn\curryfixauxn),\var(\turingfixn\var)),(\curryfixauxn\curryfixauxn,\turingfixn\var) \mid \var\text{ a variable}\}$
\end{center}

\paragraph{Equational Benchmarks} We say that a notion of program equivalence $\simeq_X$ \emph{validates} an equivalence $\equiv_Y$ if $\tm\equiv_Y\tmtwo$ implies $\tm\simeq_X\tmtwo$. Program equivalences in \cbn roughly can differ only along two axes:
\begin{enumerate}
\item \emph{$\eta$-equivalence}: the amount of $\eta$ equivalence that they validate (no $\eta$, finitely many $\eta$ expansions, infinitely many $\eta$-expansion, or even the further notion of infinite $\eta$-expansion, which is not the same of infinitely many standard $\eta$-expansions).
\item \emph{Scrutability/solvability}: the amount of identifications among inscrutable and unsolvable terms.
\end{enumerate}

Sangiorgi's \cbn normal form bisimilarity does not validate $\eta$-equivalence, because his bisimilarity is \emph{rigid}, it cannot equate different normal forms such as $\var$ and $\la\vartwo\var\vartwo$. Since $\eta$ equivalence is instead validated by contextual equivalence, $\eqcbn$ is not fully abstract. More generally, normal form bisimilarities tend to not be fully abstract. Intuitively, $\tm$ and $\tmp$ might be externally equivalent, that is, behave the same in all contexts, and yet be internally different, by having different (potentially infinitary) normal forms.


About scrutability, it follows from its characterization (\refthm{cbn-scrutability-characterization}) that \cbn normal form bisimilarity equates all inscrutable terms, that is, it validates the following equivalence. 
\begin{itemize}
\item \emph{(In)scrutable equivalence}: $\tm \equiv_{scr} \tmtwo$ if $\tm$ and $\tmtwo$ are \cbn inscrutable.
\end{itemize}
Instead, $\leqcbn$ does not equate all unsolvable terms, as it distinguishes $\Omega$ and $\la\var\Omega$ (namely $\Omega \leqcbn \la\var\Omega$ but $\la\var\Omega \not\leqcbn \Omega$). We shall see that in \cbv there is an analogous notion of \cbv inscrutability, but the reference \cbv normal form bisimulation does not equate all \cbv inscrutable terms.

% !TEX root = main.tex

\section{Plotkin's Call-by-Value and Open Terms}
\label{sect:plotkin}
Here we recall Plotkin's \cbv $\l$-calculus, and pay attention to some aspects that are often neglected, as they shall be relevant in the following sections. 

\paragraph{The \cbv $\l$-Calculus.} A \emph{value} $\val$ is a variable or an abstractions. At the rewriting level, we consider only weak evaluation, that is, out of abstractions. We define it in three variants, from left to right, from right to left, and in an unspecified order, which are discussed next.
\begin{center}
\begin{tabular}{c@{\hspace{.7cm}}c}
	$\begin{array}{r@{\hspace{.4cm}}rlll}
	\textsc{Terms} & \tm, \tmtwo, \tmthree & \grameq & \val \mid \tm\tmtwo 
	\\
	\textsc{Values} & \val  & \grameq  & \var \mid \la\var\tm

	\end{array}$
	
	&
	
	$\begin{array}{r@{\hspace{.4cm}}rlll}
	\textsc{(General) Contexts} & \ctx & \grameq &  \ctxhole\mid \tm\ctx\mid \ctx\tm \mid \la\var\ctx
	\\
	\textsc{Weak Contexts} & \wctx & \grameq &  \ctxhole\mid \tm\wctx\mid \wctx\tm
	\\
	\textsc{Left Contexts} & \lctx & \grameq &  \ctxhole\mid \val\lctx\mid \lctx\tm
	\\
	\textsc{Right Contexts} & \rctx & \grameq &  \ctxhole\mid \tm\rctx\mid \rctx\val
	\end{array}$
\end{tabular}
\end{center}

\begin{definition}[Reductions]
Let root $\betav$ reduction $\rtobv$ defined as $(\la\var\tm)\val \rtobv \tm\isub\var\val$. Then 
	$\betav$, weak, left (to right), and right (to left) reduction, noted $\tobv$, $\tow$, $\tolw$, and $\torw$, are defined as follows: if $\tm \rtobv \tmtwo$ then
	\begin{center}
		$\begin{array}{c@{\hspace{1cm}}c@{\hspace{1cm}}c@{\hspace{1cm}}c}
		\ctxp \tm \tobv \ctxp \tmtwo 
		&\wctxp \tm \tow \wctxp \tmtwo 
		&
		\lctxp \tm \tolw \lctxp \tmtwo 
		 &
		\rctxp \tm \torw \rctxp \tmtwo
		\end{array}
		$\end{center}
\end{definition}
It is standard that $\tolw$ and $\torw$ are deterministic and contained in $\tow$, which is non-deterministic but diamond, while $\tobv$ is non-deterministic and confluent but not diamond.

\paragraph{Closed Terms.} If terms are closed and evaluation is weak, which are a standard assumption in the study of functional languages, then call-by-value evaluation has some very nice properties, summed up by the next proposition.

\begin{proposition}
Let $\tm$ be a closed term.
\begin{enumerate}
\item $\tm$ is a weak/left/right normal forms if and only if $\tm$ is an abstraction.
\item On $\tm$, left and right reduction are full with respect to weak evaluation, that is, if $\tm \tow \tmtwo$ then there exist $\tmthree$ and $\tmfour$ such that $\tm \tolw \tmthree$ and $\tm \torw \tmfour$.
\end{enumerate}
\end{proposition}
Intuitively, left and right reduction are two equivalent ways of turning weak reduction into a deterministic reduction, on closed terms. 
In \cbv, contextual preorder $\leqcv$ is defined with respect to evaluation to a value, which can be equivalently expressed as termination of weak, left, or right reduction. Importantly, for $\leqcv$ one considers only the reduction of closed terms, as the definition of $\leqcv$ is based on contexts closing the terms to compare.

\paragraph{Open Terms, Stuck Redexes, and The Inequivalence of the Three Strategies.} As soon as one considers open terms, the good properties of the system break. The only ones which survive are the diamond property of weak reduction and the fact that left and right reduction are deterministic. But weak normal forms now have a complex shape, as there can be \emph{stuck redexes} such as $(\la\var\tm) (\vartwo\vartwo)$ which cannot be reduced because their argument is normal and not a value.

Stuck redexes unfortunately break the equivalence of left, right, and weak reduction: left, right, and weak normal forms are all different notions, because left and right reduction are no longer full with respect to weak reduction. For instance, $\tm \defeq \var\var(\Id\Id)$ is a left normal form which is not a weak/right normal form, because $\tm \tow \var\var\Id$, which is weak/right normal. Similarly, $\Id\Id(\var\var)$ is a right normal form which is not left/weak normal, and $\var\var(\Id\Id)(\var\var)$ is a left/right normal form which is not weak normal. Perhaps more worrying is the fact that stuck redexes introduce suspicious distinctions between contextually equivalent terms: $\Omega$ and $\Omega^L \defeq (\la\var\delta)(\vartwo\varthree)\delta$ are contextually equivalent, but the first one diverges while the second one is normal, because of the stuck redex.

%\paragraph{Improved Left and Right.} It is possible to recover the equivalence of left, right, and weak reduction by adopting less naive definitions of left and right contexts to define the corresponding reductions. Let $w$ denote a weak normal form (of which we omit the complex grammar). Considering the following variants on left and right contexts where values $\val$ have been replaced by $w$:
%\begin{center}
%	$\begin{array}{r@{\hspace{.4cm}}rlll}
%	\textsc{Improved Left Contexts} & \lctx^+ & \grameq &  \ctxhole\mid w\lctx^+\mid \lctx^+\tm
%	\\
%	\textsc{Improved Right Contexts} & \rctx^+ & \grameq &  \ctxhole\mid \tm\rctx^+\mid \rctx^+w
%	\end{array}$
%\end{center}
%The induced notions of improved left and improved right reduction are now equivalent to weak reduction. Eliminating the suspicious distinction between contextually equivalent terms instead requires to change the calculus, as we shall see in \refsect{vsc}.

\paragraph{On the Value of Variables} Another subtle aspect in the definition of \cbv is the notion of value. The $\rtobv$ root rule can be split in two root rules $\rtobabs$ and $\rtobvar$, depending on whether the substituted value is an abstraction or a variable, and all the defined reductions can be split accordingly. Note that, if terms are closed and evaluation is weak, then sub-rules based on $\rtobvar$ can never be applied, because arguments out of abstractions have to be closed, so they cannot be variables. This is one of the reasons why sometimes variables are not considered as values. Another reason is related to the efficiency of implementations, see \citet{DBLP:journals/iandc/AccattoliC17}. For us, variables are values and they are substituted, but in a later section we shall forbid them to be substituted.

% !TEX root = main.tex
\section{Equational Benchmarks for \cbv Program Equivalences}
\label{sect:benchmarks}
%%%%%%%%%%%%%%%%%%%%%%%%%%%%%%%
%%%%%%%%%%%%%%%%%%%%%%%%%%%%%%%

In \cbn, there are only two benchmarks, or degrees of freedom, for program equivalences, namely $\eta$-equivalence and the scrutable equivalence. In \cbv, the situation is richer, there are various equivalences that can be validated or not. Here we list the most relevant ones. We start by discussing the \cbv variants of $\eta$-equivalence and scrutability, and then present the equivalences that are found in known extensions of Plotkin's calculus.

For all the equivalences, we simply give the root axioms defining them, assuming that they are closed by all contexts. All the given equivalences (but \cbn erasure) are validated by \cbv contextual equivalence $\eqcv$. Some of them rather appear in the literature as reductions, but for the sake of uniformity we discuss them as equivalences. The meaning of some of the equivalences from extended \cbv calculi might seem obscure. They shall make more sense after the introduction of explicit substitutions in \refsect{vsc}.

In \cbv, $\eta$-equivalence has to be restricted, otherwise it turns non-values into values. At first sight, the by value version of $\eta$ seems to be $\val \equivetav \la\var{\val\var}$ if $\var\notin\fv\val$. But since any \cbv program equivalence validates $\betav$-reduction, the case $\val=\la\vartwo\tm$ is actually caught by $\betav$-reduction  (because $\la\var (\la\vartwo\tm)\var \tobv \la\var\tm\isub\vartwo\var =_\alpha \la\vartwo\tm$), so that $\equivetav$ simply amounts to the variable case.
\begin{itemize}
\item \emph{$\etav$ equivalence}: $\vartwo \equivetav \la\var{\vartwo\var}$ for every variable $\vartwo$.
\end{itemize}


\paragraph{\cbv Scrutability} Scrutability adapts to \cbv by simply replacing the use of $\tob$ with $\tobv$.
\begin{definition}[\cbv (in)scrutability]
\label{def:cbv-scrutability}
A term $\tm$ is \emph{\cbv scrutable} if there is a testing context $\tctx$ (see \refdef{cbn-scrutability}) and a value $\val$ such that $\tctxp\tm \tobv^* \val$, otherwise $\tm$ is \emph{\cbv inscrutable}.
\end{definition}
The equivalence to be validated here is the following one:
\begin{itemize}
\item \emph{(In)scrutable equivalence}: $\tm \equivscr \tmtwo$ if $\tm$ and $\tmtwo$ are \cbv inscrutable.
\end{itemize}
\cbv inscrutable terms are the right notion of \emph{meaningless term} in \cbv, as they are exactly the minimum ones for the contextual preorder $\leqcv$. In Plotkin's calculus, \cbv scrutable terms, however, cannot have an operational characterization analogous to that of \cbn scrutable terms (\refthm{cbn-scrutability-characterization})---they shall have one in the VSC. For instance, a term such as $\Omega^L \defeq (\la\var\delta)(\vartwo\varthree)\delta$ is \cbv inscrutable and yet it is normal for Plotkin, while it should diverge if a good characterization existed. 

The equivalence $\equivscr$ has a special role among those listed here because whether a term is (in)scrutable is \emph{undecidable}, so that the equivalence cannot be seen as computational principle to be tested via a rewriting rule. It is then all the more relevant that a program equivalence validates it.

\paragraph{Moggi} We now turn to equivalences found in extensions of Plotkin's calculus. The equivalences enriching $\betav$-conversion in Moggi's untyped computational $\l$-calculus are the following ones, here reformulated without ${\tt let}$-expressions:
\begin{itemize}
	\item \emph{Left identity}: $\Id \tm \equiv_{lid} \tm$, where $\Id=\la\var\var$ is the identity combinator;
	\item \emph{Associativity of lets}: $(\la\var\tm)((\la\vartwo\tmtwo)\tmthree) \equivass (\la\vartwo(\la\var\tm)\tmtwo)\tmthree)$ if $\vartwo \not \in \fv\tmtwo$;
	\item \emph{Left decomposition of applications}: $\tm\tmtwo \equivlad (\la\var\var\tmtwo)\tm$ if $\var\not\in\fv\tmtwo$;
	\item \emph{Right decomposition of applications}: $\val\tm \equivrad (\la\var\val\var)\tm$ if $\var\not\in\fv\val$. This one exists also in an extended form: $\tmtwo\tm \equivexrad (\la\var\tmtwo\var)\tm$ if $\var\not\in\fv\val$
\end{itemize}
Of them, the delicate one for our study is $\equivlid$, which in Plotkin's calculus holds only for values, as for values it is an instance of $\betav$-conversion, while in Moggi's holds for every term $\tm$.

\paragraph{Shuffling} Carraro's and Guerrieri's \emph{shuffling calculus} \cite{shufflingcalculus} extends Plotkin's calculus with the \cbv variants, noted $\tosone$ and $\tosthree$, of some shuffling rules introduced in \cbn by Regnier \cite{regnier94}. The induced equivalences follows.
\begin{itemize}
	\item $(\la\var\tm)\tmtwo\tmthree \equivsone (\la\var\tm\tmthree)\tmtwo$ if $\var\not\in\fv\tmthree$;
	\item $\val((\la\var\tm)\tmtwo) \equivsthree (\la\var\val\tm)\tmtwo$ if  $\var\not\in\fv\val$.
\end{itemize}
Note that Moggi's $\equiv_{ass}$ rule is an instance of $\equivsthree$.

\paragraph{Proof Nets} The \cbv translation of $\l$-calculus in linear logic proof nets, studied in detail by Accattoli \cite{Accattoli-proofnets}, equates various pairs of terms. The induced equivalences are better expressed with explicit subsitution, as we shall see in \refsect{benchmarks-vsc}, but we anticipate them here anyway. They subsume Moggi's $\equivass$ and the shuffling equivalences. Moreover, they include the following equivalences.
\begin{itemize}
	\item Extended $\sigma_3$: $\tmthree((\la\var\tm)\tmtwo) \equivexsthree (\la\var\tmthree\tm)\tmtwo$ if  $\var\not\in\fv{\tmthree}$;
	\item Commutativity: $(\la\vartwo(\la\var\tm)\tmtwo)\tmthree \equivcom (\la\var(\la\vartwo\tm)\tmthree)\tmtwo$ if $\var\notin\fv\tmthree$ and $\vartwo\notin\fv\tmtwo$.
\end{itemize}
The first one is a slight generalization of $\equivsthree$ replacing the value $\val$ with whatever term $\tmthree$. Commutativity swaps adjacent and unrelated redexes. It is a special equivalence, for at least two reasons. 
\begin{enumerate}
\item \emph{Effects}: commutativity holds in the pure \cbv setting but it often fails in extensions of \cbv with effects, because many effects are order-dependent (think of the order of writes on a memory cell). Therefore, it is an equivalence that one might want to be able to modularly add or remove from a notion of bisimilarity, rather than always validate it. 
\item \emph{Unorientable}: being symmetric, commutativity cannot be oriented as a rewriting rule. Therefore, any normal form bisimulation validating it needs to be able to compare normal forms up to some deformation of terms.
\end{enumerate}
The reason why some equivalences at times appear in restricted forms is also related to effects. The idea is that with non-commutative effects one has to fix a deterministic evaluation strategy, typically left-to-right, and this constrains the shape of equivalences forcing a sub-term to be a value $\val$, as in $\equivrad$ and $\equivsthree$. Proceeding right-to-left would relax them but force other dual constraints (such as $\tm$ being a value in $\equivlad$), and adopting a non-specified order induces the extended versions.

\paragraph{\cbn Duplication and \cbn Erasure} The last equivalences that we consider are \cbn duplication and \cbn erasure. We do not actually know how to characterize \cbn duplication independently of erasure with an axiom, or a set of axioms, but we discuss a specific case, to illustrate the idea.
\begin{itemize}
\item \emph{\cbn Duplication}: $(\la\var \vartwo \var \var) \tmtwo \equiv_{dup} \vartwo \tmtwo\tmtwo$;
\item \emph{\cbn Erasure}: $(\la\var \tm ) \tmtwo \equiv_{era} \tm$ with $\var\notin\fv\tm$.
\end{itemize}
In \cbv, duplication and erasure are included in $\betav$-conversion if $\tmtwo$ is a value, but not otherwise. For arbitrary terms, erasure is unsound in \cbv, because erasing a sub-term might turn divergence into termination: for instance $\tm\defeq (\la\var\varthree)(\vartwo\vartwo)$ is not contextually equivalent to $\tmtwo\defeq \la\var\varthree$ in \cbv, because $\ctxp\tm$ diverges while $\ctxp\tmtwo$ terminates with respect to the context $\ctx\defeq (\la\vartwo\ctxhole)\delta$. 

Duplication, instead, is sound for arbitrary terms, the idea being that terminating (resp. diverging) once, or terminating (resp. diverging) twice does not affect termination (resp. divergence). It is however a suspicious principle in \cbv. The cornerstone of \cbv is the idea that one should reduce arguments \emph{before} substituting them, and \cbn duplication does exactly the opposite. Intuitively, a \cbv program equivalence validating \cbn duplication is \emph{qualitative} or \emph{cost-insensitive}, as it only observes termination, while one rejecting it is somehow \emph{quantitative} or \emph{cost-sensitive}, as it distinguishes between differently efficient ways of realizing the same qualitative behavior.

\paragraph{Summing Up} Of all the discussed equivalences, the most relevant ones for our study are the scrutable equivalence $\equivscr$ and left identity $\equivlid$. Commutativity $\equivcom$ shall also be a source of inspiration for the modular \emph{mirror} approach of \refsect{net}. \cbn Duplication is validated by \cbv contextual equivalence while none of the program equivalences studied in this paper validates it. On the one hand, this fact shows that they are sound but not complete with respect to contextual equivalence. On the other hand, it shows that such incompleteness is not necessarily a negative fact,  because a cost-insensitive theory of \cbv is in some sense an oxymoron.
% !TEX root = main.tex
\section{Naive \cbv Bisimilarity}
\label{sect:naive}
%%%%%%%%%%%%%%%%
%%%%%%%%%%%%%%%%
If one takes Sangiorgi's \cbn nf-similarity $\leqcbn$ (\refdef{cbn-nfs}) and simply replaces weak head reduction with one of the weak \cbv reductions (weak, left, or right) then one obtains notions of \cbv nf-similarity. %Their compatibility can be proved with the method that will be explained in the next section, obtaining soundness with respect to contextual equivalence. 

We define the similarity induced by weak reduction. Left or right reductions induce different similarities but with the same pros and cons that are discussed below. Let $\bsweak $ and $\bsweakdiv$ be big-step termination and divergence with respect to weak \cbv reduction $\tow$.
\begin{definition}[Naive nf-bisimulation for Call-by-Value]
A relation $\relsym$ is a \emph{naive (\cbv) nf-simulation} if $\relsym\subseteq\relncbv$, where $\tm \relncbv \tmp$ holds whenever $\tm,\tmp$ satisfy one of the following clauses:
	\begin{center}
	$\begin{array}{r@{\hspace{.3cm}}r@{\hspace{.3cm}}l@{\hspace{.3cm}}l@{\hspace{.3cm}}lll}
	\textup{(nai 1)} & && \tm\bsweakdiv & \ie ~ \text{has no} \tow \text{-normal form.}
	\\
	\textup{(nai 2)} & \tm \bsweak \var & \text{and} & \tmp \bsweak \var &
	\\
	\textup{(nai 3)} & \tm \bsweak \la\var\tm_1 & \text{and} &\tmp \bsweak \la\var\tmp_1 & \text{with}~ \tm_1 \rel \tmp_1
	\\
	\textup{(nai 4)} & \tm \bsweak \ntm = \ntmONE\ntmTWO&\text{and}& \tmp \bsweak \ntmtwo = \ntmONEtwo\ntmTWOtwo &\text{with}~\ntmONE\rel\ntmONEtwo ~ \text{and}~ \ntmTWO\rel\ntmTWOtwo
	\end{array}$
\end{center}
%\begin{enumerate}
%		\item $\tm\bsweakdiv$;
%		\item $\tm \bsweak \var$ and $\tmp \bsweak \var$;
%		\item $\tm \bsweak \la\var\tm_1$ and $\tmp \bsweak \la\var\tmp_1$ with $\tm_1 \rel \tmp_1$;
%		\item $\tm \bsweak \ntm = \ntmONE\ntmTWO$ and $\tmp \bsweak \ntmtwo = \ntmONEtwo\ntmTWOtwo$ with $\ntmONE\rel\ntmONEtwo$ and $\ntmTWO\rel\ntmTWOtwo$.
%	\end{enumerate}
	\emph{Naive nf-similarity} $\leqncbv$ is the largest naive nf-simulation.
	\end{definition}

Naive (bi)similarity seems defined very naturally, and yet it does not validate \emph{any} of the equivalences of the previous section. Let's discuss $\equivomv$ and $\equivlid$.
\begin{enumerate}
\item
 \emph{$\Omega$-equivalence}: $\equivomv$ is not validated by $\leqncbv$, and $\Omega$-terms are not minimal for $\leqncbv$, in contrast with the fact that \cbn $\Omega$-terms are minimal for Sangiorgi's similarity $\leqcbn$. For instance, $\Omega$ is $\leqncbv$-minimal, but $\Omega^L$ is an $\Omega$-term and one has $\Omega \leqncbv \Omega^L$, because $\Omega\bsweakdiv$, but not $\Omega^L \not\leqncbv \Omega$, because $\Omega^L$ is $\tow$-normal.


\item \emph{Left identity}: the equivalence $\equivlid$ is not validated by $\leqncbv$. In Plotkin's calculus $\Id (\var\tm)$ does not reduce to $\var\tm$, because $\var\tm$ is not a value---more generally this happens for all normal open terms that are not values. Therefore, $\Id (\var\tm)$ and $\var \tm$ are $\leqncbv$-incomparable.
\end{enumerate}

\adr{\paragraph{$\beta_v$-Conversion}
	From the definition of $\leqncbv$, it immediately follows that naive bisimilarity contains the $\rtobv$ root rule, that is, that if $\tm \rtobv \tmtwo$ then $\tm\eqncbv \tmtwo$, simply because $\tm$ and $\tmtwo$ have the same left normal form. Since $\eqncbv$ is a compatible equivalence relation, it turns out that $\eqncbv$ contains the whole of $\betav$-conversion $=_{\betav}$\cadr{, thus it contains left reduction as well as weak and right reductions.}{.}
	
	\begin{proposition}[$\betav$-conversion is validated by naive bisimilarity]
		If $\tm =_{\betav} \tmtwo$ then $\tm \eqncbv \tmtwo$.
\end{proposition}}

\paragraph{Naive Similarity and Fix-Points} \cadr{It is nonetheless }{Despite its naivety, it is }possible to prove that the usual \cbv variants of Curry's and Turing's fix-point combinators $\curryfix$ and $\turingfix$ are naively similar, as we now show.
\begin{center}
$\begin{array}{c\colspace |\colspace  c}
\textsc{Curry's fix-point} & \textsc{Turing's fix-point}
\\
 \curryfix \defeq \la\var{\curryfixaux\curryfixaux}\text{, where } \curryfixaux= \la\varthree{\var\la\vartwo{\varthree\varthree\vartwo}}
 &
 \turingfix \defeq (\la\varthree{\la\var{\var\la\vartwo{\varthree\varthree\var\vartwo}}})(\la\varthree{\la\var{\var\la\vartwo{\varthree\varthree\var\vartwo}}}) 
 \end{array}$
 \end{center}
Let's build a naive (bi)simulation $\relsym$ relating $\curryfix$ and $\turingfix$. The relation $\relsym$ must contain the pair $(\curryfix,\turingfix)$.  Both terms $\tow$-evaluate to an abstraction ($\turingfix \to \la\var{\var\la\vartwo{\turingfix\var\vartwo}}$). Hence their weak normal forms are (abstractions) $\la\var{\curryfixaux\curryfixaux}$ and $\la\var{\var\la\vartwo{\turingfix\var\vartwo}}$ which must satisfy the third clause for $\leqncbv$, that is, their bodies under the abstraction $(\curryfixaux\curryfixaux,\var\la\vartwo{\turingfix\var\vartwo})$ must appear in $\relsym$. Since $\curryfixaux\curryfixaux \tow \var \la\vartwo{\curryfixaux\curryfixaux\vartwo}$, $\relsym$ must contain the fourth clause requirements, that is, $\var \rel \var$ and $\la\vartwo{\curryfixaux\curryfixaux\vartwo} \rel \la\vartwo{\turingfix\var\vartwo}$. As a result, $\var \rel \var$ needs to be added to the simulation (for all possible choice of variables when opening the abstraction at the first step), and we continue on adding $(\la\vartwo{\curryfixaux\curryfixaux\vartwo}, \la\vartwo{\turingfix\var\vartwo})$ to $\relsym$, then adding any pair of terms needed so that $\relsym\subseteq\relncbv$. By repeating this process, we eventually fall back to $\var\la\vartwo{\curryfixaux\curryfixaux\vartwo} \rel \var\la\vartwo{\turingfix\var\vartwo}$ which is already in the built relation $\relsym$, which means it now satisfies $\relsym \subseteq \relncbv$, that is, $\relsym$ is a naive simulation. Since similarly the symmetric relation $sym(\relsym)$ also satisfies $sym(\relsym) \subseteq \ncbvfp{sym(\relsym)}$, $\relsym$ is actually a naive \emph{bi}simulation.

The full relation proving the next proposition is:
\begin{center}$
\relsym \defeq \{(\var,\var) \mid \var \text{ a variable}\} ~\cup~ 
\{~~(\curryfix,\turingfix), (\la\var\curryfixaux\curryfixaux,\la\var{\var\la\vartwo{\turingfix\var\vartwo}}),\} ~~\cup $
\end{center} \begin{center}$
\{ (\curryfixaux\curryfixaux,{\var\la\vartwo{\turingfix\var\vartwo}}), ~~ (\var\la\vartwo{\curryfixaux\curryfixaux\vartwo},{\var\la\vartwo{\turingfix\var\vartwo}}),~~ (\la\vartwo{\curryfixaux\curryfixaux\vartwo},\la\vartwo{\turingfix\var\vartwo})\mid \var\text{ a variable}\}~~\cup
$\end{center}\begin{center}$
 \{({\curryfixaux\curryfixaux\vartwo},{\turingfix\var\vartwo}), ((\var\la\vartwop{\curryfixaux\curryfixaux\vartwop})\vartwo,({\var\la\vartwop{\turingfix\var\vartwop}})\vartwo)\mid \var\text{ and }\vartwo \text{ variables}\}$
\end{center} 

\begin{proposition}
\label{prop:naive-fix-points-equiv}
$\curryfix \eqncbv \turingfix$, that is, Curry's and Turing's fix-points are naive bisimilar.
\end{proposition}
% !TEX root = main.tex

\section{Lassen's eager normal form simulation}
The \cbv normal form simulation of reference in the literature is due to Lassen \cite{LassenEnf}.  Lassen's simulation is interesting because it is not defined by simply changing the notion of reduction in Sangiorgi's. Lassen indeed exploits stuck redexes in open terms, and defines a simulation which \emph{unstucks} them, a mechanism which we shall refer to as \emph{stop-and-go}.

\paragraph{Grammar of Left Normal Forms} Lassen's similarity is built using \emph{left reduction}. The starting point is the observation that, despite the limits of left reduction, it admits a description of its normal forms via left contexts which is extremely simple and elegant.
\begin{lemma}[Unique decomposition, Lassen \cite{LassenEnf}]
\label{l:las-unique-dec}
	Any (possibly open) term is either a value or admits a unique decomposition $\levctxp{\val\valtwo}$. In particular, left normal forms can be described as follows:
	\begin{center}
	$\begin{array}{c@{\hspace{.5cm}}rcc}
	\textsc{Left normal forms}  &
	\ntm,\ntmtwo & \grameq &  \val\mid \levctxp{\var\val}
	\end{array}
	$\end{center}

\end{lemma}



The simple contextual structure of left normal forms is then used by Lassen to define eager normal form simulation. The crucial clause is the fourth one, which realizes the stop-and-go mechanism.

\begin{definition}[\Enf simulation \cite{LassenEnf}]
	A relation $\relsym$ is an \emph{eager normal form (\enf) simulation} if $\relsym\subseteq\relenf$, where $\tm \relenf\tmp$ holds whenever $\tm,\tmp$ satisfy one of the following clauses:
	\begin{center}
		$\begin{array}{r@{\hspace{.3cm}}r@{\hspace{.3cm}}l@{\hspace{.3cm}}l@{\hspace{.3cm}}lll}
		\textup{(enf 1)} & &&\tm\bslasdiv & \ie ~ \text{has no} \tolw \text{-normal form.}
		\\
		\textup{(enf 2)} & \tm \bslass \var  &\text{and}& \tmp \bslass \var
		\\
		\textup{(enf 3)} & \tm \bslass \la\var\tmfirst &\text{and}& \tmp \bslass \la\var\tmpfirst 
		& \text{with} ~ \tmfirst \rel \tmpfirst
		\\
		\textup{(enf 4)} & \tm \bslass \levctxp{\var\val}  &\text{and}& \tmp \bslass \levctxtwop{\var\valtwo} 
		&
		\textup{with} ~ \val\rel\valtwo ~ \text{and}  ~ \levctxp{\varthree}\rel\levctxtwop{\varthree} 
		\\
		&& &&\text{where} ~ \varthree ~ \text{is not free in} ~ \levctx ~ \text{or} ~ \levctxtwo
		\end{array}
		$\end{center}
	\emph{\Enf similarity}, written $ \leqenf $, is defined by co-induction as the largest \enf simulation, that is, it is the union of all \enf simulations. We say that $\tm$ is \emph{\enf similar} to $\tmp$ if $\tm \leqenf \tmp$.
\end{definition}

\paragraph{Stop-and-Go, Double Task, and Left Identity.} The stop-and-go clause (enf 4) cleverly does two tasks at the same time, subsuming clause (nai 4) of naive similarity in a bottom-up way and capturing the left identity equivalence $\equivlid$. 

About (nai 4), consider for instance comparing $\tm \defeq \var\vartwo\varfour$ with itself via enf simulations. Clause (enf 4) reduces it to compare $\varthree\varfour$ and $\vartwo$ with themselves, since $\tm = \lctxp{\var\vartwo}$ with $\lctx\defeq \ctxhole\varfour$. Then, it reduces the first one to compare $\varthree'$ and $\varfour$ with themselves. In contrast, (nai 4) (or Sangiorgi's (cbn 4)) proceeds \emph{top-down}, by splitting $\var\vartwo\varfour$ into $\var\vartwo$ and $\varfour$, and then splitting $\var\vartwo$.

About the left identity equivalence, consider showing that $\tm \defeq \Id(\var\var)$ and $\tmtwo\defeq \var\var$ are enf similar. Evaluation is stuck on $\tm$, that is, it \emph{stops} because $\var\var$ is not a value. Note that $\tm$ has shape $\lctxp{\var\val}$ with $\lctx=\Id\ctxhole$ and $\val=\var$. The idea is that a term $\tmtwo$ that is $\leqenf$-similar to $\tm$ has to get stuck as well, or anyway decompose in a similar way. Now, $\tmtwo$ is not stuck, because there are no blocked redexes, but it has nonetheless shape $\lctxtwop{\var\valtwo}$ by taking $\lctxtwo = \ctxhole$ and $\valtwo = \var$. The comparison between $\tm$ and $\tmtwo$ is then reduced to compare the two pairs $\lctxp\varthree = \Id \varthree$ and $\lctxtwop\varthree=\varthree$, and $\val=\var$ and $\valtwo=\var$. The second pair trivially matches, because the identity relation is an enf simulation. About the first pair, note that $\Id\varthree$ is no longer stuck, that is, it can \emph{go}. And for the next round of comparison (of $\Id\varthree$ and $\varthree$) we have to first $\tolw$ reduce the terms, so that $\Id\varthree \tolw \varthree$ and thus also the first pair trivially matches.

Summing up, the enf simulation relating $\Id(\var\var)$ and $\var\var$ is the following  simulation $\relsym$: \[\relsym =\{(\Id(\var\var),\var\var), (\var,\var), (\Id\varthree,\varthree), (\varthree,\varthree)\}\]
This is  just an instance, but the following more general result holds.
\begin{proposition}[Enf bisimilarity validates left identity]
$\Id \tm \eqenf \tm$ for any term $\tm$.
\end{proposition}


\paragraph{Left/Right/Weak Non-Equivalent Variants} Replacing left reduction with right reduction, one obtains a unique decomposition lemma such as \reflemma{las-unique-dec} with respect to right contexts, and, accordingly, a notion of \emph{right Lassen similarity}---let us denote it with $\leqrenf$. It turns out that $\leqenf$ and $\leqrenf$ are different, incomparable similarities. For instance, $\Omega$ and $\Omega (\var\var)$ are enf bisimilar (because they are both $\tolw$-divergent) but not renf bisimilar, because $\leqrenf$ stops on $\Omega (\var\var)$ which is $\torw$ normal. Similarly, $\Omega$ and $\var\var\Omega $ are not enf bisimilar while they are renf bisimilar.

Replacing left reduction with weak reduction is instead problematic for another reason. Since $\tow$ is non-deterministic, the unique decomposition lemma (\reflemma{las-unique-dec}) fails for it. It is not clear then what would be the right definition of weak enf similarity, as the stop-and-go clause can be generalized in more than one way. An appropriate definition for weak enf similarity, as a generalization, should include terms related by enf similarity. However, it is also unclear (to us) how to prove the compatibility of some of such generalizations (we tried but failed\footnote{The definitions we can come up with for weak enf similarities that could be compatible are not able to relate as much terms as enf similarity does.}). 

The next paragraphs discuss the principles that are (in)validated by enf similarity.


\paragraph{$\beta_v$-Conversion}
From the definition of $\leqenf$, it immediately follows that enf bisimilarity contains the $\rtobv$ root rule, that is, that if $\tm \rtobv \tmtwo$ then $\tm\eqenf \tmtwo$, simply because $\tm$ and $\tmtwo$ have the same left normal form. Since $\eqenf$ is a compatible equivalence relation, it turns out that $\eqenf$ contains the whole of $\betav$-conversion $=_{\betav}$, thus it contains left reduction as well as weak and right reductions.

\begin{proposition}[$\betav$-conversion is validated by enf bisimilarity]
If $\tm =_{\betav} \tmtwo$ then $\tm \eqenf \tmtwo$.
\end{proposition}

\paragraph{Curry and Turing fix-Point Combinators are Enf Bisimilar.} It is easy to check that the relation $\relsym$ proving the naive similarity of Curry's and Turing's \cbv fix-point combinators is also a \enf bisimulation. The fact that those combinators are naive bisimilar means that the \emph{unstacking} aspect of the stop-and-go clause plays no role for their equivalence. 

\begin{proposition}
$\curryfix \eqenf \turingfix$, that is, Curry's and Turing's fix-points are enf bisimilar.
\end{proposition}

\paragraph{\cbv Scrutability} Enf bisimilarity does not validate the scrutable equivalence $\equivscr$. For instance, the same counter-example used for naive similarity works for enf, as we have $\Omega \leqenf \Omega^L$ but $\Omega^L \not\leqenf \Omega$. In fact, $\eqenf$ equates some inscrutable terms that are separated by $\eqncbv$ and vice-versa. For instance, let $\delta_3 \defeq \la\var\var\var\var$. The term $\Omega_3 \defeq \delta_3\delta_3$ is divergent and \cbv inscrutable. We have that $\Omega^L_3 \defeq (\la\var\delta_3)(\vartwo\varthree)\delta_3$ is enf bisimilar to $\Omega^L$, because they stop similarly and then both go to diverge. They are instead unrelated with respect to $\leqncbv$, because when compared as normal forms they do not have the same structure. For the vice-versa, consider $\var\var\Omega$ and $\Omega$, which are equated by $\eqncbv$ but separated by $\eqenf$, because $\var\var\Omega$ is left normal but weak divergent. 

\begin{proposition}
Enf bisimilarity does not validate scrutable equivalence $\equivscr$.
\end{proposition}


\paragraph{Further Equivalences.} The next proposition sums up the benchmarks for enf.
\begin{toappendix}
\begin{proposition}
\label{prop:enf-validation-of-equivalences}
Enf bisimilarity validates Moggi's equivalences and the shuffling equivalences, but it does not validate $\etav$, the proof nets equivalences, nor \cbn duplication.
\end{proposition}
\end{toappendix}
 Enf similarity does not validate $\etav$, since $\var$ and $\la\vartwo\var\vartwo$ are handled by different clauses in the definition of $\leqenf$. It can however be adapted to validate it, see \cite{LassenEnf,lassen+strovring-bisimilarity-eta,DBLP:journals/lmcs/BiernackiLP19}. 
About Moggi's equivalences, we have already discussed the left identity. Enf similarity validates all the other ones. Note that in \cite{LassenEnf} Lassen claims that enf validates $\equiv_{exrad}$ which is actually false, it only validates $\equiv_{rad}$ (in fact $\equiv_{exrad}$ does not correspond to Moggi's usual right decomposition rule, it shall be motivated by proof nets in \refsect{benchmarks-vsc}). Proofs of the validations are easy, one only needs to write the right relation (the identity relation $\cup$ the equivalence to validate) and show that it is indeed an \enf bisimulation.

The validation of the shuffling equivalences is an easy contribution of this paper. 
Simple inspections show that enf does not validate the proof nets equivalences $\equivexsthree$ and $\equivcom$, nor \cbn duplication. Consider now renf bisimilarity, the right variant of enf. With respect to $\sigma$-equivalences, it as a sort of dual behavior: it  does not validate $\equivsone$ but validates $\equivsthree$ and even $\equivexsthree$.
% !TEX root = main.tex

\section{Value Substitution Calculus}
\label{sect:vsc}
% !TEX root = main.tex
\begin{figure}
\begin{tabular}{c}
	\!\!\!\!\!\!\!\!\!
\begin{tabular}{cc}

$\arraycolsep=3pt
\begin{array}{rrll}
\multicolumn{4}{c}{\textsc{Language}}
\\
\textsc{Terms } & \vsubterms \ni \tm,\tmtwo, \tmthree & \grameq& \val \mid \tm\tmtwo 
\mid \tm \esub\var\tmtwo 
\\
\textsc{Values } & \val,\valtwo & \grameq & \var \mid  \la\var\tm 
\\[4pt]
\textsc{Sub. ctxs } &\sctx,\sctxtwo  &\grameq &\ctxhole \mid \sctx \esub\var\tm
\end{array}$
&
$\begin{array}{rr@{\ }l@{\ }l}
\multicolumn{4}{c}{\textsc{Root rules}}
\\
    \textsc{Mult. } & \sctxp{\la\var\tm}\tmtwo &  \rtom  & \sctxp{\tm\esub{\var}{\tmtwo}} 
    \\
    \textsc{Exp.}  & \tm\esub\var{\sctxp{\val}} &  \rtoe  & \sctxp{\tm\isub{\var}{\val}} 
    \\[4pt]
    \textsc{Exp. abs}  & \tm\esub\var{\sctxp{\la\vartwo\tmtwo}} &  \rtoeabs  & \sctxp{\tm\isub{\var}{\la\vartwo\tmtwo}} 
\\
    \textsc{Exp. var}  & \tm\esub\var{\sctxp{\vartwo}} &  \rtoevar  & \sctxp{\tm\isub{\var}{\vartwo}} 
\end{array}$    
\end{tabular}
\\[28pt]
\hline
\\[-10pt]
%%%%%%%%%%%%%%%%%
%%%%%%% begin open evaluation
\tabcolsep = 2pt
%	\fbox{
		\!\!\!\!\!\!\!\!\!\!
	\begin{tabular}{c}
\textsc{Reduction + normal forms}
	\\[6pt]
	$\begin{array}{r@{\hspace{.15cm}}r@{\hspace{.1cm}}l@{\hspace{.1cm}}ll}
	\textsc{Evaluation Contexts} & \evctx & \grameq &  \ctxhole\mid \tm\evctx\mid \evctx\tm \mid \evctx\esub{\var}{\tmtwo} \mid \tm\esub{\var}{\evctx}
\\[4pt]
	\textsc{Rewriting rules:}	
	&
	\multicolumn{3}{l}{
	\AxiomC{$\tm \rootRew{a} \tm'$}
	\UnaryInfC{$\evctxp{\tm} \Rew{a} \evctxp{\tm'}$}
	\DisplayProof
		\ \ \ 
		 a \!\in\! \set{\msym,\esym,\expoabs,\expovar}
		 }
	\\[10pt]
	\textsc{Notations} & \ \tovsc   & \defeq  & \tom \cup \toe \quad\ \  \tovscp    \ \defeq\   \tom \cup \toeabs
	\end{array}$
	\\\\[-6pt]
	$\arraycolsep = 2pt
	\begin{array}{l@{\hspace{.3cm}}lll@{\hspace{1cm}}l@{\hspace{.3cm}}llll}
	\textsc{Inert terms}  &
	 \itm,\itmtwo & \grameq & \var \mid \itm \fire \mid \itm \esub{\var}{\itmtwo}
	&
	\tovscp\textsc{-Normal forms} 
	& \ntm,\ntmtwo &\grameq &\val \mid \itm \mid \fire \esub{\var}{\itm}
	\end{array}$
\end{tabular}
\end{tabular}
\caption{Value substitution calculus and fireballs.}
\label{fig:vsc}
\end{figure}

%\begin{tabular}{ccccccc}
%\end{tabular}
%
%
%\begin{array}{ccccccc}
%\end{array}
 
%Among the many extensions of Plotkin's calculus, Accattoli and Paolini's \emph{value substitution calculus} \cite{accattoli+paolini-vsc} (abbreviated as VSC) has a special place. It was conceived as a graph-free presentation of linear logic proof nets for the \cbv $\l$-calculus by Accattoli \cite{Accattoli-proofnets}, it was used to study reasonable time for \cbv by Accattoli et al. \cite{DBLP:conf/lics/AccattoliCC21}, and to refine the understanding of meaningless terms in \cbv by Accattoli and Guerrieri \cite{DBLP:journals/pacmpl/AccattoliG22}. Additionally, Accattoli and Guerrieri  relate it to three other extensions of Plotkin's calculus, among which the mentioned shuffling calculus, and prove that they are all equivalent with respect to the study of termination \cite{accattoli+guerrieri-opencbv}.

Intuitively, the VSC is a \cbv $\lambda$-calculus extended with $\letexp$-expressions, as is common for \cbv $\l$-calculi such as \citeauthor{DBLP:conf/lics/Moggi89}'s one [\citeyear{Moggi88tech,DBLP:conf/lics/Moggi89}]. 
We do however replace a $\letexp$-expression $\letin\var\tmtwo\tm$ with a more compact  \emph{explicit substitution} (ES for short) notation $\tm\esub{\var}{\tmtwo}$, which binds $\var$ in $\tm$ and that has precedence over abstraction and application (that is, $\la\var\tm\esub\vartwo\tmtwo$ stands for $\la\var(\tm\esub\vartwo\tmtwo)$ and $\tm\tmthree\esub\vartwo\tmtwo$ for $\tm(\tmthree\esub\vartwo\tmtwo)$). Moreover, our $\letexp$/ES does not fix an order of evaluation between $\tm$ and $\tmtwo$, in contrast to many papers in the literature (\eg \citet{DBLP:journals/toplas/SabryW97,DBLP:journals/iandc/LevyPT03}) where $\tmtwo$ is evaluated first.

The reduction rules of VSC are slightly unusual as they use \emph{contexts} both to allow one to reduce redexes located in sub-terms, which is standard, \emph{and} to define the redexes themselves, which is less standard---these kind of rules is %sometimes 
called \emph{at a distance}. The rationale behind is that the rewriting rules are designed to mimic exactly cut-elimination on linear logic proof nets, via Girard's \cite{DBLP:journals/tcs/Girard87} \cbv translation $(A \Rightarrow B)^v = ! (A^v \multimap B^v)$ of intuitionistic logic into linear logic, see \citet{Accattoli-proofnets}.  


\paragraph{Root rewriting rules}
In VSC, 
there are two main rewrite rules, the \emph{multiplicative} one $\tom$ and the \emph{exponential} one $\toe$ (the terminology comes from the connection between VSC and linear logic), and both work \emph{at a distance}: they use contexts even in the definition of their \emph{root} rules (that is, before the contextual closure). Their definition is based on \emph{substitution contexts} $\sctx$, which are lists of~ES. 
In \Cref{fig:vsc}, the root rule $\rtom$ (resp. $\rtoe$) is assumed to be capture-free, so no free
 variable of $\tmtwo$ (resp. $\tm$) is captured by the substitution context $\sctx$ (by possibly $\alpha$-renaming on-the-fly). 


Examples: $(\la\var\vartwo)\esub\vartwo\tm\tmtwo \rtom \vartwo\esub\var\tmtwo\esub\vartwo\tm$ and $(\la\varthree\var\var)\esub\var{\vartwo\esub\vartwo\tm} \rtoe (\la\varthree\vartwo\vartwo)\esub\vartwo\tm$. An example with on-the-fly $\alpha$-renaming is $(\la\var\vartwo)\esub\vartwo\tm\vartwo \rtom \varthree\esub\var\vartwo\esub\varthree\tm$.

A key point is that $\beta$-redexes are decomposed via ES: the \emph{by-value} restriction is on ES-redexes, \emph{not} on $\beta$-redexes, because only values can be substituted.
The multiplicative rule
$\rtom$ fires a $\beta$-redex at a distance and generates an  ES even when the argument is not a value.
	The \cbv discipline is entirely encoded in the exponential rule $\toe$ (see \Cref{fig:vsc}): it can fire an  ES performing a substitution only when its argument is a \emph{value} (\ie a variable or an abstraction) up to a list of ES. This means that only values can be duplicated or erased.
It is useful to split the exponential root rule $\rtoe$ in two disjoint rules, depending on whether it is an abstraction (rule $\rtoeabs$) or a variable ($\rtoevar$) that it is substituted. 

\paragraph{Rewriting Rules} We close the root rules by evaluation contexts $\evctx$, which allow reduction everywhere but under abstractions. In other papers about the VSC \cite{accattoli+paolini-vsc,accattoli+guerrieri-opencbv,DBLP:conf/lics/AccattoliCC21,DBLP:journals/pacmpl/AccattoliG22}, where other contextual closures are also considered, they are called \emph{weak} or \emph{open} contexts. We consider both the reduction that includes the substitution of variables $\toevar$, noted $\tovsc$, and the one that does not, noted $\tovscp$ (note that $\tovscp \,=\, \tovsc\smallsetminus \toevar$) where $\symfont{p}$ stands for \emph{practical} because it is in implementations of \cbv that variables cannot be substituted, see \citet{DBLP:journals/iandc/AccattoliC17}.  Examples:
\begin{align*}
		\tm \esub\var{(\la\vartwo\tmtwo)\esub\varthree\tmthree \tmfour}
		& \tom \tm \esub\var{\tmtwo\esub\vartwo\tmfour\esub\varthree\tmthree}
		&
		\tm (\var\var) \esub\var{\vartwo\esub\varthree\tmtwo} & \toevar\! \tm (\vartwo\vartwo)\esub\varthree\tmtwo
		\\[-2pt]
		((\var\var) \esub\var{\la\vartwo\varthree} \tm)\esub\varfour\tmtwo 
		& \toeabs\! ((\la\vartwo\varthree) (\la\vartwo\varthree) \tm)\esub\varfour\tmtwo
		&
		\la\varthree (\var\var) \esub\var{\la\vartwo\varthree} & \not\toeabs\!  \la\varthree(\la\vartwo\varthree) \la\vartwo\varthree
\end{align*}

\paragraph{Diamond} Both $\tovsc$ and $\tovscp$ are non-deterministic, as for instance:
\begin{center}
$\delta (\delta \Id) \ \lRew{\expoabs}\  \vartwo\esub\vartwo\delta (\delta \Id) \ \tom \ \vartwo\esub\vartwo\delta((\var\var)\esub\var\Id).$
\end{center}
They are however more than confluent, they are diamond.
\begin{proposition}[Diamond \cite{DBLP:journals/pacmpl/AccattoliG22}]
\label{prop:vsc-diamond}
Let $a\in\set{\vscsym,\vscpsym}$. Then $\Rew{a}$ is diamond.
\end{proposition}
%The diamond property is a strong form of confluence (note that it holds for \emph{steps}, rather than \emph{sequences}, which is stronger) with relevant consequences. Firstly, it implies confluence. Secondly, it means that non-determinism is only apparent, because if an $\Rew{a}$-reduction sequence from $\tm$ reaches
%a $a$-normal form $\tmtwo$, then \emph{every} $\Rew{a}$-sequence from $\tm$ eventually ends in $\tmtwo$;
%and all these sequences have the \emph{same length} and \emph{same number} of $\msym$-steps and $\esym$-steps. Such a \emph{length invariance} shall be crucial for the proof of compatibility for the similarities of the next sections.

\paragraph{Irrelevance of $\toevar$} Splitting $\toe$ in $\toeabs$ and $\toevar$ is motivated by the fact that the removal of $\toevar$ does not alter the rewriting properties of the calculus, that is, $\toevar$ is postponable and the removal does not change the notion of termination, what Accattoli and Guerrieri deem (operational) \emph{irrelevance} \cite{DBLP:journals/pacmpl/AccattoliG22}. Removing $\toevar$, however, does change the equational theory, thus $\toevar$ shall be relevant for us. More precisely, at first we shall remove it, to present a simplified version of similarity for the VSC, and then we shall re-introduce it.

\begin{proposition}[$\toevar$ postponement, \cite{DBLP:journals/pacmpl/AccattoliG22}]
If $\tm \tovsc^* \tmtwo$ then there exists $\tm \tovscp^*\tmthree\toevar^* \tmtwo$. Moreover, $\tm$ is $\tovsc$-terminating if and only if it is $\tovscp$-terminating.
\end{proposition}

Actually, the reduction $\tovscp$ without $\rtoevar$ has stronger properties. In particular, it admits a neat inductive description via \emph{inert terms}, as we now explain.

\paragraph{Inert Terms and Normal Forms} 
\cbv is about \emph{values}, and, if terms are closed, normal forms are abstractions. In going beyond the closed setting,  a finer and more general view is required. First, normal forms (for $\tovscp$) are given by mutual induction with the notions of \emph{inert term}, as in \Cref{fig:vsc}. 
Second, variables are \emph{both} values and inert terms. This is on purpose, because they have the properties of both kinds of term.

Examples: $\la\var\vartwo$ is a normal form 
as an abstraction, while $\vartwo(\la\var\var)$, $\var\vartwo$, and $(\varthree(\la\var\var))(\varthree\varthree) (\la\vartwo(\varthree\vartwo))$ are normal forms as inert terms. The grammars also allow to have ES containing inert terms around abstractions and applications: $(\la\var\vartwo)\esub\vartwo{\varthree\varthree}$ is a fireball and $\var\esub\var{\vartwo(\la\var\var)}\vartwo$ is an inert term. One of the key points of inert terms is that they have a \emph{free} head variable (in particular they are open). In  
\cite{DBLP:conf/icfp/GregoireL02}, inert terms are called \emph{accumulators}, and normal forms are  called \emph{values}. In some papers about the VSC, normal forms are called \emph{fireballs}.

 Note that $\var\esub\var\vartwo$ is an inert term and it is not $\toevar$ normal, thus not $\tovsc$ normal. Normal forms for $\tovsc$ are a slightly stricter subset of $\tovscp$-normal forms (they cannot have ES of shape $\esub\var{\sctxp\vartwo}$), with a similar but less neat description that shall be given in \refsect{net}.

\begin{proposition}[Normal forms \cite{DBLP:journals/pacmpl/AccattoliG22}]
$\tm$ is $\tovscp$-normal if and only if $\tm$ is a $\ntm$-term as defined in \Cref{fig:vsc}. 
		If $\tm$ is $\tovsc$-normal then it is a $\ntm$-term.
\end{proposition}


\paragraph{\cbv Scrutability} One of the features of the VSC is that it solves the issues of Plotkin's calculus with respect to \cbv scrutability. Consider the term $\Omega^L =  (\la\var\delta)(\vartwo\varthree)\delta$ which is inscrutable and contextual equivalent to $\Omega$ but normal for Plotkin. In the VSC, instead, it diverges:
\begin{center}
$ (\la\var\delta)(\vartwo\varthree)\delta \tom \delta \esub\var{\vartwo\varthree}\delta \tom \varfour\varfour\esub\varfour\delta \esub\var{\vartwo\varthree} \toeabs \delta\delta \esub\var{\vartwo\varthree} \tom \ldots$.
\end{center}
More generally, \emph{all} \cbv inscrutable terms diverge: the following characterization of \cbv (in)scrutability holds, analogous of the one for \cbn (\refthm{cbn-scrutability-characterization}) and due to \citet{accattoli+paolini-vsc}.
\begin{theorem}[VSC characterization of \cbv scrutability, \cite{accattoli+paolini-vsc}]
	\label{thm:cbv-scrutability-characterization}
A term $\tm$ is \cbv scrutable if and only if $\tm$ is $\tovsc$ (or, equivalently, $\tovscp$) terminating.
\end{theorem}
Despite the fact that the VSC makes $\Omega^L$ diverge as in \cbn, it does \emph{not} validate \cbn duplication nor \cbn erasure, as $(\la\var\vartwo)\Omega$ is $\tovsc$-divergent and $(\la\var \vartwo \var \var) (\varthree\varthree) \tom (\vartwo \var \var)\esub\var{\varthree\varthree} \not\tovsc \vartwo (\varthree\varthree)(\varthree\varthree)$.

\paragraph{Contextual Equivalence} Terms such as $\Omega^L$ are normal for Plotkin but divergent in the VSC, that is, the VSC and Plotkin's calculus have different notions of termination. One might then suspect that contextual equivalence in the VSC is not the same as in Plotkin's calculus. This is not the case, in fact, because the VSC behaves differently \emph{only on open terms}, while contextual equivalence is defined reducing closed terms only. The following result is due to \citet{DBLP:journals/pacmpl/AccattoliG22}.

\begin{proposition}[\cite{DBLP:journals/pacmpl/AccattoliG22}]
Two $\l$-terms without ES are contextual equivalent in Plotkin's calculus if and only if they are contextual equivalent in the VSC.
\end{proposition}

%\paragraph{Next} In the next section, we see a toy notion of normal form similarity for the VSC, to familiarize the reader with the proof technique for compatibility. After that, we shall rephrase in the VSC the benchmark equivalences discussed in \refsect{degrees} and finally define the actual notion of normal form similarity for the VSC we are interested in.
% !TEX root = main.tex

\section{Toy Similarity and Lassen's Method for the VSC}
\label{sect:toy}
In this section, we introduce a toy notion of similarity for the VSC based on the reduction $\tovscp$, the one that does not substitute variables. We denote this calculus with \VSCptxt  and, in this section, we use the notation $\valp$ to refer to abstractions. The aim is to provide a gentle introduction Lassen's variant \cite{lassen1999bisimulation} of Howe's method \cite{Howe1996method,DBLP:books/cu/12/Pitts12} for proving the compatibility of similarities, and to delay some of the technicalities that we shall need to address for the similarity we are really interested in.

\paragraph{From Small-Step to Big-Step.} Normal form similarities look at normal forms, and the crucial proof in Howe's method proceeds by induction on a big-step formulation of evaluation, where \emph{big-step} means that it relates a terminating term directly with its normal form, hiding the intermediate steps. Therefore, we need to reformulate the small-step reduction $\tm\tovscp^k\ntm$ where $\ntm$ is $\tovscp$-normal, in a big-step manner as $\tm \bsvscps \ntm$. 

For the technical development, we need to keep the information about the number $k$ of small steps, that is, we shall rather write $\tm \bsvscp k \ntm$. Such a quantitative information is needed both to prove the equivalence with small-step evaluation and for the crucial proof in the method. We also need the notion of inert substitutions contexts.
\begin{center}
	$\begin{array}{c@{\hspace{.5cm}}rcc}
	\textsc{ Inert Substitution Contexts} & \isctx & \grameq &  \ctxhole\mid \isctx\esub{\var}{\itm}
	\end{array}
	$\end{center}

\begin{definition}[Big-step \VSCptxt evaluation $\bsvscps$]
The big-step \VSCptxt evaluation predicate $\tm \bsvscp k \ntm$, read \emph{$\tm$ (\VSCptxt)-converges in $k$ steps to $\ntm$}, is defined as follows.
\begin{center}
	% !TEX root = main.tex
\begin{tabular}{cccc}
	\begin{tabular}{c\colspace\colspace c\colspace\colspace ccc}
	\infer[(\bslasax)]{\val \bsvscp 0 \val}{}
	&
	\infer[(\bslasapi)]{\tm\tmtwo \bsvscp {k+h} \itm\ntm}{
	\tm \bsvscp k \itm
	&
	\tmtwo \bsvscp h \ntm
	}
	&
	\infer[(\bslasesi)]{\tm\esub\var\tmtwo \bsvscp {k+h}\ntm\esub\var\itm}{
	\tm \bsvscp k \ntm
	&
	\tmtwo \bsvscp h \itm
	}
	\end{tabular}
	\\[12pt]
	\begin{tabular}{c\colspace ccc}
	\infer[(\bslasapm)]{\tm\tmtwo \bsvscp {k+i+1} \isctxp\ntm}{
		\tm \bsvscp k \isctxp{\la\var\tmthree}
		&
		{\tmthree\esub\var\tmtwo} \bsvscp i \ntm
	}
	&
	\multicolumn{2}{c}{\infer[(\bslasese)]{\tm\esub\var{\tmtwo} \bsvscp {k+i+1} \isctxp\ntm}{
			\tmtwo \bsvscp k \isctxp{\la\vartwo\tmthree}
			&
			{\tm\isub\var{\la\vartwo\tmthree}} \bsvscp i \ntm} }
	\end{tabular}
\end{tabular}
\end{center}
\end{definition}

Notation: $\tm \bsvscps \ntm$ abbreviates \emph{there exists a $k$ such that $\tm \bsvscp k \ntm$}. Rule $(\bslasax)$ applies to both abstractions and variables ($\val = \valp$ or $\var$).
\begin{toappendix}
\begin{proposition}[Equivalence of small-step and big-step in \VSCptxt]
	\label{l:ss-bs-equivalence_vsc}
	$\tm \bsvscp k \ntm$ if and only if $\tm \tovscp^k \ntm$ with $\ntm$ is normal.
\end{proposition}
\end{toappendix}
%\begin{proof}
%	Full proof is detailed in the Appendix \ref{proof:proof-completeness-big-step-vsc-practical}.
%	
%	During this internship we developed two proofs of this proposition: one which is direct (that requires a double induction and is quite long) and one which is based on a complete subreduction which is closer to the big-step system approach. In the appendix we detail the second one, introducing a complete subreduction which has the same normal forms as $\tovsc$ and follows the same thinking as the big-step system.
%	
%\end{proof}
Let us stress an important point. We recall that $\tovscp$ is non-deterministic but diamond. The diamond property is here crucial, in order to make sense---at the big-step level---of the number of steps $k$, which for a diamond reduction does not depend on the reduction path to normal form.

\paragraph{Toy Normal Form Similarity} We adapt Sangiorgi's normal form similarity to the \VSCptxt. Since the language of terms has an additional constructor, the ES, there is one additional clause

\begin{definition}[\Naf simulations and similarity]
	Let $\relsym$ be a relation on $\l_{vsc}$-terms. $\relsym$ is a \naf simulation if $\relsym\subseteq\relnaf$, where $\tm \relnaf\tmp$ holds whenever $\tm,\tmp$ satisfy one of the following clauses:
	\begin{center}
		$\begin{array}{r@{\hspace{.3cm}}r@{\hspace{.3cm}}l@{\hspace{.3cm}}l@{\hspace{.3cm}}lll}
		\textup{(toy 1)} & &&\tm\bsvscpdiv & \ie ~ \text{has no} \tovscp \text{-normal form.}
		\\
		\textup{(toy 2)} & \tm \bsvscps \var  &\text{and}& \tmp \bsvscps \var
		\\
		\textup{(toy 3)} & \tm \bsvscps \la\var\tmfirst &\text{and}& \tmp \bsvscps \la\var\tmpfirst 
		& \text{with} ~ \tmfirst \rel \tmpfirst
		\\
		\textup{(toy 4)} & \tm \bsvscps \ntmONE \ntm &\text{and}& \tmp \bsvscps \ntmONEtwo \ntmtwo 
		& \text{with} ~ \ntmONE \rel \ntmONEtwo ~\text{and}~ \ntm \rel \ntmtwo
		\\
		\textup{(toy 5)} & \tm \bsvscps \ntm\esub\var\ntmONE &\text{and}& \tmp \bsvscps \ntmtwo\esub\var\ntmONEtwo
		& \text{with} ~ \ntmONE \rel \ntmONEtwo ~\text{and}~ \ntm \rel \ntmtwo
	\end{array}
	$\end{center}
		\Naf similarity, written $ \leqnaf $, is defined by coinduction as the largest \naf simulation, \ie it is the union of all \naf simulations.
	\end{definition}
		
	\paragraph{Making Inert Terms Explicit in the Clauses.} Cases (toy 4) and (toy 5) in the definition of toy simulations can be refined according to the grammar of \VSCptxt normal forms: for normal forms of the shape $\ntmONE\ntm$ the only possibility is that $\ntmONE$ is an inert term. Similarly, if $\ntm\esub\var\ntmONE$ is normal then $\ntmONE$ is inert. The two clauses are then refined as follows.
	\begin{center}
		$\begin{array}{r@{\hspace{.3cm}}r@{\hspace{.3cm}}l@{\hspace{.3cm}}l@{\hspace{.3cm}}lll}
		\text{(toy 4)} & \tm \bsvscps \itm \ntm &\textit{and}& \tmp \bsvscps \itmtwo \ntmtwo 
		& \textit{with} ~ \itm \rel \itmtwo ~\textit{and}~ \ntm \rel \ntmtwo
		\\
		\text{(toy 5)} & \tm \bsvscps \ntm\esub\var\itm &\textit{and}& \tmp \bsvscps \ntmtwo\esub\var\itmtwo
		& \textit{with} ~ \itm \rel \itmtwo ~\textit{and}~ \ntm \rel \ntmtwo
		\end{array}
		$\end{center}
	
\paragraph{(Howe-)Lassen's Method} Proving that a behavioral preorder $\precsim$ is compatible often cannot be done directly, that is, just by induction on the contextual closure. The idea of Howe's method is that, instead of proving compatibility of $\precsim$, one introduces a derived preorder $\howeop\precsim$ where the compatible closure is enforced in the definition, and then proves that $\precsim$ and $\howeop\precsim$ coincide. Howe introduced his method to deal with \emph{applicative} similarities \cite{Howe1996method}, Lassen adapted it for \emph{normal form} similarities \cite{lassen1999bisimulation}. The general idea is the same, but Lassen considers a different closure operation $\lassenop\precsim$.

\paragraph{Lassen's Closure.} The difficulty in proving directly that a similarity $\precsim$ is compatible comes from the applicative contextual closure, which may introduce a $\beta$-redex (when applying an abstraction to a term), that in turn can substitute over $\precsim$-related terms. The idea is to define the preorder $\lassenop\precsim$ as the compatible, substitutive, and reflexive closure of $\precsim$. 
In the case of \naf similarity, the definition of Lassen's closure needs an additional rule ($\scesub$) for the contextual closure with respect to ES.

\begin{definition}[Lassen closure]
The \emph{Lassen closure} $\lasrelsym$ of a relation $\relsym$ on terms is given by:
	\begin{center}
		% !TEX root = main.tex
\begin{tabular}{cccccc} 
%\textsc{Lassen's closure (for toy simulations)}
%\\[6pt]
\begin{tabular}{cccccc} 
	\infer[\sclift ]{\tmrone \lasrel \tmrtwo} {\tmrone \rel \tmrtwo}
	&
	\infer[\scvar]{\var \lasrel \var}	{}
	&
	\infer[\scabs ]{\la\var\tmrone \lasrel \la\var\tmrtwo} {\tmrone \lasrel \tmrtwo}
	&
		\infer[\scapp ] {\tmrone\tmrthree  \lasrel  \tmrtwo\tmrfour} {\tmrone  \lasrel \tmrtwo & \tmrthree \lasrel \tmrfour }  
\end{tabular}
\\[14pt]
\begin{tabular}{cccccc}
		\infer[\scesub ]{\tmrone\esub\var{\tmrthree} \lasrel \tmrtwo\esub\var{\tmrfour}{}} {\tmrone \lasrel \tmrtwo & \tmrthree \lasrel \tmrfour }
&
	\infer[\scsub ]{\tmrone\isub\var{\valp} \lasrel \tmrtwo\isub\var{\valptwo}{}} {\tmrone \lasrel \tmrtwo & \valp \lasrel \valptwo }	
\end{tabular}
\end{tabular}		
	\end{center}
\end{definition}
Note that rule ($\scesub$) does not constrain the terms $\tmrthree$ and $\tmrfour$ placed inside the ES, whereas rule ($\scsub$) does, because only practical values can be substituted in \VSCptxt.

\paragraph{Lassen's Closure Preserves Simulations} The proof of equivalence of $\leqnaf$ and $\lassenop\leqnaf$ reduces to prove that the closure operator $\lassenop\cdot$ preserves $\leqnaf$ simulations, that is, that $\lassenop\relsym$ is a \naf simulation if $\relsym$ is---it is often referred to as the \emph{main lemma} of the method. The proof is delicate and rests on two key intermediate properties. The first one concerns the evaluation level, and, when expressed at the big-step level, it is a sort of factorization property with respect to meta-level substitutions. In fact, it is nothing else but the substitutivity of small-step evaluation, rephrased at the big-step level.

\begin{proposition}[Small-step substitutivity]
	\label{l:stability_vsc}
	If $\tm\tovscp\tmp$ then $\tm\isubst\valp\var \tovscp \tmp\isubst\valp\var$
\end{proposition}

\begin{proof}
	By induction on $\tm\tovscp\tmp$ (induction on contexts), using the fact that a value where a variable is substituted by a value is still a value.
\end{proof}


\begin{lemma}[Big-step substitutivity]
	\label{l:splitting_vsc}
	If $\tm\isubst\valp\var \bsvscp k \ntm$ then there exist $k'$ and $\ntmtwo$ such that $ \tm \bsvscp {k'} \ntmtwo$ and $\ntmtwo\isubst\valp\var\bsvscp {k-k'} \ntm$.
\end{lemma}

\begin{proof}
If  $\tm\isubst\valp\var \bsvscp k \ntm$, then $ \tm \bsvscps \ntmtwo$ because if $\tm$ diverges then $\tm\isubst\valp\var$ diverges as well by substitutivity of $\tovscp$ {(\reflemma{stability_vsc})}.
	Then there exists $k'$ such that $\tm \bsvscp {k'} \ntmtwo$. Note that by substitutivity we have $ \tm\isubst\valp\var \tovscp^{k'} \ntmtwo\isubst\valp\var$, and so $\ntmtwo\isubst\valp\var\bsvscp {k-k'} \ntm$ because the reduction is diamond, hence all normalizing reduction sequences have the same length.
\end{proof}

%\begin{lemma}[Big-step substitutivity of $\bsvscpsym$]
%	\label{l:splitting_vsc}
%	Forall $\tm,\valp$,
%	$\tm\isubst\valp\var \bsvscp k \ntm \iff 
%	\exists k',\ntmtwo$ s.t. $ \tm \bsvscp {k'} \ntmtwo$ and $\ntmtwo\isubst\valp\var\bsvscp {k-k'} \ntm$
%\end{lemma}
%
%\begin{proof}
%	$(\Rightarrow)$ Suppose $\tm\isubst\valp\var \bsvscp k \ntm$, then $ \tm \bsvscps \ntmtwo$ because if it diverges then $\tm\isubst\valp\var$ diverges as well by substitutivity of $\tovscp$ {(\reflemma{stability_vsc})}.
%	Then there exists $k'$ such that $\tm \bsvscp {k'} \ntmtwo$. Note that by substitutivity we have $ \tm\isubst\valp\var \tovscp^{k'} \ntmtwo\isubst\valp\var$, and so $\ntmtwo\isubst\valp\var\bsvscp {k-k'} \ntm$ because the reduction is diamond, hence all normalizing reduction sequences have the same length.
%	
%	$(\Leftarrow)$ Suppose $ \tm \bsvscp {k'} \ntmtwo$ and $\ntmtwo\isubst\valp\var\bsvscp {k-k'} \ntm$.
%	By substitutivity {(\reflemma{stability_vsc})}, $ \tm\isubst\valp\var \tovscp^{k'} \ntmtwo\isubst\valp\var\tovscp^{k-k'}\ntm$. Then  $\tm\isubst\valp\var \bsvscp k \ntm$ by the equivalence of big-step and small-step evaluation ({\reflemma{ss-bs-equivalence_vsc}}).
%\end{proof}


%\paragraph{$\lasrel$ and $\lasrelnaf$ are equivalent on normal forms}
%The result for all terms is close to what we are trying to prove for the main \naf lemma we are trying to prove. Proving the equivalence for on normal forms will help for the proof.
%
%
%\begin{lemma}[Main \Naf Lemma]
%	\label{l:main-lemma-bis_vsc}
%	If $\relsym$ is a \naf simulation then
%	
%	$\tmrone\lasrel\tmrtwo, ~ \tmrone \bsvscp k \ntm \Rightarrow \tmrtwo\bsvscps \ntmtwo$ and $\ntm \lasrel \ntmtwo$
%	
%\end{lemma}

The second key intermediate property is the coherence of \naf simulations with respect to reduction and substitution.
\begin{toappendix}
\begin{proposition}[{Coherence of simulation, reduction, and substitution}]
\label{prop:naf-coherence}
Let $\rel$ be a \naf simulation, $\ntm \lasrelnaf \ntmtwo$, and $\valp\lasrelnaf\valptwo$.
\begin{enumerate}
\item \emph{Normal forms}: if $\ntm\isub\var\valp$ is $\tovscp$-normal then $\ntmtwo\isub\var\valptwo$ is $\tovscp$-normal and\\ $\ntm\isub\var\valp \lasrelnaf \ntmtwo\isub\var\valptwo$.
\item \emph{Steps}: if $\ntm\isub\var{\valp} \tovscp \tm$
	then $\ntm\isub\var{\valptwo}  \tovscp \tmtwo$ and $\tm \lasrel \tmtwo$.
\end{enumerate}
\end{proposition}
\end{toappendix}
{Note that the second point has $\lasrel$ rather than $\lasrelnaf$ in the conclusion. This is because in general $\tm$ and $\tmtwo$ are not normal. In the proof of the next proposition, it is shown that the normal forms of $\tm$ and $\tmtwo$ are in fact $\lasrelnaf$-related.}

We can now prove the crucial property of Lassen's closure.
\begin{toappendix}
\begin{proposition}
	\label{prop:main-lemma_vsc}
		Let $\relsym$ be a \naf simulation.
		\begin{enumerate}
		\item \emph{Technical auxiliary statement}: if $\tmrone\lasrel\tmrtwo$ and $\tmrone \bsvscp k \ntm$ then $\tmrtwo\bsvscps \ntmtwo$ and $\ntm \lasrelnaf \ntmtwo$.		
		\item \emph{Lassen's closure preserves \naf simulations}:  $\lassenop\relsym$ is a \naf simulation.
		\end{enumerate}
\end{proposition}
\end{toappendix}
\begin{proof} 
	\hfill
	\begin{enumerate}
	\item \emph{Sketch} (complete proof in Appendix \ref{chapter:proof-compatibility-naf}):	by induction on $(k,d)$ where $d$ is the size of the derivation of $\tmrone \lasrel \tmrtwo$. We proceed by case analysis on the last rule of the derivation $\tmrone \lasrel \tmrtwo$. Cases ($\sclift$), ($\scvar$), and ($\scabs$) are immediate by definition. The ($\scapp$) and ($\scesub$)  cases rely on a second case analysis (on the last rule of the $\tmrone \bsvscp k \ntm$ derivation). The sub-cases are routine and may depend on the ($\scsub$) rule. Case ($\scsub$) is the core of the proof. It starts by applying big-step substitutivity (\reflemma{splitting_vsc}) to $\tmrone = \tm\isub\var\valp$ and then, depending on whether the obtained $\ntmtwo\isubst\valp\var$ is normal, it applies the corresponding coherence property of \naf simulations with respect to evaluation (\refprop{naf-coherence}).
	\item Unfolding the statement one obtains exactly the statement of point 1.\qedhere
	\end{enumerate}
\end{proof}
Finally, we can use the preservation property to prove the redundancy of the closure, from which the compatibility and the soundness of \naf similarity follows.
\begin{theorem}[Compatibility and soundness of $\leqnaf$]
	\hfill
	\begin{enumerate}
	\item \emph{Redundancy of Lassen's closure}: $\leqnaf \,= \lassenop \leqnaf$.
	\item \Naf similarity is compatible and included in the \cbv contextual preorder $\leqcv$.
	\end{enumerate}
\end{theorem}

\begin{proof}
\hfill
\begin{enumerate}
\item By construction of $\lassenop\ctxhole$, $\leqnaf \subseteq \lassenop\leqnaf$ (by rule $\sclift$). Preservation of \naf simulations by Lassen's closure (\refprop{main-lemma_vsc}) and the fact that $\leqnaf$ is a \naf simulation give that $\lassenop\leqnaf$ is a \naf simulation. By definition, $\leqnaf$ is the maximal \naf simulation hence $\lassenop\leqnaf \subseteq \leqnaf$. 
\item Compatibility follows from point 1, because $\lassenop\leqnaf$ is compatible by definition. Inclusion in $\eqcv$ follows by \refprop{congruence-included-contextual-equivalence} and by adequacy of $\leqnaf$, which is trivial.\qedhere
\end{enumerate}
\end{proof}

	\paragraph{Toy Similarity Validates Scrutable Equivalence} The distinctive trait of \naf similarity with respect to naive and enf similarities is that it validates the scrutable equivalence $\equivscr$, that is, it identifies all \cbv inscrutable terms. The proof is immediate because of the characterization of inscrutability in the VSC (\refthm{cbv-scrutability-characterization}): inscrutable terms are exactly the $\tovscp$-diverging ones, which all fall in the first clause defining \naf similarity. In particular, $\Omega \eqnaf \Omega^L$.
	
	\begin{proposition}
	Toy similarity $\leqnaf$ validates scrutable equivalence $\equivscr$.
	\end{proposition}
	
	\paragraph{Toy Similarity is Useless} While $\leqnaf$ achieves the validation of $\equivscr$, it does not validate any other benchmark equivalence, and it cannot even prove that Curry's and Turing's fix-point combinators are equivalent, because for that one needs to be able to substitute variables, which is forbidden in the \VSCptxt, as its definition excludes rule $\toevar$. Therefore, in particular, there are terms that are \enf similar (or even only naively similar) but not toy similar. The two similarities are incomparable.
	
	Moggi's equivalences, as well as the shuffling and proof nets ones, are not validated by \naf similarity because they change the structure of the normal form, while $\leqnaf$ is \emph{rigid}: similar normal forms have to have the \emph{exact} same structure out of abstractions. 
	
	%\paragraph{Next} We shall then refine \Naf, adding the substitution of variables and the graphical equivalences, marrying its validation of scrutable equivalence with the other principles, thus obtaining a useful similarity. First, however, we need to re-understand the benchmark equivalences in the VSC.
% !TEX root = main.tex

\section{Equational Benchmarks and the Value Substitution Calculus}
\label{sect:benchmarks-vsc}
Here we revisit the equational benchmarks of \refsect{benchmarks} in the VSC. We begin with the proof nets equivalences, as they are one of the \emph{raison d'\^etre} of the VSC and the key to re-understand them all.

\paragraph{Structural Equivalence} The translation of the VSC to proof nets maps some terms with ES to the same proof net. The induced identification of terms is expressed by structural equivalence \cite{accattoli+paolini-vsc,Accattoli-proofnets}. 
\begin{definition}[Structural equivalence $\streq$]
Structural equivalence $\streq$ is defined as the smallest compatible equivalence relation generated by union of the following root rules.
\begin{center}
$\begin{array}{rllrcc}
	(\tm\tmthree)\esub\var\tmtwo & \equivsone&\tm\esub\var\tmtwo\tmthree 	 & \mbox{if }\var \not \in \fv\tmthree
	\\
	(\tm\tmthree)\esub\var\tmtwo &\equivexsthree& 	\tm\tmthree\esub\var\tmtwo & \mbox{if }\var \not \in \fv\tm
	\\
	\tm\esub\var\tmtwo\esub\vartwo\tmthree &\equivass& \tm\esub\var{\tmtwo\esub\vartwo\tmthree} & \mbox{if }\vartwo \not \in \fv\tm
	\\
	\tm\esub\vartwo\tmthree\esub\var\tmtwo &\equivcom& 	\tm\esub\var\tmtwo\esub\vartwo\tmthree & \mbox{if }\var \not \in \fv\tmthree \mbox{ and }\vartwo \not \in \fv\tmtwo
\end{array}$
\end{center}
\end{definition}
These axioms preserve the number and type of the constructors in terms, they only rearrange the order. In particular, structurally equivalent terms have the same number of ES. Note that the axioms simply express the constructor-and-scope-preserving commutation of ES with applications and ES themselves (but not abstractions, as that would break \refprop{strong-bisimulation} below). 

Additionally, structural equivalence behaves very well with respect to evaluation: it commutes with reduction rules---and is therefore postponable---preserving the number and kind of steps. This is expressed by the following proposition. In the literature, what is below called \emph{strong commutation} is usually called \emph{strong bisimulation}. We prefer to change the terminology here to avoid confusion with \emph{normal form (bi)simulations}, as the concept is similar and yet different (no need to observe normal forms, and it preserves the number of steps).

\begin{toappendix}
\begin{proposition}[$\streq$ strongly commutes with $\tovsc$, \cite{accattoli+paolini-vsc}]
	\label{prop:strong-bisimulation}
	Let $a \in \set{\msym,\expoabs,\expovar}$. Structural equivalence $\streq$ strongly commutes with $\tovsc$:
	if  $\tm \streq\tmtwo$ and $ \tm \Rew{a}\tmp$ then $\tmtwo \Rew{a}\tmtwop$ and $\tmp\streq\tmtwop$.
\end{proposition}
\end{toappendix}

In fact, $\streq$-equivalence classes are an isomorphic representation of proof nets, as the proof net $P_\tm$ associated to $\tm$ does the same exact rewriting steps as $\tm$, that is, the translation from $\tm$ to $P_\tm$ also strongly commutes with evaluation (turning term steps into proof nets steps, and vice-versa), see Accattoli \cite{Accattoli-proofnets}. Consequently, $\streq$-equivalent terms are \emph{indistinguishable} and should be equated by any sensible notion of program equivalence on pure terms (extensions with effects can invalidate some cases of structural equivalence, typically $\equivcom$, as we discuss below). In particular, structural equivalence is included in contextual equivalence, as shown by \citet{DBLP:journals/pacmpl/AccattoliG22}.

\paragraph{Revisiting the Benchmarks From Calculi} The equivalences of Moggi's and the shuffling calculi can be re-understood via structural equivalence. The idea is that by applying $\tom$ to the two sides of an equivalence, we can express it via ES, and many of become cases of $\streq$. Consider $\equivsone$:
	\begin{center}
		$\begin{array}{ccccc}
	(\la\var\tm)\tmtwo\tmthree &\equivsone& (\la\var\tm\tmthree)\tmtwo & \mbox{with }\var\not\in\fv\tmthree
	\\
\downarrow_{\mult} && \downarrow_{\mult}
	\\
\tm\esub\var\tmtwo\tmthree &\equivsone& (\tm\tmthree)\esub\var\tmtwo
\end{array}$
	\end{center}
Similarly, the equivalences $\equivexsthree$, $\equivass$, and $\equivcom$ of \refsect{benchmarks} become the axioms with the corresponding label of $\streq$, and $\equivsthree$ is a special case of $\equivexsthree$. Therefore, structural equivalence covers the shuffling equivalences and the proof nets equivalences. Moggi's equivalences $\equivlid$, $\equivlad$ and $\equivrad$, instead, are not covered. By applying $\tom$, we obtain the following reformulation for the VSC which corresponds to Moggi's original formulation with $\letexp$-expressions. If $\var\notin \fv\tmtwo$ for $\equivlad$ and $\var\notin \fv\val$ for $\equivrad$:
\begin{center}
\begin{tabular}{c@{\hspace{1.5cm}} c@{\hspace{1.5cm}} c}
		$\begin{array}{ccccc}
	(\la\var\var)\tm &\equivlid& \tm 
	\\
\downarrow_{\mult} && =
	\\
\var\esub\var\tm &\equivlid& \tm
\end{array}$
%%%%
&
%%%%
$\begin{array}{ccccc}
	(\la\var\var\tmtwo)\tm &\equivlad& \tm\tmtwo 
	\\
\downarrow_{\mult} && =
	\\
(\var\tmtwo)\esub\var\tm &\equivlad& \tm\tmtwo
\end{array}$
%%%%
&
%%%%
$\begin{array}{ccccc}
	(\la\var\val\var)\tm &\equivrad& \val\tm 
	\\
\downarrow_{\mult} && \downarrow_{\mult}
	\\
(\val\var)\esub\var\tm &\equivrad&  \val\tm
\end{array}$
\end{tabular}
	\end{center}
In presence of structural equivalence, the ES formulation of the application decomposition equivalences $\equivlad$ and $\equivrad$ is derivable:  $\equivlad$ follows from $\equivsone$ and $\equivlid$, $\equivrad$ follows from $\equivsthree$ and $\equivlid$. In fact, by using the extended version $\equivexsthree$ of $\equivsthree$ we can actually derive the extended version of $(\tmtwo\var)\esub\var\tm \equivexrad  \tmtwo\tm$ (if $\var\notin\fv\tmtwo$) of $\equivrad$. Therefore, the whole of Moggi's equivalences is captured by simply adding $\equivlid$ to structural equivalence.

\paragraph{A Family of Strong Commutations} By looking at the proof (in Appendix D) of strong commutation of $\streq$ (\refprop{strong-bisimulation}), it turns out that various sub-relations of structural equivalence also verify strong commutation. We here describe them by the root rules, while implicitly referring to the same closure used in the definition of $\streq$. For instance, $\equivcom$ by itself strongly commutes with $\tovsc$, as well as $\equivsone$ by itself, or $\equivexsthree\cup\equivass$, or some of the restricted version such as ${\equivsthree}\cup\equivass$. In particular, $\equivsone\cup\equivsthree\cup\equivass$, which would be the restriction of $\streq$ to a non-commutative setting for effects, also strongly commutes with $\tovsc$. In the next section, we shall craft a normal form similarity for the VSC which is parametric with respect to these variants of structural equivalence.


% !TEX root = main.tex
\section{Net Similarity for the Value Substitution Calculus}
\label{sect:net}
In this section, we finally define the nf-similarity for the VSC we are interested in, \emph{net similarity}, which extends $\leqncbv$ along two axes:
\begin{enumerate}
\item \emph{Changing the underlying calculus:} evaluation is now based on the VSC, where $\Omega$-terms have a diverging characterization, hence $\Omega$-equivalence $\equivom$ will be trivially included in the bisimilarity, and 
\item \emph{Changing the nf-bisimilarity}: allowing simulations to test terms modulo structural equivalence $\streq$, in order to avoid artificial distinctions of indistinguishable terms and to validate the commutation of $\letexp$s $\equivcom$. 
\end{enumerate}
The problematic addition of the left identity equivalence $\equivlid$ is discussed at the end of the section.

\paragraph{Changing the Underlying Calculus} Moving from Plotkin's \cbv calculus to the VSC, normal forms are harder to describe, as one can see from the grammar of normal forms in Figure \ref{fig:vsc}.

Since the case $\isctxp\var\ntm$ is quite unpleasant to manage in proofs, we go one step further and consider normal forms modulo $\equivsone$, picking the $\equivsone$-representative of each normal form where the context $\isctx$ has been pushed out of the unpleasant case for inert terms. This can be done harmlessly because $\equivsone$ by itself verifies strong \emph{commutation} with respect to $\tovsc$ and the described representant can be easily described at the big-step level. The following lemma gives a grammar for normal forms modulo $\equivsone$; we overload/redefine \emph{inert terms}, which from now on only refer to the new grammar.

\begin{lemma}
$\tm$ is $\tovsc$-normal iff there exists $\ntm$, given by the following grammar, such that $\tm \equivsone \ntm$.
\begin{center}
	$\begin{array}{r@{\hspace{.5cm}}rlll}
		\textsc{Applicative Inert terms}  &
		\itmapp,\itmapptwo & \grameq &  \var\fire\mid \itmapp\fire
		\\
		\textsc{Inert terms}  &
		\itm,\itmtwo & \grameq &  \itmapp\mid \itm\esub\var\itmtwo
		\\
		\tovsc\textsc{-normal forms modulo $\equivsone$}  &
 \ntm,\ntmtwo & \grameq &  \val \mid \itm\mid \fire\esub\var\itmtwo
	\end{array}
	$\end{center}
\end{lemma}
Note the notion of applicative inert terms, which are specific inert terms where no substitutions can be pushed outward by $\equivsone$. 
Example: the $\tovsc$-normal form $\tm =\var\esub\var{\vartwo\vartwo}\varthree$ does not belong to grammar above, but $\tm \equivsone (\var\varthree)\esub\var{\vartwo\vartwo} = \ntm$ and $\ntm$ is described by the grammar.

\paragraph{Big-Step Evaluation} We then need to express evaluation to $\tovsc$-normal form modulo $\equivsone$ as a big-step predicate. For that, we re-define the inert substitution contexts, which at first sight are defined as before, except that the notion of inert term now has changed.
\begin{center}
	$\begin{array}{r@{\hspace{.5cm}}rlll}
		\textsc{Inert Substitution Contexts} & \isctx & \grameq &  \ctxhole\mid \isctx\esub{\var}{\itm} 
	\end{array}
	$\end{center}

\begin{definition}[Big-step evaluation $\bsvsct k$]
Big-step $\VSC$ modulo $\equivsone$ evaluation $\tm \bsvsct k \ntm$ is given by:
\begin{center}
		% !TEX root = main.tex


\begin{tabular}{cccccc}

	\infer[(\bsvsctax)]{\val \bsvsct 0 \val}{}
	
	&
	
	\infer[(\bsvsctapm)]{\tm\tmtwo \bsvsct {k+i+1} \isctxp\fire}{
		\tm \bsvsct k \isctxp{\la\var\tmthree}
		&
		{\tmthree\esub\var\tmtwo} \bsvsct i \fire
	}
	
	
	
	\\[6pt]
	\infer[(\bsvsctapvar)]{\tm\tmtwo \bsvsct {k+h} \isctxp{\var\fire}}{
		\tm \bsvsct k \isctxp\var
		&
		\tmtwo \bsvsct h \fire
	}
	&
	\infer[(\bsvsctese)]{\tm\esub\var{\tmtwo} \bsvsct {k+i+1} \isctxp\fire}{
		\tmtwo \bsvsct k \isctxp{\val}
		&
		{\tm\isub\var{\val}} \bsvsct i \fire
	}

	
	
	\\[6pt]
	
	\infer[(\bsvsctapi)]{\tm\tmtwo \bsvsct {k+h} \isctxp{\itmapp\fire}}{
		\tm \bsvsct k \isctxp\itmapp
		&
		\tmtwo \bsvsct h \fire
	}
	&
\infer[(\bsvsctesi)]{\tm\esub\var\tmtwo \bsvsct {k+h} \fire\esub\var\itm}{
	\tm \bsvsct k \fire
	&
	\tmtwo \bsvsct h \itm
}
\end{tabular}


\end{center}
\emph{Notation}: $\tm \bsvscts \fire$ abbreviates \emph{there exists a $k$ such that $\tm \bsvsct k \fire$}.
\end{definition}

 The given big-step system captures $\equivsone$ via the rules ($\bsvsctapvar$) and ($\bsvsctapi$): when applying an inert term / variable surrounded by an inert context $\isctx$ to a normal term $\ntm$, the context $\isctx$ is pushed out of the application, obtaining $\isctxp{\var\fire}$ and $\isctxp{\itmapp\fire}$ instead of $\isctxp{\var}\fire$ and $\isctxp{\itmapp}\fire$. As a result, the $\bsvsctsym$-normal forms are exactly those of the given grammar for $\tovsc$-normal forms modulo $\equivsone$.

	We prove this big-step system to be correct and complete with respect to small-step reduction. 	Importantly, substitutivity also smoothly adapts.
	\begin{toappendix}
	\begin{proposition}
		\label{l:ss-bs-equivalence_vsce}
		$\tm \bsvsct k \fire$ if and only if there exists a normal form $\firep$ such that $\tm \tovsc^k \firep \equivsone \fire$.
	\end{proposition}
	\end{toappendix}
	
%	\begin{proof}
%		Similar to the one for the \VSCptxt big step system.
%	\end{proof}
	

	\begin{toappendix}
	\begin{proposition}[Substitutivity]
		\label{prop:substitutivity_vsce}
	\hfill
	\begin{enumerate}
	\item 
	\emph{Small-step}: if $\tm~\tovsc~\tmp$ then $\tm\isubst\val\var ~\tovsc~ \tmp\isubst\val\var$.
	\item 
	\emph{Big-step}: 	if $\tm\isubst\val\var \bsvsct k \ntm$ then $\exists$ $k'$ and $\ntmtwo$ such that $ \tm \bsvsct {k'} \ntmtwo$ and $\ntmtwo\isubst\val\var\bsvsct {k-k'} \ntm$.
	\end{enumerate}
\end{proposition}
\end{toappendix}

	\paragraph{Changing the Nf-Bisimilarity by Adding Structural Equivalence to Simulations, Parametrically} We are now also going to refine the definition of naive similarity by adding structural equivalence $\streq$ to the nf-bisimilarity. In fact, we are going to do something more general, in order to obtain a whole family of similarities. We abstract away structural equivalence $\streq$ as a more abstract notion of \emph{mirror equivalence} $\equivx$, defined by the properties of $\streq$ that are needed to prove that similarity modulo $\streq$ is compatible. Then, similarity is defined \emph{parametrically} in a mirror $\equivx$, and net similarity is obtained by taking $\streq$ as mirror. The terminology \emph{mirror} is meant to suggest that $\equivx$ can modify terms only in inessential ways.
	
\begin{definition}[Mirror]
An equivalence relation $\equivx$ is a \emph{mirror} for $\tovsc$ when:
\begin{enumerate}
\item \emph{Strong commutation}: if  $\tm \equivx\tmtwo$ and $ \tm \tovsc\tmp$ then $\tmtwo \tovsc\tmtwop$ and $\tmp\equivx\tmtwop$.

\item \emph{Substitutivity}: if $\tm\equivx\tmtwo$ then $\tm\isub\var\val \equivx \tmtwo\isub\var\val$ for all values $\val$.
\end{enumerate}
\end{definition}

\begin{definition}[Mirrored and \net similarities]
	Let $\relsym$ be relation and $\equivx$ be a mirror over VSC terms.	We say that $\relsym$ is a \emph{$\equivx$-mirrored (\nafex) simulation} if $\relsym\subseteq\relvscx$, where $\tm \relvscx\tmp$ holds whenever $\tm,\tmp$ satisfy one of the following clauses:
	\begin{center}
		$\begin{array}{r@{\hspace{.3cm}}r@{\hspace{.3cm}}l@{\hspace{.3cm}}l@{\hspace{.3cm}}lll}
		\textup{(\nafex 1)} & &&\tm\bsvsctdiv & \ie ~ \text{has no} \tovsc \text{-normal form.}
		\\
		\textup{(\nafex 2)} & \tm \bsvscts \var  &\text{and}& \tmp \bsvscts \var
		\\
		\textup{(\nafex 3)} & \tm \bsvscts \la\var\tmfirst &\text{and}& \tmp \bsvscts\la\var\tmpfirst
		& \text{with} ~ \tmfirst \rel \tmpfirst
		\\
		\textup{(\nafex 4)} & \tm \bsvscts \ntmONE \ntmTWO &\text{and}& \tmp \bsvscts \ntmtwo \equivx \ntmONEtwo \ntmTWOtwo
		& \text{with} ~ \ntmONE \rel \ntmONEtwo ~\text{and}~ \ntmTWO \rel \ntmTWOtwo
		\\
		\textup{(\nafex 5)} & \tm \bsvscts \ntmONE\esub\var\ntmTWO &\text{and}& \tmp \bsvscts \ntmtwo \equivx \ntmONEtwo\esub\var\ntmTWOtwo
		& \text{with} ~ \ntmONE \rel \ntmONEtwo ~\text{and}~ \ntmTWO \rel \ntmTWOtwo
	\end{array}
	$\end{center}
		$\equivx$-Mirrored (\nafex) similarity , written $ \leqvscx $, is defined the largest $\equivx$-mirrored simulation.
		
		\Net simulations and \net similarity $\leqnet$ are defined as the $\equivx$-mirrored simulations and similarities with structural equivalence $\streq$ as mirror $\equivx$. 
\end{definition}

\paragraph{Making Inert Terms Explicit in the Clauses} Cases (\nafex 4) and (\nafex 5) can be rewritten using the grammar of normal forms, which is useful for clarity in proofs. For (\nafex 4), it actually splits in two:
	\begin{center}
		$\begin{array}{r@{\hspace{.3cm}}r@{\hspace{.3cm}}l@{\hspace{.3cm}}l@{\hspace{.3cm}}lll}
		
		\text{(\nafex 4a)} & \tm \bsvscts \var \ntmONE &\textit{and}& \tmp \bsvscts \ntmtwo \equivx\var \ntmONEtwo 
		& \textit{with}~\ntmONE \rel \ntmONEtwo
		\\
		\text{(\nafex 4b)} & \tm \bsvscts \itmapp \ntmONE &\textit{and}& \tmp \bsvscts \ntmtwo \equivx\itmapptwo \ntmONEtwo 
		& \textit{with} ~ \itmapp \rel \itmapptwo ~\textit{and}~ \ntmONE \rel \ntmONEtwo
		\\
		\text{(\nafex 5)} & \tm \bsvscts \ntmONE\esub\var\itm &\textit{and}& \tmp \bsvscts \ntmtwo \equivx\ntmONEtwo\esub\var\itmtwo
	& \textit{with} ~ \itm \rel \itmtwo ~\textit{and}~ \ntmONE \rel \ntmONEtwo
	\end{array}
	$\end{center}

\paragraph{Compatibility} The compatibility proof for $\leqvscx$ follows the same structure of the one for $\leqncbv$ (in Appendix E of the additional material on HotCRP). At the evaluation level, we have already seen that substitutivity holds also for the weak reduction of the VSC and with the addition of the equivalence $\equivsone$ (\refprop{substitutivity_vsce}). At the level of the simulation, we need to refine the notion of Lassen closure, by adding rule $\mscequivx$ accounting for mirrors.
\begin{definition}[Mirrored Lassen closure]
Let the \emph{mirrored Lassen closure} $\mlasrelsym$ of $\relsym$ be:
	\begin{center}
		% !TEX root = main.tex
\begin{tabular}{cccccc} 
%\textsc{Mirrored Lassen's closure (for $\vscx$ simulations)}
%\\[6pt]
\begin{tabular}{cccccc} 
	\infer[\msclift ]{\tmrone \mlasrel \tmrtwo} {\tmrone \rel \tmrtwo}
	&
	\infer[\mscvar]{\var \mlasrel \var}	{}
	&
	\infer[\mscabs ]{\la\var\tmrone \mlasrel \la\var\tmrtwo} {\tmrone \mlasrel \tmrtwo}
	&
		\infer[\mscapp ] {\tmrone\tmrthree  \mlasrel  \tmrtwo\tmrfour} {\tmrone  \mlasrel \tmrtwo & \tmrthree \mlasrel \tmrfour }  
\end{tabular}
\\[14pt]
\begin{tabular}{cccccc}
		\infer[\mscesub ]{\tmrone\esub\var{\tmrthree} \mlasrel \tmrtwo\esub\var{\tmrfour}{}} {\tmrone \mlasrel \tmrtwo & \tmrthree \mlasrel \tmrfour }
&
	\infer[\mscsub ]{\tmrone\isub\var{\valof\tmrthree} \mlasrel \tmrtwo\isub\var{\valof\tmrfour}{}} {\tmrone \mlasrel \tmrtwo & \valof\tmrthree \mlasrel \valof\tmrfour }	
	&
\infer[\mscequivx]{\tmrone\mlasrel\tmrtwop}{\tmrone \mlasrel\tmrtwo & \tmrtwo \equivx \tmrtwop}
\end{tabular}
\end{tabular}		
	\end{center}
\end{definition}
Then the reasoning for compatibility---and in particular the coherence properties---smoothly adapts, using the mirror properties for rule $\mscequivx$ in the proof that the closure preserves mirrored simulations. In particular, strong commutation of $\equivx$ implies that it preserves normal forms and steps, that is, the coherence properties. Summing up, we obtain our main result.
\begin{toappendix}
\begin{theorem}[Compatibility and soundness of $\leqvscx$ and $\leqnet$]
	\label{thm:nafex-included-leqc}
Let $\equivx$ be a mirror.
	\begin{enumerate}
	\item \emph{Redundancy of the mirrored Lassen closure}: $\leqvscx \,= \mlassenop \leqvscx$.
	\item \Nafex similarity $\leqvscx$ is compatible and included in the \cbv contextual preorder $\leqcv$.
	\item \Net similarity $\leqnet$ is compatible and included in the \cbv contextual preorder $\leqcv$.
	\end{enumerate}
\end{theorem}
\end{toappendix}


\paragraph{Fixpoints and Benchmarks.} For any mirror $\equivx$, and in particular for $\equivx\defeq Id$ and $\equivx\defeq \streq$, one can show that Turing's and Curry's \cbv fixpoint combinators are \nafex bisimilar. The proof relies on exactly the same relation that for naive bisimilarity (\refprop{naive-fix-points-equiv}). \adr{Plotkin's $\betav$ and VSC conversions are included in \nafex similarities. }Unlike naive and enf similarities, \nafex and net similarities validate $\Omega_v$-equivalence $\equivomv$. Net and $\equivx$-mirrored bisimilarities do not however validate $\eta_v$ equivalence, one has to change the case for abstractions to accommodate it, and it does not validate \cbn duplication, as for instance $(\vartwo\var\var)\esub\var{\varthree\Id}$ and $\vartwo(\varthree\Id)(\varthree\Id)$ are both $\tovsc$-normal but not $\streq$-equivalent. 






%\paragraph{Net similarity includes structural equivalence, and more}Naturally, structurally equivalent terms are net bisimilar. However net similarity is able to relate more, in the sense that it can equate terms the \emph{strong} normal forms of which are $\streq$-equivalent: 
%
%\begin{center}
%	$\ntm_1 = \la\var{(\la\vartwo(\la\varthree\tm)\tmtwo)\tmthree}\eqnet\la\var{(\la\varthree(\la\vartwo\tm)\tmthree)\tmtwo} = \ntm_2$ whereas $\ntm_1 \not \streq \ntm_2$
%\end{center}

%\adr{add example with infinite normal forms}

%Let $A \defeq \la\var{\tm\tmtwo\esub\varthree\tmthree\var}$ and $B \defeq \la\var{(\tm\tmtwo)\esub\varthree\tmthree\var}$. We can show that $\curryfix A \eqnet \curryfix B$.


\paragraph{Left Identity Is Not Validated By \Nafex} Analogously, \net similarity does not validate Moggi's $\equivlid$ rule, because $\var\esub\var{\vartwo\Id} \not \tovsc \vartwo\Id$ and  $\var\esub\var{\vartwo\Id} \not \streq \vartwo\Id$. Thus, \enf is not included in \net similarity. About adding $\equivlid$, it is easy to define a \nafexp\equivlid bisimulation, but the current  compatibility proof does not go through, as $\equivlid$ is not a mirror for $\tovsc$ (in particular, it does not strongly commute with $\tovsc$) and the proof technique is not able (for now) to handle $\equivlid$ terms, as it breaks coherence for normal forms, that is, the fact that if  $\ntm\,\mlasrelsym \tm$ then $\tm$ is normal (and the symmetric statement).

One could also add $\equivlid$ as a reduction step of the VSC, but then the reduction is no longer diamond, and the diamond property (or at least the invariance of the number of steps to normal form) is essential in the current proof technique. It is thus unclear how to extend \net similarity as to validate $\equivlid$. Next section introduces a program equivalence including $\eqnet$ and validating $\equivlid$.

\paragraph{Net Bisimilarity Cannot Be Extensional} Lassen introduced an extension of enf bisimilarity validating $\equivetav$, enf bisimilarity up to $\eta_v$, at the same time that he introduced enf bisimilarity \cite{LassenEnf}. As of now, that modification cannot be applied to net bisimilarity. We explain the problem which boils down to, again, the fact that net does not validate the left identity law $\equivlid$.
	
	Let us consider that there exists a nf-bisimilarity $\relsym$ based on the VSC (\ie $\tm\tovsc\tmp$ implies $\tm\rel\tmp$) which is a compatible equivalence relation, and which validates $\equivetav$. Then, in particular, we have that $\var\rel\la\vartwo\var\vartwo$, as $\equivetav\subseteq\relsym$. By compatibility (for $\ctx=\ctxhole\tm$), $\var\tm\rel(\la\vartwo\var\vartwo)\tm$ and by reduction and transitivity, $\var\tm\rel\var\vartwo\esub\vartwo\tm$. This means that at least $(\var\tm,\var\vartwo\esub\vartwo\tm)$ must be included in the relation $\relsym$, which is a subcase of $\equivrad$, which itself can be implied by the $\equivlid$ rule and structural equivalence $\streq$. As net bisimilarity does not validate Moggi's laws, and we do not currently know how to include them, net bisimilarity is unable to include $\equivetav$.

%Moggi's left identity rule somehow is a $\letexp$-$\eta$-equivalence.

%Now, we call the associated bisimilarity, $\equivx$-mirrored bisimilarity up to $\eta_v$ (and \net bisimilarity up-to $\eta_v$). Using similar techniques, we prove compatibility. One has to add a rule in the mirrored Lassen's closure to account for the modified rule:
%\adr{
%	\begin{tabular}{cccccc}
%		\infer[\mscabseta]{\val\mlasrel\valtwo}{\val\vartwo\mlasrel\valtwo\vartwo & \vartwo ~\text{is fresh}}
%\end{tabular}}


\ignore{
\paragraph{Fixed point combinators are \nafex bisimilar.} As Lassen did with \enf bisimilarity, we can prove the equivalence of call-by-value versions of Curry's and Turing's fixed point combinators:

\[ \curryfix = \la\var{\curryfixaux\curryfixaux}\text{, where } \curryfixaux = \la\varthree{\var\la\vartwo{\varthree\varthree\vartwo}}\]
\[ \turingfix = (\la\varthree{\la\var{\var\la\vartwo{\varthree\varthree\var\vartwo}}})(\la\varthree{\la\var{\var\la\vartwo{\varthree\varthree\var\vartwo}}}) \]

To prove that they are \nafe bisimilar we build a bisimulation containing $\{(\curryfix,\turingfix)\}$.

\[ \relsym \defeq \{(\curryfix,\turingfix), (\la\var\curryfixaux\curryfixaux,\la\var{\var\la\vartwo{\turingfix\var\vartwo}}),(\curryfixaux\curryfixaux,{\var\la\vartwo{\turingfix\var\vartwo}}), \]\[ 
(\var\la\vartwo{\curryfixaux\curryfixaux\vartwo},{\var\la\vartwo{\turingfix\var\vartwo}}),(\var,\var),(\la\vartwo{\curryfixaux\curryfixaux\vartwo},\la\vartwo{\turingfix\var\vartwo}),\]
\[({\curryfixaux\curryfixaux\vartwo},{\turingfix\var\vartwo}),((\var\la\vartwo{\curryfixaux\curryfixaux\vartwo})\vartwo,({\var\la\vartwo{\turingfix\var\vartwo}})\vartwo),(\vartwo,\vartwo)\} \]

$\relsym \subseteq \opnafep{\relsym}$ by construction (we start with $(\curryfix,\turingfix)$ and we add to $\relsym$ what is needed for each element to satisfy one \nafe case), and sym(R) is a \nafe simulation as well, hence $\curryfix \nafebisim \turingfix$.

This relation is similar to the one defined by Lassen since those terms have "pure lambda-calculus" normal forms (nothing more can be done at the end with VSC and explicit substitutions).
}
% !TEX root = main.tex

\section{From Operational to Denotational Semantics: the Type Preorder}
In this section, we study a behavioral preorder, the \emph{type preorder} $\leqtype$, which is not defined as a normal form similarity, it is instead induced by a denotational model. Namely, Ehrhard's \cbv relational model \cite{DBLP:conf/csl/Ehrhard12} presented as a system of multi types, also known as \emph{non-idempotent intersection types}. We shall prove that both Lassen's similarity $\leqenf$ and our net similarity $\leqnet$ are included in $\leqtype$. The aim is to show that, while $\leqenf$ and $\leqnet$ are incomparable, they can be combined in a cost-sensitive preorder (the contextual preorder combines them but it is not cost-sensitive). We introduce the bare minimum about \cbv multi types. For more, see Accattoli and Guerrieri \cite{Accattoli-Guerrieri-TypesFireballs,DBLP:journals/corr/abs-2104-13979,DBLP:journals/pacmpl/AccattoliG22}.


%Let's first recall the definition of a lambda-model, in the case of Call-by-Value evaluation, adding clauses for explicit substitutions as we are considering the Value Substitution Calculus.

%\begin{definition}[Lambda-model, \cite{lambda-models-dezani}]
%	A $\l$-model is a pair $(\dom, \interp{~})$ where $\dom$ is a set, $\interp{~}$ is a mapping from $\l$-terms to elements of $\dom$ called the \emph{interpretation} and $\interp{~}$ satisfies the following:
%	
%	\begin{itemize}
%		%\item \emph{(applicative)} $\interp{\tm} \applaw \interp{\tmtwo} \subseteq \interp{\tm\tmtwo}$
%		\item \emph{(applicative)}  $\interp\tm \subseteq \interp\tmp ~\&~ \interp\tmtwo \subseteq \interp\tmtwop \Rightarrow \interp{\tm\tmtwo} \subseteq \interp{\tmp\tmtwop}$
%		\item \emph{(abstractive)} $\interp\tm \subseteq \interp\tmtwo \Rightarrow \interp{\la\var\tm} \subseteq \interp{\la\var\tmtwo}$
%		\item \emph{($\alpha$-equivalence)} $\interp{\la\var\tm} = \interp{\la\vartwo\tm\isub\var\vartwo}$
%		\item \emph{($\beta_v$-equivalence)} $\interp{(\la\var\tm)\val} = \interp{\tm\isub\var\val}$
%		\item \emph{(explicitly substitutive)} $\interp\tm \subseteq \interp\tmp ~\&~ \interp\tmtwo \subseteq \interp\tmtwop \Rightarrow \interp{\tm\esub\var\tmtwo} \subseteq \interp{\tmp\esub\var\tmtwop}$
%		\item \emph{($\alpha$-equivalence')} $\interp{\tm\esub\var\tmtwo} = \interp{(\tm\isub\var\vartwo)\esub\vartwo\tmtwo}$
%	\end{itemize}
%	
%	%We add the two last clauses because we are dealing with the Value Substitution Calculus, which uses explicit substitutions.
%\end{definition}

% !TEX root = main.tex
\begin{figure}
\begin{tabular}{c}
		$\begin{array}{ccccc}
		\textsc{Linear Types} & \ltype, \ltypetwo &\grameq&
		% \vartype \mid 
		\mtype \multimap \mtypetwo
		\\
		\textsc{Multi Types} & \mtype, \mtypetwo &\grameq& \multitype{n}{\ltype} & n\geq 0
		\end{array}$
		
		\\[10pt]
		
		
		\begin{tabular}{ccc}
			\infer[\typingruleAx]{\var \hastype [\ltype] \types \var \hastype \ltype}{}
			
			&
			
			\infer[\typingruleAbs]{\typectx \types \la\var\tm \hastype \mtype \multimap \mtypetwo}{\typectx, \var \hastype \mtype \types \tm \hastype\mtypetwo}
			
			&
			
			\infer[\typingruleMany]{\biguplus_{i\in I} \typectx_i \types \val \hastype \biguplus_{i\in I} [\ltype_i]}{(\typectx_i \types \val \hastype \ltype_i)_{i\in I}  & I~ \text{finite} }
			
			
			
			
		\end{tabular}
		\\[10pt]
		\begin{tabular}{cc}
			\infer[\typingruleApp]{\typectx \uplus \typectxtwo \types \tm\tmtwo \hastype \mtypetwo}{ \typectx \types \tm \hastype [\mtype \multimap \mtypetwo] & \typectxtwo \types \tmtwo \hastype \mtype }
			&
			\infer[\typingruleES]{\typectx \uplus \typectxtwo \types \tm\esub\var\tmtwo \hastype \mtypetwo}{ \typectx, \var \hastype \mtype \types \tm \hastype \mtypetwo & \typectxtwo \types \tmtwo \hastype \mtype }
			
		\end{tabular}

\end{tabular}
\caption{Call-by-Value Multi Type System for VSC.}
\label{fig:multi-types-vsc}
\end{figure}

%\begin{tabular}{ccccccc}
%\end{tabular}
%
%
%\begin{array}{ccccccc}
%\end{array}
 
\paragraph{Multi Types} \Cref{fig:multi-types-vsc} gives the definition of multi types $\mtype$ for the VSC, which  mutually depends on the definition of linear types $\ltype$. Multi types are defined as finite multi-sets $\multitype{n}{\ltype}$, which intuitively denote the intersection $\ltype_1 \cap \ldots \cap \ltype_n$, where the intersection $\cap$ is a commutative, associative and non-idempotent ($A \cap A \not = A$) operator, the neutral element of which is $\emptytype$, the empty multi set.
Note that there is no ground type, its role is played by the empty multi type $\emptytype$.

A typing judgment is of the shape $\typectx \types \tm \hastype T$ where $T$ is a linear or a multi type and $\typectx$ is a typing context, that is an assignment of multi types to a finite set of variables ($\typectx = \var_1 \hastype \mtype_1, \ldots, \var_n \hastype \mtype_n$). To derive a typing judgment, we follow the derivation rules defined in \Cref{fig:multi-types-vsc}.

\paragraph{Typing Rules} Linear types only type values, via the rules $\typingruleAx$ and $\typingruleAbs$. To give a multi type to value $\val$, one has to use the $\typingruleMany$ rule, turning an indexed family of linear types for $\val$ into a multi type. Note that any value can be typed with the empty multi type $\emptytype$. %The rules $\typingruleApp$ and $\typingruleES$ involve choosing a new type, which is what makes typing undecidable, but are syntax-based. 
The symbol $\uplus$ is the disjoint union operator on multi sets (which corresponds to our non-idempotent intersection on multi types).  

\paragraph{Characterization of Termination} A key property of multi types is that they characterize $\tovsc$ termination. The characterization is proved via subject reduction and expansion.

\begin{theorem}[Characterization of termination, \cite{DBLP:journals/pacmpl/AccattoliG22}]
\label{thm:mtypes-charac}
\hfill
\begin{enumerate}
\item 	\label{p:mtypes-charac-subject} \emph{Subject reduction and expansion}:	let $\tm \tovsc \tmtwo$. Then $\typectx \types \tm \hastype \mtype$ if and only if $\typectx \types \tmtwo \hastype \mtype$.

\item $\tm$ is $\tovsc$-terminating if and only if there exists $\typectx$ and $\mtype$ such that $\typectx \types \tm \hastype \mtype$.
\end{enumerate}
\end{theorem}
Since $\tovsc$-termination characterizes \cbv scrutability (\refthm{cbv-scrutability-characterization}), multi types characterize it too.

\paragraph{Multi Types Induce a Model}
Multi types induce a model
by interpreting a term %simply 
as the set of its type judgments. 
A possibly empty list of pairwise distinct variables $\vec{\var} = (\var_1, \dots, \var_n)$ is \emph{suitable for} $\tm$ if $\fv{\tm} \subseteq \{\var_1, \dots, \var_n\}$.
If $\vec{\var} = (\var_1, \dots, \var_n)$ is suitable for $\tm$, the \emph{semantics} $\sem{\tm}_{\vec{\var}}$ \emph{of} $\tm$ \emph{for} $\vec{\var}$ is given by:
\begin{align*}
	\sem{\tm}_{\vec{\var}} &\defeq \{((\mtypetwo_1,\dots, \mtypetwo_n),\mtype) \mid 
	\exists 
	\, 
	\concl{\tderiv}{\var_1 \hastype \mtypetwo_1, \dots, \var_n \hastype \mtypetwo_n}{\tm}{\mtype} \}
\end{align*}
This is exactly Ehrhard's \cbv relational model \cite{DBLP:conf/csl/Ehrhard12}. Ehrhard considers it with respect to Plotkin's calculus. We do not prove that it is a model for the VSC, because there is no formal notion of VSC model. We do have, however, that subject reduction and expansion (\refthmp{mtypes-charac}{subject}) ensure that the interpretation $\sem{\tm}_{\vec{\var}}$ is \emph{invariant} by $\tovsc$, and compatibility of the induced equational theory is proved below. These properties are what the definitions of $\l$-models or categorical models are meant to ensure. Moreover, the characterization theorem (\refthm{mtypes-charac}) ensures that $\sem{\tm}_{\vec{\var}}$ is adequate.


\begin{corollary}[\cite{DBLP:journals/pacmpl/AccattoliG22}]
	\label{thm:invariance-and-adequacy}
	Let $\tm$ be a term in the \VSC with $\vec{\var}  = (\var_1, \dots, \var_n)$ suitable for it.
\begin{enumerate}
	\item \emph{Invariance}: if $\tm (\tovsc \cup \streq) \tmtwo$ then $\sem{\tm}_{\vec{\var}} = \sem{\tmtwo}_{\vec{\var}}$.
	\item \emph{Adequacy for} $\tovsc$: $\sem{\tm}_{\vec{\var}}$ is non-empty if and only if $\tm$ is $\tovsc$-terminating.

\end{enumerate}
\end{corollary}

\paragraph{The Type Preorder} Every model $M$ induces an equational theory defined as $\tm =_M \tmtwo$ if $\sem\tm_M = \sem\tmtwo_M$. For the multi types model, we consider the \emph{preorder} induced by $\interp\tm_{\vec\var} \subseteq \interp\tmp_{\vec\var}$.
\begin{definition}[Type preorder]
The type preorder $\tm \leqtype \tmp$ holds if $\typectx \types \tm \hastype \mtype$ implies $\typectx \types \tmp \hastype \mtype$.
\end{definition}
Rephrasing the definition with respect to interpretations, we have that $\tm \leqtype \tmp$ if $\interp\tm_{\vec\var} \subseteq \interp\tmp_{\vec\var}$ for every list of suitable variables $\vec\var$. By the adequacy of $\interp\tm_{\vec\var}$, it follows the adequacy of $\leqtype$. Compatibility is easily proved directly, for once, and soundness follows.

\begin{toappendix}
\begin{proposition}[Compatibility of $\leqtype$] \label{prop:type-preorder-is-compatible}
\hfill
\begin{enumerate}
\item \emph{Compatibility}: if $\tm\leqtype\tmp$ then $\ctxp\tm \leqtype \ctxp\tmp$.
\item \emph{Soundness}: if $\tm\leqtype\tmp$ then $\tm\eqcv\tmp$.
\end{enumerate}
\end{proposition}
\end{toappendix}

\paragraph{Enf and Net Are Included in Type}
Now, we show that $\leqenf$ and $\leqnet$ are both included in $\leqtype$. For that, we prove that, if $\tm\leqenf\tmp$ or $\tm\leqnet \tmp$, then any typing derivation for $\tm$ can be transformed in a typing derivation for $\tmp$ having the same final judgement. The next two propositions and the associated lemma are proved by induction on the following notion: the size $\size\typeder$ of a type derivation $\typeder$ is the number of rule occurences in $\typeder$ except for rule $\typingruleMany$.

\begin{toappendix}
\begin{proposition}
	\label{l:bisimulation-preserves-typeder}
\hfill
\begin{enumerate}
\item \emph{Net simulations and type derivations}:
	let $\relsym$ be a \net simulation. If $\tm \rel \tmp$ and $\typeder: \typectx \types \tm \hastype \mtype$ then there exists a derivation $\typederp: \typectx \types \tmp \hastype \mtype$.
	\item \emph{Net is included in Type}: if $\tm\leqnet\tmp$ then $\tm\leqtype\tmp$.
	\end{enumerate}
\end{proposition}
\end{toappendix}

%\begin{corollary}
%	$\leqnet \subseteq \leqtype$.
%\end{corollary}

To relate \enf similarity and typability, we need a lemma to deal with Lassen's stop-and-go. 
\begin{toappendix}
\begin{lemma}[Stop-and-go and type derivations]
	\label{l:smaller-derivations-stuck}
	Let $\typeder \derives \typectx \types {\levctxp{\var\val}} \hastype \mtype$ and $\varthree$ be fresh. Then there exist $\typeder_\levctx \derives  \typectx_\levctx, \varthree \hastype \mtypetwo \types  \levctxp{\varthree} \hastype \mtype$ with $\size{\typeder_\levctx}<\size\typeder$ and $\typeder_\val \derives \typectx_\val \types \val : \mtypetwo_1$ with $\size{\typeder_\val}<\size\typeder$.
\end{lemma}
\end{toappendix}

\begin{toappendix}
\begin{proposition}
	\label{l:enf-bisimulation-preserves-typeder}
	\hfill
\begin{enumerate}
\item \emph{Enf simulations and type derivations}:
	let $\relsym$ be an \enf simulation. If $\tm \rel \tmp$ and $\typeder \derives \typectx \types \tm \hastype \mtype$ then there exists a derivation $\typederp \derives \typectx \types \tmp \hastype \mtype$.
	\item \emph{Enf is included in Type}: if $\tm\leqenf\tmp$ then $\tm\leqtype\tmp$.
	\end{enumerate}
\end{proposition}
\end{toappendix}



%
%\begin{corollary}
%	$\leqenf \subseteq \leqtype$.
%\end{corollary}

%\paragraph{Proof method}We shall now use this type equivalence as a program equivalence. For axiomatisable rules like the equivalences described in this paper, it is fairly easy to show that type equivalence validates them (for structural equivalence see \refprop{equivstruct-subseteq-equivtype}). 
%
%
%\begin{proposition}[Type equivalence validates structural equivalence]
%	\label{prop:equivstruct-subseteq-equivtype}
%	If $\tm \equiv \tmtwo$, then $\tm\equivtype\tmtwo$.
%\end{proposition}
%
%\begin{proposition}[Type equivalence validates Moggi's equivalences]
%	
%	\label{prop:equivmoggi-subseteq-equivtype}
%	If $\tm \equivlid \tmtwo$, $\tm \equiv_{rad} \tmtwo$, $\tm \equiv_{rad} \tmtwo$ or $\tm \equiv_{ass} \tmtwo$, then $\tm\equivtype\tmtwo$.
%\end{proposition}
%
%We look at an easy example, with Moggi's $\equivlid$ rule. The statement we want to prove is the following:
%\begin{center}
%	$\forall \typectx,\mtype~\,\,\,\,\,\,\, \typectx \types \tm \hastype \mtype \iff \typectx\types\var\esub\var\tm\hastype\mtype$
%\end{center}
%
%Note that at first glance, this statement seems as hard as proving contextual equivalence, because of the universal quantification over typing contexts and multi types. However, in this case we can unfold the derivation for $ \typectx\types\var\esub\var\tm\hastype\mtype$ as there is only one possibility for the last derivation rule given the syntax of the term.
%
%
%\begin{center}
%	$\forall \typectx,\mtype~\,\,\,\,\,\,\, \typectx \types \tm \hastype \mtype \iff \exists \mtypetwo\,\,\, \infer[\typingruleES]{\typectx\types\var\esub\var\tm\hastype\mtype}{\var\hastype \mtypetwo \types \var\hastype \mtype & \typectx \types \tm \hastype \mtypetwo}$
%\end{center}
%
%We cannot apply the rule $\typingruleAx$ yet on the left side of the derivation, as $\mtype$ is not \emph{linear}, hence we can only apply $\typingruleMany$. Note that there are \emph{many} possibilities on how to apply the rule (depending on the choice of $I$ for the premices).
%
%
%\begin{center}
%	$\ldots \iff \exists n, (\mtypetwo_i)_{1\leq i \leq n},(\ltype_i)_{1\leq i \leq n}\,\,\,  \infer[\typingruleES]{\typectx\types\var\esub\var\tm\hastype\multitype{n}{\ltype} = \mtype}{\infer[\typingruleMany]{\biguplus_{0\leq i\leq n}(\var\hastype \mtypetwo_i) \types \var\hastype \multitype{n}{\ltype}}{\left(\var\hastype \mtypetwo_i \types \var \hastype \ltype_i\right)_{1\leq i \leq n}} & \typectx \types \tm \hastype \biguplus_{0\leq i\leq n}\mtypetwo_i}$
%\end{center}
%
%%Note that for the second possibility, the typing judgment $\var\hastype \emptytype \types \var\hastype \emptytype$ is the same as $\emptytypectx \types \var\hastype \emptytype$. 
%We keep unfolding the rules (if $n>0$): while for now we introduced an existential quantifier for the types $\mtypetwo$ and $\ltype$, because of the left leaves of the derivation, the only possibility is that $\mtypetwo_i = [\ltype_i]$.
%
%
%\begin{center}
%	$\forall \typectx,\mtype~\,\, \typectx \types \tm \hastype \mtype \iff \exists n, (\ltype_i)_{1\leq i \leq n}\,\,\,  \infer[\typingruleES]{\typectx\types\var\esub\var\tm\hastype\multitype{n}{\ltype} = \mtype}{\infer[\typingruleMany]{\var\hastype \multitype{n}{\ltype} \types \var\hastype \multitype{n}{\ltype}}{\left(\infer[\typingruleAx]{\var\hastype [\ltype_i] \types \var \hastype \ltype_i}{}\right)_{1\leq i \leq n}} & \typectx \types \tm \hastype \multitype{n}{\ltype}}$
%\end{center}
%
%The resulting statement is trivial, as any multi type $\mtype$ can be written as $ \multitype{n}{\ltype}$.
%
%Similar arguments can be made to prove that structural equivalence, Moggi's $\equiv_{ass}$, $\equiv_{lad}$, $\equiv_{rad}$ rules are validated by type equivalence $\equivtype$. On the other hand, it seems fairly intricated to prove that Turing's and Curry's fixpoint combinators are type equivalent, as arises the problem of universal quantification as well as the general complications when trying to type fixpoint combinators.


%If one is able to develop an appropriate notion of \emph{type normal form simulation} to match denotational identity, the proof technique to prove completeness is quite satisfactory. Indeed, by coinduction, it suffices to show that $\leqtype$ is a type simulation. Depending on the definition of type simulation, such a proof could be quite easy or relying on a \cbv separability theorem.
%Depending on how type bisimulations are defined, one might need a statement close to a \emph{separability} theorem for \cbv.
%\begin{center}
%	if $\tm\not\equivtype\tmp$ then $\tm \not \equiv_{todo}\tmp$
%\end{center}
%
% Where $\equiv_{todo}$ would be a set of equivalences relating normal forms of terms $\tm$ and $\tmp$. A criteria like this seems too syntactical to be able to grasp semantics, here specifically non-idempotent intersection types semantics.
%Back to \cbv scrutability concerns, by Proposition \ref{prop:typability-normalization}, inscrutable terms are exactly minimal elements for the $\leqtype$ preorder (and therefore are equated by $\equivtype$). 

%\cadr{\paragraph{$\etav$ Equivalence and Failure of Full Abstraction} Concerning $\etav$ equivalence, the multi type system we consider does not validate it but this is standard and it is usually fixable---in \cbn---by adding a recursive equation on the ground type. To our knowledge, however, the question has not been studied in \cbv.
%About full abstraction with respect to \cbv contextual equivalence $\eqcv$, it is easily seen that it fails, as type equivalence does not validate \cbn duplication, while $\eqcv$ does.}{}

%%%%For multi types with ground type
\ignore{\paragraph{$\etav$ Reduction} Concerning $\etav$ equivalence, the multi type system we consider does not (fully) validate it. In \cbn but this is standard and it is usually fixable by adding a recursive equation on the ground type. To our knowledge, however, the question has not been studied in \cbv. 

Part of $\eta_v$ equivalence, that is $\eta_v$-reduction, is however validated by the type preorder. On the other side, $\eta_v$-expansion fails. Hence none of them is validated by the symmetric closure of the type preorder, namely type equivalence. We first show the result on $\eta_v$-reduction for variables.
Actually, $\eta_v$ equivalence on closed values is standard and validated by $\alpha$-equivalence and reduction steps below lambdas. Therefore we can describe exactly how $\eta_v$ is included in the type preorder.
\begin{toappendix}
\begin{proposition} Let $\var$ a variable and $\val$ a value. One then has:
	\label{prop:etav-for-leqtype}
	\begin{enumerate}
		\item \emph{(Variable $\eta_v$-equivalence)} $\la\vartwo\var\vartwo \leqtype \var$, but $\var \not \leqtype \la\vartwo\var\vartwo$,
		\item  \emph{(Value $\eta_v$-equivalence)} $\la\vartwo\val\vartwo \leqtype \val$ for any $\val$ and $\val \leqtype \la\vartwo\val\vartwo$ iff $\val$ is an abstraction.
	\end{enumerate}
\end{proposition}
\end{toappendix}}



\paragraph{$\eta_v$ Equivalence} 
By the fact that \enf and \net similarities are incomparable follows that they are strictly included in $\leqtype$. A further  gap between the type preorder $\leqtype$ and $\leqenf$ or $\leqnet$ is $\eta_v$ equivalence, which is included in $\leqtype$ but not in $\leqenf$ nor $\leqnet$.
%It is somewhat arbitrary to describe linear types with a ground type (see \Cref{fig:multi-types-vsc}), and the counterexample for $\var \leqtype \la\vartwo\var\vartwo$ precisely uses the existence of a ground type. We now show that if we define linear types with the grammar $\ltype \grameq [\mtype \multimap \mtypetwo]$, $\eta_v$ equivalence is included in the induced preorder $\leqtypetwo$. 
%Again, we reason first on $\eta_v$-reduced or $\eta_v$-expanded variables. In the same fashion as before, we can extend the result to all values, which gives us that $\eta_v$ equivalence is included in the $\leqtypetwo$ preorder.
%We first show the result on $\eta_v$-reduction for variables.
%Actually, $\eta_v$ equivalence on closed values is standard and validated by $\alpha$-equivalence and reduction steps below lambdas.
%The technical statement is the first part of the next proposition, as $\eta_v$ equivalence on abstractions is included in $\beta_v$ equivalence.

\begin{toappendix}
	\begin{proposition}[Variable $\eta_v$-equivalence is included in type equivalence]	\label{prop:etav-for-leqtypetwo}
	Let $\var$ a variable. Then $\la\vartwo\var\vartwo \leqtype \var$ and $\var \leqtype \la\vartwo\var\vartwo$.
	\end{proposition}
\end{toappendix}



\paragraph{Characterizing Type Equivalence} We conjecture that $\leqtype$ is exactly the sup of the \enf and \net similarities enriched with $\eta_v$ equivalence, that is, that generalizing $\leqnet$ as to validate $\equivlid$ and $\equivetav$ would match $\leqtype$. If the conjecture is false, finding a normal form similarity presentation of $\leqtype$---which corresponds to describe the equational theory of \cbv relational semantics---is anyway an interesting and challenging problem.

We leave refining the definition of \net bisimilarity to include $\equivetav$ for future work. The addition of $\equivetav$  has been investigated in the literature, as it is already present in Lassen's work about enf bisimilarity \cite{LassenEnf} and is studied in more depth by \citet{DBLP:journals/lmcs/BiernackiLP19}.

About full abstraction with respect to \cbv contextual equivalence $\eqcv$, it fails for $\equivtype$, as $\equivtype$ is cost-sensitive---it does not validate \cbn duplication---while $\eqcv$ is cost-insensitive.




% !TEX root = main.tex
\section{Conclusions}
Motivated by the fact that Lassen's enf bisimilarity $\eqenf$---the normal form bisimilarity of reference in \cbv---does not identify $\Omega$-terms and commuting $\letexp$s, we introduced \emph{net bisimilarity} $\eqnet$, which does identify them. It turns out, however, that $\eqnet$ does not validate Moggi's laws nor $\etav$, which are instead validated by $\eqenf$. Additionally, it is unclear how to extend either enf or net bisimilarity as to catch the other one.

Such a problematic duality led us to develop a sharp analysis of \cbv and of the principles that can be validated or not by normal form bisimulations. The analysis shows that the semantic landscape of \cbv is considerably richer and more sophisticated than the \cbn one. %In particular, cost-sensitive program equivalences are of interest in \cbv, yet contextual equivalence is cost-insensitive.

Concretely, our analysis contributed two further equivalences. First, a naive bisimilarity $\eqncbv$, that mainly provides a better understanding of Lassen's tricky definition of enf simulations. Second, the type equivalence $\equivtype$ induced by Ehrhard's multi types, which subsumes both enf and net bisimilarity, and includes $\etav$-equivalence, while retaining their cost-sensitive aspect. In practice, $\equivtype$ is not really usable for comparing programs, but it provides a sharp theoretical tool.

\paragraph{Future Work} Type equivalence suggests that it could be possible to find a normal form bisimilarity merging the enf and net ones. We are actively working on this challenging problem. A related question is finding an axiomatization of $\equivtype$, for which some sort of separation theorem should be developed.
We would also like to investigate how net bisimilarity $\eqnet$ relates to the topics connected to $\eqenf$, such as  game semantics \cite{DBLP:conf/lics/JaberM21}, extensions with effects \cite{DBLP:conf/esop/LagoG19,DBLP:conf/fossacs/BiernackiLP19,biernacki_et_al:LIPIcs:2020:12329}, and the $\pi$-calculus \cite{DBLP:journals/tcs/DurierHS22}.


%%
%% The acknowledgments section is defined using the "acks" environment
%% (and NOT an unnumbered section). This ensures the proper
%% identification of the section in the article metadata, and the
%% consistent spelling of the heading.
%\begin{acks}
%To Robert, for the bagels and explaining CMYK and color spaces.
%\end{acks}

%%
%% The next two lines define the bibliography style to be used, and
%% the bibliography file.
\bibliographystyle{ACM-Reference-Format}
\bibliography{main.bbl}


%%
%% If your work has an appendix, this is the place to put it.
\newpage
\appendix
% !TEX root = main.tex
\section{Removed proofs in Section 7 Lassen's Eager Normal Form Simulation}
In this sections, we give the full proof of \refprop{enf-validation-of-equivalences} concerning shuffling equivalences, detailing the $\equivsone$ case. The case for $\equivsthree$ is similar.


\begin{lemma}
	\label{l:equivsone-is-included-in-enf}
	$\tm \equivsone \tmp$ then $\tm \enfbisim \tmp$.
\end{lemma}
\begin{proof}
	We prove that $\relsym = Id \cup \{(\tm,\tmp) \mid \tm \equivsone \tmp \}$ is an \enf bisimulation. First note that $Id \subseteq \openfp{Id}$ and $sym(Id) \subseteq \openfp{sym(Id)}$. We show that $\{(\tm,\tmp) \mid \tm \equivsone \tmp \} \subseteq \relenf$ (and the same reasoning shows that $sym(\relsym)$ is also an enf simulation).
	
	Let $(\tmrone,\tmrtwo)=(((\la\var\tm)\tmtwo)\tmthree,(\la\var\tm\tmthree)\tmtwo) \in\relsym$.
	\begin{itemize}
		\item If $\tmtwo \bslasdiv$, then $\tmrone$ and $\tmrtwo$ diverge, hence $(\tmrone,\tmrtwo) \in \relenf$ by case (enf 1).
		\item If $\tmtwo \bslas k \val$, then $\tmrone \tolas^{k+1} \tm\isub\var\val\tmthree$  and $\tmrtwo \tolas^{k+1} (\tm\tmthree)\isub\var\val$. We conclude $(\tmrone,\tmrtwo) \in \relenf$ because they both reduce to $\tm\isub\var\val\tmthree = (\tm\tmthree)\isub\var\val$ -- indeed $\var\not\in\fv\tmthree$) hence have the same normal form ($Id$ part of the relation $\relsym$).
		\item If $\tmtwo \bslas k \levctxp {\vartwo\val}$, then $\tmrone \bslas k ((\la\var\tm)\levctxp {\vartwo\val})\tmthree$ and $\tmrtwo \bslas k (\la\var\tm\tmthree)\levctxp {\vartwo\val}$. By case (enf 4), $(\tmrone,\tmrtwo) \in \relenf$ since $\val \rel \val$ and $((\la\var\tm)\levctxp {\varthree})\tmthree \rel (\la\var\tm\tmthree)\levctxp {\varthree}$ (because $\{(\tm,\tmp) \mid \tm \equivsone \tmp \} \subseteq \relsym$).
	\end{itemize}	
	Hence the result by coinduction.
\end{proof}

%\begin{lemma}
%	$\tm \rtosthree \tmp$ then $\tm \enfbisim \tmp$.
%\end{lemma}
%
%\begin{proof}
%	Same reasoning that for $\sigma_1$.
%\end{proof}


\gettoappendix{prop:enf-validation-of-equivalences}

\begin{proof} Moggi's equivalences proofs are straightforward, and already included in Lassen's original paper \cite{LassenEnf}. 
	
	We deduce the result for the shuffling equivalences by an easy coinductive argument described in \reflemma{equivsone-is-included-in-enf} for $\equivsone$. For $\equivsthree$, the argument is similar.
	
	Counterexamples for the other equivalences are easy to come up with.
\end{proof}




\input{A02-ProofBigStep}
% !TEX root = main.tex

\section{Proof of Compatibility of Toy Similarity}
\label{chapter:proof-compatibility-naf}
In this section, we will prove the first part of Proposition \ref{prop:main-lemma_vsc}. 
We first prove that $\lasrel$ and $\lasrelnaf$ are equivalent on normal forms, which is somewhat the proposition we need to prove but only on normal forms.
Then we need to prove intermediate lemmas that are technicalities on the structure of terms, and Proposition \ref{prop:naf-coherence} (proven in Section \ref{section:lemmas-implicit-case-naf-proof}). Once these are proven, we can directly prove Proposition \ref{prop:main-lemma_vsc}.

We remind the reader that the letter $\ntm$ is always used only for normal forms and that for this calculus only abstractions are values (the letter $\valp$ only refers to abstractions). 
%\renewcommand{\tmrthree}{\adr{P}}
%\renewcommand{\tmrfour}{\adr{Q}}

\subsection{Proof of Proposition \ref{prop:main-lemma_vsc} but only on normal forms}
\label{section:lasrel-lasrelnaf-normal-forms}
In this section, we prove $\lasrel = \lasrelnaf$ is we restrict the relations to normal forms. One direction is easy (\reflemma{lasrelnaf-normal-forms-lasrel-left-to-right}), the other one relies on some sublemmas. The equivalence is only true if $\relsym$ is a \naf simulation.

\begin{lemma}
	\label{l:lasrelnaf-normal-forms-lasrel-left-to-right}
	If $\ntm\lasrelnaf\ntmtwo$ then $\ntm\lasrel\ntmtwo$.
\end{lemma}
\begin{proof}
	By case analysis on $\ntm = \var \mid \valp \mid \itm\ntm \mid \ntm\esub\var\itm$.
\end{proof}

To show the other direction of the equivalence (\reflemma{lasrelnaf-normal-forms-lasrel-right-to-left}), we use the two following lemmas.

\begin{lemma}
	\label{l:relnaf-to-lasrelnaf-on-normal}
If $\ntm\relnaf\ntmtwo$ then $\ntm\lasrelnaf\ntmtwo$. 
\end{lemma}

\begin{proof} By monotonicity of $\opnaf$.
%	Cases of $\ntm$.
%	\begin{itemize}
%		\item \emph{Variable}, that is, $\ntm = \var$. Then $\ntmtwo = \var$ and $\var\lasrelnaf\var$ by definition of \naf. 
%		\item \emph{Abstraction}, that is, $\ntm = \la\var\tm$. Then $\ntmtwo = \la\var\tmp$ with $\tm\rel \tmp$. Then $\tm\lasrel \tmp$ by rule $(\sclift)$. Then $\la\var \tm \lasrelnaf \la\var\tmp$ by definition of \naf. 
%		\item \emph{Applied inert}, that is, $\ntm = \itm\ntmONE$. Then $\ntmtwo = \itmtwo\ntmONEtwo$ with $\itm \rel \itmtwo$ and $\ntmONE \rel \ntmONEtwo$. By rule $(\sclift)$, $\itm \lasrel \itmtwo$  and $\ntmONE \lasrel \ntmONEtwo$. Then $\ntm =\itm\ntmONE \lasrelnaf \itmtwo\ntmONEtwo = \ntmtwo$ by definition of \naf. 
%		\item \emph{Substituted inert}, that is, $\ntm = \ntmONE\esub\var\itm$. Then $\ntmtwo = \ntmONEtwo\esub\var\itmtwo$ with $\itm \rel \itmtwo$ and $\ntmONE \rel \ntmONEtwo$. By rule $(\sclift)$, $\itm \lasrel \itmtwo$  and $\ntmONE \lasrel \ntmONEtwo$. Then $\ntm = \ntmONE\esub\var\itm \lasrelnaf \ntmONEtwo\esub\var\itmtwo = \ntmtwo$ by definition of \naf. 
%	\end{itemize}
\end{proof}

%\begin{lemma}
%	\label{l:relnaf-to-lasrel-on-normal}
%	If $\ntm\relnaf\ntmtwo$ then $\ntm\lasrel\ntmtwo $.
%\end{lemma}
%
%\begin{proof}
%	Cases of $\ntm$.
%	\begin{itemize}
%		\item \emph{Variable}, that is, $\ntm = \var$. Then $\ntmtwo = \var$ and $\var\lasrel\var$ by rule ($\scvar$).
%		\item \emph{Abstraction}, that is, $\ntm = \la\var\tm$. Then $\ntmtwo = \la\var\tmp$ with $\tm\rel \tmp$. Then $\tm\lasrel \tmp$ by rule $(\sclift)$. Then $\la\var \tm \lasrel \la\var\tmp$ by rule ($\scabs$).
%		\item \emph{Applied inert}, that is, $\ntm = \itm\ntmONE$. Then $\ntmtwo = \itmtwo\ntmONEtwo$ with $\itm \rel \itmtwo$ and $\ntmONE \rel \ntmONEtwo$. By rule $(\sclift)$, $\itm \lasrel \itmtwo$  and $\ntmONE \lasrel \ntmONEtwo$. Then $\ntm =\itm\ntmONE \lasrel \itmtwo\ntmONEtwo = \ntmtwo$ by rule ($\scapp$).
%		\item \emph{Substituted inert}, that is, $\ntm = \ntmONE\esub\var\itm$. Then $\ntmtwo = \ntmONEtwo\esub\var\itmtwo$ with $\itm \rel \itmtwo$ and $\ntmONE \rel \ntmONEtwo$. By rule $(\sclift)$, $\itm \lasrel \itmtwo$  and $\ntmONE \lasrel \ntmONEtwo$. Then $\ntm = \ntmONE\esub\var\itm \lasrel \ntmONEtwo\esub\var\itmtwo = \ntmtwo$ by rule ($\scesub$).
%	\end{itemize}
%\end{proof}

Notice that the next lemma already proves part of the conclusion of the first part of Proposition \ref{prop:naf-coherence}.


\begin{lemma}[Constrained Substitutivity of $\lasrelnaf$ on normal forms]
	\label{l:lasrelnaf-normal-forms-substitutive}
	If $\ntmONE \lasrelnaf \ntmTWO$, $\valp \lasrelnaf \valptwo$ and $\ntmONE\isub\var{\valp}$ and $\ntmTWO\isub\var{\valptwo}$ are $\tovscp$-normal then $\ntmONE\isub\var{\valp} \lasrelnaf \ntmTWO\isub\var{\valptwo}$.
\end{lemma}


\begin{proof}
	By case analysis on the shape of $\ntmONE$. Cases:
	\begin{itemize}
		\item $\ntmONE = \var$ and $\ntmTWO = \var$ then $\ntmONE\isub\var{\valp} = \valp \lasrelnaf \valptwo = \ntmTWO\isub\var{\valptwo}$.
		
		\item $\ntmONE = \vartwo$ and $\ntmTWO = \vartwo$ then $\ntmONE\isub\var{\valp} =  \vartwo \lasrelnaf \vartwo = \ntmTWO\isub\var{\valptwo}$.
		
		\item $\ntmONE = \la\vartwo\tm$ and $\ntmTWO = \la\vartwo\tmp$ with $\tm \lasrel \tmp$
		we have \[\infer{\tm\isub\var{\valp} \lasrel \tmp\isub\var{\valptwo}}{\tm \lasrel \tmp & \valp \lasrel \valptwo}\]
		hence by case (naf 3) $\ntmONE\isub\var{\valp} = \la\vartwo{\tm\isub\var{\valp}} \lasrelnaf  \la\vartwo{\tmp\isub\var{\valptwo}} = \ntmTWO\isub\var{\valptwo}$.
		
		
		\item $\ntmONE = \itmONEtwo\ntmONEtwo$ and $\ntmTWO = \itmTWOtwo\ntmTWOtwo$ with $\itmONEtwo \lasrel \itmTWOtwo$ and $\ntmONEtwo \lasrel \ntmTWOtwo$. 
		
		
		The hypothesis that $\ntmONE\isub\var\valp$ is normal is equivalent to $\itmONEtwo\isub\var\valp$ is an inert and $\ntmONEtwo\isub\var\valp$ is normal, and the hypothesis that $\ntmTWO\isub\var\valptwo$ is normal is equivalent to $\itmTWOtwo\isub\var\valptwo$ is an inert and $\ntmTWOtwo\isub\var\valptwo$ is normal.
		
		
		By \reflemma{lasrelnaf-normal-forms-lasrel-left-to-right},  $\itmONEtwo\isub\var{\valp} \lasrel \itmTWOtwo\isub\var{\valptwo}$ and $\ntmONEtwo\isub\var{\valp} \lasrel \ntmTWOtwo\isub\var{\valptwo}$. Hence $\ntmONE\isub\var{\valp} = \itmONEtwo\isub\var{\valp}\ntmONEtwo\isub\var{\valp} \lasrelnaf \itmTWOtwo\isub\var{\valptwo}\ntmTWOtwo\isub\var{\valptwo} = \ntmTWO\isub\var{\valptwo}$.
		
		
		%To conclude that $\ntmONE\isub\var\valp \lasrelnaf \ntmTWO\isub\var\valp$, what is only remaining is that $\itmONEtwo\isub\var\valp \lasrel \itmTWOtwo\isub\var\valptwo$ and  $\ntmONEtwo\isub\var\valp \lasrel \ntmTWOtwo\isub\var\valptwo$.
		
		We derive easily these facts: ($\valp\lasrelnaf\valptwo$ implies $\valp\lasrel\valptwo$ by \reflemma{lasrelnaf-normal-forms-lasrel-left-to-right})
		\[ \infer[\scsub]{\itmONEtwo\isub\var\valp \lasrel \itmTWOtwo\isub\var\valptwo}{\itmONEtwo \lasrel \itmTWOtwo & \valp \lasrel \valptwo} ~\text{and}~ \infer[\scsub]{\ntmONEtwo\isub\var\valp \lasrel \ntmTWOtwo\isub\var\valptwo}{\ntmONEtwo \lasrel \ntmTWOtwo & \valp \lasrel \valptwo}\]
		
		
		%	 By \ih we have $\itmONEtwo\isub\var{\valp} \lasrelnaf \itmTWOtwo\isub\var{\valptwo}$ and $\ntmONEtwo\isub\var{\valp} \lasrelnaf \ntmTWOtwo\isub\var{\valptwo}$. 
		%	 
		%	 \adr{Since $\ntmONE\isub\var{\valp}$ and $\ntmTWO\isub\var{\valptwo}$ are $\to$-normal, $\itmONEtwo\isub\var{\valp}$, $\itmTWOtwo\isub\var{\valptwo}$, $\ntmONEtwo\isub\var{\valp}$ and $\ntmTWOtwo\isub\var{\valptwo}$ all are $\to$-normal as well.}
		%	 
		%	 By \reflemma{lasrelnaf-normal-forms-lasrel-left-to-right},  $\itmONEtwo\isub\var{\valp} \lasrel \itmTWOtwo\isub\var{\valptwo}$ and $\ntmONEtwo\isub\var{\valp} \lasrel \ntmTWOtwo\isub\var{\valptwo}$. Hence $\ntmONE\isub\var{\valp} = \itmONEtwo\isub\var{\valp}\ntmONEtwo\isub\var{\valp} \lasrelnaf \itmTWOtwo\isub\var{\valptwo}\ntmTWOtwo\isub\var{\valptwo} = \ntmTWO\isub\var{\valptwo}$.
		%			
		
		
		
		
		\item $\ntmONE = \ntmONEtwo\esub\vartwo\itmONEtwo$ and $\ntmTWO = \ntmTWOtwo\esub\vartwo\itmTWOtwo$ with $\itmONEtwo \lasrel \itmTWOtwo$ and $\ntmONEtwo \lasrel \ntmTWOtwo$. 	The hypothesis that $\ntmONE\isub\var\valp$ is normal is equivalent to $\itmONEtwo\isub\var\valp$ is an inert and $\ntmONEtwo\isub\var\valp$ is normal and the hypothesis that $\ntmTWO\isub\var\valptwo$ is normal is equivalent to $\itmTWOtwo\isub\var\valptwo$ is an inert and $\ntmTWOtwo\isub\var\valptwo$ is normal.
		
		To conclude that $\ntmONE\isub\var\valp \lasrelnaf \ntmTWO\isub\var\valp$, what is only remaining is that $\itmONEtwo\isub\var\valp \lasrel \itmTWOtwo\isub\var\valptwo$ and  $\ntmONEtwo\isub\var\valp \lasrel \ntmTWOtwo\isub\var\valptwo$.
		
		We derive easily these facts: ($\valp\lasrelnaf\valptwo$ implies $\valp\lasrel\valptwo$ by \reflemma{lasrelnaf-normal-forms-lasrel-left-to-right})
		\[ \infer[\scsub]{\itmONEtwo\isub\var\valp \lasrel \itmTWOtwo\isub\var\valptwo}{\itmONEtwo \lasrel \itmTWOtwo & \valp \lasrel \valptwo} ~\text{and}~ \infer[\scsub]{\ntmONEtwo\isub\var\valp \lasrel \ntmTWOtwo\isub\var\valptwo}{\ntmONEtwo \lasrel \ntmTWOtwo & \valp \lasrel \valptwo}\]
		
	\end{itemize}
\end{proof}

\begin{lemma}
	\label{l:lasrelnaf-normal-forms-lasrel-right-to-left}
	If $\relsym$ is a \naf simulation.
	If $\ntm\lasrel\ntmtwo$ then $\ntm\lasrelnaf\ntmtwo$.
\end{lemma}

\begin{proof}
	By induction on the derivation $\ntm \lasrel \ntmtwo$. Cases of the last rule in the derivation of $\ntm\lasrel\ntmtwo$:
	\begin{itemize}
		\item \emph {$\scvar$}\[ \infer[\scvar]{\var \lasrel \var}{} \]
		then $\var \lasrelnaf \var$ by definition of \naf.
		\item \emph {$\scabs$} \[ \infer[\scabs]{\ntm = \la\var\tm \lasrel \la\var\tmp = \ntmtwo}{\tm \lasrel \tmp} \]
		then $\ntm \lasrelnaf \ntmtwo$ by definition of \naf with $\tm \lasrel \tmp$.
		\item \emph {$\sclift$} \[ \infer[\scabs]{\ntm \lasrel \ntmtwo}{\ntm \rel \ntmtwo} \]
		then since $\relsym$ is a \naf simulation $\ntm \relnaf \ntmtwo$, hence by \reflemma{relnaf-to-lasrelnaf-on-normal}  $\ntm \lasrelnaf \ntmtwo$.
		\item \emph {$\scapp$} \[ \infer[\scapp]{\ntm = \itmONE\ntmONE \lasrel \itmTWO\ntmTWO = \ntmtwo}{\itmONE \lasrel \itmTWO & \ntmONE \lasrel \ntmTWO} \]then $\ntm \lasrelnaf \ntmtwo$ by definition of \naf with $\itmONE \lasrel \itmTWO$ and $\ntmONE \lasrel \ntmTWO$.
		\item \emph {$\scesub$} \[ \infer[\scesub]{\ntm = \ntmONE\esub\var\itmONE \lasrel \ntmTWO\esub\var\itmTWO = \ntmtwo}{\ntmONE \lasrel \ntmTWO & \itmONE \lasrel \itmTWO} \] then $\ntm \lasrelnaf \ntmtwo$
		by definition of \naf with $\itmONE \lasrel \itmTWO$ and $\ntmONE \lasrel \ntmTWO$.
		\item \emph {$\scsub$} \[ \infer[\scsub]{\ntm = \ntmONE\isub\var\valp \lasrel \ntmTWO\isub\var\valptwo = \ntmtwo}{\ntmONE \lasrel \ntmTWO & \valp \lasrel \valptwo} \]
		by \ih we have $\ntmONE \lasrelnaf \ntmTWO$ and $\valp \lasrelnaf \valptwo$. By \reflemma{lasrelnaf-normal-forms-substitutive}, $\ntmONE\isub\var{\valp} \lasrelnaf \ntmTWO\isub\var{\valptwo}$.\qedhere
	\end{itemize}
\end{proof}


\subsection{Normal substituted terms characterization}
The two following lemmas are important to know a term behavior when a variable is substituted -- mainly for the ($\scsub$) case in the proof of Proposition \ref{prop:main-lemma_vsc}.

\begin{lemma}
	\label{l:normal-form-substituted}
	$\ntm\isub\var\valp$ is normal iff $\ntm \not = \evctxp{\isctxp\var\fire}$ and $\ntm \not  = \evctxp{\tm\esub\vartwo{\isctxp\var}}$
\end{lemma}

\begin{proof}
	($\Rightarrow$) By contraposition, $\ntm\isub\var\valp$ would not be normal in both cases.
	
	($\Leftarrow$) Suppose $\ntm\isub\var\valp$ is not normal, then find a contradiction by exposing where the reduction happens.
\end{proof}

\begin{lemma}
	$\itm\isub\var\valp$ is an inert term iff ($\itm$ is inert,) $\itm\isub\var\valp$ is normal and $\itm \not = \isctxp\var$
\end{lemma}

\begin{proof}
	($\Rightarrow$) Trivial.
	
	($\Leftarrow$) By induction on $\itm$ (using \reflemma{normal-form-substituted}).
\end{proof}

\subsection{Bottom-up lemmas for a top-down defined simulation}
These are intermediate lemmas that can be technical but do not carry much meaning for the proofs.
\reflemma{variable-inerts-stable-lasrelnaf} is essential to prove \reflemma{lasrelnaf-on-normal-subs_vsc} (one of the main sublemma for the ($\scsub$) case of the compatibility proof). \reflemma{abstraction-inerts-stable-lasrelnaf} is helpful for the compatibility proof and relating normal forms.


\paragraph{An induction principle on substitution lists.} The following \reflemma{lasrelnaf-inert-lists-induction} gives us an "induction principle" on $\isctx,\isctxtwo$ when $\isctxp\var \lasrelnaf \isctxtwop\var$ (or $\isctxp\valp \lasrelnaf \isctxtwop\valptwo$). This is quite clear when looking at how $\opnaf$ is defined (very constraining on the structure of normal forms) and because of the equivalence of $\lasrel$ and $\lasrelnaf$ on normal forms. The induction principle will facilitate proofs from now on.

\begin{lemma}
	\label{l:lasrelnaf-inert-lists-induction}
	If $\isctxp\var \lasrelnaf \isctxtwop\var$ (or $\isctxp\valp \lasrelnaf \isctxtwop\valptwo$) then either $\isctx,\isctxtwo =\ctxhole,\ctxhole$ or $\isctx,\isctxtwo = \isctxONE\esub\vartwo\itm,\isctxONEtwo\esub\vartwo\itmtwo$ with $\isctxONEp\var \lasrelnaf \isctxONEtwop\var$ (or $\isctxONEp\valp \lasrelnaf \isctxONEtwop\valptwo$) and $\itm \lasrelnaf \itmtwo$.
\end{lemma}

\begin{proof}
	By contradiction and case exhaustion, these are the only possibilities for $\isctxp\var \lasrelnaf \isctxtwop\var$ (or $\isctxp\valp \lasrelnaf \isctxtwop\valptwo$).
	\begin{itemize}
		\item If only one of the lists is empty: $\var \lasrelnaf \itm\esub\vartwo\itmtwo$ (or $\var \lasrelnaf \ntm\esub\vartwo\itmtwo$) is not possible given the definition of \naf.
		\item If $\isctx,\isctxtwo = \isctxONE\esub\vartwo\itm,\isctxONEtwo\esub\vartwo\itmtwo$ and $\neg(\itm\lasrelnaf\itmtwo)$, we again have $\neg (\ntm\esub\vartwo\itm \lasrelnaf \ntmtwo\esub\varthree\itmtwo)$.\qedhere
	\end{itemize}
\end{proof}


\begin{lemma}
	\label{l:variable-inerts-stable-lasrelnaf}
	If $\ntm \lasrelnaf \ntmtwo$ then ($\ntm =\isctxp\var$ $\iff$ $\ntmtwo = \isctxtwop\var$)
\end{lemma}

\begin{proof}
	Proof by induction on $\isctx,\isctxtwo$ using \ref{l:lasrelnaf-normal-forms-lasrel-left-to-right} and \ref{l:lasrelnaf-normal-forms-lasrel-right-to-left}.
\end{proof}

\begin{lemma}
	\label{l:abstraction-inerts-stable-lasrelnaf}
	If $\ntm \lasrelnaf \ntmtwo$ then ($\ntm =\isctxp{\la\var\tm}$ $\iff$ $\ntmtwo = \isctxtwop{\la\var\tmp}$)
\end{lemma}

\begin{proof}
	Proof by induction on $\isctx,\isctxtwo$ using \ref{l:lasrelnaf-normal-forms-lasrel-left-to-right} and \ref{l:lasrelnaf-normal-forms-lasrel-right-to-left}.
\end{proof}




\begin{lemma}
	\label{l:lasrel-stable-with-lists-subst}
	$\isctxp\var \lasrel \isctxtwop\var$, $\tm \lasrel \tmp$ and $\valp \lasrel \valptwo$ $\Rightarrow$ $\sctxp{\tm} \lasrel \sctxtwop{\tmp}$ with $\sctx \defeq \isctx\isub\var\valp$ and $\sctxtwo \defeq \isctxtwo\isub\var\valptwo$.
\end{lemma}

\begin{proof}
	$\isctxp\var$ and $\isctxtwop\var$ are normal so $\isctxp\var \lasrel \isctxtwop\var \Rightarrow \isctxp\var \lasrelnaf \isctxtwop\var$ (by lemma \ref{l:lasrelnaf-normal-forms-lasrel-right-to-left}).
	By induction on $\isctx$ ($\isctx$ and $\isctxtwo$ have same length because $\isctxp\var \lasrelnaf \isctxtwop\var$ and \reflemma{lasrelnaf-inert-lists-induction}).
	\begin{itemize}
		\item $\isctx,\isctxtwo = \ctxhole, \ctxhole$
		then \[ \tm \lasrel \tmp \]
		\item $\isctx,\isctxtwo = \isctxONE\esub\vartwo\itm, \isctxONEtwo\esub\vartwo\itmtwo$ we have $\itm \lasrel \itmtwo$ and $\isctxONEp\var \lasrel \isctxONEtwop\var$ by $\isctxp\var \lasrelnaf \isctxtwop\var$ 
		by \ih we get $\sctxONEp{\tm} \lasrel \sctxONEtwop{\tmp}$ where $\sctxONE \defeq \isctxONE\isub\var\valp$ and $\sctxONEtwo \defeq \isctxONEtwo\isub\var\valptwo$.
		Then:
		\[\infer[\scesub]{\sctxp{\tm} =\sctxONEp{\tm}\esub\vartwo{\itm\isub\var\valp} \lasrel \sctxONEtwop{\tmp}\esub\vartwo{\itmtwo\isub\var\valptwo}= \sctxtwop{\tm}}{\sctxONEp{\tm} \lasrel \sctxONEtwop{\tmp} & \infer{\itm\isub\var\valp \lasrel \itmtwo\isub\var\valptwo}{\itm \lasrel \itmtwo & \valp \lasrel \valptwo}}\]\qedhere
	\end{itemize}
\end{proof}

\begin{lemma}
	\label{l:lasrelnaf-values-isctx-decomposition}
	$\isctxp\valp \lasrelnaf \isctxtwop\valptwo \iff \valp \lasrelnaf \valptwo$ and $ \isctxp\var \lasrelnaf \isctxtwop\var$ ($\var$ is fresh)
\end{lemma}

\begin{proof}
	By induction on lists $\isctx, \isctxtwo$. (They are of the same size because of how \naf is defined - see \reflemma{lasrelnaf-inert-lists-induction})
	\begin{itemize}
		\item $\isctx,\isctxtwo =\ctxhole,\ctxhole$ then $\valp \lasrelnaf \valptwo \iff \valp \lasrelnaf \valptwo$ and $\var\lasrelnaf\var$ is always true by case (\naf 2).
		\item $\isctx,\isctxtwo =\isctxONE\esub\vartwo\itm,\isctxONEtwo\esub\vartwo\itmtwo$, then
		
		by case (\naf 5), $\isctxp\valp \lasrelnaf \isctxtwop\valptwo \iff \isctxONEp\valp \lasrel \isctxONEtwop\valptwo$ and $\itm \lasrel \itmtwo$ 
		
		by \reflemma{lasrelnaf-normal-forms-lasrel-right-to-left}, $\iff \isctxONEp\valp \lasrelnaf \isctxONEtwop\valptwo$ and $\itm \lasrel \itmtwo$
		
		by \ih, $\iff \isctxONEp\var \lasrelnaf \isctxONEtwop\var$, $\valp \lasrelnaf \valptwo$ and $\itm \lasrel \itmtwo$
		
		by \reflemma{lasrelnaf-normal-forms-lasrel-left-to-right}, $\iff \isctxONEp\var \lasrel \isctxONEtwop\var$, $\valp \lasrelnaf \valptwo$ and $\itm \lasrel \itmtwo$
		
		and finally by case (\naf 5) $\iff \isctxp\var=\isctxONEp\var\esub\vartwo\itm \lasrelnaf \isctxONEtwop\var\esub\vartwo\itmtwo= \isctxtwop\var$, $\valp \lasrelnaf \valptwo$.\qedhere
	\end{itemize}
\end{proof}

%\begin{lemma}
%	\label{l:lasrelnaf-normal-forms-isctx-decomposition}
%	$\ntm \lasrelnaf \ntmtwo$ and $ \isctxp\var \lasrelnaf \isctxtwop\var$ ($\var$ is fresh) $\Rightarrow \isctxp\ntm \lasrelnaf \isctxtwop\ntmtwo$
%\end{lemma}

%\begin{lemma}
%	\label{l:lasrel-normal-forms-isctx-decomposition}
%	$\ntm \lasrel \ntmtwo$ and $ \isctxp\var \lasrelnaf \isctxtwop\var$ ($\var$ is fresh) $\Rightarrow \isctxp\ntm \lasrel \isctxtwop\ntmtwo$
%\end{lemma}
%\begin{proof}
%	By induction on $\isctx,\isctxtwo$.
%\end{proof}

\begin{lemma}
	\label{l:lasrelnaf-normal-forms-isctx-decomposition}
	$\ntm \lasrelnaf \ntmtwo$ and $ \isctxp\var \lasrelnaf \isctxtwop\var$ ($\var$ is fresh) $\Rightarrow \isctxp\ntm \lasrelnaf \isctxtwop\ntmtwo$
\end{lemma}
\begin{proof}
	By induction on $\isctx,\isctxtwo$.
\end{proof}

\subsection{Coherence of simulation and evaluation}
\label{section:lemmas-implicit-case-naf-proof}
We divide Proposition \ref{prop:naf-coherence} into two lemmas: \reflemma{lasrelnaf-on-normal-subs_vsc} and \reflemma{lasrelnaf-not-normal-subs}. Notice that part of the conclusion for the first part of \refprop{naf-coherence} was already proven in \ref{l:lasrelnaf-normal-forms-substitutive}.
	
\gettoappendix{prop:naf-coherence}

\begin{lemma}
	\label{l:lasrelnaf-on-normal-subs_vsc}
	Let $\rel$ be an \naf simulation, $\ntm \lasrelnaf \ntmtwo$, and $\valp\lasrelnaf\valptwo$. If $\ntm\isub\var\valp$ is $\tovscp$-normal then $\ntmtwo\isub\var\valptwo$ is $\tovscp$-normal.
\end{lemma}

\begin{proof}
	By induction on normal forms $\ntm$ for which $\ntm\isub\var\valp$ is $\tovscp$-normal:
	\begin{itemize}
		\item \emph{Variable}. Two sub-cases:
		\begin{itemize}
			\item $\ntm= \var$ and so $\ntm\isub\var\valp = \valp$. Then $\ntmtwo = \var$ by case (\naf 2) and $\ntmtwo\isub\var\valptwo = \valptwo$, which is $\tovscp$-normal. 
			
			
			\item $\ntm= \vartwo$ and so $\ntm\isub\var\valp = \vartwo$. Then $\ntmtwo = \vartwo$ by case (\naf 2) and $\ntmtwo\isub\var\valptwo = \vartwo$, which is $\tovscp$-normal. 
		\end{itemize}
		
		\item \emph{Abstraction}, that is, $\ntm = \la\vartwo\tm$ and so $\ntm\isub\var\valp = \la\vartwo\tm\isub\var\valp$. Then $\ntmtwo= \la\vartwo\tmp$ with $\tm \lasrel \tmp$. We have that $\ntmtwo\isub\var\valptwo = \la\vartwo\tmp\isub\var\valptwo$, which is $\tovscp$-normal.
		
		\item \emph{Substituted Inert}, that is, $\ntm = \ntmONE\esub\vartwo\itmONE$ and so $\ntm\isub\var\valp = \ntmONE\isub\var\valp\esub\vartwo{\itmONE\isub\var\valp}$.
		$\ntm\isub\var\valp$ is normal is equivalent to $\itmONE\isub\var\valp$ is inert (\ie $\itmONE\isub\var\valp$ is normal and $\itmONE\not = \isctxp\var$) and $\ntmONE\isub\var\valp$ is normal -- else this substitution would be a redex.
		Then $\ntmtwo = \ntmONEtwo\esub\vartwo\itmONEtwo$ with $\itmONE \lasrel \itmONEtwo$ and $\ntmONE \lasrel \ntmONEtwo$, which implies $\itmONE \lasrelnaf \itmONEtwo$ and $\ntmONE \lasrelnaf \ntmONEtwo$ by \reflemma{lasrelnaf-normal-forms-lasrel-right-to-left}.
		By \ih we then have $\itmONEtwo\isub\var\valptwo$ is $\tovscp$-normal, $\ntmONEtwo\isub\var\valptwo$ is $\tovscp$-normal.
		
		We conclude by stating $\ntmtwo\isub\var\valptwo$ is $\tovscp$-normal :
		since $\itmONEtwo\isub\var\valptwo$ and $\ntmONEtwo\isub\var\valptwo$ are $\tovscp$-normal the only possibility which would trigger a $\toe$-step is that $\itmONEtwo = \isctxtwop\var$. But $\itmONE \lasrelnaf \itmONEtwo$, (by \reflemma{variable-inerts-stable-lasrelnaf}) would mean $\itmONE = \isctxp\var$ which is impossible since $\ntm\isub\var\valp = \ntmONE\isub\var\valp\esub\vartwo{\itmONE\isub\var\valp}$ is normal by assumption.
		
		
		\item \emph{Applied Inert}, that is, $\ntm = \itmONE\ntmONE$ and so $\ntm\isub\var\valp = {\itmONE\isub\var\valp}\ntmONE\isub\var\valp$.
		$\ntm\isub\var\valp$ is normal is equivalent to $\itmONE\isub\var\valp$ is inert (\ie $\itmONE\isub\var\valp$ is normal and $\itmONE\not = \isctxp\var$) and $\ntmONE\isub\var\valp$ is normal -- else this application would be a redex.
		Then $\ntmtwo = \itmONEtwo\ntmONEtwo$ with $\itmONE \lasrel \itmONEtwo$ and $\ntmONE \lasrel \ntmONEtwo$, which implies $\itmONE \lasrelnaf \itmONEtwo$ and $\ntmONE \lasrelnaf \ntmONEtwo$ by \reflemma{lasrelnaf-normal-forms-lasrel-right-to-left}.
		By \ih we then have $\itmONEtwo\isub\var\valptwo$ is $\tovscp$-normal, $\ntmONEtwo\isub\var\valptwo$ is $\tovscp$-normal.
		
		We conclude by stating $\ntmtwo\isub\var\valptwo$ is $\tovscp$-normal :
		since $\itmONEtwo\isub\var\valptwo$ and $\ntmONEtwo\isub\var\valptwo$ are $\tovscp$-normal the only possibility which would trigger a $\tom$-step is that $\itmONEtwo = \isctxtwop\var$. By $\itmONE \lasrelnaf \itmONEtwo$, (by \reflemma{variable-inerts-stable-lasrelnaf}) would mean $\itmONE = \isctxp\var$ which is impossible since $\ntm\isub\var\valp = {\itmONE\isub\var\valp}\ntmONE\isub\var\valp$ is normal by assumption.\qedhere
		
	\end{itemize}
\end{proof}

\begin{lemma} 
	\label{l:lasrelnaf-not-normal-subs}
	If $\ntm_\tmrone \lasrelnaf \ntm_\tmrtwo$, $\valp \lasrel \valptwo$
	and $\ntm_\tmrone\isub\var{\valp} \tovscp \tmronep$
	then $\ntm_\tmrtwo\isub\var{\valptwo}  \tovscp \tmrtwop$ and $\tmronep \lasrel \tmrtwop$.
\end{lemma}

\begin{proof}
	(We write $\valp=\la\vartwo\tmfour$ and $\valptwo = \la\vartwo\tmfourp$ with $\tmfour \lasrel \tmfourp$ - using the fact that ($\valp \lasrel \valptwo \Rightarrow \valp \lasrelnaf \valptwo$) by \ref{l:lasrelnaf-normal-forms-lasrel-right-to-left}.)
	
	If $\ntm_\tmrone\isub\var{\valp} \tovscp \tmronep$ then $\ntm_\tmrone = \evctxp{\isctxp\var\fire}$ or $\ntm_\tmrone = \evctxp{\fire\esub\vartwo{\isctxp\var}}$ ($\evctx$ is the context where the reduction has been done).
	Show (by induction on $\evctx$) that $\tmronep \lasrel \tmrtwop$.
	\begin{itemize}
		\item $\evctx = \ctxhole$
		then $\ntm_\tmrone = \isctxp\var\fire$ or $\ntm_\tmrone = \fire\esub\vartwo{\isctxp\var}$, 
		\begin{itemize}
			\item \emph {$\ntm_\tmrone = \isctxp\var\fire$} by $\ntm_\tmrone \lasrelnaf \ntm_\tmrtwo$ we have $\ntm_\tmrtwo = \isctxtwop\var\firetwo$ with $\isctxp\var \lasrel \isctxtwop\var$ and $\fire \lasrel \firetwo$. 
			Set $\sctx \defeq \isctx\isub\var\valp$ and $\sctxtwo \defeq \isctxtwo\isub\var\valptwo$.
			
			$\ntm_\tmrtwo\isub\var\valptwo = \sctxtwop\valptwo\firetwo\isub\var\valptwo\tovscp \sctxtwop{\tmfourp\esub\vartwo{\firetwo\isub\var\valptwo}} = \tmrtwop$
			
			We also have $\ntm_\tmrone\isub\var\valp \tovscp \sctxp{\tmfour\esub\vartwo{\fire\isub\var\valp}} = \tmronep$ and we can build the following derivation:
			\[ \infer[\scesub]{\tmfour\esub\vartwo{\fire\isub\var\valp} \lasrel \tmfourp\esub\vartwo{\firetwo\isub\var\valptwo}}{\tmfour \lasrel \tmfourp & \infer{\fire\isub\var\valp \lasrel \firetwo\isub\var\valptwo}{\fire \lasrel \firetwo & \valp \lasrel \valptwo}} \]
			and by \reflemma{lasrel-stable-with-lists-subst} we get
			\[ \tmronep = \sctxp{\tmfour\esub\vartwo{\fire\isub\var\valp}} \lasrel \sctxtwop{\tmfourp\esub\vartwo{\firetwo\isub\var\valptwo}} = \tmrtwop\]
			hence the result $\ntm_\tmrone\isub\var\valp \tovscp \tmronep$, $\ntm_\tmrtwo\isub\var\valptwo \tovscp \tmrtwop$ and $\tmronep \lasrel \tmrtwop$.
			
			
			\item \emph {$\ntm_\tmrone = \fire\esub\vartwo{\isctxp\var}$} by $\ntm_\tmrone \lasrelnaf \ntm_\tmrtwo$ we have $\ntm_\tmrtwo = \firetwo\esub\vartwo{\isctxtwop\var}$ with $\isctxp\var \lasrel \isctxtwop\var$ and $\fire \lasrel \firetwo$. 
			
			Set $\sctx \defeq \isctx\isub\var\valp$ and $\sctxtwo \defeq \isctxtwo\isub\var\valptwo$.
			
			$\ntm_\tmrtwo\isub\var\valptwo = \firetwo\isub\var\valptwo\esub\vartwo{\sctxtwop\valptwo}\tovscp \sctxtwop{{\firetwo\isub\var\valptwo}\isub\vartwo\valptwo} = \tmrtwop$
			
			We also have $\ntm_\tmrone\isub\var\valp \tovscp \sctxp{{\fire\isub\var\valp}\isub\vartwo\valp} = \tmronep$ and we can build the following derivation:
			\[ \infer[\scsub]{{{\fire\isub\var\valp}\isub\vartwo\valp} \lasrel {{\firetwo\isub\var\valptwo}\isub\vartwo\valptwo}}{\infer{\fire\isub\var\valp \lasrel \firetwo\isub\var\valptwo}{\fire \lasrel \firetwo & \valp \lasrel \valptwo} & \valp \lasrel \valptwo} \]
			and by \reflemma{lasrel-stable-with-lists-subst} we get
			\[ \tmronep = \sctxp{{\fire\isub\var\valp}\isub\vartwo\valp} \lasrel \sctxtwop{{\firetwo\isub\var\valptwo}\isub\vartwo\valptwo} = \tmrtwop\]
			
			hence the result $\ntm_\tmrone\isub\var\valp \tovscp \tmronep$, $\ntm_\tmrtwo\isub\var\valptwo \tovscp \tmrtwop$ and $\tmronep \lasrel \tmrtwop$.
			
		\end{itemize}
		\item $\evctx = \tmtwo\evctxONE$ ($\tmtwo = \itm$ because $\ntm_\tmrone$ is normal) 	then $\ntm_\tmrone = \itm\evctxONEp\tmthree$ (where $\tmthree = \isctxp\var\firetwo$ or $\tmthree = \firetwo\esub\vartwo{\isctxp\var}$).
		Then by $\ntm_\tmrone \lasrelnaf \ntm_\tmrtwo$, $\ntm_\tmrtwo = \itmtwo\ntmONE$ with $\itm \lasrel \itmtwo$ and $\evctxONEp\tmthree \lasrel \ntmONE$.
		$\evctxONEp\tmthree\isub\var\valp \tovscp \tm$ by hypothesis ($\ntm_\tmrone\isub\var\valp \tovscp \tmronep$ is in this case $\itm\isub\var\valp\evctxONEp\tmthree\isub\var\valp \tovscp \tm\fire\isub\var\valp$) and $\evctxONEp\tmthree \lasrelnaf \ntmONE$ (normal forms, apply lemma \ref{l:lasrelnaf-normal-forms-lasrel-right-to-left}), hence by \ih $\ntmONE\isub\var\valptwo \tovscp \tmp$ with $\tm \lasrel \tmp$
		\[\infer{\tmronep=\itm\isub\var\valp\tm \lasrel \itmtwo\isub\var\valptwo\tmp = \tmrtwop}{ \infer{\itm\isub\var\valp \lasrel \itmtwo\isub\var\valptwo}{\itm \lasrel \itmtwo & \valp \lasrel \valptwo} & \tm \lasrel \tmp}\]
		
		hence $\ntm_\tmrone\isub\var\valp \tovscp \tmronep$, $\ntm_\tmrtwo\isub\var\valptwo \tovscp \tmrtwop$ and $\tmronep \lasrel \tmrtwop$.
		
		\item $\evctx = \evctxONE\tmtwo$ ($\tmtwo = \fire$ because $\ntm_\tmrone$ is normal) then $\ntm_\tmrone = \evctxONEp\tmthree\fire$ (where $\tmthree = \isctxp\var\firetwo$ or $\tmthree = \firetwo\esub\vartwo{\isctxp\var}$).
		Then by $\ntm_\tmrone \lasrelnaf \ntm_\tmrtwo$, $\ntm_\tmrtwo = \ntmONE\firep$ with $\fire \lasrel \firep$ and $\evctxONEp\tmthree \lasrel \ntmONE$.
		$\evctxONEp\tmthree\isub\var\valp \tovscp \tm$ by hypothesis ($\ntm_\tmrone\isub\var\valp \tovscp \tmronep$ is in this case $\evctxONEp\tmthree\isub\var\valp\fire\isub\var\valp \tovscp \tm\fire\isub\var\valp$) and $\evctxONEp\tmthree \lasrelnaf \ntmONE$ (normal forms, apply lemma \ref{l:lasrelnaf-normal-forms-lasrel-right-to-left}), hence by \ih $\ntmONE\isub\var\valptwo \tovscp \tmp$ with $\tm \lasrel \tmp$
		\[\infer{\tmronep=\tm\fire\isub\var\valp \lasrel \tmp\firep\isub\var\valptwo = \tmrtwop}{\tm \lasrel \tmp & \infer{\fire\isub\var\valp \lasrel \firep\isub\var\valptwo}{\fire \lasrel \firep & \valp \lasrel \valptwo}}\]
		
		hence $\ntm_\tmrone\isub\var\valp \tovscp \tmronep$, $\ntm_\tmrtwo\isub\var\valptwo \tovscp \tmrtwop$ and $\tmronep \lasrel \tmrtwop$.
		
		
		\item $\evctx = \tmtwo\esub\varthree\evctxONE$ ($\tmtwo = \fire$ because $\ntm_\tmrone$ is normal) then $\ntm_\tmrone = \fire\esub\varthree{\evctxONEp\tmthree}$ (where $\tmthree = \isctxp\var\firetwo$ or $\tmthree = \firetwo\esub\vartwo{\isctxp\var}$).
		Then by $\ntm_\tmrone \lasrelnaf \ntm_\tmrtwo$, $\ntm_\tmrtwo = \firep\esub\varthree\ntmONE$ with $\fire \lasrel \firep$ and $\evctxONEp\tmthree \lasrel \ntmONE$.
		$\evctxONEp\tmthree\isub\var\valp \tovscp \tm$ by hypothesis ($\ntm_\tmrone\isub\var\valp \tovscp \tmronep$ is in this case \\ $\fire\isub\var\valp\esub\varthree{\evctxONEp\tmthree\isub\var\valp}\tovscp\fire\isub\var\valp\esub\varthree{\tm}$) and $\evctxONEp\tmthree \lasrelnaf \ntmONE$ (normal forms, apply lemma \ref{l:lasrelnaf-normal-forms-lasrel-right-to-left}), hence by \ih $\ntmONE\isub\var\valptwo \tovscp \tmp$ with $\tm \lasrel \tmp$
		\[\infer{\tmronep=\fire\isub\var\valp\esub\varthree\tm \lasrel \firep\isub\var\valptwo\esub\varthree\tmp = \tmrtwop}{ \infer{\fire\isub\var\valp \lasrel \firep\isub\var\valptwo}{\fire \lasrel \firep & \valp \lasrel \valptwo} & \tm \lasrel \tmp}\]
		
		hence $\ntm_\tmrone\isub\var\valp \tovscp \tmronep$, $\ntm_\tmrtwo\isub\var\valptwo \tovscp \tmrtwop$ and $\tmronep \lasrel \tmrtwop$.
		
		\item $\evctx = \evctxONE\esub\varthree\tmtwo$ ($\tmtwo = \itm$ because $\ntm_\tmrone$ is normal) then $\ntm_\tmrone = \evctxONEp\tmthree\esub\varthree{\itm}$ (where $\tmthree = \isctxp\var\firetwo$ or $\tmthree = \firetwo\esub\vartwo{\isctxp\var}$).
		Then by $\ntm_\tmrone \lasrelnaf \ntm_\tmrtwo$, $\ntm_\tmrtwo = \ntmONE\esub\varthree\itmtwo$ with $\itm \lasrel \itmtwo$ and $\evctxONEp\tmthree \lasrel \ntmONE$.
		$\evctxONEp\tmthree\isub\var\valp \tovscp \tm$ by hypothesis ($\ntm_\tmrone\isub\var\valp \tovscp \tmronep$ is in this case\\ ${\evctxONEp\tmthree\isub\var\valp}\esub\varthree{\itm\isub\var\valp} \tovscp \fire\isub\var\valp\esub\varthree{\tm}$) and $\evctxONEp\tmthree \lasrelnaf \ntmONE$ (normal forms, apply lemma \ref{l:lasrelnaf-normal-forms-lasrel-right-to-left}), hence by \ih $\ntmONE\isub\var\valptwo \tovscp \tmp$ with $\tm \lasrel \tmp$
		\[\infer{\tmronep=\tm\esub\varthree{\itm\isub\var\valp}\lasrel \tmp\esub\varthree{\itmtwo\isub\var\valptwo} = \tmrtwop}{ \tm \lasrel \tmp &  \infer{\itm\isub\var\valp \lasrel \itmtwo\isub\var\valptwo}{\itm \lasrel \itmtwo & \valp \lasrel \valptwo}}\]
		
		hence $\ntm_\tmrone\isub\var\valp \tovscp \tmronep$, $\ntm_\tmrtwo\isub\var\valptwo \tovscp \tmrtwop$ and $\tmronep \lasrel \tmrtwop$.\qedhere
	\end{itemize}
\end{proof}


\subsection{Proof of Proposition \ref{prop:main-lemma_vsc}}
This subsection contains the main statement and its proof, relying on all the lemmas we have proved so far in this section.


\gettoappendix{prop:main-lemma_vsc}

\begin{proof}[Proof of (1)]
	By induction on $(k,d)$ where $d$ is the size of the derivation of $\tmrone \lasrel \tmrtwo$.
	
	% !TEX root = main.tex

%%By induction on $(k,d)$ where $d=$ the size of the derivation of $\tmrone \lasrel \tmrtwo$.

By case analysis on the last rule of the derivation $\tmrone\lasrel \tmrtwo$.

The first 3 cases are identical to the last proof.
\begin{enumerate}
	\item \emph{Lifting}:
	\[ \infer[(\sclift) ]{\tmrone \lasrel \tmrtwo} {\tmrone \rel \tmrtwo}\text{ and }\tmrone\bsvscp k \ntm\]
	Since $\relsym$ is a \naf simulation, we have $\tmrone\relnaf\tmrtwo$ and $\tmrtwo \bsvscps \ntmtwo$ for some $\ntmtwo$ such that $\ntm\relnaf\ntmtwo$. Hence  $\ntm\lasrelnaf\ntmtwo$ by \reflemma{relnaf-to-lasrelnaf-on-normal} .
	
	%%%%%%%%%%%%
	\item \emph{Variables}:
	\[\infer[(\scvar) ]{\var \lasrel \var}	{} \text{ and } \var\bsvscp 0 \var\]
	
	hence the result ($\var\bsvscp 0 \var$) and $\var\lasrelnaf\var$ by definition of $\lasrelnafsym$.
	
	%%%%%%%%%%%%
	\item \emph{Abstraction}:
	\[\infer[(\scabs) ]{\la\var\tmrone \lasrel \la\var\tmrtwo} {\tmrone \lasrel \tmrtwo} \text{ and } \la\var\tmrone \bsvscp 0 \la\var\tmrone \]
	
	hence the result ($\la\var\tmrtwo \bsvscp 0 \la\var\tmrtwo$) and $\la\var\tmrone \lasrelnaf \la\var\tmrtwo$ by definition of $\lasrelnafsym$.
	%%%%%%%%%%%%
	\item \emph{Application}:
	\[ \infer[(sc.app) ] 
	{\tmrone\tmrthree  \lasrel  \tmrtwo\tmrfour} {\tmrone  \lasrel \tmrtwo & \tmrthree \lasrel \tmrfour } \text{ and }\tmrone\tmrthree \bsvscp k \ntm \]
	
	then, by case analysis on the last rule of the big-step derivation,
	\begin{enumerate}
		\item \emph{Applied inert}: 
		\[\infer{\tmrone\tmrthree \bsvscp {k+h} \itm\ntm}{
			\tmrone \bsvscp k \itm
			&
			\tmrthree \bsvscp h \ntm
		}\]
		
		by inductive hypothesis ($d$ strictly decreasing, first component not increasing) we obtain $\tmrtwo \bsvscps \itmtwo$ and $\tmrfour \bsvscps \ntmtwo$ with $\itm\lasrelnaf\itmtwo,~\ntm \lasrelnaf \ntmtwo$. In particular, by \reflemma{lasrelnaf-normal-forms-lasrel-left-to-right}, $\itm\lasrel\itmtwo,~\ntm \lasrel \ntmtwo$. Then:
		\[\infer{\tmrtwo\tmrfour \bsvscps \itmtwo\ntmtwo}{
			\tmrtwo \bsvscps  \itmtwo
			&
			\tmrfour \bsvscps \ntmtwo
		}\]
		and $\itm\ntm \lasrelnaf \itmtwo\ntmtwo$ by definition of $\lasrelnaf$ and $\itm\lasrel\itmtwo,~\ntm \lasrel \ntmtwo$.
		
%		\item \emph{Substitution of an inert}:
%		not applicable.
		
		
		\item \emph{$m$ step}:
		\[\infer{\tmrone\tmrthree \bsvscp {k+i+1} \isctxp\ntm}{
			\tmrone \bsvscp k \isctxp{\la\var\tmronep}
			&
			{\tmronep\esub\var\tmrthree} \bsvscp i \ntm
		}\]
		
		
		
		then by inductive hypothesis ($d$ strictly decreasing, first component non increasing) on $\tmrone$ we get
		$\tmrtwo\bsvscps \ntm_\tmrtwo$ with $\isctxp{\la\var\tmronep} \lasrelnaf \ntm_\tmrtwo$ ($\ntm_\tmrtwo = \isctxtwop{\la\var\tmrtwop}$ by  \reflemma{abstraction-inerts-stable-lasrelnaf}) \ie $\isctxp{\la\var\tmronep} \lasrelnaf \isctxtwop{\la\var\tmrtwop}$.
		
		
		By \reflemma{lasrelnaf-values-isctx-decomposition} we get $\isctxp{\var} \lasrelnaf \isctxtwop{\var}$ and $\la\var\tmronep\lasrelnaf \la\var\tmrtwop$ then $\tmronep\lasrel\tmrtwop$ by case (\naf 3)).
		
		Then:
		\[\infer{\tmronep\esub\var\tmrthree \lasrel \tmrtwop\esub\var\tmrfour}{\tmronep\lasrel\tmrtwop&\tmrthree\lasrel\tmrfour}\]
		
		since $\tmronep\esub\var\tmrthree \bsvscp i \ntm$ with $i < k+i+1$ we can apply the inductive hypothesis on the first component for $\tmronep\esub\var\tmrthree$ obtaining $\tmrtwop\esub\var\tmrfour \bsvscps  \ntmtwo$ for some $\ntmtwo$ such that $\ntm\lasrelnaf\ntmtwo$. Since $\isctxp{\var} \lasrelnaf \isctxtwop{\var}$ and $\ntm\lasrelnaf\ntmtwo$, by \reflemma{lasrelnaf-normal-forms-isctx-decomposition}, we get $\isctxp{\ntm} \lasrelnaf \isctxtwop{\ntmtwo}$.
		Last, note that $\tmrtwo\tmrfour\bsvscps\isctxtwop\ntmtwo$ by 
		\[\infer{\tmrtwo\tmrfour \bsvscps \isctxtwop\ntmtwo}{
			\tmrtwo \bsvscps \isctxtwop{\la\var\tmrtwop}
			&
			\tmrtwop\esub\var\tmrfour \bsvscps \ntmtwo
		}\]
		
%		
%		\item \emph{$e$ step}:
%		not applicable.
	\end{enumerate}
	
	\item \emph{Explicit Substitution}: 
	\[ \infer[(sc.esubst) ]{\tmrone\esub\var{\tmrthree} \lasrel \tmrtwo\esub\var{\tmrfour}} {\tmrone \lasrel \tmrtwo & \tmrthree \lasrel \tmrfour }\text{ and }\tmrone\esub\var{\tmrthree} \bsvscp k \ntm \]
	
	\begin{enumerate}
%		\item \emph{Applied inert}: not applicable.
		\item \emph{Substitution of an inert}:
		\[ \infer{\tmrone\esub\var\tmrthree \bsvscp {k+h} \ntm\esub\var\itm}{
			\tmrone \bsvscp k \ntm
			&
			\tmrthree \bsvscp h \itm
		} \]
		
		by inductive hypothesis ($d$ strictly decreasing, first component not increasing) we obtain $\tmrtwo \bsvscps \ntmtwo$ and $\tmrfour \bsvscps \itmtwo$ with $\itm\lasrelnaf\itmtwo,~\ntm \lasrelnaf \ntmtwo$. In particular, by \reflemma{lasrelnaf-normal-forms-lasrel-left-to-right}, $\itm\lasrel\itmtwo,~\ntm \lasrel \ntmtwo$. Then:
		\[\infer{\tmrtwo\esub\var\tmrfour \bsvscps \ntmtwo\esub\var\itmtwo}{
			\tmrtwo \bsvscps  \ntmtwo
			&
			\tmrfour \bsvscps \itmtwo
		}\]
		and $\ntm\esub\var\itm \lasrelnaf \ntmtwo\esub\var\itmtwo$ by definition of $\lasrelnafsym$.
		
%		\item \emph{m step}: not applicable.
		\item \emph{e step}: 		\[ \infer{\tmrone\esub\var{\tmrthree} \bsvscp {k+i+1} \isctxp\ntm}{
			\tmrthree \bsvscp k \isctxp{\la\vartwo\tmrthreep}
			&
			{\tmrone\isub\var{\la\vartwo\tmrthreep}} \bsvscp i \ntm
		} \]
		
		then by inductive hypothesis ($d$ strictly decreasing, first component non increasing) on $\tmrone$ and $\tmrthree$ we get
		$\tmrfour\bsvscps \ntm_\tmrfour$ with $\isctxp{\la\vartwo\tmrthreep} \lasrelnaf \ntm_\tmrfour$($\ntm_\tmrfour = \isctxtwop{\la\vartwo\tmrfourp}$ by  \reflemma{abstraction-inerts-stable-lasrelnaf}) \ie $\isctxp{\la\vartwo\tmrthreep} \lasrelnaf \isctxtwop{\la\vartwo\tmrfourp}$.
		
		By \reflemma{lasrelnaf-values-isctx-decomposition} we get $\isctxp{\var} \lasrelnaf \isctxtwop{\var}$ and $\la\vartwo\tmrthreep\lasrelnaf\la\vartwo\tmrfourp$ and in particular\\ $\la\vartwo\tmrthreep\lasrel\la\vartwo\tmrfourp$ by \ref{l:lasrelnaf-normal-forms-lasrel-left-to-right}.
		
		Then:
		\[\infer{\tmrone\isub\var{\la\vartwo\tmrthreep} \lasrel \tmrtwo\isub\var{\la\vartwo\tmrfourp}}{\tmrone\lasrel\tmrtwo&\la\vartwo\tmrthreep\lasrel\la\vartwo\tmrfourp}\]
		
		since $\tmrone\isub\var{\la\vartwo\tmrthreep} \bsvscp i \ntm$ with $i < k+i+1$ we can apply the inductive hypothesis on the first component for $\tmrone\isub\var{\la\vartwo\tmrthreep}$ obtaining $\tmrtwo\isub\var{\la\vartwo\tmrfourp} \bsvscps  \ntmtwo$ for some $\ntmtwo$ such that $\ntm\lasrelnaf\ntmtwo$. 
		Since $\isctxp{\var} \lasrelnaf \isctxtwop{\var}$ and $\ntm\lasrelnaf\ntmtwo$, by \reflemma{lasrelnaf-normal-forms-isctx-decomposition}, we get $\isctxp{\ntm} \lasrelnaf \isctxtwop{\ntmtwo}$.
		
		Last, note that $\tmrtwo\tmrfour\bsvscps\isctxtwop\ntmtwo$ by 
		\[\infer{\tmrtwo\tmrfour \bsvscps \isctxtwop\ntmtwo}{
			\tmrfour \bsvscps \isctxtwop{\la\vartwo\tmrfourp}
			&
			\tmrtwo\isub\var{\la\vartwo\tmrfourp} \bsvscps \ntmtwo
		}\]
		
	\end{enumerate}
	
	
	%%%%%%%%%%%%%
	\item \emph{Meta-level Substitution}: 
	\[ \infer[(sc.subst) ]{\tmrone\isub\var{\valp} \lasrel \tmrtwo\isub\var{\valptwo}} {\tmrone \lasrel \tmrtwo & \valp \lasrel \valptwo }\text{ and }\tmrone\isub\var{\valp} \bsvscp k \ntm \]
	then by applying Big-step substitutivity (\reflemma{splitting_vsc}), we obtain $\tmrone \bsvscp {k_1} \ntm_\tmrone$ and $\ntm_\tmrone\isub\var{\valp} \bsvscp {k_2} \ntm$ with $k=k_1+k_2$. Hence by inductive hypothesis ($d$ strictly decreasing, first component non increasing) $\tmrtwo\bsvscps\ntm_\tmrtwo$ and $\ntm_\tmrone \lasrelnaf \ntm_\tmrtwo$. In particular, by \reflemma{lasrelnaf-normal-forms-lasrel-left-to-right}, $\ntm_\tmrone \lasrel \ntm_\tmrtwo$.
	We then have: 
	\[\infer{\ntm_\tmrone\isub\var{\valp} \lasrel \ntm_\tmrtwo\isub\var{\valptwo}} {\ntm_\tmrone \lasrel \ntm_\tmrtwo & \valp \lasrel \valptwo }\]
	
	%	Note that by \ih applied to $\valp \lasrel \valptwo $ and $\valp\bsvscps \valp$ (first component not increasing, second strictly decrasing) we obtain $\valp \lasrelnaf \valptwo$.
	
	Two cases.
	\begin{enumerate}
		
		
		\item \emph{$\tmrone$ is not normal}, that is, $k_1>0$ and $k_2<k$. Then by applying the induction hypothesis to $k_2$ (first component)  and $\ntm_\tmrone\isub\var{\valp}$ we obtain $\ntm_\tmrtwo\isub\var{\valptwo} \bsvscps \ntmtwo$ with $\ntm\lasrelnaf\ntmtwo$. We conclude using substitutivity of $\tovscp$ (\ref{l:stability_vsc}) that $\tmrtwo\isub\var{\valptwo} \tovscp^* \ntm_\tmrtwo\isub\var{\valptwo} \tovscp^* \ntmtwo$, hence via the equivalence between big and small steps (\ref{l:ss-bs-equivalence_vsc}), $\tmrtwo\isub\var{\valptwo} \bsvscps \ntmtwo$.
		
		
		\item \emph{$\tmrone$ is normal}, that is, $k_1=0$ and $k_2=k$. Then $\tmrone=\ntm_\tmrone$. Two sub-cases:
		\begin{itemize}
			\item \emph{$\tmrone\isub\var{\valp} = \ntm_\tmrone\isub\var{\valp}$ is also normal}
			
			
			Since we know that $\ntm_\tmrone \lasrelnaf \ntm_\tmrtwo$ and $\valp \lasrel \valptwo $, we can apply \reflemma{lasrelnaf-on-normal-subs_vsc}, and obtain that $\ntm_\tmrtwo\isub\var{\valptwo}$ is $\tovscp$-normal and by \reflemma{lasrelnaf-normal-forms-substitutive}, $\tmrone\isub\var{\valp} \lasrelnaf \ntm_\tmrtwo\isub\var{\valptwo}$. It is only left to show that $\tmrtwo\isub\var{\valptwo} \bsvscps \ntm_\tmrtwo\isub\var{\valptwo}$, which follows from $\tmrtwo\bsvscps \ntm_\tmrtwo$, substitutivity of $\tovscp$ (\reflemma{stability_vsc}) and the fact that $\ntm_\tmrtwo\isub\var{\valptwo}$ is $\tovscp$-normal (and via \reflemma{ss-bs-equivalence_vsc}).
			
			\item   \emph{$\tmrone\isub\var{\valp} = \ntm_\tmrone\isub\var{\valp}$ is not normal}
			
			hence $\ntm_\tmrone\isub\var{\valp} \tovscp \tmronep \tovscp ^ {k-1} \ntm$ (the reduction is diamond, all reductions are of the same length, we pick any first step possible). Then by \reflemma{lasrelnaf-not-normal-subs} with $\ntm_\tmrone \lasrelnaf \ntm_\tmrtwo$, $\ntm_\tmrtwo\isub\var{\valptwo} \tovscp \tmrtwop$ with $\tmronep \lasrel \tmrtwop$.
			
			We can apply the inductive hypothesis to $\tmronep$ (first component is decreasing, as $k-1<k$) and we obtain $\tmrtwop \bsvscps \ntmtwo$ with $\ntm\lasrelnaf\ntmtwo$.
			The statement is then proved, since (using \reflemma{stability_vsc})
			$$\tmrtwo\isub\var{\valptwo} \tovscp^* \ntm_\tmrtwo\isub\var{\valptwo} \tovscp \tmrtwop \tovscp^* \ntmtwo$$ that is, $\tmrtwo\isub\var{\valptwo} \bsvscps \ntmtwo$ by \reflemma{ss-bs-equivalence_vsc}.\qedhere
			
		\end{itemize}
	\end{enumerate}
	
\end{enumerate}

\end{proof}

% !TEX root = main.tex
\section{Proofs from \refsect{benchmarks-vsc} (Equational Benchmarks and the Value Substitution Calculus)}
In this section, we recall the proof that structural equivalence strongly commutes with $\tovsc$. The proof is identical to the one given in \cite{accattoli+paolini-vsc}, it is here presented for the interested reader to be able to see which fragments of structural equivalence independently strongly commutes.


\gettoappendix{prop:strong-bisimulation}
\newcommand{\eqz}{\equiv_0}
\begin{proof}
Define $\eqz$ as the context closure of $\equivsone\cup \equivexsthree\cup \equivass \cup \equivcom$.
 We have $\streq=\eqz^*$. We prove that:
\begin{equation}
\mbox{if $t_0\eqz t_1 \Rew{a} s_1$ then there exists $w$ such that $t_0\Rew{a} w \streq s_1$}
\label{eq:bis}
\end{equation}
The statement then follows by induction on the reflexive and transitive closure of $\eqz$. Let us show that: the reflexive case is trivial and if $t_0\eqz t_0'\eqz^k t_1\Rew{a} s_1$ then by \ih\ exists $w$ such that  $t_0'\Rew{a} w \streq s_1$ and by (\ref{eq:bis}) there exists $w'$ such that  $t_0\Rew{a} w'\streq w\streq s_1$.\\

The proof of (\ref{eq:bis}) is by induction on $\eqz$. Actually, before to proceed with the proof one should first prove the following two easy substitutivity properties:
\begin{enumerate}
  \item \label{l:eqo-stability-two} If $t \eqz t'$ then   $t\isub{x}{u} \eqz t'\isub{x}{u}$.
  \item \label{l:eqo-stability-one} If $u \eqz u'$ then   $t\isub{x}{u} \streq t\isub{x}{u'}$.
  \end{enumerate}      
Used in the inductive cases for the ES. We omit their proofs, which are straightforward inductions.

\begin{itemize}
\item Base cases:
\begin{itemize}
 \item \emph{Commutativity}: let $t_0= t\esub\vartwo{u}\esub\var{s}\equivcom t\esub\var{s}\esub\vartwo{u}=t_1$ with $x\notin\fv{u}$ and $y\notin\fv{s}$. If $t_1\Rew{a} s_1$ because:
 \begin{itemize}
  \item $t\Rew{a}  t'$ then $t_0=t\esub\vartwo{u}\esub\var{s}\Rew{a}  t'\esub\vartwo{u}\esub\var{s}\equivcom t'\esub\var{s}\esub\vartwo{u}=s_1$.
  \item $u\Rew{a}  u'$ or $s\Rew{a}  s'$ then it is similar to the previous case.
  \item $s=\sctxp\val$ and $t\esub\var{\sctxp\val}\esub\vartwo{u}\toe \sctxp{t\isub\var\val}\esub\vartwo{u}=s_1$. Then:
  \[\begin{array}{llllll}
  t_0&\toe & \sctxp{t\esub\vartwo{u}\isub\var\val}\\
  &=& \sctxp{t\isub\var\val \esub\vartwo{u}} \\
    &\equivcom& \sctxp{t\isub\var\val} \esub\vartwo{u}&=&s_1
  \end{array}\]
  \item The case where $u=\sctxp\val$ and $t\esub\var{s}\esub\vartwo{\sctxp\val}\toe \sctxp{t\esub\var{s}\isub\vartwo\val}=s_1$is similar to the previous one.  \end{itemize}

\item \emph{Sigma 1}: let $t_0=t\esub\var{s} \tmtwo   \equivsone  (\tm\tmtwo)\esub\var{s}=t_1$ with $x\notin\fv{u}$. If $t_1\Rew{a} s_1$ because:

 \begin{itemize}
  \item $t\Rew{a}  t'$ then $t_0=t\esub\var{s} \tmtwo \Rew{a}  t'\esub\var{s} \tmtwo \equivsone (t' \tmtwo )\esub\var{s}=s_1$.
  \item $s\Rew{a}  s'$ or $u\Rew{a}  u'$ then it is similar to the previous case.
  \item $s=\sctxp\val$ and $t_1=(\tm\tmtwo)\esub\var{\sctxp\val}\toe \sctxp{(\tm\tmtwo)\isub\var\val}=s_1$.
Then:
  \[\begin{array}{llllll}
  t_0&=& t\esub\var{\sctxp\val} \tmtwo \\
  &\toe& \sctxp{t\isub\var\val} \tmtwo \\
    &\equivsone& \sctxp{t\isub\var\val  \tmtwo }\\
    &=& \sctxp{(\tm\tmtwo)\isub\var\val}&=&s_1
  \end{array}\]

  \item $t= \l y. t'$ and $t_1=((\l y. t') \tmtwo )\esub\var{s} \tom t'\esub\vartwo{u}\esub\var{s}$. Then:
  \[\begin{array}{llllll}
  t_0&=& (\l y. t')\esub\var{s} \tmtwo \\
  &\tom& t'\esub\vartwo{u}\esub\var{s}&=&s_1
  \end{array}\]
  \end{itemize}

Note that here it is reflexivity of $\streq$ which is used.

 \item The case symmetric to the previous one, \ie\ $t_0= (\tm\tmtwo)\esub\var{s}   \equivsone  t\esub\var{s} \tmtwo=t_1$ with $x\notin\fv{u}$, is proved analogously. It shall be so for all following cases, so we simply omit the symmetric cases.

\item \emph{Extended sigma 3}: let $t_0=\tm\tmtwo\esub\var{s}  \equivexsthree  (\tm\tmtwo)\esub\var{s}=t_1$ with $x\notin\fv{t}$. If $t_1\Rew{a} s_1$ because:

 \begin{itemize}
  \item $t\Rew{a}  t'$ then $t_0=\tm\tmtwo\esub\var{s}\Rew{a}  t' \tmtwo \esub\var{s} \equivexsthree (t' \tmtwo )\esub\var{s}=s_1$.
  \item $s\Rew{a}  s'$ or $u\Rew{a}  u'$ then it is similar to the previous case.
  \item $s=\sctxp\val$ and $t_1=(\tm\tmtwo)\esub\var{\sctxp\val}\toe \sctxp{(\tm\tmtwo)\isub\var\val}=s_1$. Then:
  \[\begin{array}{llllll}
  t_0&=& \tm\tmtwo\esub\var{\sctxp\val}\\
  &\toe & \tm\, \sctxp{\tmtwo\isub\var\val}\\
    &\equivexsthree & \sctxp{\tm\tmtwo\isub\var\val}\\
    &=& \sctxp{(\tm\tmtwo)\isub\var\val}&=&s_1
  \end{array}\]

  \item $t= \l y. t'$ and $t_1=((\l y. t') \tmtwo )\esub\var{s} \tom t'\esub\vartwo{u}\esub\var{s}$. Then:
  \[\begin{array}{llllll}
  t_0&=& (\l y. t') \tmtwo \esub\var{s}\\
  &\tom& t'\esub\vartwo{\tmtwo\esub\var{s}}\\
  &\equivass& t'\esub\vartwo{u}\esub\var{s}&=&s_1
  \end{array}\]
  \end{itemize}
 

\item \emph{Associativity of ES}: let $t_0=t\esub\vartwo{\tmtwo\esub\var{s}}  \equivass  t\esub\vartwo{u}\esub\var{s}=t_1$ with $x\notin\fv{t}$. If $t_1\Rew{a} s_1$ because:

\begin{itemize}
\item $t\Rew{a} t'$ then $t_0\Rew{a} t'\esub\vartwo{\tmtwo\esub\var{s}}\equivass t'\esub\var\tmtwo\esub\var{s}=s_1$.
\item $u\Rew{a} u'$ or $s\Rew{a} s'$ it is analogous to the previous case.
\item $s=\sctxp\val$ and $t_1 \toe \sctxp{t\esub\vartwo{u}\isub\var\val}=s_1$. Then
  \[\begin{array}{llllll}
  t_0&=& t\esub\vartwo{u\esub\var{\sctxp\val}}\\
  &\toe& t\esub\vartwo{\sctxp{u\isub\var\val}}\\
  &\equivass& \sctxp{t\esub\vartwo{u\isub\var\val}}\\
  &=& \sctxp{t\esub\vartwo{u}\isub\var\val}&=&s_1
  \end{array}\]

\item $u=\sctxp\val'$ and $t_1=\sctxp{t\esub\var{\sctxtwop\val}} \toe \sctxp{\sctxtwop{t\isub\var\val}}$. Then $t_0=t\esub\var{\sctxp{\sctxtwop\val}}\toe \sctxp{\sctxtwop{t\isub\var{\val}}}=s_1$. Note that here it is reflexivity of $\streq$ which is used.
\end{itemize}
 \end{itemize}


\item Inductive cases. We only show the interesting ones:
\begin{itemize}
%\item Lambda: $t_0=\la\var\tm\eqz \la\var\tm'=t_1\Rew{a} \la\var\tm''=s_1$. Then by \ih\ we get that there exists $w$ such that  $t\Rew{a} w\eqz t''$ and so $t_0\Rew{a} \l x.w\eqz s_1$.
\item Application: the only case where the reduction interact with the contextual closure is $t_0=\sctxp{\la\var\tm} \tmtwo  \eqz \sctxp{\la\var\tm'} \tmtwo  = t_1 \Rew{a} \sctxp{t'\esub\var\tmtwo}=s_1$. Then $t_0\Rew{a} \sctxp{t\esub\var\tmtwo} \eqz \sctxp{t'\esub\var\tmtwo}=s_1$. The variants  $t_0=\sctxp{\la\var\tm} \tmtwo  \eqz \sctxp{\la\var\tm} \tmtwo ' = t_1 \Rew{a} \sctxp{t\esub\var{\tmtwo'}}=s_1$ and $t_0=\sctxp{\la\var\tm} \tmtwo  \eqz \sctxtwop{\la\var\tm} \tmtwo  = t_1 \Rew{a} \sctxtwop{t\esub\var\tmtwo}=s_1$ are analogous. All other inductive cases for application are straightforward.

\item Explicit substitution. We only show the interesting cases. 
\begin{itemize}
\item $t_0=t\esub\var{\sctxp\val} \eqz t'\esub\var{\sctxp\val} = t_1 \Rew{a} \sctxp{t'\isub\var\val}=s_1$. Then by the first substitutivity property we obtain $t_0\Rew{a} \sctxp{t\isub\var\val} \eqz \sctxp{t'\isub\var\val}$.
\item $t_0=t\esub\var{\sctxp\val} \eqz t\esub\var{\sctxp\valtwo} = t_1 \Rew{a} \sctxp{t\isub\var\valtwo}=s_1$. Then by the second substitutivity property we obtain $t_0\Rew{a} \sctxp{t\isub\var\val}\streq \sctxp{t\isub\var\valtwo}$.\qedhere
\end{itemize}
\end{itemize}
\end{itemize}
%\qed
\end{proof}
% !TEX root = main.tex

\section{Proof of Compatibility for \nafex and \net Similarity}
\label{chapter:proof-compatibility-nafex}
The proof follows the same structure as in the case of \naf simulations. We prove the general statement for $\equivx$ a mirror.


\subsection{Lemmas concerning $\equivx$}


\begin{proposition}[$\opnafex\cdot\equivx\subseteq\opnafex$]
	\label{prop:relnafex-equivx-subseteq-relnafex}
	Let $\relsym$ be a relation on terms.
	If $\tm \relnafex \tmtwo$ and there exists $\tmtwop$ such that $\tmtwo \equivx \tmtwop$, then $\tm \relnafex \tmtwop$.
\end{proposition}

\begin{proof}
	By case analysis on $\tm \mlasrelnafex \tmtwo$.
\end{proof}


\begin{proposition}[$\equivx$ is a strong commutation wrto $\tovsc$]
	\label{prop:equivx-is-a-strong-bisimulation}
	If $\tm\equivx\tmtwo$ and $\tm \tovsc \tmp$ then $\tmtwo \tovsc \tmtwop$ and $\tmp\equivx\tmtwop$.
\end{proposition}


\begin{proposition}[$\equivx$ preserves normal forms]
	\label{prop:equivx-preserves-normal-forms}
	$\forall \tm,\tmtwo,~ \tm\equivx\tmtwo$ and $\tm$ is a normal form implies $\tmtwo$ is a normal form.
\end{proposition}

\begin{proof}
	This is derived from the fact that $\equivx$ is a strong commutation wrto $\tovsc$.
\end{proof}


\begin{proposition}[$\equivx$ is substitutive]
	\label{prop:equivx-is-substitutive}
	$\forall \tm,\tmtwo,\val,~ \tm\equivx\tmtwo$ implies $\tm\isub\var\val\equivx\tmtwo\isub\var\val$
\end{proposition}
\subsection{Compatibility proof}

A useful tool in the proof is substitutivity, with respect to small-step and big-step semantics, that is Proposition \ref{prop:substitutivity_vsce}.



\begin{proposition}[Substitutivity of $\tovsc$]
	\label{l:stability_vsce}
	$\tm\tovsc\tmp \Rightarrow \tm\isubst\val\var \tovsc \tmp\isubst\val\var$
\end{proposition}

\begin{proof}
	By induction on $\tm\tovsc\tmp$ (induction on contexts), using the fact that a value where a variable is substituted by a value is still a value.
\end{proof}

\begin{lemma}[Substitutivity of $\bsvscts$]
	\label{l:splitting_vsce}
	Forall $\tm,\val$,
	
	$\tm\isubst\val\var \bsvsct k \ntm \implies 
	\exists k',\ntmtwo$ s.t. $ \tm \bsvsct {k'} \ntmtwo$ and $\ntmtwo\isubst\val\var\bsvsct {k-k'} \ntm$
	
	
\end{lemma}

\begin{proof}
	Suppose $\tm\isubst\val\var \bsvsct k \ntm$, then $ \tm \bsvscts \ntmtwo$ because if it diverges then the first diverges as well by the stability of reduction by substitution {(\reflemma{stability_vsce})}.
	Let $k', \tm \bsvsct {k'} \ntmtwo$
	then $ \tm\isubst\val\var \tovsc^{k'}\equivsone \ntmtwo\isubst\val\var$ hence $\ntmtwo\isubst\val\var\bsvsct {k-k'} \ntm$ because the reduction $\tovsc$ is diamond (hence all reduction sequences have the same length).
\end{proof}


\subsection{Equivalence of $\mlasrelnafex$ and $\mlasrel$ on normal forms.}
As in toy's proof of compatibility, we need to show the main proof first on normal forms, then we will generalize to any term.

\begin{lemma}
	\label{l:lasrelnafex-normal-forms-lasrel-left-to-right}
	If $\fire\mlasrelnafex\firetwo$ then $\fire\mlasrel\firetwo$.
\end{lemma}
\begin{proof}
	By case analysis on $\fire =\valt \mid \var\fire\mid\itmapp\fire \mid \fire\esub\var\itm$ and using the ($\scequivx$) rule.
\end{proof}

%\begin{lemma}
%	\label{l:relnafex-to-lasrelnafex-on-normal}
%	If $\ntm\relnafex\ntmtwo$ then $\ntm\mlasrelnafex\ntmtwo$. 
%\end{lemma}
%
%\begin{proof}
%	Cases of $\ntm$.
%	\begin{itemize}
	%		\item \emph{Variable}, that is, $\ntm = \var$. Then $\ntmtwo = \var$ and $\var\mlasrelnafex\var$ by case (nafex-X 2). 
	%		\item \emph{Abstraction}, that is, $\ntm = \la\var\tm$. Then $\ntmtwo = \la\var\tmp$ with $\tm\rel \tmp$. Then $\tm\mlasrel \tmp$ by rule $(\sclift)$. Then $\la\var \tm \mlasrelnafex \la\var\tmp$ by case (nafex-X 3). 
	%		\item \emph{Applied normal form}, that is, $\ntm = \ntmONE\ntmTWO$. Then $\ntmtwo \equivx \ntmONEtwo\ntmTWOtwo$ with $\ntmONE \rel \ntmONEtwo$ and $\ntmTWO \rel \ntmTWOtwo$. By rule $(\sclift)$, $\ntmONE \mlasrel \ntmONEtwo$ and $\ntmTWO \mlasrel \ntmTWOtwo$. Then $\ntm =\ntmONE\ntmTWO \mlasrelnafex \ntmtwo$ by case (nafex 5) since $\ntmONEtwo\ntmTWOtwo \equivx \ntmtwo$.
	%		\item \emph{Substituted inert}, that is, $\ntm = \ntmONE\esub\var\itm$. Then $\ntmtwo \equivx \ntmONEtwo\esub\var\itmtwo$ with $\itm \rel \itmtwo$ and $\ntmONE \rel \ntmONEtwo$. By rule $(\sclift)$, $\itm \mlasrel \itmtwo$  and $\ntmONE \mlasrel \ntmONEtwo$. Then $\ntm = \ntmONE\esub\var\itm \mlasrelnafex \ntmtwo$ by case (nafex 6) since $ \ntmONEtwo\esub\var\itmtwo \equivx \ntmtwo$. 
	%	\end{itemize}
%\end{proof}
%
%\begin{lemma}
%	\label{l:relnafex-to-lasrel-on-normal}
%	If $\ntm\relnafex\ntmtwo$ then $\ntm\mlasrel\ntmtwo $.
%\end{lemma}
%
%\begin{proof} apply same reasoning as \ref{l:relnafex-to-lasrelnafex-on-normal}.
%	Cases of $\ntm$.
%	\begin{itemize}
	%		\item \emph{Variable}, that is, $\ntm = \var$. Then $\ntmtwo = \var$ and $\var\mlasrel\var$ by rule ($\scvar$).
	%		\item \emph{Abstraction}, that is, $\ntm = \la\var\tm$. Then $\ntmtwo = \la\var\tmp$ with $\tm\rel \tmp$. Then $\tm\mlasrel \tmp$ by rule $(\sclift)$. Then $\la\var \tm \mlasrel \la\var\tmp$ by rule ($\scabs$).
	%		\item \emph{Applied variable}, that is, $\ntm = \var\ntmONE$. Then $\ntmtwo = \var\ntmONEtwo$ with$\ntmONE \rel \ntmONEtwo$. By rule $(\sclift)$, $\ntmONE \mlasrel \ntmONEtwo$. Then $\ntm =\var\ntmONE \mlasrel \var\ntmONEtwo = \ntmtwo$ by rule ($\scapp$).
	%		\item \emph{Applied inert}, that is, $\ntm = \itm\ntmONE$. Then $\ntmtwo = \itmtwo\ntmONEtwo$ with $\itm \rel \itmtwo$ and $\ntmONE \rel \ntmONEtwo$. By rule $(\sclift)$, $\itm \mlasrel \itmtwo$  and $\ntmONE \mlasrel \ntmONEtwo$. Then $\ntm =\itm\ntmONE \mlasrel \itmtwo\ntmONEtwo = \ntmtwo$ by rule ($\scapp$).
	%		\item \emph{Substituted inert}, that is, $\ntm = \ntmONE\esub\var\itm$. Then $\ntmtwo = \ntmONEtwo\esub\var\itmtwo$ with $\itm \rel \itmtwo$ and $\ntmONE \rel \ntmONEtwo$. By rule $(\sclift)$, $\itm \mlasrel \itmtwo$  and $\ntmONE \mlasrel \ntmONEtwo$. Then $\ntm = \ntmONE\esub\var\itm \mlasrel \ntmONEtwo\esub\var\itmtwo = \ntmtwo$ by rule ($\scesub$).
	%	\end{itemize}
%\end{proof}

\begin{lemma}[Constrained Substitutivity of $\mlasrelnafex$ on normal forms]
	\label{l:lasrelnafex-normal-forms-substitutive}
	If $\ntmONE \mlasrelnafex \ntmTWO$, $\valof\tmrthree \mlasrelnafex \valof\tmrfour$ and $\ntmONE\isub\var{\valof\tmrthree}$ and $\ntmTWO\isub\var{\valof\tmrfour}$ are $\tovsce$-normal then $\ntmONE\isub\var{\valof\tmrthree} \mlasrelnafex \ntmTWO\isub\var{\valof\tmrfour}$.
\end{lemma}


\begin{proof}
	By case analysis on $\ntmONE$. Cases:
	\begin{itemize}
		\item $\ntmONE = \var$ and $\ntmTWO = \var$ then $\ntmONE\isub\var{\valof\tmrthree} = \valof\tmrthree \mlasrelnafex \valof\tmrfour = \ntmTWO\isub\var{\valof\tmrfour}$.
		
		\item $\ntmONE = \vartwo$ and $\ntmTWO = \vartwo$ then $\ntmONE\isub\var{\valof\tmrthree} =  \vartwo \mlasrelnafex \vartwo = \ntmTWO\isub\var{\valof\tmrfour}$.
		
		\item $\ntmONE = \la\vartwo\tm$ and $\ntmTWO = \la\vartwo\tmp$ with $\tm \mlasrel \tmp$
		we have \[\infer{\tm\isub\var{\valof\tmrthree} \mlasrel \tmp\isub\var{\valof\tmrfour}}{\tm \mlasrel \tmp & \valof\tmthree \mlasrel \valof\tmfour}\]
		hence by case (\nafex 3) $\ntmONE\isub\var{\valof\tmrthree} = \la\vartwo{\tm\isub\var{\valof\tmrthree}} \mlasrelnafex  \la\vartwo{\tmp\isub\var{\valof\tmrfour}} = \ntmTWO\isub\var{\valof\tmrfour}$.
		
		
		
		\item $\ntmONE = \ntmONEtwo\ntmONEthree$ and $\ntmTWO \equivx \ntmTWOtwo\ntmTWOthree$ with $\ntmONEtwo \mlasrel \ntmTWOtwo$ and $\ntmONEthree \mlasrel \ntmTWOthree$. 
		From $\ntmTWO \equivx \ntmTWOtwo\ntmTWOthree$, we deduct by substitutivity of $\equivx$ (\refprop{equivx-is-substitutive}) that $\ntmTWOtwo\isub\var{\valof\tmrfour}\ntmTWOthree\isub\var{\valof\tmrfour} \equivx \ntmTWO\isub\var{\valof\tmrfour}$.
		
		Since $\ntmONE\isub\var{\valof\tmrthree}$ and $\ntmTWO\isub\var{\valof\tmrfour}$ are $\tovsce$-normal, then by \refprop{equivx-preserves-normal-forms}, $\ntmONEtwo\isub\var{\valof\tmrthree}$, $\ntmTWOtwo\isub\var{\valof\tmrfour}$, $\ntmONEthree\isub\var{\valof\tmrthree}$ and $\ntmTWOthree\isub\var{\valof\tmrfour}$ all are $\tovsce$-normal as well and $\ntmONEtwo\isub\var{\valof\tmrthree}$, $\ntmTWOtwo\isub\var{\valof\tmrfour}$ are not almost-abstractions (\ie  $\not =\isctxp{\la\vartwo\tm}$ for any $\isctx$).
		
		To conclude that $\ntmONE\isub\var\val \mlasrelnafex \ntmTWO\isub\var\val$, what is only remaining is that\\ $\ntmONEtwo\isub\var\val \mlasrel \ntmTWOtwo\isub\var\valtwo$ and  $\ntmONEthree\isub\var\val \mlasrel \ntmTWOthree\isub\var\valtwo$.
		
		We derive easily these facts: ($\val\mlasrelnafex\valtwo$ implies $\val\mlasrel\valtwo$ by \reflemma{lasrelnafex-normal-forms-lasrel-left-to-right})
		\[ \infer[\scsub]{\ntmONEtwo\isub\var\val \mlasrel \ntmTWOtwo\isub\var\valtwo}{\ntmONEtwo \mlasrel \ntmTWOtwo & \val \mlasrel \valtwo} ~\text{and}~ \infer[\scsub]{\ntmONEthree\isub\var\val \mlasrel \ntmTWOthree\isub\var\valtwo}{\ntmONEthree \mlasrel \ntmTWOthree & \val \mlasrel \valtwo}\]
		
		
		
		\item $\ntmONE = \ntmONEtwo\esub\vartwo\itmONEtwo$ and $\ntmTWO {\equivx} \ntmTWOtwo\esub\vartwo\itmTWOtwo$ with $\itmONEtwo \mlasrel \itmTWOtwo$ and $\ntmONEtwo \mlasrel \ntmTWOtwo$. 
		
		The hypothesis that $\ntmONE\isub\var\val$ is normal is equivalent to $\itmONEtwo\isub\var\val$ is an inert and $\ntmONEtwo\isub\var\val$ is normal.
		
		From $\ntmTWO \equivx \ntmTWOtwo\esub\vartwo\itmTWOtwo$, we deduct {by substitutivity of $\equivx$} (\refprop{equivx-is-substitutive}) that $\ntmTWO\isub\var\valtwo \equivx (\ntmTWOtwo\isub\var\valtwo)\esub\vartwo{\itmTWOtwo\isub\var\valtwo}$.
		Since {$\equivx$ preserves normal forms}  (\refprop{equivx-preserves-normal-forms}), the hypothesis that $\ntmTWO\isub\var\valtwo$ is normal is equivalent to $\itmTWOtwo\isub\var\valtwo$ is an inert and $\ntmTWOtwo\isub\var\valtwo$ is normal.
		
		
		To conclude that $\ntmONE\isub\var\val \mlasrelnafex \ntmTWO\isub\var\val$, what is only remaining is that\\ $\itmONEtwo\isub\var\val \mlasrel \itmTWOtwo\isub\var\valtwo$ and  $\ntmONEtwo\isub\var\val \mlasrel \ntmTWOtwo\isub\var\valtwo$.
		
		We derive easily these facts: ($\val\mlasrelnafex\valtwo$ implies $\val\mlasrel\valtwo$ by \reflemma{lasrelnafex-normal-forms-lasrel-left-to-right})
		\[ \infer[\scsub]{\itmONEtwo\isub\var\val \mlasrel \itmTWOtwo\isub\var\valtwo}{\itmONEtwo \mlasrel \itmTWOtwo & \val \mlasrel \valtwo} ~\text{and}~ \infer[\scsub]{\ntmONEtwo\isub\var\val \mlasrel \ntmTWOtwo\isub\var\valtwo}{\ntmONEtwo \mlasrel \ntmTWOtwo & \val \mlasrel \valtwo}\]
		
	\end{itemize}
\end{proof}

\begin{lemma}
	\label{l:lasrelnafex-normal-forms-lasrel-right-to-left}
	If $\relsym$ is a \nafex simulation.
	If $\ntm\mlasrel\ntmtwo$ then $\ntm\mlasrelnafex\ntmtwo$.
\end{lemma}

\begin{proof}
	By induction on the derivation $\ntm \mlasrel \ntmtwo$. Cases of the last rule in the derivation of $\ntm\mlasrel\ntmtwo$:
	\begin{itemize}
		\item \emph {$\scvar$}\[ \infer[\scvar]{\var \mlasrel \var}{} \]
		then $\var \mlasrelnafex \var$ by case (\nafex 2).
		\item \emph {$\scabs$} \[ \infer[\scabs]{\ntm = \la\var\tm \mlasrel \la\var\tmp = \ntmtwo}{\tm \mlasrel \tmp} \]
		then $\ntm \mlasrelnafex \ntmtwo$ by case (\nafex 3) with $\tm \mlasrel \tmp$.
		\item \emph {$\sclift$} \[ \infer[\scabs]{\ntm \mlasrel \ntmtwo}{\ntm \rel \ntmtwo} \]
		then $\ntm \relnafex \ntmtwo$ since $\relsym$ is a \nafex simulation.
		By monotonicity of $\opnafex$, $\ntm \mlasrelnafex \ntmtwo$.
		
		\item \emph {$\scapp$} 
		\[ \infer[\scapp]{\ntm = \ntmONE\ntmTWO \mlasrel \ntmONEtwo\ntmTWOtwo = \ntmtwo}{\ntmONE \mlasrel \ntmONEtwo & \ntmTWO \mlasrel \ntmTWOtwo} \]
		
		
		then $\ntm \mlasrelnafex \ntmtwo$ by case (\nafex 4) with $\ntmONE \mlasrel \ntmONEtwo$ and $\ntmTWO \mlasrel \ntmTWOtwo$.
		
		
		
		\item \emph {$\scesub$} \[ \infer[\scesub]{\ntm = \ntmONE\esub\var\itmONE \mlasrel \ntmTWO\esub\var\itmTWO = \ntmtwo}{\ntmONE \mlasrel \ntmTWO & \itmONE \mlasrel \itmTWO} \] then $\ntm \mlasrelnafex \ntmtwo$
		by case (\nafex 5) with $\itmONE \mlasrel \itmTWO$ and $\ntmONE \mlasrel \ntmTWO$.
		\item \emph {$\scsub$} \[ \infer[\scsub]{\ntm = \ntmONE\isub\var\val \mlasrel \ntmTWO\isub\var\valtwo = \ntmtwo}{\ntmONE \mlasrel \ntmTWO & \val \mlasrel \valtwo} \]
		by \ih we have $\ntmONE \mlasrelnafex \ntmTWO$ and $\valof\tmrthree \mlasrelnafex \valof\tmrfour$. By \reflemma{lasrelnafex-normal-forms-substitutive}, $\ntmONE\isub\var{\valof\tmrthree} \mlasrelnafex \ntmTWO\isub\var{\valof\tmrfour}$.
		
		\item \emph {$\scequivx$} \[\infer[\scequivx]{\ntm \mlasrel \ntmtwo}{\ntm \mlasrel \ntmONE & \ntmONE \equivx \ntmtwo}\]
		
		By \ih, $\ntm \mlasrelnafex \ntmONE$ which means by \refprop{relnafex-equivx-subseteq-relnafex} $\ntm \mlasrelnafex \ntmtwo$ since $\ntmtwo \equivx \ntmONE$.\qedhere
	\end{itemize}
\end{proof}


\subsection{Bottom-up lemmas for a top-down defined simulation}



\begin{lemma}
	\label{l:lasrelnafex-inert-style-lists-induction}
	If $\isctxp\var \mlasrelnafex \isctxtwop\var$ (or $\isctxp\val \mlasrelnafex \isctxtwop\valtwo$) then either $\isctx,\isctxtwo =\ctxhole,\ctxhole$ or  (for $\tm = \var$ or $ \val$ and $\tmp=\var$ or $\val$) $\isctxp\tm = \isctxONEp\tm\esub\vartwo\itm$ and $\isctxtwop\tmp \equivx\isctxONEtwop\tmp\esub\vartwo\itmtwo$ with $\isctxONEp\var \mlasrelnafex \isctxONEtwop\var$ (or $\isctxONEp\val \mlasrelnafex \isctxONEtwop\valtwo$) and $\itm \mlasrelnafex \itmtwo$.
\end{lemma}

\begin{proof}
	By contradiction and case exhaustion, these are the only possibilities for\\ $\isctxp\var \mlasrelnafex \isctxtwop\var$ (or $\isctxp\val \mlasrelnafex \isctxtwop\valtwo$).
	\begin{itemize}
		\item If only one of the lists is empty: $\var \mlasrelnafex \itm\esub\vartwo\itmtwo$ (or $\var \mlasrelnafex \ntm\esub\vartwo\itmtwo$) is not possible given the definition of nafex. 
		\item If $\isctx= \isctxONE\esub\vartwo\itm$ and $\isctxtwo \equivx\isctxONEtwo\esub\varthree\itmtwo$, we again have $\neg (\ntm\esub\vartwo\itm \mlasrelnafex \ntmtwo\esub\varthree\itmtwo)$.\qedhere
	\end{itemize}
\end{proof}

%
%\begin{lemma}\adr{this lemma is false for now, we will do what we need}
%	\label{l:lasrelnafex-inert-app-lists-induction}
%	If $\isctxp\itmapp \mlasrelnafex \isctxtwop\itmapptwo$ then either $\isctx,\isctxtwo =\ctxhole,\ctxhole$ or  $\isctxp\itmapp = \isctxONEp\itmapp\esub\vartwo\itm$ and $\isctxtwop\itmapptwo \equivx\isctxONEtwop\itmapptwo\esub\vartwo\itmtwo$ with $\isctxONEp\itmapp \mlasrelnafex \isctxONEtwop\itmapptwo$ and $\itm \mlasrelnafex \itmtwo$.
%\end{lemma}
%
%\begin{proof}
%	By contradiction and case exhaustion, these are the only possibilities for $\isctxp\var \mlasrelnafex \isctxtwop\var$ (or $\isctxp\val \mlasrelnafex \isctxtwop\valtwo$).
%	\begin{itemize}
	%		\item If only one of the lists is empty: $\var \mlasrelnafex \itm\esub\vartwo\itmtwo$ (or $\var \mlasrelnafex \ntm\esub\vartwo\itmtwo$) is not possible given the definition of nafex. 
	%		\item If $\isctx,\isctxtwo = \isctxONE\esub\vartwo\itm,\isctxONEtwo\esub\varthree\itmtwo$, we again have $\neg (\ntm\esub\vartwo\itm \mlasrelnafex \ntmtwo\esub\varthree\itmtwo)$.
	%	\end{itemize}
%\end{proof}



This \reflemma{lasrelnafex-inert-style-lists-induction} gives us an "induction principle" on $\isctx,\isctxtwo$ when $\isctxp\var \mlasrelnafex \isctxtwop\var$ (or $\isctxp\val \mlasrelnafex \isctxtwop\valtwo$.
% \adr{or $\isctxp\itmapp \mlasrelnafex \isctxtwop\itmapptwo$}). \adr{blue version only needed in the demonstration of 3.13 ie the following lemma}


%\begin{lemma}
%	$\ntm,\ntmtwo, \ntmONE, \ntmONEtwo$ normal forms for $\to$ (not $\equivsone$).
%	Suppose $\ntm \to_{\sigma_1} \ntmONE$ and $\ntmtwo \to_{\sigma_1}  \ntmONEtwo$, then
%	$\ntm \mlasrel \ntmtwo \iff \ntmONE \mlasrel \ntmONEtwo$.
%\end{lemma}
%
%\begin{lemma}
%	\label{l:lasrel-inerts-separating-applicative-inerts-from-substitutions}
%	Suppose $\ntm = \isctxp{\itmapp}$ and $\ntmtwo = \isctxtwop{\itmapptwo}$. Then $\ntm \mlasrel \ntmtwo \iff \itmapp \mlasrel \itmapptwo$ and $\isctxp{\var} \mlasrel \isctxtwop{\var}$ ($\var$ is fresh).
%\end{lemma}
%
%\begin{proof}
%	By induction on $\isctx,\isctxtwo$.
%	\begin{itemize}
	%		\item \emph {$\isctx,\isctxtwo = \ctxhole,\ctxhole$}, then the result is obvious ($\var\mlasrel\var$ forall choice of $\var$).
	%		\item \emph {$\isctx,\isctxtwo = \isctxONE\esub\vartwo{\itm},\isctxONEtwo\esub\vartwo{\itmtwo}$}
	%		By \ref{l:lasrelnafex-normal-forms-lasrel-right-to-left} and \ref{l:lasrelnafex-normal-forms-lasrel-left-to-right}, $\ntm = \isctxONEp{\itmapp}\esub\vartwo\itmONE \mlasrel \isctxONEtwop{\itmapptwo}\esub\vartwo\itmONEtwo = \ntmtwo \iff \ntm \mlasrelnafex \ntmtwo$. The latter can only be satisfied by case (nafex 6) hence $\ntm \mlasrelnafex \ntmtwo \iff \itm \mlasrel \itmtwo$ and $\isctxONEp{\itmapp} \mlasrel \isctxONEtwop{\itmapptwo}$.
	%		By \ih, $ \isctxONEp{\itmapp} \mlasrel \isctxONEtwop{\itmapptwo} \iff \itmapp \mlasrel \itmapptwo$ and $\isctxONEp{\var} \mlasrel \isctxONEtwop{\var}$.
	%		
	%		Moreover, By \ref{l:lasrelnafex-normal-forms-lasrel-right-to-left} and \ref{l:lasrelnafex-normal-forms-lasrel-left-to-right}, $\isctxONEp{\var}\esub\vartwo\itm \mlasrel \isctxONEtwop{\var}\esub\vartwo\itmtwo \iff \isctxONEp{\var}\esub\vartwo\itm \mlasrelnafex \isctxONEtwop{\var}\esub\vartwo\itmtwo$ \ie by case (nafex 6), $\iff \isctxONEp{\var} \mlasrel \isctxONEtwop{\var}$ and $\itm \mlasrel \itmtwo$.
	%		
	%	\end{itemize}
%\end{proof}



\begin{lemma}
	\label{l:lasrelnafex-values-isctx-decomposition}
	$\isctxp\val \mlasrelnafex \isctxtwop\valtwo \iff \val \mlasrelnafex \valtwo$ and $\isctxp\var \mlasrelnafex \isctxtwop\var$ ($\var$ is fresh)
\end{lemma}

\begin{proof}
	By "induction" on lists $\isctx, \isctxtwo$. (They are of the same size because of how \nafex is defined - see \reflemma{lasrelnafex-inert-style-lists-induction})
	\begin{itemize}
		\item $\isctx,\isctxtwo =\ctxhole,\ctxhole$ then $\val \mlasrelnafex \valtwo \iff \val \mlasrelnafex \valtwo$ and $\var\mlasrelnafex\var$ is always true by case (\nafex 2).
		\item $\isctxp\val = \isctxONEp\val\esub\vartwo\itm$ and $\isctxtwop\valtwo \equivx\isctxONEtwop\valtwo\esub\vartwo\itmtwo$ with $\isctxONEp\valtwo \mlasrelnafex \isctxONEtwop\valtwo$ and $\itm \mlasrelnafex \itmtwo$, then
		
		%by case (\nafex 6), $\isctxp\val \mlasrelnafex \isctxtwop\valtwo \iff \isctxONEp\val \mlasrel \isctxONEtwop\valtwo$ and $\itm \mlasrel \itmtwo$ 
		
		%by \reflemma{lasrelnafex-normal-forms-lasrel-right-to-left}, $\iff \isctxONEp\val \mlasrelnafex \isctxONEtwop\valtwo$ and $\itm \mlasrel \itmtwo$
		
		by \ih, $\isctxONEp\valtwo \mlasrelnafex \isctxONEtwop\valtwo\iff \isctxONEp\var \mlasrelnafex \isctxONEtwop\var$ and $\val \mlasrelnafex \valtwo$
		
		by \reflemma{lasrelnafex-normal-forms-lasrel-left-to-right}, $\isctxONEp\valtwo \mlasrelnafex \isctxONEtwop\valtwo\iff \isctxONEp\var \mlasrel \isctxONEtwop\var$, $\val \mlasrelnafex \valtwo$ and $\itm \mlasrel \itmtwo$
		
		and finally by case (nafex 5)\\ $\isctxp\val \mlasrelnafex \isctxtwop\valtwo\iff \isctxp\var=\isctxONEp\var\esub\vartwo\itm \mlasrelnafex \isctxONEtwop\var\esub\vartwo\itmtwo\equivx \isctxtwop\var$, $\val \mlasrelnafex \valtwo$.
	\end{itemize}
\end{proof}

\begin{lemma}
	\label{l:values-fireballs-stable-lasrelnafex}
	If $\fire \mlasrelnafex \firep$ then ($\fire =\isctxp\valt$ $\iff$ $\firep = \isctxtwop\valttwo$)
\end{lemma}

\begin{proof}
	Proof by induction on $\isctx$ using \ref{l:lasrelnafex-normal-forms-lasrel-left-to-right} and \ref{l:lasrelnafex-normal-forms-lasrel-right-to-left}.
\end{proof}

\begin{corollary}
	\label{l:inerts-fireballs-stable-lasrelnafex}
	If $\fire \mlasrelnafex \firep$ then ($\fire =\itm$ $\iff$ $\firep = \itmtwo$)
\end{corollary}

\begin{lemma}
	\label{l:lasrelnafex-normal-forms-isctx-decomposition}
	Let $\ntm$ and $\ntmtwo$ be normal forms and $\isctx$ and $\isctxtwo$ inert substitution contexts which may capture free variables of $\ntm$ and $\ntmtwo$. $\ntm \mlasrelnafex \ntmtwo$ and $ \isctxp\var \mlasrelnafex \isctxtwop\var$ ($\var$ is fresh) $\Rightarrow \isctxp\ntm \mlasrelnafex \isctxtwop\ntmtwo$
\end{lemma}
\begin{proof}
	By induction on $\isctx,\isctxtwo$.
\end{proof}

\begin{lemma}
	\label{l:lasrelnafex-normal-forms-isctx-decomposition-2}
	$\ntm \mlasrelnafex \ntmtwo$ and $ \isctxp\var \mlasrelnafex \isctxtwop\var$ $\Rightarrow \isctxp{\var\ntm} \mlasrelnafex \isctxtwop{\var\ntmtwo}$
\end{lemma}
\begin{proof}
	By induction on $\isctx,\isctxtwo$.
\end{proof}

%unnecessary lemma if we don't do normal forms modulo sigma_1
\begin{lemma}
	\label{l:lasrelnafex-normal-forms-isctx-decomposition-applied-inerts}
	$\ntm \mlasrelnafex \ntmtwo$ and $ \isctxp\itmapp \mlasrelnafex \isctxtwop\itmapptwo$ $\Rightarrow \isctxp{\itmapp\ntm} \mlasrelnafex \isctxtwop{\itmapptwo\ntmtwo}$
\end{lemma}
\begin{proof}
	By induction on $\isctx$.
	
	Since $ \isctxp\itmapp \mlasrelnafex \isctxtwop\itmapptwo$, two cases:
	\begin{itemize}
		\item $\isctx=\ctxhole$ and we fall into case (\nafex 4), that is $\isctxtwop\itmapptwo \equivx \itmapptwo_1$ and $\itmapp \mlasrel \itmapptwo_1$. Hence $\itmapp \ntm \mlasrelnafex \itmapptwo_1\ntmtwo \equivx \isctxtwop{\itmapptwo\ntmtwo}$ follows.
		
		\item  $\isctx=\isctxONE\esub\vartwo\itm$ and we fall into case (\nafex 5), that is $\isctxtwop\itmapptwo \equivx \isctxONEtwop{\itmapptwo_1 }\esub\var\itmtwo$ (we cannot move from inerts to answers and we choose the normal form modulo $\equivsone$) where $\isctxONEp\itmapp \mlasrel \isctxONEtwop{\itmapptwo_1 }$ and $\itm \mlasrel \itmtwo$. Apply \reflemma{lasrelnafex-normal-forms-lasrel-right-to-left} and obtain by \ih that $\isctxONEp{\itmapp\ntm} \mlasrelnafex \isctxONEtwop{\itmapptwo_1\ntmtwo}$. By \reflemma{lasrelnafex-normal-forms-lasrel-left-to-right} and definition of \nafex, we can conclude.\qedhere
	\end{itemize}
\end{proof}

\subsection{Normal substituted terms characterization}
The following two lemmas characterize normal terms that are still normal when a variable is substituted by a value -- this characterization does not depend on $\equivx$ (by strong commutation and preservation of normal forms).
\begin{lemma}
	$\ntm\isub\var\val$ is normal iff $\ntm \not = \evctxp{\var\fire}$. (and $\val \not = \vartwo$)
\end{lemma}

\begin{lemma}
	\label{l:normal-subst-inert-are-inert}
	$\itm\isub\var\val$ is inert iff ($\itm$ is inert,) $\itm\isub\var\val$ is normal.
\end{lemma}

\subsection{Coherence of the \nafex simulation, evaluation and substitution}
We split the Coherence Proposition of the \nafex simulation into two lemmas to prove, knowing that part of the first statement has already been proven in \reflemma{lasrelnafex-normal-forms-substitutive}.

\begin{lemma}
	\label{l:lasrelnafex-on-normal-subs_vsce}
	Let $\rel$ be an \nafex simulation, $\ntm \mlasrelnafex \ntmtwo$, and $\val\mlasrelnafex\valtwo$. If $\ntm\isub\var\val$ is $\tovsce$-normal then $\ntmtwo\isub\var\valtwo$ is $\tovsce$-normal.
	% \adr{(and $\ntm\isub\var\val \mlasrelnafex \ntmtwo\isub\var\valtwo$ is deduced with \ref{l:lasrelnafex-normal-forms-substitutive})}.
\end{lemma}

\begin{proof}
	By induction on normal forms $\ntm$ for which $\ntm\isub\var\val$ is $\tovsce$-normal:
	\begin{itemize}
		\item \emph{Variable}. Two sub-cases:
		\begin{itemize}
			\item $\ntm= \var$ and so $\ntm\isub\var\val = \val$. Then $\ntmtwo = \var$ by case (\nafex 2) and $\ntmtwo\isub\var\valtwo = \valtwo$, which is $\tovsce$-normal. 
			
			
			\item $\ntm= \vartwo$ and so $\ntm\isub\var\val = \vartwo$. Then $\ntmtwo = \vartwo$ by case (\nafex 2) and $\ntmtwo\isub\var\valtwo = \vartwo$, which is $\tovsce$-normal. 
		\end{itemize}
		
		\item \emph{Abstraction}, that is, $\ntm = \la\vartwo\tm$ and so $\ntm\isub\var\val = \la\vartwo\tm\isub\var\val$. Then $\ntmtwo= \la\vartwo\tmp$ with $\tm \mlasrel \tmp$. We have that $\ntmtwo\isub\var\valtwo = \la\vartwo\tmp\isub\var\valtwo$, which is $\tovsce$-normal.
		
		\item \emph{Substituted Inert}, that is, $\ntm = \ntmONE\esub\vartwo\itmONE$ and so $\ntm\isub\var\val = \ntmONE\isub\var\val\esub\vartwo{\itmONE\isub\var\val}$.
		$\ntm\isub\var\val$ is normal is equivalent to $\itmONE\isub\var\val$ and $\ntmONE\isub\var\val$ are normal.
		Then $\ntmtwo \equivx \ntmONEtwo\esub\vartwo\itmONEtwo$ with $\itmONE \mlasrel \itmONEtwo$ and $\ntmONE \mlasrel \ntmONEtwo$, which implies $\itmONE \mlasrelnafex \itmONEtwo$ and $\ntmONE \mlasrelnafex \ntmONEtwo$ by \reflemma{lasrelnafex-normal-forms-lasrel-right-to-left}.
		By \ih we then have $\itmONEtwo\isub\var\valtwo$ is $\tovsce$-normal and $\ntmONEtwo\isub\var\valtwo$ is $\tovsce$-normal : which is equivalent to  $\ntmtwo\isub\var\valtwo$ is $\tovsce$-normal by \refprop{equivx-preserves-normal-forms}.
		
		
		
		\item \emph{Applied Normal forms}, that is, $\ntm = \ntmONE\ntmTWO$. Then we have three sub-cases:
		\begin{itemize}
			\item $\ntm = \var\ntmTWO$ and $\val$ is not an abstraction,
			Then $\ntmtwo \equivx \var\ntmTWOtwo$ with $\ntmTWO \mlasrel \ntmTWOtwo$, which implies $\ntmTWO \mlasrelnafex \ntmTWOtwo$ by \reflemma{lasrelnafex-normal-forms-lasrel-right-to-left}.
			By \ih we then have $\ntmTWOtwo\isub\var\valtwo$ is $\tovsce$-normal: which is equivalent to $\ntmtwo\isub\var\valtwo$ is $\tovsce$-normal by \refprop{equivx-preserves-normal-forms}.
			\item $\ntm = \vartwo\ntmTWO$,
			Then $\ntmtwo \equivx \vartwo\ntmTWOtwo$ with $\ntmTWO \mlasrel \ntmTWOtwo$, which implies $\ntmTWO \mlasrelnafex \ntmTWOtwo$ by \reflemma{lasrelnafex-normal-forms-lasrel-right-to-left}.
			By \ih we then have $\ntmTWOtwo\isub\var\valtwo$ is $\tovsce$-normal: which is equivalent to $\ntmtwo\isub\var\valtwo$ is $\tovsce$-normal by \refprop{equivx-preserves-normal-forms}.
			\item $\ntm = \itmapp\ntmTWO$, and $\itmapp\isub\var\val$ is an applied inert,
			Then $\ntmtwo \equivx \itmapptwo\ntmTWOtwo$ with $\itmapp \mlasrel \itmapptwo$ and $\ntmTWO \mlasrel \ntmTWOtwo$, which implies $\itmapp \mlasrelnafex \itmapptwo$ and $\ntmTWO \mlasrelnafex \ntmTWOtwo$ by \reflemma{lasrelnafex-normal-forms-lasrel-right-to-left}.
			
			By \ih we then have $\itmapptwo\isub\var\valtwo$ is $\tovsce$-normal, $\ntmTWOtwo\isub\var\valtwo$ is $\tovsce$-normal : which is equivalent to $\ntmtwo\isub\var\valtwo$ is $\tovsce$-normal by \refprop{equivx-preserves-normal-forms} and by \reflemma{normal-subst-inert-are-inert}.\qedhere
		\end{itemize}
		
		
	\end{itemize}
\end{proof}

\begin{lemma} 
	\label{l:lasrelnafex-not-normal-subs}
	If $\ntm_\tmrone \mlasrelnafex \ntm_\tmrtwo$, $\val \mlasrel \valtwo$
	and $\ntm_\tmrone\isub\var{\val} \to \tmronep$
	then $\ntm_\tmrtwo\isub\var{\valtwo}  \to \tmrtwop$ and $\tmronep \mlasrel \tmrtwop$
\end{lemma}

\begin{proof}
	(We write $\val=\la\vartwo\tmfour$ and $\valtwo = \la\vartwo\tmfourp$ with $\tmfour \mlasrel \tmfourp$ - using the fact that ($\val \mlasrel \valtwo \Rightarrow \val \mlasrelnafex \valtwo$) by \ref{l:lasrelnafex-normal-forms-lasrel-right-to-left}.)
	
	If $\ntm_\tmrone\isub\var{\val} \to \tmronep$ then $\ntm_\tmrone = \evctxp{\var\fire}$ ($\evctx$ is the context where the reduction has been done).
	Show (by induction on $\evctx$) that $\tmronep \mlasrel \tmrtwop$.
	\begin{itemize}
		\item $\evctx = \ctxhole$
		then $\ntm_\tmrone = {\var\fire}$ , 
		by $\ntm_\tmrone \mlasrelnafex \ntm_\tmrtwo$ we have $\ntm_\tmrtwo \equivx {\var\firetwo}$ with $\fire \mlasrel \firetwo$. 
		
		
		By substitutivity of $\equivx$ (\refprop{equivx-is-substitutive}), $\ntm_\tmrtwo\isub\var\valtwo \equivx {\valtwo\firetwo\isub\var\valtwo}\to {\tmfourp\esub\vartwo{\firetwo\isub\var\valtwo}}$
		
		By \refprop{equivx-is-a-strong-bisimulation}, there exists $\tmrtwop$ such that ${\tmfourp\esub\vartwo{\firetwo\isub\var\valtwo}} \equivx \tmrtwop$ and $\ntm_\tmrtwo\isub\var\valtwo \tovsct\tmrtwop$.
		
		We also have $\ntm_\tmrone\isub\var\val \to {\tmfour\esub\vartwo{\fire\isub\var\val}} = \tmronep$ and we can build the following derivation:
		\[ \infer[\scequivx]{\tmronep\mlasrel\tmrtwop}{\infer[\scesub]{ \tmfour\esub\vartwo{\fire\isub\var\val} \mlasrel \tmfourp\esub\vartwo{\firetwo\isub\var\valtwo}}{\tmfour \mlasrel \tmfourp & \infer{\fire\isub\var\val \mlasrel \firetwo\isub\var\valtwo}{\fire \mlasrel \firetwo & \val \mlasrel \valtwo}} & {\tmfourp\esub\vartwo{\firetwo\isub\var\valtwo}} \equivx \tmrtwop} \]
		
		hence the result $\ntm_\tmrone\isub\var\val \to \tmronep$, $\ntm_\tmrtwo\isub\var\valtwo \to \tmrtwop$ and $\tmronep \mlasrel \tmrtwop$.
		
		
		
		\item $\evctx = \tmtwo\evctxONE$ ($\tmtwo = \itm$ because $\ntm_\tmrone$ is normal) 	then $\ntm_\tmrone = \itm\evctxONEp\tmthree$ (where $\tmthree = \var\firetwo$).
		Then by $\ntm_\tmrone \mlasrelnafex \ntm_\tmrtwo$, $\ntm_\tmrtwo \equivx \itmtwo\ntmONE$ with $\itm \mlasrel \itmtwo$ and $\evctxONEp\tmthree \mlasrel \ntmONE$.
		$\evctxONEp\tmthree\isub\var\val \to \tm$ by hypothesis ($\ntm_\tmrone\isub\var\val \to \tmronep$ is in this case $\itm\isub\var\val\evctxONEp\tmthree\isub\var\val \to \itm\isub\var\val\tm$) and $\evctxONEp\tmthree \mlasrelnafex \ntmONE$ (normal forms, apply lemma \ref{l:lasrelnafex-normal-forms-lasrel-right-to-left}), hence by \ih $\ntmONE\isub\var\valtwo \to \tmp$ with $\tm \mlasrel \tmp$.
		
		By substitutivity and strong commutation of $\equivx$,
		$\ntm_\tmrtwo\isub\var\valtwo \equivx \itmtwo\isub\var\valtwo\ntmONE\isub\var\valtwo\to \itmtwo\isub\var\valtwo\tmp$ implies $\ntm_\tmrtwo\isub\var\valtwo\tovsct\tmrtwop$ with 
		$\itmtwo\isub\var\valtwo\tmp \equivx \tmrtwop$.
		\[\infer{\tmronep\mlasrel\tmrtwop}{\infer{\itm\isub\var\val\tm \mlasrel \itmtwo\isub\var\valtwo\tmp }{ \infer{\itm\isub\var\val \mlasrel \itmtwo\isub\var\valtwo}{\itm \mlasrel \itmtwo & \val \mlasrel \valtwo} & \tm \mlasrel \tmp} & \itmtwo\isub\var\valtwo\tmp \equivx \tmrtwop}\]
		
		hence $\ntm_\tmrone\isub\var\val \to \tmronep$, $\ntm_\tmrtwo\isub\var\valtwo \to \tmrtwop$ and $\tmronep \mlasrel \tmrtwop$.
		
		
		\item The rest of the induction cases ($\evctx = \evctxONE\tmtwo$, $\evctx = \tmtwo\esub\varthree\evctxONE$ or $\evctx = \evctxONE\esub\varthree\tmtwo$) follow from very similar arguments.\qedhere
		%		
		%		\item $\evctx = \evctxONE\tmtwo$ ($\tmtwo = \fire$ because $\ntm_\tmrone$ is normal) then $\ntm_\tmrone = \evctxONEp\tmthree\fire$ (where $\tmthree = \var\firetwo$).
		%		Then by $\ntm_\tmrone \mlasrelnafex \ntm_\tmrtwo$, $\ntm_\tmrtwo = \ntmONE\firep$ with $\fire \mlasrel \firep$ and $\evctxONEp\tmthree \mlasrel \ntmONE$.
		%		$\evctxONEp\tmthree\isub\var\val \to \tm$ by hypothesis ($\ntm_\tmrone\isub\var\val \to \tmronep$ is in this case $\evctxONEp\tmthree\isub\var\val\fire\isub\var\val \to \tm\fire\isub\var\val$) and $\evctxONEp\tmthree \mlasrelnafex \ntmONE$ (normal forms, apply lemma \ref{l:lasrelnafex-normal-forms-lasrel-right-to-left}), hence by \ih $\ntmONE\isub\var\valtwo \to \tmp$ with $\tm \mlasrel \tmp$
		%		\[\infer{\tmronep=\tm\fire\isub\var\val \mlasrel \tmp\firep\isub\var\valtwo = \tmrtwop}{\tm \mlasrel \tmp & \infer{\fire\isub\var\val \mlasrel \firep\isub\var\valtwo}{\fire \mlasrel \firep & \val \mlasrel \valtwo}}\]
		%		
		%		hence $\ntm_\tmrone\isub\var\val \to \tmronep$, $\ntm_\tmrtwo\isub\var\valtwo \to \tmrtwop$ and $\tmronep \mlasrel \tmrtwop$.
		%		
		%		
		%		\item $\evctx = \tmtwo\esub\varthree\evctxONE$ ($\tmtwo = \fire$ because $\ntm_\tmrone$ is normal) then $\ntm_\tmrone = \fire\esub\varthree{\evctxONEp\tmthree}$ (where $\tmthree = \var\firetwo$).
		%		Then by $\ntm_\tmrone \mlasrelnafex \ntm_\tmrtwo$, $\ntm_\tmrtwo = \firep\esub\varthree\ntmONE$ with $\fire \mlasrel \firep$ and $\evctxONEp\tmthree \mlasrel \ntmONE$.
		%		$\evctxONEp\tmthree\isub\var\val \to \tm$ by hypothesis ($\ntm_\tmrone\isub\var\val \to \tmronep$ is in this case $\fire\isub\var\val\esub\varthree{\evctxONEp\tmthree\isub\var\val} \to \fire\isub\var\val\esub\varthree{\tm}$) and $\evctxONEp\tmthree \mlasrelnafex \ntmONE$ (normal forms, apply lemma \ref{l:lasrelnafex-normal-forms-lasrel-right-to-left}), hence by \ih $\ntmONE\isub\var\valtwo \to \tmp$ with $\tm \mlasrel \tmp$
		%		\[\infer{\tmronep=\fire\isub\var\val\esub\varthree\tm \mlasrel \firep\isub\var\valtwo\esub\varthree\tmp = \tmrtwop}{ \infer{\fire\isub\var\val \mlasrel \firep\isub\var\valtwo}{\fire \mlasrel \firep & \val \mlasrel \valtwo} & \tm \mlasrel \tmp}\]
		%		
		%		hence $\ntm_\tmrone\isub\var\val \to \tmronep$, $\ntm_\tmrtwo\isub\var\valtwo \to \tmrtwop$ and $\tmronep \mlasrel \tmrtwop$.
		%		
		%		\item $\evctx = \evctxONE\esub\varthree\tmtwo$ ($\tmtwo = \itm$ because $\ntm_\tmrone$ is normal) then $\ntm_\tmrone = \evctxONEp\tmthree\esub\varthree{\itm}$ (where $\tmthree = \var\firetwo$).
		%		Then by $\ntm_\tmrone \mlasrelnafex \ntm_\tmrtwo$, $\ntm_\tmrtwo = \ntmONE\esub\varthree\itmtwo$ with $\itm \mlasrel \itmtwo$ and $\evctxONEp\tmthree \mlasrel \ntmONE$.
		%		$\evctxONEp\tmthree\isub\var\val \to \tm$ by hypothesis ($\ntm_\tmrone\isub\var\val \to \tmronep$ is in this case ${\evctxONEp\tmthree\isub\var\val}\esub\varthree{\itm\isub\var\val} \to \tm\esub\varthree{\itm\isub\var\val}$) and $\evctxONEp\tmthree \mlasrelnafex \ntmONE$ (normal forms, apply lemma \ref{l:lasrelnafex-normal-forms-lasrel-right-to-left}), hence by \ih $\ntmONE\isub\var\valtwo \to \tmp$ with $\tm \mlasrel \tmp$
		%		\[\infer{\tmronep=\tm\esub\varthree{\itm\isub\var\val}\mlasrel \tmp\esub\varthree{\itmtwo\isub\var\valtwo} = \tmrtwop}{ \tm \mlasrel \tmp &  \infer{\itm\isub\var\val \mlasrel \itmtwo\isub\var\valtwo}{\itm \mlasrel \itmtwo & \val \mlasrel \valtwo}}\]
		%		
		%		hence $\ntm_\tmrone\isub\var\val \to \tmronep$, $\ntm_\tmrtwo\isub\var\valtwo \to \tmrtwop$ and $\tmronep \mlasrel \tmrtwop$.
	\end{itemize}
\end{proof}


\subsection{Mirrored Lassen's Closure preserves \nafex simulations}
After all these preliminaries, we can finally prove \nafex compatibility, with a very similar proof than in the case of \naf.


\begin{proposition}
	\label{prop:main-lemma_vsce}
	Let $\relsym$ be a \nafex simulation.
	\begin{enumerate}
		\item \emph{Technical auxiliary statement}: if $\tmrone\mlasrel\tmrtwo$ and $\tmrone \bsvscp k \ntm$ then $\tmrtwo\bsvscps \ntmtwo$ and $\ntm \mlasrelnafex \ntmtwo$.		
		\item \emph{Mirrored Lassen's closure preserves \nafex simulations}:  $\mlassenop\relsym$ is a \nafex simulation.
	\end{enumerate}
\end{proposition}

\begin{proof}%[Proof of \reflemma{main-lemma-bis_vsce}]
	\begin{enumerate}
		\item 
	By induction on $(k,d)$ where $d$ is the size of the derivation of $\tmrone \mlasrel \tmrtwo$.
	% !TEX root = main.tex

%%By induction on $(k,d)$ where $d=$ the size of the derivation of $\tmrone \mlasrel \tmrtwo$.

By case analysis on the last rule of the derivation $\tmrone\mlasrel \tmrtwo$.

\begin{enumerate}
	\item \emph{Lifting}:
	\[ \infer[(\sclift) ]{\tmrone \mlasrel \tmrtwo} {\tmrone \rel \tmrtwo}\text{ and }\tmrone\bsvsct k \ntm\]
	Since $\relsym$ is a \nafex simulation, we have $\tmrone\relnafex\tmrtwo$ and $\tmrtwo \bsvscts \ntmtwo$ for some $\ntmtwo$ such that $\ntm\relnafex\ntmtwo$. Hence  $\ntm\mlasrelnafex\ntmtwo$ by monotonicity of $\opnafex$.
	
	%%%%%%%%%%%%
	\item \emph{Variables}:
	\[\infer[(\scvar) ]{\var \mlasrel \var}	{} \text{ and } \var\bsvsct 0 \var\]
	
	hence the result ($\var\bsvsct 0 \var$) and by the definition of \nafex, $\var \mlasrelnafex \var$.
	
	%%%%%%%%%%%%
	\item \emph{Abstraction}:
	\[\infer[(\scabs) ]{\la\var\tmrone \mlasrel \la\var\tmrtwo} {\tmrone \mlasrel \tmrtwo} \text{ and } \la\var\tmrone \bsvsct 0 \la\var\tmrone \]
	
	hence the result ($\la\var\tmrtwo \bsvsct 0 \la\var\tmrtwo$) and by the definition of \nafex, $\la\var\tmrone \mlasrelnafex \la\var\tmrtwo$.
	%%%%%%%%%%%%
	\item \emph{Application}:
	\[ \infer[(sc.app) ] 
	{\tmrone\tmrthree  \mlasrel  \tmrtwo\tmrfour} {\tmrone  \mlasrel \tmrtwo & \tmrthree \mlasrel \tmrfour } \text{ and }\tmrone\tmrthree \bsvsct k \ntm \]
	
	then, by case analysis on the last rule of the big-step derivation,
	\begin{enumerate}
		\item \emph{Applied variable}: 
		\[\infer{\tmrone\tmrthree \bsvsct {k+h} \isctxp{\var\ntm}}{
			\tmrone \bsvsct k \isctxp\var
			&
			\tmrthree \bsvsct h \ntm
		}\]
		
		by inductive hypothesis ($d$ strictly decreasing, first component not increasing) we obtain $\tmrtwo \bsvscts \firep$ with $\isctxp{\var} \mlasrelnafex \firep$ (hence by \reflemma{values-fireballs-stable-lasrelnafex}, $\firep = \isctxtwop{\var}$)and $\tmrfour \bsvscts \ntmtwo$ with $\ntm \mlasrelnafex \ntmtwo$. Then:
		\[\infer{\tmrtwo\tmrfour \bsvscts \isctxtwop{\var\ntmtwo}}{
			\tmrtwo \bsvscts  \itmtwo = \isctxtwop{\var}
			&
			\tmrfour \bsvscts \ntmtwo
		}\]
		%Hence, by \reflemma{lasrelnafex-normal-forms-lasrel-left-to-right},  $\isctxp{\var} \mlasrel \isctxtwop\var$ and $\ntm \mlasrel \ntmtwo$. 
		By definition of \nafex, and 
		%since $\isctxp{\var\ntm} \equivx \isctxp\var\ntm$ and $\isctxtwop{\var\ntmtwo} \equivx \isctxtwop\var\ntmtwo$, 
		by \reflemma{lasrelnafex-normal-forms-isctx-decomposition-2}
		we have $\isctxp{\var\ntm} \mlasrelnafex \isctxtwop{\var\ntmtwo}$.
		\item \emph{Applied inert}: 
		\[\infer{\tmrone\tmrthree \bsvsct {k+h} \isctxp{\itmappONE\ntm}}{
			\tmrone \bsvsct k \isctxp\itmappONE = \itm
			&
			\tmrthree \bsvsct h \ntm
		}\]
		
		by inductive hypothesis ($d$ strictly decreasing, first component not increasing) we obtain $\tmrtwo \bsvscts \itmtwo = \isctxtwop{\itmappONEtwo}$ (the $\tovsce$ normal form, and it is an inert by Corollary \ref{l:inerts-fireballs-stable-lasrelnafex}) and $\tmrfour \bsvscts \ntmtwo$ with $\itm\mlasrelnafex\itmtwo,~\ntm \mlasrelnafex \ntmtwo$. Then:
		\[\infer{\tmrtwo\tmrfour \bsvscts \isctxtwop{\itmappONEtwo\ntmtwo}}{
			\tmrtwo \bsvscts  \itmtwo = \isctxtwop{\itmappONEtwo}
			&
			\tmrfour \bsvscts \ntmtwo
		}\]
		%and by \reflemma{lasrel-inerts-separating-applicative-inerts-from-substitutions} applied on $\isctxp\itmappONE=\itm\mlasrel\itmtwo=\isctxtwop{\itmappONEtwo}$, $\itmappONE \mlasrel \itmappONEtwo$ and $\isctxp{\var} \mlasrel \isctxtwop{\var}$. Hence by ($\scapp$), $\itmappONE\ntm \mlasrel \itmappONEtwo\ntmtwo$. Finally, by \reflemma{lasrel-inerts-separating-applicative-inerts-from-substitutions}, 
		
		By \reflemma{lasrelnafex-normal-forms-isctx-decomposition-applied-inerts}, $\isctxp{\itmappONE\ntm} \mlasrelnafex \isctxtwop{\itmappONEtwo\ntmtwo}$.
		
		\item \emph{Substitution of an inert}:
		not applicable.
		
		
		\item \emph{$m$ step}:
		\[\infer{\tmrone\tmrthree \bsvsct {k+i+1} \isctxp\ntm}{
			\tmrone \bsvsct k \isctxp{\la\var\tmronep}
			&
			{\tmronep\esub\var\tmrthree} \bsvsct i \ntm
		}\]
		
		
		
		then by inductive hypothesis ($d$ strictly decreasing, first component non increasing) on $\tmrone$ we get
		$\tmrtwo\bsvscts \ntm_\tmrtwo$ with $\isctxp{\la\var\tmronep} \mlasrelnafex \ntm_\tmrtwo$ (hence by  \reflemma{values-fireballs-stable-lasrelnafex} $\ntm_\tmrtwo = \isctxtwop{\la\var\tmrtwop}$) \ie $\isctxp{\la\var\tmronep} \mlasrel \isctxtwop{\la\var\tmrtwop}$.
		
		Hence by \reflemma{lasrelnafex-normal-forms-lasrel-right-to-left}, $\isctxp{\la\var\tmronep} \mlasrelnafex \isctxtwop{\la\var\tmrtwop}$ and by \reflemma{lasrelnafex-values-isctx-decomposition} we get\\ $\isctxp{\var} \mlasrelnafex \isctxtwop{\var}$ and $\la\var\tmronep\mlasrelnafex \la\var\tmrtwop$ then $\tmronep\mlasrel\tmrtwop$ by case (\nafex 3).
		
		Then:
		\[\infer{\tmronep\esub\var\tmrthree \mlasrel \tmrtwop\esub\var\tmrfour}{\tmronep\mlasrel\tmrtwop&\tmrthree\mlasrel\tmrfour}\]
		
		since $\tmronep\esub\var\tmrthree \bsvsct i \ntm$ with $i < k+i+1$ we can apply the inductive hypothesis on the first component for $\tmronep\esub\var\tmrthree$ obtaining $\tmrtwop\esub\var\tmrfour \bsvscts  \ntmtwo$ for some $\ntmtwo$ such that $\ntm\mlasrelnafex\ntmtwo$. Since $\isctxp{\var} \mlasrelnafex \isctxtwop{\var}$ and $\ntm\mlasrelnafex\ntmtwo$, by \reflemma{lasrelnafex-normal-forms-isctx-decomposition}, we get $\isctxp{\ntm} \mlasrelnafex \isctxtwop{\ntmtwo}$.
		Last, note that $\tmrtwo\tmrfour\bsvscts\isctxtwop\ntmtwo$ by 
		\[\infer{\tmrtwo\tmrfour \bsvscts \isctxtwop\ntmtwo}{
			\tmrtwo \bsvscts \isctxtwop{\la\var\tmrtwop}
			&
			\tmrtwop\esub\var\tmrfour \bsvscts \ntmtwo
		}\]
		
		
		\item \emph{$e$ step}:
		not applicable.
	\end{enumerate}
	
	\item \emph{Explicit Substitution}: 
	\[ \infer[(sc.esubst) ]{\tmrone\esub\var{\tmrthree} \mlasrel \tmrtwo\esub\var{\tmrfour}} {\tmrone \mlasrel \tmrtwo & \tmrthree \mlasrel \tmrfour }\text{ and }\tmrone\esub\var{\tmrthree} \bsvsct k \ntm \]
	
	\begin{enumerate}
		\item \emph{Applied inert}: not applicable.
		\item \emph{Substitution of an inert}:
		\[ \infer{\tmrone\esub\var\tmrthree \bsvsct {k+h} \ntm\esub\var\itm}{
			\tmrone \bsvsct k \ntm
			&
			\tmrthree \bsvsct h \itm
		} \]
		
		by inductive hypothesis ($d$ strictly decreasing, first component not increasing) we obtain $\tmrtwo \bsvscts \ntmtwo$ and $\tmrfour \bsvscts \itmtwo$ with $\itm\mlasrelnafex\itmtwo$ and $\ntm \mlasrelnafex \ntmtwo$. By \reflemma{lasrelnafex-normal-forms-lasrel-left-to-right},  $\itm\mlasrel\itmtwo$ and $\ntm \mlasrel \ntmtwo$. Then:
		\[\infer{\tmrtwo\esub\var\tmrfour \bsvscts \ntmtwo\esub\var\itmtwo}{
			\tmrtwo \bsvscts  \ntmtwo
			&
			\tmrfour \bsvscts \itmtwo
		}\]
		and $\ntm\esub\var\itm \mlasrelnafex \ntmtwo\esub\var\itmtwo$ by definition of \nafex.
		
		\item \emph{m step}: not applicable.
		\item \emph{e step}: 		\[ \infer{\tmrone\esub\var{\tmrthree} \bsvsct {k+i+1} \isctxp\ntm}{
			\tmrthree \bsvsct k \isctxp{\valof{\tmrthree}}
			&
			\tmrone\isub\var{\valof{\tmrthree}}\bsvsct i \ntm
		} \]
		
		then by inductive hypothesis ($d$ strictly decreasing, first component non increasing) on $\tmrone$ and $\tmrthree$ we get
		$\tmrfour\bsvscts\tmrfourp$ with $\isctxp{\valof\tmrthree} \mlasrelnafex \tmrfourp$ (hence by \reflemma{values-fireballs-stable-lasrelnafex} $\tmrfourp = \isctxtwop{\valof\tmrfour}$) \ie $\isctxp{\valof\tmrthree} \mlasrelnafex \isctxtwop{\valof\tmrfour}$.
		Hence by \reflemma{lasrelnafex-values-isctx-decomposition} we get $\isctxp{\var} \mlasrelnafex \isctxtwop{\var}$ and ${\valof\tmrthree}\mlasrelnafex{\valof\tmrfour}$ \ie ${\valof\tmrthree}\mlasrel{\valof\tmrfour}$ by \ref{l:lasrelnafex-normal-forms-lasrel-left-to-right}.
		
		Then:
		\[\infer{\tmrone\isub\var{\valof\tmrthree} \mlasrel \tmrtwo\isub\var{\valof\tmrfour}}{\tmrone\mlasrel\tmrtwo& {\valof\tmrthree}\mlasrel{\valof\tmrfour}}\]
		
		since $\tmrone\isub\var{\valof\tmrthree} \bsvsct i \ntm$ with $i < k+i+1$ we can apply the inductive hypothesis on the first component for $\tmrone\isub\var{\valof\tmrthree}$ obtaining $\tmrtwo\isub\var{\valof\tmrfour} \bsvscts  \ntmtwo$ for some $\ntmtwo$ such that $\ntm\mlasrelnafex\ntmtwo$. 
		Since $\isctxp{\var} \mlasrelnafex \isctxtwop{\var}$ and $\ntm\mlasrelnafex\ntmtwo$, by \reflemma{lasrelnafex-normal-forms-isctx-decomposition}, we get $\isctxp{\ntm} \mlasrelnafex \isctxtwop{\ntmtwo}$.
		
		Last, note that $\tmrtwo\tmrfour\bsvscts\isctxtwop\ntmtwo$ by 
		\[\infer{\tmrtwo\tmrfour \bsvscts \isctxtwop\ntmtwo}{
			\tmrfour \bsvscts \isctxtwop{\valof\tmrfour}
			&
			\tmrtwo\isub\var{\valof\tmrfour} \bsvscts \ntmtwo
		}\]
		
	\end{enumerate}
	
	
	%%%%%%%%%%%%%
	\item \emph{Implicit Substitution}: 
	\[ \infer[(sc.subst) ]{\tmrone\isub\var{\valof\tmrthree} \mlasrel \tmrtwo\isub\var{\valof\tmrfour}} {\tmrone \mlasrel \tmrtwo & \valof\tmrthree \mlasrel \valof\tmrfour }\text{ and }\tmrone\isub\var{\valof\tmrthree} \bsvsct k \ntm \]
	then by applying the splitting lemma (\reflemma{splitting_vsce}), we obtain $\tmrone \bsvsct {k_1} \ntm_\tmrone$ and $\ntm_\tmrone\isub\var{\valof\tmrthree} \bsvsct {k_2} \ntm$ with $k=k_1+k_2$. Hence by inductive hypothesis ($d$ strictly decreasing, first component non increasing) $\tmrtwo\bsvscts\ntm_\tmrtwo$ and $\ntm_\tmrone \mlasrelnafex \ntm_\tmrtwo$, and by \ih $\valof\tmrthree \mlasrelnafex \valof\tmrfour $. By applying \reflemma{lasrelnafex-normal-forms-lasrel-left-to-right}, we have $\ntm_\tmrone \mlasrel \ntm_\tmrtwo$.
	We then have: 
	\[\infer{\ntm_\tmrone\isub\var{\valof\tmrthree} \mlasrel \ntm_\tmrtwo\isub\var{\valof\tmrfour}} {\ntm_\tmrone \mlasrel \ntm_\tmrtwo & \valof\tmrthree \mlasrel \valof\tmrfour }\]
	
	%	Note that by \ih applied to $\valof\tmrthree \mlasrel \valof\tmrfour $ and $\valof\tmrthree\bsvscts \valof\tmrthree$ (first component not increasing, second strictly decrasing) we obtain $\valof\tmrthree \mlasrelnafex \valof\tmrfour$.
	
	Two cases.
	\begin{enumerate}
		
		
		\item \emph{$\tmrone$ is not normal}, that is, $k_1>0$ and $k_2<k$. Then by applying the induction hypothesis to $k_2$ (first component)  and $\ntm_\tmrone\isub\var{\valof\tmrthree}$ we obtain $\ntm_\tmrtwo\isub\var{\valof\tmrfour} \bsvscts \ntmtwo$ with $\ntm\mlasrelnafex\ntmtwo$. We conclude using stability \ref{l:stability_vsce} and the equivalence between big and small steps, because $\tmrtwo\isub\var{\valof\tmrfour} \to^* \ntm_\tmrtwo\isub\var{\valof\tmrfour} \to^* \ntmtwo$.
		
		
		\item \emph{$\tmrone$ is normal}, that is, $k_1=0$ and $k_2=k$. Then $\tmrone=\ntm_\tmrone$. Two sub-cases:
		\begin{itemize}
			\item \emph{$\tmrone\isub\var{\valof\tmrthree} = \ntm_\tmrone\isub\var{\valof\tmrthree}$ is also normal}
			
			
			Since we know that $\ntm_\tmrone \mlasrelnafex \ntm_\tmrtwo$  and $\valof\tmrthree \mlasrelnafex \valof\tmrfour $, we can apply \reflemma{lasrelnafex-on-normal-subs_vsce}, and obtain that $\ntm_\tmrtwo\isub\var{\valof\tmrfour}$ is $\tovsce$-normal and by \reflemma{lasrelnafex-normal-forms-substitutive}, $\tmrone\isub\var{\valof\tmrthree} \mlasrelnafex \ntm_\tmrtwo\isub\var{\valof\tmrfour}$. It is only left to show that $\tmrtwo\isub\var{\valof\tmrfour} \bsvscts \ntm_\tmrtwo\isub\var{\valof\tmrfour}$, which follows from $\tmrtwo\bsvscts \ntm_\tmrtwo$, stability of reduction under substitution (\reflemma{stability_vsce}) and the fact that $\ntm_\tmrtwo\isub\var{\valof\tmrfour}$ is $\tovsce$-normal (plus the equivalence of small and big steps).
			
			\item   \emph{$\tmrone\isub\var{\valof\tmrthree} = \ntm_\tmrone\isub\var{\valof\tmrthree}$ is not normal}
			
			hence $\ntm_\tmrone\isub\var{\valof\tmrthree} \to \tmronep \to ^ {k-1} \ntm$ (the reduction is diamond, all reductions are of the same length, we pick any first step possible). Then by \reflemma{lasrelnafex-not-normal-subs} , $\ntm_\tmrtwo\isub\var{\valof\tmrfour} \to \tmrtwop$ with $\tmronep \mlasrel \tmrtwop$.
			
			We can apply the inductive hypothesis to $\tmronep$ (first component is decreasing, as $k-1<k$) and we obtain $\tmrtwop \bsvscts \ntmtwo$ with $\ntm\mlasrelnafex\ntmtwo$.
			The statement is then proved, since (using \reflemma{stability_vsce})
			$$\tmrtwo\isub\var{\valof\tmrfour} \to^* \ntm_\tmrtwo\isub\var{\valof\tmrfour} \to \tmrtwop \to^* \ntmtwo$$ that is, $\tmrtwo\isub\var{\valof\tmrfour} \bsvscts \ntmtwo$ by \reflemma{ss-bs-equivalence_vsce}.
			
		\end{itemize}
	\end{enumerate}

	\item \emph{Equivalent $X$} \[	\infer[\scequivx]{\tmrone \mlasrel \tmrtwop} {\tmrone \mlasrel \tmrtwo & \tmrtwo \equivx \tmrtwop} \text{ and }\tmrone \bsvsct k \ntm \]
	
	by \ih, $\tmrtwo \bsvscts \ntmtwo$ and $\ntm \mlasrelnafex \ntmtwo$.
	
	Since $\tmrtwo$ has a normal form and $\tmrtwo \equivx \tmrtwop$ then $\tmrtwop \bsvscts \ntmthree$ and $\ntmtwo \equivx \ntmthree$ by \refprop{equivx-is-a-strong-bisimulation}, hence $\ntm \mlasrelnafex \ntmthree$ by \refprop{relnafex-equivx-subseteq-relnafex}.
	
\end{enumerate}

	\item Reformulation of the first point.\qedhere
\end{enumerate}
\end{proof}

\begin{proposition}[\nafex similarity is adequate]
	\label{prop:adequacy-nafex}
	Suppose $\equivx$ is a mirror for $\tovsc$. If $\tm \leqnafex \tmtwo$ then $\tm \bsvscs$ implies $\tmtwo\bsvscs$
\end{proposition}

\begin{proof}
	Without using equivalences in the definition this fact was obvious. With equivalences, we need the fact that $\equivx$ preserves normal forms to conclude (Proposition \ref{prop:equivx-preserves-normal-forms}).
\end{proof}


\gettoappendix{thm:nafex-included-leqc}

\begin{proof}
	\begin{enumerate}
		\item By Proposition \ref{prop:main-lemma_vsce} and coinductive definition of $\leqvscx$.
		\item Compatibility comes from the first point of the theorem and the fact that the mirrored Lassen's closure is compatible. Inclusion in contextual preorder by compatibility and adequacy (\refprop{adequacy-nafex}).
		\item Similar argument, here we rely on the fact that $\streq$ is substitutive and a strong commutation for the VSC. \qedhere
	\end{enumerate}
\end{proof}
% !TEX root = main.tex

\section{Proofs from \refsect{type-preorder} (From Operational to Denotational Semantics: the Type Preorder)}
In this section, we prove the propositions from the Section 12 \emph{From Operational to Denotational Semantics: the Type Preorder}.


\subsection{Type preorder is compatible}
Here we show that the type preorder is compatible. The proof is quite trivial, as the type preorder is somehow \emph{compositional}. We first prove a lemma about compositionality of syntax, then compatibility follows by an induction on contexts.

\begin{lemma} Type preorder verifies:
	\begin{itemize}
		\item \emph{(applicative)}  $\tm \leqtype \tmp ~\&~ \tmtwo\leqtype\tmtwop \Rightarrow {\tm\tmtwo}  \leqtype{\tmp\tmtwop}$.
		\item \emph{(abstractive)} $\tm \leqtype \tmtwo \Rightarrow {\la\var\tm} \leqtype {\la\var\tmtwo}$.
		\item \emph{(explicitly substitutive)} $\tm \leqtype \tmp ~\&~ \tmtwo \leqtype \tmtwop \Rightarrow {\tm\esub\var\tmtwo} \leqtype {\tmp\esub\var\tmtwop}$.
		\item \emph{($\alpha$-equivalence)} ${\la\var\tm} \equivtype {\la\vartwo\tm\isub\var\vartwo}$.
		\item \emph{($\alpha$-equivalence')} ${\tm\esub\var\tmtwo} \equivtype {(\tm\isub\var\vartwo)\esub\vartwo\tmtwo}$.
	\end{itemize}
\end{lemma}

\begin{proof}
\hfill
	\begin{itemize}
		\item \emph{(applicative)}, \emph{(explicitly substitutive)} and \emph{(abstractive)} properties can be done by looking at trees since they are very syntax driven. We look at the (ES) case to sketch the idea:
		\\
		
		Let $\typeder$ be a type derivation for $\tm\esub\varthree\tmtwo$, $\typeder :~ \typectx \types \tm\esub\varthree\tmtwo \hastype \mtype$.
		
		We show that there exists a type derivation $\typederp$ such that $\typederp :~ \typectx \types \tmp\esub\varthree\tmtwop \hastype \mtype$.
		
		Since the term $\tm\esub\varthree\tmtwo$ is not a value, there is only one possibility for the last rule of the derivation: ($\typingruleES$).
		\[\infer[\typingruleES]{\typectx \types \tm\esub\varthree\tmtwo \hastype \mtype}{\typectx_1, \varthree \hastype \mtypetwo \types \tm \hastype \mtype & \typectx_2 \types \tmtwo \hastype \mtypetwo}\]
		
		where $\typectx = \typectx_1 \uplus \typectx_2$.
		
		Since $\tm$ is type equivalent to $\tmp$ and $\tmtwo$ is type equivalent to $\tmtwop$, there exists two derivations $\typectx_2 \types \tmtwop \hastype \mtypetwo$ and $\typectx_1, \varthree \hastype \mtypetwo \types \tmp \hastype \mtype$. Hence we can construct the appropriate derivation $\typederp$ for $\tmp\esub\varthree\tmtwop$.
		\[\infer[\typingruleES]{\typectx \types \tmp\esub\varthree\tmtwop \hastype \mtype}{\typectx_1, \varthree \hastype \mtypetwo \types \tmp \hastype \mtype & \typectx_2 \types \tmtwop \hastype \mtypetwo}\]
		
		
		Hence $\tm\esub\varthree\tmtwo\leqtype \tmp\esub\varthree\tmtwop$.
		
		\item \emph{($\alpha$-equivalence)} and \emph{($\alpha$-equivalence')} are obvious, because typing judgments do not depend on the representation of bounded variables.\qedhere
	\end{itemize}
\end{proof}







\gettoappendix{prop:type-preorder-is-compatible}

\begin{proof}
	\begin{enumerate}
		\item By induction on $\ctx$.
	\begin{itemize}
		\item $\ctx = \ctxhole$, $\tm\leqtype\tmp$.
		\item $\ctx \leqtype \tmtwo\ctxtwo$, then by induction ${\ctxtwop\tm} \leqtype{\ctxtwop\tmp}$ and obviously $\tmtwo \leqtype \tmtwo$, hence by the (applicative) property ${\tmtwo\ctxtwop\tm} \leqtype{\tmtwo\ctxtwop\tmp}$.
		\item $\ctx = \ctxtwo\tmtwo$, $\ctx \leqtype \ctxtwo\esub\var\tmtwo$ and $\ctx \leqtype \tmtwo\esub\var\ctxtwo$ are similar to the previous case.
		\item $\ctx = \la\var\ctxtwo$ then by induction ${\ctxtwop\tm} \leqtype{\ctxtwop\tmp}$, hence by the (abstractive) property ${\la\var\ctxtwop\tm} \leqtype{\la\var\ctxtwop\tmp}$.
	\end{itemize}
\item By compatibility and adequacy.\qedhere

\end{enumerate}
\end{proof}

\subsection{\Net similarity is included into the Type preorder}
In fact, for all \nafex similarities, such that the following propoposition is true (and not necessarily that $\equivx$ is a mirror), we get that $\leqnafex \subseteq \leqtype$. The only prerequisite is that $\equivx$ satisfies the following proposition.

\begin{proposition}[$\equivx$-equivalence implies typability-equivalence]
	\label{prop:equivx-subseteq-equivtype}
	If $\tm\equivx\tmtwo$ then $\tm\equivtype\tmtwo$.
\end{proposition}

For $\leqtype$, this proposition is true, see Theorem \ref{thm:invariance-and-adequacy}.

\gettoappendix{l:bisimulation-preserves-typeder}


\begin{proof}
	
	\begin{enumerate}
		\item By induction on the size of the derivation $\typeder: \typectx \types \tm \hastype \mtype$.
	
	
	The term $\tm$ is typable by a derivation $\typeder: \typectx \types \tm \hastype \mtype$ therefore it is normalizable by \refthm{invariance-and-adequacy}. Hence we have $\tm\tovsc^k\ntm$ and therefore (since $\relsym$ is a bisimulation) $\tmp\tovsc^*\ntmtwo$ with $\ntm \relnafex \ntmtwo$. Instead of looking for a derivation $\typederp$  of $\tmp$, we can look for a derivation $\typederp_1$ of $\ntmtwo$ and conclude by (typability) expansion for the $\tovsc$ reduction.
	
	There is a derivation $\typeder_1: \typectx \types \ntm \hastype \mtype$ whose size is at most the size of $\typeder$.
	
	By case analysis on the last rule of the derivation $\typeder_1$.
	
	\begin{enumerate}
		\item \emph{Axiom rule.} \[\typeder_1 :~~~~~ \infer[\typingruleAx]{\var \hastype [\ltype] \types \ntm = \var \hastype \ltype}{}\]
		
		Then by $\ntm=\var\relnafex \ntmtwo$, $\ntmtwo = \var$ and $\typederp_1 \defeq \typeder_1$ types $\ntmtwo$ accordingly.
		\item \emph{Abstraction rule.} \[\typeder_1 :~~~~~ \infer[\typingruleAbs]{\typectx \types \ntm = \la\var\tmtwo \hastype \mtype \multimap \mtypetwo}{\typectx, \var \hastype \mtype \types  \tmtwo \hastype \mtypetwo}\]
		
		Then by $\ntm=\la\var\tmtwo\relnafex \ntmtwo$, $\ntmtwo = \la\var\tmtwop$ with $\tmtwo\rel\tmtwop$.
		
		The derivation $\typeder_2 : \typectx, \var \hastype \mtype \types  \tmtwo \hastype \mtypetwo$ is of a strictly smaller size than $\typeder$. By induction, since $\tmtwo\rel\tmtwop$, there is a derivation $\typederp_2 : \typectx, \var \hastype \mtype \types  \tmtwop \hastype \mtypetwo$.
		
		Then, \[\typederp_1 :~~~~~ \infer[\typingruleAbs]{\typectx \types \ntmtwo = \la\var\tmtwop \hastype \mtype \multimap \mtypetwo}{\infer*{\typectx, \var \hastype \mtype \types  \tmtwop \hastype \mtypetwo}{\typederp_2}}\]
		
		\item \emph{Many rule.} \[\typeder_1 : ~~~~~
		\infer[\typingruleMany]{\biguplus_{i\in I} \typectx_i \types \ntm = \val \hastype \biguplus_{i\in I} \ltype_i}{\infer*{(\typectx_i \types \ntm= \val \hastype \ltype_i)_{i\in I}}{\typedertwo_i}  & I~ \text{finite} } \]
		
		Then by $\ntm=\val\relnafex \ntmtwo$, $\ntmtwo = \valtwo$ with $\val\relnafex\valtwo$.
		
		Two sub-cases depending on the value nature of $\val$:
		\begin{itemize}
			\item \emph{Variable}. If $\val=\var$ then, $\valtwo=\var$ as well.
			Then, $\typederp_1\defeq\typeder_1$ is a correct derivation for $\valtwo$ and concludes the proof in this case.  
			\item \emph{Abstract}. If $\val=\la\var\tmtwo$ then, $\valtwo=\la\var\tmtwop$ with $\tmtwo\rel\tmtwop$.
			
			Suppose there is at least a $\typedertwo_i$ derivation (if there are none the result is trivial).
			
			Since $\ltype_i$ is a linear type the only possibility for the last rule of $\typedertwo_i$ is a ($\typingruleAbs$) rule. 
			
			Suppose $\ltype_i = \mtype_i\multimap \mtypetwo_i$.
			\[{\typedertwo_i} :~~~~~ \infer[\typingruleAbs]{\typectx \types \la\var\tmtwo \hastype \mtype_i\multimap \mtypetwo_i}{\infer*{\typectx, \var \hastype \mtype_i\types \tmtwo \hastype\mtypetwo_i}{\typederthree_i}}\]
			
			We know that $\tmtwo\rel\tmtwop$. By \ih on $\typederthree_i$ (whose size is strictly smaller than the size of $\typeder$), we get $\typederthreep_i : \typectx, \var \hastype \mtype_i\types \tmtwop \hastype\mtypetwo_i$. Hence we can reconstruct the appropriate $\typederp_1$ derivation.
			\[\typederp_1 : ~~~~~
			\infer[\typingruleMany]{\biguplus_{i\in I} \typectx_i \types \ntmtwo = \valtwo \hastype \biguplus_{i\in I} \ltype_i}{(\infer{\typectx_i \types \valtwo = \la\var\tmtwop \hastype \ltype_i}{{\infer*{\typectx, \var \hastype \mtype_i\types \tmtwo \hastype\mtypetwo_i}{\typederthreep_i}}})_{i\in I}  & I~ \text{finite} } \]
		\end{itemize}
		
		
		
		
		\item \emph{Application rule.} \[	\infer[\typingruleApp]{\typectx \uplus \typectxtwo \types \ntm=\ntmONE\ntmTWO \hastype \mtypetwo}{ \typectx \types \ntmONE \hastype [\mtype \multimap \mtypetwo] & \typectxtwo \types \ntmTWO \hastype \mtype }
		\]
		
		Then by $\ntm=\ntmONE\ntmTWO\relnafex \ntmtwo$, $\ntmtwo \equivx \ntmONEtwo\ntmTWOtwo$ with $\ntmONE\rel\ntmONEtwo$ and $\ntmTWO\rel\ntmTWOtwo$.
		
		By \refprop{equivx-subseteq-equivtype}, $\ntmtwo$ is type equivalent with $\ntmONEtwo\ntmTWOtwo$. Hence it is enough to construct an appropriate $\typederp$ for $\ntmONEtwo\ntmTWOtwo$.
		
		By induction on $\ntmONE,\ntmONEtwo$ and $\ntmTWO,\ntmTWOtwo$, we get the appropriate derivation.
		
		\item \emph{Explicit Substitution rule.} \[
		\infer[\typingruleES]{\typectx \uplus \typectxtwo \types \ntm = \ntmONE\esub\var\ntmTWO \hastype \mtypetwo}{ \typectx, \var \hastype \mtype \types \ntmONE \hastype \mtypetwo & \typectxtwo \types \ntmTWO \hastype \mtype }\]
		
		Then by $\ntm=\ntmONE\esub\var\ntmTWO\relnafex \ntmtwo$, $\ntmtwo \equivx \ntmONEtwo\esub\var\ntmTWOtwo$ with $\ntmONE\rel\ntmONEtwo$ and $\ntmTWO\rel\ntmTWOtwo$.
		
		By \refprop{equivx-subseteq-equivtype}, $\ntmtwo$ is type equivalent with $\ntmONEtwo\esub\var\ntmTWOtwo$. Hence it is enough to construct an appropriate $\typederp$ for $\ntmONEtwo\esub\var\ntmTWOtwo$.
		
		By induction on $\ntmONE,\ntmONEtwo$ and $\ntmTWO,\ntmTWOtwo$, we get the appropriate derivation.
	\end{enumerate}
\item 
Let $\tm,\tmp$ terms such that $\tm \leqnafex\tmp$.
By the first part of the proposition, $\tm\leqtype\tmp$.\qedhere
	\end{enumerate}
\end{proof}



\subsection{Lassen's Enf similarity is included in the Type preorder}
For Lassen's bisimilarity, the proof is a little more intricated, but relies on the same reasoning. We consider Plotkin's normal form and type them using the VSC multi types. Since multi types are invariant by reduction, this does not affect type equivalence.

\gettoappendix{l:smaller-derivations-stuck}


\begin{proof}
	By induction on $\levctx$.
	\begin{itemize}
		\item $\levctx = \ctxhole$
		\[\typeder =~~~	\infer[\typingruleApp]{\typectx \uplus \typectxtwo \types \var\val \hastype \mtype}{ \typectx \types \var \hastype [\mtypetwo_1 \multimap \mtype] & \infer*{\typectxtwo \types \val \hastype \mtypetwo_1}{\typeder_{\val}} }
		\]
		
		Then, $\typeder_{\levctx}$ is of size $1$, \ie clearly smaller than $\typeder$, and it is easy to see that $\typeder_{\val}$ is of size $\size{\typeder}-2$.
		
		\item $\levctx= \valtwo\levctxtwo$
		\[\typeder =~~~	\infer[\typingruleApp]{\typectx \uplus \typectxtwo \types \levctxp{\var\val}=\valtwo\levctxtwop{\var\val} \hastype \mtype}{ \typectx \types \valtwo \hastype [\mtype_1 \multimap \mtype] & \infer*{\typectxtwo \types \levctxtwop{\var\val} \hastype \mtype_1}{\typeder_1} }
		\]
		
		By \ih, $\exists \typeder_{1\levctx} :  \typectxtwo_{a}, \varthree \hastype \mtypetwo \types  \levctxtwop{\varthree} \hastype \mtype_1$ and $\exists \typeder_{\val} : \typectx_{\val} \types \val : \mtypetwo_1$ with $\size{\typeder_{1\levctx}}<\size{\typeder_1}$ and $\size{\typeder_{\val}}<\size{\typeder_1}$.
		
		Therefore $\size{\typeder_{\val}}<\size{\typeder}$. We complete $\typeder_{\levctx}$ in the following way:
		\[\typeder_{\levctx} =~~~	\infer[\typingruleApp]{\typectx \uplus \typectxtwo_{\levctx},\varthree \hastype \mtypetwo \types \levctxp\varthree=\valtwo\levctxtwop{\varthree} \hastype \mtype}{ \typectx \types \valtwo \hastype [\mtype_1 \multimap \mtype] & \infer*{\typectxtwo_{\levctx},\varthree \hastype \mtypetwo \types \levctxtwop{\varthree} \hastype \mtype_1}{\typeder_{1\levctx}} }
		\]
		
		Then $\size{\typeder_{\levctx}}<\size\typeder$.
		
		\item $\levctx=\levctxtwo\tm$
		\[\typeder =~~~	\infer[\typingruleApp]{\typectx \uplus \typectxtwo \types \levctxtwop{\var\val}\tm \hastype \mtype}{    \infer*{\typectxtwo \types \levctxtwop{\var\val} \hastype [\mtype_1 \multimap \mtype]}{\typeder_1} &\typectx \types \tm \hastype \mtype_1}
		\]
		
		By \ih, $\exists \typeder_{1\levctx} :  \typectxtwo_{a}, \varthree \hastype \mtypetwo \types  \levctxtwop{\varthree} \hastype [\mtype_1 \multimap \mtype]$ and $\exists \typeder_{\val} : \typectx_{\val} \types \val : \mtypetwo_1$ with $\size{\typeder_{1\levctx}}<\size{\typeder_1}$ and $\size{\typeder_{\val}}<\size{\typeder_1}$.
		
		Therefore $\size{\typeder_{\val}}<\size\typeder$. We complete $\typeder_{\levctx}$ in the following way:
		\[\typeder_{\levctx} =~~~	\infer[\typingruleApp]{\typectx \uplus \typectxtwo_{\levctx},\varthree\hastype\mtypetwo \types \levctxtwop{\varthree}\tm=\levctxp\varthree \hastype \mtype}{    \infer*{\typectxtwo_{\levctx},\varthree\hastype\mtypetwo \types \levctxtwop{\varthree} \hastype [\mtype_1 \multimap \mtype]}{\typeder_{1\levctx}} &\typectx \types \tm \hastype \mtype_1}
		\]
		
		Then $\size{\typeder_{\levctx}}<\size\typeder$.\qedhere
	\end{itemize}
\end{proof}

\begin{proposition}
	\label{prop:stuck-typed-applicative-form}
	${\levctxp{\var\val}}\equivtype{(\la\varthree\levctxp{\varthree})(\var\val)}$ where $\varthree$ does not appear in $\levctx$.
\end{proposition}

\begin{proof}
	Similar arguments apply, proof by induction on $\levctx$.
\end{proof}





\gettoappendix{l:enf-bisimulation-preserves-typeder}


\begin{proof}
\begin{enumerate}
	\item 
	By induction on the size of the derivation $\typeder: \typectx \types \tm \hastype \mtype$.
	
	\newcommand{\ntmleft}{\ntm_{\mathsf{left}}}
	\newcommand{\ntmlefttwo}{\ntmtwo_{\mathsf{left}}}
	The term $\tm$ is typable by a derivation $\typeder: \typectx \types \tm \hastype \mtype$ therefore it is normalizable by \refthm{invariance-and-adequacy}. Hence we have $\tm\tovsc^k\ntm$. By the fact that all VSC terminating terms are Plotkin's left terminating, we have that $\tm\tolw\ntmleft$ and therefore (since $\relsym$ is an enf bisimulation) $\tmp\tolw^*\ntmlefttwo$ with $\ntmleft \relenf\ntmlefttwo$. Instead of looking for a derivation $\typederp$  of $\tmp$, we can look for a derivation $\typederp_1$ of $\ntmlefttwo$ and conclude by (typability) expansion for the $\tolw$ reduction.
	
	There is a derivation $\typeder_1: \typectx \types \ntmleft \hastype \mtype$ whose size is at most the size of $\typeder$.
	
	By case analysis on the last rule of the derivation $\typeder_1$.
	
	\begin{enumerate}
		\item \emph{Axiom rule.} \[\typeder_1 :~~~~~ \infer[\typingruleAx]{\var \hastype [\ltype] \types \ntmleft = \var \hastype \ltype}{}\]
		
		Then by $\ntmleft=\var\relenf \ntmlefttwo$, $\ntmlefttwo = \var$ and $\typederp_1 \defeq \typeder_1$ types $\ntmlefttwo$ accordingly.
		\item \emph{Abstraction rule.} \[\typeder_1 :~~~~~ \infer[\typingruleAbs]{\typectx \types \ntmleft = \la\var\tmtwo \hastype \mtype \multimap \mtypetwo}{\typectx, \var \hastype \mtype \types  \tmtwo \hastype \mtypetwo}\]
		
		Then by $\ntmleft=\la\var\tmtwo\relenf \ntmtwo$, $\ntmlefttwo = \la\var\tmtwop$ with $\tmtwo\rel\tmtwop$.
		
		The derivation $\typeder_2 : \typectx, \var \hastype \mtype \types  \tmtwo \hastype \mtypetwo$ is of a strictly smaller size than $\typeder$. By induction, since $\tmtwo\rel\tmtwop$, there is a derivation $\typederp_2 : \typectx, \var \hastype \mtype \types  \tmtwop \hastype \mtypetwo$.
		
		Then, \[\typederp_1 :~~~~~ \infer[\typingruleAbs]{\typectx \types \ntmlefttwo = \la\var\tmtwop \hastype \mtype \multimap \mtypetwo}{\infer*{\typectx, \var \hastype \mtype \types  \tmtwop \hastype \mtypetwo}{\typederp_2}}\]
		
		\item \emph{Many rule.} \[\typeder_1 : ~~~~~
		\infer[\typingruleMany]{\biguplus_{i\in I} \typectx_i \types \ntmleft = \val \hastype \biguplus_{i\in I} \ltype_i}{\infer*{(\typectx_i \types \ntmleft= \val \hastype \ltype_i)_{i\in I}}{\typedertwo_i}  & I~ \text{finite} } \]
		
		Then by $\ntmleft=\val\relenf \ntmtwo$, $\ntmlefttwo = \valtwo$ with $\val\relenf\valtwo$.
		
		Two sub-cases depending on the value nature of $\val$:
		\begin{itemize}
			\item \emph{Variable}. If $\val=\var$ then, $\valtwo=\var$ as well.
			Then, $\typederp_1\defeq\typeder_1$ is a correct derivation for $\valtwo$ and concludes the proof in this case.  
			\item \emph{Abstract}. If $\val=\la\var\tmtwo$ then, $\valtwo=\la\var\tmtwop$ with $\tmtwo\rel\tmtwop$.
			
			Suppose there is at least a $\typedertwo_i$ derivation (if there are none the result is trivial).
			
			Since $\ltype_i$ is a linear type the only possibility for the last rule of $\typedertwo_i$ is a ($\typingruleAbs$) rule. 
			
			Suppose $\ltype_i = \mtype_i\multimap \mtypetwo_i$.
			\[{\typedertwo_i} :~~~~~ \infer[\typingruleAbs]{\typectx \types \la\var\tmtwo \hastype \mtype_i\multimap \mtypetwo_i}{\infer*{\typectx, \var \hastype \mtype_i\types \tmtwo \hastype\mtypetwo_i}{\typederthree_i}}\]
			
			We know that $\tmtwo\rel\tmtwop$. By \ih on $\typederthree_i$ (whose size is strictly smaller than the size of $\typeder$), we get $\typederthreep_i : \typectx, \var \hastype \mtype_i\types \tmtwop \hastype\mtypetwo_i$. Hence we can reconstruct the appropriate $\typederp_1$ derivation.
			\[\typederp_1 : ~~~~~
			\infer[\typingruleMany]{\biguplus_{i\in I} \typectx_i \types \ntmlefttwo = \valtwo \hastype \biguplus_{i\in I} \ltype_i}{(\infer{\typectx_i \types \valtwo = \la\var\tmtwop \hastype \ltype_i}{{\infer*{\typectx, \var \hastype \mtype_i\types \tmtwo \hastype\mtypetwo_i}{\typederthreep_i}}})_{i\in I}  & I~ \text{finite} } \]
		\end{itemize}
		
		

		
		\item \emph{Application rule.} \[	\infer[\typingruleApp]{\typectx \uplus \typectxtwo \types \ntmleft=\tmrone\tmrtwo \hastype \mtypetwo}{ \typectx \types \tmrone \hastype [\mtype \multimap \mtypetwo] & \typectxtwo \types \tmrtwo \hastype \mtype }
		\]
		
		Then by $\ntmleft=\tmrone\tmrtwo\relenf \ntmtwo$, $\ntmleft=\levctxp{\var\val}$ and $\ntmtwo = \levctxtwop{\var\valtwo}$ with $\levctxp\varthree\rel\levctxp\varthree$ and $\val\rel\valtwo$ with $\varthree$ fresh.
		
		By \refprop{stuck-typed-applicative-form}, typability of $\ntmleft$ is equivalent to typability of $(\la\varthree\levctxp{\varthree})(\var\val)$, which the derivation (for this type) is:
		\[	\infer[\typingruleApp]{\typectx \uplus \typectxtwo \types (\la\varthree\levctxp{\varthree})(\var\val) \hastype \mtypetwo}{ \infer[\typingruleAbs]{\typectx \types \la\varthree\levctxp{\varthree} \hastype [\mtype \multimap \mtypetwo]}{\infer*{\typectx, \varthree \hastype \mtype \types  \levctxp\varthree \hastype \mtypetwo}{\typeder_{\levctx}}} & \infer[\typingruleApp]{\typectxtwo',\var \hastype [\mtype_1 \multimap \mtype] \types \var\val \hastype \mtype}{\var \hastype [\mtype_1 \multimap \mtype] \types \var \hastype [\mtype_1 \multimap \mtype] & \infer*{\typectxtwo' \types \val \hastype \mtype_1}{\typeder_{\val}}} }
		\]
		By \reflemma{smaller-derivations-stuck}, $\typeder_{\levctx}$ and $\typeder_{\val}$ are strictly smaller than $\typeder$. Hence, by \ih, there exists $\typederp_{\levctx}$ and $\typederp_{\val}$ such that we can build the following derivation tree:
			\[\typederp=~~~	\infer[\typingruleApp]{\typectx \uplus \typectxtwo \types (\la\varthree\levctxtwop{\varthree})(\var\valtwo) \hastype \mtypetwo}{ \infer[\typingruleAbs]{\typectx \types \la\varthree\levctxtwop{\varthree} \hastype [\mtype \multimap \mtypetwo]}{\infer*{\typectx, \varthree \hastype \mtype \types  \levctxtwop\varthree \hastype \mtypetwo}{\typeder_{\levctx}}} & \infer[\typingruleApp]{\typectxtwo',\var \hastype [\mtype_1 \multimap \mtype] \types \var\valtwo \hastype \mtype}{\var \hastype [\mtype_1 \multimap \mtype] \types \var \hastype [\mtype_1 \multimap \mtype] & \infer*{\typectxtwo' \types \valtwo \hastype \mtype_1}{\typeder_{\val}}} }
		\]
		
		Which concludes the proof, since by \refprop{stuck-typed-applicative-form}, $(\la\varthree\levctxtwop{\varthree})(\var\valtwo)$ and $\levctxtwop{\var\valtwo}$ are type equivalent.
		
		\item \emph{Explicit Substitution rule.} \[
		\infer[\typingruleES]{\typectx \uplus \typectxtwo \types \ntmleft = \ntmONE\esub\var\ntmTWO \hastype \mtypetwo}{ \typectx, \var \hastype \mtype \types \ntmONE \hastype \mtypetwo & \typectxtwo \types \ntmTWO \hastype \mtype }\]
		This case is not possible: $\ntmleft$ is a term without explicit substitutions.
	\end{enumerate}
\item 
Let $\tm,\tmp$ terms such that $\tm \leqenf\tmp$.
By the first part, $\tm \leqtype\tmp$.\qedhere
\end{enumerate}
\end{proof}


\subsection{Multi types by value, regarding $\eta_v$ equivalence}
We now prove that multi types as they are defined in this paper, validate $\eta_v$ equivalence. We only need to prove the result for variables, as $\eta_v$ equivalence on abstractions is contained in $\equivbetav$.

\ignore{\subsubsection{Validating $\eta_v$ preorder}

\gettoappendix{prop:etav-for-leqtype}

\begin{proof}\begin{enumerate}
		\item \begin{itemize}
		\item 
	Let $(\typectx,\mtype)$ and $\typeder$ such that $\typeder : \typectx \types \la\vartwo\var\vartwo \hastype \mtype$.
	
	It is easy to type $\var$ with the multi type $\mtype$, with the context $\typectxtwo = \var \hastype \mtype$. We then need to prove that $\typectx = \typectxtwo$ to conclude.
	
	By unfolding the derivation $\typeder$, we get:
	
	
	Let $n$, $(\ltype_i)_{1\leq i\leq n}$ such that $\mtype = \multitype{n}{\ltype}$. Let $\mtypetwo_i$ and $\mtypetwo'_i$ such that $\ltype_i = \mtypetwo_i \multimap \mtypetwo'_i$.
	Let $m_i$, $((\ltypetwo_{i,j})_{1\leq j\leq m_i})_{0 \leq i \leq n}$ such that $\mtypetwo_i = \multitype{m_i}{\ltypetwo_{i,}}$.
	
	\[\infer[\typingruleMany]{\biguplus_{0 \leq i\leq n}\typectx_i \types \la\vartwo\var\vartwo \hastype \mtype}{\ldots & \infer[\typingruleAbs]{\typectx_i \types \la\vartwo\var\vartwo \hastype \ltype_i}{\infer[\typingruleApp]{\typectx_i,\vartwo \hastype \biguplus_{0 \leq j\leq m}[\ltypetwo_{i,j}] \types \var\vartwo \hastype \mtypetwo'_i}{\typectx_i \types \var \hastype [\biguplus_{0 \leq j\leq m}[\ltypetwo_{i,j}] \multimap \mtypetwo'_i] & \infer[\typingruleMany]{\vartwo \hastype \biguplus_{0 \leq j\leq m}[\ltypetwo_{i,j}] \types \vartwo \hastype \biguplus_{0 \leq j\leq m}\ltypetwo_{i,j}}{\left(\infer[\typingruleAx]{\vartwo \hastype [\ltypetwo_{i,j}] \types \vartwo \hastype \ltypetwo_{i,j}}{}\right)_{0\leq j \leq m}}}} & \ldots ~{0 \leq i \leq n}}\]
	
	Hence, if we keep unfolding on $\typectx_i \types \var \hastype [\biguplus_{0 \leq j\leq m}[\ltypetwo_{i,j}] \multimap \mtypetwo'_i]$, we get that $\biguplus_{0 \leq i\leq n}\typectx_i = \var \hastype \mtype$.
	\item $\var \hastype [\vartype] \types \var \hastype [\vartype]$ but $\var \hastype [\vartype] \not \types \la\vartwo\var\vartwo \hastype [\vartype]$, as an abstraction cannot be typed without a linear type of the shape $\mtype \multimap \mtypetwo$.
	\end{itemize}
\item For abstractions, the result is easier as $\eta_v$ equivalence amounts then to reducing under lambdas and $\alpha$-equivalence.
	\end{enumerate}
\end{proof}

\subsubsection{Removing ground types, validating $\eta_v$ equivalence}

Consider the same multi type system but with a new grammar of types, removing ground types (the empty multi type $\emptytype$ can be seen as acting as a ground type):


\begin{center}
	$\begin{array}{ccccc}
	\textsc{Linear Types} & \ltype, \ltypetwo &\grameq& \mtype \multimap \mtypetwo
	\\
	\textsc{Multi Types} & \mtype, \mtypetwo &\grameq& \multitype{n}{\ltype} & n\geq 0
	\end{array}$
\end{center}
}

\gettoappendix{prop:etav-for-leqtypetwo}

\begin{proof}
\hfill
\begin{enumerate}
		
		\item \emph{$\var \leqtype \la\vartwo\var\vartwo$}
		
		Let $\typeder$ be a type derivation such that $\typeder : \typectx \types \var \hastype \mtype$.
		
		We show that $\la\vartwo\var\vartwo$ can be typed in the same context and with the same type.
		
		Let $n$, $(\ltype_i)_{1\leq i\leq n}$ such that $\mtype = \multitype{n}{\ltype}$. Let $\mtypetwo_i$ and $\mtypetwo'_i$ such that $\ltype_i = \mtypetwo_i \multimap \mtypetwo'_i$.
		Let $m_i$, $((\ltypetwo_{i,j})_{1\leq j\leq m_i})_{0 \leq i \leq n}$ such that $\mtypetwo_i = \multitype{m_i}{\ltypetwo_{i,}}$.
		
		\[\infer[\typingruleMany]{\biguplus_{0 \leq i\leq n}\typectx_i \types \la\vartwo\var\vartwo \hastype \mtype}{\ldots & \infer[\typingruleAbs]{\typectx_i \types \la\vartwo\var\vartwo \hastype \ltype_i}{\infer[\typingruleApp]{\typectx_i,\vartwo \hastype \biguplus_{0 \leq j\leq m}[\ltypetwo_{i,j}] \types \var\vartwo \hastype \mtypetwo'_i}{\typectx_i \types \var \hastype [\biguplus_{0 \leq j\leq m}[\ltypetwo_{i,j}] \multimap \mtypetwo'_i] & \infer[\typingruleMany]{\vartwo \hastype \biguplus_{0 \leq j\leq m}[\ltypetwo_{i,j}] \types \vartwo \hastype \biguplus_{0 \leq j\leq m}\ltypetwo_{i,j}}{\left(\infer[\typingruleAx]{\vartwo \hastype [\ltypetwo_{i,j}] \types \vartwo \hastype \ltypetwo_{i,j}}{}\right)_{0\leq j \leq m}}}} & \ldots ~{0 \leq i \leq n}}\]
		
		It only remains to show that $\typectx = \var \hastype \biguplus_{0 \leq i\leq n} [\biguplus_{0 \leq j\leq m}[\ltypetwo_{i,j}] \multimap \mtypetwo'_i]$ and then we are done.
		
		Let's unfold the derivation $\typeder : \typectx \types \var \hastype \mtype$:
		\[\infer[\typingruleMany]{\biguplus_{0 \leq i\leq n}\typectxtwo_i \types \var \hastype \mtype}{(\infer[\typingruleAx]{\typectxtwo_i \types \var \hastype \mtypetwo_i \multimap \mtypetwo'_i}{})_{0 \leq i \leq n}}\]
		
		Hence $\typectxtwo_i = \var \hastype [\mtypetwo_i \multimap \mtypetwo'_i]$, which concludes the proof.
		
		\item \emph{$\la\vartwo\var\vartwo \leqtype \var$}, 
		
		Let $(\typectx,\mtype)$ and $\typeder$ such that $\typeder : \typectx \types \la\vartwo\var\vartwo \hastype \mtype$.
		
		It is easy to type $\var$ with the multi type $\mtype$, with the context $\typectxtwo = \var \hastype \mtype$. We then need to prove that $\typectx = \typectxtwo$ to conclude.
		
		By unfolding the derivation $\typeder$, we get:
		
		
		Let $n$, $(\ltype_i)_{1\leq i\leq n}$ such that $\mtype = \multitype{n}{\ltype}$. Let $\mtypetwo_i$ and $\mtypetwo'_i$ such that $\ltype_i = \mtypetwo_i \multimap \mtypetwo'_i$.
		Let $m_i$, $((\ltypetwo_{i,j})_{1\leq j\leq m_i})_{0 \leq i \leq n}$ such that $\mtypetwo_i = \multitype{m_i}{\ltypetwo_{i,}}$.
		\[\infer[\typingruleMany]{\biguplus_{0 \leq i\leq n}\typectx_i \types \la\vartwo\var\vartwo \hastype \mtype}{\ldots & \infer[\typingruleAbs]{\typectx_i \types \la\vartwo\var\vartwo \hastype \ltype_i}{\infer[\typingruleApp]{\typectx_i,\vartwo \hastype \biguplus_{0 \leq j\leq m}[\ltypetwo_{i,j}] \types \var\vartwo \hastype \mtypetwo'_i}{\typectx_i \types \var \hastype [\biguplus_{0 \leq j\leq m}[\ltypetwo_{i,j}] \multimap \mtypetwo'_i] & \infer[\typingruleMany]{\vartwo \hastype \biguplus_{0 \leq j\leq m}[\ltypetwo_{i,j}] \types \vartwo \hastype \biguplus_{0 \leq j\leq m}\ltypetwo_{i,j}}{\left(\infer[\typingruleAx]{\vartwo \hastype [\ltypetwo_{i,j}] \types \vartwo \hastype \ltypetwo_{i,j}}{}\right)_{0\leq j \leq m}}}} & \ldots ~{0 \leq i \leq n}}\]
		
		Hence, if we keep unfolding on $\typectx_i \types \var \hastype [\biguplus_{0 \leq j\leq m}[\ltypetwo_{i,j}] \multimap \mtypetwo'_i]$, we get that $\biguplus_{0 \leq i\leq n}\typectx_i = \var \hastype \mtype$.\qedhere
		\end{enumerate}
\end{proof}


%\input{example-cbn-duplication-not-validated-by-type-equivalence}

\end{document}
\endinput
%%
%% End of file `sample-acmsmall.tex'.
