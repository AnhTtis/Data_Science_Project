% !TEX root = main.tex
\section{Preliminaries}
\paragraph{Contexts} All along the
paper we use (many notions of) \emph{contexts}, \ie terms with exactly one hole, noted $\ctxhole$. Plugging a term $\tm$ in a context $\ctx$, noted $\ctxp{\tm}$, possibly~captures free variables of $\tm$. For instance $(\la\var\ctxhole)\ctxholep\var = \la\var\var$, while $(\la\var\vartwo)\isub\vartwo\var = \la\varthree\var$.


\paragraph{Preorders} We shall mostly deal with simulations, rather than bisimulations, since the result for equivalences shall always follow by simply considering the symmetric notions. Given a preorder/similarity $\precsim_X$ for some $X$, we  denote with $\simeq_X$ the corresponding equivalence/bisimilarity.

\paragraph{(In)Equational Theories and Compatibility} Good program preorder/equivalences are \emph{(in)equational theories}, that is, they contain the reduction of the calculus and they are \emph{compatible}, defined as: if $\tm\precsim\tmp$ then $\ctxp\tm \precsim\ctxp\tmp$ for all contexts $\ctx$, that is, that they are stable by context closure. Reduction is usually trivially included in similarities while compatibility is usually non-trivial to prove.


\paragraph{Contextual Equivalence} The standard of reference for program equivalences is contextual equivalence, that can be defined abstractly as follows.
\begin{definition}[Contextual Preorder and Equivalence] Given a language of terms $\mathcal{T}$ with its associated notion of contexts $\ctx$ and predicate stating the termination of evaluation $\bs{}_{e}$, we define the associated \emph{contextual preorder} $\leqcp{e}$ and \emph{contextual equivalence} $\equivcp{e}$ as follows:
	\begin{itemize}
		\item $\tm \leqcp{e} \tmp$ if $\ctxp\tm ~{\bs{}_{e}}$ implies $\ctxp\tmp ~{\bs{}_{e}}$ for all contexts $\ctx$ such that $\ctxp{\tm}$ and $\ctxp\tmp$ are closed terms. 
		\item $\tm \equivcp{e} \tmp$ is the equivalence relation induced by $\leqc$, that is, $\tm \equivcp{e} \tmp \iff \tm \leqcp{e} \tmp$ and $\tmp \leqcp{e} \tm$.
	\end{itemize}
\end{definition}


A relation $\relsym$ is sound for the contextual preorder $\leqcp{e}$ when $\relsym \subseteq \leqcp{e}$. Soundness follows from compatibility and \emph{adequacy for $\bs{}_e$}, defined as: if $\tm\rel\tmp$  then $\tm \bs{}_e$ implies $\tmp \bs{}_e$.


\begin{proposition}
	\label{prop:congruence-included-contextual-equivalence}
	Let $\precsim$ be a compatible and adequate preorder. Then $\precsim \subseteq \leqcp{e}$.
\end{proposition}

\begin{proof}
	Suppose $\tmtwo \precsim \tmtwop$.
	Let $\ctx$ any closing context for $\tmtwo$ and $\tmtwop$.
	By compatibility, $\ctxp{\tmtwo} \precsim \ctxp{\tmtwop}$.
	By adequacy, $\ctxp{\tmtwo} \bs{}_e$ implies $\ctxp{\tmtwop} \bs{}_e$, that is, $\tmtwo \leqcp e \tmtwop$.
\end{proof}

We shall see that normal form simulations are defined in such a way that they are adequate, so that soundness follows directly from compatibility. Note that soundness without compatibility is useless: the relation $\relsym \defeq \set{(\var\vartwo,\var\vartwo)}$ is sound but not compatible. 

\paragraph{Diamond} A rewriting notion that shall play a role is the \emph{diamond property}, which is the following one-step strengthening of confluence for a reduction $\to$: if $\tm \to \tmtwo_1$, $\tm\to\tmtwo_2$, and $\tmtwo_1 \neq \tmtwo_2$ then exists $\tmthree$ such that $\tmtwo_1 \to \tmthree$ and $\tmtwo_2 \to \tmthree$. Some well-known facts: the diamond property implies confluence but not vice-versa; if $\to$ is diamond and there is a terminating reduction from $\tm$ then there are no diverging reductions from $\tm$; all reductions to normal form, if any, have the same length. Roughly, the diamond property is a relaxed form of determinism, where non-deterministic choices have no impact on the result nor on the length of the evaluation leading to it.

\paragraph{Solvability and Scrutability} A cornerstone of the theory of the (call-by-name) $\l$-calculus is the study of  \emph{meaningless terms}, which can be used to model the \emph{undefined} in the representation of partial recursive functions in the $\l$-calculus. It is well-known that one cannot define meaningless terms as the $\beta$-diverging ones, as this induces an inconsistent theory. One rather needs to define them as the \emph{unsolvable terms}, which is a notion first studied by \citet{Wad:SemPra:71,DBLP:journals/siamcomp/Wadsworth76} and \citet{DBLP:books/daglib/0016519,solvability-barendregt} . We omit the definition of unsolvable terms, but mention that, as proved by Wadsworth, they can be characterized as those terms that diverge with respect to head reduction. An alternative notion of meaningless term is given by scrutable terms.

\begin{definition}[Testing contexts and (in)scrutability]
\label{def:cbn-scrutability}
Testing context are defined by:
\begin{center}
\textsc{Testing contexts}  \ \ \ $\tctx \grameq \ctxhole \mid \tctx \tm \mid (\la\var\tctx)\tm$
\end{center}
A term $\tm$ is \emph{\cbn scrutable} if there is a testing context $\tctx$ and a value $\val$ such that $\tctxp\tm \tob^* \val$, that is, a variable or an abstraction, otherwise $\tm$ is \emph{\cbn inscrutable}.
\end{definition}

For instance, $\Omega$ is inscrutable but $\la\var\Omega$ and $\var\Omega$ are scrutable. Inscrutable terms are the minimal terms with respect to the \cbn contextual preorder (with respect to termination on weak head normal forms). There is a useful characterization of scrutable terms that avoids mentioning testing contexts, due to \citet{parametricBook}, who adapt Wadsworth's for solvability. 
\begin{theorem}[Characterization of \cbn scrutability]
\label{thm:cbn-scrutability-characterization}
A term $\tm$ is scrutable if and only if the weak head reduction of $\tm$ terminates.
\end{theorem}

From the characterization, it follows that every inscrutable term is unsolvable, but not vice-versa.

In \cbv, there are analogous notions of (un)solvable and (in)scrutable terms, thoroughly studied by \citet{DBLP:journals/pacmpl/AccattoliG22}. Surprisingly, \cbv unsolvable terms are not a good notion of meaningless term, as they lead to an inconsistent theory. The right notion of meaningless term in \cbv is given by \cbv inscrutable terms, which are the minimum terms for the \cbv contextual preorder.
