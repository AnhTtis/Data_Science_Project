% !TEX root = main.tex

\section{Equational Benchmarks and the Value Substitution Calculus}
\label{sect:benchmarks-vsc}
Here we revisit the equational benchmarks of \refsect{benchmarks} in the VSC. We begin with the proof nets equivalences, as they are one of the \emph{raison d'\^etre} of the VSC and the key to re-understand them all.

\paragraph{Structural Equivalence} The translation of the VSC to proof nets maps some terms with ES to the same proof net. The induced identification of terms is expressed by structural equivalence \cite{accattoli+paolini-vsc,Accattoli-proofnets}. 
\begin{definition}[Structural equivalence $\streq$]
Structural equivalence $\streq$ is defined as the smallest compatible equivalence relation generated by union of the following root rules.
\begin{center}
$\begin{array}{rllrcc}
	(\tm\tmthree)\esub\var\tmtwo & \equivsone&\tm\esub\var\tmtwo\tmthree 	 & \mbox{if }\var \not \in \fv\tmthree
	\\
	(\tm\tmthree)\esub\var\tmtwo &\equivexsthree& 	\tm\tmthree\esub\var\tmtwo & \mbox{if }\var \not \in \fv\tm
	\\
	\tm\esub\var\tmtwo\esub\vartwo\tmthree &\equivass& \tm\esub\var{\tmtwo\esub\vartwo\tmthree} & \mbox{if }\vartwo \not \in \fv\tm
	\\
	\tm\esub\vartwo\tmthree\esub\var\tmtwo &\equivcom& 	\tm\esub\var\tmtwo\esub\vartwo\tmthree & \mbox{if }\var \not \in \fv\tmthree \mbox{ and }\vartwo \not \in \fv\tmtwo
\end{array}$
\end{center}
\end{definition}
These axioms preserve the number and type of the constructors in terms, they only rearrange the order. In particular, structurally equivalent terms have the same number of ES. Note that the axioms simply express the constructor-and-scope-preserving commutation of ES with applications and ES themselves (but not abstractions, as that would break \refprop{strong-bisimulation} below). 

Additionally, structural equivalence behaves very well with respect to evaluation: it commutes with reduction rules---and is therefore postponable---preserving the number and kind of steps. This is expressed by the following proposition. In the literature, what is below called \emph{strong commutation} is usually called \emph{strong bisimulation}. We prefer to change the terminology here to avoid confusion with \emph{normal form (bi)simulations}, as the concept is similar and yet different (no need to observe normal forms, and it preserves the number of steps).

\begin{toappendix}
\begin{proposition}[$\streq$ strongly commutes with $\tovsc$, \cite{accattoli+paolini-vsc}]
	\label{prop:strong-bisimulation}
	Let $a \in \set{\msym,\expoabs,\expovar}$. Structural equivalence $\streq$ strongly commutes with $\tovsc$:
	if  $\tm \streq\tmtwo$ and $ \tm \Rew{a}\tmp$ then $\tmtwo \Rew{a}\tmtwop$ and $\tmp\streq\tmtwop$.
\end{proposition}
\end{toappendix}

In fact, $\streq$-equivalence classes are an isomorphic representation of proof nets, as the proof net $P_\tm$ associated to $\tm$ does the same exact rewriting steps as $\tm$, that is, the translation from $\tm$ to $P_\tm$ also strongly commutes with evaluation (turning term steps into proof nets steps, and vice-versa), see Accattoli \cite{Accattoli-proofnets}. Consequently, $\streq$-equivalent terms are \emph{indistinguishable} and should be equated by any sensible notion of program equivalence on pure terms (extensions with effects can invalidate some cases of structural equivalence, typically $\equivcom$, as we discuss below). In particular, structural equivalence is included in contextual equivalence, as shown by \citet{DBLP:journals/pacmpl/AccattoliG22}.

\paragraph{Revisiting the Benchmarks From Calculi} The equivalences of Moggi's and the shuffling calculi can be re-understood via structural equivalence. The idea is that by applying $\tom$ to the two sides of an equivalence, we can express it via ES, and many of become cases of $\streq$. Consider $\equivsone$:
	\begin{center}
		$\begin{array}{ccccc}
	(\la\var\tm)\tmtwo\tmthree &\equivsone& (\la\var\tm\tmthree)\tmtwo & \mbox{with }\var\not\in\fv\tmthree
	\\
\downarrow_{\mult} && \downarrow_{\mult}
	\\
\tm\esub\var\tmtwo\tmthree &\equivsone& (\tm\tmthree)\esub\var\tmtwo
\end{array}$
	\end{center}
Similarly, the equivalences $\equivexsthree$, $\equivass$, and $\equivcom$ of \refsect{benchmarks} become the axioms with the corresponding label of $\streq$, and $\equivsthree$ is a special case of $\equivexsthree$. Therefore, structural equivalence covers the shuffling equivalences and the proof nets equivalences. Moggi's equivalences $\equivlid$, $\equivlad$ and $\equivrad$, instead, are not covered. By applying $\tom$, we obtain the following reformulation for the VSC which corresponds to Moggi's original formulation with $\letexp$-expressions. If $\var\notin \fv\tmtwo$ for $\equivlad$ and $\var\notin \fv\val$ for $\equivrad$:
\begin{center}
\begin{tabular}{c@{\hspace{1.5cm}} c@{\hspace{1.5cm}} c}
		$\begin{array}{ccccc}
	(\la\var\var)\tm &\equivlid& \tm 
	\\
\downarrow_{\mult} && =
	\\
\var\esub\var\tm &\equivlid& \tm
\end{array}$
%%%%
&
%%%%
$\begin{array}{ccccc}
	(\la\var\var\tmtwo)\tm &\equivlad& \tm\tmtwo 
	\\
\downarrow_{\mult} && =
	\\
(\var\tmtwo)\esub\var\tm &\equivlad& \tm\tmtwo
\end{array}$
%%%%
&
%%%%
$\begin{array}{ccccc}
	(\la\var\val\var)\tm &\equivrad& \val\tm 
	\\
\downarrow_{\mult} && \downarrow_{\mult}
	\\
(\val\var)\esub\var\tm &\equivrad&  \val\tm
\end{array}$
\end{tabular}
	\end{center}
In presence of structural equivalence, the ES formulation of the application decomposition equivalences $\equivlad$ and $\equivrad$ is derivable:  $\equivlad$ follows from $\equivsone$ and $\equivlid$, $\equivrad$ follows from $\equivsthree$ and $\equivlid$. In fact, by using the extended version $\equivexsthree$ of $\equivsthree$ we can actually derive the extended version of $(\tmtwo\var)\esub\var\tm \equivexrad  \tmtwo\tm$ (if $\var\notin\fv\tmtwo$) of $\equivrad$. Therefore, the whole of Moggi's equivalences is captured by simply adding $\equivlid$ to structural equivalence.

\paragraph{A Family of Strong Commutations} By looking at the proof (in Appendix D) of strong commutation of $\streq$ (\refprop{strong-bisimulation}), it turns out that various sub-relations of structural equivalence also verify strong commutation. We here describe them by the root rules, while implicitly referring to the same closure used in the definition of $\streq$. For instance, $\equivcom$ by itself strongly commutes with $\tovsc$, as well as $\equivsone$ by itself, or $\equivexsthree\cup\equivass$, or some of the restricted version such as ${\equivsthree}\cup\equivass$. In particular, $\equivsone\cup\equivsthree\cup\equivass$, which would be the restriction of $\streq$ to a non-commutative setting for effects, also strongly commutes with $\tovsc$. In the next section, we shall craft a normal form similarity for the VSC which is parametric with respect to these variants of structural equivalence.

