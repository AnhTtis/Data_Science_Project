% !TEX root = main.tex
\section{Background About Normal Form Bisimulations}
Normal form bisimulations are program equivalences that, instead of comparing terms \emph{externally}, depending on how they behave \emph{in contexts}, they compare them \emph{internally}, by looking at the structure of their (infinitary) \emph{normal forms}. Let us give the simplest possible example.

Let $\towh$ be weak head reduction, also known as \cbn evaluation, which is a deterministic reduction and it is defined as  $(\la\var\tm)\tmtwo \tmthree_1 \ldots \tmthree_k\towh \tm\isub\var\tmtwo \tmthree_1 \ldots \tmthree_k$ with $k\geq 0$. Normal form simulations are usually based on a big-step presentation of $\towh$: we write $\tm\bswh\tmtwo$ if the $\towh$-evaluation of $\tm$ terminates on $\tmtwo$, and $\tm\bswhdiv$ otherwise.
%\adr{ We shall name $\leqcn$ and $\eqcn$ the associated notions of contextual preorder and equivalence.} 
The following notion of simulation was first considered by Sangiorgi \cite{SANGIORGI-normal-form-bisimulation}. Our presentation is slightly different but equivalent.
\begin{definition}[\cbn normal form simulations, \cite{SANGIORGI-normal-form-bisimulation}]
\label{def:cbn-nfs}
	A relation $\relsym$ is a \cbn normal form simulation if $\relsym\subseteq\relcbn$, where $\tm \relcbn \tmp$ holds whenever $\tm,\tmp$ satisfy one of the following clauses:
	\begin{center}
		$\begin{array}{r@{\hspace{.3cm}}r@{\hspace{.3cm}}l@{\hspace{.3cm}}l@{\hspace{.3cm}}lll}
	\textup{(cbn 1)} & && \tm\bswhdiv & \ie ~ \text{has no} \towh \text{-normal form.}
	\\
	\textup{(cbn 2)} & \tm \bswh \var & \text{and} & \tmp \bswh \var &
	\\
	\textup{(cbn 3)} & \tm \bswh \la\var\tm_1 & \text{and} &\tmp \bswh \la\var\tmp_1 & \text{with}~ \tm_1 \rel \tmp_1
	\\
	\textup{(cbn 4)} & \tm \bswh \ntm = \ntmONE\tmtwo&\text{and}& \tmp \bswh \ntmtwo = \ntmONEtwo \tmtwop &\text{with}~\ntmONE\rel\ntmONEtwo ~ \text{and}~ \tmtwo\rel\tmtwop
	\end{array}$
	\end{center}
%	\begin{enumerate}
%		\item $\tm\bswhdiv$;
%		\item $\tm \bswh \var$ and $\tmp \bswh \var$;
%		\item $\tm \bswh \la\var\tm_1$ and $\tmp \bswh \la\var\tmp_1$ with $\tm_1 \rel \tmp_1$;
%		\item $\tm \bswh \ntm = \ntmONE\ntmTWO$ and $\tmp \bswh \ntmtwo = \ntmONEtwo\ntmTWOtwo$ with $\ntmONE\rel\ntmONEtwo$ and $\ntmTWO\rel\ntmTWOtwo$.
%	\end{enumerate}
	\emph{\cbn normal form similarity} $\leqcbn$ is the largest \cbn normal form simulation, that is, it is the union of all normal form simulations.
	\end{definition}
Normal form bisimulations and bisimilarity are the symmetric variants of simulations and similarity, defined as expected.
A simple way of proving soundness of $\leqcbn$ is to show compatibility via the variant of Howe's method developed by Lassen in \cite{lassen1999bisimulation}. 

\paragraph{Partiality and Divergence.} Normal form simulations rely on a partial notion of evaluation (with respect to full $\beta$-reduction), such as weak head reduction $\towh$. The key point is that the partial reduction leaves some sub-terms not evaluated (arguments and abstraction bodies for $\towh$). The derived simulations compare the $\towh$-normal forms $\nfwh(\tm)$ and $\nfwh(\tmtwo)$ of $\tm$ and $\tmtwo$ by 
\begin{itemize}
\item Checking that they have the same structure for the partially evaluated part of the term, and
\item Asking that the non-evaluated sub-terms of $\nfwh(\tm)$ and $\nfwh(\tmtwo)$ in the same positions are pairwise normal form similar.
\end{itemize}
The use of a partial notion of evaluation is crucial, as it allows fine discriminations related to divergence, which would be blurred if one would only consider full normal forms. \cbn normal form similarity, indeed, discriminates between the following forms of divergence:
\begin{enumerate}
\item Looping as $\Omega$;
\item Looping only after having received an argument, as for $\la\var\Omega$;
\item Having a looping argument, as for $\var \Omega$;
\item The finite iterations of 2 and 3, and of their combination;
\item The \emph{infinite} iteration of 2 or 3, that never actually loop as $\Omega$ as they keep producing an infinite amount of head variables and/or abstractions. There exists terms $\tm$, indeed, the $\towh$ normal form of which is, for instance, $\la\var\tmtwo$ or $\var\tmtwo$ or $\var \la\vartwo\tmtwo$, and such that $\tmtwo$ has the same property. Therefore, they give rise to infinite normal forms such as $\la\var\la\var\la\var\ldots$ or $\var (\var (\var \ldots$ or $\var (\la\vartwo \var (\la\vartwo \ldots$.
\end{enumerate}

% \paragraph{Trees.} It turns out that Sangiorgi's notion of bisimilarity coincides with \levy-Longo tree equality, that is, the equality induced on terms by having the same \levy-longo tree, which is a standard notion of infinitary normal form for terms based on weak head reduction.
%
%Sangiorgi's (bi)simulations can be adapted to head reduction and then they no longer distinguish between the first and the second form of divergence above. The adapted notion of bisimilarity corresponds to \bohm tree equality, as shown by Lassen \cite{lassen1999bisimulation}, who also shows how to add $\eta$-equivalence. 


\paragraph{Open Terms.} Normal forms simulations are different from most other notions of program equivalence in that they have to deal with open terms, because, even when the terms to compare are closed, the sub-terms on which the comparison is iterated might be open.%, because the comparison goes under abstractions.



\paragraph{Easier to Use.} With respect to other notions of equivalence such as applicative bisimilarity, normal form bisimilarity is often simpler to establish, because it removes the quantification over arguments. A typical example is the proof of the equivalence of the Curry and Turing fix-point combinators $\curryfixn$ and $\turingfixn$, which is particularly simple with \cbn normal form bisimilarity, as Lassen explained in his first article relating \bohm tree equivalence and head normal form bisimulations \cite{lassen1999bisimulation}, a variant of Sangiorgi's. We can easily adapt his argument on head normal form bisimulations to weak head normal form bisimulations.
Let $\curryfixn = \la\var{\curryfixauxn\curryfixauxn}\text{, where } \curryfixauxn= \la\varthree{\var({\varthree\varthree})}$ and $ \turingfixn = (\la\varthree{\la\var{\var({\varthree\varthree\var})}})(\la\varthree{\la\var{\var({\varthree\varthree\var})}})$. It is easy to check that the following relation $\relsym$ is a \cbn normal form (bi)simulation relating $\curryfixn$ and $\turingfixn$.
\begin{center}
$\rel \defeq \{(\curryfixn,\turingfixn),(\curryfixauxn\curryfixauxn,\var(\turingfixn\var)), (\var(\curryfixauxn\curryfixauxn),\var(\turingfixn\var)),(\curryfixauxn\curryfixauxn,\turingfixn\var) \mid \var\text{ a variable}\}$
\end{center}

\paragraph{Equational Benchmarks} We say that a notion of program equivalence $\simeq_X$ \emph{validates} an equivalence $\equiv_Y$ if $\tm\equiv_Y\tmtwo$ implies $\tm\simeq_X\tmtwo$. Program equivalences in \cbn roughly can differ only along two axes:
\begin{enumerate}
\item \emph{$\eta$-equivalence}: the amount of $\eta$ equivalence that they validate (no $\eta$, finitely many $\eta$ expansions, infinitely many $\eta$-expansion, or even the further notion of infinite $\eta$-expansion, which is not the same of infinitely many standard $\eta$-expansions).
\item \emph{Scrutability/solvability}: the amount of identifications among inscrutable and unsolvable terms.
\end{enumerate}

Sangiorgi's \cbn normal form bisimilarity does not validate $\eta$-equivalence, because his bisimilarity is \emph{rigid}, it cannot equate different normal forms such as $\var$ and $\la\vartwo\var\vartwo$. Since $\eta$ equivalence is instead validated by contextual equivalence, $\eqcbn$ is not fully abstract. More generally, normal form bisimilarities tend to not be fully abstract. Intuitively, $\tm$ and $\tmp$ might be externally equivalent, that is, behave the same in all contexts, and yet be internally different, by having different (potentially infinitary) normal forms.


About scrutability, it follows from its characterization (\refthm{cbn-scrutability-characterization}) that \cbn normal form bisimilarity equates all inscrutable terms, that is, it validates the following equivalence. 
\begin{itemize}
\item \emph{(In)scrutable equivalence}: $\tm \equiv_{scr} \tmtwo$ if $\tm$ and $\tmtwo$ are \cbn inscrutable.
\end{itemize}
Instead, $\leqcbn$ does not equate all unsolvable terms, as it distinguishes $\Omega$ and $\la\var\Omega$ (namely $\Omega \leqcbn \la\var\Omega$ but $\la\var\Omega \not\leqcbn \Omega$). We shall see that in \cbv there is an analogous notion of \cbv inscrutability, but the reference \cbv normal form bisimulation does not equate all \cbv inscrutable terms.
