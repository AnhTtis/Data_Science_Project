% !TEX root = main.tex

\section{Lassen's eager normal form simulation}
The \cbv normal form simulation of reference in the literature is due to Lassen \cite{LassenEnf}.  Lassen's simulation is interesting because it is not defined by simply changing the notion of reduction in Sangiorgi's. Lassen indeed exploits stuck redexes in open terms, and defines a simulation which \emph{unstucks} them, a mechanism which we shall refer to as \emph{stop-and-go}.

\paragraph{Grammar of Left Normal Forms} Lassen's similarity is built using \emph{left reduction}. The starting point is the observation that, despite the limits of left reduction, it admits a description of its normal forms via left contexts which is extremely simple and elegant.
\begin{lemma}[Unique decomposition, Lassen \cite{LassenEnf}]
\label{l:las-unique-dec}
	Any (possibly open) term is either a value or admits a unique decomposition $\levctxp{\val\valtwo}$. In particular, left normal forms can be described as follows:
	\begin{center}
	$\begin{array}{c@{\hspace{.5cm}}rcc}
	\textsc{Left normal forms}  &
	\ntm,\ntmtwo & \grameq &  \val\mid \levctxp{\var\val}
	\end{array}
	$\end{center}

\end{lemma}



The simple contextual structure of left normal forms is then used by Lassen to define eager normal form simulation. The crucial clause is the fourth one, which realizes the stop-and-go mechanism.

\begin{definition}[\Enf simulation \cite{LassenEnf}]
	A relation $\relsym$ is an \emph{eager normal form (\enf) simulation} if $\relsym\subseteq\relenf$, where $\tm \relenf\tmp$ holds whenever $\tm,\tmp$ satisfy one of the following clauses:
	\begin{center}
		$\begin{array}{r@{\hspace{.3cm}}r@{\hspace{.3cm}}l@{\hspace{.3cm}}l@{\hspace{.3cm}}lll}
		\textup{(enf 1)} & &&\tm\bslasdiv & \ie ~ \text{has no} \tolw \text{-normal form.}
		\\
		\textup{(enf 2)} & \tm \bslass \var  &\text{and}& \tmp \bslass \var
		\\
		\textup{(enf 3)} & \tm \bslass \la\var\tmfirst &\text{and}& \tmp \bslass \la\var\tmpfirst 
		& \text{with} ~ \tmfirst \rel \tmpfirst
		\\
		\textup{(enf 4)} & \tm \bslass \levctxp{\var\val}  &\text{and}& \tmp \bslass \levctxtwop{\var\valtwo} 
		&
		\textup{with} ~ \val\rel\valtwo ~ \text{and}  ~ \levctxp{\varthree}\rel\levctxtwop{\varthree} 
		\\
		&& &&\text{where} ~ \varthree ~ \text{is not free in} ~ \levctx ~ \text{or} ~ \levctxtwo
		\end{array}
		$\end{center}
	\emph{\Enf similarity}, written $ \leqenf $, is defined by co-induction as the largest \enf simulation, that is, it is the union of all \enf simulations. We say that $\tm$ is \emph{\enf similar} to $\tmp$ if $\tm \leqenf \tmp$.
\end{definition}

\paragraph{Stop-and-Go, Double Task, and Left Identity.} The stop-and-go clause (enf 4) cleverly does two tasks at the same time, subsuming clause (nai 4) of naive similarity in a bottom-up way and capturing the left identity equivalence $\equivlid$. 

About (nai 4), consider for instance comparing $\tm \defeq \var\vartwo\varfour$ with itself via enf simulations. Clause (enf 4) reduces it to compare $\varthree\varfour$ and $\vartwo$ with themselves, since $\tm = \lctxp{\var\vartwo}$ with $\lctx\defeq \ctxhole\varfour$. Then, it reduces the first one to compare $\varthree'$ and $\varfour$ with themselves. In contrast, (nai 4) (or Sangiorgi's (cbn 4)) proceeds \emph{top-down}, by splitting $\var\vartwo\varfour$ into $\var\vartwo$ and $\varfour$, and then splitting $\var\vartwo$.

About the left identity equivalence, consider showing that $\tm \defeq \Id(\var\var)$ and $\tmtwo\defeq \var\var$ are enf similar. Evaluation is stuck on $\tm$, that is, it \emph{stops} because $\var\var$ is not a value. Note that $\tm$ has shape $\lctxp{\var\val}$ with $\lctx=\Id\ctxhole$ and $\val=\var$. The idea is that a term $\tmtwo$ that is $\leqenf$-similar to $\tm$ has to get stuck as well, or anyway decompose in a similar way. Now, $\tmtwo$ is not stuck, because there are no blocked redexes, but it has nonetheless shape $\lctxtwop{\var\valtwo}$ by taking $\lctxtwo = \ctxhole$ and $\valtwo = \var$. The comparison between $\tm$ and $\tmtwo$ is then reduced to compare the two pairs $\lctxp\varthree = \Id \varthree$ and $\lctxtwop\varthree=\varthree$, and $\val=\var$ and $\valtwo=\var$. The second pair trivially matches, because the identity relation is an enf simulation. About the first pair, note that $\Id\varthree$ is no longer stuck, that is, it can \emph{go}. And for the next round of comparison (of $\Id\varthree$ and $\varthree$) we have to first $\tolw$ reduce the terms, so that $\Id\varthree \tolw \varthree$ and thus also the first pair trivially matches.

Summing up, the enf simulation relating $\Id(\var\var)$ and $\var\var$ is the following  simulation $\relsym$: \[\relsym =\{(\Id(\var\var),\var\var), (\var,\var), (\Id\varthree,\varthree), (\varthree,\varthree)\}\]
This is  just an instance, but the following more general result holds.
\begin{proposition}[Enf bisimilarity validates left identity]
$\Id \tm \eqenf \tm$ for any term $\tm$.
\end{proposition}


\paragraph{Left/Right/Weak Non-Equivalent Variants} Replacing left reduction with right reduction, one obtains a unique decomposition lemma such as \reflemma{las-unique-dec} with respect to right contexts, and, accordingly, a notion of \emph{right Lassen similarity}---let us denote it with $\leqrenf$. It turns out that $\leqenf$ and $\leqrenf$ are different, incomparable similarities. For instance, $\Omega$ and $\Omega (\var\var)$ are enf bisimilar (because they are both $\tolw$-divergent) but not renf bisimilar, because $\leqrenf$ stops on $\Omega (\var\var)$ which is $\torw$ normal. Similarly, $\Omega$ and $\var\var\Omega $ are not enf bisimilar while they are renf bisimilar.

Replacing left reduction with weak reduction is instead problematic for another reason. Since $\tow$ is non-deterministic, the unique decomposition lemma (\reflemma{las-unique-dec}) fails for it. It is not clear then what would be the right definition of weak enf similarity, as the stop-and-go clause can be generalized in more than one way. An appropriate definition for weak enf similarity, as a generalization, should include terms related by enf similarity. However, it is also unclear (to us) how to prove the compatibility of some of such generalizations (we tried but failed\footnote{The definitions we can come up with for weak enf similarities that could be compatible are not able to relate as much terms as enf similarity does.}). 

The next paragraphs discuss the principles that are (in)validated by enf similarity.


\paragraph{$\beta_v$-Conversion}
From the definition of $\leqenf$, it immediately follows that enf bisimilarity contains the $\rtobv$ root rule, that is, that if $\tm \rtobv \tmtwo$ then $\tm\eqenf \tmtwo$, simply because $\tm$ and $\tmtwo$ have the same left normal form. Since $\eqenf$ is a compatible equivalence relation, it turns out that $\eqenf$ contains the whole of $\betav$-conversion $=_{\betav}$, thus it contains left reduction as well as weak and right reductions.

\begin{proposition}[$\betav$-conversion is validated by enf bisimilarity]
If $\tm =_{\betav} \tmtwo$ then $\tm \eqenf \tmtwo$.
\end{proposition}

\paragraph{Curry and Turing fix-Point Combinators are Enf Bisimilar.} It is easy to check that the relation $\relsym$ proving the naive similarity of Curry's and Turing's \cbv fix-point combinators is also a \enf bisimulation. The fact that those combinators are naive bisimilar means that the \emph{unstacking} aspect of the stop-and-go clause plays no role for their equivalence. 

\begin{proposition}
$\curryfix \eqenf \turingfix$, that is, Curry's and Turing's fix-points are enf bisimilar.
\end{proposition}

\paragraph{\cbv Scrutability} Enf bisimilarity does not validate the scrutable equivalence $\equivscr$. For instance, the same counter-example used for naive similarity works for enf, as we have $\Omega \leqenf \Omega^L$ but $\Omega^L \not\leqenf \Omega$. In fact, $\eqenf$ equates some inscrutable terms that are separated by $\eqncbv$ and vice-versa. For instance, let $\delta_3 \defeq \la\var\var\var\var$. The term $\Omega_3 \defeq \delta_3\delta_3$ is divergent and \cbv inscrutable. We have that $\Omega^L_3 \defeq (\la\var\delta_3)(\vartwo\varthree)\delta_3$ is enf bisimilar to $\Omega^L$, because they stop similarly and then both go to diverge. They are instead unrelated with respect to $\leqncbv$, because when compared as normal forms they do not have the same structure. For the vice-versa, consider $\var\var\Omega$ and $\Omega$, which are equated by $\eqncbv$ but separated by $\eqenf$, because $\var\var\Omega$ is left normal but weak divergent. 

\begin{proposition}
Enf bisimilarity does not validate scrutable equivalence $\equivscr$.
\end{proposition}


\paragraph{Further Equivalences.} The next proposition sums up the benchmarks for enf.
\begin{toappendix}
\begin{proposition}
\label{prop:enf-validation-of-equivalences}
Enf bisimilarity validates Moggi's equivalences and the shuffling equivalences, but it does not validate $\etav$, the proof nets equivalences, nor \cbn duplication.
\end{proposition}
\end{toappendix}
 Enf similarity does not validate $\etav$, since $\var$ and $\la\vartwo\var\vartwo$ are handled by different clauses in the definition of $\leqenf$. It can however be adapted to validate it, see \cite{LassenEnf,lassen+strovring-bisimilarity-eta,DBLP:journals/lmcs/BiernackiLP19}. 
About Moggi's equivalences, we have already discussed the left identity. Enf similarity validates all the other ones. Note that in \cite{LassenEnf} Lassen claims that enf validates $\equiv_{exrad}$ which is actually false, it only validates $\equiv_{rad}$ (in fact $\equiv_{exrad}$ does not correspond to Moggi's usual right decomposition rule, it shall be motivated by proof nets in \refsect{benchmarks-vsc}). Proofs of the validations are easy, one only needs to write the right relation (the identity relation $\cup$ the equivalence to validate) and show that it is indeed an \enf bisimulation.

The validation of the shuffling equivalences is an easy contribution of this paper. 
Simple inspections show that enf does not validate the proof nets equivalences $\equivexsthree$ and $\equivcom$, nor \cbn duplication. Consider now renf bisimilarity, the right variant of enf. With respect to $\sigma$-equivalences, it as a sort of dual behavior: it  does not validate $\equivsone$ but validates $\equivsthree$ and even $\equivexsthree$.