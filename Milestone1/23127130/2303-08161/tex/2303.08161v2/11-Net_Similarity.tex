% !TEX root = main.tex
\section{Net Similarity for the Value Substitution Calculus}
\label{sect:net}
In this section, we finally define the nf-similarity for the VSC we are interested in, \emph{net similarity}, which extends $\leqncbv$ along two axes:
\begin{enumerate}
\item \emph{Changing the underlying calculus:} evaluation is now based on the VSC, where $\Omega$-terms have a diverging characterization, hence $\Omega$-equivalence $\equivom$ will be trivially included in the bisimilarity, and 
\item \emph{Changing the nf-bisimilarity}: allowing simulations to test terms modulo structural equivalence $\streq$, in order to avoid artificial distinctions of indistinguishable terms and to validate the commutation of $\letexp$s $\equivcom$. 
\end{enumerate}
The problematic addition of the left identity equivalence $\equivlid$ is discussed at the end of the section.

\paragraph{Changing the Underlying Calculus} Moving from Plotkin's \cbv calculus to the VSC, normal forms are harder to describe, as one can see from the grammar of normal forms in Figure \ref{fig:vsc}.

Since the case $\isctxp\var\ntm$ is quite unpleasant to manage in proofs, we go one step further and consider normal forms modulo $\equivsone$, picking the $\equivsone$-representative of each normal form where the context $\isctx$ has been pushed out of the unpleasant case for inert terms. This can be done harmlessly because $\equivsone$ by itself verifies strong \emph{commutation} with respect to $\tovsc$ and the described representant can be easily described at the big-step level. The following lemma gives a grammar for normal forms modulo $\equivsone$; we overload/redefine \emph{inert terms}, which from now on only refer to the new grammar.

\begin{lemma}
$\tm$ is $\tovsc$-normal iff there exists $\ntm$, given by the following grammar, such that $\tm \equivsone \ntm$.
\begin{center}
	$\begin{array}{r@{\hspace{.5cm}}rlll}
		\textsc{Applicative Inert terms}  &
		\itmapp,\itmapptwo & \grameq &  \var\fire\mid \itmapp\fire
		\\
		\textsc{Inert terms}  &
		\itm,\itmtwo & \grameq &  \itmapp\mid \itm\esub\var\itmtwo
		\\
		\tovsc\textsc{-normal forms modulo $\equivsone$}  &
 \ntm,\ntmtwo & \grameq &  \val \mid \itm\mid \fire\esub\var\itmtwo
	\end{array}
	$\end{center}
\end{lemma}
Note the notion of applicative inert terms, which are specific inert terms where no substitutions can be pushed outward by $\equivsone$. 
Example: the $\tovsc$-normal form $\tm =\var\esub\var{\vartwo\vartwo}\varthree$ does not belong to grammar above, but $\tm \equivsone (\var\varthree)\esub\var{\vartwo\vartwo} = \ntm$ and $\ntm$ is described by the grammar.

\paragraph{Big-Step Evaluation} We then need to express evaluation to $\tovsc$-normal form modulo $\equivsone$ as a big-step predicate. For that, we re-define the inert substitution contexts, which at first sight are defined as before, except that the notion of inert term now has changed.
\begin{center}
	$\begin{array}{r@{\hspace{.5cm}}rlll}
		\textsc{Inert Substitution Contexts} & \isctx & \grameq &  \ctxhole\mid \isctx\esub{\var}{\itm} 
	\end{array}
	$\end{center}

\begin{definition}[Big-step evaluation $\bsvsct k$]
Big-step $\VSC$ modulo $\equivsone$ evaluation $\tm \bsvsct k \ntm$ is given by:
\begin{center}
		% !TEX root = main.tex


\begin{tabular}{cccccc}

	\infer[(\bsvsctax)]{\val \bsvsct 0 \val}{}
	
	&
	
	\infer[(\bsvsctapm)]{\tm\tmtwo \bsvsct {k+i+1} \isctxp\fire}{
		\tm \bsvsct k \isctxp{\la\var\tmthree}
		&
		{\tmthree\esub\var\tmtwo} \bsvsct i \fire
	}
	
	
	
	\\[6pt]
	\infer[(\bsvsctapvar)]{\tm\tmtwo \bsvsct {k+h} \isctxp{\var\fire}}{
		\tm \bsvsct k \isctxp\var
		&
		\tmtwo \bsvsct h \fire
	}
	&
	\infer[(\bsvsctese)]{\tm\esub\var{\tmtwo} \bsvsct {k+i+1} \isctxp\fire}{
		\tmtwo \bsvsct k \isctxp{\val}
		&
		{\tm\isub\var{\val}} \bsvsct i \fire
	}

	
	
	\\[6pt]
	
	\infer[(\bsvsctapi)]{\tm\tmtwo \bsvsct {k+h} \isctxp{\itmapp\fire}}{
		\tm \bsvsct k \isctxp\itmapp
		&
		\tmtwo \bsvsct h \fire
	}
	&
\infer[(\bsvsctesi)]{\tm\esub\var\tmtwo \bsvsct {k+h} \fire\esub\var\itm}{
	\tm \bsvsct k \fire
	&
	\tmtwo \bsvsct h \itm
}
\end{tabular}


\end{center}
\emph{Notation}: $\tm \bsvscts \fire$ abbreviates \emph{there exists a $k$ such that $\tm \bsvsct k \fire$}.
\end{definition}

 The given big-step system captures $\equivsone$ via the rules ($\bsvsctapvar$) and ($\bsvsctapi$): when applying an inert term / variable surrounded by an inert context $\isctx$ to a normal term $\ntm$, the context $\isctx$ is pushed out of the application, obtaining $\isctxp{\var\fire}$ and $\isctxp{\itmapp\fire}$ instead of $\isctxp{\var}\fire$ and $\isctxp{\itmapp}\fire$. As a result, the $\bsvsctsym$-normal forms are exactly those of the given grammar for $\tovsc$-normal forms modulo $\equivsone$.

	We prove this big-step system to be correct and complete with respect to small-step reduction. 	Importantly, substitutivity also smoothly adapts.
	\begin{toappendix}
	\begin{proposition}
		\label{l:ss-bs-equivalence_vsce}
		$\tm \bsvsct k \fire$ if and only if there exists a normal form $\firep$ such that $\tm \tovsc^k \firep \equivsone \fire$.
	\end{proposition}
	\end{toappendix}
	
%	\begin{proof}
%		Similar to the one for the \VSCptxt big step system.
%	\end{proof}
	

	\begin{toappendix}
	\begin{proposition}[Substitutivity]
		\label{prop:substitutivity_vsce}
	\hfill
	\begin{enumerate}
	\item 
	\emph{Small-step}: if $\tm~\tovsc~\tmp$ then $\tm\isubst\val\var ~\tovsc~ \tmp\isubst\val\var$.
	\item 
	\emph{Big-step}: 	if $\tm\isubst\val\var \bsvsct k \ntm$ then $\exists$ $k'$ and $\ntmtwo$ such that $ \tm \bsvsct {k'} \ntmtwo$ and $\ntmtwo\isubst\val\var\bsvsct {k-k'} \ntm$.
	\end{enumerate}
\end{proposition}
\end{toappendix}

	\paragraph{Changing the Nf-Bisimilarity by Adding Structural Equivalence to Simulations, Parametrically} We are now also going to refine the definition of naive similarity by adding structural equivalence $\streq$ to the nf-bisimilarity. In fact, we are going to do something more general, in order to obtain a whole family of similarities. We abstract away structural equivalence $\streq$ as a more abstract notion of \emph{mirror equivalence} $\equivx$, defined by the properties of $\streq$ that are needed to prove that similarity modulo $\streq$ is compatible. Then, similarity is defined \emph{parametrically} in a mirror $\equivx$, and net similarity is obtained by taking $\streq$ as mirror. The terminology \emph{mirror} is meant to suggest that $\equivx$ can modify terms only in inessential ways.
	
\begin{definition}[Mirror]
An equivalence relation $\equivx$ is a \emph{mirror} for $\tovsc$ when:
\begin{enumerate}
\item \emph{Strong commutation}: if  $\tm \equivx\tmtwo$ and $ \tm \tovsc\tmp$ then $\tmtwo \tovsc\tmtwop$ and $\tmp\equivx\tmtwop$.

\item \emph{Substitutivity}: if $\tm\equivx\tmtwo$ then $\tm\isub\var\val \equivx \tmtwo\isub\var\val$ for all values $\val$.
\end{enumerate}
\end{definition}

\begin{definition}[Mirrored and \net similarities]
	Let $\relsym$ be relation and $\equivx$ be a mirror over VSC terms.	We say that $\relsym$ is a \emph{$\equivx$-mirrored (\nafex) simulation} if $\relsym\subseteq\relvscx$, where $\tm \relvscx\tmp$ holds whenever $\tm,\tmp$ satisfy one of the following clauses:
	\begin{center}
		$\begin{array}{r@{\hspace{.3cm}}r@{\hspace{.3cm}}l@{\hspace{.3cm}}l@{\hspace{.3cm}}lll}
		\textup{(\nafex 1)} & &&\tm\bsvsctdiv & \ie ~ \text{has no} \tovsc \text{-normal form.}
		\\
		\textup{(\nafex 2)} & \tm \bsvscts \var  &\text{and}& \tmp \bsvscts \var
		\\
		\textup{(\nafex 3)} & \tm \bsvscts \la\var\tmfirst &\text{and}& \tmp \bsvscts\la\var\tmpfirst
		& \text{with} ~ \tmfirst \rel \tmpfirst
		\\
		\textup{(\nafex 4)} & \tm \bsvscts \ntmONE \ntmTWO &\text{and}& \tmp \bsvscts \ntmtwo \equivx \ntmONEtwo \ntmTWOtwo
		& \text{with} ~ \ntmONE \rel \ntmONEtwo ~\text{and}~ \ntmTWO \rel \ntmTWOtwo
		\\
		\textup{(\nafex 5)} & \tm \bsvscts \ntmONE\esub\var\ntmTWO &\text{and}& \tmp \bsvscts \ntmtwo \equivx \ntmONEtwo\esub\var\ntmTWOtwo
		& \text{with} ~ \ntmONE \rel \ntmONEtwo ~\text{and}~ \ntmTWO \rel \ntmTWOtwo
	\end{array}
	$\end{center}
		$\equivx$-Mirrored (\nafex) similarity , written $ \leqvscx $, is defined the largest $\equivx$-mirrored simulation.
		
		\Net simulations and \net similarity $\leqnet$ are defined as the $\equivx$-mirrored simulations and similarities with structural equivalence $\streq$ as mirror $\equivx$. 
\end{definition}

\paragraph{Making Inert Terms Explicit in the Clauses} Cases (\nafex 4) and (\nafex 5) can be rewritten using the grammar of normal forms, which is useful for clarity in proofs. For (\nafex 4), it actually splits in two:
	\begin{center}
		$\begin{array}{r@{\hspace{.3cm}}r@{\hspace{.3cm}}l@{\hspace{.3cm}}l@{\hspace{.3cm}}lll}
		
		\text{(\nafex 4a)} & \tm \bsvscts \var \ntmONE &\textit{and}& \tmp \bsvscts \ntmtwo \equivx\var \ntmONEtwo 
		& \textit{with}~\ntmONE \rel \ntmONEtwo
		\\
		\text{(\nafex 4b)} & \tm \bsvscts \itmapp \ntmONE &\textit{and}& \tmp \bsvscts \ntmtwo \equivx\itmapptwo \ntmONEtwo 
		& \textit{with} ~ \itmapp \rel \itmapptwo ~\textit{and}~ \ntmONE \rel \ntmONEtwo
		\\
		\text{(\nafex 5)} & \tm \bsvscts \ntmONE\esub\var\itm &\textit{and}& \tmp \bsvscts \ntmtwo \equivx\ntmONEtwo\esub\var\itmtwo
	& \textit{with} ~ \itm \rel \itmtwo ~\textit{and}~ \ntmONE \rel \ntmONEtwo
	\end{array}
	$\end{center}

\paragraph{Compatibility} The compatibility proof for $\leqvscx$ follows the same structure of the one for $\leqncbv$ (in Appendix E of the additional material on HotCRP). At the evaluation level, we have already seen that substitutivity holds also for the weak reduction of the VSC and with the addition of the equivalence $\equivsone$ (\refprop{substitutivity_vsce}). At the level of the simulation, we need to refine the notion of Lassen closure, by adding rule $\mscequivx$ accounting for mirrors.
\begin{definition}[Mirrored Lassen closure]
Let the \emph{mirrored Lassen closure} $\mlasrelsym$ of $\relsym$ be:
	\begin{center}
		% !TEX root = main.tex
\begin{tabular}{cccccc} 
%\textsc{Mirrored Lassen's closure (for $\vscx$ simulations)}
%\\[6pt]
\begin{tabular}{cccccc} 
	\infer[\msclift ]{\tmrone \mlasrel \tmrtwo} {\tmrone \rel \tmrtwo}
	&
	\infer[\mscvar]{\var \mlasrel \var}	{}
	&
	\infer[\mscabs ]{\la\var\tmrone \mlasrel \la\var\tmrtwo} {\tmrone \mlasrel \tmrtwo}
	&
		\infer[\mscapp ] {\tmrone\tmrthree  \mlasrel  \tmrtwo\tmrfour} {\tmrone  \mlasrel \tmrtwo & \tmrthree \mlasrel \tmrfour }  
\end{tabular}
\\[14pt]
\begin{tabular}{cccccc}
		\infer[\mscesub ]{\tmrone\esub\var{\tmrthree} \mlasrel \tmrtwo\esub\var{\tmrfour}{}} {\tmrone \mlasrel \tmrtwo & \tmrthree \mlasrel \tmrfour }
&
	\infer[\mscsub ]{\tmrone\isub\var{\valof\tmrthree} \mlasrel \tmrtwo\isub\var{\valof\tmrfour}{}} {\tmrone \mlasrel \tmrtwo & \valof\tmrthree \mlasrel \valof\tmrfour }	
	&
\infer[\mscequivx]{\tmrone\mlasrel\tmrtwop}{\tmrone \mlasrel\tmrtwo & \tmrtwo \equivx \tmrtwop}
\end{tabular}
\end{tabular}		
	\end{center}
\end{definition}
Then the reasoning for compatibility---and in particular the coherence properties---smoothly adapts, using the mirror properties for rule $\mscequivx$ in the proof that the closure preserves mirrored simulations. In particular, strong commutation of $\equivx$ implies that it preserves normal forms and steps, that is, the coherence properties. Summing up, we obtain our main result.
\begin{toappendix}
\begin{theorem}[Compatibility and soundness of $\leqvscx$ and $\leqnet$]
	\label{thm:nafex-included-leqc}
Let $\equivx$ be a mirror.
	\begin{enumerate}
	\item \emph{Redundancy of the mirrored Lassen closure}: $\leqvscx \,= \mlassenop \leqvscx$.
	\item \Nafex similarity $\leqvscx$ is compatible and included in the \cbv contextual preorder $\leqcv$.
	\item \Net similarity $\leqnet$ is compatible and included in the \cbv contextual preorder $\leqcv$.
	\end{enumerate}
\end{theorem}
\end{toappendix}


\paragraph{Fixpoints and Benchmarks.} For any mirror $\equivx$, and in particular for $\equivx\defeq Id$ and $\equivx\defeq \streq$, one can show that Turing's and Curry's \cbv fixpoint combinators are \nafex bisimilar. The proof relies on exactly the same relation that for naive bisimilarity (\refprop{naive-fix-points-equiv}). \adr{Plotkin's $\betav$ and VSC conversions are included in \nafex similarities. }Unlike naive and enf similarities, \nafex and net similarities validate $\Omega_v$-equivalence $\equivomv$. Net and $\equivx$-mirrored bisimilarities do not however validate $\eta_v$ equivalence, one has to change the case for abstractions to accommodate it, and it does not validate \cbn duplication, as for instance $(\vartwo\var\var)\esub\var{\varthree\Id}$ and $\vartwo(\varthree\Id)(\varthree\Id)$ are both $\tovsc$-normal but not $\streq$-equivalent. 






%\paragraph{Net similarity includes structural equivalence, and more}Naturally, structurally equivalent terms are net bisimilar. However net similarity is able to relate more, in the sense that it can equate terms the \emph{strong} normal forms of which are $\streq$-equivalent: 
%
%\begin{center}
%	$\ntm_1 = \la\var{(\la\vartwo(\la\varthree\tm)\tmtwo)\tmthree}\eqnet\la\var{(\la\varthree(\la\vartwo\tm)\tmthree)\tmtwo} = \ntm_2$ whereas $\ntm_1 \not \streq \ntm_2$
%\end{center}

%\adr{add example with infinite normal forms}

%Let $A \defeq \la\var{\tm\tmtwo\esub\varthree\tmthree\var}$ and $B \defeq \la\var{(\tm\tmtwo)\esub\varthree\tmthree\var}$. We can show that $\curryfix A \eqnet \curryfix B$.


\paragraph{Left Identity Is Not Validated By \Nafex} Analogously, \net similarity does not validate Moggi's $\equivlid$ rule, because $\var\esub\var{\vartwo\Id} \not \tovsc \vartwo\Id$ and  $\var\esub\var{\vartwo\Id} \not \streq \vartwo\Id$. Thus, \enf is not included in \net similarity. About adding $\equivlid$, it is easy to define a \nafexp\equivlid bisimulation, but the current  compatibility proof does not go through, as $\equivlid$ is not a mirror for $\tovsc$ (in particular, it does not strongly commute with $\tovsc$) and the proof technique is not able (for now) to handle $\equivlid$ terms, as it breaks coherence for normal forms, that is, the fact that if  $\ntm\,\mlasrelsym \tm$ then $\tm$ is normal (and the symmetric statement).

One could also add $\equivlid$ as a reduction step of the VSC, but then the reduction is no longer diamond, and the diamond property (or at least the invariance of the number of steps to normal form) is essential in the current proof technique. It is thus unclear how to extend \net similarity as to validate $\equivlid$. Next section introduces a program equivalence including $\eqnet$ and validating $\equivlid$.

\paragraph{Net Bisimilarity Cannot Be Extensional} Lassen introduced an extension of enf bisimilarity validating $\equivetav$, enf bisimilarity up to $\eta_v$, at the same time that he introduced enf bisimilarity \cite{LassenEnf}. As of now, that modification cannot be applied to net bisimilarity. We explain the problem which boils down to, again, the fact that net does not validate the left identity law $\equivlid$.
	
	Let us consider that there exists a nf-bisimilarity $\relsym$ based on the VSC (\ie $\tm\tovsc\tmp$ implies $\tm\rel\tmp$) which is a compatible equivalence relation, and which validates $\equivetav$. Then, in particular, we have that $\var\rel\la\vartwo\var\vartwo$, as $\equivetav\subseteq\relsym$. By compatibility (for $\ctx=\ctxhole\tm$), $\var\tm\rel(\la\vartwo\var\vartwo)\tm$ and by reduction and transitivity, $\var\tm\rel\var\vartwo\esub\vartwo\tm$. This means that at least $(\var\tm,\var\vartwo\esub\vartwo\tm)$ must be included in the relation $\relsym$, which is a subcase of $\equivrad$, which itself can be implied by the $\equivlid$ rule and structural equivalence $\streq$. As net bisimilarity does not validate Moggi's laws, and we do not currently know how to include them, net bisimilarity is unable to include $\equivetav$.

%Moggi's left identity rule somehow is a $\letexp$-$\eta$-equivalence.

%Now, we call the associated bisimilarity, $\equivx$-mirrored bisimilarity up to $\eta_v$ (and \net bisimilarity up-to $\eta_v$). Using similar techniques, we prove compatibility. One has to add a rule in the mirrored Lassen's closure to account for the modified rule:
%\adr{
%	\begin{tabular}{cccccc}
%		\infer[\mscabseta]{\val\mlasrel\valtwo}{\val\vartwo\mlasrel\valtwo\vartwo & \vartwo ~\text{is fresh}}
%\end{tabular}}


\ignore{
\paragraph{Fixed point combinators are \nafex bisimilar.} As Lassen did with \enf bisimilarity, we can prove the equivalence of call-by-value versions of Curry's and Turing's fixed point combinators:

\[ \curryfix = \la\var{\curryfixaux\curryfixaux}\text{, where } \curryfixaux = \la\varthree{\var\la\vartwo{\varthree\varthree\vartwo}}\]
\[ \turingfix = (\la\varthree{\la\var{\var\la\vartwo{\varthree\varthree\var\vartwo}}})(\la\varthree{\la\var{\var\la\vartwo{\varthree\varthree\var\vartwo}}}) \]

To prove that they are \nafe bisimilar we build a bisimulation containing $\{(\curryfix,\turingfix)\}$.

\[ \relsym \defeq \{(\curryfix,\turingfix), (\la\var\curryfixaux\curryfixaux,\la\var{\var\la\vartwo{\turingfix\var\vartwo}}),(\curryfixaux\curryfixaux,{\var\la\vartwo{\turingfix\var\vartwo}}), \]\[ 
(\var\la\vartwo{\curryfixaux\curryfixaux\vartwo},{\var\la\vartwo{\turingfix\var\vartwo}}),(\var,\var),(\la\vartwo{\curryfixaux\curryfixaux\vartwo},\la\vartwo{\turingfix\var\vartwo}),\]
\[({\curryfixaux\curryfixaux\vartwo},{\turingfix\var\vartwo}),((\var\la\vartwo{\curryfixaux\curryfixaux\vartwo})\vartwo,({\var\la\vartwo{\turingfix\var\vartwo}})\vartwo),(\vartwo,\vartwo)\} \]

$\relsym \subseteq \opnafep{\relsym}$ by construction (we start with $(\curryfix,\turingfix)$ and we add to $\relsym$ what is needed for each element to satisfy one \nafe case), and sym(R) is a \nafe simulation as well, hence $\curryfix \nafebisim \turingfix$.

This relation is similar to the one defined by Lassen since those terms have "pure lambda-calculus" normal forms (nothing more can be done at the end with VSC and explicit substitutions).
}