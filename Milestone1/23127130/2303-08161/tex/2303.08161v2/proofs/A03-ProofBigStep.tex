% !TEX root = main.tex

%\chapter{Proofs of Big Step systems' completeness for the VSC calculi}
%These two proofs are a bit intricated, but not very complicated. There is also another way to do so by choosing exactly one reduction strategy for the VSC with fits better with the big steps system.
%
%
%\subsection{VSC with practical values big step system completeness proof}
%
%We refine the small steps system to be more precise (which will help for the upcoming proofs).
%\paragraph{Small-Step.} VSC small-step semantics $\to$ can be formulated more clearly using the following rules for derivation. (including a finer way to write $\tom$ and $\toe$).
%\begin{center}
%	\input{VSC-practical-small-step}
%\end{center}
%
%\begin{lemma}
%	$\tm \tovsc \tmp$ and  $\tmp \bsvsct k \ntm \Rightarrow \tm \bsvsct {k+1} \ntm$
%\end{lemma}
%
%\begin{proof} \label{proof:proof-completeness-big-step-vsc-practical}
%By induction on $(k,d)$ (where $d :\tm \tovsc \tmp$) and case analysis on $\tmp \bsvsct k \ntm$, we build a derivation for $\tm \bsvsct {k+1} \ntm$. Cases of $\tm \tovsc \tmp$:
%\input{VSC-practical-bigstep-completeness-proof}
%
%\end{proof}


%
%\subsection{VSC with theoretical values big step system modulo $\sigma_1$ completeness proof}
%
%\paragraph{Small steps.} The same small steps system is used as in the previous subsection but with theoretical values instead of practical values.
%
%\begin{lemma}
%	$\tm \tovsc \tmp$ and  $\tmp \bsvsct k \fire \Rightarrow \tm \bsvsct {k+1} \fire$
%\end{lemma}
%
%\begin{proof} \label{proof:proof-completeness-big-step-vsc-theoretical}.
%	By induction on $(k,d)$ (where $d :\tm \tovsc \tmp$) and case analysis on $\tmp \bsvsct k \fire$, we build a derivation for $\tm \bsvsct {k+1} \fire$. Cases of $\tm \tovsc \tmp$:
%	\input{VSC-theoretical-sigma1-bigstep-completeness-proof}
%	
%\end{proof}

\subsection{Proof of Equivalence of Small-Step and Big-Step Operational Semantics}
In this subsection, we give the details for the proof of completeness for the big step system we introduce for the Value Substitution Calculus using the diamond property.

\subsubsection{Preliminaries}
Some generalities about subreductions and their properties with a calculus that has the diamond property (or the Random Descent property).

\paragraph{General Properties of Random Descent}
The diamond property implies the Random Descent (RD) property, that is the following:
\begin{definition}[RD property, Newman]A relation $\to$ has the 
	\emph{Random Descent (RD)} property if for each element $\tm$, all maximal sequences from $\tm$ have the same length and---if it is finite---they all end in the same element.
\end{definition}

\begin{lemma}[Completeness of subreductions]\label{lem:RD_completeness} Let $\to$ be a reduction which satisfies the RD property, and  $\tos \subseteq \to$.
	If  $\tos$ and  $\to$ have the same normal forms, and $\tmn$ is $\to$-normal, then: 
	\[\tm \to^k \tmn  \iff \tm \tos^k \tmn \]
	
	%	\begin{enumerate}
		%		\item The following are equivalent
		%		\begin{enumerate}
			%			\item  	$\tos$ and  $\to$ have the same normal forms
			%			\item  if $\tmu$ is $\tos$-normal, then $\tmu$ is $\to$-normal,  
			%			\item if $\tm \to \tm'$ then exists $\tm''$ such that $\tm \tos \tm''$
			%		\end{enumerate}
		%		
		%		\item If  $\tos$ and  $\to$ have the same normal forms, and $\tmn$ is $\to$-normal, then: 
		%		\[\tm \to^k \tmn  \iff \tm \tos^k \tmn \]
		%		
		%	\end{enumerate}
	
\end{lemma}
\begin{proof}By induction on $k$.
	\begin{itemize}
		\item $k=0$. Trivial.
		\item $k\geq 1$. Assume $\tm \to \tm' \to^{k-1} \tmn $. By   assumption, $\tm$ is not a $\tos$-$ \nf $. Hence it exists $\tm''$ such that $\tm \tos \tm''$.
		Since $\tos \subseteq \to$, then $\tm \to \tm''$ and so by RD property $\tm'' \to^{k-1} \tmn$. By \ih,  $\tm'' \tos^{k-1},   \tmn$, hence $ \tm \tos^k \tmn  $.\qedhere
	\end{itemize}
	
\end{proof}

\subsubsection{Big-Steps/Small-Steps for the Value Substitution Calculus }
\label{proof:proof-completeness-big-step-vsc-practical}
\paragraph{A constrained reduction to model the big steps semantics}
In the Value Substitution Calculus,we define a subreduction $\tos \subseteq \tovsc$ is defined as follows. 
\begin{center}
	$\begin{array}{r@{\hspace{.5cm}}rlll}
		%		\textsc{Terms} & \tm, \tmtwo, \tmthree & \grameq & \var\mid \val \mid \tm\tmtwo \mid \tm\esub\var\tmtwo
		%		\\
		%		\textsc{Values} & \val  & \grameq  & \la\var\tm
		%		\\
		%		\textsc{Evaluation Contexts} & \evctx & \grameq &  \ctxhole\mid \tm\evctx\mid \evctx\tm \mid \evctx\esub{\var}{\tmtwo} \mid \tm\esub{\var}{\evctx}
		%		\\
		%		
		%		\textsc{Substitution Contexts} & \sctx & \grameq &  \ctxhole\mid \sctx\esub{\var}{\tmtwo}
		%		\\
		%		\textsc{ Inert Substitution Contexts} & \isctx & \grameq &  \ctxhole\mid \isctx\esub{\var}{\itm} 
		%		\\	
		\textsc{Substitution-Restricted Evaluation Contexts} & \ievctx & \grameq &  \ctxhole\mid \tm\ievctx\mid \ievctx\tm \mid \ievctx\esub{\var}{\itm} \mid \tm\esub{\var}{\ievctx}\\
	\end{array}
	$\end{center}

\begin{center}
	$\begin{array}{c@{\hspace{.5cm}}rcc}
		\textsc{Rule at Top Level} & \textsc{Contextual closure} \\
		\isctxp{\la\var\tm}\tmtwo \rtos \isctxp{\tm\esub\var\tmtwo} &
		\ievctxp \tm \tos \ievctxp \tmtwo \textrm{~~~if } \tm \rtos \tmtwo
		\\
		\tm\esub\var{\isctxp{\val}} \rtos \isctxp{\tm\isub\var\val} &
		\ievctxp \tm \tos \ievctxp \tmtwo \textrm{~~~if } \tm \rtos \tmtwo
	\end{array}
	$\end{center}

\begin{remark}
	This subreduction is written with the big step system in mind to ease the proof of \reflemma{ss-bs-equivalenceaux1_vsc} and still be a complete subreduction.
\end{remark}

\begin{lemma}
	\label{l:sctx-normal-is-isctx}
	If $\sctxp\tm$ is normal then $\sctx = \isctx$.
\end{lemma}
\begin{proof}
	By induction on $\sctx$.
	\begin{itemize}
		\item $\sctx= \ctxhole$, the result is immediate.
		\item $\sctx = \sctxONE\esub\var\tmtwo$. $\sctxONEp\tm$ is also normal so by induction $\sctxONE = \isctxONE$. If $\tmtwo$ is not an inert, either $\tmtwo$ reduces or $\tmtwo$ is a value and the whole term reduces which contradicts the hypothesis that $\sctxp\tm$ is normal. Hence $\tmtwo = \itm$ and so $\sctx = \isctxONE\esub\var\itm$ \ie $\sctx = \isctx$.\qedhere
	\end{itemize}
\end{proof}

\begin{lemma}
	\label{l:tos-normal-is-tovsct-normal}
	If $\tm$ is $\tos$-normal, then $\tm$ is $\tovsc$-normal.
\end{lemma}
\begin{proof}
	By induction on the structure of $\tm$.
	\begin{itemize}
		\item $\tm =\var$ or $\tm = \la\var\tmp$, the result is immediate
		\item $\tm = \tmrone\tmrtwo$, then $\tmrone$ and $\tmrtwo$ are $\tos$-normal, and by induction are $\tovsc$-normal. There is only one possibility for $\tm$ to $\tovsc$-reduce.
		
		Suppose $\tmrone\tmrtwo \tom \tmp$ then $\tmrone =\sctxp{\la\var\tmronep}$.
		Since $\tmrone$ is a $\tovsc$ normal form, by \reflemma{sctx-normal-is-isctx} we have that $\tmrone = \isctxp{\la\var\tmronep'}$, which contradicts the assumption that $\tm$ is a $\tos$-normal form.
		\item $\tm = \tmrone\esub\var\tmrtwo$, , then $\tmrone$ and $\tmrtwo$ are $\tos$-normal, and by induction are $\tovsc$-normal. There is only one possibility for $\tm$ to $\tovsc$-reduce.
		
		Suppose $\tmrone\esub\var\tmrtwo \toeabs \tmp$ then $\tmrtwo =\sctxp{\val}$.
		Since $\tmrtwo$ is a $\tovsc$ normal form, by \reflemma{sctx-normal-is-isctx} we have $\tmrtwo = \isctxp{\valtwo}$, which contradicts the assumption that $\tm$ is a $\tos$-normal form.\qedhere
	\end{itemize}
	
\end{proof}


\begin{corollary}
	\label{cor:tovsct-and-tos-normal-forms}
	$\tovsc$ and $\tos$ have the same normal forms.
\end{corollary}

\begin{corollary}[Completeness of $\tos$] \label{prop:S_completeness}$ \tm \tovsc^k \tmn  \iff \tm \tos^k \tmn $ 
\end{corollary}
\begin{proof} 
Direct corollary of \Cref{lem:RD_completeness} and Corollary \ref{cor:tovsct-and-tos-normal-forms}.
	\end{proof}
	
	\paragraph{Big Steps/Small Steps via $\tos$}
	
	\begin{lemma}
		\label{l:ss-bs-equivalenceaux1_vsc}
		$\tm \tos \tmp$ and  $\tmp \bsvsct k \ntm \Rightarrow \tm \bsvsct {k+1} \ntm$
	\end{lemma}
	
	\begin{proof}
		Straightforward proof by structural induction. Note that substitutivity is never  required.
	\end{proof}



\gettoappendix{l:ss-bs-equivalence_vsce}

	
\begin{proof}
	$(\Rightarrow)$ by induction on the $(\tm \bsvsct k \ntm)$ derivation.
	
	$(\Leftarrow)$ by induction on $k$ using the fact that $\tm \tovsc^k \tmn$ implies $\tm \tos^k \tmn$  (by \Cref{prop:S_completeness}), and the \reflemma{ss-bs-equivalenceaux1_vsc}.
\end{proof}
