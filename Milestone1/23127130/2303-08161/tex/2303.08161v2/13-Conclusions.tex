% !TEX root = main.tex
\section{Conclusions}
Motivated by the fact that Lassen's enf bisimilarity $\eqenf$---the normal form bisimilarity of reference in \cbv---does not identify $\Omega$-terms and commuting $\letexp$s, we introduced \emph{net bisimilarity} $\eqnet$, which does identify them. It turns out, however, that $\eqnet$ does not validate Moggi's laws nor $\etav$, which are instead validated by $\eqenf$. Additionally, it is unclear how to extend either enf or net bisimilarity as to catch the other one.

Such a problematic duality led us to develop a sharp analysis of \cbv and of the principles that can be validated or not by normal form bisimulations. The analysis shows that the semantic landscape of \cbv is considerably richer and more sophisticated than the \cbn one. %In particular, cost-sensitive program equivalences are of interest in \cbv, yet contextual equivalence is cost-insensitive.

Concretely, our analysis contributed two further equivalences. First, a naive bisimilarity $\eqncbv$, that mainly provides a better understanding of Lassen's tricky definition of enf simulations. Second, the type equivalence $\equivtype$ induced by Ehrhard's multi types, which subsumes both enf and net bisimilarity, and includes $\etav$-equivalence, while retaining their cost-sensitive aspect. In practice, $\equivtype$ is not really usable for comparing programs, but it provides a sharp theoretical tool.

\paragraph{Future Work} Type equivalence suggests that it could be possible to find a normal form bisimilarity merging the enf and net ones. We are actively working on this challenging problem. A related question is finding an axiomatization of $\equivtype$, for which some sort of separation theorem should be developed.
We would also like to investigate how net bisimilarity $\eqnet$ relates to the topics connected to $\eqenf$, such as  game semantics \cite{DBLP:conf/lics/JaberM21}, extensions with effects \cite{DBLP:conf/esop/LagoG19,DBLP:conf/fossacs/BiernackiLP19,biernacki_et_al:LIPIcs:2020:12329}, and the $\pi$-calculus \cite{DBLP:journals/tcs/DurierHS22}.