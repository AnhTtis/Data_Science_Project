% !TEX root = main.tex
\section{Naive \cbv Bisimilarity}
\label{sect:naive}
%%%%%%%%%%%%%%%%
%%%%%%%%%%%%%%%%
If one takes Sangiorgi's \cbn nf-similarity $\leqcbn$ (\refdef{cbn-nfs}) and simply replaces weak head reduction with one of the weak \cbv reductions (weak, left, or right) then one obtains notions of \cbv nf-similarity. %Their compatibility can be proved with the method that will be explained in the next section, obtaining soundness with respect to contextual equivalence. 

We define the similarity induced by weak reduction. Left or right reductions induce different similarities but with the same pros and cons that are discussed below. Let $\bsweak $ and $\bsweakdiv$ be big-step termination and divergence with respect to weak \cbv reduction $\tow$.
\begin{definition}[Naive nf-bisimulation for Call-by-Value]
A relation $\relsym$ is a \emph{naive (\cbv) nf-simulation} if $\relsym\subseteq\relncbv$, where $\tm \relncbv \tmp$ holds whenever $\tm,\tmp$ satisfy one of the following clauses:
	\begin{center}
	$\begin{array}{r@{\hspace{.3cm}}r@{\hspace{.3cm}}l@{\hspace{.3cm}}l@{\hspace{.3cm}}lll}
	\textup{(nai 1)} & && \tm\bsweakdiv & \ie ~ \text{has no} \tow \text{-normal form.}
	\\
	\textup{(nai 2)} & \tm \bsweak \var & \text{and} & \tmp \bsweak \var &
	\\
	\textup{(nai 3)} & \tm \bsweak \la\var\tm_1 & \text{and} &\tmp \bsweak \la\var\tmp_1 & \text{with}~ \tm_1 \rel \tmp_1
	\\
	\textup{(nai 4)} & \tm \bsweak \ntm = \ntmONE\ntmTWO&\text{and}& \tmp \bsweak \ntmtwo = \ntmONEtwo\ntmTWOtwo &\text{with}~\ntmONE\rel\ntmONEtwo ~ \text{and}~ \ntmTWO\rel\ntmTWOtwo
	\end{array}$
\end{center}
%\begin{enumerate}
%		\item $\tm\bsweakdiv$;
%		\item $\tm \bsweak \var$ and $\tmp \bsweak \var$;
%		\item $\tm \bsweak \la\var\tm_1$ and $\tmp \bsweak \la\var\tmp_1$ with $\tm_1 \rel \tmp_1$;
%		\item $\tm \bsweak \ntm = \ntmONE\ntmTWO$ and $\tmp \bsweak \ntmtwo = \ntmONEtwo\ntmTWOtwo$ with $\ntmONE\rel\ntmONEtwo$ and $\ntmTWO\rel\ntmTWOtwo$.
%	\end{enumerate}
	\emph{Naive nf-similarity} $\leqncbv$ is the largest naive nf-simulation.
	\end{definition}

Naive (bi)similarity seems defined very naturally, and yet it does not validate \emph{any} of the equivalences of the previous section. Let's discuss $\equivomv$ and $\equivlid$.
\begin{enumerate}
\item
 \emph{$\Omega$-equivalence}: $\equivomv$ is not validated by $\leqncbv$, and $\Omega$-terms are not minimal for $\leqncbv$, in contrast with the fact that \cbn $\Omega$-terms are minimal for Sangiorgi's similarity $\leqcbn$. For instance, $\Omega$ is $\leqncbv$-minimal, but $\Omega^L$ is an $\Omega$-term and one has $\Omega \leqncbv \Omega^L$, because $\Omega\bsweakdiv$, but not $\Omega^L \not\leqncbv \Omega$, because $\Omega^L$ is $\tow$-normal.


\item \emph{Left identity}: the equivalence $\equivlid$ is not validated by $\leqncbv$. In Plotkin's calculus $\Id (\var\tm)$ does not reduce to $\var\tm$, because $\var\tm$ is not a value---more generally this happens for all normal open terms that are not values. Therefore, $\Id (\var\tm)$ and $\var \tm$ are $\leqncbv$-incomparable.
\end{enumerate}

\adr{\paragraph{$\beta_v$-Conversion}
	From the definition of $\leqncbv$, it immediately follows that naive bisimilarity contains the $\rtobv$ root rule, that is, that if $\tm \rtobv \tmtwo$ then $\tm\eqncbv \tmtwo$, simply because $\tm$ and $\tmtwo$ have the same left normal form. Since $\eqncbv$ is a compatible equivalence relation, it turns out that $\eqncbv$ contains the whole of $\betav$-conversion $=_{\betav}$\cadr{, thus it contains left reduction as well as weak and right reductions.}{.}
	
	\begin{proposition}[$\betav$-conversion is validated by naive bisimilarity]
		If $\tm =_{\betav} \tmtwo$ then $\tm \eqncbv \tmtwo$.
\end{proposition}}

\paragraph{Naive Similarity and Fix-Points} \cadr{It is nonetheless }{Despite its naivety, it is }possible to prove that the usual \cbv variants of Curry's and Turing's fix-point combinators $\curryfix$ and $\turingfix$ are naively similar, as we now show.
\begin{center}
$\begin{array}{c\colspace |\colspace  c}
\textsc{Curry's fix-point} & \textsc{Turing's fix-point}
\\
 \curryfix \defeq \la\var{\curryfixaux\curryfixaux}\text{, where } \curryfixaux= \la\varthree{\var\la\vartwo{\varthree\varthree\vartwo}}
 &
 \turingfix \defeq (\la\varthree{\la\var{\var\la\vartwo{\varthree\varthree\var\vartwo}}})(\la\varthree{\la\var{\var\la\vartwo{\varthree\varthree\var\vartwo}}}) 
 \end{array}$
 \end{center}
Let's build a naive (bi)simulation $\relsym$ relating $\curryfix$ and $\turingfix$. The relation $\relsym$ must contain the pair $(\curryfix,\turingfix)$.  Both terms $\tow$-evaluate to an abstraction ($\turingfix \to \la\var{\var\la\vartwo{\turingfix\var\vartwo}}$). Hence their weak normal forms are (abstractions) $\la\var{\curryfixaux\curryfixaux}$ and $\la\var{\var\la\vartwo{\turingfix\var\vartwo}}$ which must satisfy the third clause for $\leqncbv$, that is, their bodies under the abstraction $(\curryfixaux\curryfixaux,\var\la\vartwo{\turingfix\var\vartwo})$ must appear in $\relsym$. Since $\curryfixaux\curryfixaux \tow \var \la\vartwo{\curryfixaux\curryfixaux\vartwo}$, $\relsym$ must contain the fourth clause requirements, that is, $\var \rel \var$ and $\la\vartwo{\curryfixaux\curryfixaux\vartwo} \rel \la\vartwo{\turingfix\var\vartwo}$. As a result, $\var \rel \var$ needs to be added to the simulation (for all possible choice of variables when opening the abstraction at the first step), and we continue on adding $(\la\vartwo{\curryfixaux\curryfixaux\vartwo}, \la\vartwo{\turingfix\var\vartwo})$ to $\relsym$, then adding any pair of terms needed so that $\relsym\subseteq\relncbv$. By repeating this process, we eventually fall back to $\var\la\vartwo{\curryfixaux\curryfixaux\vartwo} \rel \var\la\vartwo{\turingfix\var\vartwo}$ which is already in the built relation $\relsym$, which means it now satisfies $\relsym \subseteq \relncbv$, that is, $\relsym$ is a naive simulation. Since similarly the symmetric relation $sym(\relsym)$ also satisfies $sym(\relsym) \subseteq \ncbvfp{sym(\relsym)}$, $\relsym$ is actually a naive \emph{bi}simulation.

The full relation proving the next proposition is:
\begin{center}$
\relsym \defeq \{(\var,\var) \mid \var \text{ a variable}\} ~\cup~ 
\{~~(\curryfix,\turingfix), (\la\var\curryfixaux\curryfixaux,\la\var{\var\la\vartwo{\turingfix\var\vartwo}}),\} ~~\cup $
\end{center} \begin{center}$
\{ (\curryfixaux\curryfixaux,{\var\la\vartwo{\turingfix\var\vartwo}}), ~~ (\var\la\vartwo{\curryfixaux\curryfixaux\vartwo},{\var\la\vartwo{\turingfix\var\vartwo}}),~~ (\la\vartwo{\curryfixaux\curryfixaux\vartwo},\la\vartwo{\turingfix\var\vartwo})\mid \var\text{ a variable}\}~~\cup
$\end{center}\begin{center}$
 \{({\curryfixaux\curryfixaux\vartwo},{\turingfix\var\vartwo}), ((\var\la\vartwop{\curryfixaux\curryfixaux\vartwop})\vartwo,({\var\la\vartwop{\turingfix\var\vartwop}})\vartwo)\mid \var\text{ and }\vartwo \text{ variables}\}$
\end{center} 

\begin{proposition}
\label{prop:naive-fix-points-equiv}
$\curryfix \eqncbv \turingfix$, that is, Curry's and Turing's fix-points are naive bisimilar.
\end{proposition}