% !TEX root = main.tex
\begin{abstract}
Normal form bisimilarities are a natural form of program equivalence resting on open terms, first introduced by Sangiorgi in call-by-name. The literature contains a normal form bisimilarity for  Plotkin's call-by-value $\lambda$-calculus, Lassen's \emph{enf bisimilarity}, which validates all of Moggi's monadic laws and can be extended to validate $\eta$. It does not validate, however, other relevant principles, such as the identification of meaningless terms---validated instead by Sangiorgi's bisimilarity---or the commutation of $\letexp$s. These shortcomings are due to issues with open terms of Plotkin's calculus. We introduce a new call-by-value normal form bisimilarity, deemed \emph{net bisimilarity}, closer in spirit to Sangiorgi's and satisfying the additional principles. We develop it on top of an existing formalism designed for dealing with open terms in call-by-value. It turns out that enf and net bisimilarities are \emph{incomparable}, as net bisimilarity does not validate Moggi's laws nor $\eta$. Moreover, there is no easy way to merge them. To better understand the situation, we provide an analysis of the rich range of possible call-by-value normal form bisimilarities, relating them to Ehrhard's relational model.
\end{abstract}