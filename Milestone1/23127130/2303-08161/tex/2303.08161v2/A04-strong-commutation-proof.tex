% !TEX root = main.tex
\section{Proofs from \refsect{benchmarks-vsc} (Equational Benchmarks and the Value Substitution Calculus)}
In this section, we recall the proof that structural equivalence strongly commutes with $\tovsc$. The proof is identical to the one given in \cite{accattoli+paolini-vsc}, it is here presented for the interested reader to be able to see which fragments of structural equivalence independently strongly commutes.


\gettoappendix{prop:strong-bisimulation}
\newcommand{\eqz}{\equiv_0}
\begin{proof}
Define $\eqz$ as the context closure of $\equivsone\cup \equivexsthree\cup \equivass \cup \equivcom$.
 We have $\streq=\eqz^*$. We prove that:
\begin{equation}
\mbox{if $t_0\eqz t_1 \Rew{a} s_1$ then there exists $w$ such that $t_0\Rew{a} w \streq s_1$}
\label{eq:bis}
\end{equation}
The statement then follows by induction on the reflexive and transitive closure of $\eqz$. Let us show that: the reflexive case is trivial and if $t_0\eqz t_0'\eqz^k t_1\Rew{a} s_1$ then by \ih\ exists $w$ such that  $t_0'\Rew{a} w \streq s_1$ and by (\ref{eq:bis}) there exists $w'$ such that  $t_0\Rew{a} w'\streq w\streq s_1$.\\

The proof of (\ref{eq:bis}) is by induction on $\eqz$. Actually, before to proceed with the proof one should first prove the following two easy substitutivity properties:
\begin{enumerate}
  \item \label{l:eqo-stability-two} If $t \eqz t'$ then   $t\isub{x}{u} \eqz t'\isub{x}{u}$.
  \item \label{l:eqo-stability-one} If $u \eqz u'$ then   $t\isub{x}{u} \streq t\isub{x}{u'}$.
  \end{enumerate}      
Used in the inductive cases for the ES. We omit their proofs, which are straightforward inductions.

\begin{itemize}
\item Base cases:
\begin{itemize}
 \item \emph{Commutativity}: let $t_0= t\esub\vartwo{u}\esub\var{s}\equivcom t\esub\var{s}\esub\vartwo{u}=t_1$ with $x\notin\fv{u}$ and $y\notin\fv{s}$. If $t_1\Rew{a} s_1$ because:
 \begin{itemize}
  \item $t\Rew{a}  t'$ then $t_0=t\esub\vartwo{u}\esub\var{s}\Rew{a}  t'\esub\vartwo{u}\esub\var{s}\equivcom t'\esub\var{s}\esub\vartwo{u}=s_1$.
  \item $u\Rew{a}  u'$ or $s\Rew{a}  s'$ then it is similar to the previous case.
  \item $s=\sctxp\val$ and $t\esub\var{\sctxp\val}\esub\vartwo{u}\toe \sctxp{t\isub\var\val}\esub\vartwo{u}=s_1$. Then:
  \[\begin{array}{llllll}
  t_0&\toe & \sctxp{t\esub\vartwo{u}\isub\var\val}\\
  &=& \sctxp{t\isub\var\val \esub\vartwo{u}} \\
    &\equivcom& \sctxp{t\isub\var\val} \esub\vartwo{u}&=&s_1
  \end{array}\]
  \item The case where $u=\sctxp\val$ and $t\esub\var{s}\esub\vartwo{\sctxp\val}\toe \sctxp{t\esub\var{s}\isub\vartwo\val}=s_1$is similar to the previous one.  \end{itemize}

\item \emph{Sigma 1}: let $t_0=t\esub\var{s} \tmtwo   \equivsone  (\tm\tmtwo)\esub\var{s}=t_1$ with $x\notin\fv{u}$. If $t_1\Rew{a} s_1$ because:

 \begin{itemize}
  \item $t\Rew{a}  t'$ then $t_0=t\esub\var{s} \tmtwo \Rew{a}  t'\esub\var{s} \tmtwo \equivsone (t' \tmtwo )\esub\var{s}=s_1$.
  \item $s\Rew{a}  s'$ or $u\Rew{a}  u'$ then it is similar to the previous case.
  \item $s=\sctxp\val$ and $t_1=(\tm\tmtwo)\esub\var{\sctxp\val}\toe \sctxp{(\tm\tmtwo)\isub\var\val}=s_1$.
Then:
  \[\begin{array}{llllll}
  t_0&=& t\esub\var{\sctxp\val} \tmtwo \\
  &\toe& \sctxp{t\isub\var\val} \tmtwo \\
    &\equivsone& \sctxp{t\isub\var\val  \tmtwo }\\
    &=& \sctxp{(\tm\tmtwo)\isub\var\val}&=&s_1
  \end{array}\]

  \item $t= \l y. t'$ and $t_1=((\l y. t') \tmtwo )\esub\var{s} \tom t'\esub\vartwo{u}\esub\var{s}$. Then:
  \[\begin{array}{llllll}
  t_0&=& (\l y. t')\esub\var{s} \tmtwo \\
  &\tom& t'\esub\vartwo{u}\esub\var{s}&=&s_1
  \end{array}\]
  \end{itemize}

Note that here it is reflexivity of $\streq$ which is used.

 \item The case symmetric to the previous one, \ie\ $t_0= (\tm\tmtwo)\esub\var{s}   \equivsone  t\esub\var{s} \tmtwo=t_1$ with $x\notin\fv{u}$, is proved analogously. It shall be so for all following cases, so we simply omit the symmetric cases.

\item \emph{Extended sigma 3}: let $t_0=\tm\tmtwo\esub\var{s}  \equivexsthree  (\tm\tmtwo)\esub\var{s}=t_1$ with $x\notin\fv{t}$. If $t_1\Rew{a} s_1$ because:

 \begin{itemize}
  \item $t\Rew{a}  t'$ then $t_0=\tm\tmtwo\esub\var{s}\Rew{a}  t' \tmtwo \esub\var{s} \equivexsthree (t' \tmtwo )\esub\var{s}=s_1$.
  \item $s\Rew{a}  s'$ or $u\Rew{a}  u'$ then it is similar to the previous case.
  \item $s=\sctxp\val$ and $t_1=(\tm\tmtwo)\esub\var{\sctxp\val}\toe \sctxp{(\tm\tmtwo)\isub\var\val}=s_1$. Then:
  \[\begin{array}{llllll}
  t_0&=& \tm\tmtwo\esub\var{\sctxp\val}\\
  &\toe & \tm\, \sctxp{\tmtwo\isub\var\val}\\
    &\equivexsthree & \sctxp{\tm\tmtwo\isub\var\val}\\
    &=& \sctxp{(\tm\tmtwo)\isub\var\val}&=&s_1
  \end{array}\]

  \item $t= \l y. t'$ and $t_1=((\l y. t') \tmtwo )\esub\var{s} \tom t'\esub\vartwo{u}\esub\var{s}$. Then:
  \[\begin{array}{llllll}
  t_0&=& (\l y. t') \tmtwo \esub\var{s}\\
  &\tom& t'\esub\vartwo{\tmtwo\esub\var{s}}\\
  &\equivass& t'\esub\vartwo{u}\esub\var{s}&=&s_1
  \end{array}\]
  \end{itemize}
 

\item \emph{Associativity of ES}: let $t_0=t\esub\vartwo{\tmtwo\esub\var{s}}  \equivass  t\esub\vartwo{u}\esub\var{s}=t_1$ with $x\notin\fv{t}$. If $t_1\Rew{a} s_1$ because:

\begin{itemize}
\item $t\Rew{a} t'$ then $t_0\Rew{a} t'\esub\vartwo{\tmtwo\esub\var{s}}\equivass t'\esub\var\tmtwo\esub\var{s}=s_1$.
\item $u\Rew{a} u'$ or $s\Rew{a} s'$ it is analogous to the previous case.
\item $s=\sctxp\val$ and $t_1 \toe \sctxp{t\esub\vartwo{u}\isub\var\val}=s_1$. Then
  \[\begin{array}{llllll}
  t_0&=& t\esub\vartwo{u\esub\var{\sctxp\val}}\\
  &\toe& t\esub\vartwo{\sctxp{u\isub\var\val}}\\
  &\equivass& \sctxp{t\esub\vartwo{u\isub\var\val}}\\
  &=& \sctxp{t\esub\vartwo{u}\isub\var\val}&=&s_1
  \end{array}\]

\item $u=\sctxp\val'$ and $t_1=\sctxp{t\esub\var{\sctxtwop\val}} \toe \sctxp{\sctxtwop{t\isub\var\val}}$. Then $t_0=t\esub\var{\sctxp{\sctxtwop\val}}\toe \sctxp{\sctxtwop{t\isub\var{\val}}}=s_1$. Note that here it is reflexivity of $\streq$ which is used.
\end{itemize}
 \end{itemize}


\item Inductive cases. We only show the interesting ones:
\begin{itemize}
%\item Lambda: $t_0=\la\var\tm\eqz \la\var\tm'=t_1\Rew{a} \la\var\tm''=s_1$. Then by \ih\ we get that there exists $w$ such that  $t\Rew{a} w\eqz t''$ and so $t_0\Rew{a} \l x.w\eqz s_1$.
\item Application: the only case where the reduction interact with the contextual closure is $t_0=\sctxp{\la\var\tm} \tmtwo  \eqz \sctxp{\la\var\tm'} \tmtwo  = t_1 \Rew{a} \sctxp{t'\esub\var\tmtwo}=s_1$. Then $t_0\Rew{a} \sctxp{t\esub\var\tmtwo} \eqz \sctxp{t'\esub\var\tmtwo}=s_1$. The variants  $t_0=\sctxp{\la\var\tm} \tmtwo  \eqz \sctxp{\la\var\tm} \tmtwo ' = t_1 \Rew{a} \sctxp{t\esub\var{\tmtwo'}}=s_1$ and $t_0=\sctxp{\la\var\tm} \tmtwo  \eqz \sctxtwop{\la\var\tm} \tmtwo  = t_1 \Rew{a} \sctxtwop{t\esub\var\tmtwo}=s_1$ are analogous. All other inductive cases for application are straightforward.

\item Explicit substitution. We only show the interesting cases. 
\begin{itemize}
\item $t_0=t\esub\var{\sctxp\val} \eqz t'\esub\var{\sctxp\val} = t_1 \Rew{a} \sctxp{t'\isub\var\val}=s_1$. Then by the first substitutivity property we obtain $t_0\Rew{a} \sctxp{t\isub\var\val} \eqz \sctxp{t'\isub\var\val}$.
\item $t_0=t\esub\var{\sctxp\val} \eqz t\esub\var{\sctxp\valtwo} = t_1 \Rew{a} \sctxp{t\isub\var\valtwo}=s_1$. Then by the second substitutivity property we obtain $t_0\Rew{a} \sctxp{t\isub\var\val}\streq \sctxp{t\isub\var\valtwo}$.\qedhere
\end{itemize}
\end{itemize}
\end{itemize}
%\qed
\end{proof}