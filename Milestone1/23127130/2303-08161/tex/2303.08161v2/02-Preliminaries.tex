% !TEX root = main.tex
\section{Preliminaries}
\label{sect:preliminaries}
\paragraph{Contexts} All along the
paper we use (many notions of) \emph{contexts}, \ie terms with exactly one hole, noted $\ctxhole$. Plugging a term $\tm$ in a context $\ctx$, noted $\ctxp{\tm}$, possibly~captures free variables of $\tm$. For instance $(\la\var\ctxhole)\ctxholep\var = \la\var\var$, while $(\la\var\vartwo)\isub\vartwo\var = \la\varthree\var$.


\paragraph{Preorders} We shall mostly deal with simulations, rather than bisimulations, since the result for equivalences shall always follow by simply considering the symmetric notions. Given a preorder/similarity $\precsim_X$ for some $X$, we  denote with $\simeq_X$ the corresponding equivalence/bisimilarity.

\paragraph{(In)Equational Theories and Compatibility} Good program preorder/equivalences are \emph{(in)equational theories}, that is, they contain the reduction of the calculus and they are \emph{compatible}, defined as: if $\tm\precsim\tmp$ then $\ctxp\tm \precsim\ctxp\tmp$ for all contexts $\ctx$, that is, that they are stable by context closure. Reduction is usually trivially included in similarities while compatibility is usually non-trivial to prove.


\paragraph{Contextual Equivalence} The standard of reference for program equivalences is contextual equivalence, that can be defined abstractly as follows.
\begin{definition}[Contextual Preorder and Equivalence] Given a language of terms $\mathcal{T}$ with its associated notion of contexts $\ctx$ and predicate stating the termination of evaluation $\bs{}_{e}$, we define the associated \emph{contextual preorder} $\leqcp{e}$ and \emph{contextual equivalence} $\equivcp{e}$ as follows:
	\begin{itemize}
		\item $\tm \leqcp{e} \tmp$ if $\ctxp\tm ~{\bs{}_{e}}$ implies $\ctxp\tmp ~{\bs{}_{e}}$ for all contexts $\ctx$ such that $\ctxp{\tm}$ and $\ctxp\tmp$ are closed terms. 
		\item $\tm \equivcp{e} \tmp$ is the equivalence relation induced by $\leqc$, that is, $\tm \equivcp{e} \tmp \iff \tm \leqcp{e} \tmp$ and $\tmp \leqcp{e} \tm$.
	\end{itemize}
\end{definition}


A relation $\relsym$ is sound for the contextual preorder $\leqcp{e}$ when $\relsym \subseteq \leqcp{e}$. Soundness follows from compatibility and \emph{adequacy for $\bs{}_e$}, defined as: if $\tm\rel\tmp$  then $\tm \bs{}_e$ implies $\tmp \bs{}_e$.


\begin{proposition}
	\label{prop:congruence-included-contextual-equivalence}
	Let $\precsim$ be a compatible and adequate preorder. Then $\precsim \subseteq \leqcp{e}$.
\end{proposition}

\begin{proof}
	Suppose $\tmtwo \precsim \tmtwop$.
	Let $\ctx$ any closing context for $\tmtwo$ and $\tmtwop$.
	By compatibility, $\ctxp{\tmtwo} \precsim \ctxp{\tmtwop}$.
	By adequacy, $\ctxp{\tmtwo} \bs{}_e$ implies $\ctxp{\tmtwop} \bs{}_e$, that is, $\tmtwo \leqcp e \tmtwop$.
\end{proof}

We shall see that normal form simulations are defined in such a way that they are adequate, so that soundness follows directly from compatibility. Note that soundness without compatibility is useless: the relation $\relsym \defeq \set{(\var\vartwo,\var\vartwo)}$ is sound but not compatible. 

\paragraph{Diamond} A rewriting notion that shall play a role is the \emph{diamond property}, which is the following one-step strengthening of confluence for a reduction $\to$: if $\tm \to \tmtwo_1$, $\tm\to\tmtwo_2$, and $\tmtwo_1 \neq \tmtwo_2$ then exists $\tmthree$ such that $\tmtwo_1 \to \tmthree$ and $\tmtwo_2 \to \tmthree$. Some well-known facts: the diamond property implies confluence but not vice-versa; if $\to$ is diamond and there is a terminating reduction from $\tm$ then there are no diverging reductions from $\tm$; all reductions to normal form, if any, have the same length. Roughly, the diamond property is a relaxed form of determinism, where non-deterministic choices have no impact on the result nor on the length of the evaluation leading to it.

\paragraph{$\Omega$-terms, Solvability, and Scrutability} A cornerstone of the theory of the (\cbn) $\l$-calculus is the study of what here we call \emph{$\Omega$-terms}, that is, terms that are contextually equivalent to $\Omega$, the paradigmatic looping term. Such a notion has been extensively studied for \emph{head} (\cbn) contextual equivalence, for which $\Omega$-terms are better known as \emph{unsolvable terms}, and often labeled as \emph{meaningless terms}. Actually, the definition of unsolvable term (here omitted) is different, as it does not mention $\Omega$ nor contextual equivalence, but equivalent to the one of $\Omega$-term. Equational theories that identify all unsolvable terms were first studied by \citet{Wad:SemPra:71,DBLP:journals/siamcomp/Wadsworth76} and \citet{DBLP:books/daglib/0016519,solvability-barendregt}, and it is the leading theme of \citeauthor{Barendregt84}'s book \citeyearpar{Barendregt84}. Unsolvable terms can be characterized as those terms that diverge with respect to head reduction, as proved by Wadsworth. 

Adopting \emph{weak} head (\cbn) contextually equivalence gives a different set of $\Omega$-terms. As for the head case, these $\Omega$-terms coincide with a natural set of terms defined without mentioning $\Omega$, namely \emph{(\cbn) inscrutable terms} (for the definition see \refapp{app-vsc}), first studied by \citet{parametricBook} (under the name \emph{not potentially valuable terms}), who provide the following useful characterization, akin to Wadsworth's for unsolvable terms. 
\begin{theorem}[Diverging characterization of \cbn scrutability]
\label{thm:cbn-scrutability-characterization}
A term $\tm$ is \cbn inscrutable (thus a weak \cbn $\Omega$-term) if and only if the weak head reduction of $\tm$ diverges.
\end{theorem}
For instance, $\Omega$ is inscrutable but $\la\var\Omega$ and $\var\Omega$ are not (while $\la\var\Omega$ is unsolvable). From the characterization, it follows that every inscrutable term is unsolvable, but not vice-versa.

In \cbv, there are analogous notions of unsolvable and inscrutable terms, but they lack analogous \emph{diverging characterizations} in Plotkin's \cbv $\l$-calculus. They have been thoroughly studied by \citet{DBLP:journals/pacmpl/AccattoliG22} in the VSC, where instead they admit diverging characterizations. Surprisingly, an equational theory identifying \cbv unsolvable terms is necessarily inconsistent. The right notion of meaningless term in \cbv is actually given by \cbv inscrutable terms. In \refapp{app-vsc}, one can find the technical definition of \cbv inscrutable terms as well as the proof that they coincide with \cbv $\Omega$-terms (relying on the VSC introduced in \refsect{vsc}). 

In this paper, we shall not consider unsolvable terms, as we only consider \emph{weak} reductions. Thus, \cbn/\cbv \emph{$\Omega$-terms} shall always be relative to weak \cbn/\cbv contextually equivalence, and coinciding with \cbn/\cbv inscrutable terms.
