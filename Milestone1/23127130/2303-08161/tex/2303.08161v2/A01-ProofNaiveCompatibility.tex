% !TEX root = main.tex

\section{Proofs from \refsect{naive} (Naive CbV Bisimilarity)}
\label{chapter:proof-compatibility-naive}
In this section, we develop the proof of compatibility for the (weak) naive similarity, following Lassen's variant of Howe method for nf-bisimulations.

\subsection{Proof of Equivalence of Small-Step and Big-Step Operational Semantics} We prove the big-step evaluation predicate sound and complete with respect to the small-step operational semantics (Proposition \ref{l:ss-bs-equivalence_weak}).

\begin{lemma}
	\label{l:aux-ss-bs-equivalence}
	If $\tm\tow\tmp$ and $\tmp\bsw k \ntm$, then $\tm\bsw {k+1} \ntm$.
\end{lemma}

\begin{proof}
	Straightforward proof by structural induction.
\end{proof}


\gettoappendix{l:ss-bs-equivalence_weak}

\begin{proof}
	Trivial using \reflemma{aux-ss-bs-equivalence}.
\end{proof}

\subsection{Lemmas about normal forms and $\lasrel$ and $\lasrelncbv$}
Actually, before proving \refprop{main-lemma_naive} for all terms, we somehow need to prove it only on normal forms. More precisely, we show that $\lasrelsym$ and $\lasrelncbvsym$ coincide on normal forms (\reflemma{lasrelncbv-normal-forms-lasrel-left-to-right} and \reflemma{lasrelncbv-normal-forms-lasrel-right-to-left}).

\begin{lemma}
	\label{l:lasrelncbv-normal-forms-lasrel-left-to-right}
	If $\ntm\lasrelncbv\ntmtwo$ then $\ntm\lasrel\ntmtwo$.
\end{lemma}
\begin{proof}
	By case analysis on the shape of $\ntm$.
\end{proof}


Notice that the next lemma already proves part of the conclusion of the first part of Proposition \ref{prop:ncbv-coherence}.


\begin{lemma}[Constrained Substitutivity of $\lasrelncbv$ on normal forms]
	\label{l:lasrelncbv-normal-forms-substitutive}
	If $\ntm \lasrelncbv \ntmtwo$, $\val \lasrelncbv \valtwo$ and $\ntm\isub\var{\val}$ and $\ntmtwo\isub\var{\valtwo}$ are $\tow$-normal then $\ntm\isub\var{\val} \lasrelncbv \ntmtwo\isub\var{\valtwo}$.
\end{lemma}


\begin{proof}
	By case analysis on the shape of $\ntm$. Cases:
	\begin{itemize}
		\item $\ntm = \var$ and $\ntmtwo = \var$ then $\ntm\isub\var{\val} = \val \lasrelncbv \valtwo = \ntmtwo\isub\var{\valtwo}$.
		
		\item $\ntm = \vartwo$ and $\ntmtwo = \vartwo$ then $\ntm\isub\var{\val} =  \vartwo \lasrelncbv \vartwo = \ntmtwo\isub\var{\valtwo}$.
		
		\item $\ntm = \la\vartwo\tm$ and $\ntmtwo = \la\vartwo\tmp$ with $\tm \lasrel \tmp$
		we have \[\infer{\tm\isub\var{\val} \lasrel \tmp\isub\var{\valtwo}}{\tm \lasrel \tmp & \val \lasrel \valtwo}\]
		hence by case (ncbv 3) $\ntm\isub\var{\val} = \la\vartwo{\tm\isub\var{\val}} \lasrelncbv  \la\vartwo{\tmp\isub\var{\valtwo}} = \ntmtwo\isub\var{\valtwo}$.
		
		
		\item $\ntm = \ntmONE\ntmTWO$ and $\ntmtwo = \ntmONEtwo\ntmTWOtwo$ with $\ntmONE\lasrel\ntmONEtwo$ and $\ntmTWO\lasrel\ntmTWOtwo$. Also, by \reflemma{lasrelncbv-normal-forms-lasrel-left-to-right}, $\val\lasrel\valtwo$.
		
			\[ \infer[\scsub]{\ntmONE\isub\var\val \lasrel \ntmONEtwo\isub\var\valtwo}{\ntmONE\lasrel\ntmONEtwo & \val \lasrel \valtwo} ~\text{and}~ \infer[\scsub]{\ntmTWO\isub\var\val \lasrel \ntmTWOtwo\isub\var\valtwo}{\ntmTWO \lasrel \ntmTWOtwo & \val \lasrel \valtwo}\]
		
		
		Hence, as $\ntm\isub\var\val$ and $\ntm\isub\var\valtwo$ are normal forms, this concludes the proof.
		
		
		
		
	\end{itemize}
\end{proof}

\begin{lemma}
	\label{l:lasrelncbv-normal-forms-lasrel-right-to-left}
	If $\relsym$ is a naive simulation.
	If $\ntm\lasrel\ntmtwo$ then $\ntm\lasrelncbv\ntmtwo$.
\end{lemma}

\begin{proof}
	By induction on the derivation $\ntm \lasrel \ntmtwo$. Cases of the last rule in the derivation of $\ntm\lasrel\ntmtwo$:
	\begin{itemize}
		\item \emph {$\scvar$}\[ \infer[\scvar]{\var \lasrel \var}{} \]
		then $\var \lasrelncbv \var$ by definition of naive.
		\item \emph {$\scabs$} \[ \infer[\scabs]{\ntm = \la\var\tm \lasrel \la\var\tmp = \ntmtwo}{\tm \lasrel \tmp} \]
		then $\ntm \lasrelncbv \ntmtwo$ by definition of naive with $\tm \lasrel \tmp$.
		\item \emph {$\sclift$} \[ \infer[\scabs]{\ntm \lasrel \ntmtwo}{\ntm \rel \ntmtwo} \]
		then since $\relsym$ is a naive simulation $\ntm \relncbv \ntmtwo$, hence by monotonicity of $\ncbvfp\cdot$, $\ntm \lasrelncbv \ntmtwo$.
		\item \emph {$\scapp$} \[ \infer[\scapp]{\ntm = \ntmONE\ntmTWO \lasrel \ntmONEtwo\ntmTWOtwo = \ntmtwo}{\ntmONE \lasrel \ntmONEtwo & \ntmTWO \lasrel \ntmTWOtwo} \]then $\ntm \lasrelncbv \ntmtwo$ by definition of naive with $\ntmONE \lasrel \ntmONEtwo$ and  $\ntmTWO \lasrel \ntmTWOtwo$.
		\item \emph {$\scsub$} \[ \infer[\scsub]{\ntm = \ntmONE\isub\var\val \lasrel \ntmTWO\isub\var\valtwo = \ntmtwo}{\ntmONE \lasrel \ntmTWO & \val \lasrel \valtwo} \]
		by \ih we have $\ntmONE \lasrelncbv \ntmTWO$ and $\val \lasrelncbv \valtwo$. By \reflemma{lasrelncbv-normal-forms-substitutive}, $\ntmONE\isub\var{\val} \lasrelncbv \ntmTWO\isub\var{\valtwo}$.\qedhere
	\end{itemize}
\end{proof}

\subsection{Coherence of simulation, reduction and substitution}
The main difficulty in the proof of \refprop{main-lemma_naive} is the case of the ($\scsub$) rule. It is dealt with the following lemma, that states that substitution behaves nicely with the simulation and the reduction of the calculus.

\gettoappendix{prop:ncbv-coherence}

\begin{proof}
	\begin{enumerate}
		\item We only prove that $\ntmtwo\isub\var\valtwo$ is a normal form, the rest of the conclusion follows by \ref{l:lasrelncbv-normal-forms-substitutive}. By induction on $\ntm$.
		\begin{itemize}
			\item $\ntm=\vartwo$, then $\ntmtwo=\vartwo$ (since $\ntm\lasrelncbv\ntmtwo$), which is a normal form.
			\item $\ntm=\var$, then $\ntmtwo=\var$ (since $\ntm\lasrelncbv\ntmtwo$), hence the result.
			\item $\ntm=\la\vartwo\tm$, then $\ntmtwo=\la\vartwo\tmp$ with $\tm\lasrel\tmp$ (since $\ntm\lasrelncbv\ntmtwo$), we conclude since $\la\vartwo\tmp\isub\var\valtwo$ is a normal form.
			\item $\ntm = \ntmONE\ntmTWO$, then $\ntmtwo=\ntmONEtwo\ntmTWOtwo$ with $\ntmONE\lasrel\ntmONEtwo$ and $\ntmTWO\lasrel\ntmTWOtwo$. By \reflemma{lasrelncbv-normal-forms-lasrel-right-to-left}, $\ntmONE\lasrelncbv\ntmONEtwo$ and $\ntmTWO\lasrelncbv\ntmTWOtwo$. Hence we can apply the \ih and get that $\ntmONEtwo\isub\var\valtwo$ and $\ntmTWOtwo\isub\var\valtwo$ are normal forms. In fact, by a quick analysis, $\ntmONEtwo\ntmTWOtwo\isub\var\valtwo$ is a normal form (otherwise $\ntm\isub\var\val$ would not be normal).
		\end{itemize}
		\item By induction on $\ntm$. Note that the only possible shape for $\ntm$ is $\ntmONE\ntmTWO$, otherwise the substitution of a value does not imply a reduction step.
		
		\begin{itemize}
			\item $\ntmONE\isub\var\val$ is normal and $\ntmTWO\isub\var\val$ is normal. Then $\ntmONE\isub\var\val = \la\vartwo\tmrone$ and $\ntmTWO\isub\var\val = \val_1$.
			By Point 1, $\ntmONEtwo = \la\vartwo\tmrtwo$ and $\ntmTWOtwo=\val_2$ such that $\tmrone\lasrel\tmrtwop$ and $\val_1 \lasrel \val_2$ (by \ref{l:lasrelncbv-normal-forms-lasrel-left-to-right}). Hence, $\ntmtwo\isub\var\val \tow \tmrtwo\isub\vartwo{\val_2}$ and we conclude using the following derivation:
			\[\infer{\tmrone\isub\vartwo{\val_1} \lasrel  \tmrtwo\isub\vartwo{\val_2}}{\tmrone\lasrel\tmrtwo & \val_1 \lasrel \val_2}\]
			
			\item $\ntmONE\isub\var\val\tow\tm_1$ or $\ntmTWO\isub\var\val\tow\tm_1$.
			
			By \ih, $\ntmONEtwo\isub\var\valtwo\tow\tmtwo_1$ or $\ntmTWOtwo\isub\var\valtwo\tow\tmtwo_1$ such that $\tm_1\lasrel\tmtwo_1$, hence we find $\tmtwo$ easily in both cases, such that $\ntmtwo\isub\var\val \tow \tmtwo$ and $\tm\lasrel\tmtwo$.
		\end{itemize}
	\end{enumerate}
\end{proof}


\subsection{Lassen's closure preserves naive simulations}

Finally, we prove that the Lassen's closure of a naive simulation is a simulation.

\gettoappendix{prop:main-lemma_naive}
\begin{proof}
	
% !TEX root = main.tex

%%By induction on $(k,d)$ where $d=$ the size of the derivation of $\tmrone \lasrel \tmrtwo$.

By case analysis on the last rule of the derivation $\tmrone\lasrel \tmrtwo$.
\begin{enumerate}
	\item \emph{Lifting}:
	\[ \infer[(\sclift) ]{\tmrone \lasrel \tmrtwo} {\tmrone \rel \tmrtwo}\text{ and }\tmrone\bsw k \ntm\]
	Since $\relsym$ is a naive simulation, we have $\tmrone\relncbv\tmrtwo$ and $\tmrtwo \bsws \ntmtwo$ for some $\ntmtwo$ such that $\ntm\relncbv\ntmtwo$. Hence  $\ntm\lasrelncbv\ntmtwo$ by monotonicity of $\ncbvfp\cdot$.
	
	%%%%%%%%%%%%
	\item \emph{Variables}:
	\[\infer[(\scvar) ]{\var \lasrel \var}	{} \text{ and } \var\bsw 0 \var\]
	
	hence the result ($\var\bsw 0 \var$) and $\var\lasrelncbv\var$ by definition of $\lasrelncbvsym$.
	
	%%%%%%%%%%%%
	\item \emph{Abstraction}:
	\[\infer[(\scabs) ]{\la\var\tmrone \lasrel \la\var\tmrtwo} {\tmrone \lasrel \tmrtwo} \text{ and } \la\var\tmrone \bsw 0 \la\var\tmrone \]
	
	hence the result ($\la\var\tmrtwo \bsw 0 \la\var\tmrtwo$) and $\la\var\tmrone \lasrelncbv \la\var\tmrtwo$ by definition of $\lasrelncbvsym$.
	%%%%%%%%%%%%
	\item \emph{Application}:
	\[ \infer[(sc.app) ] 
	{\tmrone\tmrthree  \lasrel  \tmrtwo\tmrfour} {\tmrone  \lasrel \tmrtwo & \tmrthree \lasrel \tmrfour } \text{ and }\tmrone\tmrthree \bsw k \ntm \]
	
	then, by case analysis on the last rule of the big-step derivation:
	
	\begin{itemize}
		\item \emph{Applied normal form:}
		
	\begin{center}
			$\infer{\tmrone\tmrthree\bsw {k+h} \ntm = \ntmONE\ntmTWO}{ \tmrone\bsw {k} \ntmONE & \tmrthree\bsw {h} \ntmTWO}$
	\end{center}
		
		
		by inductive hypothesis ($d$ strictly decreasing, first component not increasing) we obtain $\tmrtwo \bsws \ntmONEtwo$ and $\tmrfour \bsws \ntmTWOtwo$ with $\ntmONE\lasrelncbv\ntmONEtwo,~\ntmTWO \lasrelncbv \ntmTWOtwo$. Then we need two facts to conclude:
		\begin{itemize}
			\item $\ntmONEtwo\ntmTWOtwo$ is the normal form of $\tmrtwo\tmrfour$:
			Suppose it is not a normal form, \ie $\ntmONEtwo=\la\var\tmthree$ and $\ntmTWOtwo=\val$. By $\ntmONE\lasrelncbv\ntmONEtwo,~\ntmTWO \lasrelncbv \ntmTWOtwo$ and the definition of naive simulations, $\ntmONE$ must be an abstraction and $\ntmTWO$ must be a value, contradicting the fact that $\ntmONE\ntmTWO$ is a normal form. Hence:
			
			$\infer{\tmrtwo\tmrfour\bsw {k'+h'} \ntmtwo = \ntmONEtwo\ntmTWOtwo}{ \tmrtwo\bsw {k'} \ntmONEtwo & \tmrfour\bsw {h'} \ntmTWOtwo}$
			
			\item $\ntmONE\ntmTWO\lasrelncbv\ntmONEtwo\ntmTWOtwo$: which is clear from the first point and since, by \reflemma{lasrelncbv-normal-forms-lasrel-left-to-right}, $\ntmONE\lasrel\ntmONEtwo,~\ntmTWO \lasrel \ntmTWOtwo$. 
			
		\end{itemize}
		\item \emph{$\beta_v$ step}:
		\[\infer{\tmrone\tmrthree \bsw {k+h+i+1} \ntm}{
			\tmrone \bsw k {\la\var\tmronep}
			&
			\tmrthree \bsw h \val
			&
			{\tmronep\isub\var\val} \bsw i \ntm
		}\]
		
		
		
		then by inductive hypothesis ($d$ strictly decreasing, first component non increasing) on $\tmrone$ and $\tmrthree$ we get
		$\tmrtwo\bsws \la\var\tmrtwop$ with $\la\var\tmronep \lasrelncbv \la\var\tmrtwop$ and $\tmrfour\bsws \valtwo$ with $\val \lasrelncbv \valtwo$. In particular, by \reflemma{lasrelncbv-normal-forms-lasrel-left-to-right}, $\val\lasrel\valtwo$.
		
		
		
		Then:
		\[\infer{\tmronep\isub\var\val \lasrel \tmrtwop\isub\var\valtwo}{\tmronep\lasrel\tmrtwop&\val\lasrel\valtwo}\]
		
		since $\tmronep\isub\var\tmrthree \bsw i \ntm$ with $i < k+h+i+1$ we can apply the inductive hypothesis on the first component for $\tmronep\isub\var\val$ obtaining $\tmrtwop\isub\var\valtwo \bsws  \ntmtwo$ for some $\ntmtwo$ such that $\ntm\lasrelncbv\ntmtwo$. 
		
		Last, note that $\tmrtwo\tmrfour\bsws\ntmtwo$ by 
		\[\infer{\tmrtwo\tmrfour \bsws \ntmtwo}{
			\tmrtwo \bsws \la\var\tmrtwop
			&
			\tmrfour \bsws \valtwo
			&
			\tmrtwop\isub\var\valtwo \bsws \ntmtwo
		}\]
	\end{itemize}
	
%	
%	Let a rule of the shape (where $k\geq0$)
%	\begin{center}
%		$\infer{\tm\bswx h\ntm}{\tm_1\bswx{h_1}\ntm_1 &\ldots&\tm_k\bswx{h_k}\ntm_k   }$
%	\end{center} such that for all $1\leq i\leq k$, 
%	\begin{enumerate}
%		\item $h_i\leq h$
%		\item either $\tm_i$ is a premice of the ($\scapp$) rule or $h_i < h$
%	\end{enumerate}

	
	
	
	
	
	\ignore{
	\begin{enumerate}
		\item \emph{Applied inert}: 
		\[\infer{\tmrone\tmrthree \bsw {k+h} \itm\ntm}{
			\tmrone \bsw k \itm
			&
			\tmrthree \bsw h \ntm
		}\]
		
		by inductive hypothesis ($d$ strictly decreasing, first component not increasing) we obtain $\tmrtwo \bsws \itmtwo$ and $\tmrfour \bsws \ntmtwo$ with $\itm\lasrelncbv\itmtwo,~\ntm \lasrelncbv \ntmtwo$. In particular, by \reflemma{lasrelncbv-normal-forms-lasrel-left-to-right}, $\itm\lasrel\itmtwo,~\ntm \lasrel \ntmtwo$. Then:
		\[\infer{\tmrtwo\tmrfour \bsws \itmtwo\ntmtwo}{
			\tmrtwo \bsws  \itmtwo
			&
			\tmrfour \bsws \ntmtwo
		}\]
		and $\itm\ntm \lasrelncbv \itmtwo\ntmtwo$ by definition of $\lasrelncbv$ and $\itm\lasrel\itmtwo,~\ntm \lasrel \ntmtwo$.
		
%		\item \emph{Substitution of an inert}:
%		not applicable.
		
		
		\item \emph{$m$ step}:
		\[\infer{\tmrone\tmrthree \bsw {k+i+1} \isctxp\ntm}{
			\tmrone \bsw k \isctxp{\la\var\tmronep}
			&
			{\tmronep\esub\var\tmrthree} \bsw i \ntm
		}\]
		
		
		
		then by inductive hypothesis ($d$ strictly decreasing, first component non increasing) on $\tmrone$ we get
		$\tmrtwo\bsws \ntm_\tmrtwo$ with $\isctxp{\la\var\tmronep} \lasrelncbv \ntm_\tmrtwo$ ($\ntm_\tmrtwo = \isctxtwop{\la\var\tmrtwop}$ by  \reflemma{abstraction-inerts-stable-lasrelncbv}) \ie $\isctxp{\la\var\tmronep} \lasrelncbv \isctxtwop{\la\var\tmrtwop}$.
		
		
		By \reflemma{lasrelncbv-values-isctx-decomposition} we get $\isctxp{\var} \lasrelncbv \isctxtwop{\var}$ and $\la\var\tmronep\lasrelncbv \la\var\tmrtwop$ then $\tmronep\lasrel\tmrtwop$ by case (nai 3)).
		
		Then:
		\[\infer{\tmronep\esub\var\tmrthree \lasrel \tmrtwop\esub\var\tmrfour}{\tmronep\lasrel\tmrtwop&\tmrthree\lasrel\tmrfour}\]
		
		since $\tmronep\esub\var\tmrthree \bsw i \ntm$ with $i < k+i+1$ we can apply the inductive hypothesis on the first component for $\tmronep\esub\var\tmrthree$ obtaining $\tmrtwop\esub\var\tmrfour \bsws  \ntmtwo$ for some $\ntmtwo$ such that $\ntm\lasrelncbv\ntmtwo$. Since $\isctxp{\var} \lasrelncbv \isctxtwop{\var}$ and $\ntm\lasrelncbv\ntmtwo$, by \reflemma{lasrelncbv-normal-forms-isctx-decomposition}, we get $\isctxp{\ntm} \lasrelncbv \isctxtwop{\ntmtwo}$.
		Last, note that $\tmrtwo\tmrfour\bsws\isctxtwop\ntmtwo$ by 
		\[\infer{\tmrtwo\tmrfour \bsws \isctxtwop\ntmtwo}{
			\tmrtwo \bsws \isctxtwop{\la\var\tmrtwop}
			&
			\tmrtwop\esub\var\tmrfour \bsws \ntmtwo
		}\]
		
%		
%		\item \emph{$e$ step}:
%		not applicable.
	\end{enumerate}
	
	\item \emph{Explicit Substitution}: 
	\[ \infer[(sc.esubst) ]{\tmrone\esub\var{\tmrthree} \lasrel \tmrtwo\esub\var{\tmrfour}} {\tmrone \lasrel \tmrtwo & \tmrthree \lasrel \tmrfour }\text{ and }\tmrone\esub\var{\tmrthree} \bsw k \ntm \]
	
	\begin{enumerate}
%		\item \emph{Applied inert}: not applicable.
		\item \emph{Substitution of an inert}:
		\[ \infer{\tmrone\esub\var\tmrthree \bsw {k+h} \ntm\esub\var\itm}{
			\tmrone \bsw k \ntm
			&
			\tmrthree \bsw h \itm
		} \]
		
		by inductive hypothesis ($d$ strictly decreasing, first component not increasing) we obtain $\tmrtwo \bsws \ntmtwo$ and $\tmrfour \bsws \itmtwo$ with $\itm\lasrelncbv\itmtwo,~\ntm \lasrelncbv \ntmtwo$. In particular, by \reflemma{lasrelncbv-normal-forms-lasrel-left-to-right}, $\itm\lasrel\itmtwo,~\ntm \lasrel \ntmtwo$. Then:
		\[\infer{\tmrtwo\esub\var\tmrfour \bsws \ntmtwo\esub\var\itmtwo}{
			\tmrtwo \bsws  \ntmtwo
			&
			\tmrfour \bsws \itmtwo
		}\]
		and $\ntm\esub\var\itm \lasrelncbv \ntmtwo\esub\var\itmtwo$ by definition of $\lasrelncbvsym$.
		
%		\item \emph{m step}: not applicable.
		\item \emph{e step}: 		\[ \infer{\tmrone\esub\var{\tmrthree} \bsw {k+i+1} \isctxp\ntm}{
			\tmrthree \bsw k \isctxp{\la\vartwo\tmrthreep}
			&
			{\tmrone\isub\var{\la\vartwo\tmrthreep}} \bsw i \ntm
		} \]
		
		then by inductive hypothesis ($d$ strictly decreasing, first component non increasing) on $\tmrone$ and $\tmrthree$ we get
		$\tmrfour\bsws \ntm_\tmrfour$ with $\isctxp{\la\vartwo\tmrthreep} \lasrelncbv \ntm_\tmrfour$($\ntm_\tmrfour = \isctxtwop{\la\vartwo\tmrfourp}$ by  \reflemma{abstraction-inerts-stable-lasrelncbv}) \ie $\isctxp{\la\vartwo\tmrthreep} \lasrelncbv \isctxtwop{\la\vartwo\tmrfourp}$.
		
		By \reflemma{lasrelncbv-values-isctx-decomposition} we get $\isctxp{\var} \lasrelncbv \isctxtwop{\var}$ and $\la\vartwo\tmrthreep\lasrelncbv\la\vartwo\tmrfourp$ and in particular\\ $\la\vartwo\tmrthreep\lasrel\la\vartwo\tmrfourp$ by \ref{l:lasrelncbv-normal-forms-lasrel-left-to-right}.
		
		Then:
		\[\infer{\tmrone\isub\var{\la\vartwo\tmrthreep} \lasrel \tmrtwo\isub\var{\la\vartwo\tmrfourp}}{\tmrone\lasrel\tmrtwo&\la\vartwo\tmrthreep\lasrel\la\vartwo\tmrfourp}\]
		
		since $\tmrone\isub\var{\la\vartwo\tmrthreep} \bsw i \ntm$ with $i < k+i+1$ we can apply the inductive hypothesis on the first component for $\tmrone\isub\var{\la\vartwo\tmrthreep}$ obtaining $\tmrtwo\isub\var{\la\vartwo\tmrfourp} \bsws  \ntmtwo$ for some $\ntmtwo$ such that $\ntm\lasrelncbv\ntmtwo$. 
		Since $\isctxp{\var} \lasrelncbv \isctxtwop{\var}$ and $\ntm\lasrelncbv\ntmtwo$, by \reflemma{lasrelncbv-normal-forms-isctx-decomposition}, we get $\isctxp{\ntm} \lasrelncbv \isctxtwop{\ntmtwo}$.
		
		Last, note that $\tmrtwo\tmrfour\bsws\isctxtwop\ntmtwo$ by 
		\[\infer{\tmrtwo\tmrfour \bsws \isctxtwop\ntmtwo}{
			\tmrfour \bsws \isctxtwop{\la\vartwo\tmrfourp}
			&
			\tmrtwo\isub\var{\la\vartwo\tmrfourp} \bsws \ntmtwo
		}\]
		
	\end{enumerate}}
	
	
	%%%%%%%%%%%%%
	\item \emph{Meta-level Substitution}: 
	\[ \infer[(sc.subst) ]{\tmrone\isub\var{\val} \lasrel \tmrtwo\isub\var{\valtwo}} {\tmrone \lasrel \tmrtwo & \val \lasrel \valtwo }\text{ and }\tmrone\isub\var{\val} \bsw k \ntm \]
	then by applying Big-step substitutivity (\reflemma{splitting_weak}), we obtain $\tmrone \bsw {k_1} \ntm_\tmrone$ and $\ntm_\tmrone\isub\var{\val} \bsw {k_2} \ntm$ with $k=k_1+k_2$. Hence by inductive hypothesis ($d$ strictly decreasing, first component non increasing) $\tmrtwo\bsws\ntm_\tmrtwo$ and $\ntm_\tmrone \lasrelncbv \ntm_\tmrtwo$. In particular, by \reflemma{lasrelncbv-normal-forms-lasrel-left-to-right}, $\ntm_\tmrone \lasrel \ntm_\tmrtwo$.
	We then have: 
	\[\infer{\ntm_\tmrone\isub\var{\val} \lasrel \ntm_\tmrtwo\isub\var{\valtwo}} {\ntm_\tmrone \lasrel \ntm_\tmrtwo & \val \lasrel \valtwo }\]
	
	%	Note that by \ih applied to $\val \lasrel \valtwo $ and $\val\bsws \val$ (first component not increasing, second strictly decrasing) we obtain $\val \lasrelncbv \valtwo$.
	
	Two cases.
	\begin{enumerate}
		
		
		\item \emph{$\tmrone$ is not normal}, that is, $k_1>0$ and $k_2<k$. Then by applying the induction hypothesis to $k_2$ (first component)  and $\ntm_\tmrone\isub\var{\val}$ we obtain $\ntm_\tmrtwo\isub\var{\valtwo} \bsws \ntmtwo$ with $\ntm\lasrelncbv\ntmtwo$. We conclude using substitutivity of $\tow$ (\ref{l:stability_weak}) that $\tmrtwo\isub\var{\valtwo} \tow^* \ntm_\tmrtwo\isub\var{\valtwo} \tow^* \ntmtwo$, hence via the equivalence between big and small steps (\ref{l:ss-bs-equivalence_weak}), $\tmrtwo\isub\var{\valtwo} \bsws \ntmtwo$.
		
		
		\item \emph{$\tmrone$ is normal}, that is, $k_1=0$ and $k_2=k$. Then $\tmrone=\ntm_\tmrone$. Two sub-cases:
		\begin{itemize}
			\item \emph{$\tmrone\isub\var{\val} = \ntm_\tmrone\isub\var{\val}$ is also normal}
			
			
			Since we know that $\ntm_\tmrone \lasrelncbv \ntm_\tmrtwo$ and $\val \lasrel \valtwo $, we can apply Point 1 of \refprop{ncbv-coherence} and obtain that $\ntm_\tmrtwo\isub\var{\valtwo}$ is $\tow$-normal and  $\tmrone\isub\var{\val} \lasrelncbv \ntm_\tmrtwo\isub\var{\valtwo}$. It is only left to show that $\tmrtwo\isub\var{\valtwo} \bsws \ntm_\tmrtwo\isub\var{\valtwo}$, which follows from $\tmrtwo\bsws \ntm_\tmrtwo$, substitutivity of $\tow$ (\reflemma{stability_weak}) and the fact that $\ntm_\tmrtwo\isub\var{\valtwo}$ is $\tow$-normal (and via \reflemma{ss-bs-equivalence_weak}).
			
			\item   \emph{$\tmrone\isub\var{\val} = \ntm_\tmrone\isub\var{\val}$ is not normal}
			
			hence $\ntm_\tmrone\isub\var{\val} \tow \tmronep \tow ^ {k-1} \ntm$ (the reduction is diamond, all reductions are of the same length, we pick any first step possible). Then by Point 2 of \refprop{ncbv-coherence} with $\ntm_\tmrone \lasrelncbv \ntm_\tmrtwo$, $\ntm_\tmrtwo\isub\var{\valtwo} \tow \tmrtwop$ with $\tmronep \lasrel \tmrtwop$.
			
			We can apply the inductive hypothesis to $\tmronep$ (first component is decreasing, as $k-1<k$) and we obtain $\tmrtwop \bsws \ntmtwo$ with $\ntm\lasrelncbv\ntmtwo$.
			The statement is then proved, since (using \reflemma{stability_weak})
			$$\tmrtwo\isub\var{\valtwo} \tow^* \ntm_\tmrtwo\isub\var{\valtwo} \tow \tmrtwop \tow^* \ntmtwo$$ that is, $\tmrtwo\isub\var{\valtwo} \bsws \ntmtwo$ by \reflemma{ss-bs-equivalence_weak}.\qedhere
			
		\end{itemize}
	\end{enumerate}
	
\end{enumerate}

\end{proof}
