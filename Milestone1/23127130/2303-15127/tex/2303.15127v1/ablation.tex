\subsection{Partial Poisoning}\label{sec:results:ablation:partial}

In practical scenarios,
attackers may only have partial control
over the training data~\cite{huangunlearnable},
thus it is more practical
to consider the scenario
where only a part of the data is poisoned.
We adopt EM~\cite{huangunlearnable}
and LSP~\cite{yu2022availability}
on the CIFAR-10 dataset
as an example for our discussions.
Following the same setup,
we split varying percentages
from the clean data to carry out unlearnable poisoning
and mix it with the rest of the clean training data
for the target model training.
\Method{} is applied during model training
to explore its effectiveness
against partial poisoning.
\Cref{fig:sensitivity:em,fig:sensitivity:lsp},
show that when the poisoning ratio is low (\( <40\% \)),
the effect of the poisoning is negligible.
% From~\Cref{limit},
% we know \Method{}
% makes model convergence
% slower due to its multiple augmentation policies.
% In this case,
% we need to be more realistic in considering
% whether to use \Method{}.
Another type of partial dataset attack scenario
is the selection of a targeted class to poison.
We thus poison all training
samples of the \ordinal{9} label (``truck''),
and \Cref{fig:sensitivity:standard,fig:sensitivity:ueraser}
shows the prediction confusion matrices
of ResNet-18 trained on CIFAR-10.
In summary,
\Method{} demonstrates significant efficacy
in partial poisoning scenarios.
\begin{figure*}[htbp]
\centering
\begin{subfigure}{0.24\linewidth}
    \includegraphics[width=\linewidth]{EM-p.pdf}
    \caption{Partial poisoning of EM.}\label{fig:sensitivity:em}
\end{subfigure}
\hfill
\begin{subfigure}{0.24\linewidth}
    \includegraphics[width=\linewidth]{LSP-p.pdf}
    \caption{Partial poisoning of LSP.}\label{fig:sensitivity:lsp}
\end{subfigure}
\hfill
\begin{subfigure}{0.23\linewidth}
    \includegraphics[width=\linewidth]{unlearnconf.pdf}
    \caption{%
        Prediction matrix.
    }\label{fig:sensitivity:standard}
\end{subfigure}
\hfill
\begin{subfigure}{0.23\linewidth}
    \includegraphics[width=\linewidth]{ueraserconf.pdf}
    \caption{%
        With \Method{}.
    }\label{fig:sensitivity:ueraser}
\end{subfigure}
% \vspace{8pt}
\caption{%
    The defensive efficacy of
    \Method{} against partial poisoning.
    (\subref{fig:sensitivity:em})
    EM with different poisoning ratios;
    (\subref{fig:sensitivity:lsp})
    LSP with different poisoning ratios.
    (\subref{fig:sensitivity:standard}),
    (\subref{fig:sensitivity:ueraser})
    Prediction confusion matrices
    on the clean test set
    of ResNet-18 trained on CIFAR-10
    with an unlearnable class
    (\emph{the \ordinal{9} label `truck'}).
    (\subref{fig:sensitivity:standard}) Standard training;
    (\subref{fig:sensitivity:ueraser}) \Method{} training.
}\label{fig:sensitivity}
\end{figure*}

\subsection{Adaptive Poisoning}\label{sec:results:ablation:adaptive}

Since \Method{} is composed
of multiple data augmentations,
we should consider possible adaptive
unlearnable example attacks
which may leverage \Method{}
to craft poisons against it.
We therefore evaluate \Method{}
in a worst-case scenario
where the adversary is fully aware
of our defense mechanism,
in order to reliably assess the resilience of \Method{}
against potential adaptive attacks.
Specifically,
we design an adaptive unlearning poisoning attack
by introducing an additional data augmentation
during the training,
We adopt the error-minimization
(EM) attack~\cite{huangunlearnable}
as an example.
The EM unlearning objective
solves the following min-min optimization:
\begin{equation}
    \arg\min_{\bdelta} \min_{\btheta}
    \expect_{(\bx, y) \sim \cleanset}\bracks*{
        \loss\parens*{
            f_\btheta \parens{\bx+\bdelta_{\bx}}, y
        }
    },
    \label{eq:attack:em}
\end{equation}
where \( \|\bdelta\|_{p} \leq \epsilon \).
Similar to the REM~\cite{furobust}
that generates adaptive unlearnable examples
under adversarial training,
each image \( \bx \)
optimizes its unlearning perturbation \( \bdelta_\bx \)
before performing adversarial augmentations:
\begin{equation}
    \arg\min_{\bdelta} \min_{\btheta}
    \expect_{(\bx, y) \sim \cleanset} \bracks*{
         \loss\parens*{
            f_\btheta \parens*{
                \transform_{\textrm{adv}}\parens{\bx + \bdelta_\bx}
            }, y
        }
    },
    \label{eq:attack:adaptive}
\end{equation}
where \( \transform_{\textrm{adv}}\parens{\cdot} \)
denotes the adversarial augmentations with \Method{}.
We select different combinations of augmentations
for our experiments,
and the results are shown in~\Cref{tab:adaptive}.
The hyperparameters of the augmentations
employed in all experiments
are kept consistent
with those of \Method{}.
We observe
that the adaptive augmentation of unlearning poisons
do not significantly reduce the effectiveness
of \Method{}.
As it encompasses a diverse set of augmentation policies
and augmentation intensities,
along with loss-maximizing augmentation sampling,
adaptive poisons are hardly effective.
Moreover,
we speculate that the affine and cropping transformations
in TrivialAugment can cause unlearning perturbations
to be confined to a portion of the images,
which also limits the effectiveness
of unlearning poisons.
Because of the aggressiveness of the augmentation
in image transformations
extend beyond the \( \ell_p \) bounds,
adaptive poisons do not perform as well under \Method{}
as they do against REM\@.
To summarize,
it is challenging for the attacker
to achieve successful poisoning
against \Method{}
even if it observes the possible transformations
taken by the augmentations.
\begin{table}[!h]
\centering
\caption{%
    Adaptive poisoning with EM
    on CIFAR-10.
    `P', `C,' and `T' denote
    PlasmaTransform, ChannelShuffle,
    and TrivialAugment respectively.
    ``Standard'' represents standard training.
}\label{tab:adaptive}
% \vspace{5pt}
% \small
\begin{tabular}{l||ccc}
\toprule
    Methods & Standard & \MethodLite{} & \Method{} \\
\midrule
    Baseline & 21.21 & 90.78 & 93.38 \\
    \: + P & 24.36 & 86.05 &92.55   \\
    \: + C & 19.71 & 83.46 & 91.72   \\
    \: + P + C & 25.48 & 82.49 & 89.07 \\
    \: + P + C + T & 44.26 & 85.22 & 90.79 \\
\bottomrule
\end{tabular}
\end{table}


\subsection{%
    Larger Perturbation Scales
}\label{sec:results:ablation:larger_perturbation}

Will the performance of \Method{}
affected by large unlearnable perturbations?
To verify,
we evaluate the performance of \Method{}
on unlearnable CIFAR-10 dataset
with even larger perturbations.
We use the example
of error-maximizing (EM) attack
and increase the \( \ell_{\infty}\) perturbation
from \(8/255\) to \(24/255\)
to examine the efficacy of \Method{}
on a more severe unlearning perturbation scenario.
We also include adversarial training (AT)
as a defense baseline
with a perturbation bound of \( 8/255 \).
The experimental results in~\Cref{tab:scale}
confirm the effectiveness of \Method{}
under large unlearning noises.
\begin{figure}[t]
    \centering
    \includegraphics[width=.7\linewidth]{pics/scale.pdf}
    \caption[]{\fix{Distributions of MIG scores and reconstruction errors for low-dimensional space (blue) and high-dimensional space (green). The points in the bottom right have a better balance of disentanglement and reconstruction.}}
    \label{fig:scale}
\end{figure}

\subsection{%
    Resilience against Architecture Choices
}\label{sec:results:ablation:architectures}

Can \Method{} show resilience
against architecture choices?
In a realistic scenario,
we need to train the data
with different network architectures.
We thus explore whether \Method{}
can wipe out the unlearning effect
under different architectures.
\Cref{tab:arch} shows the corresponding results.
It is clear that \Method{}
is capable of effectively wiping out unlearning poisons,
across various network architectures.
\section{Network Architecture}
\label{sec:arch}

Throughout this section we work with general qudits of local dimension $d$. We first describe the general structure of the class of NoRA networks we consider, then we specialize to a particular network structure inspired by scaling and renormalization group (RG) considerations. We analyze both the entanglement and complexity of the scaling-adapted ground state network and discuss an extension to describe excited states. In particular, we show that a natural choice of energy scales in a toy model Hamiltonian can give rise to a power-law temperature dependence of the thermodynamic entropy and heat capacity.

\subsection{General Structure}

The NoRA network is defined by $L$ layers as in Fig.~\ref{fig:arch}, where we refer to the bottom qudits as \emph{ground state} qudits and the other qudits as \emph{excited state or thermal} qudits. When we set the thermal qudits to some fixed product state, $\ket{0}$, we obtain the \emph{ground state network} as in Fig.~\ref{fig:arch}. This nomenclature is chosen because we can view the network as a variational ansatz for the ground space of a mean-field model. From this point of view, the ground state qudits parameterize a space of states that would be identified with the degenerate ground space of the concrete model of interest.

\begin{figure}[htb]
    \centering
    \scalebox{0.8}{\tikzfig{figures/syk_circuit}}
    \caption{Basic architecture of the proposed NoRA tensor network ansatz. A code word $\ket{\psi_{\textrm{code}}}$ consisting of $n_0 \equiv k$ (logical) \enquote{ground-state} qudits is embedded  as $\ket{\Psi_{\textrm{phys}}}$ in the (physical) ground space of the $d^N$-dimensional many-body Hilbert space by the means of $L$ layers of some given depth $D$ quantum circuits. For each layer $1 \leq \ell \leq L$ the circuit $D_{\ell}$ acts on the $n_{\ell - 1}$ qudit output from the previous layer and an additional $\Delta n_{\ell}$ new ancillary \enquote{thermal} qudits initialized in state $\ket{0}$. We stress that the layer circuits $D_\ell$ do not have to respect locality structure depicted by the 1d arrangement of qudit lines.}
    \label{fig:arch}
\end{figure}

One way to think about the network is as a ``fine-graining'' circuit moving upwards from the bottom ground state qudits. This is the inverse of a conventional RG transformation since we are adding degrees of freedom. We start with $k$ of these ground state qudits. Then at each layer $\ell$ we add $\Delta n_{\ell}$ thermal qudits in the fixed state $\ket{0}^{\otimes \Delta n_{\ell}}$ and apply a depth $D$ quantum circuit to all the qudits in that layer. This circuit could also be generalized to be time evolution with a suitably normalized all-to-all Hamiltonian for a constant time (proportional to $D$). The next layer takes all the qudits from the previous layer and adds more thermal qudits to generate the hierarchical structure in Fig.~\ref{fig:arch}. The total number of qudits at layer $\ell$ is denoted $n_{\ell}$ and given by
\begin{equation}
    n_{\ell} = k + \sum_{\ell'=1}^{\ell} \Delta n_{\ell'}.
\end{equation}
The total number of qudits is therefore
\begin{equation}
    N \equiv n_L = k + \sum_{\ell=1}^L \Delta n_{\ell}.
\end{equation}


\subsection{Scaling Specialization}

As is, we have described a fairly general architecture. Motivated by scaling and renormalization group considerations, we will primarily consider the special case where $n_{\ell} \sim k + r^{\ell}$, so that the number of thermal qudits is increasing exponentially with each layer up from the bottom. Viewing the top layer as the UV or microscopic degrees of freedom and the bottom layer as the IR or emergent degrees of freedom, moving from the UV to IR (top to bottom) mimics a renormalization group transformation where we remove some fraction of the thermal degrees of freedom at each step. Indeed, borrowing the language of MERA and DMERA and viewing the circuit from top to bottom, the individual layers are like disentanglers that leave behind some decoupled degrees of freedom, the thermal qudits added at that layer. In this scheme, we choose the number of qudits at layer $\ell$ to be
\begin{equation}
    k + r^{\ell} \stackrel{!}{=} n_{\ell} = k + \sum_{\ell'=1}^{\ell} \Delta n_{\ell},
\end{equation}
implying that the number of new thermal qudits for each layer must be
\begin{align}
\begin{split}
    \Delta n_{\ell > 1} &= r^{\ell} - r^{\ell - 1}, \\
    \Delta n_{1} &= r.
\end{split}
\end{align}
For the case of $r=2$, which we primarily consider in this work, this simplifies to approximately $\Delta n_{\ell} = 2^{\ell - 1}$ for all layers $\ell$.

\subsection{Entanglement and Complexity}

We next discuss the entanglement and complexity of the RG-inspired network. There are $\text{O}(N)$ non-trivial bonds in the circuit, of which $N$ bonds connect to the same constant-depth circuit in the last layer. It is therefore straightforward to establish that the network has the potential to encode volume law entanglement for sub-regions of a \enquote{typical} UV state. We also explicitly demonstrate that this is achievable within the Clifford model discussed below in Sections~\ref{sec:clifford}, \ref{sec:numerical}, and \ref{sec:code_structure}.

Turning to the complexity, we take the number of gates in the network as an estimate of the circuit complexity of the UV state, although in general this is only an upper bound. For a layer $\ell$ with $n_{\ell}$ total qubits in it, we apply $D$ rounds of $\lfloor n_{\ell}/q \rfloor$ $q$-qudit gates, so the number of gates of layer $\ell$ is
\begin{equation}
   \text{gates at layer } \ell = D \cdot \lfloor n_{\ell}/q \rfloor.
\end{equation}
Summing this result over all layers and assuming that $q$ divides $n_{\ell}$ without remainder gives a total number of gates equal to
\begin{equation}
\label{eq:circuit_complexity}
    \text{total gates} = \frac{D}{q} \sum_{\ell=1}^L n_{\ell} = \frac{D}{q} \left(L \cdot k + \frac{r^{L+1} - r}{r-1} \right).
\end{equation}
In sections \ref{sec:numerical} and \ref{sec:code_structure} we will cast this result into simpler leading-order expressions that correspond to the respective types of ground space scaling being considered.

\subsection{Extension to Excited States}

Let us conclude this section by extending the ground state network we have so far discussed to the case of excited states. As we have repeatedly emphasized, the discussion so far is general and does not consider a particular physical Hamiltonian. We are simply trying to match certain qualitative features of the entanglement and complexity expected for mean-field models. A structure similar to what we will consider here was recently studied for non-interacting fermions and advocated for as a general approach to approximating thermal states~\cite{sewell_thermal_multi_scale_2022}.

The idea is to introduce a toy Hamiltonian for which the above network is an exact ground state for any choice of state on the $k$ ground state qudits. In other words, the toy Hamiltonian has an exactly degenerate ground space. The Hamiltonian is constructed in a standard way by introducing projectors $P = |0\rangle \langle 0|$ for each thermal qudit and defining corresponding projectors acting on the UV qudits by conjugating these elementary projectors with the network circuit. Let $\tilde{P}_i$ denote the projector for thermal qudit $i$ conjugated by the network circuit. The toy Hamiltonian is
\begin{equation}
\label{eq:stab_hamiltonian}
    H = - \sum_i J_i \tilde{P}_i,
\end{equation}
where $J_i$ are a set of free parameters that determine the energy scale associated with each thermal qudit. Note that -- just like the circuit it encodes -- this Hamiltonian is highly non-local and not necessarily few-body, thus limiting the potential for physical interpretation. The setup is described in more detail in appendix \ref{sec:entropy_scaling}. 

Again motivated by RG considerations, in which the energy scale of excitations decreases by a fixed factor after every RG step (top to bottom), we take the $J_i$ to be equal within a layer and to depend on the layer index $\ell$ as
\begin{equation}
\label{eq:energy_scaling}
    J_{\ell} = \Lambda \cdot e^{-\gamma (L-\ell)}.
\end{equation}
In this way, the UV energy scale is $\Lambda$ and the energy of excitations decreases exponentially with the layer index decreasing towards the IR. The free parameter $\gamma$ controls the rate of decrease.

As computed in appendix \ref{sec:entropy_scaling}, the entropy for the Gibbs ensemble associated to said toy Hamiltonian describing our tensor network ansatz (and for general scaling of $J_{\ell}$) is
\begin{equation}
\label{eq:stab_entropy}
    S = \log\left(d^{k} \cdot (d-1)^{\braket{N-k}}\right) + \sum_{\ell} \Delta n_{\ell} \cdot S(p_{\ell}),
\end{equation}
where we defined a probability, 
\begin{equation}
    p_{\ell} = \frac{d - 1}{e^{\beta J_{\ell}} + d - 1},
\end{equation}
$S(p_{\ell})$ is the classical binary entropy function,
\begin{equation}
    S(p_{\ell}) = - p_{\ell} \cdot \log(p_{\ell}) - (1-p_{\ell}) \cdot \log(1-p_{\ell}),
\end{equation}
and 
\begin{equation}
    \braket{N-k} = \sum_i p_i = \sum_{\ell} \Delta n_{\ell} \, p_{\ell}.
\end{equation}
Note that in the case of qubits ($d=2$), $p_{\ell}$ coincides with the ordinary Fermi-Dirac distribution, in which case $\braket{N-k}$ is analogous to a sum of occupation numbers.

Plugging in \eqref{eq:energy_scaling} and going to the low-temperature regime (relative to the energy scale $\Lambda$), \eqref{eq:stab_entropy} can be approximated in the continuum limit as
\begin{align}
\label{eq:stab_entropy_approx}
\begin{split}
    & S - k \cdot \log(d) \\ \lessapprox{}& (d-1) (N-k) \cdot \frac{\alpha}{\gamma} \cdot \Gamma\left(\frac{\alpha}{\gamma} + 1\right) (\beta \Lambda)^{-\alpha/\gamma} \\
    \propto{}& (T/\Lambda)^{\alpha/\gamma},
\end{split}
\end{align}
with $N = k + r^L$ and $\alpha = \log(r)$. This together with the specific example depicted in figure \ref{fig:entropy_scaling} confirms that in this limit the entropy does obey a power law. By choosing the parameters $\alpha$ and $\gamma$ suitable, one could even match the precise low-temperature behavior of the SYK heat capacity $C_V$ (which is proportional to $T$) due to $dS = \frac{C_V}{T} dT$:
\begin{equation}
    C_V = T \left( \frac{dS}{dT} \right) \propto (T/\Lambda)^{\alpha/\gamma}.
\end{equation}

\begin{figure}[htb]
    \centering
    \includegraphics{figures/entropy_scaling.pdf}
    \caption{Logarithmic scaling of the exact Gibbs entropy $S_{\textrm{stab}}$ associated to $H$, and the low-temperature approximation $S_{\textrm{approx}}$ for $L=20$, $r=2$, $k=1$, $d=2$, $\Lambda=1$ and $\gamma = 0.4$. Both match almost exactly for our choice of parameters and small $T/\Lambda$, confirming the existence of a scaling law. The same is also true for other choices of $\gamma$ (the only significant free parameter), as seen in figure \ref{fig:entropy_scaling_appendix}.}
    \label{fig:entropy_scaling}
\end{figure}

\subsection{Summary}

Starting from the general architecture in Figure~\ref{fig:arch}, we introduced the RG-inspired network in which the number of qudits at layer $\ell$ is $k + r^\ell$. In the special case where $k=0$, i.e. a non-degenerate ground space, the number of qudits decreases by a factor from one layer to the next into the IR. This decrease is analogous to a block decimation RG procedure applied to a quantum state. The case of $k\neq 0$ describes a generalization of such an RG procedure. The entanglement entropy of the physical states produced by the RG-inspired network can be volume-law, as expected for mean-field models. We also showed that the ground state network can be extended to provide a model of thermal excitations in which the thermodynamic heat capacity has a power-law temperature dependence at low temperature. These general features are all chosen to match characteristics of the SYK model, which also features a nearly degenerate space of highly entangled ground states and a power-law heat capacity at low temperature.

\subsection{Augmentation Options}\label{sec:results:ablation:options}

In this section,
we investigate the impact
of the \Method{} policy composition
on the mitigation of unlearning effects.
The visualization of the three policies
included is shown in~\Cref{fig:augvis}.
We conduct experiments
with the unlearnable examples from CIFAR-10
generated by the EM~\cite{huangunlearnable} method,
and~\Cref{tab:policy}
explore effectiveness of the various combinations
of augmentation policies.
ISS~\cite{liu2023image}
discovered that for the unlearnable examples
generated by the EM attack,
a grayscale filter easily
removes the unlearning poisons.
Additionally,
setting the value of each channel
to the average of all color channels
or to the value of any color channel
also considerately achieves the same effect.
However,
we show that using only ChannelShuffle
does not yield satisfactory results.
We have also discovered
an interesting phenomenon:
PlasmaTransform and ChannelShuffle
are essential for mostly restoring the accuracies,
whereas TrivialAugment,
can be substituted
with a similar policy,
\ie{}~AutoAugment~\cite{cubuk2019autoaugment}.
Only when the three policies
are employed together
can the effect of unlearning poisons
be effectively wiped out.
This also proves that the \Method{} policies
are effective and reasonable.
Recall that we also found the adoption
of error-maximizing augmentation
results in an overall improvement
on all five unlearning poisons.
Hence,
the utilization of error-maximizing
augmentation during training
serves as an effective means
to mitigate the challenges of training with unlearning examples
and improve the model's clean accuracy.
% We believe that \Method{}
% can effectively advance future research on unlearnable examples.
\begin{table}[t]
\centering
\caption{%
    Ablation analysis \Method{}
    on CIFAR-10.
    Note that all hyperparameters
    are the same,
    except for the \Method{}
    augmentation policy
    which varies.
    `P', `C', `T', and `A'
    denote ``PlasmaBrightness'', ``ChannelShuffle'',
    ``TrivialAugment'',
    and ``AutoAugment''~\cite{cubuk2019autoaugment}
    respectively.
    `Adv' denotes the full \Method{} method.
}
% \small
% \vspace{5pt}
\begin{tabular}{lcc}
\toprule
Ablation of \Method{} Policies
    & Clean
    & Unlearnable (EM) \\
\midrule
Standard Training
    & 94.78 & 21.24 \\
\midrule
+ P
    & 94.47 & 29.48 \\
+ T
    & 94.47 & 48.81 \\
+ P + C
    & 94.15 & 62.17 \\
+ P + T
    & 95.22 & 48.05 \\
+ C + T
    & 94.40 & 69.24 \\
% + C + T + (\( \ast\))
%     & 93.27 & 77.46 \\
+ C + P + A
    & 94.04 & 85.60 \\
+ C + P + T
    & 93.94 & 90.78 \\
+ C + P + T + Adv
    & 93.66 & 93.38 \\

\bottomrule
\end{tabular}
\label{tab:policy}
\end{table}


\subsection{%
    Adversarial Augmentations and Error-Maximizing Epochs
}\label{sec:results:ablation:repeat}

From~\Cref{alg:method},
the training of \Method{}
is affected by two hyperparameters, namely,
the numbers of repeated augmentation samples \( K \)
per image
and the epochs of error-maximizing augmentations \( W \).
For CIFAR-10,
The clean accuracy of the unlearnable examples
can be improved to around \( 80\% \)
after 50 epochs of training
using error-maximizing augmentation.
We explored \MethodMax{}
which applies error-maximizing augmentations
throughout the entire training phase,
and it attains best known accuracies.
The results are shown in~\Cref{tab:ablation:emaug}.
Alternatively,
one can continue the training with \MethodLite{}
can improve the clean accuracy to \( \geq 93\% \)
to save computational costs.
Although \MethodMax{}
achieves the highest clean accuracy,
we mainly focus on \Method{} in this paper,
due to its high computational cost per epoch
and longer training epochs.
\begin{table}[t]
\centering
% \vspace{-5pt}
\caption{%
    Clean test accuracies (\%)
    of different \Method{} variants.
    Note that `200' and `300' denotes
    training for \( E = W = 200 \) and \( 300 \) epochs respectively.
}\label{tab:ablation:emaug}
% \small
% \vspace{5pt}
% \resizebox{1.0\linewidth}{!}{
\begin{tabular}{c|c|cccc}
\toprule
    \multirow{2}{*}{Methods}
    & \multirow{2}{*}{Clean}
    & \multirow{2}{*}{\MethodLite{}}
    & \multirow{2}{*}{\Method{}}
    & \multicolumn{2}{c}{\MethodMax{}}
    \\
    & & & &200 &300\\
    \midrule

    {EM}
    & \multirow{2}{*}{94.78}
    & 90.78 & 93.38 & 92.12 & \textbf{95.24} \\
    {LSP}
    &
    & 84.92 & 85.07 & 91.79 & \textbf{94.95} \\
\bottomrule
\end{tabular}%}
\end{table}


Regarding the number of samples \( K \)
(by default \( K = 5 \)),
increasing it
further enhances the suppression
of unlearning shortcuts
during model training,
but also more likely to lead to gradient explosions
at the beginning of model training.
Therefore,
it may be necessary
to apply learning rate warmup
or gradient clipping
with increased number of repeated sampling.
Larger \( K \) can also results
in higher computational costs,
as it result in more samples per image
for training.
We provide a sensitivity analysis
of the number of repeated sampling \( K \)
in~\Cref{app:sensitivity}.

\subsection{Transfer Learning}\label{sec:results:ablation:transfer}

In this section,
we aim to explore the impact
of transfer learning~\cite{dai2009eigentransfer,torrey2010transfer}
on the efficacy of unlearnable examples.
We hypothesize that pretrained models
may learn certain in-distribution features
of the unaltered target distribution,
it may be able to gain accuracy
even if the training set contains unlearning poisons.

To this end,
we adopt a simple transfer learning setup,
where we use the pretrained ResNet-18
model available in the torchvision repository~\cite{torchvision}.
To fit the expected input shape
of the feature extractor,
we upsampled the input images
to \( 224 \times 224 \).
The final fully-connected classification layer
of the pretrained model
was replaced with a randomly initialized one
with 10 logit outputs.
We then fine-tune the model with
unlearnable CIFAR-10 training data.
We also further explored fine-tuning
on unlearnable data with our defenses.
For control references,
we fine-tuned a model
with clean training data,
and also trained a randomly initialized model from scratch
with poisoned training data.
\begin{figure}[ht]
    \centering
    \begin{subfigure}{0.24\linewidth}
        \includegraphics[width=\linewidth]{transfer_clean.pdf}%
        \caption{%
            Fine-tuning on a clean training set.
        }\label{fig:transfer:clean}
    \end{subfigure}
    \hfill
    \begin{subfigure}{0.24\linewidth}
        \includegraphics[width=\linewidth]{transfer_unlearn.pdf}%
        \caption{%
            Fine-tuning
            on an unlearnable training set.
        }\label{fig:transfer:unlearn}
    \end{subfigure}
    \hfill
    \begin{subfigure}{0.24\linewidth}
        \includegraphics[width=0.93\linewidth]{transfer_test.pdf}%
        \caption{%
            Different input dimensions
            on an unlearnable training set.
        }\label{fig:transfer:test}
    \end{subfigure}
    \hfill
    \begin{subfigure}{0.24\linewidth}
        \includegraphics[width=\linewidth]{transfer_uefast.pdf}%
        \caption{%
            Train and test accuracies of transfer learning
            with \Method{}.
        }\label{fig:transfer:ueem}
    \end{subfigure}
    % \subfigure[
    %     Fine-tuning with \MethodLite{}.
    % ]{%
    %     \includegraphics[scale=0.5]{transfer_uefast.pdf}%
    %     \label{fig:transfer:uefast}
    % }
    % \quad
    \caption{%
        Accuracies \wrt{} the number of training / fine-tuning epochs
        for randomly initialized / pretrained ResNet-18
        on different CIFAR-10 datasets.
        (\subref{fig:transfer:clean})
        Fine-tuning the pretrained model
        on a clean training set.
        (\subref{fig:transfer:unlearn})
        Fine-tuning the pretrained model
        with unlearnable training set
        generated with EM~\cite{huangunlearnable}.
        (\subref{fig:transfer:test})
        Comparing the test accuracies
        by training from scratch
        with either \(32 \times 32\)
        or upsampled \(224 \times 224\)
        unlearnable examples.
        (\subref{fig:transfer:ueem})
        Fine-tuning on unlearnable data
        with \Method{}.
    }\label{fig:transfer}
\end{figure}

The results of the experiments
are shown in~\Cref{fig:transfer}.
\Cref{fig:transfer:unlearn}
shows that fine-tuning
with unlearnable examples
can improve the clean test accuracy
from \(22\%\) to \(66\%\).
Additionally in~\Cref{fig:transfer:test},
we find that simply upsampling
the unlearnable samples
to use more compute and have larger feature maps
does not significantly weaken
the unlearning attack
(test accuracy increased to \(34\%\)).
Most importantly,
\Method{} successfully
eliminates the negative impact
of unlearning poisons,
which enables the model
to utilize pretrained knowledge effectively.
This enables the fine-tuned model
to achieve a test accuracy
of approximately \( 95\% \)
as shown in~\Cref{fig:transfer:ueem}.
