\subsection{Injection Parameters}

We study injected signals generated with the \textsc{IMRPhenomD\_NRTidalv2} approximant~\cite{Dietrich:2019kaq}.
The injected component masses $M_{1,2}$ are drawn from a Gaussian distribution $\mathcal{N}(\mu=1.33, \sigma=0.09)$ that is characteristic for galactic BNS systems~\cite{Ozel:2016oaf}.
They are uniformly distributed in a comoving volume with a distance cutoff at \SI{200}{Mpc}.
In a larger random sample of 1000 systems within \SI{500}{Mpc}, the average chirp mass of observable binaries settles at this distance near the injected distribution's mean, indicating that the volume is sufficiently large to characterize the underlying distribution.
We further limit our analysis to systems with a signal-to-noise ratio (SNR) above 30. This value is sufficiently high to expect measurements of significant tidal contributions without introducing a bias towards higher masses, where tidal effects would again become less prominent.
For a given mass $M$, the respective EOS fully determines radius $R$ and tidal deformability $\Lambda:=\frac{2k_2 R^5}{3M^5}$, where 
$k_2$ denotes the tidal Love number~\cite{Hinderer:2007mb}. 
Since only \textLambda\ is prominent in the waveform, we base our EOS selection on its distribution at a relatively low \SI{1.2}{M_\odot}.
This mass yet is firmly supported by observations and neutron star formation theories~\cite{Martinez:2015mya,Suwa:2018uni}.
\Cref{fig:m-r-lam-prior} shows how the parameter spaces of the EOS sets largely overlap at high masses, corresponding to core densities far beyond the breakdown of the chiral EFT approach. 
We then choose to inject an EOS from each distribution's 50th percentile which has a TOV mass closest to a fiducial value of \SI{2.2}{M_\odot}.
The dimensionless aligned spins $\chi_{1,2}$ are constrained to a uniform distribution subject to $|\chi_i|<0.05$, as implied for realistic sources of NS mergers~\cite{Stovall:2018ouw,Burgay:2003jj}.
We ultimately leave the sky location, inclination angle $\theta_{JN}$, orbital phase at coalescence $\phi$ and polarisation angle $\psi$ totally unconstrained.
Subject to the population model under consideration, the detection of 20 such systems will amount to at least two years of observation at ET~\cite{LIGOScientific:2021psn,Mandel:2021smh}.

\subsection{Bayesian Inference}
We study the resulting effects in the framework of Bayesian inference, using the EOS as a sampling parameter that is constrained by tidal terms in the observed waveform.
This makes use of Bayes' theorem \begin{align}
    \label{eq:Bayes}
    p (\theta|d)= \frac{L(d|\theta) \pi(\theta)}{\mathcal{Z}(d)}, \text{ with }\\
    \mathcal{Z}(d)=\int_\Theta L(d|\theta) \pi(\theta) \, {\rm d}\theta,
\end{align}
to determine a posterior distribution $p (\theta|d)$ of the multi-dimensional parameter space $\Theta$ that characterizes an event's GW strain.
We reweight a parameter set's prior probability $ \pi(\theta)$ by the likelihood $L$ that it is the cause of the observed data $d$.

\begin{table}[thb]
    \centering
    \begin{tabular}{clrlrl}
    &   parameter  &   \multicolumn{2}{c}{symbol} & \multicolumn{2}{c}{prior bounds} \\
      \hline
       \multirow{6}{8.5pt}{\rotatebox{90}{observational}}& luminosity distance [Mpc] &&$d_{L}$ & 5 &-- 500 \\
       & inclination &$\cos$& $\theta_{JN}$& -1 &-- 1\\
      & phase [rad] &&$\phi$ & 0 &-- $2\pi$\\
       & polarisation [rad] &&$\psi$ & 0 &-- $\pi$ \\
        & right ascension [rad] && \textalpha & 0 & -- $2\pi$\\
        & declination [rad] && \textdelta & $-\pi$ & -- $\pi$\\
      \hline
      \multirow{5}{8.5pt}{\rotatebox{90}{orbital}} & chirp mass $[\unit{M_\odot}] $ &&$\mathcal{M}$ &  \textit{1.20} &-- \textit{1.30}* \\
      & source chirp mass $[\unit{M_\odot}] $ &&$\mathcal{M}_s$ &  1.15&-- 1.30* \\
       & mass ratio && $q$ & 0.125 &-- 1\\
       &source comp. mass $[\unit{M_\odot}]$ && $M_{i,s}$ & \textit{ \textgreater 0.5}\\
        & aligned component spin & &$\chi_i$ & -0.15&-- 0.15 \\
        \hline
        hyper & Equation of State &&EOS & 1 &-- 3000
    \end{tabular}
    \caption[GW Sampling Parameters in the ET-Analysis]
    {\textbf{GW Sampling Parameters in the ET-Analysis}:
    Most priors are uniform within given bounds. 
    The declination \textdelta\ is uniform in cosine and the luminosity distance $d_{\rm L}$ is uniform within a co-moving volume of the specified dimension. 
    The EOS prior is weighted by the ability to support massive pulsars. 
    Prior ranges in italics indicate constraints that are not used as sampling parameters. 
    Starred priors are adjusted to the injected signal.}
    \label{tab:ET_priors}
\end{table}
The sampling parameters of our injection study are given in Tab. \ref{tab:ET_priors}. 
These include observational parameters (e.g., luminosity distance $d_L$, phase and inclination angles of the merger) and intrinsic binary parameters (e.g., chirp mass~$\mathcal{M}$, mass ratio, tilts).
We considerably extend the range above the injection distribution for the luminosity distance and aligned spins in order to avoid boundary effects from the prior distribution.
We note, though, that this comes at the cost of a bias towards unequal mass ratios in the parameter estimation.
We weight the EOSs conservatively by their ability to support the most firmly established pulsar masses~\cite{Antoniadis:2013pzd,Fonseca:2021wxt,Arzoumanian:2017puf}.
In order to reduce the significant computational cost, we apply a Reduced-Order-Quadrature (ROQ) rule~\cite{Smith:2016qas}. 
This requires limiting the chirp mass space to \SI{0.1}{M_\odot} intervals.
As we sample over chirp masses in the source frame, we invoke a corresponding prior adapted to the injected signal in order to avoid computational issues. 
Since the chirp mass is by far the most accurately measured quantity, the prior is still wide enough to avoid the introduction of prior-driven artefacts in the parameter estimation.

We use \textsc{parallel-bilby}~\cite{Smith:2020}, an efficient parallelisation package relying on nested sampling routines from \textsc{bilby}~\cite{Skilling:2006gxv,Ashton:2018jfp}, to obtain the evidence for either EOS set.
Nested sampling algorithms aim at calculating $\mathcal{Z}$ and yield the posterior \textit{en passant}.
Employing some minor modifications to \textsc{parallel-bilby} and \textsc{bilby}, we can sample over EOSs from the respective set.
Radii and tidal deformabilities then follow uniquely from the component masses.
The likelihood evaluations follow the usual matched filter approach employing the \textsc{IMRPhenomD\_NRTidalv2} approximant to efficiently generate waveforms including tidal effects.
To further reduce computational costs, we perform inference in the frequency range \qtyrange{30}{2048}{Hz}.
We use 2048 live points for nested sampling.
The selection of the \textsc{IMRPhenomD\_NRTidalv2} approximant is driven by efficiency and robustness considerations~\cite{Dietrich:2019kaq}.
In this setting, each inference run requires about 80,000 hours of computing time.

We can use the evidence $\mathcal{Z}$ for model selection because a higher evidence can only be achieved by the more complex one (i.e. with a less compact prior $\pi$) among two competing models if it matches the data significantly better.
Treating both models of 3N interaction as a priori equally likely, preference is expressed by the Bayes factor $\bf= \mathcal{Z}_{\ve} /\mathcal{Z}_{\rm TPE}$.
The evidence for either model explaining a suite of $N$ independent observations is given by
\begin{align}
    \mathcal{Z}&= \int \prod_{i=1}^N {L}_i(\theta_i, {\rm EOS}_i) \pi(\theta_i, {\rm EOS}_i) \,\dift\theta_i \dift{\rm EOS}_i  \\
    &= \prod_{i=1}^N \int  {L}_i(\theta_i, {\rm EOS}_i) \pi(\theta_i, {\rm EOS}_i) \,\dift\theta_i \dift{\rm EOS}_i  \\
    &= \prod_{i=1}^N \mathcal{Z}_i.\label{eq:usual_bayes}
\end{align}
We have subsumed all system parameters besides the EOS in $\theta_i$.
The corresponding Bayes factor \begin{align}
    \bf= \prod_{i=1}^N \frac{\mathcal{Z}_{\ve,i}}{\mathcal{Z}_{{\rm TPE},i}}
\end{align} then expresses the statistical support for the notion that the $N$ systems are characterised by the \mve\ description instead of TPE.

In principle, one could also include the fact that all observed systems should be explained by exactly one EOS.
This leads to the alternative expression for the evidence
\begin{align}
    \mathcal{Z}=& \int \prod_{i=1}^N {L}_i(\theta_i, {\rm EOS}_i) \pi(\theta_i, {\rm EOS}_i) \nonumber\\
    &\cdot\delta({\rm EOS}_i - {\rm EOS}_1) \,\dift\theta_i \dift{\rm EOS}_i  \\
    =& \int \prod_{i=1}^N {L}_i(\theta_i, {\rm EOS}) \pi(\theta_i, {\rm EOS}) \,\dift\theta_i \dift{\rm EOS}  \\
    =& \int \prod_{i=1}^N \mathcal{Z}_i p_i({\rm EOS}) \,\dift{\rm EOS}  \\
    =& \prod_{j=1}^N \mathcal{Z}_j\int \prod_{i=1}^N p_i({\rm EOS}) \,\dift{\rm EOS},\label{eq:alt_bayes}
\end{align}
with $p_i({\rm EOS})$ denoting the posterior of inference run $i$ marginalised over all parameters but the EOS.
However, this prescription does not reflect the construction of our EOS sets that are meant to convey current modelling uncertainties in chiral EFT. 
The EOS parameter space therefore leads to unequal prior densities in \textLambda.
Consider, for instance, a segment of \textLambda\ space at relatively low mass that is only approximately met by a single TPE EOS, whereas multiple \mve\ EOS provide similarly good agreement with observations.
This would naturally happen if \mve\ describes the true EOS, independent of the total number of EOSs in each set.
After some mergers with near solar mass NSs, eq.~\ref{eq:alt_bayes} would still suggest model preference for the inappropriate TPE description because it matches the expectation of a single true EOS better.
We see this effect in Fig.~\ref{fig:alt_result}, massively reducing model preference in case of the TPE injection.
\begin{figure}[t]
    \centering
    \includegraphics[width=0.48\textwidth]{Images/alt_evidence_TPE_inj.pdf}
    \includegraphics[width=0.48\textwidth]{Images/alt_evidence_VE1_inj.pdf}
    \caption{\textbf{Comparison of Bayes Factors:} 
    Bayes factors as in Fig.~\ref{fig:main_result}, using eq.~\ref{eq:usual_bayes} in the upper and eq.~\ref{eq:alt_bayes} in the lower part.
    Note the different scales.
    For the TPE (top) injection, the model preference becomes less decisive and the inclusion of some runs favors the \mve\ model when assuming that all observations result from the same EOS.
    For the \mve\ injection (bottom), the overall model preference remains nearly constant, while the contribution of some runs varies greatly.
    }
    \label{fig:alt_result}
\end{figure}
Nevertheless, we use the assumption that a single EOS should be responsible for all observations in the related estimate on $R_{1.4}$ and $\Lambda_{1.4}$.
As these necessarily imply an approximation based on the EOS, we employ a joint EOS posterior.
Its distribution $p({\rm EOS})$ after observing $N$ systems is given by
\begin{align*}
    p({\rm EOS})= const.\cdot \frac{\prod_{i=1}^N     p_i({\rm EOS})}{\pi({\rm EOS})^{N-1}},
\end{align*}
with $p_i({\rm EOS})$ denoting the posterior distribution obtained from the $i$-th event.

\subsection{Re-analysis of GW170817}
\begin{table}[tbhp]
    \centering
    \begin{tabular}{clrlrl}
    &   parameter  &   \multicolumn{2}{c}{symbol} & \multicolumn{2}{c}{prior bounds} \\
      \hline
       \multirow{6}{8.5pt}{\rotatebox{90}{observational}}& lum. distance [Mpc] &&$d_{\rm L}$ & 1 &-- 75 \\
       & inclination &$\cos$& $\theta_{JN}$& -1 &-- 1\\
      & phase [rad] &&$\phi$ & 0 &-- $2\pi$\\
       & polarisation [rad] &&$\psi$ & 0 &-- $\pi$ \\
        % & spin-phase angle &&$\phi_{JL}$ & 0 &-- $2\pi$\\
        & right ascension [rad]&& \textalpha &  3.44616 & (exact)\\
        & declination [rad]& & \textdelta &  -0.408084 & (exact)\\
      \hline
      \multirow{4}{8.5pt}{\rotatebox{90}{orbital}} & chirp mass $[M_\odot] $ &&$\mathcal{M}$ &  1.18 &-- 1.21 \\
       & mass ratio && $q$ & 0.125 &-- 1\\
       &component mass $[M_\odot]$ && $M_{i}$ & \textit{ \textgreater 1.0}\\
        & aligned component spin & &$\chi_i$ & -0.15&-- 0.15 \\
        \hline
        hyper & Equation of State &&EOS & 1 &-- 3000
    \end{tabular}
    \caption[GW Sampling Parameters in GW170817-Analysis]
    {\textbf{GW Sampling Parameters in GW170817-Analysis}: 
    Most priors are uniform within given bounds. 
    Luminosity distance $d_{\rm L}$ is uniform within a comoving volume of the specified radial dimension. 
    The EOS prior is weighted by the ability to support mass constraints from high-mass pulsars, NICER observations and the kilonova observations that suggested the formation of a hypermassive NS. 
    The component mass prior indicates a constraint that is not used as a sampling parameter. }
    \label{tab:gw_priors}
\end{table}
For a reanalysis of GW170817~\cite{LIGOScientific:2019lzm}, we use the available information on the GRB afterglow and kilonova, motivating the modified prior distribution given in Tab. \ref{tab:gw_priors}. 
We also use a more informative EOS prior that is weighted by minimum mass constraints from precise pulsar observations~\cite{Romani:2022,Arzoumanian:2017puf,Antoniadis:2013pzd,Fonseca:2021wxt}, evidence for the formation of a hypermassive neutron star in the merger~\cite{Rezzolla:2017aly,Margalit:2017dij} and NICER analysis of millisecond pulsars~\cite{Riley:2019yda,Riley:2021pdl,Miller:2019cac,Miller:2021qha}.
Each measurement is assumed to be subject to Gaussian errors characterized by the respectively published uncertainty.

\begin{figure*}%[thp]
    \centering
    \includegraphics[width=0.97\textwidth]{Images/GW170817_GWcorner.pdf}
    \caption[Important System Parameters for GW170817]
    {\textbf{Important System Parameters for GW170817}:
    We show a corner plot of (in reading direction) chirp mass, EOS index, mass ratio, tidal deformability and luminosity distance. 
    The blue contours represent the \mve~recovery. 
    A TPE recovery is not included because the low information gain in the EOS posterior (as indicated by the marginal deviation from the prior, shown in the corresponding histogram as a faint line) suggests that no constraints can be won. Dashed lines in the top histograms mark the 90\% CI, contours indicate 1-, 2-, and 3-\textsigma\ confidence levels in the 2D-histograms.} 
    \label{fig:GW170817_corner}
\end{figure*}
Fig. \ref{fig:GW170817_corner} displays the recovered spread of several key parameters which are consistent with the original findings and recent reanalysis~\cite{LIGOScientific:2017qsa,Narikawa:2022saj}. 
The luminosity distance peaks sharply near \SI{46}{Mpc}, matching the spread in source chirp mass.
This is slightly lower than originally reported.
We can associate this effect with the wide spread of mass ratios, falling even below 0.6.
The low mass ratios correspond to the extended prior range for the aligned spin components in comparison with Ref. \cite{LIGOScientific:2017qsa} that considered the case $|{\chi_i}|<0.05$.
Since we find the spins (not shown) to deviate only slightly from the prior distribution and to be strongly anti-correlated, we conclude that there is no evidence for significant spin effects.
The tidal deformability is relatively tightly constrained, falling way below the limits in nuclear-physics agnostic analysis of the original discovery~\cite{Abbott:2018exr,De:2018uhw,Abbott:2018wiz} and matching findings of $80\leq\Tilde{\Lambda}\leq 580$ in a similar chiral EFT framework~\cite{Tews:2018iwm}.
This is a prior-driven conclusion, though, and we find that the EOS distribution has hardly relaxed from the prior.
This is due to the fact that GW170817 and chiral EFT up to $2\rho_{\rm sat}$ provide similar information on the EOS~\cite{Capano:2019eae}.
Since this run at the upper limit of plausible deformabilities is uninformative, no better constraints can be expected from a TPE recovery on these data.
This analysis does, therefore, suggest no preference for any particular realisation of 3N interactions.
Other GW detections with neutron stars have so far proven even less informative with respect to tidal effects.
Further measurements of NS mergers are expected in the next observing runs, but current population models make it unlikely that these include signals that are considerably stronger than GW170817.

\subsection{Posterior Validation}

Our analysis comes with some caveats that we address in the following.
Ref.~\cite{Kunert:2021hgm} has shown in a comparable framework how systematic errors resulting from approximations in available waveform approximants affect the determination of tidal effects.
Given the greatly increased sensitivity of ET and the prospect of advances in waveform modelling in the upcoming years, we are optimistic that these uncertainties will be reduced significantly when analyzing future detections.
We further based our analysis on observations above \SI{30}{Hz} where tidal effects begin to contribute.
Inference on the full detection band would have further increased the high computational cost of this study by orders of magnitude.
The mass and spin parameters, however, are best determined at \qtyrange{5}{9}{Hz}~\cite{Dietrich:2020eud}.
Measuring them with high precision in this range would naturally constrain the inference of tidal parameters, too.
Similarly, a signal recorded by a GW detector network or even identified in optical counterparts would constrain the sky localisation much tighter.
To mimic these effects, we re-analyse a signal with particularly poor parameter estimation in the \mve\ injection under the assumption, that a) mass parameters and sky localisation were tightly constrained -- as expected from a full bandwidth detection in a GW detector network -- and that b) the luminosity distance was precisely known, as expected from the identification of the host galaxy to an EM counterpart.
\begin{figure*}
    \centering
    \includegraphics[width=0.75\textwidth]{Images/prior_changes.pdf}
    \caption{\textbf{Selected posteriors with adapted priors:} We show the posteriors for the EOS indices (left), the associated tidal deformability (center), and luminosity distance (right) in our original set-up (top), with tightly constrained mass parameters as expected from realistic network operation (middle), and with known distance as expected from the localisation of an EM counterpart (bottom). 
    The fainter lines in the EOS plots indicate the indices' prior weight.
    Note how the distance estimate and the tidal description improve when assuming knowledge on the mass parameters.
    Further limiting the luminosity distance does not improve the quality of other parameters.}
    \label{fig:prior_changes}
\end{figure*}
\Cref{fig:prior_changes} shows that the first option does indeed lead to a more plausible description of the tidal effects.
The major distance overestimate in our original analysis drives the EOS sets to EOSs with a lower tidal deformability to counter the underestimate in the (redshifted) source frame mass parameters. 
These are associated with more more compact neutron stars that typically have a lower TOV limit.
Because our prior penalizes low TOV limits, good waveform fits had previously worse prior support, particularly for \mve. 
Properly identifying the detected mass within narrow margins of \SI{0.01}{M_\odot} removes this source of uncertainty considerably and resolves the erroneous model preference for TPE against the injected \mve\ EOS.
Constraining the luminosity distance even further in the second step does not improve the estimation of other parameters.

Moreover, we reanalyze this signal with a different EOS prior that penalizes TOV limits above $2.16^{+0.17}_{-0.15}$\unit{M_\odot}~\cite{Rezzolla:2017aly}.
This step does not significantly improve the \mve\ parameter estimation, but leads to a modified EOS posterior.
The TPE sampling, in contrast, does not converge within acceptable runtime because the adjusted prior effectively outlaws EOSs that previously allowed suitable waveform descriptions.
This makes it much harder for the nested sampling algorithm to find parameters with better likelihood.
Enforcing convergence by allocating significantly more computing resources would certainly have removed the model preference for TPE against the injection.
This supports our conclusion that realistic GW detections in the ET era with improved priors from upcoming detections will be capable of quickly distinguishing 3N interactions.

