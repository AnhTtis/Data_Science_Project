% !TEX root=main.tex

\section{Related works}
\label{sec:related-work}


%\subsection{Object Detection}
\noindent \textbf{Object Detection.}  
General purpose visual object detectors are built primarily focusing on natural images. They can be broadly classified into two-stage \cite{faster_rcnn-Ren-2015, fpn-Lin-2017} and one-stage \cite{yolo_9000-Redmon-2017, ssd-Liu-2016} object detectors. Two-stage object detectors first extract potential object regions(ie, object proposals) in the first stage. Then in the second stage, the proposal regions are classified into object categories. Earlier approaches extract these RoIs using low-level image features \cite{rcnn-Girshick-2016, fast_rcnn-Girshick-2015}, etc. Later, a learnable component called RPN (Region Proposal Network) is proposed for object proposal extraction, giving birth to end-to-end two-stage detectors~\cite{faster_rcnn-Ren-2015, fpn-Lin-2017, rfcn-Dai-2016}. One-stage object detectors in contrast avoid the RoI extraction stage and classify and regress directly from the anchor boxes. They are  generally fast and applicable to real-time object detection \cite{yolo_9000-Redmon-2017, yolo-Redmon-2016, ssd-Liu-2016, retinanet-Lin-2017}. But generally speaking, two-stage detectors are more accurate and hence used more in aerial detection~\cite{glsan-Deng-2020, dmap-Li-2020}. Recently, one-stage detectors are being explored in aerial images~\cite{querydet-Yang-2022}. These days, anchor-free detectors~\cite{fcos-Tian-2019, centernet-Duan-2019} is getting popular, since they avoid the need for hand-crafted anchor box dimensions and their matching process, which is often specifically tuned depending on the size distribution of the objects. We performed our empirical study mostly on the two-stage detector for a fair comparison with existing approaches, but also report results on the modern anchor-free one-stage detector to validate the generality of our approach.

%\subsection{Detection of Small Objects}
\noindent \textbf{Detection of Small Objects.} 
Small object detection is studied extensively in the literature. Cheng et al. \cite{sod_survey-Cheng-2022} provides a comprehensive survey of small object detection techniques which is broadly classified into: scale aware training \cite{trident_nw-Li-2019, fpn-Lin-2017, snip-Singh-2018, sniper-Singh-2018}, super-resolution methods \cite{pgan-Li-2017}, context modeling \cite{inside_out_net-Bell-2016}, and density guided detection \cite{dmap-Duan-2021, dmap-Li-2020, clusnet-Yang-2019} among others. Due to the crowded distribution of small objects at sparse locations in the high-resolution images, density cropping is a popular strategy in aerial image detection \cite{dmap-Duan-2021, clusnet-Yang-2019, dmap-Li-2020, glsan-Deng-2020}. But this is usually obtained by additional learnable components and multi-stage training. Thus, such methods are more complex to use than uniform cropping. We also use the density cropping strategy but re-purpose the detector itself for extracting the crops, so no additional learnable components are needed. Hence, we believe that our method has the practical simplicity of the uniform cropping  strategy, yet retains the benefit of density guided detection.