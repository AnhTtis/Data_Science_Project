\documentclass[10pt]{article}
\usepackage[letterpaper, margin=0.8in]{geometry}
%\usepackage{achemso}
\usepackage{amsmath}
\usepackage{mathtools}
\usepackage[version=3]{mhchem}
\usepackage[english]{babel}
\usepackage{braket}
\usepackage{mathrsfs}
\usepackage{courier}
%\usepackage{hyperref}
\usepackage{bbold}
\usepackage{mathrsfs}
\usepackage{amssymb}
\usepackage{cancel}
\usepackage{xr}
%\usepackage[capitalize]{cleveref}
\renewcommand\vec\mathbf
\newcommand\tr{\operatorname{tr}}
\DeclareMathOperator{\erf}{erf}
\setlength\parindent{0pt}
\usepackage{color}
\usepackage{soul}

\usepackage{ifthen}
\newboolean{showannotations}   
\setboolean{showannotations}{true} 
\newcommand*{\added}[1]{%
  \ifthenelse{\boolean{showannotations}}{{\color{blue}#1}}{#1}%
}

\newcommand*{\removed}[1]{%
  \ifthenelse{\boolean{showannotations}}{{\color{red}#1}}{}%
}

\newcommand*{\change}[2]{%
  \ifthenelse{\boolean{showannotations}}{{\color{red}#1}{\color{blue}#2}}{#2}%
}

\renewcommand{\thepage}{S\arabic{page}}
\renewcommand{\theequation}{S\arabic{equation}}
\renewcommand{\thefigure}{S\arabic{figure}}
\usepackage{graphicx}% Include figure files
\usepackage{dcolumn}% Align table columns on decimal point

\usepackage{cuted}
\usepackage{bbold}
\usepackage[normalem]{ulem}
\newcommand{\stkout}[1]{\ifmmode\text{\sout{\ensuremath{#1}}}\else\sout{#1}\fi}
\newcommand{\msout}[1]{\text{\sout{\ensuremath{#1}}}}
% for Nature Comm SI style
\renewcommand{\thepage}{ \arabic{page}} 
\renewcommand{\thesection}{\normalsize \arabic{section}} 
\renewcommand{\theequation}{S\arabic{equation}} 
\renewcommand{\thetable}{\arabic{table}}  
\renewcommand{\thefigure}{S\arabic{figure}}
\renewcommand{\figurename}{Supplementary Figure}
\renewcommand{\tablename}{Supplementary Table}
 
 %\renewcommand{\thesection}{Supplementary Note \arabic{section}}   
 %\renewcommand{\thesection}{} 
 %\renewcommand{\thesubsection}{} 
 %\renewcommand{\thesubsubsection}{} 




\begin{document}
{\footnotesize


\title{Supporting Information: Microscopic Theory of Multimode Polariton Dispersion in Multilayered Materials}
 

\author{ Arkajit Mandal$^1$\footnote{am5815@columbia.edu}~, Ding Xu$^1$, Ankit Mahajan$^1$, Joonho Lee$^1$, Milan E. Delor$^1$,  David R. Reichman$^1$\footnote{drr2103@columbia.edu}}

\maketitle

{ $^1$ Department of Chemistry, Columbia University, 3000 Broadway, New York, New York, 10027,  U.S.A}



\section{\normalsize Dipole Gauge Hamiltonian beyond Long-wavelength approximation}
Here we provide details for obtaining the dipole-gauge Hamiltonian used in the main text. The p.A Hamiltonian is given by (using atomic units $\hbar = 1$),
\begin{align}\label{eqn:HpA0}
\hat{H}_\mathrm{p \cdot A} &=   \sum_{u}  \frac{(\hat{\bf P}_u - e\hat{\bf A}(\hat{\bf Q}_u))^2}{2m_\mathrm{n}} +  \sum_{j} \frac{(\hat{\bf p}_j + e\hat{\bf A}(\hat{\bf r}_j))^2}{2m_e}   + V_\mathrm{coul}(\{{ {\bf r}_j, {\bf Q}_j}\}) + \sum_{{\boldsymbol k}, {\bf\xi}} \hat{a}_{{\boldsymbol k}, {\bf\xi}}^{\dagger}\hat{a}_{{\boldsymbol k}, {\boldsymbol{\xi}}}\omega_{\boldsymbol k},
\end{align}
where  $\hat{\bf  P}_j$ and $\hat{\bf  p}_j $  are the $j$th  canonical momentum operators with the corresponding nuclear position operator $\hat{\bf Q}_j$ and electronic position operator $\hat{\bf r}_j$ for nuclei and electrons. $\hat{\bf A}({\bf r})$ is the total vector potential at ${\bf r}$ and $\hat{a}_{{\boldsymbol k}, {\bf\xi}}^{\dagger}$ ($\hat{a}_{{\boldsymbol k}, {\bf\xi}}$) is the photon creation (annihilation) operator for radiation mode ${\bf k}$ with polarization ${\boldsymbol{\xi}} \in \{  {s}, { p} \}$. Note that we assume the charges of the nuclei are unity for simplicity, but the generalization beyond this is straightforward. The vector potential is then given as~\cite{Steck}
\begin{align}
&\hat{\bf A} ({\bf r} )  = \sum_{{\boldsymbol k}, {\boldsymbol \xi}} \hat{\bf A}_{{\boldsymbol k}, {\boldsymbol \xi}} ({\bf r} )
 = \sum_{{\boldsymbol k}, {\boldsymbol \xi}} \frac{{ \boldsymbol \lambda}_{{\boldsymbol k},   {\boldsymbol \xi}}}{\omega_c({\boldsymbol k})} \left[ e^{- i {\bf k_\parallel} \cdot {\bf r}} \hat{a}_{{\boldsymbol k},   {\boldsymbol \xi}}^\dagger +  e^{ i {\bf k_\parallel}  \cdot {\bf r}} \hat{a}_{{\boldsymbol k},   {\boldsymbol \xi}}\right]\sin({ k_y}   {  r_y} )  =  \sum_{{\boldsymbol k}, {\boldsymbol \xi}}\Big[ \mathcal{A}_{{\boldsymbol k},{\boldsymbol \xi}}({\bf r})\hat{a}_{{\boldsymbol k}}^{\dagger} +\mathcal{A}_{{\boldsymbol k},{\boldsymbol \xi}}^{*}({\bf r})\hat{a}_{{\boldsymbol k}}^{\dagger}\Big] \nonumber 
% &= \sum_{{\boldsymbol k}, {\boldsymbol \xi}} \frac{{ \boldsymbol \lambda}_{{\boldsymbol k},   {\boldsymbol \xi}}}{\omega_c({\boldsymbol k})} \left[ \hat{a}_{{\boldsymbol k},   {\boldsymbol \xi}}^\dagger f^{*}({\bf r}) +  \hat{a}_{{\boldsymbol k},   {\boldsymbol \xi}}  f({\bf r}) \right]
\end{align}
where ${ \boldsymbol \lambda}_{{\boldsymbol k}, {\boldsymbol \xi}} = \sqrt{\frac{\hbar \omega_c({\boldsymbol k})}{{\epsilon_0  \epsilon_r  \mathcal{V}}}} \hat{\boldsymbol e}_{{\boldsymbol k},   {\boldsymbol \xi}} $ with $\epsilon_0$  and $ \epsilon_r $ the vacuum and material permittivity, respectively, $\mathcal{V}$ is the quantization volume, $\hat{\boldsymbol e}_{{\boldsymbol k}, {\boldsymbol \xi}} \perp  {\boldsymbol k}$ is the polarization direction of the radiation mode ${\boldsymbol k} \equiv ( k_x, k_y, k_z)$ and ${\boldsymbol k_\parallel } = k_x \hat{\bf x} + k_z \hat{\bf z}$ is the longitudinal component of the cavity mode ${\boldsymbol k}$. 

First, we perform a nuclear-centered Power-Zienau-Woolley (PZW) transformation following Ref.~\cite{Keeling2012, Mandal2022CR} for the nuclear DOF with the unitary transformation operator $\hat{U}_\mathrm{nuc} = e^{- i \hat{\bf A}({\bf R}_u) \cdot \sum_u e\hat{\bf Q}_u }$ where ${\bf R}_u$ is the equilibrium position of $\hat{\bf Q}_u$ such that we can approximate 
$\hat{\bf A}(\hat{\bf Q}_u ) \approx \hat{\bf A}({\bf R}_u)$.  With this transformation we have,

\begin{align} 
&\hat{U}_\mathrm{nuc}^{\dagger}\hat{H}_\mathrm{p \cdot A} \hat{U}_\mathrm{nuc} =   \sum_{u}  \frac{\hat{\bf P}_u^2}{2m_\mathrm{n}} +  \sum_{j} \frac{(\hat{\bf p}_j + e\hat{\bf A}(\hat{\bf r}_j))^2}{2m_e}   + V_\mathrm{coul}(\{{ {\bf r}_j, {\bf Q}_j}\}) + \sum_{{\boldsymbol k}, {\bf\xi}} \Big(\hat{U}_\mathrm{nuc}^{\dagger} \hat{a}_{{\boldsymbol k}, {\bf\xi}}^{\dagger}\hat{U}_\mathrm{nuc} \Big)\Big(\hat{U}_\mathrm{nuc}^{\dagger}\hat{a}_{{\boldsymbol k}, {\boldsymbol{\xi}}}\omega_{\boldsymbol k}\hat{U}_\mathrm{nuc}\Big) \omega_{\boldsymbol k} \nonumber \\
&= \hat{ T}_{\bf Q} +  \sum_{j} \frac{(\hat{\bf p}_j + e\hat{\bf A}(\hat{\bf r}_j))^2}{2m_e}   + V_\mathrm{coul}(\{{ {\bf r}_j, {\bf Q}_j}\}) + \sum_{{\boldsymbol k}, {\bf\xi}} \Big( \hat{a}_{{\boldsymbol k}, {\bf\xi}}^{\dagger} + i \sum_u e\hat{\bf Q}_u\mathcal{A}^{*}_{{\boldsymbol k},{\boldsymbol \xi}} ({\bf R}_u)\Big)\Big( \hat{a}_{{\boldsymbol k}, {\boldsymbol{\xi}}} - i \sum_u e\hat{\bf Q}_u\mathcal{A}_{{\boldsymbol k},{\boldsymbol \xi}} ({\bf R}_u)\Big)\omega_{\boldsymbol k}~~~~.
\end{align}

Second, we perform an orbital centered PZW transformation following Ref.~\cite{JiajunPRB2020,DmytrukPRB2021} for the multi-electron system $\hat{U}_\mathrm{el} = e^{i \int d{\bf r} \Psi^{\dagger}({\bf r})  \chi({\bf r}) \Psi({\bf r}) }$ where $ \chi({\bf r}) = \int_{0}^{{\bf r}} \hat{\bf A} ({\bf s})\cdot {\bf ds}$, $\Psi^{\dagger}({\bf r}) = \sum_{u,\alpha}\Phi_{u,\alpha}({\bf r}) \hat{c}_{u,\alpha}^{\dagger}$ is the  electronic field operator with $\hat{c}_{u,\alpha}^{\dagger}$ as the spinless fermionic creation operator and $\{\Phi_{u,\alpha}({\bf r})\}$ are localized orthonormal single-particle wavefunctions that are centered around ${\bf R}_{u}$.  Note that  we  make a simplifying approximation of dropping of all spin degrees of
freedom  which also means that we drop all degeneracies coming from the orbital part of the electronic levels~\cite{Combescot2008PRB}. Overall, here  we consider a charge neutral system such that we have equal number of electron and charged nuclei. Using this transformation one obtains~\cite{DmytrukPRB2021} $\hat{H}^{'}_\mathrm{d \cdot E} = \hat{U}_\mathrm{el}^{\dagger}\hat{U}_\mathrm{nuc}^{\dagger}\hat{H}_\mathrm{p \cdot A} \hat{U}_\mathrm{nuc}  \hat{U}_\mathrm{el} $ as

\begin{align}
  \hat{H}^{'}_\mathrm{d \cdot E} 
 &= \hat{ T}_{\bf Q} + \hat{ T}_{\bf r}  + V_\mathrm{coul}(\{{ {\bf r}_j, {\bf Q}_j}\})  + \sum_{{\boldsymbol k}, {\bf\xi}} \Big( \hat{U}_\mathrm{el}^{\dagger} \hat{a}_{{\boldsymbol k}, {\bf\xi}}^{\dagger}\hat{U}_\mathrm{el} + i \sum_u e\hat{\bf Q}_u\mathcal{A}^{*}_{{\boldsymbol k},{\boldsymbol \xi}} ({\bf R}_u)\Big)\Big( \hat{U}_\mathrm{el}^{\dagger} \hat{a}_{{\boldsymbol k}, {\boldsymbol{\xi}}}\hat{U}_\mathrm{el}  - i \sum_u e\hat{\bf Q}_u\mathcal{A}_{{\boldsymbol k},{\boldsymbol \xi}} ({\bf R}_u)\Big)\omega_{\boldsymbol k}.  
\end{align}

Assuming that the spatial variation of $\mathcal{A}^{*}_{{\boldsymbol k},{\boldsymbol \xi}} ({\bf r})$ is negligible within the spatial extent of localized orbitals $\{\Phi_{u,\alpha}({\bf r})\}$   the following expression~\cite{DmytrukPRB2021} is obtained
\begin{align}
\hat{U}_\mathrm{el}^{\dagger} \hat{a}_{{\boldsymbol k}, {\bf\xi}}^{\dagger}\hat{U}_\mathrm{el} &= \hat{a}_{{\boldsymbol k}, {\bf\xi}}^{\dagger} -i\sum_{u,\alpha}e \Bigg[ \int_0^{{\bf R}_u}\mathcal{A}^{*}_{{\boldsymbol k},{\boldsymbol \xi}} (s)ds \Bigg] \hat{c}^{\dagger}_{u, \alpha} \hat{c}_{u, \alpha} - i \sum_{u,\alpha,\alpha'} e\mathcal{A}^{*}_{{\boldsymbol k},{\boldsymbol \xi}} ( {\bf R}_u) \Bigg[\int_{-\infty}^{\infty} d{\bf r}  \Phi_{u,\alpha}({\bf r}) ({\bf r} - {\bf R}_{u}) \Phi_{u,\alpha'}({\bf r})\Bigg] \hat{c}^{\dagger}_{u, \alpha} \hat{c}_{u, \alpha'}   \nonumber\\
&= \hat{a}_{{\boldsymbol k}, {\bf\xi}}^{\dagger} -i \sum_{u,\alpha} e\bar{\boldsymbol \chi}^{*}({\bf R}_u) \hat{c}^{\dagger}_{u, \alpha} \hat{c}_{u, \alpha} - i \sum_{u,\alpha,\alpha'} \mathcal{A}^{*}_{{\boldsymbol k},{\boldsymbol \xi}} ( {\bf R}_u){\boldsymbol \mu}_{\alpha,\alpha'} \hat{c}^{\dagger}_{u, \alpha} \hat{c}_{u, \alpha'}  
\end{align}
where we have defined $\int_0^{{\bf R}_u}\mathcal{A}^{*}_{{\boldsymbol k},{\boldsymbol \xi}} (s)ds = \bar{\boldsymbol \chi}^{*}({\bf R}_u)$ and the transition dipole moment ${\boldsymbol \mu}_{\alpha,\alpha'} = e\int_{-\infty}^{\infty} d{\bf r}  \Phi_{u,\alpha}({\bf r}) ({\bf r} - {\bf R}_{u}) \Phi_{u,\alpha'}({\bf r})$ and we set $\int_{-\infty}^{\infty} d{\bf r}  \Phi_{u,\alpha}({\bf r}) {\bf r} \Phi_{v,\alpha'}({\bf r}) = 0$ for $u \ne v$. Next we perform a polaron transformation with the unitary operator 
\begin{align}
\hat{U}_\mathrm{pol} = \exp\Big[i \sum_{{\boldsymbol k}, {\boldsymbol \xi}}\Big( \hat{a}_{{\boldsymbol k}, {\bf\xi}} \sum_{u,\alpha} e\bar{\boldsymbol \chi}^{*}({\bf R}_u) \hat{c}^{\dagger}_{u, \alpha} \hat{c}_{u, \alpha}+  \hat{a}_{{\boldsymbol k}, {\bf\xi}}^{\dagger} \sum_{u,\alpha} e\bar{\boldsymbol \chi}({\bf R}_u) \hat{c}^{\dagger}_{u, \alpha} \hat{c}_{u, \alpha} \Big) \Big]~~~.
\end{align}


Note the transformation on the fermionic creation operator which induces a Peierls phase~\cite{JiajunPRB2020}, 
\begin{align}
\hat{U}_\mathrm{pol}^{\dagger} \hat{c}_{u,\alpha} \hat{U}_\mathrm{pol} =   \hat{c}_{u,\alpha} \cdot \exp\Big[{i e\sum_{{\boldsymbol k}, {\boldsymbol \xi}} \hat{a}_{{\boldsymbol k}, {\bf\xi}}   \bar{\boldsymbol \chi}^{*}({\bf R}_u)  +  \hat{a}_{{\boldsymbol k}, {\bf\xi}}^{\dagger}  \bar{\boldsymbol \chi}({\bf R}_u)  }\Big]
\end{align}
while at the same time we can write 
$ \hat{U}_\mathrm{el}^{\dagger} \hat{a}_{{\boldsymbol k}, {\bf\xi}}^{\dagger}\hat{U}_\mathrm{el} = \hat{U}_\mathrm{pol} \hat{a}_{{\boldsymbol k}, {\bf\xi}}^{\dagger}  \hat{U}_\mathrm{pol}^{\dagger}  - i \sum_{u,\alpha,\alpha'} \mathcal{A}^{*}_{{\boldsymbol k},{\boldsymbol \xi}} ( {\bf R}_u){\boldsymbol \mu}_{\alpha,\alpha'} \hat{c}^{\dagger}_{u, \alpha} \hat{c}_{u, \alpha'} $. We write a polaron transformed d.E Hamiltonian~\cite{JiajunPRB2020} as

\begin{align}
\hat{U}_\mathrm{pol}^{\dagger}\hat{H}_\mathrm{d \cdot E} \hat{U}_\mathrm{pol} &= \hat{ T}_{\bf Q} + \hat{U}_\mathrm{pol}^{\dagger}(\hat{ T}_{\bf r}  + V_\mathrm{coul}(\{{ {\bf r}_j, {\bf Q}_j}\}) ) \hat{U}_\mathrm{pol} \nonumber \\
&+ \sum_{{\boldsymbol k}, {\bf\xi}} \Big(  \hat{a}_{{\boldsymbol k}, {\bf\xi}}^{\dagger}  + i \sum_u e\hat{\bf Q}_u\mathcal{A}^{*}_{{\boldsymbol k},{\boldsymbol \xi}} ({\bf R}_u)  - i \sum_{u,\alpha,\alpha'} \mathcal{A}^{*}_{{\boldsymbol k},{\boldsymbol \xi}} ( {\bf R}_u){\boldsymbol \mu}_{\alpha,\alpha'} \hat{c}^{\dagger}_{u, \alpha} \hat{c}_{u, \alpha'} \Big)\nonumber \\
&~~~\times\Big(  \hat{a}_{{\boldsymbol k}, {\boldsymbol{\xi}}}  - i \sum_u e\hat{\bf Q}_u\mathcal{A}_{{\boldsymbol k},{\boldsymbol \xi}} ({\bf R}_u) + i \sum_{u,\alpha,\alpha'} \mathcal{A}_{{\boldsymbol k},{\boldsymbol \xi}} ( {\bf R}_u){\boldsymbol \mu}_{\alpha,\alpha'} \hat{c}^{\dagger}_{u, \alpha} \hat{c}_{u, \alpha'}\Big)\omega_{\boldsymbol k}~~~~.
\end{align}

Next, we write $\mathcal{\hat{H}}_{el} = \hat{ T}_{\bf r}  + V_\mathrm{coul}(\{{ {\bf r}_j, {\bf Q}_j}\}) =   \sum_{\alpha , \alpha'} h_{u,v,\alpha,\alpha'} \hat{c}^{\dagger}_{u, \alpha} \hat{c}_{v, \alpha'} +  \sum_{\alpha,\alpha', \beta, \beta'}  \sum_{u,v,u',v'} U_{\alpha,\alpha', \beta, \beta'}^{u,v,u',v'} \hat{c}^{\dagger}_{u, \alpha} \hat{c}_{v, \alpha'}^{\dagger} \hat{c}_{u', \beta} \hat{c}_{v', \beta'} $ in the second quantization form. Note that
\begin{align}
\hat{U}_\mathrm{pol}^{\dagger}\hat{c}^{\dagger}_{u, \alpha} \hat{c}_{v, \alpha'} \hat{U}_\mathrm{pol} &= (\hat{U}_\mathrm{pol}^{\dagger}\hat{c}^{\dagger}_{u, \alpha}\hat{U}_\mathrm{pol} ) (\hat{U}_\mathrm{pol}^{\dagger} \hat{c}_{v, \alpha'} \hat{U}_\mathrm{pol}) \nonumber \\
&= \hat{c}^{\dagger}_{u, \alpha} \hat{c}_{v, \alpha'} \cdot  e^{i \sum_{{\boldsymbol k}, {\boldsymbol \xi}} \hat{a}_{{\boldsymbol k}, {\bf\xi}}   e (\bar{\boldsymbol \chi}^{*}({\bf R}_u) - \bar{\boldsymbol \chi}^{*}({\bf R}_v))  +  \hat{a}_{{\boldsymbol k}, {\bf\xi}}^{\dagger}  e(\bar{\boldsymbol \chi}({\bf R}_u) - \bar{\boldsymbol \chi}({\bf R}_v)  )} \nonumber\\
&\approx \hat{c}^{\dagger}_{u, \alpha} \hat{c}_{v, \alpha'} \cdot  e^{i e\sum_{{\boldsymbol k}, {\boldsymbol \xi}} \hat{a}_{{\boldsymbol k}, {\bf\xi}}   ({\bf R}_u \cdot \mathcal{A}^{*}_{{\boldsymbol k},{\boldsymbol \xi}} ( {\bf R}_u) - {\bf R}_v \cdot \mathcal{A}^{*}_{{\boldsymbol k},{\boldsymbol \xi}} ( {\bf R}_v))  +  \hat{a}_{{\boldsymbol k}, {\bf\xi}}^{\dagger}  ({\bf R}_u \cdot \mathcal{A}_{{\boldsymbol k},{\boldsymbol \xi}} ( {\bf R}_u) - {\bf R}_v \cdot \mathcal{A}_{{\boldsymbol k},{\boldsymbol \xi}} ( {\bf R}_v))} \nonumber\\
&=\hat{U}_\mathrm{pol}^{' \dagger}\hat{c}^{\dagger}_{u, \alpha} \hat{c}_{v, \alpha'} \hat{U}^{'}_\mathrm{pol} ,
\end{align}
where we have assumed that the spatial variation of the vector potential is negligible over spatial extent of  $ \Phi_{u,\alpha}({\bf r}) $ and $ \Phi_{v,\alpha}({\bf r}) $. Such a approximation will hold when $h_{u,v,\alpha,\alpha'}$ is only non-negligible when ${\bf R}_{u} \approx {\bf R}_{v}$, and   $U_{\alpha,\alpha', \beta, \beta'}^{u,v,u',v'}$ is also only non-negligible when ${\bf R}_{u} \approx {\bf R}_{u'}$  and ${\bf R}_{v} \approx {\bf R}_{v'}$.  Here we have defined $\hat{U}^{'}_\mathrm{pol} $ as
\begin{align}
\hat{U}'_\mathrm{pol} = \exp\Big[i \sum_{{\boldsymbol k}, {\boldsymbol \xi}}\Big( \hat{a}_{{\boldsymbol k}, {\bf\xi}} \sum_{u,\alpha} e {\bf R}_u \cdot \mathcal{A}^{*}_{{\boldsymbol k},{\boldsymbol \xi}} ( {\bf R}_u) \hat{c}^{\dagger}_{u, \alpha} \hat{c}_{u, \alpha}+  \hat{a}_{{\boldsymbol k}, {\bf\xi}}^{\dagger} \sum_{u,\alpha} e {\bf R}_u \cdot \mathcal{A}_{{\boldsymbol k},{\boldsymbol \xi}} ( {\bf R}_u) \hat{c}^{\dagger}_{u, \alpha} \hat{c}_{u, \alpha} \Big) \Big]~~~~.
\end{align}
We thus obtain the final form of the d.E Hamiltonian by performing the transformation with $U_{\phi} = \prod_{{\boldsymbol k},{\boldsymbol \xi}} e^{i\frac{\pi}{2}\hat{a}_{{\boldsymbol k},{\boldsymbol \xi}}^{\dagger}\hat{a}_{{\boldsymbol k},{\boldsymbol \xi}}}$ such that $U_{\phi}^{\dagger} \hat{a}_{{\boldsymbol k},{\boldsymbol \xi}} U_{\phi} = -i \hat{a}_{{\boldsymbol k},{\boldsymbol \xi}}$

\begin{align}
 \hat{H}_\mathrm{d \cdot E} &= U_{\phi}^{\dagger} \hat{H}_\mathrm{d \cdot E}^{'} U_{\phi} = U_{\phi}^{\dagger} (\hat{U}_\mathrm{pol}^{' } \hat{U}_\mathrm{pol}^{\dagger} \hat{H}_\mathrm{d \cdot E}^{'}\hat{U}_\mathrm{pol}  \hat{U}_\mathrm{pol}^{' \dagger}) U_{\phi} \nonumber \\
 &= \hat{ T}_{\bf Q} + \mathcal{\hat{H}}_{el}   +  \sum_{{\boldsymbol k}, {\bf\xi}} \Big(  \hat{a}_{{\boldsymbol k}, {\bf\xi}}^{\dagger}  +  \sum_u e (\hat{\bf Q}_u - {\bf R}_u \sum_{\alpha} \hat{c}^{\dagger}_{u, \alpha} \hat{c}_{u, \alpha} )\mathcal{A}^{*}_{{\boldsymbol k},{\boldsymbol \xi}} ({\bf R}_u)  - \sum_{u,\alpha,\alpha'} \mathcal{A}^{*}_{{\boldsymbol k},{\boldsymbol \xi}} ( {\bf R}_u){\boldsymbol \mu}_{\alpha,\alpha'} \hat{c}^{\dagger}_{u, \alpha} \hat{c}_{u, \alpha'} \Big)\nonumber \\
&~~~~~~~~~~~~~~~~~~~~\times\Big(  \hat{a}_{{\boldsymbol k}, {\boldsymbol{\xi}}}  +  \sum_u e(\hat{\bf Q}_u - {\bf R}_u \sum_{\alpha} \hat{c}^{\dagger}_{u, \alpha} \hat{c}_{u, \alpha} )\mathcal{A}_{{\boldsymbol k},{\boldsymbol \xi}} ({\bf R}_u) - \sum_{u,\alpha,\alpha'} \mathcal{A}_{{\boldsymbol k},{\boldsymbol \xi}} ( {\bf R}_u){\boldsymbol \mu}_{\alpha,\alpha'} \hat{c}^{\dagger}_{u, \alpha} \hat{c}_{u, \alpha'}\Big)\omega_{\boldsymbol k}~~~.
\end{align}

We obtain the `electronic' part of the d.E Hamiltonian, so called polaritonic d.E Hamiltonian $\mathcal{\hat{H}}_\mathrm{d \cdot E}(\{{\bf R}_{u}\})$ at the nuclear configuration $\{{\bf R}_{u} \}$  as
\begin{align}
 \mathcal{\hat{H}}_\mathrm{d \cdot E}(\{{\bf R}_{u}\}) =  \mathcal{\hat{H}}_{el}    +  &\sum_{{\boldsymbol k}, {\bf\xi}} \Big(  \hat{a}_{{\boldsymbol k}, {\bf\xi}}^{\dagger}  + \sum_u e ({\bf R}_u - {\bf R}_u \sum_{\alpha} \hat{c}^{\dagger}_{u, \alpha} \hat{c}_{u, \alpha} )\mathcal{A}^{*}_{{\boldsymbol k},{\boldsymbol \xi}} ({\bf R}_u)  -  \sum_{u,\alpha,\alpha'} \mathcal{A}^{*}_{{\boldsymbol k},{\boldsymbol \xi}} ( {\bf R}_u){\boldsymbol \mu}_{\alpha,\alpha'} \hat{c}^{\dagger}_{u, \alpha} \hat{c}_{u, \alpha'} \Big)\nonumber \\
&~\times\Big(  \hat{a}_{{\boldsymbol k}, {\boldsymbol{\xi}}}  + \sum_u e({\bf R}_u - {\bf R}_u \sum_{\alpha} \hat{c}^{\dagger}_{u, \alpha} \hat{c}_{u, \alpha} )\mathcal{A}_{{\boldsymbol k},{\boldsymbol \xi}} ({\bf R}_u) - \sum_{u,\alpha,\alpha'} \mathcal{A}_{{\boldsymbol k},{\boldsymbol \xi}} ( {\bf R}_u){\boldsymbol \mu}_{\alpha,\alpha'} \hat{c}^{\dagger}_{u, \alpha} \hat{c}_{u, \alpha'}\Big)\omega_{\boldsymbol k}~~~.
\end{align}
% so the third term is both permant and transition dipole originating from the local assymetry of the wavefunction 

% The second term is the overall permanent dipole for charge separate states




\section{\normalsize  Exciton-Polariton Hamiltonian} We construct a simple (Frenkel) exciton-polariton Hamiltonian using the d.E Hamiltonian provided in the previous section. We adopt a simple two-band model with dispersionless valence and conduction bands, separated in energy by $\omega_0$. We use a matter basis containing a ground state $|G\rangle = \prod_{u} \hat{c}_{u,g}^{\dagger} |0\rangle$ where all electrons fill up the valence band and localized excitonic states $|E_u\rangle =  \hat{d}_u^\dagger  |G\rangle = \hat{c}_{u,e}^{\dagger}\hat{c}_{u,g}|G\rangle$ where one electron is placed in a localized excited state from a localized ground state at ${\bf R}_u$\cite{BassaniINCD1986,Combescot2008PRB}. These localized excitons interact via nearest-neighbor interactions as is done in the standard Frankel exciton model. We also assume that within this basis, the permanent dipoles $\mu_{\alpha,\alpha} = 0$ and only retain the transition dipoles $\mu_{\alpha,\alpha'}$ with $\alpha \ne \alpha'$. Further, using a transformation   $U_{\phi} = \prod_{{\boldsymbol k},{\boldsymbol \xi}} e^{i\frac{\pi}{2}\hat{a}_{{\boldsymbol k},{\boldsymbol \xi}}^{\dagger}\hat{a}_{{\boldsymbol k},{\boldsymbol \xi}}}$ such that $U_{\phi}^{\dagger} \hat{a}_{{\boldsymbol k},{\boldsymbol \xi}} U_{\phi} = -i \hat{a}_{{\boldsymbol k},{\boldsymbol \xi}}$ (which makes the light-matter couplings real) we obtain the following exciton-polariton dipole gauge Hamiltonian (dropping the nuclear kinetic energy operator) at the nuclear configuration $\{\hat{\bf Q}_{u}\} = \{{\bf R}_{u}\}$,
\begin{align}\label{eqn:HdE0}
\mathcal{\hat{H}}_\mathrm{d \cdot E}     &=   \omega_0 \sum_u \hat{d}_{u}^{\dagger}\hat{d}_{u}  + \sum_{\langle u , v\rangle} \tau_{uv} (\hat{d}_{u}^{\dagger}\hat{d}_{v} + \hat{d}_{u}\hat{d}_{v}^{\dagger})    + \sum_{{\boldsymbol k}, u, {\boldsymbol \xi}}   (\hat{\boldsymbol \mu}_u\cdot \hat{\boldsymbol \lambda}_{{\boldsymbol k},{\boldsymbol \xi}})  \Big(e^{- i {{ \bf k_\parallel}} \cdot {\bf R}_{u}} \hat{a}_{{\boldsymbol k},{\boldsymbol \xi}}^\dagger +  e^{ i { \bf k_\parallel} \cdot {\bf R}_{u}} \hat{a}_{{\boldsymbol k},{\boldsymbol \xi}} \Big) \sin({ k_y} (\hat{\bf y}\cdot {\bf R}_{u})) \\
&+  \sum_{{\boldsymbol k},{\boldsymbol \xi}} \hat{a}_{{\boldsymbol k},{\boldsymbol \xi}}^{\dagger}\hat{a}_{{\boldsymbol k},{\boldsymbol \xi}}\omega_c({\boldsymbol k}) + \sum_{u,v} \sum_{{\boldsymbol k},{\boldsymbol \xi}}  {\frac{ e^{i { \bf k_\parallel} \cdot ({\bf R}_{v}- {\bf R}_{u})}}{ \omega_c({\boldsymbol k}) }  }     (\hat{\boldsymbol \mu}_u\cdot {\boldsymbol \lambda}_{{\boldsymbol k},{\boldsymbol \xi}})  (\hat{\boldsymbol \mu}_v\cdot {\boldsymbol \lambda}_{{\boldsymbol k},{\boldsymbol \xi}})\sin({ k_y} (\hat{\bf y}\cdot {\bf R}_{u}))\sin({ k_y} (\hat{\bf y}\cdot {\bf R}_{v}))\nonumber~~,
\end{align}



where the last two lines describe the dipole self-energy term (DSE), and $\langle u, v\rangle$ implies the summation over the nearest neighbors within the same layer. %As a convenient illustration, consider a toy multilayered 2D material (a 2D Kronig-Penny type model~\cite{Hu2013N}, see details in the supporting information) described by a periodic harmonic potential as schematically illustrated in Fig.~\ref{3d}b. 
Here we have assumed that the transition dipole is aligned in the $\hat{\bf z}$ direction such that $(\hat{\boldsymbol \mu}_{u} \cdot  \hat{\boldsymbol e}_{{\boldsymbol k}, {s}}) =  \mu_0(\hat{d}_{u,z}^{\dagger} + \hat{d}_{u,z})$  and $(\hat{\boldsymbol \mu}_{u} \cdot  \hat{\boldsymbol e}_{{\boldsymbol k}, {p}})  = 0$.  With this simplification, we can ignore all the $p$ polarized cavity modes and write a simple dipole-gauge Hamiltonian.  We  drop the labels $s$ or $z$, such that $\hat{a}_{{\boldsymbol k}, s} \rightarrow \hat{a}_{{\boldsymbol k}}$   to reduce redundancy and write the following tight-binding (polarization-less) dipole gauge Hamiltonian 

%As a result, the total light-matter Hamiltonian is composed of two non-interacting $s$ and $p$ parts $\hat{H}_\mathrm{d \cdot E} = \mathcal{\hat{H}}_\mathrm{d \cdot E}^{s} + \mathcal{\hat{H}}_\mathrm{d \cdot E}^{p}$. In this work as we are interested in the polariton dispersion where typically~\cite{Qiu2021JPCL, Balasubrahmaniyam2021PRB,Xu2022,Michetti2005PRB,Tichauer2021JCP} $\theta ({\boldsymbol k}) \rightarrow 0$ and $\cos\theta({\boldsymbol k}) \rightarrow 1$ (as $k_y \gg k_x$) the polariton dispersion of $\mathcal{\hat{H}}_\mathrm{d \cdot E}^{s}$ and $\mathcal{\hat{H}}_\mathrm{d \cdot E}^{p}$ 
%are nearly the same. Thus in the following we only consider $\mathcal{\hat{H}}_\mathrm{d \cdot E}^{s}$ and  drop the labels $s$ or $z$, such that $\hat{a}_{{\boldsymbol k}, s} \rightarrow \hat{a}_{{\boldsymbol k}}$ and $c_{u,x}^{\dagger} \rightarrow c_{u}^{\dagger}$  to reduce redundancy and write the following tight-binding (polarization-less) dipole gauge Hamiltonian ($\mathcal{\hat{H}}_\mathrm{d \cdot E}^{s} \rightarrow \mathcal{\hat{H}}_\mathrm{d \cdot E}$)

%Further we consider  transition dipole 
% to be oriented in the $\hat{z}$ direction such that $(\hat{\boldsymbol \mu}_u \cdot  \hat{\boldsymbol e}_{{\boldsymbol k}, {s}}) =  \mu_0(c_{u}^{\dagger} + c_{u})$ and $(\hat{\boldsymbol \mu}_u \cdot  \hat{\boldsymbol e}_{{\boldsymbol k}, {p}}) =  0$ where $c_{u}^{\dagger}$ creates an electronic excitation at $u$th unit-cell. Since $p$ modes are decoupled we remove them  and we drop the label $s$, such that $\hat{a}_{{\boldsymbol k}, s} \rightarrow \hat{a}_{{\boldsymbol k}}$ for convenience. Using these simplifications, we obtain the following tight-binding dipole gauge Hamiltonian as


 \begin{align} \label{PF}
\mathcal{\hat{H}}_\mathrm{d \cdot E}  &= \sum_{{\boldsymbol k}} \hat{a}_{{\boldsymbol k}}^{\dagger}\hat{a}_{{\boldsymbol k}}\omega_c({\boldsymbol k}) + \omega_0 \sum_u \hat{d}_{u}^{\dagger}\hat{d}_{u}  + \sum_{\langle u , v\rangle} \tau_{uv} (\hat{d}_{u}^{\dagger}\hat{d}_{v} + \hat{d}_{u}\hat{d}_{v}^{\dagger})
 +\sum_{{\boldsymbol k}, u}  \mu_0 \lambda_{\boldsymbol k} (\hat{d}_{u}^{\dagger} + \hat{d}_{u})\big(e^{- i {  \boldsymbol k_\parallel} \cdot {\bf R}_{u}} \hat{a}_{{\boldsymbol k}}^\dagger +  e^{ i {\boldsymbol k_\parallel} \cdot {\bf R}_{u}} \hat{a}_{{\boldsymbol k}} \big) \sin({ k_y} (\hat{\bf y}\cdot {\bf R}_{u}))\nonumber\\
&~+  \sum_{u,v, {\boldsymbol k}}  {\frac{  \mu_0^2 \lambda_{\boldsymbol k}^2 }{{\omega_{\boldsymbol k}}}} (\hat{d}_{u}^{\dagger} + \hat{d}_{u}) (\hat{d}_{v}^{\dagger} + \hat{d}_{v})  e^{i {\bf k}_\parallel \cdot ({\bf R}_{v}- {\bf R}_{u})}  \sin({ k_y} (\hat{\bf y}\cdot {\bf R}_{u}))\sin({ k_y} (\hat{\bf y}\cdot {\bf R}_{v})).  
\end{align}
Note that the above Hamiltonian can be generalized to {\it ab initio} systems where there are multiple energetically relevant electronic transitions with transition dipole moments aligned along arbitrary directions such that both $s$ and $p$ polarization need to be taken into account.


%~\cite{Michalsky2016EPJAP} How $s$ and $p$ polarizations couple to matter will also lead to splitting between the cavity modes of different polarizations.
%Interestingly, when considering model systems with unit-cells that are symmetric, the total light-matter Hamiltonian can be written in two isolated block each containing either $s$ or $p$ polarized fields. To intuitively see this consider a single electron in a 3D Kronig-Penny type model with each unit-cell modeled as a particle in 3D finite box (with box length same in all three directions). In this scenario the first three localized excitations $\{|e_x\rangle,|e_y\rangle,|e_z\rangle \}$  within each unit-cell are degenerate but have transition dipole (from ground state) ${\boldsymbol \mu}_x  = \mu_0\hat{x}$, ${\boldsymbol \mu}_y  = \mu_0\hat{y}$ and ${\boldsymbol \mu}_z  = \mu_0\hat{z}$ along in three orthogonal directions. It is always possible define new excited states $|\tilde{e}_j\rangle = c_{jx}|e_x\rangle + c_{jy}|e_y\rangle + c_{jz}|e_z\rangle$ such that two of these state are coupled the cavity radiation ( with each coupling to  $s$ or $p$ polarized field) while one state will be decoupled from cavity radiation. Such decoupling of course will strongly dependent on the material under consideration and we plan to extend the present formalism toward {\it ab-initio} in future. That said, the results and conclusions drawn in this work are general and will qualitatively remain the same for arbitrary layered materials. 

We can further apply additional approximations to simplify the above Hamiltonian. First, we perform the rotating wave approximation (dropping   $\hat{a}_{\boldsymbol k}^{\dagger}\hat{d}_{u}^{\dagger}$ and $\hat{a}_{\boldsymbol k} \hat{d}_{u} $) and drop  the dipole self-energy term to obtain the following generalized Tavis-Cummings Hamiltonian~\cite{Keeling2020ARPC,Mandal2022CR} (GTC)  as
\begin{align}
&\mathcal{\hat{H}}_\mathrm{GTC} = \sum_{{\boldsymbol k}} \hat{a}_{\boldsymbol k}^{\dagger}\hat{a}_{\boldsymbol k}\omega_c({\boldsymbol k}) + \omega_0 \sum_u \hat{d}_{u}^{\dagger}\hat{c}_{u}  
+ \sum_{\langle u , v\rangle} \tau_{uv} (\hat{d}_{u}^{\dagger}\hat{d}_{v} + \hat{d}_{u}\hat{d}_{v}^{\dagger})  + \sum_{{\boldsymbol k}, u}  \mu_0 \lambda_{\boldsymbol k}  \big(e^{- i { \boldsymbol k_\parallel} \cdot {\bf R}_{u}} \hat{a}_{{\boldsymbol k}}^\dagger \hat{d}_{u} +  e^{ i {\boldsymbol  k_\parallel} \cdot {\bf R}_{u}} \hat{a}_{{\boldsymbol k}}\hat{d}_{u}^{\dagger} \big)\sin({ k_y}  (\hat{\bf y}\cdot {\bf R}_{u})). 
 \label{TC}
\end{align}
We investigate the validity of the approximations used to obtain the generalized Tavis-Cummings Hamiltonian below in Sec.~\ref{Sec:Mode-Trunc}. Importantly, we show that the $\mathcal{\hat{H}}_\mathrm{GTC}$ remains valid only when considering a few energetically relevant cavity modes.
% which is in agreement with recent studies in the single molecule under long-wavelength limit.~\cite{Taylor2022OL}  
 \begin{figure*}[!t]
\centering
\includegraphics[width=0.7\linewidth]{Convergence2.pdf}
\caption{\footnotesize (a) Schematic illustration of a single layer material coupled to cavity radiation.  (b) Exact band structure (dark blue) computed with the full dipole gauge Hamiltonian $\hat{\mathcal{H}}_\mathrm{d.E}$ in Eqn.~\ref{PF} compared with the approximate band  structure computed using $\hat{\mathcal{H}}_\mathrm{GTC}$ considering different number of cavity mode branches. (c) Same as (b) but plotting the polariton bands at $k_x = 0$ as a function of light-matter coupling. (d) Polariton bands computed from  $\hat{H}_\mathrm{d.E}$  (exact) compared with the case when only including the dipole-self energy terms of 4 off-resonant cavity modes (total of five cavity modes) and when only including the coupling term between these off-resonant modes and the matter. Here we used $L_y = 20000$ a.u.  and the refractive index $\eta = 1$.}
\label{fig0}
\end{figure*} 
 
 
\subsection{\normalsize d.E Hamiltonian at $k_z = 0$}
We place the localized excitations in a grid with positions ${\bf R}_{n,m,l} = X_{n} \hat{\bf x} + Y_{m} \hat{\bf y} + Z_l \hat{\bf z}$. To simplify our considerations, we will only consider nearest-neighbor coupling, such that $\hat{d}_{n,m,l}^{\dagger}$ couples to $\hat{d}_{n\pm 1,m,l}$,  $\hat{d}_{n,m\pm 1,l}$ and $\hat{d}_{n,m,l\pm 1}$ through $\tau_x$,  $\tau_y$ and $\tau_z$, respectively. Since here we focus on multi-layer materials where the inter-layer coupling is weak or negligible, we set $\tau_y = 0$.  

Next, we consider periodic boundary conditions along $\hat{x}$ and $\hat{z}$ directions, such that   ${\bf R}_{N+1,m,l} \equiv {\bf R}_{1,m,l}$ and  ${\bf R}_{n,m,N+1} \equiv {\bf R}_{n,m,1}$. Due to the periodic boundaries in the $\hat{x}$ and $\hat{z}$ directions, 
 the cavity modes ${\boldsymbol{k}} = (k_x, k_y, k_z)$ are quantized such that $k_x = \frac{2\pi}{L_x}n_x$  with  $n_x = 0, \pm 1, \pm  2, ..., \pm n_{x_\text{max}}$ and $k_z = \frac{2\pi}{L_z}n_z$ with $n_z = 0, \pm 1, \pm  2, ..., \pm n_{z_\text{max}}$.~\cite{Tichauer2021JCP} Here $L_x$ and $L_z$ are length of the periodic super-cell along $x$ and $z$ direction respectively. The unit-cell along the $\hat{\bf x}$ and $\hat{\bf z}$  direction are separated by $\delta l_x = \delta l_z$ such that $X_{n} = n\cdot \delta l_x$ with $n = 1, 2, ..., N_x$ with a similar expression for $Z_{l}$. %Note that the values of $Y_{m}$ depends on the structure of the material 

In most experiments, including ours, the polariton dispersion is plotted along $k_x$ while $k_z$ is set to 0. To obtain this dispersion, we define $\hat{d}_{k_x, m, k_z} = \frac{1}{\sqrt{N_x}} \sum_{n} \hat{d}_{n, m, k_z}e^{-i k_z  X_{n}}    = \frac{1}{\sqrt{N_x N_z}}\sum_{n, l}   e^{-i (k_x  X_{n} + k_z Z_{l})} \hat{d}_{n,m,l}$ and use it to transform the d.E Hamiltonian such that  it is block diagonalized into $|k_z|$ subspaces. Below we write the $k_z = 0$ block (which is decoupled from $k_z \ne 0$) of the d.E Hamiltonian as
\begin{align} \label{exact}
\mathcal{\hat{H}}_\mathrm{d \cdot E}(k_z=0) &= \sum_{{ k_x, k_y}} \hat{a}_{\boldsymbol k}^{\dagger}\hat{a}_{\boldsymbol k}\omega_c({\boldsymbol k}) +  \sum_{k_x,m} \epsilon(k_x) \cdot \hat{d}^{\dagger}_{k_x,m}\hat{d}_{k_x,m}  +  \sum_{{k_x, k_y}, m} g_{\boldsymbol k}   \big(  d_{k_x,m} \hat{a}_{{\boldsymbol k}}^\dagger +     d_{k_x,m}^\dagger\hat{a}_{{\boldsymbol k}} + d_{-k_x,m}^\dagger \hat{a}_{{\boldsymbol k}}^\dagger +    d_{ -k_x, m}\hat{a}_{{\boldsymbol k}}\big) \sin({ k_y} Y_m) \nonumber\\
%&~+   \sum_{{k_x, k_y}, m} g_{\boldsymbol k}   \big(  d_{-k_x,m}^\dagger \hat{a}_{{\boldsymbol k}}^\dagger +    d_{ -k_x, m}\hat{a}_{{\boldsymbol k}} \big) \sin({ k_y}  Y_m)\nonumber\\
&~+  \sum_{m,m'}\sum_{k_x, k_y}  {\frac{ g_{\boldsymbol k}^2}{{\omega_{\boldsymbol k}}}} (d_{k_x,m}^{\dagger}  + d_{-k_x, m})( d_{ -k_x, m'}^{\dagger}+ d_{k_x, m'}  )        \sin({ k_y}  Y_{m'})\sin({ k_y} Y_m),
\end{align}
where   we have simplified the indexes such that  $\hat{d}_{k_x,m, k_z = 0} \equiv \hat{d}_{k_x,m}$, we have defined $g_{\boldsymbol k} = \sqrt{N_x N_z}\mu_0 \lambda_{\boldsymbol k} $ and
 $\epsilon(k_x) = \omega_{0} - 2\tau_z- 2\tau_x\cos(k_x  \delta l_x)$. The above Hamiltonian is block diagonalized into $|k_x|$ subspaces each containing $\{\hat{a}_{k_x,k_y}, \hat{a}_{-k_x,k_y}, \hat{d}_{k_x,m}, \hat{d}_{-k_x,m}\}$ and their Hermitian conjugates. A simplified form of the Hamiltonian, the generalized Tavis-Cummings Hamiltonian, is obtained by performing the RWA and dropping the dipole self-energy (DSE) term
\begin{align}\label{TC-k.x}\nonumber
\mathcal{\hat{H}}_\mathrm{GTC}(k_z=0)
&= \sum_{k_x} 
\Bigg( 
\sum_{k_y}\hat{a}_{\boldsymbol k}^{\dagger}\hat{a}_{\boldsymbol k}\omega_c({\boldsymbol k}) + \sum_{m}\epsilon(k_x)\hat{d}^{\dagger}_{k_x,m}\hat{d}_{k_x,m}  + \sum_{{k_y},m}g_{\boldsymbol k}   \big(  \hat{a}_{{\boldsymbol k}}^\dagger \hat{d}_{k_x,m} +    \hat{a}_{{\boldsymbol k}} \hat{d}_{ k_x,m}^{\dagger} \big)\sin({ k_y}  Y_{m})\Bigg) \\
&\equiv\sum_{k_x} \hat{h}_\mathrm{GTC}(k_x) ~~,
\end{align}
which is block-diagonal in each $k_x$ subspace containing only $\{\hat{a}_{k_x,k_y},   \hat{d}_{k_x,m}\}$
and their Hermitian conjugates with ${\hat{h}}_\mathrm{GTC}(k_x)$ as the Hamiltonian in the $k_x$ block.  In this work, we use  $\mathcal{\hat{H}}_\mathrm{GTC}(k_z = 0)$ due to its simple block-diagonalized structure, as opposed to $\mathcal{\hat{H}}_\mathrm{d.E}(k_z = 0)$, and develop simple matrix models to obtain the polariton dispersion for multi-layered material inside the cavity.  

\section{\normalsize Mode truncation and dipole self-energy}\label{Sec:Mode-Trunc}
Here we discuss how to deal with energetically off-resonant cavity modes in the GTC Hamiltonian  $\mathcal{\hat{H}}_\mathrm{GTC}$ and comment on the role of missing DSE terms.  
To assess the accuracy of  $\mathcal{\hat{H}}_\mathrm{GTC}(k_z = 0)$, we consider a single layer material placed inside an optical cavity as depicted in Fig.~\ref{fig0}a.  The exact polariton dispersion is computed by solving $\mathcal{\hat{H}}_\mathrm{d.E}(k_z = 0)$ given in Eqn.~\ref{exact}, and is shown as dark blue solid lines with filled circles in Fig.~\ref{fig0}b-d. 
These exact results are computed by converging with respect to the number of cavity mode along $k_y$ and the number of Fock states for each of these modes. In Fig.~\ref{fig0} we use $\tau_x = 150$ cm$^{-1}$ and $\omega_0 - 2\tau_z = 0.62$ eV. %Note that at the coupling used in  exact polariton dispersion considering only a single mode in $\mathcal{\hat{H}}_\mathrm{d.E}$ provides nearly the same result compared to when considering many modes in $\mathcal{\hat{H}}_\mathrm{d.E}$. 

%When performing cavity mode truncation, we  restricting the sum   $\sum_{k_y} \rightarrow \sum_{k_y \le k_{y_\text{max}}}$, even for the DSE term (which ensures gauge invariance~\cite{Taylor2022OL,Taylor2020PRL}), despite the DSE being purely a matter operator. 

When using the $\mathcal{\hat{H}}_\mathrm{d.E}(k_z = 0)$,  our result shows that the energetically off-resonant cavity modes do not contribute to the low-energy polaritonic subspace, as one expects.  
However, counterintuitively, the contributions to the DSE term these off-resonant modes and the associated light-matter coupling term are non-negligible. DSE makes a positive contribution, while the coupling term makes a negative contribution. Thus, when  performing mode truncation by removing cavity modes with $k_y> k_{y_\mathrm{max}}$  that are energetically off-resonant in $\mathcal{\hat{H}}_\mathrm{d.E}(k_z = 0)$, both the corresponding coupling terms as well as the associated  DSE terms must be dropped. This means we must restrict the sum $\sum_{k_y} \rightarrow \sum_{k_y \le k_{y_\text{max}}}$ for both the coupling terms as well as the DSE terms despite the DSE being purely a matter operator (which ensures gauge invariance~\cite{Taylor2022OL,Taylor2020PRL}). 

Therefore, when considering $\mathcal{\hat{H}}_\mathrm{GTC}(k_z = 0)$ which ignores the dipole self-energy but retains the light-matter coupling term partially (since we employ the RWA), this balance is broken. As a result, a very undesirable feature of $\mathcal{\hat{H}}_\mathrm{GTC}(k_z = 0)$ (and thus the $\mathcal{\hat{H}}_\mathrm{GTC}$) is that the polariton eigenspectrum (or polariton dispersion) computed with it does not converge with respect to the number of cavity mode $k_y$. This is shown in Fig.~\ref{fig0}(b)-(d). 

In Fig.~\ref{fig0}(b), we present the polariton band structure computed exactly using direct diagonalization of $\mathcal{\hat{H}}_\mathrm{d.E}(k_z = 0)$ compared with the approximate $\mathcal{\hat{H}}_\mathrm{GTC}(k_z = 0)$ using different numbers of $k_y$ cavity modes. Here we choose a light-matter coupling   $g_c = 25$ meV where $g_\mathrm{k} = \sqrt{\frac{\omega_c({k_x,k_y})} {\omega_c(0,\pi/L_y)}}g_c$ (note that $k_z = 0$ and $\omega_c({k_x,k_y}) \equiv \omega_c({k_x,k_y, 0}) = c \sqrt{k_x^2 + k_y^2}$). When considering only one mode $\mathcal{\hat{H}}_\mathrm{GTC}(k_z = 0)$ provides visually indistinguishable  results compared to the exact benchmark. However, as number of $k_y$ cavity modes are increased  in  $\mathcal{\hat{H}}_\mathrm{GTC}(k_z = 0)$ the cavity dispersion starts to diverge. The same is observed in  Fig.~\ref{fig0}(c), where we compute the polariton bands at $k_x = 0$ as a function of the coupling $g_c$. This result corroborates previous work in the single-molecule cavity setup~\cite{Taylor2022OL} and in circuit QED~\cite{Malekakhlagh2017PRL} under the long-length approximation. Note that here, importantly, in contrast to these other works, we only make the long wavelength approximation on the atomic scale, not on the scale of the entire materials.

To illustrate the cancellation of the DSE and coupling terms for the off-resonant cavity modes, in  Fig.~\ref{fig0}(d), we considered eight cavity mode branches $k_y \in \{\frac{\pi} {L_y},\frac{2\pi} {L_y}, ..., \frac{8\pi} {L_y}\}$ where $k_y = \frac{\pi} {L_y}$ mode is resonant with matter excitations at $k_x = 0$ (the rest of the modes can be regarded as off-resonant) with $g_c = 30$ meV. We compare the full dipole gauge Hamiltonian (i.e. `exact') with an approximation where we remove all the dipole self-energy (DSE) terms associated with off-resonant cavity modes $k_y > \frac{\pi} {L_y}$ while retaining the coupling terms between matter and off-resonant cavity modes (cyan dashed lines). Finally, we compute the dispersion when including the DSE terms associated with the off-resonant cavity modes ($k_y > \frac{\pi} {L_y}$), while removing the coupling terms between matter and off-resonant cavity modes (red solid lines). As explained before, it can be seen that the DSE terms (coming from the off-resonant modes) contribute a positive amount while the coupling terms contribute a negative amount, and the magnitude of these two contributions is the same (up to the second order in light-matter coupling as shown in Ref.~\cite{Taylor2022OL}) but with opposite signs. When including both, their overall contribution (originating from the off-resonant modes) vanishes, and thus a single-mode GTC generates accurate results compared to exact. Therefore, when using the GTC model, we must restrict our consideration to a few energetically relevant cavity modes and, the (conveniently) drop all energetically off-resonant modes from the Hamiltonian. 
}


\section{\normalsize  Various cavity setups and corresponding matrix models}

\subsection{\normalsize Single Layer}
{\footnotesize 
First, we consider the simplest scenario, a single layer, such that the sum over $m$ and the index itself can be removed (such that $\hat{d}_{k_x, m} \rightarrow   \hat{d}_{k_x}$). The ${\hat{h}}_\mathrm{GTC}(k_x)$ Hamiltonian for a single layer is then written as,
\begin{align}
{\hat{h}}_\mathrm{GTC}(k_x) &= \sum_{{  k_y}} \hat{a}_{\boldsymbol k}^{\dagger}\hat{a}_{\boldsymbol k}\omega_c({\boldsymbol k})+  \epsilon(k_x)\hat{d}^{\dagger}_{k_x}\hat{d}_{k_x}     +   \sum_{k_y}{g_{\boldsymbol k}}    \big(  \hat{a}_{{\boldsymbol k}}^\dagger \hat{d}_{k_x} +    \hat{a}_{{\boldsymbol k}} \hat{d}_{ k_x}^{\dagger} \big)\sin({ k_y}  Y_{0}), 
\end{align}
where $\epsilon(k_x) = \omega_{0} - 2\tau_x\cos(k_x  \delta l_x)$, $Y_0$ is the location of the single layer along the $\hat{\bf y}$ direction inside the cavity. For this single layer placed between the two mirrors  $0 < Y_0 < L_y$, the ${\hat{h}}_\mathrm{GTC}(k_x)$ Hamiltonian can be written conveniently in the single excited subspace using the basis spanning $\{\hat{a}_{{k_x, k_y}}^\dagger| G,0\rangle,  \hat{d}_{k_x}^{^\dagger}|G,0\rangle \}$ where $|G,0\rangle$ represents the matter ground state with no photons in the cavity. 
For a given $k_x$, this is the basis of $N$ cavity modes to account for $k_y$ and one matter state for the exciton.
Therefore, ${\hat{h}}_\mathrm{GTC}(k_x)$ matrix  takes the form of the  widely used $(N+1)\times (N+1)$ matrix~\cite{ Dietrich2016SA, Coles2014APL, Michetti2005PRB, Richter2015APL,Georgiou2021JCP,Balasubrahmaniyam2021PRB,Orosz2011APE,Faure2009APL} 
\begin{align}\label{SL-Hij}
{\hat{h}}_\mathrm{N+1} = \begin{bmatrix}
\epsilon(k_x) &  {\Omega}(\frac{\pi}{L_y})  &  {\Omega}(\frac{2\pi}{L_y}) & \hdots \\
{\Omega}(\frac{\pi}{L_y})  & \omega_c({k_x, \frac{\pi}{L_y}}) & 0 & \hdots\\
{\Omega}(\frac{2\pi}{L_y}) & 0 & \omega_c({k_x, \frac{2\pi}{L_y}} )
& \hdots \\
\vdots & \vdots    & \vdots  & \ddots \end{bmatrix} ~,
\end{align}
 where $ {\Omega}({k_y})  =    g_{\boldsymbol k} \sin(k_y \cdot Y_0)$ and $\epsilon(k_x)  = \omega_{0} - 2\tau_x\cos(k_x  \delta l_x)$. Diagonalizing the above matrix provides the polariton dispersion along $k_x$ for a single layer coupled to cavity radiation modes.


}
\subsection{\normalsize Filled cavity with non-interacting layers}

{\footnotesize

Here, we consider a material filling the cavity ($l_y = L_y$). We assume that $m$th layer is centered at $Y_m = (m-\frac{1}{2})\delta L_y$ where $m = 1, 2, ... N_y$ and $k_y = n_y\frac{\pi}{L_y}$. For this setup, the collective matter excitation operators are found using the unitary matrix $[U_\mathrm{O}]_{m,n_y} = \sqrt{\frac{2}{N_y}}\sin (\frac{\pi}{N_y} n_y\cdot (m-\frac{1}{2}) ) $ for $n_y \ne N_y$
and  $[U_\mathrm{O}]_{m,N_y} = \sqrt{\frac{1}{N_y}}\sin (\frac{\pi}{N_y} N_y\cdot (m-\frac{1}{2}) )$ which is a the discrete sine transform matrix of type II. With the collective matter excitation operators $\hat{d}_{k_x,k_y} = \sum_{m} [U_\mathrm{O}]_{m,n_y} \hat{d}_{k_x,m}$ we obtain  ${\hat{h}}_\mathrm{GTC}(k_x)$. Using the single excited subspace spanning $\{\hat{a}_{{k_x, k_y}}^\dagger| G,0\rangle,  \hat{d}_{{k_x, k_y}}^{^\dagger}|G,0\rangle \}$ where $|G,0\rangle$ represents the matter ground state with no photons in the cavity, we find the following $2N\times 2N$ model 

% \begin{align}
% {\hat{h}}_{\mathrm{2N}}({k_x}) = 
% \begin{bmatrix}
% \epsilon(k_x) &  {\Omega}(\frac{\pi}{L_y})   & 0 & 0 & \hdots \\
% {\Omega}(\frac{\pi}{L_y})  & \omega_c({k_x, \frac{\pi}{L_y}}) & 0 &  0 & \hdots\\
% 0 & 0 & \epsilon(k_x) & {\Omega}(\frac{2\pi}{L_y})
% & \hdots \\
% 0   & 0 & {\Omega}(\frac{2\pi}{L_y}) &   \omega_c({k_x, \frac{2\pi}{L_y}})& \hdots\\
% \vdots & \vdots & \vdots   & \vdots  & \ddots \end{bmatrix}.
% \end{align}


\begin{align}\label{filled-noint}
{\hat{h}}_{\mathrm{2N}}({k_x}) = 
\left[\!\begin{array}{*{24}{c@{\hspace{4pt}}}}
\epsilon(k_x) &  {\Omega}(\frac{\pi}{L_y})   & 0 & 0 & \hdots \\
{\Omega}(\frac{\pi}{L_y})  & \omega_c({k_x, \frac{\pi}{L_y}}) & 0 &  0 & \hdots\\
0 & 0 & \epsilon(k_x) & {\Omega}(\frac{2\pi}{L_y})
& \hdots \\
0   & 0 & {\Omega}(\frac{2\pi}{L_y}) &   \omega_c({k_x, \frac{2\pi}{L_y}})& \hdots\\
\vdots & \vdots & \vdots   & \vdots  & \ddots \end{array}\!\right]
\end{align}
where $ {\Omega}(k_y)  =   \sqrt{\frac{N_y}{2}} g_{k_x, k_y} \propto  \sqrt{N_x N_y N_z}$ (note that $g_{\bf k} \propto \sqrt{N_x N_z}$) also  ${\Omega}_y  \propto  \sqrt{\omega_c (k_x, k_y)}$ and $\epsilon(k_x)  = \omega_{0} - 2\tau_x\cos(k_x  \delta l_x)$. Diagonalizing the above matrix provides the polariton dispersion along $k_x$ for a multi-layer (non-interacting) material filling up the entire cavity. The same results are obtained when considering interacting layers $\tau_y \ne 0$ (shown in the supporting information). This is because the same transformation also diagonalizes the inter-layer coupling term. 
 

\subsection{\normalsize Filled cavity with interacting layers}
Here we consider a filled cavity with interacting layers. For $\tau_y \ne 0$,  we focus on the light-matter coupling term (within a $k_x$ subspace),
\begin{align}
&\sum_{k_y,m}  {g_{\bf k}}  \mu_0(\hat{c}^{\dagger}_{k_x,m}\hat{a}_{\bf k}  + \hat{c}_{k_x,m}\hat{a}_{\bf k}^{\dagger} )  \sin\Big(y\frac{\pi}{N_y} \cdot \big(m-\frac{1}{2}\big)\Big) \nonumber\\
&\approx  \sum_{k_y \in \mathcal{K}_{c}}  {g_{\bf k}}  \mu_0   \sum_{m}(\hat{c}^{\dagger}_{k_x,m}\hat{a}_{\bf k}  + \hat{c}_{k_x,m}\hat{a}_{\bf k}^{\dagger} )    \Bigg[ \sin\Big(\frac{y\pi m}{N_y}     \Big)\cos( \frac{y\pi}{2N_y}) - \cos\Big(\frac{y\pi m}{N_y}    \Big)\sin\Big( \frac{y\pi}{2N_y}\Big)\Bigg] ~,\nonumber
\end{align}

where in the last-line we confined the sum  to the cavity modes that are energetically relevant, $\mathcal{K}_c = \{\frac{\pi n_\mathrm{min}}{L_y}, \frac{\pi (n_\mathrm{min} + 1)}{L_y}, ...,  \frac{\pi n_\mathrm{max}}{L_y} \}$. For $y\frac{\pi }{2N_y} = \frac{1}{2} k_y \cdot \delta l_y \rightarrow 0$ we have  $\sin\big( y\frac{\pi}{2N_y}\big) \rightarrow 0$ and $\cos\big( y\frac{\pi}{2N_y}\big) \rightarrow 1$, this is equivalently assuming that the wavelength of the relevant cavity modes are much longer than the size of the unit cells (thickness each matter layer). Additionally we assume that $N_y \gg 1$ such that $N_y \approx N_y + 1$ and we write
 \begin{align}
&\sum_{k_y,m}  {g_{\bf k}}  \mu_0  (\hat{c}^{\dagger}_{k_x,m}\hat{a}_{\bf k}  + \hat{c}_{k_x,m}\hat{a}_{\bf k}^{\dagger} )  \sin(k_y \cdot Y_m) \approx  \sum_{k_y \in \mathcal{K}_{c}}  \mu_0{g_{\bf k}}      \sum_{m}(\hat{c}^{\dagger}_{k_x,m}\hat{a}_{\bf k}  + \hat{c}_{k_x,m}\hat{a}_{\bf k}^{\dagger} )   \sin\Big(\frac{\pi y\cdot  m}{N_y+1}  \Big).
\end{align}
We use the discrete sine transform matrix of type I to construct the collective matter operator. Thus we use $[U_\mathrm{O}]_{m,y} = \sqrt{\frac{2}{N_y + 1}}\sin(\frac{y\pi m}{N_y+1})$ such that $\hat{c}_{k_x, k_y} = \sum_{m} [U_\mathrm{O}]_{m,y} \hat{c}_{k_x, m}$. Note that the inter-layer coupling term  is simultaneously diagonalized using these collective matter operators. Thus for cavity filled with multi-layered (with interacting layers) material has the following form of  $\mathcal{\hat{H}}_{k_x} $,
\begin{align} 
 \mathcal{\hat{H}}^{k_x}_\mathrm{GTC}  &= \sum_{ k_y  } \hat{a}_{\boldsymbol k}^{\dagger}\hat{a}_{\boldsymbol k}\omega_c({\boldsymbol k})  +  \Big(\omega_{0} - 2\tau_x\cos(k_x  \delta l_x) - 2\tau_y\cos(k_y  \delta l_y)\Big)\hat{c}^{\dagger}_{k_x, k_y}\hat{c}_{k_x, k_y}     + \mu_0\sqrt{ \frac{N_y + 1} {2}}\sum_{k_y \in \mathcal{K}_c}{g_{\boldsymbol k}}   \big(  \hat{a}_{k_x, k_y}^\dagger c_{ k_x, k_y} +    \hat{a}_{{k_x, k_y}}c_{ k_x, k_y}^{\dagger} \big). \nonumber
  \end{align}
  
 Using the single excited subspace spanning $\{\hat{a}_{{k_x, k_y}}^\dagger| \bar{0}\rangle,  \hat{c}_{{k_x, k_y}}^{^\dagger}|\bar{0}\rangle \}$ gives us the following Hamiltonian matrix (a $2N\times 2N$ model ) that can be diagonalized to obtain the polariton dispersion along $k_x$  
 
\begin{align}\label{filled-int}
 \mathcal{\hat{H}}_{\mathrm{2N}}({k_x}) = \begin{bmatrix}
\epsilon(k_x, \frac{\pi}{L_y}) &  {\Omega}_1    & 0 & 0 & \hdots \\
{\Omega}_1   & \omega_c({k_x, \frac{\pi}{L_y}}) & 0 &  0 & \hdots\\
0 & 0 & \epsilon(k_x, \frac{2\pi}{L_y}) & {\Omega}_2
& \hdots \\
0   & 0 & {\Omega}_2 &   \omega_c({k_x, \frac{2\pi}{L_y}})& \hdots\\
\vdots & \vdots & \vdots   & \vdots  & \ddots \end{bmatrix},
\end{align}
 where $ {\Omega}_y  =  \mu_0\sqrt{\frac{ N_y + 1}{2}} g_{k_x, k_y} \propto \sqrt{\omega_c (k_x, k_y)}$ and $\epsilon(k_x, k_y)  = \omega_{0} - 2\tau_x\cos(k_x  \delta l_x)-  2\tau_y\cos(k_y  \delta l_y)$. 



 
\subsection{\normalsize  Partially filled cavity with thin material placed next to a mirror}
Here we consider a material thickness $l_y \ll L_y$ and material is placed near one of the cavity mirrors. The $m$th layer of the material is placed at $Y_m = (m-\frac{1}{2})\delta L_y$ where $m = 1, 2, ...  N_y$ and for $l_y \ll L_y$ we approximate $\sin( k_y \cdot Y_m) \approx k_y \cdot Y_m$. With this the light-matter coupling term (in $k_x$ subspace)   Eqn.~\ref{TC-k.x} is written as
 \begin{align}
&\sum_{k_y,m}  {g_{\bf k}} (\hat{d}^{\dagger}_{k_x,m}\hat{a}_{\bf k}  + \hat{d}_{k_x,m}\hat{a}_{\bf k}^{\dagger} )  \sin(k_y \cdot Y_m) \approx  \sum_{k_y \in \mathcal{K}_{c}}   {g_{\bf k}}      k_y \sum_{m}(\hat{d}^{\dagger}_{k_x,m}\hat{a}_{\bf k}  + \hat{d}_{k_x,m}\hat{a}_{\bf k}^{\dagger} )   Y_m,
\end{align}
where in the second line we also only consider energetically relevant cavity mode subspace $\mathcal{K}_\mathrm{c}$ as before. We define $[U_\mathrm{O}]_{m,n_\mathrm{min}} = {\frac{1}{\sqrt{\mathcal{N}_B}}}   Y_m$ where $ \mathcal{N}_B = {\frac{N_y(N_y + 1 ) (2N_y +1) }{6}} - \frac{3}{4}N_y$ is a normalization factor. Following the same strategy as before we define a collective $\hat{d}_{k_x, B} = {\frac{1}{\sqrt{\mathcal{N}_B}}}   \sum_{m}Y_m \hat{d}_{k_x,m}$ and the other $N-1$ dark matter excitation operators $ \hat{d}_{k_x, D_y} = \sum_{m}[U_\mathrm{O}]_{m,n_y\ne n_\mathrm{min}} \hat{d}_{k_x,m}$. Since the dark operators as $\{\hat{d}_{k_x, D_y}\}$ are decoupled from the bright matter operator as well as photonic operators, we ignore them and write the following Hamiltonian, 

\begin{align}
{\hat{h}}_\mathrm{GTC}(k_x) &= \sum_{{  k_y  }} \hat{a}_{\boldsymbol k}^{\dagger}\hat{a}_{\boldsymbol k}\omega_c({\boldsymbol k})+  \epsilon(k_x)\hat{d}^{\dagger}_{k_x, B}\hat{d}_{k_x, B}    +  \sum_{k_y \in \mathcal{K}_c} k_y {g_{\boldsymbol k}} \sqrt{\mathcal{N}_B}   \big(  \hat{a}_{k_x, k_y}^\dagger \hat{d}_{ k_x, B} +    \hat{a}_{{k_x, k_y}}\hat{d}_{ k_x, B}^{\dagger} \big)  ~~.  
\end{align}
 Using  the single excited subspace spanning $\{\hat{a}_{{k_x, k_y}}^\dagger| G,0\rangle,  \hat{d}_{{k_x, k_y}}^{^\dagger}|G,0\rangle \}$, we obtain the following $(N+1)\times(N+1)$ model

 
 
\begin{align}\label{ML-matrix-bottom}
{\hat{h}}_{\mathrm{N+1}}({k_x})= \begin{bmatrix}
\epsilon(k_x ) &  {\Omega} (\frac{\pi}{L_y})    & {\Omega}(\frac{2\pi}{L_y})    & \hdots \\
{\Omega}(\frac{\pi}{L_y})    & \omega_c({k_x, \frac{\pi}{L_y}}) &  0 & \hdots\\
{\Omega}(\frac{2\pi}{L_y})  & 0 & \omega_c({k_x, \frac{2\pi}{L_y}})  
& \hdots \\
\vdots   & \vdots   & \vdots  & \ddots \end{bmatrix},
\end{align}
 where $ {\Omega}(k_y) =    k_y \sqrt{\mathcal{N}_B} g_{k_x, k_y} \propto k_y\cdot \sqrt{\omega_c (k_x, k_y)}$ and $\epsilon(k_x )  = \omega_{0} - 2\tau_x\cos(k_x  \delta l_x)$. 
    


\subsection{\normalsize Partially filled cavity with thin material placed at the middle}
Similar to the previous section, here we also consider $l_y \ll L_y$  but we place the material in the center of the cavity. Thus   $Y_m = m \cdot \delta l_y + L_y/2 - l_y/2$. With a geometry where $\frac{2\pi n_\mathrm{min}}{k_y} \gg l_y$, we can approximate $\sin(k_y \cdot Y_m) \approx (-1)^{\frac{n_y-1}{2}}$ for odd $n_y = 1, 3, ...$ and $\sin(k_y \cdot Y_m) \approx k_y \cdot (Y_m - L_y/2)$  for even $n_y = 2, 4, ...$ and we introduce a unitary transformation matrix $U_{\mathrm{O}}$ that is of $N_y\times2$ dimensions. We set  $[U_{\mathrm{O}}]_{m,n_\mathrm{min}} = 1/\sqrt{N_y}$ and $[U_{\mathrm{O}}]_{m,n_\mathrm{min} +1 } = (Y_m - L_y/2)/\sqrt{\sum_j (Y_m - L_y/2)^2} = (Y_m - L_y/2)/\sqrt{\mathcal{N}}$ for which we have $\sum_m [U_{\mathrm{O}}]_{m, n_\mathrm{min}} \cdot [U_{\mathrm{O}}]_{m,n_\mathrm{min} + 1} = 0$. With this transformation, we define the symmetric and asymmetric bright  excitonic operators as,
 
 \begin{align}
 \hat{d}^{\dagger}_{k_x, \mathrm{S}} &= \frac{1}{\sqrt{N_y}}  \sum_m  \hat{d}^{\dagger}_{k_x, m},  \\
 \hat{d}^{\dagger}_{k_x, \mathrm{A}} &= \frac{1}{\sqrt{\mathcal{N}}}  \sum_m    \Big(Y_m - \frac{L_y}{2}\Big) \hat{d}^{\dagger}_{k_x, m} ~~.
 \end{align}

Using the $\hat{d}^{\dagger}_{k_x, \mathrm{S}}$ and $\hat{d}^{\dagger}_{k_x, \mathrm{A}}$,  we can write ${\hat{h}}_\mathrm{GTC}(k_x) = {\hat{h}}_{\mathrm{S}}({k_x})  + {\hat{h}}_{\mathrm{A}}({k_x})$ as 
 
\begin{align} 
{\hat{h}}_{\mathrm{S}}(k_x)  &= \sum_{{  k_y \in \mathrm{odd} }} \hat{a}_{\boldsymbol k}^{\dagger}\hat{a}_{\boldsymbol k}\omega_c({\boldsymbol k}) +  \epsilon(k_x)\hat{d}^{\dagger}_{k_x, \mathrm{S}}\hat{d}_{k_x, \mathrm{S}}     + \sqrt{N_y}\sum_{k_y \in \mathrm{odd}}   (-1)^{\frac{n_y-1}{2}} {g_{\boldsymbol k}}    \big(  \hat{a}_{k_x, k_y}^\dagger \hat{d}_{ k_x, \mathrm{S}} +    \hat{a}_{{k_x, k_y}}\hat{d}_{ k_x, \mathrm{S}}^{\dagger} \big)~~,  \\
{\hat{h}}_{\mathrm{A}} ({k_x})  &= \sum_{{  k_y \in \mathrm{even} }} \hat{a}_{\boldsymbol k}^{\dagger}\hat{a}_{\boldsymbol k}\omega_c({\boldsymbol k})  \epsilon(k_x)\hat{d}^{\dagger}_{k_x, \mathrm{A}}\hat{d}_{k_x, \mathrm{A}}  +  \sqrt{\mathcal{N}}\sum_{k_y \in \mathrm{even}} k_y {g_{\boldsymbol k}}   \big(  \hat{a}_{k_x, k_y}^\dagger \hat{d}_{ k_x, \mathrm{A}} +    \hat{a}_{{k_x, k_y}}\hat{d}_{ k_x, \mathrm{A}}^{\dagger} \big)~~.  
\end{align}

The Hamiltonian matrix using a single excited subspace spanning $\{\hat{a}_{{k_x, k_y}}^\dagger| G,0\rangle,  \hat{d}_{{k_x, k_y}}^{^\dagger}|G,0\rangle \}$ is given by a $(N+2)\times (N+2)$ model 
\begin{equation}\label{ML-middle}
    {\hat{h}}_{\mathrm{N+2}}({k_x}) = {\hat{h}}_{\mathrm{S}}({k_x}) \oplus {\hat{h}}_{\mathrm{A}}({k_x})
\end{equation}
 where ${\hat{h}}_{\mathrm{S}}({k_x})$ and ${\hat{h}}_{\mathrm{A}}({k_x})$ are given as
 
\begin{align}\label{ML-middle-sym}
{\hat{h}}_{\mathrm{S}}({k_x}) =
 \begin{bmatrix}
\epsilon(k_x ) &  {\Omega}(\frac{\pi}{L_y})    & {\Omega}(\frac{3\pi}{L_y}) & \hdots \\
{\Omega}(\frac{\pi}{L_y})  & \omega_c({k_x, \frac{\pi}{L_y}}) & 0 &  \hdots  \\
{\Omega}(\frac{3\pi}{L_y})   & 0 & \omega_c({k_x, \frac{3\pi}{L_y}})&  \hdots   \\
\vdots   & \vdots & \vdots &   \ddots  \end{bmatrix} ~~,
\end{align}

\begin{align} 
{\hat{h}}_{\mathrm{A}}({k_x}) =
 \begin{bmatrix}
\epsilon(k_x ) &  {\Omega}(\frac{2\pi}{L_y})  & {\Omega}(\frac{4\pi}{L_y}) & \hdots \\
{\Omega}(\frac{2\pi}{L_y})  & \omega_c({k_x, \frac{2\pi}{L_y}}) & 0 &  \hdots  \\
{\Omega}(\frac{4\pi}{L_y})  & 0 & \omega_c({k_x, \frac{4\pi}{L_y}})&  \hdots   \\
\vdots   & \vdots & \vdots &   \ddots  \end{bmatrix}~~, 
\end{align}

 %\end{strip}
 where $ {\Omega}(k_y)  =     \sqrt{  {N}_y} g_{k_x, k_y} (-1)^{\frac{n_y-1}{2}}  \propto   \sqrt{\omega_c (k_x, k_y)}$ for odd $n_y$, ${\Omega}(k_y)  =  \mu_0  \sqrt{ \mathcal{N}} g_{k_x, k_y}  k_y  \propto   k_y \cdot  \sqrt{\omega_c (k_x, k_y)}$ for even $n_y$ and $\epsilon(k_x )  = \omega_{0} - 2\tau_x\cos(k_x  \delta l_x)$. Note that here the Hamiltonian matrix take the form of a $(N+2)\times (N+2)$ matrix which when only considering two cavity mode branches appear as an $2N\times 2N$ matrix. 



}





\section{\normalsize  Experimental Details }

 \subsection{\normalsize CH$_3$(CH$_2$)$_3$NH$_3$)$_2$(CH$_3$NH$_3$)$_2$Pb$_3$I$_{10}$  crystal synthesis}.  {\footnotesize
 
 The synthesis followed an aqueous solution crystallization procedure. Lead (II) iodide (PbI2, 99.999\% trace metals bases) powder (2720 mg, 5.9 mmol) was dissolved in   hydriodic acid solution (HI, 9 mL, 57\% wt.) with 1 mL hypophosphorous acid (H3PO2, 50\% wt.) as the reducing agent. The solution was heated to 105$^o$ C under constant magnetic stirring. Addition of methylammonium iodide (MAI, $\ge$ 99\%, anhydrous 636 mg, 4.0 mmol) and n-butylammonium iodide (BAI, 382 mg, 1.9 mmol) produced black precipitation instantly and redissolved in 5 mins. The solution was subjected to a controlled cooling rate of 0.5 $^\mathrm{o}\mathrm{C}$/h to room temperature in an oil bath. The crystals were collected by vacuum filtering and washed twice with toluene. All chemicals were purchased from Sigma-Aldrich.

 \subsection{\normalsize  Perovskites cavities fabrication.}  The three cavity samples in the main text all consist of exfoliated single crystals of CH$_3$(CH$_2$)$_3$NH$_3$)$_2$(CH$_3$NH$_3$)$_2$Pb$_3$I$_{10}$ placed between a pair of metallic mirrors. The bottom mirror, a 30 nm gold film, was deposited on cover glass (Richard-Allan Scientific, 24×50 \#1.5) by electron-beam evaporation (Angstrom EvoVac deposition system) with a deposition rate of 0.05 nm/s. This thin glass/Au mirror allows partial light transmission for spectroscopic interrogation. The top mirror was a 200 nm silver deposited by the same method. 
For the partially filled cavity of Figure 4b, thin perovskite flakes were mechanically exfoliated with PVC tape (Nitto SPV224 PVC Vinyl Surface Protection Specialty Tape) and transferred onto the bottom Au mirror. A 1040 nm thick layer of PMMA [Poly(methyl methacrylate), 950 MW, Kayaku Advanced Materials] was then deposited on the perovskite through spin-coating at 1000 rpm. The structure is then baked at 170°C for 5 mins to homogenize the interface, before depositing the top silver mirror. 

For the partially filled cavity of Figure 4c, a 515 nm PMMA spacer was spin-coated at 3000 rpm on the bottom Au mirror. The structure is baked at 170°C for 5 mins before exfoliating perovskite single crystals on top of the PMMA. A second PMMA spacer was subsequently spin-coated (3000 rpm, baked at 130°C for 2 mins) on the perovskite flakes, followed by deposition of the top silver mirror.

For the filled cavity sample of Figure 4a, the perovskite flakes were exfoliated onto the bottom Au mirror with PVC tape. The top silver mirror was evaporated on a Si substrate, then template-stripped and deposited on the Au/perovskite structure using thermal release tape.

 
 
 }


\bibliographystyle{naturemag}
\bibliography{bib.bib}
\end{document}