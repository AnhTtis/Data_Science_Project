\section{Discussion}
%
% Note:
% Summary of the study
% Related studies including Relation to Nadine: leading & following
% Future studies
% Current study is limited because the experimenter as the main author playes the role of the counterpart of the robot. Future study should examine the case of more spontaneous kinaethetic human-robot interactions by employing necessary number of human subjects wherein we may investigate possible mechanism of various social cognition including turn-taking and ...(See my future study plan used in the unit review.)
%
%  Revised by JT 2023-3-6 11am
The current study investigated human-robot bodily interactions via kinaesthesia using a PV-RNN model that was developed based on the free energy principle.
Bodily interactions between a human experimenter and a robot were conducted using Torobo, a humanoid robot equipped with a PV-RNN model that can sense excess torque exerted by a human counterpart.
We especially examined how the counter force between the robot and the human experimenter changed during movement pattern transitions physically guided by the human experimenter, depending on two different condition changes.

In experiment 1, we examined how the setting of a parameter called the meta-prior $w^i$, which regulates the KL-divergence between the approximate posterior distribution and the prior distribution in the interaction phase, affects the counter force generated in executed or attempted transitions.
Results of this experiment showed that in the case of a smaller $w^i$, while the KL-divergence between the approximate posterior and the prior distribution becomes larger, the prediction error (negative log-likelihood) becomes smaller.
Since the prediction error diminishes, the excess torque to counteract this error also decreases.
On the other hand, in the case of a larger $w^i$ setting, while the KL-divergence between the approximate posterior and the prior distribution becomes smaller, the prediction error becomes larger, which requires more excess torque for the transition.
The conflict that appeared between the movement intended by the robot and that executed by the experimenter is distributed to the prediction error and the KL-divergence between the approximate posterior and the prior in proportions determined by the meta-prior $w^i$.
With larger $w^i$ the top-down movement intention of the robot becomes stronger, which results in a stronger counter force, whereas the top-down intention as well as the counter force become weaker with smaller $w^i$.

The above is consistent with past research from our group \cite{wirkuttis2023turn}.
In \cite{wirkuttis2023turn}, Tani conducted simulation studies on synchronized imitative interaction by dyadic, vision-based robots using a PV-RNN model.
That study showed that a robot with smaller/larger $w^i$ tends to follow or lead the other robot set with a larger or smaller $w^i$ with weaker or stronger actional intention in synchronized imitative interaction.
Similarly in the current study, the experimenter easily led
the robot with a smaller $w^i$ with a smaller counter force because of the weaker top-down intention of the robot.
On the other hand, when $w^i$ is quite large, such as $>0.1$ for the robot, it was difficult for the experimenter to lead the robot because of the extremely strong counter force.
In this situation, the experimenter just followed movement patterns strongly led by the robot while grasping the robot's hands, as shown in the preliminary experiment described previously.

In experiment 2, we examined the difference in the counter force required for the experimenter to execute trained and untrained movement pattern transitions.
These experimental results showed that untrained transitions require more force since such transitions are accompanied by larger increases in free energy, the KL-divergence (between the approximate posterior and the prior), and the prediction error.
Trained transitions, on the other hand, require less force because of smaller increases in free energy, the KL-divergence, and the prediction error.

As already mentioned, there have been few studies of human-robot interactions based on the free energy principle.
Although \cite{chameICRA2020} showed that the setting of $w^t$ as meta-prior in the training phase could strongly affect characteristics of human-robot kinaesthetic interactions, the description of their experiment results did not include rigorous analysis with repeated experiments.
%the results of the study are considered to be only preliminary because of insufficient number of repeated experiments and lack of the analysis.
% On the other hand, the current study showed rigorous analysis on the dependence of the counter force on $w^i$ as well as on whether movement transitions are trained or untrained ones through repeated experiments using a real robot.

The major limitation of the present study is that presented human-robot interactions are not fully interactive since experimenter-induced sequences of movement pattern transitions were determined a priori.
In this regard, Ikegami and his colleagues \cite{ikegami2007turn, iizuka2004adaptability} investigated underlying mechanisms for turn-taking that were generated by spontaneous interaction between artificial agents as well as artificial agents and humans.
Recently, Masumori et al. \cite{Masumori2021} developed a humanoid robot platform, called Alter3, behaviour of which was controlled by sub-modules, including a self-simulator, an automatic mimicry unit, and memory storage, which were perturbed by a specific neurodynamic model for the purpose of conducting experiments on spontaneous human-robot interactions.
They showed that spontaneous turn-taking between imitator and imitated could be developed by autonomous switching of information flow between the two sides.

In future studies, we will undertake human-robot kinaesthetic interaction experiments that assume less a priori.
Such experiments should be done not with experimenters as counterparts of the robots (as in the present study) but by inviting an adequate number of human participants, since the human side also needs to be analyzed.
Such studies will focus on two research issues.
One is to investigate how spontaneous turn-taking can occur in imitative interaction based on kinaesthesis by using the active inference framework \cite{friston2010action, parr2019generalised, baltieri2019pid}.
Spontaneous turn-taking in this setting means that the role of the leader to initiate the next shared patterns switches autonomously between the two sides, such that sometimes the robot may push hard with its own intended patterns and the human counterpart may do so at other times.
This study may require development of an autonomous $w^i$ adaptation scheme, since if $w^i$ on the robot side can shift adaptively by sensing contextual flow in the interaction, the leader-follower relationship should shift accordingly.

The other focus is to investigate how novel movement patterns can be developed through repeated kinaesthetic interaction associated with continuous learning in both robots and human participants, based on the free energy principle.
One assumption is that novel patterns could develop in terms of false memory as the number of movement patterns memorized distributively in the PV-RNN model increases.
This phenomenon of false memory is due to potential non-linearity and stochasticity in the PV-RNN model.
A study on a deterministic RNN model demonstrated this property \cite{tani2004self}.
Novel patterns generated by robots could enhance improvisation of new pattern generation from human counterparts through iterative interaction.




\begin{comment}
%% Related studies by Hiroki
%%
Recently, the number of studies on the application of FEP has increased \citep{maselli2022active, pezzato2023active, millidge2022predictive, meo2022adaptation, taniguchi2023world}.
Pezzulo paper on social interaction.
\citet{maselli2022active} showed that the active inference model is able to characterise the movements generated by the agent's intention to resolve multi-sensory conflict or to achieve an external goal such as reaching its arm up to a certain point where the cognition of the agent against its arm is confused by a fake VR-arm.
\citet{baltieri2019pid} showed that PID controllers, which are used to explain biological systems, can fit a more general theory of life and cognition under the free energy principle by using generative models of the world.

\citet{tschantz2020scaling} presented a working implementation of active inference to reinforcement learning which demonstrated and efficient exploration and an order of magnitude different sample efficiency in a high-dimensional tasks such as MountainCar environment.


Also, \citet{pezzato2023active} showed that robotic tasks such as reactive action planning can be formulated as a free energy minimisation problem by introducing a hybrid combination of active inference and behaviour trees.
However, research investigating human-robot interaction using FEP remains sparse \citep{chameICRA2020, ohata2020investigation}.
Although \citet{chameICRA2020} focused on human-robot physical interaction investigating the relationship between cognitive compliance and sensitivity against observation by changing the meta-prior of learning phase $w^t$, its result lacks rigorous analysis and remains preliminary.
Current research is the next level of this, where we conducted a quantitative analysis of the relationship between cognitive compliance and the force exerted.
From our experimental result, one can claim that with a larger meta-prior, the robot performs as an agent with a stronger leading force, which coincides with the result of the leader-follower relationship developed through dyadic robot-robot interaction by \citet{wirkuttis2021leading}.

This research focused on human-robot kinaesthetic interaction, aiming to investigate the nature of interactions in which the intention of the robot and that of the human counterpart does not coincide.
Through human-robot interaction experiments, we investigated the relationship between the required amount of force exerted and the meta-prior.
The result of experiment-1 shows that the amount of force required for the robot to change its top-down intention is proportional to the meta-prior.
Experiment-2, on the other hand, shows that the amount of force required for the robot to perform unseen transitions among movement patterns is larger than that of seen transitions.
However, the current study is limited in the sense that the experimenter as the main author plays the role of the counterpart of the robot. 
Future studies should examine the case of more spontaneous kinaesthetic human-robot interactions by employing the necessary number of human subjects wherein we may investigate the possible mechanism of various social cognitive interactions.
Furthermore, it is interesting to study the development of spontaneous kinaesthetic human-robot collaborative interactions in specific tasks by further extending the PV-RNN framework.
Although various studies on human-robot collaborative interaction have been conducted \citet{sheridan1997eight, ajoudani2018progress, oguz2018hybrid}, the outcomes of these researches struggle to reconstruct a true human-human-like collaboration.
We will especially focus on concepts introduced by \citet{sheridan1997eight} for the human-robot collaborative interaction to be considered successful.
From these points of view, we aim to investigate how the robot can construct a predictive model of the human counterpart through a human-robot kinaesthetic interaction experiment.
\end{comment}







%% Original by Hiroki 2023-3-5
%%
\begin{comment}
The current study investigated how human-robot bodily interaction via kinaesthesia and the meta-prior during human-robot interaction relate.
In particular, we focused on two different perspectives, in Experiment-1, we investigated how the excess torque and meta-prior $w^i$ relate, whereas, in Experiment-2, we investigated how the exess torque and Trained/Untrained transitions (Experiment-2) are related.
In experiment-1, the robot required larger excess torque when configured with a larger meta-prior $w^i$ (see Fig.\ref{fig:TransitionForDifferentW}).
With a larger meta-prior, the top-down actional intention becomes stronger since the posterior distribution strictly follows the prior distribution which is representing the current movement intention.

As shown in the result of experiment-1, the behaviour of the network shows a significant difference between different meta-prior $w^i$ cases. 
When the experimenter attempts to induce a trained transition while Torobo is generating learnt movement patterns, the predicted joint angles and the observed ones start to diverge.
Free energy (which is the loss function of PV-RNN) begins rising at the same time, since it is expressed as a sum of the mean squared error between the predicted joint angles and the observed ones, and KL divergence between the approximate posterior and the prior (see Eq.\ref{eq:LossFunction_interaction}).
To minimise free energy, PV-RNN can either update the approximate posterior and adjust the predicted joint angle to minimise the mean squared error or keep the approximate posterior following the prior distribution to minimise the KL divergence.
The meta-prior $w^i$ is regulating the KL divergence, which balances between these two terms, the \emph{accuracy} and the \emph{complexity} when minimising free energy.
With the case of $w^i$ set to a smaller value, one can clearly see that the network immediately minimises the mean squared error by updating the approximate posterior which leads to an increase of KL-divergence (see Fig.\ref{fig:TimeDevelopmentDifferentW}a, $t_{c} = 404$ and $407$).
Whereas with the case of $w^i$ set to a larger value, the network keeps minimising the KL divergence by keeping the approximate posterior following the prior generation until the mean squared error becomes large enough (see Fig.\ref{fig:TimeDevelopmentDifferentW}c, $t_{c} = 1425$ and $1432$).

More interestingly, the required excess torque became larger when the meta-prior $w^i$ was set to a larger value (see Fig.\ref{fig:TransitionForDifferentW}a). 
When $w^i$ is set to a larger value, the error between the predicted joint angles and the observed ones becomes higher than that of the smaller $w^i$ setting.
This larger error is fed to the PID controller which results in the generation of a larger counter-force to the experimenter.
From the result, we show the relationship between the required force and the meta-prior $w^i$ that the larger $w^i$ leads the network to require a larger force when the interacting counterpart performs against the top-down intention of it.

In experiment-2, the robot required a larger excess torque while performing untrained movement pattern transitions compared to trained ones (see Fig.\ref{fig:TrainedVSUntrained}).
Also, the network showed a higher increase in both the mean squared error and KL divergence for untrained movement pattern transitions.
This indicates that the robot shows a higher resistance against an unseen situation where its free energy increases more than in a familiar situation.
% in a situation that it has encountered.
\end{comment}
%%%%%%%%%%%
%%%%%%%%%%%




\begin{comment}
------------------------Draft below this---------------------------

During the interaction phase of PV-RNN model, adaptive variables $\mathbf{A}_t^{\mu}, \mathbf{A}_t^{\sigma}$ are updated to minimise the free energy in Eq.\ref{eq:LossFunction}, which is the sum of mean squared error and KL divergence.
When the output $\Bar{\theta}$ and the observation $\theta$ begin to diverge due to external forces such as the experimenter holding Torobo's hands, the mean squared error increases, which leads to the rise of free energy.
PV-RNN model is capable of developing its output $\Bar{\theta}$ by updating adaptive variables $\mathbf{A}_t^{\mu}, \mathbf{A}_t^{\sigma}$.
However, updating the adaptive variables will lead to an increase in the KL divergence between the approximate posterior and prior, which also leads to an increase in free energy.
Hence, free energy minimisation in the interaction phase of the PV-RNN model during human-robot interaction is a balance of these two terms, which is regulated by the meta-prior $w^i$ (Eq.\ref{eq:LossFunction}).

\subsection{$w^i$ and the excess torque}
\end{comment}