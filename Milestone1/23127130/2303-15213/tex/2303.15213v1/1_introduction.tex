\section{Introduction}
Studies on social human-robot interaction have attracted considerable attention recently because of their practical applications, especially with using the linguistic modality \cite{guadarrama2013grounding, kanda2004interactive, wainer2014using}.
However, investigation of more direct interaction such as via kinaesthesia should be also indispensable when considering more embodied aspects or enactivism \cite{varela2017embodied} of human-robot interactions.
Although there have been a reasonable number of practical studies on human-robot interaction using kinaesthesia, which have contributed greatly to human-robot joint collaboration, human assistance, and user interface \cite{bahl2022human, kruger2009cooperation, peternel2017towards}, few studies have attempted to understand essential mechanisms underlying kinaesthetic interaction in light of embodied cognition, social cognition, and system-level neuroscience. 

In this regard, the current study investigated human-robot kinaesthetic interaction by applying system neuroscience theory, the free energy principle, proposed by Friston \cite{friston2005theory} which is consonant with enactivism \cite{ramstead2020free, bruineberg2018anticipating}.
Let us consider a situation in which a robot and a human dance, holding each other with both hands, executing memorized dance patterns.
If the robot initiates a particular pattern from memory with strong intention, the human counterpart might follow it without resisting because of the strong counter force.
On the other hand, if the robot generates a pattern without strong intention, the human counterpart might be able to shift to a different pattern without experiencing strong counter force.
Let us consider another situation. 
If the human counterpart attempts to induce a movement pattern that is familiar to the robot, such guidance should proceed easily without strong counter force, since the robot can infer the intended pattern immediately and can move as anticipated. On the other hand, if the human counterpart attempts to induce a movement pattern unfamiliar to the robot, such guidance should experience strong counter force, since the movement is neither inferential nor predictable for the robot.

We presume that these situations can be well explained by predictive coding \cite{friston2009predictive} and active inference \cite{friston2010action, friston2016active,parr2019generalised}, based on the free energy principle \cite{friston2005theory}.
Predictive coding provides a theory in which perception is achieved by inferring hidden causes for sensory observations that minimize the error between top-down prediction of sensation made by the generative model and the actual sensation.
On the other hand, active inference provides a theory for action generation in which an optimal action is inferred to minimize the error between the preferred sensation and the actual sensation resulting from the action.
These two are not separate, but integrated through a sensorimotor loop in embodied cognition in which a perceptual inference and action generation can be achieved simultaneously by minimising errors through iterative interaction between the top-down predictive/generative process and the bottom-up inference. 
Also in social embodied cognition, the dynamic interaction between the top-down pathway for predicting while acting on the others and the bottom-up pathway for inferring the actional intention of the counterpart through sensory observation should become a crucial element.
Accordingly, we presume that different interactions in the dance appear, depending on the balance between the strength of the top-down pathway and that of the bottom-up pathway.

Recently, the number of studies addressing application of free energy principle in cognitive robotics has increased \cite{ciria2021predictive,   lanillos2021active, taniguchi2023world}.
Maselli et al. \cite{maselli2022active} showed that the active inference model is able to characterise movements generated by the agent's intention to resolve multi-sensory conflict or to achieve an external goal, such as reaching with its arm to a certain point with the agent having a VR-vision of its arm that was tilted from the actual position to confuse the agent.
%\citet{baltieri2019pid} showed that PID controllers, which are used to explain biological systems, can fit a more general theory of life and cognition under the free energy principle by using generative models of the world.
Tschantz et al. \cite{tschantz2020scaling} presented a working implementation of active inference to reinforcement learning that demonstrated efficient exploration and an order of magnitude higher sample efficiency in a high-dimensional task, such as a mountain-car environment.
Also, Pezzato et al. \cite{pezzato2023active}, showed that a robotic task, such as reactive action planning can be formulated as a free energy minimisation problem by introducing a hybrid combination of active inference and behaviour trees.
However, studies investigating human-robot interaction using the free energy principle remain few in number.

%
The author's group has conducted neurorobotic studies related to frameworks of predictive coding and active inference to develop various types of recurrent neural network (RNN) models \cite{tani2016exploring}.
However, an essential problem in these studies is that RNN models have difficulty in dealing with probabilistic properties hidden in interactions between robots and environments.
To tackle this problem, our group developed a probabilistic variational RNN model, called PV-RNN \cite{Reza2019}, based on the free energy principle.

An indispensable feature of PV-RNN is that free energy is computed as a sum of the negative accuracy term and the complexity term, weighted by a parameter $w$, called the meta-prior.
Here the complexity term represents the divergence between the approximated posterior probability distribution and the prior probability distribution in probabilistic latent variables allocated in PV-RNN.
More intuitively, the complexity term can be understood as the internal gap between top-down expectation and bottom-up sensory reality.
The model can learn to extract probabilistic structures hidden in data either by embedding them in nonlinear, deterministic dynamics of chaos by setting a large value of the meta-prior $w$ or into stochastic processes by setting a small $w$ \cite{Reza2019}.

By following the above study, Chame et al. \cite{chameICRA2020} conducted a human-robot bodily interaction experiment using a PV-RNN model and attempted to show possible effects of $w$ on the interaction dynamics.
However, this study is considered preliminary because only some snapshots of the experiments were shown and no rigorous analysis of the experimental results was made.
Wirkuttis et al. \cite{wirkuttis2021leading} performed simulation studies on synchronized imitative interaction of dyadic robots using a PV-RNN model in which robots observed the sensation of simplified vision and proprioception, but without kinaesthesis.
Through statistical analysis of repeated simulation experiments, these studies show that a robot with a larger $w$ in both the learning phase and the test phase tends to lead the counterpart robot set with smaller $w$ by projecting stronger actional intention to the counterpart. 

The current study examined human-robot kinaesthetic interaction by implementing a PV-RNN model in a real humanoid robot.
In particular, the study attempted to translate top-down intention or prior belief, formulated using the free energy principle to the force bodily exerted on the counterpart by conducting statistical analysis on data obtained from repeated experiments.
In this experiment, a humanoid robot controlled by a PV-RNN model is trained to generate movement patterns with specific transition probability distributions among them.
After the training phase, two types of human-robot kinaesthetic interaction experiments were conducted.
In the first experiment, while the robot was generating trained patterns, a human experimenter tried to physically induce a set of trained familiar movement pattern transition in the robot. 
We examined how the robot's rigidity in terms of the counter force in its response changed depending on $w$, which was set during the interaction phase.
The second experiment compared the counter force during responses in these two cases, i.e., when the experimenter induced the robot to proceed with learned movement pattern transitions and when the experimenter initiated previously unlearned pattern transitions.

Our intensive analysis of time-development of the latent variables, complexity term, and prediction error observed in these experiments under different conditions clarifies how bodily interaction between an experimenter and a robot proceeds and what sorts of neural activities are generated through top-down and bottom-up interactions during an experimental task.

% The main contribution of this paper is to show a rigid analysis of human-robot kinaesthetic interaction experiments using PV-RNN. 
% More specifically, capture and analyse the dynamics of latent variables during human-robot physical interaction in PV-RNN. This study drastically extends the aforementioned study in a sense that we have succeeded in quantitatively showing that a change of the balance between accuracy and complexity results in a change of the humanoid robot's behaviour through kinaesthetic human-robot interaction. We have managed to measure the dynamics of the latent variable and the amount of force applied to the humanoid robot with respect to time which showed clear relationship between $w$ and the humanoid robot's resistance against human.


%\IEEEPARstart{K}{inaesthetic} interaction, in psychology, underlies the acquisition and development of social and cognitive skills in early human life \cite{psychology}. There are several ways to investigate the cognitive skills of humans (articles to be searched), but one of the reasonable ways to understand the mechanism of human cognitive skills is to reconstruct it in a humanoid robot. By reconstructing human cognitive skills in a humanoid robot, we can deductively assume that we have understood how human cognition works. There are some researches based on this assumption \cite{Idei, Takazumi} which have successfully contributed to understanding human cognitive skills. Therefore, it is plausible to investigate this aspect from the perspective of cognitive robotics.

% To understand the human cognitive mind, investigation of human-robot interaction based not only on language but also physical contact is important. Mind-body dualism by R. Descartes denotes that the mind and body are distinct. Obviously, however, there are some interactions between our mind (or brain) and body. Some research suggests that deep learning might be able to mimic aspects of the theory of mind \cite{DL_to_ToM}, which suggests that deep learning models can possibly handle interactions between mind and body. Therefore, this study focuses on the kinaesthetic interaction between humans and robots using a model that combined free energy theory (free energy principle) suggested by Friston and deep learning.

% Taking the recent study of active inference (AIF) and predictive coding (PC) based on the free energy principle into account \cite{free energy principle_recent_study1}, \cite{FEP_recent_study2}, we have hypothesized that intentionality consists of the top-down information flow that corresponds to the "action intention" given the context and the bottom-up information flow that corresponds to sensation "consequenced" by one's action. Based on this, our group introduced PV-RNN, predictive coding inspired by variational RNN \cite{PV-RNN}. During the training, this model minimizes the evidence-free energy, the weighted sum of the accuracy and the complexity mentioned above. We have shown in \cite{PV-RNN} that this model can learn the stochastic structure of time-series data by avoiding to feed inputs during the forward propagation. Instead, prediction errors are back-propagated through time (BPTT). Through experiments, we observed that the model can learn to extract stochastic structure either by embedding into deterministic or stochastic dynamics with setting the regulation parameter, so-called meta-prior $w$ large or small, respectively. Furthermore, they showed that the best generalisation in learning can be achieved when $w$ is set with moderate value \cite{Model_based_on_FEP1}-\cite{Model_based_on_FEP5}.

% Further research has been conducted on human-robot interaction using PV-RNN by our research group. Chame and Tani had conducted a study focusing on the physical interaction between humans and robots \cite{Hendry}. This research established a distinction between motor and cognitive compliance by showing that a humanoid robot tends to lead or follow a human counterpart in imitative interaction when its meta-prior is set to a larger or smaller value. However, this study only shows preliminary experimental results without quantitative analysis.

% The main contribution of this paper is to show a rigid analysis of human-robot kinaesthetic interaction experiments using PV-RNN. More specifically, capture and analyse the dynamics of latent variables during human-robot physical interaction in PV-RNN. This study drastically extends the aforementioned study in a sense that we have succeeded in quantitatively showing that a change of the balance between accuracy and complexity results in a change of the humanoid robot's behaviour through kinaesthetic human-robot interaction. We have managed to measure the dynamics of the latent variable and the amount of force applied to the humanoid robot with respect to time which showed clear relationship between $w$ and the humanoid robot's resistance against human.

% Our experiments considers a humanoid robot that are trained to generate four different movement primitive patterns. Experiments were conducted while the humanoid robot was generating the primitives based on a probabilistic finite state machine. The experiments consists of two parts. For both experiments, we measured the dynamics of the latent variable in PV-RNN and the excess torque which the humanoid robot received. In the first experiment, a humanoid robot was forced by the human counterpart through physical interaction to perform a transition learnt or unlearnt in the data set. In the second experiment, the same experiment was conducted by changing the meta-prior of PV-RNN, from flex to rigid. Through this experiment, we have observed the increase of resistance from the robot when human forces the robot to move against its intention or experience.