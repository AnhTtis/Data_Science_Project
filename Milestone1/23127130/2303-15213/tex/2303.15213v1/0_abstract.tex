\begin{abstract}
The current study investigated possible human-robot kinaesthetic interaction using a variational recurrent neural network model, called PV-RNN, which is based on the free energy principle.
Our prior robotic studies using PV-RNN showed that the nature of interactions between top-down expectation and bottom-up inference is strongly affected by a parameter, called the meta-prior, which regulates the complexity term in free energy.
% The current study examines how the behaviours of robots alter by changing the meta-prior $w$ in human-robot kinaesthetic interaction.
The current study examines how changing the meta-prior $w$ in the interaction phase affects the counter force generated when an experimenter attempts to induce movement pattern transitions familiar to the robot through its prior training.
The study also compares the counter force generated when trained transitions are induced by a human experimenter and when untrained transitions are induced.
Our experimental results indicated that (1) the human experimenter needs more/less force to induce trained transitions when $w$ is set with larger/smaller values, (2) the human experimenter needs more force to act on the robot when he attempts to induce untrained as opposed to trained movement pattern transitions.
Our analysis of time development of essential variables and values in PV-RNN during bodily interaction clarified the mechanism by which gaps in actional intentions between the human experimenter and the robot can be manifested as reaction forces between them.


%% Hiroki writing 2022-11-4
%Current study investigates the dynamics of the latent states during human-robot kinaesthetic interaction using PV-RNN.
%We have achieved to observe and analyse the internal state of an RNN model based on the free energy principle, during real-time human-robot interaction.
%Essential characteristics observed in the previous study of this variational recurrent neural network model, PV-RNN, is that by changing a meta prior $w$, the balance between the top-down intention and the bottom-up perceptual reality changes.
%In the current study, we examined how changing the weighting parameter $w$ between accuracy and complexity in free energy principle affects the humanoid robot's behaviour through human-robot interaction. We have conducted some human-robot kinaesthetic interaction experiments with various $w$ and quantitatively analysed the latent variable and the force applied to the humanoid robot. We have observed that the force required to change the robot's intention has increased, both when the top-down intention was strengthened by changing the $w$ and when corresponding switch of its primitive was against the experience of the RNN during its training. The study confirms through quantitative analysis that by increasing or decreasing the $w$ in PV-RNN, humanoid robot leads or follows the human counterpart during the human-robot kinaesthetic interaction.

\begin{comment}
Comment from Jun #2
・最後にQualitativeな結果(インパクト)が欲しい
・Current study investigates the problem on~と書き出すのが一般的
・最初の一文と最後の一文を対応させる
・最後の一文はもう少しAbstractかつ包括的に
\end{comment}

\begin{comment}
Comment from Jun #1
We investigated how the kinaesthetic human-robot interaction can affect the internal state of a model based on the free energy principle. 
=> how the internal state is affected is not the most important point in this study. This part should be rewritten.

The key function of this variational recurrent neural network model, PV-RNN, is that by changing a meta prior $w$, it takes a balance between the "complexity” term and the ”accuracy” term which corresponds to a top-down intention and a bottom-up perceptual reality in the free energy principle, respectively. 
=> This is not key function of PV-RNN. It is an essential characteristics observed in the previous study. The grammar after $w$ is something strange. Rewrite these.

This research has conducted a human-robot interaction experiment with a robotic agent in a kinaesthetic sense.
=> The sentence is not good. "in a kinaesthetic sense" is grammatically wrong.
MODIFIED => "In the current study human-robot interaction experiments using the kinaesthetic sense were conducted."

We investigated that when human forces the agent to switch primitives from one to another, larger force was required both when the human intention is conflictive against the top-down the intention of the agent and when the agent has a stronger top-down intention by modifying the $w$.
=> You should write the essential results of the experiments rather than what we investigated and also how these results could contribute to the studies on human-robot interaction.
\end{comment}

\end{abstract}