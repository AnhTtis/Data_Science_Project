% This must be in the first 5 lines to tell arXiv to use pdfLaTeX, which is strongly recommended.
\pdfoutput=1
% In particular, the hyperref package requires pdfLaTeX in order to break URLs across lines.

\documentclass[11pt]{article}

% Remove the "review" option to generate the final version.
\usepackage{acl}

% Standard package includes
\usepackage{times}
\usepackage{latexsym}
\usepackage{graphicx}
\usepackage{booktabs}


% For proper rendering and hyphenation of words containing Latin characters (including in bib files)
\usepackage[T1]{fontenc}
% For Vietnamese characters
% \usepackage[T5]{fontenc}
% See https://www.latex-project.org/help/documentation/encguide.pdf for other character sets

% This assumes your files are encoded as UTF8
\usepackage[utf8]{inputenc}


\usepackage{microtype}

\title{Salient Span Masking for Temporal Understanding}


\author{Jeremy R. Cole \qquad Aditi Chaudhary \qquad Bhuwan Dhingra \qquad Partha Talukdar \\
  Google Research \\
  \texttt{\{jrcole,aditichaud,bdhingra,partha@google.com\}}}

\begin{document}
\maketitle
\begin{abstract}
Salient Span Masking (SSM) has shown itself to be an effective strategy to improve closed-book question answering performance. SSM extends general masked language model pretraining by creating additional unsupervised training sentences that mask a single entity or date span, thus oversampling factual information.
Despite the success of this paradigm, the span types and sampling strategies are relatively arbitrary and not widely studied for other tasks. 
Thus, we investigate SSM from the perspective of temporal tasks, where learning a good representation of various temporal expressions is important. To that end, we introduce Temporal Span Masking (TSM) intermediate training.
First, we find that SSM alone improves the  downstream performance on three temporal tasks by an avg. +5.8 points. 
Further, we are able to achieve additional improvements (avg. +0.29 points) by adding the TSM task. These comprise the new best reported results on the targeted tasks. 
Our analysis suggests that the effectiveness of SSM stems from the sentences chosen in the training data rather than the mask choice: sentences with entities frequently also contain temporal expressions. Nonetheless, the additional targeted spans of TSM can still improve performance, especially in a zero-shot context.
\end{abstract}


%Improvement of SSM over best-reported baseline, avg across three tasks
%(58.52 - 52.74) = 5.78
%Improvement of SSM+TSM over SSM alone
%(86.20 - 85.88) + (100.0 - 99.73)
\section{Introduction}
\label{sec:introduction}
% \begin{itemize}
%     % Diffusion of FL
%     \item {\st{Diffusion of FL}}
%     % Security threats to FL
%     \item {\st{Security threats to FL with particular focus on model poisoning}}
%     % Limitations of existing countermeasures
%     \item {\st{Current countermeasures (e.g., KRUM) and their limitations}}
%     % Proposed method and its advantages
%     \item {\st{Intuitive description of the proposed method and its difference (i.e., advantages) w.r.t. state of the art}}
%     % Main contributions
%     \item {\st{Summary of the main contributions of this work}}
%     % Paper's structure and organization
%     \item {\st{Paper's structure and organization}}
% \end{itemize}

% Diffusion of FL
Recently, {\em federated learning} (FL) has emerged as the leading paradigm for training distributed, large-scale, and privacy-preserving machine learning (ML) systems~\cite{mcmahan2017googleai,mcmahan2017aistats}. 
The core idea of FL is to allow multiple edge clients to collaboratively train a shared, global model without disclosing their local private training data.
%Specifically, an FL system consists of a central server and many edge clients; 
A typical FL round involves the following steps: {\em(i)} the server randomly picks some clients and sends them the current, global model; {\em(ii)} each selected client locally trains its model with its own private data; then, it sends the resulting local model to the server;\footnote{Whenever we refer to global/local model, we mean global/local model {\em parameters}.} {\em(iii)} the server updates the global model by computing an \emph{aggregation function}, usually the average (FedAvg), on the local models received from clients.
% \begin{enumerate}
%     \item[{\em(i)}] the server sends the current, global model to the clients and appoints some of them for training;
%     \item[{\em(ii)}] each selected client locally trains its copy of the global model with its own private data; then, it sends the resulting local model back to the server;\footnote{Whenever we refer to global/local model, we mean global/local model {\em parameters}.}
%     \item[{\em(iii)}] the server updates the global model by computing an \emph{aggregation function} on the local models received from clients (by default, the average, also referred to as FedAvg~\cite{mcmahan2017aistats}).
% \end{enumerate}
This process goes on until the global model converges. %(e.g., after a certain number of rounds or other similar stopping criteria).
%\\
% The advantages of FL over the traditional, centralized learning paradigm are undoubtedly clear in terms of flexibility/scalability (clients can join/disconnect from the FL network dynamically), network communications (only model weights\footnote{We will use \textit{parameters} and \textit{weights} interchangeably.} are exchanged between clients and server), and privacy (each client's private training data is kept local at the client's end and not uploaded to the server).
\\
% Security threats to FL
%However, the growing adoption of FL also raises security concerns~\cite{costa2022covert}, particularly about its confidentiality, integrity, and availability.
Although its advantages over standard ML, FL also raises security concerns~\cite{costa2022covert}. %, particularly about its confidentiality, integrity, and availability~\cite{costa2022covert}.
% OLD, LONG VERSION
% Indeed, some work deals with privacy leakage that may expose the local data of some clients~\cite{melis2019sp}. 
% A large body of work, instead, investigates attacks that usually aim to detriment the predictive accuracy of the learned global model. For instance, \emph{data poisoning} attacks achieve this goal by letting an adversary pollute the training set of some corrupt FL clients with maliciously crafted examples~\cite{jagielski2018sp}.
% Similarly, in \emph{model poisoning} the attacker attempts to tweak the global model weights~\cite{bhagoji2019pmlr} by directly perturbing the local model's weights of some infected FL clients before these are sent to the central server for aggregation, usually via so-called Byzantine attacks. 
% It turns out that Byzantine model poisoning attacks severely impact standard FedAvg; therefore, more robust aggregation functions must be designed to make FL systems secure.
Here, we focus on \emph{untargeted model poisoning} attacks~\cite{bhagoji2019pmlr}, where an adversary attempts to tweak the global model weights %\footnote{We will use the terms \textit{parameters} and \textit{weights} interchangeably.} 
by directly perturbing the local model's parameters of some infected clients before these are sent to the central server for aggregation.
In doing so, the adversary aims to jeopardize the global model \textit{indiscriminately} at inference time.
Such model poisoning attacks severely impact standard FedAvg; therefore, more robust aggregation functions must be designed to secure FL systems.
\\
% In this paper, we focus on designing a novel robust aggregation scheme at the server's end to contrast the effect of Byzantine model poisoning attacks.
%
% Current countermeasures and their limitations
%Several countermeasures have been proposed in the literature to combat model poisoning attacks on FL systems.
% Some methods use simple statistics more robust than plain average to smooth the impact of malicious updates (e.g., Trimmed Mean and FedMedian~\cite{yin2018icml}). 
% Other defenses implement outlier detection techniques to discard malicious updates from the aggregation performed at the server's end. Those are either based on heuristics (e.g., Krum/Multi-Krum~\cite{blanchard2017nips} and Bulyan~\cite{mhamdi2018pmlr}) or data-driven approaches (e.g., K-means clustering~\cite{shen2016acm} or DnC via spectral analysis~\cite{shejwalkar2021ndss}). 
% Finally, some strategies rely on a centralized ``source of trust'' to spot potential malicious updates (e.g., FLTrust~\cite{cao2020fltrust}).
% Several countermeasures have been proposed in the literature to combat model poisoning attacks on FL systems, i.e., to discard possible malicious local updates from the aggregation performed at the server's end. 
% These techniques range from simple statistics more robust than plain average (e.g., Trimmed Mean and FedMedian~\cite{yin2018icml}) to outlier detection heuristics (e.g., Krum/Multi-Krum~\cite{blanchard2017nips} and Bulyan~\cite{mhamdi2018pmlr}) or data-driven approaches (e.g., spectral analysis via K-means clustering~\cite{shen2016acm} or spectral analysis), or methods based on ``source of trust'' (e.g., FLTrust~\cite{cao2020fltrust}).
% OLD, LONG VERSION
%Several countermeasures have been proposed in the literature to combat Byzantine model poisoning attacks on FL systems.
% Descriptive statistics
% For example, Trimmed Mean and FedMedian aggregate local model updates using more robust statistics than standard average~\cite{yin2018icml}.
%
% % Heuristics for outlier detection
% Many existing Byzantine-resilient strategies implement some outlier detection heuristics to discard the model updates sent by potentially malicious clients from the input of the aggregation function.
% One of the most popular heuristics is Krum~\cite{blanchard2017nips}.
% This strategy tries to mitigate the impact of Byzantine attacks by selecting as a global model the local model with the smallest sum of Euclidean distances to {\em all} the other local models.
% Although powerful, Krum requires the server to know (or, at least, estimate) the number of malicious FL clients upfront, which is generally impossible in a realistic attack scenario. %
% Moreover, Krum may become ineffective for complex, high-dimensional model parameter spaces due to the curse of dimensionality.
% Bulyan~\cite{mhamdi2018pmlr} tries to overcome this issue by combining Krum with a variant of Trimmed Mean.
% % Data-driven outlier detection
% Other strategies use data-driven outlier detection techniques -- e.g., via K-means clustering~\cite{shen2016acm} -- to spot potential malicious local model updates. 
% %For instance, Shen et al. propose to cluster local model updates with K-means and thus identify outliers.
%
% % Other techniques
% As far as the server is concerned, any local model received can be from a potential malicious client. 
% FLTrust~\cite{cao2020fltrust} assumes the server acts as a client, i.e., trains a local model on an additional {\em trustworthy} dataset at the server's end and compares it against all the local models from other clients. 
% This way, the server can rely on some ``source of trust'' when discarding potentially malicious clients.
%\\
% Limitations of existing Byzantine-resilient strategies
Unfortunately, existing defense mechanisms either rely on simple heuristics (e.g., Trimmed Mean and FedMedian by~\cite{yin2018icml}) or need strong and unrealistic assumptions to work effectively (e.g., foreknowledge or estimation of the number of malicious clients in the FL system, as for Krum/Multi-Krum~\cite{blanchard2017nips} and Bulyan~\cite{mhamdi2018pmlr}, which, however, cannot exceed a fixed threshold).
Furthermore, outlier detection methods using K-means clustering~\cite{shen2016acm} or spectral analysis like DnC~\cite{shejwalkar2021ndss} do not directly consider the temporal evolution of local model updates received.
Finally, strategies like FLTrust~\cite{cao2020fltrust} require the server to collect its own dataset and act as a proper client, thereby altering the standard FL protocol.
\\
% OLD, LONG VERSION
% Overall, existing Byzantine-resilient strategies are either simple heuristics (e.g., FedMedian) or, if they are more complex, they rely on strong and unrealistic assumptions to work effectively (e.g., knowing the number of malicious clients in the FL system in advance, as for Krum and alike).
% Furthermore, data-driven outlier detection methods do not consider the temporary evolution of local model updates received (e.g., K-means clustering). 
% Finally, strategies like FLTrust requires the server to collect its own dataset and act as a proper client, thereby altering the standard FL protocol.
%
% Description of the proposed method
This work introduces a novel pre-aggregation \textit{filter} robust to untargeted model poisoning attacks. Notably, this filter $(i)$ operates without requiring prior knowledge or constraints on the number of malicious clients and $(ii)$ inherently integrates temporal dependencies. 
The FL server can employ this filter as a preprocessing step before applying \textit{any} aggregation function, be it standard like FedAvg or robust like Krum or Bulyan.
Specifically, we formulate the problem of identifying corrupted updates as a multidimensional (i.e., matrix-valued) time series anomaly detection task. 
The key idea is that legitimate local updates, resulting from well-calibrated iterative procedures like stochastic gradient descent (SGD) with an appropriate learning rate, show \textit{higher predictability} compared to malicious updates. This hypothesis stems from the fact that the sequence of gradients (thus, model parameters) observed during legitimate training exhibit regular patterns, as validated in Section~\ref{subsec:intuition}. %until convergence. 
%This regularity may be more pronounced for smooth convex loss functions, but it can still be captured within an appropriate time window, even for more complex and convoluted loss surfaces. 
%We provide evidence of this claim in Appendix~B, where we show that the average mutual information (i.e., ``predictability''), calculated over pairs of legitimate model updates sent at different FL rounds, is significantly higher than the corresponding computation for a malicious client.
\\
Inspired by the matrix autoregressive (MAR) framework for multidimensional time series forecasting~\cite{chen2021je}, we propose the FLANDERS ({\em \textbf{F}ederated \textbf{L}earning meets \textbf{AN}omaly \textbf{DE}tection for a \textbf{R}obust and \textbf{S}ecure}) filter.
The main advantages of FLANDERS over existing strategies like FLDetector~\cite{zhao2020multivariate} are its resilience to large-scale attacks, where $50\%$ or more FL participants are hostile, and the capability of working under realistic non-iid scenarios.
We attribute such a capability to two key factors: $(i)$ FLANDERS works without knowing a priori the ratio of corrupted clients, and $(ii)$ it embodies temporal dependencies between intra- and inter-client updates, quickly recognizing local model drifts caused by evil players. Below, we summarize our main contributions:

\begin{itemize}
\item[{\em(i)}]
We provide empirical evidence that the sequence of models sent by legitimate clients is more predictable than those of malicious participants performing untargeted model poisoning attacks.
\\
\item[{\em(ii)}] 
We introduce FLANDERS, the first pre-aggregation filter for FL robust to untargeted model poisoning based on multidimensional time series anomaly detection.
\\
\item[{\em(iii)}] 
We integrate FLANDERS into Flower,\footnote{\scriptsize{\url{https://flower.dev/}}} a popular FL simulation framework for reproducibility.
\\
\item[{\em(iv)}] 
We show that FLANDERS improves the robustness of the existing aggregation methods under multiple settings: different datasets, client's data distribution (non-iid), models, and attack scenarios.
\\
\item[{\em(v)}] 
We publicly release all the implementation code of FLANDERS along with our experiments.\footnote{\scriptsize{\url{https://anonymous.4open.science/r/flanders_exp-7EEB}}}
\end{itemize}

% Paper's structure and organization
The remainder of the paper is structured as follows. %some related work and the current state-of-the-art solutions to security issues that FL entails. 
Section~\ref{sec:background} covers background and preliminaries. 
In Section~\ref{sec:related}, we discuss related work.
Section~\ref{sec:problem} and Section~\ref{sec:method} describe the problem formulation and the method proposed. % to tackle it. 
Section~\ref{sec:experiments} gathers experimental results. %, and Section~\ref{sec:limitations} discusses some limitations of this work.
Finally, we conclude in Section~\ref{sec:conclusion}.
 %discusses the limitations of this work and draws future research directions.
%reports conclusions and draws perspectives for future research directions.

%%%%%%% OLD %%%%%%%
%to overcome the resilience of Byzantine failures in distributed Stochastic Gradient Descent computations. 
% The strength of Krum is its time complexity, which is linear in the gradient dimension. 
% However, the robustness of the approach is guaranteed for gradient-based learning applications only when the majority of the clients are not compromised. 
% Besides, the aggregation mechanism of Krum, as well as that of similar methods, is robust from a coarse-grained perspective and does not provide solutions to errors and perturbations that may occur at inference time.
%A related approach to~\cite{blanchard2017nips} is the work of Su et al.~\cite{su2016dc}. Here, the authors propose an iterated approximate agreement to tackle a multi-layer scenario attacked by Byzantine agents. 
%However, the method works efficiently on the sole discrete context and it is inapplicable to continuous state environments.
%\gabri{Maybe, we should just talk about the main limitations of existing countermeasures without digging into their details (or, we can just mention Krum as this is the most popular one). I will move the description of all these methods to the Related Work section.}
\section{PoseRAC Model}
\label{sec4}

\begin{figure*}[t]
\centering
\includegraphics[width=1.0\textwidth]{figure5.pdf}
\caption{Overview of our proposed PoseRAC. For a input video, the repetitive count can be obtained through Pose Estimation, Transformer Encoder, Pose Mapping and Action-trigger, where only the Encoder and the Pose Mapping need to be trained. We use Triplet Margin Loss to train the Encoder while Binary Cross Entropy Loss to train both the Encoder and the Pose Mapping. In addition to achieving the state-of-the-art performance so far, the biggest highlight of our PoseRAC is that it is lightweight enough to be easily trained on a CPU.}
\label{fig5}
\end{figure*}

Given a video $V={\{x_i\}}^{T}_{1}\in \mathbb{R}^{C\times H\times W\times T}$ with $T$ RGB frames, repetitive action counting model aims to predict a certain value $Y$, which is the number of repetitive actions. In this section, we will introduce our PoseRAC in detail.

\subsection{Model Overview}

As shown in Figure \ref{fig5}, PoseRAC consists of four parts. 

\begin{itemize}

\item The first is a state-of-the-art and lightweight Pose Estimation Network~($\S\ref{first}$), which is used to estimate the poses represented by lots of human pose key points from each frame of the original video sequence. 

\item The second is a simple Transformer Encoder~($\S\ref{second}$) to embed the key points of poses into high-level feature space, where the same class have similar distances, while the distances of different classes are far apart.

\item The third is a Pose Mapping Module~($\S\ref{third}$), where the unique mapping relationship between the salient poses and the action classes can be learned. Each pose can be mapped to the action class with the highest probability after the previous encoding.

\item The fourth part is a lightweight Action-trigger Module~($\S\ref{fourth}$). When we get the salient action classification results of all frames of the entire video sequence, we can use this module to calculate the repetition count in a short time.

\end{itemize}

\subsection{Pose Estimation Network}
\label{first}
Our model first converts the video sequence into a sequence of human pose key points, which can be defined as: 
\begin{equation}
\begin{split}
&V={\{x_i\}}^{T}_{1}\in \mathbb{R}^{C\times H\times W\times T}\\
&V\xrightarrow{\mathrm{Pose Estimation}} P={\{p_i\}}^{T}_{1}\in \mathbb{R}^{D\times K\times T}
\end{split}
\label{eq1}
\end{equation}
where each $x_i$ represents a single RGB frame, and each $p_i$ represents the key points of each frame. To express the key points of each frame, we use $D\times K$ sequence, which includes two parts, one ($K$) is the number of key points to fully represent the current pose, the other ($D$) is the dimension of each key point, generally three, which are the two coordinates of the planes and the depth estimation.

Here we use state-of-the-art pose estimation models such as Vitpose\cite{xu2022vitpose} and BlazePose\cite{bazarevsky2020blazepose}. The pose estimation algorithms themselves are not designed by us, but we introduce pose information into the action counting task, which is a novel design not explored by previous work.

Moreover, our pose-level poses estimation processes the primitive information of video, which is similar to the feature extraction network in all video-level algorithms such as I3D\cite{carreira2017quo}, VideoSwinTransformer\cite{liu2022video}, and TSN\cite{wang2016temporal}. But the difference is that the result of video-level incorporates all information, while pose-level only produces core information, which greatly improves the performance. Additionally, using pose information can contribute to the lightweight of model. For instance, for a 1024-frame video, video-level feature extraction with an output dimension of 512 would produce a data volume of $1024\times 512=524288$, while using pose information with 33 key points produces a data volume of only $1024\times 33 \times 3=101376$.

\subsection{Encoding Poses with Transformer}
\label{second}
Here we specify our data representation for the Transformer Encoder, which requires input batch size, sequence length, and embedding dimensions. In our pose-level approach, each frame is a batch, the number of key points in each frame is the sequence length, and the feature dimension of each key point is the embedding dimension.

First we get the pose of each frame ${p_i}\in \mathbb{R}^{D\times K}$ through the Pose Estimation Network, where $i\in {1, 2, \dots, T}$ is the frame index, $K$ is the number of key points, and $D$ is the dimension of each key point. We further define $p_i = {\{k_j\}}^{K}_{1}$ to represent each key point, where $k_j\in \mathbb{R}^D$, and we embed it to obtain richer information. Our embedding projection $\mathrm{\bf{E}}$ is a simple MLP network with ReLU as the activation function. These calculations can be defined as:
\begin{equation}
\begin{split}
\mathrm{\bf{Z}}^0 = [\mathrm{\bf{E}}(k_1), \mathrm{\bf{E}}(k_2), \dots, \mathrm{\bf{E}}(k_K)]^T
\end{split}
\end{equation}
where $\mathrm{\bf{E}}(k_j)\in \mathbb{R}^{D^{\prime}}$ is the embedding feature. Then the next Transformer takes $\mathrm{\bf{Z}}^0$ as input and encodes it with self-attention. Given $\mathrm{\bf{Z}}^0\in \mathbb{R}^{K\times D^{\prime}}$ with $K$ key point features, each of which is $D^{\prime}$-dimensional, $\mathrm{\bf{Z}}^0$ is projected using $\mathrm{\bf{W}}_Q\in \mathbb{R}^{D^{\prime}\times D_q}$, $\mathrm{\bf{W}}_K\in \mathbb{R}^{D^{\prime}\times D_k}$, $\mathrm{\bf{W}}_V\in \mathbb{R}^{D^{\prime}\times D_v}$, where $D_k=D_q$, to extract feature representations query($\mathrm{\bf{Q}}$), key($\mathrm{\bf{K}}$) and value($\mathrm{\bf{V}}$), which can be defined as:
\begin{equation}
\begin{split}
&\mathrm{\bf{Q}}=\mathrm{\bf{Z}}^0\times \mathrm{\bf{W}}_Q\\
&\mathrm{\bf{K}}=\mathrm{\bf{Z}}^0\times \mathrm{\bf{W}}_K\\
&\mathrm{\bf{V}}=\mathrm{\bf{Z}}^0\times \mathrm{\bf{W}}_V
\end{split}
\end{equation}
and the output of self-attention can be computed as:
\begin{equation}
\begin{split}
\mathrm{\bf{Attn}}=\mathrm{Softmax}(\frac{\mathrm{\bf{Q}}\mathrm{\bf{K}}^T}{\sqrt{D_q}})\mathrm{\bf{V}}
\end{split}
\end{equation}
where $\mathrm{\bf{Attn}}\in \mathbb{R}^{K\times D^{\prime}}$. Also, we use common multi-head self-attention (MHSA) to make several self-attention operations calculate in parallel.

Now we introduce the overall architecture of Transformer Encoder, which has $L$ layers with each layer consisting of MHSA and MLP blocks. Also, LayerNorm and Residual Connection are applied before and after every MHSA or MLP block, respectively. Because the number of key points of each frame is  a bit less, so our encoder does not include the downsampling module that other models may have. The overall process can be defined as:
\begin{equation}
\begin{split}
&\mathrm{\bf{\hat{Z}}}^l = \mathrm{MHSA}(\mathrm{LN}(\mathrm{\bf{Z}}^{l-1})) + \mathrm{\bf{Z}}^{l-1}\\
&\mathrm{\bf{Z}}^l = \mathrm{MLP}(\mathrm{LN}(\mathrm{\bf{\hat{Z}}}^l)) + \mathrm{\bf{\hat{Z}}}^l
\end{split}
\end{equation}
where $\mathrm{\bf{Z}}^{l-1}$, $\mathrm{\bf{\hat{Z}}}^l$, $\mathrm{\bf{Z}}^l\in \mathbb{R}^{K\times D^{\prime}}$.


\subsection{Pose Mapping}
\label{third}
Taking the Encoder output $\mathrm{\bf{Z}}^L\in \mathbb{R}^{K\times D^{\prime}}$ as input, Pose Mapping module outputs probability scores $\mathrm{\bf{S}}\in \mathbb{R}^{C}$ of the current frame over all action classes. We perform binary classification after Sigmoid activation for each class, with the two salient poses of each class represented by the same bit data. To realize such a module, we use a very lightweight MLP network, which avoids the complexity. First, the two dimensions $K$ and $D^{\prime}$ of $\mathrm{\bf{Z}}^L$ are flattened into $\mathbb{R}^{KD^{\prime}}$, and then it passes through an MLP module, where the output channels is set to $C$, which can be defined as:
\begin{equation}
\begin{split}
\mathrm{\bf{S}} = \sigma(\mathrm{MLP}(\mathrm{Flatten}(\mathrm{\bf{Z}}^L)))
\end{split}
\end{equation}
where $\sigma$ represents the Sigmoid activation function.

With such Pose Mapping, we can obtain the scores of single frame. It should be noted that we extract the poses of all frames, and use the convenience of matrix operations to obtain scores in parallel, which is actually consistent with the idea of mini batch. So at last, we combine the scores of all frames to get the video score matrix $\mathrm{\bf{\hat{S}}}\in \mathbb{R}^{C\times T}$, where $T$ represents the number of frames in the current video. 


\subsection{Action-trigger Module}
\label{fourth}
We use the lightweight Action-trigger Module to obtain the final output $Y$, the repetitive action count, which has a time complexity of $\mathcal{O}(n)$. First, we get the scores $S_c\in \mathbb{R}^T$ of a given action class from $\mathrm{\bf{\hat{S}}}$. Then, we scan all frames and use the action-trigger mechanism to count when the two salient poses of the action class occur sequentially. We set upper and lower bounds to distinguish the scores of the two salient poses, which cluster non-salient poses in the middle and easily classify the salient poses to the two ends.

\subsection{Losses and Metric Learning}

The modules need to be trained are Embedding, Transformer Encoder and Pose Mapping, and because we perform binary classification for each class, so we use the Binary Cross Entropy Loss, which can be defined as follows:
\begin{gather}
\mathcal{L}_{bce} = -\frac{1}{N}\sum\limits_{i=1}^{N}(\frac{1}{C}\sum\limits_{j=1}^{C}loss(i,j))  \\
 loss(i,j)=y_{ij}\log p_{ij} + (1-y_{ij})\log(1-p_{ij})
\end{gather}
where $N$ represents the batch size (in our method, each frame is a batch), $C$ represents the number of classes, $y$ and $p$ are the labels and our predictions, respectively.

Moreover, we use Metric Learning to improve our Encoder and introduce the Pose Triplet Loss. Given a pose, Encoder produces higher-level features $\mathrm{\bf{Z}}^L$, which should be more representative. As shown in Figure \ref{fig5}, we achieve this with Triplet Margin Loss function, which selects anchors, same class positive samples, and different classes negative samples in a batch. It can be expressed as:
\begin{equation}
\begin{split}
\mathcal{L}_{tri} = \mathrm{max}(\mathrm{CS}(a,p)-\mathrm{CS}(a,n)+\mathrm{margin},0)
\end{split}
\end{equation}
where $a$, $p$, $d$ are anchors, positive and negative samples, and $\mathrm{CS}$ represents the Cosine Similarity to measure the distance between features. We pay more attention to hard samples, where the distances between anchors and negative samples are even smaller than those of positive samples. After Metric Learning, the poses of each action can be distinguishable, which cluster in the high-level space.

At last, our overall training combines these two losses:
\begin{equation}
\begin{split}
\mathcal{L} = \mathcal{L}_{bce} + \alpha\mathcal{L}_{tri}
\end{split}
\end{equation}
where $\alpha$ is the weight factor to control the two losses in the same numeric scale.
\subsection{Implementation Details}

\noindent{\bf Training.} We use the \emph{RepCount-pose} and \emph{UCFRep-pose} dataset we created to train our model. Only the frames with salient poses are inputted into the network instead of the entire video to speed up the fitting.

\noindent{\bf Inference.} During inference, the entire video sequence is inputted into the model. The poses of all frames pass through the Encoder and Pose Mapping, and then enter the Action-trigger Module to output the repetitive count.
\begin{table*}[thb]
\begin{center}
\scalebox{0.9}{
\begin{tabular}{lcccccccccc}
% \toprule
 & \multicolumn{2}{c}{SituatedQA} & \multicolumn{2}{c}{MC-TACO} & \multicolumn{2}{c}{TimeDIAL} & \multicolumn{2}{c}{TimeDIAL-0} & &  \\
\cmidrule(lr){2-3} \cmidrule(lr){4-5} \cmidrule(lr){6-7} \cmidrule{8-9}
Model & F1 & EM & F1 & EM & 1-Best & 2-Best & 1-Best & 2-Best & Overall & Overall-0  \\
\midrule
% \textsc{ALICE} \cite{pereira-etal-2020-adversarial} & -- & -- & 78.64 & 58.02 & -- & -- & -- \\
% \textsc{DeBERTa} \cite{he-etal-2020-deberta} & -- & -- & 82.92 & 63.81 & -- & -- & -- \\ 
% \textsc{BART} \cite{lewis-etal-2020-bart} & -- & 18.3 & -- & -- & -- & -- \\
% \textsc{ALBERT}  \cite{lan2019albert}  & -- & -- & -- & -- & -- & 76.10 & -- \\
\textsc{Best Reported} & -- & 18.53 & 82.92 & 63.81 & -- &  76.10 & -- & 50.60 & 52.74 & 44.31 \\
\midrule
\textsc{T5} & 25.75 & 19.78 & 84.00 & 64.56 & 99.91 & \textbf{84.50} & 90.85 & 37.59 & 56.28 & 40.64 \\
\textsc{T5-LM} & 25.38 & 19.63 & 81.99 & 59.83 & 99.91 & 80.60 & 86.87 & 32.16 & 53.35 & 37.21  \\
\textsc{SSM} & 29.92 & 23.12 & 85.88 & 68.39 & 99.73 & 84.06 & 96.74 & 67.21 & 58.52 & 52.91 \\
\midrule
\textsc{Entities} & 29.42 & 22.82 & 85.47 & 66.59 & 99.91 & 83.06 & 97.64 & 67.93 & 57.49 & 52.45 \\
\textsc{TSM} & 27.42 & 21.18 & 84.89 & 65.92 & 99.91 & 83.88 & \textbf{99.82} & \textbf{77.54} & 56.99 & 54.88  \\
\textsc{TSM+SSM} & 29.33 & 22.76 & \textbf{86.20} & 67.64 & \textbf{100.0} & 83.78 & 98.19 & 73.10 & 58.03 & 54.5  \\
\textsc{Entities+TSM} & \textbf{30.78} & \textbf{24.60} & 85.32 & \textbf{68.47} & 99.91 & 84.24 & 98.91 & 76.09 & \textbf{59.09} & \textbf{56.39} \\
\bottomrule
\end{tabular}
}
\caption{Aggregate metrics across the three datasets. \textit{Overall} performance is the simple arithmetic average of the harder metric for each approach (EM, EM, 2B); \textit{Overall-0} uses TimeDIAL-0 instead of TimeDIAL. The second section contains our runs of earlier models; the Best Reported uses best known published numbers. The third section represents our models. Note that all models (in the second and third sections) are based on T5-1.1-XXL models. Best Reported results are ALBERT \citep{lan2019albert} from \citet{abramson2022application} for TimeDIAL, BART results from \citet{zhang-choi-2021-situatedqa}, and DeBERTa \citep{he-etal-2020-deberta} results from the leaderboard for MC-TACO. Note that F1 for MC-TACO is based on the precision/recall over answers and EM is based on labeling every answer for a question correctly, while the F1 for SituatedQA is based on the token-level F1 of the answer span.}
\label{tab:overall}

\end{center}
\end{table*}
\section{Results}
\label{results}

\begin{figure*}[ht]
    \centering
    \includegraphics[scale=0.15,trim={0 2.5cm 0 5cm},clip]{images/aoi-single_burst}
    \caption{The time average peak Age of Information with burst and \gls{soa} loss values against the dynamic reliability logic for different network topologies.}
    \label{fig:aoi_burst}\vspace{-0.4cm}
\end{figure*}


This paper focuses on both transport layer and application layer metrics to determine the feasibility of dynamic reliability. For this, we have selected the session packet volume, as transmitted, retransmitted, lost and backlogged packets as \glspl{kpi} for the transport layer; while focusing on the \gls{aoi} for the application layer. The \gls{aoi} was chosen as a crucial indicator for the freshness of packets in real-time applications. More specifically, this work adopts the time average peak \gls{aoi} equation \cite{aoi_equation} depicted in Eq. \ref{aoi}, where $\Delta(r_{i+1})$ is the $i$th update at the time it was received at the server, for a session time period of $\tau$.

\begin{equation}
    \label{aoi}
    \gls{aoi}_\tau = \frac{1}{n-1}\sum_{i=1}^{n-1} \Delta(r_{i+1})
\end{equation}

We include a comparison between the vanilla QUIC implementation which does not enjoy the dynamic reliability extension, with a number of dynamic reliability policies. The tests were run a number of times for statistical significance, with the mean value of vanilla implementation used as a baseline for comparison. The topology utilised both random loss and bursty loss to explore the bounds of dynamic reliability. The \gls{soa} loss in the figures correspond to the loss values presented in Table. \ref{tab:path_char}, for ease of comparison between bursty and random loss scenarios.

\subsection{Transport-Layer KPIs}

To analyse the performance gain at the transport layer due to dynamic reliability, the volume of transmitted and backlogged packets is examined. The figures are in the form of boxplots, which take the vanilla implementation as a benchmark, depicted as the red dashed line.

As seen in Fig. \ref{fig:sent_burst}, the loss plays a crucial role in the performance of the reliability policies. The policies under random loss did incredibly well for the networks with a larger capacity, namely \gls{mmwave} and Sub-6~GHz, whereas for burst loss, the lower network capacities had a larger packet reduction. With the increase in burst loss, the behaviour of the set split reliable policies became unpredictable, if a reliable assignment happened to coincide with a burst loss, the number of transmitted packets increases, and vice versa. On the other hand, in smarter policies, such as Loss-Aware, the performance lightly matched the vanilla baseline, as the reliable assignment dominated the session to compensate for a higher burst loss. Not only that but, the burst loss also impacted the variance of the transmitted packets for the policies.

Unsurprisingly, the unreliable focused policy, 80-20 split, outperformed other policies for all topologies in random and bursty loss scenarios, with an approximate reduction of 80\%. That being said, the majority of the policies reduced the transmitted packets on the link by approximately 70\% for random loss, while the reduction started at $\approx 15\%$ and decreased as the loss increased for the burst loss scenario.

The retransmitted and lost packets, not shown due to space limitations, followed the same trend as the transmitted packets for the random loss scenarios. However, for the burst loss scenarios, the larger capacity networks had a lower reduction in the retransmitted and lost packets. This can be seen as a favorable outcome since the lower capacity networks are scarce on resources. It is important to note that the Loss-Aware policy mimicked the vanilla approach as the burst loss increased, signifying the overwhelming appointment of reliable packets in adapting to the harsh burst loss conditions.
 
Alternatively, Fig. \ref{fig:backlog_burst} clearly shows a stark comparison between the policies and loss scenario in the reduction of the backlogged packets. The Loss-Aware policy for random loss scenario reduced the backlogged packets by up to 50\%, beating all other policies by approximately 30\%. Furthermore, it is clear that the unreliability focused policies resulted in the lowest backlog for the session. In comparison, we notice that the burst loss and the backlogged frequency have a positive correlation, where the maximum reduction of the backlogged packets for the policies is at most 20\%. Much like the transmitted packets, the probability of a burst loss occurrence plays a vital role in the number of retransmissions sent and by extension the number of backlogged packets. Thus, we can conclude that the stress placed on the buffer is a result of the reliable packets which is tightly coupled with the congestion on the session. Whereas, unreliable focused policies did not encounter such a phenomenon regardless if it was experiencing a burst loss.


\subsection{Application-Layer KPIs}

The feasibility of dynamic reliability for real-time applications can be determined by the \gls{aoi}, with comparison across different topologies and policies. If we take a strict approach and consider anything below $10$~ms is real-time \cite{real-time}, then all the reliability policies passed that requirement, which is attractive for real-time applications, as shown in Fig. \ref{fig:aoi_burst}. Utilising the median as an estimate of the runs, the policies in the WLAN and Sub-6~GHz topology with random loss floated around $4-5$~ms with negligible difference, while the \gls{aoi} for \gls{mmwave} was $\approx 2-3$~ms. It is clear that the \gls{aoi} and the network capacity have a negative correlation, as the network capacity decreases, the \gls{aoi} increases. The same correlation is extended to the bursty loss scenarios, where \gls{mmwave} dominated the other topologies. That being said, it is crucial to note that the \gls{aoi} for the reliability policies is often slightly better than or equal to the \gls{aoi} of the vanilla implementation, proving that dynamic reliability reduces the congestion of the session at no cost to the \gls{aoi}.

\section{Related work}
% There is extensive recent work on speeding up analytical queries due to the need for consistent execution times in the face of the explosive growth in the volume of available data.
% In this section, we divide existing work into two categories: maintaining data freshness in MVs (\Cref{sec:server_side}) and optimizations for minimizing ad-hoc query latency (\Cref{sec:client_side}).

% \subsection{Maintaining Data Freshness in MVs}
% \label{sec:server_side}
% There exists a variety of data warehousing applications aimed at supporting low-latency analytical queries on fresh data.
% In particular, these applications require efficiency in the propagation of newly ingested data into downstream MVs.
 
\mypara{Efficient MV Refresh}
Incremental view maintenance (IVM) aims to update MVs to reflect newly ingested data, taking advantage of already computed results to perform the update in a manner more efficient than computing from scratch (full refresh)
~\cite{ahmad2012dbtoaster,mcsherry2013differential,armbrust2013generalized,zeng2016iolap, palpanas2002incremental, griffin1995incremental, agiwal2021napa, braun2015analytics}. 
There is an abundance of work in IVM, including incremental updates on duplicate values~\cite{griffin1995incremental}, non-distributive aggregate functions~\cite{palpanas2002incremental}, higher-order views~\cite{ahmad2012dbtoaster}, and sliding windows~\cite{braun2015analytics}. 
More recent works also investigate the scalability aspect of IVM, proposing scale-independent updates~\cite{armbrust2013generalized} and sampled views~\cite{zeng2016iolap}. Since \system is applicable to arbitrary SQL statements, \system is orthogonal to and is fully compatible with existing IVM techniques.

\mypara{MV Refresh Scheduling}
There exist works on scheduling the refresh of a MV set focusing on resolving cyclic dependencies~\cite{folkert2005optimizing}, minimizing weighted average staleness~\cite{golab2009scheduling}, and prioritizing MVs with the highest speedups on predicted future queries~\cite{ahmed2020automated}.
\system's scheduling to speed up the end-to-end refresh of the MV set is not addressed in existing works.

\mypara{DAG Workflow Scheduling}
The execution of workloads consisting of individual jobs with acyclic dependencies is a well-studied topic~\cite{apacheoozie,sparkdag,marchal2018parallel,bathie2020revisiting,baruah2022ilp}; many of these techniques can be applied to MV refresh runs studied in this paper.
Existing workflow scheduling systems such as Apache Oozie~\cite{apacheoozie}, Apache Airflow~\cite{airflow}, and Spark DAG scheduler~\cite{sparkdag} automate the execution of user-defined workflows following a topological order.
There are recent works aimed at finding more optimal execution orders in terms of peak memory usage~\cite{marchal2018parallel, bathie2020revisiting} and execution time on parallel platforms~\cite{baruah2022ilp}.
While \system is designed for use with MV refresh runs/workloads, our technique on joint scheduling and optimization can be reasonably applied to general workloads as a possible future direction.

% \paragraph{Incremental MV indexing}
% Update-optimized indices such as the log-structured merge-trees (LSM)~\cite{o1996log} are used for indexing MVs due to frequent updates induced by data ingestion~\cite{gupta2016mesa,agiwal2021napa}.
% \system is orthogonal to indexing: \system is capable of efficiently performing MV refresh runs regardless of whether the individual MVs are indexed or not.

% \subsection{Ad-hoc Query Latency Reduction}
% \label{sec:client_side}

% The minimization of ad-hoc analytical query response times is a well-studied topic due to latency being negatively correlated with the productivity of a data analyst during a data analysis session~\cite{liu2014effects}.
% Sessions are commonly conducted within visualization systems that contain a variety of optimization techniques to ensure that query response times fall within a certain latency tolerance.

% \mypara{Data prefetching}
% Data is often loaded into memory on a by-need basis in visualization systems to minimize interference with user-issued query computations~\cite{mani2017effective,xin2021enhancing,galakatos2017revisiting, yan2020auto, battle2016dynamic, crotty2016case, jalaparti2018netco}. 
% Query-time data retrieval can be significantly expedited by anticipating the data usage of the user in future queries and pre-loading the data into memory during the downtime between user queries (`think time'). SMART~\cite{mani2017effective} prefetches data for modified versions of current user-issued queries with different filters and dimensions. A-WARE~\cite{crotty2016case} maintains a sample store constantly refined through ingesting data based on speculations of future plots.
% ForeCache~\cite{battle2016dynamic} uses an SVM to predict the user's current analysis phase and accordingly prefetches data tiles partitioned based on different numerical values. NetCo predicts future queries via log analysis, and solves an ILP formulation to prefetch data to maximize the number of SLO-meeting queries~\cite{jalaparti2018netco}.
% In the case of MV refresh workloads, `think time' is nonexistent as individual MVs are refreshed back-to-back, rendering data prefetching techniques non-applicable.

\mypara{Intermediate Data Caching}
Some existing data visualization systems cache user-defined variables to support the typical incremental construction of data visualizations~\cite{zgraggen2016progressive, eichmann2020idebench} during data analysis sessions~\cite{jupyter, rstudio, colab}. 
Recent work proposes a management scheme for these cached variables under a memory constraint that greedily keeps variables with the highest estimated time savings based on predicted future user behavior via neural networks~\cite{xin2021enhancing}.
While useful for data visualization, a greedy approach to memory management fails to achieve satisfactory results compared to \system.

\mypara{Intermediate Result Reuse}

There exist works on storing intermediate results from computations to speedup future computations~\cite{yang2018intermediate, dursun2017revisiting, nagel2013recycling, michiardi2019memory, galakatos2017revisiting}.
Studied topics include the identification of reuse opportunities by finding overlaps in computation graphs of successive jobs~\cite{yang2018intermediate, michiardi2019memory},
selective storage under a space constraint with heuristics such as reuse probability~\cite{dursun2017revisiting}, expected savings~\cite{yang2018intermediate}, and recompute-storage cost difference~\cite{nagel2013recycling},
and rewriting incoming jobs to take advantage of stored intermediates~\cite{galakatos2017revisiting}.
These works share similarity with \system in their selection of items to store under a memory constraint, however, \system's problem setting requires it to uniquely consider the joint (re)ordering of job executions along with the selection of items.

% work that considers both job execution (re)order as well as intermediate result caching with a bounded amount of memory. but notably lack the joint aspect of \system and cannot be used to achieve immediate speedup on an incoming MV refresh run if no intermediates are stored beforehand. 

\mypara{Incremental Query Processing} Incremental processing (IQP) is useful for cases where not all data required for a query is immediately available. Similar to online aggregation~\cite{hellerstein1997online}, initial results of a query are computed on a subset of required data and progressively refined as the rest of the required data arrives in a predictable pattern~\cite{tang2019intermittent,wangtempura}. Tang et al. propose a dynamic programming formulation to pick intermediate states to store in memory given a limited memory budget~\cite{tang2019intermittent}. Tempura rewrites the query plan for more efficient execution based on predicted data arrival patterns~\cite{wangtempura}. While similarities exist between the problem setting of IQP and \system, such as management of bounded memory, \system notably includes additional joint optimization for the order of MV updates.

% \paragraph{Sampling}
% Sampling has seen wide use in visualization systems for reducing the computation time of ad-hoc queries by computing an approximate result over a subset of data as exact results are not always required by the user~\cite{crotty2016case, mani2017effective, zgraggen2014panoramicdata, kraska2021northstar, galakatos2017revisiting, kandula2016quickr}. 
% Commonly studied topics in sampling for ad-hoc queries include complex query sampling~\cite{kandula2016quickr}, rare event aggregation~\cite{kraska2021northstar, galakatos2017revisiting}, and maintaining consistency between related sampled visualizations~\cite{zgraggen2014panoramicdata}.
% Sampling server-side at the MV level compromises the assumptions of downstream applications and is thus not considered in \system.

% \paragraph{Progressive visualization}
% The latency tolerance for time-consuming queries can be circumvented by presenting a partially-computed visualization to the user within the tolerance, which is then incrementally refined until it is fully accurate~\cite{rahman2017ve, zgraggen2016progressive, crotty2015vizdom, kraska2021northstar, kamat2017infiniviz}.
% Example plots which benefit from progressive visualization include bar charts~\cite{kamat2017infiniviz} and heatmaps~\cite{rahman2017ve}.
% Similar to sampling, study on this topic is orthogonal to \system as pushing out partially-updated MVs compromises downstream assumptions.
\section{Conclusion}\label{sec:conclusion}
In this work, we focus on addressing the fundamental challenge of OOD detection tasks, which is how to fully understand the semantic discrepancy between the ID/OOD samples. We reveal that the key to success in the realistic SCOOD task is to allocate as many ID samples in the unlabeled set correctly as possible. To this end, we propose a novel uncertainty-aware optimal transport scheme that introduces class-specific energy scores as guidance for effective label assignment. Experimental results show that our method achieves better performance than previous state-of-the-art methods on SCOOD benchmarks.

\textbf{Limitations.} In addition to temperature scaling, other techniques such as feature clipping applied in ReAct~\cite{sun2021react} also enhance the performance of energy score, so how to obtain an OOD score that best fits the SCOOD task can be further explored. Moreover, a setting highly related to SCOOD has been proposed in \cite{katz2022training} and formulated as a constrained optimization problem. We will also theoretically analyze these practical OOD settings in our feature work.

% \section*{Acknowledgments}
\textbf{Acknowledgments.} 
This work is supported by National Key R\&D Program of China under Grant 2020AAA0105701, National Natural Science Foundation of China (NSFC) under Grants 61872327, Major Special Science and Technology Project of Anhui, National Natural Science Foundation of China (62033012) and Ant Group through Ant Research Intern Program.

\section{Limitations and Future Work}

We summarize the limitations we have identified for our method and propose
future research directions.

\textbf{Parallel implementation:} 
With a focus on accuracy and algorithms, our implementation for this work is
serial. Some of the most time-consuming routines in our method can easily
benefit from a parallel implementation, while the same is not obvious for the
SAP solver and the Schur complement computation. Leveraging the power of
parallelization on modern hardware for these computations is an interesting area
for future investigation.

\textbf{Rotational invariance:} 
As with all other linear constitutive models, our linearized model with lagged
rotational component is not rotationally invariant. Thus it is not suitable for
simulation of extreme deformations using large time steps. For those scenarios,
we fall back to traditional nonlinear models with Hessian positive definite
corrections proposed in \cite{bib:teran2005robust}.

\textbf{Self-contact:} 
We do not consider self-contact at the moment due to the lack of support by our
geometry engine. Self-contact can be incorporated into our method by updating the
geometry engine to augment the set of contacts reported.

\textbf{Tunneling at high speeds:} Though our method has a lower computational
cost, it could benefit from continuous collision detection strategies
\cite{bib:li2020ipc} to provide constraints before contact is established. This
would allow to mitigate issues such as objects tunneling past each other at high
speeds. Efficient solution to mitigate this issue is a topic of active research
for the authors.

\textbf{Redundant constraints:} Our geometry engine often introduces a large
number of constraints to resolve contact. Similarly, welding a large number of
deformable mesh vertices to a rigid body (as done in Section
\ref{sec:bubble_gripper}) introduces many constraints. Even though our SAP
solver \cite{bib:castro2022unconstrained} provides existence and uniqueness
guarantees, a large number of constraints hurts performance as can be observed
in the \emph{Soft-bubble} example. We are currently investigating strategies to
significantly reduce the number of constraints without sacrificing accuracy.


\section*{Acknowledgements}
We would like to thank Kelvin Guu and Srini Narayanan, as well as our anonymous reviewers, for their helpful feedback on a previous version of this manuscript.

\bibliography{anthology,custom}
\bibliographystyle{acl_natbib}

\clearpage

\appendix

\section{Training Details}
\label{sec:training_appendix}
All models are initialized from the public T5-1.1-XXL checkpoints.\footnote{https://github.com/google-research/text-to-text-transfer-transformer}

\subsection{Intermediate Training}
\label{sec:intermediate}
We use 256 Cloud TPU v3 cores for the intermediate training procedure using a batch size of 2048 and the fixed default learning rate of 0.001. Training generally proceeds for one epoch, which is between 100-150K steps depending on the precise task, though we used early stopping for the TSM and TSM+SSM models based on MC-TACO performance, as they seem to overfit.

\subsection{Finetuning}
\label{sec:finetuning}
We use 64 Cloud TPU v3 cores for finetuning and inference on all tasks. For MC-TACO, Natural Questions, and SituatedQA, we use the same fixed learning rate of 0.001 and train for 10K steps with batch size 128. For TimeDIAL, we attempt to follow their training setup more closely, and use a lower learning rate (0.0001) and train for up to 100K steps, still with batch size 128 and use early stopping on the validation set to inform when to stop. For most of the models that had intermediate training, the early stopping point was for 10K steps. However, for the basic T5 model, it was after 20K steps (improving from ($82.97 \rightarrow 84.51$)), implying that it can overcome its lack of intermediate training with additional finetuning data. Note that in the zero-shot variant, no finetuning is done.

\section{Further Discussion}
\label{app:discussion_appendix}
\subsection{Other Experiments}
We previously experimented with the T5~Large and T5-XL models, as well the as 1.0 versions of the T5 models that were first described in \citet{raffel-etal-2020-exploring}. In general, larger models and the 1.1 versions worked better. While we refrain from reporting results due to inconsistent setups, in general the smaller models were notably worse, such that distinguishing between two similar setups (such as TSM and SSM) was difficult on many tasks. While we know of no work testing salient span masking on extremely large models, it is possible it would actually show a larger impact, based on this trend. While left-to-right decoding serves as an awkward fit for the paradigm, if our hypothesis on the reason why SSM works is correct, then it should not prove to be a substantial hurdle. See also below for more discussion on said hypothesis.

\begin{table}[t!]
\begin{center}
\begin{tabular}{lc}
\toprule
Temporal Type & Number of sentences\\  
\midrule
Date & 56,520,912 \\
Duration & 8,182,819 \\
Set & 1,797,929 \\
Time & 2,281,198 \\
\bottomrule
\end{tabular}
\caption{TSM data statistics: The above table describes the distribution of temporal spans in the English Wikipedia data, which comprises of 121M sentences. 
}
\label{tab:tsmdata}

\end{center}
\end{table}
\begin{table}[t!]
\begin{center}
\begin{tabular}{lc}
\toprule
Span Type & Number of sentences\\  
\midrule
Entity & 78,139,341 \\
Date & 32,023,769 \\
\bottomrule
\end{tabular}
\caption{SSM data statistics: The above table describes the distribution of salient spans in the Wikipedia data as processed by \cite{guu-etal-2020-realm}, which comprises of 82M sentences. Each row denotes the number of sentences that contain at least one of the respective span.
}
\label{tab:ssmdata}

\end{center}
\end{table}
\begin{table}[t!]
\begin{center}
\begin{tabular}{lcc}
\toprule
TSM Span Type & \multicolumn{2}{c}{SSM Span Type}  \\  
& Named Entity & Date \\
\midrule
Date & 29,771,242 & - \\
Duration & 5,159,229 & 3,144,592\\
Set & 1,226,333 & 844,222\\
Time & 915,084 & 411,436\\
\bottomrule
\end{tabular}
\caption{We apply \textsc{SUTIME} on the SSM training data \cite{guu-etal-2020-realm} to investigate how many sentences contain temporal information. Each column denotes the number of sentences that contain the SSM identified span (e.g. \textit{named entity} or \textit{date})  and each row denotes the number of those sentences in which \textsc{SUTIME} identified the corresponding temporal span.
Number of sentences with at least one \textit{named entity}: 78,139,341 \\
Number of sentences with at least one \textit{date}: 32,023,769 (Table \ref{tab:ssmdata})
}
\label{tab:ssmdatatsmsutime}
\end{center}
\end{table}

\subsection{Span Distribution vs. Text Distribution}
Our hypothesis for SSM's effectiveness is due to it oversampling difficult sentences. This is based on the performance gain for the \textsc{Entities} intermediate training as well as the number of temporal spans that occur in the SSM training data.
\autoref{tab:ssmdatatsmsutime} shows the results of \textsc{SUTIME} parser on the SSM training data, and as we can see, significant portion of the SSM data (45\%) has temporal spans.
 \autoref{tab:ssmdatatsmsutime} shows the breakdown of different temporal spans for each SSM salient span type.
 We leave an exact test of this for future work, but if this is true, then we might expect left-to-right decoding models to also benefit from the sampling procedure of SSM, even though they do not use a masked language modeling paradigm for training. 


\subsection{Span Types in Table \ref{tab:tsmdata}}\label{sec:table2}
Mapping MC-TACO's span types is somewhat helpful to see the performance breakdown. Note that these are now based on individual answers, while MC-TACO's strict match metric is based on correctly labeling all answers for a given question.

\paragraph{Entities} While MC-TACO is a common sense dataset, it frequently relies on reasoning about relatively complicated phenomena. While it is common sense to know that a dynasty does not rule in China for only a few minutes, it is still required to know more about China and dynasties to answer the question correctly. TimeDIAL on the other hand is normally ordinary conversations that are not very entity-centric. SituatedQA is derived from Natural Questions, which is an information seeking dataset that frequently features entities.

\paragraph{Duration} MC-TACO's event duration maps well to the Duration type in \textsc{sutime}. While there may be some SituatedQA examples that include durations, we do not filter for them.

\paragraph{Set} MC-TACO's Frequency type asks question of the "How often" nature while sets frequently have answer types of that nature e.g., "every third sunday", but this is not a perfect mapping.

\paragraph{Date} MC-TACO's typical time sometimes includes dates, but it is less likely to be a specific date and more likely to be a generic date like Sunday, rather than a specific knowledge-based date. SituatedQA questions always include dates that decontextualize Natural Questions. 

\paragraph{Time} MC-TACO's typical time sometimes corresponds with times as well, but they are again less likely to be specific. Unfortunately, Date and Time are not separated in MC-TACO. 

\paragraph{Other MC-TACO Types} Note that we did not include the ``Stationarity'' or the ``Event Ordering'' MC-TACO types in the breakdown, as they do not correspond well to any \textsc{sutime} type. 



\end{document}
