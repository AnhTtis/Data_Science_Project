\documentclass[aps,twocolumn,showpacs,preprintnumbers,amsmath,amssymb,superscriptaddress,floatfix,nofootinbib]{revtex4-2}

\usepackage{graphicx}
\usepackage{epsfig}
\usepackage{epstopdf}
\usepackage{subfigure}

\usepackage{amsmath}
\usepackage{amsfonts}
\usepackage{amssymb}
\usepackage{color}
\usepackage[colorlinks,
            citecolor=blue,
            anchorcolor=red,
            menucolor=red,
            linkcolor=red,
            filecolor=red,
            runcolor=red,
            urlcolor=blue,
            frenchlinks=red]{hyperref}
            
\newcommand{\gls}[1]{{\color{blue} #1}}
\newcommand{\com}[1]{{\color{red}\bf{#1}}}
\newcommand*{\slashed}[1]{{#1\!\!\!/}}

%%%%%%%%%%%%%%%%%%%%%%%

\begin{document}
\title{The mass spectrum and strong decay properties of the charmed-strange mesons within Godfrey-Isgur model considering the coupled-channel effects}

\author{Jing-Jing Yang}
\affiliation{School of Physics and Microelectronics, Zhengzhou University, Zhengzhou, Henan 450001, China}
  
\author{Wei Hao}
\email{haowei2020@itp.ac.cn}
\affiliation{CAS Key Laboratory of Theoretical Physics, Institute of Theoretical Physics, Chinese Academy of Sciences, Beijing 100190,China}
\affiliation{School of Physical Sciences, University of Chinese Academy of Sciences (UCAS), Beijing 100049, China}

\author{Xiaoyu Wang}
\affiliation{School of Physics and Microelectronics, Zhengzhou University, Zhengzhou, Henan 450001, China}\vspace{0.5cm}
	
\author{De-Min Li}
\affiliation{School of Physics and Microelectronics, Zhengzhou University, Zhengzhou, Henan 450001, China}\vspace{0.5cm}
 
\author{Yu-Xiao Li}
\affiliation{School of Physics and Microelectronics, Zhengzhou University, Zhengzhou, Henan 450001, China}\vspace{0.5cm}

\author{En Wang}
\email{wangen@zzu.edu.cn}
\affiliation{School of Physics and Microelectronics, Zhengzhou University, Zhengzhou, Henan 450001, China}\vspace{0.5cm}
\date{\today}


\begin{abstract}
Motivated by the recently observed $D_{s0}(2590)$ state by LHCb, we investigate the mass spectrum and the strong decay properties of the charmed-strange mesons within Godfrey-Isgur model considering the coupled-channel effects. Our results support that $D_{s0}^*(2317)$ and $D_{s1}(2460)$ can be interpreted as the $D_{s}(1^3P_0)$ and $D_{s}(1^3P_1)$ states with larger $DK$ and $D^*K$ components, respectively, and $D_{s1}(2700)$, $D_{s1}(2536)$, $D^*_{s2}(2573)$, $D_{s1}^*(2860)$, $D_{s3}^*(2860)$, and $D_{sJ}^*(3040)$ can be well interpreted as the $D_s(2^3S_1)$, $D_s(1^1P_1)$, $D_s(1^3P_2)$, $D_s(1^3D_1)$, $D_s(1^3D_3)$, and $D_s(2^1P_1)$ states, respectively. Although, $D_{s0}(2590)$ mass is about 50~MeV less than our prediction for the $D_{s}(2^1S_0)$ state, its width is still in good agreement with the one of $D_{s}(2^1S_0)$. Therefore, $D_{s0}(2590)$ state needs to be further confirmed by the experimental measurements, and the more precise information about $D_{s0}(2590)$ will shed light on its assignment of $D_{s}(2^1S_0)$. Furthermore, we predict the masses and the strong decay properties of the charmed-strange mesons with masses around 3~GeV, which would be helpful to experimentally search for these states.      
\end{abstract}

\maketitle

\section{Introduction}
Recently, LHCb Collaboration observed a new excited $D_{s0}(2590)$ state with mass $M=2591\pm6\pm7$~MeV and width $\Gamma=89\pm16\pm12$~MeV in the $D^{+}K^{+}\pi^{-}$ mass distribution of the $ B^{0} \rightarrow D^{-}D^{+}K^{+}\pi^{-}$ decay using a data sample with integrated luminosity of 5.4 $ fb^{-1} $ at centre-of-mass energy of 13~TeV, and its spin-parity was determined to be $J^P=0^-$~\cite{LHCb:2020gnv}. According to the Review of Particle Physics (RPP)~\cite{ParticleDataGroup:2022pth}, there are several charmed-strange mesons, which contain the $D$, $D^*$, $D_{s0}^*(2317)$, $D_{s1}^*(2536)$, $D_{s1}(2460)$, $D_{s2}^*(2573)$, $D_{s1}^*(2860)$, $D_{s3}^*(2860)$, $D_{s1}^*(2700)$, and $D_{sJ}(3040)$, and there have been many studies about the charmed-strange mesons~\cite{Rosner_2007,Segovia:2015dia,Chen:2016spr,Song:2015nia,Song:2014mha,Li:2009qu,Li:2007px}. Although the newly observed $D_{s0}(2590)$ was suggested to be the candidate of the $D_{s0}(2S)$ state by LHCb~\cite{LHCb:2020gnv}, it still draws particular attention on the spectrum of the charmed-strange mesons due to the fact that the mass of the observed $D_{s0}(2590)$ is about 80~MeV less than the $D_{s0}(2S)$ mass predicted by the conventional quark models~\cite{Ni:2021pce,Ebert:2009ua,Zeng:1994vj,Godfrey:2015dva,Godfrey:1985xj}. 

In Ref.~\cite{Wang:2021orp}, the authors investigated the mass and the strong decay width of $D_{s0}(2590)$ and concluded that $D_{s0}(2590)$ can hardly be interpreted as the $D_{s}(2^1S_0)$ state. In Ref.~\cite{Xie:2021dwe}, it was shown that the $P$-wave $D^*K$ interaction plays an important role to obtain the mass and width of $D_{s0}(2590)$. In Ref.~\cite{Ortega:2021fem}, $D_{s0}(2590)$ can be regarded as a $D_{s}(2^1S_0)$ state plus the important effect of the nearby meson-meson thresholds by performing a coupled-channel calculation including the $ D^{(*)}K^{(*)}$, $D^{(*)}_s \omega $, and $ D^{(*)}_s\eta $ channels. In Ref.~\cite{Gao:2022bsb}, the mass and width of the $D_{s0}(2590)$ were studied within the Godfrey-Isgur (GI) relativistic quark model including screening effects and the $^{3}P_{0} $ model, which supports the interpretation of the $D_s(2^1S_0)$. In addition, Ref.~\cite{Hao:2022vwt} made a systematic calculation of the spectrum and strong decays of the charmed-strange system in a coupled-channel framework, and the mass and width of the $D_{s0}(2590)$ could be reasonably described. 

As we known, BaBar and CLEO observed $D_{s0}^*(2317)$ and $D_{s1}(2460)$ in the $D^+_s\pi^0$ channel~\cite{BaBar:2003oey,CLEO:2003ggt}, which implies there is a large violation of the isospin conservation. In addition, the masses of the observed $D_{s0}^*(2317)$ and $D_{s1}(2460)$ resonances are much lower than the corresponding predictions from the conventional quark models \cite{Godfrey:1985xj,DiPierro:2001dwf,Godfrey:2015dva} and the Lattice QCD calculations~\cite{Bali:2003jv,Dougall:2003hv}. It motivates many interpretations for their structure, such as compact $[cq][\bar{s}\bar{q}]$ tetraquark, molecular states, and the mixing of the $c\bar{s}$ and other component~\cite{Guo:2017jvc,Chen:2022asf,Oset:2016lyh,Albaladejo:2018mhb}, which implies that more components are necessary to describe the properties of the charmed-strange mesons.

%~\cite{Maiani:2004vq,Browder:2003fk}, $D^{(*)}K$ molecules~\cite{Barnes:2003dj,vanBeveren:2003kd}. 

%such as molecular states \cite{Barnes:2003dj,Xie:2010zza,Feng:2012zze,Wang:2012bu,Ortega:2016mms}, tetraquark states \cite{Cheng:2003kg,Browder:2003fk,Dmitrasinovic:2004cu, Dmitrasinovic:2005gc,Dmitrasinovic:2012zz,Maiani:2004vq,Bracco:2005kt,Kim:2005gt,Wang:2006uba,Terasaki:2003qa,Hayashigaki:2004st,Nielsen:2005zr,Terasaki:2006qd}, conventional charmed-strange mesons with coupled channel effects \cite{vanBeveren:2003jv,vanBeveren:2003kd,Coito:2011qn,Hwang:2004cd,Simonov:2004ar,Lee:2004gt,Guo:2007up,Zhou:2011sp,Badalian:2007yr,Dai:2006uz} or conventional charmed-strange mesons \cite{Dai:2003yg,Sadzikowski:2003jy,Fayyazuddin:2003aa,Song:2015nia,Liu:2015uya}.

Although the general potential models, such as GI relativistic quark model~\cite{Godfrey:1985xj}, could provide an good description for most of the meson spectra, the coupled-channel effects (or the pair-creation effects), which were usually neglected, will manifest as a coupling to meson-meson (meson-baryon) channels and lead to mass shifts. It has been shown that the coupled-channel effects play an important role for describing the mesons spectra, such as charmonium~\cite{Kalashnikova:2005ui,Li:2009ad,Ferretti:2013faa}, bottomonium~\cite{Liu:2011yp,Ferretti:2012zz,Ferretti:2013vua,Lu:2016mbb}, and charmed-strange mesons~\cite{vanBeveren:2003jv,vanBeveren:2003kd,Coito:2011qn,Hwang:2004cd,Simonov:2004ar,Lee:2004gt,Guo:2007up,Zhou:2011sp,Badalian:2007yr,Dai:2006uz}. Therefore, in this work we will investigate the mass spectrum of the charmed-strange mesons within GI relativistic quark model by taking into account the mass shifts from the coupled-channel effect, where the potential model parameters will be refitted. 

The rest of the paper is organized as follows. In Sec.~\ref{sec:model}, we will present our theoretical models, including the coupled-channel model and the GI relativistic quark model. In Sec.~\ref{sec:results}, the numerical results will be presented. We will conclude the work and give the summary in Sec.~\ref{sec:summary}.







\section{The theoretical models}
\label{sec:model}

\subsection{The coupled-channel model}
\label{sec:couple}

In the coupled-channel model, the Hamiltonian of a meson system is defined as \cite{Ferretti:2012zz,Ferretti:2013faa,Ferretti:2013vua,Ferretti:2015rsa}
\begin{equation}
		H=H_{0}+H_{BC}+H_{I} \label{eq:hamilton}
\end{equation}
where $H_0$ connects with the bare mass $M_0$ of the meson $A$, and is obtained from  the GI model. $H_{BC}$ is the Hamiltonian of the $BC$ pair, $BC$ is the intermediate virtual state composed of meson B and C coupled to the meson $A$, and $H_I$ describes the coupling of the meson state
$|A\rangle$ to the intermediate meson-meson continuum $|BC\rangle$, and  connects with the mass shifts $\Delta M$ from the coupled-channel effects.

The Hamiltonian $H_{0}$ of the GI model will give rise to the bare mass $M_0$ of the meson $A$
	\begin{eqnarray}
		H_{0}|A\rangle=M_{0}|A\rangle,
	\end{eqnarray}
and we will discuss the Hamiltonian $H_{0}$ of the GI model in next subsection. As done in Refs.~\cite{Hao:2022vwt,Xie:2021dwe}, we assume there is no interaction between the $BC$ pair, and only the kinetic energy of the intermediate $BC$ pair will be considered. The Hamiltonian $H_{BC}$ can be written as the sum of the kinetic energies of $B$ and $C$, and the Schr\"odinger equation for the $BC$ pair is derived as follows
	\begin{eqnarray}
		H_{BC}|BC\rangle=E_{BC}(p)|BC; p \rangle,
	\end{eqnarray}
	\begin{eqnarray}
		E_{BC}(\boldsymbol{p})=\sqrt{m_{B}^{2}+p^{2}}+\sqrt{m_{C}^{2}+p^{2}},
	\end{eqnarray}
where $p$ is the center of mass momentum of the meson pair $BC$ running from 0 to infinity, $m_B$ and $m_C$ are the masses of the meson $B$ and $C$, respectively.

Taking into account the coupled-channel effects $H_I$, the Schr\"odinger equation with Hamiltonian given in Eq.~(\ref{eq:hamilton}) can be written as
	\begin{eqnarray}
		H|\psi\rangle=M|\psi\rangle,
	\end{eqnarray}
where $|\psi\rangle$ is the eigen wave function of the system, which could be expressed as
	\begin{eqnarray}
		|\psi\rangle =a_{0}|A\rangle+\sum_{BC}\int d^{3}p \, c_{BC}(p)|BC, p\rangle,
	\end{eqnarray}
where the coefficients $ a_{0} $ and $c_{BC} $ are the normalized constants of the corresponding wave functions of $q\Bar{q}$ bare state and $BC$ component, respectively. 
 

Thus, the physical mass $M$ in the coupled-channel model is given by
\begin{gather}
M = M_0 + \Delta M, \label{m} \\
\Delta M = \sum_{BC\ell J} \int_0^{\infty} p^2 dp \mbox{ } \frac{\left|\left\langle BC;p \right| T^\dag \left| A \right\rangle \right|^2}{M - E_{BC}+i\epsilon}, \label{deltam}
\end{gather}
where $\left\langle BC;p \right| T^\dag \left| A \right\rangle$ is the transition amplitude for the operator $T^\dag$ between the intermediate state $|BC\rangle$ and the meson $A$. $BC$ has various channels and the sum runs over all the channels we will consider in this work. $\ell$ is the orbital angular momentum, and the total angular momentum is $J=J_B+J_C+\ell$.
In our calculation, we adopt the quark-antiquark pair-creation operator $T^\dag$ from the $^3P_0$ model~\cite{Ferretti:2013faa,Ferretti:2012zz,Ferretti:2013vua}, which could be expressed as
\begin{equation}
	\label{eqn:Tdag}
	\begin{array}{rcl}
	T^{\dagger} &=& -3 \, \gamma_0^{eff} \, \int d \boldsymbol{p}_3 \, d \boldsymbol{p}_4 \, 
	\delta(\boldsymbol{p}_3 + \boldsymbol{p}_4) \,  
	{e}^{-r_q^2 (\boldsymbol{p}_3 - \boldsymbol{p}_4)^2/6 }\,  \\
	& &  C_{34}  F_{34}  \left[ \chi_{34} \, \times \, {\cal Y}_{1}(\boldsymbol{p}_3 - \boldsymbol{p}_4) \right]^{(0)}_0 \, 
	b_3^{\dagger}(\boldsymbol{p}_3) \, d_4^{\dagger}(\boldsymbol{p}_4) ~,   
	\end{array}
\end{equation}
where $C_{34}$, $F_{34}$ and $\chi_{34}$ are the color-singlet wave function, flavor-singlet wave function, and spin-triplet wave function for the created quark and antiquark pair $q\bar{q}$, respectively. $ b_3^{\dagger}(\boldsymbol{p}_3)$ and $d_4^{\dagger}(\boldsymbol{p}_4)$ are the creation operators for a quark and an antiquark with three-momenta $\boldsymbol{p}_3$ and $\boldsymbol{p}_4$, respectively.  $\gamma_0^{eff}=\frac{m_n}{m_i}\gamma_0$ is the effective pair-creation strength, whose value is obtained by fitting to the strong decay of $D_{s2}^*(2573)$ in our calculation, which is well interpreted as the $D_s(1^3P_2)$ state. $m_n$ refers to the light quark mass $m_u$, while $m_i$ refers to the quark masses of $u$, $d$, or $s$. In the $^3P_0$ model, the operator $T^{\dagger}$ creates a pair of constituent quarks with an  effective size, the pair-creation point has to be smeared out by a Gaussian factor, whose width $r_q$ was determined from meson decays to be in the range $0.25\sim 0.35$~fm~\cite{Silvestre-Brac:1991qqx,Geiger:1991ab,Geiger:1991qe,Geiger:1996re}. In our calculation, we take the value $r_q = 0.3$ fm. 


Under the Simple Harmonic Oscillator (SHO) approximation, the meson wave function in the momentum space can be expressed as
\begin{eqnarray}
\psi_{nLM_L}^{\text{SHO}}(p)=R_{nL}^{\text{SHO}}(p)Y_{LM_L}(\Omega_p),
\end{eqnarray}
where the radial wave function is given by
\begin{eqnarray}
R_{nL}^{\text{SHO}}(p)=\frac{(-1)^n(-i)^L}{\beta^{3/2}}\sqrt{\frac{2n!}{\Gamma(n+L+3/2)}}\nonumber\\
\times\left(\frac{p}{\beta}\right)^Le^{-(p^2/2{\beta}^2)}L_n^{L+(1/2)}
\left(\frac{p^2}{\beta^2}\right),
\end{eqnarray}
here $\beta$ is the SHO wave function scale parameter, and $L_n^{L+(1/2)}\left(\frac{p^2}{\beta^2}\right)$ is an associated Laguerre polynomial. The corresponding parameters are tabulated in Table~\ref{3p0parameter}. 

\begin{table}[htpb] 
\caption{Parameters of the coupled-channel model.}
\label{3p0parameter}
\begin{center}
\begin{tabular}{cc} 
\hline 
\hline 
Parameter  &  Value     \\ 
\hline 
$\gamma_0$ & $0.478$       \\  
$\beta$   & $0.4$ GeV   \\  
$r_q$      & $0.3$ fm    \\
$m_n$      & $0.33$ GeV   \\
$m_s$      & $0.55$ GeV   \\
$m_c$      & $1.50$ GeV    \\   
\hline 
\hline
\end{tabular}
\end{center}
\end{table}



If the mass of the initial meson $A$ is above the threshold of coupled-channel $BC$, the strong decays of $A\to B C$ will happen, and the strong decay width can be expressed as
\begin{equation}
	\Gamma_{A \rightarrow BC} = \Phi_{A \rightarrow BC}(p_0) \sum_{\ell, J} 
	\left| \left\langle BC,p_0, \ell J \right| T^\dag \left| A \right\rangle \right|^2 \mbox{ },
\end{equation}
 where $\Phi_{A \rightarrow BC}(p_0)$ is the standard relativistic phase space factor \cite{Ackleh:1996yt,Barnes:2005pb}
 
\begin{equation} 
	\Phi_{A \rightarrow BC} = 2 \pi p_0 \frac{E_B(p_0) E_C(p_0)}{m_A}  \mbox{ },
\end{equation}
depending on the relative momentum $p_0$ between $B$ and $C$ and on the energies of the two intermediate state mesons, 

\begin{gather}
p_0=\frac{\sqrt{\left[m_A^2-(m_B+m_c)^2\right]\left[m_A^2-(m_B-m_c)^2\right]} }{2m_A},\\
 E_B(p_0) = \sqrt{m_B^2 + p_0^2},\\   
E_C(p_0) = \sqrt{m_C^2 + p_0^2}.
\end{gather}

For the initial states below the threshold of the coupled-channels, the possibilities of each meson-meson continuum components can be calculated by
\begin{align}
	P_{BC} =& \left[1+\sum_{BC\ell J} \int_0^{\infty} p^2 dp \mbox{ } \frac{\left|\left\langle BC;p \right| T^\dag \left| A \right\rangle \right|^2}{(M - E_{BC})^2}\right]^{-1} \\
    &\int_0^{\infty} p^2 dp \mbox{ } \frac{\left|\left\langle BC;p \right| T^\dag \left| A \right\rangle \right|^2}{(M - E_{BC})^2} .
\end{align}
The sum in the formula means all the intermediate states we considered. And the $cs$ components can be calculated by $1-P_{BC}$. 

For the coupled-channels, we consider ground state mesons, which include $DK$, $DK^*$, $D^*K$, $D^*K^*$, $D_s\eta$, $D_s\eta^\prime$, $D_s\phi$, $D_s^*\eta$, $D_s^*\eta^\prime$, and $D_s^*\phi$, as Refs.~\cite{Kalashnikova:2005ui,Ferretti:2013faa,Ferretti:2012zz,Ferretti:2013vua,Lu:2016mbb}. The physical mass $M$ and the mass shift $\Delta M$ can be simultaneously determined from Eqs.~(\ref{m}) and (\ref{deltam}). The masses of the mesons used in this work are taken from RPP~\cite{ParticleDataGroup:2022pth}.


\subsection{GI Relativistic quark model}
As mentioned above, the bare mass $M_0$ in Eq.~(\ref{m}) is calculated by potential model, which is the Godfrey-Isgur relativistic quark model in this work~\cite{Godfrey:1985xj}. In the GI model, the Hamiltonian of a meson system is defined as~\cite{Godfrey:1985xj}
\begin{eqnarray}
\tilde{H}&=&(p^2+m_1^2)^{1/2}+(p^2+m_{2}^2)^{1/2} +  \tilde{H}_{12}^{\text{conf}} \nonumber \\ 
&&+\tilde{H}_{12}^{\text{hyp}}+\tilde{H}_{12}^{\text{so}},
\label{ha}
\end{eqnarray}
where $\tilde{H}^\text{conf}_{12}$ is spin-independent potential; $\tilde{H}^\text{hyp}_{12}$ is color-hyperfine interaction which includes tensor hyperfine potintial $\tilde{H}^\text{tensor}_{12}$ and contact hyperfine potential $\tilde{H}^\text{c}_{12}$; $\tilde{H}^\text{so}_{12}$ is spin-orbit interaction which includes vector spin-orbit potential $\tilde{H}^\text{so(v)}_{12}$ and scalar spin-orbit potential $\tilde{H}^\text{so(s)}_{12}$.
In this subsection, we will first discuss the Hamiltonian terms in the non-relativistic limit, and then modified the terms to introduce the relativistic effects.
Hereafter, we will denote the terms with a tilde to be the ones considering relativistic effects, otherwise the terms without the tilde to be the ones in the non-relativistic limit.

In non-relativistic limit, the spin-independent potential $\tilde{H}^\text{conf}_{12}$ of Eq.~(\ref{ha}) will be expressed as $H_{12}^{\text{conf}}$,
\begin{eqnarray}
\tilde{H}_{12}^{\text{conf}} &\to &  H_{12}^{\text{conf}}  =G(r)+S(r),  \\
 G(r)&=&\frac{\alpha_s(r)}{r} \boldsymbol{F_1} \cdot \boldsymbol{F_2}, ~~  S(r)= br+c,
 \label{hyp}
\end{eqnarray}
where $G(r)$ stands for the short-range one-gluon-exchange potential, and $S(r)$ corresponds to the long-range confinement. The parameters $b$ and $c$ are constants, and $\alpha_s(r)$ is the running coupling constant of QCD. $\boldsymbol{F}$ is related to the Gell-Mann matrix by $\boldsymbol{F}_1=\boldsymbol{\lambda}_1/2$ for quarks and $\boldsymbol{F}_2=-\boldsymbol{\lambda}^*_2/2$ for antiquarks, with
$\langle\boldsymbol{F}_1\cdot\boldsymbol{F}_2\rangle=-4/3$ for mesons.

The color-hyperfine interaction $\tilde{H}^\text{hyp}_{12}$ of Eq.~(\ref{ha}) could be expressed as $H^{\text{hyp}}_{12}$ in the non-relativistic limit
\begin{eqnarray}
\tilde{H}_{12}^{\text{hyp}} &\to& H^{\text{hyp}}_{12} \nonumber \\
&=&-\frac{\alpha_s(r)}{m_1m_2}\Bigg[\frac{8\pi}{3}\boldsymbol{S}_1\cdot\boldsymbol{S}_2\delta^3 (\boldsymbol r)+\frac{1}{r^3}\nonumber\\
&&\Big(\frac{3\boldsymbol{S}_1\cdot\boldsymbol{r} \boldsymbol{S}_2\cdot\boldsymbol{r}}{r^2}
 -\boldsymbol{S}_1\cdot\boldsymbol{S}_2\Big)\Bigg] \boldsymbol{F_1} \cdot \boldsymbol{F_2}.
\end{eqnarray}

The spin-orbit interaction $\tilde{H}^{\text{so}}_{12}$ of Eq (\ref{ha}) will be expressed as $H^{\text{so}}_{12}$ in the non-relativistic limit,
\begin{eqnarray}
\tilde{H}_{12}^{\text{so}} \to H^{\text{so}}_{12}=H^{\text{so(cm)}}_{12}+H^{\text{so(tp)}}_{12},
\end{eqnarray}
where $H^{\text{so(cm)}}_{12}$ is the
color-magnetic term, and $H^{\text{so(tp)}}_{12}$ is the Thomas-precession term, i.e.,
\begin{eqnarray}
H^{\text{so(cm)}}_{12}&=&-\frac{\alpha_s(r)}{r^3}\left(\frac{1}{m_1}+\frac{1}{m_2}\right)\left(\frac{\boldsymbol{S}_1}{m_1}+\frac{\boldsymbol{S}_2}{m_2}\right) \cdot \boldsymbol{L}(\boldsymbol{F}_1\cdot\boldsymbol{F}_2), \nonumber \\
\end{eqnarray}
\begin{eqnarray}
H^{\text{so(tp)}}_{12}=\frac{-1}{2r}\frac{\partial H^{\text{conf}}}{\partial
r}\Bigg(\frac{\boldsymbol{S}_1}{m^2_1}+\frac{\boldsymbol{S}_2}{m^2_2}\Bigg)\cdot \boldsymbol{L}.
\end{eqnarray}

In the above expressions, $\boldsymbol{S}_1$ and $\boldsymbol{S}_2$ denote the spin of the quark and antiquark, respectively, and $\boldsymbol{L}$ is the orbital momentum between quark and antiquark.



In the GI model, the relativistic effects are introduced in two ways. Firstly, the smearing transformation is used for the non-relativistic potentials $G(r)$ and $S(r)$,
\begin{eqnarray}
    \tilde{f}(r)=\int f(r)  \rho(\boldsymbol{r}-\boldsymbol{r}')d^3\boldsymbol{r'},  \\
     \rho(\boldsymbol{r}-\boldsymbol{r}')=\frac{\sigma^3}{\pi^{3/2}}e^{-\sigma^2(\boldsymbol{r}-\boldsymbol{r}')^2}.
\end{eqnarray}
where $\rho(\boldsymbol{r}-\boldsymbol{r}')$ is smearing function, $\sigma$ is a  parameter, $f(r)$ represents $G(r)$ and $S(r)$.

Secondly, since the reflection of relativistic effects lies in the momentum dependence
of interactions between quark and anti-quark, the potentials will be modified by the momentum-dependent factor as
\begin{eqnarray}
\tilde{G}(r) &\to& \left(1+\frac{p^2}{E_1 E_2}\right)^{1/2} \tilde{G}(r) \left( 1+\frac{p^2}{ E_1 E_2}\right)^{1/2}, \\
\frac{\tilde{V}_i(r)}{m_1 m_2} &\to& \left( \frac{m_1 m_2}{E_1 E_2}\right)^{1/2+\epsilon_i}  \frac{\tilde{V}_i(r)}{m_1 m_2} \left( \frac{m_1 m_2}{E_1 E_2}\right)^{1/2+\epsilon_i},\nonumber \\ \label{eqGV}
\end{eqnarray}
where $E_1=(p^2+m^2_1)^{1/2}$, $E_2=(p^2+m^2_2)^{1/2}$ are the energies of the quark and antiquark in
the mesons. The index $i$ in the parameters $\tilde{V}_i(r)$ and $\epsilon_i$ corresponds to different types of interaction in Eq. (\ref{ha}), including $i=$ contact(c), tensor(t) vector spin-orbit[so(v)] and scalar spin-orbit[so(s)] potentials.
The details of these  effective potentials can be found in Ref.~\cite{Godfrey:1985xj}.






\section{Results and discussions}
\label{sec:results}

\begin{table}[htpb] 
\caption{Parameters of Godfrey-Isgur model.} 
\label{giparameter}
\begin{center}
\begin{tabular}{cc} 
\hline 
\hline 
Parameter  &  Value     \\ 
\hline 
$b$                   & $0.1614$ GeV$^2$      \\  
$c$                   & $0.0725$ GeV   \\  
$\sigma_0$            & $3.2666$ GeV    \\
$s$                   & $2.4980$   \\
$\epsilon_c$          & $-0.0788$    \\
$\epsilon_t$          & $0.6443$     \\
$\epsilon_{\mathrm{so(v)}}$      & $-0.2511$    \\
$\epsilon_{\mathrm{so(s)}}$      & $0.9001$    \\  
\hline 
\hline
\end{tabular}
\end{center}
\end{table}


\begin{table*}[htpb]
\begin{center}
\caption{\label{tab:dsm} The mass spectrum (in MeV) of charmed-strange mesons. Column 5 shows the our predicted masses. Column 6 and 7 show the GI model results and experimental values, respectively.}
\footnotesize
\begin{tabular}{ccccccc}
\hline\hline
  $n^{2S+1}L_J$  & states               &$M_0$  &$\Delta M$  &$M~(\mathrm{this~work})$     &GI \cite{Godfrey:1985xj}   & PDG~\cite{ParticleDataGroup:2022pth}  \\\hline
  $1^1S_0$     & $D_{s}$              &$2163$   &$-195$     &$1968$   &$1960$   &$1968.34\pm0.07$    \\
  $1^3S_1$     & $D_{s}^{*}$          &$2334$   &$-221$     &$2112$   &$2130$   &$2112.2\pm0.4$    \\
  $2^1S_0$     & $D_{s0}(2590)$       &$2859$   &$-213$     &$2646$   &$2670$   &$2591\pm6\pm7$\cite{LHCb:2020gnv}           \\
  $2^3S_1$     & $D_{s1}^*(2700)$     &$2922$   &$-201$     &$2722$   &$2730$   &$2714\pm5$                      \\
  $1^3P_0$     & $D_{s0}^*(2317)$     &$2540$   &$-223$     &$2316$   &$2480$   &$2317.8\pm0.5$      \\
  $1^1P_1$     & $D_{s1}(2536)$       &$2773$   &$-269$     &$2504$   &$2530$   &$2535.11\pm0.06$       \\ 
  $1^3P_1$     & $D_{s1}(2460)$       &$2700$   &$-244$     &$2456$   &$2570$   &$2459.5\pm0.6$     \\
  $1^3P_2$     & $D_{s2}^*(2573)$     &$2847$   &$-278$     &$2569$   &$2590$   &$2569.1\pm0.8$     \\
  $2^3P_0$     &                      &$3075$   &$-175$     &$2899$   &       &        \\
  $2^1P_1$     &$D_{sJ}^*(3040)$      &$3221$   &$-151$     &$3069$   &       &$3044\pm 8^{+30}_{-5}$   \\
  $2^3P_1$     &                      &$3166$   &$-187$     &$2979$   &       &              \\
  $2^3P_2$     &                      &$3278$   &$-153$     &$3134$   &       &        \\
  $1^3D_1$     & $D_{s1}^*(2860)$     &$3030$   &$-184$      &$2846$   &$2900$   &$2859\pm27$    \\
  $1^1D_2$     &                      &$3112$   &$-253$      &$2858$   &       &       \\
  $1^3D_2$     &                      &$3092$   &$-239$      &$2853$   &       &      \\
  $1^3D_3$     & $D_{s3}^*(2860)$     &$3154$   &$-286$      &$2868$   &$2920$   &$2860\pm7$       \\
  \hline\hline

\end{tabular}
\end{center}
\end{table*}





\begin{table*}
\caption{\label{tab:shift} Mass shift (in MeV) from the coupled-channels. } 
\begin{tabular}{ccccccccccccc} 
\hline 
\hline \\
State                & $DK$  & $DK^*$ & $D^*K$ & $D^*K^*$ & $D_s\eta$  & $D_s\eta^\prime$  & $D_s\phi$  & $D_s^*\eta$ &$D_s^*\eta^\prime$  &$D_s^*\phi$ & Total \\\hline
\hline \\
$1^1S_0$             &$0$        &$-39$    &$-44$    &$-71$    &$0$   &$0$   &$-10$    &$-8$    &$-3$   &$-19$   &$-195$  \\  
$1^3S_1$             &$-20$      &$-30$    &$-34$    &$-92$    &$-3$  &$-1$  &$-8$     &$-6$    &$-2$   &$-24$   &$-221$  \\ 
$2^1S_0$             &$0$        &$-46$    &$-62$    &$-73$    &$0$   &$0$   &$-8$     &$-8$    &$-2$   &$-14$   &$-213$  \\ 
$2^3S_1$             &$-2$       &$-38$    &$-28$    &$-96$    &$-3$  &$-1$  &$-6$     &$-7$    &$-2$   &$-17$   &$-201$  \\   
$1^3P_0$             &$-66$      &$0$      &$0$      &$-122$   &$-5$  &$-2$  &$0$      &$0$     &$0$    &$-28$   &$-223$  \\ 
$1^1P_1$             &$0$        &$-52$    &$-89$    &$-87$    &$0$   &$0$   &$-10$    &$-9$    &$-3$   &$-19$   &$-269$  \\      
$1^3P_1$             &$0$        &$-40$    &$-66$    &$-98$    &$0$   &$0$   &$-8$    &$-7$    &$-2$   &$-22$   &$-244$  \\ 
$1^3P_2$             &$-44$        &$-39$    &$-50$    &$-100$    &$-6$   &$-2$   &$-8$    &$-7$    &$-3$   &$-20$   &$-278$  \\ 

$2^3P_0$             &$-19$        &$0$    &$0$    &$-133$    &$-3$   &$-3$   &$0$    &$0$    &$0$   &$-18$   &$-175$  \\ 
$2^1P_1$             &$0$        &$-30$    &$-13$    &$-79$    &$0$   &$0$   &$-8$    &$-4$    &$-3$   &$-15$   &$-151$  \\    
$2^3P_1$             &$0$        &$-29$    &$-23$    &$-107$    &$0$   &$0$   &$-10$    &$-3$    &$-2$   &$-14$   &$-187$  \\  
$2^3P_2$             &$-10$        &$-11$    &$-8$    &$-80$    &$-1$   &$-1$   &$-7$    &$-2$    &$-2$   &$-31$   &$-153$  \\    

$1^3D_1$             &$13$        &$-19$    &$0$    &$-146$    &$-1$   &$-1$   &$-2$    &$-1$    &$0$   &$-26$   &$-184$  \\  
$1^1D_2$             &$0$        &$-66$    &$-53$    &$-98$    &$0$   &$0$   &$-9$    &$-8$    &$-3$   &$-16$   &$-253$  \\ 
$1^3D_2$             &$0$        &$-62$    &$-35$    &$-106$    &$0$   &$0$   &$-8$    &$-7$    &$-2$   &$-19$   &$-239$  \\  
$1^3D_3$             &$-44$        &$-43$    &$-52$    &$-109$    &$-6$   &$-2$   &$-7$    &$-6$    &$-2$   &$-15$   &$-286$  \\ 
\hline 
\hline
\end{tabular}
\end{table*}



\begin{table*}[htpb]
\begin{center}
\caption{ \label{tab:dsdeday1}The strong decay widths (in MeV) of charmed-strange mesons. The symbol `$-$' in the table means the the decay mode is forbidden or there is no experimental information.}
\footnotesize
\begin{tabular}{cccccccccccccc}
\hline\hline
  $n^{2S+1}L_J$  & states         & PDG~\cite{ParticleDataGroup:2022pth} & $DK$  & $DK^*$ & $D^*K$ & $D^*K^*$ & $D_s\eta$  & $D_s\eta^\prime$  & $D_s\phi$  & $D_s^*\eta$ &$D_s\eta^\prime$  &$D_s^*\phi$ & Total \\\hline
  $2^1S_0$   & $D_{s0}(2590)$     &$89\pm16\pm12$\cite{LHCb:2020gnv}    &$-$      & $-$     & 87   & $-$   & $-$  & $-$  &$-$  &$-$  &$-$  &$-$   &$87$ \\
  $2^3S_1$     & $D_{s1}^*(2700)$ &$122\pm10$       & $32$    &$-$      & $77$    &$-$     & 6    &$-$   &$-$   & 3   &$-$  &$-$  & $119$ \\
  $1^3P_0$     & $D_{s0}^*(2317)$ &$<3.8$           &$-$      &$-$     &$-$      &$-$     &$-$   &$-$   &$-$   &$-$  &$-$  &$-$  &$-$     \\
  $1^1P_1$     & $D_{s1}(2536)$  &$0.92\pm0.05$    &$-$      &$-$      & $10$    &$-$     &$-$   &$-$   &$-$   &$-$  &$-$  &$-$  & $10$  \\
  $1^3P_1$     & $D_{s1}(2460)$   &$<3.5$           &$-$      &$-$      &$-$      &$-$     &$-$   &$-$   &$-$   &$-$  & $-$  &$-$  &$-$ \\
  $1^3P_2$     & $D_{s2}^*(2573)$ &$16.9\pm0.7$     & $15$    &$-$      & $1$     &$-$     &$-$   &$-$   &$-$   &$-$  &$-$  &$-$  & $17$  \\
  $1^3D_1$     & $D_{s1}^*(2860)$ &$159\pm80$       & $46$    & $27$    & $35$    &$-$     & 7   &$-$    &$-$   & 3   &$-$  &$-$  & $118$  \\ 
  $1^1D_2$     &  $-$             &$-$              &$-$      &38       &62       &$-$     &$-$   &$-$     &$-$    &4     &$-$   &$-$   &104    \\
  $1^3D_2$     &  $-$             &$-$              &$-$      & 53      & 75      &$-$       &$-$   &$-$      &$-$     & 6    &$-$     &$-$      &134    \\
  $1^3D_3$     & $D_{s3}^*(2860)$ &$53\pm10$        & $39$    & $2$     & $22$    & $-$     & 2   &$-$   &$-$   &$-$  &$-$  &$-$  & $65$  \\
  \hline\hline

\end{tabular}
\end{center}
\end{table*}


\begin{table}[htpb]
\begin{center}
\caption{\label{tab:dsdecay2} The decay width~(in MeV) of $2P$ charmed-strange mesons. The symbol `$-$' in the table means the decay mode is forbidden or there is no experimental information.}
\footnotesize
\begin{tabular}{lccccccccc}
\hline\hline

  Channel               &$2^3P_0$   &$2^1P_1$              &$2^3P_1$   &$2^3P_2$    \\   
                        &$-$        & $D_{sJ}^*(3040)$   &$-$           &$-$         \\\hline
  $DK$                  &52         &$-$                 &$-$           &1           \\
  $DK^*$                &$-$        &67                  &25            &45             \\
  $D^*K$                &$-$        &50                  &50             &17          \\
  $D^*K^*$              &3          &77                  &29           &93        \\
  $D_s\eta$             &1          &$-$                 &$-$           &3           \\
  $D_s\eta^\prime$      &$-$        &$-$                 &$-$           &2          \\
  $D_s\phi$             &$-$        &7                   &$-$            &6           \\
  $D_s^*\eta$           &$-$        &8                   &3             &6          \\
  $D_s^*\eta^\prime$    &$-$        &$-$                   &$-$             &$-$            \\
  $D_s^*\phi$           &$-$        &$-$                &$-$            &12          \\
  $DK^*_0(1430)$        &$-$        &$-$                &$-$           &$-$           \\
  $DK_{1B}$             &$-$        &$-$                &$-$             &1           \\
  $DK_{1A}$             &$-$        &$-$                &$-$           &$-$            \\
  $DK^*_2(1430)$        &$-$        &$-$                &$-$           &$-$          \\
  $D_0^*(2300)K$        &$-$        &$-$                &$-$           &$-$         \\
  $D_1(2420)K$          &$-$        &$-$                &12             &8            \\
  $D_1(2430)K$          &$-$        &$-$                &1          &3            \\
  $D_2^*(2460)K$        &$-$        &38                 &5           &18            \\
  Total                 &57         &247                &127           &215       \\
  Exp.                  & $-$       &$239\pm35^{+46}_{-42}$         &$-$           & $-$       \\
  \hline\hline

\end{tabular}
\end{center}
\end{table}









\begin{table*}
\caption{\label{tab:probabilities} Probabilities ($\%$) of every coupled-channel component and bare $c\Bar{s}$ component for the charmed-strange mesons below threshold. } 
\begin{tabular}{ccccccccccccccc} 
\hline 
\hline \\
 State      &     & $DK$  & $DK^*$ & $D^*K$ & $D^*K^*$ & $D_s\eta$  & $D_s\eta^\prime$  & $D_s\phi$  & $D_s^*\eta$ &$D_s^*\eta^\prime$  &$D_s^*\phi$ & $P_{molecule}$   &$P_{c\bar{s}}$ \\\hline
\hline \\
$1^1S_0$    &$D_s$            &$0$   &$3$   &$4$   &$4$   &$0$   &$0$   &$1$   &$1$   &$0$   &$1$   &$15$   &$85$ \\  
$1^3S_1$    &$D_{s}^{*}$      &$2$   &$2$   &$3$   &$6$   &$0$   &$0$   &$0$   &$0$   &$0$   &$1$   &$17$   &$83$ \\  
$1^3P_0$    &$D_{s0}^*(2317)$ &$29$   &$0$   &$0$   &$7$   &$1$   &$0$   &$0$   &$0$   &$0$   &$1$   &$38$   &$62$ \\  
$1^3P_1$    &$D_{s1}(2460)$   &$0$   &$4$   &$23$   &$6$   &$0$   &$0$   &$0$   &$1$   &$0$   &$1$   &$36$   &$64$ \\  
\hline 
\hline
\end{tabular}
\end{table*}
In this section, we present our numerical calculation results of the mass spectrum and the strong decay widths for the charmed-strange mesons. Firstly, the free parameters of the GI model are obtained by fitting to the masses of the charmed-strange mesons, which are listed in Table~\ref{giparameter}. In our fitting, the input states include the $D_s$ $(1^1S_0)$, $D_s^*$ $(1^3S_1)$, $D_{s0}^*(2317)$ $(1^3P_0)$, $D_{s2}^*(2573)$ $(1^3P_2)$, $D_{s1}^*(2860)$ $(1^3D_1)$, $D_{s3}^*(2860)$ $(1^3D_3)$, $D_{s1}^*(2700)$ $(2^3S_1)$ and $D_{s0}(2590)$ $(2^1S_0)$. Then, the physical masses and the corresponding mass shifts of the mesons could be simultaneously determined, which are shown in Table~\ref{tab:dsm}. We also present the mass shifts of every channel in Table~\ref{tab:shift}. The strong decay widths of the charmed-strange mesons are given in Table~\ref{tab:dsdeday1} and Table \ref{tab:dsdecay2}. In addition, the probabilities of every coupled-channel and $c\bar{s}$ component for the states below the threshold are listed in Table~\ref{tab:probabilities}. 


Taking into account the coupled-channel effect, the masses of the ground states $D_s$ and $D_s^*$ are determined to be 1968~MeV and 2112~MeV, which are in good agreement with the experimental data. According to Table~\ref{tab:probabilities}, the probabilities of the $c\bar{s}$ component are 85\% and 83\% for the  $D_s$ and $D_s^*$, respectively, which implies that the $c\bar{s}$ component is dominant. 

The masses of the $D_{s}(1^3P_0)$ and $D_{s1}(1^3P_1)$ states are numerically estimated to be 2316~MeV and 2456~MeV, which are in good agreement with the experimental data of the $D_{s0}^*(2317)$ and $D_{s1}(2460)$, both of which have large non-$c\bar{s}$ component. 
The biggest non-$c\bar{s}$ component of $D_{s0}^*(2317)$ is $DK$, and the one of $D_{s1}(2460)$ is $D^*K$, in consistent with the phenomenological results of Ref.~\cite{Ortega:2016mms}. 
 
 %of for $D_{s0}^*(2317)$ is about $29\%$ which is lower than lattice  calculation \cite{MartinezTorres:2014kpc}, $72\pm13\pm5\%$.
%And the biggest non-cs component $D^*K$ for $D_{s1}(2460)$ is about $23\%$, which is also lower than the lattice data $57\pm21\pm6$\%. Although the possibilities of the states have different data from lattice, our results can comparable with other phenomenological results. For example, in Ref. , they obtained the $30\%$ $DK$ for $D_{s0}^*(2317)$ and $50\%$ $D^*K$ for $D_{s1}(2460)$.



%From Table \ref{tab:dsm}, the mass shifts of the $2^3P_0$, $2^1P_1$, $2^3P_1$ and $2^3P_2$ are far below from others. This is because the initial masses are too larger than the intermediate states' thresholds. We can see the property in Eq. \ref{deltam}. If the initial state masses are too larger than intermediate states' thresholds, the denominator of the formula will be large, that means the $\Delta m$ will be small. 

Our predictions for the mass and width of $D_{s}(2^3S_1)$ are 2722~MeV and 119~MeV, respectively, which are in good agreement with the experimental data for the $D_{s1}(2700)$ mass ($2714\pm5$~MeV) and width ($112\pm10$~MeV). The dominant decay modes are $DK$ and $D^*K$, in consistent with the experimental measurements~\cite{ParticleDataGroup:2022pth}. Our results support the interpretation of $D_{s1}(2700)$  as the $D_{s}(2^3S_1)$ state.

We predict that the mass and width of $D_{s}(1^1P_1)$ are 2504~MeV and 10~MeV, close to the experimental data of the $D_{s1}(2536)$, and the dominant decay mode is $D^*K$, in consistent with the experimental measurements of $D_{s1}(2536)$~\cite{ParticleDataGroup:2022pth},  which implies that $D_{s1}(2536)$ could be well interpreted as the $D_{s}(1^1P_1)$ state. For the $D_{s}(1^3P_2)$, the predicted mass and width are 2569~MeV and 17~MeV, in good agreement with the experimental data of the $D^*_{s2}(2573)$, and its dominant decay mode is $DK$, in consisteng with the experimental measurements of $D^*_{s2}(2573)$~\cite{ParticleDataGroup:2022pth}, which supports that the state could be will interpreted as the $D_{s}(1^3P_2)$ state.

The mass and width of the  $D_s(1^3D_1)$ are predicted to be 2846~MeV and 118~MeV, which are in good agreement with the experimentally measured $D^*_{s1}(2860)$ mass $2859\pm27$~MeV and width $159\pm 80$~MeV. In addition, the mass and width of the  $D_s(1^3D_3)$ are calculated to be 2868~MeV and 65~MeV, which are in good agreement with the $D^*_{s3}(2860)$ mass $2860\pm7$~MeV and width $53\pm 10$~MeV. The dominant decay mode of $D_s(1^3D_1)$ and $D_s(1^3D_3)$ is predicted to be $DK$, in consistent with the experimental measurements~\cite{ParticleDataGroup:2022pth}.  Our results suggest that  $D^*_{s1}(2860)$ and $D^*_{s3}(2860)$ are interpreted as the $D_s(1^3D_1)$ and $D_s(1^3D_3)$, respectively, supported by Refs.~\cite{Godfrey:2014fga,Song:2014mha,Song:2015nia}.

The $D_{sJ}(3040)$ was observed in $D^*K$ mass spectrum by  BaBar Collaboration~\cite{BaBar:2009rro}, and its mass and width are determined to be $3044\pm 8^{+30}_{-5}$~MeV and $239\pm 35^{+46}_{-42}$~MeV. Its mass is much close to the predicted mass (3069~MeV) of the $D_s(2^1P_1)$. We have calculated its strong decay width by regarding $D_{sJ}(3040)$ as $D_s(2^1P_1)$, which is 247~MeV, in good agreement with the experimental value. Thus, our study suggests that $D_{sJ}(3040)$ could be well interpreted as the $D_s(2^1P_1)$ state.




%be the $D_{s1}(2^3P_1)$ state, $3(1^+)$, $4(1^+)$ or a mixing state of $D_{s2}(1^1D_2)$ and $D_{s2}(1^3D_2)$. The $D_{s1}(2^3P_1)$ assignment is supported by the QPC model \cite{Sun:2009tg}, flux tube model   \cite{Chen:2009zt}, constituent quark model  \cite{Xiao:2014ura,Zhong:2009sk} and the effective approach  \cite{Colangelo:2010te}. The $3(1^+)$ and $4(1^+)$ assignments are studied in Ref. \cite{Segovia:2012cd}. The mixing state is supported by the effective Lagrangian approach  \cite{Colangelo:2010te}. 

The recently observed state $D_{s0}(2590)$ has the spin-parity quantum numbers $J^P=1^-$, and is suggested to be the candidate of $D_{s}(2^1S_0)$ by LHCb~\cite{LHCb:2020gnv}. One can see that the predicted  mass of $D_{s}(2^1S_0)$ in Table \ref{tab:dsm} is 2646~MeV, a little higher than experimental results, but the predicted width 87~MeV is in good agreement with the experimental value.
Indeed, in Ref. \cite{Godfrey:1985xj}, the predicted masses of $D_{s}(2^1S_0)$ are higher than $D_{s0}(2590)$. Thus, our results suggest that $D_{s0}(2590)$ could be the candidate of the $D_{s}(2^1S_0)$. Taking into account that $D_{s0}(2590)$ was only observed by LHCb, we strongly encourage the experimental side to search for this state in other processes, and the more precise information about the $D_{s0}(2590)$ could shed light on its assignment.

In addition, we have predicted the masses and the strong decay properties of the charmed-strange mesons around 3~GeV. For the $D$-wave states, the masses of the $D_s(1^1D_2)$ and $D_s(1^3D_2)$ are predicted to be 2858~MeV and 2853~MeV, respectively, and the their widths are predicted to be 104~MeV and 134~MeV. Their dominant decay modes are $D^*K$ and $DK^*$. For the $2P$ states, the masses of the $D_s(2^3P_0)$, $D_s(2^3P_1)$, and $D_s(2^3P_2)$ are predicted to be 2899~MeV, 2979~MeV, and 3134~MeV, respectively, and their widths are predicted to be 57~MeV, 127~MeV, and 215~MeV. The dominant decay mode of $D_s(2^3P_0)$ is $DK$, while the ones of $D_s(2^3P_1)$ and $D_s(2^3P_2)$ are $DK^*$, $D^*K$, and $D^*K^*$. Our predictions would be helpful to search for these states experimentally.



%Taking into account that our model could give an excellent description for other charmed-strange mesons, and the $D_{s0}(2590)$

%close to the experimental value $2591\pm6\pm7$ MeV.


%is a new observed state by LHCb \cite{LHCb:2020gnv} and it may be a  $D_{s0}(2^1S_0)$ state. So in our calculation, we also regard it as the $D_{s0}(2^1S_0)$ state. From Table\ref{tab:dsdeday1}, we can see that the decay width is close to the experimental value $89\pm16\pm12$ MeV, besides, the mass in Table \ref{tab:dsm} is also close to the experimental value $2591\pm6\pm7$ MeV. Although the theoretical values are a little higher than experimental results, it can also be a good candidate of the $D_{s0}(2^1S_0)$. Because the models we used are model dependent, and the $^3P_0$ model is a approximate model. 
 














%Among the charmed-strange meson family, the $D_{s1}(2536)$ and $D^{*}_{s2}(2573)$ are regarded as the candidates of the $1P$ states, and the $D^{*}_{s1}(2700)$, $D^{*}_{s1}(2860)$, and $D^{*}_{s3}(2860)$ can be assigned as the $D_{s}(2^3S_1)$, $D_{s}(1^3D_1)$, and  $D{s}(1^3D_3)$  states, respectively.






 


%The two 1$S$ states ($D$ and $D^*$) and the $1^3P_2$ state $D_{s2}^*(2573)$ have been established very well.

%The $D_{s0}^*(2317)$ and $D_{s1}(2460)$ were fist observed by BaBar and CLEO Collaborations in 2003 \cite{PhysRevLett.90.242001,PhysRevD.68.032002}. The observed masses of these two states are far lower than the corresponding predictions from the quark models \cite{Godfrey:1985xj,DiPierro:2001dwf} or lattice QCD calculations\cite{Bali:2003jv,Dougall:2003hv}. 

%The expection of the two states are far above the $DK$ or $D^*K$ threshold respectively. There are many explanations for these two states,  

%The $D_{s1}^*(2860)$ and $D_{s3}^*(2860)$ can be regard as $D_s(1^3D_1)$ and $D_s(1^3D_3)$ respectively. These assignments have been supported by the QPC model \cite{Godfrey:2014fga,Song:2014mha,Song:2015nia}, effective Lagrangian approach \cite{Wang:2014jua}, the constituent quark model \cite{Segovia:2015dia}, and QCD sum rule method \cite{Zhou:2014ytp}.

%The $D_{s1}^*(2700)$ is a good candidate of $D_{s1}(2^3S_1)$. The assignment is supported by the GI model \cite{Godfrey:1985xj} and the constituent quark model \cite{Segovia:2015dia}. Its strong decay behavior was also investigated by the QPC model \cite{Zhang:2006yj}. Besides, it can also be a mixing state of the $D_{s1}(2^3S_1)$ and $D_{s1}(1^3D_1)$ \cite{Close:2006gr,Li:2009qu,Chen:2011rr,Li:2007px,Zhong:2009sk} or a $DK^*$ molecule state \cite{Vinodkumar:2007hse}. 

%The $D_{sJ}(3040)$ could be the $D_{s1}(2^3P_1)$ state, $3(1^+)$, $4(1^+)$ or a mixing state of $D_{s2}(1^1D_2)$ and $D_{s2}(1^3D_2)$. The $D_{s1}(2^3P_1)$ assignment is supported by the QPC model \cite{Sun:2009tg}, flux tube model   \cite{Chen:2009zt}, constituent quark model  \cite{Xiao:2014ura,Zhong:2009sk} and the effective approach  \cite{Colangelo:2010te}. The $3(1^+)$ and $4(1^+)$ assignments are studied in Ref. \cite{Segovia:2012cd}. The mixing state is supported by the effective Lagrangian approach  \cite{Colangelo:2010te}. 




\section{Summary}
\label{sec:summary}
In this work, we have investigated the mass spectrum and the strong decay properties of the charmed-strange mesons within Godfrey-Isgur model by considering the coupled-channel effects.  The bare mass is obtained by Godfrey-Isgur model, and the mass shift from the coupled-channel effects is given by the interaction between the initial meson and the coupled-channel, with the interaction being described by quark-antiquark creation operator from $^3P_0$ model. 

Our results show that $D_{s0}^*(2317)$ and $D_{s1}(2460)$ can be interpreted as the $D_{s}(1^3P_0)$ and $D_{s}(1^3P_1)$ states with larger $DK$ and $D^*K$ components, respectively.
Comparing our theoretical predicted results with the experimental measurement, it is found that $D_{s1}(2700)$, $D_{s1}(2536)$, $D^*_{s2}(2573)$, $D_{s1}^*(2860)$, $D_{s3}^*(2860)$, and $D_{sJ}^*(3040)$ could be well interpreted as the $D_s(2^3S_1)$, $D_s(1^1P_1)$, $D_s(1^3P_2)$, $D_s(1^3D_1)$, $D_s(1^3D_3)$, and $D_s(2^1P_1)$ states, respectively. 

For the recently observed state $D_{s0}(2590)$,
although its mass is about 50~MeV less than our prediction for the $D_{s}(2^1S_0)$ state, its width is still in good agreement with that of $D_{s}(2^1S_0)$, which implies that $D_{s0}(2590)$ could be assigned as the candidate of $D_{s}(2^1S_0)$. We emphasis that $D_{s0}(2590)$ needs to be further confirmed by the experimental measurement, and the more precise information about $D_{s0}(2590)$ would shed light on the assignment of $D_{s}(2^1S_0)$.
Furthermore, we have predicted the masses and the strong decay properties of the charmed-strange mesons with masses around 3~GeV, which would be helpful to experimentally search for these states.  
 
\section{Acknowledgements}
This work is partly supported by the National Natural Science Foundation of China under Grants No. 12005191, 11905187,
the Key Research Projects of Henan Higher Education Institutions under No. 20A140027, Training Plan for Young Key Teachers in Higher Schools in Henan Province (2020GGJS017), the Academic Improvement Project of Zhengzhou University, and the Fundamental Research Cultivation Fund for Young Teachers of Zhengzhou University (JC202041042).


\bibliography{cite}  %参考文献库的名字Ref

\end{document}
