\section{Introduction}
In recent years, data-driven control based on the fundamental lemma \cite{Willems05} has received a significant amount of attention. This is evident by the abundance of works that use historical data generated by an unknown dynamical system to simulate trajectories, design controllers, and infer system properties among other applications (see \cite{Markovsky21} for a comprehensive survey). Central to the area of data-driven control, as well as adaptive control and system identification, is the notion of persistence of excitation (PE). An elegant definition of PE was given in \cite{Willems05}. Specifically, a discrete-time sequence is said to be persistently exciting of a certain order if a corresponding Hankel matrix of that sequence has full row rank. It was shown in \cite{Willems05} that when such a PE input is applied to a controllable LTI system, the resulting input-state or input-output Hankel matrices satisfy certain rank conditions. These conditions are crucial for a purely data-driven representation of linear systems as given by the fundamental lemma~\cite{Willems05}.\par
For certain classes of nonlinear systems, various works have proposed extensions and applications of the fundamental lemma using, e.g., basis functions. For instance, Hammerstein-Wiener systems \cite{Berberich20}, flat and feedback linearizable systems \cite{AlsaltiBerLopAll2021, Alsalti2022}, and control design of input-affine nonlinear systems \cite{DePersis22} have been considered. In these works, suitable PE conditions involving the sequence of basis functions (which depend on inputs and/or states/outputs) need to be satisfied. However, in \cite{Berberich20,AlsaltiBerLopAll2021,Alsalti2022,DePersis22} these PE conditions could only be verified a posteriori, i.e., after performing an experiment and collecting state/output data, and no a priori input design was proposed to this end.\par
%In recent years, various extensions and applications of the fundamental lemma have been developed, e.g., for linear parameter varying systems \cite{Verhoek21}, stochastic systems \cite{Pan22} and classes of nonlinear systems \cite{Berberich20,AlsaltiBerLopAll2021,Alsalti2022,DePersis22}. For the latter, a suitable set of basis functions has been employed in the extension to Hammerstein-Wiener systems \cite{Berberich20}, flat and feedback linearizable systems \cite{AlsaltiBerLopAll2021, Alsalti2022}, and for control design of input-affine nonlinear systems \cite{DePersis22}. In these works, suitable PE conditions involving the sequence of basis functions (which depend on inputs and/or states/outputs) need to be satisfied. However, in \cite{Berberich20,AlsaltiBerLopAll2021,Alsalti2022,DePersis22} these PE conditions could only be verified a posteriori, i.e., after performing an experiment and collecting state/output data, and no a priori input design was proposed to this end.\par
In general, there exist only few results on the design of suitable inputs that result in satisfaction of the required PE conditions, in particular for nonlinear systems. In \cite{vanWaarde22}, an \emph{online} method was proposed to design inputs that result in the desired rank conditions of \cite{Willems05} on input-state or input-output Hankel matrices for linear systems. However, the resulting input is not universal in the sense that it is tailored specifically to the system on which the experiment is performed. An extension of this result to the class of bilinear systems appeared in \cite{Yuan22}. Furthermore, \cite{DePersis21} shows how suitable scaling of the initial conditions and the input to a nonlinear system leads to satisfaction of certain rank conditions on the input-state data which enables local stabilization of an unknown nonlinear system in the first approximation.\par
The contributions of this paper are as follows. First, we propose a simple and highly sparse input sequence which satisfies the standard PE condition introduced in \cite{Willems05} for linear systems. Based on this result, as a second contribution we propose the a priori design of inputs such that desired PE conditions for nonlinear systems are satisfied. In particular, for the class of Hammerstein systems, we design inputs that guarantee persistence of excitation of any order for an arbitrary sequence of basis functions. Moreover, for single-input single-output (SISO) flat systems, we design inputs that guarantee persistence of excitation of order one for specific choices of basis functions, which then enables data-based design of stabilizing controllers (cf. \cite[Section VII.B]{DePersis22}). Finally, the third contribution addresses nonlinear systems that are locally reachable at the origin. Specifically, we show existence of sparse input sequences that guarantee collective PE of sequences of basis functions, and we propose a simple (heuristic) search algorithm to find such inputs.\par
%we show that there exist trains of impulses that guarantee collective persistence of excitation of a sequence of input- and state-dependent basis functions, 
Section \ref{sec_prel} introduces notation and necessary preliminaries. Sections \ref{sec_main}-\ref{sec_main3} contain the main contributions of the paper. Section \ref{sec_examples} illustrates the results for SISO flat systems with a numerical example and Section \ref{sec_conc} concludes the paper.