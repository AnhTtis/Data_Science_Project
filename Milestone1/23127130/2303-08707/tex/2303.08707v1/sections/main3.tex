\section{PE of locally reachable nonlinear systems}\label{sec_main3}
Consider an unknown nonlinear system of the form
\begin{equation}
	x_{k+1} = f(x_k,u_k),\label{eqn_NLsys}
\end{equation}%
with $x_k\in\mathbb{R}^n,u_k\in\mathbb{R}^m$ being the state and input vectors, respectively, and $f:\mathbb{R}^n\times\mathbb{R}^m\to\mathbb{R}^n$ is an unknown function satisfying $f(\mathbf{0},\mathbf{0})=\mathbf{0}$. For $\mu\in\mathbb{Z}_{>0}$, the set of all states which can be reached from $x_0$ in $\mu$ steps is defined as
\begin{align}
	&\mathcal{R}_\mu(x_0) \hspace{-1mm}=\hspace{-1mm} \left\lbrace \hspace{-0.5mm} x_\mu\in\mathbb{R}^n \,\left|\, \begin{aligned}
		&\exists\, u_{[0,\mu-1]},\, u_k\in\mathbb{R}^m,\\ &\textup{s.t.}\,\,  x_{k+1}\hspace{-0.5mm}=\hspace{-0.5mm}f(x_k,u_k),\,\forall k\hspace{-0.5mm}\in\hspace{-0.5mm}\mathbb{Z}_{[0,\mu-1]}.
	\end{aligned} \hspace{-0.5mm} \right. \right\rbrace \hspace{-1mm}.\notag
\end{align}
It was shown in \cite{Melkior20} how one can, under certain assumptions, obtain a guaranteed under-approximation of the reachable set of a nonlinear system with unknown dynamics. For the remainder of this section, we make the following assumption.
\begin{assumption}\label{assmp_reachability}
	For the system \eqref{eqn_NLsys}, there exists $\mu\in\mathbb{Z}_{>0}$ such that the origin is contained in the interior of $\mathcal{R}_\mu(\mathbf{0})$.
\end{assumption}%
%System \eqref{eqn_NLsys} is locally reachable at the origin, i.e., t

Assumption~\ref{assmp_reachability} implies that the system is locally reachable at the origin. A sufficient condition for local reachability at $x=\mathbf{0}$ is that the linearization of system~\eqref{eqn_NLsys} at the origin is controllable (cf. \cite[Lemma 3.7.8]{Sontag13}).

%This assumption is satisfied for sinusoidal functions, exponential functions (including, e.g., Gaussian radial basis functions) and monomials, among others \cite{Christensen06}.
Consider now $r$ basis functions $\theta_j:~\mathbb{R}^n\times \mathbb{R}^m\to\mathbb{R}$ which satisfy $\theta_j(\mathbf{0},\mathbf{0})=0$, $j\in\mathbb{Z}_{[1,r]}$, and denote the stacked vector of them by $\Theta(x_k,u_k)=\begingroup
\setlength\arraycolsep{1pt}\begin{bmatrix}\theta_1(x_k,u_k) & \cdots & \theta_r(x_k,u_k) \end{bmatrix}^\top\endgroup$. Suppose that the basis functions are linearly independent on arbitrary domains with non-empty interior\footnote{This assumption is satisfied for sinusoidal functions, exponential functions and monomials, among others \cite{Christensen06}. Note that basis functions that do not satisfy $\theta_j(\mathbf{0},\mathbf{0})=0$ can be suitably shifted by a constant.} $D_x\times D_u\subset\mathbb{R}^n\times\mathbb{R}^m$. The objective is to design inputs $\{u_k^{(j)}\}_{k=0}^{N_j-1}$ such that the sequences of basis functions $\{\hat{\Theta}_k^{(j)}\}_{k=0}^{N_j-1}$ (with $\hat{\Theta}_k^{(j)}\coloneqq\Theta(x_k^{(j)},u_k^{(j)})$) are collectively persistently exciting of order $L$. According to Definition \ref{def_cPE}, the following mosaic Hankel matrix must have full row rank
\begin{align}
	&\mathcal{H}_L(\vartheta) =\label{eqn_MosaicHankelTheta} \begin{bmatrix}
		\mathscr{H}_L(\hat{\Theta}^{(1)}) & \cdots & \mathscr{H}_L(\hat{\Theta}^{(r)})
	\end{bmatrix},
\end{align}
where $\vartheta = \begin{bmatrix}
	(\hat{\Theta}^{(1)})^\top & \cdots & (\hat{\Theta}^{(r)})^\top
\end{bmatrix}^\top$.
%where $\vartheta = \begin{bmatrix}
%	(\Theta^{(1)})^\top & \cdots & (\Theta^{(r)})^\top
%\end{bmatrix}^\top$.

Consider a submatrix of \eqref{eqn_MosaicHankelTheta} composed of the $L+\mu$ element of each of the $r$ sequences of basis functions
\begin{align}
	&W =
%	\begin{bmatrix}
%		\Theta(x_{L+\mu-1}^{(1)},u_{L+\mu-1}^{(1)}) & \cdots & \Theta(x_{L+\mu-1}^{(r)},u_{L+\mu-1}^{(r)})
%	\end{bmatrix}
	\begin{bmatrix}
		\hat{\Theta}^{(1)}_{L+\mu-1} & \hat{\Theta}^{(2)}_{L+\mu-1} & \cdots & \hat{\Theta}^{(r)}_{L+\mu-1}
	\end{bmatrix}.\label{eqn_W}
\end{align}
Similar to Section \ref{sec_Ham}, we would like $W$ to be invertible. Since the state at time $x_{L+\mu-1}^{(j)}$ is not a free variable, ensuring invertibility of $W$ is a control problem. In particular, one must select the inputs such that the corresponding state and input pairs at time $L+\mu-1$ result in an invertible matrix $W$. The following lemma shows that there exist $r$ sequences $u_{[0,L+\mu-1]}^{(j)}$, $j\in\mathbb{Z}_{[1,r]}$, such that $W$ is invertible.

\begin{lemma}\label{lemma_rexperimentsexist}
	Let Assumption \ref{assmp_reachability} be satisfied and suppose that $r$ basis functions $\theta_j:\mathbb{R}^n\times\mathbb{R}^m\to\mathbb{R}$ are linearly independent on $\mathcal{R}_\mu(\mathbf{0})\times\mathbb{R}^m$. Then there exist $r$ sequences $u_{[0,L+\mu-1]}^{(j)}$, $j\in\mathbb{Z}_{[1,r]}$, which when applied to \eqref{eqn_NLsys} starting from~$x_0^{(j)}=\mathbf{0}$, result in $x_{L+\mu-1}^{(j)}$, such that $W$ in \eqref{eqn_W} is invertible.
\end{lemma}
\begin{proof}
	Using similar arguments to the proof of Theorem~\ref{thm_alwaysexistslambda}, it can be shown by linear independence of the basis functions $\theta_j$ on $\mathcal{R}_\mu(\mathbf{0})\times\mathbb{R}^m$ that there exists $r$ pairs $(x_{L+\mu-1}^{(j)},u_{L+\mu-1}^{(j)})\in \mathcal{R}_\mu(\mathbf{0})\times\mathbb{R}^m$ such that $W$ is invertible. 
	
	Since $f(\mathbf{0},\mathbf{0})=\mathbf{0}$, then starting from zero initial conditions and setting the input to $u^{(j)}_{[0,L-2]}=0$, one can express the state $x_{L+\mu-1}^{(j)}$ in terms of $u^{(j)}_{[L-1,L+\mu-2]}$ only, i.e., $x_{L+\mu-1}^{(j)} = f(f\cdots(f(\mathbf{0},u_{L-1}^{(j)}),u_{L}^{(j)})\cdots,u_{L+\mu-2}^{(j)})$. Finally, since the system is locally reachable by Assumption~\ref{assmp_reachability}, there exist inputs $u_{[L-1,L+\mu-1]}^{(j)}$, $j\in\mathbb{Z}_{[1,r]}$, which steer the system from $x_{L-1}^{(j)}=\mathbf{0}$ to $x_{L+\mu-1}^{(j)}$ in $\mu$~steps.
\end{proof}

Lemma \ref{lemma_rexperimentsexist} shows that there exist $r$ input sequences such that $W$ is invertible. Notice that these inputs are sparse. Specifically, each input only has (at most) $\mu+1$ non-zero elements. However, these inputs cannot be found offline as in \eqref{eqn_feasibility} since that would require knowledge of the unknown function $f$. As an alternative, we propose a (heuristic) search algorithm to find these $r$ sequences. The idea is to apply random inputs from rest which increase the (column) rank of the matrix $W$. This is summarized in Algorithm~\ref{alg_onlinePE}.

The following theorem shows how to use the results of Lemma \ref{lemma_rexperimentsexist} to obtain collectively persistently exciting sequences of basis functions of any order $L$.
\begin{theorem}\label{thm_generalPE}
	Let Assumption~\ref{assmp_reachability} hold. Given $L,\mu\in\mathbb{Z}_{>0}$ and $r$ basis functions $\theta_j:\mathbb{R}^n\times\mathbb{R}^m\to\mathbb{R}$ that are linearly independent on $\mathcal{R}_\mu(\mathbf{0})\times\mathbb{R}^m$ and satisfy $\theta_j(\mathbf{0},\mathbf{0})=0$, let $N_j\geq2L+\mu-1$ for $j\in\mathbb{Z}_{[1,r]}$. Furthermore, let the sequences $\{u_k^{(j)}\}_{k=0}^{N_j-1}$ take the form
	\begin{equation}
		u_k^{(j)} = \begin{cases}
			\eta^{(j)}_{[0,\mu]}, \quad & k\in\mathbb{Z}_{[L-1,L+\mu-1]}, \quad \forall j\in\mathbb{Z}_{[1,r]},\\
			\mathbf{0}, \quad &\textup{otherwise},
		\end{cases}\label{eqn_PElocallyreachable}
	\end{equation}
	where $\eta^{(j)}_{[0,\mu]}$ are such that the resulting matrix $W$ in \eqref{eqn_W} is invertible. If \eqref{eqn_PElocallyreachable} are applied to \eqref{eqn_NLsys} starting from $x_0^{(j)}=\mathbf{0}$, then for the matrix in \eqref{eqn_MosaicHankelTheta} it holds that $\textup{rank}(\mathcal{H}_L(\vartheta)) = rL$.
\end{theorem}
\begingroup
\setlength{\textfloatsep}{0pt}
\begin{figure}[t]
	\vspace{-0.5em}
	\begin{algorithm}[H]
		\caption{Random search algorithm}\label{alg_onlinePE}
		\textbf{Input:} $r$ linearly independent basis functions $\theta_j$\\
		1) {Initialize} an empty matrix $W=[\,]$\\
		2) \textbf{for} $j=1$ \textbf{to} $r$ \textbf{do}
		\begin{itemize}
			\item[] \textbf{while} rank$(W)\neq j$ \textbf{do}
			\begin{itemize}
				\item[$\bullet$] Randomly choose an input $\eta^{(j)}_{[0,\mu]}$. Apply $\eta^{(j)}_{[0,\mu-1]}$ to system \eqref{eqn_NLsys} starting from rest and collect $x_{\mu}^{(j)}$.
				\item[$\bullet$] \textbf{if} rank$\left(\begin{bmatrix}
					W & \Theta(x_{\mu}^{(j)},\eta_{\mu}^{(j)})
				\end{bmatrix}\right)=j$
				\begin{itemize}
					\item[] $W\leftarrow\begin{bmatrix}
						W & \Theta(x_{\mu}^{(j)},\eta_{\mu}^{(j)})
					\end{bmatrix}$.
				\end{itemize}
				\textbf{end if}
			\end{itemize}
			\textbf{end while}
			%				\item[] Set $W \leftarrow M$.
		\end{itemize}
		3) \textbf{end for}\\
		\textbf{Output:} $r$ sequences $\eta^{(j)}_{[0,\mu]}$ which make $W$ invertible.
	\end{algorithm}
	\vspace{-0.75cm}
\end{figure}
\endgroup
\begin{proof}
	Each block row of \eqref{eqn_MosaicHankelTheta} has $r$ linearly independent columns given by the columns of $W$. Notice that each $\mathscr{H}_L(\hat{\Theta}^{(j)})$, $j\in\mathbb{Z}_{[1,r]}$, in \eqref{eqn_MosaicHankelTheta} has a lower block-anti-triangular structure due to $f(\mathbf{0},\mathbf{0})=\mathbf{0},\,\Theta(\mathbf{0},\mathbf{0})=\mathbf{0}$ and the choice of the inputs \eqref{eqn_PElocallyreachable}. As a result, every block row \eqref{eqn_MosaicHankelTheta} is linearly independent from the others. Since there are $L$ such block rows, it holds that rank$(\mathcal{H}_L(\vartheta))=rL$.
\end{proof}
%This implies that each block row of \eqref{eqn_MosaicHankelTheta} contains a different number of zero columns. 
%\begin{remark}
%	Note that if $L\leq\mu$, then one can perform $r$ experiments of length $\bar{L}>\mu$ to obtain a full row rank mosaic Hankel matrix $\mathcal{H}_{\bar{L}}(\vartheta)$ using Theorem~\ref{thm_generalPE}. Using the obtained sequences $\{\Theta(x_k^{(j)},u_k^{(j)})\}_{k=0}^{2\bar{L}-1}$, one can now build $\mathcal{H}_{L}(\vartheta)$ which is guaranteed to have full row rank.
%\end{remark}

In \cite{AlsaltiBerLopAll2021,Alsalti2022,DePersis22}, verifying PE of a sequence of basis functions requires performing a single (long) experiment of length $N~\geq~(r+1)L-1$, then checking the rank condition a posteriori. In contrast, Theorem \ref{thm_generalPE} only uses $r$ input sequences, each containing only $\mu+1$ non-zero elements, in order to build a PE sequence of basis functions. Future work will focus on formulating an experiment design procedure for PE of basis functions, i.e., to show how to \textit{select} the inputs in Lemma \ref{lemma_rexperimentsexist} online rather than using Algorithm \ref{alg_onlinePE}, which is not guaranteed to return the desired sequences. To do so, one must impose additional assumptions on the \textit{class} of systems and the \textit{choice} of basis functions under consideration. In this context, the results of \cite{vanWaarde22} and \cite{Yuan22} for LTI and bilinear systems can be interpreted as guaranteeing PE (online) of basis functions of the form $[u^\top\,x^\top]^\top$ and $[u^\top\,x^\top\,(u\otimes x)^\top]^\top$, respectively.