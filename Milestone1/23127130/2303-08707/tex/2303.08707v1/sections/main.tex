\section{Guaranteed PE for LTI systems}\label{sec_main}
Consider an unknown controllable LTI system of the form
\begin{equation}
	\begin{aligned}
		x_{k+1} = Ax_k + Bu_k,\qquad y_k = Cx_k + Du_k,
	\end{aligned}\label{eqn_LTIsys}
\end{equation}
where $x_k\in\mathbb{R}^n,\,u_k\in\mathbb{R}^m,\,y_k\in\mathbb{R}^p$ are the state, input and output vectors, respectively. The fundamental lemma \cite{Willems05} is stated in the state-space framework \cite{Berberich20, vanWaarde20} as follows.
\begin{theorem}[\cite{Willems05}]
	Let \(\{u_k,y_k\}_{k=0}^{N-1}\) be an input-output trajectory of \eqref{eqn_LTIsys}. If \(\{u_k\}_{k=0}^{N-1}\) is persistently exciting of order $L+n$, then any \(\{\bar{u}_k,\bar{y}_k\}_{k=0}^{L-1}\) is a trajectory of \eqref{eqn_LTIsys}, if and only if there exists \(\beta\in\mathbb{R}^{N-L+1}\) such that
	\begin{equation}
		\begin{bmatrix} \mathscr{H}_L(u)\\ \mathscr{H}_L(y)\end{bmatrix} \beta = \begin{bmatrix}\bar{u} \\ \bar{y}\end{bmatrix}.
		\label{eqn_fundamental_lemma}
	\end{equation}\label{thm_FL}
\end{theorem}

Theorem~\ref{thm_FL} provides a data-based representation of all finite-length input-output trajectories, provided that the linear system is controllable and that the input is persistently exciting of sufficiently high order. In the following, we propose a simple and highly sparse input sequence that is guaranteed to be PE. In particular, persistence of excitation of some order $L\in\mathbb{Z}_{>0}$ is achieved by giving a pulse at each of the $m$ input channels every $L$ instants. This is formalized in the following theorem.
\begin{theorem}\label{thm_PEinputLTI}
	Let $L\in\mathbb{Z}_{>0}$ and $N\geq(m+1)L-1$. If the sequence $\{u_k\}_{k=0}^{N-1}$ takes the form
	\begin{equation}
		u_k = \begin{cases}
			e_j, \qquad &k=jL-1, \quad \forall j\in\mathbb{Z}_{[1,m]},\\
			\mathbf{0}, \qquad &\textup{otherwise},
		\end{cases}\label{eqn_PEinputLTI}
	\end{equation}
	then it holds that $\textup{rank}(\mathscr{H}_L(u))=mL$.
\end{theorem}
\begin{proof}
	Without loss of generality, let $N=(m+1)L-1$. The following matrix has full row rank
	\begin{equation*}
		\mathscr{H}_{L}(u) \hspace{-0.5mm}= \hspace{-0.5mm}\begin{bmatrix}
			\mathbf{0} & \mathbf{0} & \cdots & e_1 & \cdots & \mathbf{0} & \mathbf{0} & \cdots  & e_m\\
			\vdots & \vdots & \reflectbox{$\ddots$} & \vdots &\cdots & \vdots& \vdots & \reflectbox{$\ddots$} & \vdots\\
			\mathbf{0} & e_1 & \cdots & \mathbf{0} & \cdots & \mathbf{0} & e_m & \cdots & \mathbf{0}\\
			e_1 & \mathbf{0} & \cdots & \mathbf{0} & \cdots & e_m & \mathbf{0} & \cdots &\mathbf{0}
		\end{bmatrix}\hspace{-0.5mm}.
	\end{equation*}%
\end{proof}

Theorem \ref{thm_PEinputLTI} provides an explicit formula \eqref{eqn_PEinputLTI} for an input which is guaranteed to be persistently exciting in the sense of Definition \ref{def_PE}. This can be used to apply the results of Theorem~\ref{thm_FL}. For PE of order $L+n$, the input in \eqref{eqn_PEinputLTI} must have length $N\geq(m+1)(L+n)-1$, which is the minimal required length such that an input $u$ can be PE of order $L+n$ (cf. Definition~\ref{def_PE}). In contrast, the online procedure from \cite{vanWaarde22} directly designs an input such that the resulting input-output data matrix in \eqref{eqn_fundamental_lemma} has rank $mL+n$ and requires $N=(m+1)L+n-1$ samples to this end, implying that it is  sample efficient. The input \eqref{eqn_PEinputLTI}, although not sample efficient, guarantees PE independently of the considered system (contrary to the online procedure of \cite{vanWaarde22}). Finally, an appealing feature of the input \eqref{eqn_PEinputLTI} is its high sparsity. Specifically, only $m$ non-zero elements (out of at least $(m+1)L-1$) are enough to obtain a PE input of order $L$. This can be useful in, e.g., networked systems with bandwidth constraints~\cite{Siami21}.
\begin{remark}
	In \cite{Coulson22}, a quantitative notion of persistence of excitation was proposed in order to establish a robust version of Theorem~\ref{thm_FL}. There, an input is said to be $\alpha-$PE of order $L$ if the minimum singular value of the Hankel matrix of the input is lower bounded by $\alpha>0$. Notice that for the input proposed in \eqref{eqn_PEinputLTI}, all singular values of the corresponding Hankel matrix are equal to one. By scaling \eqref{eqn_PEinputLTI} by $\alpha$, one obtains an input which is guaranteed to be $\alpha-$PE.
\end{remark}

In Section \ref{sec_main2}, we consider two classes of nonlinear systems and give conditions under which an input is guaranteed to result in a persistently exciting sequence of basis functions that depend on the input and state (or output) of the system.