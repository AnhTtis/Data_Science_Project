\documentclass[letterpaper, 10 pt, conference]{ieeeconf}
\usepackage{algorithm,algorithmic}
\usepackage[noadjust]{cite}
\usepackage{amssymb,mathrsfs, mathtools}%amsthm
\setcounter{MaxMatrixCols}{20}
\DeclareMathOperator*{\argmax}{arg\,max}
\DeclareMathOperator*{\argmin}{arg\,min}
\DeclarePairedDelimiter\ceil{\lceil}{\rceil}
\DeclarePairedDelimiter\floor{\lfloor}{\rfloor}
\DeclarePairedDelimiter{\norm}{\lVert}{\rVert}
\newtheorem{theorem}{Theorem}
\newtheorem{lemma}{Lemma}
\newtheorem{assumption}{Assumption}
\newtheorem{remark}{Remark}
\newtheorem{example}{Example}
\newtheorem{corollary}{Corollary}
\newtheorem{definition}{Definition}
\usepackage{algorithm}
\usepackage{algorithmic}
\usepackage{diagbox}
\IEEEoverridecommandlockouts
\overrideIEEEmargins

\makeatletter
\let\NAT@parse\undefined
\newcommand{\removelatexerror}{\let\@latex@error\@gobble}
\makeatother
\usepackage{hyperref}  %hyperref still needs to be put at the end!


\begin{document}
\title{\LARGE \bf On the design of persistently exciting inputs for data-driven control of linear and nonlinear systems}%\huge


\author{Mohammad Alsalti, Victor G. Lopez, and Matthias A. Müller %
	\thanks{Leibniz University Hannover, Institute of Automatic Control, 30167 Hannover, Germany. E-mail:\{\href{maitlo:alsalti@irt.uni-hannover.de}{alsalti},\href{maitlo:lopez@irt.uni-hannover.de}{lopez},\href{maitlo:mueller@irt.uni-hannover.de}{mueller}\}@irt.uni-hannover.de}%
	\thanks{This work has received funding from the European Research Council (ERC) under the European Union’s Horizon 2020 research and innovation programme (grant agreement No 948679).
	}
}

	\maketitle
	\thispagestyle{empty}
	\pagestyle{empty}
	
	
	\begin{abstract}%
		In the context of data-driven control, persistence of excitation (PE) of an input sequence is defined in terms of a rank condition on the Hankel matrix of the input data. For nonlinear systems, recent results employed rank conditions involving collected input and state/output data, for which no guidelines are available on how to satisfy them a priori. In this paper, we first show that a set of discrete impulses is guaranteed to be persistently exciting for any controllable LTI system. Based on this result, for certain classes of nonlinear systems, we guarantee persistence of excitation of a sequence of basis functions \textit{a priori}, by design of the physical input only. Finally, for nonlinear systems which are locally reachable at the origin, we show that there exist sparse input sequences that guarantee collective PE of sequences of basis functions.
	\end{abstract}
		
	\section{Introduction}
\label{sec:introduction}
% \begin{itemize}
%     % Diffusion of FL
%     \item {\st{Diffusion of FL}}
%     % Security threats to FL
%     \item {\st{Security threats to FL with particular focus on model poisoning}}
%     % Limitations of existing countermeasures
%     \item {\st{Current countermeasures (e.g., KRUM) and their limitations}}
%     % Proposed method and its advantages
%     \item {\st{Intuitive description of the proposed method and its difference (i.e., advantages) w.r.t. state of the art}}
%     % Main contributions
%     \item {\st{Summary of the main contributions of this work}}
%     % Paper's structure and organization
%     \item {\st{Paper's structure and organization}}
% \end{itemize}

% Diffusion of FL
Recently, {\em federated learning} (FL) has emerged as the leading paradigm for training distributed, large-scale, and privacy-preserving machine learning (ML) systems~\cite{mcmahan2017googleai,mcmahan2017aistats}. 
The core idea of FL is to allow multiple edge clients to collaboratively train a shared, global model without disclosing their local private training data.
%Specifically, an FL system consists of a central server and many edge clients; 
A typical FL round involves the following steps: {\em(i)} the server randomly picks some clients and sends them the current, global model; {\em(ii)} each selected client locally trains its model with its own private data; then, it sends the resulting local model to the server;\footnote{Whenever we refer to global/local model, we mean global/local model {\em parameters}.} {\em(iii)} the server updates the global model by computing an \emph{aggregation function}, usually the average (FedAvg), on the local models received from clients.
% \begin{enumerate}
%     \item[{\em(i)}] the server sends the current, global model to the clients and appoints some of them for training;
%     \item[{\em(ii)}] each selected client locally trains its copy of the global model with its own private data; then, it sends the resulting local model back to the server;\footnote{Whenever we refer to global/local model, we mean global/local model {\em parameters}.}
%     \item[{\em(iii)}] the server updates the global model by computing an \emph{aggregation function} on the local models received from clients (by default, the average, also referred to as FedAvg~\cite{mcmahan2017aistats}).
% \end{enumerate}
This process goes on until the global model converges. %(e.g., after a certain number of rounds or other similar stopping criteria).
%\\
% The advantages of FL over the traditional, centralized learning paradigm are undoubtedly clear in terms of flexibility/scalability (clients can join/disconnect from the FL network dynamically), network communications (only model weights\footnote{We will use \textit{parameters} and \textit{weights} interchangeably.} are exchanged between clients and server), and privacy (each client's private training data is kept local at the client's end and not uploaded to the server).
\\
% Security threats to FL
%However, the growing adoption of FL also raises security concerns~\cite{costa2022covert}, particularly about its confidentiality, integrity, and availability.
Although its advantages over standard ML, FL also raises security concerns~\cite{costa2022covert}. %, particularly about its confidentiality, integrity, and availability~\cite{costa2022covert}.
% OLD, LONG VERSION
% Indeed, some work deals with privacy leakage that may expose the local data of some clients~\cite{melis2019sp}. 
% A large body of work, instead, investigates attacks that usually aim to detriment the predictive accuracy of the learned global model. For instance, \emph{data poisoning} attacks achieve this goal by letting an adversary pollute the training set of some corrupt FL clients with maliciously crafted examples~\cite{jagielski2018sp}.
% Similarly, in \emph{model poisoning} the attacker attempts to tweak the global model weights~\cite{bhagoji2019pmlr} by directly perturbing the local model's weights of some infected FL clients before these are sent to the central server for aggregation, usually via so-called Byzantine attacks. 
% It turns out that Byzantine model poisoning attacks severely impact standard FedAvg; therefore, more robust aggregation functions must be designed to make FL systems secure.
Here, we focus on \emph{untargeted model poisoning} attacks~\cite{bhagoji2019pmlr}, where an adversary attempts to tweak the global model weights %\footnote{We will use the terms \textit{parameters} and \textit{weights} interchangeably.} 
by directly perturbing the local model's parameters of some infected clients before these are sent to the central server for aggregation.
In doing so, the adversary aims to jeopardize the global model \textit{indiscriminately} at inference time.
Such model poisoning attacks severely impact standard FedAvg; therefore, more robust aggregation functions must be designed to secure FL systems.
\\
% In this paper, we focus on designing a novel robust aggregation scheme at the server's end to contrast the effect of Byzantine model poisoning attacks.
%
% Current countermeasures and their limitations
%Several countermeasures have been proposed in the literature to combat model poisoning attacks on FL systems.
% Some methods use simple statistics more robust than plain average to smooth the impact of malicious updates (e.g., Trimmed Mean and FedMedian~\cite{yin2018icml}). 
% Other defenses implement outlier detection techniques to discard malicious updates from the aggregation performed at the server's end. Those are either based on heuristics (e.g., Krum/Multi-Krum~\cite{blanchard2017nips} and Bulyan~\cite{mhamdi2018pmlr}) or data-driven approaches (e.g., K-means clustering~\cite{shen2016acm} or DnC via spectral analysis~\cite{shejwalkar2021ndss}). 
% Finally, some strategies rely on a centralized ``source of trust'' to spot potential malicious updates (e.g., FLTrust~\cite{cao2020fltrust}).
% Several countermeasures have been proposed in the literature to combat model poisoning attacks on FL systems, i.e., to discard possible malicious local updates from the aggregation performed at the server's end. 
% These techniques range from simple statistics more robust than plain average (e.g., Trimmed Mean and FedMedian~\cite{yin2018icml}) to outlier detection heuristics (e.g., Krum/Multi-Krum~\cite{blanchard2017nips} and Bulyan~\cite{mhamdi2018pmlr}) or data-driven approaches (e.g., spectral analysis via K-means clustering~\cite{shen2016acm} or spectral analysis), or methods based on ``source of trust'' (e.g., FLTrust~\cite{cao2020fltrust}).
% OLD, LONG VERSION
%Several countermeasures have been proposed in the literature to combat Byzantine model poisoning attacks on FL systems.
% Descriptive statistics
% For example, Trimmed Mean and FedMedian aggregate local model updates using more robust statistics than standard average~\cite{yin2018icml}.
%
% % Heuristics for outlier detection
% Many existing Byzantine-resilient strategies implement some outlier detection heuristics to discard the model updates sent by potentially malicious clients from the input of the aggregation function.
% One of the most popular heuristics is Krum~\cite{blanchard2017nips}.
% This strategy tries to mitigate the impact of Byzantine attacks by selecting as a global model the local model with the smallest sum of Euclidean distances to {\em all} the other local models.
% Although powerful, Krum requires the server to know (or, at least, estimate) the number of malicious FL clients upfront, which is generally impossible in a realistic attack scenario. %
% Moreover, Krum may become ineffective for complex, high-dimensional model parameter spaces due to the curse of dimensionality.
% Bulyan~\cite{mhamdi2018pmlr} tries to overcome this issue by combining Krum with a variant of Trimmed Mean.
% % Data-driven outlier detection
% Other strategies use data-driven outlier detection techniques -- e.g., via K-means clustering~\cite{shen2016acm} -- to spot potential malicious local model updates. 
% %For instance, Shen et al. propose to cluster local model updates with K-means and thus identify outliers.
%
% % Other techniques
% As far as the server is concerned, any local model received can be from a potential malicious client. 
% FLTrust~\cite{cao2020fltrust} assumes the server acts as a client, i.e., trains a local model on an additional {\em trustworthy} dataset at the server's end and compares it against all the local models from other clients. 
% This way, the server can rely on some ``source of trust'' when discarding potentially malicious clients.
%\\
% Limitations of existing Byzantine-resilient strategies
Unfortunately, existing defense mechanisms either rely on simple heuristics (e.g., Trimmed Mean and FedMedian by~\cite{yin2018icml}) or need strong and unrealistic assumptions to work effectively (e.g., foreknowledge or estimation of the number of malicious clients in the FL system, as for Krum/Multi-Krum~\cite{blanchard2017nips} and Bulyan~\cite{mhamdi2018pmlr}, which, however, cannot exceed a fixed threshold).
Furthermore, outlier detection methods using K-means clustering~\cite{shen2016acm} or spectral analysis like DnC~\cite{shejwalkar2021ndss} do not directly consider the temporal evolution of local model updates received.
Finally, strategies like FLTrust~\cite{cao2020fltrust} require the server to collect its own dataset and act as a proper client, thereby altering the standard FL protocol.
\\
% OLD, LONG VERSION
% Overall, existing Byzantine-resilient strategies are either simple heuristics (e.g., FedMedian) or, if they are more complex, they rely on strong and unrealistic assumptions to work effectively (e.g., knowing the number of malicious clients in the FL system in advance, as for Krum and alike).
% Furthermore, data-driven outlier detection methods do not consider the temporary evolution of local model updates received (e.g., K-means clustering). 
% Finally, strategies like FLTrust requires the server to collect its own dataset and act as a proper client, thereby altering the standard FL protocol.
%
% Description of the proposed method
This work introduces a novel pre-aggregation \textit{filter} robust to untargeted model poisoning attacks. Notably, this filter $(i)$ operates without requiring prior knowledge or constraints on the number of malicious clients and $(ii)$ inherently integrates temporal dependencies. 
The FL server can employ this filter as a preprocessing step before applying \textit{any} aggregation function, be it standard like FedAvg or robust like Krum or Bulyan.
Specifically, we formulate the problem of identifying corrupted updates as a multidimensional (i.e., matrix-valued) time series anomaly detection task. 
The key idea is that legitimate local updates, resulting from well-calibrated iterative procedures like stochastic gradient descent (SGD) with an appropriate learning rate, show \textit{higher predictability} compared to malicious updates. This hypothesis stems from the fact that the sequence of gradients (thus, model parameters) observed during legitimate training exhibit regular patterns, as validated in Section~\ref{subsec:intuition}. %until convergence. 
%This regularity may be more pronounced for smooth convex loss functions, but it can still be captured within an appropriate time window, even for more complex and convoluted loss surfaces. 
%We provide evidence of this claim in Appendix~B, where we show that the average mutual information (i.e., ``predictability''), calculated over pairs of legitimate model updates sent at different FL rounds, is significantly higher than the corresponding computation for a malicious client.
\\
Inspired by the matrix autoregressive (MAR) framework for multidimensional time series forecasting~\cite{chen2021je}, we propose the FLANDERS ({\em \textbf{F}ederated \textbf{L}earning meets \textbf{AN}omaly \textbf{DE}tection for a \textbf{R}obust and \textbf{S}ecure}) filter.
The main advantages of FLANDERS over existing strategies like FLDetector~\cite{zhao2020multivariate} are its resilience to large-scale attacks, where $50\%$ or more FL participants are hostile, and the capability of working under realistic non-iid scenarios.
We attribute such a capability to two key factors: $(i)$ FLANDERS works without knowing a priori the ratio of corrupted clients, and $(ii)$ it embodies temporal dependencies between intra- and inter-client updates, quickly recognizing local model drifts caused by evil players. Below, we summarize our main contributions:

\begin{itemize}
\item[{\em(i)}]
We provide empirical evidence that the sequence of models sent by legitimate clients is more predictable than those of malicious participants performing untargeted model poisoning attacks.
\\
\item[{\em(ii)}] 
We introduce FLANDERS, the first pre-aggregation filter for FL robust to untargeted model poisoning based on multidimensional time series anomaly detection.
\\
\item[{\em(iii)}] 
We integrate FLANDERS into Flower,\footnote{\scriptsize{\url{https://flower.dev/}}} a popular FL simulation framework for reproducibility.
\\
\item[{\em(iv)}] 
We show that FLANDERS improves the robustness of the existing aggregation methods under multiple settings: different datasets, client's data distribution (non-iid), models, and attack scenarios.
\\
\item[{\em(v)}] 
We publicly release all the implementation code of FLANDERS along with our experiments.\footnote{\scriptsize{\url{https://anonymous.4open.science/r/flanders_exp-7EEB}}}
\end{itemize}

% Paper's structure and organization
The remainder of the paper is structured as follows. %some related work and the current state-of-the-art solutions to security issues that FL entails. 
Section~\ref{sec:background} covers background and preliminaries. 
In Section~\ref{sec:related}, we discuss related work.
Section~\ref{sec:problem} and Section~\ref{sec:method} describe the problem formulation and the method proposed. % to tackle it. 
Section~\ref{sec:experiments} gathers experimental results. %, and Section~\ref{sec:limitations} discusses some limitations of this work.
Finally, we conclude in Section~\ref{sec:conclusion}.
 %discusses the limitations of this work and draws future research directions.
%reports conclusions and draws perspectives for future research directions.

%%%%%%% OLD %%%%%%%
%to overcome the resilience of Byzantine failures in distributed Stochastic Gradient Descent computations. 
% The strength of Krum is its time complexity, which is linear in the gradient dimension. 
% However, the robustness of the approach is guaranteed for gradient-based learning applications only when the majority of the clients are not compromised. 
% Besides, the aggregation mechanism of Krum, as well as that of similar methods, is robust from a coarse-grained perspective and does not provide solutions to errors and perturbations that may occur at inference time.
%A related approach to~\cite{blanchard2017nips} is the work of Su et al.~\cite{su2016dc}. Here, the authors propose an iterated approximate agreement to tackle a multi-layer scenario attacked by Byzantine agents. 
%However, the method works efficiently on the sole discrete context and it is inapplicable to continuous state environments.
%\gabri{Maybe, we should just talk about the main limitations of existing countermeasures without digging into their details (or, we can just mention Krum as this is the most popular one). I will move the description of all these methods to the Related Work section.} 
	\section{Notation and Preliminaries}\label{sec_prel}
Let $\mathbb{Z}_{>0}$ denote the set of positive integers and let $\mathbb{Z}_{[a,b]}$ denote the set of integers in the interval $[a,b]$. The $m\times m$ identity matrix is denoted by $I_m$ and its columns by $e_i$ for $i\in\mathbb{Z}_{[1,m]}$. We use $\mathbf{0}$ to denote a vector or a matrix of zeros of appropriate dimensions. For a sequence $\{z_k\}_{k=0}^{N-1}$ with $z_k\in\mathbb{R}^\eta$, we denote its stacked vector as $z = \begin{bmatrix}z_0^\top &z_1^\top & \dots & z_{N-1}^\top\end{bmatrix}^\top$ and a stacked window of it as $z_{[l,j]} = \begin{bmatrix}z_l^\top &z_{l+1}^\top & \dots & z_{j}^\top\end{bmatrix}^\top$ with $0\leq l<j$.\par
Persistence of excitation of a sequence and its extension to multiple sequences \cite{vanWaarde20} are defined as follows.
\begin{definition} The sequence \(\{z_k\}_{k=0}^{N-1}\), $z_k\in\mathbb{R}^{\eta}$, is said to be persistently exciting of order \(L\) if \(\textup{rank}(\mathscr{H}_{L}(z))=\eta L\), where $\mathscr{H}_L(z) = \begin{bmatrix}
		z_{[0,L-1]} & z_{[1,L]} & \cdots & z_{[N-L,N-1]}
	\end{bmatrix}$.
	\label{def_PE}
\end{definition}
\begin{definition}[\cite{vanWaarde20}]\label{def_cPE}
	The sequences $\{z_k^{(j)}\}_{k=0}^{N_j-1}$, with $z_k^{(j)}\in\mathbb{R}^\eta$ and $j\in\mathbb{Z}_{[1,r]}$, are said to be \textit{collectively persistently exciting} of order $L$ if rank$(\mathcal{H}_L(\mathscr{Z}))=\eta L$, where $\mathscr{Z} = \begin{bmatrix}
		(z^{(1)})^\top & \cdots & (z^{(r)})^\top
	\end{bmatrix}^\top,$ and
	\begin{equation*}
		\mathcal{H}_L(\mathscr{Z}) = \begin{bmatrix}
			\mathscr{H}_L(z^{(1)}) & \cdots & \mathscr{H}_L(z^{(r)})
		\end{bmatrix}.
	\end{equation*}
\end{definition}
	% This must be in the first 5 lines to tell arXiv to use pdfLaTeX, which is strongly recommended.
\pdfoutput=1
% In particular, the hyperref package requires pdfLaTeX in order to break URLs across lines.

\documentclass[11pt]{article}

% Remove the "review" option to generate the final version.
%\usepackage[review]{ACL2023}
\usepackage{ACL2023}

% Standard package includes
\usepackage{times}
\usepackage{latexsym}

% For proper rendering and hyphenation of words containing Latin characters (including in bib files)
\usepackage[T1]{fontenc}
% For Vietnamese characters
% \usepackage[T5]{fontenc}
% See https://www.latex-project.org/help/documentation/encguide.pdf for other character sets

% This assumes your files are encoded as UTF8
\usepackage[utf8]{inputenc}

% This is not strictly necessary, and may be commented out.
% However, it will improve the layout of the manuscript,
% and will typically save some space.
\usepackage{microtype}

% This is also not strictly necessary, and may be commented out.
% However, it will improve the aesthetics of text in
% the typewriter font.
\usepackage{inconsolata}


% If the title and author information does not fit in the area allocated, uncomment the following
%
%\setlength\titlebox{10cm}
%
% and set <dim> to something 5cm or larger.

%%%%%%%%%%%%%%%%%%%%%%%%%%%%%%%%%%
\usepackage{graphicx}
\usepackage{amsfonts}
\usepackage{amsmath}
\usepackage{bigdelim}
\usepackage{diagbox}
\usepackage{amsthm}
\usepackage{makecell}
\usepackage{mathtools}
\usepackage{booktabs}
\usepackage[shortlabels]{enumitem}
\graphicspath{ {figs/} }

\theoremstyle{remark}
\newtheorem*{question}{Question}

\newcommand{\tk}[1]{\textcolor{blue}{{#1}}}
\newcommand{\sy}[1]{\textcolor{red}{{#1}}}
\newcommand{\mg}[1]{\textcolor{purple}{{#1}}}
\newcommand{\lh}[1]{\textcolor{green}{{#1}}}
\newcommand{\lc}[1]{\textcolor{green}{{#1}}}

% Rounded color box
\definecolor{light_blue}{HTML}{cfdfff}
\usepackage[most]{tcolorbox}
\tcbset{on line, 
        boxsep=1pt, left=0pt,right=0pt,top=0pt,bottom=0pt,
        colframe=white,colback=light_blue,  
        highlight math style={enhanced}
        }

\newcommand{\quash}[1]{}  %Anything in \quash is ignored
\newcommand{\gpt}{\textsc{GPT-2}}
\newcommand{\bert}{\textsc{BERT}}
\newcommand{\bertlarge}{\textsc{BERT-large}}
\newcommand{\mask}{\texttt{[MASK]}}
\newcommand{\cls}{\texttt{[CLS]}}
\newcommand{\sep}{\texttt{[SEP]}}
\newcommand{\mat}{\texttt{mat}}
\newcommand{\id}{\texttt{id}}
\newcommand{\matl}{\texttt{mat}_{\ell \rightarrow \ell'}}
\newcommand{\matattnl}{\texttt{mat\_attn}_{\ell \rightarrow \ell'}}
\newcommand{\matffl}{\texttt{mat\_ffn}_{\ell \rightarrow \ell'}}
\newcommand{\matlnl}{\texttt{mat\_ln1\_ln2}_{\ell \rightarrow \ell'}}
\newcommand{\idl}{\texttt{id}_{\ell \rightarrow \ell'}}
\newcommand{\matlL}{\texttt{mat}_{\ell \rightarrow L}}
\newcommand{\matattnlL}{\texttt{mat\_attn}_{\ell \rightarrow L}}
\newcommand{\matfflL}{\texttt{mat\_ffn}_{\ell \rightarrow L}}
\newcommand{\matlnlL}{\texttt{mat\_ln1\_ln2}_{\ell \rightarrow L}}
\newcommand{\idlL}{\texttt{id}_{\ell \rightarrow L}}

\definecolor{blue(munsell)}{rgb}{0.0, 0.5, 0.69}
%%%%%%%%%%%%%%%%%%%%%%%%%%%%%%%%%%

\title{Jump to Conclusions: Short-Cutting Transformers\\With Linear Transformations}

% Author information can be set in various styles:
% For several authors from the same institution:
% \author{Author 1 \and ... \and Author n \\
%         Address line \\ ... \\ Address line}
% if the names do not fit well on one line use
%         Author 1 \\ {\bf Author 2} \\ ... \\ {\bf Author n} \\
% For authors from different institutions:
% \author{Author 1 \\ Address line \\  ... \\ Address line
%         \And  ... \And
%         Author n \\ Address line \\ ... \\ Address line}
% To start a seperate ``row'' of authors use \AND, as in
% \author{Author 1 \\ Address line \\  ... \\ Address line
%         \AND
%         Author 2 \\ Address line \\ ... \\ Address line \And
%         Author 3 \\ Address line \\ ... \\ Address line}

\author{Alexander Yom Din$^{1}$ ~~~~~ Taelin Karidi$^{1}$ ~~~~~ Leshem Choshen$^{1}$ ~~~~~
Mor Geva$^{2}$ 
\vspace{0.2cm} \\
$^1$Hebrew University of Jerusalem ~~~ $^2$Google Research \\
\small{\texttt{\{alexander.yomdin, taelin.karidi, leshem.choshen\}@mail.huji.ac.il}}, \small{\texttt{pipek@google.com}}}

\quash{
\author{Alexander Yom Din \\
  Hebrew University of Jerusalem \\ \texttt{alexander.yomdin@mail.huji.ac.il} \\\And
  Taelin Karidi \\
  Hebrew University of Jerusalem \\
  \texttt{taelin.karidi@mail.huji.ac.il} \\\And
  Leshem Choshen \\
  Hebrew University of Jerusalem \\ \texttt{leshem.choshen@mail.huji.ac.il} \\\And
  Mor Geva \\
  Google Research \\
  \texttt{pipek@google.com} \\}
}

\begin{document}
\maketitle



\begin{abstract}
% \vspace{-1em}
The diffusion-based generative models have achieved remarkable success in text-based image generation. However, since it contains enormous randomness in generation progress, it is still challenging to apply such models for real-world visual content editing, especially in videos. 
In this paper, we propose \texttt{FateZero}, a zero-shot text-based editing method on real-world videos without per-prompt training or use-specific mask. 
\RM{Specifically, different from a pipeline of two independent inversion and then generation stages, we find the intermediate attention maps during inversions store better structure and motion information. We thus reform them to temporally casual attention and replace them in the generation progress. To further reduce the unnecessary semantic leakage of source video and enhance the editing quality, we then remix the temporally casual attentions via the cross-attention features of the source prompt as the mask.}
To edit videos consistently, we propose several techniques based on the pre-trained models. Firstly, in contrast to the straightforward DDIM inversion technique, our approach captures intermediate attention maps during inversion, which effectively retain both structural and motion information. These maps are directly fused in the editing process rather than generated during denoising. To further minimize semantic leakage of the source video, we then fuse self-attentions with a blending mask obtained by cross-attention features from the source prompt. Furthermore, we have implemented a reform of the self-attention mechanism in denoising UNet by introducing spatial-temporal attention to ensure frame consistency.
Yet succinct, our method is the first one to show the ability of zero-shot text-driven video style and local attribute editing from the trained text-to-image model. We also have a better zero-shot shape-aware editing ability based on the text-to-video model~\cite{tuneavideo}. \RM{Besides video, our unified method also achieves state-of-the-art performance in zero-shot image editing.\chenyang{Need exp or remove the zero-shot image}} Extensive experiments demonstrate our superior temporal consistency and editing capability than previous works.
% The code will be released.
% \chenyang{emphasize: our observation at inversion time} \xiaodong{replacing the bold part to the actual pipeline: \textbf{Specifically, we work on replacing and mixing the attention maps between the inversion and generation since the self-attention map keeps the structure of the original natural image and the cross-attention is semantic-related, after remixing, we replace them in the corresponding generation steps for denoising.}}
% \footnote{Since there is no general video diffusion model is publicly available, we use one-shot video generation method~(Tune-A-Video~\cite{tuneavideo}) as the pretrained video diffusion model for zero-shot video editing\xiaodong{can be removed if we actually zero-shot on video}.}.
\end{abstract}
\section{Introduction}

The ability to reason about plans is critical for performing long-horizon tasks \citep{erol1996hierarchical, sohn2018hierarchical, sharma-etal-2022-skill}, compositional generalization \citep{corona-etal-2021-modular} and generalization to unseen tasks and environments \citep{shridhar2020alfred}.
Consider a simple long-horizon planning scenario where a robot is tasked with preparing a meal and serving it on the table. 
This presents a non-trivial planning problem since the agent needs to understand the sequence of operations required to perform the task and search for the relevant objects in the unfamiliar environment by interacting with various objects. %



Large language models have been recently shown to possess commonsense knowledge about the world such as object affordances and physical dynamics \citep{ouyang2022training,chowdhery2022palm}.
Early approaches considered text based environments and fine-tuned PLMs to predict actions given the history of past observations and actions \citep{jansen-2020-visually,micheli-fleuret-2021-language,yao-etal-2020-keep}.
Recent work has used this ability to reason about plans from text instructions in simulated household environments with simplifying assumptions such as text-only environment observations or feedback \citep{huang2022language,ahn2022can,li2022pre,logeswaran-etal-2022-shot}.


We focus on \emph{visually grounded planning} with PLMs --- the ability to adapt plans based on interaction and visual feedback from the environment.
While PLMs have strong planning commonsense priors, predictions from a PLM may not be directly realizable in the environment since the observation and action spaces are unknown.
This requires \emph{grounding} the PLM in the environment and adapting it to observe visual feedback, which is highly non-trivial.
Some prior works assume the availability of a pre-trained affordance function \citep{ahn2022can} or a success detector \citep{mirchandani2021ella}.
Notably, SayCan \citep{ahn2022can} completely decouples the PLM from observation information by selecting actions that have both high affordability (through a pre-trained affordance model) and high PLM likelihood.
Although this partially addresses the grounding problem, the use of visual feedback for action affordance alone is limited.
Often an agent must choose one of many affordable actions using information from observations.
For example, a driving agent should re-navigate and possibly turn around when encountering a ``road closed'' sign, but both turning around and driving forward are indistinguishable to SayCan because they are both affordable and the PLM is blind to observations.

Another workaround explored in prior work is translating the information in the visual observations to text using a pre-trained captioning system \citep{shridhar2021alfworld,huang2022language}.
However, it can be difficult to faithfully describe an image in words and information is lost in this inherently noisy process, which limits the information available to the planner.



Recent work shows that PLMs can be adapted for various natural language tasks by inserting tunable embeddings or soft prompts at the input of the PLM (also called prompt tuning or prefix tuning)~\citep{li-liang-2021-prefix,lester-etal-2021-power}.
This approach also extends to multi-modal understanding tasks such as image captioning \citep{mokady2021clipcap} and VQA \citep{tsimpoukelli2021multimodal} where images are encoded as soft prompts and finetuned for the target task.
Transformer based architectures have also been successfully applied to offline Reinforcement Learning in recent work \citep{chen2021decision,janner2021offline,li2022pre,reid2022can}.

Taking inspiration from these works, we propose the simple approach of embedding visual observations (`visual prompts') and \textit{directly inserting them as PLM input embeddings}.
The visual encoder and PLM are jointly trained for the target task, an approach we call \textbf{\oursfull}~(\ours).
By teaching the PLM to use observations for planning in an end to end manner, we remove the dependency on external data such as captions and affordability information that was used in prior work.
We show that this simple approach performs better than prior PLM-based planning approaches on two embodied planning benchmarks based on ALFWorld~\citep{shridhar2021alfworld} and Virtualhome~\cite{puig2018virtualhome}.



\section{Related Work}

%Here we summarize prior work on transfer learning and property inference.

%\shortsection{Transfer Learning}
%%Transfer learning reuses features learned by pre-trained models for new tasks, with the pretext that inherent similarities in the generic features will be useful for the downstream tasks and hence reducing their cost of downstream training. Specifically, the downstream model trainer will use a pre-trained upstream model as the starting point for the downstream training, with inclusion of (or replacement with) the task-specific classification layer/module. The downstream model is then trained by either updating all layers of the model (including ones reused from upstream model) or freezing some earlier layers of the reused parts as the ``feature extractor'' and only updating the rest. The latter approach is more popular as the reused feature extractors can already learn useful feature representations and the training cost is also much lower and affordable for individuals with limited computational resources. We study the vulnerability of the latter transfer learning approach in this paper. 


%\shortsection{Transfer Learning} 
Several works have demonstrated risks associated with transfer learning across a variety of attack goals. Wang et al.~\cite{wang2018great} and Yao et al.~\cite{yao2019latent} consider manipulating the upstream model such that the fine-tuned downstream models contain backdoors, misclassifying test inputs that contain predefined backdoor triggers. These transfer manipulations are tailored to their particular attack goals and cannot be applied for the property inference goal considered in this paper. Zou et al.~\cite{zou2020privacy} study the threat of membership inference attacks on transfer learning, but with normally trained upstream models.  
%\dnote{its clear that the goals are different for these attacks, but how similar are the methods?} \ynote{similarity of the methods? more details about the methods? do not know what is expected here}
%In contrast, we investigate the possibility of boosting the effectiveness of property inference by manipulating the upstream model training. % Schuster et al.~\cite{schuster2020humpty} show that the attacker can modify the corpus on which the word embedding is trained such that the downstream NLP models which use that embedding will behave abnormally.

%\shortsection{Property Inference}
The risk of property inference was introduced by Ateniese et al.~\cite{ateniese2015hacking}, % introduces the threat of inferring properties of the training data from pre-trained models, 
and several subsequent works have developed property inference (also known as distribution inference) attacks~\cite{Wang2022GroupPI, suri2022formalizing, Jurez2022BlackBoxAF, Hartmann2022DistributionIR}.
% Ganju et al.~\cite{ganju2018property} and Suri and Evans~\cite{suri2022formalizing} 
These works study property inference against normally trained models, and they launch attacks using a variety of black-box and white-box attacks. All the white-box attacks use meta-classifiers, which take the permutation-invariant representation~\cite{ganju2018property} of the model parameters as the features. We use the state-of-the-art white-box attack~\cite{suri2022formalizing} in our experiments.
%We will use the state-of-the-art white-box method proposed by Ganju et al.~\cite{ganju2018property} and later extended by suri et al.~\cite{suri2022formalizing} in this paper.
%\dnote{do we use these attacks?} 
Melis et al.~\cite{melis2019exploiting} and Zhang et al.~\cite{zhang2021leakage} focus on property inference in distributed training scenarios. In their settings, the attacker is a participant in the global model training and conducts property inference using meta-classifiers that are trained on model outputs or gradients. Similarly, Suri et al.~\cite{suri2022subject} focus on federated learning settings where the attacker is a participant (or the central server) that utilizes black-box attacks for inferring membership of data from particular subjects. %\dnote{if we use black-box attacks, explain which ones, or how ours are related to previous ones} 
For our experiments, We improve the black-box meta-classifier proposed by Zhang et al.~\cite{zhang2021leakage} using the ``query tuning'' technique in Xu et al.~\cite{xu2019detecting}. 

The closest works to ours are Chase et al.~\cite{saeed} and Chaudhari et al.~\cite{Chaudhari2022SNAPEE}, which both consider a scenario where the attacker can manipulate some of the training data of the model to induce a model that significantly increases property inference risk.
% \dnote{it enables precise property inference attacks?}.
These works assume an adversary with the ability to poison the victim's training data, while the adversary in our scenario has no access to the victim's training data, and therefore, their methods are not applicable.
% \dnote{example how different from ours, and why the methods are not applicable}
%Thus, their methods are not applicable to our transfer learning scenario.
%Their methods rely on inducing certain behavior correlated with the properties to be inferred, and thus are not applicable to our transfer learning scenario. \anote{Still a bit unclear why that is the case.}
%
There are also works similar to ours that leverage ``adversarial initializations'' for attack purposes.
% \cite{grosse2019adversarial, boenisch2021curious, wen2022fishing, fowl2021robbing}.
Grosse et al.~\cite{grosse2019adversarial} focus on scenarios where the attacker can control the parameter initialization of a model, and demonstrate that the attacker can use special initializations to damage the performance of the trained model. %This attack is orthogonal to ours.
Other works \cite{boenisch2021curious, wen2022fishing, fowl2021robbing} show that the malicious central server in a federated learning protocol can reconstruct some training samples via falsifying the global model in some training rounds and then analyzing the submitted gradients. These kinds of attacks do not apply to our transfer-learning scenario since the attacker cannot access the downstream gradients, and can only manipulate the upstream training.

\iffalse %%%%%%%%%%%%%%%%%%%%%%%%%%%%%%%%

In this section, we provide the background and also the summary of prior attacks on transfer learning (Section~\ref{sec:transfer_learning}) and property inference (Section~\ref{sec:property_inference}). Then, we introduce the closely related manipulation attacks against machine learning models to boost different privacy risks in Section~\ref{sec:active_inference_attacks}.

%\anote{Do we really need a dedicated section for this? It's barely 2 paragraphs right now.}

%\dnote{the most closely related work to ours are works that attempt to amplify inference attacks by poisoning models, the two most relevant I know of are \url{https://www.computer.org/csdl/proceedings-article/sp/2022/131600b569/1CIO8nmuota} and \url{https://arxiv.org/abs/2204.00032}, but need to look thoroughly for others. We should definitely be describing this and relating it to our work, probably in the introduction. Most of what is here is Background, but should be clear what this section is for (not muddling background and related work)}

\subsection{Transfer Learning} \label{sec:transfer_learning}
Transfer learning reuses features learned by pre-trained models for new tasks, with the pretext that inherent similarities in generic features can be useful for downstream tasks, thus reducing the cost of downstream training. Specifically, the downstream model trainer uses a pre-trained upstream model as the starting point for downstream training, with the inclusion (or replacement) of task-specific classification layers/modules. The downstream model is then trained by either updating all layers of the model (including ones reused from the upstream model) or freezing some earlier layers of the reused parts as the ``feature extractor'' and only updating the rest. The latter approach is more popular as the reused feature extractors can already learn useful feature representations and the training cost is also much lower and affordable for individuals with limited computational resources. We study the vulnerability of the latter transfer learning approach in this paper. 
%mainly in two ways:  1) all the layers (including ones reused from ) and tune the full model; the other one is to freeze some earlier layers of the model as the feature extractor and only tune the rest later layers. The second update strategy could achieve better efficiency since the frozen layers can already produce meaningful feature representations~\cite{wang2018great,yao2019latent}, and we will study the transfer learning using this strategy. 

Recently, various attacks have been proposed for the transfer learning setting, but with different attack goals from ours. Wang et al.~\cite{wang2018great} generate adversarial examples against black-box student models that transfer knowledge from publicly available teacher models without repeated queries. Yao et al.~\cite{yao2019latent} propose to manipulate the upstream model such that the downstream models derived from the upstream model contain backdoors, which would misclassify test inputs that contain some predefined backdoor triggers. Zou et al.~\cite{zou2020privacy} study the threat of membership inference attacks on transfer learning and the upstream models are trained normally. In contrast, we investigate the possibility of boosting the effectiveness of property inference by manipulating the upstream model training. Schuster et al.~\cite{schuster2020humpty} show that the attacker can modify the corpus on which the word embedding is trained such that the downstream NLP models which use that embedding will behave abnormally.

%This additionally allows model trainers to achieve satisfactory performance with limited training samples, leading to reduced computational costs. The most common approach reuses parameters in the earlier layers of the pre-trained model, either by fixing them as the feature extractor or just using them for initialization, to conduct downstream training.

\subsection{Property Inference} \label{sec:property_inference}

\shortsection{Property Inference Attacks} In property inference attacks, the adversary aims to infer some sensitive properties of some data, given a model trained on it. For example, the adversary may be interested in sensitive properties like the presence of people of a specific race in the dataset~\cite{ateniese2015hacking, melis2019exploiting}), or even be curious about the 
the statistics of the training set (e.g, the ratio of people with a specific gender~\cite{saeed, ganju2018property, suri2022formalizing, zhang2021leakage}).


Ateniese et al.~\cite{ateniese2015hacking} were the first to identify the threat of inferring properties of the training data from pre-trained models. Ganju et al.~\cite{ganju2018property} and Suri and Evans~\cite{suri2022formalizing} 
study property inference against normally trained models, and they launch attacks using white-box meta-classifiers, which utilize the permutation-invariance representation~\cite{ganju2018property} of the model parameters, while other works focus on distributed training~\cite{zhang2021leakage} where the attacker is a participant in the global model training and conducts property inference using meta-classifiers trained on model outputs. Similarly, Suri et al.~\cite{suri2022subject} focus on federated learning, where the attacker is a participant (or the central server) that utilizes black-box attacks for inferring membership of data from particular subjects. Chase et al.~\cite{saeed} propose an active property inference attack for data poisoning scenarios, which we will cover and compare to in Section~\ref{sec:active_inference_attacks}.

%The closest work to ours are by Chase et al.~\cite{saeed} and Tramer et al.~\cite{tramer2022truth}. In their work, the attacker can manipulate some of the training data of the model such that a model trained (from scratch) on the poisoned data has an increased inference risk. However, their methods are not applicable to the transfer learning scenario. 
%In this work, we will focus on the property inference in transfer learning scenarios in which the attacker releases the upstream model and infer sensitive properties of the downstream models tuned from that upstream model.
% 

\shortsection{Defenses}
Defending against property inference attacks is an open problem. There are no studies in the current literature on active adversaries, and only a couple on passive ones. Ma et. al.~\cite{ma2021nosnoop} propose a defense against property inference attacks on data batches in the  collaborative learning setting. However, adversaries in the transfer-learning setting do not have access to batch-wise gradients of the downstream trainer. Chen and Ohrimenko~\cite{chen2022protecting} utilize mechanisms that add carefully-crafted noise to features to provide theoretical guarantees against inference adversaries, but focus on query-based access to the underlying dataset, not a machine learning model trained on it. These existing defenses thus do not apply to our threat model.

%propose a framework that reduces property inference to Boolean functions of individual members, posing the ratio of members satisfying the given function in a dataset as the property. These property inference attacks have since then been proposed as distribution inference attacks~\cite{suri2022formalizing}, presenting such attacks as inferring properties of the distributions used to sample datasets, differentiating them from exact inference attacks like dataset inference~\cite{maini2021dataset}. Nearly all property inference attacks use meta-classifiers to perform inference: training models on versions of datasets with and without the target property, followed by training a meta-classifier on top of these classifiers's model representations. These representations can take several forms: using model weights themselves with permutation-invariance~\cite{ganju2018property}, or model activations or logits for a generated set of query points~\cite{xu2019detecting}. However, the capability of such approaches is limited: the most that these attacks have been shown to work is medium-sized convolutional networks on the CelebA dataset~\cite{suri2022formalizing}.


\subsection{Active Privacy Attacks} \label{sec:active_inference_attacks}
% Perhaps the closely related works to ours as ones that proactively enhance the effectiveness of privacy attacks by manipulating the model training process in certain ways~\cite{saeed, melis2019exploiting, nasr2019comprehensive, tramer2022truth}. 
%shown that the adversary can, by using proactive ways, achieve stronger attacks that infer private information from deep learning systems~\cite{nasr2019comprehensive, melis2019exploiting, tramer2022truth, saeed}. In this section, we introduce the ones that are close to ours.

In the decentralized federated learning training, by submitting specially crafted gradients to the central server, malicious agents can increase membership inference risk~\cite{nasr2019comprehensive} and property inference risks~\cite{melis2019exploiting} of other benign agents' training data. However, these attacks do not apply to transfer learning scenario, as the attacker cannot control model gradients of downstream training. In the centralized setting, researchers propose attacks to poison the victim's training data such that the impacts of attribute inference and membership inference~\cite{tramer2022truth} and property inference~\cite{saeed} attacks are amplified on the poisoned model.
The ability to poison the victim's data is a threat model orthogonal to ours, since we have no access to the victim's downstream data. While there is scope to combine such approaches for stronger attacks (albeit with stronger access assumptions), we choose to focus on the scenario with no read/write access to the victim's data.

\fi %%%%%%%%%%%%%%%%%%%%%%%%%%%%%%%%

\section{Linear Shortcut Across Blocks}
\label{sec:layer_jump}

To use a hidden representation from layer $\ell<L$ as a final representation, we propose to cast it using linear regression, while skipping the computation in-between these layers. More generally, this approach can be applied to cast any $\ell$-th hidden representation to any subsequent layer $\ell'>\ell$.


\subsection{Method}
\label{subsec:methodology_linear_shortcut}

Given a source layer $\ell$ and a target layer $\ell'$ such that $0 \leq \ell < \ell' \leq L$, our goal is to learn a mapping
%$A_{\ell', \ell} \in \mathbb{R}^{d_h \times d_h}$
from hidden representations at layer $\ell$ to those at layer $\ell'$. To this end, we first collect a set of corresponding hidden representation pairs $(h^\ell, h^{\ell'})$. Concretely, we run a set $\mathcal{T}$ of input sequences through the model, and for each input $s$, we extract the hidden representations $h_{i_s}^{\ell}, h_{i_s}^{\ell'}$, where $i_s$ is a random position in $s$.
Next, we learn a matrix $A_{\ell', \ell} \in \mathbb{R}^{d_h \times d_h}$ by fitting linear regression over $\mathcal{T}$, i.e., $A_{\ell', \ell}$ is a numerical minimizer for:
$$ A \mapsto \sum_{s \in \mathcal{T}} || A \cdot h_{i_s}^\ell - h_{i_s}^{\ell'} ||^2,$$ 
and define the mapping of a representation $h$ from layer $\ell$ to layer $\ell'$ as:
\begin{equation}
\label{eq:linear_jump}
    \matl{} (h) \coloneqq A_{\ell', \ell} \cdot h.
\end{equation}


\subsection{Baseline}
\label{subsec:baseline}

We evaluate 
% our method against 
the prevalent approach of ``reading'' hidden representations directly, without any transformation. 
Namely, the propagation of a hidden representation from layer $\ell$ to layer $\ell'$ is given by the identity function, dubbed \id{}:

$$ \idl{} (h) \coloneqq h.$$

% Notably, 
This baseline 
assumes that representations at different layers operate in the same linear space.

\subsection{Quality of Fit}
\label{subsec:experiments_r2}

We first evaluate our method by measuring how well the learned linear mappings approximate the representations at the target layer. To this end, we calculate the (coordinate-averaged) $r^2$-score of our mapping's outputs with respect to the representations obtained from a full inference pass, and compare to the same for the \id{} baseline.


\paragraph{Models.}

We use \gpt{} \cite{radford2019language}, a decoder-only auto-regressive LM, with $L = 48$, $d_h = 1600$, and \bert{} \cite{devlin-etal-2019-bert}, an encoder-only model trained with masked language modeling, with $L=24$, $d_h=1024$.
% \footnote{\label{footnote:hf}We use models and data from Huggingface \cite{wolf-etal-2020-transformers,lhoest-etal-2021-datasets}.}
%For masked token prediction, we use a masked LM head pre-trained for our \bert{} model.

% \footnote{Specifically, we use the Huggingface Transformers \cite{wolf-etal-2020-transformers} implementations of all these models.}

%\sy{We use \gpt{} \cite{radford2019language}, a decoder-only auto-regressive LM, coming in four scales; $\texttt{gpt2}$ ($L = 12$, $d_h = 768$), $\texttt{gpt2-medium}$ ($L = 24$, $d_h = 1024$), $\texttt{gpt2-large}$ ($L = 36$, $d_h = 1280$) and $\texttt{gpt2-xl}$ ($L = 48$, $d_h = 1600$). Also, we use \bert{} \cite{devlin-etal-2019-bert}, an encoder-only model trained with masked language modeling, coming in two scales;  \texttt{bert-base-uncased} ($L=12$, $d_h=768$) and \texttt{bert-large-uncased} ($L=24$, $d_h=1024$). For masked token prediction, we use masked LM heads pre-trained for our models. Specifically, we use the Huggingface Transformers \cite{wolf-etal-2020-transformers} implementations of all these models. The plots presented in this section are for $48$-layered \gpt{} and $24$-layered \bert{}.}

%\sy{We use \gpt{} \cite{radford2019language}, a decoder-only auto-regressive LM, in the Huggingface \cite{wolf-etal-2020-transformers} implementation\footnote{\url{https://huggingface.co/gpt2}}, coming in four scales; $\texttt{gpt2}$ ($L = 12$, $d_h = 768$), $\texttt{gpt2-medium}$ ($L = 24$, $d_h = 1024$), $\texttt{gpt2-large}$ ($L = 36$, $d_h = 1280$) and $\texttt{gpt2-xl}$ ($L = 48$, $d_h = 1600$). Also, we use \bert{} \cite{devlin-etal-2019-bert}, an encoder-only model trained with masked language modeling, in the Hugginface implementation, coming in two scales;  \texttt{bert-base-uncased}\footnote{\url{https://huggingface.co/bert-base-uncased}} ($L=12$, $d_h=768$) and \texttt{bert-large-uncased}\footnote{\url{https://huggingface.co/bert-large-uncased}} ($L=24$, $d_h=1024$). For masked token prediction, we use the \texttt{BertForMaskedLM} heads from Huggingface, pretrained for these models. The plots presented in this section are for $48$-layered \gpt{} and $24$-layered \bert{}.}

\paragraph{Data.}
We sample random sentences from Wikipedia,
% \footref{footnote:hf} 
collecting 9,000 (resp. 3,000) sentences for the training set $\mathcal{T}$ (resp. validation set $\mathcal{V}$).\footnote{We use sentences rather than full documents to simplify the analysis.}
%\sy{We use two data sources to evaluate our method. One is Wikiepdia \cite{lhoest-etal-2021-datasets}\footnote{\url{https://huggingface.co/datasets/wikipedia}}; we use \texttt{spaCy}\footnote{\url{https://spacy.io/}} to divide documents into sentences\footnote{We use sentences rather than full documents to simplify the analysis.}\footnote{We pick randomly a Wikipedia document and then pick randomly a sentence ending in a newline character in it. \sy{[maybe this footnote is not needed?]}}, collecting 9,000 (resp. 3,000) random sentences for the training set $\mathcal{T}$ (resp. validation set $\mathcal{V}$). The second is a news article sentences dataset, the 10K English 2020 news sentences corpus
% \footnote{\url{https://downloads.wortschatz-leipzig.de/corpora/eng_news_2020_10K.tar.gz}} from the Leipzig Corpora Collection \cite{goldhahn-etal-2012-building}, which we randomly divide into a training set $\mathcal{T}$ consisting of 9,000 examples and a validation set $\mathcal{V}$ consisting of 1,000 examples.
% We truncate sentences to the maximal token length allowed by the model \mg{do we ever need to truncate? a sentence has about 10 words and the max. input len is thousands} \sy{[I surely did not need to in Leipzig, but discovered (via a transformers runtime warning) that I do need to for some (probably a minority) of the Wikipedia sentences. This probably has to do with that it is not really ``sentences" necessarily, for example, I noticed that it has some listings or something like that (bulleted items)... So some minority might get very long I guess...]}.
For each example $s$, we select a random position $i_s$ and extract the hidden representations $h_{i_s}^{\ell}$ at that position from all the layers.
For \bert{}, we first replace the input token at position $i_s$ with a \mask{} token, as our motivation is interpreting predictions, which are obtained via masked tokens in \bert{} (see \S\ref{subsec:BERT}).
Thus, in this case, the hidden representations we consider
%in the case of \bert{}
are of \mask{} tokens only.
%As we observed highly similar results for the two data sources across all our experiments, throughout the paper we will mainly report results for Wikipedia (except for \S\ref{sec:robustness}, where we cross-validate).


\begin{figure}[t]
\includegraphics[scale=0.2]{figs/r2_scores_48.pdf}
% \includegraphics[width=\columnwidth]{figs/r2_scores_48.pdf}
\caption{The coordinate-averaged $r^2$-score of $\matl{}$ (left) and $\idl{}$ (right) (\gpt{}).}
\label{fig:r2_scores}
\end{figure}


\begin{figure}[t]
\setlength{\belowcaptionskip}{-10pt}
\includegraphics[scale=0.2]{figs/bertmask_r2_scores_24.pdf}
% \includegraphics[width=\columnwidth]{figs/bertmask_r2_scores_24.pdf}
\caption{The coordinate-averaged $r^2$-score of $\matl{}$ (left) and $\idl{}$ (right) (\bert{}).}
\label{fig:bertmask_r2_scores}
\end{figure}



\paragraph{Evaluation.}
For every pair of layers $\ell, \ell'$, such that $0 \leq \ell < \ell' \leq L$, we use the training set $\mathcal{T}$ to fit linear regression as described in \S\ref{subsec:methodology_linear_shortcut}, and obtain a mapping $\matl{}$. 
Next, we evaluate the quality of $\matl{}$ as well as of $\idl{}$ using the $r^2$-coefficient, uniformly averaged over all coordinates. Concretely, we compute the $r^2$-coefficient of each of the predicted representations $\matl{} (h_{i_s}^{\ell})$ and $\idl{} (h_{i_s}^{\ell})$ versus the true representations $h_{i_s}^{\ell'}$
over all $s \in \mathcal{V}$.
%as we vary $s \in \mathcal{V}$.
%for every $s \in \mathcal{V}$.



\paragraph{Results.}
Results for \gpt{} and \bert{} are presented in Figs.~\ref{fig:r2_scores} and~\ref{fig:bertmask_r2_scores}, respectively.
In both models, \mat{} consistently yields better approximations than \id{}, as it obtains higher $r^2$-scores (in blue) across the network. 
This gap between \mat{} and \id{} is especially evident in \bert{}, where \id{} completely fails to map the representations between most layers, suggesting that hidden representations are modified  substantially by every transformer block.
Overall, this highlights the shortcoming of existing practices to inspect representations in the same linear space, and the gains from using our method to approximate future layers.
% in the network.
\section{Linear Shortcut for Language Modeling}
\label{sec:prediction}

We saw that our method approximates future hidden representations substantially better than a naive propagation. 
In this section, we will show that this improvement also translates to better predictive abilities from earlier layers. Specifically, we will use our method to estimate how often intermediate representations encode the final prediction, in the context of two fundamental LM tasks; next token prediction and masked token prediction.

\paragraph{Evaluation Metrics.}
Let $h, h' \in \mathbb{R}^{d_h}$ be a final representation and a substitute final representation obtained by some mapping, and denote by $\delta (h), \delta (h') \in \mathbb{R}^{d_v}$ their corresponding output probability distributions (obtained through projection to the output vocabulary -- see details below). 
We measure the prediction quality of $h'$ with respect to $h$ using two metrics:
\begin{itemize}
[leftmargin=*,topsep=1pt,parsep=1pt]
    \item \textbf{Precision@$k$} ($\uparrow$ is better): This checks whether the token with the highest probability according to $\delta(h')$ appears in the top-$k$ tokens according to $\delta(h)$. Namely, we sort $\delta(h)$ and assign a score of $1$ if $\arg\max(\delta(h'))$ appears in the top-$k$ tokens by $\delta(h)$, and $0$ otherwise.
    
    \item \textbf{Surprisal} ($\downarrow$ is better): We measure the minus log-probability according to $\delta(h)$, of the highest-probability token according to $\delta(h')$. Intuitively, low values mean that the model sees the substitute result as probable and hence not surprising.
\end{itemize}

\noindent We report the average Precision@$k$ and Surprisal over the validation set $\mathcal{V}$.



\subsection{Next Token Prediction}
\label{subsec:next_token_prediction_task}

Auto-regressive LMs output for every position a probability distribution over the vocabulary for the next token. Specifically, the output distribution for every position $i$ is given by $\delta (h_i^L)$, where:
\begin{equation}\label{eq:output_distribution}
    \delta (h) = \texttt{softmax} ( E^\top \cdot h) \in \mathbb{R}^{d_v}
\end{equation}
For some LMs, including \gpt{}, a layer normalization $\texttt{ln\_f}$ is applied to the final layer representation before this conversion (i.e., computing $\delta (\texttt{ln\_f}(h))$ rather than $\delta (h)$).

Recall that our goal is to measure how well this distribution can be estimated from intermediate representations, i.e. estimating $\delta (h_i^L)$ from $\delta (h_i^\ell)$ where $\ell<L$. To this end, we first run examples from the validation set through the model, while extracting for each example $s$ the hidden representation of a random position $i_s$ at every layer. Next, we apply our mappings $\matlL{}$ and the $\idlL{}$ baseline to cast the hidden representations of every layer $\ell$ to final layer substitutes (see \S\ref{sec:layer_jump}). Last, for each layer, we convert its corresponding final-layer substitute to an output distribution (Eq.~\ref{eq:output_distribution}) and compute the average Precision@$k$ (for $k=1,5,10$) and Surprisal scores with respect to the final output distribution, over the validation set.

\paragraph{Results.}
Figs.~\ref{fig:pre} and~\ref{fig:surp} show the average Precision@$k$ and Surprisal scores per layer in $48$-layered \gpt{}, respectively (the plots for the other \gpt{} models are presented in \S\ref{sec:app_scale}). Across all layers, \mat{} outperforms \id{} in terms of both scores, often by a large margin (e.g. till layer $44$ the Precision@$1$ achieved by \mat{} is bigger than that of $\id{}$ by more than $0.2$). 
This shows that linear mappings enable not just better estimation of final layer representations, but also of the predictions they induce. Moreover, the relatively high Precision@$k$ scores of \mat{} in early layers ($0.62$-$0.82$ for $k=10$, $0.52$-$0.74$ for $k=5$, and $0.28$-$0.45$ for $k=1$) suggest that early representations already encode a good estimation of the final prediction. Also, the substantially lower Surprisal scores of \mat{} compared to \id{} imply that our method allows for a more representative reading into the layer-wise prediction-formation of the model than allowed through direct projection to the vocabulary.

\begin{figure}[t]
\centering
\includegraphics[scale=0.4]{figs/pre_48.pdf}
\caption{Precision@$k$ ($k = 1,5, 10$) of $\matlL{}$ and $\idlL{}$ for next token prediction in $48$-layered \gpt{}.}
\label{fig:pre}
\end{figure}

\begin{figure}[t]
\centering
\includegraphics[scale=0.35]{figs/surp_48.pdf}
\caption{Surprisal for $\matlL$ and the baseline $\idlL{}$ ($48$-layered \gpt{} next token prediction task). A 95\% confidence interval surrounds the lines.}
\label{fig:surp}
\end{figure}

\subsection{Masked Token Prediction}
\label{subsec:BERT}

We now conduct the same experiment for the task of masked language modeling, where the model predicts a probability distribution of a masked token in the input rather than the token that follows the input. Unlike next token prediction, where the output distribution is computed from representations of varying input tokens, in masked token prediction the output is always obtained from representations of the same input token (i.e. \texttt{[MASK]}).

For this experiment, we use \bert{}, on top of which we use a pretrained masked language model head $\delta$; given a token sequence $s$, a \mask{} token inside it and its final representation $h$, $\delta (h) \in \mathbb{R}^{d_v}$
 is a probability distribution over tokens giving the model's assessment
 of the likelihood of tokens to be fitting in place of the \mask{} token in $s$.


\begin{figure}[t]
\centering
\includegraphics[scale=0.4]{figs/bertmask_pre_24.pdf}
\caption{Precision@$k$ ($k = 1,5, 10$) for  $\matlL{}$ and the baseline $\idlL{}$ ($24$-layered \bert{} masked token prediction task).}
\label{fig:bertmask_pre}
\end{figure}

\begin{figure}[t]
\centering
\includegraphics[scale=0.35]{figs/bertmask_surp_24.pdf}
\caption{Surprisal for $\matlL{}$ and the baseline $\idlL{}$ ($24$-layered \bert{} masked token prediction task). A 95\% confidence interval surrounds the lines.}
\label{fig:bertmask_surp}
\end{figure}

\paragraph{Results.}
Figs.~\ref{fig:bertmask_pre} and~\ref{fig:bertmask_surp} present the average Precision@$k$ and Surprisal scores per layer in $24$-layered \bert{} (the plots for the $12$-layered \bert{} model are presented in \S\ref{sec:app_scale}), overall showing trends similar to those observed for next token prediction in \gpt{} (\S\ref{subsec:next_token_prediction_task}). This is despite the differences between the two tasks and the considerable architectural differences between \bert{} and \gpt{}.
Notably, the superiority of \mat{} over \id{} in this setting is even more prominent; 
while \mat{}'s precision is between $0.2-0.6$ in the first ten layers (Fig.~\ref{fig:bertmask_pre}), \id{}'s precision for all values of $k$ is close to zero, again strongly indicating that our method allows for better reading into early layer hidden representations. 
More generally, \mat{} improves the Precision@$1$ of \id{} by more than $17\%$ at most layers, and unveils that a substantial amount of predictions ($>25\%$ starting from layer $3$) appear already in the very first layers.
Interestingly, the (rough) divide between the first half of layers and last half of layers for $\id{}$ in Figs.~\ref{fig:bertmask_pre},~\ref{fig:bertmask_surp} seems to align with the two-hump shape of the blue region for $\mat{}$ in Fig.~\ref{fig:bertmask_r2_scores}.

\paragraph{Analysis.}
We manually compare the predictions of our mapping $\matlL{}$ with $\idlL{}$, for a $24$-layered \bert{} model.  Concretely, we select 50 random sentences from the Leipzig dataset. Next, for each layer $\ell$, we manually analyze how many of the top-$5$ tokens according to $\matlL{}$ and $\idlL{}$ fit into context. We consider a token to fit into context if it is grammatically plausible within the sentence (see Tab.~\ref{tab:manual} for concrete examples).
In the resulting $1250$ instances (i.e. $50$ sentences $\times$ $25$ representations), we observe a substantially higher plausibility rate of $85.36\%$ for \mat{} compared to $52.8\%$ for \id{}. In fact, only in less than $4.3\%$ of the instances there are more plausible tokens among the top-$5$ tokens according to \id{} than among the top-$5$ tokens according to \mat{}, further supporting the Surprisal results above.

\begin{table*}
\footnotesize
\setlength{\belowcaptionskip}{-15pt}
\begin{tabular}{p{0.3\linewidth}ccccc}
& $\texttt{id}_{4 \rightarrow 24}$ & $\texttt{mat}_{4 \rightarrow 24}$ & $\texttt{id}_{12 \rightarrow 24}$ & $\texttt{mat}_{12 \rightarrow 24}$ & $\texttt{id}_{24 \rightarrow 24}$ \\ \midrule
\multirow{5}{=}{aldridge had shoulder surgery in \mask{}.} & fellowship & \tcbox{time} & cyclist & \tcbox{2009} & \tcbox{september} \\
& employment & \tcbox{it} & emergencies & \tcbox{2008} & \tcbox{november} \\
& agreement & her & seniors & \tcbox{2010} & \tcbox{december} \\
& \#\#ostal & them & cycling & \tcbox{2006} & \tcbox{august} \\
& \#\#com & work & \tcbox{pennsylvania} & \tcbox{2007} & \tcbox{july} \\ \midrule
\multirow{5}{=}{on your next view you will be asked to \mask{} continue reading.} & \#\#com & be & be & be & \tcbox{please} \\
& accreditation & get & undergo & \tcbox{please} & \tcbox{simply} \\ 
& $	\copyright$ & go & spartans & help & \tcbox{also} \\ 
& fellowship & \tcbox{help} & seniors & \tcbox{simply} & \tcbox{again} \\ 
& summer & have & * & say & \tcbox{immediately} \\ \bottomrule
\end{tabular}
\caption{Examples of top-$5$ predictions at layers $4$, $12$ and $24$, under the mappings $\matlL{}$ and $\idlL{}$, for a $24$-layered \bert{} model. Grammatically plausible predictions (according to a human annotator) are marked in \tcbox{blue}. Note that at layer $24$ the predictions of $\matlL{}$ and $\idlL{}$ are the same (by definition).} 
\label{tab:manual}
\end{table*}

\section{Implication to Early Exiting}
\label{sec:applications}

%The fact that it is often possible to approximate
The possibility of approximating
the final prediction already in the early layers has important implications for efficiency; applying our linear mapping instead of executing transformer blocks of quadratic time complexity, could save a substantial portion of the computation. In this section, we demonstrate this in the context of early exiting.

When 
% performing transformer model inference under 
using an early exit strategy \cite{schwartz-etal-2020-right, xin-etal-2020-deebert, schuster2022confident}, one aims at deciding dynamically at which layer to stop the computation and ``read'' the prediction from the hidden representation of that layer.
More precisely, under a confidence measure paradigm, one decides to stop the computation for a position $i$ at layer $\ell$ based on a confidence criterion, that is derived from casting the hidden representation $h_i^\ell$ as a final-layer representation and converting it to an output probability distribution. Specifically, following \citet{schuster2022confident}, a decision to exit is made if the difference between the highest and the second highest probabilities is bigger than $$ 0.9 \cdot \lambda + 0.1 \cdot {\rm exp} (-4 i / N),$$
where $N$ is the average length of the input until position $i_s$ for $s \in \mathcal{V}$, and $\lambda$ is a hyper-parameter.

\begin{figure}[t]
\setlength{\belowcaptionskip}{-10pt}
\centering
\includegraphics[width=\columnwidth]{figs/ee_gpt2bert.pdf}
\caption{Precision@$1$ with early exit and ``fixed exit'', applied to the $24$-layer \gpt{} for next token prediction (left) and the $24$-layer \bert{} for masked token prediction (right). Varying the confidence parameter $\lambda$, the $x$-coordinate is the average number of layers processed before an early exit decision is reached.}
\label{fig:ee_gpt2bert}
\end{figure}

\quash{
\begin{figure}[t]
\setlength{\belowcaptionskip}{-10pt}
\centering
\includegraphics[scale=0.35]{figs/ee_pre1_24.pdf}
\caption{Precision@$1$ for the various early exit methods, and previous ``fixed exit'' methods for comparison ($24$-layer \gpt{} next token prediction task). Varying the confidence parameter $\lambda$, the $x$-coordinate is the average number of layers processed before an early exit decision is reached.}
\label{fig:ee_pre1}
\end{figure}
}

\paragraph{Experiment.}
We assess the utility of our mapping $\matlL{}$ for early exit as a plug-and-play replacement for $\idlL{}$, through which intermediate representations are cast into final-layer representations.
We use \gpt{} for the next token prediction and \bert{} for masked token prediction (both with 24 layers).
We run each of the models over the validation set examples, while varying the confidence parameter $\lambda$ and using either $\idlL{}$ or $\matlL{}$ for casting intermediate representations.
Furthermore, we compare these early exit variants to the ``fixed exit'' strategy from \S\ref{sec:prediction}, where the computation is stopped after a pre-defined number of layers rather than relying on a dynamic decision.
We evaluate each variant in terms of both prediction's accuracy, using the Precision@$1$ metric (see \S\ref{sec:prediction}), and efficiency, measured as the average number of transformer layers processed during inference.


\paragraph{Results.}
%Figs.~\ref{fig:ee_pre1} and~\ref{fig:bertmask_ee_pre1}
Fig.~\ref{fig:ee_gpt2bert}
plots the average Precision@$1$ score against the average number of layers processed, for $24$-layer \gpt{} and $24$-layer \bert{}. For both models, under an early exit strategy our mapping \mat{} again provides a substantial improvement over \id{}.
For example, aiming at $95\%$ average precision, \mat{} saves $\sim3.3$ ($13.8$\%) layers in \gpt{} compared to only $\sim1.4$ ($5.9$\%) layers by \id{}, and $\sim4.8$ ($20$\%) layers in \bert{} versus $\sim3.5$ ($14.6$\%) layers by \id{}.
These results highlight the potential gains prominent early exit methods can obtain by using our method.
Notably, in both models and for each of the mapping methods, early exit obtains better results than fixed layer exit, as expected. 

\quash{
\begin{figure}[t]
\setlength{\belowcaptionskip}{-10pt}
\centering
\includegraphics[scale=0.35]{figs/bertmask_ee_pre1_24.pdf}
\caption{Precision@$1$ for the various early exit methods, and previous ``fixed exit'' methods for comparison ($24$-layer \bert{} masked token prediction task). Varying the confidence parameter $\lambda$, the $x$-coordinate is the average number of layers processed before an early exit decision is reached.}
\label{fig:bertmask_ee_pre1}
\end{figure}
}
\section{Linear Shortcut Across Sub-Modules}
\label{sec:submodules}

% Our experiments show that
% , despite the commonly-applied simplification by interpretability works, transformer layers do not operate in the same linear space and 
% there is a major gap in approximating future representations using an identity mapping (\S\ref{sec:layer_jump}, \S\ref{sec:prediction}).
% Here, 
In this section, we investigate whether discrepancies across layers result from specific sub-modules or are a general behaviour of all sub-modules in the network.  
This is done by extending our approach to test how well particular components in transformer blocks can be linearly approximated. 


\paragraph{Method.}

Consider \gpt{} for definiteness, then:
% we have 
$$ \texttt{b}_{\ell} = \texttt{b}_{\ell}^{\texttt{ffn}} \circ \texttt{b}_{\ell}^{\texttt{attn}}$$ 
% with
\begin{equation}\label{eq:attn} \texttt{b}^{\texttt{attn}}_{\ell} (H) = \texttt{attn}_{\ell} (\texttt{ln1}_{\ell} (H)) + H,\end{equation} 
where $\texttt{attn}_{\ell}$ is
%a multi-head self-attention
a MHSA
layer and \texttt{ln1} is a layer normalization (LN), and 
$$ \texttt{b}^{\texttt{ffn}}_{\ell} (H) = \texttt{ffn}_{\ell} (\texttt{ln2}_{\ell} (H)) + H,$$  
where $\texttt{ffn}_{\ell}$ is
%a feed-forward network
an FFN
layer and $\texttt{ln2}$ is a LN.
\quash{
Given a block $\texttt{b}_\ell$ and one of its sub-modules $\texttt{ln1}_\ell, \ \texttt{attn}_\ell, \ \texttt{ln2}_\ell$, or $\texttt{ffn}_\ell$, we fit linear regression approximating the output of the sub-module given its input and then use it in order to define mappings, as we now describe.
}
Given a block $\texttt{b}_\ell$ and one of its sub-modules $\texttt{ln1}_\ell, \ \texttt{attn}_\ell, \ \texttt{ln2}_\ell$, or $\texttt{ffn}_\ell$, we fit linear regression approximating the output of the sub-module given its input, and then use it to define mappings $\matattnl{}$, $\matlnl{}$ and $\matffl{}$.
%We provide the definition of $\matattnl{}$ below, and that of the other two in App. \ref{sec:app_submodule_skip_description}.
We provide the formal definitions of these mappings in App. \ref{sec:app_submodule_skip_description}.
\iffalse
\paragraph{$\matattnl{}$.}
%Illustrating this on $\texttt{attn}_\ell$ for definiteness,
For an input $s$, let $v^\ell_{i_s}$ be the vector at position $i_s$ in the output of $\texttt{attn}_\ell (\texttt{ln1}_\ell (H^{\ell - 1}))$. We denote by $A_\ell^{\texttt{attn}} \in \mathbb{R}^{d_h \times d_h}$ the matrix numerically minimizing 
$$ A \mapsto \sum_{s \in \mathcal{T}} || A \cdot \texttt{ln1}_\ell (h^{\ell-1}_{i_s}) - v^\ell_{i_s}||^2,$$
and define an attention sub-module replacement (Eq.~\ref{eq:attn}) by $$
\texttt{b}^{\overline{\texttt{attn}}}_\ell (h) \coloneqq A_{\ell}^{\texttt{attn}} \cdot \texttt{ln1}_\ell (h) + h. $$
We then define a mapping between two layers ${\ell \rightarrow \ell'}$ by:
$$ \matattnl{} (h) \coloneqq $$
$$ \texttt{b}^{\texttt{ffn}}_{\ell'} ( \texttt{b}^{\overline{\texttt{attn}}}_{\ell'} ( \ldots (\texttt{b}^{\texttt{ffn}}_{\ell+1} ( \texttt{b}^{\overline{\texttt{attn}}}_{\ell+1} (h)))\ldots)).$$ 
Namely, when applying each $\ell''$-th block, $\ell < \ell'' \leq \ell'$, we replace its attention sub-module $\texttt{attn}_{\ell''}$ by its linear approximation.
%In an analogous way, we consider the mappings $\matffl{}$ and $\matlnl{}$, where in the latter we perform the linear shortcut both for \texttt{ln1} and for \texttt{ln2} (see~\S\ref{sec:app_submodule_skip_description} for precise descriptions).
Importantly, unlike the original attention module, the approximation $\texttt{b}^{\overline{\texttt{attn}}}_\ell$ operates on each position independently, and therefore applying $\matattnl{}$ disables any contextualization between the layers $\ell$ and $\ell'$. Note that this is not the case for $\matffl{}$ and $\matlnl{}$, which retain the self-attention sub-modules and operate contextually.
\fi

\paragraph{Evaluation.}


We analyze the $24$-layered \gpt{}, and proceed completely analogously to \S\ref{subsec:next_token_prediction_task}, evaluating the Precision@$1$ and Surprisal metrics for the mappings $\matattnlL{}$, $\matfflL{}$ and $\matlnlL{}$.

\begin{figure}[t]
\setlength{\belowcaptionskip}{-0pt}
\centering
%\includegraphics[scale=0.2]
\includegraphics[width=\columnwidth]{figs/parts_presurp_24.pdf}
\caption{Precision@$1$ and Surprisal for the various sub-module linear mappings, and $\matlL{}$ for comparison ($24$-layer \gpt{} next token prediction task). A 95\% confidence interval surrounds the Surprisal lines.}
\label{fig:parts_presurp}
\end{figure}

\quash{
\begin{figure}[t]
\centering
\includegraphics[scale=0.4]{figs/parts_pre1_24.pdf}
\caption{Precision@$1$ for the various sub-module linear shortcut mappings, and the mapping $\matlL{}$ for comparison (\gpt{} next token prediction task).}
\label{fig:parts_pre1}
\end{figure}

\begin{figure}[t]
\centering
\includegraphics[scale=0.35]{figs/parts_surp_24.pdf}
\caption{Surprisal for the various sub-module linear shortcut mappings, and the mapping $\matlL{}$ for comparison (\gpt{} next token prediction task). A 95\% confidence interval surrounds the lines.}
\label{fig:parts_surp}
\end{figure}
}

\paragraph{Results.}
Fig.~\ref{fig:parts_presurp} shows the average Precision@$1$ and Surprisal scores per layer.
From a certain layer (\textasciitilde$7$), all sub-module mappings achieve better results than the full-block mapping $\matlL{}$. Thus, it is not just the cumulative effect of all the sub-modules in the transformer block that is amenable to linear approximation, but also individual sub-modules can be linearly approximated. 
Furthermore, the linear approximation of attention sub-modules is less harmful than that of the FFN or LN sub-modules. 
% Hypothetically, 
A possible reason is that the linear replacement of FFN or LN ``erodes'' the self-attention computation after a few layers. 
Moreover, the good performance of $\matattnlL{}$ suggests that contextualization often exhausts itself in early layers; speculatively, it is only in more delicate cases that the self-attention of late layers adds important information. Last, remark the sharp ascent of the scores for layer normalization in layers $5$-$8$, for which we do not currently see a particular reason. To conclude, we see that the possibility of linear approximation permeates
%the various
transformer components.


\section{Related Work}

Recently, there was a lot of interest in utilizing intermediate representations in transformer-based LMs, both for interpretability and for efficiency.

In the direction of interpretability, one seeks to understand the prediction construction process of the model \cite{tenney-etal-2019-bert, voita-etal-2019-bottom}.

More recent works use mechanistic interpretability and view the inference pass as a residual stream of information \cite{dar2022analyzing,geva-etal-2022-transformer}. Additionally, there are works on probing, attempting to understand what features are stored in the hidden representations \cite{adi2017finegrained, conneau-etal-2018-cram,liu-etal-2019-linguistic}. Our work is different in that it attempts to convert intermediate representations into a final-layer form, which is interpretable by design.

In the direction of efficiency, there is the thread of work on early exit, where computation is cut at a dynamically-decided earlier stage \cite{schwartz-etal-2020-right,xin-etal-2020-deebert,schuster2022confident}. Other works utilize a fixed early stage network to parallelize inference \citep{leviathan2022fast, chen2023accelerating}. However, intermediate representations are directly propagated in these works, which we show is substantially worse than our approach. Moreover, our method requires training considerably less parameters than methods such as \citet{schuster-etal-2021-consistent}, that learn a different output softmax for each intermediate layer.  

More broadly, skipping transformer layers and analyzing the linearity properties of transformer components have been discussed in prior works \cite{Zhao2021of,mickus-etal-2022-dissect,wang-etal-2022-skipbert,lamparth2023analyzing}.


\section{Conclusion and Future Work}

We present a simple and effective method for enhancing utilization of hidden representations in transformer-based LMs, that uses 
pre-fitted context-free and token-uniform linear mappings.
Through a series of experiments on different data sources, model architectures and scales, we show that our method consistently outperforms the prevalent practice of interpreting representations in the final-layer space of the model, yielding better approximations of succeeding representations and the predictions they induce, thus allowing a more faithful interpretation of the model's prediction-formation.
We demonstrate the practicality of our method for improving computation efficiency, saving a substantial amount of compute on top of prominent early exiting approaches. 
Also, by extending our method to sub-modules, 
% more specifically the attention sub-modules, 
we observe that replacing a part of the transformer inference by a non-contextual linear computation often results in a small deterioration of the prediction.
This opens new research directions for improving model efficiency,
% and parallelizability.
% including breaking the computation into several parallelizable tasks.
including breaking the computation into parallel tasks.

\section*{Limitations}

Although we see in this work that there is more linear structure to transformer inference than could be explained solely by the residual connection, we do not elucidate a reason for that. We also do not try to formulate formal criteria according to which to judge, in principle, the quality of ways of short-cutting transformer inference in-between layers. In addition, our experiments cover only English data.


%\section*{Ethics Statement}
%Scientific work published at ACL 2023 must comply with the ACL Ethics Policy.\footnote{\url{https://www.aclweb.org/portal/content/acl-code-ethics}} We encourage all authors to include an explicit ethics statement on the broader impact of the work, or other ethical considerations after the conclusion but before the references. The ethics statement will not count toward the page limit (8 pages for long, 4 pages for short papers).

\section*{Acknowledgements}

We thank Tal Schuster for constructive comments.

% Entries for the entire Anthology, followed by custom entries
\bibliography{anthology,custom}
\bibliographystyle{acl_natbib}

\appendix

\section{Descriptions of $\matattn{}$, $\matff{}$ and $\matln{}$}
\label{sec:app_submodule_skip_description}

Here we detail the definitions of the mappings $\matattnl{}$, $\matffl{}$ and $\matlnl{}$ utilized in \S\ref{sec:submodules}.

\paragraph{Description of $\matattnl{}$.}
%Illustrating this on $\texttt{attn}_\ell$ for definiteness,
For an input $s$, let $v^\ell_{i_s}$ be the vector at position $i_s$ in the output of $\texttt{attn}_\ell (\texttt{ln1}_\ell (H^{\ell - 1}))$. We denote by $A_\ell^{\texttt{attn}} \in \mathbb{R}^{d_h \times d_h}$ the matrix numerically minimizing 
$$ A \mapsto \sum_{s \in \mathcal{T}} || A \cdot \texttt{ln1}_\ell (h^{\ell-1}_{i_s}) - v^\ell_{i_s}||^2,$$
and define an attention sub-module replacement (Eq.~\ref{eq:attn}) by $$
\texttt{b}^{\overline{\texttt{attn}}}_\ell (h) \coloneqq A_{\ell}^{\texttt{attn}} \cdot \texttt{ln1}_\ell (h) + h. $$
We then define a mapping between two layers ${\ell \rightarrow \ell'}$ by:
$$ \matattnl{} (h) \coloneqq $$
$$ \texttt{b}^{\texttt{ffn}}_{\ell'} ( \texttt{b}^{\overline{\texttt{attn}}}_{\ell'} ( \ldots (\texttt{b}^{\texttt{ffn}}_{\ell+1} ( \texttt{b}^{\overline{\texttt{attn}}}_{\ell+1} (h)))\ldots)).$$ 
Namely, when applying each $\ell''$-th block, $\ell < \ell'' \leq \ell'$, we replace its attention sub-module $\texttt{attn}_{\ell''}$ by its linear approximation.
%In an analogous way, we consider the mappings $\matffl{}$ and $\matlnl{}$, where in the latter we perform the linear shortcut both for \texttt{ln1} and for \texttt{ln2} (see~\S\ref{sec:app_submodule_skip_description} for precise descriptions).
Importantly, unlike the original attention module, the approximation $\texttt{b}^{\overline{\texttt{attn}}}_\ell$ operates on each position independently, and therefore applying $\matattnl{}$ disables any contextualization between the layers $\ell$ and $\ell'$. Note that this is not the case for $\matffl{}$ and $\matlnl{}$, which retain the self-attention sub-modules and operate contextually.

\paragraph{Description of $\matffl{}$.}
Let $v^\ell_{i_s}$ be the vector at position $i_s$ in the output of $\texttt{ln2}_{\ell} (\texttt{b}_\ell^{\texttt{attn}} (H^{\ell - 1}))$, for a given input $s$. We denote by $A_\ell^{\texttt{ffn}} \in \mathbb{R}^{d_h \times d_h}$ the matrix numerically minimizing 
$$ A \mapsto \sum_{s \in \mathcal{T}} || A \cdot v^{\ell}_{i_s} - \texttt{ffn}_{\ell} (v^\ell_{i_s})||^2,$$
and define a replacement of the feed-forward sub-module $\texttt{b}_{\ell}^{\texttt{ffn}}$ by $$ \texttt{b}^{\overline{\texttt{ffn}}}_\ell (H) \coloneqq A_{\ell}^{\texttt{ffn}} \cdot \texttt{ln2}_\ell (H) + H.$$
We then define a mapping between two layers ${\ell \rightarrow \ell'}$ by:
$$ \matffl{} (H) \coloneqq $$
$$ \texttt{b}^{\overline{\texttt{ffn}}}_{\ell'} ( \texttt{b}^{\texttt{attn}}_{\ell'} ( \ldots (\texttt{b}^{\overline{\texttt{ffn}}}_{\ell+1} ( \texttt{b}^{\texttt{attn}}_{\ell+1} (H))\ldots)).$$

\paragraph{Description of $\matlnl{}$.}
Let $v^\ell_{i_s}$ be the vector at position $i_s$ in the output of $\texttt{b}^{\texttt{attn}}_{\ell} (H^{\ell - 1})$, for a given input $s$. We denote by $A_\ell^{\texttt{ln1}} \in \mathbb{R}^{d_h \times d_h}$ the matrix numerically minimizing 
$$ A \mapsto \sum_{s \in \mathcal{T}} || A \cdot h^{\ell}_{i_s} - \texttt{ln1}_{\ell} (h^\ell_{i_s})||^2$$ and we denote by $A_\ell^{\texttt{ln2}} \in \mathbb{R}^{d_h \times d_h}$ the matrix numerically minimizing $$ A \mapsto \sum_{s \in \mathcal{T}} || A \cdot v^{\ell}_{i_s} - \texttt{ln2}_{\ell} (v^\ell_{i_s})||^2.$$ We define a replacement of the block $\texttt{b}^{\texttt{attn}}_{\ell}$ by \begin{equation} \texttt{b}^{\overline{\texttt{ln1}}}_\ell (H) \coloneqq \texttt{attn}_{\ell} (A_{\ell}^{\texttt{ln1}} \cdot H) + H\end{equation} and we define a replacement of the block $\texttt{b}^{\texttt{ffn}}_{\ell}$ by \begin{equation} \texttt{b}^{\overline{\texttt{ln2}}}_\ell (H) \coloneqq \texttt{ffn}_{\ell} (A_{\ell}^{\texttt{ln2}} \cdot H) + H.\end{equation}
We then define a mapping between two layers ${\ell \rightarrow \ell'}$ by:
$$ \matlnl{} (H) \coloneqq $$
$$ \texttt{b}^{\overline{\texttt{ln2}}}_{\ell'} ( \texttt{b}^{\overline{\texttt{ln1}}}_{\ell'} ( \ldots (\texttt{b}^{\overline{\texttt{ln2}}}_{\ell+1} ( \texttt{b}^{\overline{\texttt{ln1}}}_{\ell+1} (H))\ldots)).$$


\end{document}

	 \documentclass[lettersize,journal]{IEEEtran}
\usepackage{amsmath,amsfonts}
\usepackage{algorithmic}
\usepackage{algorithm}
\usepackage{array}
%\usepackage[caption=false,font=normalsize,labelfont=sf,textfont=sf]{subfig}
\usepackage{textcomp}
\usepackage{stfloats}
\usepackage{url}
\usepackage{verbatim}
\usepackage{graphicx}
\usepackage{cite}
\usepackage{color}
\hyphenation{op-tical net-works semi-conduc-tor IEEE-Xplore}
% updated with editorial comments 8/9/2021
\usepackage{microtype}
\usepackage{graphicx}
\usepackage{booktabs}
\usepackage{hyperref}
\usepackage{tcolorbox}
%\numberwithin{equation}{section}
\usepackage{amssymb}
\usepackage{mathtools}
\usepackage{amsthm}
\usepackage{caption}
\usepackage{pifont}
\usepackage{subcaption}
\usepackage[capitalize,noabbrev]{cleveref}
\newcommand{\Ren}[1]{\textcolor{blue}{Ren: #1}}
%%%%%%%%%%%%%%%%%%%%%%%%%%%%%%%%
% THEOREMS
%%%%%%%%%%%%%%%%%%%%%%%%%%%%%%%%
\theoremstyle{plain}
\newtheorem{theorem}{Theorem}
\newtheorem{proposition}[theorem]{Proposition}
\newtheorem{lemma}{Lemma}
\newtheorem{corollary}[theorem]{Corollary}
\theoremstyle{definition}
\newtheorem{definition}[theorem]{Definition}
\newtheorem{assumption}[theorem]{Assumption}
\theoremstyle{remark}
\newtheorem{remark}[theorem]{Remark}
\newtheorem{mydef}{\bf{Definition}}

\DeclareMathOperator*{\minimize}{\text{minimize}}
\DeclareMathOperator*{\maximize}{\text{maximize}}

\DeclareMathOperator*{\st}{\text{subject to}}
\DeclareMathAlphabet\mathbfcal{OMS}{cmsy}{b}{n}
\newcommand{\Def}[0]{\mathrel{\mathop:}=}

\newcommand{\din}{\mathcal D}
%\DeclareMathOperator*{\minimize}{\text{minimize}}
%\DeclareMathOperator*{\maximize}{\text{maximize}}
\DeclareMathOperator*{\argmax}{arg\,max}
\DeclareMathOperator*{\argmin}{arg\,min}
%\DeclareMathOperator*{\st}{\text{subject to}}
\newcommand{\C}{\mathbb{C}}
\newcommand{\R}{\mathbb{R}}
\usepackage[textsize=tiny]{todonotes}

 



\begin{document}


\title{Bridging Models to Defend: A Population-Based Strategy for Robust Adversarial Defense}


\author{Ren~Wang,~\IEEEmembership{Member,~IEEE,}  Yuxuan~Li,~\IEEEmembership{Student Member,~IEEE,}
Can~Chen,~\IEEEmembership{Member,~IEEE,}        Dakuo~Wang,~\IEEEmembership{Senior~Member,~IEEE,} Jinjun~Xiong,~\IEEEmembership{Fellow,~IEEE,} Pin-Yu~Chen,~\IEEEmembership{Fellow,~IEEE,}        Sijia~Liu,~\IEEEmembership{Senior~Member,~IEEE,} Mohammad~Shahidehpour,~\IEEEmembership{Life~Fellow,~IEEE,}       and~Alfred~Hero,~\IEEEmembership{Life~Fellow,~IEEE}% <-this % stops a space
\IEEEcompsocitemizethanks{\IEEEcompsocthanksitem Ren Wang is with the Department
of Electrical and Computer Engineering, Illinois Institute of Technology, Chicago,
IL 60616.%\protect\\
% note need leading \protect in front of \\ to get a newline within \thanks as
% \\ is fragile and will error, could use \hfil\break instead.
%E-mail: rwang74@iit.edu
\IEEEcompsocthanksitem Yuxuan Li is a graduate research intern in the Department
of Electrical and Computer Engineering, Illinois Institute of Technology, Chicago,
IL 60616.
\IEEEcompsocthanksitem Can Chen is with the School of Data Science and Society, University of North Carolina at Chapel Hill, Chapel Hill, NC 27599.
\IEEEcompsocthanksitem Dakuo Wang is with the Khoury College of Computer Sciences and the College of Arts, Media and Design, Northeastern University, Boston, MA 02115.
\IEEEcompsocthanksitem Jinjun Xiong is with the Department of Computer Science and Engineering, University at Buffalo,
Buffalo, NY 14260.
\IEEEcompsocthanksitem Pin-Yu Chen is with the IBM Thomas J. Watson Research Center, NY 10598.
\IEEEcompsocthanksitem Sijia Liu is with the Department of Computer Science and Engineering,
Michigan State University, East Lansing, MI 48824.
\IEEEcompsocthanksitem Mohammad Shahidehpour is with the Department
of Electrical and Computer Engineering, Illinois Institute of Technology, Chicago,
IL 60616.
\IEEEcompsocthanksitem Alfred Hero is with the Electrical Engineering and Computer Science Department,
University of Michigan, Ann Arbor, MI 48109.}% <-this % stops an unwanted space
\thanks{The first two authors contributed equally to this paper.}
\thanks{Corresponding author: Ren Wang. E-mail: rwang74@iit.edu}
\thanks{Early versions of this work partially appeared in the conference proceedings  \cite{wang2023exploring} and \cite{wang2024deep}. %These works only present a small part of our results and only include a limited set of experiments.
}
\thanks{This work was supported in part by the National Science Foundation under grants CCF-2450414,  IIS-2246157, FMitF-2319243, by the Department of Energy under grant DE-CR0000042, and by the US Army Research Office under grant W911NF2310343.}
%\thanks{Under review at IEEE TPAMI.}
%\thanks{Manuscript received April 19, 2005; revised August 26, 2015.}
}






% The paper headers
\markboth{Journal of \LaTeX\ Class Files,~Vol.~14, No.~8, August~2021}%
{Shell \MakeLowercase{\textit{et al.}}: A Sample Article Using IEEEtran.cls for IEEE Journals}

% \IEEEpubid{0000--0000/00\$00.00~\copyright~2021 IEEE}
% Remember, if you use this you must call \IEEEpubidadjcol in the second
% column for its text to clear the IEEEpubid mark.

\maketitle

\begin{abstract}
Adversarial robustness is a critical measure of a neural network's ability to withstand adversarial attacks at inference time. While robust training techniques have improved defenses against individual $\ell_p$-norm attacks (e.g., $\ell_2$ or $\ell_\infty$), models remain vulnerable to diversified $\ell_p$ perturbations. To address this challenge, we propose a novel Robust Mode Connectivity (RMC)-oriented adversarial defense framework comprising two population-based learning phases. In Phase I, RMC searches the parameter space between two pre-trained models to construct a continuous path containing models with high robustness against multiple $\ell_p$ attacks. To improve efficiency, we introduce a Self-Robust Mode Connectivity (SRMC) module that accelerates endpoint generation in RMC. Building on RMC, Phase II presents RMC-based optimization, where RMC modules are composed to further enhance diversified robustness. To increase Phase II efficiency, we propose Efficient Robust Mode Connectivity (ERMC), which leverages $\ell_1$- and $\ell_\infty$-adversarially trained models to achieve robustness across a broad range of $p$-norms. An ensemble strategy is employed to further boost ERMC’s performance. Extensive experiments across diverse datasets and architectures demonstrate that our methods significantly improve robustness against $\ell_\infty$, $\ell_2$, $\ell_1$, and hybrid attacks. Code is available at \url{https://github.com/wangren09/MCGR}.
\end{abstract}

\begin{IEEEkeywords}
Robustness, deep learning, neural network, robust mode connectivity, adversarial training, population-based optimization.
\end{IEEEkeywords}

\section{Introduction}\label{sec:introduction}
The past decade has witnessed rapid advances in deep learning, leading to widespread adoption in high-stakes domains such as medical imaging \cite{sarvamangala2022convolutional}, defect detection \cite{jiwei2019bottom}, and power systems \cite{li2023physics}, where security is critical. Neural networks (NNs), the core of modern deep learning, learn complex mappings from data but remain highly sensitive to small, often imperceptible, input perturbations known as adversarial examples \cite{goodfellow2014explaining,wang2022ask}. Although nearly imperceptible to humans, these perturbations can cause severe model failures, raising serious concerns about the trustworthiness of NNs in safety-critical applications \cite{madry2018towards,carlini2017towards}. 


\begin{figure}[h]
  \centering
  \includegraphics[trim=0 0 0 0,clip,width=.49\textwidth]{Figures/RMC.png}
  \caption{{Overview of the Robust Mode Connectivity (RMC)-Oriented Adversarial Defense Framework}. The upper level of the panel at the top shows Phase I, illustrating that a robust path (robust to adversary types 1 and 2) in the parameter space can be found by connecting one model robust to adversary type 1 and the second model robust to adversary type 2. Selecting optimal points from the path and implementing the RMC process again can further improve robustness, as illustrated in the lower level of the panel at the top. Phase II suggests that more adversary types can be considered by using RMC as the basic unit. The right side panel at the bottom shows the efficient robust mode connectivity (ERMC), which interlaces $\ell_1$ and $\ell_\infty$ robustness into mode connectivity's structure and extends protection to perturbations from $\ell_p \in [1,\infty)$ norms. The left side panel at the bottom illustrates an ensembling method that can further boost the performance of the defense.} 
  \label{fig: framework}
\end{figure}

To address this vulnerability, adversarial training (AT) and its variants have become the most prominent defenses \cite{madry2018towards,zhang2019theoretically,shafahi2019adversarial,wang2020fast}. AT updates model parameters using adversarial examples generated on-the-fly from clean data, enabling the network to learn from adversarial distributions and become more robust during inference. However, most AT methods are designed for a single $\ell_p$ norm constraint (e.g., $\ell_\infty$), and their robustness often degrades sharply under perturbations from other norms \cite{tramer2019adversarial}. While recent works attempt to address this by training on multiple $\ell_p$ norms \cite{croce2019provable,stutz2020confidence,tramer2019adversarial,maini2020adversarial,croce2022adversarial,wang2021adversarial}, they often fall short due to the inherent limitations of single-point optimization in the model parameter space. These approaches can get trapped in local minima or saddle points when optimizing for multiple robustness objectives simultaneously.

In contrast, population-based optimization maintains a diverse set of candidate solutions, enabling broader exploration of the parameter space and better handling of complex, multi-objective tasks such as diversified $\ell_p$ robustness \cite{eiben2015evolutionary,diaz2016review,mirjalili2019evolutionary}. One particularly promising avenue is mode connectivity, which reveals that low-loss, high-accuracy paths often exist between independently trained models \cite{ren2025revisiting,garipov2018loss,freeman2017topology}. This property offers an accelerated population-based strategy for generating many viable models. However, naively applying mode connectivity is insufficient in adversarial settings.


In this work, we aim to improve a model’s robustness against perturbations constrained by different $\ell_p$ norms, with experimental focus on $p = 1, 2, \infty$. Motivated by the limitations of traditional approaches and the promise of mode connectivity, we propose a robust mode connectivity-oriented adversarial defense framework built on population-based optimization.

In Phase I, we introduce Robust Mode Connectivity (RMC), which finds high-robustness paths between adversarially trained models using a multi-steepest descent (MSD) algorithm \cite{maini2020adversarial}. To improve efficiency, we incorporate a Self-Robust Mode Connectivity (SRMC) module, which accelerates the creation of path endpoints. In Phase II, we construct RMC-based optimization, a broader framework that composes RMC modules to generate a population of candidate models and select those with the highest diversified robustness. Further, motivated by theoretical insights that affine classifiers robust to both $\ell_1$ and $\ell_\infty$ attacks can generalize to a wide range of $\ell_p$ threats, we propose Efficient Robust Mode Connectivity (ERMC). This method combines $\ell_1$- and $\ell_\infty$-robust models using a mode connectivity path and ensemble aggregation, boosting efficiency and robustness across norms (Fig.~\ref{fig: framework}).

\noindent\textit{Contributions.}
We summarize our main contributions as follows:

\begin{itemize}
    \item Robust Mode Connectivity (RMC): We propose RMC to construct paths between adversarially trained models, yielding intermediate models with high robustness to diversified $\ell_p$ perturbations. We further introduce Self-Robust Mode Connectivity (SRMC) to accelerate endpoint generation, improving the training efficiency of RMC.
(See Figures~\ref{fig: adv_mc}, \ref{fig: rmc_vgg16_cifar100}, \ref{fig: adv_self_rmc})
    \item RMC-Based Optimization: We extend RMC to a multi-stage population-based optimization framework that further improves robustness by selecting optimal models across multiple RMC units.
(Figures~\ref{fig: adv_mc_opt1}, \ref{fig: adv_mc_opt12}, \ref{fig: adv_mc_opt2})
    \item Efficient Robust Mode Connectivity (ERMC): We propose ERMC, a theoretically grounded method combining $\ell_1$ and $\ell_\infty$ robustness via mode connectivity and ensemble learning, to enhance the efficiency of the RMC-Based Optimization.
(Figure~\ref{fig: adv_self_rmc2})
    \item Comprehensive Evaluation: We conduct extensive experiments demonstrating that RMC, RMC-based optimization, and ERMC significantly outperform existing methods in achieving diversified $\ell_p$ robustness.
(Table~\ref{tab: main})
\end{itemize}

\noindent The rest of this article is organized as follows. Section~\ref{sec: related_work} introduces related works on defenses against diversified $\ell_p$ norm perturbations and population-based neural network learning. In Section~\ref{sec: pre}, we provide the definition of diversified $\ell_p$ robustness, and give introductions to adversarial attack, adversarial training, and mode connectivity. Sections~\ref{sec: rmc} and \ref{sec: opt} introduce the two phases of the proposed mode connectivity-oriented adversarial defense. The RMC method is presented in Section~\ref{sec: rmc}. The RMC-based optimization is proposed in Section~\ref{sec: opt}, and is enhanced by the ERMC method introduced in Section~\ref{sec: ermc} to improve its efficiency. Section~\ref{sec: exp} shows the experimental results. Section~\ref{sec: conclusion} concludes the article.



\section{Related Work}\label{sec: related_work}

\subsection{Adversarial Attacks}
Techniques such as the Fast Gradient Sign Method \cite{goodfellow2014explaining} and Projected Gradient Descent (PGD) \cite{madry2018towards} exploit the local gradient details of the target model to craft attacks. Building on PGD, output diversified sampling \cite{tashiro2020diversity} utilizes an enhanced initialization approach to create varied initial positions. However, these methods often provide inaccurate robustness measurements due to incorrect hyper-parameter tuning and gradient masking. To address this, Auto Attack (AA) \cite{croce2020reliable} combines four attack techniques with adjusted step sizes. To evaluate robustness under diversified $\ell_p$ norm perturbations simultaneously, Multi Steepest Descent (MSD) \cite{maini2020adversarial} incorporates various perturbation models within each step of the projected steepest descent, producing an adversary with a comprehensive understanding of the perturbation region. In this work, we consider PGD, AA, and MSD attacks to generate diversified $\ell_p$ norm perturbations.


\subsection{Adversarial Training-Based Defense}
The defense approach known as Adversarial Training (AT) \cite{madry2018towards} pioneered the use of min-max optimization for adversarial defense and has subsequently given rise to a plethora of other effective defense strategies. This includes the TRADES which delves into the trade-off between robustness and accuracy \cite{zhang2019theoretically}, dynamic adversarial training \cite{wang2019convergence}, and semi-supervised robust training approaches \cite{stanforth2019labels}. Furthermore, recent works, such as those by \cite{shafahi2019adversarial,wang2020fast,Wong2020Fast,zhang2019you}, have sought to develop faster, albeit approximate, AT algorithms. However, a common challenge across many of these methods is their concentration on a singular type of $\ell_p$ norm perturbation during AT. This specificity often culminates in a substantial decline in robustness when models are exposed to inputs with perturbations differing from the training set \cite{tramer2019adversarial}.

\subsection{Defenses on Diversified $\ell_p$ Norm Perturbations} 

Among all the works, \cite{croce2019provable} is the only one that provides a provable defense, and \cite{stutz2020confidence} considers withholding specific inputs to improve model resistance to stronger attacks. \cite{tramer2019adversarial} designs the inner loss by either selecting the type of perturbation that provides the maximum loss or averaging the loss across all types of perturbations. Extreme Norm Adversarial Training (E-AT) \cite{croce2022adversarial} leverages a fine-tuning strategy to improve robustness, while Multi Steepest Descent (MSD) Defense \cite{maini2020adversarial} incorporates various perturbation models within each step of the projected steepest descent to achieve diversified $\ell_p$ robustness. Nevertheless, despite their efforts, all the aforementioned works still depend on optimizing a single set of parameters, and the challenge of addressing the deficiency in diversified $\ell_p$ robustness remains unresolved. This work solves the challenge from a population-based optimization perspective.


\subsection{Population-Based Neural Network Learning} 
Optimizing a population of neural networks instead of a single network can prevent getting stuck at local minimums and lead to improved results. In one approach, \cite{jaderberg2017population} trained multiple instances of a model in parallel and selected the best performing instances to breed new ones. \cite{cui2018evolutionary} proposed an evolutionary stochastic gradient descent method that improved upon existing population-based methods. However, such methods typically have low learning speed and neglect adversarial robustness. Inspired by the human immune system, researchers have mimicked the key principles of the immune system in the inference phase to increase the robustness and not affect the learning speed in the training phase \cite{wang2022rails}. Mode connectivity can be treated as a faster population-based learning with two ancestor models that enhances the learning efficiency in the training phase \cite{garipov2018loss}. Researchers also analyze the mode connectivity when
networks are tampered with backdoor or error-injection attacks or under the attack of a single type of perturbation \cite{zhaobridging}. Our research extends beyond the scope of \cite{zhaobridging} by developing a novel robust, population-based optimization method for identifying candidate models with diversified $\ell_p$ robustness, and by exploring the phenomena of robust mode connectivity among different types of $\ell_p$ perturbations. Our unique approach not only facilitates the discovery of robust paths between two adversarially trained models but also generates candidates with enhanced robustness, thereby achieving state-of-the-art results in Diversified $\ell_p$ robustness. \textit{Our prior works laid the foundation for this study}: The workshop paper \cite{wang2023exploring} introduced Phase I robust path discovery but offered only limited experiments and lacked theoretical support, while the subsequent Deep Adversarial Defense \cite{wang2024deep} presented ERMC with preliminary validation. In this paper, we unify and extend these ideas into a full two-phase RMC framework, supported by SRMC, RMC-Based Optimization, refined ERMC, theoretical guarantees, and extensive experiments across datasets and architectures.



\section{Preliminaries}\label{sec: pre}


\subsection{Adversarial Attack with Different Input Perturbation Generators}
 
Recent studies indicate that conventional learning methods struggle with perturbed datasets ($\mathcal D_1,\mathcal D_2,\cdots, \mathcal D_S$) generated by
\begin{align}\label{eq: adv_atk}
    \displaystyle \arg\max {\mathcal L(\boldsymbol \theta; {\bf x}^\prime, y)}, ~~ s.t.~~ \text{Dist}_i({\bf x}^\prime, {\bf x}) \leq \delta_i, i \in [S]
\end{align}
for $\forall {\bf x} \in \mathcal D_0$, where $\mathcal D_0$ denotes the benign dataset, and $\delta_i$s are predefined bounds of perturbations corresponding to $\mathcal D_i$s with $i \in [S]$ (where $[S]=\{1, 2, \cdots, S\}$). In this paper, we restrict distance measures $\text{Dist}_i$s to be $\ell_p, p=1,2,\infty$ norms in our experiments. \eqref{eq: adv_atk} is typically termed an adversarial attack \cite{madry2018towards}. A practical approach to solving \eqref{eq: adv_atk} involves applying gradient descent and projection $P_{\delta_i}$ that maps the perturbation $\boldsymbol \epsilon_i=\bf x^\prime - \bf x$ to a feasible set, commonly referred to as a PGD attack. We use $\ell_p$-PGD to denote the PGD attack using the $\ell_p$ norm.

\subsection{Adversarial Training (AT)} 
The min-max optimization-based adversarial training (AT) is known as one of the most powerful defense methods to obtain a robust model against adversarial attacks \cite{madry2018towards}. We summarize AT below: 
\begin{align}
\begin{array}{ll}
    \displaystyle \min_{\boldsymbol \theta} \mathbb E_{(\mathbf x, y) \in \mathcal D_0} [ \displaystyle \max_{\text{Dist}_i({\bf x}^\prime, {\bf x}) \leq  \delta_i} 
 \mathcal L( \boldsymbol \theta; \mathbf x^\prime,  y ) ],
 \end{array}
 \label{eq: at}
\end{align}
Although AT can achieve relatively high robustness on $\mathcal D_i$, it does not generalize to other $\mathcal D_j, j\not=i$. Moreover, training on all $\mathcal D_i, i \in [S]$ is not scalable and will not provide robustness to all types of perturbations \cite{tramer2019adversarial}. We will use $\ell_p$-AT to denote the AT with the $\ell_p$ norm.



\subsection{Metric Definition: Diversified $\ell_p$ Robustness}

We hope that models can be robust to every $\ell_p$ adversarial type in the adversarial set of concerns, and we need to give a metric to measure such robustness. Diversified $\ell_p$ Robustness (DLR) is defined as its capacity to sustain the worst type of perturbation confined by a specific level of attack power:
\begin{mydef}\label{def_gr}

For a loss function $\mathcal L$, an input-output mapping function $f(\cdot)$, and a benign dataset $\hat{\mathcal D}_0$, the Diversified $\ell_p$ Robustness of a set of neural network parameters $\boldsymbol \theta$ is 
\begin{equation}\label{general_robustness_def}
\min_{i \in [S]} \frac{\sum_{({\bf x}^\prime, y)\in \hat{\mathcal D}_i} \boldsymbol{1}_{f({\bf x}^\prime, \boldsymbol \theta)= y}}{|\hat{\mathcal D}_i|},
\end{equation}
\end{mydef}
\noindent where $\hat{\mathcal D}_i$ represents the data generated by \eqref{eq: adv_atk} from $\hat{\mathcal D}_0$ with $\textup{Dist}_i$ as one of $\ell_p, p=1,2,\infty$ norms. We remark that there are other ways to define DLR. For example, the definition can be based on the worst-case sample-wise instead of worst-case data set-wise. Despite the difference, they essentially measure the same quantity, i.e., how well the model performs under various types of $\ell_p$ perturbations.


\subsection{Nonlinear Mode Connectivity}

\begin{figure}[h]
  \centering
  \includegraphics[trim=0 0 0 0,clip,width=.3\textwidth]{Figures/MC_Path1.png}
  \caption{{The path with near-constant loss found by mode connectivity in the parameter space}. The endpoints are two pre-trained models, and any point on the path represents a model.} 
  \label{fig: mc_path}
\end{figure}

Mode connectivity is a neural network's property that local minimums found by gradient descent methods are connected by simple paths belonging to the parameter space \cite{freeman2017topology,garipov2018loss}. Everywhere on the paths achieves a similar cost as the endpoints. The endpoints are two sets of neural network parameters $\boldsymbol \theta_1, \boldsymbol \theta_2 \in \mathbb{R}^{d}$ with the same structure and trained by minimizing the given loss $\mathcal L$. The smooth parameter curve is represented using $\phi(t; \boldsymbol \theta) \in \mathbb{R}^{d}, t \in [0,1]$, such that $\phi(0; \boldsymbol \theta)=\boldsymbol \theta_1, \phi(1; \boldsymbol \theta)=\boldsymbol \theta_2$. To find a desired low-loss path between $\boldsymbol \theta_1$ and $\boldsymbol \theta_2$, it is proposed to find parameters that minimize the following expectation over a uniform distribution on the curve:
\begin{equation}
\min_{\boldsymbol \theta} \mathbb E_{t \sim q(t; \boldsymbol \theta)}  \displaystyle \mathbb E_{({\bf x},y) \sim \mathcal D_0}  \mathcal L( \phi(t; \boldsymbol \theta); ({\bf x},y)),
\label{eq: mc_original}
\end{equation}
where $q(t; \boldsymbol \theta)$ represents the distribution for sampling the parameters on the path. Note that \eqref{eq: mc_original} is generally intractable. A computationally tractable surrogate is proposed as follows 
\begin{equation}
\min_{\boldsymbol \theta} \mathbb E_{t \sim U(0,1)}  \displaystyle \mathbb E_{({\bf x},y) \sim \mathcal D_0}  \mathcal L( \phi(t; \boldsymbol \theta); ({\bf x},y)),
\label{eq: mc}
\end{equation}
where $U(0,1)$ denotes the uniform distribution on $[0,1]$. Two common choices of $\phi(t; \boldsymbol \theta)$ in nonlinear mode connectivity are the Bezier curve \cite{farouki2012bernstein} and Polygonal chain \cite{gomes2012computer}. As an example, a quadratic Bezier curve is defined as $\phi(t; \boldsymbol \theta) = (1-t)^2 \boldsymbol \theta_1 + 2t(1-t)\boldsymbol \theta + t^2 \boldsymbol \theta_2$. Training neural networks on these curves provides many similar-performing models on low-loss paths. As shown in Fig.~\ref{fig: mc_path}, a quadratic Bezier curve obtained from \eqref{eq: mc} connects the upper two models along a path of near-constant loss.






\section{Two-Phase Robust Mode Connectivity}

\subsection{Phase I: Robust Path Search Via Robust Mode Connectivity}\label{sec: rmc}



\subsubsection{A Pilot Exploration}
Mode connectivity and adversarial training seem to be two excellent ideas for achieving high DLR that has been defined in Definition~\ref{def_gr}. If we set $\phi(0; \boldsymbol \theta)$ and $\phi(1; \boldsymbol \theta)$ to be two adversarially-trained neural networks under different types of perturbations, applying \eqref{eq: mc} may result in a path with points having high robustness for all these perturbations. Thus we ask:
\begin{tcolorbox}[before skip=2.0mm, after skip=2.0mm, boxsep=0.0cm, middle=0.1cm, top=0.1cm, bottom=0.1cm]
\noindent \textit{\textbf{(Q1)} Can simply combining adversarial training with mode connectivity provide high DLR?}
\end{tcolorbox}

Here we aim to see if implementing vanilla mode connectivity can bring us high DLR. We combine two PreResNet110 models \cite{he2016identity}, one trained with $\ell_\infty$-AT ($\delta=8/255$, 150 epochs) and the other trained with $\ell_2$-AT ($\delta=1$, 150 epochs), to find the desired path using the vanilla mode connectivity \eqref{eq: mc}. $\phi(0; \boldsymbol \theta)$ and $\phi(1; \boldsymbol \theta)$ are models trained by $\ell_\infty$-AT and $\ell_2$-AT, respectively. The mode connectivity curve is obtained with additional 50 epochs of training. The results are shown in Fig.~\ref{fig: std_mc}. The left (right) endpoint represents the model trained with $\ell_\infty$-AT ($\ell_2$-AT). One can see that the path has high loss and low robust accuracies on both types of attacks, indicating that vanilla mode connectivity fails to find the path that enjoys high DLR.

\begin{figure}[h]
  \centering
  \includegraphics[trim=0 0 0 0,clip,width=.46\textwidth]{Figures/vanilla_mc.png}
  \caption{{The vanilla mode connectivity \eqref{eq: mc} with models trained by $\ell_\infty$-AT and $\ell_2$-AT as two endpoints fails to find the path with high DLR}. $\phi(0; \boldsymbol \theta)$ and $\phi(1; \boldsymbol \theta)$ are $\ell_\infty$-AT ($\delta=8/255$, 150 epochs) and $\ell_2$-AT ($\delta=1$, 150 epochs). } 
  \label{fig: std_mc}
\end{figure}

\subsubsection{Embedding Adversarial Robustness to Mode Connectivity}

Although the vanilla mode connectivity aims to provide insight into the loss landscape geometry, it searches space following the original data distribution. Therefore it cannot provide high DLR by simply using two adversarially-trained models as two endpoints. Instead, we ask:
\begin{tcolorbox}[before skip=2.0mm, after skip=2.0mm, boxsep=0.0cm, middle=0.1cm, top=0.1cm, bottom=0.1cm]
\noindent \textit{\textbf{(Q2)} Can we develop a new method to embed adversarial robustness to mode connectivity by searching the adversarial input space?}
\end{tcolorbox}

To answer (Q2), we connect mode connectivity \eqref{eq: mc} with adversarial training under diversified $\ell_p$ adversarial perturbations. In other words, we modify the objective \eqref{eq: mc} to adjust to our high DLR purpose. An adversarial generator is added as an inner maximization loop. We adopt different types of perturbations in the generator. This is because a single type of perturbation may result in robustness bias. Formally, we obtain a model path $\phi(t; \boldsymbol \theta), t \in [0,1]$ parameterized by $\boldsymbol \theta$.  
\begin{equation}
\min_{\boldsymbol \theta} \mathbb E_{t \sim U(0,1)}  \displaystyle \mathbb E_{({\bf x},y) \sim \mathcal D_0 } \sum_{i \in I} \max_{\text{Dist}_i({\bf x}^\prime, {\bf x}) \leq  \delta_i} \mathcal L( \phi(t; \boldsymbol \theta); ({\bf x}^\prime,y)),
\label{eq: mc_adv}
\end{equation}
where $\phi(0; \boldsymbol \theta)$ and $\phi(1; \boldsymbol \theta)$ are two models trained by \eqref{eq: at}, probably under different types of perturbations. Throughout this paper, we fix the curve as a quadratic Bezier curve. Thus a model at the point $t$ can be represented by $\phi(t; \boldsymbol \theta) = (1-t)^2 \boldsymbol \theta_1 + 2t(1-t)\boldsymbol \theta + t^2 \boldsymbol \theta_2$. Similar to the nonlinear mode connectivity, \eqref{eq: mc_adv} is a computationally tractable relaxation by directly sampling $t$ from the uniform distribution $U(0,1)$ during the optimization. Data points $({\bf x}^\prime,y)$ are generated from a union of adversarial strategies $i \in I$, where $I$ is a subset of $\{1,2, 3, \cdots, S\}$. For example, data can be generated by using $\ell_2$ or $\ell_\infty$ norm distance measure, which is commonly used in adversarial attacks and adversarial training. Formulation in \eqref{eq: mc_adv} ensures that the identified path adapts to the targeted adversarial perturbations. 

We call \eqref{eq: mc_adv} the Robust Mode Connectivity (RMC), which serves as the first defense phase for robust path search. We remark that RMC itself is a defense method as we can select the model with the highest DLR in the path. One can see that a group of models (all points in the path) are generated from two initial models. Therefore RMC is a population-based optimization. 

\subsubsection{On the Benefits of Population-Based Framework}
Training a single model presents a fundamental challenge: different data points, and even the same point under varying adversarial norms, do not achieve peak robustness simultaneously. We therefore ask: 
\begin{tcolorbox}[before skip=2.0mm, after skip=2.0mm, boxsep=0.0cm, middle=0.1cm, top=0.1cm, bottom=0.1cm]
\noindent \textit{\textbf{(Q3)} Can a population-based framework address this issue by leveraging a large number of models?}
\end{tcolorbox}
Here, we define ``simultaneous'' robustness at the epoch-level, i.e., data points are considered to peak concurrently if they do so within the same training epoch. We posit that with a sufficiently large population of models, it is possible for $N$ data points to achieve peak robustness simultaneously under $S$ distinct adversarial norms.

\begin{theorem}
Let $T\geq 2$ be the number of training epochs. For each model $k$, each data-point/robustness-type pair $(i,s) \in \{1,\cdots, N\}\times \{1,\cdots, S\}$ has a random variable $X_{i,s}^{(k)} \in \{1,\cdots, T\}$ equal to the epoch at which that pair attains its highest $\ell_p$ robustness in model $k$. Assume the arrays $\{X_{i,s}^{(k)}\}_{i,s}$ are i.i.d. over $k$, the $X_{i,s}^{(k)}$ are mutuallly independent across $(i,s)$, and $\text{Pr}[X_{i,s}^{(k)}=t]=\frac{1}{T}$ for every $t \in \{1,\cdots, T\}$ for all $(i,s)$, then:

\noindent For any target confidence $1-\gamma \in (0,1)$, the minimum number of models that achieves $Pr$(at least one alignment among the $K$ models)$\geq 1-\gamma$ is $K=\lceil{\frac{\ln{\gamma}}{\ln{1-T^{-(NS-1)}}}\rceil}$.
\end{theorem}
The alignment event in each model has the probability $T^{-(NS-1)}$, so the probability of no alignment in the $K$ models is $1-T^{-(NS-1)}$. The proof is done by requiring $(1-T^{-(NS-1)})^K\le \gamma$. Note that the independence assumption may not hold in practice. Nevertheless, our objective here is solely to demonstrate the inherent advantages of the population-based method over a single model solution.

With the problem formulation and the theoretical support, the next step is to find out how to solve the RMC \eqref{eq: mc_adv}.


\begin{algorithm}[h]
\caption{Robust Mode Connectivity}
\label{alg: RMC}
\begin{algorithmic}[1]
\REQUIRE $\phi(0; \boldsymbol \theta)$, $\phi(1; \boldsymbol \theta)$ - two selected models with the same structure (potentially trained with different strategies, e.g., AT under different perturbation types); initial model $\boldsymbol \theta^0$; the perturbation types $i \in I$ and the corresponding projections $P_{\delta_i}$; training set $\mathcal D_0$; inner loop iteration number $J$; batch size $B$; initial perturbation $\epsilon^{(0)}=\mathbf 0$.
\STATE{$\boldsymbol \theta = \boldsymbol \theta^0$}
\FOR{each data batch $\mathcal D_b \in \mathcal D_0$ in each epoch $e \in E$}
\STATE{Uniformly select $t \sim U(0,1)$}
\FOR{$\forall \bf x \in \mathcal D_b$}
\FOR{$j = 1, \cdots, J$}
\FOR{$i \in I$}
\STATE{\hspace{-1mm}$\boldsymbol{\epsilon}_i^{(j)} \leftarrow P_{\delta_i}\big(\boldsymbol{\epsilon}^{(j-1)}-{\nabla_{\boldsymbol{\epsilon}}\mathcal L(\phi(t; \boldsymbol{\theta}); {\bf{x}}+{\boldsymbol{\epsilon}}^{(j-1)},y)}\big)$}
\ENDFOR
\STATE{$\boldsymbol \epsilon^{(j)} \leftarrow \arg\max_{\boldsymbol \epsilon_i^{(j)}, i \in I} {\mathcal L(\phi(t; \boldsymbol \theta); {\bf{x}} + \boldsymbol \epsilon_i^{(j)}, y)}$}
\ENDFOR
\ENDFOR
\STATE{$\boldsymbol \theta \leftarrow \boldsymbol \theta - \nabla_{\boldsymbol \theta} \sum_{\bf x \in \mathcal D_b} \mathcal L(\phi(t; \boldsymbol \theta); {\bf x} + \boldsymbol \epsilon^{(j-1)}, y)$}
\ENDFOR
\RETURN $\boldsymbol \theta$, $\phi(t; \boldsymbol \theta), \forall t \in [0,1]$
\end{algorithmic}
\end{algorithm}


\begin{figure*}[ht]
  \centering
             \subfloat[Epoch=50]{\includegraphics[trim=90 5 90 100,clip,width=0.32\textwidth]{Figures/pgd-pgd2-50.png}}
             \subfloat[Epoch=100]{\includegraphics[trim=90 5 90 100,clip,width=0.32\textwidth]{Figures/pgd-pgd2-100.png}}
             \subfloat[Epoch=150]{\includegraphics[trim=90 5 90 100,clip,width=0.32\textwidth]{Figures/pgd-pgd2-150.png}}
  \caption{{The RMC \eqref{eq: mc_adv} with models trained by $\ell_\infty$-AT and $\ell_2$-AT as two endpoints can find the path with high DLR}. MSD \cite{maini2020adversarial} with perturbations generated by $\ell_2$ and $\ell_\infty$ norm distance measures is leveraged as the inner solver. Solving \eqref{eq: mc_adv} uses 50/100/150 epochs in panel (a)/(b)/(c).} 
  \label{fig: adv_mc}
\end{figure*}


\subsubsection{Solving Robust Mode connectivity Via Multi Steepest Descent}

Solving \eqref{eq: mc_adv} is difficult as it contains multi-type perturbations. The simplest ways are using `MAX' or `AVG' strategy proposed in \cite{tramer2019adversarial}, where the inner loss is obtained by selecting the type of perturbation that provides the maximum loss or averaging the loss on all types of perturbations. However, both strategies consider perturbations independently. We leverage a Multi Steepest Descent (MSD) approach that includes the various perturbation models within each step of the projected steepest descent in order to produce a PGD adversary with complete knowledge of the perturbation region \cite{maini2020adversarial}. The key idea is to simultaneously maximize the worst-case loss overall perturbation models at each step. Algorithm~\ref{alg: RMC} shows the details, where all the perturbation types are considered in each iteration. The complexity order remains consistent when we juxtapose the Algorithm with the conventional AT. This is primarily because the number of perturbations $I$ is essentially constant in our scenarios (specifically, $I=2$ or $3$). Next we test the effectiveness of the proposed RMC algorithm.

We again use models trained with $\ell_\infty$-AT ($\delta=8/255$, 150 epochs) and $\ell_2$-AT ($\delta=1$, 150 epochs) as two endpoints $\phi(0; \boldsymbol \theta)$ and $\phi(1; \boldsymbol \theta)$. The RMC \eqref{eq: mc_adv} with MSD as the inner solver is applied to obtain the path. Fig.~\ref{fig: adv_mc} shows the results of training an additional 50/100/150 epochs with $D_{i}$s generated by $\ell_2$ and $\ell_\infty$ norm distance measures. One can find that unlike Fig.~\ref{fig: std_mc}, the paths contain points with high accuracy and high robustness against both $\ell_\infty$-PGD and $\ell_2$-PGD attacks. Although the left endpoint has low $\ell_2$ robustness and the right endpoint has relatively low $\ell_\infty$ robustness, they can achieve high DLR in the connection, where the highest DLR is $48.19\%$ in panel (a). One can also notice that when the epoch number for solving \eqref{eq: mc_adv} increases, the path becomes smoother. One can see that the robust paths also function as effective mode connectivity paths, where both the clean accuracy and loss (indicated by red lines) maintain consistent levels between the two endpoints $t=0$ and $t=1$ in panels (c). We also observe that the optimal points in panels (a), (b), and (c) yield similar DLR. The experiments indicate that RMC can find a path with high DLR. If the goal is to select an optimal model from the path, it is enough to only conduct the training with a small number of epochs.

\subsubsection{Improving Learning Efficiency of the RMC With a Self-Robust Mode Connectivity Module}\label{sec: srmc}

One drawback of the previous scheme is that it needs to initially pre-train two neural networks, which could be slow when the computational resources are limited. We ask:
\begin{tcolorbox}[before skip=2.0mm, after skip=2.0mm, boxsep=0.0cm, middle=0.1cm, top=0.1cm, bottom=0.1cm]
\noindent \textit{\textbf{(Q4)} How can we accelerate the learning process of the RMC?}
\end{tcolorbox}

Here we propose to replace RMC with a self-robust mode connectivity (SRMC) module in the learning process. SRMC can accelerate the endpoints training in the path search process and thus speed up RMC. We start with one set of neural network parameters $\phi(0; \boldsymbol \theta)=\boldsymbol \theta_1$ that is trained by \eqref{eq: at} with a fixed $\text{Dist}_i$. After the model achieves high robustness on $\mathcal D_i$, we retrain the model for a few epochs using \eqref{eq: at} with $\text{Dist}_j$. The new model we obtained will be placed at the endpoint $\phi(1; \boldsymbol \theta)=\boldsymbol \theta_2$. Now the low-loss high-robustness path can be found by optimizing \eqref{eq: mc_adv}. By leveraging the SRMC module, our proposed framework yields both high DLR and learning efficiency.











\subsection{Phase II: Robust Model Selection Via Robust Mode Connectivity-Based Optimization}\label{sec: opt}

\begin{algorithm}[h]
\caption{Robust Mode Connectivity-Based Optimization ($\ell_1, \ell_2, \ell_\infty$ perturbations)}
\label{alg: RMC_opt}
\begin{algorithmic}[1]
%\REQUIRE .
\STATE{Train three models for $T$ epochs using $\ell_1, \ell_2, \ell_\infty$ perturbations, respectively. (Training can be accelerated using the SRMC module proposed in Section~\ref{sec: srmc})}
\STATE{Apply Algorithm~\ref{alg: RMC} with $\ell_2, \ell_1$-AT trained models ($I$ includes $\ell_2, \ell_1$) and $\ell_\infty, \ell_1$-AT trained models ($I$ includes $\ell_\infty, \ell_1$) as $\phi(0; \boldsymbol \theta)$, $\phi(1; \boldsymbol \theta)$, and return model trajectories $\phi_{\boldsymbol \theta-\ell_\infty}(t), \phi_{\boldsymbol \theta-\ell_2}(t), \forall t \in [0,1]$. (pairs of perturbations can be selected in different ways)}
\STATE{Randomly pick points $t_{-\ell_\infty}$, $t_{-\ell_2}$ from  optimal regions for each model trajectory.}
\STATE{Train models for $T$ epochs using $\ell_\infty, \ell_2$ perturbations starting from $\phi_{\boldsymbol \theta-\ell_\infty}(t_{-\ell_\infty}), \phi_{\boldsymbol \theta-\ell_2}(t_{-\ell_2})$, respectively.}
\STATE{Apply Algorithm~\ref{alg: RMC} with the two models as $\phi(0; \boldsymbol \theta)$, $\phi(1; \boldsymbol \theta)$ with $I$ including $\ell_1, \ell_2, \ell_\infty$ perturbations.}
\STATE{Find the optimal point $t_{\text{opt}}$ from the model trajectory.}
\RETURN $\phi_{\boldsymbol \theta}(t_{\text{opt}})$
\end{algorithmic}
\end{algorithm}

\subsubsection{From RMC to RMC-Based Optimization}

Suppose we have neural networks that share the same structure but are trained with different settings, e.g., different types of perturbations, perturbation magnitudes, learning rates, batch size, etc. In that case, we can use RMC to search for candidates potentially leading us to better solutions or even global optimums. The intuition behind the claim is that low-loss \& high-DLR paths connect all the minimums, and thus it becomes easier for search algorithms to jump out of the sub-local minimums. We have seen the exciting property of the proposed RMC, which indicates that a larger population of NNs can result in higher DLR. Notice that RMC can serve as a component in a larger population-based optimization to select robust models with higher DLR. A natural question to ask is:
\begin{tcolorbox}[before skip=2.0mm, after skip=2.0mm, boxsep=0.0cm, middle=0.1cm, top=0.1cm, bottom=0.1cm]
\noindent \textit{\textbf{(Q5)} Can we develop a general population-based optimization method built on RMC modules to further improve the DLR of a single RMC?}
\end{tcolorbox}


The RMC-based optimization we developed below includes an evolving process of RMC units for multiple generations. As a starting point, we generate an initial population by training neural networks with data points augmented using different $\text{Dist}_i$ in \eqref{eq: at}. 
We use gradient descent to train these networks but pause the training when specific stop criteria have been met, e.g., the number of epochs or accuracy achieving the preset threshold. The initial population then varies, and the system selects candidates with the best performances. The two operations in our approach are unified through the RMC that connects two adversarially-trained neural network models on their loss landscape using a high-accuracy \& high-DLR path characterized by a simple curve. Candidates for the next generation are selected among the high DLR points on the curve. The process can be repeated and an optimal solution that enjoys the highest DLR is selected among the final candidates. 

Algorithm~\ref{alg: RMC_opt} shows the pipeline using an example of three types of perturbations. We first train three models with $\ell_\infty$-AT, $\ell_2$-AT, and $\ell_1$-AT for $T$ epochs. We then connect the $\ell_2$-AT model with the $\ell_1$-PGD model and connect the $\ell_\infty$-AT model with the $\ell_1$-AT model using the RMC for some additional epochs. The two model trajectories are denoted by $\phi_{\boldsymbol \theta-\ell_\infty}(t)$ and $\phi_{\boldsymbol \theta-\ell_2}(t), \forall t \in [0,1]$. Notice that the curves will not be perfectly flat. But there exist some regions containing points with high DLR. We will randomly pick a model from a small optimal region in each curve. In practice, we will find the point with the highest DLR and randomly pick a point around the optimal point. The rationale behind this is to increase diversity during the training. After obtaining the models $\phi_{\boldsymbol \theta-\ell_\infty}(t_{-\ell_\infty})$ and $ \phi_{\boldsymbol \theta-\ell_2}(t_{-\ell_2})$ from both trajectories, the two new endpoints are obtained by training each model $T$ epochs using the $\ell_p$-AT that is different from the types used in the previous RMC. In this specific case, we use $\ell_\infty$-AT and $\ell_2$-AT. Finally, we connect the two new endpoints with RMC for some additional epochs and find the new optimum $\phi_{\boldsymbol \theta}(t_{\text{opt}})$ at $t_{\text{opt}}$. In the case of two types of perturbations, one can start to train two models from a single optimal point or train one model from each of the two optimal points. We refer readers to Section~\ref{subsec: rmc_opt} for more details. It's pertinent to note that parameter curves derived from distinct models can be concurrently computed. This inherent parallelizability means that when we leverage parallel computing for generating independent parameter curves, the execution time is equivalent to the time of generating one parameter curve. In a more general scenario where there are $S$ types of perturbations, the process is the same, except that it contains more RMC units, as illustrated in Fig.~\ref{fig: framework}. We learn optimal points from pairs of models trained by AT under different perturbations and finally find an optimal point with the highest DLR.



\section{Enhancing Phase II efficiency: ERMC With Model Ensemble}\label{sec: ermc}

From the insights of the previous Phase II, it becomes apparent that to enhance robustness against diversified $\ell_p$ perturbations, multiple RMC procedures might be necessary. We pose the question:
\begin{tcolorbox}[before skip=2.0mm, after skip=2.0mm, boxsep=0.0cm, middle=0.1cm, top=0.1cm, bottom=0.1cm]
\noindent \textit{\textbf{(Q6)} Can enhanced robustness against diversified $\ell_p$ perturbations be attained within a single RMC process?}
\end{tcolorbox}

In the literature \cite{croce2019provable}, it is discussed that affine classifiers, including CNNs with ReLU activations, can withstand various $\ell_p$ norm attacks if they are already fortified against $\ell_1$ and $\ell_\infty$ perturbations. Theorem 3.1 in ~\cite{croce2019provable} posits that the convex hull of the union ball of the $\ell_1$ and $\ell_\infty$ provides satisfactory robustness to $\ell_p, 1 \le p\le \infty$ perturbationsTheorem 3.1 in ~\cite{croce2019provable} posits that the convex hull of the union ball of the $\ell_1$ and $\ell_\infty$ provides satisfactory robustness to $\ell_p$ perturbations, where $1 \le p\le \infty$. 

\begin{theorem}
    \cite{croce2019provable} Suppose that the classifier is affine. Let $C$ be the convex hull of the union ball of the $\ell_1$ and $\ell_\infty$. If the input dimension $d_{\mathbf x}$ is larger than or equal to two and $\delta_1 \in (\delta_\infty, d_{\mathbf x}\delta_\infty)$, then
    \begin{equation}
  \min_{\mathbb{R}^{d_{\mathbf x}}\backslash C} \|\mathbf x^\prime - \mathbf x\|_p = \frac{\boldsymbol \delta_1}{(\boldsymbol \delta_1/\boldsymbol \delta_\infty - \beta + \beta^q)^{1/q}}
    \end{equation}
where $\beta=\frac{\boldsymbol \delta_1}{\boldsymbol \delta_\infty} - \lfloor \frac{\boldsymbol \delta_1}{\boldsymbol \delta_\infty}\rfloor$ and $\frac{1}{p}+\frac{1}{q}=1$.
\end{theorem}

A recent approach, E-AT \cite{croce2022adversarial}, proposes using fine-tuning as an efficient transition from $\ell_\infty$-adversarial training (AT) to $\ell_1$-AT, aiming to improve robustness against a broader range of $\ell_p$ attacks. However, this method faces two key limitations: \ding{182} the fine-tuning process may compromise the model's original robustness to $\ell_\infty$ attacks; and \ding{183} a single model often lacks sufficient capacity to maintain strong robustness against both $\ell_\infty$ and $\ell_1$ perturbations simultaneously.

Our proposed RMC can naturally overcome these issues thanks to the power of population-based strategies. Here we propose an efficient robust mode connectivity (ERMC) strategy, leveraging SRMC to fine-tune endpoint $\phi(1; \boldsymbol \theta)$ with $\ell_1$-AT from $\phi(0; \boldsymbol \theta)$ obtained by $\ell_\infty$-AT. We then optimize the following objective to maintain robustness against both $\ell_1$ and $\ell_\infty$ attacks, effectively expanding the defense boundary and improving overall model resilience:  
\begin{equation}
\begin{aligned}
\min_{\boldsymbol \theta} &\mathbb E_{t \sim U(0,1)}   \displaystyle \mathbb E_{({\bf x},y) \sim \mathcal D_0 } \{\\&\sum_{\text{Dist}_i \in \{\|\cdot\|_1,\|\cdot\|_\infty\}} \max_{\text{Dist}_i({\bf x}^\prime, {\bf x}) \leq  \delta_i} \mathcal L( \phi(t; \boldsymbol \theta); ({\bf x}^\prime,y))\},
\label{eq: ermc_adv}
\end{aligned}
\end{equation} 
which results in a larger union ball, thereby enhancing the model's resilience against a broader range of perturbations. The detailed algorithm can be found in Algorithm~\ref{alg: ERMC}. ERMC is efficient as it only needs to conduct the connection once.

\begin{algorithm}[h]
\caption{Efficient Robust Mode Connectivity}
\label{alg: ERMC}
\begin{algorithmic}[1]
\REQUIRE A model $\phi(0; \boldsymbol \theta)$ trained with $\ell_\infty$-AT; initial model $\boldsymbol \theta^0$; the corresponding projections $P_{\boldsymbol{\delta}_1}$ and $P_{\boldsymbol{\delta}_\infty}$; training set $\mathcal D_0$; iteration number $J$; batch size $B$; initial perturbation $\boldsymbol{\delta}^{(0)}=\mathbf 0$.
\STATE{Create a copy of $\phi(0; \boldsymbol \theta)$ and retrain it with AT-$\ell_1$ for 10 epochs to obtain a model $\phi(1; \boldsymbol \theta)$.}
\STATE{$\boldsymbol \theta = \boldsymbol \theta^0$}
\FOR{each data batch $\mathcal D_b \in \mathcal D$ in each epoch $e \in E$}
\STATE{Uniformly select $t \sim U(0,1)$}
\FOR{$\forall \bf x \in \mathcal D_b$}
\FOR{$j = 1, \cdots, J$}
\STATE{\hspace{-1mm}$\boldsymbol{\delta}_1^{(j)} \leftarrow P_{\boldsymbol \epsilon_1}\big(\boldsymbol{\delta}^{(j-1)}-{\nabla_{\boldsymbol{\delta}}\mathcal L( \phi(t; \boldsymbol \theta); {\bf{x}}+{\boldsymbol{\delta}}^{(j-1)},y)}\big)$}
\STATE{\hspace{-1mm}$\boldsymbol{\delta}_\infty^{(j)} \leftarrow P_{\boldsymbol \epsilon_\infty}\big(\boldsymbol{\delta}^{(j-1)}-{\nabla_{\boldsymbol{\delta}}\mathcal L( \phi(t; \boldsymbol \theta); {\bf{x}}+{\boldsymbol{\delta}}^{(j-1)},y)}\big)$}
\ENDFOR
\STATE{$\boldsymbol \delta^{(j)} \leftarrow \arg\max_{\boldsymbol \delta_i^{(j)}, i \in \{1,\infty\}} {\mathcal L( \phi(t; \boldsymbol \theta); {\bf{x}} + \boldsymbol \delta_i^{(j)}, y)}$}
\ENDFOR
\STATE{$\boldsymbol \theta \leftarrow \boldsymbol \theta - \nabla_{\boldsymbol \theta} \sum_{\bf x \in \mathcal D_b} \mathcal L(\phi(t; \boldsymbol \theta); {\bf x} + \boldsymbol \delta^{(j)}, y)$}
\ENDFOR
\RETURN $\boldsymbol \theta$, $\phi(t; \boldsymbol \theta), \forall t \in [0,1]$
\end{algorithmic}
\end{algorithm}

Acknowledging the presence of numerous models along the trajectory that demonstrate significant resistance to $\ell_\infty$ and $\ell_1$ attacks, adopting a model ensemble technique seems a logical step to enhance robustness. By doing so, we create an aggregated model with a collective defense against both $\ell_\infty$ and $\ell_1$ disruptions. The process for selecting the ensemble involves identifying a segment $t\in [a,b]$ on the path $\phi(t; \boldsymbol \theta)$ where each point on the segment has robust accuracies surpassing two prefixed model selection thresholds $\alpha_\infty, \alpha_1$ under $\ell_\infty$ and $\ell_1$ attacks, respectively. From this segment, we select $n > 1$ models situated at intervals defined by $t = a + \frac{b - a}{n - 1}i$, with $i$ varying from $0$ to $n - 1$. In scenarios where there are several non-adjacent intervals that fulfill the selection criteria, the models are proportionally allocated based on the length of these intervals. This approach, with $n$ models chosen, is referred to as ERMC-$n$. We then calculate the class probability prediction by averaging the outputs from the final layers of these $n$ models.



\section{Experimental Results}\label{sec: exp}


Figures~\ref{fig: std_mc}, \ref{fig: adv_mc} show that using the proposed RMC can find a path with points in it enjoying high robustness on diversified $\ell_p$ perturbations. In this section, we conduct more comprehensive experiments on the Robust Mode Connectivity-Oriented Adversarial Defense.


\subsection{Settings}

We evaluate our proposed methods using the CIFAR-10, CIFAR-100 \cite{krizhevsky2009learning}, and ImageNet-100 \cite{russakovsky2015imagenet} datasets across the PreResNet110, WideResNet-28-10, and Vision Transformer-base architectures. By default, we conduct experiments on CIFAR-10 and PreResNet110. For our experiments, the considered perturbation types, denoted as $\textup{Dist}_i$s, are based on $\ell_\infty$, $\ell_2$, and $\ell_1$ norms, constrained by perturbations of $\delta = 8/255, 1$, and $12$, respectively. To obtain the endpoints' models, we employ AT. Our methods are benchmarked against the standard $\ell_\infty$-AT baseline, RMC on two randomly initialized models (RMC-RI), the Extreme norm Adversarial Training (E-AT) \cite{croce2022adversarial}, and the MSD Defense \cite{maini2020adversarial}. The evaluation methods encompass basic PGD adversarial attacks and Auto-Attack (AA) \cite{croce2020reliable} under $\ell_\infty$, $\ell_2$, $\ell_1$ norm perturbations and the MSD attack. The evaluation metrics include: (1) Standard accuracy on clean test data; (2) Robust accuracies under $\ell_\infty$, $\ell_2$, $\ell_1$-PGD adversarial attacks, MSD attack, and $\ell_\infty/\ell_2/\ell_1$ AA; (3) Accuracy on worst-case sample-wise (Union) using all three basic PGD adversarial attacks; and (4) DLR on $\ell_\infty$, $\ell_2$, $\ell_1$-PGD adversarial attacks for three types of perturbations and DLR on $\ell_\infty$, $\ell_2$ for two types of perturbations. All the following experiments are conducted on two NVIDIA RTX A100 GPUs.

\begin{figure*}[h]
  \centering
\subfloat[CIFAR100]{\includegraphics[trim=0 0 0 0,clip,width=0.25\textwidth]{Figures/cifar100.png}}
\subfloat[ImageNet-100]{\includegraphics[trim=0 0 0 0,clip,width=0.25\textwidth]{Figures/imagenet100.png}}
\subfloat[WideResNet-28-10]{\includegraphics[trim=0 0 0 0,clip,width=0.25\textwidth]{Figures/wide.png}}
\subfloat[Vision Transformer-base]{\includegraphics[trim=0 0 0 0,clip,width=0.25\textwidth]{Figures/vit.png}}
  \caption{{RMC is capable of identifying paths with points that have high DLR across various datasets and model architectures}. Figures (a) and (b), as well as (c) and (d), demonstrate that RMC performs effectively on the CIFAR-100 and ImageNet-100 datasets, as well as the WideResNet-28-10 and Vision Transformer-base architectures.}  
  \label{fig: rmc_vgg16_cifar100}
\end{figure*}

\subsection{A More Comprehensive Study of the Robust Mode Connectivity}



In this subsection, we aim to study the effectiveness of the proposed method \eqref{eq: mc_adv} on different models, architectures, and datasets. We will consider models trained under various settings. By default, we train endpoints' models 50/150 epochs and the paths are obtained by training an additional 50 epochs.


Here we evaluate the effectiveness of RMC on the CIFAR-100 and ImageNet-100 datasets, as well as the WideResNet-28-10 and Vision Transformer-base model architectures. We consider two types of perturbations that are generated from $\ell_\infty$ and $\ell_2$-PGD attacks. It can be seen from Fig.~\ref{fig: rmc_vgg16_cifar100} (a) and (b) that paths with high DLR points are obtained when CIFAR-10 is replaced with CIFAR-100 and ImageNet-100. Similarly, Fig.~\ref{fig: rmc_vgg16_cifar100} (c) and (d) demonstrate that paths with high DLR points are obtained when PreResNet110 is replaced with WideResNet-28-10 and Vision Transformer-base. These results underscore that RMC is versatile and can be effectively applied to various datasets and architectures.


\subsection{RMC with SRMC Modules}

We then tested the proposed SRMC modules to expedite the RMC. Starting with a $\ell_\infty$-AT model, we trained an additional $\ell_2$-AT model and a $\ell_1$-AT model over 5 epochs. Subsequently, we connected each of these child models with the original $\ell_\infty$-AT model. The results, depicted in Fig.~\ref{fig: adv_self_rmc}, demonstrate the presence of paths with regions of high robustness under both connections. This indicates that we don't need to train all models from scratch to obtain the desired paths. For our Phase I experiments using PreResNet110 on CIFAR-10 with a single GPU, the process of learning RMC, which involved training two endpoint AT models for 150 epochs and the parameter curve for 50 epochs, took an average of $2750$ minutes. Learning SRMC under identical settings took $1780$ minutes. In the Vision Transformer setup with one GPU, learning RMC averaged $5785$ minutes, whereas learning SRMC in the same conditions required $3592$ minutes.

\begin{figure}[h]
  \centering
  \includegraphics[trim=0 0 0 0,clip,width=.34\textwidth]{Figures/selfrmc_infty_l2.png}
  \caption{{A single SRMC module can also find paths with high DRL by connecting a $\ell_\infty$ model and a $\ell_2$ model}. } 
  \label{fig: adv_self_rmc}
\end{figure}




\begin{figure}[h]
  \centering
  \includegraphics[trim=0 0 0 0,clip,width=.46\textwidth]{Figures/rmc_2pert_opt_1p.png}
  \caption{{RMC-based optimization considering two types of perturbations (one single mid-optimal point) can result in paths with smoother and higher DLR than the path in Fig.~\ref{fig: adv_mc} left panel}. The left (right) endpoint is an $\ell_\infty$-AT ($\ell_2$-AT) trained model starting from a single optimal point of a path connected between two models, which are trained by $\ell_\infty$-AT and $\ell_2$-AT for 50 epochs.} 
  \label{fig: adv_mc_opt1}
\end{figure}

\subsection{Robust Mode Connectivity-Based Optimization}\label{subsec: rmc_opt}

As introduced in Section~\ref{sec: opt}, Phase II is an enhanced optimization process based on units of RMC (Phase I). We show the effectiveness of RMC-based optimization (Phase II) below. Training epochs for all the experiments below are 200 (allow parallel computing).



\noindent\textbf{Optimization on two types of perturbations.} We first consider $\ell_\infty$ and $\ell_2$ norm perturbations. We train two models for 50 epochs under these two types of perturbations, then leverage RMC to find a path between the two models. Initializing from a single optimal point (randomly select from $t\in [0.77,0.83]$) on the curve, we train two models parallelly with $\ell_\infty$-AT and $\ell_2$-AT for 50 epochs. Finally, we plot the mode connectivity curve based on the two AT-trained models, as shown in Fig.~\ref{fig: adv_mc_opt1}. We obtain a smoother path with higher DLR than the path in Fig.~\ref{fig: adv_mc} left panel. The optimal point achieved is  $48.8\%$ at $t=0.72$.



Now instead of selecting a single optimal point, we randomly pick two optimal points in the ranges of $t\in [0.27,0.33]$ and $t\in [0.77,0.83]$, respectively. We train two models with $\ell_\infty$-AT and $\ell_2$-AT for 50 epochs starting from each initial point. We then plot the RMC curve based on the two AT-trained models, as shown in Fig.~\ref{fig: adv_mc_opt12}. The drop in accuracy observed at approximately $t=0.5$ is attributed to the small number of epochs (50) used in RMC. Increasing the number of epochs would result in smoother curves. The optimal point achieved in this optimization process is 48.89\% (DLR) at $t=0.71$, indicating that higher robustness can be improved by using a larger population with higher diversity.

\begin{figure}[h]
  \centering
  \includegraphics[trim=0 0 0 0,clip,width=.46\textwidth]{Figures/rmc_2pert_opt.png}
  \caption{{RMC-based optimization considering two types of perturbations with two mid-optimal points is able to achieve higher robustness compared with only selecting a single mid-optimal point}. The left (right) endpoint is an $\ell_\infty$-AT ($\ell_2$-AT) trained model starting from two optimal points of a path connected between two models, which are trained by $\ell_\infty$-AT and $\ell_2$-AT for 50 epochs.} 
  \label{fig: adv_mc_opt12}
\end{figure}


\noindent\textbf{Optimization on three types of perturbations.} We take one more step by considering the $\ell_1$ norm perturbation. The process is shown in Algorithm~\ref{alg: RMC_opt}. $T=50$ and we use 50 additional epochs to learn RMC. The results of the final connection are shown in Fig.~\ref{fig: adv_mc_opt2}. The trend of the $\ell_1$-PGD curve is increasing from left to right and the trend of the $\ell_2$-PGD curve from $t=0.7$ to $t=1$ is decreasing. There exists an optimal point with DLR$=46.21\%$ at $t=0.93$. RMC-based optimization in the case of three types of perturbations can further boost models' DLR against $\ell_\infty$, $\ell_1$, $\ell_2$ adversarial attacks. Additionally, one can select multiple models from the curve and use ensemble methods to further improve performance. 

\begin{figure}[h]
  \centering
  \includegraphics[trim=0 0 0 0,clip,width=.34\textwidth]{Figures/rmc_3pert_opt.png}
  \caption{{RMC-based optimization considering three types of perturbations can further boost models' DLR against more types of attacks}. The two endpoints are trained by $\ell_\infty$-AT and $\ell_2$-AT for 50 epochs starting from the optimal points selected from two RMC curves.} 
  \label{fig: adv_mc_opt2}
\end{figure}





\begin{figure}[ht]

\begin{minipage}[b]{.48\linewidth}
  \centering
  \centerline{\includegraphics[width=5.0cm]{Figures/prs-150-10-50.png}}
  \centerline{(a) CIFAR-10}\medskip
\end{minipage}
\begin{minipage}[b]{.48\linewidth}
  \centering
  \centerline{\includegraphics[width=5.0cm]{Figures/prs100-150-10-50.png}}
  \centerline{(b) CIFAR-100}\medskip
\end{minipage}
\begin{minipage}[b]{.48\linewidth}
  \centering
  \centerline{\includegraphics[width=5.0cm]{Figures/wr-150-10-50.png}}
  \centerline{(c) WideResNet-28-10}\medskip
\end{minipage}
\hfill
\begin{minipage}[b]{0.48\linewidth}
  \centering
  \centerline{\includegraphics[width=5.0cm]{Figures/vit_pgd_pgd1.png}}
  \centerline{(d) ViT-base}\medskip
\end{minipage}
\caption{ERMC can find paths with high robustness against $\ell_\infty/\ell_2/\ell_1$ attacks by connecting a $\ell_\infty$ model and a $\ell_1$ model. The effectiveness of ERMC is validated on different datasets and model architectures.}
\label{fig: adv_self_rmc2}
\end{figure}


\subsection{Results on ERMC}

In ERMC, the models situated at the left and right endpoints undergo different training processes. The left endpoint model receives training with $\ell_\infty$-AT, whereas the right endpoint model is subsequently refined with AT fine-tuning focused on $\ell_1$-AT. These procedures' outcomes are illustrated in Fig.~\ref{fig: adv_self_rmc2}. The observations from this process are twofold: Firstly, ERMC demonstrates commendable performance across all tested datasets and architectures. Secondly, the process of fine-tuning has a noticeable impact on the models' inherent robustness. Models at each endpoint exhibit a high degree of robustness to the type of perturbations they were trained against (i.e., $\ell_\infty$ for the left endpoint and $\ell_1$ for the right) yet they show a relative vulnerability to the alternate type of perturbations (i.e., $\ell_1$ for the left endpoint and $\ell_\infty$ for the right).




\begin{table*}[!t]
\caption{Our Methods Achieve State-of-the-Art DLR Levels Under Various Perturbations. Furthermore, our methods consistently achieve the highest accuracy under Union, $\ell_\infty/\ell_2/\ell_1$ AA \cite{croce2020reliable} (with the lowest accuracies being denoted by braces), and MSD. For methods utilizing two types of perturbations, we compare them using DLR only under $\ell_\infty$-PGD and $\ell_2$-PGD attacks, marking the DLR (representing the lowest accuracy) with an underline. For those employing three types of perturbations, we assess them across all metrics, marking the DLR (the lowest accuracy) under the three basic $\ell_p$ attacks with an overline.
}
\label{tab: main}
\begin{center}

\resizebox{0.99\textwidth}{!}{
\begin{tabular}{l||c|c|c|c|c|c|c|c}
\hline
\hline
& Standard Accuracy & $\ell_\infty$-PGD ($\delta=8/255$) & $\ell_2$-PGD ($\delta=1$) & $\ell_1$-PGD ($\delta=12$) & DLR & Union & AA ($\ell_\infty/\ell_2/\ell_1$) \cite{croce2020reliable} & MSD \\
\hline
\begin{tabular}[c]{@{}c@{}} $\ell_\infty$-AT \cite{madry2018towards}   \end{tabular}  &   85.00\% & 49.03\% & 29.66\%  & 16.61\% &  / & 21.85\%  & 46.02\%/20.86\%/\{10.45\%\} & 15.27\% \\
\hline
\begin{tabular}[c]{@{}c@{}} MSD - Defense \\ (\textbf{two} types of pert)  \end{tabular}  &  81.61\% & 48.57\% & \underline{45.92\%} & 35.64\% & 45.92\% & 34.37\% & 46.6\%/42.13\%/\{31.55\%\} & 45.72\%  \\
\hline
\begin{tabular}[c]{@{}c@{}} RMC-RI \\ (\textbf{two} types of pert)  \end{tabular}  &  63.08\% & \underline{37.4\%} & 38.44\% & 30.47\% & 30.47\% & 29.22\% & 36.85\%/37.17\%/\{28.33\%\} & 36.9\%  \\
\hline
\begin{tabular}[c]{@{}c@{}} RMC \\ (ours, \textbf{two} types of pert)  \end{tabular}  &  80.90\%  & \underline{48.19\%} & 48.63\% & 38.05\% & 48.19\%  & 36.3\% & 46.74\%/45.16\%/\{34.4\%\} & 46.52\%  \\
\hline
\begin{tabular}[c]{@{}c@{}} RMC-based optimization \\ (ours, \textbf{two} types of pert)  \end{tabular}  &  81.36\%  & \underline{48.89\%} & 49.03\% & 38.83\% & 48.89\%  & 36.86\% & 47.66\%/45.73\%/\{35.18\%\} & 47.18\%  \\
\hline
\begin{tabular}[c]{@{}c@{}} MSD - Defense \cite{maini2020adversarial} \\ (\textbf{three} types of pert)  \end{tabular}  &  81.35\% & $\overline{40.14\%}$ & 48.58\% & 47.50\% & 40.14\% & 38.35\% & \{37.87\%\}/45.9\%/45.27\% & 38.20\%  \\
\hline
\begin{tabular}[c]{@{}c@{}} E-AT  \cite{croce2022adversarial} \\ (\textbf{three} types of pert)  \end{tabular}  &  79.3\% & $\overline{44.07\%}$ & 49.12\% & 49.82\% & 44.07\% & 41.08\% & \{41.41\%\}/46.5\%/47.82\% & 42.67\%  \\
\hline
\begin{tabular}[c]{@{}c@{}} RMC-based optimization \\ (ours, \textbf{three} types of pert)  \end{tabular}  &  81.76\% & $\overline{46.21\%}$ & 51.86\% & 46.23\% & 46.21\% & 41.47\% & 44.58\%/49.35\%/\{43.42\%\} & 44.75\%   \\
\hline
\begin{tabular}[c]{@{}c@{}} RMC-based optimization \\ 5-model ensemble \\ (ours, \textbf{three} types of pert)  \end{tabular}  &  78.35\% & 55.91\% & 56.78\% & $\overline{51.05\%}$ & 51.05\% & 49.39\% & 50.15\%/49.85\%/{\{\bf{49.83\%}\}} & 48.79\%   \\
\hline
\begin{tabular}[c]{@{}c@{}} RMC-based optimization \\ with SRMC modules \\ (ours, \textbf{three} types of pert)  \end{tabular}  &  80.39\%  & $\overline{46.10\%}$ & 48.92\% & 46.39\% & 46.10\%  & 42.03\% & 44.95\%/46.66\%/\{43.91\%\} & 45.07\%   \\
\hline
\begin{tabular}[c]{@{}c@{}} ERMC-1 \\ (ours, \textbf{three} types of pert) \end{tabular}  &  82.66\%  &   $\overline{46.54\%}$ & 48.76\% & 47.06\% & 46.54\% &  41.94\% & 44.88\%/45.88\%/\{43.97\%\} & 44.88\%   \\
\hline
\begin{tabular}[c]{@{}c@{}} ERMC-3 \\ (ours, \textbf{three} types of pert)  \end{tabular}  &  79.61\%  & 49.29\% & 51.32\% & $\overline{48.49\%}$ & 48.49\% &  45.27\% & \{42.88\%\}/44.57\%/47.37\% & 43.31\%   \\
\hline
\begin{tabular}[c]{@{}c@{}} ERMC-5 \\ (ours, \textbf{three} types of pert) \end{tabular}  &  79.41\%  & 55.46\% & 57.28\% & $\overline{53.97\%}$ & \bf{53.97\%} &  \bf{51.41\%} & {\{49.33\%}\}/50.55\%/52.41\% & {\bf{49.78\%}}   \\
\hline
\hline
\end{tabular}}
\end{center}
\end{table*}


\subsection{A Comprehensive Comparison}


For MSD Defense, RMC-RI, RMC, and RMC-based optimization (when considering only two types of perturbations), we evaluated them using the $\ell_\infty$-PGD and $\ell_2$-PGD attacks, given that the $\ell_1$-PGD attack was not considered during training. The DLR (representing the lowest robust accuracy) for these attacks is indicated with an underline. 

For methods that employ three types of perturbations, we assessed them under the $\ell_\infty$-PGD, $\ell_2$-PGD, $\ell_1$-PGD, AA, MSD attacks, and the union metric. In RMC-based optimization, a 5-model ensemble means that we select five models from the curve shown in Fig.~\ref{fig: adv_mc_opt2} and ensemble them. For the model ensemble, the model selection thresholds are set at $\alpha_\infty=37\%$ for $\ell_\infty$ robustness and $\alpha_1=43\%$ for $\ell_1$ robustness. We determine the lowest accuracy among the $\ell_\infty$, $\ell_2$, and $\ell_1$ norms within the AA framework and mark it with braces. The DLRs (lowest robust accuracies) for the $\ell_\infty$-PGD, $\ell_2$-PGD, and $\ell_1$-PGD attacks are denoted with an overline. Additionally, we emphasized the highest accuracy values in the union, AA, and MSD columns. 


From Table~\ref{tab: main}, the following observations can be made: (1) RMC with two types of perturbations outperforms MSD with two types of perturbations on DLR by $2.27\%$ and also surpasses RMC-RI by $10.79\%$.; (2) The RMC-based optimization with two types of perturbations yields a slightly higher DLR than RMC and excels over RMC in all other metrics; (3) When considering three types of perturbations, RMC-based optimization surpasses both MSD Defense and E-AT in DLR by $6.07\%$ and $2.14\%$. Moreover, it exhibits accuracy improvements of $3.12\%$, $5.55\%$, and $6.55\%$ ($0.39\%$, $2.01\%$, and $2.08\%$) over MSD Defense (and E-AT) under the Union, AA, and MSD Attack metrics, respectively; (4) The RMC-based optimization method shows a trade-off between clean accuracy and DLR. However, its clean accuracy drop of $3.24\%$, when benchmarked against $\ell_\infty$-AT, is less severe than that observed in other defense methods like MSD Defense and E-AT. Currently, the model selection process in RMC prioritizes robustness (DLR) over clean accuracy, but this could be adjusted in future implementations to achieve a better balance between the two; (5) The RMC-based optimization with SRMC modules can achieve similar (slightly lower) DLR performance compared to the RMC-based optimization with three types of perturbations, and even has slightly higher accuracy under the AA and Union metric; (6) Using a multi-model ensemble method can further enhance the performance of RMC-based optimization; (7) ERMC-1 reaches similar performance as RMC-based optimizataion. As the number of models $n$ increases, the performance of ERMC correspondingly improves. When $n$ reaches 5, ERMC-5 outperforms all other methods in terms of accuracy improvements under DLR, Union, and MSD.

In conclusion, our proposed Robust Mode Connectivity-Oriented Adversarial Defense shows remarkable performance across a variety of metrics. The RMC-based optimization (Phase II) delivers a higher DLR compared to RMC (Phase I) alone. ERMC can achieve high robustness by only conducting one RMC process. On the whole, the Robust Mode Connectivity-Oriented Adversarial Defense introduces a novel defense paradigm rooted in population-based optimization, effectively enhancing the robustness of Neural Networks (NNs).





\section{Conclusion}\label{sec: conclusion}
In this work, we introduced a Robust Mode Connectivity (RMC)-oriented adversarial defense framework that leverages population-based optimization to strengthen neural networks against diversified $\ell_p$ attacks. Our two-phase design enables the discovery of robust paths (Phase I) and systematic selection of high-performing models through RMC-based optimization (Phase II). To improve efficiency, we further proposed the Efficient Robust Mode Connectivity (ERMC), which combine theoretical guarantees with practical scalability. Extensive experiments across CIFAR-10, CIFAR-100, and ImageNet-100, as well as multiple architectures, demonstrated that our approach consistently outperforms existing methods, achieving superior diversified $\ell_p$ robustness while maintaining competitive accuracy. Overall, this work establishes population-based mode connectivity as a powerful and generalizable principle for adversarial defense. Future directions include extending RMC to large-scale foundation models, integrating it with certified robustness techniques, and exploring applications in safety-critical domains such as healthcare and power systems.

\bibliographystyle{ieeetr}
\bibliography{ref_adv,ref_new}

\newpage
\twocolumn
\pagestyle{empty}


\vfill
\end{document}



	\section{PE for locally reachable nonlinear systems}\label{sec_main3}
Consider an unknown nonlinear system of the form
\begin{equation}
	x_{k+1} = f(x_k,u_k),\label{eqn_NLsys}
\end{equation}%
with $x_k\in\mathbb{R}^n,u_k\in\mathbb{R}^m$ being the state and input vectors, respectively, and $f:\mathbb{R}^n\times\mathbb{R}^m\to\mathbb{R}^n$ is an unknown function satisfying $f(\mathbf{0},\mathbf{0})=\mathbf{0}$. For $\mu\in\mathbb{Z}_{>0}$, the set of all states which can be reached from $x_0$ in $\mu$ steps is defined as
\begin{align}
	&\mathcal{R}_\mu(x_0) \hspace{-1mm}=\hspace{-1mm} \left\lbrace \hspace{-0.5mm} x_\mu\in\mathbb{R}^n \,\left|\, \begin{aligned}
		&\exists\, u_{[0,\mu-1]},\, u_k\in\mathbb{R}^m,\\ &\textup{s.t.}\,\,  x_{k+1}\hspace{-0.5mm}=\hspace{-0.5mm}f(x_k,u_k),\,\forall k\hspace{-0.5mm}\in\hspace{-0.5mm}\mathbb{Z}_{[0,\mu-1]}.
	\end{aligned} \hspace{-0.5mm} \right. \right\rbrace \hspace{-1mm}.\notag
\end{align}
It was shown in \cite{Melkior20} how one can, under certain assumptions, obtain a guaranteed under-approximation of the reachable set of a nonlinear system with unknown dynamics. For the remainder of this section, we make the following assumption.
\begin{assumption}\label{assmp_reachability}
	For the system \eqref{eqn_NLsys}, there exists $\mu\in\mathbb{Z}_{>0}$ such that the origin is contained in the interior of $\mathcal{R}_\mu(\mathbf{0})$.
\end{assumption}%

Assumption~\ref{assmp_reachability} implies that the system is locally reachable at the origin. A sufficient condition for local reachability at $x=\mathbf{0}$ is that the linearization of system~\eqref{eqn_NLsys} at the origin is controllable (cf. \cite[Lemma 3.7.8]{Sontag13}).

Consider now $r$ basis functions $\theta_j:~\mathbb{R}^n\times \mathbb{R}^m\to\mathbb{R}$ which satisfy $\theta_j(\mathbf{0},\mathbf{0})=0$, $j\in\mathbb{Z}_{[1,r]}$, and denote the stacked vector of them by $\Theta(x_k,u_k)=\begingroup
\setlength\arraycolsep{1pt}\begin{bmatrix}\theta_1(x_k,u_k) & \cdots & \theta_r(x_k,u_k) \end{bmatrix}^\top\endgroup$. Suppose that the functions are linearly independent on arbitrary domains with non-empty interior\footnote{This assumption is satisfied for sinusoidal functions, exponential functions and monomials, among others \cite{Christensen06}. Note that basis functions that do not satisfy $\theta_j(\mathbf{0},\mathbf{0})=0$ can be suitably shifted by a constant.} $D_x\times D_u\subset\mathbb{R}^n\times~\mathbb{R}^m$. The objective is to design inputs $\{u_k^{(j)}\}_{k=0}^{N_j-1}$ such that the sequences of basis functions $\{\hat{\Theta}_k^{(j)}\}_{k=0}^{N_j-1}$ (with $\hat{\Theta}_k^{(j)}\coloneqq\Theta(x_k^{(j)},u_k^{(j)})$) are collectively persistently exciting of order $L$. According to Definition \ref{def_cPE}, the following mosaic Hankel matrix must have full row rank
\begin{align}
	&\mathcal{H}_L(\vartheta) =\label{eqn_MosaicHankelTheta} \begin{bmatrix}
		\mathscr{H}_L(\hat{\Theta}^{(1)}) & \cdots & \mathscr{H}_L(\hat{\Theta}^{(r)})
	\end{bmatrix},
\end{align}
where $\vartheta = \begin{bmatrix}
	(\hat{\Theta}^{(1)})^\top & \cdots & (\hat{\Theta}^{(r)})^\top
\end{bmatrix}^\top$.

Consider a submatrix of \eqref{eqn_MosaicHankelTheta} composed of the $L+\mu$ element of each of the $r$ sequences of basis functions
\begin{align}
	&W =
	\begin{bmatrix}
		\hat{\Theta}^{(1)}_{L+\mu-1} & \hat{\Theta}^{(2)}_{L+\mu-1} & \cdots & \hat{\Theta}^{(r)}_{L+\mu-1}
	\end{bmatrix}.\label{eqn_W}
\end{align}
Similar to Section \ref{sec_main2}, we would like $W$ to be invertible. Since the state at time $x_{L+\mu-1}^{(j)}$ is not a free variable, ensuring invertibility of $W$ is a control problem. In particular, one must select the inputs such that the corresponding state and input pairs at time $L+\mu-1$ result in an invertible matrix $W$. In a data-driven setting, such a control problem is difficult to solve since no model knowledge is available and - to begin with - no persistently exciting data is available yet to apply data-driven control techniques. Therefore, we first show in Lemma~\ref{lemma_rexperimentsexist} that there exist such $r$ input sequences $u_{[0,L+\mu-1]}^{(j)}$, $j\in\mathbb{Z}_{[1,r]}$, which make $W$ invertible and then prove in Theorem~\ref{thm_generalPE} how invertibility of $W$ results in collective PE of the basis functions. Later, in the following Subsection~\ref{sec_flatPE} we illustrate how, under suitable assumptions, one can indeed find the desired control inputs a priori, which guarantee collective PE of any order for sequences of basis functions that depend on input and output data of SISO flat systems.

\begin{lemma}\label{lemma_rexperimentsexist}
	Let Assumption \ref{assmp_reachability} be satisfied and suppose that $r$ basis functions $\theta_j:\mathbb{R}^n\times\mathbb{R}^m\to\mathbb{R}$ are linearly independent on $\mathcal{R}_\mu(\mathbf{0})\times\mathbb{R}^m$. Then there exist $r$ sequences $u_{[0,L+\mu-1]}^{(j)}$, $j\in\mathbb{Z}_{[1,r]}$, which when applied to \eqref{eqn_NLsys} starting from~$x_0^{(j)}=\mathbf{0}$, result in $x_{L+\mu-1}^{(j)}$, such that $W$ in \eqref{eqn_W} is invertible.
\end{lemma}
\begin{proof}
	Using similar arguments to the proof of Theorem~\ref{thm_alwaysexistslambda}, it can be shown by linear independence of the basis functions $\theta_j$ on $\mathcal{R}_\mu(\mathbf{0})\times\mathbb{R}^m$ that there exists $r$ pairs $(x_{L+\mu-1}^{(j)},u_{L+\mu-1}^{(j)})\in \mathcal{R}_\mu(\mathbf{0})\times\mathbb{R}^m$ such that $W$ is invertible. 
	
	Since $f(\mathbf{0},\mathbf{0})=\mathbf{0}$, then starting from zero initial conditions and setting the input to $u^{(j)}_{[0,L-2]}=0$, one can express the state $x_{L+\mu-1}^{(j)}$ in terms of $u^{(j)}_{[L-1,L+\mu-2]}$ only, i.e., $x_{L+\mu-1}^{(j)} = f(f\cdots(f(\mathbf{0},u_{L-1}^{(j)}),u_{L}^{(j)})\cdots,u_{L+\mu-2}^{(j)})$. Finally, since the system is locally reachable by Assumption~\ref{assmp_reachability}, there exist inputs $u_{[L-1,L+\mu-1]}^{(j)}$, $j\in\mathbb{Z}_{[1,r]}$, which steer the system from $x_{L-1}^{(j)}=\mathbf{0}$ to $x_{L+\mu-1}^{(j)}$ in $\mu$~steps.
\end{proof}

The following theorem shows how to use the results of Lemma \ref{lemma_rexperimentsexist} to obtain collectively persistently exciting sequences of basis functions of any order $L$.
\begin{theorem}\label{thm_generalPE}
	Let Assumption~\ref{assmp_reachability} hold. Given $L,\mu\in\mathbb{Z}_{>0}$ and $r$ basis functions $\theta_j:\mathbb{R}^n\times\mathbb{R}^m\to\mathbb{R}$ that are linearly independent on $\mathcal{R}_\mu(\mathbf{0})\times\mathbb{R}^m$ and satisfy $\theta_j(\mathbf{0},\mathbf{0})=0$, let $N_j\geq2L+\mu-1$ for $j\in\mathbb{Z}_{[1,r]}$. Furthermore, let the sequences $\{u_k^{(j)}\}_{k=0}^{N_j-1}$ take the form
	\begin{equation}
		u_k^{(j)} = \begin{cases}
			\eta^{(j)}_{[0,\mu]}, \quad & k\in\mathbb{Z}_{[L-1,L+\mu-1]},\\
			\mathbf{0}, \quad &\textup{otherwise},
		\end{cases}\label{eqn_PElocallyreachable}
	\end{equation}
	where $j\in\mathbb{Z}_{[1,r]}$, and $\eta^{(j)}_{[0,\mu]}$ are such that $W$ in \eqref{eqn_W} is invertible. If \eqref{eqn_PElocallyreachable} are applied to \eqref{eqn_NLsys} starting from $x_0^{(j)}=\mathbf{0}$, then for the matrix in \eqref{eqn_MosaicHankelTheta} it holds that $\textup{rank}(\mathcal{H}_L(\vartheta)) = rL$.
\end{theorem}
\begin{proof}
	Each block row of \eqref{eqn_MosaicHankelTheta} has $r$ linearly independent columns given by the columns of $W$. Notice that each $\mathscr{H}_L(\hat{\Theta}^{(j)})$, $j\in\mathbb{Z}_{[1,r]}$, in \eqref{eqn_MosaicHankelTheta} has a lower block-anti-triangular structure due to $f(\mathbf{0},\mathbf{0})=\mathbf{0},\,\Theta(\mathbf{0},\mathbf{0})=\mathbf{0}$ and the choice of the inputs \eqref{eqn_PElocallyreachable}. As a result, every block row \eqref{eqn_MosaicHankelTheta} is linearly independent from the others. Since there are $L$ such block rows, it holds that rank$(\mathcal{H}_L(\vartheta))=rL$.
\end{proof}

Notice that the results of Lemma~\ref{lemma_rexperimentsexist} for locally reachable nonlinear systems are analogous to that of Theorem~\ref{thm_alwaysexistslambda} for Hammerstein systems. However, formulating a nonlinear feasibility problem similar to \eqref{eqn_feasibility} to find $u_{[0,L+\mu-1]}^{(j)}$, $j\in\mathbb{Z}_{[1,r]}$ would require knowledge of the unknown function $f$.

It was observed in simulations that randomly sampling the input sequences $\eta^{(j)}_{[0,\mu]}$, $j\in\mathbb{Z}_{[1,r]}$, in Theorem~\ref{thm_generalPE} from a uniform distribution typically results in a corresponding invertible matrix $W$. However, such a heuristic approach is not always guaranteed to achieve this result. To systematically find the desired input sequences, one must impose additional assumptions on the {class} of systems and the {choice} of basis functions. To this end, we consider in the next subsection SISO flat nonlinear systems (which are locally reachable at the origin), and show how one can guarantee PE of any order $L>0$ a priori, for a specific choice of basis functions.
%%~~~~~~~~~~~~~~~~~~~~~~~~~~~~~~~~~~~~~~~~~~~~~~~~~~~~~~~~~~~~~~~~~~~~~~~%%
\subsection{SISO flat nonlinear systems}\label{sec_flatPE}
Consider an unknown SISO flat system of the form
\begin{equation}
	\begin{aligned}
		x_{k+1} = f(x_k,u_k), \quad y_k = h(x_k),
	\end{aligned}\label{eqn_flatsys}
\end{equation}
where \(x_k\in\mathbb{R}^n, u_k,\,y_k\in\mathbb{R}\) and $f:\mathbb{R}^n\times\mathbb{R}\to\mathbb{R}^n$, $h:~\mathbb{R}^n\to~\mathbb{R}$ are smooth unknown functions with $f(\mathbf{0},0)=\mathbf{0}$ and $h(\mathbf{0})=0$. Let $f_O^j(x_k)$ denote the $j-$th iterated composition of the undriven dynamics $f(x_k,0)$.\par
%
Since the system is flat (i.e., has a well defined relative degree equal to the system dimension $n$, cf. \cite[Sec. III.A]{AlsaltiBerLopAll2021}), it can be transformed into the discrete-time normal form provided that $0\in\textup{Im}\left(h(f_O^{n-1}(f(x,\cdot)))\right)$ holds for all $x\in\mathbb{R}^n$ (cf. \cite[Sec. 2]{MonacoNor1987} for more details). This means that there exists an invertible (w.r.t. ${v}_k$) control law ${u}_k=q({x}_k,{v}_k)$, with $q:\mathbb{R}^n\times\mathbb{R}\to\mathbb{R}$ and an invertible coordinate transformation $\xi_k=T({x}_k)=y_{[k,k+n-1]}$, such that
\begin{equation}
	\begin{matrix}
		\xi_{k+1} = {A}\xi_k + {B}{v}_k, \qquad
		{y}_k = {C}\xi_k,
	\end{matrix}
	\label{BINF}%
\end{equation}%
Furthermore, $A,B,C$ are in the Brunovsky canonical form (cf. \cite[Thm.~2]{AlsaltiBerLopAll2021}) which is a controllable/observable triplet. Hence, the system is $n$ steps locally reachable at the origin. The synthetic input $v_k$ takes the form
\begin{equation}
	\begin{aligned}
		v_k = h(f_O^{n-1}(f(x_k,u_k))).
	\end{aligned}\label{eqn_expressionforv}
\end{equation}

A sufficient condition for the analogue of Theorem~\ref{thm_FL} to flat systems \cite[Prop. 1]{AlsaltiBerLopAll2021}, and for designing controllers from data in \cite[Cor. 2]{DePersis22}, is persistence of excitation of a sequence of basis functions which contain $h\circ f_O^{n-1}\circ f$ in their span. To check the PE condition, one typically performs an experiment of length $N\geq (r+1)L-1$, collects the corresponding state or output measurements and then verifies the rank of the resulting Hankel matrix.

In this section, we illustrate how one can enforce PE of any order for a \textit{particular choice} of basis functions \textit{a priori}. A specific choice of basis functions may, in general, not contain the unknown nonlinearity \eqref{eqn_expressionforv} in its span. Nonetheless, enforcing PE of such basis functions is still useful for, e.g., designing locally stabilizing controllers for unknown SISO flat systems (cf. \cite[Sec. VII.B]{DePersis22}), and for data-driven nonlinear predictive control \cite{Alsalti2021c}, provided that the basis functions result in a good local approximation of~\eqref{eqn_expressionforv}. 

Since the map from $u$ to $v$ is invertible and since $f(\mathbf{0},0)=\mathbf{0}$ and $h(\mathbf{0})=0$, a non-zero input applied to the system from zero initial conditions results in a non-zero value of $v$ in \eqref{eqn_expressionforv}. Moreover, invertibility implies that for all $\delta_1,\delta_2\in\mathbb{R}$, the following holds
\begin{equation}
	\delta_1\hspace{-0.5mm}\neq \hspace{-0.5mm}\delta_2 \hspace{-1mm}\iff\hspace{-1mm} h(f_O^{n-1}(f(\mathbf{0},\delta_{1})))\hspace{-0.5mm}\neq\hspace{-0.5mm} h(f_O^{n-1}(f(\mathbf{0},\delta_{2}))).\label{eqn_uniquenessofv}
\end{equation}

We exploit this fact to prove the following lemma, which will be needed later for the main result of this subsection.
\begin{lemma}\label{lemma_vandermonde}
	For $t\in\mathbb{Z}_{>0}$ let $\delta_j\neq0$, $j\in\mathbb{Z}_{[1,t]}$, be mutually distinct values and define $v_{\delta_j}\coloneqq h(f_O^{n-1}(f(\mathbf{0},\delta_j)))$. Then, the following matrix is invertible:
\end{lemma}
\begin{equation}
	\Omega = \begin{bmatrix}
		v_{\delta_1} & v_{\delta_2} & \cdots & v_{\delta_t}\\
		v_{\delta_1}^2 & v_{\delta_2}^2 & \cdots & v_{\delta_t}^2\\
		\vdots & \vdots & \ddots & \vdots\\
		v_{\delta_1}^{t} & v_{\delta_2}^{t} & \cdots & v_{\delta_t}^{t}
	\end{bmatrix}.\label{eqn_OmegaVandermonde}
\end{equation}
\begin{proof}
	Since for $j\in\mathbb{Z}_{[1,t]}$, $\delta_j\neq0$ are mutually distinct values, it follows that the corresponding values $v_{\delta_j}$ are also distinct and non-zero (compare the discussion above Lemma~\ref{lemma_vandermonde}). The matrix $\Omega$ can be written as $\Omega = V^\top \Delta$, where $V\in\mathbb{R}^{t\times t}$ is a square Vandermonde matrix composed of the distinct $v_{\delta_j}$ and, hence, invertible and $\Delta\in\mathbb{R}^{t\times t}$ is a diagonal matrix containing $v_{\delta_j}$. The proof is concluded by noting that $V$ and $\Delta$ are invertible matrices.
\end{proof}

In the following, we consider monomial basis functions in the transformed state and input up to some finite order $t\in\mathbb{Z}_{>0}$, and hence $r=t(n+1)$.
\begin{align}
	&\Theta(\xi_k,u_k) = \label{eqn_specificchoice}\begingroup
	\setlength\arraycolsep{2pt}\begin{bmatrix}
		u_k & u_k^2 & \cdots & u_k^t& \xi_k^\top & (\xi_k^2)^\top & \cdots & (\xi_k^t)^\top 
	\end{bmatrix}^\top\endgroup\hspace{-2mm}.
\end{align}
The powers are defined element-wise, i.e., $\xi_k^t=[\xi_{1,k}^t \,\, \cdots \,\, \xi_{n,k}^t]^\top$. Notice that these functions depend only on the inputs and outputs of \eqref{eqn_flatsys} since $\xi_k=y_{[k,k+n-1]}$ (cf. \eqref{BINF}). In the following theorem, we show how to choose input sequences $\{u_k^{(j)}\}_{k=0}^{N_j-1}$, $j\in\mathbb{Z}_{[1,r]}$, such that the resulting sequences of basis functions $\{\hat{\Theta}_k^{(j)}\}_{k=0}^{N_j-1}$ are collectively persistently exciting of order $L>0$, i.e., that the corresponding mosaic Hankel matrix $\mathcal{H}_L(\vartheta)$ of the form \eqref{eqn_MosaicHankelTheta} has full row rank.
\begin{theorem}\label{thm_aprioriFL}
		For $t\in\mathbb{Z}_{>0}$, let $\delta_j\neq0$, $j\in~\mathbb{Z}_{[1,t(n+1)]}$, be mutually distinct values. For $L\in\mathbb{Z}_{>0}$, $N_j\geq 2L+n-1$ and the basis functions in \eqref{eqn_specificchoice}, let $\{u_k^{(j)}\}_{k=0}^{N_j-1}$ take the form in \eqref{eqn_PElocallyreachable} with the corresponding $\eta^{(j)}_{[0,n]}$ given by
		\begin{equation}
			\eta^{(j)}_{[0,n]} = \begin{cases}
				\begin{bsmallmatrix}
					\mathbf{0}_{n-j\times1}\\ \delta_j\\ \mathbf{0}_{j\times1}
				\end{bsmallmatrix}, \quad & \textup{for }j\in\mathbb{Z}_{[1,n]},\\
				&\boldsymbol{\vdots}\\
				\begin{bsmallmatrix}
					\mathbf{0}_{tn-j\times1}\\ \delta_j\\ \mathbf{0}_{j-(t-1)n\times1}
				\end{bsmallmatrix}, \quad & \textup{for }j\in\mathbb{Z}_{[(t-1)n,tn]},\\
				\begin{bsmallmatrix}
					\mathbf{0}_{n\times1}\\ \delta_j
				\end{bsmallmatrix}, \quad & \textup{for }j\in\mathbb{Z}_{[tn+1,t(n+1)]}.
			\end{cases}
			\label{eqn_PEinputSISOflat}
		\end{equation}
		If \eqref{eqn_PEinputSISOflat} are applied to \eqref{eqn_flatsys} starting from zero initial conditions, then $\textup{rank}(\mathcal{H}_L(\vartheta))=t(n+1)L$.
\end{theorem}
\begin{proof}
	Without loss of generality, let $N_j=2L+n-1$ for all $j\in\mathbb{Z}_{[1,t(n+1)]}$. For $c\in\mathbb{Z}_{[0,n-1]}$, we define $v_{\delta_j,[0,c]}~\coloneqq~\begingroup
		\setlength\arraycolsep{1pt}\begin{bmatrix}
			h(f_O^{n-1}(f(\mathbf{0},\delta_{j}))) \,\, \cdots \,\, h(f_O^{n+c-1}(f(\mathbf{0},\delta_{j})))
		\end{bmatrix}^\top\endgroup\hspace{-1.5mm}$ (with some abuse of notation we also use $v_{\delta_j,0}=v_{\delta_j}$). Since the system \eqref{BINF} is in the Brunovsky form, applying the inputs $\{u_k^{(j)}\}_{k=0}^{N_j-1}$ as defined in the theorem statement from zero initial conditions results in
		\begin{equation}
			\xi_{L+n-1}^{(j)}\hspace{-1mm} = \hspace{-1mm}\begin{cases}
				\begin{bsmallmatrix}
					\mathbf{0}_{n-j\times 1}\\ v_{\delta_j,[0,j-1]}
				\end{bsmallmatrix}, \, & \textup{for }j\in\mathbb{Z}_{[1,n]},\\
				&\boldsymbol{\vdots}\\
				\begin{bsmallmatrix}
					\mathbf{0}_{tn-j\times 1}\\ v_{\delta_j,[0,j-(t-1)n-1]}
				\end{bsmallmatrix}, \, & \textup{for }j\in\mathbb{Z}_{[(t-1)n+1,tn]},\\
				\mathbf{0},\,&\textup{for }j\in\mathbb{Z}_{[tn+1,t(n+1)]}.
			\end{cases}\label{eqn_XiStateForW}
	\end{equation}%
	Now, we consider a submatrix of $\mathcal{H}_L(\vartheta)$ of the form of $W$ in \eqref{eqn_W}. For the choice of basis functions in \eqref{eqn_specificchoice}, the inputs $\{u_k^{(j)}\}_{k=0}^{N_j-1}$ as defined in the theorem statement and the corresponding state values in \eqref{eqn_XiStateForW}, the matrix $W$ takes the form \eqref{eqn_Wflat} (see next page). Following similar arguments to the proof of Lemma~\ref{lemma_vandermonde}, one can show that $W_u\in\mathbb{R}^{t\times t}$ is invertible since $\delta_j$, $j\in\mathbb{Z}_{[tn+1,t(n+1)]}$, are non-zero and mutually distinct values.
		\begin{figure*}[!t]
			\normalsize
			\begin{align}
					W = \begin{bmatrix}
						\mathbf{0}& W_u\\
						W_\xi & \mathbf{0}
					\end{bmatrix} = \left[\begin{array}{cccc|ccc}
						0 & 0 & \cdots & 0 & \delta_{tn+1} & \cdots & \delta_{t(n+1)}\\[-1ex]
						\vdots & \vdots & \vdots & \vdots & \vdots & \vdots & \vdots\\[-0.5ex]
						0 & 0 & \cdots & 0 & \delta_{tn+1}^t & \cdots & \delta_{t(n+1)}^t\\[-0.25ex]
						\hline
						\begin{pmatrix}
							\mathbf{0}_{n-1\times1}\\[-0.5ex] v_{\delta_1}
						\end{pmatrix} & \begin{pmatrix}
							\mathbf{0}_{n-2\times1}\\[-0.5ex] v_{\delta_2,[0,1]}
						\end{pmatrix} & \cdots & \begin{pmatrix}
							v_{\delta_{tn},[0,n-1]}
						\end{pmatrix} & \mathbf{0} & \cdots & \mathbf{0}\\[-1ex]
						\vdots & \vdots & \vdots & \vdots & \vdots & \vdots & \vdots\\[-1ex]
						\begin{pmatrix}
							\mathbf{0}_{n-1\times1}\\[-0.5ex] v_{\delta_1}
						\end{pmatrix}^t & \begin{pmatrix}
							\mathbf{0}_{n-2\times1}\\[-0.5ex] v_{\delta_2,[0,1]}
						\end{pmatrix}^t & \cdots & \begin{pmatrix}
							v_{\delta_{tn},[0,n-1]}
						\end{pmatrix}^t & \mathbf{0} & \cdots & \mathbf{0}
					\end{array}
					\right].
					\label{eqn_Wflat}
				\end{align}%}
			\hrulefill
			\vspace{-1em}
		\end{figure*}
		Using the columns of $W_\xi\in\mathbb{R}^{tn\times tn}$ in \eqref{eqn_Wflat}, we construct $n$ submatrices $\overline{W}_{i,\xi}\in\mathbb{R}^{tn\times t}$, $i\in\mathbb{Z}_{[1,n]}$, of the form
		\begin{align*}
			&\overline{W}_{i,\xi} =\\
			&\begingroup
			\setlength\arraycolsep{1.5pt}\begin{bmatrix}
				\begin{pmatrix}
					\mathbf{0}_{n-i\times1}\\ v_{\delta_i,[0,i-1]}
				\end{pmatrix} & \begin{pmatrix}
					\mathbf{0}_{n-i\times1}\\ v_{\delta_{i+n},[0,i-1]}
				\end{pmatrix} & \cdots & \begin{pmatrix}
					\mathbf{0}_{n-i\times1}\\ v_{\delta_{i+(t-1)n},[0,i-1]}
				\end{pmatrix}\\
				\vdots & \vdots & \vdots & \vdots\\
				\begin{pmatrix}
					\mathbf{0}_{n-i\times1}\\ v_{\delta_{i},[0,i-1]}
				\end{pmatrix}^{\hspace{-0.25mm}t} & \begin{pmatrix}
					\mathbf{0}_{n-i\times1}\\ v_{\delta_{i+n},[0,i-1]}
				\end{pmatrix}^{\hspace{-0.25mm}t} & \cdots & \begin{pmatrix}
					\mathbf{0}_{n-i\times1}\\ v_{\delta_{i+(t-1)n},[0,i-1]}
				\end{pmatrix}^{\hspace{-0.25mm}t}
			\end{bmatrix}\endgroup\hspace{-1mm}.
		\end{align*}%
		Each matrix of this form has $t$ rows of the form $\Omega$ in \eqref{eqn_OmegaVandermonde}. Since $\delta_j$, $j\in\mathbb{Z}_{[1,tn]}$, are non-zero and distinct, it follows from Lemma \ref{lemma_vandermonde} that the corresponding $\Omega$ is invertible and hence, each matrix $\overline{W}_{i,\xi}$ has rank~$t$. Notice that the columns of each matrix $\overline{W}_{i,\xi}$ are linearly independent with respect to the columns of any other  $\overline{W}_{j,\xi}$, $i\neq j\in\mathbb{Z}_{[1,n]}$. This follows since (i) the rows of the form $\Omega$ appear in different rows in each $\overline{W}_{i,\xi}$ and (ii) the structure in which the block rows $\mathbf{0}_{n-i\times1}$ appear in each submatrix. As a result, rank$(W_\xi)=tn$. Due to the structure of $W_u$ and $W_\xi$, it follows that rank$(W)\hspace{-0.25mm}=\hspace{-0.25mm}t(n\hspace{-0.1mm}+\hspace{-0.1mm}1)$ and hence $W$ is invertible. Finally, it follows from Theorem~\ref{thm_generalPE} that rank$(\mathcal{H}_L(\vartheta))\hspace{-0.25mm}=\hspace{-0.25mm}t(n\hspace{-0.1mm}+\hspace{-0.1mm}1)L$.
\end{proof}

In the following section, we illustrate the results of Theorem~\ref{thm_aprioriFL} with an example.
	\section{Numerical Example}\label{sec_examples}
Consider a second order SISO flat system of the form in \eqref{BINF}, with $v_k = -\sin(x_{1,k}) + x_{1,k}x_{2,k}^2 - x_{1,k}^3x_{2,k} + u_k.$ In this example, we compare the performance of three nonlinear controllers: (i) An exact linearizing and stabilizing controller designed using basis functions that include $v$ in their span \cite[Cor. 2]{DePersis22}, and two locally stabilizing controllers (ii and iii) designed using the following choice of basis functions\footnote{The method described in \cite[Cor. 2]{DePersis22} requires that the unknown map \eqref{eqn_expressionforv} is linear in $u$, which is why we use the basis functions \eqref{eqn_ex_basisfunctions}. Although the choice of the basis functions in \eqref{eqn_ex_basisfunctions} is different from that in \eqref{eqn_specificchoice}, one can easily see from the proof of Theorem \ref{thm_aprioriFL} that using inputs of the form \eqref{eqn_PEinputSISOflat} also guarantees collective PE of \eqref{eqn_ex_basisfunctions}.} which do not contain $v$ in their span \cite[Cor. 2 and Sec.~III.B]{DePersis22}
\begin{equation}
	\Theta(\xi_k,u_k) = \begin{bmatrix}
		u_k & \xi_k^\top & (\xi_k^2)^\top & (\xi_k^3)^\top
	\end{bmatrix}^{\hspace{-0.5mm}\top}.\label{eqn_ex_basisfunctions}
\end{equation}
For all three controllers, PE of the basis functions of order one is a necessary and sufficient condition for the feasibility of the convex program that is solved to obtain the control gains (cf. \cite[Cor. 2, Thm. 2, and Thm. 5]{DePersis22}). For controllers (i) and (ii), PE is enforced by sampling the input randomly. For controller (iii), PE is enforced \textit{a priori} using the results of Theorem~\ref{thm_aprioriFL}. In this case, we used a straightforward extension of \cite[Cor. 2]{DePersis22} such that collected data from multiple experiments (i.e., collective PE) can be used to design the controller.

Since the system is unstable, the input data (of length $N=21$) for controllers (i) and (ii) had to be sampled from the uniform distribution $U(-0.25,0.25)$, whereas using multiple experiments as in Theorem~\ref{thm_aprioriFL} allowed us to use inputs (each of length $N_j=3$) with larger magnitudes (sampled from $U(-1,1)$). In \cite{vanWaarde20}, a similar observation was made for linear systems. As a result, a larger quantitative level of PE was attained (cf. Remark \ref{remark_qPE} and Table~\ref{table_comparison}).

The performance of the closed-loop system (over $T=20$ time instants) was compared starting from the same initial conditions (randomly sampled from $U(-1,1)\times U(-1,1)$). Table~\ref{table_comparison} shows the average cumulative stabilization errors (defined as $\sum_{k=0}^{T-1}\frac{1}{T}|x_{i,k}|$, for $i=1,2,\,T=20$) for all three controllers over 100 experiments, excluding 5 (respectively 4) unstable experiments for controllers (ii) and (iii). Controller~(i) is the best performing one since it enforces exact nonlinearity cancellation. Controller (iii) is shown to outperform controller (ii), although the same basis functions \eqref{eqn_ex_basisfunctions} were used, potentially suggesting that the region of attraction of controller (iii) is larger compared to (ii). This can be attributed to the fact that larger levels of PE were attained using multiple experiments.
	\section{Conclusion}\label{sec:conclusion}
In this work, we focus on addressing the fundamental challenge of OOD detection tasks, which is how to fully understand the semantic discrepancy between the ID/OOD samples. We reveal that the key to success in the realistic SCOOD task is to allocate as many ID samples in the unlabeled set correctly as possible. To this end, we propose a novel uncertainty-aware optimal transport scheme that introduces class-specific energy scores as guidance for effective label assignment. Experimental results show that our method achieves better performance than previous state-of-the-art methods on SCOOD benchmarks.

\textbf{Limitations.} In addition to temperature scaling, other techniques such as feature clipping applied in ReAct~\cite{sun2021react} also enhance the performance of energy score, so how to obtain an OOD score that best fits the SCOOD task can be further explored. Moreover, a setting highly related to SCOOD has been proposed in \cite{katz2022training} and formulated as a constrained optimization problem. We will also theoretically analyze these practical OOD settings in our feature work.

% \section*{Acknowledgments}
\textbf{Acknowledgments.} 
This work is supported by National Key R\&D Program of China under Grant 2020AAA0105701, National Natural Science Foundation of China (NSFC) under Grants 61872327, Major Special Science and Technology Project of Anhui, National Natural Science Foundation of China (62033012) and Ant Group through Ant Research Intern Program.


	
	\bibliographystyle{IEEEtran}
	\bibliography{references}
\end{document}