\section{Numerical Example}\label{sec_examples}
Consider a second order SISO flat system of the form in \eqref{BINF}, with $v_k = -\sin(x_{1,k}) + x_{1,k}x_{2,k}^2 - x_{1,k}^3x_{2,k} + u_k.$ In this example, we compare the performance of three nonlinear controllers: (i) An exact linearizing and stabilizing controller designed using basis functions that include $v$ in their span \cite[Cor. 2]{DePersis22}, and two locally stabilizing controllers (ii and iii) designed using the following choice of basis functions\footnote{The method described in \cite[Cor. 2]{DePersis22} requires that the unknown map \eqref{eqn_expressionforv} is linear in $u$, which is why we use the basis functions \eqref{eqn_ex_basisfunctions}. Although the choice of the basis functions in \eqref{eqn_ex_basisfunctions} is different from that in \eqref{eqn_specificchoice}, one can easily see from the proof of Theorem \ref{thm_aprioriFL} that using inputs of the form \eqref{eqn_PEinputSISOflat} also guarantees collective PE of \eqref{eqn_ex_basisfunctions}.} which do not contain $v$ in their span \cite[Cor. 2 and Sec.~III.B]{DePersis22}
\begin{equation}
	\Theta(\xi_k,u_k) = \begin{bmatrix}
		u_k & \xi_k^\top & (\xi_k^2)^\top & (\xi_k^3)^\top
	\end{bmatrix}^{\hspace{-0.5mm}\top}.\label{eqn_ex_basisfunctions}
\end{equation}
For all three controllers, PE of the basis functions of order one is a necessary and sufficient condition for the feasibility of the convex program that is solved to obtain the control gains (cf. \cite[Cor. 2, Thm. 2, and Thm. 5]{DePersis22}). For controllers (i) and (ii), PE is enforced by sampling the input randomly. For controller (iii), PE is enforced \textit{a priori} using the results of Theorem~\ref{thm_aprioriFL}. In this case, we used a straightforward extension of \cite[Cor. 2]{DePersis22} such that collected data from multiple experiments (i.e., collective PE) can be used to design the controller.

Since the system is unstable, the input data (of length $N=21$) for controllers (i) and (ii) had to be sampled from the uniform distribution $U(-0.25,0.25)$, whereas using multiple experiments as in Theorem~\ref{thm_aprioriFL} allowed us to use inputs (each of length $N_j=3$) with larger magnitudes (sampled from $U(-1,1)$). In \cite{vanWaarde20}, a similar observation was made for linear systems. As a result, a larger quantitative level of PE was attained (cf. Remark \ref{remark_qPE} and Table~\ref{table_comparison}).

The performance of the closed-loop system (over $T=20$ time instants) was compared starting from the same initial conditions (randomly sampled from $U(-1,1)\times U(-1,1)$). Table~\ref{table_comparison} shows the average cumulative stabilization errors (defined as $\sum_{k=0}^{T-1}\frac{1}{T}|x_{i,k}|$, for $i=1,2,\,T=20$) for all three controllers over 100 experiments, excluding 5 (respectively 4) unstable experiments for controllers (ii) and (iii). Controller~(i) is the best performing one since it enforces exact nonlinearity cancellation. Controller (iii) is shown to outperform controller (ii), although the same basis functions \eqref{eqn_ex_basisfunctions} were used, potentially suggesting that the region of attraction of controller (iii) is larger compared to (ii). This can be attributed to the fact that larger levels of PE were attained using multiple experiments.