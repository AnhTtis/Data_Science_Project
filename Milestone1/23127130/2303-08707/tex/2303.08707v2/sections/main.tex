\section{PE for LTI systems}\label{sec_main}
It is well known that random inputs satisfy Definition~\ref{def_PE} with high probability. However, this is not guaranteed and in some practical applications it may not be possible/desirable to apply frequent inputs. This is the case, e.g., in networked systems with bandwidth constraints~\cite{Siami21} or closed-loop control of fluid resuscitation~\cite{XinJin19}, where sparse inputs in the form of boluses are used to identify patient-specific parameters.

In the following theorem, we propose a simple and highly sparse input sequence that is guaranteed to be PE. In particular, PE of order $L\in\mathbb{Z}_{>0}$ is achieved by giving a pulse at each of the $m$ input channels every $L$ instants.
\begin{theorem}\label{thm_PEinputLTI}
	Let $L\in\mathbb{Z}_{>0}$ and $N\geq(m+1)L-1$. If the sequence $\{u_k\}_{k=0}^{N-1}$ takes the form
	\begin{equation}
		u_k = \begin{cases}
			e_j, \qquad &k=jL-1,\\
			\mathbf{0}, \qquad &\textup{otherwise},
		\end{cases}\label{eqn_PEinputLTI}
	\end{equation}
	where $j\in\mathbb{Z}_{[1,m]}$, then it holds that $\textup{rank}(\mathscr{H}_L(u))=mL$.
\end{theorem}
\begin{proof}\let\qedsymbol\relax
	Without loss of generality, let $N=(m+1)L-1$. The following matrix has full row rank
	\begin{equation*}
		\mathscr{H}_{L}(u) = \begingroup\setlength\arraycolsep{3.5pt}\begin{bmatrix}
			\mathbf{0} & \mathbf{0} & \cdots & e_1 & \cdots & \mathbf{0} & \mathbf{0} & \cdots  & e_m\\[-1ex]
			\vdots & \vdots & \reflectbox{$\ddots$} & \vdots &\cdots & \vdots& \vdots & \reflectbox{$\ddots$} & \vdots\\[-0.5ex]
			\mathbf{0} & e_1 & \cdots & \mathbf{0} & \cdots & \mathbf{0} & e_m & \cdots & \mathbf{0}\\[-0.5ex]
			e_1 & \mathbf{0} & \cdots & \mathbf{0} & \cdots & e_m & \mathbf{0} & \cdots &\mathbf{0}
		\end{bmatrix}\endgroup.
	\end{equation*}
\end{proof}
\begin{remark}For the input in \eqref{eqn_PEinputLTI}, all singular values of $\mathscr{H}_L(u)$ are equal to one. By scaling \eqref{eqn_PEinputLTI} by $\alpha$, one obtains an input which is $\alpha-$PE of order~$L$ (cf.~\cite{Coulson22}). Note that larger values of $\alpha$ imply higher levels of PE of order~$L$.\label{remark_qPE}\end{remark}

Theorem \ref{thm_PEinputLTI} provides an explicit formula \eqref{eqn_PEinputLTI} for an input which is guaranteed to be persistently exciting in the sense of Definition \ref{def_PE} (and hence, also Definition~\ref{def_oldPE}). This can be used to apply the results of Theorem~\ref{thm_FL}. A necessary condition for the input in \eqref{eqn_PEinputLTI} to be PE of order $L+n$ is that it is at least of length $N\geq(m+1)(L+n)-1$ (cf. Definition~\ref{def_PE}). In contrast, the procedure from \cite{vanWaarde22} uses online state or output measurements to design an input such that the resulting input-output data matrix in \eqref{eqn_fundamental_lemma} has rank $mL+n$ and requires $N=(m+1)L+n-1$ samples to this end, implying that it is  sample efficient. However, the resulting input only guarantees the rank condition for the system on which the experiment was done. In contrast, the input \eqref{eqn_PEinputLTI}, although not as sample efficient, guarantees the results of Theorem~\ref{thm_FL} independently of the considered system.

Apart from its use in data-driven analysis and control, Theorem~\ref{thm_FL} also allows for identification of the LTI system's matrices up to a similarity transformation (cf. \cite[Sec. 4.4]{Willems05}). In this sense, employing \eqref{eqn_PEinputLTI} to obtain \eqref{eqn_fundamental_lemma} can be interpreted as the data-driven counterpart to classical works on identification of LTI systems from their impulse response~\cite{HO66}.

In Section \ref{sec_main2}, we consider Hammerstein nonlinear systems and show how to guarantee PE of basis functions that depend on the input, by the design of the input only.