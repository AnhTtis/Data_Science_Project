\section{Introduction}
\IEEEPARstart{C}{}entral to the area of data-driven control, as well as system identification \cite{Ljung87} and adaptive control \cite{astrom08}, is the notion of persistence of excitation (PE), see, e.g., \cite{Green86}. In \cite{Willems05}, a discrete-time sequence is said to be persistently exciting of a certain order if a corresponding Hankel matrix of that sequence has full row rank. It was then shown in~\cite{Willems05} that when such a PE input is applied to a controllable LTI system, the resulting input-state or input-output Hankel matrices satisfy certain rank conditions. Now known as the fundamental lemma, this result has received a great amount of attention in recent years and has been successfully used in a wide range of applications (see \cite{Markovsky21} for a comprehensive survey).\par
%
For certain classes of nonlinear systems, various works have proposed extensions and applications of the fundamental lemma using, e.g., basis functions. For instance, Hammerstein-Wiener systems \cite{Berberich20}, flat and feedback linearizable systems \cite{AlsaltiBerLopAll2021, Alsalti2022}, and control design of input-affine nonlinear systems \cite{DePersis22} have been considered. In these works, suitable PE conditions involving the sequence of basis functions (which depend on inputs and/or states/outputs) need to be satisfied. However, in \cite{Berberich20,AlsaltiBerLopAll2021,Alsalti2022,DePersis22} these PE conditions could only be verified a posteriori, i.e., after performing an experiment and collecting state/output data, and no a priori input design was proposed to this end.\par
%
In general, there exist only few results on the design of suitable inputs that result in satisfaction of the required PE conditions, specifically for nonlinear systems. In \cite{vanWaarde22}, an \emph{online} method was proposed to design inputs that result in the desired rank conditions of \cite{Willems05} on input-state or input-output Hankel matrices for linear systems. However, the resulting input is not universal in the sense that it is tailored specifically to the system on which the experiment is performed. An extension of this result to the class of bilinear systems appeared in \cite{Yuan22}. Furthermore, \cite{DePersis21} shows how suitable scaling of the initial conditions and the input to a nonlinear system leads to satisfaction of certain rank conditions on the input-state data. This enables local stabilization of an unknown nonlinear system in the first approximation but, in general, cannot be used for  the methods in \cite{Berberich20,AlsaltiBerLopAll2021,Alsalti2022,DePersis22,Alsalti2021c}.\par
%
The contributions of this paper are as follows. First, we propose a simple input sequence which satisfies standard PE conditions for linear systems \cite{Green86,Willems05}. Based on this result, as a second contribution, we design inputs that guarantee PE of any order for an arbitrary sequence of basis functions for the class of Hammerstein systems. The third contribution addresses nonlinear systems that are locally reachable at the origin. Specifically, we show existence of sparse input sequences that guarantee collective PE of sequences of basis functions. For single-input single-output (SISO) flat systems (which are locally reachable at the origin), we systematically design inputs that guarantee collective PE of any order of specific choices of basis functions. Finally, we illustrate the results by computing data-based controllers for SISO flat systems (as proposed in~\cite{DePersis22}).\par
%
Section \ref{sec_prel} introduces notation and necessary preliminaries. Sections \ref{sec_main}-\ref{sec_main3} contain the main contributions of the paper. Section \ref{sec_examples} illustrates the results with a numerical example and Section \ref{sec_conc} concludes the paper.