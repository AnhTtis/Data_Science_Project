\section{Notation and Preliminaries}\label{sec_prel}
\emph{Notation:} Let $\mathbb{Z}_{>0}$ denote the set of positive integers and let $\mathbb{Z}_{[a,b]}$ denote the set of integers in the interval $[a,b]$. Let the $m\times m$ identity matrix be $I_m$ and its columns be $e_i$ for $i\in\mathbb{Z}_{[1,m]}$. We use $0_{n\times m}$ to denote an $n\times m$ matrix of zeros; when the dimensions are clear from the context, we omit the subscript for notational simplicity. For a sequence $\{z_k\}_{k=0}^{N-1}$ with $z_k\in\mathbb{R}^\eta$, we denote its stacked vector as $z = \begingroup\setlength\arraycolsep{2pt}\begin{bmatrix}z_0^\top &z_1^\top & \dots & z_{N-1}^\top\end{bmatrix}\endgroup^\top$ and a stacked window of it as $z_{[l,j]} = \begingroup\setlength\arraycolsep{2pt}\begin{bmatrix}z_l^\top &z_{l+1}^\top & \dots & z_{j}^\top\end{bmatrix}\endgroup^\top$ for $0\leq l<j$. We write $M\succ0$ ($M\succeq0$) if the matrix $M$ is positive (semi-)definite.\par
%
In this paper, we are concerned with the notion of persistence of excitation which originated in the fields of system identification \cite{Ljung87} and adaptive control \cite{astrom08}. Several definitions of PE have appeared in the literature. The following is one of such definitions for a finite length discrete-time sequence.
\begin{definition}[\cite{Green86}]\label{def_oldPE}
		For $N\in\mathbb{Z}_{>0}$, the sequence \(\{z_k\}_{k=0}^{N-1}\), $z_k\in\mathbb{R}^{\eta}$, is exciting over the interval $[0,N-1]$ if, for some $\nu>0$, the following holds $\sum\limits_{k=0}^{N-1}z_kz_k^\top\succeq \nu I_\eta \succ0$.
\end{definition}%
%
In recent works on data-driven control (cf. \cite{Markovsky21}), the following definition has been commonly used.
\begin{definition}[\cite{Willems05}] For $L\in\mathbb{Z}_{>0}$ and $N\geq L$, the sequence \(\{z_k\}_{k=0}^{N-1}\), $z_k\in\mathbb{R}^{\eta}$, is persistently exciting of order \(L\) if \(\textup{rank}(\mathscr{H}_{L}(z))=\eta L\), where $\mathscr{H}_L(z) = \begin{bmatrix}
		z_{[0,L-1]} & z_{[1,L]} & \cdots & z_{[N-L,N-1]}
	\end{bmatrix}$.
	\label{def_PE}
\end{definition}

The two definitions are related as follows: Notice that if a sequence is PE of order $L$ in the sense of Definition~\ref{def_PE}, then
	\begin{equation*}
		\mathscr{H}_L(z)(\mathscr{H}_L(z))^\top = \sum\limits_{k=0}^{N-L}z_{[k,k+L-1]}z_{[k,k+L-1]}^\top\succ0.
	\end{equation*}
This means that for $L=1$, the two definitions are equivalent. If a sequence is PE in the sense of Definition~\ref{def_PE} of order $L>1$, then it is also exciting in the sense of Definition~\ref{def_oldPE}, but the converse is not necessarily true. This is because if rank$(\mathscr{H}_L(z))=\eta L$, then rank$(\mathscr{H}_{\bar{L}}(z))=\eta\bar{L}$ for any depth $\bar{L}<L$, but the converse is, in general, not true.\par
%
The advantage of Definition~\ref{def_PE} is that it quantifies the order of which a signal is exciting. As shown in \cite{Willems05}, this notion has an important application. In particular, if an input to a controllable LTI system is PE of order $L+n$, then the resulting input/output data matrix contains in its span any length$-L$ input/output trajectory of the system. This became known as the fundamental lemma and is summarized below.
\begin{theorem}[\cite{Willems05}]
	Let \(\{u_k,y_k\}_{k=0}^{N-1}\) be an input-output trajectory of a controllable LTI system. If \(\{u_k\}_{k=0}^{N-1}\) is PE of order $L+n$, then any \(\{\bar{u}_k,\bar{y}_k\}_{k=0}^{L-1}\) is a trajectory of the system, if and only if there exists \(\beta\in\mathbb{R}^{N-L+1}\) such that
	\begin{equation}
		\begin{bmatrix} \mathscr{H}_L(u)\\ \mathscr{H}_L(y)\end{bmatrix} \beta = \begin{bmatrix}\bar{u} \\ \bar{y}\end{bmatrix}.
		\label{eqn_fundamental_lemma}
	\end{equation}\label{thm_FL}
\vspace{-1em}
\end{theorem}
%
The following are recent extensions of Definition~\ref{def_PE}, which we use in our paper. The notion of \textit{collective} persistence of excitation was defined in \cite{vanWaarde20}, and extends Definition~\ref{def_PE} to multiple sequences.
\begin{definition}[\cite{vanWaarde20}]\label{def_cPE}
	For $r,L\in\mathbb{Z}_{>0}$, $j\in\mathbb{Z}_{[1,r]}$, and $N_j\geq L$, the sequences $\{z_k^{(j)}\}_{k=0}^{N_j-1}$, with $z_k^{(j)}\in\mathbb{R}^\eta$, are \textit{collectively} persistently exciting of order $L$ if rank$(\mathcal{H}_L(\mathscr{Z}))=\eta L$, where $\mathscr{Z} = \begin{bmatrix}
		(z^{(1)})^\top & \cdots & (z^{(r)})^\top
	\end{bmatrix}^\top,$ and
	\begin{equation*}
		\mathcal{H}_L(\mathscr{Z}) = \begin{bmatrix}
			\mathscr{H}_L(z^{(1)}) & \cdots & \mathscr{H}_L(z^{(r)})
		\end{bmatrix}.
	\end{equation*}
\end{definition}%

Another extension was proposed in \cite{Coulson22}, where a \textit{quantitative} notion of PE is defined in order to establish a robust version of Theorem~\ref{thm_FL}. There, a sequence $\{z_k\}_{k=0}^{N-1}$ is said to be $\alpha-$PE of order $L$ if $\sigma_{\textup{min}}(\mathscr{H}_L(z))\geq\alpha>0$, provided the Hankel matrix has at least as many columns as rows, where $\sigma_{\textup{min}}$ denotes the minimum singular value.\par
%
In the next section, we propose a simple input sequence which satisfies all the above definitions of PE.