\section{PE for locally reachable nonlinear systems}\label{sec_main3}
Consider an unknown nonlinear system of the form
\begin{equation}
	x_{k+1} = f(x_k,u_k),\label{eqn_NLsys}
\end{equation}%
with $x_k\in\mathbb{R}^n,u_k\in\mathbb{R}^m$ being the state and input vectors, respectively, and $f:\mathbb{R}^n\times\mathbb{R}^m\to\mathbb{R}^n$ is an unknown function satisfying $f(\mathbf{0},\mathbf{0})=\mathbf{0}$. For $\mu\in\mathbb{Z}_{>0}$, the set of all states which can be reached from $x_0$ in $\mu$ steps is defined as
\begin{align}
	&\mathcal{R}_\mu(x_0) \hspace{-1mm}=\hspace{-1mm} \left\lbrace \hspace{-0.5mm} x_\mu\in\mathbb{R}^n \,\left|\, \begin{aligned}
		&\exists\, u_{[0,\mu-1]},\, u_k\in\mathbb{R}^m,\\ &\textup{s.t.}\,\,  x_{k+1}\hspace{-0.5mm}=\hspace{-0.5mm}f(x_k,u_k),\,\forall k\hspace{-0.5mm}\in\hspace{-0.5mm}\mathbb{Z}_{[0,\mu-1]}.
	\end{aligned} \hspace{-0.5mm} \right. \right\rbrace \hspace{-1mm}.\notag
\end{align}
It was shown in \cite{Melkior20} how one can, under certain assumptions, obtain a guaranteed under-approximation of the reachable set of a nonlinear system with unknown dynamics. For the remainder of this section, we make the following assumption.
\begin{assumption}\label{assmp_reachability}
	For the system \eqref{eqn_NLsys}, there exists $\mu\in\mathbb{Z}_{>0}$ such that the origin is contained in the interior of $\mathcal{R}_\mu(\mathbf{0})$.
\end{assumption}%

Assumption~\ref{assmp_reachability} implies that the system is locally reachable at the origin. A sufficient condition for local reachability at $x=\mathbf{0}$ is that the linearization of system~\eqref{eqn_NLsys} at the origin is controllable (cf. \cite[Lemma 3.7.8]{Sontag13}).

Consider now $r$ basis functions $\theta_j:~\mathbb{R}^n\times \mathbb{R}^m\to\mathbb{R}$ which satisfy $\theta_j(\mathbf{0},\mathbf{0})=0$, $j\in\mathbb{Z}_{[1,r]}$, and denote the stacked vector of them by $\Theta(x_k,u_k)=\begingroup
\setlength\arraycolsep{1pt}\begin{bmatrix}\theta_1(x_k,u_k) & \cdots & \theta_r(x_k,u_k) \end{bmatrix}^\top\endgroup$. Suppose that the functions are linearly independent on arbitrary domains with non-empty interior\footnote{This assumption is satisfied for sinusoidal functions, exponential functions and monomials, among others \cite{Christensen06}. Note that basis functions that do not satisfy $\theta_j(\mathbf{0},\mathbf{0})=0$ can be suitably shifted by a constant.} $D_x\times D_u\subset\mathbb{R}^n\times~\mathbb{R}^m$. The objective is to design inputs $\{u_k^{(j)}\}_{k=0}^{N_j-1}$ such that the sequences of basis functions $\{\hat{\Theta}_k^{(j)}\}_{k=0}^{N_j-1}$ (with $\hat{\Theta}_k^{(j)}\coloneqq\Theta(x_k^{(j)},u_k^{(j)})$) are collectively persistently exciting of order $L$. According to Definition \ref{def_cPE}, the following mosaic Hankel matrix must have full row rank
\begin{align}
	&\mathcal{H}_L(\vartheta) =\label{eqn_MosaicHankelTheta} \begin{bmatrix}
		\mathscr{H}_L(\hat{\Theta}^{(1)}) & \cdots & \mathscr{H}_L(\hat{\Theta}^{(r)})
	\end{bmatrix},
\end{align}
where $\vartheta = \begin{bmatrix}
	(\hat{\Theta}^{(1)})^\top & \cdots & (\hat{\Theta}^{(r)})^\top
\end{bmatrix}^\top$.

Consider a submatrix of \eqref{eqn_MosaicHankelTheta} composed of the $L+\mu$ element of each of the $r$ sequences of basis functions
\begin{align}
	&W =
	\begin{bmatrix}
		\hat{\Theta}^{(1)}_{L+\mu-1} & \hat{\Theta}^{(2)}_{L+\mu-1} & \cdots & \hat{\Theta}^{(r)}_{L+\mu-1}
	\end{bmatrix}.\label{eqn_W}
\end{align}
Similar to Section \ref{sec_main2}, we would like $W$ to be invertible. Since the state at time $x_{L+\mu-1}^{(j)}$ is not a free variable, ensuring invertibility of $W$ is a control problem. In particular, one must select the inputs such that the corresponding state and input pairs at time $L+\mu-1$ result in an invertible matrix $W$. In a data-driven setting, such a control problem is difficult to solve since no model knowledge is available and - to begin with - no persistently exciting data is available yet to apply data-driven control techniques. Therefore, we first show in Lemma~\ref{lemma_rexperimentsexist} that there exist such $r$ input sequences $u_{[0,L+\mu-1]}^{(j)}$, $j\in\mathbb{Z}_{[1,r]}$, which make $W$ invertible and then prove in Theorem~\ref{thm_generalPE} how invertibility of $W$ results in collective PE of the basis functions. Later, in the following Subsection~\ref{sec_flatPE} we illustrate how, under suitable assumptions, one can indeed find the desired control inputs a priori, which guarantee collective PE of any order for sequences of basis functions that depend on input and output data of SISO flat systems.

\begin{lemma}\label{lemma_rexperimentsexist}
	Let Assumption \ref{assmp_reachability} be satisfied and suppose that $r$ basis functions $\theta_j:\mathbb{R}^n\times\mathbb{R}^m\to\mathbb{R}$ are linearly independent on $\mathcal{R}_\mu(\mathbf{0})\times\mathbb{R}^m$. Then there exist $r$ sequences $u_{[0,L+\mu-1]}^{(j)}$, $j\in\mathbb{Z}_{[1,r]}$, which when applied to \eqref{eqn_NLsys} starting from~$x_0^{(j)}=\mathbf{0}$, result in $x_{L+\mu-1}^{(j)}$, such that $W$ in \eqref{eqn_W} is invertible.
\end{lemma}
\begin{proof}
	Using similar arguments to the proof of Theorem~\ref{thm_alwaysexistslambda}, it can be shown by linear independence of the basis functions $\theta_j$ on $\mathcal{R}_\mu(\mathbf{0})\times\mathbb{R}^m$ that there exists $r$ pairs $(x_{L+\mu-1}^{(j)},u_{L+\mu-1}^{(j)})\in \mathcal{R}_\mu(\mathbf{0})\times\mathbb{R}^m$ such that $W$ is invertible. 
	
	Since $f(\mathbf{0},\mathbf{0})=\mathbf{0}$, then starting from zero initial conditions and setting the input to $u^{(j)}_{[0,L-2]}=0$, one can express the state $x_{L+\mu-1}^{(j)}$ in terms of $u^{(j)}_{[L-1,L+\mu-2]}$ only, i.e., $x_{L+\mu-1}^{(j)} = f(f\cdots(f(\mathbf{0},u_{L-1}^{(j)}),u_{L}^{(j)})\cdots,u_{L+\mu-2}^{(j)})$. Finally, since the system is locally reachable by Assumption~\ref{assmp_reachability}, there exist inputs $u_{[L-1,L+\mu-1]}^{(j)}$, $j\in\mathbb{Z}_{[1,r]}$, which steer the system from $x_{L-1}^{(j)}=\mathbf{0}$ to $x_{L+\mu-1}^{(j)}$ in $\mu$~steps.
\end{proof}

The following theorem shows how to use the results of Lemma \ref{lemma_rexperimentsexist} to obtain collectively persistently exciting sequences of basis functions of any order $L$.
\begin{theorem}\label{thm_generalPE}
	Let Assumption~\ref{assmp_reachability} hold. Given $L,\mu\in\mathbb{Z}_{>0}$ and $r$ basis functions $\theta_j:\mathbb{R}^n\times\mathbb{R}^m\to\mathbb{R}$ that are linearly independent on $\mathcal{R}_\mu(\mathbf{0})\times\mathbb{R}^m$ and satisfy $\theta_j(\mathbf{0},\mathbf{0})=0$, let $N_j\geq2L+\mu-1$ for $j\in\mathbb{Z}_{[1,r]}$. Furthermore, let the sequences $\{u_k^{(j)}\}_{k=0}^{N_j-1}$ take the form
	\begin{equation}
		u_k^{(j)} = \begin{cases}
			\eta^{(j)}_{[0,\mu]}, \quad & k\in\mathbb{Z}_{[L-1,L+\mu-1]},\\
			\mathbf{0}, \quad &\textup{otherwise},
		\end{cases}\label{eqn_PElocallyreachable}
	\end{equation}
	where $j\in\mathbb{Z}_{[1,r]}$, and $\eta^{(j)}_{[0,\mu]}$ are such that $W$ in \eqref{eqn_W} is invertible. If \eqref{eqn_PElocallyreachable} are applied to \eqref{eqn_NLsys} starting from $x_0^{(j)}=\mathbf{0}$, then for the matrix in \eqref{eqn_MosaicHankelTheta} it holds that $\textup{rank}(\mathcal{H}_L(\vartheta)) = rL$.
\end{theorem}
\begin{proof}
	Each block row of \eqref{eqn_MosaicHankelTheta} has $r$ linearly independent columns given by the columns of $W$. Notice that each $\mathscr{H}_L(\hat{\Theta}^{(j)})$, $j\in\mathbb{Z}_{[1,r]}$, in \eqref{eqn_MosaicHankelTheta} has a lower block-anti-triangular structure due to $f(\mathbf{0},\mathbf{0})=\mathbf{0},\,\Theta(\mathbf{0},\mathbf{0})=\mathbf{0}$ and the choice of the inputs \eqref{eqn_PElocallyreachable}. As a result, every block row \eqref{eqn_MosaicHankelTheta} is linearly independent from the others. Since there are $L$ such block rows, it holds that rank$(\mathcal{H}_L(\vartheta))=rL$.
\end{proof}

Notice that the results of Lemma~\ref{lemma_rexperimentsexist} for locally reachable nonlinear systems are analogous to that of Theorem~\ref{thm_alwaysexistslambda} for Hammerstein systems. However, formulating a nonlinear feasibility problem similar to \eqref{eqn_feasibility} to find $u_{[0,L+\mu-1]}^{(j)}$, $j\in\mathbb{Z}_{[1,r]}$ would require knowledge of the unknown function $f$.

It was observed in simulations that randomly sampling the input sequences $\eta^{(j)}_{[0,\mu]}$, $j\in\mathbb{Z}_{[1,r]}$, in Theorem~\ref{thm_generalPE} from a uniform distribution typically results in a corresponding invertible matrix $W$. However, such a heuristic approach is not always guaranteed to achieve this result. To systematically find the desired input sequences, one must impose additional assumptions on the {class} of systems and the {choice} of basis functions. To this end, we consider in the next subsection SISO flat nonlinear systems (which are locally reachable at the origin), and show how one can guarantee PE of any order $L>0$ a priori, for a specific choice of basis functions.
%%~~~~~~~~~~~~~~~~~~~~~~~~~~~~~~~~~~~~~~~~~~~~~~~~~~~~~~~~~~~~~~~~~~~~~~~%%
\subsection{SISO flat nonlinear systems}\label{sec_flatPE}
Consider an unknown SISO flat system of the form
\begin{equation}
	\begin{aligned}
		x_{k+1} = f(x_k,u_k), \quad y_k = h(x_k),
	\end{aligned}\label{eqn_flatsys}
\end{equation}
where \(x_k\in\mathbb{R}^n, u_k,\,y_k\in\mathbb{R}\) and $f:\mathbb{R}^n\times\mathbb{R}\to\mathbb{R}^n$, $h:~\mathbb{R}^n\to~\mathbb{R}$ are smooth unknown functions with $f(\mathbf{0},0)=\mathbf{0}$ and $h(\mathbf{0})=0$. Let $f_O^j(x_k)$ denote the $j-$th iterated composition of the undriven dynamics $f(x_k,0)$.\par
%
Since the system is flat (i.e., has a well defined relative degree equal to the system dimension $n$, cf. \cite[Sec. III.A]{AlsaltiBerLopAll2021}), it can be transformed into the discrete-time normal form provided that $0\in\textup{Im}\left(h(f_O^{n-1}(f(x,\cdot)))\right)$ holds for all $x\in\mathbb{R}^n$ (cf. \cite[Sec. 2]{MonacoNor1987} for more details). This means that there exists an invertible (w.r.t. ${v}_k$) control law ${u}_k=q({x}_k,{v}_k)$, with $q:\mathbb{R}^n\times\mathbb{R}\to\mathbb{R}$ and an invertible coordinate transformation $\xi_k=T({x}_k)=y_{[k,k+n-1]}$, such that
\begin{equation}
	\begin{matrix}
		\xi_{k+1} = {A}\xi_k + {B}{v}_k, \qquad
		{y}_k = {C}\xi_k,
	\end{matrix}
	\label{BINF}%
\end{equation}%
Furthermore, $A,B,C$ are in the Brunovsky canonical form (cf. \cite[Thm.~2]{AlsaltiBerLopAll2021}) which is a controllable/observable triplet. Hence, the system is $n$ steps locally reachable at the origin. The synthetic input $v_k$ takes the form
\begin{equation}
	\begin{aligned}
		v_k = h(f_O^{n-1}(f(x_k,u_k))).
	\end{aligned}\label{eqn_expressionforv}
\end{equation}

A sufficient condition for the analogue of Theorem~\ref{thm_FL} to flat systems \cite[Prop. 1]{AlsaltiBerLopAll2021}, and for designing controllers from data in \cite[Cor. 2]{DePersis22}, is persistence of excitation of a sequence of basis functions which contain $h\circ f_O^{n-1}\circ f$ in their span. To check the PE condition, one typically performs an experiment of length $N\geq (r+1)L-1$, collects the corresponding state or output measurements and then verifies the rank of the resulting Hankel matrix.

In this section, we illustrate how one can enforce PE of any order for a \textit{particular choice} of basis functions \textit{a priori}. A specific choice of basis functions may, in general, not contain the unknown nonlinearity \eqref{eqn_expressionforv} in its span. Nonetheless, enforcing PE of such basis functions is still useful for, e.g., designing locally stabilizing controllers for unknown SISO flat systems (cf. \cite[Sec. VII.B]{DePersis22}), and for data-driven nonlinear predictive control \cite{Alsalti2021c}, provided that the basis functions result in a good local approximation of~\eqref{eqn_expressionforv}. 

Since the map from $u$ to $v$ is invertible and since $f(\mathbf{0},0)=\mathbf{0}$ and $h(\mathbf{0})=0$, a non-zero input applied to the system from zero initial conditions results in a non-zero value of $v$ in \eqref{eqn_expressionforv}. Moreover, invertibility implies that for all $\delta_1,\delta_2\in\mathbb{R}$, the following holds
\begin{equation}
	\delta_1\hspace{-0.5mm}\neq \hspace{-0.5mm}\delta_2 \hspace{-1mm}\iff\hspace{-1mm} h(f_O^{n-1}(f(\mathbf{0},\delta_{1})))\hspace{-0.5mm}\neq\hspace{-0.5mm} h(f_O^{n-1}(f(\mathbf{0},\delta_{2}))).\label{eqn_uniquenessofv}
\end{equation}

We exploit this fact to prove the following lemma, which will be needed later for the main result of this subsection.
\begin{lemma}\label{lemma_vandermonde}
	For $t\in\mathbb{Z}_{>0}$ let $\delta_j\neq0$, $j\in\mathbb{Z}_{[1,t]}$, be mutually distinct values and define $v_{\delta_j}\coloneqq h(f_O^{n-1}(f(\mathbf{0},\delta_j)))$. Then, the following matrix is invertible:
\end{lemma}
\begin{equation}
	\Omega = \begin{bmatrix}
		v_{\delta_1} & v_{\delta_2} & \cdots & v_{\delta_t}\\
		v_{\delta_1}^2 & v_{\delta_2}^2 & \cdots & v_{\delta_t}^2\\
		\vdots & \vdots & \ddots & \vdots\\
		v_{\delta_1}^{t} & v_{\delta_2}^{t} & \cdots & v_{\delta_t}^{t}
	\end{bmatrix}.\label{eqn_OmegaVandermonde}
\end{equation}
\begin{proof}
	Since for $j\in\mathbb{Z}_{[1,t]}$, $\delta_j\neq0$ are mutually distinct values, it follows that the corresponding values $v_{\delta_j}$ are also distinct and non-zero (compare the discussion above Lemma~\ref{lemma_vandermonde}). The matrix $\Omega$ can be written as $\Omega = V^\top \Delta$, where $V\in\mathbb{R}^{t\times t}$ is a square Vandermonde matrix composed of the distinct $v_{\delta_j}$ and, hence, invertible and $\Delta\in\mathbb{R}^{t\times t}$ is a diagonal matrix containing $v_{\delta_j}$. The proof is concluded by noting that $V$ and $\Delta$ are invertible matrices.
\end{proof}

In the following, we consider monomial basis functions in the transformed state and input up to some finite order $t\in\mathbb{Z}_{>0}$, and hence $r=t(n+1)$.
\begin{align}
	&\Theta(\xi_k,u_k) = \label{eqn_specificchoice}\begingroup
	\setlength\arraycolsep{2pt}\begin{bmatrix}
		u_k & u_k^2 & \cdots & u_k^t& \xi_k^\top & (\xi_k^2)^\top & \cdots & (\xi_k^t)^\top 
	\end{bmatrix}^\top\endgroup\hspace{-2mm}.
\end{align}
The powers are defined element-wise, i.e., $\xi_k^t=[\xi_{1,k}^t \,\, \cdots \,\, \xi_{n,k}^t]^\top$. Notice that these functions depend only on the inputs and outputs of \eqref{eqn_flatsys} since $\xi_k=y_{[k,k+n-1]}$ (cf. \eqref{BINF}). In the following theorem, we show how to choose input sequences $\{u_k^{(j)}\}_{k=0}^{N_j-1}$, $j\in\mathbb{Z}_{[1,r]}$, such that the resulting sequences of basis functions $\{\hat{\Theta}_k^{(j)}\}_{k=0}^{N_j-1}$ are collectively persistently exciting of order $L>0$, i.e., that the corresponding mosaic Hankel matrix $\mathcal{H}_L(\vartheta)$ of the form \eqref{eqn_MosaicHankelTheta} has full row rank.
\begin{theorem}\label{thm_aprioriFL}
		For $t\in\mathbb{Z}_{>0}$, let $\delta_j\neq0$, $j\in~\mathbb{Z}_{[1,t(n+1)]}$, be mutually distinct values. For $L\in\mathbb{Z}_{>0}$, $N_j\geq 2L+n-1$ and the basis functions in \eqref{eqn_specificchoice}, let $\{u_k^{(j)}\}_{k=0}^{N_j-1}$ take the form in \eqref{eqn_PElocallyreachable} with the corresponding $\eta^{(j)}_{[0,n]}$ given by
		\begin{equation}
			\eta^{(j)}_{[0,n]} = \begin{cases}
				\begin{bsmallmatrix}
					\mathbf{0}_{n-j\times1}\\ \delta_j\\ \mathbf{0}_{j\times1}
				\end{bsmallmatrix}, \quad & \textup{for }j\in\mathbb{Z}_{[1,n]},\\
				&\boldsymbol{\vdots}\\
				\begin{bsmallmatrix}
					\mathbf{0}_{tn-j\times1}\\ \delta_j\\ \mathbf{0}_{j-(t-1)n\times1}
				\end{bsmallmatrix}, \quad & \textup{for }j\in\mathbb{Z}_{[(t-1)n,tn]},\\
				\begin{bsmallmatrix}
					\mathbf{0}_{n\times1}\\ \delta_j
				\end{bsmallmatrix}, \quad & \textup{for }j\in\mathbb{Z}_{[tn+1,t(n+1)]}.
			\end{cases}
			\label{eqn_PEinputSISOflat}
		\end{equation}
		If \eqref{eqn_PEinputSISOflat} are applied to \eqref{eqn_flatsys} starting from zero initial conditions, then $\textup{rank}(\mathcal{H}_L(\vartheta))=t(n+1)L$.
\end{theorem}
\begin{proof}
	Without loss of generality, let $N_j=2L+n-1$ for all $j\in\mathbb{Z}_{[1,t(n+1)]}$. For $c\in\mathbb{Z}_{[0,n-1]}$, we define $v_{\delta_j,[0,c]}~\coloneqq~\begingroup
		\setlength\arraycolsep{1pt}\begin{bmatrix}
			h(f_O^{n-1}(f(\mathbf{0},\delta_{j}))) \,\, \cdots \,\, h(f_O^{n+c-1}(f(\mathbf{0},\delta_{j})))
		\end{bmatrix}^\top\endgroup\hspace{-1.5mm}$ (with some abuse of notation we also use $v_{\delta_j,0}=v_{\delta_j}$). Since the system \eqref{BINF} is in the Brunovsky form, applying the inputs $\{u_k^{(j)}\}_{k=0}^{N_j-1}$ as defined in the theorem statement from zero initial conditions results in
		\begin{equation}
			\xi_{L+n-1}^{(j)}\hspace{-1mm} = \hspace{-1mm}\begin{cases}
				\begin{bsmallmatrix}
					\mathbf{0}_{n-j\times 1}\\ v_{\delta_j,[0,j-1]}
				\end{bsmallmatrix}, \, & \textup{for }j\in\mathbb{Z}_{[1,n]},\\
				&\boldsymbol{\vdots}\\
				\begin{bsmallmatrix}
					\mathbf{0}_{tn-j\times 1}\\ v_{\delta_j,[0,j-(t-1)n-1]}
				\end{bsmallmatrix}, \, & \textup{for }j\in\mathbb{Z}_{[(t-1)n+1,tn]},\\
				\mathbf{0},\,&\textup{for }j\in\mathbb{Z}_{[tn+1,t(n+1)]}.
			\end{cases}\label{eqn_XiStateForW}
	\end{equation}%
	Now, we consider a submatrix of $\mathcal{H}_L(\vartheta)$ of the form of $W$ in \eqref{eqn_W}. For the choice of basis functions in \eqref{eqn_specificchoice}, the inputs $\{u_k^{(j)}\}_{k=0}^{N_j-1}$ as defined in the theorem statement and the corresponding state values in \eqref{eqn_XiStateForW}, the matrix $W$ takes the form \eqref{eqn_Wflat} (see next page). Following similar arguments to the proof of Lemma~\ref{lemma_vandermonde}, one can show that $W_u\in\mathbb{R}^{t\times t}$ is invertible since $\delta_j$, $j\in\mathbb{Z}_{[tn+1,t(n+1)]}$, are non-zero and mutually distinct values.
		\begin{figure*}[!t]
			\normalsize
			\begin{align}
					W = \begin{bmatrix}
						\mathbf{0}& W_u\\
						W_\xi & \mathbf{0}
					\end{bmatrix} = \left[\begin{array}{cccc|ccc}
						0 & 0 & \cdots & 0 & \delta_{tn+1} & \cdots & \delta_{t(n+1)}\\[-1ex]
						\vdots & \vdots & \vdots & \vdots & \vdots & \vdots & \vdots\\[-0.5ex]
						0 & 0 & \cdots & 0 & \delta_{tn+1}^t & \cdots & \delta_{t(n+1)}^t\\[-0.25ex]
						\hline
						\begin{pmatrix}
							\mathbf{0}_{n-1\times1}\\[-0.5ex] v_{\delta_1}
						\end{pmatrix} & \begin{pmatrix}
							\mathbf{0}_{n-2\times1}\\[-0.5ex] v_{\delta_2,[0,1]}
						\end{pmatrix} & \cdots & \begin{pmatrix}
							v_{\delta_{tn},[0,n-1]}
						\end{pmatrix} & \mathbf{0} & \cdots & \mathbf{0}\\[-1ex]
						\vdots & \vdots & \vdots & \vdots & \vdots & \vdots & \vdots\\[-1ex]
						\begin{pmatrix}
							\mathbf{0}_{n-1\times1}\\[-0.5ex] v_{\delta_1}
						\end{pmatrix}^t & \begin{pmatrix}
							\mathbf{0}_{n-2\times1}\\[-0.5ex] v_{\delta_2,[0,1]}
						\end{pmatrix}^t & \cdots & \begin{pmatrix}
							v_{\delta_{tn},[0,n-1]}
						\end{pmatrix}^t & \mathbf{0} & \cdots & \mathbf{0}
					\end{array}
					\right].
					\label{eqn_Wflat}
				\end{align}%}
			\hrulefill
			\vspace{-1em}
		\end{figure*}
		Using the columns of $W_\xi\in\mathbb{R}^{tn\times tn}$ in \eqref{eqn_Wflat}, we construct $n$ submatrices $\overline{W}_{i,\xi}\in\mathbb{R}^{tn\times t}$, $i\in\mathbb{Z}_{[1,n]}$, of the form
		\begin{align*}
			&\overline{W}_{i,\xi} =\\
			&\begingroup
			\setlength\arraycolsep{1.5pt}\begin{bmatrix}
				\begin{pmatrix}
					\mathbf{0}_{n-i\times1}\\ v_{\delta_i,[0,i-1]}
				\end{pmatrix} & \begin{pmatrix}
					\mathbf{0}_{n-i\times1}\\ v_{\delta_{i+n},[0,i-1]}
				\end{pmatrix} & \cdots & \begin{pmatrix}
					\mathbf{0}_{n-i\times1}\\ v_{\delta_{i+(t-1)n},[0,i-1]}
				\end{pmatrix}\\
				\vdots & \vdots & \vdots & \vdots\\
				\begin{pmatrix}
					\mathbf{0}_{n-i\times1}\\ v_{\delta_{i},[0,i-1]}
				\end{pmatrix}^{\hspace{-0.25mm}t} & \begin{pmatrix}
					\mathbf{0}_{n-i\times1}\\ v_{\delta_{i+n},[0,i-1]}
				\end{pmatrix}^{\hspace{-0.25mm}t} & \cdots & \begin{pmatrix}
					\mathbf{0}_{n-i\times1}\\ v_{\delta_{i+(t-1)n},[0,i-1]}
				\end{pmatrix}^{\hspace{-0.25mm}t}
			\end{bmatrix}\endgroup\hspace{-1mm}.
		\end{align*}%
		Each matrix of this form has $t$ rows of the form $\Omega$ in \eqref{eqn_OmegaVandermonde}. Since $\delta_j$, $j\in\mathbb{Z}_{[1,tn]}$, are non-zero and distinct, it follows from Lemma \ref{lemma_vandermonde} that the corresponding $\Omega$ is invertible and hence, each matrix $\overline{W}_{i,\xi}$ has rank~$t$. Notice that the columns of each matrix $\overline{W}_{i,\xi}$ are linearly independent with respect to the columns of any other  $\overline{W}_{j,\xi}$, $i\neq j\in\mathbb{Z}_{[1,n]}$. This follows since (i) the rows of the form $\Omega$ appear in different rows in each $\overline{W}_{i,\xi}$ and (ii) the structure in which the block rows $\mathbf{0}_{n-i\times1}$ appear in each submatrix. As a result, rank$(W_\xi)=tn$. Due to the structure of $W_u$ and $W_\xi$, it follows that rank$(W)\hspace{-0.25mm}=\hspace{-0.25mm}t(n\hspace{-0.1mm}+\hspace{-0.1mm}1)$ and hence $W$ is invertible. Finally, it follows from Theorem~\ref{thm_generalPE} that rank$(\mathcal{H}_L(\vartheta))\hspace{-0.25mm}=\hspace{-0.25mm}t(n\hspace{-0.1mm}+\hspace{-0.1mm}1)L$.
\end{proof}

In the following section, we illustrate the results of Theorem~\ref{thm_aprioriFL} with an example.