\section{Conclusion}
In this paper, we proposed a hybrid scope latent feature extraction layer, the HS-layer, and used it to construct a category-level object pose estimation framework HS-Pose. Based on the advantages of the HS-layer, HS-Pose can handle complex shapes, capture an object's size and translation, and is robust to noise. The capability of the overall framework is demonstrated in the experiments. The comparisons with the existing methods show that our HS-Pose achieves state-of-the-art performance. In future work, we plan to apply our proposed HS-layer to other problems where unstructured data needs to be processed, and the combination between the local and the global information becomes critical. 
% , with three designed structures, STE, RF-F, and ORL.
% which consists of STE, RF-F, and ORL.

% and used it to construct a category-level object pose estimation framework HS-Pose. The HS-layer 


% Inside the HS-layer, we develop three simple structures for feature extraction from different scopes. In particular, we propose the scale and translation encoding strategy (STE), a receptive field with feature distance (RF-F), and the outlier robust feature extraction layer (ORL). With these structures, the HS-layer is capable of providing scale and translation encoding as well as capturing outlier robust local and global geometric structural information. 


% Based on the advantages of the HS-layer, HS-Pose can handle complex shapes, capture an object's size and translation, and is robust to noise. In the ablation studies, we show that STE contributes to the scale and translation estimation by a large margin. In addition, the RF-F significantly boosts the rotation estimation performance. The overall framework shows its capability for complex shape handling and its resistance to noise. The comparisons with the existing methods show that our HS-Pose achieves state-of-the-art performance. 

