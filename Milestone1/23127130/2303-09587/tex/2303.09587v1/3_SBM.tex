\section{VMP on modular networks}
\label{chapter SBM}
To study the effect of the topology reflecting the biased communities of the bipopulated VMP, we analyze the model on a network with two modules of sizes $N_1$ and $N_2$, generated by a Stochastic Block Model (SBM) \cite{lee2019review}. The SBM is defined by the intra-modular ($p_{11},p_{22}$) and intermodular ($p_{12},p_{21}$) connectivities, i.e., the probabilities describing the corresponding linkings between the agents (for undirected networks $p_{12}=p_{21})$). We assume that the network results from homophilic interactions (epistemic bubbles) such that all agents within module $1$ ($2$) have bias $h_1$ ($h_2$).
\\
We start from the mean-field equations (\ref{eq:rate}) where global transition rates are now functions of the connectivities of the block model
\begin{equation}
    \begin{cases} 
    R_{+1}(\rho_1,\rho_2) =\frac{1}{\alpha p_{11} + (1-\alpha)p_{12}} (\alpha-\rho_1)\frac{1+h_1}{2}(p_{11}\rho_1+p_{12}\rho_2) \\
    R_{-1}(\rho_1,\rho_2) = \frac{1}{\alpha p_{11} + (1-\alpha)p_{12}} \rho_1\frac{1-h_1}{2}[p_{11}(\alpha-\rho_1)+p_{12}(1-\alpha -\rho_2)]\\\\
    R_{+2}(\rho_1,\rho_2) = \frac{1}{\alpha p_{12} + (1-\alpha)p_{22}} (1-\alpha-\rho_2)\frac{1+h_2}{2}(p_{12}\rho_1+p_{22}\rho_2)\\
    R_{-2}(\rho_1,\rho_2) = \frac{1}{\alpha p_{12} + (1-\alpha)p_{22}} \rho_2\frac{1-h_2}{2}[p_{12}(\alpha-\rho_1)+p_{22}(1-\alpha -\rho_2)]
    \end{cases}
    \label{SBM transition rates}
\end{equation}
For example, once a spin in down state of class 1 is selected, the probability that we find one of its neighbours in up state is 
$\frac{p_{11}\rho_1}{\alpha p_{11} + (1-\alpha)p_{12}}+\frac{p_{12}\rho_2}{\alpha p_{11} + (1-\alpha)p_{12}}$ (for more details, see the Appendix \ref{app transition rates}).\\
We end up with the mean-field evolution equations for the densities $\rho_1,\rho_2$
\begin{equation}
    \begin{cases}
    \dot{\rho_1} = C_1 \bigg[ (\alpha-\rho_1)(1+h_1)(p_{11}\rho_1+p_{12}\rho_2) - \rho_1(1-h_1)[p_{11}(\alpha-\rho_1)+p_{12}(1-\alpha -\rho_2)] \bigg]  \\
    \dot{\rho_2} = C_2 \bigg[ (1-\alpha-\rho_2)(1+h_2)(p_{12}\rho_1+p_{22}\rho_2) - \rho_2(1-h_2)[p_{12}(\alpha-\rho_1)+p_{22}(1-\alpha -\rho_2)] \bigg] 
    \end{cases}
\label{SBM mf system}
\end{equation}
where $C_1 = \frac{1}{2[\alpha p_{11} + (1-\alpha)p_{12}]}$, $C_2 = \frac{1}{2[\alpha p_{12} + (1-\alpha)p_{22}]}$.\\
It is easy to see that the consensus states are still fixed points. The numerical study of the system shows that the qualitative behaviour of the fully connected case is preserved, i.e., the different phases are separated by transcritical bifurcations, but the polarization area generally widens and the range of the polarized phase increases. Figure \ref{fig:SBMalpha}
compares the fully connected case and the topology characterized by two cliques ($p_{11}=p_{22}=1$) and intercommunity connectivities $p_{12}=p_{21}=0.3$, with equally strong opposite preferences $h_1=-h_2=h$. In general, the symmetric community structure decreases the values of the critical masses, with respect to the fully connected topology.
\begin{figure}
    \centering
    \includegraphics[width=0.7\textwidth]{Figures/SBM_p1_ritag.png}\\
    \includegraphics[width=0.7\textwidth]{Figures/SBM_p03_ritag.png}
    \caption{\textbf{VMP on modular networks with equally strong opposite preferences, $\alpha, h$ plane.} The mean field predictions are shown, i.e. the stable fixed point of the system (\ref{SBM mf system}), for two choices of connectivities of the modular network. \textit{Upper plots}: case $p_{11}=p_{22}=p_{12}=p_{21}=1$, corresponding to the fully connected topology (subplot (a) in figure \ref{fig:fc phase diagrams}). \textit{Lower plots}: case $p_{11}=p_{22}=1$, $p_{12}=p_{21}=0.3$, i.e. still symmetric probabilities but segregated communities. The polarization increases and the polarization area widens, nevertheless the general qualitative behaviour is preserved.}
    \label{fig:SBMalpha}
\end{figure}
\newpage
By considering the evolution of the normalized densities $\rho'_1 = \frac{\rho_1}{\alpha} \in [0,1]$, $\rho'_2 = \frac{\rho_2}{1-\alpha} \in [0,1]$ and defining the topological parameters
\begin{equation}
    \gamma_1 = \frac{p_{12}(1-\alpha)}{p_{11}\alpha +p_{12}(1-\alpha)} \;\;\;\;\;\;\;\;\; \gamma_2 = \frac{p_{21}\alpha}{p_{22}(1-\alpha) +p_{21}\alpha}
\end{equation}
we have a consistent reduction in the number of parameters in the mean-field system on the SBM (\ref{SBM mf system}), which reads
\begin{equation}
\begin{cases}
    \dot{\rho'_1} = \frac{1}{2}\bigg[   (1-\rho'_1)(1+h_1)[(1-\gamma_1)\rho'_1+\gamma_1\rho'_2] - \rho'_1(1-h_1)[(1-\gamma_1)(1-\rho'_1)+\gamma_1(1-\rho'_2)]\bigg]  \\
    \dot{\rho'_2} = \frac{1}{2}\bigg[ (1-\rho'_2)(1+h_2)[\gamma_2\rho'_1+(1-\gamma_2)\rho'_2) - \rho'_2(1-h_2)[\gamma_2(1-\rho'_1)+(1-\gamma_2)(1-\rho'_2)] \bigg] 
\end{cases}
\label{mf_echochambers}
\end{equation}
The interpretation of $\gamma_1,\gamma_2$ is straightforward in terms of the average internal and external degrees $z_{11},z_{12},z_{21},z_{22}$  \footnote{The internal degrees $z_{11},z_{22}$ are the numbers of connections that an agent of respectively the first and second community on average has within his community, while the external degrees $z_{12},z_{21}$ are the average number of connections towards the other community}
\begin{equation}
    \gamma_1 = \frac{z_{12}}{z_{11}+z_{12}} \;\;\;\;\;\;\;\;\; \gamma_2 = \frac{z_{21}}{z_{22}+z_{21}}
\end{equation}
Being the average fractions of external connections over the total number of connections of the agents in class 1 and 2, $\gamma_1,\gamma_2$ can be intended as the average \textit{open-mindedness} of the individuals of respectively the first and second community. Since in an epistemic bubble the agents overrepresent (i.e., are more linked to) their belonging community, we expect $\gamma_1 \in (0,1-\alpha)$ and $\gamma_2 \in (0,\alpha)$. The more far $\gamma_1,\gamma_2$ are from these upper extremes, the more the individuals of the corresponding population have unbalanced sources of information, i.e. are trapped in the bubble. If the individuals have on average more connections within their belonging community rather than towards the other, then $\gamma_1,\gamma_2\in(0,0.5)$.
Figure \ref{fig:echo_gamma} shows the stationary polarization\footnote{$P^*=\Delta^*$, since it is easy to check that at the fixed point $\Delta^*\geq0$ for $h_1\geq0,\;h_2\leq0$.} $\Delta^* = \rho'^*_1-\rho'^*_2$ in the $h_1h_2$ plane, for different choices of the open-mindedness parameters $\gamma_1,\gamma_2$, by numerically calculating the fixed points of the mean-field system (\ref{mf_echochambers}) and determining their stability.\\
\\
\\
From the linear stability analysis of the consensus fixed points of the system (\ref{mf_echochambers}), reported in Appendix \ref{Modular lsa}, we obtain the condition for the stability of the positive consensus 
\begin{equation}
    \gamma_1h_2(1-h_1)+\gamma_2h_1(1-h_2) + 2h_1h_2\geq0
\end{equation}
and for the negative one
\begin{equation}
     -\gamma_1h_2(1+h_1)-\gamma_2h_1(1+h_2)+2h_1h_2\geq0
\end{equation}
First we fix $\gamma_1,\gamma_2$ and determine the critical lines in the $h_1,h_2$ plane. The line separating the space when the positive consensus is stable (below the line) and the one for which it is unstable reads
\begin{equation}
    h_2^c(h_1) = -\frac{\gamma_2h_1}{\gamma_1(1-h_1)+h_1(2-\gamma_2)}
\end{equation}
and it is bounded superiorly by $h_2^{c+} = \frac{\gamma_2}{2-\gamma_2}$, which is reached at $h_1=1$ and does not depend on $\gamma_1$. \\
In the same way, the critical line related to the negative consensus fixed point reads
\begin{equation}
    h_1^c(h_2) = \frac{-\gamma_2h_2}{\gamma_2(1+h_2)-h_2(2-\gamma_1)}
\end{equation}
and it is bounded by $h_1^{c-} = \frac{\gamma_1}{2-\gamma_1}$, independent of $\gamma_2$. The critical lines are drawn for a choice of the open-mindedness parameters in figure \ref{fig:linear_stab_analysis}b.\\
In the $\gamma_1,\gamma_2$ plane, the critical line  for the positive consensus
\begin{equation}
    \gamma_2^c(\gamma_1) = \frac{h_1(1-h_2)}{-h_2(1-h_1)} \gamma_1 +\frac{2h_1h_2}{h_2(1-h_1)}
\label{lsa critical line gamma}
\end{equation}
is straight line in the plane, whose coefficient goes as $\sim h_1/(1-h_1)$ with $h_1$ and $\sim (1-h_2)/-h_2$ with $-h_2$, thus exploding for high $h_1$ and low $-h_2$. 
% Indeed, as shown in figure \ref{fig:linear_stab_analysis}c, once fixed $h_2 =-0.2$ the critical line becomes almost an horizontal line for high $h_1$. 
Thus, the critical points for different topologies of the first population, i.e. different $\gamma_1$, happen to be at approximately the same value of $\gamma_2$, as shown in figures \ref{fig:linear_stab_analysis}c and \ref{fig:linear_stab_analysis}d. Naturally, symmetric considerations hold for the negative consensus.\\
\\
The results of linear stability analysis allow us to conclude that if one population is very committed and the other population's bias is low, e.g. when $h_1>>-h_2$, wether or not such population manages to impose its preferred opinion depends largely on the open-mindedness of the other population $\gamma_2$,  while the dependency on the topological structure of the committed population $\gamma_1$ is negligible.\\
\\
We test this claim by numerically simulating the VMP on modular networks for a finite system of $N=1000$ agents, fixing $h_1=0.7,h_2=-0.2$ and varying the topology through the open-mindedness parameters $\gamma_2$ and $\gamma_1$. The results, shown in Figure \ref{fig:linear_stab_analysis}a, support the mean-field predictions by showing that the critical value of $\gamma_2$ separating the polarization and positive consensus phases is approximately the same for all the three lines in the $\gamma_2,\rho^*$ plane, corresponding to the three chosen values of $\gamma_1$.



\begin{figure}
    \centering
    \includegraphics[width=\textwidth]{Figures/echochamb_final_good.png}
    \caption{\textbf{VMP on modular networks.} Mean-field predictions for the polarization (eq. \ref{mf_echochambers}) in the $h_1,h_2$ plane for two equally open-minded communities ($\gamma_1=\gamma_2$, upper line), and for three values of $\gamma_2$, varying the open-mindedness of the first population $\gamma_1$. Decreasing the open-mindedness, the consensus areas shrink and the system becomes more and more polarized.}
    \label{fig:echo_gamma}
\end{figure}





\begin{figure}
    \textit{\subfloat(a)}{\includegraphics[width = 3in]{Figures/stab_analys_gammadepN.png}}
    \textit{\subfloat(b)}{\includegraphics[scale = 0.496]{Figures/stab_analys_echochN.png}}\\
    \textit{\subfloat(c)}{\includegraphics[width = 3in]{Figures/lsa_horizontal.png}} 
    \textit{\subfloat(d)}{\includegraphics[width = 3in]{Figures/lsa_gamma.png}}\\
    \caption{\textbf{Simulations and analytical results for the VMP on a modular graph.}\\ \textit{(a)} The average density of up spin at the stationary state is calculated over 30 runs of the model, with $N=1000$ agents and preferences $h_1=0.7,h_2=-0.2$, varying the open-mindedess of the second population and for 3 values of the open-mindedness of the first population, corresponding to different colors. We can see that all the three lines approach the positive consensus at approximately the same $\gamma_2$. \\
    \textit{(b)} Critical lines for the positive and negative consensus points in the $h_1h_2$ plane, for $\gamma_1=0.5,\gamma_2=0.3$ (black). On the background, the polarization values calculated numerically as in figure \ref{fig:echo_gamma} are reported, to show the consistency of the results of the linear stability analysis. The red and blue lines are the positive consensus critical lines for other values of $\gamma_1$, while $\gamma_2=0.3$ constantly. The figure is the analytical correspondant of $\gamma_1 = 0.1,0.3,0.5$, $\gamma_2=0.3$ plots in figure \ref{fig:echo_gamma}. \\
    \textit{(c)} Critical lines in the $\gamma_1\gamma_2$ plane for $h_2=-0.2$ and different $h_1$. The figure shows that the higher $h_1$, the more the critical line tends to an horizontal line. \\
    \textit{(d)}  Once fixed the biases to $h_1=0.7,h_2=-0.2$, in the $\gamma_1\gamma_2$ plane are reported the analytical critical line (\ref{lsa critical line gamma}) and the values of $\gamma_1$ (vertical lines) corresponding to the lines in figure (a): we see that the intersections with the critical line (i.e. the points at which the positive consensus becomes stable) occur almost at the same $\gamma_2$, for all the three values of $\gamma_1$.    }
    \label{fig:linear_stab_analysis}
\end{figure}