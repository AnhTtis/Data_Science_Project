\section{Introduction}
The formation of people's opinions, choices and decisions is subject to social pressure: it is a general observation that in society, individuals (agents) take into account the behaviour of others \cite{weber1978economy}. This aspect is at the root of most agent-based models of opinion dynamics \cite{castellano2009statistical,sirbu2017opinion,contucci2020statistical,peralta2022opinion}. Many such models have been introduced with the aim to understand the effects of 
different microscopic mechanisms of the opinion formation process \cite{redner2019reality,castello2006ordering}. Here, we consider the Voter model \cite{holley1975ergodic,dornic2001critical,sood2005voter}, characterized by a simple imitative mechanism, introducing personal preferences attached to single individuals \cite{masuda2010heterogeneous,masuda2011can}. \\
\\
Prejudices or personal preferences generally come from the history of the individual (e.g. ideologies and partisanship \cite{bartels2000partisanship}) and are assumed to evolve on a much longer time-scale than that of the opinion influencing interactions, in other words they 
can be considered as fixed (quenched) features of the nodes throughout the dynamics\footnote{Recent studies \cite{alford2005political,settle2009heritability,settle2010friendships} corroborate this assumption by providing evidence of a genetic contribution in the formation of
political attitudes, exhorting scientists to \textit{"incorporate genetic influences, specifically interactions between genetic heritability and social environment, into models of political attitude formation.}" \cite{alford2005political}.}. Even though this bias is a characteristic of the node, 
it is fundamentally different from other kinds of biases, such as the confirmation or algorithmic ones \cite{sirbu2019algorithmic,peralta2021effect,peralta2021opinion}, since those are dependent on the node's current opinion.\\
\\
In a model with social pressure and quenched preferences, each individual is subjected to two "forces": one inducing the individual to minimize conflicts with his neighbours, the other exhorting the individual to stick to his own prejudice.
A strong social pressure may lead the agent to adopt a public opinion in dissonance with his prejudices, a phenomenon which is defined as \textit{preference falsification} in \cite{kuran1997private}. Else, if the personal bias is strong, the individual may reject social coordination, accept conflicts more easily, and stick to his prior view even if he finds himself in disagreement with many of his neighbours. The latter mechanism, when conflicting preferences are present among the individuals, may contribute
to the emergence of \textit{polarization} \cite{valensise2022dynamics,del2017modeling}.\\
\\
In this work, we generally consider two groups of interacting individuals with opposite preferences of different intensities. We focus in particular on the role of the social network, considering a modular network with two communities corresponding to the bias assignment. The network model 
mimics the realistic setting of two \textit{epistemic bubbles}\footnote{We remark, following the subtle distinction highlighted by \cite{nguyen2020echo}, that the \textit{epistemic bubble} is a structure related to the process of acquiring knowledge that emerges principally due to an "unconscious" homophilic mechanism, and it is considerably different from the \textit{echo chamber}, which involves more active processes, e.g. the mechanisms of \textit{trust-distrust}, \textit{isolation} and \textit{disagreement-reinforcement}.}, where the agents with the same ideology share more links among themselves than with the other oppositely biased community. The result is an unbalanced choice of sources by the agent, who is systematically more influenced by his
own community than by the other. In this paper, our main focus resides in determining the level of polarization between the two groups once the system has reached the stationary state, as a function of the preferences' intensities and of the topological structure of the social network \cite{interian2022network,garimella2018political}. \\
\\
There is an analogy between the binary opinion dynamics models and the Ising model of statistical physics, where the personal preference can be achieved by a site-dependent external magnetic field. Following this analogy, we will characterize the opinion state of the individuals by a $\sigma \in \{+1,-1\}$ "spin" variable. We study the so-called \textit{Partisanship Voter Model} (PVM), in which the preference toward one of the states modifies the transition rates of the voter dynamics accordingly and breaks the original symmetry between the two opinions. This dynamics should be distinguished from that of another biased voter model that we call \textit{Voter model with media interactions} (VMMI) \cite{bhat2020polarization}, where the personal bias expresses how much an individuum follows a node with fixed opinion state connected to everybody (the "medium"). In table \ref{table:single} 
we summarize the transition rates of the Ising model and the biased voter models for homogeneous populations, where all individuals have the same personal biases. The same models in the bipopulated versions \cite{gallo2007bipartite,contucci2008phase} are defined and explained in table \ref{table:bipopulated}.\\
\\
The PVM with homogeneous preferences can be historically individuated as a specific case of the Abrams-Strogatz model for language death \cite{abrams2003modelling} and as an agent-based model in \cite{stauffer2007microscopic, vazquez2010agent}. The generalization to multiple biases has been first proposed by Masuda et al. in 2010 \cite{masuda2010heterogeneous}. The authors of \cite{borile2013effect} focused in particular on the finite-size effects for low bias intensity.  In the successive work \cite{masuda2011can}, the model was generalized for different compositions of the system and preferences' intensities. In \cite{czaplicka2022biased} the same model was considered where just a fraction of individuals was biased. Let us also point out that the introduction of \textit{zealots} \cite{mobilia2003does,mobilia2007role,mukhopadhyay2020voter}, i.e., agents that never change opinion and just try to convince others, can be traced back to the analyzed model if we properly tune the bias associated to such agents (i.e. setting $h_{z} = \{+1,-1\}$ depending on the type of commitment). The problem of social pressure and conflicting preferences has been also 
studied in evolutionary game theory \cite{hernandez2013heterogeneous,hernandez2017equilibrium,mazzoli2017equilibria}, with a focus on network effects \cite{broere2017network} and supported with various social experiments \cite{ellwardt2016conflict,goyal2021integration,broere2019experimental}. \\
\\
Despite the advancements in understanding the dynamics of systems evolving with Voter-like processes, the influence of the social network topology in the PVM with conflicting preferences remains unexplored. This paper aims to address this gap by examining the interplay between fixed individual preferences and homophilic network structures (epistemic bubbles). Indeed, we expect homophily to play a crucial role in mitigating the social cohesion induced by imitative dynamics, when conflicting preferences are present, possibly leading the system to a polarized asymptotic state.\\  
\\
In this work, we consider the model of Masuda and Redner in its most general version \cite{masuda2011can}, that we refer to as (Bipopulated) \textit{Voter Model with Preferences} (VMP) \footnote{Since personal preferences can arise from various factors, including partisanship, we use this term in the most general sense to connect our model to existing literature, such as game-theoretical models.}, where we consider two classes of agents with opposite biases of different intensities. 
The VMP is defined in section \ref{model} and solved on the fully connected network in section \ref{FC_chapter}. In section \ref{chapter SBM}, we study the model on a bi-modular network and calculate the phase diagram as a function of the model parameters using a mean-field approach.
In section \ref{PA_chapter}, we apply the Pair Approximation \cite{gleeson2011high,gleeson2013binary} to the model on the modular network, and compare its predictions to the mean-field results for sparse graphs. In the remainder of the article, we study the model on a real network with high modularity, the Political blogosphere of 2004 US elections \cite{adamic2005political}, and test the goodness of the mean-field predictions of the stationary state in the case of equally intense but opposite personal biases.\\






%comparison between personal bias in media, Ising and biased
\begin{table}[ht]
\centering
\begin{tabular}{ |p{3cm}|p{2.8cm}|p{2.8cm}|%p{4cm}|  }
} 
\hline
 & 
 %Infection rate 
 $F_{k,m}$ & 
 %Recovery rate 
 $R_{k,m}$ 
 %& Interpretation 
 \\[0.5ex] 
 \hline
 Ising Glauber 
 %with bias 
 & $\frac{1}{1+e^{-\frac{2}{T}[h+J(2m-k)]}}$ & $\frac{e^{-\frac{2}{T}[h+J(2m-k)]}}{1+e^{-\frac{2}{T}[h+J(2m-k)]}}$
 %& Each randomly selected spin is flipped with a probability depending on the difference in energy between the updated and current configurations. $J$ is the strength of pairwise interactions, $T$ is the temperature, i.e. the common factor of the bias and the pairwise parameters ($T=1$ without loss of generality).
 \\[0.5ex]
 \hline
 VMMI
 %Voter \JK{model with media}
% -like with bias in interactions 
 & $(1-h)\frac{m}{k} + h$ & $(1-h)(1-\frac{m}{k})$ 
 %&  Each selected spin interacts with his neighbours as in the standard voter model with probability $1-h$, else with a stubborn pseudoagent (always $+1$). $h\in[0,1]$ is the bias intensity towards the preferred $+1$ opinion. 
 \\  [0.5ex] 
 \hline
 %Biased Voter Model 
 PVM& $\frac{m}{k}(\frac{1+h}{2}) $ & $(1-\frac{m}{k})(\frac{1-h}{2})$ 
 %& At each iteration, the personal bias attached to every spin gives more probability to an update towards the preferred opinon, and disincentivates the opposite transition.
 \\  [0.5ex] 
 \hline\hline
\end{tabular}
\caption{\textbf{Binary-state models with homogeneous personal biases (preferences).} Ising Glauber: Ising model with Glauber dynamics, VMMI: Voter Model with Media Interaction, PVM: Partisanship Voter Model. The models are defined through the infection and recovery transition rates $F_{k,m}$ and $R_{k,m}$ for $\sigma:-1 \to +1$ and $\sigma:+1 \to -1$, respectively; the considered node has $k$ neighbors out of which $m$ are in state $+1$. The strength of the bias is $h$, which corresponds to the external field in the Ising model with temperature $T$ and pairwise couplings $J$. For the voter models $h\in [0,1].$}
%infection and recovery rates , adopting the terminology of epidemic models. 
%These are the rates at which a selected spin $i$ with $m$ 
%infected 
%neighbours ($\sigma_{j,j\in\partial i}=+1$) out of $k$ neighbours undergoes the infection $-1\rightarrow +1$ and recovery $+1\rightarrow-1$ transitions.
\label{table:single}
\end{table}




%same thing for bipopulated
%comparison between personal bias in media, Ising and biased
\begin{table}[ht]
\centering
\begin{tabular}{ |p{2.1cm}|p{2.4cm}|p{2.4cm}|p{2.4cm}|p{2.7cm}| }
 \hline
 & 
 %First population's infection rate 
 $F^{(1)}_{k,m}$ & 
 %First population's recovery rate 
 $R^{(1)}_{k,m}$ & 
 %Second population's infection rate 
 $F^{(2)}_{k,m}$ & 
 %Second population's recovery rate 
 $R^{(2)}_{k,m}$  \\[0.5ex] 
 \hline
 Ising Glauber 
 %with bias 
 & $\frac{1}{1+e^{-\frac{2}{T}[h_1+J(2m-k)]}}$ & $\frac{e^{-\frac{2}{T}[h_1+J(2m-k)]}}{1+e^{-\frac{2}{T}[h_1+J(2m-k)]}}$ &  $\frac{1}{1+e^{-\frac{2}{T}[h_2+J(2m-k)]}}$ & $\frac{e^{-\frac{2}{T}[h_2+J(2m-k)]}}{1+e^{-\frac{2}{T}[h_2+J(2m-k)]}}$ \\[0.5ex]
 \hline
 %Voter-like with bias in interactions
 VMMI
 & $(1-h_1)\frac{m}{k} + h_1$ & $(1-h_1)(1-\frac{m}{k})$ & $(1-|h_2|)\frac{m}{k}$ & $(1-|h_2|)(1-\frac{m}{k})+|h_2|$  \\  [0.5ex] 
 \hline
 %Biased Voter Model
 PVM & $\frac{m}{k}(\frac{1+h_1}{2}) $ & $(1-\frac{m}{k})(\frac{1-h_1}{2})$ & $\frac{m}{k}(\frac{1+h_2}{2}) $ & $(1-\frac{m}{k})(\frac{1-h_2}{2})$ \\
 \hline\hline
\end{tabular}
\caption{\textbf{Bipopulated binary-state models with personal biases (preferences).} Columns two and three (four and five) are for nodes in population 1 (2). For the VMMI, $h_1\in [0,1]$ and $h_2\in [-1,0]$.}
%In the Voter-like model with biases in interactions, the biases are assumed to point in two opposite directions 
%but both $h_1,h_2 \geq0$, while in the other models there are no restrictions a priori. 
\label{table:bipopulated}
\end{table}
