\documentclass{article}
\pdfoutput=1
\usepackage[utf8]{inputenc}
\usepackage{amsmath}
\usepackage[a4paper, total={6in, 8in}]{geometry}
\usepackage{graphicx}
\usepackage{color}
\DeclareMathOperator{\sign}{sign}
\usepackage{amssymb}
\usepackage[sorting=none]{biblatex}
\addbibresource{Bibliography/bibliography.bib}
\setlength{\arrayrulewidth}{0.2mm}
\setlength{\tabcolsep}{10pt}
\renewcommand{\arraystretch}{1.5}
\newcommand{\JK}[1]{{\color{red}#1}}
\newcommand{\FZ}[1]{{\color{green}#1}}
\usepackage[T1]{fontenc}
\usepackage[utf8]{inputenc}
\usepackage{authblk}
\usepackage{subfig} 
\usepackage{graphicx}


\renewcommand\Authands{ and }

% \title{Biased voter model on modular networks or Biased Voter model on echochambers: two interacting populations with opposite bias or} 
% \title{(Bipopulated) Biased Voter model on modular networks:\\ when do two biased echochambers polarize?}
% \title{Biased Voter model on epistemic bubbles: how ideologies and segregation may induce polarization?}
\title{\textbf{Voter-like model with conflicting preferences on homophilic networks}}
\title{\textbf{Voter-like dynamics with conflicting preferences on modular networks}}
\vspace{5cm}
\author{Filippo Zimmaro$^{1,2,3}$, Pierluigi Contucci$^2$, János Kertész$^3$}
\date{%
    $^1$ \textit{Department of Computer Science}, University of Pisa\\%
    $^2$ \textit{Department of Mathematics}, University of Bologna \\%
    $^3$ \textit{Department of Network and Data Science}, Central European University\\[4ex]%
    % \today
}

% \date{November 2022}
\begin{document}
\maketitle
\begin{abstract} Two of the main factors shaping an individual's opinion are social coordination and personal preferences, or personal biases. To understand the role of those and that of the topology of the network of interactions, we study an extension of the voter model proposed by Masuda and Redner (2011), where the agents are divided into two populations with opposite preferences.
We consider a modular graph with two communities that reflect the bias assignment, modeling the phenomenon of epistemic bubbles. We analyze the models by approximate analytical methods and by simulations. Depending on the network and the biases' strengths, the system can either reach a consensus or a polarized state, in which the two populations stabilize to different average opinions. The modular structure has generally the effect of increasing both the degree of 
polarization and its range in the space of parameters. 
When the difference in the bias strengths between the populations is large, the success of the very committed group in imposing its preferred opinion onto the other one depends largely on the level of segregation of the latter population, while the dependency on the topological structure of the former is negligible. We compare the simple mean-field approach with the pair approximation and test the goodness of the mean-field predictions on a real network.
\end{abstract}
\vspace{-10pt}
%%%%%%%%% BODY TEXT

\section{Introduction}
\label{section:introduction}
%% 1. why should someone care?

%The advent of advanced interactive computer vision systems~\cite{hololens} and recent progress in vision-language and multi-modal models~\cite{} opens doors for such next generation of assistive agents. 
% We envision that the future assistive agents would build up on these visual and language reasoning capabilities of today and empower users to achieve goals in their everyday lives. In particular, such agents would be able to reason about \emph{unseen} human goals... 
% We posit that such agents would require the ability to understand user goals described in natural language at high-level i.e., without complete details about as well as unseen user goals. 

%Recent progress in augmented reality systems~\cite{hololens, magicleap}, as well as vision-language and multi-modal models~\cite{}, opens doors for the next generation of assistive agents. 
Inspired by recent progress in visual systems~\cite{MagicLeap, ungureanu2020hololens}, we consider an assistive egocentric agent capable of reasoning about daily activities. When invoked via natural language commands, for e.g., while baking a cake, the agent understands the steps involved in baking, tracks progress through the various stages of the task, detects and proactively prevents mistakes by making suggestions. Such an agent would empower users to learn new skills and accomplish tasks efficiently.
% One could envision invoking such an agent merely through natural language descriptions of tasks similar to how present day assistants such as Alexa, Siri etc.~\cite{voice_assistants} are invoked. 
%We envision such agents to empower users in daily life by  invoking them naturally through 

%% 2. Why is it challenging? 
%While recent progress in vision-language and multi-modal models~\cite{} opens doors for such next generation of assistive agents, various challenges remain in making such agents a reality. 
%To make such agents a reality, 

Developing such an egocentric agent capable of tracking and verifying everyday tasks based on their natural language specification is challenging for multiple reasons. First, such an agent must reason about various ways of doing a \emph{multi-step} task specified in natural language. This entails decomposing the task into relevant actions, state changes, object interactions as well as any necessary causal and temporal relationships between these entities. Secondly, the agent must ground these entities in egocentric observations to track progress and detect mistakes. Lastly, to truly be useful, such an agent must support tracking and verification for a combination of tasks and, ideally, even unseen tasks. These three challenges -- causal and temporal reasoning about task structure from natural language, visual grounding of sub-tasks, and compositional generalization -- form the core goals of our work.

% %% 3. What are we doing? What is our approach?
% \aks{I think this is a matter of preference, but I personally don't like related work in intro. I would make this paragraph be about EgoTV and NSG. Starting with something like - "To this end, we propose...", ie, your next paragraph.}
% \nk{+1, we should move parts of this para to lit review and delete the rest.}
% Recent research on language modeling enables decomposing tasks into multiple steps from natural language descriptions~\cite{llm_zero_shot_planning,proscript}. However, such \emph{task decompositions} cannot directly be leveraged for task tracking in egocentric agents because of lack of grounding into the visual observations or context. In parallel, the computer vision community has advanced action recognition~\cite{}, object detection and tracking~\cite{}, hand object interaction and object state change detection~\cite{ego_4d,change_it,}, step classification in procedural tasks~\cite{}, and even vision language reasoning~\cite{nsvqa,nscl,star_situated_reasoning,clevrer}, which may help with the grounding challenge. However, majority of current research on identifying actions, objects, steps, or state changes does not account for the overall task structure. Likewise, predominant research on vision language understanding~\cite{} and multi-modal grounding~\cite{} does not consider the temporal and causal constraints that emerge in task tracking and verification. We therefore focus on the order-aware visual grounding problem in our work, with an eye towards compositional generalization to scale usability of these agents. In particular, we aim to achieve visual grounding of the actions and objects corresponding to each step or sub-task obtained from the task description decomposition in an order-aware manner.

%% 4. What are our results/contributions?
As our first contribution, we propose a benchmark -- \emph{\textbf{Ego}centric \textbf{T}ask \textbf{V}erification} (\etv \inlineimg{figures/TV}) -- and a corresponding dataset in the AI2-THOR~\cite{ai2thor} simulator. % \emoji{tv}
Given a natural language (NL) task description and a corresponding egocentric video of an agent, the goal of \etv is to verify whether the task was successfully completed in the video or not.
\etv contains multi-step tasks with \emph{ordering} constraints on the steps and \emph{abstracted} NL task descriptions with omitted low-level task details inspired by the needs of real-world assistants. We also provide splits of the dataset focused on different generalization aspects, e.g., unseen visual contexts, compositions of steps, and tasks (see Figure~\ref{figure:dataset}).
% Next, we create splits of the dataset focused on different aspects of generalization, ranging from generalization to unseen visual context to unseen compositions of steps and tasks. Figure~\ref{figure:dataset} shows an example task and overview of generalization splits from \etv. Succeeding at \etv tasks requires decomposing tasks into partially-ordered steps from the NL description and order-aware visual grounding of these steps into the video. 

Our second contribution is a novel approach for order-aware visual grounding~--~\emph{\textbf{N}euro-\textbf{S}ymbolic \textbf{G}rounding} (NSG), capable of compositional reasoning and generalizing to unseen tasks owing to its ability to leverage abstract NL descriptions and compositional structure of tasks (task decomposition, ordering).~In contrast, state-of-the-art vision-language models~\cite{coca,clip,videoclip,clip_hitchiker} struggle to ground NL descriptions in egocentric videos, and do not generalize to unseen tasks.~NSG outperforms these models by~$\mathbf{33.8}\%$~on compositional generalization and~$\mathbf{32.8}\%$~on abstractly described task verification. Finally, to evaluate \nsg on real-world data, we instantiate \etv on the CrossTask~\cite{cross_task} instructional video dataset. %Specifically, we synthetically create videos with mistakes in CrossTask. 
We find that it also outperforms state-of-the-art models at task verification on CrossTask. We hope that the \etv~benchmark and dataset will enable future research on egocentric agents capable of aiding in everyday tasks.

% We experiment with many for the \etv tasks. We find that while these models generalize well to unseen visual context, they struggle to perform grounding from abstracted task descriptions and to generalize to new compositions of tasks. To deal with these challenges, we take inspiration from recent research on and develop . ~\rd{unclear why neurosymbolic models would do well on abstraction.} 

% To summarize, our main contributions are:~1)~\etv: a benchmark and synthetic dataset to systematically study egocentric task verification.
% 2)~\nsg: a novel neuro-symbolic approach to enable the core reasoning capability for \etv -- order-aware visual grounding. We demonstrate \nsg's capability on our synthetic \etv dataset as well as a real-world dataset derived from CrossTask. We will release both of these datasets and our models for future research on egocentric task tracking and verification. 


% Assistive agents require the ability to track actions and state changes from an egocentric perspective for effective assistance in day-to-day tasks. For example, an agent helping a user prepare a recipe would need to both generate the steps of the recipe (\textit{plan generation}) and track the user's actions to ensure the plan is executed correctly (\textit{plan verification}). We formulate this as a Video Entailment task~\cite{violin_dataset,9710490} \rd{should we call our task video-based goal entailment?}, wherein, given an egocentric video of an agent (or human) performing a task (\textit{premise}) and a NL task description (\textit{hypothesis}), the objective is to learn a model to track whether the given task was successfully executed in the video. 
% An ideal model should also be able to seamlessly generalize to novel compositions (of actions and objects) unseen during training. \rd{add a line about what we mean by abstraction and why is it important.} To this end, we generate a novel Vision-Language dataset on the AI2-THOR simulator~\cite{ai2thor} to study compositional and abstraction-based generalization. Our dataset provides effective evaluation measures in a controlled setting, while closely reflecting the diversity of real-world events. We implement and train a variety of end-to-end models based on existing state-of-the-art approaches. We empirically demonstrate that neural models suffer from overfitting and cannot effectively generalize to novel compositions of actions, objects, and scenes. 
% To address this problem, we propose an end-to-end Neuro-Symbolic (NeSy) framework that performs plan generation and verification. At the heart of our approach is the hypothesis that symbolic reasoning models are good at generalization and capturing compositional substructure, while neural models are good at learning representations from sensory data~\cite{10.5555/3326943.3327039,nscl,clevrer}. \rd{summarize contributions in a bulleted list.} \rd{also add a line about the main result e.g., x\% improvement as compared to end-to-end models}. 

% \rd{we also evaluate NeSy with real-world data: add briefly about CrossTask experiments.}

% % \fbox{\begin{minipage}{\linewidth}
% % \textbf{Problem Statement}

% % Given:
% % (i) Premise: Egocentric video of an agent performing a task.
% % (ii) Hypothesis: NL description of the task.

% % Learn: A model to track whether the premise entails the hypothesis. The output of the model is True if the given task is executed successfully in the video.
% % \end{minipage}}

% \textbf{Contributions:} 
% \begin{itemize}
%     \item We generate a benchmark video-language dataset to study compositional and abstraction-based generalization.
%     \item We evaluate the performance of a variety of state-of-the-art models and show that these (baseline) models cannot effectively generalize to novel compositions of actions.
%     \item We propose a novel end-to-end NeSy approach that significantly outperforms the baselines on some compositional generalization splits while performing on par with them on the rest.
%     \item We also evaluate our NeSy approach with real-world data showing similar performance improvements.
% \end{itemize}

\section{Voter Model with Preferences} \label{model}
The VMP, i.e. the generalization of the PVM of Masuda et al. \cite{masuda2010heterogeneous, masuda2011can}, is defined as follows: the system of $N$ agents is divided into two populations, or \textit{classes}, of sizes $N_1$ and $N_2 = N - N_1$, the agents $i = 1,...,N_1$ belonging to the first and the remaining $i = N_1+1,...,N$ to the second one, with $\alpha = \frac{N_1}{N}\in(0,1)$ the fraction of individuals of the first population. A bias $h_i\in[-1,1]$ is assigned to each agent $i$, according to his class: in our bipopulated case, we assign the same $h_1$ to the individuals of the first population, similarly $h_2$ to all the individuals belonging to the second one. Each node's opinion is represented as a binary spin $\sigma_i=\{+1,-1\}$ for $i=1,...,N$. The dynamics obeys the following rules:
\begin{itemize}
    \item One node (agent) $i$ is 
    selected uniformly randomly.
    \item A neighbor $j$ of node $i$ is selected uniformly randomly.
    \item If $i$ and $j$ have opposite opinions, $i$ takes the opinion of $j$ with probability $\frac{1}{2}(1+\sigma_jh_i)$. Otherwise, nothing happens.
    \item Repeat the process until consensus or apparent stabilization is reached.
\end{itemize}
The dynamics is a generalization of the classical voter model - retrieved for $h_i=0,\;\forall i$ - where the individual copies his neighbour's opinion with a probability equal to $\frac{1}{2}$ (in the original voter model this probability is 1, but the factor $\frac{1}{2}$ just slows down the dynamics). The biases modify the transition probabilities, favoring the transition towards the direction of the bias and disfavoring the opposite one. It is easy to show that if in the bipopulated case both of the biases point towards the same direction, then an infinite system will always reach consensus at the preferred state. Thus, in the following we will consider $h_1\geq0$ and $h_2\leq0$, so that the individuals of the first population tend to prefer the $+1$ state, while the ones of the second class are ideologically biased towards the $-1$ opinion.



% The biases modify the transition probabilities, breaking the symmetry between the opinions: they favour the transition towards the direction of the bias and disincentivate the opposite ones. For example, let's consider a positively biased node, $h_i>0$. When it is sorted, it is in state $-1$ and one of his neighbours currently in state $+1$ is sorted, he will perform the transition $-1\rightarrow+1$ with probability $\frac{1}{2}(1+h_i)$, greater than $\frac{1}{2}$. Conversely, when it is in state $+1$ and one of his neighbours currently in state $-1$ is sorted, he will perform the transition $+1\rightarrow-1$ with probability $\frac{1}{2}(1-h_i)$, lower than $\frac{1}{2}$.\\
% It is easy to show that if in the bipopulated case both of the biases point towards the same direction, then an infinite system will always reach consensus at the preferred state\footnote{However, in this case it is still interesting to determine some quantities related to finite systems, such as the exit probability (probability to reach the positive consensus as function of the current state) and the time for consensus.}. Thus, in the following we consider $h_1\geq0$ and $h_2\leq0$; in other words, the individuals of the first population tend to prefer the $+1$ state, while the ones of the second class are ideologically biased towards the $-1$ opinion. Indeed, we will see that for $h_1h_2\leq0$ the phenomenology of the model is way more rich.
\section{VMP on the complete graph}
\label{FC_chapter}
First we study the model, for simplicity, on the complete network. This setting was already investigated in \cite{masuda2011can}, however, we complete the analysis by calculating the polarization measure at the stationary state for any choice of the parameters.\\
Let $\alpha=\frac{N_1}{N}$ be the fraction of nodes in the first class characterized by bias $h_1$. We define
\begin{equation}
    \rho_1 = \frac{\sum_{i=1}^{N_1}\frac{1+\sigma_i}{2}}{N} \in [0,\alpha]\;\;\;\;\;\;\;\;\;\;\;  \rho_2 = \frac{\sum_{i=N_1+1}^{N}\frac{1+\sigma_i}{2}}{N}\in[0,1-\alpha]
\end{equation}
as the ratios between the number of first class spins (respectively second) in current up $+1$ state and the total number of spins in the system. The system of coupled ordinary differential equations which describes the evolution of such dynamical variables can be written generally in terms of the global rates $R_{\pm1/2}$
\begin{equation}
    \begin{cases}
    \dot{\rho_1} = R_{+1}(\rho_1,\rho_2) - R_{-1}(\rho_1,\rho_2)   \\
    \dot{\rho_2} = R_{+2}(\rho_1,\rho_2) - R_{-2}(\rho_1,\rho_2)
    \end{cases}
    \label{eq:rate}
\end{equation}
For example, the global rate $R_{+1}$ represents the probability per unit time that, when the system is currently in state $\rho_1,\rho_2$, a spin of the first class undergoes the transition $-1\rightarrow+1$, increasing 
the density of up spin of the first class of $1/N$, $\rho_1\rightarrow \rho_1+\frac{1}{N}$. Considering a time unit corresponding to $N$ steps, i.e. $\delta t= N^{-1}$, the transition rates for a fully connected network are
\begin{equation}
    \begin{cases} 
    R_{+1}(\rho_1,\rho_2) = (\alpha-\rho_1)\frac{1+h_1}{2}(\rho_1+\rho_2) \\
    R_{-1}(\rho_1,\rho_2) =  \rho_1\frac{1-h_1}{2}(1-\rho_1 -\rho_2)\\\\
    R_{+2}(\rho_1,\rho_2) =  (1-\alpha-\rho_2)\frac{1+h_2}{2}(\rho_1+\rho_2)\\
    R_{-2}(\rho_1,\rho_2) =  \rho_2\frac{1-h_2}{2}(1-\rho_1-\rho_2)
    \end{cases}
\end{equation}
For example, in the first rate $R_{+1}$ the first term $\alpha-\rho_1$ is the probability of chosing uniformly randomly a spin of the first class currently in down state, while $\rho_1+\rho_2$ is the probalbility of choosing a neighbour in state $+1$ in the complete network, and eventually $\frac{1+h_1}{2}$ is the probability of the transition, according to the model dynamics. Thus we have the following mean-field equations
\begin{equation}
\begin{cases}
    \dot{\rho_1} = \frac{1}{2}\bigg[   (\alpha-\rho_1)(1+h_1)(\rho_1+\rho_2) - \rho_1(1-h_1)(1-(\rho_1+\rho_2))\bigg]  \\
    \dot{\rho_2} = \frac{1}{2}\bigg[ (1-\alpha-\rho_2)(1+h_2)(\rho_1+\rho_2) - \rho_2(1-h_2)(1-(\rho_1+\rho_2)) \bigg] 
\end{cases}
\label{fc mf rho12}
\end{equation}
For the complete network in the $N\rightarrow\infty$ limit, the mean-field equations represent exactly the evolution of the system and they
can be applied, as an approximation, 
to other networks. Not considering structural or dynamical correlations, we expect them to be still accurate on a sufficiently dense network without 
specific structural features \cite{peralta2021effect,porter2014dynamical}, such as an Erd\H os-R\'enyi random graph with probability of linkage of $O(1)$. \\
Localizing the fixed points $(\rho_1^*,\rho_2^*)$ of the system (\ref{fc mf rho12}) and characterizing their stability by the analysis of the corresponding Jacobian matrices reported in the appendix, one finds \cite{masuda2011can} that 
\begin{itemize}
    \item The positive $(1,1)$ (all up spins) and negative $(0,0)$ (all down spins) consensus points are always fixed points, for any combination of the parameters $\alpha,h_1,h_2$.
    \item When the positive (or negative) consensus is stable, it is the only stable fixed point.
    \item When both the consensus fixed points are not stable, another fixed point with $\rho_1^*,\rho_2^*\in(0,1)$ appears. Such fixed point, when it exists, is always stable.
\end{itemize}
Defining the total density of up spin $\rho= \rho_1 + \rho_2$ and $\Delta = \frac{\rho_1}{\alpha} - \frac{\rho_2}{1-\alpha} $, the polarization can be expressed\footnote{If we consider the group magnetizations $m_1=\sum\limits_{i=1}^{N_1}\sigma_i$, $m_2=\sum\limits_{i=N_1+1}^{N}\sigma_i$ it is easy to reconduct the expression before to the more conventional expression of the polarization $P=\frac{|m_1-m_2|}{2}\in[0,1]$} as $P=|\Delta|$. As a first contribution of this paper, we calculate that at the \textit{impasse} or \textit{polarized} state the average density of up spins and the polarization respectively read
\begin{align}
    &\rho^* = \frac{1}{2} \bigg(\frac{h_2-h_1}{h_1h_2}\alpha - \frac{1-h_1}{h_1}\bigg) \label{pol_rho} \\
    &\Delta^* =   \frac{1}{h_1-h_2}\bigg( 1 + \frac{\alpha^2h_1^2+(1-\alpha)^2h_2^2-h_1^2h_2^2}{2\alpha(1-\alpha)h_1h_2}\bigg)\label{pol_delta}
\end{align}
Moreover, by analyzing the Jacobian one can localize the critical value of the parameters at which the transitions from negative consensus to polarization and from polarization to positive consensus occur: taking $h_1,h_2$ fixed and letting $\alpha$ vary, we have that the critical points of the transitions above are respectively at
\begin{align*}
        &\alpha^-_c = (1-h_1)\frac{h_2}{h_2-h_1} \\
        &\alpha^+_c = (1+h_1)\frac{h_2}{h_2-h_1}  
\end{align*}
The bifurcation diagrams, taking $\rho$ and $P$ as order parameter and varying the composition $\alpha$ for various fixed $h_1,h_2$, are shown in Figure \ref{fig:fc_bifurcation}: the bifurcation is of transcritical type and the transitions are indeed continous. The presented numerical simulations confirm that the analytical solutions work well for systems defined on relatively small, $N=1000$ complete graphs as well.
\\ 
The length of the interval associated to the polarized state is $\alpha^+_c - \alpha^-_c =2\frac{h_1h_2}{h_2-h_1}$ which reduces to $\alpha^+_c - \alpha^-_c = h$ for $h_1=-h_2 = h$. The phase diagram in this case (in the $\alpha, h$ plane) is shown in Figure \ref{fig:fc phase diagrams}a, while in the remaining plots of the figure $h_2$ is fixed to different values and the phases in the plane $\alpha, h_1$ are shown. To link the bifurcation and the phase diagrams, the horizontal lines corresponding to the choices of the biases in figure \ref{fig:fc_bifurcation} are reported on the latters.\\
Defining the critical mass of a population \cite{centola2018experimental} as the minimum fraction of individuals of that population necessary to escape from consensus at the unpreferred opinion, we have that the critical masses of respectively the first and second populations are $\alpha^{-}_c$ and $1-\alpha_c^{+}$. For very low biases, the populations over the critical masses rapidly overturn the outcome of the system, switching the direction of consensus. In this model, the critical masses depend only on the biases and lay in the whole range $(0,1)$.














% First we 
% \JK{study}
% %analyze 
% the model, for simplicity, on the complete network.
% \JK{This setting was investigated in}
% %, following 
% \cite{masuda2011can}, however, we complete the \JK{analysis by presenting the phase diagram for the entire ranges of the parameters, and calculate the average opinion (magnetization) \FZ{at the stationary state}}. \JK{: IS THIS TRUE? \FZ{WHAT, LEFT OR RIGHT? Left, mmh the average magnetization at the stationary (polarized) state yes, the phase diagram is shown also in alpha h1 plane, which was not in the previous work but its not a big improvement} For the complete network in the large $N$ limit the mean-field approximation becomes exact and it }
% %the dynamical system that we obtain is of mean-field type and 
% can be applied, as an approximation, 
% %on
% \JK{to }other networks. The mean-field approximation does not consider 
% %neither 
% structural or dynamical correlations: we expect it to be 
% %exact for an infinite (no fluctuations) fully connected network, and 
% still accurate on a sufficiently dense network without 
% %remarkable
% \JK{specific }structural features \FZ{IS IT WORTH TO SAY OR IS IT DIDASCALIC AND SOMEWHAT TRIVIAL?}
% %(e.g. high local clustering or modularity) 
% \cite{peralta2021effect,porter2014dynamical}, such as an Erd\H os-R\'enyi random graph with probability of linkage of $O(1)$. 
% % In a finite fully connected network, the mean-field system reproduces exactly the evolution of the average of the dynamical variables over independent runs.
% %We choose different dynamical variables with respect to \cite{masuda2011can} that simplify the treatment and we deepen the analysis of the derived dynamical system. \\ 
% \\
% \\
% \JK{Let $\alpha=\frac{N_1}{N}$ be the fraction \FZ{of} nodes in the first class characterized by bias $h_1$.} We define
% \begin{equation}
%     \rho_1 = \frac{\sum_{i=1}^{N_1}\frac{1+\sigma_i}{2}}{N} \in [0,\alpha]\;\;\;\;\;\;\;\;\;\;\;  \rho_2 = \frac{\sum_{i=N_1+1}^{N}\frac{1+\sigma_i}{2}}{N}\in[0,1-\alpha]
% \end{equation}
% as the ratios between the number of first class spins (respectively second) in current up $+1$ state and the total number of spins in the system. The system of couple ordinary differential equations that describe the evolution of our dynamical variables can be written generally in terms of the global rate $R_{\pm1/2}$ \JK{JK: IT IS UNFORTUNATE TO USE NUMERICAL UPPER INDICES AS IT CAN BE MIXED UP WITH POWERS. PUT THE INDICES INTO PARENTHESIS LIKE $R^{(\pm1/2)}$ OR USE LOWER INDICES LIKE: $R_{\pm1/2}$. THE SAME IS TRUE FOR THE LATER INTRODUCED $\alpha_c$: USE $\alpha_{c}^{(-1)}$ OR $\alpha_{c}^{-}$ AS "1" IS NOT NEEDED HERE.}\FZ{OK, I should change $\alpha^{\pm-} \rightarrow \alpha^\pm$ also in the graph and $R^{\pm 1/2} \rightarrow R_{\pm1/2}$}
% \begin{equation}
%     \begin{cases}
%     \dot{\rho_1} = R_{+1}(\rho_1,\rho_2) - R_{-1}(\rho_1,\rho_2)   \\
%     \dot{\rho_2} = R_{+2}(\rho_1,\rho_2) - R_{-2}(\rho_1,\rho_2)
%     \end{cases}
%     \label{eq:rate}
% \end{equation}
% For example, the global rate $R_{+1}$ represents the probability \JK{per unit time} that, when the system is currently in state $\rho_1,\rho_2$, a spin of the first class undergoes the transition $-1\rightarrow+1$ 
% %(resp. $+1\rightarrow-1$ with minus sign) in a time unit, 
% increasing 
% %(resp. decreasing) 
% the density of up spin of the first class of $1/N$. Notice that such transition produces the change $\rho_1\rightarrow \rho_1+\frac{1}{N}$. Considering a time unit corresponding to $N$ steps, i.e. $\delta t= N^{-1}$, the transition rates for a fully connected network are
% \begin{equation}
%     \begin{cases} 
%     R_{+1}(\rho_1,\rho_2) = (\alpha-\rho_1)\frac{1+h_1}{2}(\rho_1+\rho_2) \\
%     R_{-1}(\rho_1,\rho_2) =  \rho_1\frac{1-h_1}{2}(1-\rho_1 -\rho_2)\\\\
%     R_{+2}(\rho_1,\rho_2) =  (1-\alpha-\rho_2)\frac{1+h_2}{2}(\rho_1+\rho_2)\\
%     R_{-2}(\rho_1,\rho_2) =  \rho_2\frac{1-h_2}{2}(1-\rho_1-\rho_2)
%     \end{cases}
% \end{equation}
% For example, in the first rate $R_{+1}$ the first term $\alpha-\rho_1$ is the probability of chosing uniformly randomly a spin of the first class currently in down state, while $\rho_1+\rho_2$ is the probalbility of choosing a neighbour in state $+1$ in fully connected, and eventually $\frac{1+h_1}{2}$ is the probability of the transition, according to the model. Thus we have 
% %obtained 
% the \JK{following }mean-field equations\\
% \begin{equation}
% \begin{cases}
%     \dot{\rho_1} = \frac{1}{2}\bigg[   (\alpha-\rho_1)(1+h_1)(\rho_1+\rho_2) - \rho_1(1-h_1)(1-(\rho_1+\rho_2))\bigg]  \\
%     \dot{\rho_2} = \frac{1}{2}\bigg[ (1-\alpha-\rho_2)(1+h_2)(\rho_1+\rho_2) - \rho_2(1-h_2)(1-(\rho_1+\rho_2)) \bigg] 
% \end{cases}
% \label{fc mf rho12}
% \end{equation}
% Such equations are exact in a fully connected network of infinite size, while \JK{they} represent exactly the evolution of the averages $\langle\rho_1\rangle,\langle\rho_2\rangle$ over infinite runs for finite size and fully connected topology. \JK{IS THE LATTER STATEMENT TRUE?! \FZ{WELL, THE PRESENCE OF THE ABSORBING STATES FOR A FINITE SYSTEM MAKES ME THINK A BIT NOW, I TRY TO RECOVER THIS IN THE LITERATURE edit: no, probably it is not true... "not represent exactly but approximate", although it is not completely clear: can it happen that fluctuations are not simmetrical?}}For other topologies, the mean-field system above is a rather crude approximation that 
% %totally 
% \JK{completely }neglects dynamical and structural correlations \cite{porter2014dynamical}.\\
% Localizing the fixed points $(\rho_1^*,\rho_2^*)$ of the system (\ref{fc mf rho12}) and characterizing them by the analysis of the corresponding Jacobian matrices reported in the appendix, one finds \cite{masuda2011can} that 
% \begin{itemize}
%     \item The positive $(1,1)$ (all up spins) and negative $(0,0)$ (all down spins) consensus points are always fixed points.
%     \item When the positive (or negative) consensus is stable, it is the only stable fixed point.
%     \item When both the consensus fixed points are not stable, another fixed point with $\rho_1^*,\rho_2^*\in(0,1)$ appears. Such fixed point, when it exists, is always stable.
% \end{itemize}
% The coordinates of this fixed point, that is associated \JK{with the}
% %to what we call 
%  \textit{impasse} or \textit{polarized} state of the system, can be analytically determined by equating to zero the r.h.s. of (\ref{fc mf rho12}) and solving the system. Defining the total density of up spin $\rho= \rho_1 + \rho_2$ and $\Delta = \frac{\rho_1}{\alpha} - \frac{\rho_2}{1-\alpha} $, the polarization can be expressed\footnote{If we consider the group magnetizations $m_1=\sum\limits_{i=1}^{N_1}\sigma_i$, $m_2=\sum\limits_{i=N_1+1}^{N}\sigma_i$ it is easy to reconduct the expression before to the more conventional expression of the polarization $P=\frac{|m_1-m_2|}{2}\in[0,1]$} as $P=|\Delta|$. We have that at the \textit{polarized} state
% \begin{align}
%     &\rho^* = \frac{1}{2} \bigg(\frac{h_2-h_1}{h_1h_2}\alpha - \frac{1-h_1}{h_1}\bigg) \label{pol_rho} \\
%     &\Delta^* =   \frac{1}{h_1-h_2}\bigg( 1 + \frac{\alpha^2h_1^2+(1-\alpha)^2h_2^2-h_1^2h_2^2}{2\alpha(1-\alpha)h_1h_2}\bigg)\label{pol_delta}
% \end{align}
% Moreover, by analyzing the Jacobian one can localize the critical value of the parameters at which the transitions from negative consensus to polarization and from polarization to positive consensus occur: taking $h_1,h_2$ fixed and letting $\alpha$ vary, we have that the critical points of the transitions above are respectively at
% \begin{align*}
%         &\alpha^-_c = (1-h_1)\frac{h_2}{h_2-h_1} \\
%         &\alpha^+_c = (1+h_1)\frac{h_2}{h_2-h_1}  
% \end{align*}
% The bifurcation diagrams, taking $\rho$ and $P$ as order parameter and varying the composition $\alpha$ for various fixed $h_1,h_2$, are shown in Figure \ref{fig:fc_bifurcation}: the bifurcation is of transcritical type and the transitions are indeed continous. \JK{The presented }numerical simulations 
% %that 
% confirm \JK{that }the analytical \JK{solutions work well for systems defined on relatively small, $N=1000$ complete graphs as well.}
% %predictions are also reported.
% \\ 
% %Interestingly, t
% The length of the interval associated to the polarized state is $\alpha^+_c - \alpha^-_c =2\frac{h_1h_2}{h_2-h_1}$ which reduces to $\alpha^+_c - \alpha^-_c = h$ 
% %in the case of equally strong opposite biases 
% \JK{for }$h_1=-h_2 = h$. The phase diagram in this case (in the $\alpha, h$ plane) is shown in 
% %the upper left plot of 
% Figure \ref{fig:fc phase diagrams}\JK{a}, while in the remaining plots of the figure $h_2$ is fixed to different values and the phases in the plane $\alpha, h_1$ are shown. To link the bifurcation and the phase diagrams, the horizontal lines corresponding to the choices of the biases in figure \ref{fig:fc_bifurcation} are reported on the latters.\\
% \FZ{!!! Maybe you did not read this part, where I first mention the critical mass}The value $\alpha^{-1}_c$ can be intended as the critical mass \cite{centola2018experimental} \FZ{related to the first populations, i.e. the minimum fraction of individuals of the first class necessary to escape from consensus towards the unpreferred $-1$ opinion,} of individuals of the first population to escape from the consensus towards the unpreferred $-1$ opinion, and same thing for $1-\alpha_c^{+1}$ but referred to the second population. Especially \FZ{Specifically} for low biases, the populations over the critical masses totally \FZ{rapidly} overturn the outcome of the system, changing \FZ{no rapidly} rapidly the direction of consensus. In this model, the critical masses depend only on the biases and lay in the whole range $(0,1)$.\\



\begin{figure}
    \subfloat{\includegraphics[width = 3in]{Figures/fc_bifurcation_rho.png}}
    \subfloat{\includegraphics[width = 3in]{Figures/fc_bifurcation_delta.png}}\\
    \caption{\textbf{Bifurcation diagrams for the bipopulated Voter Model with Preferences on the complete network.} On the right the total density of up spin is taken as order parameter, on the left the polarization $\Delta$ is shown. The solid black line indicates the stable fixed point, while the dashed gray lines indicate the unstable ones, for the choice of the preferences' intensities $h_1=0.4,h_2=-0.6$. The other solid colored lines locate just the coexistence stable fixed point for other choices of the intensities, as indicated in the legend. The points and their bars are respectively the average and the confidence interval of the order parameters calculated over $30$ independent simulations of a system with $1000$ agents, and are reported in order to test the validity of the mean field treatment also for relatively small system sizes.}
    \label{fig:fc_bifurcation}
\end{figure}




\begin{figure}
\subfloat(a){\includegraphics[width = 2.7in]{Figures/phase_diagram/phase_diagEQ.png}}
\subfloat(b){\includegraphics[width = 2.7in]{Figures/phase_diagram/phase_diag06.png}}\\
\subfloat(c){\includegraphics[width = 2.7in]{Figures/phase_diagram/phase_diag08.png}} 
\subfloat(d){\includegraphics[width = 2.7in]{Figures/phase_diagram/phase_diag09.png}}\\
\caption{\textbf{Phase diagrams for the complete network.} The polarized area is colored in light purple and indicated with $P$, in white are $+$ and $-$ consensus. Figure (a) shows the mean-field phases in the $\alpha h$ plane, for equal and opposite preferences' intensities $h_1=-h_2=h$. The dark purple dot represents the regime $\alpha =\frac{1}{2},h\rightarrow0$ in which \cite{borile2013effect} have investigated finite size effects. Figures (b),(c),(d) report the phases in the $\alpha h_1$ plane, once fixed $h_2$ respectively to $-0.6,-0.8,-0.9$. Each of the colored horizontal lines present in some of the plots represents the choice of the biases $h_1,h_2$ as in figure \ref{fig:fc_bifurcation} (the colors correspond). They are reported in order to show how the lines intersecate the different phases.}
\label{fig:fc phase diagrams}
\end{figure}



\section{VMP on modular networks}
\label{chapter SBM}
To study the effect of the topology reflecting the biased communities of the bipopulated VMP, we analyze the model on a network with two modules of sizes $N_1$ and $N_2$, generated by a Stochastic Block Model (SBM) \cite{lee2019review}. The SBM is defined by the intra-modular ($p_{11},p_{22}$) and intermodular ($p_{12},p_{21}$) connectivities, i.e., the probabilities describing the corresponding linkings between the agents (for undirected networks $p_{12}=p_{21})$). We assume that the network results from homophilic interactions (epistemic bubbles) such that all agents within module $1$ ($2$) have bias $h_1$ ($h_2$).
\\
We start from the mean-field equations (\ref{eq:rate}) where global transition rates are now functions of the connectivities of the block model
\begin{equation}
    \begin{cases} 
    R_{+1}(\rho_1,\rho_2) =\frac{1}{\alpha p_{11} + (1-\alpha)p_{12}} (\alpha-\rho_1)\frac{1+h_1}{2}(p_{11}\rho_1+p_{12}\rho_2) \\
    R_{-1}(\rho_1,\rho_2) = \frac{1}{\alpha p_{11} + (1-\alpha)p_{12}} \rho_1\frac{1-h_1}{2}[p_{11}(\alpha-\rho_1)+p_{12}(1-\alpha -\rho_2)]\\\\
    R_{+2}(\rho_1,\rho_2) = \frac{1}{\alpha p_{12} + (1-\alpha)p_{22}} (1-\alpha-\rho_2)\frac{1+h_2}{2}(p_{12}\rho_1+p_{22}\rho_2)\\
    R_{-2}(\rho_1,\rho_2) = \frac{1}{\alpha p_{12} + (1-\alpha)p_{22}} \rho_2\frac{1-h_2}{2}[p_{12}(\alpha-\rho_1)+p_{22}(1-\alpha -\rho_2)]
    \end{cases}
    \label{SBM transition rates}
\end{equation}
For example, once a spin in down state of class 1 is selected, the probability that we find one of its neighbours in up state is 
$\frac{p_{11}\rho_1}{\alpha p_{11} + (1-\alpha)p_{12}}+\frac{p_{12}\rho_2}{\alpha p_{11} + (1-\alpha)p_{12}}$ (for more details, see the Appendix \ref{app transition rates}).\\
We end up with the mean-field evolution equations for the densities $\rho_1,\rho_2$
\begin{equation}
    \begin{cases}
    \dot{\rho_1} = C_1 \bigg[ (\alpha-\rho_1)(1+h_1)(p_{11}\rho_1+p_{12}\rho_2) - \rho_1(1-h_1)[p_{11}(\alpha-\rho_1)+p_{12}(1-\alpha -\rho_2)] \bigg]  \\
    \dot{\rho_2} = C_2 \bigg[ (1-\alpha-\rho_2)(1+h_2)(p_{12}\rho_1+p_{22}\rho_2) - \rho_2(1-h_2)[p_{12}(\alpha-\rho_1)+p_{22}(1-\alpha -\rho_2)] \bigg] 
    \end{cases}
\label{SBM mf system}
\end{equation}
where $C_1 = \frac{1}{2[\alpha p_{11} + (1-\alpha)p_{12}]}$, $C_2 = \frac{1}{2[\alpha p_{12} + (1-\alpha)p_{22}]}$.\\
It is easy to see that the consensus states are still fixed points. The numerical study of the system shows that the qualitative behaviour of the fully connected case is preserved, i.e., the different phases are separated by transcritical bifurcations, but the polarization area generally widens and the range of the polarized phase increases. Figure \ref{fig:SBMalpha}
compares the fully connected case and the topology characterized by two cliques ($p_{11}=p_{22}=1$) and intercommunity connectivities $p_{12}=p_{21}=0.3$, with equally strong opposite preferences $h_1=-h_2=h$. In general, the symmetric community structure decreases the values of the critical masses, with respect to the fully connected topology.
\begin{figure}
    \centering
    \includegraphics[width=0.7\textwidth]{Figures/SBM_p1_ritag.png}\\
    \includegraphics[width=0.7\textwidth]{Figures/SBM_p03_ritag.png}
    \caption{\textbf{VMP on modular networks with equally strong opposite preferences, $\alpha, h$ plane.} The mean field predictions are shown, i.e. the stable fixed point of the system (\ref{SBM mf system}), for two choices of connectivities of the modular network. \textit{Upper plots}: case $p_{11}=p_{22}=p_{12}=p_{21}=1$, corresponding to the fully connected topology (subplot (a) in figure \ref{fig:fc phase diagrams}). \textit{Lower plots}: case $p_{11}=p_{22}=1$, $p_{12}=p_{21}=0.3$, i.e. still symmetric probabilities but segregated communities. The polarization increases and the polarization area widens, nevertheless the general qualitative behaviour is preserved.}
    \label{fig:SBMalpha}
\end{figure}
\newpage
By considering the evolution of the normalized densities $\rho'_1 = \frac{\rho_1}{\alpha} \in [0,1]$, $\rho'_2 = \frac{\rho_2}{1-\alpha} \in [0,1]$ and defining the topological parameters
\begin{equation}
    \gamma_1 = \frac{p_{12}(1-\alpha)}{p_{11}\alpha +p_{12}(1-\alpha)} \;\;\;\;\;\;\;\;\; \gamma_2 = \frac{p_{21}\alpha}{p_{22}(1-\alpha) +p_{21}\alpha}
\end{equation}
we have a consistent reduction in the number of parameters in the mean-field system on the SBM (\ref{SBM mf system}), which reads
\begin{equation}
\begin{cases}
    \dot{\rho'_1} = \frac{1}{2}\bigg[   (1-\rho'_1)(1+h_1)[(1-\gamma_1)\rho'_1+\gamma_1\rho'_2] - \rho'_1(1-h_1)[(1-\gamma_1)(1-\rho'_1)+\gamma_1(1-\rho'_2)]\bigg]  \\
    \dot{\rho'_2} = \frac{1}{2}\bigg[ (1-\rho'_2)(1+h_2)[\gamma_2\rho'_1+(1-\gamma_2)\rho'_2) - \rho'_2(1-h_2)[\gamma_2(1-\rho'_1)+(1-\gamma_2)(1-\rho'_2)] \bigg] 
\end{cases}
\label{mf_echochambers}
\end{equation}
The interpretation of $\gamma_1,\gamma_2$ is straightforward in terms of the average internal and external degrees $z_{11},z_{12},z_{21},z_{22}$  \footnote{The internal degrees $z_{11},z_{22}$ are the numbers of connections that an agent of respectively the first and second community on average has within his community, while the external degrees $z_{12},z_{21}$ are the average number of connections towards the other community}
\begin{equation}
    \gamma_1 = \frac{z_{12}}{z_{11}+z_{12}} \;\;\;\;\;\;\;\;\; \gamma_2 = \frac{z_{21}}{z_{22}+z_{21}}
\end{equation}
Being the average fractions of external connections over the total number of connections of the agents in class 1 and 2, $\gamma_1,\gamma_2$ can be intended as the average \textit{open-mindedness} of the individuals of respectively the first and second community. Since in an epistemic bubble the agents overrepresent (i.e., are more linked to) their belonging community, we expect $\gamma_1 \in (0,1-\alpha)$ and $\gamma_2 \in (0,\alpha)$. The more far $\gamma_1,\gamma_2$ are from these upper extremes, the more the individuals of the corresponding population have unbalanced sources of information, i.e. are trapped in the bubble. If the individuals have on average more connections within their belonging community rather than towards the other, then $\gamma_1,\gamma_2\in(0,0.5)$.
Figure \ref{fig:echo_gamma} shows the stationary polarization\footnote{$P^*=\Delta^*$, since it is easy to check that at the fixed point $\Delta^*\geq0$ for $h_1\geq0,\;h_2\leq0$.} $\Delta^* = \rho'^*_1-\rho'^*_2$ in the $h_1h_2$ plane, for different choices of the open-mindedness parameters $\gamma_1,\gamma_2$, by numerically calculating the fixed points of the mean-field system (\ref{mf_echochambers}) and determining their stability.\\
\\
\\
From the linear stability analysis of the consensus fixed points of the system (\ref{mf_echochambers}), reported in Appendix \ref{Modular lsa}, we obtain the condition for the stability of the positive consensus 
\begin{equation}
    \gamma_1h_2(1-h_1)+\gamma_2h_1(1-h_2) + 2h_1h_2\geq0
\end{equation}
and for the negative one
\begin{equation}
     -\gamma_1h_2(1+h_1)-\gamma_2h_1(1+h_2)+2h_1h_2\geq0
\end{equation}
First we fix $\gamma_1,\gamma_2$ and determine the critical lines in the $h_1,h_2$ plane. The line separating the space when the positive consensus is stable (below the line) and the one for which it is unstable reads
\begin{equation}
    h_2^c(h_1) = -\frac{\gamma_2h_1}{\gamma_1(1-h_1)+h_1(2-\gamma_2)}
\end{equation}
and it is bounded superiorly by $h_2^{c+} = \frac{\gamma_2}{2-\gamma_2}$, which is reached at $h_1=1$ and does not depend on $\gamma_1$. \\
In the same way, the critical line related to the negative consensus fixed point reads
\begin{equation}
    h_1^c(h_2) = \frac{-\gamma_2h_2}{\gamma_2(1+h_2)-h_2(2-\gamma_1)}
\end{equation}
and it is bounded by $h_1^{c-} = \frac{\gamma_1}{2-\gamma_1}$, independent of $\gamma_2$. The critical lines are drawn for a choice of the open-mindedness parameters in figure \ref{fig:linear_stab_analysis}b.\\
In the $\gamma_1,\gamma_2$ plane, the critical line  for the positive consensus
\begin{equation}
    \gamma_2^c(\gamma_1) = \frac{h_1(1-h_2)}{-h_2(1-h_1)} \gamma_1 +\frac{2h_1h_2}{h_2(1-h_1)}
\label{lsa critical line gamma}
\end{equation}
is straight line in the plane, whose coefficient goes as $\sim h_1/(1-h_1)$ with $h_1$ and $\sim (1-h_2)/-h_2$ with $-h_2$, thus exploding for high $h_1$ and low $-h_2$. 
% Indeed, as shown in figure \ref{fig:linear_stab_analysis}c, once fixed $h_2 =-0.2$ the critical line becomes almost an horizontal line for high $h_1$. 
Thus, the critical points for different topologies of the first population, i.e. different $\gamma_1$, happen to be at approximately the same value of $\gamma_2$, as shown in figures \ref{fig:linear_stab_analysis}c and \ref{fig:linear_stab_analysis}d. Naturally, symmetric considerations hold for the negative consensus.\\
\\
The results of linear stability analysis allow us to conclude that if one population is very committed and the other population's bias is low, e.g. when $h_1>>-h_2$, wether or not such population manages to impose its preferred opinion depends largely on the open-mindedness of the other population $\gamma_2$,  while the dependency on the topological structure of the committed population $\gamma_1$ is negligible.\\
\\
We test this claim by numerically simulating the VMP on modular networks for a finite system of $N=1000$ agents, fixing $h_1=0.7,h_2=-0.2$ and varying the topology through the open-mindedness parameters $\gamma_2$ and $\gamma_1$. The results, shown in Figure \ref{fig:linear_stab_analysis}a, support the mean-field predictions by showing that the critical value of $\gamma_2$ separating the polarization and positive consensus phases is approximately the same for all the three lines in the $\gamma_2,\rho^*$ plane, corresponding to the three chosen values of $\gamma_1$.



\begin{figure}
    \centering
    \includegraphics[width=\textwidth]{Figures/echochamb_final_good.png}
    \caption{\textbf{VMP on modular networks.} Mean-field predictions for the polarization (eq. \ref{mf_echochambers}) in the $h_1,h_2$ plane for two equally open-minded communities ($\gamma_1=\gamma_2$, upper line), and for three values of $\gamma_2$, varying the open-mindedness of the first population $\gamma_1$. Decreasing the open-mindedness, the consensus areas shrink and the system becomes more and more polarized.}
    \label{fig:echo_gamma}
\end{figure}





\begin{figure}
    \textit{\subfloat(a)}{\includegraphics[width = 3in]{Figures/stab_analys_gammadepN.png}}
    \textit{\subfloat(b)}{\includegraphics[scale = 0.496]{Figures/stab_analys_echochN.png}}\\
    \textit{\subfloat(c)}{\includegraphics[width = 3in]{Figures/lsa_horizontal.png}} 
    \textit{\subfloat(d)}{\includegraphics[width = 3in]{Figures/lsa_gamma.png}}\\
    \caption{\textbf{Simulations and analytical results for the VMP on a modular graph.}\\ \textit{(a)} The average density of up spin at the stationary state is calculated over 30 runs of the model, with $N=1000$ agents and preferences $h_1=0.7,h_2=-0.2$, varying the open-mindedess of the second population and for 3 values of the open-mindedness of the first population, corresponding to different colors. We can see that all the three lines approach the positive consensus at approximately the same $\gamma_2$. \\
    \textit{(b)} Critical lines for the positive and negative consensus points in the $h_1h_2$ plane, for $\gamma_1=0.5,\gamma_2=0.3$ (black). On the background, the polarization values calculated numerically as in figure \ref{fig:echo_gamma} are reported, to show the consistency of the results of the linear stability analysis. The red and blue lines are the positive consensus critical lines for other values of $\gamma_1$, while $\gamma_2=0.3$ constantly. The figure is the analytical correspondant of $\gamma_1 = 0.1,0.3,0.5$, $\gamma_2=0.3$ plots in figure \ref{fig:echo_gamma}. \\
    \textit{(c)} Critical lines in the $\gamma_1\gamma_2$ plane for $h_2=-0.2$ and different $h_1$. The figure shows that the higher $h_1$, the more the critical line tends to an horizontal line. \\
    \textit{(d)}  Once fixed the biases to $h_1=0.7,h_2=-0.2$, in the $\gamma_1\gamma_2$ plane are reported the analytical critical line (\ref{lsa critical line gamma}) and the values of $\gamma_1$ (vertical lines) corresponding to the lines in figure (a): we see that the intersections with the critical line (i.e. the points at which the positive consensus becomes stable) occur almost at the same $\gamma_2$, for all the three values of $\gamma_1$.    }
    \label{fig:linear_stab_analysis}
\end{figure}
\section{Pair Approximation}\label{PA_chapter}
The aim of this section is to investigate how a more refined approximation better reproduces the model's behaviour on complex networks. We implement the so called Pair Approximation \cite{gleeson2011high,gleeson2013binary}, which takes into consideration dynamical correlations at a pairwise level. \\
For the sake of simplicity, the Pair Approximation is applied on an undirected regular\footnote{regular in the sense that each node of the class has the same number of internal $z_{ii}$ and external $z_{ij}$ connections.} modular graphs with two communities, nevertheless the treatment can be easily extended to a directed modular network with heterogeneous degrees \cite{pugliese2009heterogeneous}. Generalizing the approach followed by \cite{peralta2021effect} to class-dependent infection/recovery rates, reported in the appendix \ref{app_PA}, we consider the set of probabilities of the kind $P^{ij\pm}_{z_{ij},m_{ij}}$, i.e. the probability that a node in the population $i$ currently in state $\pm1$ has $z_{ij}$  degree\footnote{trivial in the $z$-regular case, explicited only for the generalization} and $m_{ij}$ neighbours of the population $j$ currently in state $+1$.\\
The mean-field approximation consists in taking those probabilities highly peaked at the values of $\frac{m_{ij}}{z_{ij}}$ coinciding to the overall normalized densities $\rho'_1,\rho'_2$, so having the shape of delta functions explicited in (\ref{MF_probabilities_start}-\ref{MF_probabilities_end}). The Pair Approximation, instead, considers pairwise dynamical correlations by taking those probabilities as binomial distributions with single event probabilitiy corresponding to the ratio between the number of active links departing from the node's type, intended as its class and state, and the number of connections from that node's type. The argument is explained in the appendix (\ref{eq:PA_proba_start}-\ref{PA_proba_end},\ref{PA_single_proba_start}-\ref{PA_single_proba_end}).\\
The latter ratios are also dynamical variables of the system derived from the pair approximation (\ref{PA_system}), which consists indeed of six coupled differential equations. The increase in complexity is justified by a gain in accuracy, consistent for low average degrees, i.e. for sparse graphs. In figure \ref{fig:PA and MF} we compare the mean-field and Pair approximation effectiveness in reproducing the dynamics of the system, on an extremely sparse modular network. We see that the gain in accuracy is consistent, and the PA is able to predict the dynamics almost perfectly. On the right plot of the figure, we test multiple initial conditions in order to check that, as in the mean-field treatment, the stable fixed point is unique. 

\begin{figure}
\subfloat{\includegraphics[width = 0.5\textwidth]{Figures/mfANDpa.png}} 
\subfloat{\includegraphics[width = 0.5\textwidth]{Figures/multiple_initial_conditions.png}}\\
\caption{\textbf{Pair and Mean-Field approximations.} \textit{Left plot}: Dynamics of the MF (dotted line) and PA (dashed lines) approximations compared with numerical simulations (thin solid lines) averaged over $30$ independent runs. The graph is a sparse modular network of $N=3000$ nodes with two $z-$regular communities $z_{11}=4,z_{12}=2,z_{21}=1,z_{22}=3$, thus $\alpha = \frac{2}{3}$ and $\gamma_1=0.4 ,\gamma_2=0.25$. The two communities' preferences are respectively $h_1=0.3,h_2=-0.5$. The initial opinions are chosen uniformly random ($\rho_1'(0) = \rho_2'(0) = \frac{1}{2}$). \textit{Right plot}: Same setting repeated for various initial conditions $\rho'_1(0),\rho'_2(0)$. The results show that even for low connectivities the stable fixed point is unique.}
\label{fig:PA and MF}
\end{figure}


\section{Application to the network of blogs}\label{blognetwork_chapter}
We run the bipopulated VMP on a real social network with a moudular structure, to study how well the mean-field approximation performs on a real network with high modularity and potentially structural features as well as dynamical correlations. We take the network of Political Blogs during 2004 American elections \cite{adamic2005political}, characterized by the presence of two communities that reflect political bi-partisanships. After eliminating the nodes with degree less than $4$, and transforming for simplicity the original directed network in undirected (such that $p_{12}=p_{21}$), we apply a community detection algorithm \cite{kernighan1970efficient}. It turns out that the two communities, named "reds" and "blues" (shown in the upper plot of figure \ref{fig:blog net}), have sizes respectively $N_r = 413 $, $N_b= 490 $, average internal degrees $z_{rr} = 34.02 $, $z_{bb} = 32.38 $ and external degrees $z_{rb} = 2.88$, $z_{br} = 2.43$. We perform numerical simulations of the model choosing, for simplicity, opposite and equally strong preferences ($h_1 = -h_2 = h$), for multiple values of $h$, and compare them to the mean field predictions on a SBM with the same average degrees. The results are reported in figure \ref{fig:MF and blog net}: the mean-field predictions agree well with the empirical simulations, with a small gap emerging for weak preference intensities. In the lower plot of figure \ref{fig:blog net} it is reported the average polarization of each node in the asymptotic state ($t=50$) over repeated runs of the model with equal and opposite biases fixed to $h=0.3$: as one could expect, the most open-minded nodes adopt their unpreferred opinion more frequently than the ones in the rest of their community. Despite this effect due to the heterogeneity of the degrees and specifically of the local open-mindednesses, the mean-field predictions remain quite accurate.


\begin{figure}
\centering
\subfloat{\includegraphics[width = 0.8\textwidth]{Figures/commdet.png}}\\ 
\subfloat{\includegraphics[width = 0.8\textwidth]{Figures/blogmagn.png}}
\caption{\textbf{VMP on the 2004 Blogosphere.} \textit{Upper plot}: 2004 Political Blogosphere after community detection: nodes' colors reflect political partisanships. \textit{Lower plot}: each node is colored according to the average state $\bar{\rho_i}$ at time $t=50$, for a bipopulated VMP where the populations are the ones computed by community detection and the preferences are set to $h_r = -h_b = 0.2$. The colorscale goes from blue, corresponding to $\bar{\rho_i}=0$, to red ($\bar{\rho_i}=1$). The averages are computed running $100$ independent simulations.}
\label{fig:blog net}
\end{figure}



\begin{figure}
\centering
\includegraphics[width = 0.7\textwidth]{Figures/blog_polarization/blogpola_avgstd.png}
\caption{\textbf{Mean-field predictions and simulations of the model on the 2004 Blogosphere.} We compare the mean-field predictions and the numerical simulations of the asymptotic polarization on the 2004 Political Blogosphere, restricting to equally strong biases $h_r = -h_b = h$, for various $h$.}
\label{fig:MF and blog net}
\end{figure}

\section{Conclusions}
\label{sec:conclusions}

We have demonstrated that the fraction of negative event weights in
existing large high-multiplicity samples can be reduced by more than
an order of magnitude, whilst preserving predictions for observables
within statistical uncertainties. Concretely, we have employed the cell
resampling method proposed in~\cite{Andersen:2021mvw} with NLO event
samples for Z boson production with up to three jets
and W boson production with five jets produced with \textsc{Sherpa}
and \textsc{BlackHat}.

For the first time, cell resampling has been applied to samples with
up to several billions of events. This was made possible by
algorithmic improvements leading to a speed-up by several orders of
magnitude. Our updated implementation can be retreived from
\url{https://cres.hepforge.org/}.

The advances in the development of the cell resampling method
presented in this work pave the way for future applications to processes with
high-multiplicities, in particular including parton showered
predictions. It will be necessary to quantify the uncertainty
introduced by the weight smearing. Variations in the maximum cell size
parameter and different prescriptions for weight redistribution within
a cell can serve as handles to assess this uncertainty. Another
promising avenue for further exploration is the analysis of the
information on weight distribution within phase space collected during
cell resampling. Regions with insufficient Monte Carlo statistics
could be identified by their accumulated negative weight, thereby
guiding the event generation. We leave the investigation of these
questions to future work.

\section*{Acknowledgements}

AM thanks Zahari Kassabov for encouragement to reconsider the use of nearest
neighbour search trees. The work of JRA and DM is supported by the STFC under
grant ST/P001246/1.

%%% Local Variables:
%%% mode: latex
%%% TeX-master: "main"
%%% End:

\section*{Acknowledgement} 
The research reported in this work was partially supported by the EU H2020 ICT48 project "Humane AI Net" under contract \# 952026, by the European Union – Horizon 2020 Program under the scheme “INFRAIA-01-2018-2019 – Integrating Activities for Advanced Communities”, Grant Agreement n.871042, “SoBigData++: European Integrated Infrastructure for Social Mining and Big Data Analytics”, and by the CHIST-ERA grant "SAI": CHIST-ERA-19-XAI-010, FWF (grant No. I 5205). The authors thank Antonio F. Peralta for useful discussions.

\renewcommand*{\bibfont}{\small}
% % \printbibliography[title=References]
\printbibliography

\appendix
\section{Linear stability analysis}
\subsection{Fully connected network}
Here we perform the linear stability analysis of the dynamical system (\ref{fc mf rho12}) derived for the VMP on the fully connected topology, giving a proof to the considerations in chapter \ref{FC_chapter}. As said in the main text, the consensus points $(0,0)$ and $(\alpha,1-\alpha)$ are fixed points of the system for whatever choice of the parameters $\alpha, h_1,h_2$, while it is easy to prove that only for $\alpha\in(\alpha_c^{-},\alpha_c^{+})$ the fixed point corresponding to the \textit{polarized} state (\ref{pol_rho},\ref{pol_delta}) is in the rectangle $(0,\alpha)\times(0,1-\alpha)$, and thus has a physical meaning. In the following, we prove that for conflicting preferences, i.e. $h_1\geq0, h_2\leq0$, for $\alpha<\alpha_c^{-}$ the negative consensus fixed point is the only stable fixed point of the system, for $\alpha\in(\alpha_c^{-},\alpha_c^{+})$ both the consensus points are unstable and the polarized fixed point is stable, while for $\alpha>\alpha_c^{-}$ the positive consensus is the only stable fixed point.. \\
\\
The Jacobian of the dynamical system reads
\begin{equation}
    J(\rho_1,\rho_2;\alpha,h_1,h_2) = \frac{1}{2}
    \begin{pmatrix}
    \alpha-1+h_1(1+\alpha-4\rho_1-2\rho_2) & \alpha+h_1(\alpha-2\rho_1)  \\
    1-\alpha +h_2(1-\alpha-2\rho_2) & -\alpha +h_2(2-\alpha-4\rho_2-2\rho_1) \\
    \end{pmatrix}
\end{equation}
According to the linear stability theory, a fixed point is stable if both the eigenvalues of the corresponding Jacobian are negative, i.e. if the trace $T$ is negative and the determinant $D$ is positive.\\
\\
For the negative consensus fixed point we have that 
\begin{align}
    & T = \frac{1}{2}\bigg[\alpha(h_1-h_2)+2h_2+h_1-1\bigg] \\
    &D = \frac{1}{2}\bigg[h_1h_2-h_2-\alpha(h_1-h_2)\bigg] 
\end{align}
so the determinant is positive for $\alpha<\alpha^D=\frac{-h_2(1-h_1)}{h1-h_2}$ , while the trace is negative for $\alpha<\alpha^T=\frac{1-2h_2-h_1}{h_1-h_2}$. Thus the stability condition is satisfied for $\alpha<\mbox{min}(\alpha^T,\alpha^D)$. It is easy to see that $\alpha^D\leq\alpha^T$, since for the assumptions on the sign of the preferences it holds that $h_1h_2 + h_1 +h_2 \leq1$, so the negative consensus point is a stable attractive fixed point if and only if
$\alpha<\alpha^D = \alpha_c^{-}$.\\
\\
Analogous considerations apply to the positive consensus fixed points, whose determinant and trace read
\begin{align}
    & T = \frac{1}{2}\bigg[-\alpha(h_1-h_2)-h_1-2h_2-1\bigg] \\
    & D = \frac{1}{2}\bigg[\alpha(h_1-h_2) + h_1h_2 + h_2 \bigg] 
\end{align}
and by applying the same arguments as before we get that the stability condition is fulfilled when $\alpha>\mbox{max}(\alpha^D,\alpha^T) = \alpha^D = \alpha_c^{+}$.\\
\\
Last, the same considerations about the trace and determinant can be used to prove that in the range $[\alpha_c^{-},\alpha_c^{+}]$ the fixed point corresponding to the \textit{polarized} state is stable.













\subsection{Modular network} \label{Modular lsa}
We perform the linear stability analysis of the mean-field system (\ref{mf_echochambers}) derived from the VMP on a modular network with two communities and open-mindedness parameters $\gamma_1,\gamma_2$. The dynamical variables taken into consideration are now $\rho'_1,\rho'_2$ as defined in chapter \ref{chapter SBM}, both in the range $[0,1]$. The analysis focuses on the stability of the consensus points, now $(0,0)$ for the negative and $(1,1)$ for the positive, that are fixed points of the system for whatever choice of the parameters. \\
\\
The Jacobian matrix of the system (\ref{mf_echochambers}), discarding the uninfluential factor, reads
\begin{equation}
\small    J(\rho'_1,\rho'_2) = 
     \begin{pmatrix}
        \gamma_1(4h_1\rho_1-2h_1\rho_2-h_1-1)-4h_1\rho_1+2h_1 & \gamma_1[1-h_1(2\rho_1-1)]\\
        \gamma_2[1-h_2(2\rho_2-1)]  &  \gamma_2(4h_2\rho_2-2h_2\rho_1-h_2-1)-4h_2\rho_2+2h_2 \\
    \end{pmatrix}
\end{equation}
and as before we compute the traces and the determinant of the Jacobian at the fixed point in order to obtain the stability conditions. \\
\\
The traces and determinant for the positive consensus point read
\begin{align}
    & T = -\gamma_1(1-h_1)-\gamma_2(1-h_2)-2(h_1+h_2)\\
    & D = 2\big[ \gamma_1h_2(1-h_1) +\gamma_2h_1(1-h_2)+2h_1h_2 \big]
\end{align}
and we see that the trace is negative for all $h_1\geq -h_2$, while the determinant is positive for 
\begin{equation}
    \gamma_1h_2(1-h_1)+\gamma_2h_1(1-h_2) + 2h_1h_2>0
\label{det_condition}
\end{equation}
However, it never happens that $T>0$ and $D>0$ at the same time, at least in the parameters' space of interest $h_1\geq0,h_2\leq0,\gamma_{1/2}\geq0$. To prove it, we try to solve the system
\begin{equation}
\begin{cases}
    T = -\gamma_1(1-h_1)-\gamma_2(1+|h_2|)-2(h_1-|h_2|)>0\\
    D \propto -\gamma_1|h_2|(1-h_1) +\gamma_2h_1(1+|h_2|)-2h_1|h_2|>0
\end{cases}
\end{equation}
Arranging the terms we are left with the series of inequalities
\begin{equation}
    -\gamma_1(1-h_1)-2(h_1-|h_2|) > \gamma_2(1+|h_2|) > \frac{1}{h_1} \big(\gamma_1|h_2|(1-h_1) +2h_1|h_2|\big)
\end{equation}
that implies 
\begin{equation}
    -\gamma_1(1-h_1)-2(h_1-|h_2|)  > \frac{1}{h_1} \big(\gamma_1|h_2|(1-h_1) +2h_1|h_2|\big)
\end{equation}
and simplifies in 
\begin{equation}
       -\gamma_1h_1(1-h_1)-2h_1^2 >  \gamma_1|h_2|(1-h_1)
\end{equation}
which is never true, since the terms on the l.h.s. are always negative and the term on the r.h.s. is positive. Thus, the whole stability region of the positive consensus fixed point is determined by the condition derived from the determinant (\ref{det_condition}), and thus delimited by the critical curve
\begin{equation}
   \gamma_1h_2(1-h_1) +\gamma_2h_1(1-h_2)+2h_1h_2 = 0
\end{equation}
\\
For the negative consensus we have 
\begin{align}
    & T = -\gamma_1(1+h_1)-\gamma_2(1+h_2)+2(h_1+h_2)\\
    & D = 2\big[ -\gamma_1h_2(1+h_1) -\gamma_2h_1(1+h_2)+2h_1h_2  \big]
\end{align}
and applying the same arguments of the positive consensus we can claim that the stability condition is determined only by the condition on the determinant $D>0$, thus the corresponding critical curve reads
\begin{equation}
    -\gamma_1h_2(1+h_1) -\gamma_2h_1(1+h_2)+2h_1h_2 = 0
\end{equation}










\section{Mean-field transition rates for the modular network}\label{app transition rates}
As in the fully connected case, each of the global mean-field transition rates (\ref{SBM transition rates}) for the modular network is the product of three factors: the probability to randomly select an agent of class $i$ and current state $\sigma$, the probability of selecting one neighbour of such agent type currently in the opposite state $-\sigma$, and the probability of transition (imitation). With respect to the fully connected case, the first and the third factor are obviously unchanged, and in the case of $R_{+1}$ correspond respectively to $\alpha-\rho_1$ and $\frac{1+h_1}{2}$. To determine the second factor, we have to take carefully into account the modular structure and distinguish the two classes. For $R_{+1}$, once selected a spin of of the first class, the probability of randomly selecting a neighbouring agent of the first class is $\frac{\alpha p_{11}}{\alpha p_{11} + (1-\alpha)p_{12}}$, multiplied by the probability that such neighbour is in the up state $\frac{\rho_1}{\alpha}$. Analogously, the probability of randomly selecting a neighbour of the second class is $\frac{(1-\alpha)p_{12}}{\alpha p_{11} + (1-\alpha)p_{12}}$, multiplied by the probability that such neighbour is in the up state $\frac{\rho_2}{1-\alpha}$. The result is the factor  $\frac{p_{11}\rho_1}{\alpha p_{11} + (1-\alpha)p_{12}}+\frac{p_{12}\rho_2}{\alpha p_{11} + (1-\alpha)p_{12}}$. Analogous considerations apply for the other transition rates $R_{1-},R_{2+}$ and $R_{2-}$. 




\section{Derivation of the Pair Approximation system of ODEs}\label{app_PA}
For the sake of simplicity, we apply the pair approximation on a z-regular undirected modular graph with two communities, but the treatment can be easily extended to a directed modular network with heterogeneous degrees.\\
Each node of class $1$ has $z_{11}$ internal neighbours and $z_{12}$ external (of different class) ones. Same thing for class 2. Moreover, $N_1z_{12} = N_2 z_{21}$. The following quantities are defined: the density of up spins in the first and second class, respectively $\rho'_1$ and $\rho'_2,$ as in the paragraph before, while $b_{lm}^{\sigma_i\;\sigma_j}$ represents the fraction of active\footnote{An edge linking the nodes $ij$ is active at a time $t$ if $\sigma_i(t)\sigma_j(t)=-1$} links connecting a spin in state $\sigma_i$ spin of class $l$ with a spin in state $\sigma_j$ spin of class $m$, normalized by the total number of connections of the class $l$ towards the class $m$,
\begin{align}
    &\rho'_1 = \frac{1}{N_1}\sum_{i=1}^{N_1} \frac{1+\sigma_i}{2} \\
    &\rho'_2 = \frac{1}{N_2}\sum_{i=N_1+1}^{N} \frac{1+\sigma_i}{2} \\
    &b_{11}^{+-} = \frac{1}{N_1z_{11}}\sum_{i=1}^{N_1}\;\sum_{j=1}^{N_1}A_{ij}\frac{1-\sigma_i+\sigma_j-\sigma_i\sigma_j}{4}  \\
    &b_{22}^{+-} = \frac{1}{N_2z_{22}}\sum_{i=N_1+1}^{N}\;\sum_{j=N_1+1}^{N}A_{ij}\frac{1-\sigma_i+\sigma_j-\sigma_i\sigma_j}{4}  \\
    &b_{12}^{-+} = \frac{1}{N_1z_{12}}\sum_{i=1}^{N_1}\sum_{j=N_1+1}^{N}A_{ij}\frac{1-\sigma_i+\sigma_j-\sigma_i\sigma_j}{4}  \\
    &b_{12}^{+-} = \frac{1}{N_1z_{12}}\sum_{i=1}^{N_1}\sum_{j=N_1+1}^{N}A_{ij}\frac{1+\sigma_i-\sigma_j-\sigma_i\sigma_j}{4}
\end{align}
Due to the undirectedness of the network, the other quantities of interest can be expressed as functions of the ones defined above, specifically $b_{11}^{-+} = b_{11}^{+-}$, $b_{22}^{-+} = b_{22}^{+-}$, $b_{21}^{+-} = b_{12}^{-+}$, $b_{21}^{-+} = b_{12}^{+-}$.\\
\\
The first of the global rates of the possible processes $W^{1+}_{z_{11},m_{11},z_{12},m_{12}}$, i.e. the probability that in a time $dt$ (recall $dt=\frac{1}{N}d\tau$) a spin of the first class in current state $-$ and with $m_{11}$ out of $z_{11}$ internal neighbours and $m_{12}$ out of $z_{12}$ external ones currently in $+$ state flips, reads
\begin{align}
    &W^{1-\rightarrow+}_{z_{11},m_{11},z_{12},m_{12}} = N_1 (1-\rho'_1)P^{11-}_{z_{11},m_{11}}P^{12-}_{z_{12},m_{12}}F^1_{z_{11}+z_{12},m_{11}+m_{12}}
\end{align}
where $P^{11-}_{z_{11},m_{11}}$ is the probability that a node in the first population currently in state $-1$ has $z_{11}$  degree (trivial, explicited only for the generalization) and $m_{11}$ neighbours of the first population currently in state $+1$. The other rates read similarly
\begin{align}
    &W^{2-\rightarrow+}_{z_{22},m_{22},z_{21},m_{21}} = N_2 (1-\rho'_2)P^{22-}_{z_{22},m_{22}}P^{21-}_{z_{21},m_{21}}F^2_{z_{22}+z_{21},m_{22}+m_{21}}\\
    &W^{1+\rightarrow-}_{z_{11},m_{11},z_{12},m_{12}} = N_1 \rho'_1P^{11+}_{z_{11},m_{11}}P^{12+}_{z_{12},m_{12}}R^1_{z_{11}+z_{12},m_{11}+m_{12}}\\
    &W^{2+\rightarrow-}_{z_{22},m_{22},z_{21},m_{21}} = N_2 \rho'_2P^{22+}_{z_{22},m_{22}}P^{21+}_{z_{21},m_{21}}R^2_{z_{22}+z_{21},m_{22}+m_{21}}
\end{align}
The transition rates (infection $-1\rightarrow +1$ and recovery $+1\rightarrow -1$) in the expressions of the global rates, for this specific multi-class binary-state stochastic process, are 
\begin{align}
    &F^{1/2}_{z,m} = \frac{m}{z}\frac{1+h_{1/2}}{2}\\
    &R^{1/2}_{z,m} = \bigg(1-\frac{m}{z}\bigg)\frac{1-h_{1/2}}{2}
\end{align}
The two global rates of the first class $W^{1\mp\rightarrow\pm}_{z_{11},m_{11},z_{12},m_{12}}$ change the variables respectively
\begin{align}
    &\rho'_1 \rightarrow \rho'_1 \pm \frac{1}{N_1}\\
    &b_{11}^{+-} \rightarrow b_{11} \pm \frac{(z_{11}-2m_{11})}{N_1z_{11}}  \\
    &b_{12}^{-+} \rightarrow  b_{12}^{-+} \mp \frac{m_{12}}{N_1z_{12}} \\
    &b_{12}^{+-} \rightarrow  b_{12}^{+-} \pm \frac{z_{12}-m_{12}}{N_1z_{12}}
\end{align}
and correspondingly the ones the second class rates $W^{2\mp\rightarrow\pm}_{z_{22},m_{22},z_{21},m_{21}}$ change the variables 
\begin{align}
    &\rho'_2 \rightarrow \rho'_2 \pm \frac{1}{N_2}\\
    &b_{22}^{+-} \rightarrow b_{22} \pm \frac{(z_{22}-2m_{22})}{N_2z_{22}}  \\
    &b_{12}^{-+} \rightarrow  b_{12}^{-+} \pm  \frac{z_{21}-m_{21}}{N_2z_{21}}\\
    &b_{12}^{+-} \rightarrow  b_{12}^{+-} \mp  \frac{m_{21}}{N_2z_{21}} 
\end{align}
\\
Thus, the dynamical system consists of six coupled evolution equations 
\begin{equation}
    \begin{cases}
    \dot{\rho'_1} = \frac{1}{N_1} \sum\limits_{m_{11}=0}^{z_{11}}\sum\limits_{m_{12}=0}^{z_{12}}[W^{1+}_{z_{11},m_{11},z_{12},m_{12}} - W^{1-}_{z_{11},m_{11},z_{12},m_{12}}]\\
    \dot{\rho'_2} = \frac{1}{N_2} \sum\limits_{m_{22}=0}^{z_{22}}\sum\limits_{m_{21}=0}^{z_{21}}[W^{2+}_{z_{22},m_{22},z_{21},m_{21}} - W^{2-}_{z_{22},m_{22},z_{21},m_{21}}]\\
    \dot{b_{11}^{+-}} = \frac{1}{N_1z_{11}} \sum\limits_{m_{11}=0}^{z_{11}}\sum\limits_{m_{12}=0}^{z_{12}}(z_{11} - 2m_{11})[W^{1+}_{z_{11},m_{11},z_{12},m_{12}} - W^{1-}_{z_{11},m_{11},z_{12},m_{12}}]\\
    \dot{b_{22}^{+-}} = \frac{1}{N_2z_{22}} \sum\limits_{m_{22}=0}^{z_{22}}\sum\limits_{m_{21}=0}^{z_{21}}(z_{22} - 2m_{22})[W^{2+}_{z_{22},m_{22},z_{21},m_{21}} - W^{2-}_{z_{22},m_{22},z_{21},m_{21}}]\\
    \dot{b_{12}^{-+}} = \frac{1}{N_1z_{12}} \sum\limits_{m_{11}=0}^{z_{11}}\sum\limits_{m_{12}=0}^{z_{12}}(-m_{12})[W^{1+}_{z_{11},m_{11},z_{12},m_{12}} - W^{1-}_{z_{11},m_{11},z_{12},m_{12}}]  \;+\\ 
    \;\;\;\;\;\;\;\;\;\;\;\;\;+\frac{1}{N_2z_{21}} \sum\limits_{m_{22}=0}^{z_{22}}\sum\limits_{m_{21}=0}^{z_{21}}(z_{21} - m_{21})[W^{2+}_{z_{22},m_{22},z_{21},m_{21}} - W^{2-}_{z_{22},m_{22},z_{21},m_{21}}]\\
    \dot{b_{12}^{+-}} = \frac{1}{N_1z_{12}} \sum\limits_{m_{11}=0}^{z_{11}}\sum\limits_{m_{12}=0}^{z_{12}}(z_{12}-m_{12})[W^{1+}_{z_{11},m_{11},z_{12},m_{12}} - W^{1-}_{z_{11},m_{11},z_{12},m_{12}}]  \;+\\ 
    \;\;\;\;\;\;\;\;\;\;\;\;\;+\frac{1}{N_2z_{21}} \sum\limits_{m_{22}=0}^{z_{22}}\sum\limits_{m_{21}=0}^{z_{21}}( - m_{21})[W^{2+}_{z_{22},m_{22},z_{21},m_{21}} - W^{2-}_{z_{22},m_{22},z_{21},m_{21}}]\\    
    
    \end{cases}
\end{equation}
We still have to defined the probabilities e.g.  $P^{11-}_{z_{11},m_{11}}$, whose definition would close the system of ordinary differential equations above.\\
\\
The mean-field approximation corresponds to take the probabilities 
\begin{align}
    &P^{11+}_{z_{11},m_{11}} = P^{11-}_{z_{11},m_{11}} =  \delta_{\rho_1,\frac{m_{11}}{z_{11}}} \label{MF_probabilities_start} \\
    &P^{22+}_{z_{22},m_{22}} = P^{22-}_{z_{22},m_{22}} =  \delta_{\rho_2,\frac{m_{22}}{z_{22}}}\\
    &P^{12+}_{z_{12},m_{12}} = P^{12-}_{z_{12},m_{12}} = \delta_{\rho_2,\frac{m_{12}}{z_{12}}} \\
    &P^{21+}_{z_{21},m_{21}} = P^{21-}_{z_{21},m_{21}} = \delta_{\rho_1,\frac{m_{21}}{z_{21}}}
\label{MF_probabilities_end}
\end{align}
and thus decouples the evolution of the densities of up spin from the rest of the system, indeed by inserting those probabilities we would recover the mean-field evolution equations (\ref{mf_echochambers}) for $\rho'_1,\rho'_2$ .\\
\\
The pair approximation, instead, consists in assuming that those probability are binomial distributions $B_{z,m}(x) = \binom{z}{m}x^m(1-x)^{z-m}$ 
\begin{align} 
    &P^{11-}_{z_{11},m_{11}} = B_{z_{11},m_{11}}(p_{11-}) \label{eq:PA_proba_start}\\
    &P^{11+}_{z_{11},m_{11}} = B_{z_{11},z_{11}-m_{11}}(p_{11+}) \\
    &P^{22-}_{z_{22},m_{22}} = B_{z_{22},m_{22}}(p_{22-}) \\
    &P^{22+}_{z_{22},m_{22}} = B_{z_{22},z_{22}-m_{22}}(p_{22+}) \\
    &P^{12-}_{z_{12},m_{12}} = B_{z_{12},m_{12}}(p_{12-}) \\
    &P^{12+}_{z_{12},m_{12}} = B_{z_{12},z_{12}-m_{12}}(p_{12+}) \\
    &P^{21-}_{z_{21},m_{21}} = B_{z_{21},m_{21}}(p_{21-}) \\
    &P^{21+}_{z_{21},m_{21}} = B_{z_{21},z_{21}-m_{21}}(p_{21+}) \label{PA_proba_end}
\end{align}
with single events probabilities
\begin{align}
    &p_{11-} = \frac{b_{11}^{-+}}{1-\rho'_1} = \frac{b_{11}^{+-}}{1-\rho'_1}\label{PA_single_proba_start}\\
    &p_{11+} = \frac{b_{11}^{+-}}{\rho'_1} \\
    &p_{22-} = \frac{b_{22}^{-+}}{1-\rho'_2} = \frac{b_{22}^{+-}}{1-\rho'_2} \\
    &p_{22+} = \frac{b_{22}^{+-}}{\rho'_2} \\
    &p_{12-} = \frac{b_{12}^{-+}}{1-\rho'_1} \\
    &p_{12+} = \frac{b_{12}^{+-}}{\rho'_1} \\
    &p_{21-} = \frac{b_{21}^{+-}}{1-\rho'_2} = \frac{b_{12}^{-+}}{1-\rho'_2}  \\
    &p_{21+} =  \frac{b_{21}^{+-}}{\rho'_2} = \frac{b_{12}^{-+}}{\rho'_2}\label{PA_single_proba_end}   
\end{align}
The criterium is to consider as single probability, e.g. to express $p_{11-}$, the fraction of $-+$ links of the first communities (in number $b_{11}^{-+}N_1z_{11}$) over the total of the edges starting from $-$ within the first community (in number $(1-\rho'_1)N_1z_{11}$).\\
\\
Eventually we can write the final system of ODE within the pair approximation

\begin{equation}
    \begin{cases}
    \dot{\rho'_1} =   (1-\rho'_1)\sum\limits_{m_{11}=0}^{z_{11}}B_{z_{11},m_{11}}(p_{11-})\sum\limits_{m_{12}=0}^{z_{12}}B_{z_{12},m_{12}}(p_{12-})F^1_{z_{11}+z_{12},m_{11}+m_{12}} +\\
    \;\;\;\;\;\;\;\;\; -  \rho'_1 \sum\limits_{m_{11}=0}^{z_{11}}B_{z_{11},z_{11}-m_{11}}(p_{11+}) \sum\limits_{m_{12}=0}^{z_{12}}B_{z_{12},z_{12}-m_{12}}(p_{12+})R^1_{z_{11}+z_{12},m_{11}+m_{12}}\\
    \\
    
    \dot{\rho'_2} = (1-\rho'_2)\sum\limits_{m_{22}=0}^{z_{22}}B_{z_{22},m_{22}}(p_{22-}) \sum\limits_{m_{21}=0}^{z_{21}}B_{z_{21},m_{21}}(p_{21-})F^2_{z_{22}+z_{21},m_{22}+m_{21}} +\\
    \;\;\;\;\;\;\;\;\;- \rho'_2\sum\limits_{m_{22}=0}^{z_{22}}B_{z_{22},z_{22}-m_{22}}(p_{22+})\sum\limits_{m_{21}=0}^{z_{21}}B_{z_{21},z_{21}-m_{21}}(p_{21+})R^2_{z_{22}+z_{21},m_{22}+m_{21}}\\
    \\
    
    \dot{b_{11}^{+-}} = \frac{1-\rho'_1}{z_{11}} \sum\limits_{m_{11}=0}^{z_{11}}\sum\limits_{m_{12}=0}^{z_{12}}(z_{11} - 2m_{11})B_{z_{11},m_{11}}(p_{11-})B_{z_{12},m_{12}}(p_{12-})F^1_{z_{11}+z_{12},m_{11}+m_{12}} +\\
    \;\;\;\;\;\;\;\;\;\;-\frac{\rho'_1}{z_{11}} \sum\limits_{m_{11}=0}^{z_{11}}\sum\limits_{m_{12}=0}^{z_{12}}(z_{11} - 2m_{11})B_{z_{11},z_{11}-m_{11}}(p_{11+}) B_{z_{12},z_{12}-m_{12}}(p_{12+})R^1_{z_{11}+z_{12},m_{11}+m_{12}} \\
    \\
    
    \dot{b_{22}^{+-}} = \frac{1-\rho'_2}{z_{22}} \sum\limits_{m_{22}=0}^{z_{22}}\sum\limits_{m_{21}=0}^{z_{21}}(z_{22} - 2m_{22})B_{z_{22},m_{22}}(p_{22-}) B_{z_{21},m_{21}}(p_{21-})F^2_{z_{22}+z_{21},m_{22}+m_{21}} +\\ \;\;\;\;\;\;\;\;\;-\frac{\rho'_2}{z_{22}}\sum\limits_{m_{22}=0}^{z_{22}}\sum\limits_{m_{21}=0}^{z_{21}}(z_{22} - 2m_{22})B_{z_{22},z_{22}-m_{22}}(p_{22+})B_{z_{21},z_{21}-m_{21}}(p_{21+})R^2_{z_{22}+z_{21},m_{22}+m_{21}}\\
    \\
    
    \dot{b_{12}^{-+}} = \frac{1-\rho'_1}{z_{12}} \sum\limits_{m_{11}=0}^{z_{11}}\sum\limits_{m_{12}=0}^{z_{12}}(-m_{12})B_{z_{11},m_{11}}(p_{11-})B_{z_{12},m_{12}}(p_{12-})F^1_{z_{11}+z_{12},m_{11}+m_{12}} + \\ \;\;\;\;\;\;\;\;\;\;\;-\frac{\rho'_1}{z_{12}}\sum\limits_{m_{11}=0}^{z_{11}}\sum\limits_{m_{12}=0}^{z_{12}}(-m_{12}) B_{z_{11},z_{11}-m_{11}}(p_{11+}) B_{z_{12},z_{12}-m_{12}}(p_{12+})R^1_{z_{11}+z_{12},m_{11}+m_{12}}  \;+\\ 
    \;\;\;\;\;\;\;\;\;\;\;\;\;+\frac{1-\rho'_2}{z_{21}} \sum\limits_{m_{22}=0}^{z_{22}}\sum\limits_{m_{21}=0}^{z_{21}}(z_{21} - m_{21})B_{z_{22},m_{22}}(p_{22-}) B_{z_{21},m_{21}}(p_{21-})F^2_{z_{22}+z_{21},m_{22}+m_{21}}\;+\\
    \;\;\;\;\;\;\;\;\;\;\;\;- \frac{\rho'_2}{z_{21}}\sum\limits_{m_{22}=0}^{z_{22}}\sum\limits_{m_{21}=0}^{z_{21}}(z_{21} - m_{21})B_{z_{22},z_{22}-m_{22}}(p_{22+})B_{z_{21},z_{21}-m_{21}}(p_{21+})R^2_{z_{22}+z_{21},m_{22}+m_{21}}\\
    \\
    
    \dot{b_{12}^{+-}} = \frac{1-\rho'_1}{z_{12}} \sum\limits_{m_{11}=0}^{z_{11}}\sum\limits_{m_{12}=0}^{z_{12}}(z_{12}-m_{12})B_{z_{11},m_{11}}(p_{11-})B_{z_{12},m_{12}}(p_{12-})F^1_{z_{11}+z_{12},m_{11}+m_{12}} +\\
    \;\;\;\;\;\;\;\;\;\;\;\;-  \frac{\rho'_1}{z_{12}}\sum\limits_{m_{11}=0}^{z_{11}}\sum\limits_{m_{12}=0}^{z_{12}}(z_{12}-m_{12}) B_{z_{11},z_{11}-m_{11}}(p_{11+}) B_{z_{12},z_{12}-m_{12}}(p_{12+})R^1_{z_{11}+z_{12},m_{11}+m_{12}}   \;+\\ 
    \;\;\;\;\;\;\;\;\;\;\;\;\;+\frac{1-\rho'_2}{z_{21}} \sum\limits_{m_{22}=0}^{z_{22}}\sum\limits_{m_{21}=0}^{z_{21}}( - m_{21})B_{z_{22},m_{22}}(p_{22-}) B_{z_{21},m_{21}}(p_{21-})F^2_{z_{22}+z_{21},m_{22}+m_{21}} +\\
    \;\;\;\;\;\;\;\;\;\;\;\;- \frac{\rho'_2}{z_{21}}\sum\limits_{m_{22}=0}^{z_{22}}\sum\limits_{m_{21}=0}^{z_{21}}( - m_{21})B_{z_{22},z_{22}-m_{22}}(p_{22+})B_{z_{21},z_{21}-m_{21}}(p_{21+})R^2_{z_{22}+z_{21},m_{22}+m_{21}}\\    
    \end{cases}
    \label{PA_system}
\end{equation}
to be solve numerically with standard methods.\\
From the initial conditions $\rho'_1(0),\rho'_2(0)$, the other initial conditions are determined as follows
\begin{align}
    & b_{11}^{+-}(0) = \rho'_1(0)(1-\rho'_1(0))\\
    & b_{22}^{+-}(0) = \rho'_2(0)(1-\rho'_2(0))\\
    & b_{12}^{-+}(0) = \rho'_1(0)(1-\rho'_2(0))\\
    & b_{12}^{+-}(0) = \rho'_2(0)(1-\rho'_1(0))
\end{align}






\end{document}
