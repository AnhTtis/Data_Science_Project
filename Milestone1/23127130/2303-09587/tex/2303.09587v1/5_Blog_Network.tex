\section{Application to the network of blogs}\label{blognetwork_chapter}
We run the bipopulated VMP on a real social network with a moudular structure, to study how well the mean-field approximation performs on a real network with high modularity and potentially structural features as well as dynamical correlations. We take the network of Political Blogs during 2004 American elections \cite{adamic2005political}, characterized by the presence of two communities that reflect political bi-partisanships. After eliminating the nodes with degree less than $4$, and transforming for simplicity the original directed network in undirected (such that $p_{12}=p_{21}$), we apply a community detection algorithm \cite{kernighan1970efficient}. It turns out that the two communities, named "reds" and "blues" (shown in the upper plot of figure \ref{fig:blog net}), have sizes respectively $N_r = 413 $, $N_b= 490 $, average internal degrees $z_{rr} = 34.02 $, $z_{bb} = 32.38 $ and external degrees $z_{rb} = 2.88$, $z_{br} = 2.43$. We perform numerical simulations of the model choosing, for simplicity, opposite and equally strong preferences ($h_1 = -h_2 = h$), for multiple values of $h$, and compare them to the mean field predictions on a SBM with the same average degrees. The results are reported in figure \ref{fig:MF and blog net}: the mean-field predictions agree well with the empirical simulations, with a small gap emerging for weak preference intensities. In the lower plot of figure \ref{fig:blog net} it is reported the average polarization of each node in the asymptotic state ($t=50$) over repeated runs of the model with equal and opposite biases fixed to $h=0.3$: as one could expect, the most open-minded nodes adopt their unpreferred opinion more frequently than the ones in the rest of their community. Despite this effect due to the heterogeneity of the degrees and specifically of the local open-mindednesses, the mean-field predictions remain quite accurate.


\begin{figure}
\centering
\subfloat{\includegraphics[width = 0.8\textwidth]{Figures/commdet.png}}\\ 
\subfloat{\includegraphics[width = 0.8\textwidth]{Figures/blogmagn.png}}
\caption{\textbf{VMP on the 2004 Blogosphere.} \textit{Upper plot}: 2004 Political Blogosphere after community detection: nodes' colors reflect political partisanships. \textit{Lower plot}: each node is colored according to the average state $\bar{\rho_i}$ at time $t=50$, for a bipopulated VMP where the populations are the ones computed by community detection and the preferences are set to $h_r = -h_b = 0.2$. The colorscale goes from blue, corresponding to $\bar{\rho_i}=0$, to red ($\bar{\rho_i}=1$). The averages are computed running $100$ independent simulations.}
\label{fig:blog net}
\end{figure}



\begin{figure}
\centering
\includegraphics[width = 0.7\textwidth]{Figures/blog_polarization/blogpola_avgstd.png}
\caption{\textbf{Mean-field predictions and simulations of the model on the 2004 Blogosphere.} We compare the mean-field predictions and the numerical simulations of the asymptotic polarization on the 2004 Political Blogosphere, restricting to equally strong biases $h_r = -h_b = h$, for various $h$.}
\label{fig:MF and blog net}
\end{figure}
