\begin{abstract}
Two of the main factors shaping an individual's opinion are social pressure and personal preference.
% , that may drive a system of interacting individuals towards the convergence to a polarized state.
% When these two happen to be in contrast, people can either behave in dissonance with their personal believes in order to favour social consensus or, when their bias is dominant, persist in their preferred opinion and accept conflict more easily. 
We study an extension of the voter model proposed by Masuda and Redner (2011), where the interacting agents are divided into two populations with opposite external biases of different intensities. We analyze the role of the underlying social network, considering a modular graph with two communities that reflect the bias assignment, modeling the phenomenon of epistemic bubbles (Nguyen, 2020). Depending on the network and the biases' strengths, the system can either reach a consensus (common agreement, zero polarization) or a polarized state, in which the two populations stabilize to different average opinions. It turns out that generally the modular structure has the effect of increasing both the value of the polarization and the size of the polarization region in the space of parameters. We also show that, when the discrepancy in bias between the populations is high, the success of the very committed one in imposing its preferred opinion depends largely on the level of segregation of the other population, while the dependency on its own topological structure is negligible. We compare the mean-field dynamics with the one deriving from the Pair approximation and test the goodness of the mean-field predictions on a real network.
\end{abstract}