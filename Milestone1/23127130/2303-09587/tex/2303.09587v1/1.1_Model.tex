\section{Voter Model with Preferences} \label{model}
The VMP, i.e. the generalization of the PVM of Masuda et al. \cite{masuda2010heterogeneous, masuda2011can}, is defined as follows: the system of $N$ agents is divided into two populations, or \textit{classes}, of sizes $N_1$ and $N_2 = N - N_1$, the agents $i = 1,...,N_1$ belonging to the first and the remaining $i = N_1+1,...,N$ to the second one, with $\alpha = \frac{N_1}{N}\in(0,1)$ the fraction of individuals of the first population. A bias $h_i\in[-1,1]$ is assigned to each agent $i$, according to his class: in our bipopulated case, we assign the same $h_1$ to the individuals of the first population, similarly $h_2$ to all the individuals belonging to the second one. Each node's opinion is represented as a binary spin $\sigma_i=\{+1,-1\}$ for $i=1,...,N$. The dynamics obeys the following rules:
\begin{itemize}
    \item One node (agent) $i$ is 
    selected uniformly randomly.
    \item A neighbor $j$ of node $i$ is selected uniformly randomly.
    \item If $i$ and $j$ have opposite opinions, $i$ takes the opinion of $j$ with probability $\frac{1}{2}(1+\sigma_jh_i)$. Otherwise, nothing happens.
    \item Repeat the process until consensus or apparent stabilization is reached.
\end{itemize}
The dynamics is a generalization of the classical voter model - retrieved for $h_i=0,\;\forall i$ - where the individual copies his neighbour's opinion with a probability equal to $\frac{1}{2}$ (in the original voter model this probability is 1, but the factor $\frac{1}{2}$ just slows down the dynamics). The biases modify the transition probabilities, favoring the transition towards the direction of the bias and disfavoring the opposite one. It is easy to show that if in the bipopulated case both of the biases point towards the same direction, then an infinite system will always reach consensus at the preferred state. Thus, in the following we will consider $h_1\geq0$ and $h_2\leq0$, so that the individuals of the first population tend to prefer the $+1$ state, while the ones of the second class are ideologically biased towards the $-1$ opinion.



% The biases modify the transition probabilities, breaking the symmetry between the opinions: they favour the transition towards the direction of the bias and disincentivate the opposite ones. For example, let's consider a positively biased node, $h_i>0$. When it is sorted, it is in state $-1$ and one of his neighbours currently in state $+1$ is sorted, he will perform the transition $-1\rightarrow+1$ with probability $\frac{1}{2}(1+h_i)$, greater than $\frac{1}{2}$. Conversely, when it is in state $+1$ and one of his neighbours currently in state $-1$ is sorted, he will perform the transition $+1\rightarrow-1$ with probability $\frac{1}{2}(1-h_i)$, lower than $\frac{1}{2}$.\\
% It is easy to show that if in the bipopulated case both of the biases point towards the same direction, then an infinite system will always reach consensus at the preferred state\footnote{However, in this case it is still interesting to determine some quantities related to finite systems, such as the exit probability (probability to reach the positive consensus as function of the current state) and the time for consensus.}. Thus, in the following we consider $h_1\geq0$ and $h_2\leq0$; in other words, the individuals of the first population tend to prefer the $+1$ state, while the ones of the second class are ideologically biased towards the $-1$ opinion. Indeed, we will see that for $h_1h_2\leq0$ the phenomenology of the model is way more rich.