\section{Pair Approximation}\label{PA_chapter}
The aim of this section is to investigate how a more refined approximation better reproduces the model's behaviour on complex networks. We implement the so called Pair Approximation \cite{gleeson2011high,gleeson2013binary}, which takes into consideration dynamical correlations at a pairwise level. \\
For the sake of simplicity, the Pair Approximation is applied on an undirected regular\footnote{regular in the sense that each node of the class has the same number of internal $z_{ii}$ and external $z_{ij}$ connections.} modular graphs with two communities, nevertheless the treatment can be easily extended to a directed modular network with heterogeneous degrees \cite{pugliese2009heterogeneous}. Generalizing the approach followed by \cite{peralta2021effect} to class-dependent infection/recovery rates, reported in the appendix \ref{app_PA}, we consider the set of probabilities of the kind $P^{ij\pm}_{z_{ij},m_{ij}}$, i.e. the probability that a node in the population $i$ currently in state $\pm1$ has $z_{ij}$  degree\footnote{trivial in the $z$-regular case, explicited only for the generalization} and $m_{ij}$ neighbours of the population $j$ currently in state $+1$.\\
The mean-field approximation consists in taking those probabilities highly peaked at the values of $\frac{m_{ij}}{z_{ij}}$ coinciding to the overall normalized densities $\rho'_1,\rho'_2$, so having the shape of delta functions explicited in (\ref{MF_probabilities_start}-\ref{MF_probabilities_end}). The Pair Approximation, instead, considers pairwise dynamical correlations by taking those probabilities as binomial distributions with single event probabilitiy corresponding to the ratio between the number of active links departing from the node's type, intended as its class and state, and the number of connections from that node's type. The argument is explained in the appendix (\ref{eq:PA_proba_start}-\ref{PA_proba_end},\ref{PA_single_proba_start}-\ref{PA_single_proba_end}).\\
The latter ratios are also dynamical variables of the system derived from the pair approximation (\ref{PA_system}), which consists indeed of six coupled differential equations. The increase in complexity is justified by a gain in accuracy, consistent for low average degrees, i.e. for sparse graphs. In figure \ref{fig:PA and MF} we compare the mean-field and Pair approximation effectiveness in reproducing the dynamics of the system, on an extremely sparse modular network. We see that the gain in accuracy is consistent, and the PA is able to predict the dynamics almost perfectly. On the right plot of the figure, we test multiple initial conditions in order to check that, as in the mean-field treatment, the stable fixed point is unique. 

\begin{figure}
\subfloat{\includegraphics[width = 0.5\textwidth]{Figures/mfANDpa.png}} 
\subfloat{\includegraphics[width = 0.5\textwidth]{Figures/multiple_initial_conditions.png}}\\
\caption{\textbf{Pair and Mean-Field approximations.} \textit{Left plot}: Dynamics of the MF (dotted line) and PA (dashed lines) approximations compared with numerical simulations (thin solid lines) averaged over $30$ independent runs. The graph is a sparse modular network of $N=3000$ nodes with two $z-$regular communities $z_{11}=4,z_{12}=2,z_{21}=1,z_{22}=3$, thus $\alpha = \frac{2}{3}$ and $\gamma_1=0.4 ,\gamma_2=0.25$. The two communities' preferences are respectively $h_1=0.3,h_2=-0.5$. The initial opinions are chosen uniformly random ($\rho_1'(0) = \rho_2'(0) = \frac{1}{2}$). \textit{Right plot}: Same setting repeated for various initial conditions $\rho'_1(0),\rho'_2(0)$. The results show that even for low connectivities the stable fixed point is unique.}
\label{fig:PA and MF}
\end{figure}

