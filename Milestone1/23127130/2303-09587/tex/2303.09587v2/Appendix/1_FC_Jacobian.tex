\section{Linear stability analysis}
\subsection{Fully connected network}
Here we perform the linear stability analysis of the dynamical system (\ref{fc mf rho12}) derived for the VMP on the fully connected topology, giving a proof to the considerations in chapter \ref{FC_chapter}. As said in the main text, the consensus points $(0,0)$ and $(\alpha,1-\alpha)$ are fixed points of the system for whatever choice of the parameters $\alpha, h_1,h_2$, while it is easy to prove that only for $\alpha\in(\alpha_c^{-},\alpha_c^{+})$ the fixed point corresponding to the \textit{polarized} state (\ref{pol_rho},\ref{pol_delta}) is in the rectangle $(0,\alpha)\times(0,1-\alpha)$, and thus has a physical meaning. In the following, we prove that for conflicting preferences, i.e. $h_1\geq0, h_2\leq0$, for $\alpha<\alpha_c^{-}$ the negative consensus fixed point is the only stable fixed point of the system, for $\alpha\in(\alpha_c^{-},\alpha_c^{+})$ both the consensus points are unstable and the polarized fixed point is stable, while for $\alpha>\alpha_c^{-}$ the positive consensus is the only stable fixed point.. \\
\\
The Jacobian of the dynamical system reads
\begin{equation}
    J(\rho_1,\rho_2;\alpha,h_1,h_2) = \frac{1}{2}
    \begin{pmatrix}
    \alpha-1+h_1(1+\alpha-4\rho_1-2\rho_2) & \alpha+h_1(\alpha-2\rho_1)  \\
    1-\alpha +h_2(1-\alpha-2\rho_2) & -\alpha +h_2(2-\alpha-4\rho_2-2\rho_1) \\
    \end{pmatrix}
\end{equation}
According to the linear stability theory, a fixed point is stable if both the eigenvalues of the corresponding Jacobian are negative, i.e. if the trace $T$ is negative and the determinant $D$ is positive.\\
\\
For the negative consensus fixed point we have that 
\begin{align}
    & T = \frac{1}{2}\bigg[\alpha(h_1-h_2)+2h_2+h_1-1\bigg] \\
    &D = \frac{1}{2}\bigg[h_1h_2-h_2-\alpha(h_1-h_2)\bigg] 
\end{align}
so the determinant is positive for $\alpha<\alpha^D=\frac{-h_2(1-h_1)}{h1-h_2}$ , while the trace is negative for $\alpha<\alpha^T=\frac{1-2h_2-h_1}{h_1-h_2}$. Thus the stability condition is satisfied for $\alpha<\mbox{min}(\alpha^T,\alpha^D)$. It is easy to see that $\alpha^D\leq\alpha^T$, since for the assumptions on the sign of the preferences it holds that $h_1h_2 + h_1 +h_2 \leq1$, so the negative consensus point is a stable attractive fixed point if and only if
$\alpha<\alpha^D = \alpha_c^{-}$.\\
\\
Analogous considerations apply to the positive consensus fixed points, whose determinant and trace read
\begin{align}
    & T = \frac{1}{2}\bigg[-\alpha(h_1-h_2)-h_1-2h_2-1\bigg] \\
    & D = \frac{1}{2}\bigg[\alpha(h_1-h_2) + h_1h_2 + h_2 \bigg] 
\end{align}
and by applying the same arguments as before we get that the stability condition is fulfilled when $\alpha>\mbox{max}(\alpha^D,\alpha^T) = \alpha^D = \alpha_c^{+}$.\\
\\
Last, the same considerations about the trace and determinant can be used to prove that in the range $[\alpha_c^{-},\alpha_c^{+}]$ the fixed point corresponding to the \textit{polarized} state is stable.













\subsection{Modular network} \label{Modular lsa}
We perform the linear stability analysis of the mean-field system (\ref{mf_echochambers}) derived from the VMP on a modular network with two communities and open-mindedness parameters $\gamma_1,\gamma_2$. The dynamical variables taken into consideration are now $\rho'_1,\rho'_2$ as defined in chapter \ref{chapter SBM}, both in the range $[0,1]$. The analysis focuses on the stability of the consensus points, now $(0,0)$ for the negative and $(1,1)$ for the positive, that are fixed points of the system for whatever choice of the parameters. \\
\\
The Jacobian matrix of the system (\ref{mf_echochambers}), discarding the uninfluential factor, reads
\begin{equation}
\small    J(\rho'_1,\rho'_2) = 
     \begin{pmatrix}
        \gamma_1(4h_1\rho_1-2h_1\rho_2-h_1-1)-4h_1\rho_1+2h_1 & \gamma_1[1-h_1(2\rho_1-1)]\\
        \gamma_2[1-h_2(2\rho_2-1)]  &  \gamma_2(4h_2\rho_2-2h_2\rho_1-h_2-1)-4h_2\rho_2+2h_2 \\
    \end{pmatrix}
\end{equation}
and as before we compute the traces and the determinant of the Jacobian at the fixed point in order to obtain the stability conditions. \\
\\
The traces and determinant for the positive consensus point read
\begin{align}
    & T = -\gamma_1(1-h_1)-\gamma_2(1-h_2)-2(h_1+h_2)\\
    & D = 2\big[ \gamma_1h_2(1-h_1) +\gamma_2h_1(1-h_2)+2h_1h_2 \big]
\end{align}
and we see that the trace is negative for all $h_1\geq -h_2$, while the determinant is positive for 
\begin{equation}
    \gamma_1h_2(1-h_1)+\gamma_2h_1(1-h_2) + 2h_1h_2>0
\label{det_condition}
\end{equation}
However, it never happens that $T>0$ and $D>0$ at the same time, at least in the parameters' space of interest $h_1\geq0,h_2\leq0,\gamma_{1/2}\geq0$. To prove it, we try to solve the system
\begin{equation}
\begin{cases}
    T = -\gamma_1(1-h_1)-\gamma_2(1+|h_2|)-2(h_1-|h_2|)>0\\
    D \propto -\gamma_1|h_2|(1-h_1) +\gamma_2h_1(1+|h_2|)-2h_1|h_2|>0
\end{cases}
\end{equation}
Arranging the terms we are left with the series of inequalities
\begin{equation}
    -\gamma_1(1-h_1)-2(h_1-|h_2|) > \gamma_2(1+|h_2|) > \frac{1}{h_1} \big(\gamma_1|h_2|(1-h_1) +2h_1|h_2|\big)
\end{equation}
that implies 
\begin{equation}
    -\gamma_1(1-h_1)-2(h_1-|h_2|)  > \frac{1}{h_1} \big(\gamma_1|h_2|(1-h_1) +2h_1|h_2|\big)
\end{equation}
and simplifies in 
\begin{equation}
       -\gamma_1h_1(1-h_1)-2h_1^2 >  \gamma_1|h_2|(1-h_1)
\end{equation}
which is never true, since the terms on the l.h.s. are always negative and the term on the r.h.s. is positive. Thus, the whole stability region of the positive consensus fixed point is determined by the condition derived from the determinant (\ref{det_condition}), and thus delimited by the critical curve
\begin{equation}
   \gamma_1h_2(1-h_1) +\gamma_2h_1(1-h_2)+2h_1h_2 = 0
\end{equation}
\\
For the negative consensus we have 
\begin{align}
    & T = -\gamma_1(1+h_1)-\gamma_2(1+h_2)+2(h_1+h_2)\\
    & D = 2\big[ -\gamma_1h_2(1+h_1) -\gamma_2h_1(1+h_2)+2h_1h_2  \big]
\end{align}
and applying the same arguments of the positive consensus we can claim that the stability condition is determined only by the condition on the determinant $D>0$, thus the corresponding critical curve reads
\begin{equation}
    -\gamma_1h_2(1+h_1) -\gamma_2h_1(1+h_2)+2h_1h_2 = 0
\end{equation}










\section{Mean-field transition rates for the modular network}\label{app transition rates}
As in the fully connected case, each of the global mean-field transition rates (\ref{SBM transition rates}) for the modular network is the product of three factors: the probability to randomly select an agent of class $i$ and current state $\sigma$, the probability of selecting one neighbour of such agent type currently in the opposite state $-\sigma$, and the probability of transition (imitation). With respect to the fully connected case, the first and the third factor are obviously unchanged, and in the case of $R_{+1}$ correspond respectively to $\alpha-\rho_1$ and $\frac{1+h_1}{2}$. To determine the second factor, we have to take carefully into account the modular structure and distinguish the two classes. For $R_{+1}$, once selected a spin of of the first class, the probability of randomly selecting a neighbouring agent of the first class is $\frac{\alpha p_{11}}{\alpha p_{11} + (1-\alpha)p_{12}}$, multiplied by the probability that such neighbour is in the up state $\frac{\rho_1}{\alpha}$. Analogously, the probability of randomly selecting a neighbour of the second class is $\frac{(1-\alpha)p_{12}}{\alpha p_{11} + (1-\alpha)p_{12}}$, multiplied by the probability that such neighbour is in the up state $\frac{\rho_2}{1-\alpha}$. The result is the factor  $\frac{p_{11}\rho_1}{\alpha p_{11} + (1-\alpha)p_{12}}+\frac{p_{12}\rho_2}{\alpha p_{11} + (1-\alpha)p_{12}}$. Analogous considerations apply for the other transition rates $R_{1-},R_{2+}$ and $R_{2-}$. 



