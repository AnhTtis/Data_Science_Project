\section{Annealing approximation}
We rewrite the dynamical system in terms of the total density of up spin $\rho$ and the polarization $\Delta$, defined as
\begin{align*}
&\rho = \rho_1 + \rho_2 \\
&\Delta = \frac{\rho_1}{\alpha} - \frac{\rho_2}{1-\alpha}    
\end{align*}
Substituing those in the system (\ref{fc mf rho12}), we get
\begin{equation}
    \begin{cases}
        \dot{\rho} = [\alpha h_1 +(1-\alpha)h_2]\rho(1-\rho) - \Delta\alpha(1-\alpha)(h_1-h_2)(\rho-\frac{1}{2}) \\
        \dot{\Delta} = (h_1-h_2)\rho(1-\rho) - \Delta\bigg[[(1-\alpha)h_1+\alpha h_2](\rho-\frac{1}{2}) + \frac{1}{2}\bigg]
    \end{cases}
\label{fc rhodelta}
\end{equation}
The "annealing" approximation consists in assuming that there is no distinction, on average, between the magnetizations of two classes, in other words that $\Delta(t)\approx0$ at all times. \\
Within the annealing approximation, the evolution equation is single (only on the density $\rho$) and becomes
\begin{equation}
    \dot{\rho} = [\alpha h_1 +(1-\alpha)h_2]\rho(1-\rho)
\end{equation}
defining the "effective bias"
\begin{equation}
    H = \alpha h_1 +(1-\alpha)h_2
\end{equation}
we get 
\begin{equation}
    \frac{d\rho}{dt} = H\rho(1-\rho)
\end{equation}
Indeed another way to interpret the annealed setting is to assume that the features of the classes, in this model the biases, are random variable extracted at every iteration with a probability correspondent to their density in the system, i.e. at each iteration each node has bias $h_1$ with probability $\alpha$ and $h_2$ with probability $1-\alpha$.\\
The annealing approximation is applied but uncontrolled in \cite{}. To control it, we can derive an upper bound on $\Delta(t)$, namely $\Delta_{max}$, and assuming that as long as $\Delta_{max}$ is small, the annealing approximation $\Delta(t)\simeq 0\;\forall t$ works well. Eventually, we test this assumption by comparing with numerical simulations. Notice that since by hypothesis $h_1\geq 0$ and $h_2\leq0$, and starting from a randomic initial distribution of the opinions, $<\Delta>(t)$ is always non-negative and thus there is no need to set a lower bound.\\
We have defined $\Delta_{max}$ as the maximum value reachable by $\Delta$, i.e. the value over which $\Delta$ is systematically pushed back towords zero ($\dot{\Delta}<0$). Looking at the second equation of (\ref{fc rhodelta}), since the first term is always positive and so is the term multiplying $\Delta$, we get that
\begin{equation}
    \Delta_{max} = max_{\rho\in[0,1]} \frac{A\rho(1-\rho)}{B(\rho-\frac{1}{2})+\frac{1}{2}}
\end{equation}
with $A=h_1-h_2$ and $B=(1-\alpha)h_1+\alpha h_2$.\\
The latter maximization is satisfied by 
\begin{equation}
    \rho^* \equiv argmax_{\rho\in[0,1]}  \frac{A\rho(1-\rho)}{B(\rho-\frac{1}{2})+\frac{1}{2}} = \frac{1}{2} \bigg(1-\frac{1-\sqrt{1-B^2}}{B}\bigg)
\end{equation}
Notice that $\rho^*\in[0,1]$ is satisfied ($|B|\leq1$ always), and $\rho^*<\frac{1}{2}$ if $B>0$, viceversa $\rho^*>\frac{1}{2}$ if $B<0$, eventually $\rho^*=\frac{1}{2}$ if $B=0$.\\
Substituting back, we have that
\begin{equation}
    \Delta_{max} = (h_1-h_2) \frac{1-\sqrt{1-B^2}}{B^2}
\end{equation}
Notice that $\Delta_{max}\geq\frac{h_1-h_2}{2}$, and expanding for small $B$ we have that $\Delta_{max} = \frac{h_1-h_2}{2} +O(B^2)$.
\\
\\
The single evolution equation derived by the annealing approximation is solvable and the solution, i.e. the evolution of the density of up spins starting from an initial density $\rho_0$, reads 
\begin{equation}
    \rho(t) = \frac{1}{1+(\frac{1-\rho_0}{\rho_0})e^{-Ht}}
\end{equation}
which for uniformly random initial conditions ($\rho_0=0$) is simply
\begin{equation}
    \rho(t) = \frac{1}{1+e^{-Ht}}
\end{equation}
Indeed the annealed approximation is equivalent to the single population biased voter model with effective bias $H$, which always predicts the approach to consensus as long as $H\neq 0$.\\
\\
Within the annealing approximation, it is possible to calculate the exit probability $E(\rho)$, i.e. the probability to reach positive consensus in a finite system (an infinite system it always approaches consensus) as a function of the initial density of up spins. Following \cite{}, one writes the Backward-Kolmogorov equation for the exit probability
\begin{equation}
    E(\rho) = R^+(\rho)E(\rho+\delta\rho)+R^-(\rho)E(\rho-\delta\rho) +[1-R^+(\rho)-R^-(\rho)]E(\rho)
\end{equation}
with $\Delta\rho = \frac{1}{N}$, and boundary conditions $E(1)=1,E(0)=0$. The global transition rates are considered in the annealed setting:
\begin{align*}
    \begin{cases} 
    R^{+}(\rho) = (1-\rho)\bigg(\alpha\frac{1+h_1}{2} + (1-\alpha)\frac{1+h_2}{2}\bigg)\rho = (1-\rho)\frac{1+H}{2}\rho \\
    R^{-}(\rho) = \rho\bigg(\alpha\frac{1-h_1}{2} + (1-\alpha)\frac{1-h_2}{2}\bigg)(1-\rho) = \rho\frac{1-H}{2}(1-\rho)\\
    \end{cases}
\end{align*}
Using $\rho = \frac{n}{N}$, the equation becomes
\begin{equation}
    E(n) = \frac{1+H}{2}E(n+1)+\frac{1-H}{2}E(n-1) 
\end{equation}
whose solution satisfying boundary conditions is 
\begin{equation}
    E(n) = \frac{1-(\frac{1-H}{1+H})^n}{1-(\frac{1-H}{1+H})^N}
\end{equation}
and thus substituting back the density
\begin{equation}
    E(\rho) = \frac{1-(\frac{1-H}{1+H})^{\rho N}}{1-(\frac{1-H}{1+H})^N}
\end{equation}
Starting from uniformly random initial conditions, the expression above becomes
\begin{equation}
    E(\small\frac{1}{2}) = \frac{1}{1+(\frac{1-H}{1+H})^{\frac{N}{2}}}
\label{exit_homo}
\end{equation}
In figure \ref{}, we check test the validity of (\ref{exit_homo}) for multiple values of $\alpha$ in a small system $N=50$, comparing numerical estimations with the analytical predictions. In both the figures we fix the value of $h_2$ and represent the analytical prediction and the empirical simulations of the exit probability varying the first population's bias $h_1$. In the first figure we set $h_2=0$, which corresponds to the biased vs unbiased case studied in \ref{}, while in the second $h_2=-0.1$. The results confirm the validity of the assumption at the root of the annealing approximation: when $\Delta_{max}$ is low (i.e. when $h_1$ is low, see subfigure), the values of the exit probability predicted within the annealing approximation match with the empirical ones. For larger values of $\Delta_{max}$, instead, the predictions are far from being correct.\\

% We see that in general the prediction of the exit probability calculated within the anneanling apporximation is good for low $\Delta_{max}$, which correspond for both cases to small $h_1$. However, for $h_2=-0.1$ $\Delta_{max}$ is larger for small $h_1$ and thus it turns out that the approximation loses in accuracy before,increasing $h_1$, with respect to the $h_2=0$ case.
\\
% One could also derive and test a prediction on the mean time to reach consensus, within the annealing approximation.

\begin{figure}
\subfloat{\includegraphics[width = 0.5\textwidth]{Figures/exitproba1.png}} 
\subfloat{\includegraphics[width = 0.5\textwidth]{Figures/exitproba2.png}}\\
\subfloat{\includegraphics[width = 0.5\textwidth]{Figures/deltamax1.png} }
\subfloat{\includegraphics[width = 0.5\textwidth]{Figures/deltamax2.png}} \\ \\
\caption{\textbf{Annealing approximation.} \textit{Upper plots}: numerical simulations (triangles) and analytical predictions (lines) of the exit probability at $\rho = \frac{1}{2}$ of a system with $N=50$ agents, fixing $h_2 = 0$ (left) and $h_2 = -0.1$ (right), for different fractions of agents of the first populations $\alpha$ (colors), as function of the first population's bias $h_1$. \textit{Lower plots}: value of the quantity that controls the validity of the annealing approximation $\Delta_{max}$ as a function of $h_1$, obtained from equation \ref{}, respectively for $h_2 = 0$ (left) and $h_2 = -0.1$ (right). In order to calculate numerically the exit probability, we performed $10000$ runs of the model and iterated until convergence, thus we took the fraction of runs that reached the up consensus state.}
\label{fig:annealing}
\end{figure}
