\section{Conclusions and perspectives} \label{chapter_discussion}
In this work, we have studied a system of individuals whose process of  opinion formation is influenced by three factors: the imitative mechanism at the root of most of the models of opinion dynamics, the heterogeneous personal preferences of individuals for one opinion rather than the other, and the homophilic phenomenon at the root of the so-called \textit{epistemic bubbles}, which implies that each individual is more connected to individuals with the same preference and leads to the formation of a social network with a modular structure. We have considered two oppositely biased populations with preferences of different intensities, interacting through a social network that reflects the phenomenon of epistemic bubbles, where two individuals with the same bias happen more likely to get connected. We have derived the system of differential equations governing the opinion dynamics within the mean-field and pair approximations. In the mean-field framework, we have analytically determined the conditions under which the individuals of one population manage to induce the whole system to converge to their preferred opinion. Moreover, we have shown that the achievement of consensus depends mostly on the topological structure  of the "losing" population rather than on the one of the "winning" population. This disparity is more evident the greater is the bias intensity of the "winning" group with respect to the "losing" one.\\
\\
The appropriatness of the assumptions at the basis of the abstract and minimalistic Voter Model with Preferences needs to be tested in a real setting through the implementation of social experiments \cite{peralta2022opinion,centola2015spontaneous}. 
On the other hand, in perspective of an application of the VMP on real data, one has to cope with the problem of identifying and quantifying the external preferences attached to individuals. This is of course an open and difficult problem: a practical method would be to analyze historical positions of each individual on other topics (for example, in the context of misinformation \cite{masuda2011can}, the attached bias may correspond to the frequency at which conspiracy theories has been preferred to mainstream news in the past, by the individual). Another approach would be to first identify communities in a social network of interactions and then infer the average preferences of the individuals of the communities by analyzing the opinion uploads during internal and external interactions (similarly to \cite{pansanella2022change})\footnote{Such inverse problem has been extensively studied with mathematical rigour for the multipopulated Ising-like models on the fully connected graph \cite{fedele2013inverse,contucci2022inverse} and applied \cite{burioni2015enhancing}.}.\\
\\
Additionally, the bipopulated VMP presented in this work serves as a foundation for more sophisticated and realistic models. For example, preferences can be formulated to depend on the current opinions of the two groups. Moreover, the assumption of assigning the same preference to all individuals in a community can be relaxed to explore the impact of different distributions on the asymptotic state of the system. Furthermore, one can consider the presence of noise or other biases (e.g. the algorithmic one \cite{peralta2021effect}) or consider comparable time-scales for personal preferences and opinion uploads.\\
\\
Finally, the VMP can be compared to other models with similar settings (e.g. \cite{malarz2006truth,hernandez2013heterogeneous}), to study the roles of preferences and homophily in the opinion formation process through mutiple perspectives and approaches, as suggested by the authors of \cite{flache2017models}.