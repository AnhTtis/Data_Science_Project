\section{VMP on the complete graph}
\label{FC_chapter}
First we study the model, for simplicity, on the complete network. This setting was already investigated in \cite{masuda2011can}, however, we complete the analysis by calculating the polarization measure at the stationary state for any choice of the parameters.\\
Let $\alpha=\frac{N_1}{N}$ be the fraction of nodes in the first class characterized by bias $h_1$. We define
\begin{equation}
    \rho_1 = \frac{\sum_{i=1}^{N_1}\frac{1+\sigma_i}{2}}{N} \in [0,\alpha]\;\;\;\;\;\;\;\;\;\;\;  \rho_2 = \frac{\sum_{i=N_1+1}^{N}\frac{1+\sigma_i}{2}}{N}\in[0,1-\alpha]
\end{equation}
as the ratios between the number of first class spins (respectively second) in current up $+1$ state and the total number of spins in the system. The system of coupled ordinary differential equations which describes the evolution of such dynamical variables can be written generally in terms of the global rates $R_{\pm1/2}$
\begin{equation}
    \begin{cases}
    \dot{\rho_1} = R_{+1}(\rho_1,\rho_2) - R_{-1}(\rho_1,\rho_2)   \\
    \dot{\rho_2} = R_{+2}(\rho_1,\rho_2) - R_{-2}(\rho_1,\rho_2)
    \end{cases}
    \label{eq:rate}
\end{equation}
For example, the global rate $R_{+1}$ represents the probability per unit time that, when the system is currently in state $\rho_1,\rho_2$, a spin of the first class undergoes the transition $-1\rightarrow+1$, increasing 
the density of up spin of the first class of $1/N$, $\rho_1\rightarrow \rho_1+\frac{1}{N}$. Considering a time unit corresponding to $N$ steps, i.e. $\delta t= N^{-1}$, the transition rates for a fully connected network are
\begin{equation}
    \begin{cases} 
    R_{+1}(\rho_1,\rho_2) = (\alpha-\rho_1)\frac{1+h_1}{2}(\rho_1+\rho_2) \\
    R_{-1}(\rho_1,\rho_2) =  \rho_1\frac{1-h_1}{2}(1-\rho_1 -\rho_2)\\\\
    R_{+2}(\rho_1,\rho_2) =  (1-\alpha-\rho_2)\frac{1+h_2}{2}(\rho_1+\rho_2)\\
    R_{-2}(\rho_1,\rho_2) =  \rho_2\frac{1-h_2}{2}(1-\rho_1-\rho_2)
    \end{cases}
\end{equation}
For example, in the first rate $R_{+1}$ the first term $\alpha-\rho_1$ is the probability of chosing uniformly randomly a spin of the first class currently in down state, while $\rho_1+\rho_2$ is the probalbility of choosing a neighbour in state $+1$ in the complete network, and eventually $\frac{1+h_1}{2}$ is the probability of the transition, according to the model dynamics. Thus we have the following mean-field equations
\begin{equation}
\begin{cases}
    \dot{\rho_1} = \frac{1}{2}\bigg[   (\alpha-\rho_1)(1+h_1)(\rho_1+\rho_2) - \rho_1(1-h_1)(1-(\rho_1+\rho_2))\bigg]  \\
    \dot{\rho_2} = \frac{1}{2}\bigg[ (1-\alpha-\rho_2)(1+h_2)(\rho_1+\rho_2) - \rho_2(1-h_2)(1-(\rho_1+\rho_2)) \bigg] 
\end{cases}
\label{fc mf rho12}
\end{equation}
For the complete network in the $N\rightarrow\infty$ limit, the mean-field equations represent exactly the evolution of the system and they
can be applied, as an approximation, 
to other networks. Not considering structural or dynamical correlations, we expect them to be still accurate on a sufficiently dense network without 
specific structural features \cite{peralta2021effect,porter2014dynamical}, such as an Erd\H os-R\'enyi random graph with probability of linkage of $O(1)$. \\
Localizing the fixed points $(\rho_1^*,\rho_2^*)$ of the system (\ref{fc mf rho12}) and characterizing their stability by the analysis of the corresponding Jacobian matrices reported in the appendix, one finds \cite{masuda2011can} that 
\begin{itemize}
    \item The positive $(1,1)$ (all up spins) and negative $(0,0)$ (all down spins) consensus points are always fixed points, for any combination of the parameters $\alpha,h_1,h_2$.
    \item When the positive (or negative) consensus is stable, it is the only stable fixed point.
    \item When both the consensus fixed points are not stable, another fixed point with $\rho_1^*,\rho_2^*\in(0,1)$ appears. Such fixed point, when it exists, is always stable.
\end{itemize}
Defining the total density of up spin $\rho= \rho_1 + \rho_2$ and $\Delta = \frac{\rho_1}{\alpha} - \frac{\rho_2}{1-\alpha} $, the polarization can be expressed\footnote{If we consider the group magnetizations $m_1=\sum\limits_{i=1}^{N_1}\sigma_i$, $m_2=\sum\limits_{i=N_1+1}^{N}\sigma_i$ it is easy to reconduct the expression before to the more conventional expression of the polarization $P=\frac{|m_1-m_2|}{2}\in[0,1]$} as $P=|\Delta|$. As a first contribution of this paper, we calculate that at the \textit{impasse} or \textit{polarized} state the average density of up spins and the polarization respectively read
\begin{align}
    &\rho^* = \frac{1}{2} \bigg(\frac{h_2-h_1}{h_1h_2}\alpha - \frac{1-h_1}{h_1}\bigg) \label{pol_rho} \\
    &\Delta^* =   \frac{1}{h_1-h_2}\bigg( 1 + \frac{\alpha^2h_1^2+(1-\alpha)^2h_2^2-h_1^2h_2^2}{2\alpha(1-\alpha)h_1h_2}\bigg)\label{pol_delta}
\end{align}
Moreover, by analyzing the Jacobian one can localize the critical value of the parameters at which the transitions from negative consensus to polarization and from polarization to positive consensus occur: taking $h_1,h_2$ fixed and letting $\alpha$ vary, we have that the critical points of the transitions above are respectively at
\begin{align*}
        &\alpha^-_c = (1-h_1)\frac{h_2}{h_2-h_1} \\
        &\alpha^+_c = (1+h_1)\frac{h_2}{h_2-h_1}  
\end{align*}
The bifurcation diagrams, taking $\rho$ and $P$ as order parameter and varying the composition $\alpha$ for various fixed $h_1,h_2$, are shown in Figure \ref{fig:fc_bifurcation}: the bifurcation is of transcritical type and the transitions are indeed continous. The presented numerical simulations confirm that the analytical solutions work well for systems defined on relatively small, $N=1000$ complete graphs as well.
\\ 
The length of the interval associated to the polarized state is $\alpha^+_c - \alpha^-_c =2\frac{h_1h_2}{h_2-h_1}$ which reduces to $\alpha^+_c - \alpha^-_c = h$ for $h_1=-h_2 = h$. The phase diagram in this case (in the $\alpha, h$ plane) is shown in Figure \ref{fig:fc phase diagrams}a, while in the remaining plots of the figure $h_2$ is fixed to different values and the phases in the plane $\alpha, h_1$ are shown. To link the bifurcation and the phase diagrams, the horizontal lines corresponding to the choices of the biases in figure \ref{fig:fc_bifurcation} are reported on the latters.\\
Defining the critical mass of a population \cite{centola2018experimental} as the minimum fraction of individuals of that population necessary to escape from consensus at the unpreferred opinion, we have that the critical masses of respectively the first and second populations are $\alpha^{-}_c$ and $1-\alpha_c^{+}$. For very low biases, the populations over the critical masses rapidly overturn the outcome of the system, switching the direction of consensus. In this model, the critical masses depend only on the biases and lay in the whole range $(0,1)$.














% First we 
% \JK{study}
% %analyze 
% the model, for simplicity, on the complete network.
% \JK{This setting was investigated in}
% %, following 
% \cite{masuda2011can}, however, we complete the \JK{analysis by presenting the phase diagram for the entire ranges of the parameters, and calculate the average opinion (magnetization) \FZ{at the stationary state}}. \JK{: IS THIS TRUE? \FZ{WHAT, LEFT OR RIGHT? Left, mmh the average magnetization at the stationary (polarized) state yes, the phase diagram is shown also in alpha h1 plane, which was not in the previous work but its not a big improvement} For the complete network in the large $N$ limit the mean-field approximation becomes exact and it }
% %the dynamical system that we obtain is of mean-field type and 
% can be applied, as an approximation, 
% %on
% \JK{to }other networks. The mean-field approximation does not consider 
% %neither 
% structural or dynamical correlations: we expect it to be 
% %exact for an infinite (no fluctuations) fully connected network, and 
% still accurate on a sufficiently dense network without 
% %remarkable
% \JK{specific }structural features \FZ{IS IT WORTH TO SAY OR IS IT DIDASCALIC AND SOMEWHAT TRIVIAL?}
% %(e.g. high local clustering or modularity) 
% \cite{peralta2021effect,porter2014dynamical}, such as an Erd\H os-R\'enyi random graph with probability of linkage of $O(1)$. 
% % In a finite fully connected network, the mean-field system reproduces exactly the evolution of the average of the dynamical variables over independent runs.
% %We choose different dynamical variables with respect to \cite{masuda2011can} that simplify the treatment and we deepen the analysis of the derived dynamical system. \\ 
% \\
% \\
% \JK{Let $\alpha=\frac{N_1}{N}$ be the fraction \FZ{of} nodes in the first class characterized by bias $h_1$.} We define
% \begin{equation}
%     \rho_1 = \frac{\sum_{i=1}^{N_1}\frac{1+\sigma_i}{2}}{N} \in [0,\alpha]\;\;\;\;\;\;\;\;\;\;\;  \rho_2 = \frac{\sum_{i=N_1+1}^{N}\frac{1+\sigma_i}{2}}{N}\in[0,1-\alpha]
% \end{equation}
% as the ratios between the number of first class spins (respectively second) in current up $+1$ state and the total number of spins in the system. The system of couple ordinary differential equations that describe the evolution of our dynamical variables can be written generally in terms of the global rate $R_{\pm1/2}$ \JK{JK: IT IS UNFORTUNATE TO USE NUMERICAL UPPER INDICES AS IT CAN BE MIXED UP WITH POWERS. PUT THE INDICES INTO PARENTHESIS LIKE $R^{(\pm1/2)}$ OR USE LOWER INDICES LIKE: $R_{\pm1/2}$. THE SAME IS TRUE FOR THE LATER INTRODUCED $\alpha_c$: USE $\alpha_{c}^{(-1)}$ OR $\alpha_{c}^{-}$ AS "1" IS NOT NEEDED HERE.}\FZ{OK, I should change $\alpha^{\pm-} \rightarrow \alpha^\pm$ also in the graph and $R^{\pm 1/2} \rightarrow R_{\pm1/2}$}
% \begin{equation}
%     \begin{cases}
%     \dot{\rho_1} = R_{+1}(\rho_1,\rho_2) - R_{-1}(\rho_1,\rho_2)   \\
%     \dot{\rho_2} = R_{+2}(\rho_1,\rho_2) - R_{-2}(\rho_1,\rho_2)
%     \end{cases}
%     \label{eq:rate}
% \end{equation}
% For example, the global rate $R_{+1}$ represents the probability \JK{per unit time} that, when the system is currently in state $\rho_1,\rho_2$, a spin of the first class undergoes the transition $-1\rightarrow+1$ 
% %(resp. $+1\rightarrow-1$ with minus sign) in a time unit, 
% increasing 
% %(resp. decreasing) 
% the density of up spin of the first class of $1/N$. Notice that such transition produces the change $\rho_1\rightarrow \rho_1+\frac{1}{N}$. Considering a time unit corresponding to $N$ steps, i.e. $\delta t= N^{-1}$, the transition rates for a fully connected network are
% \begin{equation}
%     \begin{cases} 
%     R_{+1}(\rho_1,\rho_2) = (\alpha-\rho_1)\frac{1+h_1}{2}(\rho_1+\rho_2) \\
%     R_{-1}(\rho_1,\rho_2) =  \rho_1\frac{1-h_1}{2}(1-\rho_1 -\rho_2)\\\\
%     R_{+2}(\rho_1,\rho_2) =  (1-\alpha-\rho_2)\frac{1+h_2}{2}(\rho_1+\rho_2)\\
%     R_{-2}(\rho_1,\rho_2) =  \rho_2\frac{1-h_2}{2}(1-\rho_1-\rho_2)
%     \end{cases}
% \end{equation}
% For example, in the first rate $R_{+1}$ the first term $\alpha-\rho_1$ is the probability of chosing uniformly randomly a spin of the first class currently in down state, while $\rho_1+\rho_2$ is the probalbility of choosing a neighbour in state $+1$ in fully connected, and eventually $\frac{1+h_1}{2}$ is the probability of the transition, according to the model. Thus we have 
% %obtained 
% the \JK{following }mean-field equations\\
% \begin{equation}
% \begin{cases}
%     \dot{\rho_1} = \frac{1}{2}\bigg[   (\alpha-\rho_1)(1+h_1)(\rho_1+\rho_2) - \rho_1(1-h_1)(1-(\rho_1+\rho_2))\bigg]  \\
%     \dot{\rho_2} = \frac{1}{2}\bigg[ (1-\alpha-\rho_2)(1+h_2)(\rho_1+\rho_2) - \rho_2(1-h_2)(1-(\rho_1+\rho_2)) \bigg] 
% \end{cases}
% \label{fc mf rho12}
% \end{equation}
% Such equations are exact in a fully connected network of infinite size, while \JK{they} represent exactly the evolution of the averages $\langle\rho_1\rangle,\langle\rho_2\rangle$ over infinite runs for finite size and fully connected topology. \JK{IS THE LATTER STATEMENT TRUE?! \FZ{WELL, THE PRESENCE OF THE ABSORBING STATES FOR A FINITE SYSTEM MAKES ME THINK A BIT NOW, I TRY TO RECOVER THIS IN THE LITERATURE edit: no, probably it is not true... "not represent exactly but approximate", although it is not completely clear: can it happen that fluctuations are not simmetrical?}}For other topologies, the mean-field system above is a rather crude approximation that 
% %totally 
% \JK{completely }neglects dynamical and structural correlations \cite{porter2014dynamical}.\\
% Localizing the fixed points $(\rho_1^*,\rho_2^*)$ of the system (\ref{fc mf rho12}) and characterizing them by the analysis of the corresponding Jacobian matrices reported in the appendix, one finds \cite{masuda2011can} that 
% \begin{itemize}
%     \item The positive $(1,1)$ (all up spins) and negative $(0,0)$ (all down spins) consensus points are always fixed points.
%     \item When the positive (or negative) consensus is stable, it is the only stable fixed point.
%     \item When both the consensus fixed points are not stable, another fixed point with $\rho_1^*,\rho_2^*\in(0,1)$ appears. Such fixed point, when it exists, is always stable.
% \end{itemize}
% The coordinates of this fixed point, that is associated \JK{with the}
% %to what we call 
%  \textit{impasse} or \textit{polarized} state of the system, can be analytically determined by equating to zero the r.h.s. of (\ref{fc mf rho12}) and solving the system. Defining the total density of up spin $\rho= \rho_1 + \rho_2$ and $\Delta = \frac{\rho_1}{\alpha} - \frac{\rho_2}{1-\alpha} $, the polarization can be expressed\footnote{If we consider the group magnetizations $m_1=\sum\limits_{i=1}^{N_1}\sigma_i$, $m_2=\sum\limits_{i=N_1+1}^{N}\sigma_i$ it is easy to reconduct the expression before to the more conventional expression of the polarization $P=\frac{|m_1-m_2|}{2}\in[0,1]$} as $P=|\Delta|$. We have that at the \textit{polarized} state
% \begin{align}
%     &\rho^* = \frac{1}{2} \bigg(\frac{h_2-h_1}{h_1h_2}\alpha - \frac{1-h_1}{h_1}\bigg) \label{pol_rho} \\
%     &\Delta^* =   \frac{1}{h_1-h_2}\bigg( 1 + \frac{\alpha^2h_1^2+(1-\alpha)^2h_2^2-h_1^2h_2^2}{2\alpha(1-\alpha)h_1h_2}\bigg)\label{pol_delta}
% \end{align}
% Moreover, by analyzing the Jacobian one can localize the critical value of the parameters at which the transitions from negative consensus to polarization and from polarization to positive consensus occur: taking $h_1,h_2$ fixed and letting $\alpha$ vary, we have that the critical points of the transitions above are respectively at
% \begin{align*}
%         &\alpha^-_c = (1-h_1)\frac{h_2}{h_2-h_1} \\
%         &\alpha^+_c = (1+h_1)\frac{h_2}{h_2-h_1}  
% \end{align*}
% The bifurcation diagrams, taking $\rho$ and $P$ as order parameter and varying the composition $\alpha$ for various fixed $h_1,h_2$, are shown in Figure \ref{fig:fc_bifurcation}: the bifurcation is of transcritical type and the transitions are indeed continous. \JK{The presented }numerical simulations 
% %that 
% confirm \JK{that }the analytical \JK{solutions work well for systems defined on relatively small, $N=1000$ complete graphs as well.}
% %predictions are also reported.
% \\ 
% %Interestingly, t
% The length of the interval associated to the polarized state is $\alpha^+_c - \alpha^-_c =2\frac{h_1h_2}{h_2-h_1}$ which reduces to $\alpha^+_c - \alpha^-_c = h$ 
% %in the case of equally strong opposite biases 
% \JK{for }$h_1=-h_2 = h$. The phase diagram in this case (in the $\alpha, h$ plane) is shown in 
% %the upper left plot of 
% Figure \ref{fig:fc phase diagrams}\JK{a}, while in the remaining plots of the figure $h_2$ is fixed to different values and the phases in the plane $\alpha, h_1$ are shown. To link the bifurcation and the phase diagrams, the horizontal lines corresponding to the choices of the biases in figure \ref{fig:fc_bifurcation} are reported on the latters.\\
% \FZ{!!! Maybe you did not read this part, where I first mention the critical mass}The value $\alpha^{-1}_c$ can be intended as the critical mass \cite{centola2018experimental} \FZ{related to the first populations, i.e. the minimum fraction of individuals of the first class necessary to escape from consensus towards the unpreferred $-1$ opinion,} of individuals of the first population to escape from the consensus towards the unpreferred $-1$ opinion, and same thing for $1-\alpha_c^{+1}$ but referred to the second population. Especially \FZ{Specifically} for low biases, the populations over the critical masses totally \FZ{rapidly} overturn the outcome of the system, changing \FZ{no rapidly} rapidly the direction of consensus. In this model, the critical masses depend only on the biases and lay in the whole range $(0,1)$.\\



\begin{figure}
    \subfloat{\includegraphics[width = 3in]{Figures/fc_bifurcation_rho.png}}
    \subfloat{\includegraphics[width = 3in]{Figures/fc_bifurcation_delta.png}}\\
    \caption{\textbf{Bifurcation diagrams for the bipopulated Voter Model with Preferences on the complete network.} On the right the total density of up spin is taken as order parameter, on the left the polarization $\Delta$ is shown. The solid black line indicates the stable fixed point, while the dashed gray lines indicate the unstable ones, for the choice of the preferences' intensities $h_1=0.4,h_2=-0.6$. The other solid colored lines locate just the coexistence stable fixed point for other choices of the intensities, as indicated in the legend. The points and their bars are respectively the average and the confidence interval of the order parameters calculated over $30$ independent simulations of a system with $1000$ agents, and are reported in order to test the validity of the mean field treatment also for relatively small system sizes.}
    \label{fig:fc_bifurcation}
\end{figure}




\begin{figure}
\subfloat(a){\includegraphics[width = 2.7in]{Figures/phase_diagram/phase_diagEQ.png}}
\subfloat(b){\includegraphics[width = 2.7in]{Figures/phase_diagram/phase_diag06.png}}\\
\subfloat(c){\includegraphics[width = 2.7in]{Figures/phase_diagram/phase_diag08.png}} 
\subfloat(d){\includegraphics[width = 2.7in]{Figures/phase_diagram/phase_diag09.png}}\\
\caption{\textbf{Phase diagrams for the complete network.} The polarized area is colored in light purple and indicated with $P$, in white are $+$ and $-$ consensus. Figure (a) shows the mean-field phases in the $\alpha h$ plane, for equal and opposite preferences' intensities $h_1=-h_2=h$. The dark purple dot represents the regime $\alpha =\frac{1}{2},h\rightarrow0$ in which \cite{borile2013effect} have investigated finite size effects. Figures (b),(c),(d) report the phases in the $\alpha h_1$ plane, once fixed $h_2$ respectively to $-0.6,-0.8,-0.9$. Each of the colored horizontal lines present in some of the plots represents the choice of the biases $h_1,h_2$ as in figure \ref{fig:fc_bifurcation} (the colors correspond). They are reported in order to show how the lines intersecate the different phases.}
\label{fig:fc phase diagrams}
\end{figure}


