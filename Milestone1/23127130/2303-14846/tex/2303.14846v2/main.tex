\documentclass[reprint,prl,author-numerical]{revtex4-2}
\pdfoutput=1

\usepackage{graphicx}% Include figure files
\usepackage{dcolumn}% Align table columns on decimal point
\usepackage{hyperref} % add hypertext capabilities
\usepackage{mathtools}
\usepackage{siunitx}

\usepackage{graphicx}
\usepackage{float}
\usepackage{bm}
\usepackage{color}
\usepackage[dvipsnames]{xcolor}
\newcommand{\HB}[1]{\textcolor{SkyBlue}{(HB: #1)}}
\newcommand{\GM}[1]{\textcolor{Peach}{(GM: #1)}}
\renewcommand{\vec}[1]{\mathbf{#1}}
\newcommand{\red}{\color{red}}
\begin{document}
	
\title{Dielectric properties of aqueous electrolytes at the nanoscale}

\author{Maximilian R. Becker}
\affiliation{Fachbereich Physik, Freie Universität Berlin, Arnimallee 14, Berlin, 14195, Germany}

\author{Philip Loche}
\affiliation{Laboratory of Computational Science and Modeling, IMX, École Polytechnique Fédérale de Lausanne, 1015 Lausanne, Switzerland}
\affiliation{Fachbereich Physik, Freie Universität Berlin, Arnimallee 14, Berlin, 14195, Germany}

\author{ Roland R. Netz}
\email[]{rnetz@physik.fu-berlin.de}
\affiliation{Fachbereich Physik, Freie Universität Berlin, Arnimallee 14, Berlin, 14195, Germany}


\author{Douwe Jan Bonthuis}
\affiliation{Institute of Theoretical and Computational Physics, Graz University of Technology, Graz, Austria}

\author{Dominique Mouhanna}
\affiliation{Sorbonne Universit{\'e}, CNRS, Laboratoire de Physique Th{\'e}orique de la Mati{\`e}re Condens{\'e}e (LPTMC, UMR 7600), F-75005 Paris, France}

\author{H{\'e}l{\`e}ne Berthoumieux}
\email[]{helene.berthoumieux@espci.fr}
\affiliation{Gulliver, CNRS, {\'E}cole Sup{\'e}rieure de Physique et Chimie Industrielles de Paris, Paris Sciences et Lettres Research University, Paris 75005, France}

\begin{abstract}
Despite the ubiquity of aqueous electrolytes, the effect of salt on water organization remains controversial. We introduce a nonlocal and nonlinear field theory for the nanoscale polarization of ions and water and derive the electrolyte dielectric response as a function of salt concentration to first order in a loop expansion. By comparison with molecular dynamics simulations, we show that rising salt concentration induces a dielectric permittivity decrement and Debye screening in the longitudinal susceptibility but leaves the water structure remarkably unchanged. 
\end{abstract}


\maketitle



{\it Introduction - }
Studying electrolytes at the nanoscale is exciting for both the experimental relevance and the theoretical challenge~\cite{bjorneholm16,kavokine20,backus21}.
The nanometer scale is typical of technological and biological devices, it is the screening length of medium-concentrated
ionic solutions, as well as the range at which water
starts to behave as a discrete molecular medium~\cite{schlaich2016,fumagalli2018,monetprl}.  In the standard Poisson-Boltzmann (PB) approach, water is described as a linear dielectric medium, and its permittivity $\epsilon_w$ is wavenumber independent. This model cannot capture the complexity of water-ion interaction at the nanoscale.
A nonlocal dielectric medium is characterized by the wavenumber dependence of the susceptibility tensor $\chi(\vec{q})$, where $q$ is the scattering wavenumber, $q=2\pi/\lambda$. This kernel is defined by the correlations of the polarization field $\bm{\mathcal{P}}$ as ${\chi_{ij}(\vec{q})=\beta\langle \bm{\mathcal{P}}_i(\vec{q})\cdot\bm{\mathcal{P}}_j(-\vec{q})\rangle/\epsilon_0}$, with $\beta$ the inverse temperature and $\epsilon_0$ the vacuum permittivity~\cite{bopp1996,kornyshevtrans}.\par For pure water, figure \ref{fig:0} shows the longitudinal $\chi^{\rm w}_{\parallel}$ and transverse $\chi^{\rm w}_\perp$  susceptibilities as a function of $q$ (panel b and c, blue markers and blue broken line) derived from force-field molecular dynamics (MD) simulations with the TIP4p/$\epsilon$ water model~\cite{sm_paper,gromacs,neuman83,flyvbjerg1989}, designed to reproduce its macroscopic permittivity~\cite{azcatl2014}. 
\begin{figure}
	\includegraphics[scale=1]{Fig1-1.pdf} % AR : 0.6
	\caption{a) Sketch of the system under study. Water acts as a nonlocal, nonlinear dielectric medium in the presence of ions. Using FT, we  evaluate the ionic-strength effect on the water
		longitudinal $\bm{\mathcal{P}_\parallel}$ (red arrow) and transverse $\bm{\mathcal{P}_\perp}$ (blue arrow) polarization correlations and compare with simulated response functions. Longitudinal (b) and transverse (c) MD simulated and FT-derived susceptibilities for pure water in Fourier space.   FT predictions follow from Eq.~(\ref{chiw}) for $K$= 1/76,  $\kappa_l$= -0.218~\AA$^{2}$ , $\alpha$=0.012~\AA$^{4}$ and $\kappa_t$=0.013~\AA$^{2}$. MD simulated susceptibilities are calculated from the charge structure factor  from radial distribution functions for larger $q$ (dashed line) and for discrete wavenumbers at low $q$ (data points) with a minimal $q=2\pi/L\approx$0.1\AA$^{-1}$ fixed by the size $L$ of the simulation box~\cite{sm_paper}. 
	}
	\label{fig:0}
\end{figure}
MD data display a pronounced maximum with $\chi^{\rm w}_{\parallel}(q)\gg$1 around $q$=3~\AA$^{-1}$, which corresponds to a $q$-region associated with a negative permittivity, as $\epsilon(q)$ obeys $\epsilon(q)=1/(1-\chi^{\rm w}_{\parallel}(q))$, named overscreening~\cite{bopp1996}. This phenomenon can be attributed to the short-range H-bond network of water~\cite{kornyshev1997}. 
	The transverse susceptibility is rather featureless and decays monotonically. \par
	Adding ions induces a decrement of the macroscopic permittivity $\epsilon(q=0)$ of electrolytes which is well-documented experimentally~\cite{hasted48}.  This effect can be recovered theoretically using a nonlinear PB equation~\cite{levy2012, levy2013}, which accounts for the low permittivity due to the water orientation in the hydration shells. \par
	The description of water-ion interactions has been the subject of many experimental~\cite{omta,chen2016,zhang2022,balos2022} and simulation works~\cite{galli2017,pluharova2017}. However, the effect of salt on the H-bond network - breaking or reinforcing  -  and the spatial range on which it plays a role remains controversial.  Coarse-grained theories that account for nonlocal dielectric properties, {\it ie} the $q$-dependence of the susceptibility,  provide a useful framework for describing correlated  fluids~\cite{kornyshev1986,Basilevsky1998,hildebrandt2004,maggs2006}. They have been applied to electrolytes to study the coupling between the correlation length of simple fluids and the Debye screening length by deriving a nonlocal linear PB equation~\cite{paillusson2010,benyaakov2011}. Recently, a general theory for electrolytes, including electrostatic and structural interactions for the solvent, has been derived ~\cite{blossey,blossey2,blossey3}. However, a nonlocal field theory (FT) for aqueous electrolytes, including the water-salt interaction at the nanoscale, validated by MD simulations, is missing. This is what we address in this letter. \par
{\it Nonlocal model for pure water - } 
 The electrostatic energy of pure water $\mathcal{U}_{\rm el}$ can be written as a functional of the polarization~\cite{blossey} $\bm{\mathcal{P}}(\vec{r})$ as:
\begin{eqnarray}
\label{Hframework}
\mathcal{U}_{\rm el}[\bm{\mathcal{P}}]=\frac{1}{2}\int d\vec{r} d\vec{r}' \frac{ \nabla\cdot \bm{ \mathcal{P}}(\vec{r}) \nabla\cdot \bm{ \mathcal{P}}(\vec{r}')}{4\pi \epsilon_0|\vec{r}-\vec{r}'|} 
+\mathcal{U}_{\rm conf}[\bm{\mathcal{P}}].
\end{eqnarray}
The first term corresponds to the bare Coulomb interactions between the partial charges $-\nabla\cdot\bm{\mathcal{P}}(\vec{r})$ of the fluid and the second term to a phenomenological configurational energy of the fluid~\cite{maggs2006,blossey} written as follows:
\begin{eqnarray}
\label{Hpw}
\mathcal{U}_{\rm conf}[\bm{\mathcal{P}}]&=&\frac{1}{2\epsilon_0}\int d\vec{r}\Big [\gamma \bm{ \mathcal{P}}(\bm{r})^4+ K \bm{ \mathcal{P}}(\bm{r})^2
+\kappa_l(\bm{\nabla}\cdot \bm{ \mathcal{P}}(\bm{r}))^2\nonumber\\&+&\kappa_t(\bm{\nabla}\times \bm{ \mathcal{P}}(\bm{r}))^2+ \alpha (\bm{\nabla}
(\bm{\nabla} \cdot \bm{ \mathcal{P}}(\bm{r}))  )^2 \Big ].
\end{eqnarray}
First, we discuss this nonlocal model at the Gaussian level, obtained by setting $\gamma$=0. In this case,  $\mathcal{U}_{\rm conf}[\bm{\mathcal{P}}]$ is a Landau-Ginzburg expansion up to the second spatial derivative for the longitudinal part (terms in $\kappa_l$ and $\alpha$) and up to the first spatial derivative for the transverse part (term in $\kappa_t$). This functional captures the main features of dielectric properties of water at the nanoscale~\cite{maggs2006,berthoumieux2015,monetprl}.  
The polarization susceptibility $\chi^{\rm w}$ is obtained by inversion of Eq.~(\ref{Hframework})~\cite{sm_paper}. We decompose it in Fourier space into a longitudinal $\chi_{\parallel}^{\rm w}$ and a transverse $\chi_{\perp}^{\rm w}$ response using translational invariance of the system, such that $\chi_{ij}(\vec{q})=\chi^{\rm w}_{\parallel}(q) q_iq_j/q^2+\chi^{\rm w}_{\perp}(q)(\delta_{ij}-q_iq_j/q^2)$, with $(i,j)=(x,y,z)$.  Their expressions follow from Eq.~(\ref{Hpw}) as 
\begin{equation}
\label{chiw}
	\chi_{\parallel}^{\rm w}(q)=\frac{1}{1+K+\kappa_l q^2+ \alpha q^4}, \quad \chi_{\perp}^{\rm w}(q)=\frac{1}{K+\kappa_t q^2}.
\end{equation}
For $\chi_\parallel^{\rm w}(q)$, the case $(\kappa_l<0$, $\alpha>0)$, generates a maximum at finite $q$. First, we determine the the parameters  ($K$, $\kappa_l$, $\alpha$) by fitting $\chi^{\rm w}_{\parallel}(0)$, the position and the value of the maximum of $\chi^{\rm w}_{\parallel}(q)$ to the MD data shown in Fig.\ref{fig:0} (b). Note that the secondary peak observed for $q=5$~\AA$^{-1}$ corresponds to intramolecular correlations and to length scales that are not addressed in our formula of $\mathcal{U}_{\rm conf}$.
Second, we determine  $\kappa_t$ by fitting the decay of $\chi_{\perp}^{\rm w}(q)$ at low $q$ shown in the MD data in Fig.~\ref{fig:0} (c). The values of $(K,\kappa_l,\alpha,\kappa_t)$ are given in the caption of Fig.~\ref{fig:0}.

 
For $\gamma> 0$, the term scaling like $\bm{\mathcal{P}}^4$ in Eq.~(\ref{Hpw}) describes the nonlinear water response. As a result, when submitted to a strong electrostatic field $\vec{E}$, the polarization no longer scales linearly with  $\vec{E}$ but as $\vec{E}^{1/3}$   ~\cite{alper90,fedorov2007,levy2013,berthoumieux2019}.
This accounts for the polarization of water in the ionic solvation shells~\cite{vorotyntsev2019,berthoumieux2019} and produces a threshold polarization, $P_0=\sqrt{K/2\gamma}$, between a linear-response  ($|\bm{\mathcal{P}}|\ll P_0$) and a saturation-response ($|\bm{\mathcal{P}}|\gg P_0$) regime. The parameter $\gamma$ tunes this saturation threshold.\par
In this letter, we compute the free energy and the nonlocal dielectric susceptibility of an aqueous electrolyte for which water is modeled by Eq.~(\ref{Hpw}). We perform MD simulations and compare the longitudinal and transverse permittivity dependence for varying salt concentrations with the FT predictions. Finally, we identify the essential building blocks of a FT that accurately models electrolytes at the nanoscale and reproduces the longitudinal
and transverse dielectric susceptibilities computed by MD simulations. \par
{\it Nonlocal model for electrolytes - } We consider an electrolyte with $N$ point-like cations of charge $e$ and $N$ point-like anions of charge $-e$ solvated in water. The ionic charge density reads $\rho(\vec{r})=\Sigma_{i=1}^{N} e \delta(\vec{r}-\vec{r}_i^+)-\Sigma_{j=1}^{N} e \delta(\vec{r}-\vec{r}_j^-)$.
In the canonical ensemble, the partition function of the electrolyte can be written as 
\begin{eqnarray}
\label{Zloc}
&&\mathcal{Z}=\frac{1}{(N!)^2}\left[\prod_{i=1}^{N}\int d\vec{r}_i^+ \right]\left[\prod_{j=1}^{N}\int d\vec{r}_j^- \right]\nonumber\\& &\times\int \mathcal{D}[\bm{\mathcal{P}}]e^{-\beta \mathcal{U}_{\rm conf}[\bm{ \mathcal{P}}]}
 e^{-\frac{\beta}{2}\int d\vec{r} d\vec{r}'\rho_{\rm tot}(\vec{r})v(\vec{r}-\vec{r}')\rho_{\rm tot}(\vec{r}')}\,
\end{eqnarray}
with the total charge density $\rho_{\rm tot}(\vec{r})=\rho(\vec{r})-\nabla\cdot\bm{\mathcal{P}}(\vec{r}) $ and $v(\vec{r})=1/4\pi \epsilon_0|\vec{r}|$. $\mathcal{Z}$ includes the configurational degrees of freedom of the polarizable water, with $\int\mathcal{D}[\bm{\mathcal{P}}]$ the path integral over the fluctuating field $\bm{\mathcal{P}}$, and the Coulomb interactions between free and partial charges. Following the work of Orland and co-workers~\cite{levy2012}, we perform a Hubbard-Stratonovich transformation for $\mathcal{Z}$ to get rid of the long-range potential $v$ and switch to the grand-canonical ensemble for a more tractable expression of the partition function. The grand-canonical partition function, ${\Xi=\int \mathcal{D}[\bm{\mathcal{P}}]\, \mathcal{D}[\Psi]e^{ - \beta F_u[\bm{{\mathcal{P}}},\Psi]}}$, takes in the Gaussian limit ( $\gamma=0$ ) a simple form as a function of the action $F_u[{\bm {\mathcal P}},\Psi]$ which is a functional of ${\bm {\mathcal P}}$ and of the electrostatic potential $\Psi$~\cite{levy2012}. Assuming a 1:1 electrolyte and an ionic density $n=N/V$ with $V$ the volume of the system, we find:
\begin{eqnarray}
\label{action}
F_u[{\bm {\mathcal P}},\Psi]&=& \mathcal{U}_{\rm conf}[\bm{\mathcal{P}}] -\int d\vec{r}\Big(\frac{\epsilon_0}{2} (\nabla \Psi(\vec{r}))^2-\Psi(\vec{r}) \nabla \cdot \bm{\mathcal{P}}(\vec{r})\nonumber\\&-&\frac{2 n}{\beta}\cosh(\beta \Psi(\vec{r}) e)\Big).
\end{eqnarray}
Details of the calculations are given in the subsection S2.2 of the SM. The free energy $\mathcal{F}$ of the system follows from $\mathcal{F}=-k_BT\rm{ln} \Xi$ and the generalized susceptibility is a 4$\times$4 matrix defined as:
\begin{eqnarray}
\label{DefSusc}
&&\left(\begin{array}{cc} \epsilon_0\chi & \chi_{P,\psi} \\ \chi_{\psi, P} & \chi_{\psi,\psi}/\epsilon_0
	\end{array}\right)({\vec{r}_1-\vec{r}_2})= \nonumber\\ & &\left(\begin{array}{cc}
\frac{\delta^2 F_u(\bm{P},\psi)}{\delta \mathcal{P}_i(\vec{r}_1) \delta \mathcal{P}_j(\vec{r}_2) }   & \frac{\delta^2 F_u(\bm{P},\psi)}{\delta \mathcal{P}_i(\vec{r}_1) \delta \Psi(\vec{r}_1)} \\
\frac{\delta^2 F_u(\bm{P},\psi)}{\delta \Psi(\vec{r}_1) \delta \mathcal{P}_j(\vec{r}_2)}  & \frac{\delta^2 F_u(\bm{P},\psi)}{\delta \Psi(\vec{r}_1) \delta \Psi(\vec{r}_2)}
\end{array}\right) ^{-1}
\end{eqnarray}
where the second functional derivatives of $F_u[{\bm {\mathcal P}},\Psi]$ are evaluated at the mean field point 
($\bm{P},\psi$), defined by $\delta F_u(\bm{P},\psi)/\delta{\mathcal{P}_i}=\delta F_u(\bm{P},\psi)/\delta \psi=0$~\cite{sm_paper}.  We focus here on the polarization susceptibility. The longitudinal part is written in Fourier space as
\begin{figure}
\includegraphics[scale=0.9]{Fig1.pdf} % AR : 0.6
\caption{Dielectric susceptibility for electrolytes. (a) Longitudinal susceptibility $\chi_{\parallel}$ - Eq.~(\ref*{SuscepElectMT}) - as a function of $q$ for water and solutions of increasing salt concentration. The arrow indicates the increase of the peak maximum with an increase of the concentration. The inset zooms into the low $q$ part of the plot. (b) Transverse susceptibility $\chi_{\perp}=\chi_{\perp}^{\rm w}$ given by Eq~(\ref{chiw}) as a function of $q$, which is identical for water and electrolytic solutions. The parameter values of the FT
	model given in Eq. (\ref{Hpw}) are $K$= 1/76,  $\kappa_l$= -0.218~\AA$^{2}$, $\alpha$=0.012~\AA$^{4}$ and $\kappa_t$=0.013~\AA$^{2}$.}
\label{fig:1}
\end{figure}

\begin{equation}
\label{SuscepElectMT}
\chi_{\parallel}(q)=\frac{\frac{\epsilon_w}{\lambda_D^2}+q^2}{\left(\frac{\epsilon_w}{\lambda_D^2}+q^2\right)(K+\kappa_l q^2+\alpha q^4)+q^2}, 
\end{equation}
with the Debye length $\lambda_D=\sqrt{\epsilon_0\epsilon_w/2\beta n e^2}$~\cite{sm_paper}. Fig.~\ref{fig:1} a) shows $\chi_{\parallel}(q)$ for increasing molar salt concentration $c=n/\mathcal{N}_a$, where $\mathcal{N}_a$ is the Avogadro number.  The inset zooms on the low $q$ part of $\chi_{\parallel}(q)$. It reveals that for low $q$, $\chi_{\parallel}(q)$ decreases with a characteristic wavenumber $q^{\rm char}$, which scales as $1/\lambda_D$. The Gaussian model predicts an enhancement of the water ordering with an increase of $c$, as indicated by the magnitude increase of the peak at $q$=3~\AA$^{-1}$. This can be understood as follows: in the nonlocal Gaussian framework, an ion, located at $r=0$, generates an oscillating exponentially decaying polarization response~\cite{Basilevsky1998,vatin21}. This induces the organization of the free ionic charges, which in turn generates longer range correlations in water that increase with the salt concentration until a nonphysical crystallization of the system occurs, corresponding to a divergence of $\chi_{\parallel}(q)$ (see Fig. S2 of the SM). In contrast, for a very dilute solution, the "overscreening" peak at $q$=3~\AA$^{-1}$ is identical to the peak of neat water as shown by the green line in fig.~(\ref{fig:1}) a).  In this case, the Debye length is several orders of magnitude larger than the polarization correlation lengths in water. $\chi_{\parallel}(q)$  in Eq.~(\ref{SuscepElectMT}) takes a simple expression in the low $q$ regime,  $\chi_{\parallel}(q)\approx(\frac{\epsilon_w}{\lambda_D^2}+q^2)/(K\left(q^2+\frac{\epsilon_w}{\lambda_D^2}\right)+q^2)$, and becomes a function of only $\lambda_D$ and the large-scale ($q=0$) parameter $K$. In the large $q$ regime,   $\chi_{\parallel}(q)$ equals the neat water  susceptibility, $\chi_{\parallel}(q)\approx \chi_\parallel^w(q)$ in Eq.~(\ref{chiw}). Up to a negligible constant, we can write $\chi_{\parallel}(q)$ for dilute electrolytes as:
 \begin{equation}
 \label{decoupled_chi}
 	 \chi_{\parallel}(q)\approx\frac{\frac{\epsilon_w}{\lambda_D^2}+q^2}{K\left(q^2+\frac{\epsilon_w}{\lambda_D^2}\right)+q^2}+\frac{1}{1+K+\kappa_l q^2+ \alpha q^4}.
 \end{equation}
The transverse susceptibility $\chi_\perp(q)$ of electrolytes is unaffected by the presence of salt and obeys $\chi_{\perp}(q)$=$\chi_\perp^{\rm w}(q)$, see Fig.~\ref{fig:1} (b). Indeed, the coupling between the salt and the solvent occurs via the Coulomb interactions and involves only the longitudinal part of the polarization, as seen in Eq.~(\ref{Zloc}). Finally, we note that the Gaussian model predicts that the dielectric large-scale properties of electrolyte solutions $\chi_{\parallel}(0)=\chi_{\perp}(0)=1/K$ are independent of the salt concentration, which contradicts the permittivity decrement due to added salt~\cite{hasted48}, and is a consequence of the simple Gaussian model with $\gamma$=0.\par% 
{\it Comparison with MD simulations - }
To check the validity of the Gaussian model, we compare its predictions with the dielectric properties of simulated solutions of NaCl in TIP4p/$\epsilon$ water for concentrations $c$ up to 1.5~mol.L$^{-1}$~\cite{azcatl2014,loche2021}.  We compute the $q=0$ permittivity and plot $\epsilon(c)$ in Fig.~\ref{fig:2}~(a). We observe a linear decay not described by the Gaussian model, which predicts a constant $q$=0 permittivity. Panels (b) and (c) of Fig.~\ref{fig:2} show $\chi_{\perp}^{\rm MD}(q)$ and  $\chi_{\parallel}^{\rm MD}(q)$ for $c$=0.15~mol.L$^{-1}$ ($\lambda_D$=7.8~\AA), $c$=0.75~mol.L$^{-1}$ ($\lambda_D$=3.5~\AA), and $c$=1.5~mol.L$^{-1}$ ($\lambda_D$=2.5~\AA). The blue markers and blue broken line show the response for pure water.  $\chi^{\rm MD}_{\perp}(q\rightarrow 0)$ decays for an increasing salt concentration. $\chi^{\rm MD}_{\parallel}(q)$ shows a decay at low $q$  ($q\leq$2~\AA $^{-1}$) associated with a cut-off wavenumber of about $1/\lambda_D$. At large $q$ ($q>$2~\AA$^{-1}$), $\chi_{\parallel}^{\rm MD}(q)$ for electrolytes surprisingly remains almost identical to the one of pure water. It is only at high salt concentration (yellow markers and line, $c$=1.5~mol.L$^{-1}$) that an effect of the salt is visible: the peak at $q\approx$3~\AA$^{-1}$ slightly decreases and flattens - see inset of Fig.~\ref{fig:2} (c), in contradiction to the Gaussian model predictions. \par
{\it Nonlinear model for water - } 
To obtain a better agreement between FT and MD simulations, we consider the nonlinear configuration energy  $\mathcal{U}_{\rm conf}$ with $\gamma > 0$. The action $F_u$ contains then a non-quadratic term, $\gamma\bm{{\mathcal{P}}}^4$. Its effect on response functions can be estimated by using a {\it loop expansion} around the mean field~\cite{netz2000} that we perform here at first order~\cite{levy2013}.  $F_u$ is expanded up to the second order in $(\bm{ \mathcal{P}},\Psi)$ around the mean field solution according to $F_u[\bm{\mathcal{P}},\Psi]\approx F_u[{\bm P},\psi] +1/2\int d\vec{r}d\vec{r}'(\delta \bm{P}(\vec{r}),\delta \psi(\vec{r}))\cdot F_u^{(2)}({\bm P}, \psi)\cdot (\delta \bm{P}(\vec{r'}),\delta \psi(\vec{r}'))$, with $F_u^{(2)}$ the second functional derivative of $F_u$ with respect to $(\bm{\mathcal{P}}, \Psi)$ and $\delta \bm{P}=\bm{\mathcal{P}}-{\bm P}$, $\delta \psi=\Psi-\psi $. The partition function thus follows as 
\begin{eqnarray}	
	 \Xi\approx \exp \left\{- \beta F_u[\bm{{\bm{P}}},\psi]-\frac{1}{2} \ln \left[\det \beta F_u^{(2)}({\bm P}, \psi)\right]\right\}
	 \end{eqnarray}
and the free energy is written  as 
$\mathcal{F}\approx F_u[{\bm P},\psi]+{1/2}{\rm Tr}{\rm ln}(\beta F_u^{(2)}[{\bm P},\psi])/\beta$~\cite{sm_paper}. The inverse susceptibility follows as $
\chi^{-1}(\vec{r}_1-\vec{r}_2)=\chi^{-1}_{(0)}(\vec{r}_1-\vec{r}_2)+\chi^{-1}_{(1)}(\vec{r}_1-\vec{r}_2)$, with  $\chi_{(0)}^{-1}$ the inverse susceptibility in the Gaussian limit given in Eq.~(\ref{DefSusc}). The one-loop correction term $\chi^{-1}_{(1)}(\vec{r}_1-\vec{r}_2)$ obeys:
\begin{eqnarray}
\label{chifirstloop}
(\chi^{-1}_{(1)})_{i,j}(\vec{r}_1-\vec{r}_2)=\frac{1}{2\beta}\frac{\delta ^2 {\rm Tr}\ln \beta F_u^{(2)}({\bm P},\psi)}{\delta \mathcal{P}_i(\vec{r}_1) \delta \mathcal{P}_j(\vec{r}_2)}.
\end{eqnarray}
 Performing the functional derivatives and calculating the trace in Eq.~(\ref{chifirstloop}), one obtains~\cite{sm_paper}
\begin{eqnarray}
\label{chim11}
&&(\chi_{(1)}^{-1})_{i,j}(\vec{r}_1-\vec{r}_2)=\delta K \delta(\vec{r}_1-\vec{r}_2)\delta_{ij}\quad {\rm with}\nonumber\\
\delta K&=&\frac{20\gamma \epsilon_0}{3 \beta }\left( \hat{\chi}_{\parallel}(r=0)+2\hat{\chi}_{\perp}(r=0)\right).
\end{eqnarray}
Note that here $\hat{\chi}_\parallel$ and $\hat{\chi}_\perp$  are  back-Fourier transforms of Eqs.~(\ref{chiw}, \ref{SuscepElectMT}) at $r=0$.
 The first-order correction of the susceptibility is purely local and depends on $c$ via $\chi_\parallel$.
We expand $\epsilon=1+1/(K+\delta K)$ linearly in the salt concentration $c$ and obtain: 
\begin{eqnarray}
\label{epsoneloop}
	\epsilon(c)&=&\epsilon_w-\frac{\delta K_c}{K^2}c.
\end{eqnarray}
with $\delta K_c$ given in S2.8 of the SM.
We determine $\delta K_c$ by fitting the permittivity of simulated solutions as shown in fig.~\ref{fig:2} (a). The value of $\delta K_c$ is given in the caption of Fig.\ref{fig:2}.\par

{\it Susceptibility kernels for electrolytes - } We now compare the simulated $q$-dependent susceptibilities with the nonlinear FT ones, obtained by replacing $K$ by ${K + \delta K_c c}$ in Eq.~(\ref*{SuscepElectMT}) for $\chi_{\parallel}$ and in Eq.~(\ref{chiw}) for $\chi_{\perp}$, and using the values $(\alpha, \kappa_l, \kappa_t, K)$ fitted for pure water. 
\begin{figure}
\includegraphics[scale=0.9]{Fig2.pdf} % AR : 0.6
\caption{Response functions of electrolytes. (a)  MD simulated (markers) and FT-derived (line) permittivity as a function of salt concentration $c$. FT expression is given in Eq.~(\ref{epsoneloop}) for $\delta K_c$=0.0028~mol$^{-1}$.L. (b) MD simulated (markers) and FT-derived (line, Eq~(\ref{chiperpFT}) transverse permittivity. (c) MD simulated longitudinal susceptibility $\chi_{\parallel}^{\rm MD}$.  Markers and dashed lines correspond to two extraction methods (see caption of Fig.{\ref{fig:0}} and S1 of the SM). (d) FT expression of the longitudinal susceptibility $\chi_{\parallel}^{\rm FT}$ given in Eq.~(\ref{chiparallelFT}). The insets zoom on the $q$=3\AA$^{-1}$ peak. For (d), the blue and green plots overlap completely.} 
\label{fig:2}
\end{figure}
We plot the one-loop corrected transverse response,
\begin{equation}
\label{chiperpFT}
\chi^{\rm FT }_\perp(q)=\frac{1}{K+\delta K_c c +\kappa_t q^2},
\end{equation}
in Fig.~\ref{fig:2} (b)  for $c$=0, 0.15, 0.75, 1.5 mol.L$^{-1}$ and find a very reasonable
agreement with simulations. \par In the SM (see Fig. S4), we show that the renormalized longitudinal susceptibility exhibits an enhancement of the "overscreening" peak, which is weaker than for the Gaussian limit but still contradictory to the MD data. To account for the structural decoupling of water and ion polarization, we use
the decoupled expression for $\chi_\parallel(q)$ given in Eq.~(\ref{decoupled_chi}) and replace $K$ by $K+\delta K_cc$. We obtain:
\begin{eqnarray}
\label{chiparallelFT}
\chi^{\rm FT }_\parallel(q)&=&\frac{\frac{\epsilon_w}{\lambda_D^2}+q^2}{(K+\delta K_c c)(\frac{\epsilon_w}{\lambda_D^2}+q^2)+q^2}\nonumber\\&+&\frac{1}{1+K+\delta K_cc+\kappa_l q^2+\alpha q^4}.
\end{eqnarray}
The first term accounts for Debye screening, the second term for the susceptibility for pure water. In both terms, the $q=0$ parameter $K$ is renormalized.  We plot $\chi^{\rm FT }_\parallel(q)$ for $c$=0, 0.15, 0.75, 1.5 mol.L$^{-1}$ in Fig.~\ref{fig:2} (d). It reproduces the MD main features, in particular, the decay at low $q$ and the flattening of the peak at $q$=3~\AA$^{-1}$ with an increase of the salt concentration.\par
{\it Discussion - }
In this letter, we describe the behavior of the susceptibility of water with added salt.  For this, we compare the susceptibility calculated from FT, including nonlocal and nonlinear terms, with MD data. We highlight two regimes in $q$-space for the longitudinal correlations: for small $q$, $q\leq$2~\AA$^{-1}$, the electrolyte is described with a renormalized permittivity $\epsilon(c)$. This long-range interaction regime shows the standard Debye screening. At larger $q$, the longitudinal susceptibility in electrolytes equals the one of pure water but is associated with the renormalized permittivity $\epsilon(c)$. 
These two decoupled regimes reflect two distinct water structures in electrolytes: water molecules solvating ions are "electrically" frozen by the ionic field, creating a solvation shell of vanishing permittivity~\cite{levy2012}. Outside of this shell, the water structure is neither strengthened nor destroyed, but is remarkably unaffected.~\cite{monetJCP2021,zhang2022}. 
Our work reveals the absence of coupling between salt screening and transverse	polarization modes of water, which could have consequences on the interactions between objects immersed in electrolytes~\cite{pires21} that are assumed to be screened~\cite{MahantyNinham1976,Parsegian2006}.
This study gives a clear picture of the nature and range of salt's effect on water organization at the nanoscale for unconfined solutions. It is the first step toward an accurate FT describing the properties of
nanoconfined electrolytes~\cite{tuladhar2020,robert23,martinjimenez2016,lee2021}.\\

HB thanks H. Orland for fruitful discussions. H.B. acknowledges funding from the Humboldt Research Fellowship Program for Experienced Researchers.  M.B. acknowledges support by Deutsche Forschungsgemeinschaft, Grant No. CRC 1349, Code No. 387284271, Project No. C04. This work was funded by the Deutsche Forschungsgemeinschaft (DFG, German Research Foundation) in project GRK 2662 -  434130070.


\bibliographystyle{unsrt}
\bibliography{RPfinal}
\end{document}