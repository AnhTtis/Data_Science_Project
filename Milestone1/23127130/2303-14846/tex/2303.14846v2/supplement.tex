\documentclass[reprint, amsmath,amssymb, onecolumn, aps, linenumbers]{revtex4}
\pdfoutput=1

\usepackage{epsfig}
\usepackage{hyperref}
\usepackage{amsfonts}
\usepackage{amsmath}
\usepackage{color}
\usepackage{mathtools}
\usepackage{siunitx}


\usepackage{graphicx}
\usepackage{float}
\usepackage{bm}
\usepackage[dvipsnames]{xcolor}
\newcommand{\HB}[1]{\textcolor{SkyBlue}{(HB: #1)}}
\renewcommand{\vec}[1]{\mathbf{#1}}

% Usual (decimal) numbering
\renewcommand{\thesection}{\arabic{section}}
\renewcommand{\thesubsection}{\thesection.\arabic{subsection}}
\renewcommand{\thesubsubsection}{\thesubsection.\arabic{subsubsection}}

\renewcommand{\thesection}{S\arabic{section}}
\renewcommand{\thefigure}{S\arabic{figure}}
\renewcommand{\thetable}{S\arabic{table}}
\renewcommand{\theequation}{S\arabic{equation}}

\usepackage{verbatim}
\immediate\write18{texcount -tex -sum  main.tex > \jobname.wordcount.tex}

\begin{document}
	\title{Supplementary Materials for\\Dielectric properties of aqueous electrolytes at the nanoscale}
	
\author{Maximilian R. Becker}
\affiliation{Fachbereich Physik, Freie Universität Berlin, Arnimallee 14, Berlin, 14195, Germany}

\author{Philip Loche}
\affiliation{Laboratory of Computational Science and Modeling, IMX, École Polytechnique Fédérale de Lausanne, 1015 Lausanne, Switzerland}
\affiliation{Fachbereich Physik, Freie Universität Berlin, Arnimallee 14, Berlin, 14195, Germany}

\author{ Roland R. Netz}
\email[]{rnetz@physik.fu-berlin.de}
\affiliation{Fachbereich Physik, Freie Universität Berlin, Arnimallee 14, Berlin, 14195, Germany}


\author{Douwe Jan Bonthuis}
\affiliation{Institute of Theoretical and Computational Physics, Graz University of Technology, Graz, Austria}

\author{Dominique Mouhanna}
\affiliation{Sorbonne Universit{\'e}, CNRS, Laboratoire de Physique Th{\'e}orique de la Mati{\`e}re Condens{\'e}e (LPTMC, UMR 7600), F-75005 Paris, France}

\author{H{\'e}l{\`e}ne Berthoumieux}
\email[]{helene.berthoumieux@espci.fr}
\affiliation{Gulliver, CNRS, {\'E}cole Sup{\'e}rieure de Physique et Chimie Industrielles de Paris, Paris Sciences et Lettres Research University, Paris 75005, France}

	
	\maketitle
	
	\tableofcontents


% \address{$^1$ CNRS, UMR 7600, LPTMC, F-75005, Paris, France\\
% $^2$ Sorbonne Universit\'es, UPMC Univ Paris 06, UMR 7600, LPTMC, F-75005, Paris, France}  
\section{Supplementary information for molecular dynamics simulation}

\paragraph{Molecular dynamics simulations}
We simulate a cubic water box of side $L$=6.5 nm composed of N$_w$ water molecules, N$_w$ going from   9033 to 8527 for increasing salt concentration. See a snapshot of the simulated system in Fig.~\ref{fig:5}. The 0.15 mol.l$^{-1}$ solution contains 25 ion pairs, the 0.75 mol.l$^{-1}$ solution contains 124 ion pairs and the 1.5 mol.l$^{-1}$ solution, 248 ion pairs.
We perform simulations with the TIP4p/$\epsilon$~\cite{azcatl2014}, which contains 4-interaction sites,  three point-charges, and one Lennard-Jones reference site model. The Lennard-Jones (LJ) center is placed on the oxygen. Charges are placed on the hydrogen atoms, and an additional interaction site, M, carries the negative charge. This model is derived from the TIP4p model with parameters modified to reproduce the macroscopic permittivity of water. The ions (Na$^+$ and Cl$^-$) are treated according to the force field developed in the reference~\cite{loche2021}.\par
\begin{figure}
	\includegraphics[scale=0.6]{simuvisu.png}
	\caption{Snapshot of the simulation box of an electrolyte. Red and white sticks represent TIP4p/$\epsilon$ water molecules, purple spheres chloride Cl$^-$ ions, and green spheres sodium Na$^+$ ions. This picture corresponds to a 0.75 mol.l$^{-1}$ solution. Size of the box: $L$=6.5~nm.}
	\label{fig:5}
\end{figure}
Simulations are performed using the GROMACS 2021 molecular dynamics simulation package~\cite{gromacs}, and the integration time steps are set to $\Delta t$=2 fs. Simulation boxes are periodically replicated in all directions, and long-range electrostatics are handled using the smooth particle mesh Ewald (SPME) technique. Lennard-Jones interactions are cut off at a distance $r_{\rm cut}$=0.9 nm.  A potential shift is used at the cut-off distance. All systems are coupled to a heat bath at 300 K using a v-rescale thermostat with a time constant of 0.5 ps. We use MDAnanlysis to treat the trajectories. After creating the simulation box, we perform a first energy minimization. We equilibrate the system in the NVT ensemble for 200 ps and afterward in the NPT ensemble for another 200 ps using a Berendsen barostat at 1 bar.
Production runs are performed in the NVT ensemble for 20 ns. 


\paragraph{Statistical treatment}
For the longitudinal and the transverse susceptibility, error bars are derived following the reblocking method~\cite{flyvbjerg1989}.
For the $q=0$ permittivity, we cut the trajectory in 5 statistically independent blocks, compute the permittivity of each block,  estimate the sample variance $\sigma^2$ and define the error bar as $\sqrt{\sigma^2/5}$. \par

\paragraph{Macroscopic permittivity}
Permittivity is calculated from the total system dipole moment~\cite{neuman83} $\mathbf{M}$ 
according to:
\begin{equation}
{\epsilon} -1 = \frac{\left\langle \mathbf{M} \cdot \mathbf{M} \right\rangle - \left\langle \mathbf{M} \right\rangle \cdot \left\langle \mathbf{M} \right\rangle}{3\epsilon_0 k_B T V},
\end{equation}
as implemented in the GROMACS dipoles module. $\mathbf{M}$ is the volume integral of the polarization as $\mathbf{M}=\int_Vd\vec{r} \bm{\mathcal{P}(\vec{r})}$ and $V$ is the volume of the box. Note that the ion polarization is not taken into account.
\paragraph{q-dependent susceptibilities}
To compute the $q$-dependent susceptibilities, we use the fluctuation-dissipation theorem, relating the response functions to the polarization fluctuations as follows:
\begin{equation}
\label{FDT}
\chi_{ij}(\vec{q})=\frac{\langle \mathcal{P}_i(\vec{q})\cdot\mathcal{P}_j(-\vec{q})\rangle}{\epsilon_0 k_BT}.
\end{equation}

One can express the longitudinal susceptibility as a function of the charge structure factor $S(q)$ and obtains
\begin{equation}
\chi_\parallel(q)=\frac{S(q)}{q^2 \epsilon_0 k_BT}.
\end{equation}

The charge structure factor in Fourier space can be decomposed into an intramolecular and an intermolecular part,
\begin{equation}
S(q)=S_{\rm int}(q)+S_{\rm inter}(q)\,
\end{equation}
with $S_{\rm inter}(q)$ the intermolecular contribution
\begin{equation}
S_{\rm inter}(q)=4 n_w z^2 e^2\left(h_{\rm MM}(q)+h_{\rm HH}(q)-2h_{\rm HM}(q)\right)
\end{equation}
$z$ is the valency, $e$ being the elementary charge, $n_w$ the molecular density. $h_{\rm IJ}$ is the Fourier transform of $g_{\rm IJ}(r)-1$, $g_{\rm IJ}(r)$ being the radial distribution function associated with the atoms couple IJ. 
The intramolecular contribution can be written as
\begin{equation}
S_{\rm intra}(q)=4 n_w z^2 e^2\left(\frac{\sin(q d_{\rm HH})}{q d_{\rm HH}}-4\frac{\sin(q d_{\rm HM})}{q d_{\rm HM}}+3\right)
\end{equation}
where $ d_{\rm IJ}$ is the intramolecular distance between atoms I and J. 



At low $q$, the numerical precision of $S(q)$ becomes pretty low as the function $h_{\rm IJ}(r)$ is obtained on a finite range imposed by the box size.  To solve this problem, we proceed as follows. 
For $q<$ 2.5~\AA$^{-1}$, we take into account the periodicity of the system, calculate the charge structure factor for discretized values of the wavenumber $q$, $q=2\pi/L\sqrt{n_x^2+n_y^2+n_z^2}$. We compute  directly the charge structure factor from the charge distribution $	\tilde{\rho}(q)$ in the Fourier space, 
\begin{eqnarray}
\label{rhoq}
\tilde{\rho}(q)&=&\sum_{i=1}^{Nw} ez  e^{i \vec{q} \cdot \vec{r}}\left(-2 e^{-i\vec{q}\cdot\vec{r}_{{\rm M},i}}+e^{-i\vec{q}\cdot\vec{r}_{{\rm H1},i}}+e^{-i\vec{q}\cdot\vec{r}_{{\rm H2},i}} \right)
\end{eqnarray}
where H$_{1,i}$ and H$_{2,i}$ stand for the two hydrogens and M$_i$ for the site carrying the negative charge of the molecule $i$. The charge structure factor $S(q)=\langle \tilde{\rho}(q) \tilde{\rho}(-q) \rangle/ V$ follows

\begin{eqnarray}
S(q) &=&\frac{2q_H^2}{V}\sum_{i,j, j\leq i}\Big(4 \cos(\vec{q}\cdot\vec{d}_{\rm MiMj})-2\cos(\vec{q}\cdot\vec{d}_{\rm MiH1j})-2\cos(\vec{q}\cdot\vec{d}_{\rm H1iMj})\nonumber\\&-&2\cos(\vec{q}\cdot\vec{d}_{\rm MiH2j})-2\cos(\vec{q}\cdot\vec{d}_{\rm H2iMj})+\cos(\vec{q}\cdot\vec{d}_{\rm H1iH1j})+\cos(\vec{q}\cdot\vec{d}_{\rm H2iH2j})\nonumber\\&+&\cos(\vec{q}\cdot\vec{d}_{\rm H1iH2j})+\cos(\vec{q}\cdot\vec{d}_{\rm H2iH1j})\Big)
\end{eqnarray}
where $q$ is a vector and $d_{AiAj}$ stands for $\vec{r}_{Ai}-\vec{r}_{Aj}$.


The transverse susceptibility is calculated following ref.~\cite{kornyshevtrans}.
The polarization of the medium in Fourier space
\begin{equation}
{\bf P}({\bf q})=\Sigma_j {\bf p}_j({\bf q})e^{-i {\bf q}\cdot  {\bf r}_j}
\end{equation}
can be written as a sum over the molecular polarization ${\bf p}_j({\bf q})$ of the molecule $j$, which reads as
\begin{equation}
{\bf p}_j({\bf q})=	\frac{1}{\sqrt{V}}\Sigma_\alpha \frac{e z_\alpha {\bf \delta r}_{\alpha j}}{i {\bf q}\cdot {\bf \delta r}_{\alpha j } }\left(1-e^{-i{\bf q}\cdot {\bf \delta r}_{\alpha j }}\right)
\end{equation}
with ${\bf \delta r}_{\alpha j } $ the distance between the charge $\alpha$ and the center of mass of the molecule and $V$ the volume of the simulation box. 
We then take the transverse part of the polarization ${\bf P}_\perp({\bf q})={\bf q}\times {\bf P}(q)/q$ and define the transverse susceptibility as
\begin{equation}
\label{chiperpMD}
\chi_\perp(q)=\frac{\langle{\bf P}_\perp({\bf q})\cdot{\bf P}_\perp({\bf -q})\rangle}{2k_BT\epsilon_0}.
\end{equation}

Note that we replace $\left(1-e^{-i{\bf q}\cdot {\bf \delta r}_{\alpha j }}\right)/i {\bf q}\cdot {\bf \delta r}_{\alpha j } $ by 1 for $ {\bf q}\cdot {\bf \delta r}_{\alpha j } <$ 10$^{-5}$ to prevent numerical errors.  



\section{Supplementary information for the field theory model}
\subsection{Derivation of $\chi^{w}(\vec{q})$, the susceptibility of pure water}
The dielectric susceptibility $\chi(\vec{r}-\vec{r}')$ of the system  described by Eq. (1) of the main text is defined as 

\begin{equation}\label{chiparaperp}
\begin{aligned}
	\mathcal{U}_{\rm el}[\bm{\mathcal{P}}]&=\frac{1}{2\epsilon_0}\int d\vec{r} d\vec{r}'\bm{\mathcal{P}}(\vec{r}) \cdot \chi^{w, -1}(\vec{r}-\vec{r}')\cdot \bm{\mathcal{P}}(\vec{r}'),\\
 &=\frac{1}{2\epsilon_0}\int d\vec{r} d\vec{r}'\left(\bm{\mathcal{P}}_\parallel(\vec{r}) \cdot\chi_\parallel^{w, -1}(\vec{r}-\vec{r}')\cdot \bm{\mathcal{P}}_\parallel(\vec{r}')+\bm{\mathcal{P}}_\perp(\vec{r}) \cdot\chi_\perp^{ w, -1}(\vec{r}-\vec{r}')\cdot \bm{\mathcal{P}}_\perp(\vec{r}')\right),
 \end{aligned}
\end{equation}
where we have split the polarization field into a longitudinal part $\bm{\mathcal{P}}_\parallel$ and a transverse part $\bm{\mathcal{P}}_\perp$, which respectively satisfy
$\bm{\nabla}_{\vec{r}}\times  \bm{\mathcal{P}}_\parallel(\vec{r})=0$ and $\bm{\nabla}_{\vec{r}} \cdot  \bm{\mathcal{P}}_\perp(\vec{r})=0$.
We rewrite the electrostatic energy as a functional of the longitudinal $\bm{\mathcal{P}}_\parallel(\vec{q})$ and transverse $\bm{\mathcal{P}}_\perp(\vec{q})$ polarization fields in Fourier space as
\begin{equation}
\begin{aligned}
	\mathcal{U}_{\rm el}[\bm{\mathcal{P}}]&=\frac{1}{2\epsilon_0}\int d\vec{q} \left(\bm{\mathcal{P}}_\parallel(\vec{q}) \cdot (1+K+\kappa_l q^2+\alpha q^4)\frac{q_iq_j}{q^2}\cdot \bm{\mathcal{P}}_\parallel(\vec{-\vec{q}})+\bm{\mathcal{P}}_\perp(\vec{q})\cdot (K+\kappa_t q^2)(\delta_{ij}-\frac{q_iq_j}{q^2} )\cdot \bm{\mathcal{P}}_\perp(-\vec{q})\right),
 \end{aligned}
\end{equation}
and invert the kernel $\chi^{-1,w}(\vec{q})$ to get Eq. (3) of the main text:
\begin{equation}
	\chi_{\parallel}^{\rm w}(q)=\frac{1}{1+K+\kappa_l q^2+ \alpha q^4}, \quad \chi_{\perp}^{\rm w}(q)=\frac{1}{K+\kappa_t q^2}.
\end{equation}
The $q$-dependent permittivity of the system obeys:
	\begin{equation}
	\label{epsq}
	\epsilon(q)=\frac{1}{1-\chi(q)}
	\end{equation}
	and is plotted in Fig.~(\ref{fig:6}). The $q$-dependent permittivity has two poles, and exhibits a region of negative values, the "overscreening" effect, in agreement with the results in the literature~\cite{bopp1996}. $\epsilon^{\rm w}$ tends to one, the vacuum response, for large q. The forbidden zone, $\epsilon^{\rm w}\notin [0,1[$, is shaded in red. 
	\begin{figure}
		\includegraphics{FigS4.pdf}
		\caption{$q$-dependent permittivity of water. The permittivity $\epsilon^{\rm w}(q)$ is given in Eq.~(\ref{epsq}) and plotted as a function of $q$ for $K$= 1/76,  $\kappa_l$= -0.218~\AA$^{2}$, $\alpha$=0.012~\AA$^{4}$. The forbidden zone $[0,1[$ for $\epsilon^{\rm w}$~\cite{bopp1996} is shaded in red.}
		\label{fig:6}
\end{figure}
\subsection{Derivation of $\Xi$,  the grand partition function in the grand canonical ensemble}
This subsection presents the main steps for deriving the grand partition function $\Xi$ in the grand canonical ensemble. 
We start from the partition function of the canonical ensemble, $\mathcal{Z}$,
\begin{eqnarray}
\label{Zloc}
\mathcal{Z}=\frac{1}{N_+!}\frac{1}{N_-!}\left[\prod_{i=1}^{N+}\int d\vec{r}_i^+ \right]\left[\prod_{j=1}^{N-}\int d\vec{r}_j^- \right]\int \mathcal{D}[\bm{\mathcal{P}}]e^{-\beta \mathcal{U}_{\rm conf}[\bm{ \mathcal{P}}]}
e^{-\frac{\beta}{2}\int d\vec{r} d\vec{r}'\rho_{\rm tot}(\vec{r})v(\vec{r}-\vec{r}')\rho_{\rm tot}(\vec{r}')}
\end{eqnarray} 
with $v(\vec{r}-\vec{r}')=1/4\pi \epsilon_0|\vec{r}-\vec{r}'|$ the Coulomb potential and $\rho_{\rm tot}(\vec{r})=\rho(\vec{r})-\nabla\cdot\bm{\mathcal{P}}(\vec{r})$.
We introduce an auxiliary field $\Phi$ and perform a Hubbard-Stratonovich transformation using the relation $v(\vec{r}-\vec{r}')^{-1}=-\epsilon_0 \nabla^2 \delta (\vec{r}-\vec{r}')$~\cite{levy2012}. Dropping the prefactor, we obtain
\begin{eqnarray}
\mathcal{Z}=\frac{1}{N_+!}\frac{1}{N_-!}\left[\prod_{i=1}^{N+}\int d\vec{r}_i^+ \right]\left[\prod_{j=1}^{N-}\int d\vec{r}_j^- \right]\int \mathcal{D}[\bm{\mathcal{P}}]e^{-\beta \mathcal{U}_{\rm conf}[\bm{ \mathcal{P}}]}\int \mathcal{D}[\Phi]e^{-\frac{\beta}{2}\int d\vec{r} \epsilon_0 (\nabla \Phi)^2-i\beta \int d\vec{r} \Phi(\vec{r})\left(\rho(\vec{r})-\nabla \cdot\bm{ \mathcal{P}}(\vec{r})\right)}.
\end{eqnarray}
Introducing the expression of the charge density $\rho(\vec{r})=\Sigma_{i=1}^{N+} e \delta(\vec{r}-\vec{r}_i^+)-\Sigma_{j=1}^{N-} e \delta(\vec{r}-\vec{r}_j^-)$, we obtain
\begin{eqnarray}
\mathcal{Z}&=&\int\mathcal{D}[\Phi]\frac{1}{N_+!}\left(\int d\vec{r} e^{-i\beta e \Phi(\vec{r})}\right)^{N_+}\frac{1}{N_-!}\left(\int d\vec{r} e^{i\beta e \Phi(\vec{r})}\right)^{N_-}e^{-\frac{\beta}{2}\int d\vec{r} \epsilon_0(\nabla \Phi(\vec{r}))^2}\nonumber \\
&\times& \int\mathcal{D}[\bm{\mathcal{P}}]e^{-\beta \mathcal{U}_{\rm conf}[\bm{ \mathcal{P}}]} e^{-i \beta\int d\vec{r} \Phi(\vec{r})(-\nabla \cdot \bm{ \mathcal{P}}(\vec{r}))}
\end{eqnarray}
The partition function is brought into a more manageable form by going to the grand canonical ensemble, where we get
\begin{eqnarray}
\Xi&=&\sum_{N+=0}^{\infty}\frac{e^{\beta \mu_+ N_+}}{N_+!}\left(\int d\vec{r} e^{-i \beta e \Phi(\vec{r})}\right)^{N+}
 \times \sum_{N-=0}^{\infty}\frac{e^{\beta \mu_- N_-}}{N_-!}\left(\int d\vec{r} e^{i \beta e \Phi(\vec{r})}\right)^{N-}\nonumber\\
&\times&\int \mathcal{D}[\bm{\mathcal{P}}]e^{-\beta \mathcal{U}_{\rm conf}[\bm{\mathcal{P}}]}
\int \mathcal{D}[\Phi] e^{-\frac{\beta}{2}\int d\vec{r}\epsilon_0\left(\nabla \Phi(\vec{r})\right)^2
	-i \beta\int d\vec{r} \Phi(\vec{r})\left(-\nabla \cdot {\bm{ \mathcal{P}}}(\vec{r})\right)},
\end{eqnarray}
with $\mu_i$, $(i=+,-)$, the chemical potential for cations and anions.  We introduce the field $\Psi=i\Phi$ that can be identified as the electrostatic potential~\cite{levy2012}, we perform the sums over $N_\pm$ and identify the action $F_u[\Psi,\bm{\mathcal{P}}]$, such that
\begin{equation}
\label{GrandPartFunc}
\Xi=\int \mathcal{D}[\bm{\mathcal{P}}]\,\mathcal{D}[\Psi]e^{ - \beta F_u [\bm{{\mathcal{P}}},\Psi]}. 
\end{equation}
Setting $e^{\beta \mu_{\pm}}=n$, the ionic density, we obtainf for the action:
\begin{eqnarray}
\label{action}
F_u[{\bm {\mathcal P}},\Psi]=\mathcal{U}_{\rm conf}[\bm{\mathcal{P}}] -\int d\vec{r}\Big(\frac{\epsilon_0}{2} (\nabla \Psi)^2-\Psi \nabla \cdot \bm{\mathcal{P}}-\frac{2 n}{\beta}\cosh(\beta \Psi e)\Big).
\end{eqnarray}
 
\subsection{Mean field  ($\psi$, P) for the Gaussian limit and the nonlinear model}
In this subsection, we derive  $(\psi, \bm{P})$, the fields minimizing the action $F_u$ given in Eq.~(\ref{action}). $\mathcal{U}_{\rm conf}$ equals to:
\begin{eqnarray}
\label{HnG}
\mathcal{U}_{\rm conf}[\bm{\mathcal{P}}]=\frac{1}{2\epsilon_0}\int d\vec{r}\Big[ \gamma \bm{ \mathcal{P}}(\vec{r})^4+K\bm{\mathcal{P}}(\vec{r})^2
+\kappa_l(\bm{\nabla}\cdot \bm{\mathcal{P}}(\vec{r}))^2+\kappa_t(\bm{\nabla}\times \bm{ \mathcal{P}}(\vec{r}))^2+\alpha (\bm{\nabla}
(\bm{\nabla} \cdot \bm{ \mathcal{P}}(\vec{r})))^2 \Big ].
\end{eqnarray}
The mean fields obey the following equations:
\begin{equation}
\frac{\delta F_u(\vec{P},\psi)}{ \delta \Psi}=0, \quad \frac{\delta F_u(\vec{P},\psi)}{ \delta \mathcal{P}_i}=0, \quad i=x,y,z. 
\end{equation}
The functional derivative with respect to $\Psi$ gives:
\begin{eqnarray}
\frac{\delta F_u}{ \delta \Psi}&=& \epsilon_0 \Delta \Psi -\nabla\cdot \bm{{\mathcal {P}}}-2en  \sinh(\beta e \Psi), 
\end{eqnarray}
The functional derivative with respect to ${\mathcal{P}}_i$ leads to 

\begin{eqnarray}
\frac{\delta F_u}{\delta {\mathcal{P}}_i}=\frac{1}{\epsilon_0}\Big(2\gamma{\mathcal{P}_i}{\bm{\mathcal{P}}}^2+K{\mathcal{P}_i}-\kappa_l\partial_i \nabla \cdot {\bm{\mathcal{P}}} +\kappa_t \left(\partial_i \nabla \cdot {\bm{\mathcal{P}}}-\Delta \mathcal{P}_i\right)+\alpha\partial_i \Delta \nabla \cdot {\bm{\mathcal{P}}} \Big)+\partial_i \Psi.
\end{eqnarray}
The Gaussian limit is obtained by setting $\gamma=0$.

We obtain:
\begin{equation}
	\psi=0, \quad {\bm P}=0.
\end{equation}
The mean fields are vanishing both for Gaussian and nonlinear configuration energy.
 \subsection{Susceptibility in the Gaussian limit ($\gamma$=0) for an electrolyte}
 In this section, we derive the expressions of the longitudinal $\chi_{\parallel}(q)$ and transverse $\chi_\perp(q)$ polarization susceptibility of an electrolyte. \\
 We first define the total susceptibility of the system as the 4x4 matrix
 \begin{eqnarray}
 \label{DefSusc}
\chi_{\rm tot}({\vec{r}_1-\vec{r}_2})= \left(\begin{array}{cc} \epsilon_0\chi & \chi_{P,\psi} \\ \chi_{\psi, P} & \chi_{\psi,\psi}/\epsilon_0
 \end{array}\right)({\vec{r}_1-\vec{r}_2})=\left(\begin{array}{cccc}
 \frac{\delta^2 F_u(\bm{P},\Psi)}{\delta \mathcal{P}_x(\vec{r}_1) \delta \mathcal{P}_x(\vec{r}_2) }  &  \frac{\delta^2 F_u(\bm{P},\Psi)}{\delta \mathcal{P}_x(\vec{r}_1) \delta \mathcal{P}_y(\vec{r}_2) } & \frac{\delta^2 F_u(\bm{P},\Psi)}{\delta \mathcal{P}_x(\vec{r}_1) \delta \mathcal{P}_z(\vec{r}_2) }   & \frac{\delta^2 F_u(\bm{P},\Psi)}{\delta \mathcal{P}_x(\vec{r}_1) \delta \Psi(\vec{r}_1)} \\
 \frac{\delta^2 F_u(\bm{P},\Psi)}{\delta \mathcal{P}_y(\vec{r}_1) \delta \mathcal{P}_x(\vec{r}_2) }  &  \frac{\delta^2 F_u(\bm{P},\Psi)}{\delta \mathcal{P}_y(\vec{r}_1) \delta \mathcal{P}_y(\vec{r}_2) } & \frac{\delta^2 F_u(\bm{P},\Psi)}{\delta \mathcal{P}_y(\vec{r}_1) \delta \mathcal{P}_z(\vec{r}_2) }   & \frac{\delta^2 F_u(\bm{P},\Psi)}{\delta \mathcal{P}_y(\vec{r}_1) \delta \Psi(\vec{r}_1)} \\
 \frac{\delta^2 F_u(\bm{P},\Psi)}{\delta \mathcal{P}_z(\vec{r}_1) \delta \mathcal{P}_x(\vec{r}_2) }  &  \frac{\delta^2 F_u(\bm{P},\Psi)}{\delta \mathcal{P}_z(\vec{r}_1) \delta \mathcal{P}_y(\vec{r}_2) } & \frac{\delta^2 F_u(\bm{P},\Psi)}{\delta \mathcal{P}_z(\vec{r}_1) \delta \mathcal{P}_z(\vec{r}_2) }   & \frac{\delta^2 F_u(\bm{P},\Psi)}{\delta \mathcal{P}_z(\vec{r}_1) \delta \Psi(\vec{r}_1)} \\
 \frac{\delta^2 F_u(\bm{P},\Psi)}{\delta \Psi(\vec{r}_1) \delta \mathcal{P}_x(\vec{r}_2)} &  \frac{\delta^2 F_u(\bm{P},\Psi)}{\delta \Psi(\vec{r}_1) \delta \mathcal{P}_y(\vec{r}_2)} &  \frac{\delta^2 F_u(\bm{P},\Psi)}{\delta \Psi(\vec{r}_1) \delta \mathcal{P}_z(\vec{r}_2)} & \frac{\delta^2 F_u(\bm{P},\Psi)}{\delta \Psi(\vec{r}_1) \delta \Psi(\vec{r}_2)}
 \end{array}\right) ^{-1}
 \end{eqnarray}
 with ($\bm{P},\psi$) denoting the mean fields minimizing the action $F_u$.
We perform the second variational derivatives of $F_u$ as follows:
 \begin{eqnarray}
 \label{F21G}
 \frac{\delta^2 F_u(\bm{P},\psi)}{\delta \Psi(\vec{r}) \delta \Psi(\vec{r}')}&=&\left( \epsilon_0\Delta_\vec{r}-2\Lambda\beta e^2{\rm cosh}(\beta e\psi) \right)\delta(\vec{r}-\vec{r}')\\
 \label{F22G}
 \frac{\delta^2 F_u(\bm{P},\psi)}{\delta \mathcal{P}_i(\vec{r}) \delta \mathcal{P}_j(\vec{r}')}&=&\frac{1}{\epsilon_0}\Big(K\delta_{ij}-\kappa_l \partial_i\partial_j  +\kappa_t\left(\partial_i\partial_j-\Delta \delta_{ij}\right)+\alpha\Delta \partial_i\partial_j \Big)\delta(\vec{r}-\vec{r}'),\quad (i,j)\in(x,y,z)\\
 \label{F23G}
 \frac{\delta^2 F_u(\bm{P},\psi)}{\delta \Psi(\vec{r}) \delta \mathcal{P}_i(\vec{r}')}&=&- \partial_i \delta(\vec{r}-\vec{r}'), \quad
 \frac{\delta^2 F_u(\bm{P},\psi)}{\delta \mathcal{P}_i(\vec{r}) \delta \Psi(\vec{r}')}= \partial_i \delta(\vec{r}-\vec{r}'), \quad i\in(x,y,z).
 \end{eqnarray}
 
 
We thus find for the inverse susceptibility of the system, in Fourier space,
 \begin{eqnarray}
 \label{chiinverse}
 \chi^{-1}_{\rm tot}(\vec{q})=\left(\begin{array}{cccc}
 & & & iq_x \\
 &\frac{\chi^{-1}}{\epsilon_0}(\vec{q})&  & iq_y\\
 & & &iq_z\\
 -iq_x &-iq_y&-iq_z & -(2n e^2\beta+\epsilon_0q^2)
 \end{array}\right),
 \end{eqnarray}
 with the matrix $\chi^{-1}(\vec{q})$ given by
 \begin{eqnarray}
 \label{chim1G}
\chi^{-1}_{ij}(\vec{q})=(K+\kappa_l q^2+\alpha q^4)\frac{q_iq_j}{q^2}+(K+\kappa_t q^2)\left(\delta_{ij}-\frac{q_iq_j}{q^2}\right).   
 \end{eqnarray}
 We invert the matrix $\chi_{\rm tot}^{-1}(q)$ by using a block inversion and obtain
 \begin{eqnarray}
 \chi_{\rm tot}(\vec{q})=\left(\begin{array}{cc} \epsilon_0\chi(\vec{q}) & iQ^{\top} \\
 -i Q & \frac{\chi_{\psi,\psi}(q)}{\epsilon_0}	\end{array}\right)
 \end{eqnarray}
 with $Q=(q_x,q_y,q_z)$ and
 \begin{eqnarray}
 \label{chiG}
 \chi_{ij}(\vec{q})&=&\chi_{\parallel}(q)\frac{q_iq_j}{q^2}+\chi_\perp(q)(\delta_{ij}-\frac{q_iq_j}{q^2})\\
 \label{chiparaG}
 \chi_{\parallel}(q)&=&\frac{\frac{\epsilon_w}{\lambda_D^2}+q^2}{\left(\frac{\epsilon_w}{\lambda_D^2}+q^2\right)(K+\kappa_l q^2+\alpha q^4)+q^2}, \quad
\chi_{\perp}(q)=\frac{1}{K+\kappa_tq^2}\\
 \chi_{\psi,\psi}(q)&=&-\frac{K+\kappa_l q^2+\alpha q^4}{\left(\frac{\epsilon_w}{\lambda_D^2}+q^2\right)(K+\kappa_l q^2+\alpha q^4)+q^2}
 \end{eqnarray}
where we have introduced the Debye length $\lambda_D=\sqrt{\epsilon_0\epsilon_w/2\beta n e^2}$. 
We plot the susceptibility $\chi_{\parallel}(q)$ for increasing salt concentrations ($c=n/\mathcal{N}_a$, $\mathcal{N}_a$ the Avogadro number) in Fig.~\ref{fig:2}. As we have ($\kappa_l<$0, $\alpha>$0), the denominator of $\chi_{\parallel}(q)$ will vanish for a small enough Debye length inducing a divergence of the susceptibility. For the chosen set of parameters, the divergence occurs for $c$=22 mmol.l$^{-1}$. At this concentration, the longitudinal correlations are purely oscillating, illustrating an unphysical crystalization of the medium.
 \begin{figure}
	\includegraphics{FigSI2.pdf}
	\caption{Longitudinal susceptibility of electrolytes. The susceptibility in the Gaussian limit $\chi_\parallel(q)$ given in Eq.~(\ref{chiparaG}) is plotted for increasing salt concentration. Parameters are given in the caption of Fig.~S1.}
	\label{fig:2}
\end{figure}
  \subsection{Susceptibility for Gaussian model in real space}
We express the Gaussian susceptibility $\chi$ given in Eq.~(\ref{chiparaG}) in real space.  We perform the back Fourier transform   $\hat{\chi}_{ij}(\vec{r})$=$\mathcal{F}_T(\chi_{ij}(\vec{q}))$ defined as
 \begin{equation}
 \label{chifourier}
 \hat{\chi}_{ij}(\vec{r})=\frac{1}{2\pi^3}\int_0^{2\pi} d\phi \int_0^{\pi} d\theta \sin(\theta) \int_0^\infty dq q^2 e^{iqr\cos(\theta)}\chi_{ij}(\vec{q}).
 \end{equation}
$\hat{\chi}$ is here expressed in the intrinsic basis in which the vector $\vec{r}$, joining the two correlated points, is aligned with the $\vec{e}_z$ direction of wavenumber $\vec{q}$ basis, as illustrated in Fig.~\ref{fig:3}.
 \begin{figure}
 	\includegraphics[scale=0.27]{figSIbases.jpg} % AR : 0.6
 	\caption{Illustration of the basis considered to calculate the susceptibility tensor $\chi$.}
 	\label{fig:3}
 \end{figure}
 We express the longitudinal projector $q_iq_j/q^2$ $(i,j=x,y,z)$ in the  spherical basis, 
 \begin{equation}
 \label{longq}
 \frac{q_iq_j}{q^2}=\left(\begin{array}{ccc}
 \sin(\theta)^2\cos(\phi)^2 & 	\sin(\theta)^2\cos(\phi)\sin(\phi) & 	\sin(\theta)\cos(\theta)\cos(\phi) \\
 \sin(\theta)^2\cos(\phi)\sin(\phi) & \sin(\theta)^2\sin(\phi)^2 & 	\sin(\theta)\cos(\theta)\sin(\phi) \\
 \sin(\theta)\cos(\theta)\cos(\phi) & 	\sin(\theta)\cos(\theta)\sin(\phi) & \cos(\theta)^2
 \end{array}\right),
 \end{equation}
Using Eqs.~(\ref{chiG},\ref{chiparaG}), $\hat{\chi}_{ij}(r)$ is split into two contributions: $\hat{\chi}_{ij}(r)=\mathcal{F}_T(\chi_{\parallel}(q)q_iq_j/q^2)+\mathcal{F}_T(\chi_{\perp}(q)(\delta_{ij}-q_iq_j/q^2))$. We perform the integrals in Eq.~(\ref{chifourier}) using (\ref{longq}). We obtain:
 \begin{equation}
 \label{spericcorrelation}
 \mathcal{F}_T(\chi_{\parallel}(q)q_iq_j/q^2)= \left(\begin{array}{ccc}\frac{1}{2}(I_{1,\parallel}(r)-I_{2,\parallel}(r)) &0& 0\\
 0 & \frac{1}{2}(I_{1,\parallel}(r)-I_{2,\parallel}(r))  & 0 \\
 0 & 0 & I_{2,\parallel}(r)\end{array}\right),
 \end{equation}
and


 \begin{eqnarray}
\mathcal{F}_T\left(\chi_{\perp}(q)(\delta_{ij}-q_iq_j/q^2)\right)=\left(\begin{array}{ccc}
 \frac{1}{2}\left(I_{1,\perp}(r) + I_{2,\perp}(r) \right) &0 &\\0&	\frac{1}{2}\left(I_{1,\perp}(r)+ I_{2,\perp}(r)\right) &0 \\
 0 & 0 & 	I_{1,\perp}(r) -I_{2,\perp}(r) 
 \end{array}\right).
 \end{eqnarray}
We have introduced four elementary functions ($I_{\parallel,1}$, $I_{\perp,1}$ $I_{\parallel,2}$, $I_{\perp,2}$), defined as follow:
 \begin{eqnarray}
 \label{Iexp}
 	I_{i,1}(r)&=&\frac{1}{(2 \pi)^3}\int_0^\infty dq q^2 \int_0^\pi d\theta \sin(\theta) \int_0^{2\pi} d\phi\, \chi_i(q)e^{iqr\cos(\theta)},\nonumber\\ \quad I_{i,2}(r)&=&\frac{1}{(2 \pi)^3}\int_0^\infty dq q^2\int_0^\pi d\theta\sin(\theta)\cos(\theta)^2 \int_0^{2\pi}  d\phi\, \chi_i(q)e^{iqr\cos(\theta)}, 
 \end{eqnarray}
with $\quad i=\parallel,\perp$.
  The susceptibility finally reads:
 \begin{eqnarray}
 \hat{\chi}(r)=\left(\begin{array}{ccc}
\hat{\chi}_{\perp}(r) & 0 & 0\\ 
 0 &\hat{\chi}_{\perp}(r) & 0\\
 0 &0 & \hat{\chi}_{\parallel}(r) 
 \end{array}\right)
 \end{eqnarray}
with 
 \begin{eqnarray}
 \label{chidiagr}
 \hat{\chi}_{\parallel}(r)&=&I_{2,\parallel}(r)+(I_{1,\perp}(r)-I_{2,\perp}(r)) \\
 \label{chidiagangl}
 \hat{\chi}_{\perp}(r)&=&\frac{I_{1,\parallel}(r)-I_{2,\parallel}(r)}{2}+\frac{I_{1,\perp}(r)-I_{2,\perp}(r)}{2},
 \end{eqnarray}
 in the basis $\{\vec{e}_i\}$, $(i=x,y,z)$, defined such that $\vec{r}$ is aligned with  $\vec{e}_z$, see sketch in Fig.~\ref{fig:3}. \par
The susceptibility associated with a distance vector $\vec{r}=(x,y,z))$ is  obtained by performing the following change of basis:
\begin{equation}
\label{chicart}
	\hat{\chi}_{\rm cart}=R^{-1}\cdot \hat{\chi}(r)\cdot R,
\end{equation}
$R$ is the change-of-basis matrix from Cartesian to spherical coordinates.
\subsection{One-loop free energy expansion}
In the case of nonvanishing $\gamma$, the configurational integrals cannot be performed exactly. The partition function can be expanded to the second order around the mean field point $(\bm{P},\psi)$, as
\begin{eqnarray}
\Xi&\approx&e^{-\beta F_u[\bm{P},\psi]}\int \mathcal{D}[\delta\bm{\mathcal{P}}]\, \mathcal{D}[\delta\Psi]e^{-\frac{\beta}{2}\int d\vec{r}d\vec{r}'\left(\delta \bm{\mathcal{P}}(\vec{r}), \delta\Psi(\vec{r}) \right)\cdot F_u^{(2)}({\bm P},\psi) \cdot \left(\delta \bm{\mathcal{P}}(\vec{r}), \delta\Psi(\vec{r}) \right) }\nonumber\\&=&e^{-\beta F_u[\psi,P]}\left(\beta |F_u^{(2)}|\right)^{-1/2}
\end{eqnarray}
where we have dropped the prefactor.
$F_u^{(2)}({\bm r}-{\bm r}')$ is the second functional derivative of the action and is defined as follows:
\begin{eqnarray}
F_u^{(2)}(\vec{r}-\vec{r}')=\left(\begin{array}{cc}
\frac{\delta^2 F_u(\bm{P},\psi)}{\delta \mathcal{P}_i(\vec{r}) \delta \mathcal{P}_j(\vec{r}')}    & \frac{\delta^2 F_u(\bm{P},\psi)}{\delta \mathcal{P}_i(\vec{r}) \delta \Psi(\vec{r}')} \\
\frac{\delta^2 F_u(\bm{P},\psi)}{\delta \Psi(\vec{r}) \delta \mathcal{P}_i(\vec{r}')}  & \frac{\delta^2 F_u(\bm{P},\psi)}{\delta \Psi(\vec{r}) \delta \Psi(\vec{r}')}
\end{array}\right)
\end{eqnarray}
with
\begin{eqnarray}
\label{F21}
\frac{\delta^2 F_u(\bm{P},\psi)}{\delta \Psi(\vec{r}) \delta \Psi(\vec{r}')}&=&\left( \epsilon_0\Delta_\vec{r}-2\Lambda\beta e^2{\rm cosh}(\beta e\psi) \right)\delta(\vec{r}-\vec{r}')\\
\label{F22}
\frac{\delta^2 F_u(\bm{P},\psi)}{\delta \mathcal{P}_i(\vec{r}) \delta \mathcal{P}_j(\vec{r}')}&=&\frac{1}{\epsilon_0}\Big(\left(2\gamma P_{i}^2+K\right)\delta_{ij}+4\gamma P_i P_j-\kappa_l \partial_i\partial_j  +\kappa_t\left(\partial_i\partial_j-\Delta \delta_{ij}\right)+\alpha\Delta \partial_i\partial_j \Big)\delta(\vec{r}-\vec{r}')\\
\label{F23}
\frac{\delta^2 F_u(\bm{P},\psi)}{\delta \Psi(\vec{r}) \delta \mathcal{P}_i(\vec{r}')}&=&- \partial_i \delta(\vec{r}-\vec{r}'), \quad
\frac{\delta^2 F_u(\bm{P},\psi)}{\delta \mathcal{P}_i(\vec{r}) \delta \Psi(\vec{r}')}= \partial_i \delta(\vec{r}-\vec{r}')
\end{eqnarray}
 The free energy at first in the loop expansion follows: 
\begin{equation}
\label{FreeEnergyOneLoop}
\mathcal{F}\approx F_u[{\bm P},\psi]+\frac{1}{2\beta}{\rm Tr}{\rm ln}{\beta F_u^{(2)}}[{\bm P},\psi].
\end{equation}

 \subsection{One-loop correction for the inverse susceptibility}
 In this subsection, we expand the inverse polarization susceptibility to the first loop order, $\chi^{-1}\approx \chi_{(0)}^{-1}+\chi_{(1)}^{-1}$. 
Using its expression as a function of the free energy of the system,
 \begin{eqnarray}
 \label{DefSusc}
 \frac{1}{\epsilon_0}\chi^{-1}_{ij}(\vec{r}_1-\vec{r}_2)=
 \frac{\delta^2 \mathcal{F}(\bm{P},\psi)}{\delta \mathcal{P}_i(\vec{r}_1) \delta \mathcal{P}_j(\vec{r}_2) }
 \end{eqnarray}
 and the first order expansion of $\mathcal{F}$, (Eq.  \ref{FreeEnergyOneLoop}), we obtain:
 \begin{eqnarray}
 \label{MatrDeriv}
\frac{1}{\epsilon_0}(\chi_{(1)}^{-1})_{xx}(\vec{r}_1-\vec{r}_2) & =&\frac{\delta {\rm Tr} {\rm ln} (\beta F_u^{(2)})}{\delta \mathcal{P}_x(\vec{r}_1)\delta \mathcal{P}_x(\vec{r}_2)}(\bm{P},\psi)\nonumber \\ &=&{\rm Tr}\left( \left(F_u^{(2)}\right)^{-1} \cdot \frac{\partial^2F_u^{(2)} }{\partial \mathcal{P}_x(\vec{r}_1)\partial \mathcal{P}_{x}(\vec{r}_2)}-\frac{\delta F_u^{(2)} }{\delta \mathcal{P}_x(\vec{r}_1)}\frac{\delta F_u^{(2)} }{\delta \mathcal{P}_x(\vec{r}_2)}\cdot \left(F_u^{(2)}\right)^{-2}\right)\Big(\bm{P},\psi\Big) .
 \end{eqnarray}
 The second matrix product in the right-hand term is vanishing as $\delta F_u^{(2)} / \delta \mathcal{P}_x(\vec{r}_2)(\bm{P},\psi)=0$. We thus obtain:
 \begin{equation}
 \label{MatrDerivxx0}
 \frac{\delta {\rm Tr} {\rm ln} (\beta F_u^{(2)})}{\delta \mathcal{P}_x(\vec{r}_1)\delta \mathcal{P}_{x}(\vec{r}_2)}(\bm{P},\psi) ={\rm Tr}\left( \left(F_u^{(2)}\right)^{-1} \cdot \frac{\partial^2F_u^{(2)} }{\partial \mathcal{P}_x(r_1)\partial \mathcal{P}_{x}(r_2)}\right)(\bm{P},\psi) .
 \end{equation}
We use the expression of $F_u^{(2)}$ given in Eq. (\ref{F22}) and obtain
 \begin{equation}
 \label{MatrDerivxx}
 \frac{\partial^2F_u^{(2)}(\bm{P},\psi) }{\partial \mathcal{P}_x(r_1)\partial \mathcal{P}_x(r_2)}=	\left(\begin{array}{cccc} 12 \gamma /\epsilon_0 & 0 &0 & 0 \\
 0 & 4  \gamma /\epsilon_0& 0 & 0 \\
 0 & 0 & 4  \gamma /\epsilon_0 & 0 \\
 0 & 0 & 0 & 0
 \end{array}\right)\delta(\vec{r}-\vec{r}_1)\delta(\vec{r}-\vec{r}_2)\delta(\vec{r}'-\vec{r})
 \end{equation}
 and 
 \begin{equation}
 \label{d2F2}
 \frac{\partial^2F_u^{(2)}(\bm{P},\psi) }{\partial P_x(\vec{r}_1)\partial P_y(\vec{r}_2)}=\left(\begin{array}{cccc} 0 & 4 \gamma/\epsilon_0 &0 & 0 \\
 4 \gamma/\epsilon_0 & 0 & 0 & 0 \\
 0 & 0 & 0 & 0 \\
 0 & 0 & 0 & 0
 \end{array}\right)\delta(\vec{r}-\vec{r}_1)\delta(\vec{r}-\vec{r}_2)\delta(\vec{r}'-\vec{r})
 \end{equation}
 The other matrices are easily deduced by symmetry.
 
 
 We now have to calculate the trace of the matrices in (\ref{MatrDerivxx}) by integrating over the continuous indices ${\int d\vec{r}\,d\vec{r}'}$ and summing over the discrete indices. 
 The polarization correlations depend only on the distance $u=|\vec{r}-\vec{r}'|$.
 We thus write:
 \begin{eqnarray}
\int d\vec{r} d\vec{r}' \left(F_u^{(2)}\right)^{-1} \cdot \frac{\partial^2F_u^{(2)} }{\partial \mathcal{P}_i(\vec{r}_1)\partial \mathcal{P}_j(\vec{r}_2)}= \int d\vec{r}\int du u^2\left( \int d\phi\, d\theta\sin(\theta)  \left(F_u^{(2)}\right)^{-1} \right) \frac{\partial^2F_u^{(2)} }{\partial \mathcal{P}_i(\vec{r}_1)\partial \mathcal{P}_j(\vec{r}_2)}. 
 \end{eqnarray}  
Using $\delta (\vec{r}-\vec{r}')=\delta(u)/2 \pi u^2$ and $\Big(F_u^{(2)}\Big)^{-1}(\bm{P},\psi)=\chi_{\bm{P},\psi}(\vec{r}-\vec{r}')$ given in Eq.~(\ref{DefSusc}),  
 we perform the integral over $\theta$ and $\phi$ and find for the polarization correlation:
 \begin{eqnarray}
 \label{chirbasis}
 \int_0^{2\pi}d\phi\int_0^{\pi}d\theta \sin(\theta) R^{-1}\cdot\chi_{ij}(u)\cdot R=\frac{4\pi}{3}\left(\chi_\parallel(u)+2\chi_{\perp}(u)\right)\delta_{ij} .
\end{eqnarray}
 We now evaluate Eq.(\ref{MatrDerivxx0}) and find: 
 \begin{eqnarray}
 \label{chim11}
 (\chi_{(1)}^{-1})_{xx}(\vec{r}_1-\vec{r}_2)&=&	\frac{\pi}{\beta}\int d\vec{r} \int_0^{\infty} du 4 \gamma\epsilon_0\left(\frac{10}{3} \hat{\chi}_\parallel(u)+\frac{20}{3}\hat{\chi}_\perp(u)\right)\frac{\delta(u)}{2\pi}\delta(\vec{r}-\vec{r}_1)\delta(\vec{r}-\vec{r}_2)\nonumber\\
 \label{finalres}
 &=&\frac{20 \gamma\epsilon_0}{ 3 \beta }\left(\hat{\chi}_\parallel(0)+2\hat{\chi}_\perp(0)\right)\delta(\vec{r}_1-\vec{r}_2).
 \end{eqnarray}
 Note that we find the same value for $(\chi_{(1)}^{-1})_{yy}$ and  $(\chi_{(1)}^{-1})_{zz}$  and that the cross terms are vanishing. We calculate the susceptibility at $r=0$, $\chi(0)$, using the  elementary functions defined in Eq.~(\ref{Iexp}) as follows,
 \begin{eqnarray}
 \label{i1para}
 	I_{1,\parallel}(0)&=&\frac{1}{2\pi^2}\int_0^\infty dq q^2\frac{\epsilon_w/\lambda_D^2+q^2}{(\epsilon_w/\lambda_D^2+q^2)(K+\kappa_lq^2+\alpha q^4)+q^2},\\
 	\label{i1perp}
 	I_{1,\perp}(r_c)&=&\frac{1}{2\pi^2}\int_0^{2\pi/r_c}dq\frac{q^2}{K+\kappa_t q^2}=\frac{1}{\pi K \lambda_t^2r_c}\\
 	\label{i2}
 	I_{2,\parallel}(0)&=&\frac{I_{1,\parallel}(0)}{3}, \quad I_{2,\perp}(r_c)=\frac{I_{1,\perp}(r_c)}{3}
\end{eqnarray}
  where we have introduced a cutoff length $r_c$ to remove the divergence for $I_{1/2,\perp}$ in $r=0$. \par  The Gaussian susceptibility tensor reads,
  \begin{equation}
  \label{chirm}
  	\hat{\chi}_{\parallel}(r_c)=I_{2,\parallel}(0)+I_{1,\perp}(r_c)-I_{2,\perp}(r_c),\quad \hat{\chi}_{\perp}(r_c)=\frac{1}{2}\left(I_{1,\parallel}(0)-I_{2,\parallel}(0)+I_{1,\perp}(r_c)-I_{2,\perp}(r_c)\right).
  \end{equation} 
 Using Eq. (\ref{finalres}), we obtain in Fourier space,
 \begin{equation}
 \label{deltaK}
 	\chi^{-1}_{(1)}(\vec{q})=\delta K\frac{q_iq_j}{q^2}+\delta K\left(\delta_{ij}-\frac{q_iq_j}{q^2}\right),\quad {\rm with}\quad \delta K=\frac{20\gamma\epsilon_0}{3\beta}\left(\hat{\chi}_{\parallel}(r_c)+2\hat{\chi}_{\perp}(r_c)\right).
 \end{equation}
 Finally, we obtain the expression for the inverse polarization susceptibility to first order,
 \begin{eqnarray}
 	\chi_{ij}^{-1}(\vec{q})=(K+\delta K+\kappa_l q^2+\alpha q^4)\frac{q_iq_j}{q^2}+(K+\delta K+\kappa_t q^2)\left(\delta_{ij}-\frac{q_iq_j}{q^2}\right).	
 \end{eqnarray}
 where we have used the expression of $\chi(\vec{q})$ given in Eq. (\ref{chim1G}).
  \subsection{Linear dependence of $\chi^{-1}_{(1)}(q)$ in salt concentration $c$}
The correction $\delta K$ is split into two contributions, 
\begin{equation}
	\delta K=\delta K_w +\delta K_c c+\tau(c^2)
\end{equation}  
a pure water one $\delta K_w$, and a second one, $\delta K_c c$, depending on the salt concentration and expanded linearly in $c$.

To obtain explicit expressions of $\delta K_w$ and $\delta K_c$, we expand the functions $I_{i,x}(0)$, $i=1,2$, $x=\parallel, \perp$ given in Eq.~(\ref{i1para}-\ref{i2}) linearly in $c$, using $\epsilon_w/\lambda_D^2$=$c\times2\mathcal{N}_ae^2\beta/\epsilon_0$. We obtain
\begin{eqnarray}
	I_{i,\parallel}(0)&=&	I^w_{i,\parallel}(0)+c\times I^1_{i,\parallel}(0)+\tau(c^2)\\
		I_{i,\perp}(r_c)&=&	I^w_{i,\perp}(r_c), \quad i=1,2
\end{eqnarray}
where the functions are split into a pure water contribution and a linear correction in $c$. Note that the $I_{i,\perp}$ functions do not depend on the salt concentration, the associated salt correction is thus vanishing. \\ 
The functions for pure water obey
\begin{eqnarray}
I^w_{1\parallel}(0)&=&\frac{1}{2\pi^2}\int_0^\infty dq q^2\frac{1}{1+K+\kappa_l q^2 +\alpha q^4},\quad 	I^w_{2\parallel}(0)=\frac{I_{1,\parallel(0)}^w}{3},\\
I^w_{1\perp}(r_c)&=&\frac{1}{2\pi^2}\int_0^{\frac{2\pi}{r_c}} dq q^2\frac{1}{K+\kappa_t q^2 },\quad 	I^w_{2\perp}(r_c)=\frac{I_{1,\perp}^w(r_c)}{3},
\end{eqnarray}
and the linear correction for the parallel function reads
\begin{equation}
\label{i1salt}
	I^1_{1,\parallel}(0)=\frac{1}{2\pi^2\epsilon_0}\int_0^\infty dq \frac{2\mathcal{N}_ae^2\beta}{(1+K+\kappa_lq^2+\alpha q^4)^2}, \quad I^1_{2,\parallel}(0)=\frac{I^1_{1,\parallel}(0)}{3}. 
\end{equation}
The expressions of $\hat{\chi}_{\parallel}^{\rm w}(r_c)$ and $\hat{\chi}_{\perp}^{\rm w}(r_c)$ are obtained by replacing $I_{i, \parallel}(0)$ by $I^w_{i, \parallel}(0)$ and $I_{i, \perp}(0)$ by $I^{\rm w}_{i, \perp}(0)$ in Eq.~(\ref{finalres}). Finally, we obtain:
\begin{equation}
	\delta K_w=\frac{20\gamma\epsilon_0}{3\beta}\left(\hat{\chi}^{\rm w}_{\parallel}(0)+\hat{\chi^{\rm w}}_{\perp}(r_c)\right).
\end{equation}
We set $\delta K_w$ to zero as it is included in the fitted parameter $K$. Using the expressions for $\delta K$ in Eq.~(\ref{deltaK}) and ($\hat{\chi}_{\perp}$, $ \hat{\chi}_{\parallel}$) in Eq.~(\ref{chirm}), we obtain
\begin{equation}
	\delta K_c=\frac{20\gamma\epsilon_0}{3\beta}I^1_{1,\parallel}(0).
\end{equation}
To obtain more physical insight, we express the function $I^1_{1,\parallel}(0)$ as a function of the characteristic lengths of the problem.
The longitudinal susceptibility $\chi^{\rm w}_{\parallel}(q)$  is associated with two correlation lengths: a longitudinal decay, $\lambda_d$ and an oscillation length $\lambda_o$, defined as the imaginary and real part of the inverse of the poles of the function. The transverse susceptibility $\chi^{\rm w}_{\perp}(q)$  is associated with a decay length $\lambda_t$
which is the inverse of its pole. We can express these lengths as functions of the parameters of the model. Their expressions obey:
\begin{equation}
\label{longueur}
\lambda_d=\frac{2\sqrt{\alpha}}{\sqrt{2\sqrt{\alpha(1+K)}+\kappa_l}}, \quad \lambda_t=\sqrt{\frac{\kappa_t}{K}}, \quad \lambda_o=\frac{4\pi\sqrt{\alpha}}{\sqrt{2\sqrt{\alpha(1+K)}-\kappa_l}}.
\end{equation}
Using the estimated values of the parameters, we get a longitudinal decay length $\lambda_d$=4.7~\AA, an oscillating length $\lambda_o$=2.1~\AA, and a transverse decay length, $\lambda_t$=1.05\AA.\par
We perform the integral in Eq.~(\ref{i1salt}) and obtain
\begin{equation}
	I_{1,\parallel}^1(0)=\frac{\beta \mathcal{N}_a e^2(\epsilon_w-1)^2}{64\pi \epsilon_0\epsilon_w}\frac{(4\pi^2\lambda_d^2+\lambda_o^2)(4\pi^2\lambda_d^2+5\lambda_o^2)}{\lambda_d\lambda_o^4}
\end{equation}
after having expressed $K$, $\kappa_l$, $\alpha$ as functions of $\epsilon_w$, $\lambda_o$, $\lambda_d$ by inverting Eq.~(\ref{longueur}).
We deduce that,
\begin{equation}
	\delta K_c=\gamma\frac{5 \mathcal{N}_a e^2(\epsilon_w-1)^2}{24\pi \epsilon_w}\frac{(4\pi^2\lambda_d^2+\lambda_o^2)(4\pi^2\lambda_d^2+5\lambda_o^2)}{\lambda_d\lambda_o^4}
\end{equation}
Expanding $\epsilon(c)=1+1/(K+\delta K_c c)$ to linear order in $c$, we obtain 
for the permittivity of the electrolytes
\begin{equation}
	\epsilon(c)=\epsilon_w-\frac{\delta K_c}{K^2}c.
\end{equation}
The inverse susceptibility can thus be written as:
 \begin{eqnarray}
\chi^{-1}(q)&=&\chi^{-1}_{(0)}(q)+\chi^{-1}_{(1)}(q)\nonumber\\
&=&(K+\delta K_c c+\kappa_l q^2+\alpha q^4)\frac{q_iq_j}{q^2}+(K+\delta K_cc+\kappa_t q^2)\left(\delta_{ij}-\frac{q_iq_j}{q^2}\right).	
\end{eqnarray}
Using Eq.~(\ref{chiinverse}) and following the derivation detailed in S2.3, we find:
\begin{equation}
\label{oneloopchi}
	\chi_\parallel(q)=\frac{\frac{\epsilon_w}{\lambda_D^2}+q^2}{\left(\frac{\epsilon_w}{\lambda_D^2}+q^2\right)(K+\delta K_c c+\kappa_l q^2+\alpha q^4)+q^2}.
\end{equation}
We plot the renormalized susceptibility $\chi_\parallel(q)$ for increasing salt concentration in Fig.~\ref{fig:4} (solid lines). By comparing the susceptibilities obtained for the
Gaussian model (dashed lines), we see that the enhancement of the
pseudo-resonant peak at $q$=3~\AA$^{-1}$ is attenuated but not canceled by the one-loop correction.
 \begin{figure}
	\includegraphics{FigSI3.pdf}
	\caption{Renormalized longitudinal susceptibility compared with the Gaussian predictions. The susceptibility $\chi_\parallel(q)$ given in Eq. (\ref{oneloopchi}) is plotted for increasing concentration with $\delta K_c$=0.028 mol$^{-1}$.L (solid lines) and compared to the Gaussian susceptibility given in Eq.~(\ref{chiparaG}) (dashed lines). Other parameters are given in the caption of Fig.~S1.}
	\label{fig:4}
\end{figure}

% Bibliographies
% \bibliographystyle{unsrt}
% \bibliography{references}
\medskip
\bibliographystyle{unsrt}
\bibliography{RPfinal}

\end{document}