\documentclass[reprint, amsmath,amssymb, onecolumn, aps, linenumbers]{revtex4}
\pdfoutput=1

\usepackage{epsfig}
\usepackage{hyperref}
\usepackage{amsfonts}
\usepackage{amsmath}
\usepackage{color}
\usepackage{mathtools}
\usepackage{siunitx}


\usepackage{graphicx}
\usepackage{float}
\usepackage{bm}
\usepackage[dvipsnames]{xcolor}
\newcommand{\HB}[1]{\textcolor{SkyBlue}{(HB: #1)}}
\renewcommand{\vec}[1]{\mathbf{#1}}

% Usual (decimal) numbering
\renewcommand{\thesection}{\arabic{section}}
\renewcommand{\thesubsection}{\thesection.\arabic{subsection}}
\renewcommand{\thesubsubsection}{\thesubsection.\arabic{subsubsection}}

\renewcommand{\thesection}{S\arabic{section}}
\renewcommand{\thefigure}{S\arabic{figure}}
\renewcommand{\thetable}{S\arabic{table}}
\renewcommand{\theequation}{S\arabic{equation}}

\usepackage{verbatim}
\immediate\write18{texcount -tex -sum  main.tex > \jobname.wordcount.tex}

\begin{document}
	\title{Supplementary Materials for\\Dielectric properties of aqueous electrolytes at the nanoscale}
	
\author{Maximilian R. Becker}
\affiliation{Fachbereich Physik, Freie Universität Berlin, Arnimallee 14, Berlin, 14195, Germany}

\author{Philip Loche}
\affiliation{Laboratory of Computational Science and Modeling, IMX, École Polytechnique Fédérale de Lausanne, 1015 Lausanne, Switzerland}
\affiliation{Fachbereich Physik, Freie Universität Berlin, Arnimallee 14, Berlin, 14195, Germany}

\author{ Roland R. Netz}
\email[]{rnetz@physik.fu-berlin.de}
\affiliation{Fachbereich Physik, Freie Universität Berlin, Arnimallee 14, Berlin, 14195, Germany}


\author{Douwe Jan Bonthuis}
\affiliation{Institute of Theoretical and Computational Physics, Graz University of Technology, Graz, Austria}

\author{Dominique Mouhanna}
\affiliation{Sorbonne Universit{\'e}, CNRS, Laboratoire de Physique Th{\'e}orique de la Mati{\`e}re Condens{\'e}e (LPTMC, UMR 7600), F-75005 Paris, France}

\author{H{\'e}l{\`e}ne Berthoumieux}
\email[]{helene.berthoumieux@espci.fr}
\affiliation{Gulliver, CNRS, {\'E}cole Sup{\'e}rieure de Physique et Chimie Industrielles de Paris, Paris Sciences et Lettres Research University, Paris 75005, France}

	
	\maketitle
	
	\tableofcontents


% \address{$^1$ CNRS, UMR 7600, LPTMC, F-75005, Paris, France\\
% $^2$ Sorbonne Universit\'es, UPMC Univ Paris 06, UMR 7600, LPTMC, F-75005, Paris, France}  


\section{Supplementary information for the field theory model}
\subsection{Model for nonlocal dielectric properties of water}
We consider a field theory (FT) model describing the dielectric properties of water at the nanoscale.
The electrostatic energy of the system is written as a functional of the polarization $\bm{\mathcal{P}}$ as follows~\cite{maggs2006}:
\begin{eqnarray}
\label{Hframework}
\mathcal{U}_{\rm el}[\bm{\mathcal{P}}]=\frac{1}{2}\int d\vec{r} d\vec{r}' \frac{ \nabla\cdot \bm{ \mathcal{P}}(\vec{r}) \nabla\cdot \bm{\mathcal{P}}(\vec{r}')}{4\pi \epsilon_0|\vec{r}-\vec{r}'|} 
+\mathcal{U}^{\rm G}_{\rm conf}[\bm{\mathcal{P}}] \ .
\end{eqnarray}
The first term corresponds to the bare Coulomb interactions between the partial charges $-\bm{\nabla}\cdot\bm{\mathcal{P}}(\vec{r})$ of the fluid and the second term $\mathcal{U}^{\rm G}_{\rm conf}$ - where the index G stands for Gaussian - is a phenomenological configuration energy. 
It can be written following a Landau-Ginzburg approach. We consider it here up to the second spatial derivative in the polarization $\bm{\mathcal{P}}$ to reproduce the dielectric response of water and set
\begin{eqnarray}
\label{Hpw}
\mathcal{U}^{\rm G}_{\rm conf}[\bm{\mathcal{P}}]&=&\frac{1}{2\epsilon_0}\int d\vec{r}d\vec{r}'\bm{\mathcal{P}}(\vec{r})\cdot K(\vec{r}-\vec{r}')\cdot \bm{\mathcal{P}}(\vec{r}')\quad {\rm with} \nonumber\\
\bm{\mathcal{P}}(\vec{r})K(\vec{r}-\vec{r}')\bm{ \mathcal{P}}(\vec{r}')&=&	\Big [ K \bm{ \mathcal{P}}(\vec{r})^2
+\kappa_l(\bm{\nabla}\cdot \bm{ \mathcal{P}}(\vec{r}))^2+\kappa_t(\bm{\nabla}\times \bm{ \mathcal{P}}(\vec{r}))^2+ \alpha (\bm{\nabla}
(\bm{\nabla} \cdot \bm{ \mathcal{P}}(\vec{r}))  )^2 \Big ]\delta(\vec{r}-\vec{r}') .  
\end{eqnarray}
Note that as we consider bulk water which is homogeneous and isotropic, the two-point kernel $K$ depends simply on the distance $\vec{r}-\vec{r}'$.
We introduce the two-point dielectric susceptibility $\chi^{\rm w}(\vec{r}-\vec{r}')$ defined as: 
\begin{equation}
\mathcal{U}_{\rm el} [\bm{\mathcal{P}}]=\frac{1}{2\epsilon_0}\int d\vec{r}d\vec{r}'\bm{\mathcal{P}}(\vec{r})\left(\chi^{\rm w}(\vec{r}-\vec{r}')\right)^{-1}\bm{\mathcal{P}}(\vec{r}')
\end{equation}
where the superscript w stands for water considered here as a pure solvent.
The matrix $\chi^{\rm w}$ can be decomposed in Fourier space into a longitudinal $\chi_\parallel^{\rm w}$ and a transverse $\chi_\perp^{\rm w}$ function, such that
\begin{eqnarray}
\label{chiwat}
&	&\chi^{\rm w}_{ij}(\vec{q})=\chi_{\parallel}^{\rm w}(q)\frac{q_iq_j}{q^2}+\chi^{\rm w}_\perp(q)(\delta_{ij}-\frac{q_iq_j}{q^2})\nonumber\\
&	&\chi^{\rm w}_\parallel(q)=\frac{1}{1+K+\kappa_l q^2 +\alpha q^4}, \quad \chi_\perp^{\rm w}(q)=\frac{1}{K+\kappa_t q^2}.
\end{eqnarray}
\par
The figure \ref{fig:1} shows the MD simulated longitudinal a) and transverse b) response function for the 3 charge water model TIP4p/$\epsilon$~\cite{azcatl2014} (blue markers). MD data show an overesponse of the system $\chi^{\rm w}_{\parallel}(q)>$1 around $q$=3~\AA$^{-1}$, which corresponds to a negative permittivity, as $\epsilon(q)$ obeys $\epsilon(q)=1/(1-\chi^{\rm w}_{\parallel}(q))$~\cite{bopp1996}. This overscreening phenomenon can be attributed to the H-bond network structuring water at short range~\cite{kornyshev1997}. Note that the secondary peak for $q>$4~\AA$^{-1}$ corresponds to intramolecular correlations and to  length scales that are not addressed in this study. The
transverse susceptibility  is also extracted from MD simulations and shows a spectrum of simpler shape, as it monotonously decays. \par
The minimal model given in Eq.~(\ref{Hpw}) can reproduce these main features. For the longitudinal susceptibility, we consider the case $(\kappa_l<0$, $\alpha>0)$. The value of the parameters  $K$, $\kappa_l$, and $\alpha$,  are adjusted to fit the bulk value, $\chi^{\rm w}_{\parallel}(0)$, the position and the value of the maximum of $\chi^{\rm w}_{\parallel}(q)$ of the the MD simulated response. 
The parameter $\kappa_t$ is chosen to fit the decay of MD data. Fig.~\ref{fig:1} shows the FT response functions (black line). The values of the parameters are given in the caption. \par 
The longitudinal susceptibility $\chi^{\rm w}_{\parallel}(q)$  is associated with two correlation lengths: a longitudinal decay, $\lambda_d$ and an oscillation length $\lambda_o$, defined as the imaginary and real part of the inverse of the poles of the function. The transverse susceptibility $\chi^{\rm w}_{\perp}(q)$  is associated with a decay length $\lambda_t$
which is the inverse of its pole. Their expressions obey:
\begin{equation}
\label{longueur}
\lambda_d=\frac{2\sqrt{\alpha}}{\sqrt{2\sqrt{\alpha(1+K)}+\kappa_l}}, \quad \lambda_t=\sqrt{\frac{\kappa_t}{K}}, \quad \lambda_o=\frac{4\pi\sqrt{\alpha}}{\sqrt{2\sqrt{\alpha(1+K)}-\kappa_l}}.
\end{equation}
The polarization-polarization correlation function in the real space can be derived by Fourier transforming the susceptibility given in Eq.(\ref{chiwat}) and has been studied in a previous work~\cite{berthoumieux2015}.
Using the estimated values of the parameters we get a longitudinal decay length $\lambda_d$=4.7~\AA, an oscillating length $\lambda_o$=2.1~\AA, and a transverse decay length, $\lambda_t$=1.05\AA.
 \begin{figure}
 	\includegraphics{FigSI1.pdf}
	\caption{MD simulated and FT derived susceptibilities for bulk water. (a) Longitudinal and (b) transverse susceptibilities in
	Fourier space from FT model Eq.~(\ref{chiwat}) and from MD simulations (details given in S2). The parameter values of the FT
	model are $K$= 1/76,  $\kappa_l$= -0.218~\AA$^{-2}$ , $\alpha$=0.012~\AA$^{-4}$ and $\kappa_t$=0.013~\AA$^{-2}$.}
	\label{fig:1}
\end{figure}

\subsection{Derivation of $\Xi$,  the partition function in the grand canonical ensemble}
In this subsection, we present the main steps of the derivation of the partition function $\Xi^{\rm G}$ in the grand canonical ensemble. 
We start from the partition function of the canonical ensemble, $\mathcal{Z}^{\rm G}$,
\begin{eqnarray}
\label{Zloc}
\mathcal{Z}^{\rm  G}=\frac{1}{N_+!}\frac{1}{N_-!}\left[\prod_{i=1}^{N+}\int d\vec{r}_i^+ \right]\left[\prod_{j=1}^{N-}\int d\vec{r}_j^- \right]\int \mathcal{D}[\bm{\mathcal{P}}]e^{-\beta \mathcal{U}^{\rm G}_{\rm conf}[\bm{ \mathcal{P}}]}
e^{-\frac{\beta}{2}\int d\vec{r} d\vec{r}'\rho_{\rm tot}(\vec{r})v(\vec{r}-\vec{r}')\rho_{\rm tot}(\vec{r}')}
\end{eqnarray} 
with $v(\vec{r}-\vec{r}')=1/4\pi \epsilon_0|\vec{r}-\vec{r}'|$, the Coulomb potential and $\rho_{\rm tot}(\vec{r})=\rho(\vec{r})-\nabla\cdot\bm{\mathcal{P}}(\vec{r})$.
We introduce an auxiliary field $\Phi$ and perform a Hubbard-Stratonovich transformation using the relation $v(\vec{r}-\vec{r}')^{-1}=-\epsilon_0 \nabla^2 \delta (\vec{r}-\vec{r}')$~\cite{levy2012}. Dropping the prefactor, we get
\begin{eqnarray}
\mathcal{Z}^{\rm  G}=\frac{1}{N_+!}\frac{1}{N_-!}\left[\prod_{i=1}^{N+}\int d\vec{r}_i^+ \right]\left[\prod_{j=1}^{N-}\int d\vec{r}_j^- \right]\int \mathcal{D}[\bm{\mathcal{P}}]e^{-\beta \mathcal{U}^{\rm G}_{\rm conf}[\bm{ \mathcal{P}}]}\int \mathcal{D}[\Phi]e^{-\frac{\beta}{2}\int d\vec{r} \epsilon_0 (\nabla \Phi)^2-i\beta \int d\vec{r} \Phi(\vec{r})\left(\rho(\vec{r})-\nabla \cdot\bm{ \mathcal{P}}(\vec{r})\right)}.
\end{eqnarray}
Introducing the expression of the charge density $\rho(\vec{r})=\Sigma_{i=1}^{N+} e \delta(\vec{r}-\vec{r}_i^+)-\Sigma_{j=1}^{N-} e \delta(\vec{r}-\vec{r}_j^-)$, we get,
\begin{eqnarray}
\mathcal{Z}^{\rm  G}&=&\int\mathcal{D}[\Phi]\frac{1}{N_+!}\left(\int d\vec{r} e^{-i\beta e \Phi(\vec{r})}\right)^{N_+}\frac{1}{N_-!}\left(\int d\vec{r} e^{i\beta e \Phi(\vec{r})}\right)^{N_-}e^{-\frac{\beta}{2}\int d\vec{r} \epsilon_0(\nabla \Phi(\vec{r}))^2}\nonumber \\
&\times& \int\mathcal{D}[\bm{\mathcal{P}}]e^{-\beta \mathcal{U}^{\rm G}_{\rm conf}[\bm{ \mathcal{P}}]} e^{-i \beta\int d\vec{r} \Phi(\vec{r})(-\nabla \cdot \bm{ \mathcal{P}}(\vec{r}))}
\end{eqnarray}
The partition function is brought into a more manageable form by going to the grand canonical ensemble, where we get
\begin{eqnarray}
\Xi^{\rm  G}&=&\sum_{N+=0}^{\infty}\frac{e^{-\beta \mu_+ N_+}}{N_+!}\left(\int d\vec{r} e^{-i \beta e \Phi(\vec{r})}\right)^{N+}
 \times \sum_{N-=0}^{\infty}\frac{e^{-\beta \mu_- N_-}}{N_-!}\left(\int d\vec{r} e^{i \beta e \Phi(\vec{r})}\right)^{N-}\nonumber\\
&\times&\int \mathcal{D}[\bm{\mathcal{P}}]e^{-\beta \mathcal{U}^{\rm G}_{\rm conf}[\bm{\mathcal{P}}]}
\int \mathcal{D}[\Phi] e^{-\frac{\beta}{2}\int d\vec{r}\epsilon_0\left(\nabla \Phi(\vec{r})\right)^2
	-i \beta\int d\vec{r} \Phi(\vec{r})\left(-\nabla \cdot {\bm{ \mathcal{P}}}(\vec{r})\right)},
\end{eqnarray}
with $\mu_i$, $(i=+,-)$, the chemical potential for cations and anions.  We introduce the field $\Psi=i\Phi$ that can be identified as the electrostatic potential~\cite{levy2012}, we perform the sums over $N_\pm$ and identify the action $F^{\rm G}_u[\Psi,\bm{\mathcal{P}}]$, such that
\begin{equation}
\label{GrandPartFunc}
\Xi^{\rm G}=\int \mathcal{D}[\bm{\mathcal{P}}]\,\mathcal{D}[\Psi]e^{ - \beta F_u^{\rm  G}[\bm{{\mathcal{P}}},\Psi]}. 
\end{equation}
Setting $e^{-\beta \mu_{\pm}}=n$, the ionic density, we obtain
\begin{eqnarray}
\label{action}
F_u^{\rm  G}[{\bm {\mathcal P}},\Psi]=\mathcal{U}^{\rm  G}_{\rm conf}[\bm{\mathcal{P}}] -\int d\vec{r}\Big(\frac{\epsilon_0}{2} (\nabla \Psi)^2-\Psi \nabla \cdot \bm{\mathcal{P}}-\frac{2 n}{\beta}\cosh(\beta \Psi e)\Big).
\end{eqnarray}
 
\subsection{Mean field  ($\psi$, P) for the linear and nonlinear models}
In this subsection, we derive  $(\psi, \bm{P})$, the fields minimizing the action $F^{\rm  G}_u$ given in Eq.~(\ref{action}), and the action $F_u$ obtained by replacing $\mathcal{U}_{\rm conf}^{\rm G}$ in Eq.~(\ref{action}) by the non Gaussian  configuration energy, 
\begin{eqnarray}
\label{HnG}
\mathcal{U}_{\rm conf}[\bm{\mathcal{P}}]=\frac{1}{2\epsilon_0}\int d\vec{r}\Big[ \gamma (\bm{ \mathcal{P}}(\vec{r})^2+P_0^2)^2
+\kappa_l(\bm{\nabla}\cdot \bm{\mathcal{P}}(\vec{r}))^2+\kappa_t(\bm{\nabla}\times \bm{ \mathcal{P}}(\vec{r}))^2+\alpha (\bm{\nabla}
(\bm{\nabla} \cdot \bm{ \mathcal{P}}(\vec{r})))^2 \Big ].
\end{eqnarray}
The mean fields obey the following equations:
\begin{equation}
\frac{\delta F^k_u}{ \delta \Psi}\big\vert{ \bm{\mathcal{P}}}=0, \quad \frac{\delta F^k_u}{ \delta \mathcal{P}_i}\big\vert{ \Psi }=0, \quad k={\rm G}, \emptyset, \quad i=x,y,z. 
\end{equation}
The functional derivative with respect to $\Psi$ gives:
\begin{eqnarray}
\frac{\delta F^k_u}{ \delta \Psi}&=& \epsilon_0 \Delta \Psi -\nabla\cdot \bm{{\mathcal {P}}}-2en  \sinh(\beta e \Psi), 
\end{eqnarray}
The functional derivative with respect to ${\mathcal{P}}_i$ leads to 
\begin{eqnarray}
\frac{\delta F^{\rm G}_u}{\delta {\mathcal{P}}_i}=\frac{1}{\epsilon_0}\Big(K{\mathcal{P}_i}-\kappa_l\partial_i \nabla \cdot {\bm{\mathcal{P}}} +\kappa_t \left(\partial_i \nabla \cdot {\bm{\mathcal{P}}}-\Delta {\mathcal P}_i\right)+\alpha\partial_i \Delta \nabla \cdot {\bm{\mathcal{P}}} \Big)+\partial_i \Psi\,
\end{eqnarray}
or the Gaussian action, $F^{\rm  G}_u$, and to 

\begin{eqnarray}
\frac{\delta F_u}{\delta {\mathcal{P}}_i}=\frac{1}{\epsilon_0}\Big(2\gamma{\mathcal{P}_i}\left ({\bm{\mathcal{P}}}^2+P_0^2\right)-\kappa_l\partial_i \nabla \cdot {\bm{\mathcal{P}}} +\kappa_t \left(\partial_i \nabla \cdot {\bm{\mathcal{P}}}-\Delta {\mathcal P}_i\right)+\alpha\partial_i \Delta \nabla \cdot {\bm{\mathcal{P}}} \Big)+\partial_i \Psi\,
\end{eqnarray}
for the non Gaussian one, $F_u$:

In the two cases, we obtain
\begin{equation}
	\psi=0, \quad {\bm P}=0.
\end{equation}
The mean fields are vanishing both for Gaussian and non Gaussian configuration energy.
 \subsection{Gaussian susceptibility for an electrolyte}
 In this section we derive the expressions of the longitudinal $\chi^{\rm G}_{\parallel}(q)$ and transverse $\chi^{\rm G}_\perp(q)$ polarization susceptibility of an electrolyte. \\
 We first define the total susceptibility of the system as
 \begin{eqnarray}
 \label{DefSusc}
\chi_{\rm tot}({\vec{r}_1-\vec{r}_2})= \left(\begin{array}{cc} \epsilon_0\chi^{\rm G} & \chi^{\rm G}_{P,\psi} \\ \chi^{\rm G}_{\psi, P} & \chi^{\rm G}_{\psi,\psi}/\epsilon_0
 \end{array}\right)({\vec{r}_1-\vec{r}_2})= \left(\begin{array}{cc}
 \frac{\delta^2 F_u^{\rm  G}(\bm{P},\Psi)}{\delta \mathcal{P}_i(\vec{r}_1) \delta \mathcal{P}_j(\vec{r}_2) }   & \frac{\delta^2 F_u^{\rm  G}(\bm{P},\Psi)}{\delta \mathcal{P}_i(\vec{r}_1) \delta \Psi(\vec{r}_1)} \\
 \frac{\delta^2 F_u^{\rm  G}(\bm{P},\Psi)}{\delta \Psi(\vec{r}_1) \delta \mathcal{P}_j(\vec{r}_2)}  & \frac{\delta^2 F_u^{\rm  G}(\bm{P},\Psi)}{\delta \Psi(\vec{r}_1) \delta \Psi(\vec{r}_2)}
 \end{array}\right) ^{-1}
 \end{eqnarray}
 with ($\bm{P},\psi$)  the mean fields minimizing the action $F^{\rm G}_u$.
Using the expressions given in Eq.~(\ref{chiwat}), we find for the inverse susceptibility of the system, in Fourier space,
 \begin{eqnarray}
 \chi^{-1}_{\rm tot}(\vec{q})=\left(\begin{array}{cccc}
 & & & iq_x \\
 &\frac{\chi^{-1\rm G}}{\epsilon_0}(\vec{q})&  & iq_y\\
 & & &iq_z\\
 -iq_x &-iq_y&-iq_z & -(2n e^2\beta+\epsilon_0q^2)
 \end{array}\right),
 \end{eqnarray}
 with the matrix $\chi^{-1\rm G}(\vec{q})$ given by
 \begin{eqnarray}
 \label{chim1G}
\chi^{-1,\rm{G}}_{ij}(\vec{q})=(K+\kappa_l q^2+\alpha q^4)\frac{q_iq_j}{q^2}+(K+\kappa_t q^2)\left(\delta_{ij}-\frac{q_iq_j}{q^2}\right).   
 \end{eqnarray}
 We inverse the matrix $\chi_{\rm tot}^{-1}(q)$ by using a block inversion and obtain
 \begin{eqnarray}
 \chi_{\rm tot}(\vec{q})=\left(\begin{array}{cc} \epsilon_0\chi^{\rm G}(\vec{q}) & iQ^{\top} \\
 -i Q & \frac{\chi_{\psi,\psi}(q)}{\epsilon_0}	\end{array}\right)
 \end{eqnarray}
 with $Q=(q_x,q_y,q_z)$ and
 \begin{eqnarray}
 \label{chiG}
 \chi^{\rm G}_{ij}(\vec{q})&=&\chi_{\parallel}^{\rm G}(q)\frac{q_iq_j}{q^2}+\chi^{\rm G}_\perp(q)(\delta_{ij}-\frac{q_iq_j}{q^2})\\
 \label{chiparaG}
 \chi^{ \rm G}_{\parallel}(q)&=&\frac{\frac{\epsilon_w}{\lambda_D^2}+q^2}{\left(\frac{\epsilon_w}{\lambda_D^2}+q^2\right)(K+\kappa_l q^2+\alpha q^4)+q^2}, \quad
\chi^{\rm G}_{\perp}(q)=\frac{1}{K+\kappa_tq^2}\\
 \chi_{\psi,\psi}(q)&=&-\frac{K+\kappa_l q^2+\alpha q^4}{\left(\frac{\epsilon_w}{\lambda_D^2}+q^2\right)(K+\kappa_l q^2+\alpha q^4)+q^2}
 \end{eqnarray}
where we have introduced the Debye length $\lambda_D=\sqrt{\epsilon_0\epsilon_w/2\beta n e^2}$. 
We plot the susceptibility $\chi_{\parallel}^{\rm G}(q)$ for increasing concentrations ($c=n/\mathcal{N}_a$, $\mathcal{N}_a$ the Avogadro number) in Fig.~\ref{fig:2}. As we have ($\kappa_l<$0, $\alpha>$0) the denominator of $\chi_{\parallel}^{\rm G}(q)$ will vanish for a small enough Debye length inducing a divergence of the susceptibility. See Fig.~\ref{fig:2} , grey dashed line. For the chosen set of parameters, the divergence occurs for $c$=22 mmol.l$^{-1}$. At this concentration, the longitudinal correlations are purely oscillating illustrating an unphysical crystalization of the medium.
 \begin{figure}
	\includegraphics{FigSI2.pdf}
	\caption{Longitudinal susceptibility of electrolytes. The Gaussian susceptibility $\chi_\parallel^{\rm G}(q)$ given in Eq.~(\ref{chiparaG}), for increasing concentration. Parameters are given in caption of Fig.~S1.}
	\label{fig:2}
\end{figure}
  \subsection{Susceptibility for Gaussian model in real space}
We express the Gaussian susceptibility $\chi^{\rm G}$ given in Eq.~(\ref{chiparaG}) in real space.  We perform the Fourier transform   $\chi_{ij}^{\rm G}(\vec{r})$=$\mathcal{F}_T(\chi_{ij}^{\rm G}(\vec{q}))$ defined as
 \begin{equation}
 \label{chifourier}
 \chi^{\rm G}_{ij}(\vec{r})=\frac{1}{2\pi^3}\int_0^{2\pi} d\phi \int_0^{\pi} d\theta \sin(\theta) \int_0^\infty dq q^2 e^{iqr\cos(\theta)}\chi^{\rm G}_{ij}(\vec{q})
 \end{equation}
$\chi^{\rm G}$ is here expressed in the intrinsic basis in which the vector $\vec{r}$, joining the two correlated points, is aligned with the $\vec{e}_z$ direction of wavemode $\vec{q}$ basis, as illustrated in Fig.~\ref{fig:3}.
 \begin{figure}
 	\includegraphics[scale=0.27]{figSIbases.jpg} % AR : 0.6
 	\caption{Illustration of the basis considered to calculate the susceptibility tensor $\chi^{\rm G}$.}
 	\label{fig:3}
 \end{figure}
 We develop the longitudinal projector $q_iq_j/q^2$ $(i,j=x,y,z)$ in the  spherical basis, 
 \begin{equation}
 \label{longq}
 \frac{q_iq_j}{q^2}=\left(\begin{array}{ccc}
 \sin(\theta)^2\cos(\phi)^2 & 	\sin(\theta)^2\cos(\phi)\sin(\phi) & 	\sin(\theta)\cos(\theta)\cos(\phi) \\
 \sin(\theta)^2\cos(\phi)\sin(\phi) & \sin(\theta)^2\sin(\phi)^2 & 	\sin(\theta)\cos(\theta)\sin(\phi) \\
 \sin(\theta)\cos(\theta)\cos(\phi) & 	\sin(\theta)\cos(\theta)\sin(\phi) & \cos(\theta)^2
 \end{array}\right),
 \end{equation}
Using Eqs.~(\ref{chiG},\ref{chiparaG}), $\chi_{ij}^{\rm G}(r)$ is splitted into two contributions: $\chi_{ij}^{\rm G}(r)=\mathcal{F}_T(\chi_{\parallel}^{\rm G}(q)q_iq_j/q^2)+\mathcal{F}_T(\chi_{\perp}^{\rm G}(q)(\delta_{ij}-q_iq_j/q^2))$. We perform the integrals in Eq.~(\ref{chifourier}) using (\ref{longq}). We get:
 \begin{equation}
 \label{spericcorrelation}
 \mathcal{F}_T(\chi_{\parallel}^{\rm G}(q)q_iq_j/q^2)= \left(\begin{array}{ccc}\frac{1}{2}(I_{1,\parallel}(r)-I_{2,\parallel}(r)) &0& 0\\
 0 & \frac{1}{2}(I_{1,\parallel}(r)-I_{2,\parallel}(r))  & 0 \\
 0 & 0 & I_{2,\parallel}(r)\end{array}\right),
 \end{equation}
and


 \begin{eqnarray}
\mathcal{F}_T\left(\chi_{\perp}^{\rm G}(q)(\delta_{ij}-q_iq_j/q^2)\right)=\left(\begin{array}{ccc}
 \frac{1}{2}\left(I_{1,\perp}(r) + I_{2,\perp}(r) \right) &0 &\\0&	\frac{1}{2}\left(I_{1,\perp}(r)+ I_{2,\perp}(r)\right) &0 \\
 0 & 0 & 	I_{1,\perp}(r) -I_{2,\perp}(r) 
 \end{array}\right).
 \end{eqnarray}
We have introduced four elementary functions ($I_{\parallel,1}$, $I_{\perp,1}$ $I_{\parallel,2}$, $I_{\perp,2}$), defined as follow:
 \begin{eqnarray}
 \label{Iexp}
 	I_{i,1}(r)&=&\frac{1}{(2 \pi)^3}\int_0^\infty dq q^2 \int_0^\pi d\theta \sin(\theta) \int_0^{2\pi} d\phi\, \chi^{\rm G}_i(q)e^{iqr\cos(\theta)},\nonumber\\ \quad I_{i,2}(r)&=&\frac{1}{(2 \pi)^3}\int_0^\infty dq \int_0^\pi d\theta\sin(\theta) \int_0^{2\pi}  d\phi\, \chi^{\rm G}_i(q)e^{iqr\cos(\theta)}, 
 \end{eqnarray}
with $\quad i=\parallel,\perp$.
  The Gaussian susceptibility finally reads
 \begin{eqnarray}
 \chi^{\rm G}(r)=\left(\begin{array}{ccc}
 \chi_{\perp}^{\rm G}(r) & 0 & 0\\ 
 0 &\chi_{\perp}^{\rm G}(r) & 0\\
 0 &0 & \chi_{\parallel}^{\rm G}(r) 
 \end{array}\right)
 \end{eqnarray}
with 
 \begin{eqnarray}
 \label{chidiagr}
 \chi^{\rm G}_{\parallel}(r)&=&I_{2,\parallel}(r)+(I_{1,\perp}(r)-I_{2,\perp}(r)) \\
 \label{chidiagangl}
 \chi^{\rm G}_{\perp}(r)&=&\frac{I_{1,\parallel}(r)-I_{2,\parallel}(r)}{2}+\frac{I_{1,\perp}(r)-I_{2,\perp}(r)}{2}
 \end{eqnarray}
 in the basis $\{\vec{e}_i\}, i=x,y,z)$ defined such that $\vec{r}$ is aligned with  $\vec{e}_z$. See sketch in Fig.~\ref{fig:3}. \par
The suscpetibility associated with any distance vector $\vec{r}=(x,y,z))$ is  obtained by performing the following change of basis:
\begin{equation}
\label{chicart}
	\chi^{\rm G}_{\rm cart}=R^{-1}\cdot \chi^{\rm G}(r)\cdot R,
\end{equation}
$R$ being the change-of-basis matrix from Cartesian to spherical coordinates.
\subsection{One-loop free energy expansion}
The partition function can be expanded to the second order around the mean field point $(\bm{P},\psi)$, as
\begin{eqnarray}
\Xi&\approx&e^{-\beta F_u[\bm{P},\psi]}\int \mathcal{D}[\delta\bm{\mathcal{P}}]\, \mathcal{D}[\delta\Psi]e^{-\frac{\beta}{2}\int d\vec{r}d\vec{r}'\left(\delta \bm{\mathcal{P}}(\vec{r}), \delta\Psi(\vec{r}) \right)\cdot F_u^{(2)}({\bm P},\psi) \cdot \left(\delta \bm{\mathcal{P}}(\vec{r}), \delta\Psi(\vec{r}) \right) }\nonumber\\&=&e^{-\beta F_u[\psi,P]}\left(\beta |F_u^{(2)}|\right)^{-1/2}
\end{eqnarray}
where we have droped the prefactor.
$F_u^{(2)}({\bm r}-{\bm r'})$ is the second functional derivative of the action and can be defined as follows:
\begin{eqnarray}
F_u^{(2)}(\vec{r}-\vec{r}')=\left(\begin{array}{cc}
\frac{\delta^2 F_u(\bm{P},\psi)}{\delta \mathcal{P}_i(\vec{r}) \delta \mathcal{P}_j(\vec{r}')}    & \frac{\delta^2 F_u(\bm{P},\psi)}{\delta \mathcal{P}_i(\vec{r}) \delta \Psi(\vec{r}')} \\
\frac{\delta^2 F_u(\bm{P},\psi)}{\delta \Psi(\vec{r}) \delta \mathcal{P}_i(\vec{r}')}  & \frac{\delta^2 F_u(\bm{P},\psi)}{\delta \Psi(\vec{r}) \delta \Psi(\vec{r}')}
\end{array}\right)
\end{eqnarray}
with
\begin{eqnarray}
\label{F21}
\frac{\delta F_u(\bm{P},\psi)}{\delta \Psi(\vec{r}) \delta \Psi(\vec{r}')}&=&\left( \epsilon_0\Delta_\vec{r}-2\Lambda\beta e^2{\rm cosh}(\beta e\psi) \right)\delta(\vec{r}-\vec{r}')\\
\label{F22}
\frac{\delta F_u(\bm{P},\psi)}{\delta \mathcal{P}_i(\vec{r}) \delta \mathcal{P}_j(\vec{r}')}&=&\frac{1}{\epsilon_0}\left(2\gamma\Big(P_{ij}^2+P_0^2\right)\delta_{ij}+4\gamma P_i P_j-\kappa_l \partial_i\partial_j  +\kappa_t\left(\partial_i\partial_j-\Delta \delta_{ij}\right)+\alpha\Delta \partial_i\partial_j \Big)\delta(\vec{r}-\vec{r}')\\
\label{F23}
\frac{\delta F_u(\bm{P},\psi)}{\delta \Psi(\vec{r}) \delta \mathcal{P}_i(\vec{r}')}&=&- \partial_i \delta(\vec{r}-\vec{r}'), \quad
\frac{\delta F_u(\bm{P},\psi)}{\delta \mathcal{P}_i(\vec{r}) \delta \Psi(\vec{r}')}= \partial_i \delta(\vec{r}-\vec{r}')
\end{eqnarray}
We set $2\gamma P_0^2=K$ so that the Gaussian asymptotic behavior of the $P^4$-model, reached
in the low polarization limit, is similar to the Gaussian model. The Gibbs free energy at the first order with a loop expansion can be written as: 
\begin{equation}
\label{FreeEnergyOneLoop}
\mathcal{F}\approx F_u[{\bm P},\psi]+\frac{1}{2\beta}{\rm Tr}{\rm ln}{\beta F_u^{(2)}}[{\bm P},\psi].
\end{equation}

 \subsection{One-loop correction for the inverse susceptibility}
 In this subsection, we expand the inverse polarization susceptibility to the first order, $\chi^{-1}\approx \chi^{-1,\rm{G}}+\chi^{-1,1}$. 
Using its expression as a function of the Gibbs free energy of the system,
 \begin{eqnarray}
 \label{DefSusc}
 \frac{1}{\epsilon_0}\chi^{-1}_{ij}(\vec{r}_1-\vec{r}_2)=
 \frac{\delta^2 \mathcal{F}(\bm{P},\psi)}{\delta \mathcal{P}_i(\vec{r}_1) \delta \mathcal{P}_j(\vec{r}_2) }
 \end{eqnarray}
 and the first order expansion of $\mathcal{F}$, (Eq.  \ref{FreeEnergyOneLoop}), we get:
 \begin{eqnarray}
 \label{MatrDeriv}
\frac{1}{\epsilon_0}\chi^{-1,1}_{xx}(\vec{r}_1-\vec{r}_2) & =&\frac{\delta {\rm Tr} {\rm ln} (\beta F_u^{(2)})}{\delta \mathcal{P}_x(\vec{r}_1)\delta \mathcal{P}_x(\vec{r}_2)}(\bm{P},\psi)\nonumber \\ &=&{\rm Tr}\left( \left(F_u^{(2)}\right)^{-1} \cdot \frac{\partial^2F_u^{(2)} }{\partial \mathcal{P}_x(\vec{r}_1)\partial \mathcal{P}_{x}(\vec{r}_2)}-\frac{\delta F_u^{(2)} }{\delta \mathcal{P}_x(\vec{r}_1)}\frac{\delta F_u^{(2)} }{\delta \mathcal{P}_x(\vec{r}_2)}\cdot \left(F_u^{(2)}\right)^{-2}\right)\Big(\bm{P},\psi\Big) .
 \end{eqnarray}
 The second matrix product in the right hand term is vanishing as $\delta F_u^{(2)} / \delta \mathcal{P}_x(\vec{r}_2)(\bm{P},\psi)=0$. We thus get:
 \begin{equation}
 \label{MatrDerivxx}
 \frac{\delta {\rm Tr} {\rm ln} (\beta F_u^{(2)})}{\delta \mathcal{P}_x(\vec{r}_1)\delta \mathcal{P}_{x}(\vec{r}_2)}(\bm{P},\psi) ={\rm Tr}\left( \left(F_u^{(2)}\right)^{-1} \cdot \frac{\partial^2F_u^{(2)} }{\partial \mathcal{P}_x(r_1)\partial \mathcal{P}_{x}(r_2)}\right)(\bm{P},\psi) .
 \end{equation}
We use the expression of $F_u^{(2)}$ given in Eq. (\ref{F22}) and obtain
 \begin{equation}
 \label{MatrDerivxx}
 \frac{\partial^2F_u^{(2)}(\bm{P},\psi) }{\partial \mathcal{P}_x(r_1)\partial \mathcal{P}_x(r_2)}=	\left(\begin{array}{cccc} 12 \gamma /\epsilon_0 & 0 &0 & 0 \\
 0 & 4  \gamma /\epsilon_0& 0 & 0 \\
 0 & 0 & 4  \gamma /\epsilon_0 & 0 \\
 0 & 0 & 0 & 0
 \end{array}\right)\delta(\vec{r}-\vec{r}_1)\delta(\vec{r}-\vec{r}_2)\delta(\vec{r}'-\vec{r})
 \end{equation}
 and 
 \begin{equation}
 \label{d2F2}
 \frac{\partial^2F_u^{(2)}(\bm{P},\psi) }{\partial P_x(\vec{r}_1)\partial P_y(\vec{r}_2)}=\left(\begin{array}{cccc} 0 & 4 \gamma/\epsilon_0 &0 & 0 \\
 4 \gamma/\epsilon_0 & 0 & 0 & 0 \\
 0 & 0 & 0 & 0 \\
 0 & 0 & 0 & 0
 \end{array}\right)\delta(\vec{r}-\vec{r}_1)\delta(\vec{r}-\vec{r}_2)\delta(\vec{r}'-\vec{r})
 \end{equation}
 The other matrices are easily deduced by symmetry.
 
 
 We now have to calculate the trace of matrices in (\ref{MatrDerivxx}) by integrating over the continuous indices ${\int d\vec{r}\,d\vec{r}'}$ and summing over the discrete indices. 
 The polarization correlations depend only on the distance $u=|\vec{r}-\vec{r}'|$.
 We thus write:
 \begin{eqnarray}
\int d\vec{r} d\vec{r}' \left(F_u^{(2)}\right)^{-1} \cdot \frac{\partial^2F_u^{(2)} }{\partial \mathcal{P}_i(\vec{r}_1)\partial \mathcal{P}_j(\vec{r}_2)}= \int d\vec{r}\int du u^2\left( \int d\phi\, d\theta\sin(\theta)  \left(F_u^{(2)}\right)^{-1} \right) \frac{\partial^2F_u^{(2)} }{\partial \mathcal{P}_i(\vec{r}_1)\partial \mathcal{P}_j(\vec{r}_2)} 
 \end{eqnarray}  
Using $\delta (\vec{r}-\vec{r}')=\delta(u)/2 \pi u^2$ and $\Big(F_u^{(2)}\Big)^{-1}(\bm{P},\psi)=\chi_{\bm{P},\psi}^{\rm G}(\vec{r}-\vec{r}')$ given in Eq.~(\ref{DefSusc}),  
 we perform the integral over $\theta$ and $\phi$ and find for the polarization correlation:
 \begin{eqnarray}
 \label{chirbasis}
 \int_0^{2\pi}d\phi\int_0^{\pi}d\theta \sin(\theta) R^{-1}\cdot\chi_{ij}^{\rm G}(u)\cdot R=\frac{4\pi}{3}\left(\chi_\parallel(u)+2\chi_{\perp}(u)\right)\delta_{ij} .
\end{eqnarray}
 We now perform the matrix product of matrix given in Eqs.~(\ref{MatrDerivxx}, \ref{chirbasis}), calculate its trace and find: 
 \begin{eqnarray}
 \label{chim11}
 \chi^{-1,1}_{x,x'}(\vec{r}_1-\vec{r}_2)&=&	\frac{\pi}{\beta}\int d\vec{r} \int_0^{\infty} du 4 \gamma\epsilon_0\left(\frac{10}{3} \chi^{\rm G}_\parallel(u)+\frac{20}{3}\chi^{\rm G}_\perp(u)\right)\frac{\delta(u)}{2\pi}\delta(\vec{r}-\vec{r}_1)\delta(\vec{r}-\vec{r}_2)\nonumber\\
 \label{finalres}
 &=&\frac{20 \gamma\epsilon_0}{ 3 \beta }\left(\chi^{\rm G}_\parallel(0)+2\chi^{\rm G}_\perp(0)\right)\delta(\vec{r}_1-\vec{r}_2).
 \end{eqnarray}
 Note that we find the same value for $\chi_{yy}^{-1,1}$ and $\chi_{zz}^{-1,1}$  and that the cross terms are vanishing. We calculate the susceptibility at $r=0$, $\chi^{\rm G}(0)$, using the  elementary functions defined in Eq.~(\ref{Iexp}) and taken in 0 as follows,
 \begin{eqnarray}
 \label{i1para}
 	I_{1,\parallel}(0)&=&\frac{1}{2\pi^2}\int_0^\infty dq q^2\frac{\epsilon_w/\lambda_D^2+q^2}{(\epsilon_w/\lambda_D^2+q^2)(K+\kappa_lq^2+\alpha q^4)+q^2},\\
 	\label{i1perp}
 	I_{1,\perp}(r_c)&=&\frac{1}{2\pi^2}\int_0^{2\pi/r_c}dq\frac{q^2}{K+\kappa_t q^2}=\frac{1}{\pi K \lambda_t^2r_c}\\
 	\label{i2}
 	I_{2,\parallel}(0)&=&\frac{I_{1,\parallel}(0)}{3}, \quad I_{2,\perp}(0)=\frac{I_{1,\perp}(r_c)}{3}
\end{eqnarray}
  where we have introduced a cutoff length $r_c$ to remove the divergence for $I_{1/2,\perp}$ in $r=0$. \par  The Gaussian susceptibility tensor reads,
  \begin{equation}
  \label{chirm}
  	\chi_{\parallel}^{\rm G}(r_c)=I_{2,\parallel}(0)+I_{1,\perp}(r_c)-I_{2,\perp}(r_c),\quad \chi_{\perp}^{\rm G}(r_c)=\frac{1}{2}\left(I_{1,\parallel}(0)-I_{2,\parallel}(0)+I_{1,\perp}(r_c)-I_{2,\perp}(r_c)\right).
  \end{equation} 
 Using Eq. (\ref{finalres}), we obtain in Fourier space,
 \begin{equation}
 \label{deltaK}
 	\chi^{-1,1}(\vec{q})=\delta K\frac{q_iq_j}{q^2}+\delta K\left(\delta_{ij}-\frac{q_iq_j}{q^2}\right),\quad {\rm with}\quad \delta K=\frac{20\gamma\epsilon_0}{3\beta}\left(\chi_{\parallel}^{\rm G}(r_c)+2\chi_{\perp}^{\rm G}(r_c)\right).
 \end{equation}
 Finally, we get the expression for the inverse polarization susceptibility at the first order,
 \begin{eqnarray}
 	\chi_{ij}^{-1}(\vec{q})=(K+\delta K+\kappa_l q^2+\alpha q^4)\frac{q_iq_j}{q^2}+(K+\delta K+\kappa_t q^2)\left(\delta_{ij}-\frac{q_iq_j}{q^2}\right).	
 \end{eqnarray}
 where we have used the expression of $\chi^{\rm G}(\vec{q})$ given in Eq. (\ref{chim1G}).
  \subsection{Linear dependence of $\chi^{-1,1}(q)$ in salt concentration $c$}
The correction $\delta K$ is now splitted into two contributions, 
\begin{equation}
	\delta K=\delta K_w +\delta K_c c+\tau(c^2)
\end{equation}  
a pure water one $\delta K_w$ and a second one, $\delta K_c c$ depending on the salt concentration and expanded linearly in $c$.

To get explicit expression of $\delta K_w$ and $\delta K_c$, we expand linearly in $c$ the functions $I_{i,x}(0)$, $i=1,2$, $x=\parallel, \perp$ given in Eq.~(\ref{i1para}-\ref{i2}), using $\epsilon_w/\lambda_D^2$=$c\times2\mathcal{N}_ae^2\beta/\epsilon_0$. We get
\begin{eqnarray}
	I_{i,\parallel}(0)&=&	I^w_{i,\parallel}(0)+c\times I^1_{i,\parallel}(0)+\tau(c^2)\\
		I_{i,\perp}(r_c)&=&	I^w_{i,\parallel}(r_c), \quad i=1,2
\end{eqnarray}
in which the functions are split into a pure water contribution and a linear correction in $c$. Note that $I_{i,\perp}$ functions do not depend of the salt concentration, the associated salt correction is thus vanishing. \\ 
The functions for pure water obey
\begin{eqnarray}
I^w_{1\parallel}(0)&=&\frac{1}{2\pi^2}\int_0^\infty dq q^2\frac{1}{1+K+\kappa_l q^2 +\alpha q^4},\quad 	I^w_{2\parallel}(0)=\frac{I_{1,\parallel(r_c)}^w}{3},\\
I^w_{1\perp}(0)&=&\frac{1}{2\pi^2}\int_0^{\frac{2\pi}{r_c}} dq q^2\frac{1}{K+\kappa_t q^2 },\quad 	I^w_{2\perp}(r_c)=\frac{I_{1,\perp}^w(r_c)}{3},
\end{eqnarray}
and the linear correction for the parallel function,
\begin{equation}
\label{i1salt}
	I^1_{1,\parallel}(0)=\frac{1}{2\pi^2\epsilon_0}\int_0^\infty dq \frac{2\mathcal{N}_ae^2\beta}{(1+K+\kappa_lq^2+\alpha q^4)^2}, \quad I^1_{2,\parallel}(0)=\frac{I^1_{1,\parallel}(0)}{3}. 
\end{equation}
The expression of $\chi_{\parallel}^{\rm w}(r_c)$ and $\chi_{\perp}^{\rm w}(r_c)$ are obtained by replacing $I_{i, \parallel}(0)$ by $I^w_{i, \parallel}(0)$ and $I_{i, \perp}(0)$ by $I^{\rm w}_{i, \perp}(0)$ in Eq.~(\ref{finalres}). Finally, we get:
\begin{equation}
	\delta K_w=\frac{20\gamma\epsilon_0}{3\beta}\left(\chi^{\rm w}_{\parallel}(r_c)+\chi^{\rm w}_{\perp}(r_c)\right).
\end{equation}
We set $\delta K_w$ to zero as it is included in the fitted parameter $K$. Using the expressions of $\delta K$ in Eq.~(\ref{deltaK}) and ($\chi_{\perp}^{\rm G}$, $ \chi_{\parallel}^{\rm G}$) in Eq.~(\ref{chirm}), we obtain
\begin{equation}
	\delta K_c=\frac{20\gamma\epsilon_0}{3\beta}I_{1,\parallel}(0).
\end{equation}
We perform the integral in Eq.~(\ref{i1salt}) and obtain
\begin{equation}
	I_{1,\parallel}^1(0)=\frac{\beta \mathcal{N}_a e^2(\epsilon_w-1)^2}{64\pi \epsilon_0\epsilon_w}\frac{(4\pi^2\lambda_d^2+\lambda_o^2)(4\pi^2\lambda_d^2+5\lambda_o^2)}{\lambda_d\lambda_o^4}
\end{equation}
after having expressed $K$, $\kappa_l$, $\alpha$ as functions of $\epsilon_w$, $\lambda_o$, $\lambda_d$ by inverting Eq.~(\ref{longueur}).
We deduce that,
\begin{equation}
	\delta K_c=\gamma\frac{5 \mathcal{N}_a e^2(\epsilon_w-1)^2}{24\pi \epsilon_w}\frac{(4\pi^2\lambda_d^2+\lambda_o^2)(4\pi^2\lambda_d^2+5\lambda_o^2)}{\lambda_d\lambda_o^4}
\end{equation}
Expanding linearly the relation $\epsilon(c)=1+1/(K+\delta K_c c_)$, we obtain 
for the permittivity of the electrolytes
\begin{equation}
	\epsilon(c)=\epsilon_w-\frac{\delta K_c}{K^2}c.
\end{equation}
The inverse susceptibility can thus be written as:
 \begin{eqnarray}
\chi^{-1}(q)&=&\chi^{-1,\rm{G}}(q)+\chi^{-1,1}(q)\nonumber\\
&=&(K+\delta K_c c+\kappa_l q^2+\alpha q^4)\frac{q_iq_j}{q^2}+(K+\delta K_cc+\kappa_t q^2)\left(\delta_{ij}-\frac{q_iq_j}{q^2}\right).	
\end{eqnarray}
that is inverted to get the one-loop corrected susceptibility
\begin{equation}
\label{oneloopchi}
	\chi_\parallel(q)=\frac{1}{1+K+\delta K_c c+\kappa_l q^2+\alpha q^4}
\end{equation}
which we plot for increasing salt concentration in Fig.~\ref{fig:4}. One sees comparing the susceptibilities obtained for the
Gaussian model plotted in Fig.~\ref{fig:2}  and the one-loop expanded one plotted in Fig (S5) that the enhancement of the
pseudo-resonant peak at $q$=3~\AA$^{-1}$ is attenuated but not canceled by the one-loop correction.
 \begin{figure}
	\includegraphics{FigSI3.pdf}
	\caption{One-loop expanded longitudinal susceptibility of electrolytes. The  susceptibility $\chi_\parallel(q)$ given in Eq. (\ref{oneloopchi}), for increasing concentration with $\delta K_c$=0.028 mol$^{-1}$.L. Other parameters are given in caption of Fig.~S1.}
	\label{fig:4}
\end{figure}
\section{Supplementary information for molecular dynamic simulation}
\paragraph{Molecular dynamics simulations}
We simulate a cubic water box of side size $L$=6.5 nm composed of N$_w$ water molecules, N$_w$ going from   9033 to 8527 for increasing salt concentration. See a snapshot of the simulated system in Fig.~\ref{fig:5}. The 0.15 mol.l$^{-1}$ solution contains 25 ion pairs, the 0.75 mol.l$^{-1}$ solution contains 124 ion pairs and the 1.5 mol.l$^{-1}$ solution, 248 ion pairs.
 \begin{figure}
	\includegraphics[scale=0.6]{simuvisu.png}
	\caption{Snapshot of a simulation box of electrolytes. Red and white sticks represent TIP4p/$\epsilon$ water molecules, purple spheres chloride Cl$^-$ ions and green spheres sodium Na$^+$ ions. This picture corresponds to a 0.75 mol.l$^{-1}$ solution. Side size of the box: $L$=6.5~nm.}
	\label{fig:5}
\end{figure}
Simulations are performed using GROMACS 2021 molecular dynamics simulation package (\cite{gromacs}), the integration time steps is set to $\Delta t$=2 fs. Simulation boxes are periodically replicated in all directions and long range electrostatics are handled using the smooth particle mesh Ewald (SPME) technique. Lennard-Jones interactions are cut off at a distance $r_{\rm cut}$=0.9 nm.  A potential shift is used at the cut-off distance. All systems are coupled to a heat bath at 300 K using v-rescale thermostat with a time constant of 0.5 ps. We use MDAnanlysis to treat the trajectories. After creating the simulation box, we perform a first energy minimization. We equilibrate the system in the NVT ensemble for 200 ps, and afterwards in the NPT ensemble for another 200 ps using a Berendsen barostat at 1 bar.
Production runs are performed in the NVT ensemble for 20 ns. 

We performe simulations with TIP4p/$\epsilon$ (\cite{azcatl2014}), a 4 interaction site,  three point-charges and one Lennard Jones reference site model. The Lenard-Jones (LJ) center is placed on the oxygen. Charges are placed on the hydrogen atoms and on an additional interaction site, M, carrying the negative charge. The ions (Na$^+$ and Cl$^-$) were treated according to the force field developed in the reference~\cite{loche2021}.

\paragraph{Statistical treatment}
For the longitudinal and transverse susceptibility, the error bars are derived following the reblocking method~\cite{flyvbjerg1989}.
For the bulk permittivity, we cut the trajectory in 5 statistically independent blocks, compute the bulk permittivity of each block,  estimate the sample variance $\sigma^2$ and define the error bar as $\sqrt{\sigma^2/5}$. \par

\paragraph{Permittivity}
Bulk permittivity is calculated from the total system dipole moments~\cite{neuman83} $\mathbf{M}$ 
according to:
\begin{equation}
\chi = \frac{\left\langle \mathbf{M} \cdot \mathbf{M} \right\rangle - \left\langle \mathbf{M} \right\rangle \cdot \left\langle \mathbf{M} \right\rangle}{3\epsilon_0 k_B T V},
\end{equation}
as implemented in the GROMACS dipoles module. $\mathbf{M}$ is the  is the volume integral of the polarization as $\mathbf{M}=\int_Vd\vec{r} \bm{\mathcal{P}(\vec{r})}$ and $V$ is the volume of the box. Note that the ion polarization is not taken into account.
\paragraph{Susceptibilities}
To compute the $q$-dependent susceptibilities, we use the fluctuation-dissipation theorem, relating the response functions to the polarization fluctuations as follows:
\begin{equation}
\label{FDT}
\chi_{ij}(\vec{q})=\frac{\langle \bm{\mathcal{P}}(\vec{q})\cdot\bm{\mathcal{P}}(-\vec{q})\rangle}{\epsilon_0 k_BT}
\end{equation}

One can express the longitudinal susceptibility as a function of the charge structure factor $S(q)$ and gets
\begin{equation}
\chi_\parallel(q)=\frac{S(q)}{q^2 \epsilon_0 k_BT}
\end{equation}

The charge structure factor in the Fourier space can be decomposed into a intramolecular and intermolecular part,
\begin{equation}
S(q)=S_{\rm int}(q)+S_{inter}(q)
\end{equation}
with $S_{\rm inter}(q)$ the intermolecular contribution
\begin{equation}
S_{\rm inter}(q)=\frac{4 n_w z^2 e^2}{q^2}\left(h_{\rm MM}(q)+h_{\rm HH}(q)-2h_{\rm HM}(q)\right)
\end{equation}
$z$ is the valency, $e$ being the elementary charge, $n_w$ the molecular number density. $h_{\rm IJ}$ is the Fourier transform of $g_{\rm IJ}(r)-1$, $g_{\rm IJ}(r)$ being the radial distribution function associated with the atoms couple IJ. 
The intramolecular contribution can be written as
\begin{equation}
S_{\rm intra}(q)=\frac{4 n_w z^2 e^2}{q^2}\left(\frac{\sin(q d_{\rm HH})}{q d_{\rm HH}}-4\frac{\sin(q d_{\rm HM})}{q d_{\rm HM}}+3\right)
\end{equation}
where $ d_{\rm IJ}$ is the intramolecular distance between atom I and J. 



At low $q$ the precision of this expression of the structure factor become pretty low as the function $h_{\rm IJ}(r)$ is obtained on a finite range imposed by the box size.  To solve this problem, we proceed as follows. 
For $q<$ 2.5~\AA$^{-1}$, we take into account the periodicity of the system, calculate the charge structure factor for discretized values of the wave length $q$, $q=2\pi/L\sqrt{n_x^2+n_y^2+n_z^2}$. We compute  directly the charge structure factor from the charge distribution $	\tilde{\rho}(q)$ in the Fourier space, 
\begin{eqnarray}
\label{rhoq}
%	\rho(r)&=&\Sigma_{i=1}^N q_H\left(-2\delta(r-r_{\rm O,i})+\delta(r-r_{\rm H1,i})+\delta(r-r_{\rm H2,i})\right)\nonumber\\
\tilde{\rho}(q)&=&\sum_{i=1}^{Nw} ez  e^{i \vec{q} \cdot \vec{r}}\left(-2 e^{-i\vec{q}\cdot\vec{r}_{{\rm M},i}}+e^{-i\vec{q}\cdot\vec{r}_{{\rm H1},i}}+e^{-i\vec{q}\cdot\vec{r}_{{\rm H2},i}} \right)
\end{eqnarray}
where H$_{1,i}$ and H$_{2,i}$ stand for the two hydrogens of the molecule $j$. the charge structure factor $S(q)=\langle \tilde{\rho}(q) \tilde{\rho}(-q) \rangle/ V$,

\begin{eqnarray}
S(q) &=&\frac{2q_H^2}{V}\sum_{i,j, j\leq i}\Big(4 \cos(\vec{q}\cdot\vec{d}_{\rm OiOj})-2\cos(\vec{q}\cdot\vec{d}_{\rm OiH1j})-2\cos(\vec{q}\cdot\vec{d}_{\rm H1iOj})\nonumber\\&-&2\cos(\vec{q}\cdot\vec{d}_{\rm OiH2j})-2\cos(\vec{q}\cdot\vec{d}_{\rm H2iOj})+\cos(\vec{q}\cdot\vec{d}_{\rm H1iH1j})+\cos(\vec{q}\cdot\vec{d}_{\rm H2iH2j})\nonumber\\&+&\cos(\vec{q}\cdot\vec{d}_{\rm H1iH2j})+\cos(\vec{q}\cdot\vec{d}_{\rm H2iH1j}))\Big)
\end{eqnarray}
where $q$ is a vector and $d_{AiAj}$ stands for $\vec{r}_{Ai}-\vec{r}_{Aj}$.
%\begin{eqnarray}
%\langle  \rho(q)\rho(-q) \rangle&=&q_H^2\Sigma_{i,j, j\neq i}\Big(4 \cos(q \cdot d_{\rm OiOj})-4\cos(q \cdot d_{\rm OiH1j}))\nonumber\\&-&4\cos(q \cdot d_{\rm OiH2j})+\cos(q \cdot d_{\rm H1iH1j})+\cos(q \cdot d_{\rm H2iH2j})\nonumber\\&+&2\cos(q \cdot d_{\rm H1iH2j})+\cos(q \cdot d_{\rm H2iH1j}))\Big)
%\end{eqnarray}

The transverse susceptibility is performed following (\cite{kornyshevtrans}).
The polarization of the medium in the Fourier space
\begin{equation}
{\bf P}({\bf q})=\Sigma_j {\bf p}_j({\bf q})e^{-i {\bf q}\cdot  {\bf r}_j}
\end{equation}
can be written as a sum over the molecular polarization ${\bf p}_j({\bf q})$ of the molecule $j$ which reads as
\begin{equation}
{\bf p}_j({\bf q})=	\frac{1}{\sqrt{V}}\Sigma_\alpha \frac{e z_\alpha {\bf \delta r}_{\alpha j}}{i {\bf q}\cdot {\bf \delta r}_{\alpha j } }\left(1-e^{-i{\bf q}\cdot {\bf \delta r}_{\alpha j }}\right)
\end{equation}
with ${\bf \delta r}_{\alpha j } $ the distance between the charge $\alpha$ and the center of mass of the molecule and $V$ the volume of the simulation box. 
We then take the transverse part of the polarization ${\bf P}_\perp({\bf q})={\bf q}\times {\bf P}(q)/q$ and define the transverse susceptibility as
\begin{equation}
\label{chiperpMD}
\chi_\perp(q)=\frac{\langle{\bf P}_\perp({\bf q})\cdot{\bf P}_\perp({\bf -q})\rangle}{2k_BT\epsilon_0}.
\end{equation}

Note that we replace $\left(1-e^{-i{\bf q}\cdot {\bf \delta r}_{\alpha j }}\right)/i {\bf q}\cdot {\bf \delta r}_{\alpha j } $ by 1 for $ {\bf q}\cdot {\bf \delta r}_{\alpha j } <$ 10$^{-5}$ to prevent numerical errors.  


% Bibliographies
% \bibliographystyle{unsrt}
% \bibliography{references}
\medskip
\bibliographystyle{unsrt}
\bibliography{RPfinal}

\end{document}